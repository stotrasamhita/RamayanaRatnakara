\sect{पुराकल्पीयरामायणकथनम्}

\src{पद्म-पुराणम्}{सृष्टिखण्डम्}{अध्यायः ३३}{१--१८५}
% \tags{concise, complete}
\notes{This chapter describes the Ramayana of another Kalpa, as told by Shiva to Rama, at the request of the latter. A key difference is the absence of Setubandhanam; instead, the army crosses the ocean by meaans of Shiva's massive bow, the Ājagava!}
\textlink{https://sa.wikisource.org/wiki/पद्मपुराणम्/खण्डः_५_(पातालखण्डः)/अध्यायः_११६}
\translink{https://www.wisdomlib.org/hinduism/book/the-padma-purana/d/doc365826.html}

\storymeta


सूत उवाच-

सन्ध्यावन्दनकर्मक्रियतामिति रामो मुनिमाचष्टायम्

उष्णद्युतिरप्यस्तमुपैति द्विजकुलमेतन्नीडमुपैति १

स्वयमपि सन्ध्यावन्दनकामोऽव्रजदुत्तरदिशमुज्झितयानः

हाहाहूहूकृतसङ्गीतिर्बन्दीप्रमुखप्रस्तुतकीर्तिः २

गौतमीतटमुपेत्य राघवो वायुनन्दनसुधौतपद्युगः

जाम्बवत्कृतकरावलम्बनः प्रापदुत्तमनदीं तु गौतमीम् ३

करद्वये धृतकुशः स राघवः प्रागमद्वरुणदिशामथोत्तमाम्

दत्वा ततोऽर्घत्रितयं यथाविधि प्रहृष्टरोमाथ जजाप सोऽन्तरे ४

सम्प्रार्थयित्वा वरुणं यथाक्रमं शम्भुं वसिष्ठं प्रणनाम राघवः

ताभ्यां कृताशीरगमन्मनःपदं हनूमता क्षालित पादपङ्कजः

जुहाव वह्नीनथ बन्दिमागधैः संस्तूयमानोऽथ विनिर्ययौ बहिः ५

प्रहसच्चन्द्र किरणैः सुधालिप्तमिवाम्बरम्

प्रसन्नताराकुसुमं वितानमिव सर्वतः ६

अथागच्छत्सौधतलं वृद्धामात्येन कल्पितम्

नानासनसमोपेतं सभास्थानं ययौ नृपः ७

अथ मुनिं ह्युपवेश्य स राघवः स्वयमपि प्रथमासनमाभजत्

कपिगणाः परितः पृथुविग्रहा रचनयास्थितिमाप्रतिपेदिरे ८

सुखस्थितं नृपमभिवीक्ष्य स द्विजो वचस्तदा समुचितमाह शम्भुः

इहस्थितो भवति समस्तपूजितः कथं कथा नृपवर वर्तते गुहायाम् ९

आकर्ण्याथ रघूद्वहो द्विजवचः शुश्रूषुरासीत्कथां

तत्रस्थो निपुणं निवार्यवचनं सर्वैः श्रुतं तत्क्षणात्

शुश्रूषामि कथं महाद्भुततया स्वात्माश्रयामन्यथा

रक्षोबाधनवादिनीमथनृपः किन्त्वेतदित्याह च १०

कुम्भश्रोत्रवधः पुरा समजनि प्राप्तो दशास्यो वधं

पश्चादित्ययमन्यथा विरचितं रामायणं भाषते

कोऽयं विप्रवरः समस्तजनता नास्तिक्य सम्पादको

राज्ञांस्थानमुपेत्य वक्ति समया दण्ड्योऽथ पूज्योऽथवा ११

अथाह जाम्बवानमुं रघूत्तमं कथां प्रति

रामायणं न तावकं त्विदं हि कल्पितं मतम्

समस्तमत्र विस्तराद्वदामि देव तच्छृणु

पङ्केरुहस्यसूनुतो मया श्रुतं पुराह्यभूत् १२

जाम्बवन्तं विज्ञाप्य रामचन्द्रो वचनमाह १३

श्रीराम उवाच-

कीर्तय पुराणं मे शुश्रूषुः कुतूहलादहं प्रणीतं तत्केन च विज्ञातम् १४

जाम्बवानथ बभाषे हि विधात्रे नमो नमस्तथैव विधुभूषणकेशवाभ्याम् १५

अथ पुरातनं रामायणं कथयामि १६

यस्य श्रवणेनाखिलजन्मसम्पादित पापक्षयो जायते १७

अथ तथापि दशरथो दशरथसमानरथी महीयसा बलेन सुमानसनामनगरजिगीषया पङ्केरुहसुतसुतं
वसिष्ठमाहूय नमस्कृत्वा मुनिदत्तानुज्ञः शताक्षौहिणीसेनया सहारुह्य तुरङ्गमं चन्द्र
समानशरीरमतिरोषसमाविष्टो विष्टरश्रवसमाराध्य दण्डयात्रां चकार १८

साध्यो नाम स्वीयया सेनयावृतो दशरथाभिमुखमाययौ योद्धुं युद्धं चान्योन्यमभूत् १९

मासमेकं युद्धं कृत्वा दशरथस्तं साध्यं जग्राह २०

अथ साध्यसूनुर्भूषणोनामाल्पपरिवारो युयुधे दशरथेन २१

दशरथोऽपि साध्यसूनुं भुवोभूषणमवलोक्य योद्धुमेव नैच्छत् २२

कथमेतादृशं हन्मि चास्मिन्हतेऽस्य कथं पिता भविष्यति कथं तन्माता कथमप्रौढयौवनाप्रियार्या
 २३

अमुष्य हि देहे समालिङ्गनचुम्बनपरिवर्तन नवीनतरदलारविन्दपदानि कुसुमानीव दृश्यन्ते २४

एतत्समानवर्णवया एतादृशः सुभगः परमप्रीतिवर्द्धनोनाम पुत्रो भल्लूकभक्षितोमृतःस्मृतिपथं
प्राप्यापि मां रक्षयितुमिच्छतीव मम हृदयमन्यथा करोति इति मनसा वितर्क्यातिबालकं
ग्रहीतुमारभत् २५

स च साध्योपि पराधीनो बभूव २६

स च कुमारेण सह पराजय खेदमपि मत्वा सुखमध्युवास च २७

दशरथोऽपि तत्र मासं स्थित्वा तत्पुत्रसन्दर्शनसुखमवलोक्याचिन्तयत् २८

अहो सर्वदुःखापनोदनक्षममेतन्मुखावलोकनं पुत्रसंवर्द्धनं नाम सर्वराष्ट्रिको मम जयः
पुत्रवियोगमनुस्मरतो दुःखाय केवलं भवति २९

तदस्य पृच्छां करोमि कथमीदृशो जायते पुत्र इति वितर्क्य तमपृच्छत् ३०

साध्योऽपि सकलमोक्षमार्गं क्षितीशायादिशत् ३१

हरीशानौ सहाराध्य सर्वैकादशीरुपोष्य द्वादशीषु ब्राह्मणानाराध्य तत्तत्कालभवं
फलपूर्वमन्नाद्यं व्यञ्जनं पुष्पं च न्यायेन सम्पाद्य कपिलाघृतेन केशवं स्नापयित्वा मुद्गचूर्णेन संलिप्य
स्वादूदकेन स्नापयित्वा सुरभिपाटीरं स्वयमुद्घृष्टं मृगनाभ्यागुरुसारेण वासमेतं देवाङ्गं सर्वमुपलिप्य
तुलसीदलैर्यूथिकाकरवीरनीलोत्पलकमल कोकनदद्रो णकुसुम मरुवदमनक
गिरिकर्णिकाकेतकीदलपूर्वैर्यथासम्भवमभ्यर्च्य द्वादशाक्षरेण पुरुषसूक्तेन वा नाम्ना षोडशोपचारेण
वाराध्य प्रणम्य नृत्यं कृत्वा देवं क्षमापयेत् ३२

तथा व्रतानि विचित्राणि नारायणप्रीणनाय कुर्य्यात् ३३

प्रसन्नो भगवान्मुनिरीप्सितं पुत्रं यच्छति तदमुमाराधयस्वेति दशरथमुक्तवान् ३४

स चापि साध्यं तत्र स्थाप्य गत्वायोध्यां तथा सर्वं कृतवान् ३५

अथ पुत्रकामेष्टौ समाप्तायामाहवनीयाद्यज्ञमूर्तिः शङ्खचक्रगदापाणिरुदतिष्ठत्

राजानं च वरं वृणीष्वेत्युक्तवान् ३६

स च राजा वव्रे पुत्रानतिधार्म्मिकान्दीर्घायुषश्चतुरोलोकोपकारकान्देहीति ३७

अथ राजमहिष्यश्चतस्रः कौशल्या सुमित्रा सुरूपा सुवेषा चेति ३८

राजानमब्रुवन्देव प्रतियोषमेकेन पुत्रेण भवितव्यम् ३९

अथ कौशल्योवाच एष यदि प्रसन्नो देवस्तदायमुत्पद्यतां मम ४०

राजोवाच-

मम यदिष्टं तदयं प्रार्थ्यते हरिः

विष्णोप्रसीददेवेश कमलापते शङ्खचक्रगदाधरविभीषणसृष्टिसमस्तलोकपालादिपूजित पादयुगलशाश्वतहरे
नमस्ते नमस्ते एवं स्तुतो भगवानथ राजानमाह ४१

माधव उवाच-

तव पुत्रो भविष्यामि कौशल्यायाम्

अथ चरुं प्राविशद्धरिः

तं चरुं हि चतुर्धा विभज्य भार्य्याभ्यो दत्तवान् ४२

अथ कौशल्यायां रामो लक्ष्मणः सुमित्रायां सुरूपायां भरतः सुवेषायां शत्रुघ्नो जज्ञे
खात्पुष्पवृष्टिश्च पपात ४३

अथ चतुराननः स्वयमुपेत्य जातकर्मादिकाः क्रियाश्चक्रे ४४

त्रिभुवनाभिरामतया राम इति नाम चक्रे रूपशौर्य्यादिलक्ष्मीयोग्यतया लक्ष्मण इत्यपरस्य भुवं
भारात्तारयतीति भरतः शत्रून्हन्तीति शत्रुघ्न इति नामानि कृत्वा ब्रह्मा स्वभवनं जगाम
शिशवश्च वृद्धिमीयुः ४५

अथ पादसञ्चारिणं बालचन्द्र सङ्काशदर्शनं बिम्बाधरमुन्नततिलप्रसूननासं पुरश्चूलिकालम्बमान-रत्नपत्रकं
श्रवणलोललम्बमानकुण्डलं वक्षःस्थलविचलित स्थूलमुक्ताहारं विलसत्कार्तस्वरो ब-ह्वलयं
सिञ्जन्मणिकङ्कणरत्नाङ्गुलीयकं हेममणिरचितश्रोणिसूत्रं
सिञ्जन्ननूपुरोपशोभितपादमङ्गुलीयोपशोभितपादमध्याङ्गुलीयकं वज्राङ्कुशसरोजलाञ्छनशोभितोरुपादतलं
तूणीरसदृश जङ्घं करिकरसदृशोरुं विस्तृतजघनं सूक्ष्ममध्यंवर्तुलावर्तकं
गम्भीरनाभिमिन्द्रनीलशिलाविशा-ल वक्षःस्थलं कम्बुग्रीवञ्चन्द्रबिम्बसदृशवदनमर्द्धचन्द्र सदृशललाटं
नीलकुटिलकुन्तलं क्रोडासक्तन्धूलिभिरापाण्डुरं फुल्लपद्मदलारक्तविलोललोचनं महेश्वरमिवोद्धूलितभूतिं
महेश्वरमिव दिगम्बरं रामं कुमारं राजा दशरथो दृष्ट्वा हर्षपरिपूर्णहृदयः पुत्रमालिङ्ग्य चुम्बित्वा
वक्षस्यालिलिङ्ग दृढम् ४६

अथ कुमारोपि पार्श्वेनास्याङ्कमारोप्य कलकलितलोचनो यत्किञ्चिदुवाच ४७

याचमानमितस्ततो वीक्षमाणः तात गच्छेशये तात क्रीडामि तातेत्यादिपुत्रसुखमनुभूयानुभूय
निर्वृतिं ययौ ४८

अथ कदाचिद्भोक्तुमागते राजनि रामचन्द्रो बालक्रीडासक्तहृदयो बहुक्रीडनकरकमल उत्प्लुत्य
धावमानो नरपतिपुरःस्थितमणिखचितसुवर्णभाजनस्थमन्नं वामकरेण गृहीत्वा राजनि चिक्षेप

इदमपि राजा सुखाय मेने

एतादृशान्यन्यानि चकार रामचन्द्रः ४९

अथ कदाचित्क्रीडमाने रामे वात्या राममपातयद्रामश्च रुदन्नपतत् ५०

एतस्मिन्नन्तरे ब्रह्मराक्षसो राममगृह्णाद्रामश्च मूर्च्छामाप ५१

अथ सहचरो बाल इतस्ततो रोरूयमाणो रामं तथाविधं राज्ञे व्यज्ञापयत् ५२

अथ राजा राममादाय वसिष्ठमाह किमिदं रामस्येति पप्रच्छ ५३

अथ वसिष्ठो भस्मादायाभिमन्त्र्य ब्रह्मराक्षसं मोचयामास ५४

पप्रच्छ को भवानिति स चाहाहं वेदगर्वितो ब्राह्मणो बहुशः परधनमपहृत्य ब्रह्मराक्षसो जातो
मे निष्कृतिं विचारय ५५

वसिष्ठ उवाच-

इदानीमितः परमेकवर्षशतोपभोग्यं राक्षसत्वं नरकं भागीरथीस्नानमेकं शिवाय बिल्वपत्रशतं
समर्प्य ततः स्नात्वा पापाद्विमुक्तो भवसीति ५६

कदाचित्तादृशं कृतपुण्यं तव पदं प्रयच्छामि तदुपरिशिष्टां गतिं भजेति वसिष्ठवाक्यमाकर्ण्य
ब्रह्मराक्षसो वसिष्ठोपदिष्टपुण्यवशाद्दिव्यशरीरो भूत्वा नमस्कृत्वा स्वर्गं जगाम ५७

अथ रामं प्राप्तेकाले उपनीय वसिष्ठो वेदानध्यापयामास षडङ्गानि मीमांसाद्वयं नीतिशास्त्रं
चाध्यापयामास ५८

अथ धनुर्वेदमायुर्वेदं भरत गान्धर्व वास्तु शाकुन विविधयुद्धशास्त्राणि च ५९

अथ विवाहं कर्तुकामेन राज्ञा दशरथे नानानादेशजनपतीन्प्रति दूताः प्रेरिताः ६०

अथ कश्चिच्छीघ्रमागत्य राजानमिदमब्रवीत् ६१

राजन्विदर्भदेशाधिपतिर्विदेहो नाम राजा तस्य पुत्री वैदेही होमलब्धारूपेण लक्ष्मीसमा
सर्वलक्षणसम्पन्ना रामयोग्या विद्यते स च तां दातुं राजा रामायोद्यतः तद्गम्यतां शीघ्रमिति
 ६२

अथ वसिष्ठादीन्प्रेषयामास ते च तत्र गत्वा तां च निरीक्ष्य लग्नं निश्चित्यायोध्यामेत्य
राजानमुक्त्वा रामसहिताः पृथिवीपतिसमेताः शीघ्रं विविध करि तुरग शकट
शिबिकान्दोलिकाभिरतिसुभगरूपभोगविलासक्रियानिपुणा विदितविविधचेष्टागन्धर्वाः
कामशास्त्रसुकुशलाः मृदुकठिनपृथुपयोधरासन्न कण्ठाः स्थूलसूक्ष्मललाटबिम्बदशनच्छदमुखपङ्कजाः
कुटिलकुन्तलदीर्घकेशधम्मिल्लाः कनकपत्रकर्णाः स्नानचेष्टयोत्थितरोमशोभित जपाकुसुमरक्तदशना
विशदविस्फुरच्छफरीलोचनाः शुक्तिकासदृशश्रवणाः नक्षत्रसदृशस्थूलमुक्ताफलोपशोभितनासापुटा
मुकुरसदृशकपोलास्तिलप्रसूननासिका आनम्रमध्यप्रदेशचूचुका इन्द्रगोपप्रतीकाशाधरपुटदशनक्षताः
समदीर्घकाङ्गप्रदर्शनास्थितसर्वप्रदेशवर्तुला नातिमांसलाः पिण्डकाग्रन्थिनीव्योवलितबाहुमूला
अनतिचिरकालोत्थितरोमतया हरिद्रा वर्णतया च कर्णिकारदलसदृशबाहुमूला
मृदुस्निग्धवर्तुलसूक्ष्ममध्यप्रदेशाः कठिनस्थूलवर्तुलामग्नचूचुकपरस्परस्थानाक्रमणस्पर्द्धि
पयोधरमध्यलब्धपदकपयोधरोपरिचञ्चलविविधमणिमय हारोपशोभित वक्षःस्थलाः पयोधरपरितो
लब्धपदतया तरुणदृष्टिपरम्परतया असमानयानाभिकूपोपरितन रोमराज्योपशोभितोदरप्रदेशा
भज्यमान मध्यस्थलीकरणएव वलीत्रयोपशोभित मुष्टिग्राह्यमध्याः करिकरोपमजघनप्रदेशा
अरोमसदृशमृदुस्निग्धामला समजान्व्यः कदलीस्तम्भसन्निभोरुयुगला
आमग्नजानुकृशकुशवर्तुलपिण्डिकारहितजङ्घा आमग्नगुल्फा आसूक्ष्मस्निग्धा दीर्घदीर्घाङ्गुली
पादानूपुररवाह्वयमानमदना हंसमतङ्गजगमना दक्षिणाङ्गुष्ठस्पर्शिकच्छाग्रा उपरिकच्छं नीवीं कृत्वा
करद्वययुता वस्त्रप्रदेशकण्ठमप्रावृत्या परवसनपरिभागा वृतस्तनवसना परभागे वामांस एव
दक्षिणपार्श्वगतेन दशाभागेन नाभिप्रान्तेन प्रवेशितोपशोभितगात्रयष्टयो योषितो
विवाहमङ्गलकर्मकरणायानेकश आगच्छन् ६३

बालिकाश्च विद्युल्लतांशुशोधितगात्रयष्टय उद्भिन्नकुचकमलकुड्मलविविधहारोपशोभिवक्षसो
यत्किञ्चिद्भाषिण्योऽतिचपलमृदुगतयो वृद्धवनिताश्चागच्छन् ६४

अथ विदेहे पुरतः क्रोशमात्रे चूतवनिकायां
विविधविटपविस्तरप्रदेशविविधविहङ्गकूजिता-कर्णनदत्तकर्णवनहरिणशाववत्यां
महारजतनिर्मितोच्चनीचप्रासादोपशोभितप्रदेशविविधविहङ्गायां
हेमवल्कलसंवीतभसितो-द्धूलितशरीरजटिलमुनिगणध्यानोपासनोपशोभितवृक्षमालायां
विविधविद्यधरवधूपयोधरभाराभिभूतविचरिततरङ्गसरसीयुतायां
सरस्तीरमिलितसैरन्ध्रीयुवतिभिराहूयमान तरुणजनायां
नानावर्णकुसुमसौरभवासिताशेषप्रदेशायामितस्ततो रिरंसया
प्रदर्शितस्फारशफरीविलोचनतरलचक्षुषा प्रभाविलसितशरीरवेश्याजनायां विविधाश्चर्ययुतायां
दशरथः सामात्यपुरोहिताभिरामरामादिपुत्रसहितः सुखमुवास ६५

अथ वैदेहोऽपि मिथिलां नानापताकोपशोभितां
विविधप्रासादगोपुरोद्यानदेवतायतनोपशोभितामन्योन्यकेलिचतुरयुवतिजनानुकीर्णामुशीरविरचितमहाप्रपां
सुकेलीजनोपशोभितविशिखां

विविधपण्योपशोभितरथ्यां तत्रतत्र ब्रह्मघोषशोभितमठां प्रतिमन्दिरं मीमांसादिव्याख्यानसम्पादि
सामाध्ययनां
सुपुण्यहविर्गन्धसामादिस्वरपदक्रमश्रुतिब्राह्मणवाटिकामनेकपरिवृढमन्दिरप्रवेशनिष्क्रीतागुरुकुङ्कुमाध्वर्य्युवेषां
मृदुलवसन ताम्बूल रक्तदन्तच्छदकामिनीमृदुवचनकठिनवचनकरसंज्ञानिर्द्धारित प्रतिवचनविविधोपायना
हरणकरजनोपशोभितां मृदुधवलजघनपरिवीतवस्त्रोपरिभागेन
स्निग्धवर्तुलपरस्परसङ्घर्षपयोधरमध्यप्रदेशोपशोभित वामांसकन्दोपशोभितवनितां
विविधमुक्ताहारजपासङ्काशदशनच्छद मन्दहासमालाकारसहस्रोपशोभितां पुण्यासवसाधनमन्दिरां
तत्रतत्र विचित्रतोरणां विशुद्धवीथिकां तत्रतत्र स्थापितकल्पपादपां रम्भाविभूषितद्वारां पुरीं
शोभितां शोभयामास ६६

अथाभिकलनार्थं विलासिन्यो निशादूर्वाक्षतामन्त्रमङ्गलकज्जलितकैशिकधमिल्ललताग्रन्थि-त
जटोपशोभित सीमन्तशीर्षशोभितनासामुखविचित्राभरणार्हणहेमपात्रावस्थिताज्यगुग्गलु
फलादिसौभाग्यद्रव्यमुद्वहन्तीभिः स्त्रीभिरन्यैरपिशोभितजनैः स राजा निर्जगाम ६७

तदानीं मङ्गलतुर्यघोषादेव दुन्दुभिभेरीनिःसाणमर्दलशङ्खादिनादाः प्रादुर्बभूवुः ६८

गायकाश्च मङ्गलानि जगुः ६९

मङ्गलवेदवाक्यानुपाठेन वैदिकाः ब्राह्मणाः

कुलपाठका भेरीघोषेण कृत्स्नमाकाशमापूरयन् ७०

अथान्योन्याक्षतपूर्वमङ्गीकुर्वन्तः सूतबन्दिजनादिभिः स्तूयमानाः पुरं प्रविविशुः ७१

विदेहनगरात्पश्चिमभागे निर्मितं मन्दिरं दशरथः प्रविवेश ७२

अवशिष्टाश्च यथायोग्यं भवनं विविशुः ७३

अथ नारदो मिथिलां तदानीमेवागच्छत् ७४

विदेहोऽपि देवर्षिमभिपूज्य स्वागतं दृष्ट्वा भोजनं कारयित्वा

सुखासीनाय मुनये सघनसारं ताम्बूलं दत्वा व्यज्ञापयत् ७५

श्वो विवाहे भवान्स्थातुमर्हति कारयितुं विवाहम् ७६

नारद उवाच ७७

श्वो हि नक्षत्रं सूर्यनक्षत्रदर्शनं तत्र विवाहो न कर्तव्य इति ७८

अथ मौहूर्तिकं वृद्धगार्ग्यमाहूय राजा पप्रच्छ क्व विवाहमुहूर्तः ७९

श्व इति गार्ग्य उवाच ८०

राजा च नारदं गार्ग्यं चोदीक्ष्य भो इदमित्थमिति पप्रच्छ ८१

अथ नारदो गार्ग्यमुवाच ८२

कथमुक्तं लग्नं दास्यसि ८३

अथ गार्ग्यो विषघटिकाश्च विहाय लग्नं दास्यामि इत्युवाच ८४

नारदोपि ब्रह्मवचनानि किं न जानासीत्युक्तवान्गार्ग्यम् ८५

गार्ग्यस्तुष्टस्तान्दोषानपठत् ८६

उल्का च ब्रह्मदण्डश्च मोघः कम्पस्तथैव च

सर्वकार्यविनाशाय दृष्टा वै ब्रह्मणा पुरा ८७

प्रतिष्ठासु विवाहेषु मौञ्जीबन्धाभिषेकतः

अन्येषु सर्वकार्येषु विषनाडीर्विवर्जयेत् ८८

अतः परं तु कार्य्याणां करणेन च दोषभाक्

विवाहादिषु कार्येषु दोषमेव वदाम्यतः ८९

उल्का दहेत्कुलं सर्वं ब्रह्मदण्डो विनाशयेत्

मोघस्तु मरणा यस्यात्कम्पः कम्पाय कर्म्मणः ९०

इति नारदोक्तमाकर्ण्य गार्ग्यो मुनिर्मौनपरोऽभवत् ९१

दध्यौ रविं ग्रहपतिं विहाय विषनाडीर्विवाहः क्रियतमिति ९२

नारद उवाच-

कथं ब्रह्मवचनम् ९३

सूर्य उवाच-

देशभेदेन व्यवस्थोदिता तदस्मिन्देशे विवाहो विषघटिकां विहाय कर्तव्य एव ९४

नारदोप्यनुमेने ९५

उवाच श्वः पराह्णे च क्षत्त्रविवाहश्च भवेदतः स्वयं वरार्थं नृपा आगच्छन्तु तन्नृपदूतान्प्रेषय ९६

अथ राजा दशरथानुमतेन सर्वानेव नृपानागमय्याचिन्तयत्

कथं सर्वानेव तिरस्कृत्य वैदेही रामाय देयेति ९७

अथ रात्रौ मुहुर्मुहुर्निःश्वस्य निद्रालुरपि न निद्रामाप ९८

अथ मध्ये निशं राजा शुचिर्भूत्वा त्र्यम्बकं साम्बिकं मङ्गलदुकूलधारिणं
कमलभवपुरुषोत्तमशक्र-प्रमुखनिखिलदेवैर्भृगुप्रमुखमुनिवरैर्हाहाप्रमुखगन्धर्वैस्तुम्बुरुप्रमुखैश्च
श्रुतिस्मृतीतिहासपुराणैर्मूर्तिमद्भिश्च सिद्धविद्याधरादिमातृकागणैश्च नन्दीप्रमुखगणैश्च सेव्यमान
पादकमलं सर्वामङ्गलपरिमोचकमतिपुण्यसलिलया गङ्गया निष्कलं केन चन्द्रमसा सेव्यमान शिरोभागं
वामाङ्कारूढया गिरिजया प्रदीयमानवीटिं सहासं सकामं सहावेक्षणमाददानं गोक्षीरसदृशं
प्रतिकूलकस्तूरिकासदृशकण्ठं मृदुसूक्ष्मस्निग्धजटाभिर्विरचितकपर्दं
विशुद्धकार्तस्वरकुण्डलोपशोभितगण्डभागं द्विरष्टवर्षवयसं गोक्षीरसदृश
स्थूलमुक्ताफलकौसुम्भवर्णाञ्चलेनावेष्टितशिरोभागं
विविधरत्नविरचितकार्तस्वरभूषितवक्षःस्थलमतिधवलोपवीतेनोपशोभितशरीरमम्बिकानुलग्नकुङ्कुमारुणसुगन्धिशरीरमीक्षमाणं
तर्जमानकाममार्गणं कोटिकन्दर्पसदृशं मनसाचिन्तयत् ९९

जजाप शतरुद्रियं जुहाव च तेनैव कामाहुतीः प्रास्तुवच्च पुरुषसूक्तेन १००

अथ तादृश एव महेश्वरस्तत्र प्रादुरभूत् १०१

अथ राजा नमस्कृत्यास्तुवीत १०२

राजोवाच-

क्षितिसलिलगगनपवनदहनरविशशियजमानमूर्तिभिरष्टमूर्ते विश्वमूर्ते लोकमूर्त्ते त्रिभुवनमूर्ते
वेदपुराणमूर्त्ते यज्ञमूर्ते स्तोत्रमूर्त्ते शास्त्रमूर्त्तेस्वधामूर्त्ते नारायणमूर्त्ते सर्वदेवतामूर्त्ते
त्रयीमयत्रयीप्रमाणत्रयीनेत्रसा-मप्रियवसुधाराप्रियभक्तिप्रियभक्तसुलभाभक्तविदूरस्तुतिप्रिय
धूपप्रिय दीपप्रिय घृतक्षीरप्रिय द्रो णकरवीरप्रिय श्रीपत्रप्रिय कमलकह्लारप्रिय
नन्द्यावर्तप्रिय बकुलप्रिय यूथिकाप्रिय कोकनदप्रिय ग्रीष्मजलावासप्रिय यमनियमप्रिय
नियतेन्द्रियप्रिय जपप्रिय श्राद्धप्रिय गायनप्रिय गायत्रीप्रिय पञ्चब्रह्मप्रिय सदाचारप्रिय
गोत्रोत्सादिकमलभवहरिहरनयनसमर्चित पादकमलजयप्रद
हरिप्रार्थितजलोत्पादितचक्रप्रदर्शकृत्स्मृतियुक्तिप्रद स्मृतमङ्गलप्रद मृत्युञ्जय नमस्ते नमस्ते १०३

इति स्तोत्रमाकर्ण्य भगवान्भवो राजानमुवाच वरदोऽहं वरं वृणु १०४

राजोवाच श्रीमन्मम कन्या वैदेही रामाय दित्सिता स्वयंवरे कुलरूपबलोत्साह
सम्पन्नानेकभूपराक्षसविप्रादिसर्वप्राणिसमागमे रामाधिकबलो यदि तामग्रहीष्यत्

तदा वचनमनृतं मम पापं च भविष्यति १०५

प्रत्युत दशरथोऽपि सर्वानेवागतान्विजेतुमलं क्षत्रकदनश्च रामो यद्यायास्यति तर्हि मम सुतां
किं करिष्यति वा किङ्किं वा प्रेषयिष्यति कीदृशं कारयिष्यति मम किंवा करिष्यति सर्वथा हि
प्रभूतबलवाहनो नरपतिरशेषमपि त्रिभुवनं हन्यात् किमुत मामल्पसत्वं किमुत बहुना भवानेव शरणं
ममोपायं वद यथा विवाहे श्रेयो भविष्यति रामश्च जामाता भविष्यति १०६

शम्भुरपि तथा करोमीत्युवाच १०७

रामएव नाथः सीताया भविष्यति

रामं च कृत्वा स्वस्त्यद्यैव करिष्यामि गृहाणाजगवं धनुरिदम् १०८

राजोवाच-

किमेतेनाजगवेन धनुषा स्वयंवरे सीतां रामं प्रापय १०९

शङ्कर उवाच-

इदं धनुरसज्यं मे यस्तु सज्यं करिष्यति

तस्मै देया मया सीता प्रतिज्ञामेवमाचर ११०

इत्येवमुक्त्वा भगवान्गणैरन्तर्दधे हरः

अथादातुं धनू राजा न शशाकातियत्नतः १११

अथोज्ज्वलं शतसहस्रगजबलं समाहूय गृहाणेत्युवाच ११२

स चापि मातुलं नत्वाट्टहासं कृत्वोत्प्लुत्य धनुर्द्वाभ्यां कराभ्यामुद्दधार जानुपर्यन्तं मातुलो

मारीचः श्रुत्वा एकाकी विप्रवेषं कृत्वा विदेहमयाचत

वैश्वदेवान्ते प्राप्तमतिथिं मामवैहि ११३

राजोवाच-

स्वागतं भो इदं ब्रह्मन्नासनं तत्र निषीदेति ११४

स चातिथिस्तथेत्युक्त्वा निषसाद ११५

अथ राजा जलमादाय पादौ प्रक्षाल्य गन्धपुष्पाक्षतैरभ्यर्च्य महाजं तस्मै निवेद्य भोजनाय
प्रार्थयामास ११६

स चापि तदन्नं षड्रसोपेतं सौवर्णभाजनगतमीक्षमाणइवेतस्ततो विलोकयामास ११७

तस्मिन्नेवावसरे सीता पद्मकिञ्जल्कप्रभेषदरुणवसनं बिभ्रती नीलकुटिलकुन्तलैश्चलद्भिर्यूनां
मनांस्याकर्षयद्भिः प्रेक्षमाणदृष्टिभग्नकलैरिव स्त्रीणां चित्तमीदृशमिति
दर्शयद्भिरिवोपशोभित-ललाटानङ्गचापसुभ्रूपद्मपत्रारुणविलोचना तिलप्रसूननासामृदुस्निग्धरो
मशकपोलानन्तरा रक्तोष्ठराक्तासनमाणिक्यनिभदाडिमीदशना
जपाकुसुमारुणाधरातिशोभितचिबुकाशुक्ति-कर्णासमदीर्घकण्ठातिमांसलवक्षाः
पीनोद्भिन्नकुचकुड्मलानेकहारोपशोभिता सुभगाका-

रानतिमांसलबाहुलता
मुग्धायतसमानाङ्गुलिशिखापद्मारुणपल्लवाविविधबहुरत्नाङ्गुलिभूषणामुष्टिग्राह्यमध्यासु
रोमराजिगम्भीरनाभिः पृथुजघनाकरिकरोरुस्तूणीरजङ्घासुपादकमला-नूपुरादि पादविभूषणा
पादाङ्गुलीभूषिता विकसितसौगन्धिकं विदधती भुञ्जानमारीचस्य पुरतश्चागता ११८

वीक्ष्यासावचिन्तयदेनां कथमपहरामि कथमालिङ्गामि
कथमन्यत्किञ्चित्करोमीत्येवमवसरमलभमानस्तूष्णीमेव विनिर्गतः ११९

अथ देवा धनुःसज्जीकरणाय यतमाना अहम्पूर्विकया विद्यमाना अन्योन्यतिरस्कारेण महेन्द्रः प्राप
धनुरुत्तमं प्रान्तद्वयात्परं नावनमयितुं शशाक १२०

अथ सूर्यो धनुरादाय नमयन्नेव निपपात १२१

वायुर्बलवतां श्रेष्ठो जग्राहाजगवमथ स्वेनैव करेणोत्कर्षयन्नधः पपात धनुश्च वायोरुपरि पपात
अहसंस्तदा सर्वे १२२

एतस्मिन्नन्तरे तुरगवरमारुह्य बाणासुरः सहस्रबाहुरनेकानेकशिरोभिर्दैत्यैः परिवृतः प्रह्लादसमेतो
विदेहपुरीमाजगाम १२३

अथ स्वविभूषणोद्भासितां दिशं कुर्वन्स्वतेजसापयशसो देवताः कुर्वन्नाना-

विधगीतं शृण्वन्द्व्यङ्गुलमात्रेण शक्तो विरराम १२४

प्रह्लादोबलिश्चैवधावातेऽथविरेमतुः १२५

अथ राक्षसेषु तूष्णीम्भूतेषु राजानोऽतिबलिनः समागता ज्याबन्धाशक्ता अपसृत्य तस्थुः १२६

अथ ब्राह्मणाः समागताः १२७

अथ विश्वामित्रो धनुरादाय एकाङ्गुलपर्यन्तं सज्यं कृत्वा विरराम

निवृत्ताश्चापरे १२८

अथ दिनमात्रे धनुषि तूष्णीम्भूतेषु राघवः सहानुजैरागत्य धनुर्निरीक्ष्यास्पृशत् १२९

अथ राजकुमाराः शतशः समागताः

सर्वाभरणभूषिता धनुर्दृष्ट्वापस्पृशुर्न चालनक्षमाः १३०

अथ दाशरथिप्रमुखाःकुमाराः समागताः १३१

अथ वेत्रझर्झरपाणयः समागमन्सर्वानेवापसारयामासुः १३२

अथ रामो लक्ष्मरणहस्तं गृहीत्वा सर्वाभरणभूषितो धनुरासाद्य स्पृष्ट्वा नत्वा प्रदक्षिणीकृत्य
धनुरादायोद्दधार १३३

तदादानसमये सर्व एवैत्य सहासमूचुः अत्र भग्ना महारथा इति १३४

अथ स रामो धनुर्ज्यास्थानमवनमय्य धनुषि जानुं कृत्वा सज्यमेककरेणोत्पादयन्कोट्यामनामयत् १३५

अथसज्जीकृतं दृष्ट्वा सर्व एव नासाग्रन्यस्ताङ्गुलयोऽभवन् १३६

रामोऽपि ज्यामन्वनादयत्

तेन नादेन सर्वेषां मनांसि क्षुभितान्यासन् १३७

रामेणसज्यितं धनुरिति सर्वत्र वादः सञ्जातः १३८

जनकोऽपि सीतां रामाय ददौ राजभिश्च युद्धं कृत्वा तान्निर्ज्जित्य स्वपुरीमागात् १३९

अथैकदा दशरथो रामं यौवराज्येभिषिच्य सुखी बभूव सर्वप्रजारञ्जनाच्च रामो राजानुमत इति
सर्वप्रजावादोऽभूत् १४०

अथ कैकयदेशाधिपतितनया सुवेषा रामं राजानमसहमाना राजानमुवाच मम वरदानावसर इति
राजाचिन्तयत्किं देयमिति १४१

देव्युवाच-

चतुर्दशवर्षाणि रामो वनं विशतु पालयतु राज्यं भरतः १४२

राजानृतवचनदोषभयात्कथङ्कथमपि स्वीचकार १४३

अथ वसिष्ठं भावितयावोचत रामो वनाय निर्गच्छति अस्य किंवा भवेदिति विचार्य शुभाशुभं
ब्रूहि १४४

वसिष्ठो विचार्य सहर्षं राजानमुवाच १४५

गत्वा वनं निखिलदानववीरहन्ता शम्भोरनेकविधपूजनमातनोति

सीतावियोगरुषितः कपिसेनया च तीर्त्वोदधिं दशमुखं च निहन्ति रामः १४६

आगम्य राज्यं रघुनन्दनोऽपि बहूनि वर्षाणि समातनोति

प्रशस्तकीर्तिर्निखिलेपि लोके शर्वेण देवेन चिरं न्यवात्सीत् १४७

सुपुत्रयुक्तो बहुयज्ञयाजी परिवृढः सर्वगुणाधिकश्च १४८

इति वसिष्ठवचनं श्रुत्वा दशरथो रामगुणाननुस्मरन्नित्युवाच श्रेयो मे मरणं रामस्य निर्गमने इति
 १४९

अथ रामो मातरं पितरं गुरुं च वसिष्ठं पितृपत्नीर्नमस्कृत्य वनाय जगाम १५०

अथोपवने दिनमेकं स्थित्वा जटाः कारयित्वा वल्कलं वासो धृत्वैकोपवीती कृतदन्तशुद्धिरेकेनोपवीतेन
जटां बद्ध्वा भस्मोद्धूलितसर्वाङ्गो भसित निष्ठुरकायो मुक्ताफलदाममणिव्यत्यस्त
रुद्राक्षमालामुरसि दधानोऽल्पभूषणाधिभूषित सीतासहायो लक्ष्मणानुचरो विवेश वनान्तरम् १५१

अथानेक राक्षसांस्तस्मिन्निजघान भवानिव निखिलं चकार सीतापहरणादिनिखिलमपि भवता
यथातथा स्याथ सुग्रीवाश्रममृष्यमूकपर्वतं रामो जगाम निबिडच्छायाचूतवृक्षमासाद्य
लक्ष्मणसहायः परिश्रयमकल्पयत्

वृक्षे तु धनुषी आरोप्यासीनलक्ष्मणाङ्के शिरः कृत्वा हरिचर्मशय्याशयनोलक्षिताङ्गीतिं शृण्वन्वृक्षफलं
निरीक्षमाणो वानरमेकं मणिकुण्डलं हेमपिङ्गलं सदृढबद्धमौञ्जीकौपीनमच्छोपवीतिनमतिचञ्चलं
फलमादायात्मनिविक्षिपन्तं पुष्पमञ्जरीश्च किरन्तं गानमनुकुर्वन्तं व्यजनेन रामं वीजयन्तमारुह्य
शाखामपि तथा वीजयन्तमाबद्धचूतफलमात्रं रामो वीक्ष्य लक्ष्मणमभाषत १५२

लक्ष्मण कोऽयं कपिरिति १५३

लक्ष्मणोऽपि न जान इत्युवाच

अथ रामः समाहूय कस्य त्वं किं नामत्येपृच्छत् १५४

स च सुग्रीवस्य हनूमानित्युवाच १५५

रामं नत्वा सुग्रीवमेत्य नत्वा देवनारायणइवापरः पुरुषो युवा मेघश्यामो जटी
आजानुबाहुरतियशस्वी सूर्यसङ्काशेन सहापरेण नरेण इहास्ते १५६

अथ तरुच्छायाधः संस्थितौ सर्वलक्षणसम्पन्नौ राजपुत्रौ दृष्ट्वा उक्तश्च ताभ्यां सुग्रीवाय
निवेदयेति तत्त्वयि निवेदितम् १५७

अथ सुग्रीवः सत्वरमुत्थाय पुष्पसलिलादिद्रव्यमादाय पादप्रक्षालनादिकं कृत्वा फलानि समर्प्य
व्यज्ञापयत् १५८

कौ युवां किमर्थमागतौ राजपुत्रौ तपस्विनाविति सुग्रीववचनमाकर्ण्य लक्ष्मणेनाभाषत रामः
 १५९

दशरथतनयावावां रामलक्ष्मणौ दुष्टनिग्रह शिष्टपरिपालनाय वनं गताविति १६०

अथ सुग्रीव आह युवयोरुपकारमपकारं कार्यमस्तीति लक्ष्यते १६१

अन्यथा सेनासमेतावागमिष्यतः लक्ष्मण आह अस्ति कार्यान्तरम्

अमुष्य भार्या केनापहृता न ज्ञायते तामन्वेष्टुमागतौ तदेवावयोः कार्य्यमन्यदानुषङ्गिकम्

तदर्थमपि जलधिं तराव अपि पातालं प्रविशाव अपि नाकं साधयावः अपि महेन्द्रं पातयावः अपि
बलिनं हनावः किमपि कुर्वहे १६२

सुग्रीव उवाच-

रावणेनापहृतया कयाचिद्ध्रियमाणागतया विभूषणानि कानिचित्परित्यक्ता निगतानि मया
सङ्गृहीतानि तानि दर्शयामीत्याभाष्य रामं मन्दिरमागमय्य दर्शयामास १६३

रामोऽपि निरीक्ष्य निश्चित्य प्ररुद्य क्व गतोऽसौ रावण इति पप्रच्छ स च दक्षिणामाशां गत
इति बभाषे १६४

अथ रामस्तेन सख्यमकरोत्

अपृच्छच्च किमर्थमिह भार्य्याहीनः स्थित इति १६५

सुग्रीव उवाच-

मम भ्राता वाली महाबलो ममभार्य्यां राज्यं चापहृत्य किष्किन्धायामास्ते युद्धेन चाहं
पराजितः तद्वधाय सर्वथा मम चिन्ता यथासौ त्वया निहन्यते तथाहमपि सागरं बद्ध्वा परतीरे
लङ्कायां स्थितां सीतां रावणेनापहृतां तव समर्पयामीत्याभाष्य शपथं कृत्वा सुग्रीवो
वालिनातिबलिना युद्धायाहूतेन युयुधे १६६

रामोप्यनन्तरमनिश्चयाद्वालिनं नाहनत् १६७

अथ सुग्रीवः पलायितो राममिदमभाषत १६८

तव चित्तमविज्ञाय प्रवृत्तोऽहमरणाय १६९

रामोपि युवयोर्विशेषाज्ञानान्मया तूष्णीं भूतं चिह्नितं त्वा निरीक्ष्यतं हन्मि १७०

अथ सुग्रीवश्चिह्नं कृत्वा वालिनं युद्धायाहूय समतिष्ठत १७१

तारा बभाषे वालिनम् १७२

सहायवानिव लक्ष्यते सुग्रीवो नोचेदेवं नाह्वयति ज्ञातं मया रामलक्ष्मणौ दशरथतनयौ
नारायणांशौ भूभारावतरणाय समागतौ तावस्य सहायभूतौ १७३

वाल्युवाच-

नीतिमान्राम इति मया श्रुतः

नहि बलवन्तं विहाय दुर्बलं भजते तादृशः समायातु वा रामः प्रतिपन्नमधिकं कृत्वा बिभेति वीरो
यदि रामः स्वयं युद्धाय यातस्तदा युद्धं कर्तव्यमित्याभाष्य तारां सम्भाव्य सुग्रीवयुद्धाय
निर्य्यातः १७४

अथ मुष्टियुद्धमन्योन्यमभूत् १७५

रामोऽपि वालिनं जघान १७६

पपात च वाल्याह चाशस्त्रयुद्धे वा बाणघातोऽथ शोणितसर्वाङ्गो बभूव १७७

अथ तारा चाङ्गदश्च समागत्य व्यथितौ बभूवतुः १७८

अथ राघवं वानराः समायाता वाल्युपान्ते निपेतूरुरुदुश्च १७९

अथ तारा रामं बभाषे शास्त्रकुशलाः शूरा धार्मिका राघवाः पुरा चापि राम कथं
पापमकार्षीः १८०

न क्षत्रधर्म्मं जानीषे राजगणसेवितम् १८१

अन्योन्यं युद्ध्यतोर्युद्धे जयो वा मरणं भवेत्

अन्यो यदितयोर्हन्याद्ब्रह्महा स निगद्यते १८२

किं वैरेण वालिनमाहनः किं वानरमांसाशया १८३

अभोज्यं वानरं मांसं यद्यात्मनोऽप्रियात्सुखाभावादपरेषामपि तथाभावं मन्यसे अहो
विमोहाद्यदिमामादातुमिदं कृतमेकपत्नीव्रतं तव १८४

यदि रावणहृतां सीतामानेतुं सुग्रीवसहायाय कृतमेवमेव हा महदन्तरं बलवृद्धेन महाबलेन वालिना
सद्भावेन दिनकरावर्तितान्तरे सीतामानेतुं समर्थेन स्मरणागतरावणदानमर्थेन वानरराजेन
पञ्चाशत्परार्द्धवानरभल्लूकसेनावता आत्मकार्येण सिद्ध्यत इति किं सुग्रीवेणाल्पवी-र्येण
सप्तपरार्द्धसेनापतिना कपिना किं सिद्ध्यति कार्यं वचनवतः १८५

अहो ज्ञातं सर्वदेव भद्रं यदुक्तोसि १८६

वक्ति च रामः पृथिवीपतिना मया दुष्टनिग्रहणं कार्यं शिष्टपरिपालनं च वालिना
सुग्रीवमहिषी रुमापहृता राज्यं च अतश्च न तादृग्वधे दोषः १८७

तारोवाच-

सुग्रीवोऽपि तर्हि वध्यो दुन्दुभिना युद्ध्यता वालिना बिलेप्रविष्टेन वत्सरं तत्रोषितं तदन्तरे च
मामपहृत्य राज्यं च कृतं सुग्रीवेण तं पूर्वमपि पश्चात्तं हन्तु १८८

राम उवाच-

कियत्कालात्पूर्वमिदं च वद १८९

तारोवाच-

षष्टिवर्षसहस्रादर्वाकशीतितमे वर्षे रक्षोयुद्धे सुग्रीवेण राज्यमपहृतम् १९०

पुनश्च वर्षान्तरे प्राप्तेन वालिना सुग्रीवः पलायितः १९१

अपहृता तस्य भार्या राज्यं चापहृतम् १९२

तस्मिन्नेव दिने भवतः पितुर्दशरथस्याभिषेकः १९३

राघव उवाच-

मया पितुरनुशासनाद्राज्यगतदुष्टनिग्रहणं कृतम्

गुरुवचनस्यानुल्लङ्घनीयत्वात्तदपहरणवेलायां यो राजासनाचरत् १९४

अथवा स्वतन्त्रौ मृगौ मृगयोर्हतश्च वाली मृगाणामन्योन्यं दारणाद्य जगुप्सा च १९५

यतो मम मृगयावदाथवा मृगाणाम्

चलितस्थितबद्धानां चलद्भ्रान्तपलायिनाम्

अथावसृजतासङ्गमुज्झिता मृगया तथा १९६

मृगयाशास्त्रविधितो मृगयेयं मया कृता

दर्शनादर्शनाभ्यां च धावताधावता तथा १९७

अवरोहात्परं स्थानं सात्विकानां प्रभिद्यते

राज्ञश्च मृगयाधर्मो विना आमिषभोजनम् १९८

अथ रामवचनमाकर्ण्य सर्व एव प्राकम्पयञ्छिरांसि १९९

वाली बभाषे राममञ्जलिमस्तके निधाय नमस्ते राम शृणु वचनं मम २००

शङ्खचक्रगदापाणिः पीतवासा जगद्गुरुः

नारायणः स्वयं साक्षाद्भवानिति मया श्रुतम् २०१

त्वां योगिनश्चिन्तयन्ति त्वां यजन्ति च यज्विनः

हव्यकव्यभुगे कस्त्वं पितृदेवस्वरूपधृक् २०२

मरणे चिन्तयानस्य त्वां विमुक्तिरदूरतः

सत्वं मे दर्शनं प्राप्तो राम मे पापसङ्क्षयः २०३

गृहाण बाणं काकुत्स्थ व्यथितो भृशमस्म्यहम् २०४

अथ रामस्तथेति बाणमादाय वालिनमुवाच किमिष्टं दीयतां वद २०५

कपिरुवाच-

यदि प्रसन्नो भगवान्मम सद्गतिं देहि २०६

अयं सुग्रीवस्तथा रक्षणीयोऽङ्गदोऽथ तारा च मया पापिनापराधः कृतस्तत्फलमनुभूतम् २०७

अथ रामं पश्यन्नेव वाली ममार स्वर्गं च गतः २०८

अथ सुग्रीवं राज्येऽभिषिच्य स्वयं वनं विवेश २०९

अथ तेन सहायेन जलधिसमीपं गत्वा क्व लङ्का क्व सीता क्व चारातिरिति सुग्रीवमाह रामः २१०

अथ हनुमानाह प्रविश्य लङ्कां विचित्य सीतां सर्वतत्त्वमवगत्य युद्धं सन्धिर्वा कर्तव्यः
तदुदधिलङ्घनाय किञ्चित्समादिशतु भगवान् २११

अथ सुग्रीवमाह रामः

कथमेतद्घटत इति २१२

कपिरुवाच-

मम वानरा भल्लूप्रमुखाः कोटिशः सन्ति २१३

एकं नियुज्य सर्वमाकलय्य यथायुक्तं तथा करणीयम् २१४

अथ जाम्बवानाह

हनूमानेको गच्छतु बुध्यतु लङ्काम् २१५

अथ हनूमानगमल्लङ्कापुरं विचित्य सीतामशोकवनिकायामासीनां तथा च सम्भाष्य विश्वासं कृत्वा वनं
बभञ्ज वनरक्षकांश्च २१६

बद्धो रक्षसा लङ्कां दग्ध्वा उत्तरकूलं गत्वा रामं दृष्ट्वा वृत्तान्तं कथयित्वा तूष्णीमतिष्ठत् २१७

अथ रामः सर्वैर्विचारयामास जाम्बवानुवाच रामेण लङ्का कपिभिर्विनश्यतीति नारदेन ममोक्तम्
 २१८

अथ सागरोत्तरणे यत्नतया स्थेयम् २१९

अथ रामः शङ्करमाराध्य सर्वं निवेदयित्वा त्वदुक्तं करोमीति वचनमुक्त्वा शिवमभ्यर्च्य प्रणतो
भूत्वा व्यजिज्ञपत् २२०

हे महादेव महाभूतग्रास महाप्रलयकारण महाहिभूषण महारुद्र शङ्कर परमेश्वर विरूपाक्ष
नागयज्ञोपवीतकरि कृत्तिवसन ब्रह्मशिरः कपालमालाभरभूषण –नरकास्थिभूषणभसित
परनारायणप्रिय शुभचरित पञ्चब्रह्मादिदेव पञ्चानन चतुर्वदन वेदवेद्य भक्तसुलभा भक्तदुर्लभ
परमानन्दविज्ञान पर पूषदन्तपातन दक्षशिरश्छेदन ब्रह्मपञ्चमशिरोहरण पार्वतीवल्लभ
नारदोपगीयमान शुभचरित शर्व त्रिनेत्र त्रिशूलधर पिनाकपाणे कपर्दिन्ननेकरूपधर वृषभवाहन
शुद्धस्फटिकसङ्काश चतुर्भुज नानायुध दक्षिणामूर्ते ईश्वर देवपते गङ्गाधर त्रिपुरहर श्रीशैलनिवास
काशीनाथ केदारेश्वर भूषणसिद्धेश्वर गोकर्णेश्वर कनखलेश्वर पर्वतेश्वर चक्रप्रद बाणचिन्तापादक
मुरहर पूजितचरणकमल सोमसोमभूषण सर्वज्ञ ज्योतिर्मय जगन्मय नमस्ते नमस्ते २२१

एवं स्तुवतो रामस्य पुरतो लिङ्गमध्यकोपेतस्तेजोमयमूर्तिराविर्बभूव २२२

अभयवानथ पुनः पद्मासनासीनमुमाधिष्ठिताङ्कमीशमामुक्तसर्वाभरणं सुकान्तिकिरीटिनं
हैमवतीकटिस्पर्शं करद्वयेनाभयवरप्रदं तरङ्गितानेकदिशाभिः

पूर्णतेजस्विनं हासमुखं प्रसन्नवदनं ददर्श रामः २२३

परमेशितारं ननाम बद्धाञ्जलिपुनश्च दण्डवत्पपात २२४

अथ रामं परमेश्वरोऽपि वरं वृणु त्वं वरदोऽहमित्युक्तवान् २२५

राम उवाच-

लङ्कां गमिष्यामि समुद्रतरणे उपायमेकं मम देहि शम्भो २२६

शम्भुरुवाच-

ममाजगवं धनुरस्ति तत्कालरूपमविकल्पं वा भवति

तदारुह्य समुद्रं तीर्त्वा लङ्कामाप्नुहि २२७

रामस्तथेति निश्चित्य सस्माराजगवम् २२८

आगतं धनुस्ततश्च रामोऽपूजयत् २२९

अथ हरो धनुरादाय रामाय दत्तवान् २३०

रामोऽपि जलधावपातयत् २३१

आरुरुहुः सर्वे वानरा रामलक्ष्मणौ च षष्टिपरार्द्धं तेषामसङ्ख्येषु वानरेषु धनुरारूढेषु निकामं ययौ
 २३२

धनुस्तटं वानराश्च ततस्ततो गत्वा निरीक्षयामासुः २३३

अथातिकायो नाम रक्षः कपिबलमालोक्य रावणायोक्तवान् २३४

रावणोऽपि किं कपिभिः शाखामृगैः किं वा मानुषाभ्यां रामलक्ष्मणाभ्यां किमायातं

दैवागतमसकं भोजनमित्युवाच २३५

अथ सुग्रीवः पश्चिमावलम्बिनि भास्वति हनूमज्जाम्बवदादिमहाबलैश्चातिकायैरसङ्ख्यातैर्लङ्कापार्श्वं
गत्वा उपवनं प्रवश्यि नानाफलानि खादित्वा पयः पीत्वोपवनरक्षिराक्षसान्विद्राव्य
सर्वविपिनमेकैकशो गृहीत्वा प्राद्र वन्लङ्कां गोपुरं च गत्वा समारुह्य प्रासादं च विशीर्यैकैकशः
केचित्स्तम्भमादाय रक्षोभिर्युयुधुः २३६

एके च शालां बभञ्जुर्गृहाणि चूर्णयामासुर्बालवृद्धस्त्रीजनादिकं सर्वमेव निजघ्नुः २३७

अथैकं प्राकारं निर्जितमाज्ञाय रावण इन्द्रजितं सन्दिदेश २३८

इन्द्रजिता च युद्धं वानराः कृत्वा भीताः पलायिताश्च २३९

अथ हनूमानखिलं निर्गतमाज्ञाय रावणं ज्ञात्वा वानरानाहूय निर्भर्त्स्य सेनां महतीं कारयित्वा
दशमुखं कल्पयित्वा मोदयामास २४०

अथ खस्थ एवेन्द्रजिद्युयुधे न च वानरास्तं दृष्टवन्तः २४१

अथ हनूमज्जाम्बवन्तौ खमुत्पत्य पर्वतशिखराभ्यामिन्द्रजितं निजघ्नतुः २४२

अथ भुविपपात तं लक्ष्मणश्च यमलोकगामिनं चकार २४३

अथातिकायमहाकायौ वानरसैन्यं बहुशो हत्वा लक्ष्मणं पीडयित्वा रामेण संयुध्य सुग्रीवं कृत्वा
हनूमज्जाम्बवद्भ्यां युयुधाते पराजितौ गृहीत्वा तौ च योद्धारावादाय रामसमीपं गत्वा रामाय
न्यवेदयताम् २४४

अतिकायमभाषत रामः २४५

रावणस्य मम युद्धं ब्रूहि सचिवानामन्येषां महाभयानां च २४६

अतिकाय उवाच-

निश्चितमिदं पुरास्माभिः कार्यं सेनां विभागशः कृत्वा विद्युन्माली नाम राक्षसो महाबलो
विचित्रयोधी दर्शनादर्शनयोधी वानरैः सर्वैरेक एव युध्यते २४७

अपरे च बलिनो महान्तः शिक्षितास्त्राश्चागता आवां च युवाभ्यां युध्यावो रावणः
पुष्पकमारुह्यापरभागेन त्वामेव निहनिष्यति २४८

अन्ये च राक्षसाः कुम्भकर्णमुखाश्चात्मरूपं कृत्वा त्वां परिवार्य गृहीत्वा सीतायै दर्शयित्वा
तत्सन्निधावेव हनिष्यति २४९

रामः प्राह अहो बलवतां किमसाध्यमेवं भवति दैवगतिः कुटिला सुग्रीवोऽतिकोपनः सक्रोधं
दृष्ट्वा राममुवाच २५०

वध्यावेतौ न मोचनीयौ २५१

रामः प्राहावध्यौ मोचनीयावेतौ वसनानि भूषणान्यानयेत्युक्तमात्रे हनूमता तान्यानीतानि
रामस्ताभ्यां दत्तवान् २५२

नत्वा यदेतल्लङ्काद्वारे दृश्यते दारुपञ्चवक्त्रं शुक्रेणोक्तमेतेन च्छिन्नेन रावणो हन्यते २५३

अथ च दारुच्छेदनसमनन्तरं पातालं गन्तव्यमिति भार्गवभाषितं शासनं लिखितम् २५४

तस्मात्त्वमिदं दार्वेकप्रयत्नैकबाणनिपातेन पञ्चधाच्छिन्धि ततस्तव शक्तिं ज्ञात्वा युद्धमतिदृढं
कुर्वीमहि २५५

अथ भार्गववचोविज्ञाय रामः पूर्वकोट्यां स्पर्शमात्रेण सज्यं कृत्वा धनुषि बाणं संयोज्य
रक्षोभ्यां हनूमताश्रावयन्नेव बाणं मुमोच २५६

बाणं धनुषश्चलितं तौ राक्षसौ बाणमार्गं निरीक्षमाणौ दारु बाणेन पञ्चधा च्छिन्नं निरीक्ष्य
रामं व्यज्ञापयतामावयोः शिशवो रक्षणीयास्त्वयेति तथेत्याह रामः राक्षसौ लङ्कां प्रविष्टौ
 २५७

अथ प्राकारयुद्धं कर्तुं वानरा गत्वा सर्वतो वरणमात्रं हि पार्ष्णिभिः पादैर्जानुभिः करैः
पृष्ठैश्च तलसमं कृत्वा द्वितीयप्राकारं गतास्तदा च रावणः समागत्य सर्वानेवेषुभिर्द्रावयित्वा
तदनुगच्छन्राममगात् २५८

अथ राममपि पञ्चभिर्बाणैर्विव्याध २५९

अथ रामो दशभिर्बाणै रावणं सव्रणं चकार २६०

अनयोरतिदारुणमन्योन्यं युद्धं बभूव २६१

रावणो दशभिर्बाणैर्विव्याध २६२

अथ रामबाणैश्च क्षतशरीरो राक्षसः पलायनपरोऽभवत् २६३

वानरा लक्ष्मणश्च कोटिकोटिराक्षसानघ्नन् २६४

अथपरस्मिन्नहनि विभीषणो रावणं विचार्येदमुवाच २६५

तृतीयोपायकालोऽयं चतुर्थं न विचारय

चतुर्थो विपरीतो न शस्तः शस्तार्थकारिणः २६६

परस्य चाऽत्मनः शक्तिं विदित्वा चाऽत्मनोऽधिकाम्

तदा युद्धं प्रशस्तं स्याद्विपरीतं विनाशकम् २६७

रामेण बलिनानैव युद्धं ते दुर्बलस्य च

एकेषुवालिहन्ताऽसौ वालिर्ज्ञातस्त्वया पुरा २६८

मारीचमेकबाणेन भवानपि पलायितः

निहता राक्षसाः शूरा इन्द्र जिच्च सुतो हतः २६९

वरेण्यत्रितयं भग्नं तेन युद्धं च नैव ते

दासभावमथो वाऽपि दत्त्वा सीतामथाऽप्नुहि २७०

गोपुरस्थं तथा दारु पञ्चवक्त्रमथेषुणा

चिच्छेद पञ्चधा तेन रामस्त्वां मारयिष्यति २७१

त्वदर्थं बहवो नष्टा नाशमेष्यन्ति चापरे

एको न्यायः सुखार्थाय न च मौढ्यं सहोदर २७२

मानुषीं मृत्युसंयुक्तामनिच्छन्तीं पतिव्रताम्

पत्नीं बलवतश्चापि पूजयित्वा विसर्जय २७३

अनिच्छन्त्याः समायोगे भवेद्दुःखपरम्परा

दुर्गन्धमलसंयुक्तो नारीसङ्गो जुगुप्सितः २७४

विरक्तिरथ चेज्जाता दुःखायाकार्यवर्तनम्

अनुरागोयदि भवेन्मरणं नरकं ततः २७५

आत्मनो मरणं व्यर्थं तस्याश्चाद्य समागमे

त्यागो वा मरणं तात धर्मपत्न्यास्तथा भवेत् २७६

एवमादि तथाऽन्यच्च कश्मलं सम्भविष्यति

अन्यदाख्यामि ते वाक्यं सर्वेषां च प्रियं हितम् २७७

गत्वा रामान्तिकं नत्वा स्तुत्वा विज्ञाप्य राघवम्

क्षम राम महावीर शरणागतवत्सल २७८

तामसा राक्षसाः सर्वे वयमेते सुपापिनः

सीतापहारजं दोषं त्यक्त्वा पुत्रानवेहि नः २७९

त्वदधीना वयं राम रक्ष वा मारयेच्छया

इत्युदीर्य पुरस्तस्य राघवस्य स्थिता वयम् २८०

स्थिरायुषो भविष्यामः स्थिरराज्या दशानन

अथाऽहं रावणो वाक्यमहो नो राक्षसो भवान् २८१

न शूरो राजधर्मं च न च जानासि शाश्वतम्

परनारीपरद्र व्यपरराज्यनिषेवया २८२

शूराणामुत्तमो धर्मो न षण्ढानां भवादृशाम्

शत्रुपक्षं समालिग्य निर्गच्छेच्छा हि चेन्नृप २८३

अथ बिभीषणो मन्दिरं गत्वा रामान्तिकं गत्वा तं शरणमभजत् २८४

अथ रावणः पुरान्निर्गत्य रामेण लक्ष्मणवानरै राक्षसा अपि युयुधिरे २८५

अथ रावणं महाबलं हन्तुमशक्तोरामो विभीषणमुखमवलोक्य तदुक्तचिह्नपदं बाणेन निर्भिद्यामारयत्
 २८६

अथ कुम्भकर्णो महागदामादाय सर्वं निष्पाद्य वानराननेकशो भक्षयित्वा रामोत्तमाङ्गं
गदयाऽहन् २८७

अथ रामो निशितबाणशतेन तमहन्ममार कुम्भकर्णः २८८

अथ विभीषणेन रावणादेः श्राद्धादिकं कारयित्वा शिवालयं तन्नाम्ना कारयित्वा तमेव
लङ्काराज्ये विभीषणमभिषिच्य सीतामग्निप्रवेशशुद्धामुमामहेश्वराभ्यां नमयित्वा पुरहरेण
दत्ताखिलामृतबलायुष्यः सुपुष्पकमारुह्य जलधिमुत्तीर्य पारावारतटे सेनां समवस्थाप्य
शिवप्रतिष्ठां तत्र कृत्वा मुनिभिर्देवैरभ्यर्चितोऽयोध्यामगमत् २८९

अथ भरतादिसमुपेतो नागरैर्वसिष्ठेन मुनिभिश्चाभ्यर्चितःस्वगृहमगमत् २९०

आत्मनाऽगतानिन्द्रा दिदेवानासनादिनाऽभ्यर्च्य वानरान्सम्पूज्य मुक्तजटोऽभिषिक्तो राज्ये
रावणवधहर्षिता देवा राममूचुः २९१

त्वयाऽत्मराज्ये स्थापिता वयं नः सर्वदा परिपालय त्वमादिनारायणो देवो
निखिलदुष्टनिग्रहार्थमवतीर्णो रावणं सबान्धवं हत्वा लोकत्रयरक्षकोऽसि श्रिया सह सुखी
भवेत्युदीर्य स्वर्गं गताः २९२

अथायोध्यावासिनो रामं प्रहर्षिता ऊचुः २९३

हत्वा शत्रून्समायातो दृष्ट्वा प्राप्तोऽसि वै शिवम्

दिष्ट्या त्वं राजसे राम दिष्ट्या पालयसे प्रजाः २९४

त्वया यज्ञाः करिष्यन्ते त्वया धर्मो विवर्धते

इतिपौरवचःश्रुत्वारामो राजीवलोचनः २९५

वस्त्रादिभिरथोसर्वान्नागरान्समपूजयत्

मुनीनुवाचधर्म्मात्मापूजयित्वाखिलैर्जनैः २९६

कच्चित्तपःसमृद्धंवःकच्चिद्यज्ञःस्वनुष्ठितः

कच्चित्स्वदारनिरताःकच्चिदीशोभिपूज्यते २९७

कच्चित्सुप्रजसोभार्याःकच्चित्सर्वंसुखोत्तरम् २९८

मुनय ऊचुः -

त्वयि राजनि काकुत्स्थ सर्वं स्वस्थं तपस्विनाम्

गच्छामहे पदमितः किं वा त्वं मन्यसे नृप २९९

राम उवाच-

यस्य विप्राः प्रसीदन्ति तस्य शम्भुः प्रसीदति

यस्य प्रसीदतीशानस्तस्य भद्रं भविष्यति ३००

तत्कृत्वा भोजनमिह गन्तुमर्हा अनन्तरम्

अथेत्युक्त्वा मुनिगणाः कृत्वा भोजनमुत्तमम् ३०१

अभिवर्ध्यतमाशीर्भिर्हृष्टाः स्वं स्वं पदं ययुः

रामोऽपि परमप्रीतः सभार्यश्च सहानुजः ३०२

अकण्टकं स कृतवान्राज्यं सर्वजनप्रियः

शृणोत्येतदुपाख्यानं यः कश्चिदपि पातकी ३०३

सर्वपापविनिर्मुक्तः परं ब्रह्माधिगच्छति

न दुर्गतिर्भवेत्तस्य यश्चेदं स्मरते नरः ३०४

यश्चापि कीर्तयेत्तस्य एवमेतदुदीरितम् ३०५

इति श्रीपद्मपुराणे पातालखण्डे शिवराघवसंवादे पुराकल्पीयरामायणकथनं नाम षोडशोत्तरशततमोऽध्यायः॥११६॥


===
