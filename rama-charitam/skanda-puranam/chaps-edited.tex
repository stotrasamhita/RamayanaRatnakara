\sect{द्व्युत्तरशततमोऽध्यायः --- लक्ष्मणादिप्रासादपञ्चकनिर्माणप्रतिष्ठापनवर्णनम्}

\src{स्कन्दपुराणम्}{खण्डः ६ (नागरखण्डः)}{अध्यायः ०९९}
\vakta{}
\shrota{}
\tags{}
\notes{}
\textlink{https://sa.wikisource.org/wiki/स्कन्दपुराणम्/खण्डः_६_(नागरखण्डः)/अध्यायः_०९९}
\translink{https://www.wisdomlib.org/hinduism/book/the-skanda-purana/d/doc493463.html}

\storymeta




\uvacha{ऋषय ऊचुः}

\twolineshloka
{यदेतद्भवता प्रोक्तं तत्र रामेण निर्मितः}
{रामेश्वरस्तथा सीता तेन तत्र विनिर्मिता}%॥ १ ॥

\twolineshloka
{तथा च लक्ष्मणार्थाय निर्मितस्तेन संश्रयः}
{एतन्महद्विरुद्धं ते प्रतिभाति वचोऽखिलम्}%॥ २ ॥

\twolineshloka
{त्वया सूत पुरा प्रोक्तं रामो लक्ष्मणसंयुतः}
{सीतया सहितः प्राप्तः क्षेत्रेऽत्र प्रस्थितो वने}%॥ ३ ॥

\twolineshloka
{श्राद्धं कृत्वा गयाशीर्षे लक्ष्मणेन विरुद्ध्य च}
{पुनः सम्प्रस्थितोऽरण्यं क्रोधाविष्टश्च तं प्रति}%॥ ४ ॥

\twolineshloka
{यत्त्वयोक्तं तदा तेन निर्मितोऽत्र महेश्वरः}
{एतच्च सर्वमाचक्ष्व सन्देहं सूतनन्दन}%॥ ५ ॥

\uvacha{सूत उवाच}


\threelineshloka
{अत्र मे नास्ति सन्देहो युष्माकं च पुनः स्थितः}
{ततो वक्ष्याम्यशेषेण श्रूयतां द्विजसत्तमाः}
{एतत्क्षेत्रं पुनश्चाद्यं न क्षयं याति कुत्रचित्}%॥ ६ ॥

\twolineshloka
{अन्यस्मिन्दिवसे प्राप्ते स तदा रघुनन्दनः}
{यदा विरोधमापन्नः सार्धं सौमित्रिणा सह}%॥ ७ ॥

\twolineshloka
{एतत्पुनर्दिनं चान्यद्यत्र तेन प्रतिष्ठितः}
{रामेश्वरः स्वयं भक्त्या दुःखितेन महात्मना}%॥ ८ ॥

\uvacha{ऋषय ऊचुः}

\twolineshloka
{अन्यस्मिन्दिवसे तत्र कस्मिन्काले रघूत्तमः}
{सम्प्राप्तस्तस्य किं दुःखं सञ्जातं तत्प्रकीर्तय}%॥ ९ ॥

\uvacha{सूत उवाच}

\twolineshloka
{कृत्वा सीतापरित्यागं रामो राजीवलोचनः}
{लोकापवादसन्त्रस्तस्ततो राज्यं चकार सः}%॥ १० ॥

\twolineshloka
{कृत्वा स्वर्णमयीं सीतां पत्नीं यज्ञप्रसिद्धये}
{न स चक्रे महाभागो भार्यामन्यां कथञ्चन}%॥ ११ ॥

\twolineshloka
{दशवर्षसहस्राणि दशवर्षशतानि च}
{ब्रह्मचर्येण चक्रे स राज्यं निहतकण्टकम्}%॥ १२ ॥

\twolineshloka
{ततो वर्षसहस्रान्ते प्राप्ते चैकादशे द्विजाः}
{देवदूतः समायातो रामस्य सदनं प्रति}%॥ १३ ॥

\twolineshloka
{तेनोक्तं देवराजेन प्रेषितोऽहं तवान्तिकम्}
{तस्मात्कुरु समालोकं विजने त्वं मया सह}%॥ १४ ॥

\twolineshloka
{एवमुक्तस्तदा तेन दूतेन रघुनन्दनः}
{परं रहः समासाद्य मन्त्रं चक्रे ततः परम्}%॥ १५ ॥

\twolineshloka
{तस्यैवमुपविष्टस्य मन्त्रस्थाने महात्मनः}
{बहुत्वादिष्टलोकस्य न रहस्यं प्रजायते}%॥ १६ ॥

\twolineshloka
{ततः कोपपरीतात्मा दूतः प्रोवाच सादरम्}
{विहस्य जनसंसर्गं दृष्ट्वैकान्तेऽपि संस्थिते}%॥ १७ ॥

\twolineshloka
{यथा दंष्ट्राच्युतः सर्पो नागो वा मदवर्जितः}
{आज्ञाहीनस्तथा राजा मानवैः परिभूयते}%॥ १८ ॥

\twolineshloka
{सेयं तव रघुश्रेष्ठ नाज्ञास्ति प्रतिवेद्म्यहम्}
{शक्रालापमपि त्वं च नैकान्ते श्रोतुमर्हसि}%॥ १९ ॥

\twolineshloka
{तस्य तद्वचनं श्रुत्वा कोपसंरक्तलोचनः}
{त्रिशाखां भृकुटीं कृत्वा ततः स प्राह लक्ष्मणम्}%॥ २० ॥

\threelineshloka
{ममात्र सन्निविष्टस्य सहानेन प्रजल्पतः}
{यदि कश्चिन्नरो मोहादागमिष्यति लक्ष्मण}
{स्वहस्तेन न सन्देहः सूदयिष्यामि तं द्रुतम्}%॥ २१ ॥

\twolineshloka
{न हन्मि यदि तं प्राप्तमत्र मे दृष्टिगोचरम्}
{तन्मा भून्मे गतिः श्रेष्ठा धर्मिणां या प्रपद्यते}%॥ २२ ॥

\twolineshloka
{एवं ज्ञात्वा प्रयत्नेन त्वया भाव्यमसंशयम्}
{राजद्वारि यथा कश्चिन्न मया वध्यतेऽधुना}%॥ २३ ॥

\twolineshloka
{तमोमित्येव सम्प्रोच्य लक्ष्मणः शुभलक्षणः}
{राजद्वारं समासाद्य चकार विजनं ततः}%॥ २४ ॥

\twolineshloka
{देवदूतोऽपि रामेण समं चक्रे ततः परम्}
{मन्त्रं शक्रसमादिष्टं तथान्यैः स्वर्गवासिभिः}%॥ २५ ॥

\uvacha{देवदूत उवाच}

\twolineshloka
{त्वं रावणविनाशार्थमवतीर्णो धरातले}
{स च व्यापादितो दुष्टः पापस्त्रैलोक्यकण्टकः}%॥ २६ ॥

\twolineshloka
{कृतं सर्वं महाभाग देव कृत्यं त्वयाऽधुना}
{तस्मात्सन्तु सनाथास्ते देवाः शक्रपुरोगमाः}%॥ २७ ॥

\threelineshloka
{यदि ते रोचते चित्ते नोपरोधेन साम्प्रतम्}
{प्रसादं कुरु देवानां तस्मादागच्छ सत्वरम्}
{स्वर्गलोकं परित्यज्य मर्त्यलोकं सुनिन्दितम्}%॥ २८ ॥

\uvacha{सूत उवाच}

\twolineshloka
{एतस्मिन्नन्तरे प्राप्तो दुर्वासा मुनिसत्तमः}
{प्रोवाचाथ क्षुधाविष्टः क्वासौ क्वासौ रघूत्तमः}%॥ २९ ॥

\uvacha{लक्ष्मण उवाच}

\twolineshloka
{व्यग्रः स पार्थिवश्रेष्ठो देवकार्येण केनचित्}
{तस्मादत्रैव विप्रेन्द्र मुहूर्तं परिपालय}%॥ ३० ॥

\twolineshloka
{यावत्सान्त्वयते रामो दूतं शक्रसमुद्भवम्}
{ममोपरि दयां कृत्वा विनयावनतस्य हि}%॥ ३१ ॥

\uvacha{दुर्वासा उवाच}

\twolineshloka
{यदि यास्यति नो दृष्टिं मम द्राक्स रघूत्तमः}
{शापं दत्त्वा कुलं सर्वं तद्धक्ष्यामि न संशयः}%॥ ३२ ॥

\twolineshloka
{ममापि दर्शनादन्यन्न किञ्चिद्विद्यते गुरु}
{कृत्यं लक्ष्मण यावत्त्वमन्यन्मूढ़ प्रकत्थसे}%॥ ३३ ॥

\twolineshloka
{तच्छ्रुत्वा लक्ष्मणश्चित्ते चिन्तयामास दुःखितः}
{वरं मे मृत्युरेकस्य मा भूयात्कुलसङ्क्षयः}%॥ ३४ ॥

\twolineshloka
{एवं स निश्चयं कृत्वा ततो राममुपाद्रवत्}
{उवाच दण्डवद्भूमौ प्रणिपत्य कृताञ्जलिः}%॥ ३५ ॥

\twolineshloka
{दुर्वासा मुनिशार्दूलो देव ते द्वारि तिष्ठति}
{दर्शनार्थी क्षुधाविष्टः किं करोमि प्रशाधि माम्}%॥ ३६ ॥

\threelineshloka
{तस्य तद्वचनं श्रुत्वा ततो दूतमुवाच तम्}
{गत्वेमं ब्रूहि देवेशं मम वाक्यादसंशयम्}
{अहं संवत्सरस्यान्त आगमिष्यामि तेंऽतिके}%॥ ३७ ॥

\twolineshloka
{एवमुक्त्वा विसृज्याथ तं दूतं प्राह लक्ष्मणम्}
{प्रवेशय द्रुतं वत्स तं त्वं दुर्वाससं मुनिम्}%॥ ३८ ॥

\twolineshloka
{ततश्चार्घ्यं च पाद्यं च गृहीत्वा सम्मुखो ययौ}
{रामदेवः प्रहृष्टात्मा सचिवैः परिवारितः}%॥ ३९ ॥

\twolineshloka
{दत्त्वार्घ्यं विधिवत्तस्य प्रणिपत्य मुहुर्मुहुः}
{प्रोवाच रामदेवोऽथ हर्षगद्गदया गिरा}%॥ ४० ॥

\twolineshloka
{स्वागतं ते मुनिश्रेष्ठ भूयः सुस्वागतं च ते}
{एतद्राज्यममी पुत्रा विभवश्च तव प्रभो}%॥ ४१ ॥

\threelineshloka
{कृत्वा मम प्रसादं च गृहाण मुनिसत्तम}
{धन्योऽस्म्यनुगृहीतोऽस्मि यत्त्वं मे गृहमागतः}
{पूज्यो लोकत्रयस्यापि निःशेषतपसान्निधिः}%॥ ४२ ॥

\uvacha{मुनिरुवाच}

\twolineshloka
{चातुर्मास्यव्रतं कृत्वा निराहारो रघूत्तम}
{अद्य ते भवनं प्राप्य आहारार्थं बुभुक्षितः}%॥ ४३ ॥

\twolineshloka
{तस्मात्त्वं यच्छ मे शीघ्रं भोजनं रघुनन्दन}
{नान्येन कारणं किञ्चित्सन्न्यस्तस्य धनादिना}%॥ ४४ ॥

\twolineshloka
{ततस्तं भोजयामास श्रद्धापूतेन चेतसा}
{स्वयमेवाग्रतः स्थित्वा मृष्टान्नैर्विविधैः शुभैः}%॥ ४५ ॥

\twolineshloka
{लेह्यैश्चोष्यैस्तथा चर्व्यैः खाद्यैरेव पृथग्विधैः}
{यावदिच्छा मुनेस्तस्य तथान्नैर्विविधैरपि}%॥ ४६ ॥
॥इति श्रीस्कान्दे महापुराण एकाशीतिसाहस्र्यां संहितायां षष्ठे नागरखण्डे हाटकेश्वरक्षेत्रमाहात्म्ये रामेश्वरस्थापनप्रस्तावे श्रीरामम्प्रति दुर्वासः समागमनवृत्तान्तवर्णनं नामैकोनशततमोऽध्यायः॥९९॥

\uvacha{सूत उवाच}

\twolineshloka
{एवं भुक्त्वा स विप्रर्षिर्वाञ्छया राममन्दिरे}
{दत्ताशीर्निर्गतः पश्चादामन्त्र्य रघुनन्दनम्}%॥ १ ॥

\twolineshloka
{अथ याते मुनौ तस्मिन्दुर्वाससि तदन्तिकात्}
{लक्ष्मणः खङ्गमादाय रामदेवमुवाच ह}%॥ २ ॥

\twolineshloka
{एतत्खङ्गं गृहीत्वाशु मां प्रभो विनिपातय}
{येन ते स्यादृतं वाक्यं प्रतिज्ञातं च यत्पुरा}%॥ ३ ॥

\twolineshloka
{ततो रामश्चिरात्स्मृत्वा तां प्रतिज्ञां स्वयं कृताम्}
{वधार्थं सम्प्रविष्टस्य समीपे पुरुषस्य च}%॥ ४ ॥

\twolineshloka
{ततोऽतिचिन्तयामास व्याकुलेनान्तरात्मना}
{बाष्पव्याकुलनेत्रश्च निःष्वसन्पन्नगो यथा}%॥ ५ ॥

\twolineshloka
{तं दीनवदनं दृष्ट्वा निःष्वसन्तं मुहुर्मुहुः}
{भूयः प्रोवाच सौमित्रिर्विनयावनतः स्थितः}%॥ ६ ॥

\twolineshloka
{एष एव परो धर्मो भूपतीनां विशेषतः}
{यथात्मीयं वचस्तथ्यं क्रियते निर्विकल्पितम्}%॥ ७ ॥

\twolineshloka
{तस्मात्त्वया प्रभो प्रोक्तं स्वयमेव ममाग्रतः}
{तस्यैव देवदूतस्य तारनादेन कोपतः}%॥ ९ ॥

\twolineshloka
{योऽत्रागच्छति सौमित्रे मम दूतस्य सन्निधौ}
{तं चेद्धन्मि स्वहस्तेन नाहं तस्मात्सुपापकृत्}%॥ ९ ॥

\twolineshloka
{तदहं चागतस्तात भयाद्दुर्वाससो मुनेः}
{निषिद्धोऽपि त्वयातीव तस्माच्छीघ्रं तु घातय}%॥ १० ॥

\twolineshloka
{ततः सम्मन्त्र्य सुचिरं मन्त्रिभिः सहितो नृपः}
{ब्राह्मणैर्धर्मशास्त्रज्ञैस्तथान्यैर्वेदपारगैः}%॥ ११ ॥

\twolineshloka
{प्रोवाच लक्ष्मणं पश्चाद्विनयावनतं स्थितम्}
{वाष्पक्लिन्नमुखो रामो गद्गदं निःश्वसन्मुहुः}%॥ १२ ॥

\twolineshloka
{व्रज लक्ष्मण मुक्तस्त्वं मया देशातरं द्रुतम्}
{त्यागो वाथ वधो वाथ साधूनामुभयं समम्}%॥ १३ ॥

\twolineshloka
{न मया दर्शनं भूयस्तव कार्यं कथञ्चन}
{न स्थातव्यं च देशेऽपि यदि मे वाञ्छसि प्रियम्}%॥ १४ ॥

\twolineshloka
{तस्य तद्वचनं श्रुत्वा प्रणिपत्य ततः परम्}
{निर्ययौ नगरात्तस्मात्तत्क्षणादेव लक्ष्मणः}%॥ १५ ॥

\twolineshloka
{अकृत्वापि समालापं केनचिन्निजमन्दिरे}
{मात्रा वा भार्यया वाथ सुतेन सुहृदाथवा}%॥ १६ ॥

\twolineshloka
{ततोऽसौ सरयूं गत्वाऽवगाह्याथ च तज्जलम्}
{शुचिर्भूत्वा निविष्टोथ तत्तीरे विजने शुभे}%॥ १७ ॥

\twolineshloka
{पद्मासनं विधायाथ न्यस्यात्मानं तथात्मनि}
{ब्रह्मद्वारेण तं पश्चात्तेजोरूपं व्यसर्जयत्}%॥ १८ ॥

\twolineshloka
{अथ तद्राघवो दृष्ट्वा महत्तेजो वियद्गतम्}
{विस्मयेन समायुक्तोऽचिन्तयत्किमिदं ततः}%॥ १९ ॥

\twolineshloka
{अथ मर्त्ये परित्यक्ते तेजसा तेन तत्क्षणात्}
{वैष्णवेन तुरीयेण भागेन द्विजसत्तमाः}%॥ २० ॥

\twolineshloka
{पपात भूतले कायं काष्ठलोष्टोपमं द्रुतम्}
{लक्ष्मणस्य गतश्रीकं सरय्वाः पुलिने शुभे}%॥ २१ ॥

\twolineshloka
{ततस्तु राघवः श्रुत्वा लक्ष्मणं गतजीवितम्}
{पतितं सरितस्तीरे विललाप सुदुःखितः}%॥ २२ ॥

\twolineshloka
{स्वयं गत्वा तमुद्देशं सामात्यः ससुहृज्जनः}
{लक्ष्मणं पतितं दृष्ट्वा करुणं पर्यदेवयत्}%॥ २३ ॥

\twolineshloka
{हा वत्स मां परित्यज्य किं त्वं सम्प्रस्थितो दिवम्}
{प्राणेष्टं भ्रातरं श्रेष्ठं सदा तव मते स्थितम्}%॥ २४ ॥

\twolineshloka
{तस्मिन्नपि महारण्ये गच्छमानः पुरादहम्}
{अपि सन्धार्यमाणेन अनुयातस्त्वया तदा}%॥ २५ ॥

\twolineshloka
{सम्प्राप्तेऽपि कबन्धाख्ये राक्षसे बलवत्तरे}
{त्वया रात्रिमुखे घोरे सभार्योऽहं प्ररक्षितः}%॥ २६ ॥

\twolineshloka
{येनेन्द्रजिद्धतो युद्धे तादृग्रूपो निशाचरः}
{स एष पतितः शेते गतासुर्धरणीतले}%॥ ९७ ॥

\twolineshloka
{येन शूर्पणखा ध्वस्ता राक्षसी सा च दारुणा}
{लीलयापि ममादेशात्सोयमेवंविधः स्थितः}%॥ २८ ॥

\twolineshloka
{यद्बाहुबलमाश्रित्य मया ध्वस्ता निशाचराः}
{सोऽयं निपतितः शेते मम भ्राता ह्यनाथवत्}%॥ २९ ॥

\twolineshloka
{हा वत्स क्व गतो मां त्वं विमुच्य भ्रातरं निजम्}
{ज्येष्ठं प्राणसमं किं ते स्नेहोऽद्य विगतः क्वचित्}%॥ ३० ॥

\uvacha{सूत उवाच}

\twolineshloka
{एवं बहुविधान्कृत्वा प्रलापान्रघुनन्दनः}
{मातृभिः सहितो दीनः शोकेन महतान्वितः}%॥ ३१ ॥

\twolineshloka
{ततस्ते मन्त्रिणस्तस्य प्रोचुस्तं वीक्ष्य दुःखितम्}
{विलपन्तं रघुश्रेष्ठं स्त्रीजनेन समन्वितम्}%॥ ३२ ॥

\uvacha{मन्त्रिण ऊचुः}

\twolineshloka
{मा शोकं कुरु राजेन्द्र यथान्यः प्राकृतः स्थितः}
{कुरुष्व च यथेदं स्यात्साम्प्रतं चौर्ध्वदैहिकम्}%॥ ३३ ॥

\twolineshloka
{नष्टं मृतमतीतं च ये शोचन्ति कुबुद्धयः}
{धीराणां तु पुरा राजन्नष्टं नष्टं मृतं मृतम्}%॥ ३४ ॥

\twolineshloka
{एवं ते मन्त्रिणः प्रोच्य ततस्तस्य कलेवरम्}
{लक्ष्मणस्य विलप्यौच्चैश्चन्दनोशीरकुङ्कुमैः}%॥ ३५ ॥

\twolineshloka
{कर्पूरागुरुमिश्रैश्च तथान्यैः सुसुगन्धिभिः}
{परिवेष्ट्य शुभैर्वस्त्रैः पुष्पैः सम्भूष्य शोभनैः}%॥ ३६ ॥

\twolineshloka
{चन्दनागुरुकाष्ठैश्च चितिं कृत्वा सुविस्तराम्}
{न्यदधुस्तस्य तद्गात्रं तत्र दक्षिणदिङ्मुखम्}%॥ ३७ ॥

\twolineshloka
{एतस्मिन्नन्तरे जातं तत्राश्चर्यं द्विजोत्तमाः}
{तन्मे निगदतः सर्वं शृण्वन्तु सकलं द्विजाः}%॥ ३८ ॥

\twolineshloka
{यावत्तेंऽतः समारोप्य चितां तस्य कलेवरम्}
{प्रयच्छन्ति हविर्वाहं तावन्नष्टं कलेवरम्}%॥ ३९ ॥

\twolineshloka
{एतस्मिन्नन्तरे वाणी निर्गता गगनाङ्गणात्}
{नादयन्ती दिशः सर्वाः पुष्पवर्षादनन्तरम्}%॥ ४० ॥

\twolineshloka
{रामराम महाबाहो मा त्वं शोकपरो भव}
{न चास्य युज्यते वह्निर्दातुं चैव कथञ्चन}%॥ ४१ ॥

\twolineshloka
{ब्रह्मज्ञानप्रयुक्तस्य सन्न्यस्तस्य विशेषतः}
{अग्निदानं न युक्तं स्यात्सर्वेषामपि योगिनाम्}%॥ ४२ ॥

\twolineshloka
{तवायं बान्धवो राम ब्रह्मणः सदनं गतः}
{ब्रह्मद्वारेण चात्मानं निष्क्रम्य सुमहायशाः}%॥ ४३ ॥

\threelineshloka
{अथ ते मन्त्रिणः प्रोचुस्तच्छ्रुत्वाऽऽकाशगं वचः}
{अशोच्यो यं महाराज संसिद्धिं परमां गतः}
{लक्ष्मणो गम्यतां शीघ्रं तस्मात्स्वभवने विभो}%॥ ४४ ॥

\twolineshloka
{चिन्त्यन्तां राजकार्याणि तथा यच्चौर्ध्वदैहिकम्}
{कुरु स्नेहोचितं तस्य पृष्ट्वा ब्राह्मणसत्तमान्}%॥ ४५ ॥

\uvacha{राम उवाच}

\twolineshloka
{नाहं गृहं गमिष्यामि लक्ष्मणेन विनाऽधुना}
{प्राणानत्र विहास्यामि यथा तेन महात्मना}%॥ ४६ ॥

\twolineshloka
{एष पुत्रो मया दत्तः कुशाख्यो मम सम्मतः}
{युष्मभ्यं क्रियतां राज्ये मदीये यदि रोचते}%॥ ४७ ॥

\twolineshloka
{एवमुक्त्वा ततो रामो गन्तुकामो दिवालयम्}
{चिन्तयामास भूयोऽपि स्मृत्वा मित्रं विभीषणम्}%॥ ४८ ॥

\twolineshloka
{मया तस्य तदा दत्तं लङ्कायां राज्यमक्षयम्}
{बहुभक्तिप्रतुष्टेन यावच्चन्द्रार्कतारकाः}%॥ ४९ ॥

\twolineshloka
{अतिक्रूरतरा जाती राक्षसानां यतः स्मृता}
{विशेषाद्वरपुष्टानां जायतेऽत्र धरातले}%॥ ५० ॥

\twolineshloka
{तच्चेद्राक्षसभावेन स महात्मा विभीषणः}
{करिष्यति सुरैः सार्धं विरोधं रावणो यथा}%॥ ५१ ॥

\twolineshloka
{तं देवाः सूदयिष्यन्ति उपायैः सामपूर्वकैः}
{त्रैलोक्यकण्टको यद्वत्तस्य भ्राता दशाननः}%॥ ५२ ॥

\twolineshloka
{ततो मे स्यान्मृषा वाणी तस्माद्गत्वा तदन्तिकम्}
{शिक्षां ददामि तस्याहं यथा देवान्न दूषयेत्}%॥ ५३ ॥

\twolineshloka
{तथा मे परमं मित्रं द्वितीयं वानरः स्थितः}
{सुग्रीवाख्यो महाभागो जाम्बवांश्च तथाऽपरः}%॥ ५४ ॥

\twolineshloka
{सभृत्यो वायुपुत्रश्च वालिपुत्रसमन्वितः}
{कुमुदाख्यश्च तारश्च तथान्येऽपि च वानराः}%॥ ५५ ॥

\twolineshloka
{तस्मात्तानपि सम्भाष्य सर्वान्सम्मन्त्र्य सादरम्}
{ततो गच्छामि देवानां कृतकृत्यो गृहं प्रति}%॥ ५६ ॥

\twolineshloka
{एवं सञ्चिन्त्य सुचिरं समाहूय च पुष्पकम्}
{तत्रारुह्य ययौ तूर्णं किष्किन्धाख्यां पुरीं प्रति}%॥ ५७ ॥

\twolineshloka
{अथ ते वानरा दृष्ट्वा प्रोद्द्योतं पुष्पकोद्भवम्}
{विज्ञाय राघवं प्राप्तं सत्वरं सम्मुखा ययुः}%॥ ५८ ॥

\twolineshloka
{ततः प्रणम्य ते दूराज्जानुभ्यामवनिं गताः}
{जयेति शब्दमादाय मुहुर्मुहुरितस्ततः}%॥ ५९ ॥

\twolineshloka
{ततस्तेनैव संयुक्ताः किष्किन्धां तां महापुरीम्}
{विविशुः सत्पताकाभिः समन्तात्समलङ्कृताम्}%॥ ६० ॥

\twolineshloka
{अथोत्तीर्य विमानाग्र्यात्सुग्रीवभवने शुभे}
{प्रविवेश द्रुतं रामः सर्वतः सुविभूषिते}%॥ ६१ ॥

\twolineshloka
{तत्र रामं निविष्टं ते विश्रान्तं वीक्ष्य वानराः}
{अर्घ्यादिभिश्च सम्पूज्य पप्रच्छुस्तदनन्तरम्}%॥ ६२ ॥

\uvacha{वानरा ऊचुः}

\twolineshloka
{तेजसा त्वं विनिर्मुक्तो दृश्यसे रघुनन्दन}
{कृशोऽस्यतीव चोद्विग्नः कच्चित्क्षेमं गृहे तव}%॥ ६३ ॥

\twolineshloka
{काये वाऽनुगतो नित्यं तथा ते लक्ष्मणोऽनुजः}
{न दृश्यते समीपस्थः किमद्य तव राघव}%॥ ६४ ॥

\twolineshloka
{तथा प्राणसमाऽभीष्टा सीता तव प्रभो}
{दृश्यते किं न पार्श्वस्था एतन्नः कौतुकं परम्}%॥ ६५ ॥

\uvacha{सूत उवाच}

\twolineshloka
{तेषां तद्वचनं श्रुत्वा चिरं निःश्वस्य राघवः}
{वाष्पपूर्णेक्षणो भूत्वा सर्वं तेषां न्यवेदयत्}%॥ ६६ ॥

\twolineshloka
{अथ सीता परित्यक्ता तथा भ्राता स लक्ष्मणः}
{यदर्थं तत्र सम्प्राप्तः स्वयमेव द्विजोत्तमाः}%॥ ६७ ॥

\twolineshloka
{तच्छ्रुत्वा वानराः सर्वे सुग्रीवप्रमुखास्ततः}
{रुरुदुस्ते सुदुःखार्ताः समालिङ्ग्य ततः परम्}%॥ ६८ ॥

\twolineshloka
{एवं चिरं प्रलप्योच्चैस्ततः प्रोचू रघूत्तमम्}
{आदेशो दीयतां राजन्योऽस्माभिरिह सिध्यति}%॥ ६९ ॥

\twolineshloka
{धन्या वयं धरापृष्ठे येषां त्वं रघुसत्तम}
{ईदृक्स्नेहसमायुक्तः समागच्छसि मन्दिरे}%॥ ७० ॥

\uvacha{राम उवाच}

\twolineshloka
{उषित्वा रजनीमेकां सुग्रीव तव मन्दिरे} 
{प्रातर्लङ्कां गमिष्यामि यत्रास्ते स विभीषणः}% ७१ ॥

\twolineshloka
{प्रधानामात्ययुक्तेन त्वयापि कपिसत्तम}
{आगन्तव्यं मया सार्धं विभीषणगृहं प्रति}%॥ ७२ ॥
॥इति श्रीस्कान्दे महापुराण एकाशीतिसाहस्र्यां संहितायां षष्ठे नागरखण्डे हाटकेश्वरक्षेत्रमाहात्म्ये श्रीरामेश्वरस्थापन प्रस्तावे लक्ष्मणनिर्वाणोत्तरं श्रीरामस्य सुग्रीवनगरीं प्रति गमनवर्णनं नाम शततमोऽध्यायः॥१००॥

\uvacha{सूत उवाच}

\twolineshloka
{एवं तां रजनीं तत्र स उषित्वा रघूत्तमः}
{उपास्यमानः सर्वैस्तैः सद्भक्त्या वानरोत्तमैः}%॥ १ ॥

\twolineshloka
{ततः प्रभाते विमले प्रोद्गते रविमण्डले}
{कृत्वा प्राभातिकं कर्म समाहूयाथ पुष्पकम्}%॥ २ ॥

\twolineshloka
{सुग्रीवेण सुषेणेन तारेण कुमुदेन च}
{अङ्गदेनाथ कुण्डेन वायुपुत्रेण धीमता}%॥ ३ ॥

\twolineshloka
{गवाक्षेण नलेनेव तथा जाम्बवतापि च}
{दशभिर्वानरैः सार्धं समारूढः स पुष्पके}%॥ ४ ॥

\twolineshloka
{ततः सम्प्रस्थितः काले लङ्कामुद्दिश्य राघवः}
{मनोजवेन तेनैव विमानेन सुवर्चसा}%॥ ५ ॥

\twolineshloka
{सम्प्राप्तस्तत्क्षणादेव लङ्काख्यां च महापुरीम्}
{वीक्षयंस्तान्प्रदेशांश्च यत्र युद्धं पुराऽभवत्}%॥ ६ ॥

\threelineshloka
{ततो विभीषणो दृष्ट्वा प्रोद्द्योतं पुष्पकोद्भवम्}
{रामं विज्ञाय सम्प्राप्तं प्रहृष्टः सम्मुखो ययौ}
{मन्त्रिभिः सकलैः सार्धं तथा भृत्यैः सुतैरपि}%॥ ७ ॥

\twolineshloka
{अथ दृष्ट्वा सुदूरात्तं रामदेवं विभीषणः}
{पपात दण्डवद्भूमौ जयशब्दमुदीरयन्}%॥ ८ ॥

\twolineshloka
{तथागतं परिष्वज्य सादरं स विभीषणम्}
{तेनैव सहितः पश्चाल्लङ्कां तां प्रविवेश ह}%॥ ९ ॥

\twolineshloka
{विभीषणगृहं प्राप्य तत्र सिंहासने शुभे}
{निविष्टो वानरैस्तैश्च समन्तात्परिवारितः}%॥ १० ॥

\twolineshloka
{ततो निवेदयामास तस्मै सर्वं विभीषणः}
{राज्यं पुत्रकलत्रादि यच्चान्यदपि किञ्चन}%॥ ११ ॥

\twolineshloka
{ततः प्रोवाच विनयात्कृताञ्जलिपुटः स्थितः}
{आदेशो दीयतां देव ब्रूहि कृत्यं करोमि किम्}%॥ १२ ॥

\twolineshloka
{अकस्मादेव सम्प्राप्तः किमर्थं वद मे प्रभो}
{किं नायातः स सौमित्रिस्त्वया सार्ध च जानकी}%॥ १३ ॥

\uvacha{सूत उवाच}

\twolineshloka
{निवेद्य राघवस्तस्मै सर्वं गद्गदया गिरा}
{वाष्पपूरप्रतिच्छन्नवक्त्रो भूयो विनिःश्वसन्}%॥ १४ ॥

\twolineshloka
{ततः प्रोवाच सत्यार्थं विभीषणकृते हितम्}
{तं चापि शोकसन्तप्तं सम्बोध्य रघुनन्दनः}%॥ १५ ॥

\twolineshloka
{अहं राज्यं परित्यज्य साम्प्रतं राक्षसोत्तम}
{यास्यामि त्रिदिवं तूर्णं लक्ष्मणो यत्र संस्थितः}%॥ १६ ॥

\twolineshloka
{न तेन रहितो मर्त्ये मुहूर्तमपि चोत्सहे}
{स्थातुं राक्षसशार्दूल बान्धवेन महात्मना}%॥ १७ ॥

\twolineshloka
{अहं शिक्षापणार्थाय तव प्राप्तो विभीषण}
{तस्मादव्यग्रचित्तेन संशृणुष्व कुरुष्व च}%॥ १८ ॥

\twolineshloka
{एषा राज्योद्भवा लक्ष्मीर्मदं सञ्जनयेन्नृणाम्}
{मद्यवत्स्वल्पबुद्धीनां तस्मात्कार्यो न स त्वया}%॥ १९ ॥

\twolineshloka
{शक्राद्या अमराः सर्वे त्वया पूज्याः सदैव हि}
{मान्याश्च येन ते राज्यं जायते शाश्वतं सदा}%॥ २० ॥

\twolineshloka
{मम सत्यं भवेद्वाक्य मेतस्मादहमागतः}
{प्राप्तराज्यप्रतिष्ठोऽपि तव भ्राता महाबलः}%॥ २१ ॥

\threelineshloka
{विनाशं सहसा प्राप्तस्तस्मान्मान्याः सुराः सदा}
{यदि कश्चित्समायाति मानुषोऽत्र कथञ्चन}
{मत्काय एव द्रष्टव्यः सर्वैरेव निशाचरैः}%॥ २२ ॥

\twolineshloka
{तथा निशाचराः सर्वे त्वया वार्या विभीषण}
{मम सेतुं समुल्लङ्घ्य न गन्तव्यं धरातले}%॥ २३ ॥

\uvacha{विभीषण उवाच}

\twolineshloka
{एवं विभो करिष्यामि तवादेशमसंशयम्}
{परं त्वया परित्यक्ते मर्त्ये मे जीवितं व्रजेत्}%॥ २४ ॥

\twolineshloka
{तस्मान्मामपि तत्रैव त्वं विभो नेतुमर्हसि}
{आत्मना सह यत्रास्ते प्राग्गतो लक्ष्मणस्तव}%॥ २५ ॥

\uvacha{श्रीराम उवाच}

\twolineshloka
{मया तेऽक्षयमादिष्टं राज्यं राक्षससत्तम}
{तस्मान्नार्हसि मां कर्तुं मिथ्याचारं कथञ्चन}%॥ २६ ॥

\threelineshloka
{अहमस्मिन्स्वके सेतौ शङ्करत्रितयं शुभम्}
{स्थापयिष्यामि कीर्त्यर्थं तत्पूज्यं भवता सदा}
{भक्तिमान्प्रतिसन्धाय यावच्चन्द्रार्कतारकम्}%॥ २७ ॥

\twolineshloka
{एवमुक्त्वा रघुश्रेष्ठो राक्षसेन्द्रं विभीषणम्}
{दशरात्रं तत्र तस्थौ लङ्कायां वानरैः सह}%॥ २८ ॥

\twolineshloka
{कुर्वन्युद्धकथाश्चित्रा याः कृताः पूर्वमेव हि}
{पश्यन्युद्धस्य सर्वाणि स्थानानि विविधानि च}%॥ २९ ॥

\twolineshloka
{शंसमानः प्रवीरांस्तान्राक्षसान्बलवत्तरान्}
{कुम्भकर्णेन्द्रजित्पूर्वान्सङ्ख्ये चाभिमुखागतान्}%॥ ३० ॥

\twolineshloka
{ततश्चैकादशे प्राप्ते दिवसे रघुनन्दनः}
{पुष्पकं तत्समारुह्य प्रस्थितः स्वपुरीं प्रति}%॥ ३१ ॥

\twolineshloka
{वानरैस्तैः समोपेतो विभीषणपुरःसरः}
{ततः संस्थापयामास सेतुप्रान्ते महेश्वरम्}%॥ ३२ ॥

\twolineshloka
{मध्ये चैव तथादौ च श्रद्धापूतेन चेतसा}
{रामेश्वरत्रयं राम एवं तत्र विधाय सः}%॥ ३३ ॥

\twolineshloka
{सेतुबन्धं तथासाद्य प्रस्थितः स्वगृहं प्रति}
{तावद्विभीषणेनोक्तः प्रणिपत्य मुहुर्मुहुः}%॥ ३४ ॥

\uvacha{विभीषण उवाच}

\twolineshloka
{अनेन सेतुमार्गेण रामेश्वरदिदृक्षया}
{मानवा आगमिष्यन्ति कौतुकाच्छ्रद्धयाविताः}%॥ ३५ ॥

\twolineshloka
{राक्षसानां महाराज जातिः क्रूरतमा मता}
{दृष्ट्वा मानुषमायान्तं मांसस्येच्छा प्रजायते}%॥ ३६ ॥

\twolineshloka
{यदा कश्चिज्जनं कश्चिद्राक्षसो भक्षयिष्यति}
{आज्ञाभङ्गो ध्रुवं भावी मम भक्तिरतस्य च}%॥ ३७ ॥

\twolineshloka
{भविष्यन्ति कलौ काले दरिद्रा नृपमानवाः}
{तेऽत्र स्वर्णस्य लोभेन देवतादर्शनाय च}%॥ ३८ ॥

\twolineshloka
{नित्यं चैवागमिष्यन्ति त्यक्त्वा रक्षःकृतं भयम्}
{तेषां यदि वधं कश्चिद्राक्षसात्प्रापयिष्यति}%॥ ३९ ॥

\threelineshloka
{भविष्यति च मे दोषः प्रभुद्रोहोद्भवः प्रभो}
{तस्मात्कञ्चिदुपायं त्वं चिन्तयस्व यथा मम}
{आज्ञाभङ्गकृतं पापं जायते न गुरो क्वचित्}%॥ ४० ॥

\twolineshloka
{तस्य तद्वचनं श्रुत्वा ततः स रघुसत्तमः}
{बाढमित्येव चोक्त्वाथ चापं सज्जीचकार सः}%॥ ४१ ॥

\twolineshloka
{ततस्तं कीर्तिरूपं च मध्यदेशे रघूत्तमः}
{अच्छिनन्निशितैर्बाणैर्दशयोजनविस्तृतम्}%॥ ४२ ॥

\twolineshloka
{तेन संस्थापितो यत्र शिखरे शङ्करः स्वयम्}
{शिखरं तत्सलिङ्गं च पतितं वारिधेर्जले}%॥ ४३ ॥

\twolineshloka
{एवं मार्गमगम्यं तं कृत्वा सेतुसमुद्भवम्}
{वानरै राक्षसैः सार्धं ततः सम्प्रस्थितो गृहम्}%॥ ४४ ॥
॥इति श्रीस्कान्दे महापुराण एकाशीतिसाहस्र्यां संहितायां षष्ठे नागरखण्डे हाटकेश्वरक्षेत्रमाहात्म्ये सेतुमध्ये श्रीरामकृतरामेश्वरप्रतिष्ठावर्णनं नामैकोत्तरशततमोऽध्यायः॥१०१॥

\uvacha{सूत उवाच}

\twolineshloka
{सम्प्रस्थितस्य रामस्य स्वकीयं सदनं प्रति}
{यदाश्चर्यमभून्मार्गे श्रूयतां द्विजसत्तमाः}%॥ १ ॥

\twolineshloka
{नभोमार्गेण गच्छत्तद्विमानं पुष्पकं द्विजाः}
{अकस्मादेव सञ्जातं निश्चलं चित्रकृन्नृणाम्}%॥ २ ॥

\twolineshloka
{अथ तन्निश्चलं दृष्ट्वा पुष्पकं गगनाङ्गणे}
{रामो वायुसुतस्येदं वचनं प्राह विस्मयात्}%॥ ३ ॥

\twolineshloka
{त्वं गत्वा मारुते शीघ्रं भूमिं जानीहि कारणम्}
{किमेतत्पुष्पकं व्योम्नि निश्चलत्वमुपागतम्}%॥ ४ ॥

\twolineshloka
{कदाचिद्धार्यते नास्य गतिः कुत्रापि केनचित्}
{ब्रह्मदृष्टिप्रसूतस्य पुष्पकस्य महात्मनः}%॥ ५ ॥

\twolineshloka
{बाढमित्येव स प्रोच्य हनूमान्धरणीतलम्}
{गत्वा शीघ्रं पुनः प्राह प्रणिपत्य रघूत्तमम्}%॥ ६ ॥

\twolineshloka
{अत्रास्याधः शुभं क्षेत्रं हाटकेश्वर संज्ञितम्}
{यत्र साक्षाज्जगत्कर्ता स्वयं ब्रह्मा व्यवस्थितः}%॥ ७ ॥

\twolineshloka
{आदित्या वसवो रुद्रा देववैद्यौ तथाश्विनौ}
{तत्र तिष्ठन्ति ते सर्वे तथान्ये सिद्धकिन्नराः}%॥ ८ ॥

\twolineshloka
{एतस्मात्कारणान्नैतदतिक्रामति पुष्पकम्}
{तत्क्षेत्रं निश्चलीभूतं सत्यमेतन्मयोदितम्}%॥ ९ ॥

\uvacha{सूत उवाच}

\twolineshloka
{तस्य तद्वचनं श्रुत्वा कौतूहलसमवितः} 
{पुष्पकं प्रेरयामास तत्क्षेत्रं प्रति राघवः} 

\twolineshloka
{सर्वैस्तैर्वानरैः सार्धं राक्षसैश्च पृथग्विधैः}
{अवतीर्य ततो हृष्टस्तस्मिन्क्षेत्रे समन्ततः}%॥ ११ ॥

\threelineshloka
{तीर्थमालोकयामास पुण्यान्यायतनानि च}
{ततो विलोकयामास पितामहविनिर्मिताम्}
{चामुण्डां तत्र च स्नात्वा कुण्डे कामप्रदायिनि}%॥ १२ ॥

\twolineshloka
{ततो विलोकयामास पित्रा तस्य विनिर्मितम्}
{रामः स्वमिव देवेशं दृष्ट्वा देवं चतुर्भुजम्}%॥ १३ ॥

\twolineshloka
{राजवाप्यां शुचिर्भूत्वा स्नात्वा तर्प्य निजान्पितॄन्}
{ततश्च चिन्तयामास क्षेत्रे त्र बहुपुण्यदे}%॥ १४ ॥

\twolineshloka
{लिङ्गं संस्थापयाम्येव यद्वत्तातेन केशवः}
{तथा मे दयितो भ्राता लक्ष्मणो दिवमाश्रितः}%॥ १५ ॥

\threelineshloka
{यस्तस्य नामनिर्दिष्टं लिङ्गं संस्थापयाम्यहम्}
{तं चापि मूर्तिमन्तं च सीतया सहितं शुभम्}
{क्षेत्रे मेध्यतमे चात्र तथात्मानं दृषन्मयम्}%॥ १६ ॥

\twolineshloka
{एवं स निश्चयं कृत्वा प्रासादानां च पञ्चकम्}
{स्थापयामास सद्भक्त्या रामः शस्त्रभृतां वरः}%॥ १७ ॥

\twolineshloka
{ततस्ते वानराः सर्वे राक्षसाश्च विशेषतः}
{लिङ्गानि स्थापयामासुः स्वानिस्वानि पृथक्पृथक्}%॥ १८ ॥

\twolineshloka
{तत्रैव सुचिरं कालं स्थितास्ते श्रद्धयाऽन्विताः}
{ततो जग्मुरयोध्यायां विमानवरमाश्रिताः}%॥ १९ ॥

\twolineshloka
{एतद्वः सर्वमाख्यातं यथा रामेश्वरो महान्}
{लक्ष्मणेश्वरसंयुक्तस्तस्मिंस्तीर्थे सुशोभने}%॥ २० ॥

\twolineshloka
{यस्तौ प्रातः समुत्थाय सदा पश्यति मानवः}
{स कृत्स्नं फलमाप्नोति श्रुते रामायणेऽत्र यत्}%॥ २१ ॥

\twolineshloka
{अथाष्टम्यां चतुर्दश्यां यो रामचरितं पठेत्}
{तदग्रे वाजिमेधस्य स कृत्स्नं लभते फलम्}%॥ २२ ॥
॥इति श्रीस्कान्दे महापुराण एकाशीतिसाहस्र्यां संहितायां षष्ठे नागरखण्डे हाटकेश्वरक्षेत्रमाहात्म्ये श्रीरामचन्द्रेण हाटकेश्वरक्षेत्रे लक्ष्मणादिप्रासादपञ्चकनिर्माणप्रतिष्ठापनवर्णनं नाम द्व्युत्तरशततमोऽध्यायः॥१०२॥

===

\sect{एकादशोत्तरशततमोऽध्यायः --- रामेश्वरक्षेत्रमाहात्म्यवर्णनम्}

\src{स्कन्दपुराणम्}{खण्डः ७ (प्रभासखण्डः)}{प्रभासक्षेत्र माहात्म्यम्}{अध्यायः १११}
\vakta{}
\shrota{}
\tags{}
\notes{}
\textlink{https://sa.wikisource.org/wiki/स्कन्दपुराणम्/खण्डः_७_(प्रभासखण्डः)/प्रभासक्षेत्र_माहात्म्यम्/अध्यायः_१११}
\translink{https://www.wisdomlib.org/hinduism/book/the-skanda-purana/d/doc626899.html}

\storymeta




\uvacha{ईश्वर उवाच}

\twolineshloka
{ततो गच्छेन्महादेवि पुष्करारण्यमुत्तमम्}
{तस्मादीशानकोणस्थं धनुषां षष्टिभिः स्थितम्}%॥ १ ॥

\twolineshloka
{तत्र कुण्डं महादेवि ह्यष्टपुष्करसंज्ञितम्}
{सर्व पापहरं देवि दुष्प्राप्यमकृतात्मभिः}%॥ २ ॥

\twolineshloka
{तत्र कुण्डसमीपे तु पुरा रामेशधीमता}
{स्थापितं तन्महालिङ्गं रामेश्वर इति स्मृतम्}%॥ ३

\onelineshloka
{तस्य पूजनमात्रेण मुच्यते ब्रह्महत्यया}%॥ ४ ॥

\uvacha{श्रीदेव्युवाच}

\twolineshloka
{भगवन्विस्तराद्ब्रूहि रामेश्वरसमुद्भवम्}
{कथं तत्रागमद्रामः ससीतश्च सलक्ष्मणः}%॥ ५ ॥

\twolineshloka
{कथं प्रतिष्ठितं लिङ्गं पुष्करे पापतस्करे}
{एतद्विस्तरतो ब्रूहि फलं माहात्म्यसंयुतम्}%॥ ६ ॥

\uvacha{ईश्वर उवाच}

\twolineshloka
{चतुर्विंशयुगे रामो वसिष्ठेन पुरोधसा}
{पुरा रावणनाशार्थं जज्ञे दशरथात्मजः}%॥ ७ ॥

\twolineshloka
{ततः कालान्तरे देवि ऋषिशापान्महातपाः}
{ययौ दाशरथी रामः ससीतः सहलक्ष्मणः}%॥ ८ ॥

\twolineshloka
{वनवासाय निष्क्रान्तो दिव्यैर्ब्रह्मर्षिभिर्वृतः}
{ततो यात्राप्रसङ्गेन प्रभासं क्षेत्रमागतः}%॥ ९ ॥

\twolineshloka
{तं देशं तु समासाद्य सुश्रान्तो निषसाद ह}
{अस्तं गते ततः सूर्ये पर्णान्यास्तीर्य भूतले}%॥ १० ॥

\twolineshloka
{सुष्वापाथ निशाशेषे ददृशे पितरं स्वकम्}
{स्वप्ने दशरथं देवि सौम्यरूपं महाप्रभम्}%॥ ११ ॥

\twolineshloka
{प्रातरुत्थाय तत्सर्वं ब्राह्मणेभ्यो न्यवेदयत्}
{यथा दशरथः स्वप्ने दृष्टस्तेन महात्मना}%॥ १२ ॥

\uvacha{ब्राह्मणा ऊचुः}

\twolineshloka
{वृद्धिकामाश्च पितरो वरदास्तव राघव}
{दर्शनं हि प्रयच्छन्ति स्वप्नान्ते हि स्ववंशजे}%॥ १३ ॥

\twolineshloka
{एतत्तीर्थं महापुण्यं सुगुप्तं शार्ङ्गधन्वनः}
{पुष्करेति समाख्यातं श्राद्धमत्र प्रदीयताम्}%॥ १४ ॥

\twolineshloka
{नूनं दशरथो राजा तीर्थे चास्मिन्समीहते}
{त्वया दत्तं शुभं पिण्डं ततः स दर्शनं गतः}%॥ १५ ॥

\uvacha{ईश्वर उवाच}

\twolineshloka
{तेषां तद्वचनं श्रुत्वा रामो राजीवलोचनः}
{निमन्त्रयामास तदा श्राद्धार्हान्ब्राह्मणाञ्छुभान्}%॥ १६ ॥

\twolineshloka
{अब्रवील्लक्ष्मणं पार्श्वे स्थितं विनतकन्धरम्}
{फलार्थं व्रज सौमित्रे श्राद्धार्थं त्वरयाऽन्वितः}%॥ १७ ॥

\twolineshloka
{स तथेति प्रतिज्ञाय जगाम रघुनन्दनः}
{आनयामास शीघ्रं स फलानि विविधानि च}%॥ १८ ॥

\twolineshloka
{बिल्वानि च कपित्थानि तिन्दुकानि च भूरिशः}
{बदराणि करीराणि करमर्दानि च प्रिये}%॥ १९ ॥

\twolineshloka
{चिर्भटानि परूषाणि मातुलिङ्गानि वै तथा}
{नालिकेराणि शुभ्राणि इङ्गुदीसम्भवानि च}%॥ २० ॥

\twolineshloka
{अथैतानि पपाचाशु सीता जनकनन्दिनी}
{ततस्तु कुतपे काले स्नात्वा वल्कलभृच्छुचिः}%॥ २१ ॥

\twolineshloka
{ब्राह्मणानानयामास श्राद्धार्हान्द्विजसत्तमान्}
{गालवो देवलो रैभ्यो यवक्रीतोऽथ पर्वतः}%॥ २२ ॥

\twolineshloka
{भरद्वाजो वसिष्ठश्च जावालिर्गौतमो भृगुः}
{एते चान्ये च बहवो ब्राह्मणा वेदपारगाः}%॥ २३ ॥

\twolineshloka
{श्राद्धार्थं तस्य सम्प्राप्ता रामस्याक्लिष्टकर्मणः}
{एतस्मिन्नेव काले तु रामः सीतामभाषत}%॥ २४ ॥

\twolineshloka
{एहि वैदेहि विप्राणां देहि पादावनेजनम्}
{एतच्छ्रुत्वाऽथ सा सीता प्रविष्टा वृक्षमध्यतः}%॥ २५ ॥

\twolineshloka
{गुल्मैराच्छाद्य चात्मानं रामस्यादर्शने स्थिता}
{मुहुर्मुहुर्यदा रामः सीतासीतामभाषत}%॥ २६ ॥

\twolineshloka
{ज्ञात्वा तां लक्ष्मणो नष्टां कोपाविष्टं च राघवम्}
{स्वयमेव तदा चक्रे ब्राह्मणार्ह प्रतिक्रियाम्}%॥ २७ ॥

\twolineshloka
{अथ भुक्तेषु विप्रेषु कृत पिण्डप्रदानके}
{आगता जानकी सीता यत्र रामो व्यवस्थितः}%॥ २८ ॥

\threelineshloka
{तां दृष्ट्वा परुषैर्वाक्यैर्भर्त्सयामास राघवः}
{धिग्धिक्पापे द्विजांस्त्यक्त्वा पितृकृत्यमहोदयम्}
{क्व गताऽसि च मां हित्वा श्राद्धकाले ह्युपस्थिते}%॥ २९ ॥

\uvacha{ईश्वर उवाच}

\onelineshloka
{तस्य तद्वचनं श्रुत्वा भयभीता च जानकी}%॥ ३० ॥

\twolineshloka
{कृताञ्जलिपुटा भूत्वा वेपमाना ह्यभाषत}
{मा कोपं कुरु कल्याण मा मां निर्भर्त्सय प्रभो}%॥ ३१ ॥

\twolineshloka
{शृणु यस्माद्विभोऽन्यत्र गता त्यक्त्वा तवान्तिकम्}
{दृष्टस्तत्र पिता मेऽद्य तथा चैव पितामहः}%॥ ३२ ॥

\twolineshloka
{तस्य पूर्वतरश्चापि तथा मातामहादयः}
{अङ्गेषु ब्राह्मणेन्द्राणामाक्रान्तास्ते पृथक्पृथक्}%॥ ३३ ॥

\twolineshloka
{ततो लज्जा समभवत्तत्र मे रघुनन्दन}
{पित्रा तत्र महाबाहो मनोज्ञानि शुभानि च}%॥ ३४ ॥

\threelineshloka
{भक्ष्याणि भक्षितान्येव यानि वै गुणवन्ति च}
{स कथं सुकषायाणि क्षाराणि कटुकानि च}
{भक्षयिष्यति राजेन्द्र ततो मे दुःखमाविशत्}%॥ ३६ ॥

\twolineshloka
{एतस्मात्कारणान्नष्टा लज्जयाऽहं रघूद्वह}
{दृष्ट्वा श्वशुरवर्गं स्वं तस्मात्कोपं परित्यज}%॥ ३६ ॥

\twolineshloka
{तस्यास्तद्वचनं श्रुत्वा विस्मितो राघवोऽभवत्}
{विशेषेण ददौ तस्मिञ्छ्राद्धं तीर्थे तु पुष्करे}%॥ ३७ ॥

\twolineshloka
{तत्र पुष्करसान्निध्ये दक्षिणे धनुषां त्रये}
{लिङ्गं प्रतिष्ठयामास रामेश्वरमिति श्रुतम्}%॥ ३५ ॥

\twolineshloka
{यस्तं पूजयते भक्त्या गन्धपुष्पादिभिः क्रमात्}
{स प्राप्नोति परं स्थानं य्रत्र देवो जनार्दनः}%॥ ३९ ॥

\twolineshloka
{किमत्र बहुनोक्तेन द्वादश्यां यत्प्रदापयेत्}
{न तत्र परिसङ्ख्यानं त्रिषु लोकेषु विद्यते}%॥ ४० ॥

\twolineshloka
{शुक्राङ्गारकसंयुक्ता चतुर्थी या भवेत्क्वचित्}
{षष्ठी वात्र वरारोहे तत्र श्राद्धे महत्फलम्}%॥ ४१ ॥

\twolineshloka
{यावद्द्वादशवर्षाणि पितरश्च पितामहाः}
{तर्पिता नान्यमिच्छन्ति पुष्करे स्वकुलोद्भवे}%॥ ४२ ॥

\twolineshloka
{तत्र यो वाजिनं दद्यात्सम्यग्भक्तिसमन्वितः}
{अश्वमेधस्य यज्ञस्य फलं प्राप्नोति मानवः}%॥ ४३ ॥

\twolineshloka
{इति ते कथितं सम्यङ्माहात्म्यं पापनाशनम्}
{रामेश्वरस्य देवस्य पुष्करस्य च भामिनि}%॥ ४४ ॥
॥इति श्रीस्कान्दे महापुराण एकाशीतिसाहस्र्यां संहितायां सप्तमे प्रभासखण्डे प्रथमे प्रभासक्षेत्रमाहात्म्ये पुष्करमाहात्म्ये रामेश्वरक्षेत्रमाहात्म्यवर्णनं नामैकादशोत्तरशततमोऽध्यायः॥१११॥

===

\sect{द्वादशोत्तरशततमोऽध्यायः --- लक्ष्मणेश्वरमाहात्म्यवर्णनम्}

\src{स्कन्दपुराणम्}{खण्डः ७ (प्रभासखण्डः)}{प्रभासक्षेत्र माहात्म्यम्}{अध्यायः ११२}
\vakta{}
\shrota{}
\tags{}
\notes{}
\textlink{https://sa.wikisource.org/wiki/स्कन्दपुराणम्/खण्डः_७_(प्रभासखण्डः)/प्रभासक्षेत्र_माहात्म्यम्/अध्यायः_११२}
\translink{https://www.wisdomlib.org/hinduism/book/the-skanda-purana/d/doc626900.html}

\storymeta




\uvacha{ईश्वर उवाच}

\twolineshloka
{ततो गच्छेन्महादेवि लक्ष्मणेश्वरमुत्तमम्}
{रामेशात्पूर्वदिग्भागे धनुस्त्रिंशकसंस्थितम्}%॥ १ ॥

\twolineshloka
{स्थापितं लक्ष्मणेनैव तत्र यात्रागतेन वै}
{महापापहरं देवि तल्लिङ्गं सुरपूजितम्}%॥ २ ॥

\twolineshloka
{यस्तं पूजयते भक्त्या नृत्यगीतादिवादनैः}
{होमजाप्यैः समाधिस्थः स याति परमां गतिम्}%॥ ३ ॥

\twolineshloka
{अन्नोदकं हिरण्यं च तत्र देयं द्विजातये}
{सम्पूज्य देवदेवेशं गन्धपुष्पादिभिः क्रमात्}%॥ ४ ॥

\twolineshloka
{माघे कृष्णचतुर्दश्यां विशेषस्तत्र पूजने}
{स्नानं दानं जपस्तत्र भवेदक्षयकारकम्}%॥ ५ ॥
॥इति श्रीस्कान्दे महापुराण एकाशीतिसाहस्र्यां संहितायां सप्तमे प्रभासखण्डे प्रथमे प्रभासक्षेत्रमाहात्म्ये रामेश्वरक्षेत्रमाहात्म्ये लक्ष्मणेश्वरमाहात्म्यवर्णनं नाम द्वादशोत्तरशततमोऽध्यायः॥११२॥

===

\sect{त्रयोदशोत्तरशततमोऽध्यायः --- जानकीश्वरमाहात्म्यवर्णनम्}

\src{स्कन्दपुराणम्}{खण्डः ७ (प्रभासखण्डः)}{प्रभासक्षेत्र माहात्म्यम्}{अध्यायः ११३}
\vakta{}
\shrota{}
\tags{}
\notes{}
\textlink{https://sa.wikisource.org/wiki/स्कन्दपुराणम्/खण्डः_७_(प्रभासखण्डः)/प्रभासक्षेत्र_माहात्म्यम्/अध्यायः_११३}
\translink{https://www.wisdomlib.org/hinduism/book/the-skanda-purana/d/doc626901.html}

\storymeta




\uvacha{ईश्वर उवाच}

\twolineshloka
{ततो गच्छेन्महादेवि जानकीश्वरमुत्तमम्}
{रामेशान्नैऋते भागे धनुस्त्रिंशकसंस्थितम्}%॥ १ ॥

\twolineshloka
{पापघ्नं सर्वजन्तूनां जानक्याऽऽराधितं पुरा}
{प्रतिष्ठितं विशेषेण सम्यगाराध्यशङ्करम्}%॥ २ ॥

\twolineshloka
{पूर्वं तस्यैव लिङ्गस्य वसिष्ठेशेति नाम वै}
{तत्पश्चाज्जानकीशेति त्रेतायां प्रथितं क्षितौ}%॥ ३ ॥

\twolineshloka
{ततः षष्टिसहस्राणि वालखिल्या महर्षयः}
{तत्र सिद्धिमनुप्राप्तास्तेन सिद्धेश्वरेति च}%॥ ४ ॥

\twolineshloka
{ख्यातं कलौ महादेवि युगलिङ्गं महाप्रभम्}
{तद्दृष्ट्वा मुच्यते पापैर्दुःखदौर्भाग्यसम्भवैः}%॥ ५ ॥

\twolineshloka
{यस्तं पूजयते भक्त्या नारी वा पुरुषोऽपि वा}
{संस्नाप्य विधिवद्भक्त्या स मुक्तः पातकैर्भवेत्}%॥ ६ ॥

\twolineshloka
{स्नात्वा च पुष्करे तीर्थे यस्तल्लिगं प्रपूजयेत्}
{नियतो नियताहारो मासमेकं निरन्तरम्}%॥ ७ ॥

\threelineshloka
{दिनेदिने भवेत्तस्य वाजिमेधाधिकं फलम्}
{माघे मासि तृतीयायां या नारी तं प्रपूजयेत्}
{तदन्वयेऽपि दौर्भाग्यं दुःखं शोकश्च नो भवेत्}%॥ ८ ॥

\twolineshloka
{इति ते कथितं देवि माहात्म्यं पापनाशनम्}
{श्रुतं हरति पापानि सौभाग्यं सम्प्रयच्छति}%॥ ९ ॥
॥इति श्रीस्कान्दे महापुराण एकाशीतिसाहस्र्यां संहितायां सप्तमे प्रभासखण्डे प्रथमे प्रभासक्षेत्रमाहात्म्ये जानकीश्वरमाहात्म्यवर्णनं नाम त्रयोदशोत्तरशततमोऽध्यायः॥११३॥

===

\sect{एकसप्तत्युत्तरशततमोऽध्यायः --- दशरथेश्वरमाहात्म्यवर्णनम्}

\src{स्कन्दपुराणम्}{खण्डः ७ (प्रभासखण्डः)}{प्रभासक्षेत्र माहात्म्यम्}{अध्यायः १७१}
\vakta{}
\shrota{}
\tags{}
\notes{}
\textlink{https://sa.wikisource.org/wiki/स्कन्दपुराणम्/खण्डः_७_(प्रभासखण्डः)/प्रभासक्षेत्र_माहात्म्यम्/अध्यायः_१७१}
\translink{https://www.wisdomlib.org/hinduism/book/the-skanda-purana/d/doc626959.html}

\storymeta




\uvacha{ईश्वर उवाच}

\twolineshloka
{ततो गच्छेन्महादेवि देवीमेकल्लवीरिकाम्}
{एकल्लवीरायाम्ये तु नातिदूरे व्यवस्थिताम्}%॥ १ ॥

\twolineshloka
{पूर्वं दशरथो योऽसौ सूर्यवंशविभूषणः}
{प्रभासं क्षेत्रमासाद्य तपश्चक्रे सुदुश्चरम्}%॥ २ ॥

\twolineshloka
{लिङ्गं तत्र प्रतिष्ठाप्य तोषयामास शाङ्करम्}
{स देवं प्रार्थयामास पुत्रं चैवामितौजसम्}%॥ ३ ॥

\twolineshloka
{ददौ तस्य तदा पुत्रं देवं त्रैलोक्यपूजितम्}
{रामेति नाम यस्यासीत्त्रैलोक्ये प्रथितं यशः}%॥ ४ ॥

\twolineshloka
{यस्याद्यापीह गायन्ति भूर्भुवःस्वर्नि वासिनः}
{देवदैत्यासुराः सर्वे वाल्मीक्याद्या महर्षयः}%॥ ५ ॥

\threelineshloka
{तल्लिङ्गस्य प्रभावेन प्राप्तं राज्ञा महद्यशः}
{कार्तिक्यां कार्तिके मासि विधिना यस्तमर्चयेत्}
{दीपपूजोपहारेण यशस्वी सोऽपि जायते}%॥ ६ ॥
॥इति श्रीस्कान्दे महापुराण एकाशीतिसाहस्र्यां संहितायां सप्तमे प्रभासखण्डे प्रथमे प्रभासक्षेत्रमाहात्म्ये दशरथेश्वरमाहात्म्यवर्णनं नामैकसप्तत्युत्तरशततमोऽध्यायः॥१७१॥

===

\sect{अष्टादशोऽध्यायः --- बलिनिग्रहवृत्तान्तवर्णनम्}

\src{स्कन्दपुराणम्}{खण्डः ७ (प्रभासखण्डः)}{वस्त्रापथक्षेत्रमाहात्म्यम्}{अध्यायः १८}
\vakta{वामनः}
\shrota{नारदः}
\tags{}
\notes{}
\textlink{https://sa.wikisource.org/wiki/स्कन्दपुराणम्/खण्डः_७_(प्रभासखण्डः)/वस्त्रापथक्षेत्रमाहात्म्यम्/अध्यायः_१८}
\translink{https://www.wisdomlib.org/hinduism/book/the-skanda-purana/d/doc627173.html}

\storymeta

\twolineshloka
{लङ्कायां रावणो राज्यं करिष्यति महाबलः}
{त्रैलोक्यकण्टकं नाम यदासौ धारयिष्यति}%॥ १८१ ॥

\twolineshloka
{तदा दाशरथी रामः कौसल्यानन्दवर्द्धनः}
{भविष्ये भ्रातृभिः सार्द्धं गमिष्ये यज्ञमण्डपे}%॥ १८२ ॥

\twolineshloka
{ताडकां ताडयित्वाहं सुबाहुं यज्ञमन्दिरे}
{नीत्वा यज्ञाद्गमिष्यामि सीतायास्तु स्वयंवरे}%॥ १८३ ॥

\twolineshloka
{परिणेष्याभि तां सीतां भङ्क्त्वा माहेश्वरं धनुः}
{त्यक्त्वा राज्यं गमिष्यामि वने वर्षांश्चतुर्दश}%॥ १८४ ॥

\twolineshloka
{सीताहरणजं दुःखं प्रथमं मे भविष्यति}
{नासाकर्णविहीनां तां करिष्ये राक्षसीं वने}%॥ १८५ ॥

\twolineshloka
{चतुर्द्दशसहस्राणि त्रिशिरःखरदूषणान्}
{द्हत्वा हनिष्ये मारीचं राक्षसं मृगरूपिणम्}%॥ १८६ ॥

\twolineshloka
{हृतदारो गमिष्यामि दग्ध्वा गृध्रं जटायुषम्}
{सुग्रीवेण समं मैत्रीं कृत्वा हत्वाऽथ वालिनम्}%॥ १८७ ॥

\twolineshloka
{समुद्रं बन्धयिष्यामि नलप्रमुखवानरैः}
{लङ्कां संवेष्टयिष्यामि मारयिष्यामि राक्षसान्}%॥ १८८ ॥

\twolineshloka
{कुम्भकर्णं निहत्याजौ मेघनादं ततो रणे}
{निहत्य रावणं रक्षः पश्यतां सर्वरक्षसाम्}%॥ १८९ ॥

\twolineshloka
{विभीषणाय दास्यामि लङ्कां देवविनिर्मिताम्}
{अयोध्यां पुनरागत्य कृत्वा राज्यमकण्टकम्}%॥ १९० ॥

\twolineshloka
{कालदुर्वाससोश्चित्रचरित्रेणामरावतीम्}
{यास्येऽहं भ्रातृभिः सार्धं राज्यं पुत्रे निवेद्य च}%॥ १९१ ॥

॥इति श्रीस्कान्दे महापुराण एकाशीतिसाहस्र्यां संहितायां सप्तमे प्रभासखण्डे द्वितीये वस्त्रापथक्षेत्रमाहात्म्ये बलिनिग्रहवृत्तान्तवर्णनं नामाष्टादशोऽध्यायः॥१८॥
