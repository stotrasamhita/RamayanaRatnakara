\sect{अष्टादशोऽध्यायः --- बलिनिग्रहवृत्तान्तवर्णने रामावतारवर्णनम्}

\src{स्कन्दपुराणम्}{खण्डः ७ (प्रभासखण्डः)}{वस्त्रापथक्षेत्रमाहात्म्यम्}{अध्यायः १८}
\vakta{वामनः}
\shrota{नारदः}
\tags{}
\notes{Vāmana Bhagavān narrates the context/story of different avatāras to Nārada.}
\textlink{https://sa.wikisource.org/wiki/स्कन्दपुराणम्/खण्डः_७_(प्रभासखण्डः)/वस्त्रापथक्षेत्रमाहात्म्यम्/अध्यायः_१८}
\translink{https://www.wisdomlib.org/hinduism/book/the-skanda-purana/d/doc627173.html}

\storymeta

\addtocounter{shlokacount}{180}

\twolineshloka
{लङ्कायां रावणो राज्यं करिष्यति महाबलः}
{त्रैलोक्यकण्टकं नाम यदासौ धारयिष्यति}%॥ १८१ ॥

\twolineshloka
{तदा दाशरथी रामः कौसल्यानन्दवर्द्धनः}
{भविष्ये भ्रातृभिः सार्द्धं गमिष्ये यज्ञमण्डपे}%॥ १८२ ॥

\twolineshloka
{ताडकां ताडयित्वाहं सुबाहुं यज्ञमन्दिरे}
{नीत्वा यज्ञाद्गमिष्यामि सीतायास्तु स्वयंवरे}%॥ १८३ ॥

\twolineshloka
{परिणेष्याभि तां सीतां भङ्क्त्वा माहेश्वरं धनुः}
{त्यक्त्वा राज्यं गमिष्यामि वने वर्षांश्चतुर्दश}%॥ १८४ ॥

\twolineshloka
{सीताहरणजं दुःखं प्रथमं मे भविष्यति}
{नासाकर्णविहीनां तां करिष्ये राक्षसीं वने}%॥ १८५ ॥

\twolineshloka
{चतुर्द्दशसहस्राणि त्रिशिरःखरदूषणान्}
{द्हत्वा हनिष्ये मारीचं राक्षसं मृगरूपिणम्}%॥ १८६ ॥

\twolineshloka
{हृतदारो गमिष्यामि दग्ध्वा गृध्रं जटायुषम्}
{सुग्रीवेण समं मैत्रीं कृत्वा हत्वाऽथ वालिनम्}%॥ १८७ ॥

\twolineshloka
{समुद्रं बन्धयिष्यामि नलप्रमुखवानरैः}
{लङ्कां संवेष्टयिष्यामि मारयिष्यामि राक्षसान्}%॥ १८८ ॥

\twolineshloka
{कुम्भकर्णं निहत्याजौ मेघनादं ततो रणे}
{निहत्य रावणं रक्षः पश्यतां सर्वरक्षसाम्}%॥ १८९ ॥

\twolineshloka
{विभीषणाय दास्यामि लङ्कां देवविनिर्मिताम्}
{अयोध्यां पुनरागत्य कृत्वा राज्यमकण्टकम्}%॥ १९० ॥

\twolineshloka
{कालदुर्वाससोश्चित्रचरित्रेणामरावतीम्}
{यास्येऽहं भ्रातृभिः सार्धं राज्यं पुत्रे निवेद्य च}%॥ १९१ ॥

॥इति श्रीस्कान्दे महापुराण एकाशीतिसाहस्र्यां संहितायां सप्तमे प्रभासखण्डे द्वितीये वस्त्रापथक्षेत्रमाहात्म्ये बलिनिग्रहवृत्तान्तवर्णनं नामाष्टादशोऽध्यायः॥१८॥

\closesection