\sect{रामाय देवीवरदानम्}

\src{देवी-भागवतम्}{तृतीयः स्कन्धः}{अध्यायाः २८}{श्लोकाः १--६३}
\vakta{व्यासः}
\shrota{जनमेजयः}
\notes{This passage describes how the sage Narada appeared to console the grieving Rama, revealed that Sita's abduction was destined for Ravana's destruction, instructed Rama to perform the nine-day Devi worship (Navaratri) in the month of Ashvin, during which the Goddess appeared and blessed Rama with the assurance that he would defeat Ravana with the help of the vānaras, after which Rama successfully built the bridge across the ocean and killed Ravana.}
\textlink{https://sa.wikisource.org/wiki/देवीभागवतपुराणम्/स्कन्धः_०३/अध्यायः_३०}
\translink{https://www.wisdomlib.org/hinduism/book/devi-bhagavata-purana/d/doc57163.html}

\storymeta

\uvacha{व्यास उवाच}

\twolineshloka
{एवं तौ संविदं कृत्वा यावत्तूष्णीं बभूवतुः}
{आजगाम तदाऽऽकाशान्नारदो भगवानृषिः}% ॥ १ ॥

\twolineshloka
{रणयन्महतीं वीणां स्वरग्रामविभूषिताम्}
{गायन्बृहद्रथं साम तदा तमुपतस्थिवान्}% ॥ २ ॥

\twolineshloka
{दृष्ट्वा तं राम उत्थाय ददावथ वृषं शुभम्}
{आसनं चार्घ्यपाद्यञ्च कृतवानमितद्युतिः}% ॥ ३ ॥

\twolineshloka
{पूजां परमिकां कृत्वा कृताञ्जलिरुपस्थितः}
{उपविष्टः समीपे तु कृताज्ञो मुनिना हरिः}% ॥ ४ ॥

\twolineshloka
{उपविष्टं तदा रामं सानुजं दुःखमानसम्}
{पप्रच्छ नारदः प्रीत्या कुशलं मुनिसत्तमः}% ॥ ५ ॥

\twolineshloka
{कथं राघव शोकार्तो यथा वै प्राकृतो नरः}
{हृतां सीतां च जानामि रावणेन दुरात्मना}% ॥ ६ ॥

\twolineshloka
{सुरसद्मगतश्चाहं श्रुतवाञ्जनकात्मजाम्}
{पौलस्त्येन हृतां मोहान्मरणं स्वमजानता}% ॥ ७ ॥

\twolineshloka
{तव जन्म च काकुत्स्थ पौलस्त्यनिधनाय वै}
{मैथिलीहरणं जातमेतदर्थं नराधिप}% ॥ ८ ॥

\twolineshloka
{पूर्वजन्मनि वैदेही मुनिपुत्री तपस्विनी}
{रावणेन वने दृष्टा तपस्यन्ती शुचिस्मिता}% ॥ ९ ॥

\twolineshloka
{प्रार्थिता रावणेनासौ भव भार्येति राघव}
{तिरस्कृतस्तयाऽसौ वै जग्राह कबरं बलात्}% ॥ १० ॥

\twolineshloka
{शशाप तत्क्षणं राम रावणं तापसी भृशम्}
{कुपिता त्यक्तुमिच्छन्ती देहं संस्पर्शदूषितम्}% ॥ ११ ॥

\twolineshloka
{दुरात्मंस्तव नाशार्थं भविष्यामि धरातले}
{अयोनिजा वरा नारी त्यक्त्वा देहं जहावपि}% ॥ १२ ॥

\twolineshloka
{सेयं रमांशसम्भूता गृहीता तेन रक्षसा}
{विनाशार्थं कुलस्यैव व्याली स्रगिव सम्भ्रमात्}% ॥ १३ ॥

\twolineshloka
{तव जन्म च काकुत्स्थ तस्य नाशाय चामरैः}
{प्रार्थितस्य हरेरंशादजवंशेऽप्यजन्मनः}% ॥ १४ ॥

\twolineshloka
{कुरु धैर्यं महाबाहो तत्र सा वर्ततेऽवशा}
{सती धर्मरता सीता त्वां ध्यायन्ती दिवानिशम्}% ॥ १५ ॥

\twolineshloka
{कामधेनुपयः पात्रे कृत्वा मघवता स्वयम्}
{पानार्थं प्रेषितं तस्याः पीतं चैवामृतं यथा}% ॥ १६ ॥

\twolineshloka
{सुरभीदुग्धपानात्सा क्षुत्तुड्‌दुःखविवर्जिता}
{जाता कमलपत्राक्षी वर्तते वीक्षिता मया}% ॥ १७ ॥

\twolineshloka
{उपायं कथयाम्यद्य तस्य नाशाय राघव}
{व्रतं कुरुष्व श्रद्धावानाश्विने मासि साम्प्रतम्}% ॥ १८ ॥

\twolineshloka
{नवरात्रोपवासञ्च भगवत्याः प्रपूजनम्}
{सर्वसिद्धिकरं राम जपहोमविधानतः}% ॥ १९ ॥

\twolineshloka
{मेघ्यैश्च पशुभिर्देव्या बलिं दत्त्वा विशंसितैः}
{दशांशं हवनं कृत्वा सशक्तस्त्वं भविष्यसि}% ॥ २० ॥

\twolineshloka
{विष्णुना चरितं पूर्वं महादेवेन ब्रह्मणा}
{तथा मघवता चीर्णं स्वर्गमध्यस्थितेन वै}% ॥ २१ ॥

\twolineshloka
{सुखिना राम कर्तव्यं नवरात्रव्रतं शुभम्}
{विशेषेण च कर्तव्यं पुंसा कष्टगतेन वै}% ॥ २२ ॥

\twolineshloka
{विश्वामित्रेण काकुत्स्थ कृतमेतन्न संशयः}
{भृगुणाऽथ वसिष्ठेन कश्यपेन तथैव च}% ॥ २३ ॥

\twolineshloka
{गुरुणा हृतदारेण कृतमेतन्महाव्रतम्}
{तस्मात्त्वं कुरु राजेन्द्र रावणस्य वधाय च}% ॥ २४ ॥

\twolineshloka
{इन्द्रेण वृत्रनाशाय कृतं व्रतमनुत्तमम्}
{त्रिपुरस्य विनाशाय शिवेनापि पुरा कृतम्}% ॥ २५ ॥

\twolineshloka
{हरिणा मधुनाशाय कृतं मेरौ महामते}
{विधिवत्कुरु काकुत्स्थ व्रतमेतदतन्द्रितः}% ॥ २६ ॥

\uvacha{श्रीराम उवाच}


\twolineshloka
{का देवी किं प्रभावा सा कुतो जाता किमाह्वया}
{व्रतं किं विधिवद्‌ब्रूहि सर्वज्ञोऽसि दयानिधे}% ॥ २७ ॥

\uvacha{नारद उवाच}


\twolineshloka
{शृणु राम सदा नित्या शक्तिराद्या सनातनी}
{सर्वकामप्रदा देवी पूजिता दुःखनाशिनी}% ॥ २८ ॥

\twolineshloka
{कारणं सर्वजन्तूनां ब्रह्मादीनां रघूद्वह}
{तस्याः शक्तिं विना कोऽपि स्पन्दितुं न क्षमो भवेत्}% ॥ २९ ॥

\twolineshloka
{विष्णोः पालनशक्तिः सा कर्तृशक्तिः पितुर्मम}
{रुद्रस्य नाशशक्तिः सा त्वन्याशक्तिः परा शिवा}% ॥ ३० ॥

\twolineshloka
{यच्च किञ्चित्क्वचिद्वस्तु सदसद्‌भुवनत्रये}
{तस्य सर्वस्य या शक्तिस्तदुत्पत्तिः कुतो भवेत्}% ॥ ३१ ॥

\twolineshloka
{न ब्रह्मा न यदा विष्णुर्न रुद्रो न दिवाकरः}
{न चेन्द्राद्याः सुराः सर्वे न धरा न धराधराः}% ॥ ३२ ॥

\twolineshloka
{तदा सा प्रकृतिः पूर्णा पुरुषेण परेण वै}
{संयुता विहरत्येव युगादौ निर्गुणा शिवा}% ॥ ३३ ॥

\twolineshloka
{सा भूत्वा सगुणा पश्चात्करोति भुवनत्रयम्}
{पूर्वं संसृज्य ब्रह्मादीन्दत्त्वा शक्तीश्च सर्वशः}% ॥ ३४ ॥

\twolineshloka
{तां ज्ञात्वा मुच्यते जन्तुर्जन्मसंसारबन्धनात्}
{सा विद्या परमा ज्ञेया वेदाद्या वेदकारिणी}% ॥ ३५ ॥

\twolineshloka
{असङ्ख्यातानि नामानि तस्या ब्रह्मादिभिः किल}
{गुणकर्मविधानैस्तु कल्पितानि च किं ब्रुवे}% ॥ ३६ ॥

\twolineshloka
{अकारादिक्षकारान्तैः स्वरैर्वर्णैस्तु योजितैः}
{असङ्ख्येयानि नामानि भवन्ति रघुनन्दन}% ॥ ३७ ॥


\uvacha{राम उवाच}


\twolineshloka
{विधिं मे ब्रूहि विप्रर्षे व्रतस्यास्य समासतः}
{करोम्यद्यैव श्रद्धावाञ्छ्रीदेव्याः पूजनं तथा}% ॥ ३८ ॥

\uvacha{नारद उवाच}


\twolineshloka
{पीठं कृत्वा समे स्थाने संस्थाप्य जगदम्बिकाम्}
{उपवासान्नवैव त्वं कुरु राम विधानतः}% ॥ ३९ ॥

\twolineshloka
{आचार्योऽहं भविष्यामि कर्मण्यस्मिन्महीपते}
{देवकार्यविधानार्थमुत्साहं प्रकरोम्यहम्}% ॥ ४० ॥

\uvacha{व्यास उवाच}


\twolineshloka
{तच्छ्रुत्वा वचनं सत्यं मत्वा रामः प्रतापवान्}
{कारयित्वा शुभं पीठं स्थापयित्वाम्बिकां शिवाम्}% ॥ ४१ ॥

\twolineshloka
{विधिवत्पूजनं तस्याश्चकार व्रतवान् हरिः}
{सम्प्राप्ते चाश्विने मासि तस्मिन्गिरिवरे तदा}% ॥ ४२ ॥

\twolineshloka
{उपवासपरो रामः कृतवान्व्रतमुत्तमम्}
{होमञ्च विधिवत्तत्र बलिदानञ्च पूजनम्}% ॥ ४३ ॥

\twolineshloka
{भ्रातरौ चक्रतुः प्रेम्णा व्रतं नारदसम्मतम्}
{अष्टम्यां मध्यरात्रे तु देवी भगवती हि सा}% ॥ ४४ ॥

\twolineshloka
{सिंहारूढा ददौ तत्र दर्शनं प्रतिपूजिता}
{गिरिशृङ्गे स्थितोवाच राघवं सानुजं गिरा}% ॥ ४५ ॥

\onelineshloka*
{मेघगम्भीरया चेदं भक्तिभावेन तोषिता}

\uvacha{देव्युवाच}

\onelineshloka
{राम राम महाबाहो तुष्टाऽस्म्यद्म व्रतेन ते}% ॥ ४६ ॥

\twolineshloka
{प्रार्थयस्व वरं कामं यत्ते मनसि वर्तते}
{नारायणांशसम्भूतस्त्वं वंशे मानवेऽनघे}% ॥ ४७ ॥

\twolineshloka
{रावणस्य वधायैव प्रार्थितस्त्वमरैरसि}
{पुरा मत्स्यतनुं कृत्वा हत्वा घोरञ्च राक्षसम्}% ॥ ४८ ॥

\twolineshloka
{त्वया वै रक्षिता वेदाः सुराणां हितमिच्छता}
{भूत्वा कच्छपरूपस्तु धृतवान्मन्दरं गिरिम्}% ॥ ४९ ॥

\twolineshloka
{अकूपारं प्रमन्थानं कृत्वा देवानपोषयः}
{कोलरूपं परं कृत्वा दशनाग्रेण मेदिनीम्}% ॥ ५० ॥

\twolineshloka
{धृतवानसि यद्‌राम हिरण्याक्षं जघान च}
{नारसिंहीं तनुं कृत्वा हिरण्यकशिपुं पुरा}% ॥ ५१ ॥

\twolineshloka
{प्रह्लादं राम रक्षित्वा हतवानसि राघव}
{वामनं वपुरास्थाय पुरा छलितवान्बलिम्}% ॥ ५२ ॥

\twolineshloka
{भूत्वेन्द्रस्यानुजः कामं देवकार्यप्रसाधकः}
{जमदग्निसुतस्त्वं मे विष्णोरंशेन सङ्गतः}% ॥ ५३ ॥

\twolineshloka
{कृत्वान्तं क्षत्रियाणां तु दानं भूमेरदाद्‌द्विजे}
{तथेदानीं तु काकुत्स्थ जातो दशरथात्मज}% ॥ ५४ ॥

\twolineshloka
{प्रार्थितस्तु सुरैः सर्वै रावणेनातिपीडितैः}
{कपयस्ते सहाया वै देवांशा बलवत्तराः}% ॥ ५५ ॥

\twolineshloka
{भविष्यन्ति नरव्याघ्र मच्छक्तिसंयुता ह्यमी}
{शेषांशोऽप्यनुजस्तेऽयं रावणात्मजनाशकः}% ॥ ५६ ॥

\twolineshloka
{भविष्यति न सन्देहः कर्तव्योऽत्र त्वयाऽनघ}
{वसन्ते सेवनं कार्यं त्वया तत्रातिश्रद्धया}% ॥ ५७ ॥

\twolineshloka
{हत्वाऽथ रावणं पापं कुरु राज्यं यथासुखम्}
{एकादश सहस्राणि वर्षाणि पृथिवीतले}% ॥ ५८ ॥

\onelineshloka*
{कृत्वा राज्यं रघुश्रेष्ठ गन्ताऽसि त्रिदिवं पुनः}

\uvacha{व्यास उवाच}

\onelineshloka
{इत्युक्त्वान्तर्दधे देवी रामस्तु प्रीतमानसः}% ॥ ५९ ॥

\twolineshloka
{समाप्य तद्‌व्रतं चक्रे प्रयाणं दशमीदिने}
{विजयापूजनं कृत्वा दत्त्वा दानान्यनेकशः}% ॥ ६० ॥

\fourlineindentedshloka
{कपिपतिबलयुक्तः सानुजः श्रीपतिश्च}
{प्रकटपरमशक्त्या प्रेरितः पूर्णकामः}
{उदधितटगतोऽसौ सेतुबन्धं विधाया-}
{प्यहनदमरशत्रुं रावणं गीतकीर्तिः}% ॥ ६१ ॥

\twolineshloka
{यः शृणोति नरो भक्त्या देव्याश्चरितमुत्तमम्}
{स भुक्त्वा विपुलान्भोगान्प्राप्नोति परमं पदम्}% ॥ ६२ ॥

\twolineshloka
{सन्त्यन्यानि पुराणानि विस्तराणि बहूनि च}
{श्रीमद्‌भागवतस्यास्य न तुल्यानीति मे मतिः}% ॥ ६३ ॥

॥इति श्रीदेवीभागवते महापुराणेऽष्टादशसाहस्र्यां संहितायां तृतीयस्कन्धे रामाय देवीवरदानं नाम त्रिंशोऽध्यायः॥

\closesection