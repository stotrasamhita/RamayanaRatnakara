\chapt{देवी-भागवतम्}

\src{देवी-भागवतम्}{तृतीयः स्कन्धः}{अध्यायाः २८}{श्लोकाः १--६९}
\vakta{व्यासः}
\shrota{जनमेजयः}
\tags{concise, complete}
\notes{This passage from the Devi Bhagavatam describes how Rama and Lakshmana accompanied sage Vishvamitra, killed demons including Tadaka, led to Rama winning Sita by breaking Shiva's bow, their exile to the forest due to Queen Kaikeyi's demands, and culminates with Ravana disguising himself to approach Sita while She was alone after tricking Rama and Lakshman away.}
\textlink{https://sa.wikisource.org/wiki/देवीभागवतपुराणम्/स्कन्धः_०३/अध्यायः_२८}
\translink{}

\storymeta

\sect{रामचरित्रवर्णनम्}


\uvacha{जनमेजय उवाच}


\twolineshloka
{कथं रामेण तच्चीर्णं व्रतं देव्याः सुखप्रदम्}
{राज्यभ्रष्टः कथं सोऽथ कथं सीता हृता पुनः}% ॥ १ ॥

\uvacha{व्यास उवाच}


\twolineshloka
{राजा दशरथः श्रीमानयोध्याधिपतिः पुरा}
{सूर्यवंशधरश्चासीद्देवब्राह्मणपूजकः}% ॥ २ ॥

\twolineshloka
{चत्वारो जज्ञिरे तस्य पुत्रा लोकेषु विश्रुताः}
{रामलक्ष्मणशत्रुघ्ना भरतश्चेति नामतः}% ॥ ३ ॥

\twolineshloka
{राज्ञः प्रियकराः सर्वे सदृशा गुणरूपतः}
{कौसल्यायाः सुतो रामः कैकेय्या भरतः स्मृतः}% ॥ ४ ॥

\twolineshloka
{सुमित्रातनयौ जातौ यमलौ द्वौ मनोहरौ}
{ते जाता वै किशोराश्च धनुर्बाणधराः किल}% ॥ ५ ॥

\twolineshloka
{सूनवः कृतसंस्कारा भूपतेः सुखवर्धकाः}
{कौशिकेन तदाऽऽगत्य प्रार्थितो रघुनन्दनः}% ॥ ६ ॥

\twolineshloka
{राघवं मखरक्षार्थं सूनुं षोडशवार्षिकम्}
{तस्मै सोऽयं ददौ रामं कौशिकाय सलक्ष्मणम्}% ॥ ७ ॥

\twolineshloka
{तौ समेत्य मुनिं मार्गे जग्मतुश्चारुदर्शनौ}
{ताटका निहता मार्गे राक्षसी घोरदर्शना}% ॥ ८ ॥

\twolineshloka
{रामेणैकेन बाणेन मुनीनां दुःखदा सदा}
{यज्ञरक्षा कृता तत्र सुबाहुर्निहतः शठः}% ॥ ९ ॥

\twolineshloka
{मारीचोऽथ मृतप्रायो निक्षिप्तो बाणवेगतः}
{एवं कृत्वा महत्कर्म यज्ञस्य परिरक्षणम्}% ॥ १० ॥

\twolineshloka
{गतास्ते मिथिलां सर्वे रामलक्ष्मणकौशिकाः}
{अहल्या मोचिता शापान्निष्पापा सा कृताऽबला}% ॥ ११ ॥

\twolineshloka
{विदेहनगरे तौ तु जग्मतुर्मुनिना सह}
{बभञ्ज शिवचापञ्च जनकेन पणीकृतम्}% ॥ १२ ॥

\twolineshloka
{उपयेमे ततः सीतां जानकीञ्च रमांशजाम्}
{लक्ष्मणाय ददौ राजा पुत्रीमेकां तथोर्मिलाम्}% ॥ १३ ॥

\twolineshloka
{कुशध्वजसुते कन्ये प्रापतुर्भ्रातरावुभौ}
{तथा भरतशत्रुघ्नौ सुशिलौ शुभलक्षणौ}% ॥ १४ ॥

\twolineshloka
{एवं दारक्रियास्तेषां भ्रातॄणां चाभवन्नृप}
{चतुर्णां मिथिलायां तु यथाविधि विधानतः}% ॥ १५ ॥

\twolineshloka
{राज्ययोग्यं सुतं दृष्ट्वा राजा दशरथस्तदा}
{राघवाय धुरं दातुं मनश्चक्रे निजाय वै}% ॥ १६ ॥

\twolineshloka
{सम्भारं विहितं दृष्ट्वा कैकेयी पूर्वकल्पितौ}
{वरौ सम्प्रार्थयामास भर्तारं वशवर्तिनम्}% ॥ १७ ॥

\twolineshloka
{राज्यं सुताय चैकेन भरताय महात्मने}
{रामाय वनवासञ्च चतुर्दशसमास्तथा}% ॥ १८ ॥

\twolineshloka
{रामस्तु वचनात्तस्याः सीतालक्ष्मणसंयुतः}
{जगाम दण्डकारण्यं राक्षसैरुपसेवितम्}% ॥ १९ ॥

\twolineshloka
{राजा दशरथः पुत्रविरहेण प्रपीडितः}
{जहौ प्राणानमेयात्मा पूर्वशापमनुस्मरन्}% ॥ २० ॥

\twolineshloka
{भरतः पितरं दृष्ट्वा मृतं मातृकृतेन वै}
{राज्यमृद्धं न जग्राह भ्रातुः प्रियचिकीर्षया}% ॥ २१ ॥

\twolineshloka
{पञ्चवट्यां वसन् रामो रावणावरजां वने}
{शूर्पणखां विरूपां वै चकारातिस्मरातुराम्}% ॥ २२ ॥

\twolineshloka
{खरादयस्तु तां दृष्ट्वा छिन्ननासां निशाचराः}
{चक्रुः सङ्ग्राममतुलं रामेणामिततेजसा}% ॥ २३ ॥

\twolineshloka
{स जघान खरादींश्च दैत्यानतिबलान्वितान्}
{मुनीनां हितमन्विच्छन् रामः सत्यपराक्रमः}% ॥ २४ ॥

\twolineshloka
{गत्वा शूर्पणखा लङ्कां खरदूषणघातनम्}
{दूषिता कथयामास रावणाय च राघवात्}% ॥ २५ ॥

\twolineshloka
{सोऽपि श्रुत्वा विनाशं तं जातः क्रोधवशः खलः}
{जगाम रथमारुह्य मारीचस्याश्रमं तदा}% ॥ २६ ॥

\twolineshloka
{कृत्वा हेममृगं नेतुं प्रेषयामास रावणः}
{सीताप्रलोभनार्थाय मायाविनमसम्भवम्}% ॥ २७ ॥

\twolineshloka
{सोऽथ हेममृगो भूत्वा सीतादृष्टिपथं गतः}
{मायावी चातिचित्राङ्गश्चरन्प्रबलमन्तिके}% ॥ २८ ॥

\twolineshloka
{तं दृष्ट्वा जानकी प्राह राघवं दैवनोदिता}
{चर्मानयस्व कान्तेति स्वाधीनपतिका यथा}% ॥ २९ ॥

\twolineshloka
{अविचार्याथ रामोऽपि तत्र संस्थाप्य लक्ष्मणम्}
{सशरं धनुरादाय ययौ मृगपदानुगः}% ॥ ३० ॥

\twolineshloka
{सारङ्गोऽपि हरिं दृष्ट्वा मायाकोटिविशारदः}
{दृश्यादृश्यो बभूवाथ जगाम च वनान्तरम्}% ॥ ३१ ॥

\twolineshloka
{मत्वा हस्तगतं रामः क्रोधाकृष्टधनुः पुनः}
{जघान चातितीक्ष्णेन शरेण कृत्रिमं मृगम्}% ॥ ३२ ॥

\twolineshloka
{स हतोऽतिबलात्तेन चुक्रोश भृशदुःखितः}
{हा लक्ष्मण हतोऽस्मीति मायावी नश्वरः खलः}% ॥ ३३ ॥

\twolineshloka
{स शब्दस्तुमुलस्तावज्जानक्या संश्रुतस्तदा}
{राघवस्येति सा मत्वा दीना देवरमब्रवीत्}% ॥ ३४ ॥

\twolineshloka
{गच्छ लक्ष्मण तूर्णं त्वं हतोऽसौ रघुनन्दनः}
{त्वामाह्वयति सौ‌मित्रे साहाय्यं कुरु सत्वरम्}% ॥ ३५ ॥

\twolineshloka
{तत्राह लक्ष्मणः सीतामम्ब रामवधादपि}
{नाहं गच्छेऽद्य मुक्त्वा त्वामसहायामिहाश्रमे}% ॥ ३६ ॥

\twolineshloka
{आज्ञा मे राघवस्यात्र तिष्ठेति जनकात्मजे}
{तदतिक्रमभीतोऽहं न त्यजामि तवान्तिकम्}% ॥ ३७ ॥

\twolineshloka
{दूरं वै राघवं दृष्ट्वा वने मायाविना किल}
{त्यक्त्वा त्वां नाधिगच्छामि पदमेकं शुचिस्मिते}% ॥ ३८ ॥

\twolineshloka
{कृरु धैर्यं न मन्येऽद्य रामं हन्तुं क्षमं क्षिप्तौ}
{नाहं त्यक्त्वा गमिष्यामि विलङ्घ्य रामभाषितम्}% ॥ ।३९ ॥

\uvacha{व्यास उवाच}


\twolineshloka
{रुदती सुदती प्राह ते तदा विधिनोदिता}
{अक्रूरा वचनं क्रूरं लक्ष्मणं शुभलक्षणम्}% ॥ ४० ॥

\twolineshloka
{अहं जानामि सौ‌मित्रे सानुरागं च मां प्रति}
{प्रेरितं भरतेनैव मदर्थमिह सङ्गतम्}% ॥ ४१ ॥

\twolineshloka
{नाहं तथाविधा नारी स्वैरिणी कुहकाधम}
{मृते रामे पतिं त्वां न कर्तुमिच्छामि कामतः}% ॥ ४२ ॥

\twolineshloka
{नागमिष्यति चेद्रामो जीवितं सन्त्यजाम्यहम्}
{विना तेन न जीवामि विधुरा दुःखिता भृशम्}% ॥ ४३ ॥

\twolineshloka
{गच्छ वा तिष्ठ सौ‍मित्रे न जानेऽहं तवेप्सितम्}
{क्व गतं तेऽद्य सौहार्दं ज्येष्ठे धर्मरते किल}% ॥ ४४ ॥

\twolineshloka
{तच्छ्रुत्वा वचनं तस्या लक्ष्मणो दीनमानसः}
{प्रोवाच रुद्धकण्ठस्तु तां तदा जनकात्मजाम्}% ॥ ४५ ॥

\twolineshloka
{किमात्थ क्षितिजे वाक्यं मयि क्रूरतरं किल}
{किं वदस्यत्यनिष्टं ते भावि जाने धिया ह्यहम्}% ॥ ४६ ॥

\twolineshloka
{इत्युक्त्वा निर्ययौ वीरस्तां त्यक्त्वा प्ररुदन्भृशम्}
{अग्रजस्य ययौ पश्यञ्छोकार्तः पृथिवीपते}% ॥ ४७ ॥

\twolineshloka
{गतेऽथ लक्ष्मणे तत्र रावणः कपटाकृतिः}
{भिक्षुवेषं ततः कृत्वा प्रविवेश तदाश्रमे}% ॥ ४८ ॥

\twolineshloka
{जानकी तं यतिं मत्वा दत्त्वार्घ्यं वन्यमादरात्}
{भैक्ष्यं समर्पयामास रावणाय दुरात्मने}% ॥ ४९ ॥

\twolineshloka
{तां पप्रच्छ स दुष्टात्मा नम्रपूर्वं मृदुस्वरम्}
{काऽसि पद्मपलाशाक्षि वने चैकाकिनी प्रिये}% ॥ ५० ॥

\twolineshloka
{पिता कस्तेऽथ वामोरु भ्राता कः कः पतिस्तव}
{मूढेवैकाकिनी चात्र स्थिताऽसि वरवर्णिनि}% ॥ ५१ ॥

\twolineshloka
{निर्जने विपिने किं त्वं सौधार्हा त्वमसि प्रिये}
{उटजे मुनिपत्‍नीवद्देवकन्यासमप्रभा}% ॥ ५२ ॥

\uvacha{व्यास उवाच}


\twolineshloka
{इति तद्वचनं श्रुत्वा प्रत्युवाच विदेहजा}
{दिव्यं दिष्ट्या यतिं ज्ञात्वा मन्दोदर्याः पतिं तदा}% ॥ ५३ ॥

\twolineshloka
{राजा दशरथः श्रीमांश्चत्वारस्तस्य वै सुताः}
{तेषां ज्येष्ठः पतिर्मेऽस्ति रामनामेति विश्रुतः}% ॥ ५४ ॥

\twolineshloka
{विवासितोऽथ कैकेय्या कृते भूपतिना वरे}
{चतुर्दश समा रामो वसतेऽत्र सलक्ष्मणः}% ॥ ५५ ॥

\twolineshloka
{जनकस्य सुता चाहं सीतानाम्नीति विश्रुता}
{भङ्क्त्वा शैवं धनुः कामं रामेणाहं विवाहिता}% ॥ ५६ ॥

\twolineshloka
{रामबाहुबलेनात्र वसामो निर्भया वने}
{काञ्चनं मृगमालोक्य हन्तुं मे निर्गतः पतिः}% ॥ ५७ ॥

\twolineshloka
{लक्ष्मणोऽपि पुनः श्रुत्वा रवं भ्रातुर्गतोऽधुना}
{तयोर्बाहुबलादत्र निर्भयाऽहं वसामि वै}% ॥ ५८ ॥

\twolineshloka
{मयेदं कथितं सर्वं वृत्तान्तं वनवासके}
{तेऽत्रागत्यार्हणां ते वै करिष्यन्ति यथाविधि}% ॥ ५९ ॥

\twolineshloka
{यतिर्विष्णुस्वरूपोऽसि तस्मात्त्वं पूजितो मया}
{आश्रमो विपिने घोरे कृतोऽस्ति रक्षसां कुले}% ॥ ६० ॥

\twolineshloka
{तस्मात्त्वां परिपृच्छामि सत्यं ब्रूहि ममाग्रतः}
{कोऽसि त्रिदण्डिरूपेण विपिने त्वं समागतः}% ॥ ६१ ॥


\uvacha{रावण उवाच}


\twolineshloka
{लङ्केशोऽहं मरालाक्षि श्रीमान्मन्दोदरीपतिः}
{त्वत्कृते तु कृतं रूपं मयेत्थं शोभनाकृते}% ॥ ६२ ॥

\twolineshloka
{आगतोऽहं वरारोहे भगिन्या प्रेरितोऽत्र वै}
{जनस्थाने हतौ श्रुत्वा भ्रातरौ खरदूषणौ}% ॥ ६३ ॥

\twolineshloka
{अङ्गीकुरु नृपं मां त्वं त्यक्त्वा तं मानुषं पतिम्}
{हृतराज्यं गतश्रीकं निर्बलं वनवासिनम्}% ॥ ६४ ॥

\twolineshloka
{पट्टराज्ञी भव त्वं मे मन्दोदर्युपरि स्फुटम्}
{दासोऽस्मि तव तन्वङ्‌गि स्वामिनी भव भामिनि}% ॥ ६५ ॥

\twolineshloka
{जेताऽहं लोकपालानां पतामि तव पादयोः}
{करं गृहाण मेऽद्य त्वं सनाथं कुरु जानकि}% ॥ ६६ ॥

\twolineshloka
{पिता ते याचितः पूर्वं मया वै त्वत्कृतेऽबले}
{जनको मामुवाचेत्थं पणबन्धो मया कृतः}% ॥ ६७ ॥

\twolineshloka
{रुद्रचापभयान्नाहं सम्प्राप्तस्तु स्वयंवरे}
{मनो मे संस्थितं तावन्निमग्नं विरहातुरम्}% ॥ ६८ ॥

\twolineshloka
{वनेऽत्र संस्थितां श्रुत्वा पूर्वानुरागमोहितः}
{आगतोऽस्म्यसितापाङ्‌गि सफलं कुरु मे श्रमम्}% ॥ ६९ ॥


॥इति श्रीदेवीभागवते महापुराणेऽष्टादशसाहस्र्यां संहितायां तृतीयस्कन्धे रामचरित्रवर्णनं नाम अष्टाविंशोऽध्यायः॥
