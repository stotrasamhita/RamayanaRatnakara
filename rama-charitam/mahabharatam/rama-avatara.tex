\chapt{महाभारतम्}

\sect{भीष्मेण रामावतारकथनम्}

\src{श्रीमन्महाभारतम्}{सभा-पर्व}{अर्घाहरणपर्व}{अध्यायः ५०}
\vakta{भीष्मः}
\shrota{युधिष्ठिरादयः}
\tags{concise, complete}
\notes{Bhishma narrates the story of Rāma, while recounting the greatness of Vishnu. He goes on to then talk about Krishna, and subsequently, Krishnavatara too.}
% \textlink{http://stotrasamhita.net/wiki/Narayaniyam/Dashaka_34}
\translink{}

\storymeta


\uvacha{भीष्म उवाच}

\twolineshloka
{शृणु राजंस्ततो विष्णोः प्रादुर्भावं महात्मनः}
{अष्टाविंशे युगे चापि मार्कण्डेयपुरः सरः}


\twolineshloka
{तिथौ नावमिके जज्ञे तथा दशरथादपि}
{कृत्वाऽऽत्मानं महाबाहुश्चतुर्धा विष्णुरव्ययः}


\twolineshloka
{लोके राम इति ख्यातस्तेजसा भास्करोपमः}
{प्रसादनार्थं लोकस्य विष्णुस्तत्र सनातनः}


\twolineshloka
{धर्मार्थमेव कौन्तेय जज्ञे तत्र महायशाः}
{तमप्याहुर्मनुष्येन्द्रं सर्वभूतपतेस्तनुम्}


\twolineshloka
{यज्ञविघ्नकरस्तत्र विश्वामित्रस्य भारत}
{सुबाहुर्निहतस्तेन मारीचस्ताडितो भृशम्}


\twolineshloka
{तस्मै दत्तानि चास्राणि विश्वमित्रेण धीमता}
{वधार्थं सर्वशत्रूणां दुर्वाराणि सुरैरपि}


\twolineshloka
{वर्तमाने महायज्ञे जनकस्य महात्मनः}
{भग्नं माहेश्वरं चापं क्रीडता लीलया भृशम्}


\twolineshloka
{ततस्तु सीतां जग्राह भार्यार्थे जानकीं विभुः}
{नगरीं पुनरासाद्य मुमुदे तत्र सीतया}


\twolineshloka
{कस्यचित्त्वथ कालस्य पित्रा तत्राभिचोदितः}
{कैकेय्याः प्रियमन्विच्छन्वनमभ्यवपद्यत}


\twolineshloka
{यः समाः सर्वधर्मज्ञश्चतुर्दश वने वसन्}
{लक्ष्मणानुचरो रामः सर्वभूतहिते रतः}


\twolineshloka
{चतुर्दश वने तीर्त्वा तदा वर्षाणि भारत}
{रूपिणी यस्य पार्श्वस्था सीतेत्यभिहिता जनैः}


\twolineshloka
{पूर्वोचितत्वात्सा लक्ष्मीर्भर्तारमनुशोचति}
{जनस्थाने वसन्कार्यं त्रिदशानां चकार सः}


\twolineshloka
{मारीचं दूषणं हत्वा खरं त्रिशिरसं तथा}
{चतुर्दश सहस्राणि रक्षसां घोरकर्मणाम्}


\twolineshloka
{जघान रामो धर्मात्मा प्रजानां हितकाम्यया}
{विराधं च कबन्धं च राक्षसौ घोरकर्मिणौ}


\twolineshloka
{जघान च तदा रामो गन्धर्वौ शापविक्षतौ}
{स रावणस्य भगिनी नासाच्छेदमकारयत्}


\twolineshloka
{भार्यावियोगं तं प्राप्य मृगयन्व्यचरद्वनम्}
{स तस्मादृश्यमूकं तु गत्वा पम्पामतीत्य च}


\twolineshloka
{सुग्रीवं मारुतिं दृष्ट्वा चक्रे मैत्रीं तयोः स वै}
{अथ गत्वा स किष्किन्धां सुग्रीवेण तदा सह}


\twolineshloka
{निहत्य वालिनं युद्धे वानरेन्द्रं महाबलम्}
{अभ्यषिञ्चत्तदा रामः सुग्रीवं वानरेश्वरम्}


\twolineshloka
{ततः स वीर्यवान्राजंस्त्वरया वै समुत्सुकः}
{विचित्य वायुपुत्रेण लङ्कादेशं निवेदितः}


\twolineshloka
{सेतुं बद्ध्वा समुद्रस्य वानरैः स समुत्सुकः}
{सीतायाः पदमन्विच्छन्रामो लङ्कां विवेश वै}


\twolineshloka
{देवोरगगणानां हि यक्षराक्षसपक्षिणाम्}
{तत्रावद्यं राक्षसेन्द्रं रावणं युधि दुर्जयम्}


\twolineshloka
{युक्तं राक्षसकोटीभिर्भिन्नाञ्जनचयोपमम्}
{दुर्निरीक्ष्यं सुरगणैर्वरदानेन दर्पितम्}


\onelineshloka
{जघान सचिवैः सार्धं सान्वयं रावणं रणे}


\twolineshloka
{त्रैलोक्यकण्टकं वीरं महाकायं महाबलम्}
{रावणं सगणं हत्वा रामो भूतपतिः पुरा}


\twolineshloka
{लङ्कायां तं महात्मानं राक्षसेन्द्रं विभीषणम्}
{अभिषिच्य ततो राम अमरत्वं ददौ तदा}


\twolineshloka
{आरुह्य पुष्पकं रामः सीतामादाय पाण्डव}
{सबलं स्वपुरं गत्वा धर्मराज्यमपालयत्}


\twolineshloka
{दानवो लवणो नाम मधोः पुत्रो महाबलः}
{शत्रुघ्नेन हतो राजंस्तदा रामस्य शासनात्}


\twolineshloka
{एवं बहूनि कर्माणि कृत्वा लोकहिताय सः}
{राजं चकार विधिवद्रामो धर्मभृतां वरः}


\twolineshloka
{शताश्वमेधानाजह्रे ज्योतिरुक्थ्यान्निरर्गलान्}
{नाश्रूयन्ताशुभा वाचो नात्ययः प्राणिनां तदा}


\twolineshloka
{न दस्युजं भयं चासीद्रामे राज्यं प्रशसति}
{ऋषीणां देवतानां च मनुष्याणां तथैव च}


\twolineshloka
{पृथिव्यां धार्मिकाः सर्वे रामे राज्यं प्रशासति}
{नाधर्मिष्ठो नरः कश्चिद्बभूव प्राणिनां क्वचित्}


\twolineshloka
{प्राणापानौ समौ ह्यास्तां रामे राज्यं प्रशासति}
{गाधामप्यत्र गायन्ति ये पुराणविदो जनाः}


\twolineshloka
{श्यामो युवा लोहिताक्षो मातङ्गानामिवर्षभः}
{आजानुबाहुः सुमुखः सिंहस्कन्धो महाबलः}


\twolineshloka
{दशवर्षसहस्राणि दशवर्षशतानि च}
{राज्यं भोगं च सम्प्राप्य शशास पृथिवीमिमाम्}


\twolineshloka
{रामो रामो राम इति प्राजानामभवन्कथाः}
{रामभूतं जगदिदं रामे राज्यं प्रशासति}


\twolineshloka
{ऋग्यजुः सामहीनाश्च न तदाऽसन्द्विजायः}
{उषित्वा दण्डके कार्यं त्रिदशानां चकार सः}


\twolineshloka
{पूर्वापकारिणं तं तु पौलस्त्यं मनुजर्षभम्}
{देवगन्धर्वनागानामरिं स निजघान ह}


\twolineshloka
{सत्ववान्गुणसम्पन्नो दीप्यमानः स्वतेजसा}
{एवमेव महाबाहुरिक्ष्वाकुकुलवर्धनः}


\twolineshloka
{रावणं सगणं हत्वा दिवमाक्रमताभिभूः}
{इति दाशरथेः ख्यातः प्रादुर्भावो महात्मनः}


\twolineshloka
{ततः कृष्णो महाबाहुर्भीतानामभयङ्करः}
{अष्टाविंशे युगे राजञ्जज्ञे श्रीवत्सलक्षणः}


\twolineshloka
{पेशलश्च वदान्यश्चलोके बहुमतो नृषु}
{स्मृतिमान्देशकालज्ञः शङ्खचक्रगदासिभृत्}


\twolineshloka
{वासुदेव इति ख्यातो लोकानां हितकृत्सदा}
{वृष्णीनां च कुले जातो भूमेः प्रियचिकीर्षया}


\twolineshloka
{शत्रूणां भयकृद्दाता मधुहेति स विश्रुतः}
{शकटार्जुनरामाणां कीलस्थानान्यसूदयत्}


\twolineshloka
{कंसादीन्निजघानाऽऽजौ दैत्यान्मानुषविग्रहान्}
{अयं लोकहितार्थाय प्रादुर्भावो महात्मनः}


\twolineshloka
{कल्की विष्णुयशा नाम भूयश्चोत्पत्स्यते हरिः}
{लेर्युगान्ते सम्प्राप्ते धर्मे शिथिलतां गते}


\twolineshloka
{पाषण्डिनां गणानां हि वधार्थं भरतर्षभ}
{धर्मस्य च विवृद्ध्यर्थं विप्राणां हितकाम्यया}


\twolineshloka
{एते चान्ये च बहवो विष्णोर्देवगणैर्युताः}
{प्रादुर्भावाः पुराणेषु गीयन्ते ब्रह्मवादिभिः}


॥इति श्रीमन्महाभारते सभापर्वणि अर्घाहरण-पर्वणि पञ्चाशोऽध्यायः॥६०॥

\closesection