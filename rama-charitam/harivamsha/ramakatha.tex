\chapt{हरिवंशः}

\src{हरिवंशः}{हरिवंशपर्व}{अध्यायः ४१}{श्लोकाः १२१--१५५}
\notes{This chapter is part of the Harivamsha Purana, which is a supplement to the Mahabharata. Along with the rest of the avataras, this extract from the 41st chapter narrates the life and exploits of Bhagavān Rāma.}
\textlink{https://sa.wikisource.org/wiki/हरिवंशपुराणम्/पर्व_१_(हरिवंशपर्व)/अध्यायः_४१}
\translink{https://www.wisdomlib.org/hinduism/book/harivamsha-purana-dutt/d/doc485519.html}

\storymeta


\sect{रामकथा}

\addtocounter{shlokacount}{120}

\twolineshloka
{चतुर्विंशे युगे चापि विश्वामित्रपुरस्सरः}
{राज्ञो दशरथस्याथ पुत्रः पद्मायतेक्षणः}%।। १२१।।

\twolineshloka
{कृत्वाऽऽत्मानं महाबाहुश्चतुर्धा प्रभुरीश्वरः}
{लोके राम इति ख्यातस्तेजसा भास्करोपमः}%।। १२२।।

\twolineshloka
{प्रसादनार्थं लोकस्य रक्षसां निधनाय च}
{धर्मस्य च विवृद्धद्यर्थं जज्ञे तत्र महायशाः}%।। १२३।।

\twolineshloka
{तमप्याहुर्मनुष्येन्द्रं सर्वभूतपतेस्तनुम्}
{यस्मै दत्तानि चास्त्राणि विश्वामित्रेण धीमता}%।। १२४।।

\twolineshloka
{वधार्थं देवशत्रूणां दुर्धराणि सुरैरपि}
{यज्ञविध्नकरो येन मुनीनां भावितात्मनाम्}%।। १२५।।

\twolineshloka
{मारीचश्च सुबाहुश्च बलेन बलिनां वरौ}
{निहतौ च निराशी च कृतौ तेन महात्मना}%।। १२६।।

\twolineshloka
{वर्तमाने मखे येन जनकस्य महात्मनः} 
{भग्नं माहेश्वरं चापं क्रीडता लीलया पुरा} %।। १२७।

\twolineshloka
{यः समाः सर्वधर्मज्ञश्चतुर्दश वनेऽवसत्}
{लक्ष्मणानुचरो रामः सर्वभूतहिते रतः}%।। १२८।।

\twolineshloka
{रूपिणी यस्य पार्श्वस्था सीतेति प्रथिता जनैः}
{पूर्वोचिता तस्य लक्ष्मीर्भर्तारमनुगच्छति}%।। १२९।।

\twolineshloka
{चतुर्दश तपस्तप्त्वा वने वर्षाणि राघवः}
{जनस्थाने वसन् कार्यं त्रिदशानां चकार ह}%।।१३०।।


\threelineshloka
{सीतायाः पदमन्विच्छल्लँक्ष्मणानुचरो विभुः}
{विराधं च कबन्धं च राक्षसौ भीमविक्रमौ}
{जघान पुरुषव्याघ्रौ गन्धर्वौ शापवीक्षितौ}%।। १३१।।

\twolineshloka
{हुताशनार्केन्दुतडिद्घनाभैः प्रतप्तजाम्बूनदचित्रपुङ्खैः}
{महेन्द्रवज्राशनितुल्यसारैः शरैः शरीरेण वियोजितौ बलात्}%।। १३२।।

\twolineshloka
{सुग्रीवस्य कृते येन वानरेन्द्रो महाबलः}
{वाली विनिहतो युद्धे सुग्रीवश्चाभिषेचितः}%।। १३३।।

\twolineshloka
{देवासुरगणानां हि यक्षगन्धर्वभोगिनाम्}
{अवध्यं राक्षसेन्द्रं तं रावणं युद्धदुर्मदम्}%।। १३४।।

\twolineshloka
{युक्तं राक्षसकोटीभिर्नीलाञ्जनचयोपमम्}
{त्रैलोक्यरावणं घोरं रावणं राक्षसेश्वरम्}%।। १३५।।

\twolineshloka
{दुर्जयं दुर्धरं दृप्तं शार्दूलसमविक्रमम्}
{दुर्निरीक्ष्यं सुरगणैर्वरदानेन दर्पितम्}%।।१३६।।

\twolineshloka
{जघान सचिवैः सार्द्धं ससैन्यं रावणं युधि}
{महाभ्रघनसङ्काशं महाकायं महाबलम्}%।। १३७।।

\twolineshloka
{तमागस्कारिणं घोरं पौलस्त्यं युधि दुर्जयम्}
{सभ्रातृपुत्रसचिवं ससैन्यं क्रूरनिश्चयम्}%।। १३८ ।।

\twolineshloka
{रावणं निजघानाशु रामो भूतपतिः पुरा}
{मधोश्च तनयो दृप्तो लवणो नाम दानवः}%।।१३९।।

\twolineshloka
{हतो मधुवने वीरो वरदृप्तो महासुरः}
{समरे युद्धशौण्डेन तथा चान्येऽपि राक्षसाः}%।।१४०।।

\twolineshloka
{एतानि कृत्वा कर्माणि रामो धर्मभृतां वरः}
{दशाश्वमेधाञ्जारूथ्यानाजहार निरर्गलान्}%।।१४१।।

\twolineshloka
{नाश्रूयन्ताशुभा वाचो नाकुलं मारुतो ववौ}
{न वित्तहरणं त्वासीद् रामे राज्यं प्रशासति}%।। १४२।।

\twolineshloka
{पर्यदेवन्न विधवा नानर्थास्ताभवंस्तदा}
{सर्वमासीज्जगद् दान्तं रामे राज्यं प्रशासति}%।। १४३ ।।

\twolineshloka
{न प्राणिनां भयं चापि जलानलनिघातजम्}
{न च स्म वृद्धा बालानां प्रेतकार्याणि कुर्वते}%।। १४४।।


\threelineshloka
{ब्रह्म पर्यचरत् क्षत्र विशः क्षत्रमनुव्रताः}
{शूद्राश्चैव हि वर्णांस्त्रीञ्छुश्रूषन्त्यनहङ्कृताः}
{नार्यो नात्यचरन्भर्तॄन् भार्यां नात्यचरत् पतिः}%।। १४५।।

\twolineshloka
{सर्वमासीञ्जगद् दान्तं निर्दस्युरभवन्मही}
{राम एकोऽभवद् भर्त्ता रामः पालयिताभवत्}%।।१४६।।

\twolineshloka
{आयुर्वर्षसहस्राणि तथा पुत्रसहस्रिणः}
{अरोगाः प्राणिनश्चासन् रामे राज्यं प्रशासति}%।।१४७।।

\twolineshloka
{देवतानामृषीणां च मनुष्याणां च सर्वशः}
{पृथिव्यां समवायोऽभूद्रामे राज्यं प्रशासति}%।। १४८।।

\twolineshloka
{गाथा अप्यत्र गायन्ति ये पुराणविदो जनाः}
{रामे निबद्धतत्त्वार्था माहात्म्यं तस्य धीमतः}%।।१४९।।

\twolineshloka
{श्यामो युवा लोहिताक्षो दीप्तास्यो मितभाषिता}
{आजानुबाहुः सुमुखः सिंहस्कन्धो महाभुजः}%।।१५०।।

\twolineshloka
{दश वर्षसहस्राणि दश वर्षशतानि च}
{अयोध्याधिपतिर्भूत्वा रामो राज्यमकारयत्}%।। १५१।।

\twolineshloka
{ऋक्सामयजुषां घोषो ज्याघोषश्च महात्मनः}
{अव्युच्छिन्नोऽभवद्राज्ये दीयतां भुज्यतामिति}%।। १५२।।

\twolineshloka
{सत्त्ववान् गुणसम्पन्नो दीप्यमानः स्वतेजसा}
{अति चन्द्रं च सूर्यं च रामो दाशरथिर्बभौ}%।।१५३।।

\twolineshloka
{ईजे क्रतुशतैः पुण्यैः समाप्तवरदक्षिणैः}
{हित्वायोध्यां दिवं यातो राघवः समहाबलः}%।।१५४।।

\twolineshloka
{एवमेष महाबाहुरिक्ष्वाकुकुलनन्दनः}
{रावणं सगणं हत्वा दिवमाचक्रमे प्रभुः}%।। १५५।।

\ldots

॥इति श्रीमहाभारते खिलभागे हरिवंशे हरिवंशपर्वणि प्रादुर्भावानुसङ्ग्रहो नामैकचत्वारिंशोऽध्यायः॥४१॥


\closesection