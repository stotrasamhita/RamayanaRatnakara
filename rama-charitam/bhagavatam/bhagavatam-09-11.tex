\sect{श्रीरामोपाख्यानम्}

\src{श्रीमद्-भागवतम्}{नवमः स्कन्धः}{अध्यायः ११}{श्लोकाः १---३६}
\vakta{शुकः}
\shrota{परीक्षितः}
\tags{concise, complete}
\notes{This chapter summarises how Lord Rāmachandra exemplified supreme dharma through performing extensive yajñas, generous gifts to brāhmaṇas, deep love for His subjects, and painful renunciation of Sītādevī , ultimately culminating in His departure to Vaikuntham after establishing His sons in the kingdom.}
\textlink{http://stotrasamhita.net/wiki/Bhagavatam/Skandha_09/Adhyaya_11}
\translink{https://www.wisdomlib.org/hinduism/book/the-bhagavata-purana/d/doc1128850.html}

\storymeta

\uvacha{श्रीशुक उवाच}

\twolineshloka
{भगवानात्मनात्मानं राम उत्तमकल्पकैः}
{सर्वदेवमयं देवमीजेऽथाचार्यवान्मखैः} %1

\twolineshloka
{होत्रेऽददाद्दिशं प्राचीं ब्रह्मणे दक्षिणां प्रभुः}
{अध्वर्यवे प्रतीचीं वा उत्तरां सामगाय सः} %2

\twolineshloka
{आचार्याय ददौ शेषां यावती भूस्तदन्तरा}
{अन्यमान इदं कृत्स्नं ब्राह्मणोऽर्हति निःस्पृहः} %3

\twolineshloka
{इत्ययं तदलङ्कार वासोभ्यामवशेषितः}
{तथा राज्ञ्यपि वैदेही सौमङ्गल्यावशेषिता} %4

\twolineshloka
{ते तु ब्राह्मणदेवस्य वात्सल्यं वीक्ष्य संस्तुतम्}
{प्रीताः क्लिन्नधियस्तस्मै प्रत्यर्प्येदं बभाषिरे} %5

\twolineshloka
{अप्रत्तं नस्त्वया किं नु भगवन्भुवनेश्वर}
{यन्नोऽन्तर्हृदयं विश्य तमो हंसि स्वरोचिषा} %6

\twolineshloka
{नमो ब्रह्मण्यदेवाय रामायाकुण्ठमेधसे}
{उत्तमश्लोकधुर्याय न्यस्तदण्डार्पिताङ्घ्रये} %7

\twolineshloka
{कदाचिल्लोकजिज्ञासुर्गूढो रात्र्यामलक्षितः}
{चरन्वाचोऽशृणोद्रामो भार्यामुद्दिश्य कस्यचित्} %8

\twolineshloka
{नाहं बिभर्मि त्वां दुष्टामसतीं परवेश्मगाम्}
{स्त्रैणो हि बिभृयात्सीतां रामो नाहं भजे पुनः} %9

\twolineshloka
{इति लोकाद्बहुमुखाद्दुराराध्यादसंविदः}
{पत्या भीतेन सा त्यक्ता प्राप्ता प्राचेतसाश्रमम्} %10

\twolineshloka
{अन्तर्वत्न्यागते काले यमौ सा सुषुवे सुतौ}
{कुशो लव इति ख्यातौ तयोश्चक्रे क्रिया मुनिः} %11

\twolineshloka
{अङ्गदश्चित्रकेतुश्च लक्ष्मणस्यात्मजौ स्मृतौ}
{तक्षः पुष्कल इत्यास्तां भरतस्य महीपते} %12

\twolineshloka
{सुबाहुः श्रुतसेनश्च शत्रुघ्नस्य बभूवतुः}
{गन्धर्वान्कोटिशो जघ्ने भरतो विजये दिशाम्} %13

\threelineshloka
{तदीयं धनमानीय सर्वं राज्ञे न्यवेदय}
{शत्रुघ्नश्च मधोः पुत्रं लवणं नाम राक्षस}
{हत्वा मधुवने चक्रे मथुरां नाम वै पुरीम्॥१४} %14

\twolineshloka
{मुनौ निक्षिप्य तनयौ सीता भर्त्रा विवासिता}
{ध्यायन्ती रामचरणौ विवरं प्रविवेश ह} %15

\twolineshloka
{तच्छ्रुत्वा भगवान्रामो रुन्धन्नपि धिया शुचः}
{स्मरंस्तस्या गुणांस्तांस्तान्नाशक्नोद्रोद्धुमीश्वरः} %16

\twolineshloka
{स्त्रीपुम्प्रसङ्ग एतादृक्सर्वत्र त्रासमावहः}
{अपीश्वराणां किमुत ग्राम्यस्य गृहचेतसः} %17

\twolineshloka
{तत ऊर्ध्वं ब्रह्मचर्यं धार्यन्नजुहोत्प्रभुः}
{त्रयोदशाब्दसाहस्रमग्निहोत्रमखण्डितम्} %18

\twolineshloka
{स्मरतां हृदि विन्यस्य विद्धं दण्डककण्टकैः}
{स्वपादपल्लवं राम आत्मज्योतिरगात्ततः} %19

\fourlineindentedshloka
{नेदं यशो रघुपतेः सुरयाच्ञयात्त}
{लीलातनोरधिकसाम्यविमुक्तधाम्नः}
{रक्षोवधो जलधिबन्धनमस्त्रपूगैः}
{किं तस्य शत्रुहनने कपयः सहायाः} %20

\fourlineindentedshloka
{यस्यामलं नृपसदःसु यशोऽधुनापि}
{गायन्त्यघघ्नमृषयो दिगिभेन्द्रपट्टम्}
{तं नाकपालवसुपालकिरीटजुष्ट}
{पादाम्बुजं रघुपतिं शरणं प्रपद्ये} %21

\twolineshloka
{स यैः स्पृष्टोऽभिदृष्टो वा संविष्टोऽनुगतोऽपि वा}
{कोसलास्ते ययुः स्थानं यत्र गच्छन्ति योगिनः} %22

\twolineshloka
{पुरुषो रामचरितं श्रवणैरुपधारयन्}
{आनृशंस्यपरो राजन्कर्मबन्धैर्विमुच्यते} %23

\uvacha{श्रीराजोवाच}


\twolineshloka
{कथं स भगवान्रामो भ्रात्न्वा स्वयमात्मनः}
{तस्मिन्वा तेऽन्ववर्तन्त प्रजाः पौराश्च ईश्वरे} %24

\uvacha{श्रीबादरायणिरुवाच}


\twolineshloka
{अथादिशद्दिग्विजये भ्रात्ंस्त्रिभुवनेश्वरः}
{आत्मानं दर्शयन्स्वानां पुरीमैक्षत सानुगः} %25

\twolineshloka
{आसिक्तमार्गां गन्धोदैः करिणां मदशीकरैः}
{स्वामिनं प्राप्तमालोक्य मत्तां वा सुतरामिव} %26

\twolineshloka
{प्रासादगोपुरसभा चैत्यदेवगृहादिषु}
{विन्यस्तहेमकलशैः पताकाभिश्च मण्डिताम्} %27

\twolineshloka
{पूगैः सवृन्तै रम्भाभिः पट्टिकाभिः सुवाससाम्}
{आदर्शैरंशुकैः स्रग्भिः कृतकौतुकतोरणाम्} %28

\twolineshloka
{तमुपेयुस्तत्र तत्र पौरा अर्हणपाणयः}
{आशिषो युयुजुर्देव पाहीमां प्राक्त्वयोद्धृताम्} %29

\fourlineindentedshloka
{ततः प्रजा वीक्ष्य पतिं चिरागतं}
{दिदृक्षयोत्सृष्टगृहाः स्त्रियो नराः}
{आरुह्य हर्म्याण्यरविन्दलोचनम्}
{अतृप्तनेत्राः कुसुमैरवाकिरन्} %30

\twolineshloka
{अथ प्रविष्टः स्वगृहं जुष्टं स्वैः पूर्वराजभिः}
{अनन्ताखिलकोषाढ्यमनर्घ्योरुपरिच्छदम्} %31

\twolineshloka
{विद्रुमोदुम्बरद्वारैर्वैदूर्यस्तम्भपङ्क्तिभिः}
{स्थलैर्मारकतैः स्वच्छैर्भ्राजत्स्फटिकभित्तिभिः} %32

\twolineshloka
{चित्रस्रग्भिः पट्टिकाभिर्वासोमणिगणांशुकैः}
{मुक्ताफलैश्चिदुल्लासैः कान्तकामोपपत्तिभिः} %33

\twolineshloka
{धूपदीपैः सुरभिभिर्मण्डितं पुष्पमण्डनैः}
{स्त्रीपुम्भिः सुरसङ्काशैर्जुष्टं भूषणभूषणैः} %34

\twolineshloka
{तस्मिन्स भगवान्रामः स्निग्धया प्रिययेष्टया}
{रेमे स्वारामधीराणामृषभः सीतया किल} %35

\twolineshloka
{बुभुजे च यथाकालं कामान्धर्ममपीडयन्}
{वर्षपूगान्बहून्नॄणामभिध्याताङ्घ्रिपल्लवः} % 36


॥इति श्रीमद्भागवते महापुराणे पारमहंस्यां संहितायां नवमस्कन्धे श्रीरामोपाख्याने एकादशोऽध्यायः॥


\closesection