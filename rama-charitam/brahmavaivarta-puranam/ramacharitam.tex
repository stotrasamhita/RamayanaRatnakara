\chapt{ब्रह्मवैवर्त-पुराणम्}

\sect{श्रीरामचरितम्}

\src{ब्रह्मवैवर्त-पुराणम्}{खण्डः ४ (श्रीकृष्णजन्मखण्डः)}{अध्यायः ६२}{श्लोकाः १--१००}
% \tags{concise, complete}
\notes{This chapter describes the story of Rama in brief.}
\textlink{https://sa.wikisource.org/wiki/ब्रह्मवैवर्तपुराणम्/खण्डः_४_(श्रीकृष्णजन्मखण्डः)/अध्यायः_०६२}
\translink{https://archive.org/details/brahma-vaivarta-purana-all-four-kandas-english-translation/page/n1309/mode/2up}

\storymeta


\uvacha{नारद उवाच}

\twolineshloka
{ब्रह्मन् केन प्रकारेण रामो दाशरथिः स्वयम्}
{चकार मोक्षणं कुत्र युगे गौतमयोषितः}% ॥१॥


\twolineshloka
{रामावतारं सुखदं समासेन मनोहरम्}
{कथायस्व महाभाग श्रोतुं कौतूहलं मम}% ॥२॥

\uvacha{नारायण उवाच}

\twolineshloka
{ब्रह्मणा प्रार्थितो विष्णुर्जातो दशरथात्स्वयम्}
{कौसल्यायां च भगवांस्त्रेतायां च मृदाऽन्वितः}% ॥३॥


\twolineshloka
{कैकेय्यां भरतश्चैव रामतुल्यो गुणेन च}
{लक्ष्मणश्चापि शत्रुघ्नः सुमित्रायां गुणार्णवः}% ॥४॥


\twolineshloka
{पिश्वामित्रप्रेषितश्च श्रीरामश्च सलक्ष्मणः}
{प्रययौ मिथिलां रम्यां सीताग्रहणहेतवे}% ॥५॥


\twolineshloka
{दृष्ट्वा पाषाणारूपा च रामो वर्त्मनि कामिनीम्}
{विश्वामित्रं च पप्रच्छ कारणं जगदीश्वरः}% ॥६॥


\twolineshloka
{रामस्य वचनं श्रुत्वा विश्वामित्रो महातपाः}
{उवाच तत्र धर्मिष्ठो रहस्यं सर्वमेव च}% ॥७॥


\twolineshloka
{कारणं तन्मुखाच्छ्रृत्वा रामो भुवनपावनः}
{पस्पर्श पादाङ्गुलिना सा बभूव च पद्मिनी}% ॥८॥


\twolineshloka
{सा राममाशिषं कृत्वा प्रययौ भर्तृ मन्दिरम्}
{शुभाशिषं ददौ तस्मै भार्या सम्प्राप्य गौतमः}% ॥९॥


\twolineshloka
{रामश्च मिथिलां गत्वा धनुर्भङ्गं शिवस्या च}
{चकार पाणिग्रहणं सीतायाश्चैव नारद}% ॥१०॥


\twolineshloka
{कृत्वा विवाहं राजेन्द्रो भृगुदर्प निहत्य च}
{णयोध्यां प्रययौ रम्यां क्रीडाकौतुकमङ्गलैः}% ॥११॥


\twolineshloka
{राजा पुत्रं नृपं कर्तुमिषेय कृतसादरम्}
{सप्ततीर्थोदकं तूर्णमानीय मुनिपुङ्गवान्}% ॥१२॥


\twolineshloka
{कृताधिवासं श्रीरामं सर्व मङ्गलसंयुतम्}
{दृष्ट्वा भरतमाता च कैकेयी शोकविह्वला}% ॥१३॥


\twolineshloka
{वरयामास राजानं पूर्वमङ्गीकृतं वरम्}
{रामस्य वनवासं च राजत्वं भरतस्य च}% ॥१४॥


\twolineshloka
{वरं दातुं महाराजो नेयेष प्रेममोहितः}
{धर्मसत्यभयेनैवोवाच रामो नृपं सुधी}% ॥१५॥

\uvacha{श्रीराम उवाच}

\twolineshloka
{तडागशतदानेन यत्पुण्यं लभते नरः}
{ततोऽधिकं च लभते वापीदानेन निश्चितम्}% ॥१६॥


\twolineshloka
{दशवापीप्रदानेन यत्पुण्यं लभते नरः}
{ततोऽधिकं च लभते पुण्यं कन्याप्रदानतः}% ॥१७॥


\twolineshloka
{दशकन्याप्रदानेन यत्पुण्यं लभते नरः}
{ततोऽधिकं च लभते यज्ञैकेन नराधिप}% ॥१८॥


\twolineshloka
{शतयज्ञेन यत्पुण्यं लभते पुण्यकृज्जनः}
{ततोऽधिकं च लभते पुत्रास्यदर्शनेन च}% ॥१९॥


\twolineshloka
{दर्शने शतपुत्राणां यत्पुण्यं लभते नरः}
{तत्पुण्यं लभते नूनं पुण्यवान्सत्यबालनात्}% ॥२०॥


\twolineshloka
{न हि सत्यात्परो धर्मो नानृतात्पातकं परम्}
{न हि गह्घासमं तीर्यं म देवः केशवात्परः}% ॥२१॥


\twolineshloka
{नास्ति धर्मात्परो बन्धुर्नास्ति धर्मात्परं धनम्}
{धर्मात्प्रियः परः को वा स्पधर्मं रक्ष यत्नतः}% ॥२२॥


\twolineshloka
{स्वधर्मे रक्षिते तात शश्वत्सर्वत्र मङ्गलम्}
{यशस्यं सुप्रतिष्ठा च प्रतापः पूजनं परम्}% ॥२३॥


\twolineshloka
{यतुर्दशाब्दं धर्मेण त्यक्त्वा गृहसुखं म्रमन्}
{वनवासं करिष्यामि सत्यस्य पालनाय ते}% ॥२४॥


\twolineshloka
{कृत्वा सत्यं च शपथमिच्छयाऽनिच्छयाऽथवा}
{न कृर्यात्पालनं यो हि भस्मान्तं तस्य सूतकम्}% ॥२५॥


\twolineshloka
{कुम्भीपाके स पचति यावच्चन्द्रदिवाकरौ}
{ततो मूको भवेत्कुषठी मानवः सप्तजन्मसु}% ॥२६॥


\twolineshloka
{इत्योवमुक्त्वा श्रीरामो विधाय कल्कलं जटाम्}
{प्रययौ च महारण्ये सीतया लक्ष्मणेन च}% ॥२७॥


\twolineshloka
{पुत्रशोकान्महाराजास्तत्याज स्वतनुं मुने}
{पालनाय पितुः सत्यं रामो बभ्रामकानन}% ॥२८॥


\twolineshloka
{कालान्तरे महारण्ये भगिनी रावणस्य च}
{भ्रमन्ती कानने घोरे भर्त्रा सार्धं सुगौतुकात्}% ॥२९॥


\twolineshloka
{ददर्श रामं कुलटा कामार्ता राक्षसी तदा}
{पुलकाञ्चितसर्वाङ्गी सूर्च्छमाप स्मरेण च}% ॥३०॥


\twolineshloka
{श्रीरामनिकटं गत्वा सस्मितोवाच कामुकी}
{शस्वधौवनसंयुक्ताऽतिप्रौढा कामदुर्मदा}% ॥३१॥

\uvacha{शूर्पणखोवाच}

\twolineshloka
{हेराम हे धनश्याम रूपधाम सुणान्वित}
{मायानुरक्तां वनितां मां गृहाण सुनिर्जने}% ॥३२॥


\twolineshloka
{श्रुत्वा शूर्पणखावाक्यं धर्मं संस्मृत्य धार्मिकः}
{उवाच सधुरें वाक्यं शावभीतश्च नारद}% ॥३३॥

\uvacha{श्रीराम उवाच}

\twolineshloka
{अम्बमातः सभार्योऽहमभार्यं गच्छ मेऽनुजम्}
{दुःखं प्रियोऽन्यां प्रभजेदितरं च सुखालयम्}% ॥३४॥


\twolineshloka
{रामस्य वचनं श्रुत्वा प्रययौ लक्ष्मणं मुदा}
{ददर्श लक्षमणं शान्तं कान्तं च लक्षणान्वितम्}% ॥३५॥


\twolineshloka
{मां भजस्व महाभागेत्युवाच च पुनः पुनः}
{लक्ष्मणास्तद्वचः श्रुत्वा तामुवाच कुतूहलात्}% ॥३६॥

\uvacha{लक्ष्मण उवाच}

\twolineshloka
{विहाय रामं सर्वेशं हे मूढे दासमिच्छसि}
{सीतादासी च मत्पत्नी सीतादासोऽहमेव च}% ॥३७॥


\twolineshloka
{भव सीतासपत्नी त्वं गच्छ रामं मदीश्वरम्}
{तव पुत्रो भविष्यामि सीतायाश्चयथा सति}% ॥३८॥


\twolineshloka
{लक्ष्मणास्य वचः श्रुत्वा कामेन हृतमानसा}
{उवाच लक्ष्मणं मूढा शुष्ककण्ठेष्ठतालुका}% ॥३९॥

\uvacha{शूर्पणखोवाच}

\twolineshloka
{यदी त्यजसि मां मूढ कामात्स्वयमुपस्थिताम्}
{युवयोश्च विपत्तिश्च भविष्य ति न संशयः}% ॥४०॥


\twolineshloka
{ब्रह्मा च मोहिनीं त्यक्तवा विश्वेऽपूज्ये बभूव सः}
{रम्भाशापेन दक्षश्च छागमस्तो बभूव सः}% ॥४१॥


\twolineshloka
{स्वर्वैद्यश्चोर्वशीशापाद्यज्ञभागविवर्जितः}
{रूपहीनः कुबेरश्च मेनाशापेन लक्ष्मण}% ॥४२॥


\twolineshloka
{कामो घृताचीशापेन बभूव भस्मसाच्छिवात्}
{बलिर्मदालसाशापाद्भ्रष्टराज्यो बभूव ह}% ॥४३॥


\twolineshloka
{शापेन मिश्रकेश्याश्च हृतभार्यो बृहस्पतिः}
{मम शापात्तथा रामो हृतभार्यो भविष्यति}% ॥४४॥


\twolineshloka
{कामातुरां यौवनस्थां भार्या स्वयमुपस्थिताम्}
{न त्यजेद्धर्मभीतश्च श्रुतं माध्यन्दिने पुरा}% ॥४५॥


\twolineshloka
{इह त्यक्त्वा पिबद्ग्रस्तः परत्र नरकं व्रजेत्}
{श्रुत्वा शूर्पणखावाक्यमर्धयन्द्रेण लक्ष्मणः}% ॥४६॥


\twolineshloka
{चिच्छेद नासिकां तस्याः क्षुरधारेण लीलया}
{तस्या भ्राता च युयुधे बलवान्खरदूषणः}% ॥४७॥


\twolineshloka
{ससैन्यो लक्ष्मणास्त्रेण स जगाम यमालयम्}
{चतुर्दशसहस्रं च राक्षसान्खरदूषणम्}% ॥४८॥


\twolineshloka
{मृतान्दृष्ट्वा शूर्पणखा भर्त्सयामास रावणम्}
{सर्वं निवेदनं कृत्वा जगाम पुष्करं तदा}% ॥४९॥


\twolineshloka
{ब्रह्मणश्च वरं प्राप कृत्वा च दुष्करं तपः}
{उवाच तादृशीं दृष्ट्वा निराहारां तपस्विनीम्}% ॥५०॥

\onelineshloka
{सर्वज्ञस्तन्मनो मत्वा कृपासिन्धुश्च नारद}% ॥५१॥

\uvacha{ब्रह्मोवाच}

\twolineshloka
{अप्राप्य रामं दृष्प्रापं करोषि दुष्करं तपः}
{जितेन्द्रियाणां प्रवरं लक्ष्मणं धर्मलक्षणम्}% ॥५२॥


\twolineshloka
{ब्रह्मविष्णुशिवादीनामीश्वरं प्रकृतेः परम्}
{जन्मान्तरे च भर्तारं प्राप्स्यसि त्वं वरानने}% ॥५३॥


\twolineshloka
{इत्येवमुक्त्वा ब्रह्म च जगाम स्वालयं मुदा}
{देहं तत्याज सा वह्नो सा च कुब्जा बभूव ह}% ॥५४॥


\twolineshloka
{अथ शूर्पणखा वाक्यात्कोपात्कम्पितविग्रहः}
{जहार मायया सीतां मायावी राक्षसेश्वरः}% ॥५५॥


\twolineshloka
{सीतां न दृष्ट्वा रामश्च मूर्च्छां प्राप चिरं मुने}
{चेतनां कारयामास भ्राता चऽऽध्यात्मिकेन च}% ॥५६॥


\twolineshloka
{ततो बभ्राम गहनं शैलं च कन्दरं नदम्}
{अहर्निशं च शोकार्तो मुनीनामाश्रमं मुने}% ॥५७॥


\twolineshloka
{चिरमन्वेषणं कृत्वा न दृष्ट्वा जानकीं विभुः}
{चकार मित्रतां रामः सुग्रीवेण स्वयं प्रभुः}% ॥५८॥


\twolineshloka
{निहत्य वालिनं बाणैर्ददौ राज्यं च लीलया}
{सुग्रीवाय च मित्राय स्वीकारपालनाय वै}% ॥५९॥


\twolineshloka
{दूतान्प्रस्थापयामास सर्वत्र वानरेश्वाः}
{तस्थौ सुग्रीवभवने श्रीरामश्च सलक्ष्मणः}% ॥६०॥


\twolineshloka
{हनूमते वरं दत्तवा रम्यं रत्नाङ्गुलीयकम्}
{सीतायै शुभसन्देशं प्राणधारणकारणम्}% ॥६१॥


\twolineshloka
{तं च प्रस्थापयामास दक्षिणां दिशमुत्तमाम्}
{सुप्रीत्याऽऽलिङ्गनं दत्त्वा पादरेणून्सुदुर्लभान्}% ॥६२॥


\twolineshloka
{हनुमान्पययौ लङ्कां सीतान्वेषणहेतवे}
{रामादधीतसन्देशो ययौ रुद्रकलोद्भवः}% ॥६३॥


\twolineshloka
{अशोककानने सीतां ददर्श शोककर्शिताम्}
{निराहारामतिकृशां कुह्वां चन्द्रकलामिव}% ॥६४॥


\twolineshloka
{सततं राम रामेति जपन्तीं भक्तिबूर्वकम्}
{बिभ्रतीं च जटाभारं तप्तकाञ्चनसन्निभाम्}% ॥६५॥


\twolineshloka
{ध्यायमानां पदाब्जं च श्रीरामस्य दिपानिशम्}
{शुद्धाशयां सुशीलां च सुव्रतां च पतिव्रताम्}% ॥६६॥


\twolineshloka
{महालक्ष्मीलक्ष्मयुक्तां प्रज्वलन्तीं स्वतेजसा}
{पुण्यदां सर्वतीर्थानां दृष्ट्वा भुवनपावतीम्}% ॥६७॥


\twolineshloka
{प्रणम्य मातरं दृष्ट्वा रुदतीं वायुनन्दनः}
{रत्नाङ्गुलीयं रामस्य ददौ तस्यै मुदाऽन्वितः}% ॥६८॥


\twolineshloka
{रुरोद धर्मी तां दृष्ट्वा धृत्वा तच्चरणाम्बुजम्}
{उवाच रामसन्देशं सीताजीवनरक्षणम्}% ॥६९॥

\uvacha{हनुमानुवाच}

\twolineshloka
{पारेसमुद्रं श्रीरामः सन्नद्धश्च सलक्ष्मणः}
{बभूव राममित्रं च सुग्रीवो बलवान्कपिः}% ॥७०॥


\twolineshloka
{रामश्च वालिनं हत्वा राज्यं निष्कण्टकं ददौ}
{सुग्रीवाय च मित्राय तद्भार्यं वालिना हृताम्}% ॥७१॥


\twolineshloka
{सुग्रीवश्च तवोद्धारं स्वीचकार च धर्मतः}
{वानराश्च ययुः सर्वे तवान्वेषणकारणात्}% ॥७२॥


\twolineshloka
{प्राप्य मङ्गलवार्तां च मत्तो राजीवलोचनः}
{गम्भीरं सागरं बद्ध्वा सोऽचिरेणाऽऽगमिष्यति}% ॥७३॥


\twolineshloka
{निहत्य रावणं पापं सपुत्रं च सबान्धवम्}
{करिष्यत्यचिरेणैव हे मातस्तव मोक्षणम्}% ॥७४॥


\twolineshloka
{अद्य रत्नमयीं लङ्कां निःशङ्कस्त्वतप्रसादतः}
{भस्मीभूतां करिष्यामि मातः पश्य च सस्मितम्}% ॥७५॥


\twolineshloka
{मर्कटीडिम्भतुल्यां च लङ्कां पश्यामि सुव्रते}
{मूत्रतुल्यं समुद्रं च शरावमिव भूतलम्}% ॥७६॥


\twolineshloka
{पिपीलिकासङ्घमिव ससैन्यं रावणं तथा}
{संहर्तुं च समर्थोंऽहं मुहूर्तार्धेन लीलया}% ॥७७॥


\twolineshloka
{रामप्रतिज्ञारक्षार्थं न हनिष्यामि साम्प्रतम्}
{स्वस्था भव महाभागे त्यज भीतिं मदीश्वरि}% ॥७८॥


\twolineshloka
{पानरस्य वचः श्रुत्वा रुरोदोच्चैर्मुहुर्मुहुः}
{उवाच वचनं भीता रामपतिव्रता}% ॥७९॥


\uvacha{सीतोवाच}
\onelineshloka
{अये जीवति मे रामो मच्छोकार्णवदारुणात्}% ॥८०॥


\twolineshloka
{अपि मे कुशली नाथः कौसल्यानन्दनः प्रभु}
{कीदृशश्च कृशाङ्गश्च जानकीजीवनोऽधुना}% ॥८१॥


\twolineshloka
{किमाहारश्च किं भुङ्क्ते मम प्रणाधिकः प्रियः}
{अपि पारे समुद्रस्य सत्यं सीतापतिः स्वयम्}% ॥८२॥


\twolineshloka
{अपि सत्यं क सन्नद्धो न शोकेन हतः प्रभुः}
{अपि स्मरति मां पापां स्वामिनो दुःखरूपिणीम्}% ॥८३॥


\twolineshloka
{मदर्थे कति दुःखं वा सम्प्राप स मदीश्वरः}
{हारो नाऽऽरोर्पितः कण्ठे पुरा व्यवहितो रतौ}% ॥८४॥


\twolineshloka
{अधुनैवाऽऽवयोर्मध्ये समुद्रः शतयोजनः}
{अपि द्रक्ष्यामि तं रामं करुणासागरं प्रभुम्}% ॥८५॥


\twolineshloka
{कान्तं शान्तं नितान्तं च धर्मिष्ठं धर्मकर्मणि}
{अपि सेवां करिष्यामि पादपद्मे पुनः प्रभोः}% ॥८६॥


\twolineshloka
{पतिसेवाविहीनाया मूढाया जीवनं दृथा}
{अपि मे धर्मपुत्रश्च सत्यं जीवति लक्ष्मणः}% ॥८७॥


\twolineshloka
{मच्छोकसागरे मग्नो भग्नदर्पो मया विना}
{वीराणां प्रवरो धर्मी देवकल्पश्च देवरः}% ॥८८॥


\threelineshloka
{अपि सत्यं ससन्न्नद्धो सत्प्रभोरनुजः सदा} 
{अपि द्रक्ष्यामि सत्यं तं लक्ष्मणं धर्मलक्षणम्}
{प्राणानामधिकं प्रेम्णा धन्यं पुण्यस्वरूपिणम्}% ॥८९॥


\twolineshloka
{इत्येवं वचनं श्रुत्वा दत्त्वा प्रत्युत्तरं शुभम्}
{भस्मीभूतां च तां लङ्कां चकार लीलया मुने}% ॥९०॥


\twolineshloka
{पुनः प्रबोधं तस्यै च दत्तवा वायुसुतः कपिः}
{प्रययौ लीलया वेगाद्यत्र राजीवलोचनः}% ॥९१॥


\twolineshloka
{सर्वं तत्कथायामास वृत्तान्तं मातुरेव च}
{सीतामङ्गलवृत्तान्तं श्रुत्वा रामो रुरोद च}% ॥९२॥


\twolineshloka
{रुतोदोच्चैर्लक्ष्मणश्च सुग्रीवश्चापि नारद}
{वानरा रुरुदुः सर्वे सहबलपराक्रमाः}% ॥९३॥


\twolineshloka
{निबध्य सेतुं लङ्कां च प्रययौ रघुनन्दनः}
{ससैन्यः सानुजः शीघ्रं सन्नद्धश्चापि नारद}% ॥९४॥


\twolineshloka
{निहत्य रावणं रामो रणं कृत्वा सबान्धवम्}
{चकार मोक्षणं ब्रह्मन् सितायश्च शुभेक्षणे}% ॥९५॥


\twolineshloka
{कृत्वा पुष्पकयानेन सीतां सत्यपरायणाम्}
{अयोध्यां प्रययौ शीघ्रं क्रीडाकौतुकमङ्गलैः}% ॥९६॥


\twolineshloka
{क्रीडां चकार भगवान् सीतां कृत्वा च वक्षसि}
{विजहौ विरहज्वालां सीता रामश्च तत्क्षणम्}% ॥९७॥


\twolineshloka
{सप्तद्वीपेश्वरो रामो बभूव पृथिवीतले}
{बभूव निखिला पृथ्वी आधिव्याधिविवर्जिता}% ॥९८॥


\twolineshloka
{बभूवतू रामपुत्रौ धार्मिकौ च कुशीलवौ}
{तयोश्च पुत्रैः पौत्रेश्च सूर्यवंशोद्भवा नृपाः}% ॥९९॥


\twolineshloka
{इति ते कथितं वत्स श्रीरामचरितं शुभम्}
{सुखदं मोक्षदं सारं पारपोतं भवार्णवे}% ॥१००॥

॥इति श्रीब्रह्मवैवर्ते महापुराणे श्रीकृष्णजन्मखण्डे उत्तरार्धे नारदनारायणसंवादे श्रीरामाचरितं नाम द्विषष्टितमोऽध्यायः॥६२॥

\closesection