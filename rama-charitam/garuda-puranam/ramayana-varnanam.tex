\chapt{गरुड-पुराणम्}

\sect{रामायणवर्णनम्}

\src{गरुड-पुराणम्}{पूर्वखण्डः}{अध्यायः १४३}{श्लोकाः १---५१}
\tags{concise, complete}
\notes{Summary of Ramayana.}
\textlink{}
\translink{}

\storymeta


\uvacha{ब्रह्मोवाच}

\twolineshloka
{रामायणमतो वक्ष्ये श्रुतं पापविनाशनम्}
{विष्णुनाभ्यब्जतो ब्रह्मा मरीचिस्तत्सुतोऽभवत्} %॥१,१४३.१॥

\twolineshloka
{मरीचेः कश्यपस्तस्माद्रविस्तस्मान्मनुः स्मृतः}
{मनोरिक्ष्वाकुरस्याभूद्वंशे राजा रघुः स्मृतः} %॥१,१४३.२॥

\twolineshloka
{रघोरजस्ततो जातो राजा दशरथो बली}
{तस्य पुत्रास्तु चत्वारो महाबलपराक्रमाः} %॥१,१४३.३॥

\twolineshloka
{कौसल्यायाम भूद्रामो भरतः कैकयीसुतः}
{सुतौ लक्ष्मणशक्षुघ्नौ सुमित्रायां बभूवतुः} %॥१,१४३.४॥

\twolineshloka
{रामो भक्तः पितुर्मातुर्विश्वामित्रादवाप्तवान्}
{अस्त्रग्रामं ततो यक्षीं ताटकां प्रजघान ह} %॥१,१४३.५॥

\twolineshloka
{विशावमित्रस्य यज्ञे वै सुबाहुं न्यवधीद्बली}
{जनकस्य क्रतुं गत्वा उपयेमेऽथ जानकीम्} %॥१,१४३.६॥

\twolineshloka
{ऊर्मिलां लक्ष्मणो वीरो भरतो माण्डवीं सुताम्}
{शत्रुघ्नो वै कीर्तिमतीं कुशध्वजसुते उभे} %॥१,१४३.७॥

\twolineshloka
{पित्रादिभिरयोध्यायां गत्वा रामादयः स्थिताः}
{युधाजितं मातुलञ्च शत्रुघ्नभरतौ गतौ} %॥१,१४३.८॥

\twolineshloka
{गतयोर्नृपवर्योऽसौ राज्यं दातुं समुद्यतः}
{स रामाय तत्पुत्राय कैकेय्या प्रार्थितस्तदा} %॥१,१४३.९॥

\twolineshloka
{चतुर्दशसमावासो वनेरामस्य वाञ्छितः}
{रामः पितृहितार्थञ्च लक्ष्मणेन च सीतया} %॥१,१४३.१०॥

\twolineshloka
{राज्यञ्च तृणवत्त्यक्त्वा शृङ्गवेरपुरं गतः}
{रथं त्यक्त्वा प्रयागञ्च चित्रकूटगिरिं गतः} %॥१,१४३.११॥

\twolineshloka
{रामस्य तु वियोगेन राजा स्वर्गं समाश्रितः}
{संस्कृत्य भरतश्चागाद्राममाह बलान्वितः} %॥१,१४३.१२॥

\twolineshloka
{अयोध्यान्तु समागत्य राज्यं कुरु महामते}
{स नैच्छत्पादुके दत्त्वा राज्याय भरताय तु} %॥१,१४३.१३॥

\twolineshloka
{विसर्जितोऽथ भरतो रामराज्यमपालयत्}
{नन्दिग्रामे स्थितो भक्तो ह्ययोध्यां नाविशद्व्रती} %॥१,१४३.१४॥

\twolineshloka
{रामोऽपि चित्रकूटाच्च ह्यत्रेराश्रममाययौ}
{नत्वा सुतीक्ष्णं चागस्त्यं दण्डकारण्यमागतः} %॥१,१४३.१५॥

\twolineshloka
{तत्र शूर्पणखा नाम राक्षसी चात्तुमागता}
{निकृत्य कर्णो नासे च रामेणाथापवारिता} %॥१,१४३.१६॥

\twolineshloka
{तत्प्रेरितः खरश्चागाद्दूषणस्त्रिशिरास्तथा}
{चतुर्दशसहस्रेण रक्षसान्तु बलेन च} %॥१,१४३.१७॥

\twolineshloka
{रामोऽपि प्रेषयामास बाणैर्यमपुरञ्च तान्}
{राक्षस्या प्रेरितोऽभ्यागाद्रावणो हरणाय हि} %॥१,१४३.१८॥

\twolineshloka
{मृगरूपं स मारीचं कृत्वाग्रेऽथ त्रिदण्डधृक्}
{सीतया प्रेरितो रामो मारीचं निजघान ह} %॥१,१४३.१९॥

\twolineshloka
{म्रियमाणः स च प्राह हा सीते ! लक्ष्मणोति च}
{सीतोक्तो लक्ष्मणोऽथागाद्रामश्चानुददर्श तम्} %॥१,१४३.२०॥

\twolineshloka
{उवाच राक्षसी माया नूनं सीता हृतेति सः}
{रावणोऽन्तरमासाद्य ह्यङ्केनादाय जानकीम्} %॥१,१४३.२१॥

\twolineshloka
{जटायुषं विनिर्भिद्य ययौ लङ्कां ततो बली}
{अशोकवृक्षच्छायायां रक्षितां तामधारयत्} %॥१,१४३.२२॥

\twolineshloka
{आगत्य रामः सून्याञ्च पर्णशालां ददर्श ह}
{शोकं कृत्वाथ जानक्या मार्गणं कृतवान्प्रभुः} %॥१,१४३.२३॥

\twolineshloka
{जटायुषञ्च संस्कृत्य तदुक्तो दक्षिणां दिशम्}
{गत्वा सख्यं ततश्चक्रे सुग्रीवेण च राघवः} %॥१,१४३.२४॥

\twolineshloka
{सप्त तालान्विनिर्भिद्य शरेणानतपर्वणा}
{वालिनञ्च विनिर्भिद्य किष्किन्धायां हरीश्वरम्} %॥१,१४३.२५॥

\twolineshloka
{सुग्रीवं कृतवान्राम ऋश्यमूके स्वयं स्थितः}
{सुग्रीवः प्रेषयामास वानरान्पर्वतोपमान्} %॥१,१४३.२६॥

\twolineshloka
{सीताया मार्गणं कर्तुं पूर्वाद्याशासु सोत्सवान्}
{प्रतीचीमुत्तरां प्राचीं दिशं गत्वा समागताः} %॥१,१४३.२७॥

\twolineshloka
{दक्षिणान्तु दिशं ये च मार्गयन्तोऽथ जानकीम्}
{वनानि पर्वतान्द्वीपान्नदीनां पुलिनानि च} %॥१,१४३.२८॥

\twolineshloka
{जानकीन्ते ह्यपश्यन्तो मरणे कृतनिश्चयाः}
{सम्पातिवचनाज्ज्ञात्वा हनूमान्कपिकुञ्जरः} %॥१,१४३.२९॥

\twolineshloka
{शतयोजनविस्तीर्णं पुप्लुवे मकरालयम्}
{अपश्यज्जानकीं तत्र ह्यशोकवनिकास्थिताम्} %॥१,१४३.३०॥

\twolineshloka
{भर्त्सितां राक्षसीभिश्च रावणेन च रक्षसा}
{भव भार्येति वदता चिन्तयन्तीञ्च राघवम्} %॥१,१४३.३१॥

\twolineshloka
{अङ्गुलीयं कपिर्दत्त्वा सीतां कौशल्यमब्रवीत्}
{रामस्य तस्य दूतोऽहं शोकं मा कुरु मैथिलि} %॥१,१४३.३२॥

\twolineshloka
{स्वाभिज्ञानञ्च मे देहि येन रामः स्मरिष्यति}
{तच्छ्रुत्वा प्रददौ सीता वेणीरत्नं हनूमते} %॥१,१४३.३३॥

\twolineshloka
{यथा रामो नयेच्छीघ्रं तथा वाच्यं त्वया कपे}
{तथेत्युक्त्वा तु हनुमान्वनं दिव्यं बभञ्ज ह} %॥१,१४३.३४॥

\twolineshloka
{हत्वाक्षं राक्षसांश्चान्यान्बन्धनं स्वयमागतः}
{सर्वैरिन्द्रजितो बाणैर्दृष्ट्वा रावणमब्रवीत्} %॥१,१४३.३५॥

\twolineshloka
{रामदूतोऽस्मि हनुमान्देहि रामाय मैथिलीम्}
{एतच्छ्रुत्वा प्रकुपितो दीपयामास पुच्छकम्} %॥१,१४३.३६॥

\twolineshloka
{कपिर्ज्वलितलाङ्गूलो लङ्कां देहेऽ महाबलः}
{दग्ध्वा लङ्कां समायातो रामपार्श्वं स वानरः} %॥१,१४३.३७॥

\twolineshloka
{जग्ध्वा फलं मधुवने दृष्टा सीतत्यवेदयत्}
{वेणीरत्नञ्च रामाय रामो लङ्कापुर्री ययौ} %॥१,१४३.३८॥

\twolineshloka
{ससुग्रीवः स हनुमान्साङ्गदश्च सलक्ष्मणः}
{विभीषणोऽपि सम्प्राप्तः शरणं राघवं प्रति} %॥१,१४३.३९॥

\twolineshloka
{लङ्कैश्वर्येष्वभ्यषिञ्चद्रामस्तं रावणानुजम्}
{रामो नलेन सेतुञ्च कृत्वाब्धौ चोत्ततार तम्} %॥१,१४३.४०॥

\twolineshloka
{सुवेलावस्थितश्चैव पुरीं लङ्कां ददर्शह}
{अथ ते वानरा वीरा नीलाङ्गदनलादयः} %॥१,१४३.४१॥

\twolineshloka
{धूम्रधूम्राक्षवीरेन्द्रा जाम्बवत्प्रमुखास्तदा}
{मैन्दद्विविदमुख्यास्ते पुरीं लङ्कां बभञ्जिरे} %॥१,१४३.४२॥

\twolineshloka
{राक्षसांश्च महाकायान्कालाञ्जनचयोपमान्}
{रामः सलक्ष्मणो हत्वा सकपिः सर्वराक्षसान्} %॥१,१४३.४३॥

\twolineshloka
{विद्युज्जिह्वञ्च धूम्राक्षं देवान्तकनरान्त कौ}
{महोदरमहापार्श्वावतिकायं महाबलम्} %॥१,१४३.४४॥

\twolineshloka
{कुम्भं निकुम्भं मत्तञ्च मकराक्षं ह्यकम्पनम्}
{प्रहस्तं वीरमुन्मत्तं कुम्भकर्णं महाबलम्} %॥१,१४३.४५॥

\twolineshloka
{रावणिं लक्ष्मणोऽच्छिन्त ह्यस्त्राद्यै राघवो बली}
{निकृत्य बाहुचक्राणि रावणन्तु न्यपातयन्} %॥१,१४३.४६॥

\twolineshloka
{सीतां शुद्धां गृहीत्वाथ विमाने पुष्पके स्थितः}
{सवानरः समायातो ह्ययोध्यां प्रवरां पुरीम्} %॥१,१४३.४७॥

\twolineshloka
{तत्र राज्यं चकाराथ पुत्त्रवत्पालयन्प्रजाः}
{दशाश्वमेधानाहृत्य गयाशिरसि पातनम्} %॥१,१४३.४८॥

\twolineshloka
{पिण्डानां विधिवत्कृत्वा दत्त्वा दानानि राघवः}
{पुत्रौ कुशलवौ दृष्ट्वा तौ च राज्येऽभ्यषेचयत्} %॥१,१४३.४९॥

\twolineshloka
{एकादशसहस्राणि रामो राज्यमकारयत्}
{शत्रुघ्नो लवणं जघ्ने शैलूषं भतस्ततः} %॥१,१४३.५०॥

\twolineshloka
{अगस्त्यादीन्मुनीन्नत्वा श्रुत्वोत्पत्तिञ्च रक्षसाम्}
{स्वर्गं गतो जनैः सार्धमयोध्यास्थैः कृतार्थकः} %॥१,१४३.५१॥

॥इति श्रीगारुडे महापुराणे पूर्वखण्डे प्रथमांशाख्ये आचारकाण्डे रामायणवर्णनं नाम
त्रिचत्वारिंशदुत्तरशततमोऽध्यायः॥