\chapt{ब्रह्म-पुराणम्}

\sect{अनन्तवासुदेवमाहात्म्यवर्णनम्}

\src{ब्रह्म-पुराणम्}{पूर्वखण्डः}{अध्यायः १७६}{श्लोकाः ३७---५१}
\tags{concise, complete}
\notes{Summary of Ramayana, during the narration of AnantaVasudeva Mahatmyam.}
\textlink{https://sa.wikisource.org/wiki/ब्रह्मपुराणम्/अध्यायः_१७६}
\translink{}

\storymeta


\uvacha{मुनय ऊचुः}

\twolineshloka
{न हि नस्तृप्तिरस्तीह शृण्वतां भगवत्कथाम्}
{पुनरेव परं गुह्यं वक्तुमर्हस्यशेषतः} %॥१॥

\twolineshloka
{अनन्तवासुदेवस्य न सम्यग्वर्णितं त्वया}
{श्रोतुमिच्छामहे देव विस्तरेण वदस्व नः} %॥२॥

\uvacha{ब्रह्मोवाच}

\twolineshloka
{प्रवक्ष्यामि मुनिश्रेष्ठाः सारात्सारतरं परम्}
{अनन्तवासुदेवस्य माहात्म्यं भुवि दुर्लभम्} %॥३॥

\twolineshloka
{आदिकल्पे पुरा विप्रास्त्वहमव्यक्तजन्मवान्}
{विश्वकर्माणमाहूय वचनं प्रोक्तवानिदम्} %॥४॥

\twolineshloka
{वरिष्ठं देवशिल्पीन्द्रं विश्वकर्माग्रकर्मिणम्}
{प्रतिमां वासुदेवस्य कुरु शैलमयीं भुवि} %॥५॥

\twolineshloka
{यां प्रेक्ष्य विधिवद्‌भक्ताः सेन्द्रा वै मानुषादयः}
{येन दानवरक्षोभ्यो विज्ञाय सुमहद्‌भयम्} %॥६॥

\twolineshloka
{त्रिदिवं समनुप्राप्य सुमेरुशिखरं चिरम्}
{वासुदेवं समाराध्य निरातङ्का वसन्ति ते} %॥७॥

\twolineshloka
{मम तद्वचनं श्रुत्वा विश्वकर्मा तु तत्क्षणात्}
{चकार प्रतिमां शुद्धां शङ्खचक्रगदाधराम्} %॥८॥

\twolineshloka
{सर्वलक्षणसंयुक्तां पुण्डरीकायतेक्षणाम्}
{श्रीवत्सलक्ष्संयुक्तामत्युग्रां प्रतिमोत्तमाम्} %॥९॥

\twolineshloka
{वनमालावृतोरस्कां मुकुटाङ्गदधारिणीम्}
{पीतवस्त्रां सुपीनांसां कुण्डलाभ्यामलङ्कृताम्} %॥१०॥

\twolineshloka
{एवं सा प्रतिमा दिव्या गुह्यमन्त्रैस्तदा स्वयम्}
{प्रतिष्ठाकालमासाद्य मयाऽऽसौ निर्मिता पुरा} %॥११॥

\twolineshloka
{तस्मिन्काले तदा शक्रो देवराट्खेचरैः सह}
{जगाम ब्रह्मसदनमारुह्य गजमुत्तमम्} %॥१२॥

\twolineshloka
{प्रसाद्य प्रतिमां शक्रः स्नानदानैः पुनः पुनः}
{प्रतिमां तां समादाय स्वपुरं पुनरागमत्} %॥१३॥

\twolineshloka
{तां समाराध्य सुचिरं यतवाक्कायमानसः}
{वृत्राद्यानसुरान्क्रूरान्नमुचिप्रमुखान्स च} %॥१४॥

\twolineshloka
{निहत्य दानवान्भीमान्भूक्तवान्भुवनत्रयम्}
{द्वितीये च युगे प्राप्ते त्रेतायां राक्षसाधिपः} %॥१५॥

\twolineshloka
{बभूव सुमहावीर्यो दशग्रीवः प्रतापवान्}
{दश वर्षसहस्राणि निराहारो जितेन्द्रियः} %॥१६॥

\twolineshloka
{चचार व्रतमत्युग्रं तपः परमदुश्चरम्}
{तपसा तेन तुष्टोऽहं वरं तस्मै प्रदत्तवान्} %॥१७॥

\twolineshloka
{अवध्यः सर्वदेवानां स दैत्योरगयक्षसाम्}
{शापप्रहरणैरुग्रैरवध्यो यमकिङ्करैः} %॥१८॥

\twolineshloka
{वरं प्राप्य तदा रक्षो यक्षान्सर्वगणानिमान्}
{धनाध्यक्षं विनिर्जित्य शक्रं जेतुं समुद्यतः} %॥१९॥

\twolineshloka
{सङ्ग्रामं सुमहाघोरं कृत्वा देवैः स राक्षसः}
{देवराजं विनिर्जित्य तदा इन्द्रिजितेति वै} %॥२०॥

\twolineshloka
{राक्षसस्तत्सुरो नाम मेघनादः प्रलब्धवान्}
{अमरावतीं ततः प्राप्य देवराजगृहे शुभे} %॥२१॥

\twolineshloka
{ददर्शाञ्जनसङ्काशां रावणस्तु बलान्वितः}
{प्रतिमां वासुदेवस्य सर्वलक्षणसंयुताम्} %॥२२॥

\twolineshloka
{श्रीवत्सलक्ष्मसंयुक्तं पद्मपत्रायतेक्षणाम्}
{वनमालावृतोरस्कां सर्वकामफलप्रदाम्} %॥२३॥

\twolineshloka
{शङ्खचक्रगदाहस्तां पीतवस्त्रां चतुर्भुजाम्}
{सर्वाभरणसंयुक्तां सर्वकामफलप्रदाम्} %॥२४॥

\twolineshloka
{विहाय रत्नसङ्घांश्च प्रतिमां शुभलक्षणाम्}
{पुष्पकेण विमानेन लङ्कां प्रास्थापयद्‌द्रुतम्} %॥२५॥

\twolineshloka
{पुराध्यक्षः स्थितः श्रीमान्धर्मात्मा स विभीषणः}
{रावणस्यानुजो मन्त्री नारायणपरायणः} %॥२६॥

\twolineshloka
{दृष्ट्वा तां प्रतिमां दिव्यां देवेन्द्रभवनच्युताम्}
{रोमाञ्चिततनुर्भूत्वा विस्मयं समपद्यत} %॥२७॥

\twolineshloka
{प्रणम्य शिरसा देवं प्रहृष्टेनान्तरात्मना}
{अद्य मे सफलं जन्म अद्य मे सफलं तपः} %॥२८॥

\twolineshloka
{इत्युक्त्वा स तु धर्मात्मा प्रणिपत्य मुहुर्मुहुः }
{ज्येष्ठं भ्रातरमासाद्य कृताञ्जलिरभाषत} %॥२९॥

\twolineshloka
{राजन्प्रतिमया त्वं मे प्रसादं कर्तुमर्हसि}
{यामाराध्य जगन्नाथ निस्तरेयं भवार्णवम्} %॥३०॥

\twolineshloka
{भ्रातुर्वचनमाकर्ण्य रावणस्तं तदाऽब्रवीत्}
{गृहाण प्रतिमां वीर त्वनया किं करोम्यहम्} %॥३१॥

\twolineshloka
{स्वयम्भुवं समाराध्य त्रैलोक्यं विजये त्वहम्}
{नानाश्चर्यमयं देवं सर्वभूतभवोद्भवम्} %॥३२॥

\twolineshloka
{विभीषणो महाबुद्धिस्तदा तां प्रतिमां शुभाम्}
{शतमष्टोत्तरं चाब्दं समाराध्य जनार्दनम्} %॥३३॥

\twolineshloka
{अजरामरणं प्राप्तमणिमादिगुणैर्युतम्}
{राज्यं लङ्काधिपत्यं च भोगान्भुङ्क्ते यथेप्सितान्} %॥३४॥

\uvacha{मुनय ऊचुः}

\twolineshloka
{अहो नो विस्मयो जातः श्रुत्वेदं परमामृतम्}
{अनन्तवासुदेवस्य सम्भवं भुवि दुर्लभम्} %॥३५॥

\twolineshloka
{श्रोतुमिच्छामहे देव विस्तरेण यथातथम्}
{तस्य देवस्य माहात्म्यं वक्तुमर्हस्यशेषतः} %॥३६॥

\dnsub{रामकथासारः}

\uvacha{ब्रह्मोवाच}

\twolineshloka
{तदा स राक्षसः क्रूरो देवगन्धर्वकिन्नरान्}
{लोकपालान्समनुजान्मुनिसिद्धांश्च पापकृत्} %॥३७॥

\twolineshloka
{विजित्य समरे सर्वानाजहार तदङ्गनाः}
{संस्थाप्य नगरीं लङ्कां पुनः सीतार्थमोहितः} %॥३८॥

\twolineshloka
{शङ्कितो मृगरूपेण सौवर्णेन च रावणः}
{ततः क्रुद्धेन रामेण रणे सौमित्रिणा सह} %॥३९॥

\twolineshloka
{रावणस्य वधार्थाय हत्वा वालिं मनोजवम्}
{अभिषिक्तश्च सुग्रीवो युवराजोऽङ्गदस्तथा} %॥४०॥

\twolineshloka
{हनुमान्नलनीलश्च जाम्बवान्पनसस्तथा}
{गवयश्च गवाक्षश्च पाठीनः परमौजसः} %॥४१॥

\twolineshloka
{एतैश्चान्यैश्च बहुभिर्वानरैः समहाबलैः}
{समावृतो महाघोरै रामो राजीवलोचनः} %॥४२॥

\twolineshloka
{गिरीणां सर्वसङ्घातैः सेतुं बद्‌ध्वा महोदधौ}
{बलेन महता रामः समुत्तीर्य महोदधिम्} %॥४३॥

\twolineshloka
{सङ्ग्राममतुलं चक्रे रक्षोगणसमन्वितः}
{यमहस्तं प्रहस्तं च निकुम्भं कुम्भमेव च} %॥४४॥

\twolineshloka
{नरान्तकं महावीर्यं तथा चैव यमान्तकम्}
{मालाढ्यं मालिकाढ्यं च हत्वा रामस्तु वीर्यवान्} %॥४५॥

\twolineshloka
{पुनरिन्द्रजितं हत्वा कुम्भकर्णं सरावणम्}
{वैदेहीं चाग्निनाऽऽसोध्य दत्त्वा राज्यं विभिषणे} %॥४६॥

\twolineshloka
{वासुदेवं समादाय यानं पुष्पकमारुहत्}
{लीलया समनुप्रापदयोध्यां पूर्वपालिताम्} %॥४७॥

\twolineshloka
{कनिष्ठं भरतं स्नेहाच्छत्रुघ्नं भक्तवत्सलः}
{लीलया समनुप्रापदयोध्यां पूर्वपालिताम्} %॥४८॥

\twolineshloka
{पुरातनीं स्वमूर्तिं च समाराध्य ततो हरिः}
{दश वर्षसहस्राणि दश वर्षशतानि च} %॥४९॥

\twolineshloka
{भुक्ताव सागरपर्यन्तां मेदिनीं स तु राघवः}
{राज्यमासाद्य सुगतिं वैष्णवं पदमाविशत्} %॥५०॥

\twolineshloka
{तां चापि प्रतिमां रामः समुद्रेशाय दत्तवान्}
{धन्यो रक्षयितासि त्वं तोयरत्नसमन्वितः} %॥५१॥

\closesub

\twolineshloka
{द्वापरं युगमासाद्य यदा देवो जगत्पतिः}
{धरण्याश्चानुरोधेन भावशैथिल्यकारणात्} %॥५२॥

\twolineshloka
{अवतीर्णः स भगवान्वसुदेवकुले प्रभुः}
{कंसादीनां वधार्थाय सङ्कर्षणसहायवान्} %॥५३॥

\twolineshloka
{तदा तां प्रतिमां विप्राः सर्ववाञ्छाफलप्रदाम्}
{सर्वलोकहितार्थाय कस्यचित्कारणान्तरे} %॥५४॥

\twolineshloka
{तस्मिन्क्षेत्रवरे पुण्ये दुर्लभे पुरुषोत्तमे}
{उज्जहार स्वयं तोयात्समुद्रः सरितां पतिः} %॥५५॥

\twolineshloka
{तदा प्रभृति तत्रैव क्षेत्रे मुक्तिप्रदे द्विजाः}
{आस्ते स देवो देवानां सर्वकामफलप्रदः} %॥५६॥

\twolineshloka
{ये संश्रयन्ति चानन्तं भक्त्या सर्वेश्वरं प्रभुम्}
{वाङ्मनः कर्मभिर्नित्यं ते यान्ति परमं पदम्} %॥५७॥

\twolineshloka
{दृष्ट्वाऽनन्तं सकृद्‌भक्त्या सम्पूज्य प्रणिपत्य च}
{राजसूयाश्वमेधाभ्यां फलं दशगुणं लभेत्} %॥५८॥

\twolineshloka
{सर्वकामसमृद्धेन कामगेन सुवर्चसा}
{विमानेनार्कवर्णेन किङ्किणीजालमालिना} %॥५९॥

\twolineshloka
{त्रिःसप्तकुलमुद्धृत्य दिव्यस्त्रीगणसेवितः}
{उपगीयमानो गन्धर्वैर्नरो विष्णुपुरं व्रजेत्} %॥६०॥

\twolineshloka
{तत्र भुक्त्वा वरान्भोगाञ्जरामरणवर्जितः}
{दिव्यरूपधरः श्रीमान्यावदाभूतसम्प्लवम्} %॥६१॥

\twolineshloka
{पुण्यक्षयादिहाऽऽयातश्चतुर्वेदी द्विजोत्तमः}
{वैष्णवं योगमास्थाय ततो मोक्षमवाप्नुयात्} %॥६२॥

\twolineshloka
{एवं मया त्वनन्तोऽसौ कीर्तितो मुनिसत्तमाः}
{कः शक्नोति गुणान्वक्तुं तस्य वर्षशतैरपि} %॥६३ 

॥इति श्रीमहापुराणे आदिब्राह्मे स्वयम्भ्वृषिसंवादेऽनन्तवासुदेवमाहात्म्यनिरूपणं नाम षट्सप्तत्यधिकशततमोऽध्यायः॥१७६॥