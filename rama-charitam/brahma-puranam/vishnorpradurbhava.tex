\sect{विष्णोर्प्रादुर्भावः --- रामावतारः}

\src{ब्रह्म-पुराणम्}{पूर्वखण्डः}{अध्यायः २१३}{श्लोकाः १२४---१५८}
\tags{concise, complete}
\notes{Summary of Ramayana, during the narration of various Vishnu avataras.}
\textlink{https://sa.wikisource.org/wiki/ब्रह्मपुराणम्/अध्यायः_२१३}
\translink{}

\storymeta

\uvacha{व्यास उवाच}

\addtocounter{shlokacount}{123}

\twolineshloka
{चतुर्विंशे युगे वाऽपि विश्वामित्रपुरःसरः}
{जज्ञे दशरथस्याथ पुत्रः पद्मयतेक्षणः} %॥१२४॥

\twolineshloka
{कृत्वाऽत्मानं महाबाहुश्चतुर्धा प्रभुरीश्वरः}
{लोके राम इति ख्यातस्तेजसा भास्करोपमः} %॥१२५॥

\twolineshloka
{प्रसादनार्थं लोकस्य रक्षसां निग्रहाय च}
{धर्मस्य च विवृद्ध्यर्थं जज्ञे तत्र महयशाः} %॥१२६॥

\twolineshloka
{तमप्याहुर्मनुष्येन्द्रं सर्वभूतहिते रतम्}
{यः समाः सर्वधर्मज्ञश्चतुर्दश वनेऽवसत्} %॥१२७॥

\twolineshloka
{लक्ष्मणानुचरो रामः सर्वभूतहिते रतः}
{चतुर्दश वने तप्त्वा तपो वर्षणि राघवः} %॥१२८॥

\twolineshloka
{रूपिणी तस्य पार्श्वस्था सीतेति प्रथिता जने}
{पूर्वोदिता तु या लक्ष्मीर्भर्तारमनुगच्छति} %॥१२९॥

\twolineshloka
{जनस्थाने वसन्कार्यं त्रिदशानां चकार सः}
{तस्यापकारिणं क्रूरं पौलस्त्यं मनुजर्षभः} %॥१३०॥

\twolineshloka
{सीतायाः पदमन्विच्छन्निजघान महायशाः}
{देवासुरगणानां च यक्षराक्षसभोगिनाम्} %॥१३१॥

\twolineshloka
{यत्रावध्यं राक्षसेन्द्रं रावणं युधि दुर्जयम्}
{युक्तं राक्षसकोटीभिर्नीलाञ्जनचयोपमम्} %॥१३२॥

\twolineshloka
{त्रैलाक्यद्रावणं क्रूरं रावणं राक्षसेश्वरम्}
{दुर्जयं दुर्धरं दृप्तं शार्दूलसमविक्रमम्} %॥१३३॥

\twolineshloka
{दुर्निरीक्ष्यं सुरगणैर्वरदानेन दर्पितम्}
{जघान सचिवैः सार्धं ससैन्यं रावणं युधि} %॥१३४॥

\twolineshloka
{महाभ्रगणसङ्काशं महाकायं महाबलम्}
{रावणं निजघानाऽऽशु रामो भूतपतिः पुरा} %॥१३५॥

\twolineshloka
{सुग्रीवस्य कृते येन वानरेन्द्रो महाबलः}
{वाली विनिहतः सङ्ख्ये सुग्रीवश्चाभिषेचितः} %॥१३६॥

\twolineshloka
{मधोश्च तनयो दृप्तो लवणो नाम दानवः}
{हतो मधुवने वीरो वरमत्तो महासुरः} %॥१३७॥

\twolineshloka
{यज्ञविघ्नकरौ येन मुनीनां भावितात्मनाम्}
{मारीचश्च सुबाहुश्च बलेन बलिनां वरौ} %॥१३८॥

\twolineshloka
{निहतौ च निराशौ च कृतौ तेन महात्मना}
{समरे युद्धशौण्डेन तथाऽन्ये चापि राक्षसाः} %॥१३९॥

\twolineshloka
{विराधश्च कबन्धश्च राक्षसौ भीमविक्रमौ}
{जघान पुरुषव्याघ्रो गन्धवौ शापमोहितौ} %॥१४०॥

\twolineshloka
{हुताशनार्कांशुतडिद्‌गुणाभैः प्रतप्तजाम्बूनदचित्रपुङ्खैः}
{महेन्द्रवज्राशनितुल्यसारै रिपून्स रामः समरे निजघ्ने} %॥१४१॥

\twolineshloka
{तस्मै दत्तानि शस्त्राणि विश्वामित्रेण धीमता}
{वधार्थं देवशत्रूणां दुर्धर्षाणां सुरैरपि} %॥१४२॥

\twolineshloka
{वर्तमाने मखे येन जनकस्य महात्मनः}
{भग्नं माहेश्वरं चापं क्रीडता लीलया पुरा} %॥१४३॥

\twolineshloka
{एतानि कृत्वा कर्माणि रामो धर्मभृतां वरः}
{दशाश्वमेधाञ्जारूथ्यानाजहार निरर्गलान्} %॥१४४॥

\twolineshloka
{नाश्रूयन्ताशुभा वाचो नाऽऽकुलं मारुतो ववौ}
{न वित्तहरणं चाऽऽसीद्रामे राज्यं प्रशासति} %॥१४५॥

\twolineshloka
{परिदेवन्ति विधवा नानर्थाश्च कदाचन}
{सर्वमासीच्छुभं तत्र रामे राज्यं प्रशासति} %॥१४६॥

\twolineshloka
{न प्राणिनां भयं चाऽऽसीज्जलाग्न्यनिलघातजम्}
{न चापि वृद्धा बालानां प्रेतकार्याणि चक्रिरे} %॥१४७॥

\twolineshloka
{ब्रह्मचर्यपरं क्षत्रं विशस्तु क्षत्रिये रताः}
{शूद्राश्चैव हि वर्णास्त्रीञ्शुश्रूषन्त्यनहङ्कृताः} %॥१४८॥

\twolineshloka
{नार्यो नात्यचरन्भर्तॄन्भार्यां नात्यचरत्पतिः}
{सर्वमासीज्जगद्दान्तं निर्दस्युरभवन्मही} %॥१४९॥

\twolineshloka
{राम एकोऽभवद्भर्ता रामः पालयिताऽभवत्}
{आसन्वर्षसहस्राणि तथा पुत्रसहस्रिणः} %॥१५०॥

\twolineshloka
{अरोगाः प्राणिनश्चाऽऽसन्रामे राज्यं प्रशासति}
{देवतानामृषीणां च मनुष्याणां च सर्वशः} %॥१५१॥

\twolineshloka
{पृथिव्यां समवायोऽभूद्रामे राज्यं प्रशासति}
{गाथामप्यत्र गायन्ति ये पुराणविदो जनाः} %॥१५२॥

\twolineshloka
{रामे निबद्धतत्त्वार्था माहात्म्यं तस्य धीमतः}
{श्यामो युवा लोहिताक्षो दीप्तास्यो मितभाषितः} %॥१५३॥

\twolineshloka
{आजानुबाहुः सुमुखः सिंहस्कन्धो महाभुजः}
{दश वर्षसहस्राणि रामो राज्यमकारयत्} %॥१५४॥

\twolineshloka
{ऋक्सामयजुषां घोषो ज्याघोषश्च महात्मनः}
{अव्युच्छिन्नोऽभवद्राष्ट्रे दीयतां भुज्यतामिति} %॥१५५॥

\twolineshloka
{सत्त्ववान्गुणसम्पन्नो दीप्यमानः स्वतेजसा}
{अतिचन्द्रं च सूर्यं च रामो दाशरथिर्बभौ} %॥१५६॥

\twolineshloka
{ईजे क्रतुशतैः पुण्यैः समाप्तवरदक्षिणैः}
{हित्वाऽयोध्यां दिवं यातो राघवो हि महाबलः} %॥१५७॥

\twolineshloka
{एवमेव महाबाहुरिक्ष्वाकुकुलनन्दनः}
{रावणं सगणं हत्वा दिवमाचक्रमे विभुः} %॥१५८॥

॥इति श्रीमहापुराणे आदिब्राह्मे विष्णोः प्रादुर्भावानुकीर्तनं नाम त्रयोदशाधिकद्विशततमोऽध्यायः॥२१३॥
