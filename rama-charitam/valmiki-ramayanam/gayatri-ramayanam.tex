\sect{गायत्री रामयाणम्}

\annotwolineshloka
{तपः स्वाध्यायनिरतं तपस्वी वाग्विदां वरम्}
{नारदं परिपप्रच्छ वाल्मीकिर्मुनिपुङ्गवम्}{१-१-१}

\annotwolineshloka
{स हत्वा राक्षसान् सर्वान् यज्ञघ्नान् रघुनन्दनः}
{ऋषिभिः पूजितः सम्यक् यथेन्द्रो विजये पुरा}{१-३०-२३}

\annotwolineshloka
{विश्वामित्रस्तु धर्मात्मा श्रुत्वा जनकभाषितम्}
{वत्स राम धनुः पश्य इति राघवमब्रवीत्}{१-६७-१२}

\annotwolineshloka
{तुष्टावास्य तदा वंशं  प्रविश्य च विशाम्पतेः}
{शयनीयं नरेन्द्रस्य तदासाद्य व्यतिष्ठत}{२-१५-२०}

\annotwolineshloka
{वनवासं हि सङ्ख्याय वासांस्याभरणानि च}
{भर्तारमनुगच्छन्त्यै सीतायै श्वशुरो ददौ}{२-४०-१५}

\annotwolineshloka
{राजा सत्यं च धर्मं च  राजा कुलवतां कुलम्}
{राजा माता पिता चैव राजा हितकरो नृणाम्}{२-६७-३४}

\annotwolineshloka
{निरीक्ष्य स मुहूर्तं तु ददर्श भरतो गुरुम्}
{उटजे राममासीनं जटामण्डलधारिणम्}{२-९९-२५}

\annotwolineshloka
{यदि बुद्धिः कृता द्रष्टुम् अगस्त्यं तं महामुनिम्}
{अद्यैव गमने बुद्धिं रोचयस्व महायशाः}{३-११-४४}

\annotwolineshloka
{भरतस्यार्यपुत्रस्य श्वश्रूणां मम च प्रभो}
{मृगरूपमिदं व्यक्तं विस्मयं जनयिष्यति}{३-४३-१७}

\annotwolineshloka
{गच्छ शीघ्रमितो राम सुग्रीवं तं महाबलम्}
{वयस्यं तं कुरु क्षिप्रमितो गत्वाऽद्य राघव}{३-७२-१७}

\annotwolineshloka
{देशकालौ प्रतीक्षस्व क्षममाणः प्रियाप्रिये}
{सुखदुःखसहः काले  सुग्रीववशगो भव}{४-२२-२०}

\annotwolineshloka
{वन्द्यास्ते तु तपः सिद्धास्तपसा वीतकल्मषाः}
{प्रष्टव्याश्चापि सीतायाः प्रवृत्तिं विनयान्वितैः}{४-४३-३४}

\annotwolineshloka
{स निर्जित्य पुरीं श्रेष्ठां लङ्कां तां कामरूपिणीम्}
{विक्रमेण महातेजा हनूमान्मारुतात्मजः}{५-४-१}

\annotwolineshloka
{धन्या देवाः सगन्धर्वाः सिद्धाश्च परमर्षयः}
{मम पश्यन्ति ये नाथं रामं राजीवलोचनम्}{५-२६-४१}

\annotwolineshloka
{मङ्गलाभिमुखी तस्य सा तदासीन्महाकपेः}
{उपतस्थे विशालाक्षी प्रयता हव्यवाहनम्}{५-५३-२६}

\annofourlineindentedshloka
{हितं महार्थं मृदु हेतुसंहितम्}
{व्यतीतकालायतिसम्प्रतिक्षमम्}
{निशम्य तद्वाक्यमुपस्थितज्वरः}
{प्रसङ्गवानुत्तरमेतदब्रवीत्}{६-१०-२७}

\annotwolineshloka
{धर्मात्मा रक्षसां श्रेष्ठः सम्प्राप्तोऽयं विभीषणः}
{लङ्कैश्वर्यं ध्रुवं श्रीमानयं प्राप्नोत्यकण्टकम्}{६-४१-६८}

\annofourlineindentedshloka
{यो वज्रपाताशनिसन्निपातान्}{न चुक्षुभे नापि चचाल राजा}
{स रामबाणाभिहतो भृशार्तः}{चचाल चापं च मुमोच वीरः}{६-५९-१४०}

\annotwolineshloka
{यस्य विक्रममासाद्य राक्षसा निधनं गताः}
{तं मन्ये राघवं वीरं नारायणमनामयम्}{६-७२-११}

\annotwolineshloka
{न ते ददर्शिरे रामं दहन्तमरिवाहिनीम्}
{मोहिताः परमास्त्रेण गान्धर्वेण महात्मना}{६-९४-२६}

\annotwolineshloka
{प्रणम्य देवताभ्यश्च ब्राह्मणेभ्यश्च मैथिली}
{बद्धाञ्जलिपुटा चेदमुवाचाग्निसमीपतः}{६-११९-२३}

\annotwolineshloka
{चलनात्पर्वतेन्द्रस्य गणा देवाश्च कम्पिताः}
{चचाल पार्वती चापि तदाऽऽश्लिष्टा महेश्वरम्}{७-१६-२६}

\annotwolineshloka
{दाराः पुत्राः पुरं राष्ट्रं भोगाच्छादनभोजनम्}
{सर्वमेवाविभक्तं नौ भविष्यति हरीश्वर}{७-३४-४१}

\annotwolineshloka
{यामेव रात्रिं शत्रुघ्नः पर्णशालां समाविशत्}
{तामेव रात्रिं सीताऽपि प्रसूता दारकद्वयम्}{७-६६-१}

\twolineshloka*
{इदं रामायणं कृत्स्नं गायत्रीबीजसंयुतम्}
{त्रिसन्ध्यं यः पठेन्नित्यं सर्वपापैः प्रमुच्यते}

॥इति श्री-गायत्री रामायणं सम्पूर्णम्॥

\closesection