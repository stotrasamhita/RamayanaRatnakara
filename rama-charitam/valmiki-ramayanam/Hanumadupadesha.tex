\sect{हनूमदुपदेशः}

\src{श्रीमद्-वाल्मीकि-रामायणम्}{सुन्दरकाण्डः}{अध्यायः ५१}{श्लोकाः १---४६}
\vakta{हनुमान्}
\shrota{रावणादयः}
\tags{concise, part}
\notes{Narration of Rama's story and prowess by Hanuman, in Ravana's court.}
\textlink{}
\translink{}

\storymeta

\twolineshloka
{तं समीक्ष्य महासत्त्वं सत्त्ववान् हरिसत्तमः}
{वाक्यमर्थवदव्यग्रस्तमुवाच दशाननम्}

\twolineshloka
{अहं सुग्रीवसन्देशादिह प्राप्तस्तवालयम्}
{राक्षसेन्द्र हरीशस्त्वां भ्राता कुशलमब्रवीत्}

\twolineshloka
{भ्रातुः शृणु समादेशं सुग्रीवस्य महात्मनः}
{धर्मार्थोपहितं वाक्यमिह चामुत्र च क्षमम्}

\twolineshloka
{राजा दशरथो नाम रथकुञ्जरवाजिमान्}
{पितेव बन्धुर्लोकस्य सुरेश्वरसमद्युतिः}

\twolineshloka
{ज्येष्ठस्तस्य महाबाहुः पुत्रः प्रियकरः प्रभुः}
{पितुर्निदेशान्निष्क्रान्तः प्रविष्टो दण्डकावनम्}

\twolineshloka
{लक्ष्मणेन सह भ्रात्रा सीतया चापि भार्यया}
{रामो नाम महातेजा धर्म्यं पन्थानमाश्रितः}

\twolineshloka
{तस्य भार्या वने नष्टा सीता पतिमनुव्रता}
{वैदेहस्य सुता राज्ञो जनकस्य महात्मनः} 

\twolineshloka
{स मार्गमाणस्तां देवीं राजपुत्रः सहानुजः}
{ऋश्यमूकमनुप्राप्तः सुग्रीवेण समागतः}

\twolineshloka
{तस्य तेन प्रतिज्ञातं सीतायाः परिमार्गणम्}
{सुग्रीवस्यापि रामेण हरिराज्यं निवेदितम्}

\twolineshloka
{ततस्तेन मृधे हत्वा राजपुत्रेण वालिनम्}
{सुग्रीवः स्थापितो राज्ये हर्यृक्षाणां गणेश्वरः}

\twolineshloka
{त्वया विज्ञातपूर्वश्च वाली वानरपुङ्गवः}
{रामेण निहतः सङ्ख्ये शरेणैकेन वानरः}

\twolineshloka
{स सीतामार्गणे व्यग्रः सुग्रीवः सत्यसङ्गरः}
{हरीन् सम्प्रेषयामास दिशः सर्वा हरीश्वरः}

\twolineshloka
{तां हरीणां सहस्राणि शतानि नियुतानि च}
{दिक्षु सर्वासु मार्गन्ते ह्यधश्चोपरि चाम्बरे}

\twolineshloka
{वैनतेयसमाः केचित् केचित्तत्रानिलोपमाः}
{असङ्गगतयः शीघ्रा हरिवीरा महाबलाः}

\twolineshloka
{अहं तु हनुमान्नाम मारुतस्यौरसः सुतः}
{सीतायास्तु कृते तूर्णं शतयोजनमायतम्}

\twolineshloka
{समुद्रं लङ्घयित्वैव तां दिदृक्षुरिहागतः}
{भ्रमता च मया दृष्टा गृहे ते जनकात्मजा}

\twolineshloka
{तद्भवान् दृष्टधर्मार्थस्तपः कृतपरिग्रहः}
{परदारान् महाप्राज्ञ नोपरोद्धुं त्वमर्हसि}

\twolineshloka
{न हि धर्मविरुद्धेषु बह्वपायेषु कर्मसु}
{मूलघातिषु सज्जन्ते बुद्धिमन्तो भवद्विधाः}

\twolineshloka
{कश्च लक्ष्मणमुक्तानां रामकोपानुवर्तिनाम्}
{शराणामग्रतः स्थातुं शक्तो देवासुरेष्वपि}

\twolineshloka
{न चापि त्रिषु लोकेषु राजन् विद्येत कश्चन}
{राघवस्य व्यलीकं यः कृत्वा सुखमवाप्नुयात्}

\twolineshloka
{तत् त्रिकालहितं वाक्यं धर्म्यमर्थानुबन्धि च}
{मन्यस्व नरदेवाय जानकी प्रतिदीयताम्}

\twolineshloka
{दृष्टा हीयं मया देवी लब्धं यदिह दुर्लभम्}
{उत्तरं कर्म यच्छेषं निमित्तं तत्र राघवः}

\twolineshloka
{लक्षितेयं मया सीता तथा शोकपरायणा}
{गृह्य यां नाभिजानासि पञ्चास्यामिव पन्नगीम्}

\twolineshloka
{नेयं जरयितुं शक्या सासुरैरमरैरपि}
{विषसंसृष्टमत्यर्थं भुक्तमन्नमिवौजसा}

\twolineshloka
{तपःसन्तापलब्धस्ते योऽयं धर्मपरिग्रहः}
{न स नाशयितुं न्याय्य आत्मप्राणपरिग्रहः}

\twolineshloka
{अवध्यतां तपोभिर्यां भवान् समनुपश्यति}
{आत्मनः सासुरैर्देवैर्हेतुस्तत्राप्ययं महान्}

\twolineshloka
{सुग्रीवो न हि देवोऽयं नासुरो न च राक्षसः}
{न दानवो न गन्धर्वो न यक्षो न च पन्नगः}

\twolineshloka
{तस्मात् प्राणपरित्राणं कथं राजन् करिष्यसि}
{ननु धर्मोपसंहारमधर्मफलसंहितम्}

\twolineshloka
{तदेव फलमन्वेति धर्मश्चाधर्मनाशनः}
{प्राप्तं धर्मफलं तावद्भवता नात्र संशयः}

\twolineshloka
{फलमस्याप्यधर्मस्य क्षिप्रमेव प्रपत्स्यसे}
{जनस्थानवधं बुद्ध्वा बुद्ध्वा वालिवधं तथा}

\twolineshloka
{रामसुग्रीवसख्यं च बुध्यस्व हितमात्मनः}
{कामं खल्वहमप्येकः सवाजिरथकुञ्जराम्}

\twolineshloka
{लङ्कां नाशयितुं शक्तस्तस्यैष तु न निश्चयः}
{रामेण हि प्रतिज्ञातं हर्यृक्षगणसन्निधौ}

\twolineshloka
{उत्सादनममित्राणां सीता यैस्तु प्रधर्षिता}
{अपकुर्वन् हि रामस्य साक्षादपि पुरन्दरः}

\twolineshloka
{न सुखं प्राप्नुयादन्यः किं पुनस्त्वद्विधो जनः}
{यां सीतेत्यभिजानासि येयं तिष्ठति ते वशे}

\twolineshloka
{कालरात्रीति तां विद्धि सर्वलङ्काविनाशिनीम्}
{तदलं कालपाशेन सीताविग्रहरूपिणा}

\twolineshloka
{स्वयं स्कन्धावसक्तेन क्षममात्मनि चिन्त्यताम्}
{सीतायास्तेजसा दग्धां रामकोपप्रपीडिताम्}

\twolineshloka
{दह्यमानामिमां पश्य पुरीं साट्टप्रतोलिकाम्}
{स्वानि मित्राणि मन्त्रींश्च ज्ञातीन्भ्रातॄन्सुतान्हितान्}

\twolineshloka
{भोगान् दारांश्च लङ्कां च मा विनाशमुपानय}
{सत्यं राक्षसराजेन्द्र शृणुष्व वचनं मम}

\twolineshloka
{रामदासस्य दूतस्य वानरस्य विशेषतः}
{सर्वाँल्लोकान् सुसंहृत्य सभूतान् सचराचरान्}

\twolineshloka
{पुनरेव तदा स्रष्टुं शक्तो रामो महायशाः}
{देवासुरनरेन्द्रेषु यक्षरक्षोगणेषु च}

\twolineshloka
{विद्याधरेषु सर्वेषु गन्धर्वेषूरगेषु च}
{सिद्धेषु किन्नरेन्द्रेषु पतत्रिषु च सर्वतः}

\twolineshloka
{सर्वभूतेषु सर्वत्र सर्वकालेषु नास्ति सः}
{यो रामं प्रतियुध्येत विष्णुतुल्यपराक्रमम्}

\twolineshloka
{सर्वलोकेश्वरस्यैवं कृत्वा विप्रियमीदृशम्}
{रामस्य राजसिंहस्य दुर्लभं तव जीवितम्}

\fourlineindentedshloka
{देवाश्च दैत्याश्च निशाचरेन्द्र}
{गन्धर्वविद्याधरनागयक्षाः}
{रामस्य लोकत्रयनायकस्य}
{स्थातुं न शक्ताः समरेषु सर्वे}

\fourlineindentedshloka
{ब्रह्मा स्वयम्भूश्चतुराननो वा}
{रुद्रस्त्रिनेत्रस्त्रिपुरान्तको वा}
{इन्द्रो महेन्द्रः सुरनायको वा}
{त्रातुं न शक्ता युधि रामवध्यम्}

\fourlineindentedshloka
{स सौष्ठवोपेतमदीनवादिनः}
{कपेर्निशम्याप्रतिमोऽप्रियं वचः}
{दशाननः कोपविवृत्तलोचनः}
{समादिशत्तस्य वधं महाकपेः}

इत्यार्षे श्रीमद्रामायणे वाल्मीकीये आदिकाव्ये चतुर्विंशतिसहस्रिकायां संहितायाम् सुन्दरकाण्डे हनूमदुपदेशो नाम एकपञ्चाशः सर्गः॥

\closesection