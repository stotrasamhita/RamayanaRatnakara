\sect{काव्य-सङ्क्षेपः}

\src{श्रीमद्-वाल्मीकि-रामायणम्}{बालकाण्डः}{अध्यायः ३}{श्लोकाः १---२९}
% \vakta{वाल्मीकिः}
\tags{concise, complete}
\notes{Summary of the Kavya by Valmiki.}
\textlink{}
\translink{}

\storymeta

\twolineshloka
{श्रुत्वा वस्तु समग्रं तद्धर्मात्मा धर्मसंहितम्}
{व्यक्तमन्वेषते भूयो यद्वृत्तं तस्य धीमतः} %||1-3-1||

\twolineshloka
{उपस्पृश्योदकं संयन्मुनिः स्थित्वा कृताञ्जलिः}
{प्राचीनाग्रेषु दर्भेषु धर्मेणान्वेषते गतिम्} %||1-3-2||

\twolineshloka
{जन्म रामस्य सुमहद्वीर्यं सर्वानुकूलताम्}
{लोकस्य प्रियतां क्षान्तिं सौम्यतां सत्यशीलताम्} %||1-3-3||

\twolineshloka
{नानाचित्राः कथाश्चान्या विश्वामित्रसहायने}
{जानक्याश्च विवाहं च धनुषश्च विभेदनम्} %||1-3-4||

\twolineshloka
{रामरामविवादं च गुणान्दाशरथेस्तथा}
{तथाभिषेकं रामस्य कैकेय्या दुष्टभावताम्} %||1-3-5||

\twolineshloka
{व्याघातं चाभिषेकस्य रामस्य च विवासनम्}
{राज्ञः शोकं विलापं च परलोकस्य चाश्रयम्} %||1-3-6||

\twolineshloka
{प्रकृतीनां विषादं च प्रकृतीनां विसर्जनम्}
{निषादाधिपसंवादं सूतोपावर्तनं तथा} %||1-3-7||

\twolineshloka
{गङ्गायाश्चाभिसन्तारं भरद्वाजस्य दर्शनम्}
{भरद्वाजाभ्यनुज्ञानाच्चित्रकूटस्य दर्शनम्} %||1-3-8||

\twolineshloka
{वास्तुकर्मनिवेशं च भरतागमनं तथा}
{प्रसादनं च रामस्य पितुश्च सलिलक्रियाम्} %||1-3-9||

\twolineshloka
{पादुकाग्र्याभिषेकं च नन्दिग्राम निवासनम्}
{दण्डकारण्यगमनं सुतीक्ष्णेन समागमम्} %||1-3-10||

\twolineshloka
{अनसूयासमस्यां च अङ्गरागस्य चार्पणम्}
{शूर्पणख्याश्च संवादं विरूपकरणं तथा} %||1-3-11||

\twolineshloka
{वधं खरत्रिशिरसोरुत्थानं रावणस्य च}
{मारीचस्य वधं चैव वैदेह्या हरणं तथा} %||1-3-12||

\twolineshloka
{राघवस्य विलापं च गृध्रराजनिबर्हणम्}
{कबन्धदर्शनं चैव पम्पायाश्चापि दर्शनम्} %||1-3-13||

\twolineshloka
{शबर्या दर्शनं चैव हनूमद्दर्शनं तथा}
{विलापं चैव पम्पायां राघवस्य महात्मनः} %||1-3-14||

\twolineshloka
{ऋष्यमूकस्य गमनं सुग्रीवेण समागमम्}
{प्रत्ययोत्पादनं सख्यं वालिसुग्रीवविग्रहम्} %||1-3-15||

\twolineshloka
{वालिप्रमथनं चैव सुग्रीवप्रतिपादनम्}
{ताराविलापसमयं वर्षरात्रिनिवासनम्} %||1-3-16||

\twolineshloka
{कोपं राघवसिंहस्य बलानामुपसङ्ग्रहम्}
{दिशः प्रस्थापनं चैव पृथिव्याश्च निवेदनम्} %||1-3-17||

\twolineshloka
{अङ्गुलीयकदानं च ऋक्षस्य बिलदर्शनम्}
{प्रायोपवेशनं चैव सम्पातेश्चापि दर्शनम्} %||1-3-18||

\twolineshloka
{पर्वतारोहणं चैव सागरस्य च लङ्घनम्}
{रात्रौ लङ्काप्रवेशं च एकस्यापि विचिन्तनम्} %||1-3-19||

\twolineshloka
{आपानभूमिगमनमवरोधस्य दर्शनम्}
{अशोकवनिकायानं सीतायाश्चापि दर्शनम्} %||1-3-20||

\twolineshloka
{अभिज्ञानप्रदानं च सीतायाश्चापि भाषणम्}
{राक्षसीतर्जनं चैव त्रिजटास्वप्नदर्शनम्} %||1-3-21||

\twolineshloka
{मणिप्रदानं सीताया वृक्षभङ्गं तथैव च}
{राक्षसीविद्रवं चैव किङ्कराणां निबर्हणम्} %||1-3-22||

\twolineshloka
{ग्रहणं वायुसूनोश्च लङ्कादाहाभिगर्जनम्}
{प्रतिप्लवनमेवाथ मधूनां हरणं तथा} %||1-3-23||

\twolineshloka
{राघवाश्वासनं चैव मणिनिर्यातनं तथा}
{सङ्गमं च समुद्रस्य नलसेतोश्च बन्धनम्} %||1-3-24||

\twolineshloka
{प्रतारं च समुद्रस्य रात्रौ लङ्कावरोधनम्}
{विभीषणेन संसर्गं वधोपायनिवेदनम्} %||1-3-25||

\twolineshloka
{कुम्भकर्णस्य निधनं मेघनादनिबर्हणम्}
{रावणस्य विनाशं च सीतावाप्तिमरेः पुरे} %||1-3-26||

\twolineshloka
{विभीषणाभिषेकं च पुष्पकस्य च दर्शनम्}
{अयोध्यायाश्च गमनं भरतेन समागमम्} %||1-3-27||

\twolineshloka
{रामाभिषेकाभ्युदयं सर्वसैन्यविसर्जनम्}
{स्वराष्ट्ररञ्जनं चैव वैदेह्याश्च विसर्जनम्} %||1-3-28||

\twolineshloka
{अनागतं च यत्किञ्चिद्रामस्य वसुधातले}
{तच्चकारोत्तरे काव्ये वाल्मीकिर्भगवानृषिः} %||1-4-29||


{॥इत्यार्षे श्रीमद्रामायणे वाल्मीकीये आदिकाव्ये बालकाण्डे तृतीयः सर्गः॥}

\closesection