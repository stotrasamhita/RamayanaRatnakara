\sect{हनूमज्जानकीसंवादोपक्रमः}

\src{श्रीमद्-वाल्मीकि-रामायणम्}{सुन्दरकाण्डः}{अध्यायः १२९}{श्लोकाः १६---३२}
\vakta{सीता}
\shrota{हनुमान्}
\tags{concise, part}
\notes{Narration of Rama's story until Sita Haranam, by Sita Herself to Hanuman.}
\textlink{}
\translink{}

\storymeta


\twolineshloka
{सोऽवतीर्य द्रुमात्तस्माद्विद्रुमप्रतिमाननः}
{विनीतवेषः कृपणः प्रणिपत्योपसृत्य च}

\twolineshloka
{तामब्रवीन्महातेजा हनूमान् मारुतात्मजः}
{शिरस्यञ्जलिमाधाय सीतां मधुरया गिरा}

\twolineshloka
{का नु पद्मपलाशाक्षि क्लिष्टकौशेयवासिनि}
{द्रुमस्य शाखामालम्ब्य तिष्ठसि त्वमनिन्दिते}

\twolineshloka
{किमर्थं तव नेत्राभ्यां वारि स्रवति शोकजम्}
{पुण्डरीकपलाशाभ्यां विप्रकीर्णमिवोदकम्}

\twolineshloka
{सुराणामसुराणां वा नागगन्धर्वरक्षसाम्}
{यक्षाणां किन्नराणां वा का त्वं भवसि शोभने}

\twolineshloka
{का त्वं भवसि रुद्राणां मरुतां वा वरानने}
{वसूनां वा वरारोहे देवता प्रतिभासि मे}

\twolineshloka
{किं नु चन्द्रमसा हीना पतिता विबुधालयात्}
{रोहिणी ज्योतिषां श्रेष्ठा श्रेष्ठा सर्वगुणान्विता}

\twolineshloka
{का त्वं भवसि कल्याणि त्वमनिन्दितलोचने}
{कोपाद्वा यदि वा मोहाद्भर्तारमसितेक्षणे}

\twolineshloka
{वसिष्ठं कोपयित्वा त्वं वासि कल्याण्यरुन्धती}
{को नु पुत्रः पिता भ्राता भर्ता वा ते सुमध्यमे}

\twolineshloka
{अस्माल्लोकादमुं लोकं गतं त्वमनुशोचसि}
{रोदनादतिनिःश्वासाद्भूमिसंस्पर्शनादपि}

\twolineshloka
{न त्वां देवीमहं मन्ये राज्ञः संज्ञावधारणात्}
{व्यञ्जनानि च ते यानि लक्षणानि च लक्षये}

\twolineshloka
{महिषी भूमिपालस्य राजकन्यासि मे मता}
{रावणेन जनस्थानाद्बलादपहृता यदि}

\twolineshloka
{सीता त्वमसि भद्रं ते तन्ममाचक्ष्व पृच्छतः}
{यथा हि तव वै दैन्यं रूपं चाप्यतिमानुषम्}

\twolineshloka
{तपसा चान्वितो वेषस्त्वं राममहिषी धुवम्}
{सा तस्य वचनं श्रुत्वा रामकीर्तनहर्षिता}

\twolineshloka
{उवाच वाक्यं वैदेही हनुमन्तं द्रुमाश्रितम्}
{पृथिव्यां राजसिंहानां मुख्यस्य विदितात्मनः}

\twolineshloka
{स्नुषा दशस्थस्याहं शत्रुसैन्यप्रमाथिनः}
{दुहिता जनकस्याहं वैदेहस्य महात्मनः}

\twolineshloka
{सीता च नाम नाम्नाहं भार्या रामस्य धीमतः}
{समा द्वादश तत्राहं राघवस्य निवेशने}

\twolineshloka
{भुआना मानुषान् भोगान् सर्वकामसमृद्धिनी}
{ततस्त्रयोदशे वर्षे राज्ये चेक्ष्वाकुनन्दनम्}

\twolineshloka
{अभिषेचयितुं राजा सोपाध्यायः प्रचक्रमे}
{तस्मिन् सम्भ्रियमाणे तु राघवस्याभिषेचने}

\twolineshloka
{कैकेयी नाम भर्तारमिदं वचनमब्रवीत्}
{न पिबेयं न खादेयं प्रत्यहं मम भोजनम्}

\twolineshloka
{एष मे जीवितस्यान्तो रामो यद्यभिषिच्यते}
{यत्तदुक्तं त्वया वाक्यं प्रीत्या नृपतिसत्तम}

\twolineshloka
{तच्चेन्न वितथं कार्यं वनं गच्छतु राघवः}
{स राजा सत्यवाग्देव्या वरदानमनुस्मरन्}

\twolineshloka
{मुमोह वचनं श्रुत्वा कैकेय्याः क्रूरमप्रियम्}
{ततस्तु स्थविरो राजा सत्ये धर्मे व्यवस्थितः}

\twolineshloka
{ज्येष्ठं यशस्विनं पुत्रं रुदन् राज्यमयाचत}
{स पितुर्वचनं श्रीमानभिषेकात् परं प्रियम्}

\twolineshloka
{मनसा पूर्वमासाद्य वाचा प्रतिगृहीतवान्}
{दद्यान्न प्रतिगृह्णीयान्न ब्रूयात् किञ्चिदप्रियम्}

\twolineshloka
{अपि जीवितहेतोर्वा रामः सत्यपराक्रमः}
{स विहायोत्तरीयाणि महार्हाणि महायशाः}

\twolineshloka
{विसृज्य मनसा राज्यं जनन्यै मां समादिशत्}
{साहं तस्याग्रतस्तूर्णं प्रस्थिता वनचारिणी}

\twolineshloka
{न हि मे तेन हीनाया वासः स्वर्गेऽपि रोचते}
{प्रागेव तु महाभागः सौमित्रिर्मित्रनन्दनः}

\twolineshloka
{पूर्वजस्यानुयात्रार्थे द्रुमचीरैरलङ्कृतः}
{ते वयं भर्तुरादेशं बहुमान्य दृढव्रताः}

\twolineshloka
{प्रविष्टाः स्म पुरादृष्टं वनं गम्भीरदर्शनम्}
{वसतो दण्डकारण्ये तस्याहममितौजसः}

\threelineshloka
{रक्षसापहृता भार्या रावणेन दुरात्मना}
{द्वौ मासौ तेन मे कालो जीवितानुग्रहः कृतः}
{ऊर्ध्वं द्वाभ्यां तु मासाभ्यां ततस्त्यक्ष्यामि जीवितम्}

इत्यार्षे श्रीमद्रामायणे वाल्मीकीये आदिकाव्ये चतुर्विंशतिसहस्रिकायां संहितायाम्
सुन्दरकाण्डे हनूमज्जानकीसंवादो नाम त्रयस्त्रिंशः सर्गः॥

\closesection