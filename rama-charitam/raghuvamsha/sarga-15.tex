\sect{पञ्चदशः सर्गः}

\twolineshloka
{कृतसीतापरित्यागः स रत्नाकरमेखलाम्}
{बुभुजे पृथिवीपालः पृथिवीमेव केवलाम्} % १५-१

\twolineshloka
{लवणेन विलुप्तेज्यास्तामिस्रेण तमभ्ययुः}
{मुनयो यमुनाभाजः शरण्यम् शरणार्थिनः} % १५-२

\twolineshloka
{अवेक्ष्य रामम् ते तस्मिन्न प्रजह्रुः स्वतेजसा}
{त्राणाभावे हि शापास्त्राः कुर्वन्ति तपसो व्ययम्} % १५-३

\twolineshloka
{प्रतिशुश्राव काकुत्स्थस्तेभ्यो विघ्नप्रतिक्रियाम्}
{धर्मसंरक्षणार्थेव प्रवृत्तिर्भुवि शार्ङ्गिणः} % १५-४

\twolineshloka
{ते रामाय वधोपायमाचख्युर्विबुधद्विषः}
{दुर्जयो लवणः शूली विशूलः प्रार्थ्यतामिति} % १५-५

\twolineshloka
{आदिदेशाथ शत्रुघ्नम् तेषाम् क्षेमाय राघवः}
{करिष्यन्निव नामास्य यथार्थमरिनिग्रहात्} % १५-६

\twolineshloka
{यः कश्चन रघूणाम् हि परमेकः परन्तपः}
{अपवाद इवोत्सर्गम् व्यावर्तयितुमीश्वरः} % १५-७

\twolineshloka
{अग्रजेन प्रयुक्ताशीस्ततो दाशरथी रथी}
{ययौ वनस्थलीः पश्यन्पुष्पिताः सुरभीरभीः} % १५-८

\twolineshloka
{रामादेशादनुगता सेना तस्यार्थसिद्धये}
{पश्चादध्यनार्थस्य धातोरधिरिवाभवत्} % १५-९

\twolineshloka
{आदिष्टवर्त्मा मुनिभिः स गच्छंस्तपसाम् वरः}
{विरराज रथप्रष्ठैर्वालखिल्यैरिवांशुमान्} % १५-१०

\twolineshloka
{तस्य मार्गवशादेका बभूव वसतिर्यतः}
{रथस्वनोत्कण्ठमृगे वाल्मीकीये तपोवने} % १५-११

\twolineshloka
{तमृषिः पूजयामास कुमारम् क्लान्तवाहनम्}
{तपःप्रभावसिद्धाभिर्विशेषप्रतिपत्तिभिः} % १५-१२

\twolineshloka
{तस्यामेवास्य यामिन्यामन्तर्वन्ती प्रजावती}
{सुतावसूत सम्पन्नौ कोशदण्डाविव क्षितिः} % १५-१३

\twolineshloka
{सन्तानश्रवणाद्भ्रातुः सौमित्रिः सौमनस्यवान्}
{प्राञ्जलिर्मुनिमामन्त्र्य प्रातर्युक्तरथो ययौ} % १५-१४

\twolineshloka
{स च प्राप मधूपघ्नम् कुम्भीनस्याश्च कुक्षिजः}
{वनात्करमिवादाय सत्त्वराशिमुपस्थितः} % १५-१५

\twolineshloka
{धूमधूम्रो वसागन्धी ज्वालाबभ्रुशिरोरुहः}
{क्रव्याद्गणपरीवारश्चिताग्निरिव जङ्गमः} % १५-१६

\twolineshloka
{अपशूलम् तमासाद्य लवणम् लक्ष्मणानुजः}
{रुरोध सम्मुखीनो हि जयो रन्ध्रप्रहारिणाम्} % १५-१७

\twolineshloka
{नातिपर्याप्तमालक्ष्य मत्कुक्षेरद्य भोजनम्}
{दिष्ट्या त्वमसि मे धात्रा भीतेनेवोपपादितः} % १५-१८

\twolineshloka
{इति सन्तर्ज्य शत्रुघ्नम् राक्षसस्तज्जिघांसया}
{प्रांशुमुत्पाटयामास मुस्तास्तम्बमिव द्रुमम्} % १५-१९

\twolineshloka
{सौमित्रेर्निशितैर्बाणैरन्तरा शकलीकृतः}
{गात्रम् पुष्परजः प्राप न शाखी नैरृतेरितः} % १५-२०

\twolineshloka
{विनाशात्तस्य वृक्षस्य रक्षस्तस्मै महोपलम्}
{प्रजिघाय कृतान्तस्य मुष्टिम् पृथगिव स्थितम्} % १५-२१

\twolineshloka
{ऐन्द्रमस्त्रमुपादाय शत्रुघ्नेन स ताडितः}
{सिकतात्वादपि परम् प्रपेदे परमाणुताम्} % १५-२२

\twolineshloka
{तमुपाद्रवदुद्यम्य दक्षिणम् दोर्निशाचरः}
{एकताल इवोत्पातपवनप्रेरितो गिरिः} % १५-२३

\twolineshloka
{कार्ष्णेन पत्रिणा शत्रुः स भिन्नहृदयः पतन्}
{अनिनाय भुवः कम्पम् जहाराश्रमवासिनाम्} % १५-२४

\twolineshloka
{वयसाम् पङ्क्तयः पेतुर्हतस्योपरि विद्विषः}
{तत्प्रतिद्वन्द्विनो मूर्ध्नि दिव्याः कुसुमवृष्टयः} % १५-२५

\twolineshloka
{स हत्वा लवणम् वीरस्तदा मेने महौजसः}
{भ्रातुः सोदर्यमात्मानमिन्द्रजिद्वधशोभिनः} % १५-२६

\twolineshloka
{तस्य संस्तूयमानस्य चरितार्थैस्तपस्विभिः}
{शुशुभे विक्रमोदग्रम् व्रीडयावनतम् शिरः} % १५-२७

\twolineshloka
{उपकूलम् च कालिन्द्याः पुरीम् पौरुषभूषणः}
{निर्ममे निर्ममोऽर्थेषु मथुराम् मधुराकृतिः} % १५-२८

\twolineshloka
{या सौराज्यप्रकाशाभिर्बभौ पौरविभूतिभिः}
{स्वर्गाभिष्यन्दवमनम् कृत्वेवोपनिवेशिता} % १५-२९

\twolineshloka
{तत्र सौधगतः पश्यन्यमुनाम् चक्रवाकिनीम्}
{हेमभक्तिमतीम् भूमेः प्रवेणीमिव पिप्रिये} % १५-३०

\twolineshloka
{सखा दशरथस्यापि जनकस्य च मन्त्रकृत्}
{सञ्चस्कारोभयप्रीत्या मैथिलेयौ यथाविधि} % १५-३१

\twolineshloka
{स तौ कुशलवोन्मृष्टगर्भक्लेदौ तदाख्यया}
{कविः कुशलवावेव चकार किल नामतः} % १५-३२

\twolineshloka
{साङ्गम् च वेदमध्याप्य किञ्चिदुत्क्रान्तशैशवौ}
{स्वकृतिम् गापयामास कविप्रथमपद्धतिम्} % १५-३३

\twolineshloka
{रामस्य मधुरम् वृत्तम् गायन्तो मातुरग्रतः}
{तद्वियोगव्यथाम् किञ्चिच्छिथिलीचक्रतुः सुतौ} % १५-३४

\twolineshloka
{इतरेऽपि रघोर्वंश्यास्त्रयस्त्रेताग्नितेजसः}
{तद्योगात्पतिवत्नीषु पत्नीष्वासन्द्विसूनवः} % १५-३५

\twolineshloka
{शत्रुघातिनि शत्रुघ्नः सुबाहौ च बहुश्रुते}
{मथुराविदिशे सून्वोर्निदधे पूर्वजोत्सुकः} % १५-३६

\twolineshloka
{भूयस्तपोव्ययो मा भूद्वाल्मीकेरिति सोऽत्यगात्}
{मैथिलीतनयोद्गीतनिःस्पन्दमृगमाश्रमम्} % १५-३७

\twolineshloka
{वशी विवेश चायोध्याम् रथ्यसंस्कारशोभिनीम्}
{लवणस्य वधात्पौरैरीक्षितोऽत्यन्तगौरवम्} % १५-३८

\twolineshloka
{स ददर्श सभामध्ये सभासद्भिरुपस्थितम्}
{रामम् सीतापरित्यागादसामान्यपतिम् भुवः} % १५-३९

\twolineshloka
{तमभ्यनन्दत्प्रणतम् लवणान्तकमग्रजः}
{कालनेमिवधात्प्रीतस्तुराषाडिव शार्ङ्गिणम्} % १५-४०

\twolineshloka
{स पृष्टः सर्वतो वार्तमाख्यद्राज्ञे न सन्ततिम्}
{प्रत्यर्पयिष्यतः काले कवेराद्यस्य शासनात्} % १५-४१

\twolineshloka
{अथ जानपदो विप्रः शिशुमप्राप्तयौवनम्}
{अवतार्याङ्कशय्यास्थम् द्वारि चक्रन्द भूपतेः} % १५-४२

\twolineshloka
{शोचनीयासि वसुधे या त्वम् दशरथाच्च्युता}
{रामहस्तमनुप्राप्य कष्टात्कष्टतरम् गता} % १५-४३

\twolineshloka
{श्रुत्वा तस्य शुचो हेतुम् गोप्ता जिह्राय राघवः}
{न ह्यकालभवो मृत्युरिक्ष्वाकुपदमस्पृशत्} % १५-४४

\twolineshloka
{स मुहूर्तम् क्षमस्वेति द्विजमाश्वास्य दुःखितम्}
{यानम् सस्मार कौबेरम् वैवस्वतजिगीषया} % १५-४५

\twolineshloka
{आत्तशस्त्रस्तदध्यास्य प्रस्थितः स रघूद्वहः}
{उच्चचार पुरस्तस्य गूढरूपा सरस्वती} % १५-४६

\twolineshloka
{राजन् प्रजासु ते कश्चिदपचारः प्रवर्तते}
{तमन्विष्य प्रशमयेर्भविष्यसि ततः कृती} % १५-४७

\twolineshloka
{इत्याप्तवचनाद्रामो विनेष्यन्वर्णविक्रियाम्}
{दिशः पपात पत्रेण वेगनिष्कम्पहेतुना} % १५-४८

\twolineshloka
{अथ धूमाभिताम्राक्षम् वृक्षशाखावलम्बिनम्}
{ददर्श कञ्चिदैक्ष्वाकस्तपस्यन्तमधोमुखम्} % १५-४९

\twolineshloka
{पृष्टनामान्वयो राज्ञा स किलाचष्ट धूमपः}
{आत्मानम् शम्बुकम् नाम शूद्रम् सुरपदार्थिनम्} % १५-५०

\twolineshloka
{तपस्यनधिकारित्वात्प्रजानाम् तमघावहम्}
{शीर्षच्छेद्यम् परिच्छिद्य नियन्ता शस्त्रमाददे} % १५-५१

\twolineshloka
{स तद्वक्त्रम् हिमक्लिष्टकिञ्जल्कमिव पङ्कजम्}
{ज्योतिष्कणाहतश्मश्रु कण्ठनालादपातयत्} % १५-५२

\twolineshloka
{कृतदण्डः स्वयम् राज्ञा लेभे शूद्रः सताम् गतिम्}
{तपसा दुश्चरेणापि न स्वमार्गविलङ्घिना} % १५-५३

\twolineshloka
{रघुनाथोऽप्यगस्त्येन मार्गसन्दर्शितात्मना}
{महौजसा संयुयुजे शरत्काल इवेन्दुना} % १५-५४

\twolineshloka
{कुम्भयोनिरलङ्कारम् तस्मै दिव्यपरिग्रहम्}
{ददौ दत्तम् समुद्रेण पीतेनेवात्मनिष्क्रयम्} % १५-५५

\twolineshloka
{स दधन्मैथिलीकण्ठनिर्व्यापारेण बाहुना}
{पश्चान्निववृते रामः प्राक्परासुर्द्विजात्मजः} % १५-५६

\twolineshloka
{तस्य पूर्वोदिताम् निन्दाम् द्विजः पुत्रसमागतः}
{स्तुत्वा निवर्तयामास त्रातुर्वैवस्वतादपि} % १५-५७

\twolineshloka
{तमध्वराय मुक्ताश्वम् रक्षःकपिनरेश्वराः}
{मेघाः सस्यमिवाम्भोभिरभ्यवर्षन्नुपायनैः} % १५-५८

\twolineshloka
{दिग्भ्यो निमन्त्रिताश्चैनमभिजग्मुर्महर्षयः}
{न भौमान्येव धिष्ण्यानि हित्वा ज्योतिर्मयान्यपि} % १५-५९

\twolineshloka
{उपशल्यनिविष्टैस्तैश्चतुर्द्वारमुखी बभौ}
{अयोध्या सृष्टलोकेव सद्यः पैतामही तनुः} % १५-६०

\twolineshloka
{श्लाघ्यस्त्यागोऽपि वैदेह्याः पत्युः प्राग्वंशवासिनः}
{अनन्यजानेः सैवासीद्यस्माज्जाया हिरण्मयी} % १५-६१

\twolineshloka
{विधेरधिकसम्भारस्ततः प्रववृते मखः}
{आसन्यत्र क्रियाविघ्ना राक्षसा एव रक्षिणः} % १५-६२

\twolineshloka
{अथ प्राचेतसोपज्ञम् रामायणमितस्ततः}
{मैथिलेयौ कुशलवौ जगतुर्गुरुचोदितौ} % १५-६३

\twolineshloka
{वृत्तम् रामस्य वाल्मीकेः कृतिस्तौ किन्नरस्वनौ}
{किम् तद्येन मनो हर्तुमलम् स्याताम् न शृण्वताम्} % १५-६४

\twolineshloka
{रूपे गीते च माधुर्यन्तयोस्तज्ज्ञैर्निवेदितम्}
{ददर्श सानुजो रामः शुश्राव च कुतूहली} % १५-६५

\twolineshloka
{तद्गीतश्रवणैकाग्रा संसदश्रुमुखी बभौ}
{हिमनिष्यन्दिनी प्रातर्निर्वातेव वनस्थली} % १५-६६

\twolineshloka
{वयोवेषविसंवादि रामस्य च तयोस्तदा}
{जनता प्रेक्ष्य सादृश्यम् नाक्षिकम्पम् व्यतिष्ठत} % १५-६७

\twolineshloka
{उभयोर्न तथा लोकः प्रावीण्येन विसिष्मिये}
{नृपतेः प्रीतिदानेषु वीतस्पृहतया यथा} % १५-६८

\twolineshloka
{गेये को नु विनेता वाम् कस्य चेयम् कृतिः कवेः}
{इति राज्ञा स्वयम् पृष्टौ तौ वाल्मीकिमशंसताम्} % १५-६९

\twolineshloka
{अथ सावरजो रामः प्राचेतसमुपेयिवान्}
{ऊरीकृत्यात्मनो देहम् राज्यमस्मै न्यवेदयत्} % १५-७०

\twolineshloka
{स तावाख्याय रामाय मैथिलीयौ तदात्मजौ}
{कविः कारुणिको वव्रे सीतायाः सम्परिग्रहम्} % १५-७१

\twolineshloka
{तात शुद्धा समक्षम् नः स्नुषा ते जातवेदसि}
{दौरात्म्याद्रक्षसस्ताम् तु नात्रत्याः श्रद्दधुः प्रजाः} % १५-७२

\twolineshloka
{ताः स्वचारित्र्यमुद्दिश्य प्रत्याययतु मैथिली}
{ततः पुत्रवतीमेनाम् प्रतिपत्स्ये त्वदाज्ञया} % १५-७३

\twolineshloka
{इति प्रतिश्रुते राज्ञा जानकीमाश्रमान्मुनिः}
{शिष्यैरानाययामास स्वसिद्धिम् नियमैरिव} % १५-७४

\twolineshloka
{अन्येद्युरथ काकुत्स्थः सन्निपात्य पुरौकसः}
{कविमाह्वाययामास प्रस्तुतप्रतिपत्तये} % १५-७५

\twolineshloka
{स्वरसंस्कारवत्यासौ पुत्राभ्यामथ सीतया}
{ऋचेवोदर्चिषम् सूर्यम् रामम् मुनिरुपस्थितः} % १५-७६

\twolineshloka
{काषायपरिवीतेन स्वपदार्पितचक्षुषा}
{अन्वमीयत शुद्धेति शान्तेन वपुषैव सा} % १५-७७

\twolineshloka
{जनास्तदालोकपथात्प्रतिसंहृतचक्षुषः}
{तस्थुस्तेऽवाङ्मुखाः सर्वे फलिता इव शालयः} % १५-७८

\twolineshloka
{ताम् दृष्टिविषये भर्तुर्मुनिरास्थितविष्टरः}
{कुरु निःसंशयम् वत्से स्ववृत्ते लोकमित्यशात्} % १५-७९

\twolineshloka
{अथ वाल्मीकिशिष्येण पुण्यमावर्जितम् पयः}
{आचम्योदीरयामास सीता सत्याम् सरस्वतीम्} % १५-८०

\twolineshloka
{वाङ्मनःकर्मभिः पत्यौ व्यभिचारो यथा न मे}
{तथा विश्वम्भरे देवि मामन्तर्धातुमर्हसि} % १५-८१

\twolineshloka
{एवमुक्ते तया साध्व्या रन्ध्रात्सद्योभवाद्भुवः}
{शातह्रदमिव ज्योतिः प्रभामण्डलमुद्ययौ} % १५-८२

\twolineshloka
{तत्र नागफणोत्क्षिप्तसिंहासननिषेदुषी}
{समुद्ररशना साक्षात्प्रादुरासीद्वसुन्धरा} % १५-८३

\twolineshloka
{सा सीतामङ्कमारोप्य भर्तृप्रणिहितेक्षणाम्}
{मामेति व्याहरत्येव तस्मिन्पातालमभ्यगात्} % १५-८४

\twolineshloka
{धरायाम् तस्य संरम्भम् सीताप्रत्यर्पणैषिणः}
{गुरुर्विधिबलापेक्षी शमयामास धन्विनः} % १५-८५

\twolineshloka
{ऋषीन्विसृज्य यज्ञान्ते सुहृदश्च पुरस्कृतान्}
{रामः सीतागतम् स्नेहम् निदधे तदपत्ययोः} % १५-८६

\twolineshloka
{युधाजितस्य सन्देशात्स देशम् सिन्धुनामकम्}
{ददौ दत्तप्रभावाय भरताय भृतप्रजः} % १५-८७

\twolineshloka
{भरतस्तत्र गन्धर्वान्युधि निर्जित्य केवलम्}
{आतोद्यम् ग्राहयामास समत्याजयदायुधम्} % १५-८८

\twolineshloka
{स तक्षपुष्कलौ पुत्रौ राजधान्योस्तदाख्ययोः}
{अभिषिच्याभिषेकार्हौ रामान्तिकमगात्पुनः} % १५-८९

\twolineshloka
{अङ्गदम् चन्द्रकेतुम् च लक्ष्मणोऽप्यात्मसम्भवौ}
{शासनाद्रघुनाथस्य चक्रे कारापथेश्वरौ} % १५-९०

\twolineshloka
{इत्यारोपितपुत्रास्ते जननीनाम् जनेश्वराः}
{भर्तृलोकप्रपन्नानाम् निवापान्विदधुः क्रमात्} % १५-९१

\twolineshloka
{उपेत्य मुनिवेषोऽथ कालः प्रोवाच राघवम्}
{रहःसंवादिनौ पश्येदावाम् यस्तम् त्यजेरिति} % १५-९२

\twolineshloka
{तथेति प्रतिपन्नाय विवृतात्मा नृपाय सः}
{आचख्यौ दिवमध्यास्व शासनात्परमेष्ठिनः} % १५-९३

\twolineshloka
{विद्वानपि तयोर्द्वाःस्थः समयम् लक्ष्मणोऽभिनत्}
{भीतो दुर्वाससः शापाद्रामसन्दर्शनार्थिना} % १५-९४

\twolineshloka
{स गत्वा सरयूतीरम् देहत्यागेन योगवित्}
{चकारावितथाम् भ्रातुः प्रतिज्ञाम् पूर्वजन्मनः} % १५-९५

\twolineshloka
{तस्मिन्नात्मचतुर्भागे प्राङ्नाकमधितस्थुषि}
{राघवः शिथिलम् तस्थौ भुवि धर्मस्त्रिपादिव} % १५-९६

\twolineshloka
{स निवेश्य कुशावत्याम् रिपुनागाङ्कुशम् कुशम्}
{शरावत्याम् सताम् सूक्तैर्जनिताश्रुलवम् लवम्} % १५-९७

\twolineshloka
{उदक्प्रतस्थे स्थिरधीः सानुजोऽग्निपुरःसरः}
{अन्वितः पतिवात्सल्याद्गृहवर्जमयोध्यया} % १५-९८

\twolineshloka
{जगृहुस्तस्य चित्तज्ञाः पदवीम् हरिराक्षसाः}
{कदम्बमुकुलस्थूलैरभिवृष्टाम् प्रजाश्रुभिः} % १५-९९

\twolineshloka
{उपस्थितविमानेन तेन भक्तानुकम्पिना}
{चक्रे त्रिदिवनिःश्रेणिः सरयूरनुयायिनाम्} % १५-१००

\twolineshloka
{यद्गोप्रतरकल्पोऽभूत्सम्मर्दस्तत्र मज्जताम्}
{अतस्तदाख्यया तीर्थम् पावनम् भुवि पप्रथे} % १५-१०१

\twolineshloka
{स विभुर्विबुधांशेषु प्रतिपन्नात्ममूर्तिषु}
{त्रिदशीभूतपौराणाम् स्वर्गान्तरमकल्पयत्} % १५-१०२

\fourlineindentedshloka
{निर्वर्त्यैवम् दशमुखशिरश्छेदकार्यम्}
{सुराणाम्विष्वक्सेनः स्वतनुमविशत्सर्वलोकप्रतिष्ठाम्}
{लङ्कानाथम् पवनतनयम् चोभयम् स्थापयित्वा}
{कीर्तिस्तम्भद्वयमिव गिरौ दक्षिणे चोत्तरे च} % १५-१०३

॥इति श्री-महाकवि-कालिदास-कृत-रघुवंश-महाकाव्ये पञ्चदशः सर्गः॥