\src{रघुवंशः}{सर्गः १०--१५}{}{}
\notes{In 6 cantos, Mahākavi Kālidāsa narrates the story of Rāma, inside the description of the entire lineage of Raghu.}
\textlink{https://sa.wikisource.org/wiki/रघुवंशम्/}
\translink{https://sanskritdocuments.org/sites/giirvaani/giirvaani/rv/intro_rv.htm}
\storymeta

\sect{दशमः सर्गः}

\twolineshloka
{पृथिवीं शासतस्तस्य पाकशासनतेजसः}
{किंचिदूनं अनूनर्द्धेः शरदां अयुतं ययौ}%॥१०.१॥

\twolineshloka
{न चोपलेभे पूर्वेषां ऋणनिर्मोक्षसाधनम्}
{सुताभिधानं स ज्योतिः सद्यः शोकतमोपहम्}%॥१०.२॥

\twolineshloka
{अतिष्ठत्प्रत्ययापेक्षसंततिः स चिरं नृपः}
{प्राङ्मन्थादनभिव्यक्तरत्नोत्पत्तिरिवार्णवः}%॥१०.३॥

\twolineshloka
{ऋष्यशृङ्गादयस्तस्य सन्तः संतानकाङ्क्षिणः}
{आरेभिरे जितात्मानः पुत्रीयां इष्टिं ऋत्विजः}%॥१०.४॥

\twolineshloka
{तस्मिन्नवसरे देवाः पौलस्त्योपप्लुता हरिम्}
{अभिजग्मुर्निदाघार्ताश्छायावृक्षं इवाध्वगाः}%॥१०.५॥

\twolineshloka
{ते च प्रापुरुदन्वतं बुबुधे चादिपूरुषः}
{अव्याक्षेपो भविष्यन्त्याः कार्यसिद्धेर्हि लक्षणम्}%॥१०.६॥

\twolineshloka
{भोगिभोगादनासीनं ददृशुस्तं दिवौकसः}
{तत्फणामण्डलोदर्चिर्मणिद्योतितविग्रहम्}%॥१०.७॥

\twolineshloka
{श्रियः पद्मनिषण्णायाः क्ष्ॐआन्तरितमेखले}
{अङ्के निक्षिप्तचरणं आस्तीर्णकरपल्लवे}%॥१०.८॥

\twolineshloka
{प्रबुद्धपुण्डरीकाक्षं बालातपनिभांशुकम्}
{दिवसं शारदं इव प्रारम्भसुखदर्शनम्}%॥१०.९॥

\twolineshloka
{प्रभानुलिप्तश्रीवत्सं लक्ष्मीविभ्रमदर्पणम्}
{कौत्सुभाख्यं अपां सारं बिभ्राणं बृहतोरसा}%॥१०.१०॥

\twolineshloka
{बाहुभिर्विटपाकारैर्दिव्याभरणभूषितैः}
{आविर्भूतं अपां मध्ये पारिजातं इवापरम्}%॥१०.११॥

\twolineshloka
{दैत्यस्त्रीगण्डलेखानां मदरागविलोपिभिः}
{हेतिभिश्चेतनावद्भिरुदीरितजयस्वनम्}%॥१०.१२॥

\twolineshloka
{मुक्तशेषविरोधेन कुलिशव्रणलक्ष्मणा}
{उपस्थितं प्राञ्जलिना विनीतेन गरुत्मता}%॥१०.१३॥

\twolineshloka
{योगनिद्रान्तविशदैः पावनैरवलोकनैः}
{भृग्वादीननुगृह्णन्तं सौख शायनिकानृषीन्}%॥१०.१४॥

\twolineshloka
{प्रणिपत्य सुरास्तस्मै शमयित्रे सुरद्विषाम्}
{अथैनं तुष्टुवुः स्तुत्यं अवाङ्मनसगोचरम्}%॥१०.१५॥

\twolineshloka
{नमो विश्वसृजे पूर्वं विश्वं तदनु बिभ्रते}
{अथ विश्वस्य संहर्त्रे तुभ्यं त्रेधास्थितात्मने}%॥१०.१६॥

\twolineshloka
{रसान्तराण्येकरसं यथा दिव्यं पयोऽश्नुते}
{देशे देशे गुणेष्वेवं अवस्थास्त्वं अविक्रियः}%॥१०.१७॥

\twolineshloka
{अमेयो मितलोकस्त्वं अनर्थी प्रार्थनावहः}
{अजितो जिष्णुरत्यन्तं अव्यक्तो व्यक्तकारणम्}%॥१०.१८॥

\twolineshloka
{एकः कारणतस्तां तां अवस्थां प्रतिपद्यसे}
{नानात्वं रागसंयोगात्स्फटिकस्य्ऽएव ते स्मृतम्}%॥१०.१९॥

\twolineshloka
{हृदयस्थं अनासन्नं अकामं त्वां तपस्विनम्}
{दयालुं अनघस्पृष्टं पुराणं अजरं विदुः}%॥१०.२०॥

\twolineshloka
{सर्वज्ञस्त्वं अविज्ञातः सर्वयोनिस्त्वं आत्मभूः}
{सर्वप्रभुरनीशस्त्वं एकस्त्वं सर्वरूपभाक्}%॥१०.२१॥

\twolineshloka
{सप्तसामोपगीतं त्वां सप्तार्णवजलेशयम्}
{सप्तार्चिर्मुखं आचख्युः सप्तलोकैकसंश्रयम्}%॥१०.२२॥

\twolineshloka
{चतुर्वर्गफलं ज्ञानं कालावस्था चतुर्युगा}
{चतुर्वर्णमयो लोकस्त्वत्तः सर्वं चतुर्मुखाथ्}%॥१०.२३॥

\twolineshloka
{अभ्यासनिगृहीतेन मनसा हृदयाश्रयम्}
{ज्योतिर्मयं विचिन्वन्ति योगिनस्त्वां विमुक्तये}%॥१०.२४॥

\twolineshloka
{अजस्य गृह्णतो जन्म निरीहस्य हतद्विषः}
{स्वपतो जागरूकस्य याथात्म्यं वेद कस्तव}%॥१०.२५॥

\twolineshloka
{शब्दादीन्विषयान्भोक्तुं चरितुं दुश्चरं तपः}
{पर्याप्तोऽसि प्रजाः पातुं औदासीन्येन वर्तितुम्}%॥१०.२६॥

\twolineshloka
{बहुधाप्यागमैर्भिन्नाः पन्थानः सिद्धिहेतवः}
{त्वय्येव निपतन्त्योघा जाह्नवीया इवार्णवे}%॥१०.२७॥

\twolineshloka
{त्वय्यावेशितचित्तानां त्वत्समर्पितकर्मणाम्}
{गतिस्त्वं वीतरागाणां अभूयःसंनिवृत्तये}%॥१०.२८॥

\twolineshloka
{प्रत्यक्षोऽप्यपरिच्छेद्यो मह्यादिर्महिमा तव}
{आप्तवागनुमानाभ्यां साध्यं त्वां प्रति का कथा}%॥१०.२९॥

\twolineshloka
{केवलं स्मरणेनैव पुनासि पुरुषं यतः}
{अनेन वृत्तयः शेषा निवेदितफलास्त्वयि}%॥१०.३०॥

\twolineshloka
{उदधेरिव रत्नानि तेजांसीव विवस्वतः}
{स्तुतिभ्यो व्यतिरिच्यन्ते दूरेण चरितानि ते}%॥१०.३१॥

\twolineshloka
{अनवाप्तं अवाप्तव्यं न ते किंचन विद्यते}
{लोकानुग्रह एवैको हेतुस्ते जन्मकरमणोः}%॥१०.३२॥

\twolineshloka
{महिमानं यदुत्कीर्त्य तव संह्रियते वचः}
{श्रमेण तदशक्त्या वा न गुणानां इयत्तया}%॥१०.३३॥

\twolineshloka
{इति प्रसादयां आसुस्तव संह्रियते वचः}
{भूतार्थव्याहृतिः सा हि न स्तुतिः परमेष्ठिनः}%॥१०.३४॥

\twolineshloka
{तस्मै कुशलसंप्रश्नव्यञ्जितप्रीतये सुराः}
{भयं अप्रलयोद्वेलादाचख्युर्नैरृतोदधेः}%॥१०.३५॥

\twolineshloka
{अथ वेलासमासन्नशैलरन्ध्रानुनादिना}
{स्वरेणोवाच भगवान्परिभूतार्णवध्वनिः}%॥१०.३६॥

\twolineshloka
{पुराणस्य कवेस्तस्य वर्णस्थानसमीरिता}
{बभूव कृतसंस्कारा चरितार्थैव भारती}%॥१०.३७॥

\twolineshloka
{बभौ स दशनज्योत्स्ना सा विभोर्वदनोद्गता}
{निर्यातशेषा चरणाद्गङ्गेवोर्ध्वप्रवर्तिनी}%॥१०.३८॥

\twolineshloka
{जाने वो रक्षसाक्रान्तावनुभावपराक्रमौ}
{अङ्गिनां तमसेवोभौ गुणौ प्रथ्ममध्यमौ}%॥१०.३९॥

\twolineshloka
{विदितं तप्यमानं च तेन मे भुवन्त्रयम्}
{अकामोपनतेनेव साधोर्हृदयं एनसा}%॥१०.४०॥

\twolineshloka
{कार्येषु चैककार्यत्वादभ्यर्थ्योऽस्मि न वज्रिणा}
{स्वयं एव हि वातोऽग्नेः सारथ्यं प्रतिपद्यते}%॥१०.४१॥

\twolineshloka
{स्वासिधारापरिहृतः कामं चक्रस्य तेन मे}
{स्थापितो दशमो मूर्धा लव्यांश इव रक्षसा}%॥१०.४२॥

\twolineshloka
{स्रष्टुर्वरातिसर्गात्तु मया तस्य दुरात्मनः}
{अत्यारूढं रिपोः सोढं चन्दनेव भोगिनः}%॥१०.४३॥

\twolineshloka
{धातारं तपसा प्रीतं ययाचे स हि राक्षसः}
{दैवात्सर्गादवध्यत्वं मर्त्येष्वास्थापराङ्मुखः}%॥१०.४४॥

\twolineshloka
{सोऽहं दाशरथिर्भूत्वा रणभूमेर्बलिक्षमम्}
{करिष्यामि शरैस्तीक्ष्णैस्तच्छिरःकमलोच्चयम्}%॥१०.४५॥

\twolineshloka
{अचिराद्वज्वभिर्भागं कल्पितं विधिवत्पुनः}
{मायाविभिरनालीढं आदास्यध्वे मिशाचरैः}%॥१०.४६॥

\twolineshloka
{वैमानिकाः पुण्यकृतस्त्यजन्तु मरुतां पथि}
{पुष्पकालोकसंक्षोभं मेघावरणतत्पराः}%॥१०.४७॥

\twolineshloka
{मोष्यध्वे स्वर्गबन्दीनां वेणीबन्धानदूषितान्}
{शापयन्त्रितपौलस्त्यबलात्कारकचग्रहैः}%॥१०.४८॥

\twolineshloka
{रावणावग्रहक्लान्तं इति वागमृतेन सः}
{अभिवृष्य मरुत्सस्यं कृष्णमेघस्तिरोदधे}%॥१०.४९॥

\twolineshloka
{पुरुहूतप्रभृतयः सुरकार्योद्यतं सुराः}
{अंशैरनुययुर्विष्णुं पुष्पैर्वायुं इव द्रुमाः}%॥१०.५०॥

\twolineshloka
{अथ तस्य विशांपत्युरन्ते काम्यस्य कर्मणः}
{पुरुषः प्रबभूवाग्नेर्विस्मयेन सहर्त्विजाम्}%॥१०.५१॥

\twolineshloka
{हेमपात्रगतं दोर्भ्यां आदधानः पयश्चरुम्}
{अनुप्रवेशादाद्यस्य पुंसस्तेनापि दुर्वहम्}%॥१०.५२॥

\twolineshloka
{प्राजापत्योपनीतं तद्(?) अन्नं प्रत्यग्रहीन्नृपः}
{वृषेव पयसां सारं आविष्कृतं उदन्वता}%॥१०.५३॥

\twolineshloka
{अनेन कथिता राज्ञो गुणास्तस्यान्यदुर्लभाः}
{प्रसूतिं चकमे तस्मिंस्त्रैलोक्यप्रभवोऽपि यथ्}%॥१०.५४॥

\twolineshloka
{स तेजो वैश्नवं पत्न्योर्विभेजे चरुसंज्ञितम्}
{द्यावापृथिव्योः प्रत्यग्रं अहर्पतिरिवातपम्}%॥१०.५५॥

\twolineshloka
{अर्चिता तस्य कौसल्या प्रिया केकयवंशजा}
{अतः संभावितां ताभ्यां सुमित्रां ऐच्छदीश्वरः}%॥१०.५६॥

\twolineshloka
{ते बहुज्ञस्य चित्तज्ञे पत्न्यौ पत्युर्महीषितः}
{चरोरर्धार्धभागाभ्यां तां अयोजयतां उभे}%॥१०.५७॥

\twolineshloka
{सापि प्रणयवत्यासीत्सपत्न्योरुभयोरपि}
{भ्रमरी वारणस्येव मदनिस्यन्दलेखयोः}%॥१०.५८॥

\twolineshloka
{ताभिर्गर्भः प्रजाभूत्यै दध्रे देवांशसम्भवः}
{सौरीभिरिव नाडीभिरमृताख्याभिरम्मयः}%॥१०.५९॥

\twolineshloka
{समं आपन्नसत्त्वास्ता रेजुरापाण्डुरत्विषः}
{अन्तर्गतफलारम्भाः सस्यानां इव संपदः}%॥१०.६०॥

\twolineshloka
{गुप्तं ददृशुरात्मानं सर्वाः स्वप्नेषु वामनैः}
{जलजासिगदाशार्ङ्गचक्रलाञ्छितमूर्तिभिः}%॥१०.६१॥

\twolineshloka
{हेमपक्षप्रभाजालं गगने च वितन्वता}
{उह्यन्ते स्म सुपर्णेन वेगाकृष्टपय्ॐउचा}%॥१०.६२॥

\twolineshloka
{बिभ्रत्या कौस्तुभं न्यासं स्तनान्तरविलम्बिनम्}
{पर्युपास्यन्त लक्ष्म्या च पद्मव्यजनहस्तया}%॥१०.६३॥

\twolineshloka
{कृताभिषेकैर्दिव्यायां त्रिस्रोतसि च सप्तभिः}
{ब्रह्म र्षिभिः परं ब्रह्म गृणध्बिरुपतस्थिरे}%॥१०.६४॥

\twolineshloka
{ताभ्यस्तथाविधान्स्वप्नाञ्छ्रुत्वा प्रीतो हि पार्थिवः}
{मेने परार्ध्यं आत्मानं गुरुत्वेन जग्द्गुरोः}%॥१०.६५॥

\twolineshloka
{विभक्तात्मा विभुस्तासां एकः कुषिष्वनेकधा}
{उवास प्रतिमाचन्द्रः प्रसन्नानां अपां इव}%॥१०.६६॥

\twolineshloka
{अथाग्रमहिषी राज्ञः प्रसूतिसमये सती}
{पुत्रं तमोऽपहं लेभे नक्तं ज्योतिरिवौषधिः}%॥१०.६७॥

\twolineshloka
{राम इत्यभिरामेण तेनाप्रतिम तेजसा}
{नामधेयं गुरुश्चक्रे जगत्प्रथममङ्गलम्}%॥१०.६८॥

\twolineshloka
{रघुवंशप्रदीपेन तेनाप्रतिम तेजसा}
{रक्षागृहगता दीपाः प्रत्यादिष्ट इवाभवन्}%॥१०.६९॥

\twolineshloka
{शय्यागतेन रामेण माता शातोदरी बभौ}
{सैकताम्भोजबलिना जाह्नवीव शरत्कृशा}%॥१०.७०॥

\twolineshloka
{कैकेय्यास्तनयो जज्ञे भरतो नाम शीलवान्}
{जनयित्रीं अलंचक्रे यः प्रश्रय इव श्रियम्}%॥१०.७१॥

\twolineshloka
{सुतौ लक्ष्मणशत्रुघ्नौ सुमित्रा सुषुवे यमौ}
{सम्यगागमिता विद्या प्रबोधविनयाविव}%॥१०.७२॥

\twolineshloka
{निर्दोषं अभवत्सर्वं आविष्कृतगुणं जगथ्}
{अन्वगादिव हि स्वर्गो गां गतं पुरुषोत्तमम्}%॥१०.७३॥

\twolineshloka
{तस्योदये चतुर्मूर्तेः पौलस्त्यचकितेश्वराः}
{विरजस्कैर्नभस्वद्भिर्दिश उच्छ्वसिता इव}%॥१०.७४॥

\twolineshloka
{कृशानुरपधूमत्वात्प्रसन्नत्वात्प्रभाकरः}
{रक्षिविप्रकृतावास्तां अपविद्धशुचाविव}%॥१०.७५॥

\twolineshloka
{दशाननकिरीटेभ्यस्तत्क्षणं राक्षसश्रियः}
{मणिव्याजेन पर्यस्ताः पृथिव्यां अश्रुबिन्दवः}%॥१०.७६॥

\twolineshloka
{पुत्रजन्मप्रवेश्यानां तूर्याणां तस्य पुत्रिणः}
{आरम्भं प्रथमं चक्रुर्देवधुन्दुभयो दिवि}%॥१०.७७॥

\twolineshloka
{संतानकमयी वृष्टिर्भवने चास्य पेतुषी}
{समङ्गलोपचाराणां सैवादिरचनाभवथ्}%॥१०.७८॥

\twolineshloka
{कुमाराः कृतसंस्कारास्ते धात्रिस्तन्य पायिनः}
{आनन्देनाग्रजेनेव समं ववृधिरे पितुः}%॥१०.७९॥

\twolineshloka
{स्वाभाविकं विनीतत्वं तेषं विनयकर्मणा}
{मुमूर्छ सहजं तेजो हविषेव हविर्भुजाम्}%॥१०.८०॥

\twolineshloka
{परस्पराविरुद्धास्ते तद्रघोरनघं कुलम्}
{अलं उद्द्योतयां आसुर्देवारण्यं इवर्तवः}%॥१०.८१॥

\twolineshloka
{समानेऽपि हि सौभ्रात्रे यथोभौ रामलक्ष्मणौ}
{तथा भरतशत्रुघ्नौ प्रीत्या द्वन्द्वं बभूवतुः}%॥१०.८२॥

\twolineshloka
{तेषां द्वयोर्द्वयोरैक्यं बिभिदे न कदाचन}
{यथा वायुविभावस्वोर्यथा चन्द्रसमुद्रयोः}%॥१०.८३॥

\twolineshloka
{ते प्रजानां प्रजानाथास्तेजसा प्रश्रयेण च}
{मनो जह्रुर्निदाघान्ते श्यामाभ्रा दिवसा इव}%॥१०.८४॥

\twolineshloka
{स चतुर्धा बभौ व्यस्तः प्रसवः पृथिवीपतेः}
{धर्मार्थकाममोक्षाणां अवतार इवाङ्गभाक्}%॥१०.८५॥

\twolineshloka
{गुणैराराधयां आसुस्ते गुरुं गुरुवत्सलाः}
{तं एव चतुर्नतेशं रत्नैरिव महार्णवाः}%॥१०.८६॥

\twolineshloka
{सुरगज इव दन्तैर्भग्नदैत्यासिधारैर्नय इव पणबन्धव्यक्तयोगैरुपायैः}
{हरिरिव युगदीर्घैर्दोर्भिरंषैस्तदीयैः पतिरवनिपतीनां तैश्चकाशे चतुर्भिः}%॥१०.८७॥

॥इति श्री-महाकवि-कालिदास-कृत-रघुवंश-महाकाव्ये दशमः सर्गः॥

\sect{एकादशः सर्गः}

\twolineshloka
{कौशिकेन स किल क्षितीश्वरो राममध्वरविघातशान्तये}
{काकपक्षधरमेत्य याचितस्तेजसां हि न वयः समीक्ष्यते}%॥ ११.१ ॥

\twolineshloka
{कृच्छ्रलब्धमपि लब्धवर्णभाक्तं दिदेश मुनये सलक्ष्मणम्}
{अप्यसुप्रणयिनां रघोः कुले न व्यहन्यत कदाचिदर्थिता}%॥ ११.२ ॥

\twolineshloka
{यावदादिशति पार्थिवस्तयोर्निर्गमाय पुरमार्गसत्क्रियाम्}
{तावदाशु विदधे मरुत्सखैः सा सपुष्पजलवर्षिभिर्घनैः}%॥ ११.३ ॥

\twolineshloka
{तौ निदेशकरणोद्यतौ पितुर्धन्विनौ चरणयोर्निपेततुः}
{भूपतेरपि तयोः प्रवत्स्यतोर्नम्रयोरुपरि बाष्पबिन्दवः}%॥ ११.४ ॥

\twolineshloka
{तौ पितुर्नयनजेन वारिणा किंचिदुक्षितशिखण्डकावुभौ}
{धन्विनौ तमृषिमन्वगच्छतां पौरदृष्टिकृतमार्गतोरणौ}%॥ ११.५ ॥

\twolineshloka
{लक्ष्मणानुचरमेव राघवं नेतुमैच्छदृषिरित्यसौ नृपः}
{आशिषं प्रयुयुजे न वाहिनीं सा हि रक्षणविधौ तयोः क्षमा}%॥ ११.६ ॥

\twolineshloka
{रेजतुश्च सुतरां महौजसः कौशिकस्य पदवीमनुद्रुतौ}
{उत्तरां प्रति दिशं विवस्वतः प्रस्थितस्य मधुमाधवाविव}%॥ ११.६* ॥

\twolineshloka
{मातृवर्गचरणस्पृषौ मुनेस्तौ प्रपद्य पदवीं महौजसः}
{रेजतुर्गतिवशात्प्रवर्तिनौ भास्करस्य मधुमाधवाविव}%॥ ११.७ ॥

\twolineshloka
{वीचिलोलभुजयोस्तयोर्गतं शैशवाच्चपलमप्यशोभत}
{तोयदागम इवोद्ध्यभिद्ययोर्नामधेयसदृशं विचेष्टितम्}%॥ ११.८ ॥

\twolineshloka
{तौ बलातिबलयोः प्रभावतो विद्ययोः पथि मुनिप्रदिष्टयोः}
{मम्लतुर्न मणिकुट्टिमोचितौ मातृपार्श्वपरिवर्तिनाविव}%॥ ११.९ ॥

\twolineshloka
{पूर्ववृत्तकथितैः पुराविदः सानुजः पितृसखस्य राघवः}
{उह्यमान इव वाहनोचितः पादचारमपि न व्यभावयत्}%॥ ११.१० ॥

\twolineshloka
{तौ सरांसि रसवद्भिरम्बुभिः कूजितैः श्रुतिसुखैः पतत्रिणः}
{वायवः सुरभिपुष्परेणुभिश्छायया च जलदाः सिषेविरे}%॥ ११.११ ॥

\twolineshloka
{नाम्भसां कमलशोभिनां तथा शाखिनां न च परिश्रमच्छिदाम्}
{दर्शनेन लघुना यथा तयोः प्रीतिमापुरुभयोस्तपस्विनः}%॥ ११.१२ ॥

\twolineshloka
{स्थाणुदग्धवपुषस्तपोवनं प्राप्य दाशरथिरात्तकार्मुकः}
{विग्रहेण मदनस्य चारुणा सोऽभवत्प्रतिनिधिर्न कर्मणा}%॥ ११.१३ ॥

\twolineshloka
{तौ सुकेतुसुतया खिलीकृते कौशिकाद्विदितशापया पथि}
{निन्यतुः स्थलनिवेशितातटनी लीलयैव धनुषी अधिज्यताम्}%॥ ११.१४ ॥

\twolineshloka
{ज्यानिनिआदमथ गृह्णती तयोः प्रादुरास बहूलक्षपा छविः}
{ताडका चलकपालकुण्डला कालिकेव निबिडा बलाकिनी}%॥ ११.१५ ॥

\twolineshloka
{तीव्रवेगधुतमार्गवृक्षया प्रेतचीवरवसा स्वनोग्रया}
{अभ्यभावि भरताग्रजस्तया वात्ययेव पितृकाननोत्थया}%॥ ११.१६ ॥

\twolineshloka
{उद्यतैकभुजयष्टिमायतीं श्रोणिलम्बिपुरुषान्त्रमेखलाम्}
{तां विलोक्य वनितावधे घृणां पत्त्रिणा सह मुमोच राघवः}%॥ ११.१७ ॥

\twolineshloka
{यच्चकार विवरं शिलाघने ताडकोरसि स रामसायकः}
{अप्रविष्टविषयस्य रक्षसां द्वारतामगमदन्तकस्य तत्}%॥ ११.१८ ॥

\twolineshloka
{बाणभिन्नहृदया निपेतुषी सा स्वकानभुवं न केवलाम्}
{विष्टपत्रयपराजयस्थिरां रावणश्रियमपि व्यकम्पयत्}%॥ ११.१९ ॥

\twolineshloka
{राममन्मथशरेण ताडिता दुःसहेन दृदये निशाचरी}
{गन्धवद्रुधिरचन्दनोक्षिता जीवितेशवसतिं जगाम सा}%॥ ११.२० ॥

\twolineshloka
{नैरृतघ्नमथ मन्त्रवन्मुनेः प्रापदस्त्रमवदानतोषितात्}
{ज्योतिरिन्धनैपाति भास्करात्सूर्यकान्त इव ताडकान्तकः}%॥ ११.२१ ॥

\twolineshloka
{वामनाश्रमपदं ततः परं पावनं श्रुअमृषेरुपेयिवान्}
{उन्मनाः प्रथमजन्मचेष्टितान्यस्मरन्नपि बभूव राघवः}%॥ ११.२२ ॥

\twolineshloka
{आससाद मुनिरात्मनस्ततः शिष्यवर्गपरिकल्पितार्हणम्}
{बद्धपल्लवपुटाञ्जलिद्रुमं दर्शनोन्मुख मृगं तपोवनम्}%॥ ११.२३ ॥

\twolineshloka
{तत्र दीक्षितमृषिं रकषतुर्विघ्नतो दशरथात्मजौ शरैः}
{लोकमन्धतमसात्क्रमोदितौ रशामिभिः शशिदिवाकराविव}%॥ ११.२४ ॥

\twolineshloka
{वीक्ष्य वेदिमथ रक्तबिन्दुभिर्बन्धुजीवपृथुभिः प्रदूषिताम्}
{संभ्रमोऽभवदपोढकर्मणामृत्विजां च्युतविकङ्कतस्रुचाम्}%॥ ११.२५ ॥

\twolineshloka
{उन्मुखः सपदि लक्ष्मणाग्रजो बाणमाश्रयमुखात्समुद्धरन्}
{रक्षसां बलमपश्यदम्बरे गृध्रपक्षपवनेरितध्वजम्}%॥ ११.२६ ॥

\twolineshloka
{तत्र यावधिपती मखद्विषां तौ शरव्यमकरोत्स नेतरान्}
{किं महोरगविसर्पिविक्रमो राजिलेषु गरुडः प्रवर्तते}%॥ ११.२७ ॥

\twolineshloka
{सोऽस्त्रमुग्रजवमस्त्रकोविदः संदधे धनुषि वायुदैवतम्}
{तेन शैलगुरुमप्यपातयत्पाण्डुपत्त्रमिव ताडकासुतम्}%॥ ११.२८ ॥

\twolineshloka
{यः सुबाहुरिति राक्षसोऽपरस्तत्र तत्र विससर्प मायया}
{तं क्षुरप्रशकलीकृतां कृती पत्त्रिणां व्यभजदाश्रमाद्बहिः}%॥ ११.२९ ॥

\twolineshloka
{इत्यपास्तमखविघ्नयोस्तयोः सांयुगीनमभिनन्द्य विक्रमम्}
{ऋत्विजः कुलपतेर्यथाक्रमं वाग्यतस्य निरवर्तयन् क्रियाः}%॥ ११.३० ॥

\twolineshloka
{तौ प्रणामचलकाकपक्षकौ भ्रातरावभृथाप्लुतो मुनिः}
{आशिषामनुपदं समस्पृशद्दर्भपाटिततलेन पाणिना}%॥ ११.३१ ॥

\twolineshloka
{तं न्यमन्त्रयत संभृतक्रतुर्मैथिलः स मिघिलां व्रजन् वशी}
{राघवावपि निनाय बिभ्रतौ तद्धनुःश्रवणजं कुतूहलम्}%॥ ११.३२ ॥

\twolineshloka
{तैः शिवेषु वसतिर्गताध्वभिः सायमाश्रमतरुष्वगृह्यत}
{येषु दीर्घतपसः परिग्रहो वासवक्षणकलत्रतां ययौ}%॥ ११.३३ ॥

\twolineshloka
{प्रत्यपद्यत चिराय यत्पुनश्चारु गौतमवधूः शिलामयी}
{स्वं वपुः स किल किलिबिषच्छिदां रामपादरजसामनुग्रहः}%॥ ११.३४ ॥

\twolineshloka
{राघवान्वितमुपस्थितं मुनिं तं निशम्य जनको जनेश्वरः}
{अर्थकामसहितं सपर्यया देहबद्धमिव धर्ममभ्यगात्}%॥ ११.३५ ॥

\twolineshloka
{तौ विदेहनगरीनिवासिनां गां गताविव दिवः पुनर्वसू}
{मन्यते स्म पिबतां विलोचनैः पक्ष्मपातमपि वञ्चनां मनः}%॥ ११.३६ ॥

\twolineshloka
{यूपवत्यवसिते किर्याविधौ कालवित्कुशिकवंशवर्धनः}
{राममिष्वसनदर्शनोत्सुकं मैथिलाय कथयां ब्वभूव सः}%॥ ११.३७ ॥

\twolineshloka
{तस्य वीक्ष्य ललितं वपुः शिशोः पार्थिवः प्रथितवंशजन्मनः}
{स्वं विचिन्त्य च धनुर्दुरानमं पीडितो दुहिट्शुल्कसंस्थया}%॥ ११.३८ ॥

\twolineshloka
{अब्रवीच्च भगवन्मतङ्गजैर्यद्बृहद्भिरपि कर्म दुष्करम्}
{तत्र नाहमनुमन्तुमुत्सहे मोघवृत्ति कलभस्य चेष्टितम्}%॥ ११.३९ ॥

\twolineshloka
{ह्रेपिता हि बहवो नरेश्वरास्तेन तात धनुषा धनुर्भृतः}
{ज्यानिघातकठिनत्वचो भुजान् स्वान् विधूय धिगिति प्रतस्थिरे}%॥ ११.४० ॥

\twolineshloka
{प्रत्युवाच तमृषिर्निशम्यतां सारतोऽयमथ वा कृतं गिरा}
{चाप एव भवतो भविष्यति व्यक्तशक्तिरशनिर्गिराविव}%॥ ११.४१ ॥

\twolineshloka
{एवमाप्तवचनात्स पौरुषं काकपक्षकधरेऽपि राघवे}
{श्रद्दधे त्रिदशगोपमात्रके दाहशक्तिमिव कृष्णवर्त्मनि}%॥ ११.४२ ॥

\twolineshloka
{व्यादिदेश गणः सपार्श्वगान् कर्मुकाभरणाय मैथिलः}
{तैजसय धनुषः प्रवृत्तये तोयदानिव सहस्रलोचनः}%॥ ११.४३ ॥

\twolineshloka
{तत्प्रसुप्तभुजगेन्द्रभीषणं वीक्ष्य दाशरथिराददे धनुः}
{विद्रुतक्रतुमृगानौसारिणं येन बाणमसृजद्वृषध्वजः}%॥ ११.४४ ॥

\twolineshloka
{आततज्यमकरोत्स संसदा विस्मयस्तिमितनेत्रमीक्षितः}
{शैलसारमपि नातियत्नतः पुष्पचापमिव पेशलं स्मरः}%॥ ११.४५ ॥

\twolineshloka
{भज्यमानमतिमात्रकर्षणात्तेन वज्रपरुषस्वनं धनुः}
{भार्गवाय दृढमन्यवे पुनः क्षत्रमुद्यतमिति न्यवेदयत्}%॥ ११.४६ ॥

\twolineshloka
{दृष्टसारमथ रुद्रकार्मुके वीर्यशुल्कमभिनन्द्य मैथिलः}
{राघवाय तनयामयोनिजां रूपिणीं श्रियमिव न्यवेदयत्}%॥ ११.४७ ॥

\twolineshloka
{मैथिलः सपदि सत्यसंगरो राघवाय तनयामयोनिजां}
{संनिधौ द्युतिमतस्तपोनिधेरग्निसाक्षिक इवातिसृष्टवान्}%॥ ११.४८ ॥

\twolineshloka
{प्राहिणोच्च महितं महाद्युतिः कोसलाधिपतये पुरोधसम्}
{भृत्यभावि दुहितुः परिग्रहाद्दिश्यतां कुलमिदं निमेरिति}%॥ ११.४९ ॥

\twolineshloka
{उत्सुकश्च सुतदारकर्मणा सोऽभवद्गुरुरुपागतश्च तम्}
{गौतमस्य तनयोऽनुकूलवाक्प्रार्थितं हि सुकृतामकालहृत्}%॥ ११.४९* ॥

\twolineshloka
{अन्वियेष सदृशीं स च स्नुषां प्राप चैनमनुकूलवाग्द्विजः}
{सद्य एव सुकृतां हि पच्यते कल्पवृक्षफल धर्मि काङ्क्षितम्}%॥ ११.५० ॥

\twolineshloka
{तस्य कल्पितपुरस्क्रियाविधेः शुश्रुवान् वचनमग्रजन्मनः}
{उच्चचाल वलभितसखो वशी सैन्यरेणुमुषितार्कदीधितिः}%॥ ११.५१ ॥

\twolineshloka
{आससाद मिथिलां स वेष्टयन् पिडितोपवनपादपां बलैः}
{प्रीतिरोधमसहिष्ट सा पुरी स्त्रीव कान्तपरिभोगमायतम्}%॥ ११.५२ ॥

\twolineshloka
{तौ समेत्य समयस्थितावुभौ भूपती वरुणवासवोपमौ}
{कन्यकातनयकौतुकक्रियां स्वप्रभावसदृशीं वितेनतुः}%॥ ११.५३ ॥

\twolineshloka
{पार्थिवीमुदवहद्रघूद्वहो लक्ष्मणस्तदनुजामथोर्मिलाम्}
{यौ तयोरवरजौ वरौजसौ तौ कुशध्वजसुते सुमध्यमे}%॥ ११.५४ ॥

\twolineshloka
{ते चतुर्थसहितास्त्रयो बभुः सूनवो नववधूपरिग्रहाः}
{सामदानविधिभेदविग्रहाः सिद्धिमन्त इव तस्य भूपतेः}%॥ ११.५५ ॥

\twolineshloka
{ता नराधिपसुता नृपात्मजैस्ते च ताभिरगमन् कृतार्थताम्}
{सोऽभवद्वरवधूसमागमः प्रत्ययप्रकृतियोग्संनिभः}%॥ ११.५६ ॥

\twolineshloka
{एवमात्तरतिरात्मसंभवांस्तान्निवेश्य चतुरोऽपि तत्र सः}
{अध्वसु त्रिषु विसृष्टमैथिलः स्वां पुरीं दशरथो न्यवर्तत}%॥ ११.५७ ॥

\twolineshloka
{तस्य जातु मरुतः प्रतीपगा वर्त्मसु धव्जतरुप्रमाथिनः}
{चिक्लिशुर्भृशतया वरूथिनीमुत्तटा इव नदीरयाः स्थलीम्}%॥ ११.५८ ॥

\twolineshloka
{लक्ष्यते स्म तदनन्तरं रविर्बद्धभीमपैर्वेषमण्डलः}
{वैनतेयशमितस्य भोगिनो भोगवेष्टित इव च्युतो मणिः}%॥ ११.५९ ॥

\twolineshloka
{श्येनपक्षपरिधूसरालकाः सांध्यमेघरुधिरार्द्रवाससः}
{अङ्गना इव रजस्वला दिशो नो बभूवुरवलोकनक्षमाः}%॥ ११.६० ॥

\twolineshloka
{भास्करश्च दिशमध्युवास यां तां श्रिताः प्रतिभयं ववाशिरे}
{क्षत्रशोणितपितृक्रियोचितं चोदयन्त्य इव भार्गवं शिवाः}%॥ ११.६१ ॥

\twolineshloka
{तत्प्रतीपपवनादि वैकृतं प्रेक्ष्य शान्तिमधिकृत्य कृत्यवित्}
{अन्वयुङ्क्त गुरुमीश्वरः क्षितेः स्वन्तमित्यलघयत्स तद्व्यथाम्}%॥ ११.६२ ॥

\twolineshloka
{तेजसः सपदि राशिरुत्थितः प्रादुरास किल वाहिनीमुखे}
{यः प्रमृज्य नयनानि सैनिकैर्लक्षणीयपुरुषाकृतिश्चिरात्}%॥ ११.६३ ॥

\twolineshloka
{पित्र्यमंशमुपवीतलक्षणं मातृकं च धनुरूर्जितं दधत्[१]}
{यः ससोम इव घर्मदीधितिः सद्विजिह्व इव चन्दनद्रुमः}%॥ ११.६४ ॥

\twolineshloka
{येन रोषपरुषात्मनः पितुः शासने स्थिभिदोऽपि तस्थुषा}
{वेपमानजननीशिरश्छिदा प्रागजीयत घृणा ततो मही}%॥ ११.६५ ॥

\twolineshloka
{अक्षभीजवलयेन निबभौ दक्षिणश्रवणसंस्थितेन यः}
{क्षत्रियान्तकरणैकविंशतेर्व्याजपूर्वगणनामिवोद्वहन्}%॥ ११.६६ ॥

\twolineshloka
{तं पितुर्वधभवेन मन्युना राजवंशनिधनाय दीक्षितम्}
{बालसूनुरवलोक्य भार्गवं स्वां दशां च विषसाद पार्थिवः}%॥ ११.६७ ॥

\twolineshloka
{रामनाम इति तुल्यमात्मजे वर्तमानमहिते च दारुणे}
{हृद्यमस्य भयदायि चाभवद्रत्नजातमिव हारसर्पयोः}%॥ ११.६८ ॥

\twolineshloka
{अर्घ्यमर्घ्यमिति वादिनं नृपं सोऽनवेक्ष्य भरताग्रजो यतः}
{क्षत्रकोपदहनार्चिषं ततः संदधे दृशमुदग्रतारकाम्}%॥ ११.६९ ॥

\twolineshloka
{तेन कार्मुकनिषक्तमुष्टिना राघवो विगतभीः पुरोगतः}
{अङ्गुलीविवरचारिणं शरं कुर्वता निजगदे युयुत्सुना}%॥ ११.७० ॥

\twolineshloka
{क्षत्रजातमपकारि वैरि मे तन्निहत्य बहुशः शमं गतः}
{सुप्तसर्प इव दण्डघट्टनाद्रोषितोऽस्मि तव विक्रमश्रवात्}%॥ ११.७१ ॥

\twolineshloka
{मैथिलस्य धनुरन्यपार्थिवैस्त्वं किलानमितपूर्वमक्षणोः}
{तन्निशम्य बहवता समर्थये वीर्यशृङ्गमिव भग्नमात्मनः}%॥ ११.७२ ॥

\twolineshloka
{अन्यदा जगति राम इत्ययं शब्द उच्चरित एव मामगात्}
{व्रीडमावहति मे स संप्रति व्यस्तवृत्तिरुदयोन्मुखे त्वयि}%॥ ११.७३ ॥

\twolineshloka
{बिभ्रतोऽस्त्रमचलेऽप्यकुण्ठितं द्वौ मतौ मम रिपू समागसौ}
{होमधेनुहरणाच्च हैहयस्त्वं च इर्तिमपहर्तुमुद्यतः}%॥ ११.७४ ॥

\twolineshloka
{क्षत्रियान्तकरणोऽपि विक्रमस्तेन मामवति नाजिते त्वयि}
{पावकस्य महिमा स गण्यते कक्षवज्ज्वलति सागरेऽपि यः}%॥ ११.७५ ॥

\twolineshloka
{विद्धि चात्तबलमोजसा हरेरैश्वरं धनुरभाजि यत्त्वया}
{खातमूलमनिलो नदीरयैः पातयत्यपि मृदुस्तटद्रुमम्}%॥ ११.७६ ॥

\twolineshloka
{तन्मदीयमिदमायुधं ज्यया संगमय्य सशरं विकृष्यताम्}
{तिष्ठतु प्रधनमेवमप्यहं तुल्यबाहुतरसा जितस्त्वया}%॥ ११.७७ ॥

\twolineshloka
{कातरोऽसि यदि वोद्गतार्चिषा तर्जितः परशुधारया मम}
{ज्यानिघातकठिनाङ्गुलिर्वृथा बध्यतामभययाचनाञ्जलिः}%॥ ११.७८ ॥

\twolineshloka
{एवमुक्तवति भीमदर्शने भार्गवे स्मितविकम्पिताधरः}
{तद्धनुर्ग्रहणमेव राघवः प्रत्यपद्यत समर्थमुत्तरम्}%॥ ११.७९ ॥

\twolineshloka
{पूर्वजन्मधनुषा समागतः सोऽतिमात्रलघुदर्शनोऽभवत्}
{केवलोऽपि सुभगो नवाम्बुदः किं पुनस्त्रिदशचापलाञ्छितः}%॥ ११.८० ॥

\twolineshloka
{तेन भूमिनिहितैककोटि तत्कार्मुकं च बलिनाधिरोपितम्}
{निष्प्रभश्च रिपुरास भूभृतां धूमशेष इव धूमकेतनः}%॥ ११.८१ ॥

\twolineshloka
{तावुभावपि परस्परस्थितौ वर्धमानपरिहीनतेजसौ}
{पश्यति स्म जनता दिनात्यये पार्वणौ शशिदिवाकराविव}%॥ ११.८२ ॥

\twolineshloka
{तं कृपामृदुरवेक्ष्य भार्गवं राघवः स्खलितवीर्यमात्मनि}
{स्वं च संहितममोघमाशुगं व्याजहार हरसूनसंनिभः}%॥ ११.८३ ॥

\twolineshloka
{न प्रहर्तुमलमस्मि निदयं विप्र इत्यभिभवत्यपि त्वयि}
{शंष किं गतिमनेन पत्त्रिणा हन्मि लोकमुत ते मखार्जितम्}%॥ ११.८४ ॥

\twolineshloka
{प्रत्युवाच तमृषिर्न तत्त्वतस्त्वां न वेद्मि पुरुषं पुरातनम्}
{गां गतस्य तव धाम वैष्णवं कोपितो ह्यसि मया दिदृक्षुणा}%॥ ११.८५ ॥

\twolineshloka
{भस्मसात्कृतवतः पितृद्विषः पात्रसाच्च वसुधां ससागराम्}
{आहितो जयविपर्ययोऽपि मे श्लाघ्य एव परमेष्ठिना त्वया}%॥ ११.८६ ॥

\twolineshloka
{तद्गतिं मतिमतां वरेप्सितां पुण्यतीर्थगमनाय रक्ष मे}
{पीडयिष्यति न मां खिलीकृता स्वर्गपद्धतिरभोगलोलुपम्}%॥ ११.८७ ॥

\twolineshloka
{प्रत्यपद्यत तथेति राघवः प्राङ्मुखश्च विससर्ज सायकम्}
{भार्गवस्य सुकृतोऽपि सोऽभवत्स्वर्गमार्गपरिघो दुरत्ययः}%॥ ११.८८ ॥

\twolineshloka
{राघवोऽपि चरणौ तपोनिधेः क्षम्यतामिति वदन् समस्पृषत्}
{निर्जितेषु तरसा तरस्विनां शत्रुषु प्रणतिरेव कीर्तये}%॥ ११.८९ ॥

\twolineshloka
{राजसत्वमवधूय मातृकं पित्र्यमस्मि गमितः शमं यदा}
{नन्वनिन्दितफलो मम त्वया निग्रहोऽप्ययमनुग्रहीकृतः}%॥ ११.९० ॥

\twolineshloka
{साधु याम्यहमविघ्नमस्तु ते देवकार्यमुपपादयिष्यतः}
{ऊचिवानिति वचः सलक्ष्मणं लक्ष्मणाग्रजमृषिस्तिरोदधे}%॥ ११.९१ ॥

\twolineshloka
{स्वं निवेश्य किल धाम राघवे वैष्णवं विदितविष्णुतेजसि}
{स्वस्तिदानमधिकृत्य चाक्षयं भार्गवोऽथ निजमाश्रमं ययौ}%॥ ११.९१* ॥

\twolineshloka
{तस्मिन् गते विजयिनं परिरभ्य रामं स्नेहादमन्यत पिता पुनरेव जातम्}
{तस्याभवत्क्षणशुचः परितोषलाभः कक्षाग्निलङ्घिततरोरिव वृष्टिपातः}%॥ ११.९२ ॥

\twolineshloka
{अथ पथि गमयित्वा कॢपरम्योपकार्ये कतिचिदवनिपालः शर्वरीः शर्वकल्पह्}
{पुरमविशदयोध्यां मैथिलीदर्शनीनां कुवलयितगवाक्षां लोचनैरङ्गनानाम्}%॥ ११.९३ ॥


॥इति श्री-महाकवि-कालिदास-कृत-रघुवंश-महाकाव्ये एकादशः सर्गः॥

\sect{द्वादशः सर्गः}

\twolineshloka
{निर्विष्टविषयस्नेहः स दशान्तं उपेयिवान्}
{आसीदासन्ननिर्वाणः प्रदीपार्चिरिवोषसि}%॥१२.१॥

\twolineshloka
{तं कर्णमूलं आगत्य रामे श्रीर्नस्यतां इति}
{कैकेयीशङ्कयेवाह पलितच्छद्मना जरा}%॥१२.२॥

\twolineshloka
{सा पौरान्पौरकान्तस्य रामस्याभ्युदयश्रुतिः}
{प्रत्येकं ह्लादयां चक्रे कुल्येवोद्यानपादपान्}%॥१२.३॥

\twolineshloka
{तस्याभिषेकसंभारं कल्पितं क्रूरनिश्चया}
{दूषयां आस कैकेयी शोकोष्णैः पार्थिवाश्रुभिः}%॥१२.४॥

\twolineshloka
{सा किलाश्वासिता चण्डी भर्त्रा तत्संश्रुतौ वरौ}
{उद्ववामेन्द्रसिक्ता भूर्बिलमग्नाविवोरगौ}%॥१२.५॥

\twolineshloka
{तयोश्चतुर्दशैकेन रामं प्राव्राजयत्समाः}
{द्वितीयेन सुतस्यैच्छद्वैधव्यैकफलां श्रियम्}%॥१२.६॥

\twolineshloka
{पित्रा दत्तां रुदन्रामः प्राङ्महीं प्रत्यपद्यत}
{पश्चाद्वनाय गच्छेति तदाज्ञां मुदितोऽग्रहीथ्}%॥१२.७॥

\twolineshloka
{दधतो मङ्गलक्ष्ॐए वसानस्य च वल्कले}
{ददृशुर्विस्मितास्तस्य मुखरागं समं जनाः}%॥१२.८॥

\twolineshloka
{स सीतालक्ष्मणसखः सत्याद्गुरुं अलोपयन्}
{विवेश दण्डकारण्यं प्रत्येकं च सतां मनः}%॥१२.९॥

\twolineshloka
{राजापि तद्वियोगार्तः स्मृत्वा शापं स्वकर्मजम्}
{शरीरत्यागमात्रेण शुद्धिलाभं अमन्यत}%॥१२.१०॥

\twolineshloka
{विप्रोषितकुमारं तद्(?) राज्यं अस्तमितेश्वरम्}
{रन्ध्रान्वेषणदक्षाणां द्विषां आमिषतां ययौ}%॥१२.११॥

\twolineshloka
{अथानाथाः प्रकृतयो मातृबन्धुनिवासिनम्}
{मौलैरानाययां आसुर्भर्तं स्तम्भिताश्रुभिः}%॥१२.१२॥

\twolineshloka
{श्रुत्वा तथाविधं मृत्युं कैकेयीतनयः पितुः}
{मातुर्न केवलं स्वस्याः श्रियोऽप्यासीत्पराङ्मुखः}%॥१२.१३॥

\twolineshloka
{ससैन्यश्चान्वगाद्रामं दर्शितानाश्रमालयैः}
{तस्य पश्यन्सस्ॐइत्रेरुदश्रुर्वसतिद्रुमान्}%॥१२.१४॥

\twolineshloka
{चित्रकूटवनस्थं च कथितस्वर्गतिर्गुरोः}
{लक्ष्म्या निमन्त्रयां चक्रे तं अनुच्छिष्टसंपदा}%॥१२.१५॥

\twolineshloka
{स हि प्रथमजे तस्मिन्नकृतश्रीपरिग्रहे}
{परिवेत्तारं आत्मानं मेने स्वीकरणाद्भुवः}%॥१२.१६॥

\twolineshloka
{तं अशक्यं अपाक्रष्टुं निर्देशात्स्वर्गिणः पितुः}
{ययाचे पादुके पश्चात्कर्तुं राज्याधिदेवते}%॥१२.१७॥

\twolineshloka
{स विसृष्टस्तथेत्युक्त्वा भ्रात्रा नैवाविशत्पुरीम्}
{नन्दिग्रामगतस्तस्य राज्यं न्यासं इवाभुनक्}%॥१२.१८॥

\twolineshloka
{दृढभक्तिरिति ज्येष्ठे राज्यतृष्णापराङ्मुखः}
{मातुः पापस्य शुद्ध्यर्थं प्रायश्चित्तं इवाकरोथ्}%॥१२.१९॥

\twolineshloka
{रामोऽपि सह वैदेह्या वने वन्येन वर्तयन्}
{चचार सानुजः शान्तो वृद्धेक्ष्वाकुव्रतं युवा}%॥१२.२०॥

\twolineshloka
{प्रभावस्तम्भितच्छायं आश्रितः स वनस्पतिम्}
{कदाचिदङ्के सीतायाः शिश्ये किंचिदिव श्रमाथ्}%॥१२.२१॥

\twolineshloka
{ऐन्द्रिः किल नखैस्तस्या विददार स्तनौ द्विजः}
{प्रियोपभोगचिह्नेषु पौरोभाग्यं इवाचरन्}%॥१२.२२॥

\twolineshloka
{मृगमांसं ततः सीतां रक्षन्तीं आतपे धृतम्}
{पक्षतुण्डनखाघातैर्बबाधे वायसो बलाथ्}%॥१२.२२*॥

\twolineshloka
{तस्मिन्नास्थदिषीकास्त्रं रामो रामावभोधितः}
{आत्मानं मुमुचे तस्मादेकनेत्रव्ययेन सः}%॥१२.२३॥

\twolineshloka
{रामस्त्वासन्नदेशत्वाद्भरतागमनं पुनः}
{आशङ्क्योत्सुकसारङ्गां चित्रकूटस्थलीं जहौ}%॥१२.२४॥

\twolineshloka
{प्रययावातिथेयेषु वसन्नृषिकुलेषु सः}
{दक्षिणां दिशं ऋक्षेषु वार्षिकेष्विव भास्करः}%॥१२.२५॥

\twolineshloka
{बभौ तं अनुगच्छन्ती विदेहाधिपतेः सुता}
{प्रतिषिद्धापि कैकेय्या लक्ष्मीरिव गुणोन्मुखी}%॥१२.२६॥

\twolineshloka
{अनुसूयातिसृष्टेन पुण्यगन्धेन काननम्}
{सा चकाराङ्गरागेण पुष्पोच्चलित षट्पदम्}%॥१२.२७॥

\twolineshloka
{संध्याभ्रकपिषस्तत्र विराधो नाम राक्षसः}
{अतिष्ठन्मार्गं आवृत्य रामस्येन्दोरिव ग्रहः}%॥१२.२८॥

\twolineshloka
{स जहरा तयोर्मध्ये मैथिलीं लोकशोषणः}
{नभोनभस्ययोर्वृष्टिं अवग्रह इवान्तरे}%॥१२.२९॥

\twolineshloka
{तं विनिष्पिष्य काकुत्स्थौ पुरा दूषयति श्तलीम्}
{गन्धेनाशुचिना चेति वसुधायां निचख्नतुः}%॥१२.३०॥

\twolineshloka
{पञ्चवट्यां ततो रामः शासनात्कुम्भजन्मनः}
{अनपोष्हस्थितिस्तस्थौ विन्ध्याद्रिः प्रकृताविव}%॥१२.३१॥

\twolineshloka
{रावणावरजा तत्र राघवं मदनातुरा}
{अभिपेदे निदाघार्ता व्यालीव मलयद्रुमम्}%॥१२.३२॥

\twolineshloka
{सा सीतासंनिधावेव तं वव्रे कथितान्वया}
{अत्यारूढो हि नारीणां अकालज्ञो मओभवः}%॥१२.३३॥

\twolineshloka
{कलत्रवानहं बाले कनीयांसं भजस्व मे}
{इति रामो वृषस्यन्तीं वृषस्कन्धः शशास ताम्}%॥१२.३४॥

\twolineshloka
{ज्येष्ठाभिगमनात्पूर्वं तेनाप्यनभिनन्दिता}
{साभूद्रामाश्रया भूयो नदीवोभयकूलभाक्}%॥१२.३५॥

\twolineshloka
{संरम्भं मैथिलीहासः क्षणं स्ॐयां निनाय ताम्}
{निवातस्तिमितां वेलां चन्द्रोदय इवोदधेः}%॥१२.३६॥

\twolineshloka
{फलं अस्योपहासस्य सद्यः प्राप्स्यसि पश्य माम्}
{मृग्यः परिभवो व्याघ्र्यां इत्यवेहि त्वया कृतम्}%॥१२.३७॥

\twolineshloka
{इत्युक्त्वा मैथिलीं भर्तुरङ्के निर्विशतीं भयाथ्}
{रूपं शूर्पणखा-नाम्नः सदृशं प्रत्यपद्यत}%॥१२.३८॥

\twolineshloka
{लक्ष्मणः प्रथमं श्रुत्वा कोकिलामञ्जुभाषिणीम्}
{शिवाघोरस्वनां पश्चाद्बुबुधे विकृतेति ताम्}%॥१२.३९॥

\twolineshloka
{पर्णशालां अथ क्षिप्रं विधृतासिः प्रविश्य सः}
{वैरूप्यपौनरुक्त्येन भीषणां तां अयोजयथ्}%॥१२.४०॥

\twolineshloka
{सा वक्रनखधारिण्या वेणुकर्कशपर्वया}
{अङ्कुशाकारयाङ्गुल्या तावतर्जयदम्बरे}%॥१२.४१॥

\twolineshloka
{प्राप्य चाशु जन्स्थानं खरादिभ्यस्तथाविधम्}
{रामोपक्रमं आचख्यौ रक्षःपरिभवं नवम्}%॥१२.४२॥

\twolineshloka
{मुखावयवलूणां तां नैरृता यत्पुरोदधुः}
{रामाभियायिनां तेषां तदेवाभूदमङ्गलम्}%॥१२.४३॥

\twolineshloka
{उदायुधानापततस्तान्दृप्तान्प्रेक्ष्य राघवः}
{निदधे विजयाशंसां चापे सीतां च लक्ष्मणे}%॥१२.४४॥

\twolineshloka
{एको दाशरथी रामो यातुधानाः सहस्रशः}
{ते तु यावन्त एवाजौ तावांश्च ददृशे स तैः}%॥१२.४५॥

\twolineshloka
{असज्जनेन काकुत्स्थः प्रयुक्तं अथ दूषणम्}
{न चक्षमे शुभाचारः स दूषणं इवात्मनः}%॥१२.४६॥

\twolineshloka
{तं शरैः प्रतिजग्राह खरतिशिरसौ च सः}
{ख्रमशस्ते पुनस्तस्य चापात्समं इवोद्ययुः}%॥१२.४७॥

\twolineshloka
{तैस्त्रयाणां शितैर्बाणैर्यथापूर्वविशुद्धिभिः}
{आयुर्देहातिगैः पीतं रुधिरं तु पतत्रिभिः}%॥१२.४८॥

\twolineshloka
{तस्मिन्रामशरोत्कृत्ते बले महति रक्षसाम्}
{उत्थितं ददृशेऽन्यच्च कबन्धेभ्यो न किंचन}%॥१२.४९॥

\twolineshloka
{सा बाणवर्षिणं रामं योधयित्वा सुरद्विषाम्}
{अप्रबोधाय सुष्वाप गृध्रच्छाये वरूथिनी}%॥१२.५०॥

\twolineshloka
{राघवास्त्रविदीर्णानां रावणं प्रति रक्षसाम्}
{तेषां शूर्पणखैवैका दुष्प्रत्वृत्तिहराभवथ्}%॥१२.५१॥

\twolineshloka
{निग्रहात्स्वसुराप्तानां वधाच्च धनदानुजः}
{रामेण निहतं मेने पदं दशसु मूर्धसु}%॥१२.५२॥

\twolineshloka
{रक्षसा मृगरूपेण वञ्चयित्वा स राघवौ}
{जहरा सीतां पक्षीन्द्रप्रयासक्षणविघ्नितः}%॥१२.५३॥

\twolineshloka
{तौ सीतान्वेषिणौ गृध्रं लूनपक्षं अपश्यताम्}
{प्राणैर्दशरथप्रीतेरनृणं कण्ठवर्तिभिः}%॥१२.५४॥

\twolineshloka
{स रावणहृतां ताभ्यां वचसाचष्ट मैथिलीम्}
{आत्मनः सुमहत्कर्म व्रणैरावेद्य संस्थितः}%॥१२.५५॥

\twolineshloka
{तयोस्रावणहृतां ताभ्यां पितृव्यापत्तिशोकयोः}
{पितरीवाग्निसंस्कारात्परा ववृतिरे क्रियाः}%॥१२.५६॥

\twolineshloka
{वधनिर्धूतशापस्य कबन्धस्योपदेशतः}
{मुमूर्छ सख्यं रामस्य समानव्यसने हरौ}%॥१२.५७॥

\twolineshloka
{स हत्वा वालिनं वीरस्तत्पदे चिरकाङ्क्षिते}
{धातोः स्थान इवादेशं सुग्रीवं संन्यवेशयथ्}%॥१२.५८॥

\twolineshloka
{इतस्ततश्च वैदेहीं अन्वेष्टुं भर्तृचोदिताः}
{कपयश्चेरुरार्तस्य रामस्येव मनोरथाः}%॥१२.५९॥

\twolineshloka
{प्रवृत्तावुपलब्धायां तस्याः संपातिदर्शनाथ्}
{मारुतिः सागरं तीर्णः संसारं इव निर्ममः}%॥१२.६०॥

\twolineshloka
{दृष्टा विचिन्वता तेन लङ्कायां राक्षसीवृता}
{जानकी विषवल्लीभिः परीतेव महौषधिः}%॥१२.६१॥

\twolineshloka
{तस्यै भर्तुरभिज्ञामं अङ्गुलीयं ददौ कपिः}
{प्रत्युद्गतं इवानुष्णैस्तदानन्दाश्रुभिन्दुभिः}%॥१२.६२॥

\twolineshloka
{निर्वाप्य प्रियसंदेशैः सीतां अक्षवधोद्धतः}
{स ददाह पुरीं लङ्कां क्षणसोढारिनिग्रहः}%॥१२.६३॥

\twolineshloka
{प्रत्यभिज्ञानरत्नं च रामायादर्शयत्कृती}
{हृदयं स्वयं आयातं वैदेह्या इव मूर्तिमथ्}%॥१२.६४॥

\twolineshloka
{स प्राप हृदयन्यस्तमणिपर्शनिमीलितः}
{अपयोधरसंसर्गं प्रियालिङ्गननिर्वृतिम्}%॥१२.६५॥

\twolineshloka
{श्रुत्वा रामः प्रियोदन्तं मेने तत्संगमोत्सुकः}
{महार्णवपरिक्षेपं लङ्कायाः परिखालघुम्}%॥१२.६६॥

\twolineshloka
{स प्रतस्थेऽरिनाशाय हरिसैन्यैरनुद्रुतः}
{न केवलं धरा-पृष्ठे व्य्ॐनि संबाधवर्तिभिः}%॥१२.६७॥

\twolineshloka
{निर्विष्टं उदधेः कूले तं प्रपेदे विभीषणः}
{स्नेहाद्राक्षसलक्ष्म्येव बुद्धिं आदिश्य चोदितः}%॥१२.६८॥

\twolineshloka
{तस्मै निशाचरैश्वर्यं प्रतिशुश्राव राघवः}
{काले खलु समारब्धाः फलं बध्नन्ति नीतयः}%॥१२.६९॥

\twolineshloka
{स सेतुं बन्धयां आस प्लवगैर्लवणाम्भसि}
{रसातलादिवोन्मग्नं शेषं स्वप्नाय शार्ङ्गिणः}%॥१२.७०॥

\twolineshloka
{तेनोत्तीर्य पथा लङ्कां रोधयां आस पिङ्गलैः}
{द्वितीयं हेमप्राकारं कुर्वद्भिरिव वानरैः}%॥१२.७१॥

\twolineshloka
{रणः प्रववृते तत्र भीमः प्लवगरक्षसाम्}
{दिग्विजृम्भितकाकुत्स्थपौलस्त्यजयघोषणः}%॥१२.७२॥

\twolineshloka
{पादपाविद्धपरिघः शिलानिष्पिष्टमुद्गरः}
{अतिशस्त्रनखन्यासः शैलरुग्ण मतङ्गजः}%॥१२.७३॥

\twolineshloka
{अथ रामशिरश्छेददर्शनोद्भ्रान्तचेतनाम्}
{सीतां मायेति शंसन्ती त्रिजटा समजीवयथ्}%॥१२.७४॥

\twolineshloka
{कामं जीवति मे नाथ इति सा विजहौ शुचम्}
{प्राङ्मत्वा सत्यं अस्यान्तं जीवितास्मीति लज्जिता}%॥१२.७५॥

\twolineshloka
{गरुडापातविश्लिष्टमेघनादास्त्रबन्धनः}
{दाशरथ्योः क्षणक्लेशः स्वप्नवृत्त इवाभवथ्}%॥१२.७६॥

\twolineshloka
{ततो बिभेद पौलस्त्यः शक्त्या वक्षसि लक्ष्मणम्}
{रामस्त्वनाहतोऽप्यासीद्विदीर्णहृदयः शुचा}%॥१२.७७॥

\twolineshloka
{स मारुतिसमानीतमहौषधिहतव्यथः}
{लङ्कास्त्रीणां पुनश्चक्रे विलापाचार्यकं शरैः}%॥१२.७८॥

\twolineshloka
{स नादं मेघनादस्य धनुश्चेन्द्रायुधप्रभम्}
{मेघस्येव शरत्कालो न किंचित्पर्यशेषयथ्}%॥१२.७९॥

\twolineshloka
{क्लेशेन महता निद्रां त्याजितं रणदुर्जयम्}
{रावणः प्रेषयां आस युद्धायानुजं आत्मनः}%॥१२.७९आ॥

\twolineshloka
{जघान स तदादेशात्कपीनुग्राननेकशः}
{विवेश च पुरीं लङ्कां समादाय हरीश्वरम्}%॥१२.७९भ्॥

\twolineshloka
{कुम्भकर्णः कपीन्द्रेण तुल्यावस्थः स्वसुः कृतः}
{रुरोध रामं शृङ्गीव टङ्कच्छिन्नमनःशिलः}%॥१२.८०॥

\twolineshloka
{अकाले बोधितो भ्रात्रा प्रियस्वप्नो वृथा भवान्}
{रामेषुभिरितीवासौ दीर्घनिद्रां प्रवेशितः}%॥१२.८१॥

\twolineshloka
{इतराण्यपि रक्षांसि पेतुर्वानरकोटिषु}
{रजांसि समरोत्थानि रच्छोणितनन्दीष्विव}%॥१२.८२॥

\twolineshloka
{निर्ययावथ पौलस्थ्यः पुनर्युद्धाय मन्दिराथ्}
{अरावणं अरामं वा जगदद्येति निश्चितः}%॥१२.८३॥

\twolineshloka
{रामं पदातिं आलोक्य लङ्केषं च वरूथिनम्}
{हरियुग्यं रथं तस्मै पर्जिघाय पुरंदरः}%॥१२.८४॥

\twolineshloka
{तं आधूतद्वजपटं व्य्ॐअगङ्गोर्मिवायुभिः}
{देवसूतभुजालम्बी जैत्रं अध्यास्त राघवः}%॥१२.८५॥

\twolineshloka
{मातलिस्तस्य माहेन्द्रं आमुमोच तनुच्छदम्}
{यत्रोत्पलददलक्लैब्यं अस्त्राण्यापुः सुरद्विषाम्}%॥१२.८६॥

\twolineshloka
{अन्योन्यदर्शनप्राप्तविक्रमावसरं चिराथ्}
{रामरावणयोर्युद्धं चरितार्थं इवाभवथ्}%॥१२.८७॥

\twolineshloka
{भुजमूर्धोरुबाहुल्यादेकोऽपि धन्दानुजः}
{ददृशे सोऽयथापूर्वो मातृवंश इव स्थितः}%॥१२.८८॥

\twolineshloka
{जेतारं लोकपालानां स्वमुखैरर्चितेश्वरम्}
{रामस्तुलितकैलासं अरातिं बह्वमन्यत}%॥१२.८९॥

\twolineshloka
{तस्य स्फुरति पौलस्त्यह्सीतासंगमशंसिनि}
{निचखानाधिकक्रोधः शरं सव्येतरे भुजे}%॥१२.९०॥

\twolineshloka
{रावणस्यापि रामास्तो भित्त्वा हृदयं आशुगः}
{विवेश भुवं आख्यातुं उरगेभ्य इव प्रियम्}%॥१२.९१॥

\twolineshloka
{वचसैव तयोर्वाक्यं अस्त्रं अस्त्रेण निघ्नतोः}
{अन्योन्यजयसंरम्भो ववृधे वादिनोरिव}%॥१२.९२॥

\twolineshloka
{विक्रमव्यतिहारेण अस्त्रं अस्त्रेण निघ्नतोः}
{जयश्रीरन्तरा वेदिर्मत्तवारणयोरिव}%॥१२.९३॥

\twolineshloka
{कृतप्रतिकृतप्रीतैस्तयोर्मुक्तां सुरासुरैः}
{परस्परं शरव्राताः पुष्पवृष्टिं न सेहिरे}%॥१२.९४॥

\twolineshloka
{अयःशङ्कुचितां रक्षः शतघ्नीं अथ शत्रवे}
{हृतां वैवस्वतस्येव कूटशाल्मलिं अक्षिपथ्}%॥१२.९५॥

\twolineshloka
{राघवो रथं अप्राप्तां तां आशां च सुरद्विषाम्}
{अर्धचन्द्रमुखैर्बाणैश्चिच्छेद कदलीसुखम्}%॥१२.९६॥

\twolineshloka
{अमोघं संदधे चास्मै धनुष्यकेअध्नुर्धरः}
{ब्राह्मं अस्त्रं प्रियाशोकशल्यनिष्कर्षणौषधम्}%॥१२.९७॥

\twolineshloka
{तद्व्य्ॐनि दशधा भिन्नं ददृशे दीप्तिमन्मुखम्}
{वपुर्महोरगस्येव करालफणमण्डलम्}%॥१२.९८॥

\twolineshloka
{तेन मन्त्रप्रयुक्तेन निमेषार्धादपातयथ्}
{स रावणशिरःपङ्क्तिं अज्ञातव्रणवेदनाम्}%॥१२.९९॥

\twolineshloka
{बालार्कप्रतिमेवाप्सु वीचिभिन्ना पतिष्यतः}
{रराज रक्षःकायस्य कण्ठच्छेदप्रंपरा}%॥१२.१००॥

\twolineshloka
{मरुतां पश्यतां तस्य शिरांसि पतितान्यपि}
{मनो नातिविशश्वास पुनः संधानशङ्किनाम्}%॥१२.१०१॥

\twolineshloka
{अथ मदगुरुपक्षैर्लोकपालद्विपानां अनुगतं अलिवृन्दैर्गण्डभित्तीर्विहाय}
{उपनतमणिबन्धे मूर्ध्नि पौलस्त्यशत्रोः सुरभि सुरविमुक्तं पुष्पवर्षं पपात}%॥१२.१०२॥

\twolineshloka
{यन्ता हरेः सपदि संहृतकार्मुकज्यं आपृच्छ्य राघवं अनुष्ठितदेवकार्यम्}
{नामाङ्करावणशराङ्कितकेतुयष्टिं ऊर्ध्वं रथं हरिसहस्रयुजं निनाय}%॥१२.१०३॥

\twolineshloka
{रघुपतिरपि जातवेदोविशुद्धां प्रगृह्य प्रियां प्रियसुहृदि विभीषणे संगमय्य श्रियं वैरिणः}
{रविसुतसहितेन तेनानुयातः सस्ॐइत्रिणा भुजविजितविमानरत्नाधिरूढः प्रतस्थे पुरीम्}%॥१२.१०४ ।।

॥इति श्री-महाकवि-कालिदास-कृत-रघुवंश-महाकाव्ये द्वादशः सर्गः॥

\sect{त्रयोदशः सर्गः}

\twolineshloka
{अथात्मनः शब्दगुणं गुणज्ञः पदं विमानेन विगाहमानः}
{रत्नाकरं वीक्ष्य मिथः स जायां रामाभिधानो हरिरित्युवाच}%॥१३.१॥

\twolineshloka
{वैदेहि पश्य्ऽ आमलयाद्विभक्तं मत्सेतुना फेनिलं अम्बुराशिम्}
{छायापथेनेव शरत्प्रसन्नं आकाशं आविष्कृतचारुतारम्}%॥१३.२॥

\twolineshloka
{गुरोर्यियक्षोः कपिलेन मेध्ये रसातलं संक्रमिते तुरंगे}
{तदर्थं उर्वीं अवदारयद्भिः पूर्वैः किलायं परिवर्धितो नः}%॥१३.३॥

\twolineshloka
{गर्भं दधत्यर्कमरीचयोऽस्माद्विवृद्धिं अत्राश्नुवते वसूनि}
{अबिन्धनं वह्निं असौ बिभर्ति प्रह्लादनं ज्योतिरजन्यनेन}%॥१३.४॥

\twolineshloka
{तां तां अवस्थां प्रतिपद्यमानं स्थितं दश व्याप्य दिशो महिम्ना}
{विष्णोरिवास्यानवधारणीयं ईदृक्तया रूपं इयत्तया वा}%॥१३.५॥

\twolineshloka
{नाभिप्ररूढाम्बुरुहासनेन संस्तूयमानः प्रथमेन धात्रा}
{अमुं युगान्तोचितयोगन्दिरः संहृत्य लोकान्पुरुषोऽधिशेते}%॥१३.६॥

\twolineshloka
{पक्षच्छिदा गोत्रभिदात्तगन्धाः शरण्यं एनं शतशो महीध्राः}
{नृपा इवोपप्लविनः परेभ्यो धर्मोत्तरं मध्यमं आश्रयन्ते}%॥१३.७॥

\twolineshloka
{रसातलादादिभवेन पुंसा भुवः प्रयुक्तोद्वहनक्रियायाः}
{अस्याच्छं अम्भः प्रलयप्रवृद्धं मुहूर्तवक्त्रावरणं बभूव}%॥१३.८॥

\twolineshloka
{मुखार्पणेषु प्रकृतिप्रगल्भाः स्वयं तरङ्गाधरदानदक्षः}
{अनन्यसामान्यकलत्रवृत्तिः पिबत्यसौ पाययते च सिन्धूः}%॥१३.९॥

\twolineshloka
{ससत्त्वं आदाय नदीमुखाम्भः संमीलयन्तो विवृताननत्वाथ्}
{अमी शिरोभिस्तिमयः सरन्ध्रैरूर्ध्वं वितन्वन्ति जलप्रवाहान्}%॥१३.१०॥

\twolineshloka
{मातङ्गनक्रैः सहसोत्पतद्भिर्भिन्नान्द्विधा पश्य समुद्रफेनान्}
{कपोलसंसर्पितया य एषां व्रजन्ति कर्ण क्षणचामरत्वम्}%॥१३.११॥

\twolineshloka
{वेलानिलाय प्रसृता भुजंगा महोर्मिविस्फूर्जथुनिर्विशेषाः}
{सूर्यांशुसंपर्कसमृद्धरागैर्व्यज्यन्त एते मणिभिः फणस्थैः}%॥१३.१२॥

\twolineshloka
{तवाधरस्परधिषु विद्रुमेषु पर्यस्तं एतत्सहसोर्मिवेगाथ्}
{ऊर्ध्वाङ्कुरप्रोतमुखं कथंचित्क्लेशदपक्रामति शङ्खयूथम्}%॥१३.१३॥

\twolineshloka
{प्रवृत्तमात्रेण पयांसि पातुं आवर्तवेगाद्भ्रमता घनेन}
{आभाति भूयिष्ठं अयं समुद्रः प्रमथ्यमानो गिरिणेव भूयः}%॥१३.१४॥

\twolineshloka
{दूरादयश्चक्रनिभस्य तन्वी तमालतालीवनराजिनीला}
{आभाति वेला लवणाम्बुराशेर्धारानिबद्धेव कलङ्कलेखा}%॥१३.१५॥

\twolineshloka
{वेलानिलः केतकरेणुभिस्ते संभावयत्याननं आयताक्षि}
{मां अक्षमं मण्डनकालहानेर्वेत्तीव बिम्बाधरबद्धतृष्णम्}%॥१३.१६॥

\twolineshloka
{एते वयं सैकतभिन्नशुक्तिपर्यस्तमुक्तापटलं पयोधेः}
{प्राप्ता मुहूर्तेन विमानवेगात्कूलं फलावर्जितपूगमालम्}%॥१३.१७॥

\twolineshloka
{कुरुष्व तावत्करभोरु पश्चान्मार्गे मृगप्रेक्षिणि दृष्तिपातम्}
{एषा विदूरीभवतः समुद्रात्सकानना निष्पततीव भूमिः}%॥१३.१८॥

\twolineshloka
{क्वचित्पथा संचरते सुराणां क्वचिद् घनानां पततां क्वचिच्च}
{यथाविधो मे मन्सोऽभिलाषः प्रवर्तते पश्य तथा विमानम्}%॥१३.१९॥

\twolineshloka
{असौ महेन्द्रद्विपदानगन्धी त्रिमार्गगावीचिविमर्दशीतः}
{आकाशवायुर्दिनयौवनोत्थानाचामति स्वेदलवान्मुखे ते}%॥१३.२०॥

\twolineshloka
{करेण वातायनलम्बितेन स्पृष्टस्त्वया चण्डि कुतूहलिन्या}
{आमुञ्चतीवाभरणं द्वितीयं उद्भिन्नविद्युद्वलयो घनस्ते}%॥१३.२१॥

\twolineshloka
{अमी जनस्थानं अपोढविघ्नं मत्वा समारब्धनवोटजानि}
{अध्यासते चीरभृतो यथास्वं चिरोज्झितान्याश्रममण्डलानि}%॥१३.२२॥

\twolineshloka
{सैषा स्थली यत्र विचिन्वता त्वां भ्रष्टं मया नूपुरं एकं उर्व्याम्}
{अदृश्यत त्वच्चरणारविन्दविश्लेषदुःखादिव बद्धमौनम्}%॥१३.२३॥

\twolineshloka
{त्वं रक्षसा भीरु यतोऽपनीता तं मार्गं एताः कृपया लता मे}
{अदर्शयन्वक्तुं अशक्नुवत्यः शाखाभिरावर्जितपल्लवाभिः}%॥१३.२४॥

\twolineshloka
{मृग्यश्च दर्भाङ्कुरनिर्व्यपेक्षास्तवागतिज्ञं समबोधयन्माम्}
{व्यापारयन्त्यो दिशि दक्षिणस्यां उत्पक्ष्मराजीनि विलोचनानि}%॥१३.२५॥

\twolineshloka
{एतद्गिरेर्मालयवतः पुरस्तादाविर्भवत्यम्बर्लेखि शृङ्गम्}
{नवं यत्र घनैर्मया च त्वद्विप्रयोगाश्रु समं विसृष्टम्}%॥१३.२६॥

\twolineshloka
{गन्धश्च धाराहतपल्वलानां कादम्बं अर्धोद्गतकेसरं च}
{स्निग्धाश्च केकाः शिखिनां बभूवुर्यस्मिनसह्यानि विना त्वया मे}%॥१३.२७॥

\twolineshloka
{पूर्वानुभूतं स्मरता च यत्र कम्पोत्तरं भीरु तवोपगूढम्}
{गुहाविसारीण्यतिवाहितानि मया कथंचिद् घनगर्जितानि}%॥१३.२८॥

\twolineshloka
{आसारसिक्तक्षितिबाष्पयोगान्मां अक्षिणोद्यत्र विभिन्नकोशैः}
{विडम्ब्यमाना नवकन्दलैस्ते विवाहधूमारुणलोचनश्रीः}%॥१३.२९॥

\twolineshloka
{उपान्तवानीरवनोपगूधान्यालक्ष्यपारिप्लवसारसानि}
{दूरावतीर्णा पिबतीव खेदादमूनि पम्पासलिलानि दृष्टिः}%॥१३.३०॥

\twolineshloka
{अत्रावियुक्तानि रथाङ्गनाम्नां अन्योन्यदत्तोत्पलकेसराणि}
{द्वन्द्वानि दूरान्तरवर्तिना ते मया प्रिये सस्पृहं ईक्षितानि}%॥१३.३१॥

\twolineshloka
{इमां तताशोकलतां च तन्वीं स्तनाभिरामस्तबकाभिनम्राम्}
{त्वत्प्राप्तिबुद्ध्या परिरिप्समानः स्ॐइत्रिणा सास्रं अहं निषिद्धः}%॥१३.३२॥

\twolineshloka
{अमूर्विमानान्तरलम्बिनीनां श्रुत्वा स्वनं काञ्चनकिङ्किणीनाम्}
{प्रत्युद्व्रजन्तीव खं उत्पतन्त्यो गोदावरीसारसपङ्क्तयस्त्वाम्}%॥१३.३३॥

\twolineshloka
{एषा त्वया पेशलमध्ययापि घटाम्बुसंवर्धितबालचूता}
{आह्लादयत्युन्मुखकृष्णसारा दृष्ट चिरात्पञ्चवटी मनो मे}%॥१३.३४॥

\twolineshloka
{अत्रानुगोदं मृगयानिवृत्तस्तरङ्गवातेम विनीतखेदः}
{रहस्त्वदुत्सङ्गनिषण्णमूर्धा स्मरामि वानीरगृहेषु सुप्तः}%॥१३.३५॥

\twolineshloka
{भ्रूभेद मात्रेण पदान्मघोनः प्रभ्रंशयां यो नहुषं चकार}
{तस्याविलाम्भःपरिशुद्धिहेतोर्भ्ॐओ मुनेः स्थानपरिग्रहोऽयम्}%॥१३.३६॥

\twolineshloka
{त्रेताग्निधूमाग्रं अनिन्द्यकीर्तेस्तस्येदं आक्रान्तविमानमार्गम्}
{घ्रात्वा हविर्गन्धि रजोविमुक्तः समश्नुते मे लघिमानं आत्मा}%॥१३.३७॥

\twolineshloka
{एतन्मुनेर्मानिनि शातकर्णेः पञ्चाप्सरो नाम विहारवारि}
{आभाति पर्यन्तवनं विदूरान्मेघान्तरालक्ष्यं इवेन्दुबिम्बम्}%॥१३.३८॥

\twolineshloka
{पुरा स दर्भाङ्कुरमात्रवृत्तिश्चरन्मृगैः सार्धं ऋषिर्मघोना}
{समाधिभीतेन किलोपनीतः पञ्चाप्सरोयौवनकूटभन्धम्}%॥१३.३९॥

\twolineshloka
{तस्यायं अन्तर्हितसौधभाजः प्रसक्तसंगीतमृदङ्गघोषः}
{वियद्गतः पुष्पकचन्द्रशालाः क्षणं प्रतिष्रुन्मुखराः करोति}%॥१३.४०॥

\twolineshloka
{हविर्भुजां एधवतां चतुर्णां मध्ये ललाटंतपसप्तसप्तिः}
{असौ तपस्यत्यपरस्तपस्वी नाम्ना सुतीक्ष्णश्चरितेन दान्तः}%॥१३.४१॥

\twolineshloka
{अमुं सहासप्रहितेक्षणानि व्याजार्धसंदर्शितमेखलानि}
{नालं विकर्तुं जनितेन्द्रशङ्कं सुराङ्गनाविभ्रमचेष्टितानि}%॥१३.४२॥

\twolineshloka
{एषोऽक्षमालावलयं मृगाणां कण्डूयितारं कुशसूचिलावम्}
{सभाजने मे भुजं ऊर्ध्वबाहुः सव्येतरं प्राध्वं इतः प्रयुङ्क्ते}%॥१३.४३॥

\twolineshloka
{वाचंयमत्वात्प्रणतिं ममैष कम्पेन किंचित्प्रतिगृह्य मूर्ध्नः}
{दृष्टिं विमानव्यवधानमुक्तां पुनः सहस्राचिषि संनिधत्ते}%॥१३.४४॥

\twolineshloka
{अदः शरण्यं शरभङ्गनाम्नस्तपोवनं पावनं आहिताग्नेः}
{चिराय संतर्प्य समिधिरग्निं यो मन्त्रपूतां तनुं अप्यहौषीथ्}%॥१३.४५॥

\twolineshloka
{छायाविनीताध्वपरिश्रमेषु भूयिष्ठसंभाव्यफलेष्वमीषु}
{तस्यातिथीनां अधुना सपर्या स्थिता सुपुत्रेष्विव पादपेषु}%॥१३.४६॥

\twolineshloka
{धारास्वनोद्गारिदरीमुखोऽसौ शृङ्गाग्रलग्नाम्बुदवप्रपङ्कः}
{बध्नाति मे बन्धुरगात्रि चक्षुर्दृप्तः ककुद्मानिव चित्रकूटः}%॥१३.४७॥

\twolineshloka
{एषा प्रसन्नस्तिमितप्रवाहा सरिद्विदूरान्तरभावतन्वी}
{मन्दाकिनी भाति नगोपकण्ठे मुक्तावली कण्ठगतेव भूमेः}%॥१३.४८॥

\twolineshloka
{अयं सुजातोऽनुगिरं तमालः प्रवालं आदाय सुगन्धि यस्य}
{कर्णार्पितेन्ऽ ठाकरवं कपोलं अपार्थ्यकालागुरुपत्त्रलेखं Vथ्}%॥१३.४९॥

\twolineshloka
{अनिग्रहत्रासविनीतसत्त्वं अपुष्पलिङ्गात्फलबन्धिवृक्षम्}
{वनं तपःसाधनं एतदत्रेराविष्कृतोदग्रतरप्रभावम्}%॥१३.५०॥

\twolineshloka
{अत्राभिषेकाय तपोधनानां सप्तर्शिहस्तोद्धृतहेमपद्माम्}
{प्रवर्तयां आस किल्ऽ आनुसूया त्रिस्रोतसं त्र्यम्बकमौलिमालाम्}%॥१३.५१॥

\twolineshloka
{वीरासनैर्ध्यानजुषां ऋषीनां अमी समाध्यासितवेदिमध्याः}
{निवातनिष्कम्पतया विभान्ति योगाधिरूढा इव शाखिनोऽपि}%॥१३.५२॥

\twolineshloka
{त्वया पुरस्तादुपयाचितो यः सोऽयं वटः श्याम इति प्रतीतः}
{राशिर्मणीनां इव गारुडानां सपद्मरागः फलितो विभाति}%॥१३.५३॥

\twolineshloka
{क्वचित्प्रभा चान्द्रमसी तमोभिश्मुक्तामयी यष्टिरिवानुविद्धा}
{अन्यत्र माला सितपङ्कजानां इन्दीवरैरुत्खचितान्तरेव}%॥१३.५४॥

\twolineshloka
{क्वचित्खगानां प्रियमानसानां कादम्बसंसर्गवतीव}
{अन्यत्र शुभ्रा शरदभ्रलेखा भक्तिर्भुवश्चन्दनकल्पितेव}%॥१३.५५॥

\twolineshloka
{क्वचित्प्रभा चान्द्रमसी तमोभिश्छायाविलीनैः शबलीकृतेव}
{अन्यत्र शुभ्रा शरदभ्रलेखा रन्ध्रेष्विवालक्ष्यनभःप्रदेशा}%॥१३.५६॥

\twolineshloka
{क्वचिच्च कृष्णोरगभूषणेव भस्माङ्गरागा तनुरीश्वरस्य}
{पश्यानवद्याङ्गि विभाति गङ्गा भिन्नप्रवाहा यमुनातरङ्गैः}%॥१३.५७॥

\twolineshloka
{तमिस्रया शुभ्रनिशेव भिन्ना कुन्दस्रगिन्दीवरमालयेव}
{कृत्तिर्हरेः कृष्णमृगत्वचेव भूतिः स्मरारेरिव कण्ठभासा}%॥१३.५७आ॥

\twolineshloka
{दृश्यार्धया शारदमेघलेखा निर्धूतनिस्त्रिंशरुचा विशेव}
{गवाक्षकालागुरुधूमराज्या हर्म्यस्थलीलेपसुधा नवेव}%॥१३.५७भ्॥

\twolineshloka
{तुषारसंघातशिला हिमाद्रेर्जात्याञ्जनप्रस्तरशोभयेव}
{पतत्रिणां मनसगोचराणां ठ्श्रेणीव कादम्बविहंगपङ्क्त्या}%॥१३.५७C॥

\twolineshloka
{नितान्तशुद्धस्फुटिकाशयोगाद्वैडूर्यकान्त्या रशनावलीव}
{गङ्गा रवेरात्मजया समेता पुष्प्यत्युदारं परभागलेखा}%॥१३.५७ढ्॥

\twolineshloka
{समुद्रपत्न्योर्जलसंनिपाते पूतात्मनां अत्र किलाभिषेकाथ्}
{तत्त्वावबोधेन विनापि भूयस्तनुत्यजां नास्ति शरीरबन्धः}%॥१३.५८॥

\twolineshloka
{पुरं निषादाधिपतेरिदं तद्यस्मिन्मया मौलिमणिं विहाय}
{जटासु बद्धास्वरुदत्सुमन्त्रः कैकेयि कामाः फलितास्तवेति}%॥१३.५९॥

\twolineshloka
{पयोधरैः पुण्यजनाङ्गनानां निर्विष्टहेमाम्बुजरेणु यस्याः}
{ब्राह्मं सरः कारणं आप्तवाचो बुद्धेरिवाव्यक्तं उदाहरन्ति}%॥१३.६०॥

\twolineshloka
{जलानि या तीरनिखातयूपा वहत्ययोध्यां अनु राजधानीम्}
{तुरंगमेधावभृतवतीर्णैरिक्ष्वाकुभिः पुण्यतरीकृतानि}%॥१३.६१॥

\twolineshloka
{यां सैकतोत्सङ्गसुखोचितानां प्राज्यैः पयोभिः परिवर्धितानाम्}
{सामान्यधात्रीं इव मानसं मे संभावयत्युत्तरकोसलानाम्}%॥१३.६२॥

\twolineshloka
{सेयं मदीया जननीव तेन मान्येन राज्ञा सरयूर्वियुक्ता}
{दूरे वसन्तं शिशिरानिलैर्मां तरङ्गहस्तैरुपगूहतीव}%॥१३.६३॥

\twolineshloka
{विरक्तसंध्याकपिशं पुरस्ताद्यतो रजः पार्थिवं उज्जिहीते}
{शङ्के हनूमत्कथितप्रवृत्तिः प्रत्युद्गतो मां भरतः ससैन्यः}%॥१३.६४॥

\twolineshloka
{अद्धा श्रियं पालितसंगराय प्रत्यर्पयिष्यत्यनघां स साधुः}
{हत्वा निवृत्ताय मृधे खरादीन्संरक्षितां त्वां इव लक्ष्मणो मे}%॥१३.६५॥

\twolineshloka
{असौ पुरस्कृत्य गुरुं पदातिः पश्चादवस्थापितवाहिनीकः}
{वृद्धैरमात्यैः सह चीरवासा मां अर्घ्यपाणिर्भरतोऽभ्युपैति}%॥१३.६६॥

\twolineshloka
{पित्रा निसृष्टां मदपेक्षया यः श्रियं युवाप्यङ्कगतां अभोक्ता}
{इयन्ति वर्षाणि तया सहोग्रं अभ्यस्यतीव व्रतं आसिधारम्}%॥१३.६७॥

\twolineshloka
{एतावदुक्तवति दाशरथौ तदीयां इच्छां विमानं अधिदेवतया विदित्वा}
{द्योतिष्पथादवततार सविस्मयाभिरुद्वीक्षितं प्रकृतिभिर्भरतानुगाभिः}%॥१३.६८॥

\twolineshloka
{तस्मात्पुरःसरविभीषणदर्शितेन सेवाविचक्षणहरीश्वरदत्तहस्तः}
{यानादवातरददूरमहीतलेन मार्गेण भङ्गिरचितस्फटिकेन रामः}%॥१३.६९॥

\twolineshloka
{इक्ष्वाकुवंशगुरवे प्रयतः प्रणम्य स भ्रातरं भरतं अर्घ्यपरिग्रहान्ते}
{पर्यश्रुरस्वजत मूर्धनि चोपजघ्रौ तद्भक्त्यपोढपितृराह्यमहाभिषेके}%॥१३.७०॥

\twolineshloka
{श्मश्रुप्रवृद्धिजनिताननविक्रियांश्च प्लक्षान्प्ररोहजटिलानिव मन्त्रिवृद्धान्}
{अन्वग्रहीत्प्रणमतः शुभदृष्टिपातैर्वार्त्तानुयोगमधुराक्षरया च वाचा}%॥१३.७१॥

\twolineshloka
{दुर्जातबन्धुरयं ऋक्षहरीश्वरो मे पौलस्त्य एष समरेषु पुरः प्रहर्ता}
{इत्यादृतेन कथितौ रघुनन्दनेन व्युत्क्रम्य लक्ष्मणं उभौ भरतो ववन्दे}%॥१३.७२॥

\twolineshloka
{स्ॐइत्रिणा तदनु संससृजे स चैनं उत्थाप्य नम्रशिरसं भृशं आलिनिङ्ग}
{रूढेन्द्रजित्प्रहरणव्रणकर्कशेन क्लिश्यन्निवास्य भुजमध्यं उरःस्थलेन}%॥१३.७३॥

\twolineshloka
{रामाज्ञया हरिचमूपतयस्तदानीं कृत्वा मनुष्यवपुरारुरुहुर्गजेन्द्रान्}
{तेषु क्षरत्सु बहुधा मदवारिधाराः शैलाधिरोहणसुखान्युपलेभिरे ते}%॥१३.७४॥

\twolineshloka
{सानुप्लवः प्रभुरपि क्षणदाचराणां भेजे रथान्दशरथप्रभवानुशिष्टः}
{मायाविकल्परचितैरपि ये तदीयैर्न स्यन्दनैस्तुलितकृत्रिमभक्तिशोभाः}%॥१३.७५॥

\twolineshloka
{भूयस्ततो रघुपतिर्विलसत्पताकं अध्यास्त कामगति सावरजो विमानम्}
{दोषातनं बुधबृहस्पतियोगदृश्यस्तारापतिस्तरलविद्युदिव्ऽआभ्रवृन्दम्}%॥१३.७६॥

\twolineshloka
{तत्रेश्वरेण जगतां प्रलयादिवोर्वीं वर्षात्ययेन रुचं अभ्रघनादिवेन्दोः}
{रामेण मैथिलसुतां दशकण्ठकृच्छ्रात्प्रत्युद्धृतां धृतिमतीं भरतो ववन्दे}%॥१३.७७॥

\twolineshloka
{लङ्केश्वरप्रणतिभङ्गदृढव्रतं तद्(?) वन्द्यं युगं चरणयोर्जनकात्मजायाः}
{ज्येष्ठानुवृत्तिजटिलं च शिरोऽस्य साधोरन्योन्यपावनं अभूदुभयं समेत्य}%॥१३.७८॥

\twolineshloka
{क्रोशार्धं प्रकृतिपुरःसरेण गत्वा काकुत्स्थः स्तिमितजवेन पुष्पकेण}
{शत्रुघ्नप्रतिविहितोपकार्यं आर्यः साकेतोपवनं उदारं अध्युवास}%॥१३.७९॥

॥इति श्री-महाकवि-कालिदास-कृत-रघुवंश-महाकाव्ये त्रयोदशः सर्गः॥

\sect{चतुर्दशः सर्गः}

\twolineshloka
{उत्तिष्ठ वत्से ननु सानुजोऽसौ दशान्तरं तत्र समं प्रपन्ने}
{अपश्यतां दाशरथी जनन्यौ छेदादिवोपघ्नतरोर्व्रतत्यौ}%॥१४.१॥

\twolineshloka
{उभावुभाभ्यां प्रणतौ हतारी यथाक्रमं विक्रमशोभिनौ तौ}
{विस्पष्टमस्रान्धतया न दृष्टौ ज्ञातौ सुतस्पर्शसुखोपलम्भात्}%॥१४.२॥

\twolineshloka
{आनन्दजः शोकजं अश्रु बाष्पस्तयोरशीतं शिशिरो बिभेद}
{गङ्गासरय्वोर्जलं उष्णत्पतं हिमाद्रिनिस्यन्द इवावतीर्णः}%॥१४.३॥

\twolineshloka
{ते पुत्रयोर्नैरृतशस्त्रमार्गानार्द्रानिवाङ्गे सदयं स्पृशन्त्यौ}
{अपीप्सितं क्षत्रकुलाङ्गनानां न वीरसूशब्दं अकामयेताम्}%॥१४.४॥

\twolineshloka
{क्लेशावहा भर्तुरलक्षणाहं सीतेति नाम स्वं उदीरयन्ती}
{स्वर्गप्रतिष्ठस्य गुरोर्महिष्यावभक्तिभेदेन वधूर्ववन्दे}%॥१४.५॥

\twolineshloka
{उत्तिष्ठ वत्से ननु सानुजोऽसौ वृत्तेन भर्ता शुचिना तवैव}
{कृच्छ्रं महत्तीर्ण इति प्रियार्हां तां ऊचतुस्ते प्रियं अप्यमिथ्या}%॥१४.६॥

\twolineshloka
{अथाभिषेकं रघुवंशकेतोः प्रारब्धं आनन्दजलैर्जनन्योः}
{निर्वर्तयां आसुरमात्यवृद्धास्तीर्थाहृतैः काञ्चनकुम्भतोयैः}%॥१४.७॥

\twolineshloka
{सरित्समुद्रान्सरसीश्च गत्वा रक्षःकपीन्द्रैरुपपादितानि}
{तस्यापतन्मूर्ध्नि जलानि जिष्णोर्विन्ध्यस्य मेघप्रभवा इवापः}%॥१४.८॥

\twolineshloka
{तपस्विवेषक्रिययापि तावद्यः प्रेक्षणीयः सुतरां बभूव}
{राजेन्द्रनेपथ्यविधानशोभा रस्योदितासीत्पुनरुक्तदोषा}%॥१४.९॥

\twolineshloka
{स मौलरक्षोहरिमिश्रसैन्यस्तूर्यस्वनानन्दितपौरवर्गः}
{विवेष सौधोद्गतलाजवर्षां उत्तोरणां अन्वयराजधानीम्}%॥१४.१०॥

\twolineshloka
{सौमित्रिणा सावरजेन मन्दं आधूतवालव्यजनो रथस्थः}
{धृतातपत्रो भरतेन साक्षादुपायसंघात इव प्रवृद्धः}%॥१४.११॥

\twolineshloka
{प्रासादकालागुरुधूमराजिस्तस्याः पुरो वायुवशेन भिन्ना}
{वनान्निवृत्तेन रघूद्वहेन मुक्ता स्वयं वेणिरिवाभासे}%॥१४.१२॥

\twolineshloka
{श्वश्रूजनानुष्ठितचारुवेषां कर्णीरथस्थां रघुवीरपत्नीम्}
{प्रासादवातायनदृश्यबन्धैः साकेतनार्योऽञ्जलिभिः प्रणेमुः}%॥१४.१३॥

\twolineshloka
{स्फुरत्प्रभामण्डलं आनुसूयं सा बिभ्रती शाश्वतं अङ्गरागम्}
{रराज शुद्धेति पुनः स्वपुर्यै संदर्शित वह्निगतेव भर्त्रा}%॥१४.१४॥

\twolineshloka
{वेश्मानि रामः परिबर्हवन्ति विश्राण्य सौहार्धनिधिः सुहृद्ब्यः}
{बाष्पायमाणो बलिमन्निकेतं आलेख्यशेषस्य पितुर्विवेश}%॥१४.१५॥

\twolineshloka
{कृताञ्जलिस्तत्र यदम्ब सत्यान्नाभ्रश्यत स्वर्गफलाद्गुरुर्नः}
{तच्चिन्त्यमानं सुकृतं तवेति जहार लज्जां भरतस्य मातुः}%॥१४.१६॥

\twolineshloka
{तथा च सुग्रीवविभीषणादीनुपाचरत्कृत्रिमसंविधाभिः}
{संकल्पमात्रोदितसिद्धयस्ते क्रान्ता यथा चेतसि विस्मयेन}%॥१४.१७॥

\twolineshloka
{सभाजनायोपगतान्स दिव्यान्मुनीन्पुरस्कृत्य हतस्य शत्रोः}
{शुश्राव तेभ्यः प्रभवादि वृत्तं स्वविक्रमे गौरवं आदधानम्}%॥१४.१८॥

\twolineshloka
{प्रतिप्रयातेषु तपोधनेषु सुखादविज्ञातगतार्धमासान्}
{सीतास्वहस्तोपहृताग्र्यपूजान्रक्षःकपीन्द्रान्विससर्ज रामः}%॥१४.१९॥

\twolineshloka
{तच्चात्मचिन्तासुलभं विमानं हृतं सुरारेः सह जीवितेन}
{कैलासनाथोद्वहनाय भूयः पुष्पं दिवः पुष्पकं अन्वमंस्त}%॥१४.२०॥

\twolineshloka
{पितुर्नियोगाद्वनवासं एवं निस्तीर्य रामः प्रतिपन्नराज्यः}
{धर्मार्थकामेषु समां प्रपेदे यथा तथैवावरजेषु वृत्तिम्}%॥१४.२१॥

\twolineshloka
{सर्वासु मातृष्वपि वत्सलत्वात्स निर्विशेषप्रतिपत्तिरासीथ्}
{षडाननापीतपयोधरासु नेता चमूनां इव कृत्तिकासु}%॥१४.२२॥

\twolineshloka
{तेनार्थवांल्लोभपराङ्मुखेन तेन घ्नता विघ्नभयं क्रियावान्}
{तेनास लोकः पितृमान्विनेत्रा तेनैव शोकापनुदेन पुत्री}%॥१४.२३॥

\twolineshloka
{स पौरकार्याणि समीक्ष्य काले रेमे विदेहाधिपतेर्दुहित्रा}
{उपस्थितश्चारु वपुस्तदीयं कृत्वोपभोगोत्सुकयेव लक्ष्म्या}%॥१४.२४॥

\twolineshloka
{तयोर्यथाप्रार्थितं इन्द्रियार्थानासेदुषोः सद्मसु चित्रवत्सु}
{प्राप्तानि दुःखान्यपि दण्डकेषु संचिन्त्यमानानि सुखान्यभूवन्}%॥१४.२५॥

\twolineshloka
{अथाधिकस्निग्धविलोचनेन मुखेन सीता शरपाण्डुरेण}
{आनन्दयित्री परिणेतुरासीदनक्षरव्यञ्जितदोहदेन}%॥१४.२६॥

\twolineshloka
{तां अङ्कं आरोप्य कृशाङ्गयष्टिं वर्णान्तराक्रान्तपयोधराग्राम्}
{विलज्जमानां रहसि प्रतीतः प्रप्रच्छ रामां रमणोऽभिलाषम्}%॥१४.२७॥

\twolineshloka
{सा दष्टनीवारबलीनिहिंस्रैः संबद्ध वैखानसकन्यकानि}
{इयेष भूयः कुशवन्ति गन्तुं भागीरथीतीरतपोवनानि}%॥१४.२८॥

\twolineshloka
{तस्यै प्रतिश्रुत्य रघुप्रवीरस्तद्(?) ईप्सितं पार्श्वचरानुयातः}
{आलोकयिष्यन्मुदितां अयोध्यां प्रासादं अभ्रंलिहं आरुरोह}%॥१४.२९॥

\twolineshloka
{ऋद्धापणं राजपथं स पश्यन्विगाह्यमानां सरयूं च नौभिः}
{विलासिभिश्चाध्युषितानि पौरैः पुरोपकण्ठोपवनानि रेमे}%॥१४.३०॥

\twolineshloka
{स किंवदन्तीं वदतां पुरोगः स्वऋत्तं उद्दिश्य विशुद्धवृत्तः}
{सर्पाधिराजोरुभुजोऽपसर्पं पप्रच्छ भद्रं विजितारिभद्रः}%॥१४.३१॥

\twolineshloka
{निर्बन्धपृष्टः स जगाद सर्वं स्तुवन्ति पौराश्चरितं त्वदीयम्}
{अन्यत्र रक्षोभवनोषितायाः परिग्रहान्मानवदेव देव्याः}%॥१४.३२॥

\twolineshloka
{कलत्रनिन्दागुरुणा किलैवं अभ्याहतं कीर्तिविपर्ययेण}
{अयोघनेनाय इवाभितप्तं वैदेहिबन्धोर्हृदयं विदद्रे}%॥१४.३३॥

\twolineshloka
{किं आत्मनिर्वादकथां उपेक्षे जायां अदोषां उत संत्यजामि}
{इत्येकपक्षाश्रयविक्लवत्वादासीत्स दोलाचलचित्तवृत्तिः}%॥१४.३४॥

\twolineshloka
{निश्चित्य चानन्यनिवृत्ति वाच्यं त्यागेन पत्न्याः परिमार्ष्टुं ऐच्छथ्}
{अपि स्वदेहात्किं उतेन्द्रियार्थाद्यशोधनानां हि यशो गरीयः}%॥१४.३५॥

\twolineshloka
{स संनिपात्यावरजान्हतौजास्तद्विक्रियादर्शनलुप्तहर्षान्}
{कौलीनं आत्माश्रयं आचचक्षे तेभ्यः पुनश्चेदं उवाच वाक्यम्}%॥१४.३६॥

\twolineshloka
{राजर्षिवंशस्य रविप्रसूतेरुपस्थितः पश्यत कीदृशोऽयम्}
{मत्तः सदाचारशुचेः कलङ्कः पयोदवातादिव दर्पणस्य}%॥१४.३७॥

\twolineshloka
{पौरेषु सोऽहं वहुलीभवन्तं अपां तरङ्गेष्विव तैलबिन्दुम्}
{सोढुं न तत्पूर्वं अवर्णं ईशे आलानिकं स्थाणुं इव द्विएपेन्द्रः}%॥१४.३८॥

\twolineshloka
{तस्यापनोदाय फलप्रवृत्तावुपस्थितायां अपि निर्व्यपेक्षः}
{त्यक्ष्यामि वैदेहसुतां पुरस्तात्समुद्रनेमिं पितुराज्ञयेव}%॥१४.३९॥

\twolineshloka
{अवैमि चैनां अनघेति किं तु लोकापवादो बलवान्मतो मे}
{छाया हि भूमेः शशिनो मलत्वेनारोपिता शुद्धिमतः प्रजाभिः}%॥१४.४०॥

\twolineshloka
{रक्षोवधान्तो न च मे प्रयासो व्यर्थः स वैरप्रतिमोचनाय}
{अमर्षणः शोणितकाङ्क्षया किं पदा स्पृशन्तं दशति द्विजिह्वः}%॥१४.४१॥

\twolineshloka
{तदेष सर्गः करुणार्द्रचित्तैर्न मे भवद्भिः प्रतिषेधनीयः}
{यद्यर्थिता निर्हृतवाच्यशल्यान्प्राणान्मया धारयितुं चिरं वः}%॥१४.४२॥

\twolineshloka
{इत्युक्तवन्तं जनकात्मजायां नितान्तरूक्षाभिनिवेशं ईशम्}
{न कश्चन भ्रातृषु तेषु शक्तो निषेद्धुं आसीदनुवर्तितुं वा}%॥१४.४३॥

\twolineshloka
{स लक्ष्मणं लक्ष्मणपूर्वजन्मा विलोक्य लोकत्रयगीतकीर्तिः}
{स्ॐयेति चाभाष्य यथार्थभाषी स्थितं निदेशे पृथगादिदेश}%॥१४.४४॥

\twolineshloka
{प्रजावती दोहदशंसिनी ते तपोवनेषु स्पृहयालुरेव}
{स्ॐयेति चाभाष्य यथार्थभाषी प्रापय्य वाल्मीकिपदं त्यजैनाम्}%॥१४.४५॥

\twolineshloka
{स शुश्रुवान्मातरि भार्गवेण पितुर्नियोगात्प्रहृतं द्विषद्वथ्}
{प्रत्यग्रहीदग्रजशासनं तदाज्ञा गुरूणां ह्यविचारणीया}%॥१४.४६॥

\twolineshloka
{अथानुकूलश्रवणप्रतीतां अत्रस्नुभिर्युक्तधुरं तुरंगैः}
{रथं सुमन्त्र प्रतिपन्नरश्मिं आरोप्य वैदेहसुतां प्रतस्थे}%॥१४.४७॥

\twolineshloka
{सा नीयमाना रुचिरान्प्रदेशान्प्रियंकरो मे प्रिय इत्यनन्दथ्}
{नाबुद्ध कल्पद्रुमतां विहाय जातं तं आत्मन्यसिपत्त्रवृक्षम्}%॥१४.४८॥

\twolineshloka
{जुगूह तस्याः पथि लक्ष्मणो यत्सव्येतरेण स्फुरता तदक्ष्णा}
{आख्यातं अस्यै गुरु भावि दुःखं अत्यन्तलुप्तप्रियदर्शनेन}%॥१४.४९॥

\twolineshloka
{सा दुर्निमित्तोपगताद्विषादात्सद्यः परिम्लानमुखारविन्दा}
{राज्ञः शिवं सावरजस्य भूयादित्याशशंसे करणैरबाह्यैः}%॥१४.५०॥

\twolineshloka
{गुरोर्नियोगाद्वनितां वनान्ते साध्वीं सुमित्रातनयो विहास्यन्}
{अवार्यतेवोत्थितवीचिहस्तैर्जह्नोर्दुहित्रा शितया पुरस्ताथ्}%॥१४.५१॥

\twolineshloka
{रथात्स यन्त्रा निगृहीतवाहात्तां भ्रातृह्यायां पुलिनेऽवतार्य}
{गङ्गां निषादाहृतनौविशेषस्ततार संधां इव सत्यसंधः}%॥१४.५२॥

\twolineshloka
{अथ व्यवस्थापितवाक्कथंचित्स्ॐइत्रिरन्तर्गतबाष्पकण्ठः}
{औत्पातिको मेघ इवाश्मवर्षं महीपतेः शासनं उज्जगार}%॥१४.५३॥

\twolineshloka
{ततोऽभिषङ्गानिलविप्रविद्धा प्रभ्रश्यमानाभरणस्प्रसूना}
{स्वमूर्तिलाभप्रकृतिं धरित्रीं लतेव सीता सहसा जगाम}%॥१४.५४॥

\twolineshloka
{इक्ष्वाकुवंशप्रभवः कथं त्वां त्यजेदकस्मात्पतिरार्यवृत्तः}
{इति क्षितिः संशयितेव तस्यै ददौ प्रवेशं जननी न तावथ्}%॥१४.५५॥

\twolineshloka
{सा लुप्तसंज्ञा न विवेद दुःखं प्रत्यागतासुः समतप्यतान्तः}
{तस्याः सुमित्रात्मजयत्नलब्धो मोहादभूत्कष्टतरः प्रबोधः}%॥१४.५६॥

\twolineshloka
{न चावदद्भर्तुरवर्णं आर्या निराकरिष्णोर्वृजिनादृतेऽपि}
{आत्मानं एव स्थिरदुःखबाजं पुनः पुनर्दुष्कृतिनं निनिन्द}%॥१४.५७॥

\twolineshloka
{आश्वास्य रामावरजः सतीं तां आख्यातवाल्मीकिनिकेतमार्गः}
{निघ्नस्य मे भर्तृनिदेशरौक्ष्यं देवि क्षमस्वेति बभूव नम्रः}%॥१४.५८॥

\twolineshloka
{सीता समुत्थाप्य जगाद वाक्यं प्रीतास्मि ते स्ॐयचिराय जीव}
{विडौजसा विष्णुरिवाग्रजेन भ्रात्रा यदित्थं परवानसि त्वम्}%॥१४.५९॥

\twolineshloka
{श्वश्रूजनं सर्वं अनुक्रमेण विज्ञापय प्रापितमत्प्रणामः}
{प्रजानिशेकं मयि वर्तमानं सूनोरनुध्यायत चेतसेति}%॥१४.६०॥

\twolineshloka
{वाच्यस्त्वया मद्वचनात्स राजा वह्नौ विशुद्धां अपि यत्समक्षम्}
{मां लोकवादश्रवणादहासीः श्रुतस्य किं तत्सदृशं कुलस्य}%॥१४.६१॥

\twolineshloka
{कल्याणबुद्धेरथ वा तवायं न कामचारो मयि शङ्कनीयः}
{ममैव जन्मान्तरपातकानां विपाकविस्फूर्जथुरप्रसह्यः}%॥१४.६२॥

\twolineshloka
{उपस्थितां पूर्वं अपास्य लक्ष्मीं वनं मया सार्धं असि प्रपन्नः}
{तदास्पदं प्राप्य तयातिरोषात्सोढास्मि न त्वद्भवने वसन्ती}%॥१४.६३॥

\twolineshloka
{निशाचरोपप्लुतभर्तृकाणां तपस्विनीनां भवतः प्रसादाथ्}
{भूत्वा शरण्या शरणार्थं अन्यां कथं प्रपत्स्ये त्वयि दीप्यमाने}%॥१४.६४॥

\twolineshloka
{किं वा तवात्यन्तवियोगमोघे कुर्यां उपेक्षां हतजीवितेऽस्मिन्}
{स्याद्रक्षणीयं यदि मे न तेजस्त्वदीयं अन्तर्गतं अन्तरायः}%॥१४.६५॥

\twolineshloka
{साहं तपः सूर्यनिविष्टदृष्टिरूर्ध्वं प्रसूतेस्चरितुं यतिष्ये}
{तथा यथा मे जननान्तरेऽपि त्वं एव भर्ता न च विप्रयोगः}%॥१४.६६॥

\twolineshloka
{नृपस्य वर्णाश्रमरक्षणं यत्स एव धर्मो मनुना प्रणीतः}
{निर्वासिताप्येवं अतस्त्वयाहं तपस्विसामान्यं अवेक्षणीया}%॥१४.६७॥

\twolineshloka
{तथेति तस्याः प्रतिगृह्य वाचं रामानुजे दृश्टिपथं व्यतीते}
{सा मुक्तकण्ठं व्यसनातिभाराच्चक्रन्द विग्ना कुररीव भूयः}%॥१४.६८॥

\twolineshloka
{नृत्यं मयूराः कुसुमानि वृक्षा दर्भानुपात्तान्विजहुर्हरिण्यः}
{तस्याः प्रपन्ने समदुःखभावं अत्यन्तं आसीद्रुदितं वनेऽपि}%॥१४.६९॥

\twolineshloka
{तां अभ्यगच्छद्रुदितानुसारी कविः कुशेध्माहरणाय यातः}
{निषादविद्धाण्डजदर्शनोत्थः श्लोकत्वं आपद्यत यस्य शोकः}%॥१४.७०॥

\twolineshloka
{तं अश्रु नेत्रावरणं प्रमृज्य सीता विलापाद्विरता ववन्दे}
{तस्यै मुनिर्दोहदलिङ्गदर्शी दाश्वान्सुपुर्त्राशिषं इत्युवाच}%॥१४.७१॥

\twolineshloka
{जाने विषृश्टां प्रणिधानतस्त्वां मिथ्यापवादक्षुभितेन भर्त्रा}
{तन्मा व्यथिष्ठा विषयान्तरस्थं प्राप्तासि वैदेहि पितुर्निकेतम्}%॥१४.७२॥

\twolineshloka
{उथ्कातलोकत्रयकण्टकेऽपि सत्यप्रतिज्ञेऽप्यविकत्थनेऽपि}
{त्वां प्रत्यकस्मात्कलुषप्रवृत्तावस्त्येव मन्युर्भरताग्रजे मे}%॥१४.७३॥

\twolineshloka
{तवेन्दुकीर्तिः श्वशुरः सखा मे सतां भवोच्छेदकरः पिता ते}
{धुरि स्थिता त्वं पतिदेवतानां किं तन्न येनासि ममानुकम्प्या}%॥१४.७४॥

\twolineshloka
{तपस्विसंसर्गविनितत्सत्त्वे तपोवने वीतभया वसास्मिन्}
{इतो भविष्यत्यनघप्रसूतेरपत्यसंस्कारमयो विधिस्ते}%॥१४.७५॥

\twolineshloka
{अशून्यतीरां मुनिसंनिवेशैस्तमोऽपहन्त्रीं तमसां विगाह्य}
{तत्सैकतोत्सङ्गबलिक्रियाभिः संपत्स्यते ते मनसः प्रसादः}%॥१४.७६॥

\twolineshloka
{पुष्पं फलं चार्तवं आहरन्त्यो बीजं च बालेयं अकृष्टरोहि}
{विनोदयिष्यन्ति नवाभिषङ्गां उदारवाचो मुनिकन्यकास्त्वाम्}%॥१४.७७॥

\twolineshloka
{पयोघटैराश्रमबालवृक्षान्संवर्धयन्ती स्वबलानुरूपैः}
{असंशयं प्राक्तनयोपपत्तेः स्तनंधयप्रीतिं अवाप्स्यसि त्वम्}%॥१४.७८॥

\twolineshloka
{अनुग्रहप्रत्यभिनन्दिनीं तां वाल्मीकिरादाय दयार्द्रचेताः}
{सायं मृगाध्यासितवेदिपार्श्वं स्वं आश्रमं शान्तमृगं निनाय}%॥१४.७९॥

\twolineshloka
{तां अर्पयां आस च शोकदीनां तदागमप्रीतिषु तापसीषु}
{निर्विष्टसारां पितृभिर्हिमांशोरन्त्यां कलां दर्श इवौषधीषु}%॥१४.८०॥

\twolineshloka
{ता इङ्गुदीस्नेहकृतप्रदीपं आस्तीर्णमेध्याजिनतल्पं अन्तः}
{तस्यै सपर्यानुपदं दिनान्ते निवासहेतोरुटजं वितेरुः}%॥१४.८१॥

\twolineshloka
{तत्राभिषेकप्रयता वसन्ती प्रयुक्तपूजा विधिनातिथिभ्यः}
{वन्येन सा वल्कलिनी शरीरं पत्युः प्रजासंततये बभार}%॥१४.८२॥

\twolineshloka
{अपि प्रभुः सानुशयोऽधुना स्यात्किं उत्सुकः शक्रजितोऽपि हन्ता}
{शशंस सीतापरिदेवनान्तं अनुष्ठितं शासनं अग्रजाय}%॥१४.८३॥

\twolineshloka
{बभूव रामः सहसा सबाष्पस्तुषारवर्षीव सहस्यचन्द्रः}
{कौलीनभीतेन गृहान्निरस्ता न तेन वैदेहसुता मनस्तः}%॥१४.८४॥

\twolineshloka
{निगृह्य शोकं स्वयं एव धीमान्वर्णाश्रमावेक्षणजागरूकः}
{स भ्रातृसाधारणभोगं ऋद्धं राज्यं रजोरिक्तमनाः शशास}%॥१४.८५॥

\twolineshloka
{तां एकभार्यां परिवादभीरोः साध्वीं अपि त्यक्तवतो नृपस्य}
{वक्षस्यसंघट्टसुखं वसन्ती रेजे सपत्नीरहितेव लक्ष्मीः}%॥१४.८६॥

\twolineshloka
{सीतां हित्वा दशमुखरिपुर्नोपयेम यदन्यां तस्या एव प्रतिकृतिसखो यत्क्रतूनाजहार}
{वृत्तान्तेन श्रवणविषयप्रापिणा तेन भर्तुः सा दुर्वारं कथं अपि परित्यागदुःखं विषेहे}%॥१४.८७॥

॥इति श्री-महाकवि-कालिदास-कृत-रघुवंश-महाकाव्ये चतुर्दशः सर्गः॥

\sect{पञ्चदशः सर्गः}

\twolineshloka
{कृतसीतापरित्यागः स रत्नाकरमेखलाम्}
{बुभूजे पृथिवीपालः पृथिवीं एव केवलाम्}%॥१५.१॥

\twolineshloka
{लवणेन विलुप्तेज्यास्तामिस्रेण तं अभ्ययुः}
{मुनयो यमुनाभाजः शरण्यं शरणार्थिनः}%॥१५.२॥

\twolineshloka
{अवेक्ष्य रामं ते तस्मिन्न प्रजह्रुः स्वतेजसा}
{त्राणाभावे हि शापास्त्राः कुर्वन्ति तपसो व्ययम्}%॥१५.३॥

\twolineshloka
{प्रतिशुश्राव काकुत्स्थस्तेभ्यो विघ्नप्रतिक्रियाम्}
{धर्मसंरक्षणायैव प्रवृत्तिर्भुवि शार्ङ्गिणः}%॥१५.४॥

\twolineshloka
{ते रामाय वधोपायं आचख्युर्विबुधविषः}
{दुर्जयो लवणः शूली विशूलः प्रार्थ्यतां इति}%॥१५.५॥

\twolineshloka
{आदिदेशाथ शत्रुघ्नं तेषां क्षेमाय राघवः}
{करिष्यन्निव नामास्य यथार्थं अरिनिग्रहाथ्}%॥१५.६॥

\twolineshloka
{यः कश्चन रघूणां हि परं एकः परंतपः}
{अपवाद इवोत्सर्गं व्यावर्तयितुं ईश्वरः}%॥१५.७॥

\twolineshloka
{अग्रजेन प्रयुक्ताशीस्तदा दाशरथी रथी}
{ययौ वन्स्तह्लिः पश्यन्पुष्पिताः सुरभीरभीः}%॥१५.८॥

\twolineshloka
{रामादेशादनुपदं सेनाङ्गं तस्य सिद्धये}
{पश्चादध्ययनार्थस्य धातोरधिरिवाभवथ्}%॥१५.९॥

\twolineshloka
{आदिष्टवर्त्मा मुनिभिः स गच्छंस्तपतां वरः}
{विरराज रथपृष्ठैर्वालखिल्यैरिवांशुमान्}%॥१५.१०॥

\twolineshloka
{तस्य मार्गवशादेका बभूव वसतिर्यतः}
{रथस्वनोत्कण्ठमृगे वाल्मीकीये तपोवने}%॥१५.११॥

\twolineshloka
{तं ऋषिः पूजयां आस कुमारं क्लान्तवाहनम्}
{तपःप्रभावसिद्धाभिर्विशेषप्रतिपत्तिभिः}%॥१५.१२॥

\twolineshloka
{तस्यां एवास्य यामिन्यां अन्तर्वत्नी प्रजावती}
{सुतावसूत संपन्नौ कोशदण्डाविव क्षितिः}%॥१५.१३॥

\twolineshloka
{संतानश्रवणाद्भ्रातुः स्ॐइत्रिः स्ॐअनस्यवान्}
{प्राञ्जलिर्मुनिं आमन्त्र्य प्रातर्युक्तरथो ययौ}%॥१५.१४॥

\twolineshloka
{स च प्राप मधूपघ्नं कुम्भीनस्याश्च कुक्षिजः}
{वनात्करं इवादाय सत्त्वराशिं उपस्थितः}%॥१५.१५॥

\twolineshloka
{धूमधूम्रो वसाघन्धी ज्वालाबभ्रुशिरोरुहः}
{क्रव्याद्गणपरीवारश्चिताग्निरिव जङ्गमः}%॥१५.१६॥

\twolineshloka
{अपशुलं तं आसाद्य लवणं लक्ष्मणानुजः}
{रुरोध संमुखीनो हि जयो रन्ध्रप्रहारिणाम्}%॥१५.१७॥

\twolineshloka
{नातिपर्याप्तं आलक्ष्य मत्कुक्षेरद्य भोजनम्}
{दिष्ट्या त्वं असि मे धात्रा भीग्तेनेवोपपादितः}%॥१५.१८॥

\twolineshloka
{इति संतर्ज्य शत्रुघ्नं राक्षसस्तज्जिघांसया}
{प्रांशुं उत्पाटयां आस मुस्तास्तम्बं इव द्रुमम्}%॥१५.१९॥

\twolineshloka
{स्ॐइत्रेर्निशितैर्बाणैरन्तरा शकलीकृतः}
{गात्रं पुष्परजः प्राप न शाखी नैरृतेरितः}%॥१५.२०॥

\twolineshloka
{विनाशात्तस्य वृक्षस्य रक्षस्तस्मै महोपलम्}
{प्रजिघाय कृतान्तस्य मुष्टिं पृथगिव स्थितम्}%॥१५.२१॥

\twolineshloka
{ऐन्द्रं अस्त्रं उपादाय शत्रुघ्नेन स ताडितः}
{सिकताभ्योऽपि हि परां प्रपेदे परमाणुताम्}%॥१५.२२॥

\twolineshloka
{दक्षिणं दोषं उद्यम्य राक्षसस्तं उपाद्रवथ्}
{एकताल इवोपातपवनप्रेरितो गिरिः}%॥१५.२३॥

\twolineshloka
{कार्ष्नेन पत्त्रिना शत्रुः स भिन्नर्हृदयः पतन्}
{आनिनाय भुवः कम्पं जहाराश्रमवासिनाम्}%॥१५.२४॥

\twolineshloka
{वयसां पङ्क्तयः पेतुर्हतस्योपरि रक्षसः}
{तत्प्रतिद्वन्दिनो मूर्ध्नि दिव्याः कुसुमवृष्टयः}%॥१५.२५॥

\twolineshloka
{स हत्वा लवणं वीरस्तदा मेने महौजसः}
{भ्रातुः सोदर्यं आत्मानं इन्द्रजिद्वधशोभिनः}%॥१५.२६॥

\twolineshloka
{तस्य संस्तूयमानस्य चरितार्थैस्तपस्विभिः}
{शुशुभे विक्रमोदग्रं व्रीडयावनतं शिरः}%॥१५.२७॥

\twolineshloka
{उपकूलं स कालिन्द्याः पुरीं पौरुषभूषणः}
{निर्ममे निर्ममोऽर्थेषु मथुरां मधुराकृतिः}%॥१५.२८॥

\twolineshloka
{या सौराज्यप्रकाशाभिर्बभौ पौरविभूतिभिः}
{स्वर्गाभिष्यन्दवमनं कृत्वेवोपनिवेशिता}%॥१५.२९॥

\twolineshloka
{तत्र सौधगतः पश्यन्यमुनां चक्रवाकिनीम्}
{हेम भक्तिमतीं भूमेः प्रवेणीं इव प्रिपिये}%॥१५.३०॥

\twolineshloka
{सखा दशरथस्याथ जनकस्य च मन्त्रकृथ्}
{संचस्कारोभयप्रीत्या मैथिलेयौ यथाविधि}%॥१५.३१॥

\twolineshloka
{स तौ कुशलवोन्मृष्टगर्भक्लेदौ तदाख्यया}
{कविः कुशलवावेव चकार किल नामतः}%॥१५.३२॥

\twolineshloka
{साङ्गं च वेदं अध्याप्य किंचिदुत्क्रान्तशैशवौ}
{स्वकृतिं गापयां आस कविप्रथमपद्धतिम्}%॥१५.३३॥

\twolineshloka
{रामस्य मधुरं वृत्तं गायन्तौ मातुरग्रतः}
{तद्वियोगव्यथां किंचिच्छिथिलीचक्रतुः सुतौ}%॥१५.३४॥

\twolineshloka
{इतरेऽपि रघोर्वंश्यास्त्रयस्त्रेताग्नितेजसः}
{तद्योगात्पतिवत्नीषु पत्नीष्वासन्द्विसूनवः}%॥१५.३५॥

\twolineshloka
{शत्रुघातिनि शत्रुघ्नः सुबाहौ च बहुश्रुते}
{मथुराविदिशे सून्वोर्निदधे पूर्वजोत्सुकः}%॥१५.३६॥

\twolineshloka
{भूयस्तपोव्ययो मा भूद्वाल्मीकेरिति सोऽत्यगाथ्}
{मैथिलीतनयोद्गीतनिष्पन्दमृगं आश्रमम्}%॥१५.३७॥

\twolineshloka
{वशी विवेश चायोध्यां रथ्यासंस्कारशोभिनीम्}
{लवणस्य वधात्पौरैरतिगौरवं ईक्षितः}%॥१५.३८॥

\twolineshloka
{स ददर्श सभामध्ये सभासद्भिरुपस्थितम्}
{रामं सीतापरित्यागादसामण्यपतिं भुवः}%॥१५.३९॥

\twolineshloka
{तं अभ्यनन्दत्प्रणतं लवणान्तकं अग्रजः}
{कालनेमिवधात्प्रीतस्तुराषाडिव शार्ङ्गिणम्}%॥१५.४०॥

\twolineshloka
{स पृष्टः सर्वतो वार्त्तां आख्याद्राज्ञे न संततिम्}
{प्रत्यर्पयिष्यतः काले कवेराद्यस्य शासनाथ्}%॥१५.४१॥

\twolineshloka
{अथ जानपदो विप्रः शिशुं अप्राप्तयौवनम्}
{अवतार्य्ऽ आङ्कशय्यास्थं द्वारि चक्रन्द भूपतेः}%॥१५.४२॥

\twolineshloka
{शोचनीयासि वसुधे या त्वं दशरथाच्च्युता}
{रामहस्तं अनुप्राप्य कष्टात्कष्टतरं गता}%॥१५.४३॥

\twolineshloka
{श्रुत्वा तस्य शुचो हेतुं गोप्ता जिह्राय राघवः}
{न ह्यकालभवो मृत्युरिक्ष्वाकुपदं अस्पृशथ्}%॥१५.४४॥

\twolineshloka
{स मुहूर्तं सहस्वेति द्विजं आश्वास्य दुःखितम्}
{यानं सस्मार कौबेरं वैवस्वतजिगीषया}%॥१५.४५॥

\twolineshloka
{आत्तशस्त्रस्तदध्यास्य प्रतिस्थः स रघूद्वहः}
{उच्चचार पुरस्तस्य गूढरूपा सरस्वती}%॥१५.४६॥

\twolineshloka
{राजन्प्रजासु ते कश्चिदपचारः प्रवर्तते}
{तं अन्विष्य प्रशमयेर्भवितासि ततः कृती}%॥१५.४७॥

\twolineshloka
{इत्याप्तवचनाद्रामो विनेष्यन्वर्णविक्रियाम्}
{दिशः पपात पत्त्रेण वेगनिष्कम्पकेतुना}%॥१५.४८॥

\twolineshloka
{अथ धूमाभिताम्राक्षं वृक्षाखावलम्बिनम्}
{ददर्श कंचिदैक्श्वाकस्तपस्यन्तं अध्ॐउखम्}%॥१५.४९॥

\twolineshloka
{पृष्टनामान्वयो राज्ञा स किलाचष्ट धूमपः}
{आत्मानं शम्बुकं नाम शूद्रं सुरपदार्थिनम्}%॥१५.५०॥

\twolineshloka
{तपस्यनधिकारित्वात्प्रजानां तं अघावहम्}
{शीर्षच्छेद्यं परिच्छिद्य नियन्ता शस्त्रं आददे}%॥१५.५१॥

\twolineshloka
{स तद्वक्त्रं हिमक्लिष्टकिञ्जल्कं इव पङ्कजम्}
{ज्योतिष्कणाहतश्मश्रु कण्ठनालादपाहरथ्}%॥१५.५२॥

\twolineshloka
{कृतण्डः स्वयं राज्ञा लेभे शूद्रः सतां गतिम्}
{तपसा दुश्चरेणापि न स्वमार्गविलङ्घिना}%॥१५.५३॥

\twolineshloka
{रघुनाथोऽप्यगस्त्येन मार्गसंदर्शितात्मना}
{महौजसा संयुयुजे शरत्काल इवेन्दुना}%॥१५.५४॥

\twolineshloka
{कुम्भयोनिरलंकारं तस्मै दिव्यपरिग्रहम्}
{ददौ दत्तं समुद्रेण पीतेनेवात्मनिष्क्रयम्}%॥१५.५५॥

\twolineshloka
{तं दधन्मैथिलीकण्ठनिर्व्यापारेण बाहुना}
{पश्चान्निववृते रामः प्राक्परासुर्द्विजात्मजः}%॥१५.५६॥

\twolineshloka
{तस्य पूर्वोदितां निन्दां द्विजः पुत्रसमागतः}
{स्तुत्या निवर्तयां आस त्रातुर्वैवस्वतादपि}%॥१५.५७॥

\twolineshloka
{तं अध्वराय मुक्ताश्वं रक्षःकपिनरेश्वराः}
{मेघाः सस्यं इवाम्भोभिरभ्यवर्षन्नुपायनैः}%॥१५.५८॥

\twolineshloka
{दिग्भ्यो निमन्त्रिताश्चैनं अभिजग्मुर्महर्षयः}
{न भ्ॐआन्येव धिष्ण्यानि हित्वा ज्योतिर्मयान्यपि}%॥१५.५९॥

\twolineshloka
{उपशल्यनिविष्टैस्तैश्चतुर्द्वारमुखी बभौ}
{अयोध्या सृष्टलोकेव सद्यः पैतामही तनुः}%॥१५.६०॥

\twolineshloka
{श्लाघ्यस्त्यागोऽपि वैदेह्याः पत्युः प्राग्वंशवासिनः}
{अनन्यहानेः तस्यासीत्सैव जाया हिरण्मयी}%॥१५.६१॥

\twolineshloka
{विधेरधिकसंभारस्ततः प्रववृते मखः}
{आसन्यत्र क्रियाविघ्ना राक्षसा एव रक्षिणः}%॥१५.६२॥

\twolineshloka
{अथ प्राचेतसोपज्ञं रामायणं इतस्ततः}
{मैथिलेयौ कुशलवौ जगतुर्गुरुचोदितौ}%॥१५.६३॥

\twolineshloka
{वृत्तं रामस्य वाल्मीकेः कृतिस्तौ किंनरस्वनौ}
{किं तद्येन मनो हर्तुं अलं स्यातां न शृण्वताम्}%॥१५.६४॥

\twolineshloka
{रूपे गीते च माधुर्यं तयोस्तज्ज्ञैर्निवेदितम्}
{ददर्श सानुजो रामः शुश्राव च कुतूहली}%॥१५.६५॥

\twolineshloka
{तद्गीतश्रवणैकाग्रा संसदश्रुमुखी बभौ}
{हिमनिस्यन्दिनी प्रातर्निवाग्तेव वनस्थली}%॥१५.६६॥

\twolineshloka
{वयोवेषविसंवादि रामस्य च तयोश्च सा}
{जनता प्रेक्ष्य सादृश्यं नाक्शिकम्पं व्यतिष्ठत}%॥१५.६७॥

\twolineshloka
{उभयोर्न तथा लोकः प्रावीण्येन विसिष्मिये}
{नृपतेः प्रीतिदानेषु वीतस्पृहतया यथा}%॥१५.६८॥

\twolineshloka
{गेये केन विनीतौ वां कस्य चेयं कवेः कृतिः}
{इति राज्ञा स्वयं पृष्टौ तौ वाल्मीकिं अशंसताम्}%॥१५.६९॥

\twolineshloka
{अथ सावरजो रामः प्राचेतसं उपेयिवान्}
{उरिक्र्त्यात्मनो देहं राज्यं अस्मै न्यवेदयथ्}%॥१५.७०॥

\twolineshloka
{स तावाख्याय रामाय मैथिलेयौ तदात्मजौ}
{कविः कारुणिको वव्रे सीतायाः संपरिग्रहम्}%॥१५.७१॥

\twolineshloka
{तात शुद्धा समक्षं नः स्नुषा ते जातवेदसि}
{दौरात्म्याद्रक्षसस्तां तु नात्रत्याः श्रद्दधुः प्रजाः}%॥१५.७२॥

\twolineshloka
{ताः स्वचारित्रं उद्दिश्य प्रत्याययतु मैथिली}
{ततः पुत्रवतीं एनां प्रतिपत्स्ये तदाज्ञया}%॥१५.७३॥

\twolineshloka
{इति प्रतिश्रुते राज्ञा जानकीं आस्रमान्मुनिः}
{शिष्यैरानाययां आस स्वसिद्धिं नियमैरिव}%॥१५.७४॥

\twolineshloka
{अन्येद्युरथ काकुत्स्थः संनिपात्य पुरौकसः}
{कविं आह्वाययां आस प्रस्तुतप्रतिपत्तये}%॥१५.७५॥

\twolineshloka
{स्वरसंस्कारवत्येव पुत्राभ्यां सह सीतया}
{ऋचेवोदर्चिषं सूर्यं रामं मुनिरुपस्थितः}%॥१५.७६॥

\twolineshloka
{काषायपरिवीतेन स्वपदार्पितचक्षुषा}
{कविं आह्वाययां आस शान्तेन वपुषैव सा}%॥१५.७७॥

\twolineshloka
{जनास्तदालोकपथात्प्रतिसंहृतचक्षुषः}
{तस्थुस्तेऽवाङ्मुखाः सर्वे फलिता इव सालयः}%॥१५.७८॥

\twolineshloka
{तां दृष्टिविषये भर्तुर्मुनिरास्थितविष्टरः}
{कुरु निःसंशयं वत्से स्ववृत्ते लोकं इत्यशाथ्}%॥१५.७९॥

\twolineshloka
{अथ वाल्मीक्शिष्येण पुण्यं आवर्जितं पयः}
{आचम्योदीरयां आस सीता सत्यां सरस्वतीम्}%॥१५.८०॥

\twolineshloka
{वाङ्मनःकर्मभिः पत्यौ व्यभिचारो यथा न मे}
{तथा विश्वंभरे देवि मां अन्तर्धातुं अर्हसि}%॥१५.८१॥

\twolineshloka
{एवं उक्ते तया साध्व्या रन्ध्रात्सद्योभवाद्भुवः}
{शातह्रदं इव ज्योतिः प्रभामण्डलं उद्ययौ}%॥१५.८२॥

\twolineshloka
{तत्र नागफणोत्क्षिप्तसिंहासननिषेदुषी}
{समुद्ररशना साक्षात्प्रादुरासीद्वसुंधरा}%॥१५.८३॥

\twolineshloka
{सा सीतां अङ्कं आरोप्य भर्तरि प्रहितेक्षणाम्}
{मा मेति व्याहरत्येव तस्मिन्पातालं अभ्यगाथ्}%॥१५.८४॥

\twolineshloka
{धरायां तस्य संरम्भं सीताप्रत्यर्पणैषिणैः}
{गुरुर्विधिबलापेक्षी शमयां आस धन्विनः}%॥१५.८५॥

\twolineshloka
{ऋषीन्विसृज्य यज्ञान्ते सुहृदश्च पुरस्कृतान्}
{रामः सीतागतं स्नेहं निदधे तदपत्ययोः}%॥१५.८६॥

\twolineshloka
{युधाजितस्तु संदेशात्स देश सिन्धुनामकम्}
{ददौ दत्तप्रभावाय भरताय धृतप्रजः}%॥१५.८७॥

\twolineshloka
{भरतस्तत्र गन्धर्वान्युधि निजित्य केवलम्}
{आतोद्यं ग्राहयां आस समत्याजयदायुधम्}%॥१५.८८॥

\twolineshloka
{स तक्षपुष्कलौ पुत्रौ राजधान्योस्तदाख्ययोः}
{अभिषिच्याभिषेकार्हौ रामान्तिकं अगात्पुनः}%॥१५.८९॥

\twolineshloka
{अङ्गदं चन्द्रकेतुं च लक्ष्मणोऽप्यात्मसंभवौ}
{शासनाद्रघुनाथस्य चक्रे कारापथेशावरौ}%॥१५.९०॥

\twolineshloka
{इत्यारोपितपुत्रास्ते जननीनां जनेश्वराः}
{भर्तृलोकप्रपन्नानां निवापान्विदधुः क्रमाथ्}%॥१५.९१॥

\twolineshloka
{उपेत्य मुनिवेषोऽथ कालः प्रोवाच राघवम्}
{रहःसंवादिनौ पास्येदावां यस्तं त्यजेरिति}%॥१५.९२॥

\twolineshloka
{तथेति प्रतिपन्नाय विवृतात्मा नृपाय सः}
{आचख्यौ दिवं अध्यास्व शासनात्परमेष्ठिनः}%॥१५.९३॥

\twolineshloka
{विद्वानपि तयोर्द्वाःस्तहः समयं लक्ष्मणोऽभिनथ्}
{भीतो दुर्वाससः शापाद्रामसंदर्शनार्थिनः}%॥१५.९४॥

\twolineshloka
{स गत्वा सरयूतीरं देहत्यागेन योगविथ्}
{चकार वितथां भ्रातुः प्रतिज्ञां पूर्वजन्मनः}%॥१५.९५॥

\twolineshloka
{तस्मिन्नात्मचतुर्भागे प्राङ्नाकं अधितस्थुषि}
{राघवः शिथिलं तस्थौ भुवि धर्मस्त्रिपादिव}%॥१५.९६॥

\twolineshloka
{स निवेश्य कुशावत्यां रिपुनागाङ्कुषं कुशम्}
{शरावत्यां सतां सूक्तैर्जनिताश्रुलवं लवम्}%॥१५.९७॥

\twolineshloka
{उदक्प्रतस्थे स्थिरधीः सानुजोऽग्निपुरःसरः}
{अन्वितः पतिवात्सल्याद्गृहवर्जं अयोध्यया}%॥१५.९८॥

\twolineshloka
{जगृहुस्तस्य चित्तज्ञाः पदवीं हरिराक्षसाः}
{कदम्बमुकुलस्थूलैरभिवृष्टं प्रजाश्रुभिः}%॥१५.९९॥

\twolineshloka
{उपस्थितविमानेन तेन भक्तानुकम्पिना}
{चक्रे त्रिदिवनिःष्रेणिः सरयूरनुयायिनाम्}%॥१५.१००॥

\twolineshloka
{यद्गोप्रतरकल्पोऽभुत्संमर्दस्तत्र मज्जताम्}
{अतस्तदाख्यया तीर्थं पावनं भुवि पप्रथे}%॥१५.१०१॥

\twolineshloka
{स विभुर्विबुधांशेषु प्रतिपन्नात्ममूर्तिषु}
{त्रिदशीभूतपौराणां स्वर्गान्तरं अकल्पयथ्}%॥१५.१०२॥

\twolineshloka
{निर्वर्त्यैवं दशमुखशिरश्छेदकार्यं सुराणां विष्वक्सेनः स्वतनुं अविशत्सर्वलोकप्रतिष्ठाम्}
{लङ्कानाथं पवनतनयं चोभयं स्थापयित्वा कीर्तिस्तम्भद्वयं इव गिरौ दक्षिणे चोत्तरे च}%॥१५.१०३॥

॥इति श्री-महाकवि-कालिदास-कृत-रघुवंश-महाकाव्ये पञ्चदशः सर्गः॥