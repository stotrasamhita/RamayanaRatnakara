\sect{एकादशः सर्गः}

\fourlineindentedshloka
{कौशिकेन स किल क्षितीश्वरो}
{राममध्वरविघातशान्तये}
{काकपक्षधरमेत्य याचितः}
{तेजसाम् हि न वयः समीक्ष्यते} % ११-१

\fourlineindentedshloka
{कृच्छ्रलब्धमपि लब्धवर्णभाक्}
{तम् दिदेश मुनये सलक्ष्मणम्}
{अप्यसुप्रणयिनाम् रघोः कुले}
{न व्यहन्यत कदाचिदर्थिता} % ११-२

\fourlineindentedshloka
{यावदादिशति पार्थिवस्तयोः}
{निर्गमाय पुरमार्गसंस्क्रियाम्}
{तावदाशु विदधे मरुत्सखैः}
{सा सपुष्पजलवर्षिभिर्घनैः} % ११-३

\fourlineindentedshloka
{तौ निदेशकरणोद्यतौ पितुः}
{धन्विनौ चरणयोर्निपेततुः}
{भूपतेरपि तयोः प्रवत्स्यतोः}
{नम्रयोरुपरि बाष्पबिन्दवः} % ११-४

\fourlineindentedshloka
{तौ पितुर्नयनजेन वारिणा}
{किञ्चिदुक्षितशिखण्डकावुभौ}
{धन्विनौ तमृषिमन्वगच्छताम्}
{पौरदृष्टिकृतमार्गतोरणौ} % ११-५

\fourlineindentedshloka
{लक्ष्मणानुचरमेव राघवम्}
{नेतुमैच्छदृषिरित्यसौ नृपः}
{आशिषम् प्रयुयुजे न वाहिनीम्}
{सा हि रक्षणविधौ तयोः क्षमा} % ११-६

\fourlineindentedshloka
{मातृवर्गचरणस्पृशौ मुनेः}
{तौ प्रपद्य पदवीम् महौजसः}
{रेजतुर्गतिवशात्प्रवर्तिनौ}
{भास्करस्य मधुमाधवाविव} % ११-७

\fourlineindentedshloka
{वीचिलोलभुजयोस्तयोर्गतम्}
{शैशवाच्चपलमप्यशोभत}
{तोयदागमे इवोद्ध्यभिद्ययोः}
{नामधेयसदृशम् विचेष्टितम्} % ११-८

\fourlineindentedshloka
{तौ बलातिबलयोः प्रभावतो}
{विद्ययोः पथि मुनिप्रदिष्टयोः}
{मम्लतुर्न मणिकुट्टिमोचितौ}
{मातृपार्श्वपरिवर्तिनाविव} % ११-९

\fourlineindentedshloka
{पूर्ववृत्तकथितैः पुराविदः}
{सानुजः पितृसखस्य राघवः}
{उह्यमान इव वाहनोचितः}
{पादचारमपि न व्यभावयत्} % ११-१०

\fourlineindentedshloka
{तौ सरांसि रसवद्भिरम्बुभिः}
{कूजितैः श्रुतिसुखैः पतत्रिणः}
{वायवः सुरभिपुष्परेणुभिः}
{छायया च जलदाः सिषेविरे} % ११-११

\fourlineindentedshloka
{नाम्भसाम् कमलशोभिनाम् तथा}
{शाखिनाम् च न परिश्रमच्छिदाम्}
{दर्शनेन लघुना यथा तयोः}
{प्रीतिमापुरुभयोस्तपस्विनः} % ११-१२

\fourlineindentedshloka
{स्थाणुदग्धवपुषस्तपोवनम्}
{प्राप्य दाशरथिरात्तकार्मुकः}
{विग्रहेण मदनस्य चारुणा}
{सोऽभवत्प्रतिनिधिर्न कर्मणा} % ११-१३

\fourlineindentedshloka
{तौ सुकेतुसुतया खिलीकृते}
{कौशिकाद्विदितशापया पथि}
{निन्यतुः स्थलनिवेशिताटनी}
{लीलयैव धनुषी अधिज्यताम्} % ११-१४

\fourlineindentedshloka
{ज्यानिनादमथ गृह्णती तयोः}
{प्रादुरास बहुलक्षपाछविः}
{ताडका चलकपालकुण्डला}
{कालिकेव निबिडा बलाकिनी} % ११-१५

\fourlineindentedshloka
{तीव्रवेगधुतमार्गवृक्षया}
{प्रेतचीवरवसा स्वनोग्रया}
{अभ्यभावि भरताग्रजस्तया}
{वात्ययेव पितृकाननोत्थया} % ११-१६

\fourlineindentedshloka
{उद्यतैकभुजयष्टिमायतीम्}
{श्रोणिलम्बिपुरुषान्त्रमेखलाम्}
{ताम् विलोक्य वनितावधे घृणाम्}
{पत्रिणा सह मुमोच राघवः} % ११-१७

\fourlineindentedshloka
{यच्चकार विवरम् शिलाघने}
{ताडकोरसि स रामसायकः}
{अप्रविष्टविषयस्य रक्षसाम्}
{द्वारतामगमदन्तकस्य तत्} % ११-१८

\fourlineindentedshloka
{बाणभिन्नहृदया निपेतुषी}
{सा स्वकाननभुवम् न केवलाम्}
{विष्टपत्रयपराजयस्थिराम्}
{रावणश्रियमपि व्यकम्पयत्} % ११-१९

\fourlineindentedshloka
{राममन्मथशरेण ताडिता}
{दुःसहेन हृदये निशाचरी}
{गन्धवद्रुधिरचन्दनोक्षिता}
{जीवितेशवसतिम् जगाम सा} % ११-२०

\fourlineindentedshloka
{नैरृतघ्नमथ मन्त्रवन्मुनेः}
{प्रापदस्त्रमवदानतोषितात्}
{ज्योतिरिन्धननिपाति भास्करात्}
{सूर्यकान्त इव ताडकान्तकः} % ११-२१

\fourlineindentedshloka
{वामनाश्रमपदम् ततः परम्}
{पावनम् श्रुतमृषेरुपेयिवान्}
{उन्मनाः प्रथमजन्मचेष्टिता}
{न्यस्मरन्नपि बभूव राघवः} % ११-२२

\fourlineindentedshloka
{आससाद मुनिरात्मनस्ततः}
{शिष्यवर्गपरिकल्पितार्हणम्}
{बद्धपल्लवपुटाञ्जलिद्रुमम्}
{दर्शनोन्मुखमृगम् तपोवनम्} % ११-२३

\fourlineindentedshloka
{तत्र दीक्षितमृषिम् ररक्षतुः}
{विघ्नतो दशरथात्मजौ शरैः}
{लोकमन्धतमसात्क्रमोदितौ}
{रश्मिभिः शशिदिवाकराविव} % ११-२४

\fourlineindentedshloka
{वीक्ष्य वेदिमथ रक्तबिन्दुभिः}
{बन्धुजीवपृथुभिः प्रदूषिताम्}
{सम्भ्रमोऽभवदपोढकर्मणाम्}
{ऋत्विजाम् च्युतविकङ्कतस्रुचाम्} % ११-२५

\fourlineindentedshloka
{उन्मुखः सपदि लक्षमणाग्रजो}
{बाणमाश्रयमुखात्समुद्धरन्}
{रक्षसाम् बलमपश्यदम्बरे}
{गृध्रपक्षपवनेरितध्वजम्} % ११-२६

\fourlineindentedshloka
{तत्र यावधिपती मखद्विषाम्}
{तौ शरव्यमकरोत्स नेतरान्}
{किम् महोरगविसर्पिविक्रमो}
{राजिलेषु गरुडः प्रवर्तते} % ११-२७

\fourlineindentedshloka
{सोऽस्त्रमुग्रजवमस्त्रकोविदः}
{सन्दधे धनुषि वायुदैवतम्}
{तेन शैलगुरुमप्यपातयत्}
{पाण्डुपत्रमिव ताडकासुतम्} % ११-२८

\fourlineindentedshloka
{यः सुबाहुरिति राक्षसोऽपरः}
{तत्र तत्र विससर्प मायया}
{तम् क्षुरप्रशकलीकृतम् कृती}
{पत्रिणाम् व्यभजदाश्रमाद्बहिः} % ११-२९

\fourlineindentedshloka
{इत्यपास्तमखविघ्नयोस्तयोः}
{सांयुगीनमभिनन्द्य विक्रमम्}
{ऋत्विजः कुलपतेर्यथाक्रमम्}
{वाग्यतस्य निरवर्तयन्क्रियाः} % ११-३०

\fourlineindentedshloka
{तौ प्रणामचलकाकपक्षकौ}
{भ्रातराववभृथाप्लुतो मुनिः}
{आशिषामनुपदं समस्पृशत्}
{दर्भपाटलतलेन पाणिना} % ११-३१

\fourlineindentedshloka
{तम् न्यमन्त्रयत सम्भृतक्रतुः}
{मैथिलः स मिथिलाम् व्रजन्वशी}
{राघवावपि निनाय बिभ्रतौ}
{तद्धनुःश्रवणजम् कुतूहलम्} % ११-३२

\fourlineindentedshloka
{तैः शिवेषु वसतिर्गताध्वभिः}
{सायमाश्रमतनुष्वगृह्यत}
{येषु दीर्घतपसः परिग्रहो}
{वासवक्षणकलत्रताम् ययौ} % ११-३३

\fourlineindentedshloka
{प्रत्यपद्यत चिराय यत्पुन}
{श्चारु गौतमवधूः शिलामयी}
{स्वम् वपुः स किल किल्बिषच्छिदाम्}
{रामपादरजसामनुग्रहः} % ११-३४

\fourlineindentedshloka
{राघवान्वितमुपस्थितम् मुनिम्}
{तम् निशम्य जनको जनेश्वरः}
{अर्थकामसहितम् सपर्यया}
{देहबद्धमिव धर्ममभ्यगात्} % ११-३५

\fourlineindentedshloka
{तौ विदेहनगरीनिवासिनाम्}
{गाम् गताविव दिवः पुनर्वसू}
{मन्यते स्म पिबताम् विलोचनैः}
{पक्ष्मपातमपि वञ्चनाम् मनः} % ११-३६

\fourlineindentedshloka
{यूपवत्यवसिते क्रियाविधौ}
{कालवित्कुशिकवंशवर्धनः}
{राममिष्वसनदर्शनोत्सुकम्}
{मैथिलाय कथयाम्बभूव सः} % ११-३७

\fourlineindentedshloka
{तस्य वीक्ष्य ललितम् वपुः शिशोः}
{पार्थिवः प्रथितवंशजन्मनः}
{स्वम् विचिन्त्य च धनुर्दुरानमम्}
{पीडितो दुहितृशुल्कसंस्थया} % ११-३८

\fourlineindentedshloka
{अब्रवीच्च भगवन्मतङ्गजै}
{र्यद्बृहद्भिरपि कर्म दुष्करम्}
{तत्र नाहमनुमन्तुमुत्सहे}
{मोघवृत्ति कलभस्य चेष्टितम्} % ११-३९

\fourlineindentedshloka
{ह्रेपिता हि बहवो नरेश्वरास्तेन}
{तात धनुषा धनुर्भृतः}
{ज्यानिघातकठिनत्वचो भुजान्}
{स्वान्विधूय धिगिति प्रतस्थिरे} % ११-४०

\fourlineindentedshloka
{प्रत्युवाच तमृषिर्निशम्यताम्}
{सारतोऽयमथवा गिरा कृतम्}
{चाप एव भवतो भविष्यति}
{व्यक्तशक्तिरशनिर्गिराविव} % ११-४१

\fourlineindentedshloka
{एवमाप्तवचनात्स पौरुषम्}
{काकपक्षकधरेऽपि राघवे}
{श्रद्दधे त्रिदशगोपमात्रके}
{दाहशक्तिमिव कृष्णवर्त्मनि} % ११-४२

\fourlineindentedshloka
{व्यादिदेश गणशोऽथ पार्श्वगान्}
{कार्मुकाभिहरणाय मैथिलः}
{तैजसस्य धनुषः प्रवृत्तये}
{तोयदानिव सहस्रलोचनः} % ११-४३

\fourlineindentedshloka
{तत्प्रसुप्तभुजगेन्द्रभीषणम्}
{वीक्ष्य दाशरथिराददे धनुः}
{विद्रुतक्रतुमृगानुसारिणम्}
{येन बाणमसृजत्वृषध्वजः} % ११-४४

\fourlineindentedshloka
{आततज्यमकरोत्स संसदा}
{विस्मयस्तिमितनेत्रमीक्षितः}
{शैलसारमपि नातियत्नतः}
{पुष्पचापमिव पेशलम् स्मरः} % ११-४५

\fourlineindentedshloka
{भज्यमानमतिमात्रकर्षणात्}
{तेन वज्रपरुषस्वनम् धनुः}
{भार्गवाय दृढमन्यवे पुनः}
{क्षत्रमुद्यतमिव न्यवेदयत्} % ११-४६

\fourlineindentedshloka
{दृष्टसारमथ रुद्रकार्मुके}
{वीर्यशुल्कमभिनन्द्य मैथिलः}
{राघवाय तनयामयोनिजाम्}
{रूपिणीम् श्रियमिव न्यवेदयत्} % ११-४७

\fourlineindentedshloka
{मैथिलः सपदि सत्यसङ्गरो}
{राघवाय तनयामयोनिजाम्}
{सन्निधौ द्युतिमतस्तपोनिधे}
{रग्निसाक्षिक इवातिसृष्टवान्} % ११-४८

\fourlineindentedshloka
{प्राहिणोच्च महितम् महाद्युतिः}
{कोसलाधिपतये पुरोधसम्}
{भृत्यभावि दुहितुः परिग्रहात्}
{दिश्यताम् कुलमिदम् निमेरिति} % ११-४९

\fourlineindentedshloka
{अन्वियेष सदृशीम् स च स्नुषाम्}
{प्राप चैनमनुकूलवाग्द्विजः}
{सद्य एव सुकृताम् हि पच्यते}
{कल्पवृक्षफलधर्मि काङ्क्षितम्} % ११-५०

\fourlineindentedshloka
{तस्य कल्पितपुरस्क्रियाविधेः}
{शुश्रुवान्वचनमग्रजन्मनः}
{उच्चचाल बलभित्सखो वशी}
{सैन्यरेणुमुषितार्कदीधितिः} % ११-५१

\fourlineindentedshloka
{आससाद मिथिलाम् स वेष्टयन्}
{पीडितोपवनपादपाम् बलैः}
{प्रीतिरोधमसहिष्ट सा पुरी}
{स्त्रीव कान्तपरिभोगमायतम्} % ११-५२

\fourlineindentedshloka
{तौ समेत्य समये स्थितावुभौ}
{भूपती वरुणवासवोपमौ}
{कन्यकातनयकौतुकक्रियाम्}
{स्वप्रभावसदृशीम् वितेनतुः} % ११-५३

\fourlineindentedshloka
{पार्थिवीमुदवहद्रघूद्वहो}
{लक्ष्मणस्तदनुजामथोर्मिलाम्}
{यौ तयोरवरजौ वरौजसौ}
{तौ कुशध्वजसुते सुमध्यमे} % ११-५४

\fourlineindentedshloka
{ते चतुर्थसहितास्त्रयो बभुः}
{सूनवो नववधूपरिग्रहाः}
{सामदानविधिभेदनिग्रहाः}
{सिद्धिमन्त इव तस्य भूपतेः} % ११-५५

\fourlineindentedshloka
{ता नराधिपसुता नृपात्मजैः}
{ते च ताभिरगमन्कृतार्थताम्}
{सोऽभवद्वरवधूसमागमः}
{प्रत्ययप्रकृतियोगसन्निभः} % ११-५६

\fourlineindentedshloka
{एवमात्तरतिरात्मसम्भवां}
{तान्निवेश्य चतुरोऽपि तत्र सः}
{अध्वसु त्रिषु विसृष्टमैथिलः}
{स्वाम् पुरीम् दशरथो न्यवर्तत} % ११-५७

\fourlineindentedshloka
{तस्य जातु मरुतः प्रतीपगा}
{वर्त्मसु ध्वजतरुप्रमाथिनः}
{चिक्लिशुर्भृशतया वरूथिनीम्}
{उत्तटा इव नदीरयाः स्थलीम्} % ११-५८

\fourlineindentedshloka
{तस्य जातु मरुतः प्रतीपगा}
{वर्त्मसु ध्वजतरुप्रमाथिनः}
{चिक्लिशुर्भृशतया वरूथिनीम्}
{उत्तटा इव नदीरयाः स्थलीम्} % ११-५८

\fourlineindentedshloka
{लक्ष्यते स्म तदनन्तरम् रविः}
{बद्धभीमपरिवेषमण्डलः}
{वैनतेयशमितस्य भोगिनो}
{भोगवेष्टित इव च्युतो मणिः} % ११-५९

\fourlineindentedshloka
{श्येनपक्षपरिधूसरालकाः}
{सान्ध्यमेघरुधिरार्द्रवाससः}
{अङ्गना इव रजस्वला दिशो}
{नो बभूवुरवलोकनक्षमाः} % ११-६०

\fourlineindentedshloka
{भास्करश्च दिशमध्युवास याम्}
{ताम् श्रिताः प्रतिभयम् ववासिरे}
{क्षत्रशोणितपितृक्रियोचितम्}
{चोदयन्त्य इव भार्गवम् शिवाः} % ११-६१

\fourlineindentedshloka
{तत्प्रतीपपवनादिवैकृतम्}
{प्रेक्ष्य शान्तिमधिकृत्य कृत्यविद्}
{अन्वयुङ्क्त गुरुमीश्वरः क्षितेः}
{स्वन्तमित्यलघयत्स तद्व्यथाम्} % ११-६२

\fourlineindentedshloka
{तेजसः सपदि राशिरुत्थितः}
{प्रादुरास किल वाहिनीमुखे}
{यः प्रमृज्य नयनानि सैनिकैः}
{लक्षणीयपुरुषाकृतिश्चिरात्} % ११-६३

\fourlineindentedshloka
{पित्र्यमंशमुपवीतलक्षणम्}
{मातृकम् च धनुरूर्जितम् दधत्}
{यः ससोम इव घर्मदीधितिः}
{सद्विजिह्व इव चन्दनद्रुमः} % ११-६४

\fourlineindentedshloka
{येन रोषपरुषात्मनः पितुः}
{शासने स्थितिभिदोऽपि तस्थुषा}
{वेपमानजननीशिरश्छिदा}
{प्रागजीयत घृणा ततो मही} % ११-६५

\fourlineindentedshloka
{अक्षबीजवलयेन निर्बभौ}
{दक्षिणश्रवणसंस्थितेन यः}
{क्षत्रियान्तकरणैकविंशतेः}
{व्याजपूर्वगणनामिवोद्वहन्} % ११-६६

\fourlineindentedshloka
{तम् पितुर्वधभवेन मन्युना}
{राजवंशनिधनाय दीक्षितम्}
{बालसूनुरवलोक्य भार्गवम्}
{स्वाम् दशाम् च विषसाद पार्थिवः} % ११-६७

\fourlineindentedshloka
{नाम राम इति तुल्यमात्मजे}
{वर्तमानमहिते च दारुणे}
{हृद्यमस्य भयदायि चाभवद्}
{रत्नजातमिव हारसर्पयोः} % ११-६८

\fourlineindentedshloka
{अर्घ्यमर्घ्यमिति वादिनम् नृपम्}
{सोऽनवेक्ष्य भरताग्रजो यतः}
{क्षत्रकोपदहनार्चिषम् ततः}
{सन्दधे दृशमुदग्रतारकाम्} % ११-६९

\fourlineindentedshloka
{तेन कार्मुकनिषक्तमुष्टिना}
{राघवो विगतभीः पुरोगतः}
{अङ्गुलीविवरचारिणम् शरम्}
{कुर्वता निजगदे युयुत्सुना} % ११-७०

\fourlineindentedshloka
{क्षत्रजातमपकारवैरि मे}
{तन्निहत्य बहुशः शमम् गतः}
{सुप्तसर्प इव दण्डघट्टनात्}
{रोषितोऽस्मि तव विक्रमश्रवात्} % ११-७१

\fourlineindentedshloka
{मैथिलस्य धनुरन्यपार्थिवैः}
{त्वम् किलानमितपूर्वमक्षणोः}
{तन्निशम्य भवता समर्थये}
{वीर्यशृङ्गमिव भग्नमात्मनः} % ११-७२

\fourlineindentedshloka
{अन्यदा जगति राम इत्ययम्}
{शब्द उच्चरित एव मामगात्}
{व्रीडमावहति मे स सम्प्रति}
{व्यस्तवृत्तिरुदयोन्मुखे त्वयि} % ११-७३

\fourlineindentedshloka
{बिभ्रतोऽस्त्रमचलेऽप्यकुण्ठितम्}
{द्वौ रुपू मम मतौ समागसौ}
{धेनुवत्सहरणाच्च हैहय}
{स्त्वम् च कीर्तिमपहर्तुमुद्यतः} % ११-७४

\fourlineindentedshloka
{क्षत्रियान्तकरणोऽपि विक्रमः}
{तेन मामवति नाजिते त्वयि}
{पावकस्य महिमा स गण्यते}
{कक्षवज्जलति सागरेऽपि यः} % ११-७५

\fourlineindentedshloka
{विद्धि चात्तबलमोजसा हरेः}
{ऐश्वरम् धनुरभाजि यत्त्वया}
{खातमूलमनिलो नदीरयैः}
{पातयत्यपि मृदुस्तटद्रुमम्} % ११-७६

\fourlineindentedshloka
{तन्मदीयमिदमायुधम् ज्यया}
{सङ्गमय्य सशरम् विकृष्यताम्}
{तिष्ठतु प्रधनमेवमप्यहम्}
{तुल्यबाहुतरसा जितस्त्वया} % ११-७७

\fourlineindentedshloka
{कातरोऽसि यदि वोद्गतार्चिषा}
{तर्जितः परशुधारया मम}
{ज्यानिघातकठिनाङ्गुलिर्वृथा}
{बध्यतामभययाचनाञ्जलिः} % ११-७८

\fourlineindentedshloka
{एवमुक्तवति भीमदर्शने}
{भार्गवे स्मितविकम्पिताधरः}
{तद्धनुर्ग्रहणमेव राघवः}
{प्रत्यपद्यत समर्थमुत्तरम्} % ११-७९

\fourlineindentedshloka
{पूर्वजन्मधनुषा समागतः}
{सोऽतिमात्रलघुदर्शनोऽभवत्}
{केवलोऽपि सुभगो नवाम्बुदः}
{किम् पुनस्त्रिदशचापलाञ्छितः} % ११-८०

\fourlineindentedshloka
{तेन भूमिनिहितैककोटि}
{तत्कार्मुकम् च बलिनाधिरोपितम्}
{निष्प्रभश्च रिपुरास भूभृताम्}
{धूमशेष इव धूमकेतनः} % ११-८१

\fourlineindentedshloka
{तावुभावपि परस्परस्थितौ}
{वर्धमानपरिहीनतेजसौ}
{पश्यति स्म जनता दिनात्यये}
{पार्वणौ शशिदिवाकराविव} % ११-८२

\fourlineindentedshloka
{तम् कृपामृदुरवेक्ष्य भार्गवम्}
{राघवः स्खलितवीर्यमात्मनि}
{स्वम् च संहितममोघमाशुगम्}
{व्याजहार हरसूनुसन्निभः} % ११-८३

\fourlineindentedshloka
{न प्रहर्तुमलमस्मि निर्दयम्}
{विप्र इत्यभिभत्यपि त्वयि}
{शंस किम् गतिमनेन पत्रिणा}
{हन्मि लोकमुत ते मखार्जितम्} % ११-८४

\fourlineindentedshloka
{प्रत्युवाच तमृषिर्न तत्त्वतः}
{त्वाम् न वेद्मि पुरुषम् पुरातनम्}
{गाम् गतस्य तव धाम वैष्णवम्}
{कोपितो ह्यसि मया दिदृक्षुणा} % ११-८५

\fourlineindentedshloka
{भस्मसात्कृतवतः पितृद्विषः}
{पात्रसाच्च वसुधाम् ससागराम्}
{आहितो जयविपर्ययोऽपि मे}
{श्लाघ्य एव परमेष्ठिना त्वया} % ११-८६

\fourlineindentedshloka
{तत् गतिम् मतिमताम् वरेप्सिताम्}
{पुण्यतीर्थगमनाय रक्ष मे}
{पीडयिष्यति न माम् खिलीकृता}
{स्वर्गपद्धतिः अभोघलोलुपम्} % ११-८७

\fourlineindentedshloka
{प्रत्यपद्यत तथेति राघवः}
{प्राङ्मुखश्च विससर्ज सायकम्}
{भार्गवस्य सुकृतोऽपि सोऽभवत्}
{स्वर्गमार्गपरिघो दुरत्ययः} % ११-८८

\fourlineindentedshloka
{राघवोऽपि चरणौ तपोनिधेः}
{क्षम्यतामिति वदन्समस्पृशत्}
{निर्जितेषु तरसा तरस्विनाम्}
{शत्रुषु प्रणतिरेव कीर्तये} % ११-८९

\fourlineindentedshloka
{राजसत्वमवधूय मातृकम्}
{पित्र्यमस्मि गमितः शमम् यदा}
{नन्वनिन्दितफलो मम त्वया}
{निग्रहोऽप्ययमनुगृहीकृतः} % ११-९०

\fourlineindentedshloka
{साधयाम्यहमविघ्नमस्तु ते}
{देवकार्यमुपपादयिष्यतः}
{ऊचिवानिति वचः सलक्ष्मणम्}
{लक्ष्मणाग्रजमृषितिरोदधे} % ११-९१

\fourlineindentedshloka
{तस्मिन्गते विजयिनम् परिरभ्य रामम्}
{स्नेहादमन्यत पिता पुनरेव जातम्}
{तस्याभवत्क्षणशुचः परितोषलाभः}
{कक्षाग्निलङ्घिततरोरिव वृष्टिपातः} % ११-९२

\fourlineindentedshloka
{अथ पथि गमयित्वा कॢप्तरम्योपकार्ये}
{कतिचिदवनिपालः शर्वरीः शर्वकल्पः}
{पुरमविशदयोध्याम् मैथिलीदर्शनीनाम्}
{कुवलयितगवाक्षाम् लोचनैरङ्गनानाम्} % ११-९३

॥इति श्री-महाकवि-कालिदास-कृत-रघुवंश-महाकाव्ये एकादशः सर्गः॥
