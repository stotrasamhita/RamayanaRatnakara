
\src{अग्निपुराणम्}{सर्गः १०--१५}{}{}
\notes{In 6 cantos, Mahākavi Kālidāsa narrates the story of Rāma, inside the description of the entire lineage of Raghu.}
\textlink{https://sanskritdocuments.org/sites/giirvaani/giirvaani/rv/intro_rv.htm}
\translink{https://sanskritdocuments.org/sites/giirvaani/giirvaani/rv/intro_rv.htm}
\storymeta

\sect{दशमः सर्गः}

\twolineshloka
{पृथिवीम् शासतस्तस्य पाकशासनतेजसः}
{किन्चिदूनमनूनर्द्धेः शरदामयुतम् ययौ} % १०-१

\twolineshloka
{न चोपलेभे पूर्वेषामृणनिर्मोक्षसाधनम्}
{सुताभिधानम् स ज्योतिः सद्यः शोकतमोपहम्} % १०-२

\twolineshloka
{अतिष्ठत्प्रत्ययापेक्षसम्ततिः स चिरम् नृपः}
{प्राङ्मन्थादनभिव्यक्तरत्नोत्पत्तिरिवार्णवः} % १०-३

\twolineshloka
{ऋष्यशृङ्गादयस्तस्य सन्तः सन्तानकाङ्क्षिणः}
{आरेभिरे जितात्मानः पुत्रीयामिष्टिमृत्विजः} % १०-४

\twolineshloka
{तस्मिन्नवसरे देवाः पौलस्त्योपप्लुता हरिम्}
{अभिजग्मुर्निदाघार्ताश्छायावृक्षमिवाध्वगाः} % १०-५

\twolineshloka
{ते च प्रापुरुदन्वन्तम् बुबुधे चादिपूरुषः}
{अव्याक्षेपो भविष्यन्त्याः कार्यसिद्धेर्हि लक्षणम्} % १०-६

\twolineshloka
{भोगिभोगासनासीनम् ददृशुस्तम् दिवौकसः}
{तत्फणामण्डलोदर्चिर्मणिद्योतितविग्रहम्} % १०-७

\twolineshloka
{श्रियः पद्मनिषण्णायाः क्षौमान्तरितमेखले}
{अङ्के निक्षिप्तचरणमास्तीर्णकरपल्लवे} % १०-८

\twolineshloka
{प्रबुद्धपुण्डरीकाक्षम् बालातपनिभाम्शुकम्}
{दिवसम् शारदमिव प्रारम्भसुखदर्शनम्} % १०-९

\twolineshloka
{प्रभानुलिप्तश्रीवत्सम् लक्ष्मीविभ्रमदर्पणम्}
{कौस्तुभाख्यमपाम् सारम् बिभ्राणम् बृहतोरसा} % १०-१०

\twolineshloka
{बहुभिर्विटपाकारैर्दिव्याभरणभूषितैः}
{आविर्भूतमपाम् मध्ये पारिजातमिवापरम्} % १०-११

\twolineshloka
{दैत्यस्त्रीगण्डलेखानाम् मदरागविलोपिभिः}
{हेतिभिश्चेतनावद्भिरुदीरितजयस्वनम्} % १०-१२

\twolineshloka
{मुक्तशेषविरोधेन कुलिशव्रणलक्ष्मणा}
{उपस्थितम् प्राञ्जलिना विनीतेन गरुत्मता} % १०-१३

\twolineshloka
{योगनिद्रान्तविशदैः पावनैरवलोकनैः}
{भृग्वादीननुगृह्णन्तम् सौखशायनिकानृषीन्} % १०-१४

\twolineshloka
{प्रणिपत्य सुरास्तस्मै शमयित्रे सुरद्विषाम्}
{अथैनम् तुष्टुवुः स्तुत्यमवाङ्मनसगोचरम्} % १०-१५

\twolineshloka
{नमो विश्वसृजे पूर्वम् विश्वम् तदनु बिभ्रते}
{अथ विश्वस्य सम्हर्त्रे तुभ्यम् त्रेधास्थितात्मने} % १०-१६

\twolineshloka
{रसान्तराण्येकरसम् यथा दिव्यम् पयोऽश्नुते}
{देशे देशे गुणेष्वेवमवस्थास्त्वमविक्रियः} % १०-१७

\twolineshloka
{अमेयो मितलोकस्त्वमनर्थी प्रार्थनावहः}
{अजितो जिष्णुरत्यन्तमव्यक्तो व्यक्तकारणम्} % १०-१८

\twolineshloka
{हृदयस्थमनासन्नमकामम् त्वाम् तपस्विनम्}
{दयालुमनघस्पृष्टम् पुराणमजरम् विदुः} % १०-१९

\twolineshloka
{सर्वज्ञस्त्वमविज्ञातः सर्वयोनिस्त्वमात्मभूः}
{सर्वप्रभुरनीशस्त्वमेकस्त्वम् सर्वरूपभाक्} % १०-२०

\twolineshloka
{सप्तसामोपगीतम् त्वाम् सप्तार्णवजलेशयम्}
{सप्तार्चिमुखमाचख्युः सप्तलोकैकसम्श्रयम्} % १०-२१

\twolineshloka
{चतुर्वर्गफलम् ज्ञानम् कालावस्थाश्चतुर्युगाः}
{चतुर्वर्णमयो लोकस्त्वत्तः सर्वम् चतुर्मुखात्} % १०-२२

\twolineshloka
{अभ्यासनिगृहीतेन मनसा हृदयाश्रयम्}
{ज्योतिर्मयम् विचिन्वन्ति योगिनस्त्वाम् विमुक्तये} % १०-२३

\twolineshloka
{अजस्य गृह्णतो जन्म निरीहस्य हतद्विषः}
{स्वपतो जागरूकस्य याथार्थ्यम् वेद कस्तव} % १०-२४

\twolineshloka
{शब्दादीन्विषयान्भोक्तुम् चरितुम् दुश्चरम् तपः}
{पर्याप्तोऽसि प्रजाः पातुमौदासीन्येन वर्तितुम्} % १०-२५

\twolineshloka
{बहुधाप्यागमैर्भिन्नाः पन्थानः सिद्धिहेतवः}
{त्वय्येव निपतन्त्योघा जाह्नवीया इवार्णवे} % १०-२६

\twolineshloka
{त्वय्यावेशितचित्तानाम् त्वत्समर्पितकर्मणाम्}
{गतिस्त्वम् वीतरागाणामभूयःसम्निवृत्तये} % १०-२७

\twolineshloka
{प्रत्यक्षोऽप्यपरिच्छेद्यो मह्यादिर्महिमा तव}
{आप्तवागनुमानाभ्याम् साध्यम् त्वाम् प्रति का कथा} % १०-२८

\twolineshloka
{केवलम् स्मरणेनैव पुनासि पुरुषम् यतः}
{अनेन वृत्तयः शेषा निवेदितफलास्त्वयि} % १०-२९

\twolineshloka
{उदधेरिव रत्नानि तेजाम्सीव विवस्वतः}
{स्तुतिभ्यो व्यतिरिच्यन्ते दूराणि चरितानि ते} % १०-३०

\twolineshloka
{अनवाप्तमवाप्तव्यम् न ते किञ्चन विद्यते}
{लोकानुग्रह एवैको हेतुस्ते जन्मकर्मणोः} % १०-३१

\twolineshloka
{महिमानम् यदुत्कीर्त्य तव सम्ह्रियते वचः}
{श्रमेण तदशक्त्या वा न गुणानामियत्तया} % १०-३२

\twolineshloka
{इति प्रसादयामासुस्ते सुरास्तमधोक्षजम्}
{भूतार्थव्याहृतिः सा हि न स्तुतिः परमेष्ठिनः} % १०-३३

\twolineshloka
{तस्मै कुशलसम्प्रश्नव्यञ्जितप्रीतये सुराः}
{भयमप्रलयोद्वेलादाचख्युर्नैरृतोदधेः} % १०-३४

\twolineshloka
{अथ वेलासमासन्नशैलरन्ध्रानुवादिना}
{स्वरेणोवाच भगवान् परिभूतार्णवध्वनिः} % १०-३५

\twolineshloka
{पुराणस्य कवेस्तस्य वर्णस्थानसमीरिता}
{बभूव कृतसम्स्कारा चरितार्थैव भारती} % १०-३६

\twolineshloka
{बभौ सदशनज्योत्स्ना सा विभोर्वदनोद्गता}
{निर्यातशेषा चरणाद्गङ्गेवोर्ध्वप्रवर्तिनी} % १०-३७

\twolineshloka
{जाने वो रक्षसाक्रान्तावनुभावपराक्रमौ}
{अङ्गिनाम् तमसेवोभौ गुणौ प्रथममध्यमौ} % १०-३८

\twolineshloka
{विदितम् तप्यमानम् च तेन मे भुवनत्रयम्}
{अकामोपनतेनेव साधोर्हृदयमेनसा} % १०-३९

\twolineshloka
{कार्येषु चैककार्यत्वादभ्यर्थ्योऽस्मि न वज्रिणा}
{स्वयमेव हि वातोऽग्नेः सारथ्यम् प्रतिपद्यते} % १०-४०

\twolineshloka
{स्वासिधारापरिहृतः कामम् चक्रस्य तेन मे}
{स्थापितो दशमो मूर्धा लभ्याम्श इव रक्षसा} % १०-४१

\twolineshloka
{स्रष्टुर्वरातिसर्गात्तु मया तस्य दुरात्मनः}
{अत्यारूढम् रिपोः सोढम् चन्दनेनेव भोगिनः} % १०-४२

\twolineshloka
{धातारम् तपसा प्रीतम् ययाचे स हि राक्षसः}
{दैवात्सर्गादवध्यत्वम् मर्त्येष्वास्थापराङ्मुखः} % १०-४३

\twolineshloka
{सोऽहम् दाशरथिर्भूत्वा रणभूमेर्बलिक्षमम्}
{करिष्यामि शरैस्तीक्ष्णैस्तच्छिरःकमलोच्चयम्} % १०-४४

\twolineshloka
{अचिराद्यज्वभिर्भागम् कल्पितम् विधिवत्पुनः}
{मायाविभिरनालीढमादास्यध्वे निशाचरैः} % १०-४५

\twolineshloka
{वैमानिकाः पुण्यकृतस्त्यजन्तु मरुताम् पथि}
{पुष्पकालोकसम्क्षोभम् मेघावरणतत्पराः} % १०-४६

\twolineshloka
{मोक्ष्यध्वे स्वर्गबन्दीनाम् वेणीबन्धनदूषितान्}
{शापयन्त्रितपौलस्त्यबलात्कारकचग्रहैः} % १०-४७

\twolineshloka
{रावणावग्रहक्लान्तमिति वागमृतेन सः}
{अभिवृष्य मरुत्सस्यम् कृष्णमेघस्तिरोदधे} % १०-४८

\twolineshloka
{पुरहूतप्रभृतयः सुरकार्योद्यतम् सुराः}
{अम्शैरनुययुर्विष्णुम् पुष्पैर्वायुमिव द्रुमाः} % १०-४९

\twolineshloka
{अथ तस्य विशाम्पत्युरन्ते कामस्य कर्मणः}
{पुरुषः प्रबभूवाग्नेर्विस्मयेन सहर्त्विजाम्} % १०-५०

\twolineshloka
{हेमपात्रगतम् दोर्भ्यामादधानः पयश्चरुम्}
{अनुप्रवेशादाद्यस्य पुम्सस्तेनापि दुर्वहम्} % १०-५१

\twolineshloka
{प्राजापत्योपनीतम् तदन्नम् प्रत्यग्रहीन्नृपः}
{वृषेव पयसाम् सारमाविष्कृतमुदन्वता} % १०-५२

\twolineshloka
{अनेन कथिता राज्ञो गुणास्तस्यान्यदुर्लभाः}
{प्रसूतिम् चकमे तस्मिम्स्त्रैलोक्यप्रभवोऽपि यत्} % १०-५३

\twolineshloka
{स तेजो वैष्णवम् पत्न्योर्विभेजे चरुसम्ज्ञितम्}
{द्यावापृथिव्योः प्रत्यग्रमहर्पतिरिवातपम्} % १०-५४

\twolineshloka
{अर्चिता तस्य कौसल्या प्रिया केकयवम्शजा}
{अतः सम्भाविताम् ताभ्याम् सुमित्रामैच्छदीश्वरः} % १०-५५

\twolineshloka
{ते बहुज्ञस्य चित्तज्ञे पत्नौ पत्युर्महीक्षितः}
{चरोरर्धार्धभागाभ्याम् तामयोजयतामुभे} % १०-५६

\twolineshloka
{सा हि प्रणयवत्यासीत्सपत्न्योरुभयोरपि}
{भ्रमरी वारणस्येव मदनिस्यन्दरेखयोः} % १०-५७

\twolineshloka
{ताभिर्गर्भः प्रजाभूत्यै दध्रे देवाम्शसम्भवः}
{सौरीभिरिव नाडीभिरमृताख्याभिरम्मयः} % १०-५८

\twolineshloka
{सममापन्नसत्त्वास्ता रेजुरापाण्डुरत्विषः}
{अन्तर्गतफलारम्भाः सस्यानामिव सम्पदः} % १०-५९

\twolineshloka
{गुप्तम् ददृशुरात्मानम् सर्वाः स्वप्नेषु वामनैः}
{जलजासिकदाशार्ङ्गचक्रलाञ्छितमूर्तिभिः} % १०-६०

\twolineshloka
{हेमपक्षप्रभाजालम् गगने च वितन्वता}
{उह्यन्ते स्म सुपर्णेन वेगाकृष्टपयोमुचा} % १०-६१

\twolineshloka
{बिभ्रत्या कौस्तुभन्यासम् स्तनान्तरविलम्बितम्}
{पर्युपास्यन्त लक्ष्म्या च पद्मव्यजनहस्तया} % १०-६२

\twolineshloka
{कृताभिषेकैर्दिव्यायाम् त्रिस्रोतसि च सप्तभिः}
{ब्रह्मर्षिभिः परम् ब्रह्म गृणद्भिरुपतस्थिरे} % १०-६३

\twolineshloka
{ताभ्यस्तथाविधान्स्वप्नाञ्छ्रुत्वा प्रीतो हि पार्थिवः}
{मेने परार्ध्यमात्मानम् गुरुत्वेन जगद्गुरोः} % १०-६४

\twolineshloka
{विभक्त आत्मा विभुस्तासामेकः कुक्षिष्वनेकधा}
{उवास प्रतिमाचन्द्रः प्रसन्नानामपामिव} % १०-६५

\twolineshloka
{अथाग्र्यमहिषी राज्ञः प्रसूतिसमये सती}
{पुत्रम् तमोऽपहम् लेभे नक्तम् ज्योतिरिवौषधिः} % १०-६६

\twolineshloka
{राम इत्यभिरामेण वपुषा तस्य चोदितः}
{नामधेयम् गुरुश्चक्रे जगत्प्रथममङ्गलम्} % १०-६७

\twolineshloka
{रघुवम्शप्रदीपेन तेनाप्रतिमतेजसा}
{रक्षागृहगता दीपाः प्रत्यादिष्टा इवाभवन्} % १०-६८

\twolineshloka
{शय्यागतेन रामेण माता शातोदरी बभौ}
{सैकताम्भोजबलिना जाह्नवीव शरत्कृशा} % १०-६९

\twolineshloka
{कैकेय्यास्तनयो जज्ञे भरतो नाम शीलवान्}
{जनयित्रीमलम्चक्रे यः प्रश्रय इव श्रियम्} % १०-७०

\twolineshloka
{सुतौ लक्ष्मणशत्रुघ्नौ सुमित्रा सुषुवे यमौ}
{सम्यगाराधिता विद्या प्रबोधविनयाविव} % १०-७१

\twolineshloka
{निर्दोषमभवत्सर्वमाविष्कृतगुणम् जगत्}
{अन्वगादिव हि स्वर्गो गाम् गतम् पुरुषोत्तमम्} % १०-७२

\twolineshloka
{तस्योदये चतुर्मूर्तेः पौलस्त्यचकितेश्वराः}
{विरजस्कैर्नभस्वद्भिर्दिश उच्छ्वसिता इव} % १०-७३

\twolineshloka
{कृशानुरपधूमत्वात्प्रसन्नत्वात्प्रभाकरः}
{रक्षोविप्रकृतावास्तामपविद्धशुचाविव} % १०-७४

\twolineshloka
{दशाननकिरीटेभ्यस्तत्क्षणम् राक्षसश्रियः}
{मणिव्याजेन पर्यस्ताः पृथिव्यामश्रुबिन्दवः} % १०-७५

\twolineshloka
{पुत्रजन्मप्रवेश्यानाम् तूर्याणाम् तस्य पुत्रिणः}
{आरम्भम् प्रथमम् चक्रुर्देवदुन्दुभयो दिवि} % १०-७६

\twolineshloka
{सन्तानकमयी वृष्टिर्भवने चास्य पेतुषी}
{सन्मङ्गलोपचाराणाम् सैवादिरचनाऽभवत्} % १०-७७

\twolineshloka
{कुमाराः कृतसम्स्कारास्ते धात्रीस्तन्यपायिनः}
{आनन्देनाग्रजेनेव समम् ववृधिरे पितुः} % १०-७८

\twolineshloka
{स्वाभाविकम् विनीतत्वम् तेषाम् विनयकर्मणा}
{मुमूर्च्छ सहजम् तेजो हविषेव हविर्भुजाम्} % १०-७९

\twolineshloka
{परस्परविरुद्धास्ते तद्रगोरनघम् कुलम्}
{अलमुद्योतयामासुर्देवारण्यमिवर्तवः} % १०-८०

\twolineshloka
{समानेऽपि च सौभ्रात्रे यथेभौ रामलक्ष्मणौ}
{तथा भरतशत्रुघ्नौ प्रीत्या द्वन्द्वम् बभूवतुः} % १०-८१

\twolineshloka
{तेषाम् द्वयोर्द्वयोरैक्यम् बिभिदे न कदाचन}
{यथा वायुर्विभावस्वोर्यथा चन्द्रसमुद्रयोः} % १०-८२

\twolineshloka
{ते प्रजानाम् प्रजानाथास्तेजसा प्रश्रयेण च}
{मनो जह्रुर्निदाघान्ते श्यामाभ्रा दिवसा इव} % १०-८३

\twolineshloka
{स चतुर्धा बभौ व्यस्तः प्रसवः पृथिवीपतेः}
{धर्मार्थकाममोक्षाणामवतार इवाङ्गवान्} % १०-८४

\twolineshloka
{गुणैराराधयामासुस्ते गुरुम् गुरुवत्सलाः}
{तमेव चतुरन्तेशम् रत्नैरिव महार्णवाः} % १०-८५

\fourlineindentedshloka
{सुरगज इव दन्तैर्भग्नदैत्यासिधारै}
{र्नय इव पणबन्धव्यक्तयोगैरुपायैः}
{हरिरिव युगदीर्घैर्दोर्भिरम्शैस्तदीयैः}
{पतिरवनिपतीनाम् तैश्चकाशे चतुर्भिः} % १०-८६

॥इति श्री-महाकवि-कालिदास-कृत-रघुवंश-महाकाव्ये दशमः सर्गः॥
