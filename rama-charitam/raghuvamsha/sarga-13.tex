\sect{त्रयोदशः सर्गः}

\fourlineindentedshloka
{अथात्मनः शब्दगुणम्गुणज्ञ्यः}
{पदम्विमानेन विगाहमानः}
{रत्नाकरम्वीक्ष्य मिथः स जायाम्}
{रामाभिधानो हरिरित्युवाच} % १३-१

\fourlineindentedshloka
{वैदेहि पश्यऽऽमलयाद्विभक्तम्}
{मत्सेतुना फेनिलमम्बुराशिम्}
{छायापथेनेव शरत्प्रसन्नम्}
{आकाशमाविष्कृतचारुतारम्} % १३-२

\fourlineindentedshloka
{गुरोर्यियक्षोः कपिलेन मेध्ये}
{रसातलम्सङ्क्रमिते तुरङ्गे}
{तदर्थमुर्वीमवधारयद्भिः}
{पूर्वैः किलायम्परिवर्धितो नः} % १३-३

\fourlineindentedshloka
{गर्भम्दधत्यर्कमरीचयोऽस्मात्}
{विवृद्धिमत्राश्नुवते वसूनि}
{अबिन्धनम्वह्निमसौ बिभर्ति}
{प्रह्लादनम्ज्योतिरजन्यनेन} % १३-४

\fourlineindentedshloka
{ताम्तामवस्थाम्प्रतिपद्यमानम्}
{स्थितम्दश व्याप दिशो महिम्ना}
{विष्णोरिवास्यानवधारणीयम्}
{ईदृक्तया रूपमियत्ताया वा} % १३-५

\fourlineindentedshloka
{नाभिप्ररूढाम्बुरुहासनेन}
{संस्तूयमानः प्रथमेन धात्रा}
{अमुम्युगान्तोचितयोगनिद्रः}
{संहृत्य लोकान्पुरुषोऽधिशेते} % १३-६

\fourlineindentedshloka
{पक्षच्छिदा गोत्रभिदात्तगन्धाः}
{शरण्यमेनम्शतशो महीध्राः}
{नृपा इवोपप्लविनः परेभ्यो}
{धर्मोत्तरम्मध्यममाश्रयन्ते} % १३-७

\fourlineindentedshloka
{रसातलादादिभवेन पुंसा}
{भुवः प्रयुक्तोद्वहनक्रियायाः}
{अस्याच्छमम्भः प्रलयप्रवृद्धम्}
{मुहूर्तवक्त्राभरणम्बभूव} % १३-८

\fourlineindentedshloka
{मुखार्पणेषु प्रकृतिप्रगल्भाः}
{स्वयम्तरङ्गाधरदानदक्षः}
{अनन्यसामान्यकलत्रवृत्तिः}
{पिबत्यसौ पाययते च सिन्धूः} % १३-९

\fourlineindentedshloka
{ससत्वमादाय नदीमुखाम्भः}
{सम्मीलयन्तो विवृताननत्वात्}
{अमी तिरोभिस्तिमयः सरन्ध्रैः}
{ऊध्वम्वितन्वन्ति जलप्रवाहान्} % १३-१०

\fourlineindentedshloka
{मातङ्गनक्रैः सहसोत्पतद्भिर्ः}
{भिन्नान्द्विधा पश्य समुद्रफेनान्}
{कपोलसंसर्पितया य एषाम्}
{व्रजन्ति कर्णक्षणचामरत्वम्} % १३-११

\fourlineindentedshloka
{वेलानिलाय प्रसृता भुजङ्गा}
{वहोर्मिविस्फूर्जथुनिर्विशेषाः}
{सूर्यांशुसम्पर्कविवृद्धरागैः}
{व्यज्यन्त एते मणिभिः फणस्थैः} % १३-१२

\fourlineindentedshloka
{तवाधरस्पर्धिषु विद्रुमेषु}
{पर्यस्तमेतत्सहसोर्मिवेगात्}
{ऊर्ध्वाङ्कुरप्रोतमुखम्कथञ्चित्}
{क्लेशादपक्रामति शङ्खयूथम्} % १३-१३

\fourlineindentedshloka
{प्रवृत्तमात्रेण पयांसि पातुम्}
{आवर्तवेगाद्भ्रमता घनेन}
{आभाति भूयिष्ठमयम्समुद्रः}
{प्रमथ्यमानो गिरिणेव भूयः} % १३-१४

\fourlineindentedshloka
{दूरादयश्चक्रनिभस्य तन्वी}
{तमालतालीवनराजिनीला}
{आभाति वेला लवणाम्बुराशेः}
{धारानिबद्धेव कलङ्करेखा} % १३-१५

\fourlineindentedshloka
{वेलानिलः केतकरेणुभिस्ते}
{सम्भावयत्याननमायताक्षि}
{मामक्षमम्मण्डनकालहानेः}
{वेत्तीव बिम्बाधरबद्धतृष्णम्} % १३-१६

\fourlineindentedshloka
{एते वयम्सैकतभिन्नशुक्ति}
{पर्यस्तमुक्तापटलम्पयोधेः}
{प्राप्ता मुहूर्तेन विमानवेगात्}
{कूलम्फलावर्जितपूगमालम्} % १३-१७

\fourlineindentedshloka
{कुरुष्व तावत्करभोरु पश्चात्}
{मार्गे मृगप्रेक्षिणि दृष्टिपातम्}
{एषा विदूरीभवतः समुद्रात्}
{सकानना निष्पततीव भूमिः} % १३-१८

\fourlineindentedshloka
{क्वचित्पथा सञ्चरते सुराणाम्}
{क्वचिद्घनानाम्पतताम्क्वचिच्च}
{यथाविधो मे मनसोऽभिलाषः}
{प्रवर्तते पश्य तथा विमानम्} % १३-१९

\fourlineindentedshloka
{असौ महेन्द्रद्विपदानगन्धि}
{स्त्रिमार्गगावीचिविमर्दशीतः}
{आकाशवायुर्दिनयौवनोत्थान्}
{आचामति स्वेदलवान्मुखे ते} % १३-२०

\fourlineindentedshloka
{करेण वातायनलम्बितेन}
{स्पृष्टस्त्वया चण्डि कुतूहलिन्या}
{आमुञ्चतीवाभरणम्द्वितीयम्}
{उद्भिन्नविद्युद्वलयो घनस्ते} % १३-२१

\fourlineindentedshloka
{अमी जनस्थानमपोढविघ्नम्}
{मत्वा समारब्धनवोटजानि}
{अध्यासते चीरभृतो यथास्वम्}
{चिरोज्झितान्याश्रममण्डलानि} % १३-२२

\fourlineindentedshloka
{सैषा स्थली यत्र विचिन्वता त्वाम्}
{भ्रष्टम्मया नूपुरमेकमुर्व्याम्}
{अदृष्यत त्वच्चरणारविन्द}
{विश्लेषदुःखादिव बद्धमौनम्} % १३-२३

\fourlineindentedshloka
{त्वम्रक्षसा भीरु यतोऽपनीता}
{तम्मार्गमेताः कृपया लता मे}
{अदर्शयन्वक्तुमशक्नुवत्यः}
{शाखाभिरावर्जितपल्लवाभिः} % १३-२४

\fourlineindentedshloka
{मृग्यश्च दर्भाङ्कुरनिर्व्यपेक्षाः}
{तवागतिज्ञ्यम्समबोधयन्माम्}
{व्यापारयन्त्यो दिशि दक्षिणस्याम्}
{उत्पक्षराजीनि विलोचनानि} % १३-२५

\fourlineindentedshloka
{एतद्गिरेर्माल्यवतः पुरस्तात्}
{आविर्भत्यम्बरलेखि शृङ्गम्}
{नवम्पयो यत्र घनैर्मया च}
{त्वद्विप्रयोगाश्रु समम्विसृष्टम्} % १३-२६

\fourlineindentedshloka
{गन्धश्च धाराहतपल्वलानाम्}
{कादम्बमर्धोद्गतकेसरम्च}
{स्निग्धाश्च केकाः शिखिनाम्बभूवुः}
{यस्मिन्नसह्यानि विना त्वया मे} % १३-२७

\fourlineindentedshloka
{पूर्वानुभूतम्स्मरता च यत्र}
{कम्पोत्तरम्भीरु तवोपगूढम्}
{गुहाविसारीण्यतिवाहितानि}
{मया कथञ्चिद्घनगर्जितानि} % १३-२८

\fourlineindentedshloka
{आसारसिक्तक्षितिबाष्पयोगात्}
{मामक्षिणोद्यत्र विभिन्नकोशैः}
{विडम्ब्यमाना नवकन्दलैस्ते}
{विवाहधूमारुणलोचनश्रीः} % १३-२९

\fourlineindentedshloka
{उपान्तवानीरवनोपगूढानि}
{आलक्ष्यपारिप्लवसारसानि}
{दूरावतीर्णा पिबतीव खेदात्}
{अमूनि पम्पासलिलानि दृष्टिः} % १३-३०

\fourlineindentedshloka
{अत्रावियुक्तानि रथङ्गनाम्ना}
{मन्योन्यदत्तोत्पलकेसराणि}
{द्वन्द्वानि दूरान्तरवर्तिना ते}
{मया प्रिये सस्मितमीक्षितानि} % १३-३१

\fourlineindentedshloka
{इमाम्तटाशोकलताम्च तन्वीम्}
{स्तनाभिरामस्तबकाभिनम्राम्}
{त्वत्प्राप्तिबुद्ध्या परिरब्धुकामः}
{सौमित्रिणा साश्रुरहम्निषिद्धः} % १३-३२

\fourlineindentedshloka
{अमूर्विमानान्तरलम्बिनीनाम्}
{श्रुत्वा स्वनम्काञ्चनकिङ्किणीनाम्}
{प्रत्युद्व्रजन्तीव खमुत्पतन्त्यो}
{गोदावरीसारसपङ्क्तयस्त्वाम्} % १३-३३

\fourlineindentedshloka
{एषा त्वया पेशलमध्ययापि}
{घटाम्बुसंवर्धितबालचूता}
{आनन्दयत्युन्मुखकृष्णसारा}
{दृष्टा चिरात्पञ्चवटी मनो मे} % १३-३४

\fourlineindentedshloka
{अत्रानुगोदम्मृगयानिवृत्तः}
{तरङ्गवातेन विनीतखेदः}
{रहस्त्वदुत्सङ्गनिषण्णमूर्धा}
{स्मरामि वानीरगृहेषु सुप्तः} % १३-३५

\fourlineindentedshloka
{भ्रूभेदमात्रेण पदान्मघोनः}
{प्रभ्रंशयाम् यो नहुषम्चकार}
{तस्याविलाम्भःपरिशुद्धिहेतोः}
{भौमो मुनेः स्थानपरिग्रहोऽयम्} % १३-३६

\fourlineindentedshloka
{त्रेताग्निधूमाग्रमनिन्द्यकीर्तेः}
{तस्येदमाक्रान्तविमानमार्गम्}
{घ्रात्वा हविर्गन्धि रजोविमुक्तः}
{समश्नुते मे लघिमानमात्मा} % १३-३७

\fourlineindentedshloka
{एतन्मुमुनेर्मानिनि शातकर्णेः}
{पञ्चाप्सरो नाम विहारवारि}
{आभाति पर्यन्तवनम्विदूरात्}
{मेघान्तरालक्ष्यमिवेन्दुबिम्बम्} % १३-३८

\fourlineindentedshloka
{पुरा स दर्भाङ्कुरमात्रवृत्तिः}
{चरन्मृगैः सार्धमृषिर्मघोना}
{समाधिभीतेन किलोपनीतः}
{पञ्चाप्सरोयौवनकूटबन्धम्} % १३-३९

\fourlineindentedshloka
{तस्यायमन्तर्हितसौधभाजः}
{प्रसक्तसङ्गीतमृदङ्गघोषः}
{वियद्गतः पुष्पकचन्द्रशालाः}
{क्षणम्प्रतिश्रुन्मुखराः करोति} % १३-४०

\fourlineindentedshloka
{हविर्भुजामेधवताम्चतुर्णाम्}
{मध्ये ललाटन्तपसप्तसप्तिः}
{असौ तपस्यत्यपरस्तपस्वी}
{नाम्ना सुतीक्ष्णश्चरितेन दान्तः} % १३-४१

\fourlineindentedshloka
{अमुम्सहासप्रहितेक्षणानि}
{व्याजार्धसन्दर्शितमेखलानि}
{नालम्विकर्तुम्जनितेन्द्रशङ्कम्}
{सुराङ्गनाविभ्रमचेष्टितानि} % १३-४२

\fourlineindentedshloka
{एषोऽक्षमालावलयम्मृगाणाम्}
{कण्डूयितारम्कुशसूचिलावम्}
{सभाजने मे भुजमूर्ध्वबाहुः}
{सव्येतरम्प्राध्वमितः प्रयुङ्क्ते} % १३-४३

\fourlineindentedshloka
{वाचंयमत्वात्प्रणतिम्ममैष}
{कम्पेन किञ्चित्प्रतिगृह्य मूर्ध्नः}
{दृष्टिम्विमानव्यवधानमुक्ताम्}
{पुनः सहस्रार्चिषि सन्निधत्ते} % १३-४४

\fourlineindentedshloka
{अदः शरण्यम्शरभङ्गनाम्नः}
{तपोवनम्पावनमाहिताग्नेः}
{चिराय सन्तर्प्य समिद्भिरग्निम्}
{यो मन्त्रपूताम्तनुमप्यहौषीत्} % १३-४५

\fourlineindentedshloka
{छायाविनीताध्वपरिश्रमेषु}
{भूयिष्ठसम्भाव्यफलेष्वमीषु}
{तस्यातिथीनामधुनासपर्या}
{स्थिता सुपुत्रेष्विव पादपेषु} % १३-४६

\fourlineindentedshloka
{धारास्वनोद्गारिदरीमुखोऽसौ}
{शृङ्गाङ्गलग्नाम्बुजवप्रपङ्कः}
{बध्नाति मे बन्धुरगात्रि चक्षुः}
{दृप्तः ककुद्मानिव चित्रकूटः} % १३-४७

\fourlineindentedshloka
{एषा प्रसन्नस्तिमितप्रवाहा}
{सरिद्विदूरान्तरभावतन्वी}
{मन्दाकिनी भाति नगोपकण्ठे}
{मुक्तावली कण्ठगतेव भूमेः} % १३-४८

\fourlineindentedshloka
{अयम्सुजातोऽनुगिरम्तमालः}
{प्रवालमादाय सुगन्धि यस्य}
{यवाङ्कुरापाण्डुकपोलशोभी}
{मयावतंसः परिकल्पितस्ते} % १३-४९

\fourlineindentedshloka
{अनिग्रहत्रासविनीतसत्त्व}
{मपुष्पलिङ्गात्फलबन्धिवृक्षम्}
{वनम्तपःसाधनमेतदत्रेः}
{आविष्कृतोदग्रतरप्रभावम्} % १३-५०

\fourlineindentedshloka
{अत्राभिषेकाय तपोधनानाम्}
{सप्तर्षिहस्तोद्धृतहेमपद्माम्}
{प्रवर्तयामास किलानसूया}
{त्रिस्रोतसम्त्र्यम्बकमौलिमालाम्} % १३-५१

\fourlineindentedshloka
{वीरासनैर्ध्यानजुषामृषीणाम्}
{अमी समध्यासितवेदिमध्याः}
{निवातनिष्कम्पतया विभान्ति}
{योगाधिरूढा इव शाखिनोऽपि} % १३-५२

\fourlineindentedshloka
{त्वया पुरस्तादुपयाचितो यः}
{सोऽयम्वटः श्याम इति प्रतीतः}
{राशिर्मणीनामिव गारुडानाम्}
{सपद्मरागः फलितो विभाति} % १३-५३

\fourlineindentedshloka
{क्वचित्प्रभालेपिभिरिन्द्रनीलैः}
{मुक्तामयी यष्टिरिवानुविद्धा}
{अन्यत्र माला सितपङ्कजानाम्}
{इन्दीवरैरुत्खचितान्तरेव} % १३-५४

\fourlineindentedshloka
{क्वचित्खगानाम्प्रियमानसानाम्}
{कादम्बसंसर्गवतीव पङ्क्तिः}
{अन्यत्र कालागुरुदतापत्रा}
{भक्तिर्भुवश्चन्दनकल्पितेव} % १३-५५

\fourlineindentedshloka
{क्वचित्प्रभा चान्द्रमसी}
{तमोभिश्छायाविलीनैः शबलीकृतेव}
{अन्यत्र शुभ्रा शरदभ्रलेखा}
{रन्ध्रेष्विवालक्ष्यनभःप्रदेशा} % १३-५६

\fourlineindentedshloka
{क्वचिच्च कृष्णोरगभूषणेव}
{भस्माङ्गरागा तनुरीश्वरस्य}
{पश्यानवद्याङ्गि विभाति गङ्गा}
{भिन्नप्रवाहा यमुनातरङ्गैः} % १३-५७

\fourlineindentedshloka
{समुद्रपत्योर्जलसन्निपाते}
{पूतात्मनामत्र किलाभिषेकात्}
{तत्त्वावबोधेन विनापि भूयः}
{तनुत्यजाम्नास्ति शरीरबन्धः} % १३-५८

\fourlineindentedshloka
{पुरम्निषादाधिपतेरिदम्तत्}
{यस्मिन्मया मौलिमणिम्विहाय}
{जटासु बद्धास्वरुदत्सुमन्त्रः}
{कैकेयि कामाः फलितास्तवेति} % १३-५९

\twolineshloka
{पयोधरैः पुण्यजनाङ्गनानाम्निर्विष्टहेमाम्बुजरेणु यस्याः}
{ब्राह्मम्सरः कारणमाप्तवाचो बुद्धेरिवाव्यक्तमुदाहरन्ति} % १३-६०

\fourlineindentedshloka
{जलानि या तीरनिखातयूपा}
{वहत्ययोध्यामनु राजधानीम्}
{तुरङ्गमेधावभृथावतीर्णैः}
{इक्ष्वाकुभिः पुण्यतरीकृतानि} % १३-६१

\fourlineindentedshloka
{याम्सैकतोत्सङ्गसुखोचितानाम्}
{प्राज्यैः पयोभिः परिवर्धितानाम्}
{सामान्यधात्रीमिव मानसम्मे}
{सम्भावयत्युत्तरकोसलानाम्} % १३-६२

\fourlineindentedshloka
{सेयम्मदीया जननीव तेन}
{मान्येन राज्ञ्या सरयूर्वियुक्ता}
{दूरे वसन्तम्शिशिरानिलैर्माम्}
{तरङ्गहस्तैरुपगूहतीव} % १३-६३

\fourlineindentedshloka
{विरक्तसन्ध्याकपिशम्परस्तात्}
{यतो रजः पार्थिवमुज्जिहीते}
{शङ्के हनूमत्कथितप्रवृत्तिः}
{प्रत्युद्गतो माम्भरतः ससैन्यः} % १३-६४

\fourlineindentedshloka
{अद्धा श्रियम्पालितसङ्गराय}
{प्रत्यर्पयिष्यत्यनघाम्स साधुः}
{हत्वा निवृत्ताय मृधे खरादीन्}
{संरक्षिताम्त्वामिव लक्ष्मणो मे} % १३-६५

\fourlineindentedshloka
{असौ पुरस्कृत्य गुरुम्पदातिः}
{पश्चादवस्थापितवाहिनीकः}
{वृद्धैरमात्यैः सह चीरवासा}
{मामर्घ्यपाणिर्भरतोऽप्युपैति} % १३-६६

\fourlineindentedshloka
{पित्रा विसृष्टाम्मदपेक्षया यः}
{श्रियम्युवाप्यङ्कगतामभोक्ता}
{इयन्ति वर्षाणि तया सहोग्रम्}
{अभ्यस्यतीव व्रतमासिधारम्} % १३-६७

\fourlineindentedshloka
{एतावदुक्तवति दाशरथौ तदीया}
{मिच्छाम्विमानमधिदेवतया विदित्वा}
{ज्योतिष्पथादवततार सविस्मयाभि}
{रुद्वीक्षितम्प्रकृतिभिर्भरतानुगाभिः} % १३-६८

\fourlineindentedshloka
{तस्मात्पुरःसरबिभीषणदर्शनेन}
{सेवाविचक्षणहरीश्वरदत्तहस्तः}
{यानादवातरददूरमहीतलेन}
{मार्गेण भङ्गिरचितस्फटिकेन रामः} % १३-६९

\fourlineindentedshloka
{इक्ष्वाकुवंशगुरवे प्रयतः प्रणम्य}
{स भ्रातरम्भरतमर्घ्यपरिग्रहान्ते}
{पर्यश्रुरस्वजत मूर्धनि चोपजघ्रौ}
{तद्भक्त्यपोढपितृराज्यमहाभिषेके} % १३-७०

\fourlineindentedshloka
{श्मश्रुप्रवृद्धिजनिताननविक्रियांश्च}
{प्लक्षान्प्ररोहजटिलानिव मन्त्रिवृद्धान्}
{अन्वग्रहीत्प्रणमतः शुभदृष्टिपातै}
{र्वातानुयोगमधुराक्षरया च वाचा} % १३-७१

\fourlineindentedshloka
{दुर्जातबन्धुरयमृक्षहरीश्वरो मे}
{पौलस्त्य एष समरेषु पुरःप्रहर्ता}
{इत्यादृतेन कथितौ रघुनन्दनेन}
{व्युत्क्रम्य लक्ष्मणमुभौ भरतो ववन्दे} % १३-७२

\fourlineindentedshloka
{सौमित्रिणा तदनु संससृजे स चैन}
{मुत्थाप्य नम्रशिरसम्भृशमालिलिङ्ग}
{रूढेन्द्रजित्प्रहरणव्रणकर्कशेन}
{क्लिश्यन्निवास्य भुजमध्यमुरस्थलेन} % १३-७३

\fourlineindentedshloka
{रामाज्ञ्यया हरिचमूपतयस्तदानीम्}
{कृत्वा मनुष्यवपुरारुरुहुर्गजेन्द्रान्}
{तेषु क्षरत्सु बहुधा मदवारिधाराः}
{शैलाधिरोहणसुखान्युपलेभिरेते} % १३-७४

\fourlineindentedshloka
{सानुप्लवः प्रभुरपि क्षणदाचराणाम्}
{भेजे रथान्दशरथप्रभवानुशिष्टः}
{मायाविकल्परचितैरपि ये तदीयै}
{र्न स्यन्दनैस्तुलितकृत्रिमभक्तिशोभाः} % १३-७५

\fourlineindentedshloka
{भूयस्ततो रघुपतिर्विलसत्पताक}
{मध्यास्त कामगति सावरजो विमानम्}
{दोषातनम्बुधबृहस्पतियोगदृश्य}
{स्तारापतिस्तरलविद्युदिवाभ्रवृन्दम्} % १३-७६

\fourlineindentedshloka
{तत्रेश्वरेण जगताम्प्रलयादिवोर्वीं}
{वर्षात्ययेन रुचमभ्रघनादिवेन्दोः}
{रामेण मैथिलसुताम्दशकण्ठकृच्छ्रा}
{त्प्रत्युद्धृताम्धृतिमतीम्भरतो ववन्दे} % १३-७७

\fourlineindentedshloka
{लङ्केश्वरप्रणतिभङ्गदृढव्रतम्त}
{द्वन्द्यम्युगम्चरणयोर्जनकात्मजायाः}
{जेष्ठानुवृत्तिजटिलम्च शिरोऽस्य साधो}
{रन्योन्यपावनमभूदुभयम्समेत्य} % १३-७८

\fourlineindentedshloka
{क्रोशार्धम्प्रकृतिपुरःसरेण गत्वा}
{काकुत्स्थः स्तिमितजवेन पुष्पकेण}
{शत्रुघ्नप्रतिविहितोपकार्यमार्यः}
{साकेतोपवनमुदारमध्युवास} % १३-७९

॥इति श्री-महाकवि-कालिदास-कृत-रघुवंश-महाकाव्ये त्रयोदशः सर्गः॥
