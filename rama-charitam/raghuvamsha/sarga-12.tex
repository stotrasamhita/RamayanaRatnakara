\sect{द्वादशः सर्गः}

\twolineshloka
{निर्विष्टविषयस्नेहः स दशान्तमुपेयिवान्}
{आसीदासन्ननिर्वाणः प्रदीपार्चिरिवोषसि} % १२-१

\twolineshloka
{तम् कर्णमूलमागत्य रामे श्रीर्न्यस्यतामिति}
{कैकेयीशङ्कयेवाह पलितच्छद्मना जरा} % १२-२

\twolineshloka
{सा पौरान्पौरकान्तस्य रामस्याभ्युदयश्रुतिः}
{प्रत्येकम् ह्रादयाञ्चक्रे कुल्येवोद्यानपादपान्} % १२-३

\twolineshloka
{तस्याभिषेकसम्भारम् कल्पितम् क्रूरनिश्चया}
{दूषयामास कैकेयी शोकोष्णैः पार्थिवाश्रुभिः} % १२-४

\twolineshloka
{सा किलाश्वासिता चण्डी भर्त्रा तत् संश्रुतौ वरौ}
{उद्ववामेन्द्रसिक्ता भूर्बिलमग्नाविवोरगौ} % १२-५

\twolineshloka
{तयोश्चतुर्दशैकेन रामम् प्राव्राजयत्समाः}
{द्वितीयेन सुतस्यैच्छद्वैधव्यैकफलाम् श्रियम्} % १२-६

\twolineshloka
{पित्रा दत्तम् रुदन्रामः प्राङ्महीम् प्रत्यपद्यत}
{पश्चाद्वनाय गच्छेति तदाज्ञाम् मुदितोऽग्रहीत्} % १२-७

\twolineshloka
{दधतो मङ्गलक्षौमे वसानस्य च वल्कले}
{ददृशुर्विस्मितास्तस्य मुखरागम् समम् जनाः} % १२-८

\twolineshloka
{स सीतालक्ष्मणसखः सत्याद्गुरुमलोपयन्}
{विवेश दण्डकारण्यम् प्रत्येकम् च सताम् मनः} % १२-९

\twolineshloka
{राजापि तद्वियोगार्तः स्मृत्वा शापम् स्वकर्मजम्}
{शरीरत्यागमात्रेण शुद्धिलाभमन्यत} % १२-१०

\twolineshloka
{विप्रोषितकुमारम् तद्राज्यमस्तमितेश्वरम्}
{रन्ध्रान्वेषणदक्षाणाम् द्विषामामिषतताम् ययौ} % १२-११

\twolineshloka
{अथानाथाः प्रकृतयो मातृबन्धुनिवासिनम्}
{मौलैरानाययामासुर्भरतम् स्तम्भिताश्रुभिः} % १२-१२

\twolineshloka
{श्रुत्वा तथाविधम् मृत्युम् कैकेयीतनयः पितुः}
{मातुर्न केवलम् स्वस्याः श्रियोऽप्यासीत्पराङ्मुखः} % १२-१३

\twolineshloka
{ससैन्यश्चान्वगाद्रामम् दर्शितानाश्रमालयैः}
{तस्य पश्यन्ससौमित्रेरुदश्रुर्वसतिद्रुमान्} % १२-१४

\twolineshloka
{चित्रकूटवनस्थम् च कथितस्वर्गतिर्गुरोः}
{लक्ष्म्या निमन्त्रयाञ्चक्रे तमनुच्छिष्टसम्पदा} % १२-१५

\twolineshloka
{स हि प्रथमजे तस्मिन्नकृतश्रीपरिग्रहे}
{परिवेत्तारमात्मानम् मेने स्वीकरणाद्भुवः} % १२-१६

\twolineshloka
{तमशक्यमपाक्रष्टुम् निदेशात्स्वर्गिणः पितुः}
{ययाचे पादुके पश्चात्कर्तुम् राज्याधिदेवते} % १२-१७

\twolineshloka
{स विसृष्टस्तथेत्युक्त्वा भ्रात्रा नैवाविशत्पुरीम्}
{नन्दिग्रामगतस्तस्य राज्यम् न्यासमिवाभुनक्} % १२-१८

\twolineshloka
{दृढभक्तिरिति ज्येष्ठे राज्यतृष्णापराङ्मुखः}
{मातुः पापस्य भरतः प्रायश्चित्तमिवाकरोत्} % १२-१९

\twolineshloka
{रामोऽपि सह वैदेह्या वने वन्येन वर्तयन्}
{चचार सानुजः शान्तो वृद्धेक्ष्वाकुव्रतम् युवा} % १२-२०

\twolineshloka
{प्रभावस्तम्भितच्छायमाश्रितः स वनस्पतिम्}
{कदाचिदङ्के सीतायाः शिश्ये किञ्चिदिव श्रमात्} % १२-२१

\twolineshloka
{ऐन्द्रिः किल नखैस्तस्या विददार स्तनौ द्विजः}
{प्रियोपभोगचिह्नेषु पौरोभाग्यमिवाचरन्} % १२-२२

\twolineshloka
{तस्मिन्नस्थदिषीकास्त्रम् रामो रामावबोधितः}
{आत्मानम् मुमुचे तस्मादेकनेत्रव्ययेन सः} % १२-२३

\twolineshloka
{रामस्त्वासन्नदेशत्वाद्भरतागमनम् पुनः}
{आशङ्क्योत्सुकसारङ्गाम् चित्रकूटस्थलीम् जहौ} % १२-२४

\twolineshloka
{प्रययावातिथेयेषु वसन्नृषिकुलेषु सः}
{दक्षिणाम् दिशमृक्षेषु वार्षिकेष्विव भास्करः} % १२-२५

\twolineshloka
{बभौ तमनुगच्छन्ती विदेहाधिपतेः सुता}
{प्रतिषिद्धापि कैकेय्या लक्ष्मीरिव गुणोन्मुखी} % १२-२६

\twolineshloka
{अनसूयातिसृष्टेन पुण्यगन्धेन काननम्}
{सा चकाराङ्गरागेण पुष्पोच्चलितषट्पदम्} % १२-२७

\twolineshloka
{सन्ध्याभ्रकपिशस्तस्य विराधो नाम राक्षसः}
{अतिष्ठन्मार्गमावृत्य रामस्येन्दोरिव ग्रहः} % १२-२८

\twolineshloka
{स जहार तयोर्मध्ये मैथिलीम् लोकशोषणः}
{नभोनभस्ययोर्वृष्टिमवग्रह इवान्तरे} % १२-२९

\twolineshloka
{तम् विनिष्पिष्य काकुत्स्थौ पुरा दूषयति स्थलीम्}
{गन्धेनाशुचिना चेति वसुधायाम् निचख्नतुः} % १२-३०

\twolineshloka
{पञ्चवट्याम् ततो रामः शासनात्कुम्भजन्मनः}
{अनपोढस्थितिस्तस्थौ विन्ध्याद्रिः प्रकृताविव} % १२-३१

\twolineshloka
{रावणावरजा तत्र राघवम् मदनातुरा}
{अभिपेदे निदाघार्ता व्यालीव मलयद्रुमम्} % १२-३२

\twolineshloka
{सा सीतासन्निधावेव तम् वव्रे कथितान्वया}
{अत्यारूढो हि नारीणामकालज्ञो मनोभवः} % १२-३३

\twolineshloka
{कलत्रवानहम् बाले कनीयांसम् भजस्व मे}
{इति रामो वृषस्यन्तीम् वृषस्कन्धः शशास ताम्} % १२-३४

\twolineshloka
{ज्येष्ठाभिगमनात्पूर्वम् तेनाप्यनभिनन्दिताम्}
{साऽभूद्रामाश्रया भूयो नदीवोभयकूलभाक्} % १२-३५

\twolineshloka
{संरम्भम् मैथिलीहासः क्षणसौम्याम् निनाय ताम्}
{निवातस्तिमिताम् वेलाम् चन्द्रोदय इवोदधेः} % १२-३६

\twolineshloka
{फलमस्योपहासस्य सद्यः प्राप्स्यसि पश्य माम्}
{मृग्याः परिभवो व्याघ्र्यामित्यवेहि त्वया कृतम्} % १२-३७

\twolineshloka
{इत्युक्त्वा मैथिलीम् भर्तुरङ्के निविशतीम् भयात्}
{रूपम् शूर्पणखा नाम्नः सदृशम् प्रत्यपद्यत} % १२-३८

\twolineshloka
{लक्ष्मणः प्रथमम् श्रुत्वा कोकिलामञ्जुवादिनीम्}
{शिवाघोरस्वनाम् पश्चाद्बुबुधे विकृतेति ताम्} % १२-३९

\twolineshloka
{पर्णशालामथ क्षिप्रम् विकृष्टासिः प्रविश्य सः}
{वैरूप्यपौनरुक्त्येन भीषणाम् तामयोजयत्} % १२-४०

\twolineshloka
{सा वक्रनखधारिण्या वेणुकर्कशपर्वया}
{अङ्कुशाकारयाङ्गुल्या तावतर्जयदम्बरे} % १२-४१

\twolineshloka
{प्राप्य चाशु जनस्थानम् खरादिभ्यस्तथाविधम्}
{रामोपक्रममाचख्यौ रक्षःपरिभवम् नवम्} % १२-४२

\twolineshloka
{मुखावयवलूनाम् ताम् नैरृता यत्पुरो दधुः}
{रामाभियायिनाम् तेषाम् तदेवाभूदमङ्गलम्} % १२-४३

\twolineshloka
{उदायुधानात्पततस्तान्दृप्तान्प्रेक्ष्य राघवः}
{निदधे विजयाशंसाम् चापे सीताम् च लक्ष्मणे} % १२-४४

\twolineshloka
{एको दाशरथिः कामम् यातुधानाः सहस्रशः}
{ते तु यावन्त एवाजौ तावांश्च ददृशे स तैः} % १२-४५

\twolineshloka
{असज्जनेन काकुत्स्थः प्रयुक्तमथ दूषणम्}
{न चक्षमे शुभाचारः स दूषणमिवात्मनः} % १२-४६

\twolineshloka
{तम् शरैः प्रतिजग्राह खरत्रिशिरसौ च सः}
{क्रमशस्ते पुनस्तस्य चापात्सममिवोद्ययुः} % १२-४७

\twolineshloka
{तैस्त्रयाणाम् शितैर्बाणैर्यथापूर्वविशुद्धिभिः}
{आयुर्देहातिगैः पीतम् रुधिरम् तु पतत्रिभिः} % १२-४८

\twolineshloka
{तस्मिन्रामशरोत्कृत्ते बले महति रक्षसाम्}
{उत्थितम् ददृशेऽन्न्यच्च कबन्धेभ्यो न किञ्चन} % १२-४९

\twolineshloka
{सा बाणवर्षिणम् रामम् योधयित्वा सुरद्विषाम्}
{अप्रबोधाय सुष्वाप गृध्रच्छाये वरूथिनी} % १२-५०

\twolineshloka
{राघवास्त्रविदीर्णानाम् रावणम् प्रति रक्षसाम्}
{तेषाम् शूर्पणखैवैका दुष्प्रवृत्तिहराऽभवत्} % १२-५१

\twolineshloka
{निग्रहात्स्वसुराप्तानाम् वधाच्च धनदानुजः}
{रामेण निहितम् मेने पदम् दशसु मूर्धसु} % १२-५२

\twolineshloka
{रक्षसा मृगरूपेण वञ्चयित्वा स राघवौ}
{जहार सीताम् पक्षीन्द्रप्रयासक्षणविघ्नितः} % १२-५३

\twolineshloka
{तौ सीतावेषिणौ गृध्रम् लूनपक्षमपश्यताम्}
{प्राणैर्दशरथप्रीतेरनृणम् कण्ठवर्तिभिः} % १२-५४

\twolineshloka
{स रावणहृताम् ताभ्याम् वचसाचष्ट मैथिलीम्}
{आत्मनः सुमहत्कर्म व्रणैरावेद्य संस्थितः} % १२-५५

\twolineshloka
{तयोस्तस्मिन्नवीभूतपितृव्यापतिशोकयोः}
{पितरीवाग्निसंस्कारात्परा ववृतिरे क्रियाः} % १२-५६

\twolineshloka
{वधनिर्धूतशापस्य कबन्धस्योपदेशतः}
{मुमूर्च्छ सख्यम् रामस्य समानव्यसने हरौ} % १२-५७

\twolineshloka
{स हत्वा वालिनम् वीरस्तत्पदे चिरकाङ्क्षिते}
{धातोः स्थान इवादेशम् सुग्रीवम् सन्न्यवेशयत्} % १२-५८

\twolineshloka
{इतस्ततश्च वैदेहीमन्वेष्टुम् भर्तृचोदिताः}
{कपयश्चेरुरार्तस्य रामस्येव मनोरथाः} % १२-५९

\twolineshloka
{प्रवृत्तावुपलब्धायाम् तस्याः सम्पातिदर्शनात्}
{मारुतिः सागरम् तीर्णः संसारमिव निर्ममः} % १२-६०

\twolineshloka
{दृष्टा विचिन्वता तेन लङ्कायाम् राक्षसीवृता}
{जानकी विषवल्लीभिः परितेव महौषधिः} % १२-६१

\twolineshloka
{तस्यै भर्तुरभिज्ञ्यानमङ्गुलीयम् ददौ कपिः}
{प्रत्युद्गतमिवानुष्णैस्तदानन्दाश्रुबिन्दुभिः} % १२-६२

\twolineshloka
{निर्वाप्य प्रियसन्देशैः सीतामक्षवधोद्धतः}
{स ददाह पुरीम् लङ्काम् क्षणसोढारिनिग्रहः} % १२-६३

\twolineshloka
{प्रत्यभिज्ञानरत्नम् च रामायादर्शयत्कृती}
{हृदयम् स्वयमायातम् वैदेह्या इव मूर्तिमत्} % १२-६४

\twolineshloka
{स प्राप हृदयन्यस्तमणिस्पर्शनिमीलितः}
{अपयोधरसंसर्गाम् प्रियालिङ्गननिर्वृतिम्} % १२-६५

\twolineshloka
{श्रुत्वा रामः प्रियोदन्तम् मेने तत्सङ्गमोत्सुकः}
{महार्णवपरिक्षेपम् लङ्कायाः परिखालघुम्} % १२-६६

\twolineshloka
{स प्रतस्थेऽरिनाशाय हरिसैन्यैरनुद्रुतः}
{न केवलम् भुवः पृष्ठे व्योम्नि सम्बाधवर्तिभिः} % १२-६७

\twolineshloka
{निविष्टमुदधेः कूले तम् प्रपाद बिभीषणः}
{स्नेहाद्राक्षसलक्ष्म्येव बुद्धिमाविश्य चोदितः} % १२-६८

\twolineshloka
{तस्मै निशाचरैश्वर्यम् प्रतिशुश्राव राघवः}
{काले खलु समालब्धाः फलम् बध्नन्ति नीतयः} % १२-६९

\twolineshloka
{स सेतुम् बन्धयामास प्लवगैर्लवणाम्भसि}
{रसातलादिवोन्मग्नम् शेषम् स्वप्नाय शार्ङ्गिणः} % १२-७०

\twolineshloka
{तेनोत्तीर्य पथा लङ्काम् रोधयामास पिङ्गलैः}
{द्वितीयम् हेमप्राकारम् कुर्वद्भिरिव वानरैः} % १२-७१

\twolineshloka
{रणः प्रववृते तत्र भीमः प्लवग रक्षसाम्}
{दिग्विजृम्भितकाकुत्स्थपौलस्त्यजयघोषणः} % १२-७२

\twolineshloka
{पादपाविद्धपरिघः शिलानिष्पिष्टमुद्गरः}
{अतिशस्त्रनखन्यासः शैलरुग्णमतङ्गजः} % १२-७३

\twolineshloka
{अथ रामशिरश्छेददर्शनोद्भ्रान्तचेतनाम्}
{सीताम् मायेति शंसन्ती त्रिजटा समजीवयत्} % १२-७४

\twolineshloka
{कामम् जीवति मे नाथ इति सा विजहौ शुचम्}
{प्राङ्मत्वा सत्यमस्यान्तम् जीवितास्मीति लज्जिता} % १२-७५

\twolineshloka
{गरुडापातविश्लिष्टमेघनादास्त्रबन्धनः}
{दाशरथ्योः क्षणक्लेशः स्वप्नवृत्त इवाभवत्} % १२-७६

\twolineshloka
{ततो बिभेद पौलस्त्यः शक्त्या वक्षसि लक्ष्मणम्}
{रामस्त्वनाहतोऽप्यासीद्विदीर्णहॄदयः शुचा} % १२-७७

\twolineshloka
{स मारुतिसमानीतमहौषधिहतव्यथः}
{लङ्कास्त्रीणाम् पुनश्चक्रे विलापाचार्यकम् शरैः} % १२-७८

\twolineshloka
{स नादम् मेघनादस्य धनुश्चेन्द्रायुधप्रभम्}
{मेघस्येव शरत्कालो न किञ्चित्पर्यशेषयत्} % १२-७९

\twolineshloka
{कुम्भकर्णः कपीन्द्रेण तुल्यावस्थः स्वसुः कृतः}
{रुरोध रामम् शृङ्गीव टङ्कच्छिन्नमनःशिलः} % १२-८०

\twolineshloka
{अकाले बोधितो भ्राता प्रियस्वप्नो वृथा भवान्}
{रामेषुभिरितीवासौ दीर्घनिद्राम् प्रवेशितः} % १२-८१

\twolineshloka
{इतराण्यपि रक्षांसि पेतुर्वानरकोटिषु}
{रजांसि समरोत्थानि तच्छोणितनदीष्विव} % १२-८२

\twolineshloka
{निर्ययावथ पौलस्त्यः पुनर्युद्धाय मन्दिरात्}
{अरावणमरामम् वा जगदद्येति निश्चितः} % १२-८३

\twolineshloka
{रामम् पदातिमालोक्य लङ्केशम् च वरूथिनम्}
{हरियुग्यम् रथम् तस्मै प्रजिघाय पुरन्दरः} % १२-८४

\twolineshloka
{तमाधूतध्वजपटम् व्योमगङ्गोर्मिवायुभिः}
{देवसूतभुजालम्बी जैत्रमध्यास्त राघवः} % १२-८५

\twolineshloka
{मातलिस्तस्य माहेन्द्रमामुमोच तनुच्छदम्}
{यत्रोत्पलदलक्लैब्यमस्त्राण्यापुः सुरद्विषाम्} % १२-८६

\twolineshloka
{अन्योन्यदर्शनप्राप्तविक्रमावसरम् चिरात्}
{रामरावणयोर्युद्धम् चरितार्थमिवाभवत्} % १२-८७

\twolineshloka
{भुजमूर्धोरुबाहुल्यादेकोऽपि धनदानुजः}
{ददृशे ह्ययथापूर्वो मातृवंश इव स्थितः} % १२-८८

\twolineshloka
{जेतारम् लोकपालानाम् स्वमुखैरर्चितेश्वरम्}
{रामस्तुलितकैलासमारातिम् बह्वमन्यत} % १२-८९

\twolineshloka
{तस्य स्फुरति पौलस्त्यः सीतासङ्गमशंसिनि}
{निचखानाधिकक्रोधः शरम् सव्येतरे भुजे} % १२-९०

\twolineshloka
{रावणस्यापि रामास्तो भित्त्वा हृदयमाशुगः}
{विवेश भुवमाख्यातुमुरगेभ्य इव प्रियम्} % १२-९१

\twolineshloka
{वचसैव तयोर्वाक्यमस्त्रमस्त्रेण निघ्नतोः}
{अन्योन्यजयसंरम्भो ववृधे वादिनोरिव} % १२-९२

\twolineshloka
{विक्रमव्यतिहारेण सामान्याऽभूद्द्वयोरपि}
{जयश्रीरन्तरा वेदिर्मत्तवारणयोरिव} % १२-९३

\twolineshloka
{कृतप्रतिकृतप्रीतैस्तयोर्मुक्ताम् सुरासुरैः}
{परस्परशरव्राताः पुष्पवृष्टिम् न सेहिरे} % १२-९४

\twolineshloka
{अयःशङ्कुचिताम् रक्षः शतघ्नीमथ शत्रवे}
{हृताम् वैवस्वतस्येव कूटशाल्मलिमक्षिपत्} % १२-९५

\twolineshloka
{राघवो रथमप्राप्ताम् तामाशाम् च सुरद्विषाम्}
{अर्धचन्द्रमुखैर्बाणैश्चिच्छेद कदलीसुखम्} % १२-९६

\twolineshloka
{अमोघम् सन्दधे चास्मै धनुष्येकधनुर्धरः}
{ब्राह्ममस्त्रम् प्रियाशोकशल्यनिष्कर्षणौषधम्} % १२-९७

\twolineshloka
{तद्व्योम्नि शतधा भिन्नम् ददृशे दीप्तिमन्मुखम्}
{वपुर्महोरगस्येव करालफलमण्डलम्} % १२-९८

\twolineshloka
{तेन मन्त्रप्रयुक्तेन निमेषार्धादपातयत्}
{स रावणशिरःपङ्क्तिमज्ञातव्रणवेदनाम्} % १२-९९

\twolineshloka
{बालार्कप्रतिमेवाप्सु वीचिभिन्ना पतिष्यतः}
{रराज रक्षःकायस्य कण्ठच्छेदपरम्परा} % १२-१००

\twolineshloka
{मरुताम् पश्यताम् तस्य शिरांसि पतितान्यपि}
{मनो नातिविशस्वास पुनःसन्धानशङ्किनाम्} % १२-१०१

\fourlineindentedshloka
{अथ मदगुरुपक्षैलोकपालद्विपाना}
{मनुगतमलिवृन्दैर्गण्डभित्तीर्विहाय}
{उपनतमणिबन्धे मूर्ध्नि पौलस्त्यशत्रोः}
{सुरभि सुरविमुक्तम् पुष्पवर्षम् पपात} % १२-१०२

\fourlineindentedshloka
{यन्ता हरेः सपदि संहृतकार्मुकज्य}
{मापृच्छ्य राघवमनुष्ठितदेवकार्यम्}
{नामाङ्करावणशराङ्कितकेतुयष्टि}
{मूर्ध्वम् रथम् हरिसहस्रयुजम् निनाय} % १२-१०३

\fourlineindentedshloka
{रघुपतिरपि जातवेदोविशुद्धाम् प्रगृह्य प्रियाम्}
{प्रियसुहृदि बिभीषणे सङ्गमय्य श्रियम् वैरिणः}
{रविसुतसहितेन तेनानुयातः ससौमित्रिणा}
{भुजविजितविमानरत्नाधिरूढः प्रतस्थे पुरीम्} % १२-१०४

॥इति श्री-महाकवि-कालिदास-कृत-रघुवंश-महाकाव्ये द्वादशः सर्गः॥
