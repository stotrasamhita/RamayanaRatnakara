\src{रघुवंशः}{सर्गः १०--१५}{}{}
\notes{In 6 cantos, Mahākavi Kālidāsa narrates the story of Rāma, inside the description of the entire lineage of Raghu.}
\textlink{https://sanskritdocuments.org/sites/giirvaani/giirvaani/rv/intro_rv.htm}
\translink{https://sanskritdocuments.org/sites/giirvaani/giirvaani/rv/intro_rv.htm}
\storymeta

\sect{दशमः सर्गः}

\twolineshloka
{पृथिवीं शासतस्तस्य पाकशासनतेजसः}
{किञ्चिदूनमनूनर्द्धेः शरदामयुतं ययौ} % १०-१

\twolineshloka
{न चोपलेभे पूर्वेषामृणनिर्मोक्षसाधनम्}
{सुताभिधानं स ज्योतिः सद्यः शोकतमोपहम्} % १०-२

\twolineshloka
{अतिष्ठत्प्रत्ययापेक्षसन्ततिः स चिरं नृपः}
{प्राङ्मन्थादनभिव्यक्तरत्नोत्पत्तिरिवार्णवः} % १०-३

\twolineshloka
{ऋष्यशृङ्गादयस्तस्य सन्तः सन्तानकाङ्क्षिणः}
{आरेभिरे जितात्मानः पुत्रीयामिष्टिमृत्विजः} % १०-४

\twolineshloka
{तस्मिन्नवसरे देवाः पौलस्त्योपप्लुता हरिम्}
{अभिजग्मुर्निदाघार्ताश्छायावृक्षमिवाध्वगाः} % १०-५

\twolineshloka
{ते च प्रापुरुदन्वन्तं बुबुधे चादिपूरुषः}
{अव्याक्षेपो भविष्यन्त्याः कार्यसिद्धेर्हि लक्षणम्} % १०-६

\twolineshloka
{भोगिभोगासनासीनं ददृशुस्तं दिवौकसः}
{तत्फणामण्डलोदर्चिर्मणिद्योतितविग्रहम्} % १०-७

\twolineshloka
{श्रियः पद्मनिषण्णायाः क्षौमान्तरितमेखले}
{अङ्के निक्षिप्तचरणमास्तीर्णकरपल्लवे} % १०-८

\twolineshloka
{प्रबुद्धपुण्डरीकाक्षं बालातपनिभांशुकम्}
{दिवसं शारदमिव प्रारम्भसुखदर्शनम्} % १०-९

\twolineshloka
{प्रभानुलिप्तश्रीवत्सं लक्ष्मीविभ्रमदर्पणम्}
{कौस्तुभाख्यमपां सारं बिभ्राणं बृहतोरसा} % १०-१०

\twolineshloka
{बाहुभिर्विटपाकारैर्दिव्याभरणभूषितैः}
{आविर्भूतमपां मध्ये पारिजातमिवापरम्} % १०-११

\twolineshloka
{दैत्यस्त्रीगण्डलेखानां मदरागविलोपिभिः}
{हेतिभिश्चेतनावद्भिरुदीरितजयस्वनम्} % १०-१२

\twolineshloka
{मुक्तशेषविरोधेन कुलिशव्रणलक्ष्मणा}
{उपस्थितं प्राञ्जलिना विनीतेन गरुत्मता} % १०-१३

\twolineshloka
{योगनिद्रान्तविशदैः पावनैरवलोकनैः}
{भृग्वादीननुगृह्णन्तं सौखशायनिकानृषीन्} % १०-१४

\twolineshloka
{प्रणिपत्य सुरास्तस्मै शमयित्रे सुरद्विषाम्}
{अथैनं तुष्टुवुः स्तुत्यमवाङ्मनसगोचरम्} % १०-१५

\twolineshloka
{नमो विश्वसृजे पूर्वं विश्वं तदनु बिभ्रते}
{अथ विश्वस्य संहर्त्रे तुभ्यं त्रेधास्थितात्मने} % १०-१६

\twolineshloka
{रसान्तराण्येकरसं यथा दिव्यं पयोऽश्नुते}
{देशे देशे गुणेष्वेवमवस्थास्त्वमविक्रियः} % १०-१७

\twolineshloka
{अमेयो मितलोकस्त्वमनर्थी प्रार्थनावहः}
{अजितो जिष्णुरत्यन्तमव्यक्तो व्यक्तकारणम्} % १०-१८

\twolineshloka*
{एकः कारणतस्तां तामवस्थां प्रतिपद्यसे}
{नानात्वं रागसंयोगात्स्फटिकस्य्ऽएव ते स्मृतम्} % १०-१८*

\twolineshloka
{हृदयस्थमनासन्नमकामं त्वां तपस्विनम्}
{दयालुमनघस्पृष्टं पुराणमजरं विदुः} % १०-१९

\twolineshloka
{सर्वज्ञस्त्वमविज्ञातः सर्वयोनिस्त्वमात्मभूः}
{सर्वप्रभुरनीशस्त्वमेकस्त्वं सर्वरूपभाक्} % १०-२०

\twolineshloka
{सप्तसामोपगीतं त्वां सप्तार्णवजलेशयम्}
{सप्तार्चिर्मुखमाचख्युः सप्तलोकैकसंश्रयम्} % १०-२१

\twolineshloka
{चतुर्वर्गफलं ज्ञानं कालावस्थाश्चतुर्युगाः}
{चतुर्वर्णमयो लोकस्त्वत्तः सर्वं चतुर्मुखात्} % १०-२२

\twolineshloka
{अभ्यासनिगृहीतेन मनसा हृदयाश्रयम्}
{ज्योतिर्मयं विचिन्वन्ति योगिनस्त्वां विमुक्तये} % १०-२३

\twolineshloka
{अजस्य गृह्णतो जन्म निरीहस्य हतद्विषः}
{स्वपतो जागरूकस्य याथार्थ्यं वेद कस्तव} % १०-२४

\twolineshloka
{शब्दादीन्विषयान्भोक्तुं चरितुं दुश्चरं तपः}
{पर्याप्तोऽसि प्रजाः पातुमौदासीन्येन वर्तितुम्} % १०-२५

\twolineshloka
{बहुधाप्यागमैर्भिन्नाः पन्थानः सिद्धिहेतवः}
{त्वय्येव निपतन्त्योघा जाह्नवीया इवार्णवे} % १०-२६

\twolineshloka
{त्वय्यावेशितचित्तानां त्वत्समर्पितकर्मणाम्}
{गतिस्त्वं वीतरागाणामभूयःसन्निवृत्तये} % १०-२७

\twolineshloka
{प्रत्यक्षोऽप्यपरिच्छेद्यो मह्यादिर्महिमा तव}
{आप्तवागनुमानाभ्यां साध्यं त्वां प्रति का कथा} % १०-२८

\twolineshloka
{केवलं स्मरणेनैव पुनासि पुरुषं यतः}
{अनेन वृत्तयः शेषा निवेदितफलास्त्वयि} % १०-२९

\twolineshloka
{उदधेरिव रत्नानि तेजांसीव विवस्वतः}
{स्तुतिभ्यो व्यतिरिच्यन्ते दूराणि चरितानि ते} % १०-३०

\twolineshloka
{अनवाप्तमवाप्तव्यं न ते किञ्चन विद्यते}
{लोकानुग्रह एवैको हेतुस्ते जन्मकर्मणोः} % १०-३१

\twolineshloka
{महिमानं यदुत्कीर्त्य तव संह्रियते वचः}
{श्रमेण तदशक्त्या वा न गुणानामियत्तया} % १०-३२

\twolineshloka
{इति प्रसादयामासुस्ते सुरास्तमधोक्षजम्}
{भूतार्थव्याहृतिः सा हि न स्तुतिः परमेष्ठिनः} % १०-३३

\twolineshloka
{तस्मै कुशलसम्प्रश्नव्यञ्जितप्रीतये सुराः}
{भयमप्रलयोद्वेलादाचख्युर्नैरृतोदधेः} % १०-३४

\twolineshloka
{अथ वेलासमासन्नशैलरन्ध्रानुवादिना}
{स्वरेणोवाच भगवान् परिभूतार्णवध्वनिः} % १०-३५

\twolineshloka
{पुराणस्य कवेस्तस्य वर्णस्थानसमीरिता}
{बभूव कृतसंस्कारा चरितार्थैव भारती} % १०-३६

\twolineshloka
{बभौ सदशनज्योत्स्ना सा विभोर्वदनोद्गता}
{निर्यातशेषा चरणाद्गङ्गेवोर्ध्वप्रवर्तिनी} % १०-३७

\twolineshloka
{जाने वो रक्षसाक्रान्तावनुभावपराक्रमौ}
{अङ्गिनां तमसेवोभौ गुणौ प्रथममध्यमौ} % १०-३८

\twolineshloka
{विदितं तप्यमानं च तेन मे भुवनत्रयम्}
{अकामोपनतेनेव साधोर्हृदयमेनसा} % १०-३९

\twolineshloka
{कार्येषु चैककार्यत्वादभ्यर्थ्योऽस्मि न वज्रिणा}
{स्वयमेव हि वातोऽग्नेः सारथ्यं प्रतिपद्यते} % १०-४०

\twolineshloka
{स्वासिधारापरिहृतः कामं चक्रस्य तेन मे}
{स्थापितो दशमो मूर्धा लभ्यांश इव रक्षसा} % १०-४१

\twolineshloka
{स्रष्टुर्वरातिसर्गात्तु मया तस्य दुरात्मनः}
{अत्यारूढं रिपोः सोढं चन्दनेनेव भोगिनः} % १०-४२

\twolineshloka
{धातारं तपसा प्रीतं ययाचे स हि राक्षसः}
{दैवात्सर्गादवध्यत्वं मर्त्येष्वास्थापराङ्मुखः} % १०-४३

\twolineshloka
{सोऽहं दाशरथिर्भूत्वा रणभूमेर्बलिक्षमम्}
{करिष्यामि शरैस्तीक्ष्णैस्तच्छिरःकमलोच्चयम्} % १०-४४

\twolineshloka
{अचिराद्यज्वभिर्भागं कल्पितं विधिवत्पुनः}
{मायाविभिरनालीढमादास्यध्वे निशाचरैः} % १०-४५

\twolineshloka
{वैमानिकाः पुण्यकृतस्त्यजन्तु मरुतां पथि}
{पुष्पकालोकसङ्क्षोभं मेघावरणतत्पराः} % १०-४६

\twolineshloka
{मोक्ष्यध्वे स्वर्गबन्दीनां वेणीबन्धनदूषितान्}
{शापयन्त्रितपौलस्त्यबलात्कारकचग्रहैः} % १०-४७

\twolineshloka
{रावणावग्रहक्लान्तमिति वागमृतेन सः}
{अभिवृष्य मरुत्सस्यं कृष्णमेघस्तिरोदधे} % १०-४८

\twolineshloka
{पुरुहूतप्रभृतयः सुरकार्योद्यतं सुराः}
{अंशैरनुययुर्विष्णुं पुष्पैर्वायुमिव द्रुमाः} % १०-४९

\twolineshloka
{अथ तस्य विशाम्पत्युरन्ते काम्यस्य कर्मणः}
{पुरुषः प्रबभूवाग्नेर्विस्मयेन सहर्त्विजाम्} % १०-५०

\twolineshloka
{हेमपात्रगतं दोर्भ्यामादधानः पयश्चरुम्}
{अनुप्रवेशादाद्यस्य पुंसस्तेनापि दुर्वहम्} % १०-५१

\twolineshloka
{प्राजापत्योपनीतं तदन्नं प्रत्यग्रहीन्नृपः}
{वृषेव पयसां सारमाविष्कृतमुदन्वता} % १०-५२

\twolineshloka
{अनेन कथिता राज्ञो गुणास्तस्यान्यदुर्लभाः}
{प्रसूतिं चकमे तस्मिंस्त्रैलोक्यप्रभवोऽपि यत्} % १०-५३

\twolineshloka
{स तेजो वैष्णवं पत्न्योर्विभेजे चरुसंज्ञितम्}
{द्यावापृथिव्योः प्रत्यग्रमहर्पतिरिवातपम्} % १०-५४

\twolineshloka
{अर्चिता तस्य कौसल्या प्रिया केकयवंशजा}
{अतः सम्भावितां ताभ्यां सुमित्रामैच्छदीश्वरः} % १०-५५

\twolineshloka
{ते बहुज्ञस्य चित्तज्ञे पत्न्यौ पत्युर्महीक्षितः}
{चरोरर्धार्धभागाभ्यां तामयोजयतामुभे} % १०-५६

\twolineshloka
{सा हि प्रणयवत्यासीत्सपत्न्योरुभयोरपि}
{भ्रमरी वारणस्येव मदनिस्यन्दरेखयोः} % १०-५७

\twolineshloka
{ताभिर्गर्भः प्रजाभूत्यै दध्रे देवांशसम्भवः}
{सौरीभिरिव नाडीभिरमृताख्याभिरम्मयः} % १०-५८

\twolineshloka
{सममापन्नसत्त्वास्ता रेजुरापाण्डुरत्विषः}
{अन्तर्गतफलारम्भाः सस्यानामिव सम्पदः} % १०-५९

\twolineshloka
{गुप्तं ददृशुरात्मानं सर्वाः स्वप्नेषु वामनैः}
{जलजासिगदाशार्ङ्गचक्रलाञ्छितमूर्तिभिः} % १०-६०

\twolineshloka
{हेमपक्षप्रभाजालं गगने च वितन्वता}
{उह्यन्ते स्म सुपर्णेन वेगाकृष्टपयोमुचा} % १०-६१

\twolineshloka
{बिभ्रत्या कौस्तुभन्यासं स्तनान्तरविलम्बिनम्}
{पर्युपास्यन्त लक्ष्म्या च पद्मव्यजनहस्तया} % १०-६२

\twolineshloka
{कृताभिषेकैर्दिव्यायां त्रिस्रोतसि च सप्तभिः}
{ब्रह्मर्षिभिः परं ब्रह्म गृणद्भिरुपतस्थिरे} % १०-६३

\twolineshloka
{ताभ्यस्तथाविधान्स्वप्नाञ्छ्रुत्वा प्रीतो हि पार्थिवः}
{मेने परार्ध्यमात्मानं गुरुत्वेन जगद्गुरोः} % १०-६४

\twolineshloka
{विभक्तात्मा विभुस्तासामेकः कुक्षिष्वनेकधा}
{उवास प्रतिमाचन्द्रः प्रसन्नानामपामिव} % १०-६५

\twolineshloka
{अथाग्र्यमहिषी राज्ञः प्रसूतिसमये सती}
{पुत्रं तमोऽपहं लेभे नक्तं ज्योतिरिवौषधिः} % १०-६६

\twolineshloka
{राम इत्यभिरामेण वपुषा तस्य चोदितः}
{नामधेयं गुरुश्चक्रे जगत्प्रथममङ्गलम्} % १०-६७

\twolineshloka
{रघुवंशप्रदीपेन तेनाप्रतिमतेजसा}
{रक्षागृहगता दीपाः प्रत्यादिष्टा इवाभवन्} % १०-६८

\twolineshloka
{शय्यागतेन रामेण माता शातोदरी बभौ}
{सैकताम्भोजबलिना जाह्नवीव शरत्कृशा} % १०-६९

\twolineshloka
{कैकेय्यास्तनयो जज्ञे भरतो नाम शीलवान्}
{जनयित्रीमलञ्चक्रे यः प्रश्रय इव श्रियम्} % १०-७०

\twolineshloka
{सुतौ लक्ष्मणशत्रुघ्नौ सुमित्रा सुषुवे यमौ}
{सम्यगाराधिता विद्या प्रबोधविनयाविव} % १०-७१

\twolineshloka
{निर्दोषमभवत्सर्वमाविष्कृतगुणं जगत्}
{अन्वगादिव हि स्वर्गो गां गतं पुरुषोत्तमम्} % १०-७२

\twolineshloka
{तस्योदये चतुर्मूर्तेः पौलस्त्यचकितेश्वराः}
{विरजस्कैर्नभस्वद्भिर्दिश उच्छ्वसिता इव} % १०-७३

\twolineshloka
{कृशानुरपधूमत्वात्प्रसन्नत्वात्प्रभाकरः}
{रक्षोविप्रकृतावास्तामपविद्धशुचाविव} % १०-७४

\twolineshloka
{दशाननकिरीटेभ्यस्तत्क्षणं राक्षसश्रियः}
{मणिव्याजेन पर्यस्ताः पृथिव्यामश्रुबिन्दवः} % १०-७५

\twolineshloka
{पुत्रजन्मप्रवेश्यानां तूर्याणां तस्य पुत्रिणः}
{आरम्भं प्रथमं चक्रुर्देवदुन्दुभयो दिवि} % १०-७६

\twolineshloka
{सन्तानकमयी वृष्टिर्भवने चास्य पेतुषी}
{सन्मङ्गलोपचाराणां सैवादिरचनाऽभवत्} % १०-७७

\twolineshloka
{कुमाराः कृतसंस्कारास्ते धात्रीस्तन्यपायिनः}
{आनन्देनाग्रजेनेव समं ववृधिरे पितुः} % १०-७८

\twolineshloka
{स्वाभाविकं विनीतत्वं तेषां विनयकर्मणा}
{मुमूर्छ सहजं तेजो हविषेव हविर्भुजाम्} % १०-७९

\twolineshloka
{परस्पराविरुद्धास्ते तद्रगोरनघं कुलम्}
{अलमुद्द्योतयामासुर्देवारण्यमिवर्तवः} % १०-८०

\twolineshloka
{समानेऽपि च सौभ्रात्रे यथोभौ रामलक्ष्मणौ}
{तथा भरतशत्रुघ्नौ प्रीत्या द्वन्द्वं बभूवतुः} % १०-८१

\twolineshloka
{तेषां द्वयोर्द्वयोरैक्यं बिभिदे न कदाचन}
{यथा वायुविभावस्वोर्यथा चन्द्रसमुद्रयोः} % १०-८२

\twolineshloka
{ते प्रजानां प्रजानाथास्तेजसा प्रश्रयेण च}
{मनो जह्रुर्निदाघान्ते श्यामाभ्रा दिवसा इव} % १०-८३

\twolineshloka
{स चतुर्धा बभौ व्यस्तः प्रसवः पृथिवीपतेः}
{धर्मार्थकाममोक्षाणामवतार इवाङ्गवान्} % १०-८४

\twolineshloka
{गुणैराराधयामासुस्ते गुरुं गुरुवत्सलाः}
{तमेव चतुरन्तेशं रत्नैरिव महार्णवाः} % १०-८५

\fourlineindentedshloka
{सुरगज इव दन्तैर्भग्नदैत्यासिधारैर्-}
{नय इव पणबन्धव्यक्तयोगैरुपायैः}
{हरिरिव युगदीर्घैर्दोर्भिरंशैस्तदीयैः}
{पतिरवनिपतीनां तैश्चकाशे चतुर्भिः} % १०-८६

॥इति श्री-महाकवि-कालिदास-कृत-रघुवंश-महाकाव्ये दशमः सर्गः॥
\sect{एकादशः सर्गः}

\fourlineindentedshloka
{कौशिकेन स किल क्षितीश्वरो}
{राममध्वरविघातशान्तये}
{काकपक्षधरमेत्य याचितः}
{तेजसां हि न वयः समीक्ष्यते} % ११-१

\fourlineindentedshloka
{कृच्छ्रलब्धमपि लब्धवर्णभाक्}
{तं दिदेश मुनये सलक्ष्मणम्}
{अप्यसुप्रणयिनां रघोः कुले}
{न व्यहन्यत कदाचिदर्थिता} % ११-२

\fourlineindentedshloka
{यावदादिशति पार्थिवस्तयोः}
{निर्गमाय पुरमार्गसंस्क्रियाम्}
{तावदाशु विदधे मरुत्सखैः}
{सा सपुष्पजलवर्षिभिर्घनैः} % ११-३

\fourlineindentedshloka
{तौ निदेशकरणोद्यतौ पितुः}
{धन्विनौ चरणयोर्निपेततुः}
{भूपतेरपि तयोः प्रवत्स्यतोः}
{नम्रयोरुपरि बाष्पबिन्दवः} % ११-४

\fourlineindentedshloka
{तौ पितुर्नयनजेन वारिणा}
{किञ्चिदुक्षितशिखण्डकावुभौ}
{धन्विनौ तमृषिमन्वगच्छताम्}
{पौरदृष्टिकृतमार्गतोरणौ} % ११-५

\fourlineindentedshloka
{लक्ष्मणानुचरमेव राघवम्}
{नेतुमैच्छदृषिरित्यसौ नृपः}
{आशिषं प्रयुयुजे न वाहिनीम्}
{सा हि रक्षणविधौ तयोः क्षमा} % ११-६

\fourlineindentedshloka
{मातृवर्गचरणस्पृशौ मुनेः}
{तौ प्रपद्य पदवीं महौजसः}
{रेजतुर्गतिवशात्प्रवर्तिनौ}
{भास्करस्य मधुमाधवाविव} % ११-७

\fourlineindentedshloka
{वीचिलोलभुजयोस्तयोर्गतम्}
{शैशवाच्चपलमप्यशोभत}
{तोयदागमे इवोद्ध्यभिद्ययोः}
{नामधेयसदृशं विचेष्टितम्} % ११-८

\fourlineindentedshloka
{तौ बलातिबलयोः प्रभावतो}
{विद्ययोः पथि मुनिप्रदिष्टयोः}
{मम्लतुर्न मणिकुट्टिमोचितौ}
{मातृपार्श्वपरिवर्तिनाविव} % ११-९

\fourlineindentedshloka
{पूर्ववृत्तकथितैः पुराविदः}
{सानुजः पितृसखस्य राघवः}
{उह्यमान इव वाहनोचितः}
{पादचारमपि न व्यभावयत्} % ११-१०

\fourlineindentedshloka
{तौ सरांसि रसवद्भिरम्बुभिः}
{कूजितैः श्रुतिसुखैः पतत्रिणः}
{वायवः सुरभिपुष्परेणुभिः}
{छायया च जलदाः सिषेविरे} % ११-११

\fourlineindentedshloka
{नाम्भसां कमलशोभिनां तथा}
{शाखिनां च न परिश्रमच्छिदाम्}
{दर्शनेन लघुना यथा तयोः}
{प्रीतिमापुरुभयोस्तपस्विनः} % ११-१२

\fourlineindentedshloka
{स्थाणुदग्धवपुषस्तपोवनम्}
{प्राप्य दाशरथिरात्तकार्मुकः}
{विग्रहेण मदनस्य चारुणा}
{सोऽभवत्प्रतिनिधिर्न कर्मणा} % ११-१३

\fourlineindentedshloka
{तौ सुकेतुसुतया खिलीकृते}
{कौशिकाद्विदितशापया पथि}
{निन्यतुः स्थलनिवेशिताटनी}
{लीलयैव धनुषी अधिज्यताम्} % ११-१४

\fourlineindentedshloka
{ज्यानिनादमथ गृह्णती तयोः}
{प्रादुरास बहुलक्षपाछविः}
{ताडका चलकपालकुण्डला}
{कालिकेव निबिडा बलाकिनी} % ११-१५

\fourlineindentedshloka
{तीव्रवेगधुतमार्गवृक्षया}
{प्रेतचीवरवसा स्वनोग्रया}
{अभ्यभावि भरताग्रजस्तया}
{वात्ययेव पितृकाननोत्थया} % ११-१६

\fourlineindentedshloka
{उद्यतैकभुजयष्टिमायतीम्}
{श्रोणिलम्बिपुरुषान्त्रमेखलाम्}
{तां विलोक्य वनितावधे घृणाम्}
{पत्रिणा सह मुमोच राघवः} % ११-१७

\fourlineindentedshloka
{यच्चकार विवरं शिलाघने}
{ताडकोरसि स रामसायकः}
{अप्रविष्टविषयस्य रक्षसाम्}
{द्वारतामगमदन्तकस्य तत्} % ११-१८

\fourlineindentedshloka
{बाणभिन्नहृदया निपेतुषी}
{सा स्वकाननभुवं न केवलाम्}
{विष्टपत्रयपराजयस्थिराम्}
{रावणश्रियमपि व्यकम्पयत्} % ११-१९

\fourlineindentedshloka
{राममन्मथशरेण ताडिता}
{दुःसहेन हृदये निशाचरी}
{गन्धवद्रुधिरचन्दनोक्षिता}
{जीवितेशवसतिं जगाम सा} % ११-२०

\fourlineindentedshloka
{नैरृतघ्नमथ मन्त्रवन्मुनेः}
{प्रापदस्त्रमवदानतोषितात्}
{ज्योतिरिन्धननिपाति भास्करात्}
{सूर्यकान्त इव ताडकान्तकः} % ११-२१

\fourlineindentedshloka
{वामनाश्रमपदं ततः परम्}
{पावनं श्रुतमृषेरुपेयिवान्}
{उन्मनाः प्रथमजन्मचेष्टिता}
{न्यस्मरन्नपि बभूव राघवः} % ११-२२

\fourlineindentedshloka
{आससाद मुनिरात्मनस्ततः}
{शिष्यवर्गपरिकल्पितार्हणम्}
{बद्धपल्लवपुटाञ्जलिद्रुमम्}
{दर्शनोन्मुखमृगं तपोवनम्} % ११-२३

\fourlineindentedshloka
{तत्र दीक्षितमृषिं ररक्षतुः}
{विघ्नतो दशरथात्मजौ शरैः}
{लोकमन्धतमसात्क्रमोदितौ}
{रश्मिभिः शशिदिवाकराविव} % ११-२४

\fourlineindentedshloka
{वीक्ष्य वेदिमथ रक्तबिन्दुभिः}
{बन्धुजीवपृथुभिः प्रदूषिताम्}
{सम्भ्रमोऽभवदपोढकर्मणाम्}
{ऋत्विजां च्युतविकङ्कतस्रुचाम्} % ११-२५

\fourlineindentedshloka
{उन्मुखः सपदि लक्षमणाग्रजो}
{बाणमाश्रयमुखात्समुद्धरन्}
{रक्षसां बलमपश्यदम्बरे}
{गृध्रपक्षपवनेरितध्वजम्} % ११-२६

\fourlineindentedshloka
{तत्र यावधिपती मखद्विषाम्}
{तौ शरव्यमकरोत्स नेतरान्}
{किं महोरगविसर्पिविक्रमो}
{राजिलेषु गरुडः प्रवर्तते} % ११-२७

\fourlineindentedshloka
{सोऽस्त्रमुग्रजवमस्त्रकोविदः}
{सन्दधे धनुषि वायुदैवतम्}
{तेन शैलगुरुमप्यपातयत्}
{पाण्डुपत्रमिव ताडकासुतम्} % ११-२८

\fourlineindentedshloka
{यः सुबाहुरिति राक्षसोऽपरः}
{तत्र तत्र विससर्प मायया}
{तं क्षुरप्रशकलीकृतं कृती}
{पत्रिणां व्यभजदाश्रमाद्बहिः} % ११-२९

\fourlineindentedshloka
{इत्यपास्तमखविघ्नयोस्तयोः}
{सांयुगीनमभिनन्द्य विक्रमम्}
{ऋत्विजः कुलपतेर्यथाक्रमम्}
{वाग्यतस्य निरवर्तयन्क्रियाः} % ११-३०

\fourlineindentedshloka
{तौ प्रणामचलकाकपक्षकौ}
{भ्रातराववभृथाप्लुतो मुनिः}
{आशिषामनुपदं समस्पृशत्}
{दर्भपाटलतलेन पाणिना} % ११-३१

\fourlineindentedshloka
{तं न्यमन्त्रयत सम्भृतक्रतुः}
{मैथिलः स मिथिलां व्रजन्वशी}
{राघवावपि निनाय बिभ्रतौ}
{तद्धनुःश्रवणजं कुतूहलम्} % ११-३२

\fourlineindentedshloka
{तैः शिवेषु वसतिर्गताध्वभिः}
{सायमाश्रमतनुष्वगृह्यत}
{येषु दीर्घतपसः परिग्रहो}
{वासवक्षणकलत्रतां ययौ} % ११-३३

\fourlineindentedshloka
{प्रत्यपद्यत चिराय यत्पुन-}
{श्चारु गौतमवधूः शिलामयी}
{स्वं वपुः स किल किल्बिषच्छिदाम्}
{रामपादरजसामनुग्रहः} % ११-३४

\fourlineindentedshloka
{राघवान्वितमुपस्थितं मुनिम्}
{तं निशम्य जनको जनेश्वरः}
{अर्थकामसहितं सपर्यया}
{देहबद्धमिव धर्ममभ्यगात्} % ११-३५

\fourlineindentedshloka
{तौ विदेहनगरीनिवासिनाम्}
{गां गताविव दिवः पुनर्वसू}
{मन्यते स्म पिबतां विलोचनैः}
{पक्ष्मपातमपि वञ्चनां मनः} % ११-३६

\fourlineindentedshloka
{यूपवत्यवसिते क्रियाविधौ}
{कालवित्कुशिकवंशवर्धनः}
{राममिष्वसनदर्शनोत्सुकम्}
{मैथिलाय कथयाम्बभूव सः} % ११-३७

\fourlineindentedshloka
{तस्य वीक्ष्य ललितं वपुः शिशोः}
{पार्थिवः प्रथितवंशजन्मनः}
{स्वं विचिन्त्य च धनुर्दुरानमम्}
{पीडितो दुहितृशुल्कसंस्थया} % ११-३८

\fourlineindentedshloka
{अब्रवीच्च भगवन्मतङ्गजै}
{र्यद्बृहद्भिरपि कर्म दुष्करम्}
{तत्र नाहमनुमन्तुमुत्सहे}
{मोघवृत्ति कलभस्य चेष्टितम्} % ११-३९

\fourlineindentedshloka
{ह्रेपिता हि बहवो नरेश्वरास्तेन}
{तात धनुषा धनुर्भृतः}
{ज्यानिघातकठिनत्वचो भुजान्}
{स्वान्विधूय धिगिति प्रतस्थिरे} % ११-४०

\fourlineindentedshloka
{प्रत्युवाच तमृषिर्निशम्यताम्}
{सारतोऽयमथवा गिरा कृतम्}
{चाप एव भवतो भविष्यति}
{व्यक्तशक्तिरशनिर्गिराविव} % ११-४१

\fourlineindentedshloka
{एवमाप्तवचनात्स पौरुषम्}
{काकपक्षकधरेऽपि राघवे}
{श्रद्दधे त्रिदशगोपमात्रके}
{दाहशक्तिमिव कृष्णवर्त्मनि} % ११-४२

\fourlineindentedshloka
{व्यादिदेश गणशोऽथ पार्श्वगान्}
{कार्मुकाभिहरणाय मैथिलः}
{तैजसस्य धनुषः प्रवृत्तये}
{तोयदानिव सहस्रलोचनः} % ११-४३

\fourlineindentedshloka
{तत्प्रसुप्तभुजगेन्द्रभीषणम्}
{वीक्ष्य दाशरथिराददे धनुः}
{विद्रुतक्रतुमृगानुसारिणम्}
{येन बाणमसृजत्वृषध्वजः} % ११-४४

\fourlineindentedshloka
{आततज्यमकरोत्स संसदा}
{विस्मयस्तिमितनेत्रमीक्षितः}
{शैलसारमपि नातियत्नतः}
{पुष्पचापमिव पेशलं स्मरः} % ११-४५

\fourlineindentedshloka
{भज्यमानमतिमात्रकर्षणात्}
{तेन वज्रपरुषस्वनं धनुः}
{भार्गवाय दृढमन्यवे पुनः}
{क्षत्रमुद्यतमिव न्यवेदयत्} % ११-४६

\fourlineindentedshloka
{दृष्टसारमथ रुद्रकार्मुके}
{वीर्यशुल्कमभिनन्द्य मैथिलः}
{राघवाय तनयामयोनिजाम्}
{रूपिणीं श्रियमिव न्यवेदयत्} % ११-४७

\fourlineindentedshloka
{मैथिलः सपदि सत्यसङ्गरो}
{राघवाय तनयामयोनिजाम्}
{सन्निधौ द्युतिमतस्तपोनिधे}
{रग्निसाक्षिक इवातिसृष्टवान्} % ११-४८

\fourlineindentedshloka
{प्राहिणोच्च महितं महाद्युतिः}
{कोसलाधिपतये पुरोधसम्}
{भृत्यभावि दुहितुः परिग्रहात्}
{दिश्यतां कुलमिदं निमेरिति} % ११-४९

\fourlineindentedshloka
{अन्वियेष सदृशीं स च स्नुषाम्}
{प्राप चैनमनुकूलवाग्द्विजः}
{सद्य एव सुकृतां हि पच्यते}
{कल्पवृक्षफलधर्मि काङ्क्षितम्} % ११-५०

\fourlineindentedshloka
{तस्य कल्पितपुरस्क्रियाविधेः}
{शुश्रुवान्वचनमग्रजन्मनः}
{उच्चचाल बलभित्सखो वशी}
{सैन्यरेणुमुषितार्कदीधितिः} % ११-५१

\fourlineindentedshloka
{आससाद मिथिलां स वेष्टयन्}
{पीडितोपवनपादपां बलैः}
{प्रीतिरोधमसहिष्ट सा पुरी}
{स्त्रीव कान्तपरिभोगमायतम्} % ११-५२

\fourlineindentedshloka
{तौ समेत्य समये स्थितावुभौ}
{भूपती वरुणवासवोपमौ}
{कन्यकातनयकौतुकक्रियाम्}
{स्वप्रभावसदृशीं वितेनतुः} % ११-५३

\fourlineindentedshloka
{पार्थिवीमुदवहद्रघूद्वहो}
{लक्ष्मणस्तदनुजामथोर्मिलाम्}
{यौ तयोरवरजौ वरौजसौ}
{तौ कुशध्वजसुते सुमध्यमे} % ११-५४

\fourlineindentedshloka
{ते चतुर्थसहितास्त्रयो बभुः}
{सूनवो नववधूपरिग्रहाः}
{सामदानविधिभेदनिग्रहाः}
{सिद्धिमन्त इव तस्य भूपतेः} % ११-५५

\fourlineindentedshloka
{ता नराधिपसुता नृपात्मजैः}
{ते च ताभिरगमन्कृतार्थताम्}
{सोऽभवद्वरवधूसमागमः}
{प्रत्ययप्रकृतियोगसन्निभः} % ११-५६

\fourlineindentedshloka
{एवमात्तरतिरात्मसम्भवां}
{तान्निवेश्य चतुरोऽपि तत्र सः}
{अध्वसु त्रिषु विसृष्टमैथिलः}
{स्वां पुरीं दशरथो न्यवर्तत} % ११-५७

\fourlineindentedshloka
{तस्य जातु मरुतः प्रतीपगा}
{वर्त्मसु ध्वजतरुप्रमाथिनः}
{चिक्लिशुर्भृशतया वरूथिनीम्}
{उत्तटा इव नदीरयाः स्थलीम्} % ११-५८

\fourlineindentedshloka
{तस्य जातु मरुतः प्रतीपगा}
{वर्त्मसु ध्वजतरुप्रमाथिनः}
{चिक्लिशुर्भृशतया वरूथिनीम्}
{उत्तटा इव नदीरयाः स्थलीम्} % ११-५८

\fourlineindentedshloka
{लक्ष्यते स्म तदनन्तरं रविः}
{बद्धभीमपरिवेषमण्डलः}
{वैनतेयशमितस्य भोगिनो}
{भोगवेष्टित इव च्युतो मणिः} % ११-५९

\fourlineindentedshloka
{श्येनपक्षपरिधूसरालकाः}
{सान्ध्यमेघरुधिरार्द्रवाससः}
{अङ्गना इव रजस्वला दिशो}
{नो बभूवुरवलोकनक्षमाः} % ११-६०

\fourlineindentedshloka
{भास्करश्च दिशमध्युवास याम्}
{तां श्रिताः प्रतिभयं ववासिरे}
{क्षत्रशोणितपितृक्रियोचितम्}
{चोदयन्त्य इव भार्गवं शिवाः} % ११-६१

\fourlineindentedshloka
{तत्प्रतीपपवनादिवैकृतम्}
{प्रेक्ष्य शान्तिमधिकृत्य कृत्यविद्}
{अन्वयुङ्क्त गुरुमीश्वरः क्षितेः}
{स्वन्तमित्यलघयत्स तद्व्यथाम्} % ११-६२

\fourlineindentedshloka
{तेजसः सपदि राशिरुत्थितः}
{प्रादुरास किल वाहिनीमुखे}
{यः प्रमृज्य नयनानि सैनिकैः}
{लक्षणीयपुरुषाकृतिश्चिरात्} % ११-६३

\fourlineindentedshloka
{पित्र्यमंशमुपवीतलक्षणम्}
{मातृकं च धनुरूर्जितं दधत्}
{यः ससोम इव घर्मदीधितिः}
{सद्विजिह्व इव चन्दनद्रुमः} % ११-६४

\fourlineindentedshloka
{येन रोषपरुषात्मनः पितुः}
{शासने स्थितिभिदोऽपि तस्थुषा}
{वेपमानजननीशिरश्छिदा}
{प्रागजीयत घृणा ततो मही} % ११-६५

\fourlineindentedshloka
{अक्षबीजवलयेन निर्बभौ}
{दक्षिणश्रवणसंस्थितेन यः}
{क्षत्रियान्तकरणैकविंशतेः}
{व्याजपूर्वगणनामिवोद्वहन्} % ११-६६

\fourlineindentedshloka
{तं पितुर्वधभवेन मन्युना}
{राजवंशनिधनाय दीक्षितम्}
{बालसूनुरवलोक्य भार्गवम्}
{स्वां दशां च विषसाद पार्थिवः} % ११-६७

\fourlineindentedshloka
{नाम राम इति तुल्यमात्मजे}
{वर्तमानमहिते च दारुणे}
{हृद्यमस्य भयदायि चाभवद्}
{रत्नजातमिव हारसर्पयोः} % ११-६८

\fourlineindentedshloka
{अर्घ्यमर्घ्यमिति वादिनं नृपम्}
{सोऽनवेक्ष्य भरताग्रजो यतः}
{क्षत्रकोपदहनार्चिषं ततः}
{सन्दधे दृशमुदग्रतारकाम्} % ११-६९

\fourlineindentedshloka
{तेन कार्मुकनिषक्तमुष्टिना}
{राघवो विगतभीः पुरोगतः}
{अङ्गुलीविवरचारिणं शरम्}
{कुर्वता निजगदे युयुत्सुना} % ११-७०

\fourlineindentedshloka
{क्षत्रजातमपकारवैरि मे}
{तन्निहत्य बहुशः शमं गतः}
{सुप्तसर्प इव दण्डघट्टनात्}
{रोषितोऽस्मि तव विक्रमश्रवात्} % ११-७१

\fourlineindentedshloka
{मैथिलस्य धनुरन्यपार्थिवैः}
{त्वं किलानमितपूर्वमक्षणोः}
{तन्निशम्य भवता समर्थये}
{वीर्यशृङ्गमिव भग्नमात्मनः} % ११-७२

\fourlineindentedshloka
{अन्यदा जगति राम इत्ययम्}
{शब्द उच्चरित एव मामगात्}
{व्रीडमावहति मे स सम्प्रति}
{व्यस्तवृत्तिरुदयोन्मुखे त्वयि} % ११-७३

\fourlineindentedshloka
{बिभ्रतोऽस्त्रमचलेऽप्यकुण्ठितम्}
{द्वौ रुपू मम मतौ समागसौ}
{धेनुवत्सहरणाच्च हैहय}
{स्त्वं च कीर्तिमपहर्तुमुद्यतः} % ११-७४

\fourlineindentedshloka
{क्षत्रियान्तकरणोऽपि विक्रमः}
{तेन मामवति नाजिते त्वयि}
{पावकस्य महिमा स गण्यते}
{कक्षवज्जलति सागरेऽपि यः} % ११-७५

\fourlineindentedshloka
{विद्धि चात्तबलमोजसा हरेः}
{ऐश्वरं धनुरभाजि यत्त्वया}
{खातमूलमनिलो नदीरयैः}
{पातयत्यपि मृदुस्तटद्रुमम्} % ११-७६

\fourlineindentedshloka
{तन्मदीयमिदमायुधं ज्यया}
{सङ्गमय्य सशरं विकृष्यताम्}
{तिष्ठतु प्रधनमेवमप्यहम्}
{तुल्यबाहुतरसा जितस्त्वया} % ११-७७

\fourlineindentedshloka
{कातरोऽसि यदि वोद्गतार्चिषा}
{तर्जितः परशुधारया मम}
{ज्यानिघातकठिनाङ्गुलिर्वृथा}
{बध्यतामभययाचनाञ्जलिः} % ११-७८

\fourlineindentedshloka
{एवमुक्तवति भीमदर्शने}
{भार्गवे स्मितविकम्पिताधरः}
{तद्धनुर्ग्रहणमेव राघवः}
{प्रत्यपद्यत समर्थमुत्तरम्} % ११-७९

\fourlineindentedshloka
{पूर्वजन्मधनुषा समागतः}
{सोऽतिमात्रलघुदर्शनोऽभवत्}
{केवलोऽपि सुभगो नवाम्बुदः}
{किं पुनस्त्रिदशचापलाञ्छितः} % ११-८०

\fourlineindentedshloka
{तेन भूमिनिहितैककोटि}
{तत्कार्मुकं च बलिनाधिरोपितम्}
{निष्प्रभश्च रिपुरास भूभृताम्}
{धूमशेष इव धूमकेतनः} % ११-८१

\fourlineindentedshloka
{तावुभावपि परस्परस्थितौ}
{वर्धमानपरिहीनतेजसौ}
{पश्यति स्म जनता दिनात्यये}
{पार्वणौ शशिदिवाकराविव} % ११-८२

\fourlineindentedshloka
{तं कृपामृदुरवेक्ष्य भार्गवम्}
{राघवः स्खलितवीर्यमात्मनि}
{स्वं च संहितममोघमाशुगम्}
{व्याजहार हरसूनुसन्निभः} % ११-८३

\fourlineindentedshloka
{न प्रहर्तुमलमस्मि निर्दयम्}
{विप्र इत्यभिभत्यपि त्वयि}
{शंस किं गतिमनेन पत्रिणा}
{हन्मि लोकमुत ते मखार्जितम्} % ११-८४

\fourlineindentedshloka
{प्रत्युवाच तमृषिर्न तत्त्वतः}
{त्वां न वेद्मि पुरुषं पुरातनम्}
{गां गतस्य तव धाम वैष्णवम्}
{कोपितो ह्यसि मया दिदृक्षुणा} % ११-८५

\fourlineindentedshloka
{भस्मसात्कृतवतः पितृद्विषः}
{पात्रसाच्च वसुधां ससागराम्}
{आहितो जयविपर्ययोऽपि मे}
{श्लाघ्य एव परमेष्ठिना त्वया} % ११-८६

\fourlineindentedshloka
{तत् गतिं मतिमतां वरेप्सिताम्}
{पुण्यतीर्थगमनाय रक्ष मे}
{पीडयिष्यति न मां खिलीकृता}
{स्वर्गपद्धतिः अभोघलोलुपम्} % ११-८७

\fourlineindentedshloka
{प्रत्यपद्यत तथेति राघवः}
{प्राङ्मुखश्च विससर्ज सायकम्}
{भार्गवस्य सुकृतोऽपि सोऽभवत्}
{स्वर्गमार्गपरिघो दुरत्ययः} % ११-८८

\fourlineindentedshloka
{राघवोऽपि चरणौ तपोनिधेः}
{क्षम्यतामिति वदन्समस्पृशत्}
{निर्जितेषु तरसा तरस्विनाम्}
{शत्रुषु प्रणतिरेव कीर्तये} % ११-८९

\fourlineindentedshloka
{राजसत्वमवधूय मातृकम्}
{पित्र्यमस्मि गमितः शमं यदा}
{नन्वनिन्दितफलो मम त्वया}
{निग्रहोऽप्ययमनुगृहीकृतः} % ११-९०

\fourlineindentedshloka
{साधयाम्यहमविघ्नमस्तु ते}
{देवकार्यमुपपादयिष्यतः}
{ऊचिवानिति वचः सलक्ष्मणम्}
{लक्ष्मणाग्रजमृषितिरोदधे} % ११-९१

\fourlineindentedshloka
{तस्मिन् गते विजयिनं परिरभ्य रामं}
{स्नेहादमन्यत पिता पुनरेव जातम्}
{तस्याभवत्क्षणशुचः परितोषलाभः}
{कक्षाग्निलङ्घिततरोरिव वृष्टिपातः} % ११-९२

\fourlineindentedshloka
{अथ पथि गमयित्वा कॢप्तरम्योपकार्ये}
{कतिचिदवनिपालः शर्वरीः शर्वकल्पः}
{पुरमविशदयोध्यां मैथिलीदर्शनीनाम्}
{कुवलयितगवाक्षां लोचनैरङ्गनानाम्} % ११-९३

॥इति श्री-महाकवि-कालिदास-कृत-रघुवंश-महाकाव्ये एकादशः सर्गः॥
\sect{द्वादशः सर्गः}

\twolineshloka
{निर्विष्टविषयस्नेहः स दशान्तमुपेयिवान्}
{आसीदासन्ननिर्वाणः प्रदीपार्चिरिवोषसि} % १२-१

\twolineshloka
{तं कर्णमूलमागत्य रामे श्रीर्न्यस्यतामिति}
{कैकेयीशङ्कयेवाह पलितच्छद्मना जरा} % १२-२

\twolineshloka
{सा पौरान्पौरकान्तस्य रामस्याभ्युदयश्रुतिः}
{प्रत्येकं ह्रादयाञ्चक्रे कुल्येवोद्यानपादपान्} % १२-३

\twolineshloka
{तस्याभिषेकसम्भारं कल्पितं क्रूरनिश्चया}
{दूषयामास कैकेयी शोकोष्णैः पार्थिवाश्रुभिः} % १२-४

\twolineshloka
{सा किलाश्वासिता चण्डी भर्त्रा तत् संश्रुतौ वरौ}
{उद्ववामेन्द्रसिक्ता भूर्बिलमग्नाविवोरगौ} % १२-५

\twolineshloka
{तयोश्चतुर्दशैकेन रामं प्राव्राजयत्समाः}
{द्वितीयेन सुतस्यैच्छद्वैधव्यैकफलां श्रियम्} % १२-६

\twolineshloka
{पित्रा दत्तं रुदन्रामः प्राङ्महीं प्रत्यपद्यत}
{पश्चाद्वनाय गच्छेति तदाज्ञां मुदितोऽग्रहीत्} % १२-७

\twolineshloka
{सम्बाधवर्तिभिः क्षौमे वसानस्य च वल्कले}
{ददृशुर्विस्मितास्तस्य मुखरागं समं जनाः} % १२-८

\twolineshloka
{स सीतालक्ष्मणसखः सत्याद्गुरुमलोपयन्}
{विवेश दण्डकारण्यं प्रत्येकं च सतां मनः} % १२-९

\twolineshloka
{राजापि तद्वियोगार्तः स्मृत्वा शापं स्वकर्मजम्}
{शरीरत्यागमात्रेण शुद्धिलाभममन्यत} % १२-१०

\twolineshloka
{विप्रोषितकुमारं तद्राज्यमस्तमितेश्वरम्}
{रन्ध्रान्वेषणदक्षाणां द्विषामामिषतां ययौ} % १२-११

\twolineshloka
{अथानाथाः प्रकृतयो मातृबन्धुनिवासिनम्}
{मौलैरानाययामासुर्भरतं स्तम्भिताश्रुभिः} % १२-१२

\twolineshloka
{श्रुत्वा तथाविधं मृत्युं कैकेयीतनयः पितुः}
{मातुर्न केवलं स्वस्याः श्रियोऽप्यासीत्पराङ्मुखः} % १२-१३

\twolineshloka
{ससैन्यश्चान्वगाद्रामं दर्शितानाश्रमालयैः}
{तस्य पश्यन्ससौमित्रेरुदश्रुर्वसतिद्रुमान्} % १२-१४

\twolineshloka
{चित्रकूटवनस्थं च कथितस्वर्गतिर्गुरोः}
{लक्ष्म्या निमन्त्रयाञ्चक्रे तमनुच्छिष्टसम्पदा} % १२-१५

\twolineshloka
{स हि प्रथमजे तस्मिन्नकृतश्रीपरिग्रहे}
{परिवेत्तारमात्मानं मेने स्वीकरणाद्भुवः} % १२-१६

\twolineshloka
{तमशक्यमपाक्रष्टुं निदेशात्स्वर्गिणः पितुः}
{ययाचे पादुके पश्चात्कर्तुं राज्याधिदेवते} % १२-१७

\twolineshloka
{स विसृष्टस्तथेत्युक्त्वा भ्रात्रा नैवाविशत्पुरीम्}
{नन्दिग्रामगतस्तस्य राज्यं न्यासमिवाभुनक्} % १२-१८

\twolineshloka
{दृढभक्तिरिति ज्येष्ठे राज्यतृष्णापराङ्मुखः}
{मातुः पापस्य भरतः प्रायश्चित्तमिवाकरोत्} % १२-१९

\twolineshloka
{रामोऽपि सह वैदेह्या वने वन्येन वर्तयन्}
{चचार सानुजः शान्तो वृद्धेक्ष्वाकुव्रतं युवा} % १२-२०

\twolineshloka
{प्रभावस्तम्भितच्छायमाश्रितः स वनस्पतिम्}
{कदाचिदङ्के सीतायाः शिश्ये किञ्चिदिव श्रमात्} % १२-२१

\twolineshloka
{ऐन्द्रिः किल नखैस्तस्या विददार स्तनौ द्विजः}
{प्रियोपभोगचिह्नेषु पौरोभाग्यमिवाचरन्} % १२-२२

\twolineshloka
{तस्मिन्नस्थदिषीकास्त्रं रामो रामावबोधितः}
{आत्मानं मुमुचे तस्मादेकनेत्रव्ययेन सः} % १२-२३

\twolineshloka
{रामस्त्वासन्नदेशत्वाद्भरतागमनं पुनः}
{आशङ्क्योत्सुकसारङ्गां चित्रकूटस्थलीं जहौ} % १२-२४

\twolineshloka
{प्रययावातिथेयेषु वसन्नृषिकुलेषु सः}
{दक्षिणां दिशमृक्षेषु वार्षिकेष्विव भास्करः} % १२-२५

\twolineshloka
{बभौ तमनुगच्छन्ती विदेहाधिपतेः सुता}
{प्रतिषिद्धापि कैकेय्या लक्ष्मीरिव गुणोन्मुखी} % १२-२६

\twolineshloka
{अनसूयातिसृष्टेन पुण्यगन्धेन काननम्}
{सा चकाराङ्गरागेण पुष्पोच्चलितषट्पदम्} % १२-२७

\twolineshloka
{सन्ध्याभ्रकपिशस्तस्य विराधो नाम राक्षसः}
{अतिष्ठन्मार्गमावृत्य रामस्येन्दोरिव ग्रहः} % १२-२८

\twolineshloka
{स जहार तयोर्मध्ये मैथिलीं लोकशोषणः}
{नभोनभस्ययोर्वृष्टिमवग्रह इवान्तरे} % १२-२९

\twolineshloka
{तं विनिष्पिष्य काकुत्स्थौ पुरा दूषयति स्थलीम्}
{गन्धेनाशुचिना चेति वसुधायां निचख्नतुः} % १२-३०

\twolineshloka
{पञ्चवट्यां ततो रामः शासनात्कुम्भजन्मनः}
{अनपोढस्थितिस्तस्थौ विन्ध्याद्रिः प्रकृताविव} % १२-३१

\twolineshloka
{रावणावरजा तत्र राघवं मदनातुरा}
{अभिपेदे निदाघार्ता व्यालीव मलयद्रुमम्} % १२-३२

\twolineshloka
{सा सीतासन्निधावेव तं वव्रे कथितान्वया}
{अत्यारूढो हि नारीणामकालज्ञो मनोभवः} % १२-३३

\twolineshloka
{कलत्रवानहं बाले कनीयांसं भजस्व मे}
{इति रामो वृषस्यन्तीं वृषस्कन्धः शशास ताम्} % १२-३४

\twolineshloka
{ज्येष्ठाभिगमनात्पूर्वं तेनाप्यनभिनन्दिताम्}
{साऽभूद्रामाश्रया भूयो नदीवोभयकूलभाक्} % १२-३५

\twolineshloka
{संरम्भं मैथिलीहासः क्षणसौम्यां निनाय ताम्}
{निवातस्तिमितां वेलां चन्द्रोदय इवोदधेः} % १२-३६

\twolineshloka
{फलमस्योपहासस्य सद्यः प्राप्स्यसि पश्य माम्}
{मृग्याः परिभवो व्याघ्र्यामित्यवेहि त्वया कृतम्} % १२-३७

\twolineshloka
{इत्युक्त्वा मैथिलीं भर्तुरङ्के निविशतीं भयात्}
{रूपं शूर्पणखा नाम्नः सदृशं प्रत्यपद्यत} % १२-३८

\twolineshloka
{लक्ष्मणः प्रथमं श्रुत्वा कोकिलामञ्जुवादिनीम्}
{शिवाघोरस्वनां पश्चाद्बुबुधे विकृतेति ताम्} % १२-३९

\twolineshloka
{पर्णशालामथ क्षिप्रं विकृष्टासिः प्रविश्य सः}
{वैरूप्यपौनरुक्त्येन भीषणां तामयोजयत्} % १२-४०

\twolineshloka
{सा वक्रनखधारिण्या वेणुकर्कशपर्वया}
{अङ्कुशाकारयाङ्गुल्या तावतर्जयदम्बरे} % १२-४१

\twolineshloka
{प्राप्य चाशु जनस्थानं खरादिभ्यस्तथाविधम्}
{रामोपक्रममाचख्यौ रक्षःपरिभवं नवम्} % १२-४२

\twolineshloka
{मुखावयवलूनां तां नैरृता यत्पुरो दधुः}
{रामाभियायिनां तेषां तदेवाभूदमङ्गलम्} % १२-४३

\twolineshloka
{उदायुधानात्पततस्तान्दृप्तान्प्रेक्ष्य राघवः}
{निदधे विजयाशंसां चापे सीतां च लक्ष्मणे} % १२-४४

\twolineshloka
{एको दाशरथिः कामं यातुधानाः सहस्रशः}
{ते तु यावन्त एवाजौ तावांश्च ददृशे स तैः} % १२-४५

\twolineshloka
{असज्जनेन काकुत्स्थः प्रयुक्तमथ दूषणम्}
{न चक्षमे शुभाचारः स दूषणमिवात्मनः} % १२-४६

\twolineshloka
{तं शरैः प्रतिजग्राह खरत्रिशिरसौ च सः}
{क्रमशस्ते पुनस्तस्य चापात्सममिवोद्ययुः} % १२-४७

\twolineshloka
{तैस्त्रयाणां शितैर्बाणैर्यथापूर्वविशुद्धिभिः}
{आयुर्देहातिगैः पीतं रुधिरं तु पतत्रिभिः} % १२-४८

\twolineshloka
{तस्मिन्रामशरोत्कृत्ते बले महति रक्षसाम्}
{उत्थितं ददृशेऽन्न्यच्च कबन्धेभ्यो न किञ्चन} % १२-४९

\twolineshloka
{सा बाणवर्षिणं रामं योधयित्वा सुरद्विषाम्}
{अप्रबोधाय सुष्वाप गृध्रच्छाये वरूथिनी} % १२-५०

\twolineshloka
{राघवास्त्रविदीर्णानां रावणं प्रति रक्षसाम्}
{तेषां शूर्पणखैवैका दुष्प्रवृत्तिहराऽभवत्} % १२-५१

\twolineshloka
{निग्रहात्स्वसुराप्तानां वधाच्च धनदानुजः}
{रामेण निहितं मेने पदं दशसु मूर्धसु} % १२-५२

\twolineshloka
{रक्षसा मृगरूपेण वञ्चयित्वा स राघवौ}
{जहार सीतां पक्षीन्द्रप्रयासक्षणविघ्नितः} % १२-५३

\twolineshloka
{तौ सीतावेषिणौ गृध्रं लूनपक्षमपश्यताम्}
{प्राणैर्दशरथप्रीतेरनृणं कण्ठवर्तिभिः} % १२-५४

\twolineshloka
{स रावणहृतां ताभ्यां वचसाचष्ट मैथिलीम्}
{आत्मनः सुमहत्कर्म व्रणैरावेद्य संस्थितः} % १२-५५

\twolineshloka
{तयोस्तस्मिन्नवीभूतपितृव्यापतिशोकयोः}
{पितरीवाग्निसंस्कारात्परा ववृतिरे क्रियाः} % १२-५६

\twolineshloka
{वधनिर्धूतशापस्य कबन्धस्योपदेशतः}
{मुमूर्च्छ सख्यं रामस्य समानव्यसने हरौ} % १२-५७

\twolineshloka
{स हत्वा वालिनं वीरस्तत्पदे चिरकाङ्क्षिते}
{धातोः स्थान इवादेशं सुग्रीवं सन्न्यवेशयत्} % १२-५८

\twolineshloka
{इतस्ततश्च वैदेहीमन्वेष्टुं भर्तृचोदिताः}
{कपयश्चेरुरार्तस्य रामस्येव मनोरथाः} % १२-५९

\twolineshloka
{प्रवृत्तावुपलब्धायां तस्याः सम्पातिदर्शनात्}
{मारुतिः सागरं तीर्णः संसारमिव निर्ममः} % १२-६०

\twolineshloka
{दृष्टा विचिन्वता तेन लङ्कायां राक्षसीवृता}
{जानकी विषवल्लीभिः परितेव महौषधिः} % १२-६१

\twolineshloka
{तस्यै भर्तुरभिज्ञ्यानमङ्गुलीयं ददौ कपिः}
{प्रत्युद्गतमिवानुष्णैस्तदानन्दाश्रुबिन्दुभिः} % १२-६२

\twolineshloka
{निर्वाप्य प्रियसन्देशैः सीतामक्षवधोद्धतः}
{स ददाह पुरीं लङ्कां क्षणसोढारिनिग्रहः} % १२-६३

\twolineshloka
{प्रत्यभिज्ञानरत्नं च रामायादर्शयत्कृती}
{हृदयं स्वयमायातं वैदेह्या इव मूर्तिमत्} % १२-६४

\twolineshloka
{स प्राप हृदयन्यस्तमणिस्पर्शनिमीलितः}
{अपयोधरसंसर्गां प्रियालिङ्गननिर्वृतिम्} % १२-६५

\twolineshloka
{श्रुत्वा रामः प्रियोदन्तं मेने तत्सङ्गमोत्सुकः}
{महार्णवपरिक्षेपं लङ्कायाः परिखालघुम्} % १२-६६

\twolineshloka
{स प्रतस्थेऽरिनाशाय हरिसैन्यैरनुद्रुतः}
{न केवलं भुवः पृष्ठे व्योम्नि सम्बाधवर्तिभिः} % १२-६७

\twolineshloka
{निविष्टमुदधेः कूले तं प्रपाद बिभीषणः}
{स्नेहाद्राक्षसलक्ष्म्येव बुद्धिमाविश्य चोदितः} % १२-६८

\twolineshloka
{तस्मै निशाचरैश्वर्यं प्रतिशुश्राव राघवः}
{काले खलु समालब्धाः फलं बध्नन्ति नीतयः} % १२-६९

\twolineshloka
{स सेतुं बन्धयामास प्लवगैर्लवणाम्भसि}
{रसातलादिवोन्मग्नं शेषं स्वप्नाय शार्ङ्गिणः} % १२-७०

\twolineshloka
{तेनोत्तीर्य पथा लङ्कां रोधयामास पिङ्गलैः}
{द्वितीयं हेमप्राकारं कुर्वद्भिरिव वानरैः} % १२-७१

\twolineshloka
{रणः प्रववृते तत्र भीमः प्लवग रक्षसाम्}
{दिग्विजृम्भितकाकुत्स्थपौलस्त्यजयघोषणः} % १२-७२

\twolineshloka
{पादपाविद्धपरिघः शिलानिष्पिष्टमुद्गरः}
{अतिशस्त्रनखन्यासः शैलरुग्णमतङ्गजः} % १२-७३

\twolineshloka
{अथ रामशिरश्छेददर्शनोद्भ्रान्तचेतनाम्}
{सीतां मायेति शंसन्ती त्रिजटा समजीवयत्} % १२-७४

\twolineshloka
{कामं जीवति मे नाथ इति सा विजहौ शुचम्}
{प्राङ्मत्वा सत्यमस्यान्तं जीवितास्मीति लज्जिता} % १२-७५

\twolineshloka
{गरुडापातविश्लिष्टमेघनादास्त्रबन्धनः}
{दाशरथ्योः क्षणक्लेशः स्वप्नवृत्त इवाभवत्} % १२-७६

\twolineshloka
{ततो बिभेद पौलस्त्यः शक्त्या वक्षसि लक्ष्मणम्}
{रामस्त्वनाहतोऽप्यासीद्विदीर्णहॄदयः शुचा} % १२-७७

\twolineshloka
{स मारुतिसमानीतमहौषधिहतव्यथः}
{लङ्कास्त्रीणां पुनश्चक्रे विलापाचार्यकं शरैः} % १२-७८

\twolineshloka
{स नादं मेघनादस्य धनुश्चेन्द्रायुधप्रभम्}
{मेघस्येव शरत्कालो न किञ्चित्पर्यशेषयत्} % १२-७९

\twolineshloka
{कुम्भकर्णः कपीन्द्रेण तुल्यावस्थः स्वसुः कृतः}
{रुरोध रामं शृङ्गीव टङ्कच्छिन्नमनःशिलः} % १२-८०

\twolineshloka
{अकाले बोधितो भ्राता प्रियस्वप्नो वृथा भवान्}
{रामेषुभिरितीवासौ दीर्घनिद्रां प्रवेशितः} % १२-८१

\twolineshloka
{इतराण्यपि रक्षांसि पेतुर्वानरकोटिषु}
{रजांसि समरोत्थानि तच्छोणितनदीष्विव} % १२-८२

\twolineshloka
{निर्ययावथ पौलस्त्यः पुनर्युद्धाय मन्दिरात्}
{अरावणमरामं वा जगदद्येति निश्चितः} % १२-८३

\twolineshloka
{रामं पदातिमालोक्य लङ्केशं च वरूथिनम्}
{हरियुग्यं रथं तस्मै प्रजिघाय पुरन्दरः} % १२-८४

\twolineshloka
{तमाधूतध्वजपटं व्योमगङ्गोर्मिवायुभिः}
{देवसूतभुजालम्बी जैत्रमध्यास्त राघवः} % १२-८५

\twolineshloka
{मातलिस्तस्य माहेन्द्रमामुमोच तनुच्छदम्}
{यत्रोत्पलदलक्लैब्यमस्त्राण्यापुः सुरद्विषाम्} % १२-८६

\twolineshloka
{अन्योन्यदर्शनप्राप्तविक्रमावसरं चिरात्}
{रामरावणयोर्युद्धं चरितार्थमिवाभवत्} % १२-८७

\twolineshloka
{भुजमूर्धोरुबाहुल्यादेकोऽपि धनदानुजः}
{ददृशे ह्ययथापूर्वो मातृवंश इव स्थितः} % १२-८८

\twolineshloka
{जेतारं लोकपालानां स्वमुखैरर्चितेश्वरम्}
{रामस्तुलितकैलासमारातिं बह्वमन्यत} % १२-८९

\twolineshloka
{तस्य स्फुरति पौलस्त्यः सीतासङ्गमशंसिनि}
{निचखानाधिकक्रोधः शरं सव्येतरे भुजे} % १२-९०

\twolineshloka
{रावणस्यापि रामास्तो भित्त्वा हृदयमाशुगः}
{विवेश भुवमाख्यातुमुरगेभ्य इव प्रियम्} % १२-९१

\twolineshloka
{वचसैव तयोर्वाक्यमस्त्रमस्त्रेण निघ्नतोः}
{अन्योन्यजयसंरम्भो ववृधे वादिनोरिव} % १२-९२

\twolineshloka
{विक्रमव्यतिहारेण सामान्याऽभूद्द्वयोरपि}
{जयश्रीरन्तरा वेदिर्मत्तवारणयोरिव} % १२-९३

\twolineshloka
{कृतप्रतिकृतप्रीतैस्तयोर्मुक्तां सुरासुरैः}
{परस्परशरव्राताः पुष्पवृष्टिं न सेहिरे} % १२-९४

\twolineshloka
{अयःशङ्कुचितां रक्षः शतघ्नीमथ शत्रवे}
{हृतां वैवस्वतस्येव कूटशाल्मलिमक्षिपत्} % १२-९५

\twolineshloka
{राघवो रथमप्राप्तां तामाशां च सुरद्विषाम्}
{अर्धचन्द्रमुखैर्बाणैश्चिच्छेद कदलीसुखम्} % १२-९६

\twolineshloka
{अमोघं सन्दधे चास्मै धनुष्येकधनुर्धरः}
{ब्राह्ममस्त्रं प्रियाशोकशल्यनिष्कर्षणौषधम्} % १२-९७

\twolineshloka
{तद्व्योम्नि शतधा भिन्नं ददृशे दीप्तिमन्मुखम्}
{वपुर्महोरगस्येव करालफलमण्डलम्} % १२-९८

\twolineshloka
{तेन मन्त्रप्रयुक्तेन निमेषार्धादपातयत्}
{स रावणशिरःपङ्क्तिमज्ञातव्रणवेदनाम्} % १२-९९

\twolineshloka
{बालार्कप्रतिमेवाप्सु वीचिभिन्ना पतिष्यतः}
{रराज रक्षःकायस्य कण्ठच्छेदपरम्परा} % १२-१००

\twolineshloka
{मरुतां पश्यतां तस्य शिरांसि पतितान्यपि}
{मनो नातिविशस्वास पुनःसन्धानशङ्किनाम्} % १२-१०१

\fourlineindentedshloka
{अथ मदगुरुपक्षैलोकपालद्विपाना}
{मनुगतमलिवृन्दैर्गण्डभित्तीर्विहाय}
{उपनतमणिबन्धे मूर्ध्नि पौलस्त्यशत्रोः}
{सुरभि सुरविमुक्तं पुष्पवर्षं पपात} % १२-१०२

\fourlineindentedshloka
{यन्ता हरेः सपदि संहृतकार्मुकज्य}
{मापृच्छ्य राघवमनुष्ठितदेवकार्यम्}
{नामाङ्करावणशराङ्कितकेतुयष्टि}
{मूर्ध्वं रथं हरिसहस्रयुजं निनाय} % १२-१०३

\fourlineindentedshloka
{रघुपतिरपि जातवेदो विशुद्धां प्रगृह्य प्रियां}
{प्रियसुहृदि बिभीषणे सङ्गमय्य श्रियं वैरिणः}
{रविसुतसहितेन तेनानुयातः ससौमित्रिणा}
{भुजविजितविमानरत्नाधिरूढः प्रतस्थे पुरीम्} % १२-१०४

॥इति श्री-महाकवि-कालिदास-कृत-रघुवंश-महाकाव्ये द्वादशः सर्गः॥
\sect{त्रयोदशः सर्गः}

\fourlineindentedshloka
{अथात्मनः शब्दगुणम्गुणज्ञ्यः}
{पदम्विमानेन विगाहमानः}
{रत्नाकरम्वीक्ष्य मिथः स जायाम्}
{रामाभिधानो हरिरित्युवाच} % १३-१

\fourlineindentedshloka
{वैदेहि पश्यऽऽमलयाद्विभक्तम्}
{मत्सेतुना फेनिलमम्बुराशिम्}
{छायापथेनेव शरत्प्रसन्नम्}
{आकाशमाविष्कृतचारुतारम्} % १३-२

\fourlineindentedshloka
{गुरोर्यियक्षोः कपिलेन मेध्ये}
{रसातलम्सङ्क्रमिते तुरङ्गे}
{तदर्थमुर्वीमवधारयद्भिः}
{पूर्वैः किलायम्परिवर्धितो नः} % १३-३

\fourlineindentedshloka
{गर्भम्दधत्यर्कमरीचयोऽस्मात्}
{विवृद्धिमत्राश्नुवते वसूनि}
{अबिन्धनम्वह्निमसौ बिभर्ति}
{प्रह्लादनम्ज्योतिरजन्यनेन} % १३-४

\fourlineindentedshloka
{ताम्तामवस्थाम्प्रतिपद्यमानम्}
{स्थितम्दश व्याप दिशो महिम्ना}
{विष्णोरिवास्यानवधारणीयम्}
{ईदृक्तया रूपमियत्ताया वा} % १३-५

\fourlineindentedshloka
{नाभिप्ररूढाम्बुरुहासनेन}
{संस्तूयमानः प्रथमेन धात्रा}
{अमुं युगान्तोचितयोगनिद्रः}
{संहृत्य लोकान्पुरुषोऽधिशेते} % १३-६

\fourlineindentedshloka
{पक्षच्छिदा गोत्रभिदात्तगन्धाः}
{शरण्यमेनं शतशो महीध्राः}
{नृपा इवोपप्लविनः परेभ्यो}
{धर्मोत्तरम्मध्यममाश्रयन्ते} % १३-७

\fourlineindentedshloka
{रसातलादादिभवेन पुंसा}
{भुवः प्रयुक्तोद्वहनक्रियायाः}
{अस्याच्छमम्भः प्रलयप्रवृद्धं}
{मुहूर्तवक्त्राभरणं बभूव} % १३-८

\fourlineindentedshloka
{मुखार्पणेषु प्रकृतिप्रगल्भाः}
{स्वयम्तरङ्गाधरदानदक्षः}
{अनन्यसामान्यकलत्रवृत्तिः}
{पिबत्यसौ पाययते च सिन्धूः} % १३-९

\fourlineindentedshloka
{ससत्वमादाय नदीमुखाम्भः}
{सम्मीलयन्तो विवृताननत्वात्}
{अमी तिरोभिस्तिमयः सरन्ध्रैः}
{ऊध्वम्वितन्वन्ति जलप्रवाहान्} % १३-१०

\fourlineindentedshloka
{मातङ्गनक्रैः सहसोत्पतद्भिर्ः}
{भिन्नान्द्विधा पश्य समुद्रफेनान्}
{कपोलसंसर्पितया य एषाम्}
{व्रजन्ति कर्णक्षणचामरत्वम्} % १३-११

\fourlineindentedshloka
{वेलानिलाय प्रसृता भुजङ्गा}
{वहोर्मिविस्फूर्जथुनिर्विशेषाः}
{सूर्यांशुसम्पर्कविवृद्धरागैः}
{व्यज्यन्त एते मणिभिः फणस्थैः} % १३-१२

\fourlineindentedshloka
{तवाधरस्पर्धिषु विद्रुमेषु}
{पर्यस्तमेतत्सहसोर्मिवेगात्}
{ऊर्ध्वाङ्कुरप्रोतमुखम्कथञ्चित्}
{क्लेशादपक्रामति शङ्खयूथम्} % १३-१३

\fourlineindentedshloka
{प्रवृत्तमात्रेण पयांसि पातुम्}
{आवर्तवेगाद्भ्रमता घनेन}
{आभाति भूयिष्ठमयम्समुद्रः}
{प्रमथ्यमानो गिरिणेव भूयः} % १३-१४

\fourlineindentedshloka
{दूरादयश्चक्रनिभस्य तन्वी}
{तमालतालीवनराजिनीला}
{आभाति वेला लवणाम्बुराशेः}
{धारानिबद्धेव कलङ्करेखा} % १३-१५

\fourlineindentedshloka
{वेलानिलः केतकरेणुभिस्ते}
{सम्भावयत्याननमायताक्षि}
{मामक्षमम्मण्डनकालहानेः}
{वेत्तीव बिम्बाधरबद्धतृष्णम्} % १३-१६

\fourlineindentedshloka
{एते वयम्सैकतभिन्नशुक्ति}
{पर्यस्तमुक्तापटलम्पयोधेः}
{प्राप्ता मुहूर्तेन विमानवेगात्}
{कूलम्फलावर्जितपूगमालम्} % १३-१७

\fourlineindentedshloka
{कुरुष्व तावत्करभोरु पश्चात्}
{मार्गे मृगप्रेक्षिणि दृष्टिपातम्}
{एषा विदूरीभवतः समुद्रात्}
{सकानना निष्पततीव भूमिः} % १३-१८

\fourlineindentedshloka
{क्वचित्पथा सञ्चरते सुराणाम्}
{क्वचिद् घनानाम्पतताम्क्वचिच्च}
{यथाविधो मे मनसोऽभिलाषः}
{प्रवर्तते पश्य तथा विमानम्} % १३-१९

\fourlineindentedshloka
{असौ महेन्द्रद्विपदानगन्धि}
{स्त्रिमार्गगावीचिविमर्दशीतः}
{आकाशवायुर्दिनयौवनोत्थान्}
{आचामति स्वेदलवान्मुखे ते} % १३-२०

\fourlineindentedshloka
{करेण वातायनलम्बितेन}
{स्पृष्टस्त्वया चण्डि कुतूहलिन्या}
{आमुञ्चतीवाभरणम्द्वितीयम्}
{उद्भिन्नविद्युद्वलयो घनस्ते} % १३-२१

\fourlineindentedshloka
{अमी जनस्थानमपोढविघ्नम्}
{मत्वा समारब्धनवोटजानि}
{अध्यासते चीरभृतो यथास्वम्}
{चिरोज्झितान्याश्रममण्डलानि} % १३-२२

\fourlineindentedshloka
{सैषा स्थली यत्र विचिन्वता त्वां}
{भ्रष्टं मया नूपुरमेकमुर्व्याम्}
{अदृश्यत त्वच्चरणारविन्द-}
{विश्लेषदुःखादिव बद्धमौनम्} % १३-२३

\fourlineindentedshloka
{त्वं रक्षसा भीरु यतोऽपनीता}
{तं मार्गमेताः कृपया लता मे}
{अदर्शयन्वक्तुमशक्नुवत्यः}
{शाखाभिरावर्जितपल्लवाभिः} % १३-२४

\fourlineindentedshloka
{मृग्यश्च दर्भाङ्कुरनिर्व्यपेक्षाः}
{तवागतिज्ञ्यम्समबोधयन्माम्}
{व्यापारयन्त्यो दिशि दक्षिणस्याम्}
{उत्पक्षराजीनि विलोचनानि} % १३-२५

\fourlineindentedshloka
{एतद्गिरेर्माल्यवतः पुरस्तात्}
{आविर्भत्यम्बरलेखि शृङ्गम्}
{नवम्पयो यत्र घनैर्मया च}
{त्वद्विप्रयोगाश्रु समम्विसृष्टम्} % १३-२६

\fourlineindentedshloka
{गन्धश्च धाराहतपल्वलानाम्}
{कादम्बमर्धोद्गतकेसरम्च}
{स्निग्धाश्च केकाः शिखिनाम्बभूवुः}
{यस्मिन्नसह्यानि विना त्वया मे} % १३-२७

\fourlineindentedshloka
{पूर्वानुभूतम्स्मरता च यत्र}
{कम्पोत्तरम्भीरु तवोपगूढम्}
{गुहाविसारीण्यतिवाहितानि}
{मया कथञ्चिद् घनगर्जितानि} % १३-२८

\fourlineindentedshloka
{आसारसिक्तक्षितिबाष्पयोगात्}
{मामक्षिणोद्यत्र विभिन्नकोशैः}
{विडम्ब्यमाना नवकन्दलैस्ते}
{विवाहधूमारुणलोचनश्रीः} % १३-२९

\fourlineindentedshloka
{उपान्तवानीरवनोपगूढानि}
{आलक्ष्यपारिप्लवसारसानि}
{दूरावतीर्णा पिबतीव खेदात्}
{अमूनि पम्पासलिलानि दृष्टिः} % १३-३०

\fourlineindentedshloka
{अत्रावियुक्तानि रथङ्गनाम्ना}
{मन्योन्यदत्तोत्पलकेसराणि}
{द्वन्द्वानि दूरान्तरवर्तिना ते}
{मया प्रिये सस्मितमीक्षितानि} % १३-३१

\fourlineindentedshloka
{इमाम्तटाशोकलताम्च तन्वीम्}
{स्तनाभिरामस्तबकाभिनम्राम्}
{त्वत्प्राप्तिबुद्ध्या परिरब्धुकामः}
{सौमित्रिणा साश्रुरहम्निषिद्धः} % १३-३२

\fourlineindentedshloka
{अमूर्विमानान्तरलम्बिनीनाम्}
{श्रुत्वा स्वनम्काञ्चनकिङ्किणीनाम्}
{प्रत्युद्व्रजन्तीव खमुत्पतन्त्यो}
{गोदावरीसारसपङ्क्तयस्त्वाम्} % १३-३३

\fourlineindentedshloka
{एषा त्वया पेशलमध्ययापि}
{घटाम्बुसंवर्धितबालचूता}
{आनन्दयत्युन्मुखकृष्णसारा}
{दृष्टा चिरात्पञ्चवटी मनो मे} % १३-३४

\fourlineindentedshloka
{अत्रानुगोदम्मृगयानिवृत्तः}
{तरङ्गवातेन विनीतखेदः}
{रहस्त्वदुत्सङ्गनिषण्णमूर्धा}
{स्मरामि वानीरगृहेषु सुप्तः} % १३-३५

\fourlineindentedshloka
{भ्रूभेदमात्रेण पदान्मघोनः}
{प्रभ्रंशयां यो नहुषं चकार}
{तस्याविलाम्भःपरिशुद्धिहेतोः}
{भौमो मुनेः स्थानपरिग्रहोऽयम्} % १३-३६

\fourlineindentedshloka
{त्रेताग्निधूमाग्रमनिन्द्यकीर्तेः}
{तस्येदमाक्रान्तविमानमार्गम्}
{घ्रात्वा हविर्गन्धि रजोविमुक्तः}
{समश्नुते मे लघिमानमात्मा} % १३-३७

\fourlineindentedshloka
{एतन्मुमुनेर्मानिनि शातकर्णेः}
{पञ्चाप्सरो नाम विहारवारि}
{आभाति पर्यन्तवनम्विदूरात्}
{मेघान्तरालक्ष्यमिवेन्दुबिम्बम्} % १३-३८

\fourlineindentedshloka
{पुरा स दर्भाङ्कुरमात्रवृत्तिः}
{चरन्मृगैः सार्धमृषिर्मघोना}
{समाधिभीतेन किलोपनीतः}
{पञ्चाप्सरोयौवनकूटबन्धम्} % १३-३९

\fourlineindentedshloka
{तस्यायमन्तर्हितसौधभाजः}
{प्रसक्तसङ्गीतमृदङ्गघोषः}
{वियद्गतः पुष्पकचन्द्रशालाः}
{क्षणम्प्रतिश्रुन्मुखराः करोति} % १३-४०

\fourlineindentedshloka
{हविर्भुजामेधवताम्चतुर्णाम्}
{मध्ये ललाटन्तपसप्तसप्तिः}
{असौ तपस्यत्यपरस्तपस्वी}
{नाम्ना सुतीक्ष्णश्चरितेन दान्तः} % १३-४१

\fourlineindentedshloka
{अमुम्सहासप्रहितेक्षणानि}
{व्याजार्धसन्दर्शितमेखलानि}
{नालम्विकर्तुम्जनितेन्द्रशङ्कम्}
{सुराङ्गनाविभ्रमचेष्टितानि} % १३-४२

\fourlineindentedshloka
{एषोऽक्षमालावलयम्मृगाणाम्}
{कण्डूयितारम्कुशसूचिलावम्}
{सभाजने मे भुजमूर्ध्वबाहुः}
{सव्येतरम्प्राध्वमितः प्रयुङ्क्ते} % १३-४३

\fourlineindentedshloka
{वाचंयमत्वात्प्रणतिम्ममैष}
{कम्पेन किञ्चित्प्रतिगृह्य मूर्ध्नः}
{दृष्टिम्विमानव्यवधानमुक्ताम्}
{पुनः सहस्रार्चिषि सन्निधत्ते} % १३-४४

\fourlineindentedshloka
{अदः शरण्यंशरभङ्गनाम्नः}
{तपोवनम्पावनमाहिताग्नेः}
{चिराय सन्तर्प्य समिद्भिरग्निम्}
{यो मन्त्रपूताम्तनुमप्यहौषीत्} % १३-४५

\fourlineindentedshloka
{छायाविनीताध्वपरिश्रमेषु}
{भूयिष्ठसम्भाव्यफलेष्वमीषु}
{तस्यातिथीनामधुनासपर्या}
{स्थिता सुपुत्रेष्विव पादपेषु} % १३-४६

\fourlineindentedshloka
{धारास्वनोद्गारिदरीमुखोऽसौ}
{शृङ्गाङ्गलग्नाम्बुजवप्रपङ्कः}
{बध्नाति मे बन्धुरगात्रि चक्षुः}
{दृप्तः ककुद्मानिव चित्रकूटः} % १३-४७

\fourlineindentedshloka
{एषा प्रसन्नस्तिमितप्रवाहा}
{सरिद्विदूरान्तरभावतन्वी}
{मन्दाकिनी भाति नगोपकण्ठे}
{मुक्तावली कण्ठगतेव भूमेः} % १३-४८

\fourlineindentedshloka
{अयम्सुजातोऽनुगिरम्तमालः}
{प्रवालमादाय सुगन्धि यस्य}
{यवाङ्कुरापाण्डुकपोलशोभी}
{मयावतंसः परिकल्पितस्ते} % १३-४९

\fourlineindentedshloka
{अनिग्रहत्रासविनीतसत्त्व}
{मपुष्पलिङ्गात्फलबन्धिवृक्षम्}
{वनम्तपःसाधनमेतदत्रेः}
{आविष्कृतोदग्रतरप्रभावम्} % १३-५०

\fourlineindentedshloka
{अत्राभिषेकाय तपोधनानाम्}
{सप्तर्षिहस्तोद्धृतहेमपद्माम्}
{प्रवर्तयामास किलानसूया}
{त्रिस्रोतसम्त्र्यम्बकमौलिमालाम्} % १३-५१

\fourlineindentedshloka
{वीरासनैर्ध्यानजुषामृषीणाम्}
{अमी समध्यासितवेदिमध्याः}
{निवातनिष्कम्पतया विभान्ति}
{योगाधिरूढा इव शाखिनोऽपि} % १३-५२

\fourlineindentedshloka
{त्वया पुरस्तादुपयाचितो यः}
{सोऽयम्वटः श्याम इति प्रतीतः}
{राशिर्मणीनामिव गारुडानाम्}
{सपद्मरागः फलितो विभाति} % १३-५३

\fourlineindentedshloka
{क्वचित्प्रभालेपिभिरिन्द्रनीलैः}
{मुक्तामयी यष्टिरिवानुविद्धा}
{अन्यत्र माला सितपङ्कजानाम्}
{इन्दीवरैरुत्खचितान्तरेव} % १३-५४

\fourlineindentedshloka
{क्वचित्खगानाम्प्रियमानसानाम्}
{कादम्बसंसर्गवतीव पङ्क्तिः}
{अन्यत्र कालागुरुदतापत्रा}
{भक्तिर्भुवश्चन्दनकल्पितेव} % १३-५५

\fourlineindentedshloka
{क्वचित्प्रभा चान्द्रमसी}
{तमोभिश्छायाविलीनैः शबलीकृतेव}
{अन्यत्र शुभ्रा शरदभ्रलेखा}
{रन्ध्रेष्विवालक्ष्यनभःप्रदेशा} % १३-५६

\fourlineindentedshloka
{क्वचिच्च कृष्णोरगभूषणेव}
{भस्माङ्गरागा तनुरीश्वरस्य}
{पश्यानवद्याङ्गि विभाति गङ्गा}
{भिन्नप्रवाहा यमुनातरङ्गैः} % १३-५७

\fourlineindentedshloka
{समुद्रपत्योर्जलसन्निपाते}
{पूतात्मनामत्र किलाभिषेकात्}
{तत्त्वावबोधेन विनापि भूयः}
{तनुत्यजाम्नास्ति शरीरबन्धः} % १३-५८

\fourlineindentedshloka
{पुरम्निषादाधिपतेरिदम्तत्}
{यस्मिन्मया मौलिमणिम्विहाय}
{जटासु बद्धास्वरुदत्सुमन्त्रः}
{कैकेयि कामाः फलितास्तवेति} % १३-५९

\twolineshloka
{पयोधरैः पुण्यजनाङ्गनानाम्निर्विष्टहेमाम्बुजरेणु यस्याः}
{ब्राह्मम्सरः कारणमाप्तवाचो बुद्धेरिवाव्यक्तमुदाहरन्ति} % १३-६०

\fourlineindentedshloka
{जलानि या तीरनिखातयूपा}
{वहत्ययोध्यामनु राजधानीम्}
{तुरङ्गमेधावभृथावतीर्णैः}
{इक्ष्वाकुभिः पुण्यतरीकृतानि} % १३-६१

\fourlineindentedshloka
{याम्सैकतोत्सङ्गसुखोचितानाम्}
{प्राज्यैः पयोभिः परिवर्धितानाम्}
{सामान्यधात्रीमिव मानसम्मे}
{सम्भावयत्युत्तरकोसलानाम्} % १३-६२

\fourlineindentedshloka
{सेयम्मदीया जननीव तेन}
{मान्येन राज्ञ्या सरयूर्वियुक्ता}
{दूरे वसन्तंशिशिरानिलैर्माम्}
{तरङ्गहस्तैरुपगूहतीव} % १३-६३

\fourlineindentedshloka
{विरक्तसन्ध्याकपिशम्परस्तात्}
{यतो रजः पार्थिवमुज्जिहीते}
{शङ्के हनूमत्कथितप्रवृत्तिः}
{प्रत्युद्गतो माम्भरतः ससैन्यः} % १३-६४

\fourlineindentedshloka
{अद्धा श्रियम्पालितसङ्गराय}
{प्रत्यर्पयिष्यत्यनघाम्स साधुः}
{हत्वा निवृत्ताय मृधे खरादीन्}
{संरक्षिताम्त्वामिव लक्ष्मणो मे} % १३-६५

\fourlineindentedshloka
{असौ पुरस्कृत्य गुरुम्पदातिः}
{पश्चादवस्थापितवाहिनीकः}
{वृद्धैरमात्यैः सह चीरवासा}
{मामर्घ्यपाणिर्भरतोऽप्युपैति} % १३-६६

\fourlineindentedshloka
{पित्रा विसृष्टाम्मदपेक्षया यः}
{श्रियम्युवाप्यङ्कगतामभोक्ता}
{इयन्ति वर्षाणि तया सहोग्रम्}
{अभ्यस्यतीव व्रतमासिधारम्} % १३-६७

\fourlineindentedshloka
{एतावदुक्तवति दाशरथौ तदीया}
{मिच्छाम्विमानमधिदेवतया विदित्वा}
{ज्योतिष्पथादवततार सविस्मयाभि}
{रुद्वीक्षितम्प्रकृतिभिर्भरतानुगाभिः} % १३-६८

\fourlineindentedshloka
{तस्मात्पुरःसरबिभीषणदर्शनेन}
{सेवाविचक्षणहरीश्वरदत्तहस्तः}
{यानादवातरददूरमहीतलेन}
{मार्गेण भङ्गिरचितस्फटिकेन रामः} % १३-६९

\fourlineindentedshloka
{इक्ष्वाकुवंशगुरवे प्रयतः प्रणम्य}
{स भ्रातरम्भरतमर्घ्यपरिग्रहान्ते}
{पर्यश्रुरस्वजत मूर्धनि चोपजघ्रौ}
{तद्भक्त्यपोढपितृराज्यमहाभिषेके} % १३-७०

\fourlineindentedshloka
{श्मश्रुप्रवृद्धिजनिताननविक्रियांश्च}
{प्लक्षान्प्ररोहजटिलानिव मन्त्रिवृद्धान्}
{अन्वग्रहीत्प्रणमतः शुभदृष्टिपातै}
{र्वातानुयोगमधुराक्षरया च वाचा} % १३-७१

\fourlineindentedshloka
{दुर्जातबन्धुरयमृक्षहरीश्वरो मे}
{पौलस्त्य एष समरेषु पुरःप्रहर्ता}
{इत्यादृतेन कथितौ रघुनन्दनेन}
{व्युत्क्रम्य लक्ष्मणमुभौ भरतो ववन्दे} % १३-७२

\fourlineindentedshloka
{सौमित्रिणा तदनु संससृजे स चैन}
{मुत्थाप्य नम्रशिरसम्भृशमालिलिङ्ग}
{रूढेन्द्रजित्प्रहरणव्रणकर्कशेन}
{क्लिश्यन्निवास्य भुजमध्यमुरस्थलेन} % १३-७३

\fourlineindentedshloka
{रामाज्ञ्यया हरिचमूपतयस्तदानीम्}
{कृत्वा मनुष्यवपुरारुरुहुर्गजेन्द्रान्}
{तेषु क्षरत्सु बहुधा मदवारिधाराः}
{शैलाधिरोहणसुखान्युपलेभिरेते} % १३-७४

\fourlineindentedshloka
{सानुप्लवः प्रभुरपि क्षणदाचराणाम्}
{भेजे रथान्दशरथप्रभवानुशिष्टः}
{मायाविकल्परचितैरपि ये तदीयै}
{र्न स्यन्दनैस्तुलितकृत्रिमभक्तिशोभाः} % १३-७५

\fourlineindentedshloka
{भूयस्ततो रघुपतिर्विलसत्पताक}
{मध्यास्त कामगति सावरजो विमानम्}
{दोषातनम्बुधबृहस्पतियोगदृश्य}
{स्तारापतिस्तरलविद्युदिवाभ्रवृन्दम्} % १३-७६

\fourlineindentedshloka
{तत्रेश्वरेण जगताम्प्रलयादिवोर्वीं}
{वर्षात्ययेन रुचमभ्रघनादिवेन्दोः}
{रामेण मैथिलसुताम्दशकण्ठकृच्छ्रा}
{त्प्रत्युद्धृताम्धृतिमतीम्भरतो ववन्दे} % १३-७७

\fourlineindentedshloka
{लङ्केश्वरप्रणतिभङ्गदृढव्रतम्त}
{द्वन्द्यम्युगम्चरणयोर्जनकात्मजायाः}
{जेष्ठानुवृत्तिजटिलम्च शिरोऽस्य साधो}
{रन्योन्यपावनमभूदुभयम्समेत्य} % १३-७८

\fourlineindentedshloka
{क्रोशार्धम्प्रकृतिपुरःसरेण गत्वा}
{काकुत्स्थः स्तिमितजवेन पुष्पकेण}
{शत्रुघ्नप्रतिविहितोपकार्यमार्यः}
{साकेतोपवनमुदारमध्युवास} % १३-७९

॥इति श्री-महाकवि-कालिदास-कृत-रघुवंश-महाकाव्ये त्रयोदशः सर्गः॥
\sect{चतुर्दशः सर्गः}

\fourlineindentedshloka
{भर्तुः प्रणाशादथ शोचनीयम्}
{दशान्तरं तत्र समं प्रपन्ने}
{अपश्यतां दाशरथी जनन्यौ}
{छेदादिवोपघ्नतरोर्व्रतत्यौ} % १४-१

\fourlineindentedshloka
{उभावु भाभ्यां प्रणतौ हतारी}
{यथाक्रमं विक्रमशोभिनौ तौ}
{विस्पष्टमस्रान्धतया न दृष्टौ}
{ज्ञातौ सुतस्पर्श सुखोपलम्भात्} % १४-२

\fourlineindentedshloka
{आनन्दजः शोकजमश्रुबाष्पः}
{तयोरशीतं शिशिरो बिभेद}
{गङ्गासरय्वोर्जलमुष्णतप्तम्}
{हिमाद्रिनिस्यन्द इवावतीर्णः} % १४-३

\fourlineindentedshloka
{ते पुत्रयोर्नैर्ऋतशस्त्रमार्गान्}
{आर्द्रानिवाङ्गे सदयं स्पृशन्तौ}
{अपीप्सितं क्षत्रकुलाङ्गनानाम्}
{न वीरसूशब्दमकामयेताम्} % १४-४

\fourlineindentedshloka
{क्लेशावहा भर्तुरलक्षणाहम्}
{सीतेति नाम स्वमुदीरयन्ती}
{स्वर्गप्रतिष्ठस्य गुरोर्महिष्या}
{वभक्तिभेदेन वधूर्ववन्दे} % १४-५

\fourlineindentedshloka
{उत्तिष्ठ वत्से ननु स अनुजोः असौ}
{वृत्तेन भर्ता शुचिना तवैव}
{कृच्छ्रं महत्तीर्ण इव प्रियार्हाम्}
{तामूचतुस्ते प्रियमप्यमिथ्या} % १४-६

\fourlineindentedshloka
{अथाभिषेकं रघुवंशकेतोः}
{प्रारब्धमानन्दजलैर्जनन्योः}
{निर्वर्तयामासुरमात्यवृद्धाः}
{तीर्थाहृतैः काञ्चनकुम्भतोयैः} % १४-७

\fourlineindentedshloka
{सरित्समुद्रान्सरसीश्च गत्वा}
{रक्षःकपीन्द्रैरुपपादितानि}
{तस्यापतन्मूर्ध्नि जलानि जिष्णोः}
{विन्ध्यस्य मेघप्रभवा इवापः} % १४-८

\fourlineindentedshloka
{तपस्विवेषक्रिययापि तावत्}
{यः प्रेक्षणीयः सुतरां बभूव}
{राजेन्द्रनेपथ्यविधानशोभा}
{तस्योदितासीत्पुनरुक्तदोषा} % १४-९

\fourlineindentedshloka
{स मौलरक्षोहरिभिः ससैन्यः}
{तूर्यस्वनानन्दितपौरवर्गः}
{विवेश सौधोद्गतलाजवर्षाम्}
{उत्तोरणामन्वयराजधानीम्} % १४-१०

\fourlineindentedshloka
{सौमित्रिणा सावरजेन मन्दम्}
{आधूतबालव्यजनो रथस्थः}
{धृतातपत्रो भरतेन साक्षात्}
{उपायसङ्घात इव प्रवृद्धः} % १४-११

\fourlineindentedshloka
{प्रासादकालागुरुधूमराजिः}
{तस्याः पुरो वायुवशेन भिन्ना}
{वनान्निवृत्तेन रघूत्तमेन}
{मुक्ता स्वयं वेणिरिवाबभासे} % १४-१२

\fourlineindentedshloka
{श्वश्रूजनानुष्ठितचारुवेषाम्}
{कर्णीरथस्थां रघुवीरपत्नीम्}
{प्रासादवातायनदृश्यबन्धैः}
{साकेतनार्योऽञ्जलिभिः प्रणेमुः} % १४-१३

\fourlineindentedshloka
{स्फुरत्प्रभामण्डनमानसूयम्}
{सा बिभ्रती शाश्वतमङ्गरागम्}
{रराज शुद्धेति पुनः स्वपुर्यै}
{सन्दर्शिता वह्निगतेव भर्त्रा} % १४-१४

\fourlineindentedshloka
{वेश्मानि रामः परिबर्हवन्ति}
{विश्राण्य सौहार्दनिधिः सुहृद्भ्यः}
{बाष्पायमानो बलिमन्निकेतम्}
{आलेख्यशेषस्य पितुर्विवेश} % १४-१५

\fourlineindentedshloka
{कृताञ्जलिस्तत्र यदम्ब सत्यात्}
{नाभ्रश्यत स्वर्गफलाद्गुरुर्नः}
{तच्चिन्त्यमानं सुकृतं तवेति}
{जहार लज्जां भरतस्य मातुः} % १४-१६

\fourlineindentedshloka
{तथैव सुग्रीवबिभीषणादीन्}
{उपाचरत्कृत्रिमसंविधाभिः}
{सङ्कल्पमात्रोदितसिद्धयस्ते}
{क्रान्ता यथा चेतसि विस्मयेन} % १४-१७

\fourlineindentedshloka
{सभाजनायोपगतान्स दिव्यान्}
{मुनीन्पुरस्कृत्य हतस्य शत्रोः}
{शुश्राव तेभ्यः प्रभवादि वृत्तम्}
{स्वविक्रमे गौरवमादधानम्} % १४-१८

\fourlineindentedshloka
{प्रतिप्रयातेषु तपोधनेषु}
{सुखादविज्ञ्यातगतार्धमासान्}
{सीतास्वहस्तोपहृताग्र्यपूजान्}
{रक्षःकपीन्द्रान्विससर्ज रामः} % १४-१९

\fourlineindentedshloka
{तच्चात्मचिन्तासुलभं विमानम्}
{हृतं सुरारेः सह जीवितेन}
{कैलासनाथोद्वहनाय भूयः}
{पुष्पं दिवः पुष्पकमन्वमंस्त} % १४-२०

\fourlineindentedshloka
{पितुर्नियोगाद्वनवासमेवम्}
{निस्तीर्य रामः प्रतिपन्नराज्यः}
{धर्मार्थकामेषु समां प्रपेदे}
{यथा तथैवावरजेषु वृत्तिम्} % १४-२१

\fourlineindentedshloka
{सर्वासु मातुष्वपि वत्सलत्वात्}
{स निर्विशेषप्रतिपत्तिरासीत्}
{षडाननापीतपयोधरासु}
{नेता चमूनामिव कृत्तिकासु} % १४-२२

\fourlineindentedshloka
{तेनार्थवान् लोभपराङ्मुखेन}
{तेन घ्नता विघ्नभयं क्रियावान्}
{तेनास लोकः पितृमान्विनेत्रा}
{तेनैव शोकापनुदेव पुत्री} % १४-२३

\fourlineindentedshloka
{स पौरकार्याणि समीक्ष्य काले}
{रेमे विदेहाधिपतेर्दुहित्रा}
{उपस्थितश्चारु वपुस्तदीयम्}
{कृत्वोपभोगोत्सुकयेव लक्ष्म्या} % १४-२४

\fourlineindentedshloka
{तयोर्यथाप्रार्थितमिन्द्रियार्थान्}
{आसेदुषोः सद्मसु चित्रवत्सु}
{प्राप्तानि दुःखान्यपि दण्डकेषु}
{सञ्चिन्त्यमानानि सुखान्यभूवन्} % १४-२५

\fourlineindentedshloka
{अथाधिकस्निग्धविलोचनेन}
{मुखेन सीता शरपाण्डुरेण}
{आनन्दयित्री परिणेतुरासीत्}
{अनक्षरव्यञ्जितदोहदेन} % १४-२६

\fourlineindentedshloka
{तामङ्कमारोप्य कृशाङ्गयष्टिम्}
{वर्णान्तराक्रान्तपयोधराग्राम्}
{विलज्जमानां रहसि प्रतीतः}
{पप्रच्छ रामां रमणोऽभिलाषम्} % १४-२७

\fourlineindentedshloka
{सा दष्टनीवारबलीनि हिंस्रैः}
{सम्बद्धवैखानसकन्यकानि}
{इयेष भूयः कुशवन्ति गन्तुम्}
{भागीरथीतीरतपोवनानि} % १४-२८

\fourlineindentedshloka
{तस्यै प्रतिश्रुत्य रघुप्रवीरः}
{तदीप्सितं पार्श्वचरानुयातः}
{आलोकयिष्यन्मुदितामयोध्याम्}
{प्रासादमभ्रंलिहमारुरोह} % १४-२९

\fourlineindentedshloka
{ऋद्धापणं राजपथं स पश्यन्}
{विगाह्यमानां सरयूं च नौभिः}
{विलासिभिश्चाध्युषितानि पौरैः}
{पुरोपकण्ठोपवनानि रेमे} % १४-३०

\fourlineindentedshloka
{स किंवदन्तीं वदतां पुरोगः}
{स्ववृत्तमुद्दिश्य विशुद्धवृत्तः}
{सर्पाधिराजोरुभुजोऽपसर्पम्}
{पप्रच्छ भद्रं विजितारिभद्रः} % १४-३१

\fourlineindentedshloka
{निर्बन्धपृष्टः स जगाद सर्वम्}
{स्तुवन्ति पौराश्चरितं त्वदीयम्}
{अन्यत्र रक्षोभवनोषितायाः}
{परिग्रहान्मानवदेव देव्याः} % १४-३२

\fourlineindentedshloka
{कलत्रनिन्दागुरुणा किलैवम्}
{अभ्याहतं कीर्तिविपर्ययेण}
{अयोघनेनाय इवाभितप्तम्}
{वैदेहिबन्धोर्हृदयं विदद्रे} % १४-३३

\fourlineindentedshloka
{किमात्मनिर्वादकथामुपेक्षे}
{जायामदोषामुत सन्त्यजामि}
{इत्येकपक्षाश्रयविक्लवत्वात्}
{आसीत्स दोलाचलचित्तवृत्तिः} % १४-३४

\fourlineindentedshloka
{निश्चित्य चानन्यनिवृत्ति वाच्यम्}
{त्यागेन पत्न्याः परिमार्ष्टुमैच्छत्}
{अपि स्वदेहात्किमुतेन्द्रियार्थात्}
{अशोधनानां हि यशो गरीयः} % १४-३५

\fourlineindentedshloka
{स सन्निपत्यावरजान्हतौजाः}
{तद्विक्रियादर्शनलुप्तहर्षान्}
{कौलीनमात्माश्रयमाचचक्षे}
{तेभ्यः पुनश्चेदमुवाच वाक्यम्} % १४-३६

\fourlineindentedshloka
{राजर्षिवंशस्य रविप्रसूतेः}
{उपस्थितः पश्यत कीदृशोऽयम्}
{मत्तः सदाचारशुचेः कलङ्कः}
{पयोदवातादिव दर्पणस्य} % १४-३७

\fourlineindentedshloka
{पौरेषु सोऽहं बहुलीभवन्तम्}
{अपां तरङ्गेष्विव तैलबिन्दुम्}
{सोढुं न तत्पूर्वमवर्णमीशे}
{आलानिकं स्थाणुरिव द्विपेन्द्रः} % १४-३८

\fourlineindentedshloka
{तस्यापनोदाय फलप्रवृत्तौ}
{उपस्थितायामपि निर्व्यपेक्षः}
{त्यक्षामि वैदेहसुतां पुरस्तात्}
{समुद्रनेमिं पितुराज्ञ्ययेव} % १४-३९

\fourlineindentedshloka
{अवैमि चैनामनघेति किन्तु}
{लोकापवादो बलवान्मतो मे}
{छाया हि भूमेः शशिनो मलत्वे-}
{नाऽऽरोपिता शुद्धिमतः प्रजाभिः} % १४-४०

\fourlineindentedshloka
{रक्षोवधान्तो न च मे प्रयासो}
{व्यर्थः स वैरप्रतिमोचनाय}
{अमर्षणः शोणितकाङ्क्षया किम्}
{पदा स्पृशन्तं दशति द्विजिह्वः} % १४-४१

\fourlineindentedshloka
{तदेष सर्गः करुणार्द्रचित्तैः}
{न मे भवद्भिः प्रतिषेधनीयः}
{यद्यर्थिता निर्हृतवाच्यशल्यान्}
{प्राणान्मया धारयितुं चिरं वः} % १४-४२

\fourlineindentedshloka
{इत्युक्तवन्तं जनकात्मजायाम्}
{नितान्तरूक्षाभिनिवेशमीशम्}
{न कश्चन भ्रातृषु तेषु शक्तो}
{निषेद्धुमासीदनुमोदितुं वा} % १४-४३

\fourlineindentedshloka
{स लक्ष्मणं लक्ष्मणपूर्वजन्मा}
{विलोक्य लोकत्रयगीतकीर्तिः}
{सौम्येति चाभाष्य यथार्थभाषी}
{स्थितं निदेशे पृथगादिदेश} % १४-४४

\fourlineindentedshloka
{प्रजावती दोहदशंसिनी ते}
{तपोवनेषु स्पृहयालुरेव}
{स त्वं रथी तद्व्यपदेशनेयाम्}
{प्रापय्य वाल्मीकिपदं त्यजैनाम्} % १४-४५

\fourlineindentedshloka
{स शुश्रुवान्मातरि भार्गवेन}
{पितुर्नियोगात्प्रहृतं द्विषद्वत्}
{प्रत्यग्रहीदग्रजशासनं तत्}
{आज्ञ्या गुरूणां ह्यविचारणीया} % १४-४६

\fourlineindentedshloka
{अथानुकूलश्रवणप्रतीता}
{मत्रस्नुभिर्युक्तधुरं तुरङ्गैः}
{रथं सुमन्त्रप्रतिपन्नरश्मिम्}
{आरोप्य वैदेहसुतां प्रतस्थे} % १४-४७

\fourlineindentedshloka
{सा नीयमाना रुचिरान्प्रदेशान्}
{प्रियङ्करो मे प्रिय इत्यनन्दत्}
{नाबुद्ध कल्पद्रुमतां विहाय}
{जातं तमात्मन्यसिपत्रवृक्षम्} % १४-४८

\fourlineindentedshloka
{जुगूह तस्याः पथि लक्ष्मणो यत्}
{सव्येतरेण स्फुरता तदक्ष्णा}
{आख्यातमस्यै गुरु भावि दुःखम्}
{अत्यन्तलुप्तप्रियदर्शनेन} % १४-४९

\fourlineindentedshloka
{सा दुर्निमित्तोपगताद्विषादात्}
{सद्यःपरिम्लानमुखारविन्दा}
{राज्ञ्यः शिवं सावरजस्य भूयादित्य्}
{आशशंसे करणैरबाह्यैः} % १४-५०

\fourlineindentedshloka
{गुरोर्नियोगाद्वनितां वनान्ते}
{साध्वीं सुमित्रातनयो विहास्यन्}
{अवार्यतेवोत्थितवीचिहस्तैः}
{जह्नोर्दुहित्रा स्थितया पुरस्तात्} % १४-५१

\fourlineindentedshloka
{रथात्स यन्त्रा निगृहीतवाहात्}
{तां भ्रातृजायां पुलिनेऽवतार्य}
{गङ्गां निषादाहृतनौविशेषः}
{ततार सन्धामिव सत्यसन्धः} % १४-५२

\fourlineindentedshloka
{अथ व्यवस्थापितवाक्कथञ्चित्}
{सौमित्रिरन्तर्गतबाष्पकण्ठः}
{औत्पातिकं मेघ इवाश्मवर्षम्}
{महीपतेः शासनमुज्जगार} % १४-५३

\fourlineindentedshloka
{ततोऽभिषङ्गानिलविप्रविद्धा}
{प्रभ्रश्यमानाभरणप्रसूना}
{स्वमूर्तिलाभप्रकृतिं धरित्रीम्}
{लतेव सीता सहसा जगाम} % १४-५४

\fourlineindentedshloka
{इक्ष्वाकुवंशप्रभवः कथं त्वाम्}
{त्यजेदकस्मात्पतिरार्यवृत्तः}
{इति क्षितिः संशयितेव तस्यै}
{ददौ प्रवेशं जननी न तावत्} % १४-५५

\fourlineindentedshloka
{सा लुप्तसञ्ज्ञ्या न विवेद दुःखम्}
{प्रत्यागतासुः समतप्यतान्तः}
{तस्याः सुमित्रात्मजयत्नलब्धो}
{मोहादभूत्कष्टतरः प्रबोधः} % १४-५६

\fourlineindentedshloka
{न चावदद्भर्तुरवर्णमार्या}
{निराकरिष्णोर्वृजिनादृतेऽपि}
{आत्मानमेवं स्थिरदुःखभाजम्}
{पुनः पुनर्दुष्कृतिनं निनिन्द} % १४-५७

\fourlineindentedshloka
{आश्वास्य रामावरजः सतीं ताम्}
{आख्यातवाल्मीकिनिकेतमार्गः}
{निघ्नस्य मे भर्तृनिदेशरौक्ष्यम्}
{देवि क्षमस्वेति बभूव नम्रः} % १४-५८

\fourlineindentedshloka
{सीता तमुत्थाप्य जगाद वाक्यम्}
{प्रीतास्मि ते सौम्य चिराय जीव}
{बिडौजसा विष्णुरिवाग्रजेन}
{भ्रात्रा यदित्थं परवानसि त्वम्} % १४-५९

\fourlineindentedshloka
{श्वश्रूजनं सर्वमनुक्रमेण}
{विज्ञ्यापय प्रापितमत्प्रणामः}
{प्रजानिषेकं मयि वर्तमानम्}
{सूनोरनुध्यायत चेतसेति} % १४-६०

\fourlineindentedshloka
{वाच्यस्त्वया मद्वचनात्स राजा}
{वह्नौ विशुद्धामपि यत्समक्षम्}
{मां लोकवादश्रवणादहासीः}
{श्रुतस्य किं तत्सदृशं कुलस्य} % १४-६१

\fourlineindentedshloka
{कल्याणबुद्धेरथवा तवायम्}
{न कामचारो मयि शङ्कनीयः}
{ममैव जन्मान्तरपातकानाम्}
{विपाकविस्फूर्जिथुरप्रसह्यः} % १४-६२

\fourlineindentedshloka
{उपस्थितां पूर्वमपास्य लक्ष्मीम्}
{वनं मया सार्धमसि प्रपन्नः}
{तदास्पदं प्राप्य तयातिरोषात्}
{सोढास्मि न त्वद्भवने वसन्ती} % १४-६३

\fourlineindentedshloka
{निशाचरोपप्लुतभर्तृकाणाम्}
{तपस्विनीनां भवतः प्रसादात्}
{भूत्वा शरण्या शरणार्थमन्यम्}
{कथं प्रपत्स्ये त्वयि दीप्यमाने} % १४-६४

\fourlineindentedshloka
{किंवा तवात्यन्तवियोगमोघे}
{कुर्यामुपेक्षां हतजीवितेऽस्मिन्}
{स्याद्रक्षणीयं यदि मे न तेजः}
{तदीयमन्तर्गतमन्तरायः} % १४-६५

\fourlineindentedshloka
{साहं तपः सूर्यनिविष्टदृष्टिः}
{ऊर्ध्वं प्रसूतेश्चरितुं यतिष्ये}
{भूयो यथा मे जननान्तरेऽपि}
{त्वमेव भर्ता न च विप्रयोगः} % १४-६६

\fourlineindentedshloka
{नृपस्य वर्णाश्रमपालनं यत्}
{स एव धर्मो मनुना प्रतीतः}
{निर्वासिताप्येवमतस्त्वयाहम्}
{तपस्विसामान्यमवेक्षणीया} % १४-६७

\fourlineindentedshloka
{तथेति तस्याः प्रतिगृह्य वाचम्}
{रामानुजे दृष्टिपथं व्यतीते}
{सा मुक्तकण्ठं व्यसनातिभारा}
{च्चक्रन्द विग्ना कुररीव भूयः} % १४-६८

\fourlineindentedshloka
{नृत्यं मयूराः कुसुमानि वृक्षा}
{दर्भानुपात्तान्विजहुर्हरिण्यः}
{तस्याः प्रपन्ने समदुःखभावम्}
{अत्यन्तमासीद्रुदितं वनेऽपि} % १४-६९

\fourlineindentedshloka
{तामभ्यगच्छद्रुदितानुकारी}
{कविः कुशेध्माहरणाय यातः}
{निषादविद्धाण्डजदर्शनोत्थः}
{श्लोकत्वमापद्यत यस्य शोकः} % १४-७०

\fourlineindentedshloka
{तमश्रु नेत्रावरणं प्रमृज्य}
{सीता विलापाद्विरता ववन्दे}
{तस्यै मुनिर्दोहदलिङ्गदर्शी}
{दाश्वान्सुपुत्राशिषमित्युवाच} % १४-७१

\fourlineindentedshloka
{जाने विसृष्टां प्रणिधानतस्त्वाम्}
{मिथ्यापवादक्षुभितेन भर्त्रा}
{तन्मा व्यथिष्ठा विषयान्तरस्थम्}
{प्राप्तासि वैदेहि पितुर्निकेतम्} % १४-७२

\fourlineindentedshloka
{उत्खातलोकत्रयकण्टकेऽपि}
{सत्यप्रतिज्ञ्येऽप्यविकत्थनेऽपि}
{त्वां प्रत्यकस्मात्कलुषप्रवृत्ता}
{वस्त्येव मन्युर्भरताग्रजे मे} % १४-७३

\fourlineindentedshloka
{तवोरुकीर्तिः श्वशुरः सखा मे}
{सतां भवोच्छेदकरः पिता ते}
{धुरि स्थिता त्वं पतिदेवतानाम्}
{किं तन्न येनासि ममानुकम्प्या} % १४-७४

\fourlineindentedshloka
{तपस्विसंसर्गविनीतसत्त्वे}
{तपोवने वीतभया वसास्मिन्}
{इतो भविष्यत्यनघप्रसूतेः}
{अपत्यसंस्कारमयोविधिस्ते} % १४-७५

\fourlineindentedshloka
{अशून्यतीरां मुनिसन्निवेशैः}
{तमोनिहन्त्रीं तमसां वगाह्य}
{तत्सैकतोत्सङ्गबलिक्रियाभिः}
{सम्पत्स्यते ते मनसः प्रसादः} % १४-७६

\fourlineindentedshloka
{पुष्पं फलं चार्तवमाहरन्त्यो}
{बीजं च बालेयमकृष्टरोहि}
{विनोदयिष्यन्ति नवाभिषङ्गाम्}
{उदारवाचो मुनिकन्यकास्त्वाम्} % १४-७७

\fourlineindentedshloka
{पयोघटैराश्रमबालवृक्षान्}
{संवर्धयन्ती स्वबलानुरूपैः}
{असंशयं प्राक्तनयोपपत्तेः}
{स्तनन्धयप्रीतिमवाप्स्यसि त्वम्} % १४-७८

\fourlineindentedshloka
{अनुग्रहप्रत्यभिनन्दिनीं ताम्}
{वाल्मीकिरादाय दयार्द्रचेताः}
{सायं मृगाध्यासितवेदिपार्श्वम्}
{स्वमाश्रमं शान्तमृगं निनाय} % १४-७९

\fourlineindentedshloka
{तामर्पयामास च शोकदीनाम्}
{तदागमप्रीतिषु तापसीषु}
{निर्विष्टसारां पितृभिर्हिमांशोः}
{न्त्यां कलां दर्श इवौषधीषु} % १४-८०

\fourlineindentedshloka
{ता इङ्गुदीस्नेहकृतप्रदीपम्}
{आस्तीर्णमेध्याजिनतल्पमन्तः}
{तस्यै सपर्यानुपदं दिनान्ते}
{निवासहेतोरुटजं वितेरुः} % १४-८१

\fourlineindentedshloka
{तत्राभिषेकप्रयता वसन्ती}
{प्रयुक्तपूजा विधिनातिथिभ्यः}
{वन्येन सा वल्कलिनी शरीरम्}
{पत्युः प्रजासन्ततये बभार} % १४-८२

\fourlineindentedshloka
{अपि प्रभुः सानुशयोऽधुना स्यात्}
{किमुत्सुकः शक्रजितोऽपि हन्ता}
{शशंस सीतापरिदेवनान्तम्}
{अनुष्ठितं शासनमग्रजाय} % १४-८३

\fourlineindentedshloka
{बभूव रामः सहसा सबाष्पः}
{तुषारवर्षीव सहस्यचन्द्रः}
{कौलीनभीतेन गृहान्निरस्ता}
{न तेन वैदेहसुता मनस्तः} % १४-८४

\fourlineindentedshloka
{निगृह्य शोकं स्वयमेव धीमान्}
{वर्णाश्रमावेक्षणजागरूकः}
{स भ्रातृसाधारणभोगमृद्धम्}
{राज्यं रजोरिक्तमनाः शशास} % १४-८५

\fourlineindentedshloka
{तामेकभार्यां परिवादभीरोः}
{साध्वीमपि त्यक्तवतो नृपस्य}
{वक्षस्यसङ्घट्टसुखं वसन्ती}
{रेजे सपत्नीरहितेव लक्ष्मीः} % १४-८६

\fourlineindentedshloka
{सीतां हित्वा दशमुखरिपुर्नोपमेये यदन्याम्}
{तस्या एव प्रतिकृतिसखो यत्क्रतूनाजहार}
{वृत्तान्तेन श्रवणविषयप्रापिणा तेन भर्तुः}
{सा दुर्वारं कथमपि परित्यागदुःखं विषेहे} % १४-८७

॥इति श्री-महाकवि-कालिदास-कृत-रघुवंश-महाकाव्ये चतुर्दशः सर्गः॥
\sect{पञ्चदशः सर्गः}

\twolineshloka
{कृतसीतापरित्यागः स रत्नाकरमेखलाम्}
{बुभुजे पृथिवीपालः पृथिवीमेव केवलाम्} % १५-१

\twolineshloka
{लवणेन विलुप्तेज्यास्तामिस्रेण तमभ्ययुः}
{मुनयो यमुनाभाजः शरण्यं शरणार्थिनः} % १५-२

\twolineshloka
{अवेक्ष्य रामं ते तस्मिन्न प्रजह्रुः स्वतेजसा}
{त्राणाभावे हि शापास्त्राः कुर्वन्ति तपसो व्ययम्} % १५-३

\twolineshloka
{प्रतिशुश्राव काकुत्स्थस्तेभ्यो विघ्नप्रतिक्रियाम्}
{धर्मसंरक्षणार्थेव प्रवृत्तिर्भुवि शार्ङ्गिणः} % १५-४

\twolineshloka
{ते रामाय वधोपायमाचख्युर्विबुधद्विषः}
{दुर्जयो लवणः शूली विशूलः प्रार्थ्यतामिति} % १५-५

\twolineshloka
{आदिदेशाथ शत्रुघ्नं तेषां क्षेमाय राघवः}
{करिष्यन्निव नामास्य यथार्थमरिनिग्रहात्} % १५-६

\twolineshloka
{यः कश्चन रघूणां हि परमेकः परन्तपः}
{अपवाद इवोत्सर्गं व्यावर्तयितुमीश्वरः} % १५-७

\twolineshloka
{अग्रजेन प्रयुक्ताशीस्ततो दाशरथी रथी}
{ययौ वनस्थलीः पश्यन्पुष्पिताः सुरभीरभीः} % १५-८

\twolineshloka
{रामादेशादनुगता सेना तस्यार्थसिद्धये}
{पश्चादध्यनार्थस्य धातोरधिरिवाभवत्} % १५-९

\twolineshloka
{आदिष्टवर्त्मा मुनिभिः स गच्छंस्तपसां वरः}
{विरराज रथप्रष्ठैर्वालखिल्यैरिवांशुमान्} % १५-१०

\twolineshloka
{तस्य मार्गवशादेका बभूव वसतिर्यतः}
{रथस्वनोत्कण्ठमृगे वाल्मीकीये तपोवने} % १५-११

\twolineshloka
{तमृषिः पूजयामास कुमारं क्लान्तवाहनम्}
{तपःप्रभावसिद्धाभिर्विशेषप्रतिपत्तिभिः} % १५-१२

\twolineshloka
{तस्यामेवास्य यामिन्यामन्तर्वन्ती प्रजावती}
{सुतावसूत सम्पन्नौ कोशदण्डाविव क्षितिः} % १५-१३

\twolineshloka
{सन्तानश्रवणाद्भ्रातुः सौमित्रिः सौमनस्यवान्}
{प्राञ्जलिर्मुनिमामन्त्र्य प्रातर्युक्तरथो ययौ} % १५-१४

\twolineshloka
{स च प्राप मधूपघ्नं कुम्भीनस्याश्च कुक्षिजः}
{वनात्करमिवादाय सत्त्वराशिमुपस्थितः} % १५-१५

\twolineshloka
{धूमधूम्रो वसागन्धी ज्वालाबभ्रुशिरोरुहः}
{क्रव्याद्गणपरीवारश्चिताग्निरिव जङ्गमः} % १५-१६

\twolineshloka
{अपशूलं तमासाद्य लवणं लक्ष्मणानुजः}
{रुरोध सम्मुखीनो हि जयो रन्ध्रप्रहारिणाम्} % १५-१७

\twolineshloka
{नातिपर्याप्तमालक्ष्य मत्कुक्षेरद्य भोजनम्}
{दिष्ट्या त्वमसि मे धात्रा भीतेनेवोपपादितः} % १५-१८

\twolineshloka
{इति सन्तर्ज्य शत्रुघ्नं राक्षसस्तज्जिघांसया}
{प्रांशुमुत्पाटयामास मुस्तास्तम्बमिव द्रुमम्} % १५-१९

\twolineshloka
{सौमित्रेर्निशितैर्बाणैरन्तरा शकलीकृतः}
{गात्रं पुष्परजः प्राप न शाखी नैरृतेरितः} % १५-२०

\twolineshloka
{विनाशात्तस्य वृक्षस्य रक्षस्तस्मै महोपलम्}
{प्रजिघाय कृतान्तस्य मुष्टिं पृथगिव स्थितम्} % १५-२१

\twolineshloka
{ऐन्द्रमस्त्रमुपादाय शत्रुघ्नेन स ताडितः}
{सिकतात्वादपि परं प्रपेदे परमाणुताम्} % १५-२२

\twolineshloka
{तमुपाद्रवदुद्यम्य दक्षिणं दोर्निशाचरः}
{एकताल इवोत्पातपवनप्रेरितो गिरिः} % १५-२३

\twolineshloka
{कार्ष्णेन पत्रिणा शत्रुः स भिन्नहृदयः पतन्}
{अनिनाय भुवः कम्पं जहाराश्रमवासिनाम्} % १५-२४

\twolineshloka
{वयसां पङ्क्तयः पेतुर्हतस्योपरि विद्विषः}
{तत्प्रतिद्वन्द्विनो मूर्ध्नि दिव्याः कुसुमवृष्टयः} % १५-२५

\twolineshloka
{स हत्वा लवणं वीरस्तदा मेने महौजसः}
{भ्रातुः सोदर्यमात्मानमिन्द्रजिद्वधशोभिनः} % १५-२६

\twolineshloka
{तस्य संस्तूयमानस्य चरितार्थैस्तपस्विभिः}
{शुशुभे विक्रमोदग्रं व्रीडयावनतं शिरः} % १५-२७

\twolineshloka
{उपकूलं च कालिन्द्याः पुरीं पौरुषभूषणः}
{निर्ममे निर्ममोऽर्थेषु मथुरां मधुराकृतिः} % १५-२८

\twolineshloka
{या सौराज्यप्रकाशाभिर्बभौ पौरविभूतिभिः}
{स्वर्गाभिष्यन्दवमनं कृत्वेवोपनिवेशिता} % १५-२९

\twolineshloka
{तत्र सौधगतः पश्यन्यमुनां चक्रवाकिनीम्}
{हेमभक्तिमतीं भूमेः प्रवेणीमिव पिप्रिये} % १५-३०

\twolineshloka
{सखा दशरथस्यापि जनकस्य च मन्त्रकृत्}
{सञ्चस्कारोभयप्रीत्या मैथिलेयौ यथाविधि} % १५-३१

\twolineshloka
{स तौ कुशलवोन्मृष्टगर्भक्लेदौ तदाख्यया}
{कविः कुशलवावेव चकार किल नामतः} % १५-३२

\twolineshloka
{साङ्गं च वेदमध्याप्य किञ्चिदुत्क्रान्तशैशवौ}
{स्वकृतिं गापयामास कविप्रथमपद्धतिम्} % १५-३३

\twolineshloka
{रामस्य मधुरं वृत्तं गायन्तो मातुरग्रतः}
{तद्वियोगव्यथां किञ्चिच्छिथिलीचक्रतुः सुतौ} % १५-३४

\twolineshloka
{इतरेऽपि रघोर्वंश्यास्त्रयस्त्रेताग्नितेजसः}
{तद्योगात्पतिवत्नीषु पत्नीष्वासन्द्विसूनवः} % १५-३५

\twolineshloka
{शत्रुघातिनि शत्रुघ्नः सुबाहौ च बहुश्रुते}
{मथुराविदिशे सून्वोर्निदधे पूर्वजोत्सुकः} % १५-३६

\twolineshloka
{भूयस्तपोव्ययो मा भूद्वाल्मीकेरिति सोऽत्यगात्}
{मैथिलीतनयोद्गीतनिःस्पन्दमृगमाश्रमम्} % १५-३७

\twolineshloka
{वशी विवेश चायोध्यां रथ्यसंस्कारशोभिनीम्}
{लवणस्य वधात्पौरैरीक्षितोऽत्यन्तगौरवम्} % १५-३८

\twolineshloka
{स ददर्श सभामध्ये सभासद्भिरुपस्थितम्}
{रामं सीतापरित्यागादसामान्यपतिं भुवः} % १५-३९

\twolineshloka
{तमभ्यनन्दत्प्रणतं लवणान्तकमग्रजः}
{कालनेमिवधात्प्रीतस्तुराषाडिव शार्ङ्गिणम्} % १५-४०

\twolineshloka
{स पृष्टः सर्वतो वार्तमाख्यद्राज्ञे न सन्ततिम्}
{प्रत्यर्पयिष्यतः काले कवेराद्यस्य शासनात्} % १५-४१

\twolineshloka
{अथ जानपदो विप्रः शिशुमप्राप्तयौवनम्}
{अवतार्याङ्कशय्यास्थं द्वारि चक्रन्द भूपतेः} % १५-४२

\twolineshloka
{शोचनीयासि वसुधे या त्वं दशरथाच्च्युता}
{रामहस्तमनुप्राप्य कष्टात्कष्टतरं गता} % १५-४३

\twolineshloka
{श्रुत्वा तस्य शुचो हेतुं गोप्ता जिह्राय राघवः}
{न ह्यकालभवो मृत्युरिक्ष्वाकुपदमस्पृशत्} % १५-४४

\twolineshloka
{स मुहूर्तं क्षमस्वेति द्विजमाश्वास्य दुःखितम्}
{यानं सस्मार कौबेरं वैवस्वतजिगीषया} % १५-४५

\twolineshloka
{आत्तशस्त्रस्तदध्यास्य प्रस्थितः स रघूद्वहः}
{उच्चचार पुरस्तस्य गूढरूपा सरस्वती} % १५-४६

\twolineshloka
{राजन् प्रजासु ते कश्चिदपचारः प्रवर्तते}
{तमन्विष्य प्रशमयेर्भविष्यसि ततः कृती} % १५-४७

\twolineshloka
{इत्याप्तवचनाद्रामो विनेष्यन्वर्णविक्रियाम्}
{दिशः पपात पत्रेण वेगनिष्कम्पहेतुना} % १५-४८

\twolineshloka
{अथ धूमाभिताम्राक्षं वृक्षशाखावलम्बिनम्}
{ददर्श कञ्चिदैक्ष्वाकस्तपस्यन्तमधोमुखम्} % १५-४९

\twolineshloka
{पृष्टनामान्वयो राज्ञा स किलाचष्ट धूमपः}
{आत्मानं शम्बुकं नाम शूद्रं सुरपदार्थिनम्} % १५-५०

\twolineshloka
{तपस्यनधिकारित्वात्प्रजानां तमघावहम्}
{शीर्षच्छेद्यं परिच्छिद्य नियन्ता शस्त्रमाददे} % १५-५१

\twolineshloka
{स तद्वक्त्रं हिमक्लिष्टकिञ्जल्कमिव पङ्कजम्}
{ज्योतिष्कणाहतश्मश्रु कण्ठनालादपातयत्} % १५-५२

\twolineshloka
{कृतदण्डः स्वयं राज्ञा लेभे शूद्रः सतां गतिम्}
{तपसा दुश्चरेणापि न स्वमार्गविलङ्घिना} % १५-५३

\twolineshloka
{रघुनाथोऽप्यगस्त्येन मार्गसन्दर्शितात्मना}
{महौजसा संयुयुजे शरत्काल इवेन्दुना} % १५-५४

\twolineshloka
{कुम्भयोनिरलङ्कारं तस्मै दिव्यपरिग्रहम्}
{ददौ दत्तं समुद्रेण पीतेनेवात्मनिष्क्रयम्} % १५-५५

\twolineshloka
{स दधन्मैथिलीकण्ठनिर्व्यापारेण बाहुना}
{पश्चान्निववृते रामः प्राक्परासुर्द्विजात्मजः} % १५-५६

\twolineshloka
{तस्य पूर्वोदितां निन्दां द्विजः पुत्रसमागतः}
{स्तुत्वा निवर्तयामास त्रातुर्वैवस्वतादपि} % १५-५७

\twolineshloka
{तमध्वराय मुक्ताश्वं रक्षःकपिनरेश्वराः}
{मेघाः सस्यमिवाम्भोभिरभ्यवर्षन्नुपायनैः} % १५-५८

\twolineshloka
{दिग्भ्यो निमन्त्रिताश्चैनमभिजग्मुर्महर्षयः}
{न भौमान्येव धिष्ण्यानि हित्वा ज्योतिर्मयान्यपि} % १५-५९

\twolineshloka
{उपशल्यनिविष्टैस्तैश्चतुर्द्वारमुखी बभौ}
{अयोध्या सृष्टलोकेव सद्यः पैतामही तनुः} % १५-६०

\twolineshloka
{श्लाघ्यस्त्यागोऽपि वैदेह्याः पत्युः प्राग्वंशवासिनः}
{अनन्यजानेः सैवासीद्यस्माज्जाया हिरण्मयी} % १५-६१

\twolineshloka
{विधेरधिकसम्भारस्ततः प्रववृते मखः}
{आसन्यत्र क्रियाविघ्ना राक्षसा एव रक्षिणः} % १५-६२

\twolineshloka
{अथ प्राचेतसोपज्ञं रामायणमितस्ततः}
{मैथिलेयौ कुशलवौ जगतुर्गुरुचोदितौ} % १५-६३

\twolineshloka
{वृत्तं रामस्य वाल्मीकेः कृतिस्तौ किन्नरस्वनौ}
{किं तद्येन मनो हर्तुमलं स्यातां न शृण्वताम्} % १५-६४

\twolineshloka
{रूपे गीते च माधुर्यन्तयोस्तज्ज्ञैर्निवेदितम्}
{ददर्श सानुजो रामः शुश्राव च कुतूहली} % १५-६५

\twolineshloka
{तद्गीतश्रवणैकाग्रा संसदश्रुमुखी बभौ}
{हिमनिष्यन्दिनी प्रातर्निर्वातेव वनस्थली} % १५-६६

\twolineshloka
{वयोवेषविसंवादि रामस्य च तयोस्तदा}
{जनता प्रेक्ष्य सादृश्यं नाक्षिकम्पं व्यतिष्ठत} % १५-६७

\twolineshloka
{उभयोर्न तथा लोकः प्रावीण्येन विसिष्मिये}
{नृपतेः प्रीतिदानेषु वीतस्पृहतया यथा} % १५-६८

\twolineshloka
{गेये को नु विनेता वां कस्य चेयं कृतिः कवेः}
{इति राज्ञा स्वयं पृष्टौ तौ वाल्मीकिमशंसताम्} % १५-६९

\twolineshloka
{अथ सावरजो रामः प्राचेतसमुपेयिवान्}
{ऊरीकृत्यात्मनो देहं राज्यमस्मै न्यवेदयत्} % १५-७०

\twolineshloka
{स तावाख्याय रामाय मैथिलीयौ तदात्मजौ}
{कविः कारुणिको वव्रे सीतायाः सम्परिग्रहम्} % १५-७१

\twolineshloka
{तात शुद्धा समक्षं नः स्नुषा ते जातवेदसि}
{दौरात्म्याद्रक्षसस्तां तु नात्रत्याः श्रद्दधुः प्रजाः} % १५-७२

\twolineshloka
{ताः स्वचारित्र्यमुद्दिश्य प्रत्याययतु मैथिली}
{ततः पुत्रवतीमेनां प्रतिपत्स्ये त्वदाज्ञया} % १५-७३

\twolineshloka
{इति प्रतिश्रुते राज्ञा जानकीमाश्रमान्मुनिः}
{शिष्यैरानाययामास स्वसिद्धिं नियमैरिव} % १५-७४

\twolineshloka
{अन्येद्युरथ काकुत्स्थः सन्निपात्य पुरौकसः}
{कविमाह्वाययामास प्रस्तुतप्रतिपत्तये} % १५-७५

\twolineshloka
{स्वरसंस्कारवत्यासौ पुत्राभ्यामथ सीतया}
{ऋचेवोदर्चिषं सूर्यं रामं मुनिरुपस्थितः} % १५-७६

\twolineshloka
{काषायपरिवीतेन स्वपदार्पितचक्षुषा}
{अन्वमीयत शुद्धेति शान्तेन वपुषैव सा} % १५-७७

\twolineshloka
{जनास्तदालोकपथात्प्रतिसंहृतचक्षुषः}
{तस्थुस्तेऽवाङ्मुखाः सर्वे फलिता इव शालयः} % १५-७८

\twolineshloka
{तां दृष्टिविषये भर्तुर्मुनिरास्थितविष्टरः}
{कुरु निःसंशयं वत्से स्ववृत्ते लोकमित्यशात्} % १५-७९

\twolineshloka
{अथ वाल्मीकिशिष्येण पुण्यमावर्जितं पयः}
{आचम्योदीरयामास सीता सत्यां सरस्वतीम्} % १५-८०

\twolineshloka
{वाङ्मनःकर्मभिः पत्यौ व्यभिचारो यथा न मे}
{तथा विश्वम्भरे देवि मामन्तर्धातुमर्हसि} % १५-८१

\twolineshloka
{एवमुक्ते तया साध्व्या रन्ध्रात्सद्योभवाद्भुवः}
{शातह्रदमिव ज्योतिः प्रभामण्डलमुद्ययौ} % १५-८२

\twolineshloka
{तत्र नागफणोत्क्षिप्तसिंहासननिषेदुषी}
{समुद्ररशना साक्षात्प्रादुरासीद्वसुन्धरा} % १५-८३

\twolineshloka
{सा सीतामङ्कमारोप्य भर्तृप्रणिहितेक्षणाम्}
{मामेति व्याहरत्येव तस्मिन्पातालमभ्यगात्} % १५-८४

\twolineshloka
{धरायां तस्य संरम्भं सीताप्रत्यर्पणैषिणः}
{गुरुर्विधिबलापेक्षी शमयामास धन्विनः} % १५-८५

\twolineshloka
{ऋषीन्विसृज्य यज्ञान्ते सुहृदश्च पुरस्कृतान्}
{रामः सीतागतं स्नेहं निदधे तदपत्ययोः} % १५-८६

\twolineshloka
{युधाजितस्य सन्देशात्स देशं सिन्धुनामकम्}
{ददौ दत्तप्रभावाय भरताय भृतप्रजः} % १५-८७

\twolineshloka
{भरतस्तत्र गन्धर्वान्युधि निर्जित्य केवलम्}
{आतोद्यं ग्राहयामास समत्याजयदायुधम्} % १५-८८

\twolineshloka
{स तक्षपुष्कलौ पुत्रौ राजधान्योस्तदाख्ययोः}
{अभिषिच्याभिषेकार्हौ रामान्तिकमगात्पुनः} % १५-८९

\twolineshloka
{अङ्गदं चन्द्रकेतुं च लक्ष्मणोऽप्यात्मसम्भवौ}
{शासनाद्रघुनाथस्य चक्रे कारापथेश्वरौ} % १५-९०

\twolineshloka
{इत्यारोपितपुत्रास्ते जननीनां जनेश्वराः}
{भर्तृलोकप्रपन्नानां निवापान्विदधुः क्रमात्} % १५-९१

\twolineshloka
{उपेत्य मुनिवेषोऽथ कालः प्रोवाच राघवम्}
{रहःसंवादिनौ पश्येदावां यस्तं त्यजेरिति} % १५-९२

\twolineshloka
{तथेति प्रतिपन्नाय विवृतात्मा नृपाय सः}
{आचख्यौ दिवमध्यास्व शासनात्परमेष्ठिनः} % १५-९३

\twolineshloka
{विद्वानपि तयोर्द्वाःस्थः समयं लक्ष्मणोऽभिनत्}
{भीतो दुर्वाससः शापाद्रामसन्दर्शनार्थिना} % १५-९४

\twolineshloka
{स गत्वा सरयूतीरं देहत्यागेन योगवित्}
{चकारावितथां भ्रातुः प्रतिज्ञां पूर्वजन्मनः} % १५-९५

\twolineshloka
{तस्मिन्नात्मचतुर्भागे प्राङ्नाकमधितस्थुषि}
{राघवः शिथिलं तस्थौ भुवि धर्मस्त्रिपादिव} % १५-९६

\twolineshloka
{स निवेश्य कुशावत्यां रिपुनागाङ्कुशं कुशम्}
{शरावत्यां सतां सूक्तैर्जनिताश्रुलवं लवम्} % १५-९७

\twolineshloka
{उदक्प्रतस्थे स्थिरधीः सानुजोऽग्निपुरःसरः}
{अन्वितः पतिवात्सल्याद्गृहवर्जमयोध्यया} % १५-९८

\twolineshloka
{जगृहुस्तस्य चित्तज्ञाः पदवीं हरिराक्षसाः}
{कदम्बमुकुलस्थूलैरभिवृष्टां प्रजाश्रुभिः} % १५-९९

\twolineshloka
{उपस्थितविमानेन तेन भक्तानुकम्पिना}
{चक्रे त्रिदिवनिःश्रेणिः सरयूरनुयायिनाम्} % १५-१००

\twolineshloka
{यद्गोप्रतरकल्पोऽभूत्सम्मर्दस्तत्र मज्जताम्}
{अतस्तदाख्यया तीर्थं पावनं भुवि पप्रथे} % १५-१०१

\twolineshloka
{स विभुर्विबुधांशेषु प्रतिपन्नात्ममूर्तिषु}
{त्रिदशीभूतपौराणां स्वर्गान्तरमकल्पयत्} % १५-१०२

\fourlineindentedshloka
{निर्वर्त्यैवं दशमुखशिरश्छेदकार्यम्}
{सुराणाम्विष्वक्सेनः स्वतनुमविशत्सर्वलोकप्रतिष्ठाम्}
{लङ्कानाथं पवनतनयं चोभयं स्थापयित्वा}
{कीर्तिस्तम्भद्वयमिव गिरौ दक्षिणे चोत्तरे च} % १५-१०३

॥इति श्री-महाकवि-कालिदास-कृत-रघुवंश-महाकाव्ये पञ्चदशः सर्गः॥