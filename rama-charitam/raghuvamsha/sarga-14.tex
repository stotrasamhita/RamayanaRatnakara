\sect{चतुर्दशः सर्गः}

\fourlineindentedshloka
{भर्तुः प्रणाशादथ शोचनीयम्}
{दशान्तरम् तत्र समम् प्रपन्ने}
{अपश्यताम् दाशरथी जनन्यौ}
{छेदादिवोपघ्नतरोर्व्रतत्यौ} % १४-१

\fourlineindentedshloka
{उभावु भाभ्याम् प्रणतौ हत अरी}
{यथा क्रमम् विक्रम शोभिनौ तौ}
{विस्पष्टम् अस्र अन्धतया न दृष्टौ}
{ज्ञ्यातौ सुत-स्पर्श सुख उपलम्भात्} % १४-२

\fourlineindentedshloka
{आनन्दजः शोकजमश्रुबाष्पः}
{तयोरशीतम् शिशिरो बिभेद}
{गङ्गासरय्वोर्जलमुष्णतप्तम्}
{हिमाद्रिनिस्यन्द इवावतीर्णः} % १४-३

\fourlineindentedshloka
{ते पुत्रयोःनैऋतशस्त्रमार्गान्}
{आर्द्रानिवाङ्गे सदयम् स्पृशन्तौ}
{अपीप्सितम् क्षत्रकुलाङ्गनानाम्}
{न वीरसूशब्दमकामयेताम्} % १४-४

\fourlineindentedshloka
{क्लेशावहा भर्तुरलक्षणाहम्}
{सीतेति नाम स्वमुदीरयन्ती}
{स्वर्गप्रतिष्ठस्य गुरोर्महिष्या}
{वभक्तिभेदेन वधूर्ववन्दे} % १४-५

\fourlineindentedshloka
{उत्तिष्ठ वत्से ननु स अनुजोः असौ}
{वृत्तेन भर्ता शुचिना तवैव}
{कृच्छ्रम् महत्तीर्ण इव प्रियार्हाम्}
{तामूचतुस्ते प्रियमप्यमिथ्या} % १४-६

\fourlineindentedshloka
{अथाभिषेकम् रघुवंशकेतोः}
{प्रारब्धमानन्दजलैर्जनन्योः}
{निर्वर्तयामासुरमात्यवृद्धाः}
{तीर्थाहृतैः काञ्चनकुम्भतोयैः} % १४-७

\fourlineindentedshloka
{सरित्समुद्रान्सरसीश्च गत्वा}
{रक्षःकपीन्द्रैरुपपादितानि}
{तस्यापतन्मूर्ध्नि जलानि जिष्णोः}
{विन्ध्यस्य मेघप्रभवा इवापः} % १४-८

\fourlineindentedshloka
{तपस्विवेषक्रिययापि तावत्}
{यः प्रेक्षणीयः सुतराम् बभूव}
{राजेन्द्रनेपथ्यविधानशोभा}
{तस्योदितासीत्पुनरुक्तदोषा} % १४-९

\fourlineindentedshloka
{स मौलरक्षोहरिभिः ससैन्यः}
{तूर्यस्वनानन्दितपौरवर्गः}
{विवेश सौधोद्गतलाजवर्षाम्}
{उत्तोरणामन्वयराजधानीम्} % १४-१०

\fourlineindentedshloka
{सौमित्रिणा सावरजेन मन्दम्}
{आधूतबालव्यजनो रथस्थः}
{धृतातपत्रो भरतेन साक्षात्}
{उपायसङ्घात इव प्रवृद्धः} % १४-११

\fourlineindentedshloka
{प्रासादकालागुरुधूमराजिः}
{तस्याः पुरो वायुवशेन भिन्ना}
{वनान्निवृत्तेन रघूत्तमेन}
{मुक्ता स्वयम् वेणिरिवाबभासे} % १४-१२

\fourlineindentedshloka
{श्वश्रूजनानुष्ठितचारुवेषाम्}
{कर्णीरथस्थाम् रघुवीरपत्नीम्}
{प्रासादवातायनदृश्यबन्धैः}
{साकेतनार्योऽञ्जलिभिः प्रणेमुः} % १४-१३

\fourlineindentedshloka
{स्फुरत्प्रभामण्डनमानसूयम्}
{सा बिभ्रती शाश्वतमङ्गरागम्}
{रराज शुद्धेति पुनः स्वपुर्यै}
{सन्दर्शिता वह्निगतेव भर्त्रा} % १४-१४

\fourlineindentedshloka
{वेश्मानि रामः परिबर्हवन्ति}
{विश्राण्य सौहार्दनिधिः सुहृद्भ्यः}
{बाष्पायमानो बलिमन्निकेतम्}
{आलेख्यशेषस्य पितुर्विवेश} % १४-१५

\fourlineindentedshloka
{कृताञ्जलिस्तत्र यदम्ब सत्यात्}
{नाभ्रश्यत स्वर्गफलाद्गुरुर्नः}
{तच्चिन्त्यमानम् सुकृतम् तवेति}
{जहार लज्जाम् भरतस्य मातुः} % १४-१६

\fourlineindentedshloka
{तथैव सुग्रीवबिभीषणादीन्}
{उपाचरत्कृत्रिमसंविधाभिः}
{सङ्कल्पमात्रोदितसिद्धयस्ते}
{क्रान्ता यथा चेतसि विस्मयेन} % १४-१७

\fourlineindentedshloka
{सभाजनायोपगतान्स दिव्यान्}
{मुनीन्पुरस्कृत्य हतस्य शत्रोः}
{शुश्राव तेभ्यः प्रभवादि वृत्तम्}
{स्वविक्रमे गौरवमादधानम्} % १४-१८

\fourlineindentedshloka
{प्रतिप्रयातेषु तपोधनेषु}
{सुखादविज्ञ्यातगतार्धमासान्}
{सीतास्वहस्तोपहृताग्र्यपूजान्}
{रक्षःकपीन्द्रान्विससर्ज रामः} % १४-१९

\fourlineindentedshloka
{तच्चात्मचिन्तासुलभम् विमानम्}
{हृतम् सुरारेः सह जीवितेन}
{कैलासनाथोद्वहनाय भूयः}
{पुष्पम् दिवः पुष्पकमन्वमंस्त} % १४-२०

\fourlineindentedshloka
{पितुर्नियोगाद्वनवासमेवम्}
{निस्तीर्य रामः प्रतिपन्नराज्यः}
{धर्मार्थकामेषु समाम् प्रपेदे}
{यथा तथैवावरजेषु वृत्तिम्} % १४-२१

\fourlineindentedshloka
{सर्वासु मातुष्वपि वत्सलत्वात्}
{स निर्विशेषप्रतिपत्तिरासीत्}
{षडाननापीतपयोधरासु}
{नेता चमूनामिव कृत्तिकासु} % १४-२२

\fourlineindentedshloka
{तेनार्थवान् लोभपराङ्मुखेन}
{तेन घ्नता विघ्नभयम् क्रियावान्}
{तेनास लोकः पितृमान्विनेत्रा}
{तेनैव शोकापनुदेव पुत्री} % १४-२३

\fourlineindentedshloka
{स पौरकार्याणि समीक्ष्य काले}
{रेमे विदेहाधिपतेर्दुहित्रा}
{उपस्थितश्चारु वपुस्तदीयम्}
{कृत्वोपभोगोत्सुकयेव लक्ष्म्या} % १४-२४

\fourlineindentedshloka
{तयोर्यथाप्रार्थितमिन्द्रियार्थान्}
{आसेदुषोः सद्मसु चित्रवत्सु}
{प्राप्तानि दुःखान्यपि दण्डकेषु}
{सञ्चिन्त्यमानानि सुखान्यभूवन्} % १४-२५

\fourlineindentedshloka
{अथाधिकस्निग्धविलोचनेन}
{मुखेन सीता शरपाण्डुरेण}
{आनन्दयित्री परिणेतुरासीत्}
{अनक्षरव्यञ्जितदोहदेन} % १४-२६

\fourlineindentedshloka
{तामङ्कमारोप्य कृशाङ्गयष्टिम्}
{वर्णान्तराक्रान्तपयोधराग्राम्}
{विलज्जमानाम् रहसि प्रतीतः}
{पप्रच्छ रामाम् रमणोऽभिलाषम्} % १४-२७

\fourlineindentedshloka
{सा दष्टनीवारबलीनि हिंस्रैः}
{सम्बद्धवैखानसकन्यकानि}
{इयेष भूयः कुशवन्ति गन्तुम्}
{भागीरथीतीरतपोवनानि} % १४-२८

\fourlineindentedshloka
{तस्यै प्रतिश्रुत्य रघुप्रवीरः}
{तदीप्सितम् पार्श्वचरानुयातः}
{आलोकयिष्यन्मुदितामयोध्याम्}
{प्रासादमभ्रंलिहमारुरोह} % १४-२९

\fourlineindentedshloka
{ऋद्धापणम् राजपथम् स पश्यन्}
{विगाह्यमानाम् सरयूम् च नौभिः}
{विलासिभिश्चाध्युषितानि पौरैः}
{पुरोपकण्ठोपवनानि रेमे} % १४-३०

\fourlineindentedshloka
{स किंवदन्तीम् वदताम् पुरोगः}
{स्ववृत्तमुद्दिश्य विशुद्धवृत्तः}
{सर्पाधिराजोरुभुजोऽपसर्पम्}
{पप्रच्छ भद्रम् विजितारिभद्रः} % १४-३१

\fourlineindentedshloka
{निर्बन्धपृष्टः स जगाद सर्वम्}
{स्तुवन्ति पौराश्चरितम् त्वदीयम्}
{अन्यत्र रक्षोभवनोषितायाः}
{परिग्रहान्मानवदेव देव्याः} % १४-३२

\fourlineindentedshloka
{कलत्रनिन्दागुरुणा किलैवम्}
{अभ्याहतम् कीर्तिविपर्ययेण}
{अयोघनेनाय इवाभितप्तम्}
{वैदेहिबन्धोर्हृदयम् विदद्रे} % १४-३३

\fourlineindentedshloka
{किमात्मनिर्वादकथामुपेक्षे}
{जायामदोषामुत सन्त्यजामि}
{इत्येकपक्षाश्रयविक्लवत्वात्}
{आसीत्स दोलाचलचित्तवृत्तिः} % १४-३४

\fourlineindentedshloka
{निश्चित्य चानन्यनिवृत्ति वाच्यम्}
{त्यागेन पत्न्याः परिमार्ष्टुमैच्छत्}
{अपि स्वदेहात्किमुतेन्द्रियार्थात्}
{अशोधनानाम् हि यशो गरीयः} % १४-३५

\fourlineindentedshloka
{स सन्निपत्यावरजान्हतौजाः}
{तद्विक्रियादर्शनलुप्तहर्षान्}
{कौलीनमात्माश्रयमाचचक्षे}
{तेभ्यः पुनश्चेदमुवाच वाक्यम्} % १४-३६

\fourlineindentedshloka
{राजर्षिवंशस्य रविप्रसूतेः}
{उपस्थितः पश्यत कीदृशोऽयम्}
{मत्तः सदाचारशुचेः कलङ्कः}
{पयोदवातादिव दर्पणस्य} % १४-३७

\fourlineindentedshloka
{पौरेषु सोऽहम् बहुलीभवन्तम्}
{अपाम् तरङ्गेष्विव तैलबिन्दुम्}
{सोढुम् न तत्पूर्वमवर्णमीशे}
{आलानिकम् स्थाणुरिव द्विपेन्द्रः} % १४-३८

\fourlineindentedshloka
{तस्यापनोदाय फलप्रवृत्तौ}
{उपस्थितायामपि निर्व्यपेक्षः}
{त्यक्षामि वैदेहसुताम् पुरस्तात्}
{समुद्रनेमिम् पितुराज्ञ्ययेव} % १४-३९

\fourlineindentedshloka
{अवैमि चैनामनघेति किन्तु}
{लोकापवादो बलवान्मतो मे}
{छाया हि भूमेः शशिनो मलत्वेन}
{आरोपिता शुद्धिमतः प्रजाभिः} % १४-४०

\fourlineindentedshloka
{रक्षोवधान्तो न च मे प्रयासो}
{व्यर्थः स वैरप्रतिमोचनाय}
{अमर्षणः शोणितकाङ्क्षया किम्}
{पदा स्पृशन्तम् दशति द्विजिह्वः} % १४-४१

\fourlineindentedshloka
{तदेष सर्गः करुणार्द्रचित्तैः}
{न मे भवद्भिः प्रतिषेधनीयः}
{यद्यर्थिता निर्हृतवाच्यशल्यान्}
{प्राणान्मया धारयितुम् चिरम् वः} % १४-४२

\fourlineindentedshloka
{इत्युक्तवन्तम् जनकात्मजायाम्}
{नितान्तरूक्षाभिनिवेशमीशम्}
{न कश्चन भ्रातृषु तेषु शक्तो}
{निषेद्धुमासीदनुमोदितुम् वा} % १४-४३

\fourlineindentedshloka
{स लक्ष्मणम् लक्ष्मणपूर्वजन्मा}
{विलोक्य लोकत्रयगीतकीर्तिः}
{सौम्येति चाभाष्य यथार्थभाषी}
{स्थितम् निदेशे पृथगादिदेश} % १४-४४

\fourlineindentedshloka
{प्रजावती दोहदशंसिनी ते}
{तपोवनेषु स्पृहयालुरेव}
{स त्वम् रथी तद्व्यपदेशनेयाम्}
{प्रापय्य वाल्मीकिपदम् त्यजैनाम्} % १४-४५

\fourlineindentedshloka
{स शुश्रुवान्मातरि भार्गवेन}
{पितुर्नियोगात्प्रहृतम् द्विषद्वत्}
{प्रत्यग्रहीदग्रजशासनम् तत्}
{आज्ञ्या गुरूणां ह्यविचारणीया} % १४-४६

\fourlineindentedshloka
{अथानुकूलश्रवणप्रतीता}
{मत्रस्नुभिर्युक्तधुरम् तुरङ्गैः}
{रथम् सुमन्त्रप्रतिपन्नरश्मिम्}
{आरोप्य वैदेहसुताम् प्रतस्थे} % १४-४७

\fourlineindentedshloka
{सा नीयमाना रुचिरान्प्रदेशान्}
{प्रियङ्करो मे प्रिय इत्यनन्दत्}
{नाबुद्ध कल्पद्रुमताम् विहाय}
{जातम् तमात्मन्यसिपत्रवृक्षम्} % १४-४८

\fourlineindentedshloka
{जुगूह तस्याः पथि लक्ष्मणो यत्}
{सव्येतरेण स्फुरता तदक्ष्णा}
{आख्यातमस्यै गुरु भावि दुःखम्}
{अत्यन्तलुप्तप्रियदर्शनेन} % १४-४९

\fourlineindentedshloka
{सा दुर्निमित्तोपगताद्विषादात्}
{सद्यःपरिम्लानमुखारविन्दा}
{राज्ञ्यः शिवम् सावरजस्य भूयादित्य्}
{आशशंसे करणैरबाह्यैः} % १४-५०

\fourlineindentedshloka
{गुरोर्नियोगाद्वनिताम् वनान्ते}
{साध्वीम् सुमित्रातनयो विहास्यन्}
{अवार्यतेवोत्थितवीचिहस्तैः}
{जह्नोर्दुहित्रा स्थितया पुरस्तात्} % १४-५१

\fourlineindentedshloka
{रथात्स यन्त्रा निगृहीतवाहात्}
{ताम् भ्रातृजायाम् पुलिनेऽवतार्य}
{गङ्गाम् निषादाहृतनौविशेषः}
{ततार सन्धामिव सत्यसन्धः} % १४-५२

\fourlineindentedshloka
{अथ व्यवस्थापितवाक्कथञ्चित्}
{सौमित्रिरन्तर्गतबाष्पकण्ठः}
{औत्पातिकम् मेघ इवाश्मवर्षम्}
{महीपतेः शासनमुज्जगार} % १४-५३

\fourlineindentedshloka
{ततोऽभिषङ्गानिलविप्रविद्धा}
{प्रभ्रश्यमानाभरणप्रसूना}
{स्वमूर्तिलाभप्रकृतिम् धरित्रीम्}
{लतेव सीता सहसा जगाम} % १४-५४

\fourlineindentedshloka
{इक्ष्वाकुवंशप्रभवः कथम् त्वाम्}
{त्यजेदकस्मात्पतिरार्यवृत्तः}
{इति क्षितिः संशयितेव तस्यै}
{ददौ प्रवेशम् जननी न तावत्} % १४-५५

\fourlineindentedshloka
{सा लुप्तसञ्ज्ञ्या न विवेद दुःखम्}
{प्रत्यागतासुः समतप्यतान्तः}
{तस्याः सुमित्रात्मजयत्नलब्धो}
{मोहादभूत्कष्टतरः प्रबोधः} % १४-५६

\fourlineindentedshloka
{न चावदद्भर्तुरवर्णमार्या}
{निराकरिष्णोर्वृजिनादृतेऽपि}
{आत्मानमेवम् स्थिरदुःखभाजम्}
{पुनः पुनर्दुष्कृतिनम् निनिन्द} % १४-५७

\fourlineindentedshloka
{आश्वास्य रामावरजः सतीम् ताम्}
{आख्यातवाल्मीकिनिकेतमार्गः}
{निघ्नस्य मे भर्तृनिदेशरौक्ष्यम्}
{देवि क्षमस्वेति बभूव नम्रः} % १४-५८

\fourlineindentedshloka
{सीता तमुत्थाप्य जगाद वाक्यम्}
{प्रीतास्मि ते सौम्य चिराय जीव}
{बिडौजसा विष्णुरिवाग्रजेन}
{भ्रात्रा यदित्थम् परवानसि त्वम्} % १४-५९

\fourlineindentedshloka
{श्वश्रूजनम् सर्वमनुक्रमेण}
{विज्ञ्यापय प्रापितमत्प्रणामः}
{प्रजानिषेकम् मयि वर्तमानम्}
{सूनोरनुध्यायत चेतसेति} % १४-६०

\fourlineindentedshloka
{वाच्यस्त्वया मद्वचनात्स राजा}
{वह्नौ विशुद्धामपि यत्समक्षम्}
{माम् लोकवादश्रवणादहासीः}
{श्रुतस्य किम् तत्सदृशम् कुलस्य} % १४-६१

\fourlineindentedshloka
{कल्याणबुद्धेरथवा तवायम्}
{न कामचारो मयि शङ्कनीयः}
{ममैव जन्मान्तरपातकानाम्}
{विपाकविस्फूर्जिथुरप्रसह्यः} % १४-६२

\fourlineindentedshloka
{उपस्थिताम् पूर्वमपास्य लक्ष्मीम्}
{वनम् मया सार्धमसि प्रपन्नः}
{तदास्पदम् प्राप्य तयातिरोषात्}
{सोढास्मि न त्वद्भवने वसन्ती} % १४-६३

\fourlineindentedshloka
{निशाचरोपप्लुतभर्तृकाणाम्}
{तपस्विनीनाम् भवतः प्रसादात्}
{भूत्वा शरण्या शरणार्थमन्यम्}
{कथम् प्रपत्स्ये त्वयि दीप्यमाने} % १४-६४

\fourlineindentedshloka
{किंवा तवात्यन्तवियोगमोघे}
{कुर्यामुपेक्षाम् हतजीवितेऽस्मिन्}
{स्याद्रक्षणीयम् यदि मे न तेजः}
{तदीयमन्तर्गतमन्तरायः} % १४-६५

\fourlineindentedshloka
{साहम् तपः सूर्यनिविष्टदृष्टिः}
{ऊर्ध्वम् प्रसूतेश्चरितुम् यतिष्ये}
{भूयो यथा मे जननान्तरेऽपि}
{त्वमेव भर्ता न च विप्रयोगः} % १४-६६

\fourlineindentedshloka
{नृपस्य वर्णाश्रमपालनम् यत्}
{स एव धर्मो मनुना प्रतीतः}
{निर्वासिताप्येवमतस्त्वयाहम्}
{तपस्विसामान्यमवेक्षणीया} % १४-६७

\fourlineindentedshloka
{तथेति तस्याः प्रतिगृह्य वाचम्}
{रामानुजे दृष्टिपथम् व्यतीते}
{सा मुक्तकण्ठम् व्यसनातिभारा}
{च्चक्रन्द विग्ना कुररीव भूयः} % १४-६८

\fourlineindentedshloka
{नृत्यम् मयूराः कुसुमानि वृक्षा}
{दर्भानुपात्तान्विजहुर्हरिण्यः}
{तस्याः प्रपन्ने समदुःखभावम्}
{अत्यन्तमासीद्रुदितम् वनेऽपि} % १४-६९

\fourlineindentedshloka
{तामभ्यगच्छद्रुदितानुकारी}
{कविः कुशेध्माहरणाय यातः}
{निषादविद्धाण्डजदर्शनोत्थः}
{श्लोकत्वमापद्यत यस्य शोकः} % १४-७०

\fourlineindentedshloka
{तमश्रु नेत्रावरणम् प्रमृज्य}
{सीता विलापाद्विरता ववन्दे}
{तस्यै मुनिर्दोहदलिङ्गदर्शी}
{दाश्वान्सुपुत्राशिषमित्युवाच} % १४-७१

\fourlineindentedshloka
{जाने विसृष्टाम् प्रणिधानतस्त्वाम्}
{मिथ्यापवादक्षुभितेन भर्त्रा}
{तन्मा व्यथिष्ठा विषयान्तरस्थम्}
{प्राप्तासि वैदेहि पितुर्निकेतम्} % १४-७२

\fourlineindentedshloka
{उत्खातलोकत्रयकण्टकेऽपि}
{सत्यप्रतिज्ञ्येऽप्यविकत्थनेऽपि}
{त्वाम् प्रत्यकस्मात्कलुषप्रवृत्ता}
{वस्त्येव मन्युर्भरताग्रजे मे} % १४-७३

\fourlineindentedshloka
{तवोरुकीर्तिः श्वशुरः सखा मे}
{सताम् भवोच्छेदकरः पिता ते}
{धुरि स्थिता त्वम् पतिदेवतानाम्}
{किम् तन्न येनासि ममानुकम्प्या} % १४-७४

\fourlineindentedshloka
{तपस्विसंसर्गविनीतसत्त्वे}
{तपोवने वीतभया वसास्मिन्}
{इतो भविष्यत्यनघप्रसूतेः}
{अपत्यसंस्कारमयोविधिस्ते} % १४-७५

\fourlineindentedshloka
{अशून्यतीराम् मुनिसन्निवेशैः}
{तमोनिहन्त्रीम् तमसाम् वगाह्य}
{तत्सैकतोत्सङ्गबलिक्रियाभिः}
{सम्पत्स्यते ते मनसः प्रसादः} % १४-७६

\fourlineindentedshloka
{पुष्पम् फलम् चार्तवमाहरन्त्यो}
{बीजम् च बालेयमकृष्टरोहि}
{विनोदयिष्यन्ति नवाभिषङ्गाम्}
{उदारवाचो मुनिकन्यकास्त्वाम्} % १४-७७

\fourlineindentedshloka
{पयोघटैराश्रमबालवृक्षान्}
{संवर्धयन्ती स्वबलानुरूपैः}
{असंशयम् प्राक्तनयोपपत्तेः}
{स्तनन्धयप्रीतिमवाप्स्यसि त्वम्} % १४-७८

\fourlineindentedshloka
{अनुग्रहप्रत्यभिनन्दिनीम् ताम्}
{वाल्मीकिरादाय दयार्द्रचेताः}
{सायम् मृगाध्यासितवेदिपार्श्वम्}
{स्वमाश्रमम् शान्तमृगम् निनाय} % १४-७९

\fourlineindentedshloka
{तामर्पयामास च शोकदीनाम्}
{तदागमप्रीतिषु तापसीषु}
{निर्विष्टसाराम् पितृभिर्हिमांशोः}
{न्त्याम् कलाम् दर्श इवौषधीषु} % १४-८०

\fourlineindentedshloka
{ता इङ्गुदीस्नेहकृतप्रदीपम्}
{आस्तीर्णमेध्याजिनतल्पमन्तः}
{तस्यै सपर्यानुपदम् दिनान्ते}
{निवासहेतोरुटजम् वितेरुः} % १४-८१

\fourlineindentedshloka
{तत्राभिषेकप्रयता वसन्ती}
{प्रयुक्तपूजा विधिनातिथिभ्यः}
{वन्येन सा वल्कलिनी शरीरम्}
{पत्युः प्रजासन्ततये बभार} % १४-८२

\fourlineindentedshloka
{अपि प्रभुः सानुशयोऽधुना स्यात्}
{किमुत्सुकः शक्रजितोऽपि हन्ता}
{शशंस सीतापरिदेवनान्तम्}
{अनुष्ठितम् शासनमग्रजाय} % १४-८३

\fourlineindentedshloka
{बभूव रामः सहसा सबाष्पः}
{तुषारवर्षीव सहस्यचन्द्रः}
{कौलीनभीतेन गृहान्निरस्ता}
{न तेन वैदेहसुता मनस्तः} % १४-८४

\fourlineindentedshloka
{निगृह्य शोकम् स्वयमेव धीमान्}
{वर्णाश्रमावेक्षणजागरूकः}
{स भ्रातृसाधारणभोगमृद्धम्}
{राज्यम् रजोरिक्तमनाः शशास} % १४-८५

\fourlineindentedshloka
{तामेकभार्याम् परिवादभीरोः}
{साध्वीमपि त्यक्तवतो नृपस्य}
{वक्षस्यसङ्घट्टसुखम् वसन्ती}
{रेजे सपत्नीरहितेव लक्ष्मीः} % १४-८६

\fourlineindentedshloka
{सीताम् हित्वा दशमुखरिपुर्नोपमेये यदन्याम्}
{तस्या एव प्रतिकृतिसखो यत्क्रतूनाजहार}
{वृत्तान्तेन श्रवणविषयप्रापिणा तेन भर्तुः}
{सा दुर्वारम् कथमपि परित्यागदुःखम् विषेहे} % १४-८७

॥इति श्री-महाकवि-कालिदास-कृत-रघुवंश-महाकाव्ये चतुर्दशः सर्गः॥
