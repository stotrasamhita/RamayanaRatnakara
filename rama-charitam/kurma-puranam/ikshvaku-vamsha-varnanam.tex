\chapt{कूर्म-पुराणम्}

\sect{इक्ष्वाकु-वंश-वर्णनम्}

\src{कूर्मपुराणम्}{}{अध्यायः २१}{श्लोकाः १६--}
\vakta{}
\shrota{}
\notes{Brief story of Rama, in the context of Ikshvaku dynasty. Notable is the mention of Rameshvaram temple, and the Shiva linga installed by Rama.}
\textlink{https://sa.wikisource.org/wiki/कूर्मपुराणम्-पूर्वभागः/एकविंशतितमोऽध्यायः}
\translink{}

\storymeta

\uvacha{सूत उवाच}
\twolineshloka
{त्रिधन्वा राजपुत्रस्तु धर्मेणापालयन्महीम्}
{तस्य पुत्रोऽभवद् विद्वांस्त्रय्यारुण इति स्मृतः} %२१.१

\twolineshloka
{तस्य सत्यव्रतो नाम कुमारोऽभून्महाबलः}
{भार्या सत्यधना नाम हरिश्चन्द्रमजीजनत्} %२१.२

\twolineshloka
{हरिश्चन्द्रस्य पुत्रोऽभूद् रोहितो नाम वीर्यवान्}
{रोहितस्य वृकः पुत्रः तस्मात्बाहुरजायत} %२१.३

\threelineshloka*
{हरितो रोहितस्याथ धुन्धुस्तस्य सुतोऽभवत्}
{विजयश्च सुदेवश्च धुन्धुपुत्रौ बभूवतुः}
{विजयस्याभवत् पुत्रः कारुको नाम वीर्यवान्}

\twolineshloka
{सगरस्तस्य पुत्रौऽभूद् राजा परमधार्मिकः}
{द्वे भार्ये सगरस्यापि प्रभा भानुमती तथा} %२१.४

\twolineshloka
{ताभ्यामाराधितः वह्निः प्रादादौ वरमुत्तमम्}
{एकं भानुमती पुत्रमगृह्णादसमञ्जसम्} %२१.५

\twolineshloka
{प्रभा षष्टिसहस्त्रं तु पुत्राणां जगृहे शुभा}
{असमञ्सस्य तनयो ह्यंशुमान् नाम पार्थिवः} %२१.६

\twolineshloka
{तस्य पुत्रो दिलीपस्तु दिलीपात् तु भगीरथः}
{येन भागीरथी गङ्गा तपः कृत्वाऽवतारिता} %२१.७

\twolineshloka
{प्रसादाद् देवदेवस्य महादेवस्य धीमतः}
{भगीरथस्य तपसा देवः प्रीतमना हरः} %२१.८

\twolineshloka
{बभार शिरसा गङ्गां सोमान्ते सोमभूषणः}
{भगीरथसुतश्चापि श्रुतो नाम बभूव ह} %२१.९

\twolineshloka
{नाभागस्तस्य दायादः सिन्धुद्वीपस्ततोऽभवत्}
{अयुतायुः सुतस्तस्य ऋतुपर्णस्तु तत्सुतः} %२१.१०

\twolineshloka
{ऋतुपर्णस्य पुत्रोऽभूत् सुदासो नाम धार्मिकाः}
{सौदासस्तस्य तनयः ख्यातः कल्माषपादकः} %२१.११

\twolineshloka
{वसिष्ठस्तु महातेजाः क्षेत्रे कल्माषपादके}
{अश्मकं जनयामसा तमिक्ष्वाकुकुलध्वजम्} %२१.१२

\twolineshloka
{अश्मकस्योत्कलायां तु नकुलो नाम पार्थिवः}
{स हि रामभयाद् राजा वनं प्राप सुदुः खितः} %२१.१३

\twolineshloka
{विभ्रत् स नारीकवचं तस्माच्छतरथोऽभवत्}
{तस्माद् बिलिबिलिः श्रीमान्‌वृद्धशर्माचतत्सुतः} %२१.१४

\twolineshloka
{तस्माद् विश्वसहस्तस्मात् खट्वाङ्ग इति विश्रुतः}
{दीर्घबाहुः सुतस्तस्य रघुस्तस्मादजायत} %२१.१५

\twolineshloka
{रघोरजः समुत्पन्नो राजा दशरथस्ततः}
{रामो दाशरथिर्वोरो धर्मज्ञो लोकविश्रुतः} %२१.१६

\twolineshloka
{भरतो लक्ष्मणश्चैव शत्रुघ्नश्च महाबलः}
{सर्वे शक्रसमा युद्धे विष्णुशक्तिसमन्विताः} %२१.१७

\twolineshloka
{जज्ञे रावणनाशार्थं विष्णुरंशेन विश्वकृत्}
{रामस्य सुभगा भार्या जनकस्यात्मजा शुभा} %२१.१८

\twolineshloka
{सीता त्रिलोकविख्याता शीलौदार्यगुणान्विता}
{तपसा तोषिता देवी जनकेन गिरीन्द्रजा} %२१.१९

\twolineshloka
{प्रायच्छज्जानकीं सीतां राममेवाश्रितां पतिम्}
{प्रीतश्च भगवानीशस्त्रिशूली नीललोहितः} %२१.२०

\twolineshloka
{प्रददौ शत्रुनाशार्थं जनकायाद्‌भुतं धनुः}
{स राजा जनको विद्वान् दातुकामः सुतामिमाम्} %२१.२१

\twolineshloka
{अघोषयदमित्रघ्नो लोकेऽस्मिन् द्विजपुङ्गवाः}
{इदं धनुः समादातुं यः शक्नोति जगत्त्रये} %२१.२२

\twolineshloka
{देवो वा दानवो वाऽपि स सीतां लब्धुमर्हति}
{विज्ञाय रामो बलवान् जनकस्य गृहं प्रभुः} %२१.२३

\twolineshloka
{भञ्जयामास चादाय गत्वाऽसौ लीलयैव हि}
{उद्ववाह च तां कन्यां पार्वतीमिव शङ्करः} %२१.२४

\twolineshloka
{रामः परमधर्मात्मा सेनामिव च षण्मुखः}
{ततो बहुतिथे काले राजा दशरथः स्वयम्} %२१.२५

\twolineshloka
{रामं ज्येष्ठं सुतं वीरं राजानं कर्तुमारभत्}
{तस्याथ पत्नी सुभगा कैकेयी चारुभाषिणी} %२१.२६

\twolineshloka
{निवारयामास पतिं प्राह सम्भ्रान्तमानसा}
{मत्सुतं भरतं वीरं राजानं कर्त्तुमर्हसि} %२१.२७

\twolineshloka
{पूर्वमेव वरो यस्माद् दत्तो मे भवता यतः}
{स तस्या वचनं श्रुत्वा राजा दुःखितमानसः} %२१.२८

\twolineshloka
{बाढमित्यब्रवीद् वाक्यं तथा रामोऽपि धर्मवित्}
{प्रणम्याथ पितुः पादौ लक्ष्मणेन सहाच्युतः} %२१.२९

\twolineshloka
{ययौ वनं सपत्नीकः कृत्वा समयमात्मवान्}
{संवत्सराणां चत्वारि दश चैव महाबलः} %२१.३०

\twolineshloka
{उवास तत्र मतिमान् लक्ष्मणेन सह प्रभुः}
{कदाचिद् वसतोऽरण्ये रावणो नाम राक्षसः} %२१.३१

\twolineshloka
{परिव्राजकवेषेण सीतां हृत्वा ययौ पुरीम् ।।}
{अदृष्ट्वा लक्ष्मणो रामः सीतामाकुलितेन्द्रियौ} %२१.३२

\twolineshloka
{दुः खशोकाभिसन्तप्तौ बभूवतुररिन्दमौ}
{ततः कदाचित् कपिना सुग्रीवेण द्विजोत्तमाः} %२१.३३

\twolineshloka
{वानराणामभूत् सख्यं रामस्याक्लिष्टकर्मणः ।।}
{सुग्रीवस्यानुगो वीरो हनुमान्नाम वानरः} %२१.३४

\twolineshloka
{वायुपुत्रौ महातेजा रामस्यासीत् प्रियः सदा}
{स कृत्वा परमं धैर्यं रामाय कृतनिश्चयः} %२१.३५

\twolineshloka
{आनयिष्यामि तां सीतामित्युक्त्वा विचचार ह}
{महीं सागरपर्यन्तां सीतादर्शनतत्परः} %२१.३६

\twolineshloka
{जगाम रावणपुरीं लङ्कां सागरसंस्थिताम्}
{तत्राथ निर्जने देशे वृक्ष्मूले शुचिस्मिताम्} %२१.३७

\twolineshloka
{अपश्यदमलां सीतां राक्षसीभिः समावृताम्}
{अश्रुपूर्णेक्षणां हृद्यां संस्मरन्तीमनिन्दिताम्} %२१.३८

\twolineshloka
{राममिन्दीवरश्यामं लक्ष्मणं चात्मसंस्थितम्}
{निवेदयित्वा चात्मानं सीतायै रहसि स्वयम्} %२१.३९

\twolineshloka
{असंशयाय प्रददावस्यै रामाङ्‌गुलीयकम्}
{दृष्ट्वाऽङ्‌गुलीयकं सीता पत्युः परमशोभनम्} %२१.४०

\twolineshloka
{मेने समागतं रामं प्रीतिविस्फारितेक्षणा}
{समाश्वास्य तदा सीतां दृष्ट्वा रामस्य चान्तिकम्} %२१.४१

\twolineshloka
{नयिष्ये त्वां महाबाहुरुक्त्वा रामं ययौ पुनः}
{निवेदयित्वा रामाय सीतादर्शनमात्मवान्} %२१.४२

\twolineshloka
{तस्थौ रामेण पुरतो लक्ष्मणेन च पूजितः}
{ततः स रामो बलवान् सार्द्धं हनुमता स्वयम्} %२१.४३

\twolineshloka
{लक्ष्मणेन च युद्धाय बुद्धिं चक्रे हि रक्षसाम्}
{कृत्वाऽथ वानरशतैर्लङ्कामार्गं महोदधेः} %२१.४४

\twolineshloka
{सेतुं परमधर्मात्मा रावणं हतवान् प्रभुः}
{सपत्नीकं च ससुतं सभ्रातृकमरिन्दमः} %२१.४५

\twolineshloka
{आनयामास तां सीतां वायुपुत्रसहायवान्}
{सेतुमध्ये महादेवमीशानं कृत्तिवाससम्} %२१.४६

\twolineshloka
{स्थापयामास लिङ्गस्थं पूजयामास राघवः}
{तस्य देवो महादेवः पार्वत्या सह शङ्करः} %२१.४७

\twolineshloka
{प्रत्यक्षमेव भगवान् दत्तवान् वरमुत्तमम्}
{यत् त्वया स्थापितं लिङ्गं द्रक्ष्यन्तीह द्विजातयः} %२१४८

\twolineshloka
{महापातकसंयुक्तास्तेषां पापं विनङ्क्ष्यति}
{अन्यानि चैव पापानि स्नातस्यात्र महोदधौ} %२१.४९

\twolineshloka
{दर्शनादेव लिङ्गस्य नाशं यान्ति न संशयः}
{यावत् स्थास्यन्ति गिरयो यावदेषा च मेदिनी} %२१.५०

\twolineshloka
{यावत् सेतुश्च तावच्च स्थास्याम्यत्र तिरोहितः}
{स्नानं दानं जपः श्राद्धं भविष्यत्यक्षयं महत्} %२१.५१

\twolineshloka
{स्मरणादेव लिङ्गस्य दिनपापं प्रणश्यति}
{इत्युक्त्वा भगवाञ्छम्भुः परिष्वज्य तु राघवम्} %२१.५२

\twolineshloka
{सनन्दी सगणो रुद्रस्तत्रैवान्तरधीयत}
{रामोऽपि पालयामास राज्यं धर्मपरायणः} %२१.५३

\twolineshloka
{अभिषिक्तो महातेजा भरतेन महाबलः}
{विशेषाढ् ब्राह्मणान् सर्वान् पूजयामसचेश्वरम्} %२१.५४

\twolineshloka
{यज्ञेन यज्ञहन्तारमश्वमेधेन शङ्करम्}
{रामस्य तनयो जज्ञे कुश इत्यभिविश्रुतः} %२१.५५

\twolineshloka
{लवश्च सुमहाभागः सर्वतत्त्वार्थवित् सुधीः}
{अतिथिस्तु कुशाज्जज्ञे निषधस्तत्सुतोऽभवत्} %२१.५६

\twolineshloka
{नलस्तु निषधस्याभून्नभास्तमादजायत}
{नभसः पुण्डरीकाक्षः क्षेमधन्वा च तत्सुतः} %२१.५७

\twolineshloka
{तस्य पुत्रोऽभवद् वीरो देवानीकः प्रतापवान्}
{अहीनगुस्तस्य सुतो सहस्वांस्तत्सुतोऽभवत्} %२१.५८

\twolineshloka
{तस्माच्चन्द्रावलोकस्तु तारापीडस्तु तत्सुतः}
{तारापीडाच्चन्द्रगिरिर्भानुवित्तस्ततोऽभवत्} %२१.५९

\twolineshloka
{श्रुतायुरभवत् तस्मादेते इक्ष्वाकुवंशजाः}
{सर्वे प्राधान्यतः प्रोक्ताः समासेन द्विजोत्तमाः} %२१.६०

\twolineshloka
{य इमं श्रृणुयान्नित्यमिक्ष्वाकोर्वंशमुत्तमम्}
{सर्वपापविनिर्मुक्तो स्वर्गलोके महीयते} %२१.६१

॥इति श्रीकूर्मपुराणे षट्‌साहस्त्र्यां संहितायां पूर्वविभागे इक्ष्वाकुवंशवर्णनं नाम एकविंशोऽध्यायः॥

\closesection