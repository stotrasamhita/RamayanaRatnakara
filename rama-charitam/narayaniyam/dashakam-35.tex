\sect{दशकं ३५ --- श्रीरामचरितवर्णनम्}

\src{श्रीमन्नारायणीयम्}{पञ्चत्रिंश-दशकं}{}{श्लोकाः १--१०}
\vakta{शुकः}
\shrota{परीक्षितः}
\tags{concise, complete}
\notes{This chapter summarises the events following the death of Vāli, including the alliance with Sugrīva, the search for Sītā, Setubandhanam, the eventual victory over Rāvaṇa, and the establishment of Rāma's rule in Ayodhyā.}
\textlink{http://stotrasamhita.net/wiki/Narayaniyam/Dashaka_35}
\translink{}

\storymeta


\fourlineindentedshloka
{नीतस्सुग्रीवमैत्रीं तदनु दुन्दुभेः कायमुच्चैः}
{क्षिप्त्वाङ्गुष्ठेन भूयो लुलविथ युगपत्पत्रिणा सप्त सालान्}
{हत्वा सुग्रीवघातोद्यतमतुलबलं वालिनं व्याजवृत्त्या}
{वर्षावेलामनैषीर्विरहतरळितस्त्वं मतङ्गाश्रमान्ते} %॥१॥

\fourlineindentedshloka
{सुग्रीवेणानुजोक्त्या सभयमभियता व्यूहितां वाहिनीं ता-}
{मृक्षाणां वीक्ष्य दिक्षु द्रुतमथ दयितामार्गणायावनम्राम्}
{सन्देशं चान्गुलीयं पवनसुतकरे प्रादिशो मोदशाली}
{मार्गे मार्गे ममार्गे कपिभिरपि तदी त्वत्प्रिया सप्रयासैः} %॥२॥

\fourlineindentedshloka
{त्वद्वार्ताकर्णनोद्यद्गरुदुरुजवसम्पातिसम्पातिवाक्य-}
{प्रोत्तीर्णार्णोधिरन्तर्नगरि जनकजां वीक्ष्य दत्त्वाऽङ्गुलीयम्}
{प्रक्षुद्योद्यानमक्षक्षपणचणरणः सोढबन्धो दशास्यम्}
{दृष्ट्वा प्लुष्ट्वा च लङ्कां झटिति स हनुमान्मौलिरत्नं ददौ ते} %॥३॥

\fourlineindentedshloka
{त्वं सुग्रीवाङ्गदादिप्रबलकपिचमूचक्रविक्रान्तभूमी-}
{चक्रोऽभिक्रम्य पारेजलधि निशिचरेन्द्रानुजाश्रीयमाणः}
{तत्प्रोक्तां शत्रुवार्तां रहसि निशमयन्प्रार्थनापार्थ्यरोष-}
{प्रास्ताग्नेयास्त्रतेजस्त्रसदुदधिगिरा लब्धवान्मध्यमार्गम्} %॥४॥

\fourlineindentedshloka
{कीशैराशान्तरोपाहृतगिरिनिकरैः सेतुमाधाप्य यातो}
{यातून्यामर्द्य दंष्ट्रानखशिखरिशिलासालशस्त्रैः स्वसैन्यैः}
{व्याकुर्वन्सानुजस्त्वं समरभुवि परं विक्रमं शक्रजेत्रा}
{वेगान्नागास्त्रबद्धः पतगपतिगरुन्मारुतैर्मोचितोऽभूः} %॥५॥

\fourlineindentedshloka
{सौमित्रिस्त्वत्र शक्तिप्रहृतिगळदसुर्वातजानीतशैल-}
{घ्राणात्प्रणानुपेतो व्यकृणुत कुसृतिश्लाघिनं मेघनादम्}
{मायाक्षोभेषु वैभीषणवचनहृतस्तम्भनः कुम्भकर्णम्}
{सम्प्राप्तं कम्पितोर्वीतलमखिलचमूभक्षिणं व्यक्षिणोस्त्वम्} %॥६॥

\fourlineindentedshloka
{गृह्णन् जम्भारिसम्प्रेषितरथकवचौ रावणेनाभियुध्यन्}
{ब्रह्मास्त्रेणास्य भिन्दन् गळततिमबलामग्निशुद्धां प्रगृह्णन्}
{देव श्रेणीवरोज्जीवितसमरमृतैरक्षतैऱ्क्षसङ्घैर्-}
{लङ्काभर्त्रा च साकं निजनगरमगाः सप्रियः पुष्पकेण} %॥७॥

\fourlineindentedshloka
{प्रीतो दिव्याभिषेकैरयुतसमधिकान्वत्सरान्पर्यरंसी-}
{र्मैथिल्यां पापवाचा शिव शिव किल तां गर्भिणीमभ्यहासीः}
{शत्रुघ्नेनार्दयित्वा लवणनिशिचरं प्रार्दयः शूद्रपाशम्}
{तावद्वाल्मीकिगेहे कृतवसतिरुपासूत सीता सुतौ ते} %॥८॥

\fourlineindentedshloka
{वाल्मीकेस्त्वत्सुतोद्गापितमधुरकृतेराज्ञया यज्ञवाटे}
{सीतां त्वय्याप्तुकामे क्षितिमविशदसौ त्वं च कालार्थितोऽभूः}
{हेतोः सौमित्रिघाती स्वयमथ सरयूमग्ननिश्शेषभृत्यैः}
{साकं नाकं प्रयातो निजपदमगमो देव वैकुण्ठमाद्यम्} %॥९॥

\fourlineindentedshloka
{सोऽयं मर्त्यावतारस्तव खलु नियतं मर्त्यशिक्षार्थमेवम्}
{विश्लेषार्तिर्निरागस्त्यजनमपि भवेत्कामधर्मातिसक्त्या}
{नो चेत्स्वात्मानुभूतेः क्वनु तव मनसो विक्रिया चक्रपाणे}
{स त्वं सत्त्वैकमूर्ते पवनपुरपते व्याधुनु व्याधितापान्} %॥१०॥

॥इति श्रीमन्नारायणीये श्रीरामचरितवर्णनं नाम पञ्चत्रिंश-दशकं सम्पूर्णम्॥

\closesection