\sect{रामनाथलिङ्गप्रतिष्ठाविधिवर्णनप्रसङ्गे राम-स्तोत्रम्}

\uvacha{मुनय ऊचुः}

\twolineshloka
{नमस्ते रामचन्द्राय लोकानुग्रहकारिणे}
{अरावणं जगत्कर्तुमवतीर्णाय भूतले}%॥ ६३ ॥

\twolineshloka
{ताटिकादेहसंहर्त्रे गाधिजाध्वररक्षिणे}
{नमस्ते जितमारीच सुबाहुप्राणहारिणे}%॥ ६४ ॥

\twolineshloka
{अहल्यामुक्तिसन्दायिपादपङ्कजरेणवे}
{नमस्ते हरकोदण्डलीलाभञ्जनकारिणे}%॥ ६५ ॥

\twolineshloka
{नमस्ते मैथिलीपाणिग्रहणोत्सवशालिने}
{नमस्ते रेणुकापुत्रपराजयविधायिने}%॥ ६६ ॥

\twolineshloka
{सहलक्ष्मणसीताभ्यां कैकेय्यास्तु वरद्वयात्}
{सत्यं पितृवचः कर्तुं नमो वनमुपेयुषे}%॥ ६७ ॥

\twolineshloka
{भरतप्रार्थनादत्तपादुकायुगुलाय ते}
{नमस्ते शरभङ्गस्य स्वर्गप्राप्त्यैकहेतवे}%॥ ६८ ॥

\twolineshloka
{नमो विराधसंहर्त्रे गृधराजसखाय ते}
{मायामृगमहाक्रूरमारीचाङ्गविदारिणे}%॥ ६९ ॥

\twolineshloka
{सीतापहारिलोकेशयुद्धत्यक्तकलेवरम्}
{जटायुषं तु सन्दह्य तत्कैवल्यप्रदायिने}%॥ ७० ॥

\twolineshloka
{नमः कबन्धसंहर्त्रे शबरीपूजिताङ्घ्रये}
{प्राप्तसुग्रीवसख्याय कृतवालिवधाय ते}%॥ ७१ ॥

\twolineshloka
{नमः कृतवते सेतुं समुद्रे वरुणालये}
{सर्वराक्षससंहर्त्रे रावणप्राणहारिणे}%॥ ७२ ॥

\twolineshloka
{संसाराम्बुधिसन्तारपोतपादाम्बुजाय ते}
{नमो भक्तार्तिसंहर्त्रे सच्चिदानन्दरूपिणे}%॥ ७३ ॥

\twolineshloka
{नमस्ते रामभद्राय जगतामृद्धिहेतवे}
{रामादिपुण्यनामानि जपतां पापहारिणे}%॥ ७४ ॥

\twolineshloka
{नमस्ते सर्वलोकानां सृष्टिस्थित्यन्तकारिणे}
{नमस्ते करुणामूर्ते भक्तरक्षणदीक्षित}%॥ ७८५ ॥

\twolineshloka
{ससीताय नमस्तुभ्यं विभीषणसुखप्रद}
{लङ्केश्वरवधाद्राम पालितं हि जगत्त्वया}%॥ ७६ ॥

\twolineshloka
{रक्ष रक्ष जगन्नाथ पाह्यस्माञ्जानकीपते}
{स्तुत्वैवं मुनयः सर्वे तूष्णीं तस्थुर्द्विजोत्तमाः}%॥ ७७ ॥

\uvacha{श्रीसूत उवाच}

\twolineshloka
{य इदं रामचन्द्रस्य स्तोत्रं मुनिभिरीरितम्}
{त्रिसन्ध्यं पठते भक्त्या भुक्तिं मुक्तिं च विन्दति}%॥ ७८ ॥

\twolineshloka
{प्रयाणकाले पठतो न् भीतिरुपजायते}
{एतत्स्तोत्रस्य पठनाद्भूतवेतालकादयः}%॥ ७९ ॥

\twolineshloka
{नश्यन्ति रोगा नश्यन्ति नश्यते पापसञ्चयः}
{पुत्रकामो लभेत्पुत्रं कन्या विन्दति सत्पतिम्}%॥ ८० ॥

\twolineshloka
{मोक्षकामो लभेन्मोक्षं धनकामो धनं लभेत्}
{सर्वान्कामानवाप्नोति पठन्भक्त्या त्विमं स्तवम्}%॥ ८१ ॥

॥इति श्रीस्कान्दे महापुराण एकाशीतिसाहस्र्यां संहितायां तृतीये ब्रह्मखण्डे सेतुमाहात्म्ये रामनाथलिङ्गप्रतिष्ठाविधिवर्णनं नाम चतुश्चत्वारिंशोऽध्याये मुनिभिरीरित-राम-स्तोत्रं सम्पूर्णम्॥

