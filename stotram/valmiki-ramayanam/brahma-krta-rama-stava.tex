% !TeX program = XeLaTeX
% !TeX root = ../../shloka.tex

\sect{ब्रह्मकृतरामस्तवः}

\twolineshloka
{ततो हि दुर्मना रामः श्रुत्वैवं वदतां गिरः}
{दध्यौ मुहूर्तं धर्मात्मा बाष्पव्याकुललोचनः}% ॥ १ ॥


\twolineshloka
{ततो वैश्रवणो राजा यमश्च पितृभिः सह}
{सहस्राक्षश्च देवेशो वरुणश्च जलेश्वरः}% ॥ २ ॥


\twolineshloka
{षडर्धनयनः श्रीमान्महादेवो वृषध्वजः}
{कर्ता सर्वस्य लोकस्य ब्रह्मा ब्रह्मविदां वरः}% ॥ ३ ॥


\twolineshloka
{एते सर्वे समागम्य विमानैः सूर्यसन्निभैः}
{आगम्य नगरीं लङ्काम् अभिजग्मुश्च राघवम्}% ॥ ४ ॥


\twolineshloka
{ततः सहस्ताभरणान् प्रगृह्य विपुलान् भुजान्}
{अब्रुवंस्त्रिदशश्रेष्ठाः प्राञ्जलिं राघवं स्थितम्}% ॥ ५ ॥


\twolineshloka
{कर्ता सर्वस्य लोकस्य श्रेष्ठो ज्ञानविदां विभुः}
{उपेक्षसे कथं सीतां पतन्तीं हव्यवाहने}% ॥ ६ ॥


\twolineshloka
{कथं देवगणश्रेष्ठम् आत्मानं नावबुध्यसे}
{ऋतधामा वसुः पूर्वं वसूनां त्वं प्रजापतिः}% ॥ ७ ॥


\twolineshloka
{त्रयाणां त्वं हि लोकानाम् आदिकर्ता स्वयम्प्रभुः}
{रुद्राणामष्टमो रुद्रः साध्यानामसि पञ्चमः}% ॥ ८ ॥


\twolineshloka
{अश्विनौ चापि ते कर्णौ चन्द्रसूर्यौ च चक्षुषी}
{अन्ते चादौ च लोकानाम् दृश्यसे त्वं परन्तप}% ॥ ९ ॥


\twolineshloka
{उपेक्षसे च वैदेहीं मानुषः प्राकृतो यथा}
{इत्युक्तो लोकपालैस्तैः स्वामी लोकस्य राघवः}% ॥ १० ॥


\twolineshloka
{अब्रवीत् त्रिदशश्रेष्ठान् रामो धर्मभृतां वरः}
{आत्मानं मानुषं मन्ये रामं दशरथात्मजम्}% ॥ ११ ॥


\twolineshloka
{योऽहं यस्य यतश्चाहं भगवांस्तद् ब्रवीतु मे}
{इति ब्रुवन्तं काकुत्स्थं ब्रह्मा ब्रह्मविदां वरः}% ॥ १२ ॥


\twolineshloka
{अब्रवीच्छृणु मे राम सत्यं सत्यपराक्रम}
{भवान् नारायणो देवः श्रीमांश्चक्रायुधः प्रभुः}% ॥ १३ ॥


\twolineshloka
{एकशृङ्गो वराहस्त्वं भूतभव्यसपत्नजित्}
{अक्षरं ब्रह्म सत्यं च मध्ये चान्ते च राघव}% ॥ १४ ॥


\twolineshloka
{लोकानां त्वं परो धर्मो विष्वक्सेनश्चतुर्भुजः}
{शार्ङ्गधन्वा हृषीकेशः पुरुषः पुरुषोत्तमः}% ॥ १५ ॥


\twolineshloka
{अजितः खड्गधृद् विष्णुः कृष्णश्चैव सनातनः}
{सेनानीर्ग्रामणीश्च त्वं बुद्धिः सत्त्वं क्षमा दमः}% ॥ १६ ॥


\twolineshloka
{प्रभवश्चाप्ययश्च त्वम् उपेन्द्रो मधुसूदनः}
{इन्द्रकर्मा महेन्द्रस्त्वं पद्मनाभो रणान्तकृत्}% ॥ १७ ॥


\twolineshloka
{शरण्यं शरणं च त्वाम् आहुर्दिव्या महर्षयः}
{सहस्रशृङ्गो वेदात्मा शतशीर्षो महर्षभः}% ॥ १८ ॥


\twolineshloka
{त्वं त्रयाणां हि लोकानाम् आदिकर्ता स्वयम्प्रभुः}
{सिद्धानामपि साध्यानाम् आश्रयश्चासि पूर्वजः}% ॥ १९ ॥


\twolineshloka
{त्वं यज्ञस्त्वं वषट्कारस्त्वमोङ्कारः परात्परः}
{प्रभवं निधनं वा ते नो विदुः को भवानिति}% ॥ २० ॥


\twolineshloka
{दृश्यसे सर्वभूतेषु ब्राह्मणेषु च गोषु च}
{दिक्षु सर्वासु गगने पर्वतेषु वनेषु च}% ॥ २१ ॥


\twolineshloka
{सहस्रचरणः श्रीमान् शतशीर्षः सहस्रदृक्}
{त्वं धारयसि भूतानि पृथिवीं च सपर्वताम्}% ॥ २२ ॥


\twolineshloka
{अन्ते पृथिव्याः सलिले दृश्यसे त्वं महोरगः}
{त्रीन् लोकान् धारयन् राम देवगन्धर्वदानवान्}% ॥ २३ ॥


\twolineshloka
{अहं ते हृदयं राम जिह्वा देवी सरस्वती}
{देवा गात्रेषु रोमाणि ब्रह्मणा निर्मिताः प्रभो}% ॥ २४ ॥


\twolineshloka
{निमेषस्ते भवेद् रात्रिरुन्मेषस्ते भवेद् दिवा}
{संस्कारास्तेऽभवन् वेदा न तदस्ति त्वया विना}% ॥ २५ ॥


\twolineshloka
{जगत् सर्वं शरीरं ते स्थैर्यं ते वसुधातलम्}
{अग्निः कोपः प्रसादस्ते सोमः श्रीवत्सलक्षणः}% ॥ २६ ॥


\twolineshloka
{त्वया लोकास्त्रयः क्रान्ताः पुरा स्वैर्विक्रमैस्त्रिभिः}
{महेन्द्रश्च कृतो राजा बलिं बद्ध्वा महासुरम्}% ॥ २७ ॥


\twolineshloka
{सीता लक्ष्मीर्भवान् विष्णुः देवः कृष्णः प्रजापतिः}
{वधार्थं रावणस्येह प्रविष्टो मानुषीं तनुम्}% ॥ २८ ॥


\twolineshloka
{तदिदं नः कृतं कार्यं त्वया धर्मभृतां वर}
{निहतो रावणो राम प्रहृष्टो दिवमाक्रम}% ॥ २९ ॥


\twolineshloka
{अमोघं बलवीर्यं ते अमोघस्ते पराक्रमः}
{अमोघं दर्शनं राम अमोघस्तव संस्तवः}% ॥ ३० ॥


\twolineshloka
{अमोघास्ते भविष्यन्ति भक्तिमन्तो नरा भुवि}
{ये त्वां देवं ध्रुवं भक्ताः पुराणं पुरुषोत्तमम्}% ॥ ३१ ॥


\threelineshloka
{प्राप्नुवन्ति सदा कामान् इह लोके परत्र च}
{इममार्षं स्तवं नित्यम् इतिहासं पुरातनम्}
{ये नराः कीर्तयिष्यन्ति तेषां नास्ति पराभवः} %॥ ३२ ॥

॥इत्यार्षे श्रीमद्रामायणे वाल्मीकीये आदिकाव्ये चतुर्विंशतिसहस्रिकायां संहितायां युद्धकाण्डे ब्रह्मकृतरामस्तवो नाम विंशत्यधिकशततमः सर्गः॥
