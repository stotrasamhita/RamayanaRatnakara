% !TeX program = XeLaTeX
% !TeX root = ./ramayana-sangraha-mulam.tex

\begingroup
\fontspec[Script=Devanagari]{Adobe Devanagari}
\fontsize{12pt}{14.4pt}\selectfont
\centerline{\large{ॐ}}
\centerline{॥श्री-गणेशाय नमः॥}
\centerline{॥श्री-गुरुभ्यो नमः॥}
\centerline{॥श्री-सीता-लक्ष्मण-भरत-शत्रुघ्न-हनुमत्-समेत-श्री-रामचन्द्राय नमः॥}

\thispagestyle{empty}

\begin{center}
\chapter*{{प्रस्तावना}}
\end{center}

\twolineshloka*{सदाशिवसमारम्भां शङ्कराचार्यमध्यमाम्}
{अस्मदाचार्यपर्यन्तां वन्दे गुरुपरम्पराम्}

\twolineshloka*
{एष सेतुर्विधरणो लोकासम्भेदहेतवे}
{कोदण्डेन च दण्डेन रामेण गुरुणा कृतः}

रामायण-श्रोतॄणां कदापि तृप्तिर्न जायते! यथा महर्षिः वाल्मीकिः वदति--- ``रामो रामो राम इति प्रजानामभवन् कथाः'', तद्वत् इतिहासपुराणेष्वपि श्रीरामचन्द्रस्य बहवः कथाः लभ्यन्ते। तेषाम् एकत्र प्रस्तुतिं कर्तुम् एषः प्रयासः। सीतादेवी अपि अध्यात्मरामायणे रामस्य अरण्यगमनप्रसङ्गे वदति---

\centerline{``रामायणानि बहुशः श्रुतानि बहुभिर्द्विजैः॥२-४-७७॥''} 

अस्मिन् ग्रन्थे अनेकरामकथाः प्रस्तुताः सन्ति। यद्यपि बहवः कथाः श्रीमाद्वाल्मीकिरामायण\-मनुसृत्य एव वर्तन्ते, काश्चित् कथाः तदतिक्रम्य अपि अन्यकल्पेषु ये केचित्विचित्राः कथाप्रसङ्गाः सन्ति तान् वर्णयन्ति (यथा पद्मपुराणे श्रीरामचन्द्रः स्वयं महादेवं पृच्छति!)। काश्चित् कथाः श्रीमद्वाल्मीकि\-रामायणस्य अन्तर्गत-घट्टानां विस्तृतप्रस्तुतिं कुर्वन्ति। महाभारतेऽपि भीष्मः हनुमान् (स्वानुजं भीमं प्रति) नारदः च विभिन्नेषु प्रसङ्गेषु रामकथां कथयन्ति। विशेषतः वनपर्वणि मारकण्डेयमहर्षिः रामोपाख्यानपर्वणि रामकथां विस्तरेण वर्णयति।

एतासां कथानां वक्तॄन् व्यासं वाल्मीकिं च नमस्कृत्य एतस्य ग्रन्थस्य पारायणम् आरभामहे। रामे अनन्यभक्तिः सदा भवतु नः। 

\twolineshloka*
{नारायणं नमस्कृत्य नरं चैव नरोत्तमम्}
{देवीं सरस्वतीं चैव ततो जयमुदीरयेत्}

\twolineshloka*
{कूजन्तं राम रामेति मधुरं मधुराक्षरम्}
{आरुह्य कविताशाखां वन्दे वाल्मीकिकोकिलम्}


\twolineshloka*
{यत्र यत्र रघुनाथकीर्तनं तत्र तत्र कृतमस्तकाञ्जलिम्}
{बाष्पवारिपरिपूर्णलोचनं मारुतिं नमत राक्षसान्तकम्}

\twolineshloka*
{रामं रामानुजं सीतां भरतं भरतानुजम्}
{सुग्रीवं वायुसूनुं च प्रणमामि पुनः पुनः}

\twolineshloka*
{नमोऽस्तु रामाय सलक्ष्मणाय देव्यै च तस्यै जनकात्मजायै}
{नमोऽस्तु रुद्रेन्द्रयमानिलेभ्यो नमोऽस्तु चन्द्रार्कमरुद्गणेभ्यः}

यथा श्रीमद्भगवद्\-गीतायां भगवान् श्रीकृष्ण आह, ``कथयन्तश्च मां नित्यं तुष्यन्ति च रमन्ति च॥'' तथा वयं सर्वेऽपि रामकथामृतं श्रुत्वा परस्परं च कथयित्वा रामस्य अनन्तकल्याणगुणान् अनुभूय तुष्टिं प्राप्नुयामः! बलं विष्णोः प्रवर्धताम्!\\



\centerline{सर्वम् श्री-सीतारामचन्द्रार्पणमस्तु॥}
\endgroup
\medskip
\noindent{आषाढ-कृष्ण-द्वादशी} \hfill कार्तिकः रामसूनूः\\
विश्वावसु-संवत्सरः ५१२७ कर्कटकः ७ \hfill सर्वज्ञात्म-प्रतिष्ठानम्\\
July 22, 2025


\vfill

\centerline{\large \textbf{Acknowledgments}}

\textsf{\scriptsize Really grateful to all the selfless volunteers who have scanned various old texts, performed OCR, uploaded on wikisource and other repositories. What I present here is a compilation of several such texts. Also very grateful to wisdomlib.org for some fantastic indices (and translations) to the Puranas, which enabled an easier search. %This is an ongoing work, and I will continue to add more texts (incl. Skanda Purana, Brahmavaivarta Purana, etc.) as I find time to typeset them.
 I initially began proofreading the texts as I inserted them, but could not keep pace, especially as I encountered some of the massive texts like Padma Puranam --- in the fullness of time, and with the grace of Bhagavan Rama, hope that this project will see fruition at some point. But it's always important to get the version 0.1 going. Jaya Shri Rama!}

\bigskip

\centerline{\textit{\scriptsize Last updated: \textbf{\today}}}