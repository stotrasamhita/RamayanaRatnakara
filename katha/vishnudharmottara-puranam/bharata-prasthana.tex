\chapt{विष्णुधर्मोत्तर-पुराणम्}

\src{विष्णुधर्मोत्तर-पुराणम्}{प्रथमखण्डः}{अध्यायः २०२}{}
\tags{concise, complete}
\notes{Bharata is sent by Rāma with a vast army to drive out the Gandharvas from Yudhājit’s lands. He marches in splendour, crossing sacred rivers and reaching the rich city of Takṣaśilā. After a fierce battle, the Gandharvas are defeated, Yudhājit’s sons are crowned, and Bharata returns to Ayodhyā in triumph.}
\textlink{}
\translink{}

\storymeta

\sect{भरतप्रस्थानवर्णनम् --- द्व्यधिकद्विशततमोऽध्यायः}

\uvacha{वज्र उवाच}

\twolineshloka
{शैलूषपुत्रा गन्धर्वा भरतेन कथं हताः}
{किमर्थं तु महाभाग तन्ममाचक्ष्व पृच्छतः}%।।१।।

\uvacha{मार्कण्डेय उवाच}

\twolineshloka
{अयोध्यायामयोध्यायां रामे दशरथात्मजे}
{कैकेयाधिपतिः श्रीमान्युधाजिन्नाम पार्थिवः}%।। २।।

\twolineshloka
{रामाय प्रेषयामास दूतं भरत मातुलः}
{वृद्धं पुरोहितं गार्ग्यं येन कार्येण तच्छृणु}%।।३।।

\twolineshloka
{सिन्धोरुभयकूलेषु रामदेशो मनोहरः}
{हत्वा रणे मनुष्येन्द्रान्गन्धर्वैर्विनिवेशित}%।।४।।

\twolineshloka
{गन्धर्वास्ते च राजेन्द्र राज्ञां विप्रियकारकाः}
{लक्ष्मणं भरतं वापि शत्रुघ्नमथवा नृप}%।। ५ ।।

\twolineshloka
{विसर्जयित्वा गन्धर्वांस्तान्विनाशय राघव}
{राजानो निर्भयाः सन्तु देशश्चास्तु तथा तव}%।। ६ ।।

\twolineshloka
{दूतस्य वचनं श्रुत्वा चिन्तयामास राघवः}
{मेघनादवधे कर्म लक्ष्मणेन महत्कृतम्}%।। ७ ।।

\twolineshloka
{शत्रुघ्नेन कृतं कर्म लवणं च विनिघ्नता}
{प्रेषयिष्यामि भरतं गन्धर्वस्य च निग्रहे}%।।८।।

\twolineshloka
{इत्येवं मनसि ध्यात्वा रामो भरतमब्रवीत्}
{गच्छ गार्ग्यं पुरस्कृत्य वत्स राजगृहं स्वयम्}%।।९।।

\twolineshloka
{मातुलेन समायुक्तः कैकयेन्द्रगृहात्ततः}
{जहि शैलूषतनयान्गन्धर्वान्पापनिश्चयान्}% ।।१० ।।

\twolineshloka
{एवमुक्तः स धर्मात्मा भरतो भ्रातृवत्सलः}
{रामस्य पादौ शिरसा चाभिवन्द्य कृताञ्जलिः}%।। ११ ।।

\twolineshloka
{गृहं गत्वा चकाराथ सर्वं प्रास्थानिकं विधिम्}
{ओषधीनां कषायेण तदोत्प्लावितविग्रहः}%।। १२ ।।

\twolineshloka
{गौरसर्षपकल्केन प्रसादितशिरोरुहः}
{तीर्थसारसनादेयैः सलिलैश्च स सागरैः}%।। १३ ।।

\twolineshloka
{चन्दनस्रावसम्मिश्रैः कुङ्कुमाक्षोदसंयुतैः}
{सर्वौषधिसमायुक्तैः सर्वगन्धसमन्वितैः}%।। १४ ।।

\twolineshloka
{मन्त्रपूतैर्महातेजाः सस्नौ राघववर्धनः}
{शङ्खभेरी निनादेन पणवानां स्वनेन च}%।। १५ ।।

\twolineshloka
{आनकानां च शब्देन निस्वनेन च बन्दिनाम्}
{सूतमागधशब्देन जयकारैस्तथैव च}%।। १६ ।।

\twolineshloka
{तुष्टवुः स्नानकाले तं स्तवैर्मङ्गलपाठकाः}
{तथोपतस्थुर्गीतेन गन्धर्वाप्सरसां गणाः}%।। १७ ।।

\twolineshloka
{स्नातः स भरतो लक्ष्म्या युवराजाभिरूपया}
{विलिप्य चारुसर्वाङ्गं चन्दनेन सुगन्धिना}%।। १८ ।।

\twolineshloka
{अहताम्बरसंवीतः श्वेतमाल्यविभूषणः}
{कुण्डली साङ्गदी मौली सर्वरत्नविभूषितः}%।। १९ ।।

\twolineshloka
{अन्तस्थं पूजयामास देवदेवं त्रिवक्रमम्}
{गन्धमाल्यनमस्कारधूपदीपादिकर्मणा} %॥२०॥

\twolineshloka
{पूजयित्वा जगन्नाथमुपतस्थे हुताशनम्}
{सुहुतं ब्राह्मणेन्द्रेण राघवाणां पुरोधसा}%।। २१ ।।

\twolineshloka
{गोभिर्वस्त्रैर्हिरण्यैश्च तुरङ्गकरिपुङ्गवैः}
{मोदकैः सफलैर्दध्ना गन्धैर्माल्यैस्तथाक्षतैः}%।। २२ ।।

\twolineshloka
{स्वस्तिवाद्यांस्तथा विप्रान्महात्मा भूरिदक्षिणः}
{व्याघ्रचर्मोत्तरे रम्ये सूपविष्टो वरासने}%।। २३ ।।

\twolineshloka
{आयुधानन्तरं चक्रे ध्वजच्छत्राभिपूजनम्}
{स्वस्तिकान्वर्धमानांश्च नन्द्यावर्तास्तथैव च}%।। २४ ।।

\twolineshloka
{नद्यः काञ्चनविन्यस्ताः शङ्खः सत्कमलाञ्जनम्}
{पूर्णकुम्भं गजमदं दूर्वाः सार्द्रं च गोमयम्}%।। २५ ।।

\twolineshloka
{रत्नान्यादाय बिल्वं च चापमादाय सत्वरः}
{सशरं राघवश्रेष्ठः पदद्वात्रिंशकं ययौ}%।। २६ ।।

\twolineshloka
{श्रेष्ठमश्वं सुचन्द्राभं हेमभाण्डपरिच्छदम्}
{आरुह्य निर्ययौ श्रीमाञ्जयकाराभिपूजितः}%।। २७ ।।

\twolineshloka
{पौरजानपदामात्यैर्वाद्यघोषेण भूरिणा}
{ह्रादेन गजघण्टानां बृंहितेन पुनःपुनः}%।। २८।।

\twolineshloka
{हेषितेन तुरङ्गाणां नराणां क्ष्वेडितेन च}
{द्विजपुण्याहघोषेण प्रयातो भूरिदक्षिणः}%।। २९ ।।

\twolineshloka
{भरतस्य प्रयाणे तु देवाः शक्पुरोगमाः}
{मुमुचुः पुष्पवर्षाणि वाच ऊचुः शुभास्तथा} %॥३०॥

\twolineshloka
{एष विग्रहवान्धर्म एष सत्यवतां वरः}
{एष वीर्यवतां श्रेष्ठो रूपेणाप्रतिमो भुवि}%।। ३१ ।।

\twolineshloka
{अनेन यत्कृतं कर्म रामे वनमुपागते}
{न तस्य कर्ता लोकेऽस्मिन्दिवि वा विद्यते क्वचित्}%।। ३२ ।।

\twolineshloka
{अनेन राज्यं सन्त्यक्तं गृहं दग्धमिवाग्निना}
{अनेन दुःखशय्यासु शयितं समहात्मना}%।। ३३ ।।

\twolineshloka
{नित्यमासन्नभोगेन जटावल्कलधारिणा}
{फलमूलाशिनानेन रामराज्यं हि पालितम्}%।। ३४ ।।

\twolineshloka
{शृण्वन्सुवाक्यानि सुरेरितानि रामानुजो रामगृहं जगाम}
{शूरार्यविद्वत्पुरुषोपकीर्णं रत्नैर्यथा सागरमप्रमेयम्}%।। ३५ ।।

॥इति श्रीविष्णुधर्मोत्तरे प्रथमखण्डे मार्कण्डेयवज्रसंवादे भरतप्रस्थानवर्णनं नाम द्व्यधिकद्विशततमोऽध्यायः॥२०२॥

\sect{आनुयात्रिकवर्णनम् --- त्र्यधिकद्विशततमोऽध्यायः}

\uvacha{मार्कण्डेय उवाच}

\twolineshloka
{गृहात्प्रयाते भरते प्रस्थानार्थिनि यादव}
{प्रागेव लक्ष्मणं रामो नित्योद्युक्तमभाषत}%।। १ ।।

\twolineshloka
{अनुयानं कुमारस्य भरतस्य महात्मनः}
{प्रेषयाश्वसहस्राणां शतानि त्रीणि राघव}%।। २ ।।

\twolineshloka
{दश दन्तिसहस्राणि रथानां षड्गुणानि च}
{कोट्यः पञ्च पदातीनां समरेष्वनिवर्तिनाम्}%।। ३ ।।

\twolineshloka
{धनाध्यक्षास्तथा वत्समनुगच्छन्तु लक्ष्मण}
{गोरथैश्च तथा पुष्टैर्गोभिरुष्ट्रैस्तथैव च}%।। ४ ।।

\twolineshloka
{व्यायतैः पुरुषैरश्वैर्गर्दभैश्च तथा वरैः}
{वस्त्ररूप्यसुवर्णानां मणीनामपि भागशः}%।। ५ ।।

\twolineshloka
{विसर्गाय कुमारस्य परिपूर्णा यथासुखम्}
{ब्राह्मणाः कथया मुख्यास्तथैव नटवर्तकाः}%।। ६ ।।

\twolineshloka
{गीते नृत्ते तथा लास्ये प्रवीणाश्च वराङ्गनाः}
{प्रास्थानिकाश्च ये केचित्पानविक्रयिणश्च ये}%।। ७ ।।

\twolineshloka
{रूपाजीवाश्च वणिजो नानापण्योपजीविनः}
{नानारूपमुपादाय बहुपण्यं व्रजन्तु वै}%।।८।।

\twolineshloka
{विषवैद्याः शल्यवैद्यास्तथा कायचिकित्सकाः}
{कर्मन्तिका स्थपतयो मार्गिणो वृक्षरोपकाः}%।। ९ ।।

\twolineshloka
{कूपकाराः सुधाकारा वंशकर्मकतस्तथा}
{वाणिक्काराश्च ये केचित्कूर्चकाराश्च शोभनाः} %॥१०॥

\twolineshloka
{परिकर्मकृतश्चैव तथा वस्त्रोपजीविनः}
{मायूरिकास्तैत्तिरिकाश्चेतका भेदकाश्च ये}%।। ११ ।।

\twolineshloka
{रञ्जका दन्तकाराश्च तथा दन्तोपजीविनः}
{एरण्डवेत्रकाराश्च कटकाराश्च शोभनाः}%।। १२ ।।

\twolineshloka
{आरकूटकृतश्चैव तात्रकूटास्तथैव च}
{भूर्जकूटाः खड्गकारा गुडसीधुप्रपाचकाः}%।। १३ ।।

\twolineshloka
{औरभ्रका माहिषकास्तुन्नवायाश्च लक्ष्मण}
{ये चाभिष्टावकाः केचित्सूतमागधबन्दिनः}%।। १४ ।।

\twolineshloka
{चैलनिर्णेजकाश्चैव चर्मकारास्तथैव च}
{अङ्गारकोराश्च तथा लुब्धका ये च धीवराः}%।। १५ ।।

\twolineshloka
{कबन्धधारिणो ये च ये च काष्ठप्रपाटकाः}
{वस्त्रसीवनसक्ताश्च गृहकाराश्च ये नराः}%।। १६ ।।

\twolineshloka
{कुम्भकाराश्च ये केचिच्छ्मश्रुवर्धकिनश्च ये}
{लेखका गणका ये च तथा तन्दुलकारकाः}%।। १७ ।।

\twolineshloka
{सक्तुकाराश्च ये केचिच्छाकपण्योपजीविनः}
{तैलिका गान्धिकाश्चैव तीर्थसंशुद्धिकारकाः}%।।१६।।

\twolineshloka
{चित्रकर्मविदो ये च ये च लाङ्गूलिका जनाः}
{सूताः पौरोगवाश्चैव सौविदल्लाश्च लक्ष्मण}%।। १९ ।।

\twolineshloka
{गोपा वनचरा ये च नदीतीरविचारणाः}
{गोसङ्घैर्महिषीसङ्घैस्तेऽनुयान्तु यथासुखम्}% ।।२० ।।

\twolineshloka
{श्रेणीमहत्तरा ये च ग्रामघोषमहत्तराः}
{तथैवाटविका ये च ये च शैलविचारिणः}%।। २१ ।।

\twolineshloka
{सौमित्रे सर्व एवैते सुभृताश्च सुपूजिताः}
{अनुयान्तु कुमारं मे भरतं भ्रातृवत्सलम्}%।। २२ ।।

\twolineshloka
{इत्येवमुक्तः स तु तान्समस्तानाज्ञापयामास नरेन्द्रवाक्यात्}
{आमन्त्र्य रामं शिरसा च सर्वे विनिर्ययुस्ते नगरात्प्रहृष्टाः}%।।२३।।

॥इति श्रीविष्णुधर्मोत्तरे प्रथमखण्डे मार्कण्डेयवज्रसंवादे आनुयात्रिकवर्णनन्नाम त्र्यधिकद्विशततमोऽध्यायः॥२०३॥

\sect{भरतनिर्याणवर्णनम् --- चतुरधिकद्विशततमोऽध्यायः}

\uvacha{मार्कण्डेय उवाच}

\twolineshloka
{एतस्मिन्नेव काले तु रामः शुश्राव तन्महत्}
{शङ्खवाद्यरवोन्मिश्रं भरतस्यानुयात्रिकम्}%।। १ ।।

\twolineshloka
{राजद्वारमुपागत्य भरतोऽपि महायशाः}
{पदभ्यां जगाम राजानमवतीर्य तुरङ्गमात्}%।।२।।

\twolineshloka
{स ददर्श तदा रामं रत्नसिंहासनस्थितम्}
{अनुलिप्तं परार्ध्येन चन्दनेन सुगन्धिना}%।।३।।

\twolineshloka
{सूक्ष्मं वसानं वसनं सर्वाभरणभूषितम}
{तेजसा भास्कराकारं सौन्दर्येणोडुपोपमम्}%।। ४ ।।

\twolineshloka
{क्षमया पृथिवीतुल्यं क्रोधे कालानलोपमम्}
{बृहस्पतिसमं बुद्ध्या विष्णुतुल्यं पराक्रमे}%।।५।।

\twolineshloka
{सत्ये दानेऽप्यनौपम्यं दमे शीले च राघवम्}
{पुरोहितैरमात्यैश्च युतं प्रकृतिभिस्तथा}%।। ६ ।।

\twolineshloka
{दृष्ट्वा तं भरतः श्रीमाञ्जगाम शिरसा महीम्}
{भरतं युवराजानं रामाय विदितात्मने}%।। ७ ।।

\twolineshloka
{न्यवेदयत धर्मात्मा प्रतीहारः सलक्ष्मणम्}
{उत्थाय कण्ठे जग्राह रामोऽपि भरतं तदा}%।। ८ ।।

\twolineshloka
{मूर्ध्नि चैनमुपाघ्राय आदिदेशास्य शासनम्}
{भरतं तु सुखासीनं रामो वचनमब्रवीत्}%।। ९ ।।

\twolineshloka
{गन्धर्वपुत्रांस्तान्हत्वा कर्तव्यं नगद्वयम्}
{सिन्धो रुभयपार्श्वे तु पुत्रयोरुभयोः कृते} %॥१०॥

\twolineshloka
{अभिषिच्य तदा वत्स पुत्रौ नगरयोस्तयोः}
{युधाजिति परीधाय क्षिप्रमागन्तुमर्हसि}%।।११।।

\twolineshloka
{त्वया विना नरव्याघ्र नाहं जीवितुमुत्सहे}
{क्षत्रधर्मं पुरस्कृत्य तत्र त्वं प्रेषितो मया}%।।१२।।

\twolineshloka
{स त्वं गच्छ महाभाग मा ते कालात्ययो भवेत्}
{स्वस्त्यस्तु तेन्तरिक्षेभ्यः पार्थिवेभ्यश्च गच्छतः}%।। १३}

\onelineshloka
{दिव्येभ्यश्चैव भूतेभ्यः समरे च तथानघ}%।। १४ ।।

\twolineshloka
{स्वांस्वां दिशमधिष्ठाय दिक्पाला दीप्ततेजसः}
{पालयन्तु सदा तुभ्यं दीप्तविग्रहधारिणः}%।। १५ ।।

\twolineshloka
{ब्रह्मा विष्णुश्च रुद्रश्च साध्याश्च समरुद्गणाः}
{आदित्या वसवो रुद्रा अश्विनौ च भिषग्वरौ}%।। १६ ।।

\twolineshloka
{भृगवोङ्गिरसश्चैव कालस्यावयवास्तथा}
{सरितः सागराः शैलाः समुद्राश्च सरांसि च}%।। १७ ।।

\twolineshloka
{दैत्यदानवगन्धर्वा पिशाचोरगराक्षसाः}
{देवपत्न्यस्तथा सर्वा देवमातर एव च}%।। १८ ।।

\twolineshloka
{शस्त्राण्यस्त्राणि शास्त्राणि मङ्गलाय भवन्तु ते}
{विजयं दीर्घमायुश्च भोगांश्चान्यान्दिशन्तु ते}%।। १९ ।।

\twolineshloka
{इति स्वस्त्ययनं श्रुत्वा राज्ञा स समदीरितम्}
{रामस्य पादौ शिरसा त्वभिवन्द्य धनुर्धरः} %॥२०॥

\twolineshloka
{निर्गत्य राजभवनाद्रामाज्ञाकल्पितं जगत}
{हिमाद्रिकूटसङ्काशं चारुदंष्ट्रोज्ज्वलाननम्}%।। २१ ।।

\twolineshloka
{मदेन सिञ्चमानं च नृपवेश्माजिरं नृप}
{समाक्रान्तकटं चापि पानलुब्धशिलीमुखैः}%।। २२ ।।

\twolineshloka
{स्तब्धचारुमहाकर्णं मधुपकृपयैव तु}
{दीर्घाग्रमध्वक्षकृतं कृतशृङ्गावतंसकम्}%।। २३ ।।

\twolineshloka
{स्वासनं व्यूढकुम्भं च तथोदग्रं महाबलम्}
{नक्षत्रमालां शिरसा धारयानं तु काञ्चनीम्}%।। २४ ।।

\twolineshloka
{पटुस्वने तथा घण्टे दर्शनीये मनोहरे}
{कुथं विचित्रं रम्यं च कोविदारं महाध्वजं}%।। २५ ।।

\twolineshloka
{वैजयन्त्यः पताकाश्च किङ्किणीजालमालिताः}
{समारूढं नयविदा महामात्रेण धीमता}%।। २६ ।।

\twolineshloka
{वैडूर्यदण्डतीक्ष्णाग्र काञ्चनाङ्कुशधारिणा}
{जघनस्थेन चान्येन वरतोमरधारिणा}%।। २७ ।।

\twolineshloka
{तथा वैजयिकैर्मन्त्रैर्देवज्ञेनाभिमन्त्रितम्}
{आरुरोह महातेजा जयत्काराभिनन्दिनः}%।।२८।।

\twolineshloka
{पूर्णेन्दुमण्डलाकारं रुक्मदण्डं मनोहरम्}
{छत्रमादाय तं प्रेम्णा चारुरोह स लक्ष्मणः}%।। २९ ।।

\twolineshloka
{चामरौ द्वौ समादाय चन्द्ररश्मिमप्रभौ}
{आरूढं योषितोर्युग्मं रूपेणाऽप्रतिमं भुवि} %॥३०॥

\twolineshloka
{तं समारुह्य नागेन्द्रं मदलेखाभिगामिनम्}
{जगाम सह गार्ग्येण रथारूढेन यादव}%।। ३१ ।।

\twolineshloka
{तमन्वयौ महाभाग चतुरङ्गमहाबलम्}
{पताकाध्वजसम्बाधं कल्पयन्तं वसुन्धराम्}%।। ३२ ।।

\twolineshloka
{श्येनाः काकवहाः कङ्काः पिशाचा यक्षराक्षसाः}
{ययुः पुरःसरास्तस्य भरतस्य महात्मनः}%।। ३३ ।।

\twolineshloka
{गन्धर्वपुत्रमांसानां लुब्धा मांसोपजीविनः}
{तूर्यघोषेण महता बन्दिनां निःस्वनेन च}%।। ३४ ।।

\twolineshloka
{वायुना चानुलोमेन सेव्यमानः सुगन्धिना ।। ।}
{मङ्गलानां च मुख्यानां दर्शनाद्धृष्टमानसः}%।। ३९ ।।

\twolineshloka
{निर्ययौ राजमार्गेण जनसम्बाधशालिना}
{तेजस्विनां च तेजांसि हृदयानि च योषि ताम्}%।। ३६ ।।

\twolineshloka
{आददानो महातेजा नगरात्स विनिर्ययौ}
{क्रोशमात्रं ततो गत्वा समे देशे च सोदके}%।। ३७ ।।

\twolineshloka
{प्रशस्तद्रुमसङ्कीर्णे शिबिरं प्राङ्निषेवितम्}
{सेनाध्यक्षैः सुनिपुणैर्विवेश भरतस्तदा}%।। ३८ ।।

\twolineshloka
{स प्रविश्य महातेजाः शिबिरं स्वं निवेशनम्}
{मङ्गलालम्भनं कृत्वा वरासनगतः प्रभुः}%।। ३९ ।।

\twolineshloka
{पौरजानपदं सर्वं प्रेषयामा यादव}
{परिष्वज्य ततः पश्चाल्लक्ष्मणं शुभलक्षणम्} %॥४०॥

\twolineshloka
{मूर्ध्नि चैवमुपाघ्राय प्रेषयामास धर्मवित्}
{प्रायाणकं च श्वोभूते दुन्दुभिस्ताड्यतां मम}%।। ४१ ।।

\twolineshloka
{दैशिकाः पुरतो यान्तु ये च वृक्षावरोहकाः}
{आज्ञाप्य सकलं चैव शिबिरं च तथाविधम्}%।। ४२ ।।

\twolineshloka
{अवृक्षेषु तु देशेषु रोपयन्तु द्रुमाञ्जनाः}
{द्रुमाः कण्टकिनश्चैव ये च मार्गप्ररोधकाः}%।। ४३ ।।

\twolineshloka
{छिन्दन्तु गत्वा तानद्य तीक्ष्णैः शीघ्रं परश्वधैः}
{तोयहीनेषु देशेषु कूपान्कुर्वन्तु मे तथा}%।। ४४ ।।

\twolineshloka
{विषमांश्च तथा देशान्समान्कुर्वन्तु मे पथि}
{तीराणि सरितां चैव कुर्वन्तु पुलिनानि च}%।। ४५ ।।

\twolineshloka
{न भवेद्येन सङ्घट्टो जनस्य पथि यास्यतः}
{पुरः प्रयान्तु सैन्येन विजये रतिवर्धनाः}%।। ४६ ।।

\twolineshloka
{नीलश्च नक्रदेवश्च वसुमान्मुनयस्तथा}
{मध्यप्रयाणे गान्धारो जयनाभो रणोत्कटः}%।। ४७ ।।

\twolineshloka
{सुशीलः कामपालश्च यान्तु सैन्येन दंशिताः}
{जघनं कटकस्याहं पालयानो यथासुखम्}%।। ४८ ।।

\twolineshloka
{सैन्येन सह यास्यामि विजयाय नृपस्य तु}
{पश्यन्तु दैशिकाः स्थानं प्रभूतयवसेन्धनम् ।।}%।। ४९ ।।

\twolineshloka
{सोदकं च समं यत्र सेनावासो वेन्मम}
{एवमाज्ञाप्य भरतो विजहार यथासुखम्} %॥५०॥

\onelineshloka
{सुष्वाप च निशां तत्र घर्मकाले मनोहराम्}%।।५१।।

\twolineshloka
{सुप्तस्य सा तत्र रघूद्वहस्य पुण्या ययौ रात्रिरदीनसत्त्वा}
{सम्पूर्णचन्द्राभरणा प्रतीता ज्येष्ठस्य मासस्य रणोत्सुकस्य}%।। ५२।।

॥इति श्रीविष्णुधर्मोत्तरे प्रथमखण्डे मार्कण्डेयवज्रसंवादे भरतनिर्याणवर्णनं नाम चतुरधिकद्विशततमोऽध्यायः॥२०४॥

\sect{प्रयाणवर्णनम् --- पञ्चाधिकद्विशततमोऽध्यायः}

\uvacha{मार्कण्डेय उवाच}

\twolineshloka
{प्रभातायां तु शर्वर्यां दुन्दुभिः समहन्यत}
{प्रयाणिको महाराज भरतस्य महास्वनः}%।। १ ।।

\twolineshloka
{तस्य शब्देन महता विबुद्धः कटके जनः}
{अवश्यकरणीयानि कृत्वा राजंस्त्वरान्विताः}%।। २ ।।

\twolineshloka
{त्वरिता गमनार्थाय समाहूयेतरेतरम्}
{पटवेश्मानि रम्याणि सहन्तुमुपचक्रमुः}%।। ३ ।।

\twolineshloka
{महान्ति सुमनोज्ञानि वर्तितोर्णाकृतानि च}
{चक्रुस्तानि च राजेन्द्र सुखवाह्यान्ययत्नतः}%।। ४ ।।

\twolineshloka
{ततस्त्वारोपयाञ्चक्रुः करभेषु खरेषु च}
{गोरथेषु तु मुख्येषु तथा दन्तिषु सत्वराः}%।। ५ ।।

\twolineshloka
{भाण्डमुच्चावचं चैव शयनानि मृदूनि च}
{आसनानि च मुख्यानि भाण्डं यच्च महानसे}%।। ६ ।।

\twolineshloka
{पेयं च यवसं चैव शस्त्राणि विविधानि च}
{कवचानि तुरङ्गाणां शिल्पभाण्डानि यानि च}%।। ७ ।।

\twolineshloka
{धनं च विविधं राजन्सर्वोप करणानि च}
{आरोप्यमाणे भाण्डे तु करभाणां विकृष्यताम्}%।। ८ ।।

\twolineshloka
{शुश्रुवे तुमुलः शब्दः खराणां च खरस्वनः}
{गजानां युज्यमानानां तुरङ्गाणां रथैः सह}%।। ९ ।।

\twolineshloka
{वाद्यानां हन्यमानानां शुश्रुवे तुमुलंस्वनम्}
{नादेन गजघण्टानां बृंहितेन च पार्थिव} %॥१०॥

\twolineshloka
{ह्रेषितेन तुरङ्गाणां बभूव तुमुलः स्वनः}
{प्रायाणिकमुपादाय ताडयन्नेव दुन्दुभिम्}%।। ११ ।।

\twolineshloka
{अग्रे प्रयाणमान्येन ययौ दुन्दुभिभिः सह}
{पण्यानि च समादाय वणिजस्त्वरिता ययुः}%।। १२ ।।

\twolineshloka
{ग्रहीतुकामाश्चान्यानि सोदकानि समानि च}
{स्थानानि वरमुख्यानां ययुश्चाग्रेसरा नराः}%।। १३ ।।

\twolineshloka
{महानसिकमुख्यास्तु त्वरिताश्च तथा ययुः}
{सुखयानासु रम्यासु तथैवाश्वतरीषु च}%।। १४ ।।

\twolineshloka
{आरोप्य योषितो जग्मुः प्रत्यूषे मुदिता जनाः}
{आरूढाश्चापरा नार्यः सवितानाः करेणवः}%।।१५।।

\twolineshloka
{कञ्चुकोष्णीषिभिर्गुप्ता गुप्ता वर्षवरैस्तथा}
{ययुः ससैन्या राजेन्द्र गीतवाद्य पुरस्सराः}%।। १६ ।।

\twolineshloka
{दीनान्धकृपणानाथांस्तर्पयन्त्यो धनेन ताः}
{नरेन्द्रयोषितो राज्ञां दिव्यालङ्गारभूषिताः}%।। १७ ।।

\twolineshloka
{तथान्ये बद्धनिस्त्रिंशाः पुरुषाश्च कलापिनः}
{आदीप्य तृणवेश्मानि जग्मुस्त्वरितमानसाः}%।। १८ ।।

\twolineshloka
{भरतोऽपि समारुह्य शिबिकां रत्नभूषिताम्}
{विनिर्ययौ महातेजा स्तूर्यघोषपुरःसरः}%।। १९ ।।

\twolineshloka
{शून्यं च शिबिरस्थानं गृध्रमण्डसङ्कुलम्}
{बहुक्रव्यादसङ्कीर्णं क्षणेन समपद्यत} %॥२०॥

\twolineshloka
{गजोष्ट्रगर्दभाश्वानां शरीरावयवैर्युतम्}
{भग्नभाण्डसमाकीर्णं शरीरावयवैर्युतम्}%।।२१।।

\twolineshloka
{बहुक्रव्यादसङ्कीर्णं करीषोत्करसंयुतम्}
{खातैर्महानसस्थानैर्दग्धमृत्तिकया युतैः ।। ।}%।।२२।।

\twolineshloka
{समण्डकर्दमोपेतैर्मक्षिकासहितैर्युतम्}
{सन्त्यज्य निर्ययुः सर्वे भरतस्य तु सैनिकाः}%।।२३।।

\twolineshloka
{प्रयाणे तस्य सैन्यस्य बलीवर्दाञ्छ्रमान्वितान्}
{नागानुत्थापयामासुरुपविष्टान्प्रयत्नतः}%।।२४।।

\twolineshloka
{केचिदुष्ट्रपरित्रस्तातान्गार्दभेन निपातितान्}
{भाण्डमारोपयाञ्चक्रुर्भूय एव नरोत्तमाः}%।। २५ ।।

\twolineshloka
{नद्युत्तारेषु महिषान्केचित्सूर्यांशुतापितान्}
{निषण्णान्सह भारेण ताडयाञ्चक्रिरे जनाः}%।। २६ ।। 

\twolineshloka
{केचिदश्वतरांस्त्रस्तान्नागबृंहितनिस्वनैः}
{आरूढयोषितो यत्नाज्जगृहुर्नृप रश्मिषु}%।। २७ ।।

\twolineshloka
{केचिदश्वैर्गजत्रस्तैराक्षिप्ता भुवि मानवाः}
{जानुविश्रमणार्थाय वाजिग्रीवकृताङघ्रयः}%।। २८ ।।

\twolineshloka
{तुरङ्गांश्च तदोद्भ्रान्तान्स्रस्तचर्माश्च सादिनः}
{केचिदाक्रम्य वेगेन जगृहुस्तत्र यादव}%।। २९ ।।

\twolineshloka
{विशश्रमुस्तथा केचिद्वृक्षच्छायास मानवाः}
{केचिच्चोदकतीरेषु चक्रिरे भोजनक्रियाम्} %॥३०॥

\twolineshloka
{केचित्सन्त्रस्ततुरगसन्निकृष्टसमुत्थितैः}
{द्रुतान्कापिञ्जलैर्यत्नाज्जगृहुस्तांस्तुरङ्गमान्}%।। ३१ ।।
 
\twolineshloka
{केचित्कटकसन्त्रस्तान्मृगयूथान्प्रधावतः}
{वेगेनाक्रम्य विशिखैर्जघ्नुर्यदुकुलोद्वह}%।। ३२ ।।

\twolineshloka
{केचिच्च यवसं चक्रुः केचिच्चक्रुरथेन्धनम्}
{तथान्यैर्द्विगुणीभूतं तथा दुन्दुभिनिस्वनम्}%।। ३३ ।।

\twolineshloka
{प्रायाणिकं जहुः शीघ्रं श्रमं यदुकुलोद्वह}
{केचिदापणवीथ्यग्रमहावंशसमुच्छ्रितम्}%।।३४ ।।

\twolineshloka
{सपताकं नरा दृष्ट्वा प्राप्ताः स्म इति मेनिरे}
{चक्रुः केचिच्च च्छन्दांसि सहायानां पुनःपुनः}%।।३५ ।।

\twolineshloka
{पुरोगतानां राजेन्द्र स्थानलब्धिचिकीर्षया}
{केचित्पटकुटीं दृष्ट्वा स्वकीयां त्वरिता ययुः}%।। ३६ ।।

\twolineshloka
{वर्तितोर्णाकृतं दृष्ट्वा गृहांश्चान्ये ययुर्द्रुतम्}
{केषाञ्चित्तत्र वेश्मानि तीर्णानि नृपसत्तम}%।। ३७ ।।

\twolineshloka
{कृतानि क्रियमाणानि ददृशुस्तत्र मानवाः}
{द्रुमैर्विश्राम्यमाणैश्च क्रियमाणैस्तथा कटैः}%।। ३८ ।।

\twolineshloka
{गृहैरारोप्यमाणैश्च पटोर्णातृणसंस्कृतैः}
{शुशुभे तन्महाराज कटकं शुभकर्मणः}%।। ३९ ।।

\twolineshloka
{शस्त्रेण संशोधयतां भुवं भूमिपत नृणाम्}
{राजसाक्रान्तवपुषामप्रकाशं वपुर्बभौ} %॥४०॥

\twolineshloka
{पार्श्वस्थतोयसम्पूर्णदृतयश्च तथा जनाः}
{धावमानाः प्रदृश्यन्ते वर्धिता गृहशोधने}%।। ४१ ।।

\twolineshloka
{अभ्युक्षयन्ति चाप्यन्ये तृणवेश्मानि पार्थिव}
{तप्तानां शीतकामानां दृतिवक्त्रोद्गतैर्जलैः}%।। ४२ ।।

\twolineshloka
{अभिश्रयेण धूपेन समन्तादाकुलीकृतम्}
{बभूव तस्य कटकं नीहारेणैव संयुतम्}%।। ४३ ।।

\twolineshloka
{अवरोपितभाण्डानां दान्तानां यदुनन्दन}
{पृष्ठान्यभ्युक्षयामासुर्गोमयेन जलेन च}%।। ४४}

\onelineshloka
{खरोष्ट्रसबलीवर्द चरणार्थे विनिर्गतम्}%।। ४५ ।।

\twolineshloka
{ददृशे बहुसाहस्रं कटके भरतस्य तु}
{रथेभ्यस्तुरगानन्ये विमुच्य हयकोविदाः}%।। ४६ ।।

\twolineshloka
{अवरोपितभाण्डानि सान्त्वयाञ्चक्रिरे शनैः}
{समुत्थाय रजो भौमं तुरङ्गपरिवर्तनैः}%।। ४७ ।।

\twolineshloka
{खमारुरोह राजेन्द्र कपोतारुणपाण्डुरम्}
{स्थापयाञ्चक्रिरे चान्ये जलस्थाने च वाजिनः}%।। ४८ ।।

\twolineshloka
{श्रेणी चक्रुस्तथैवान्ये कटच्छायासु पार्थिव}
{वितानाधस्तथा केचित्स्थापयाञ्चक्रिरे हयान्}%।। ४९ ।।

\twolineshloka
{हयेभ्यो यवसं दत्त्वा केचिद्बुभुजरे जनाः}
{कटच्छायाश्च नागानां चक्रुश्चान्ये सहस्रशः} %॥५०॥

\twolineshloka
{वितानानि च मुख्यानि सूर्यतापप्रशान्तये}
{स्नाताञ्जलाशये नागाँल्लब्धतोयाञ्जनाधिप}%।। ५१।। 

\twolineshloka
{सच्छन्नान्स्वपरीधानकुम्भान्निन्युः स्वमालयम्}
{कटकाद्दूरतश्चक्रुरालानं नृपदन्तिनाम्}%।। ५२।।

\twolineshloka
{आलानानि महाराज महावृक्षेषु मानवाः}
{आदाय गोपिनस्तत्र कटकाच्च विदूरतः}%।। ५३ ।।

\twolineshloka
{गोसङ्घान्महिषीसङ्घांश्चक्रुर्व्यस्तान्यथासुखम्}
{कटके च तथा निन्युर्गोरसानि नराधिप}%।। ५४ ।।

\twolineshloka
{कटकापण्यवीथीं च सर्वपण्यविभूषिताम्}
{ददृशुः पुरुषास्तत्र अयोध्यामिव चापराम्}%।। ५५ ।।

\twolineshloka
{स्थानानि सर्ववैद्यानांसध्वजानि नराधिप}
{सागदानि च दृश्यन्ते कटके भरतस्य तु}%।।५६।।

\twolineshloka
{सेनाध्यक्षेण वीरेण विजयेन महात्मना}
{कृतं शास्त्रानुसारेण स्कन्धावारनिवेशनम्}%।। ५७ ।।

\twolineshloka
{विवेश भरतः श्रीमांश्चतुरङ्गबलान्वितः}
{अन्वीयमानो वीराभ्यां पुत्राभ्यां यदुनन्दन}%।। ५८ ।।

\twolineshloka
{पुष्करेण च वीरेण तक्षेण सुमहात्मना}
{बलमुख्यैस्तथैवान्यैः सूतमागधबन्दिभिः}%।। ५९ ।।

\twolineshloka
{शङ्खवादित्रशब्देन पटहानां स्नेन च}
{भरतस्य गृहद्वारतोरणान्तिकमागताः} %॥६०॥

\twolineshloka
{बभूबुर्बलमुख्यास्ते दिक्षु ये यदुनन्दन}
{सन्त्यज्य मध्यमां वीथीं प्रवेशाय महामनाः}%।। ६१ ।।

\twolineshloka
{बन्दिभिः ख्याप्यमानांस्तान्नामकर्मावदानतः}
{शिरःकम्पेन भरतः प्रैरयत्स्वान्निवेशनान्}%।। ६२ ।।

\twolineshloka
{बलमुख्यान्विवेशाथ स्वगृहं सर्वऋद्धिमत्}
{द्वियोजनाध्वना श्रान्ता भरतस्य तु सैनिकाः}%।। ६३ ।। 

\twolineshloka
{विविशुर्भवनान्स्वान्स्वान्भेजिरे शयनानि च}
{प्रविश्य वेश्मप्रवरं भरतोऽपि यथासुखम्}%।। ६४ ।।


\threelineshloka
{विजहार महाराज देवराजसमद्युतिः}
{क्रमेणानेन धर्मात्मा भूमिपाल दिनेदिने}
{आससाद व्रजन्नेव गङ्गां त्रिपथगां नदीम्}%।। ६५}

\twolineshloka
{सुराङ्गनापीनपयोधरस्थसच्चन्दनक्षालनलब्धलक्ष्मीम्} 
{ग्रीष्मार्कतापाद्विगलत्तुषारविवृद्धशीतोदपटोत्तरीयाम्}%।। ६६ ।।

॥इति श्रीविष्णुधर्मोत्तरे प्रथमखण्डे मार्कण्डेयवज्रसंवादे प्रयाणवर्णनं नाम पञ्चाधिकद्विशततमोऽध्यायः॥२०५॥

\sect{गङ्गावतरण वर्णनम् --- षडुत्तरद्विशततमोऽध्यायः}

\uvacha{मार्कण्डेय उवाच}

\twolineshloka
{निवेशमकरोद्राजन् गङ्गातीरे स राघवः}
{भरतस्य तदा चक्रुर्गङ्गायां जलमण्डपान्}%।। १ ।।

\twolineshloka
{तथैव बलमुख्यानां प्राधान्येन नराधिप}
{सैनिकश्च जनस्तस्य प्राप्य तां सुरनिम्नगाम्}%।। २ ।।

\twolineshloka
{साफल्यं जन्मनो मेने तत्याज च तथा क्लमम्}
{सस्नौ पपौ पयः कामं हर्षेण महता युतः}%।। ३ ।।।

\twolineshloka
{स्नापयाञ्चक्रिरे तत्र तुरगांस्तुरगप्रियाः}
{कुञ्जरान्स्नापयमासुर्महामात्रास्तथैव च}%।। ४ ।।

\twolineshloka
{मज्जद्भिर्बहुसाहस्रैः कुञ्जरैर्जाह्नवी नदी}
{किरातविषयैः सैव विरराज यथोपलैः}%।। ५ ।।

\twolineshloka
{समुत्थितमहामात्रान्दृष्ट्वाभ्युक्षणछद्मना}
{चिक्रीडुर्दृतिभिः केचिद्विविशुर्बाहुभिस्तथा}%।। ६ ।।

\twolineshloka
{गङ्गामासाद्य संहृष्टा भरतस्य तु सैनिकाः}
{भरतोऽपि तथा स्नातः कृतदैवतपूजनः}%।।७।।

\twolineshloka
{श्राद्धं चक्रे महातेजा ददौ दानं तथैव च}
{दत्त्वा तत्र महादानानाज्ञापयति यादव}%।।८।।

\twolineshloka
{शिल्पिनो मम कुर्वन्तु कूटागारान्मनोरमान्}
{रात्रौ दीपनिवेशार्थं शतशोऽथ सहस्रशः}%।।९।।

\twolineshloka
{आनीयन्तां तथा नावः सैन्यस्य तणाय मे}
{क्रियन्तां चर्मनावश्च प्लवाश्च शतशस्तथा} %॥१०॥

\twolineshloka
{उत्तरं पारमासाद्य निवेशः क्रियतां तथा}
{परं पारं जना यान्तु उदयास्तमयान्तरे}%।।११।।

\twolineshloka
{एवमाज्ञाय भरतो विजहार यथासुखम्}
{प्रसारितकरो राजन्सर्वत्रैव दिवाकरः}%।। १२ ।।

\twolineshloka
{ययावदर्शनं तत्र भरतस्यैव लज्जया}
{पद्मपत्रदलाग्राभा प्रतीची चाभवत्क्षणात्}%।।१३।।

\twolineshloka
{आदित्येस्तमनुप्राप्ते सन्ध्यारागानुरञ्जिता}
{ततस्तु तमसा व्याप्ते न प्राज्ञायत किञ्चन}%।। १४ ।।

\twolineshloka
{भरताज्ञाकृतान्पूर्वं कूटागारान्सदीपकान्}
{ध्वजमालापरिक्षिप्तांश्चिक्षिपुर्जाह्नवीजले}%।।१५।।

\twolineshloka
{सैनिकैश्च तथा मुख्यैर्भरतस्य पृथक्पृथक् ।}
{गङ्गाम्भसि परिक्षिप्ता दीपवृक्षाः सहस्रशः}%।।१६।।

\twolineshloka
{सा दीपमालिनी गङ्गा तीरद्योतितविग्रही}
{जहास फेनहासेन जिह्वेव गगनप्रिया}%।। १७ ।।

\twolineshloka
{जलान्तरागतै र्दीपैर्दीपवृक्षैर्मनोहरैः}
{तथा कल्लोलसङ्क्रान्तैर्गङ्गा दीप्तेव लक्ष्यते}%।। १८ ।।

\twolineshloka
{एवं हि क्रीडतां तत्र गङ्गातीरे तथा नृणाम्}
{रात्रावेवाञ्जसा जग्मुर्नावो बहुविधा नृप}%।। १९ ।।

\twolineshloka
{शशाङ्कराजहंसेन दृष्ट्वा खससीं जनाः}
{आक्रम्यमाणा सुषुपुर्निशीथे निद्रयान्विताः} %॥२०॥

\twolineshloka
{शशाङ्कोदयसंसुप्तबलं पद्मवनोपमम्}
{निबोधयामास तथा दिवाकरकरोत्करम्}%।। २१ ।।

\twolineshloka
{चकृषुस्ते तदा नावः कर्णधारा यथा स्वकाः}
{बलं च सकलं पारं तारया ञ्चक्रिरे तदा}%।। २२ ।।

\twolineshloka
{नावस्तां किङ्किणीजालैः पताकाभिश्च राजिता}
{निन्युर्बलं परं पारं कर्णधारस्फिगाहताः}%।। २३ ।।

\twolineshloka
{भाण्डैः पूर्णास्तथा काश्चित्काश्चित्पूर्णा जनेन च}
{तुरङ्गमैस्तथा पूर्णा गोखरोष्ट्रैस्तथा पराः}%।। २४ ।।

\twolineshloka
{काश्चित्सकुञ्जरा नावो ययुः पारं तदा परम्}
{तथा पद्मदलाक्षीणां स्त्रीणां नावश्च पूरिताः}%।। २५ ।।

\twolineshloka
{विमानाभाः प्रदृश्यन्ते गङ्गाम्भसि नरेश्वर}
{कर्णधारवरोपेता दण्डिभिः पुरुषैर्वृताः}%।। २६ ।।

\twolineshloka
{लोकसन्तारणार्थाय भूयो जग्मुस्तथा पराः}
{चर्मनौभिस्तथा केचित्प्लवैः केचित्सुयन्त्रितैः}%।। २७ ।।

\twolineshloka
{जग्मुरादाय भाण्डानि दृतिभिश्च तथा क्वचित्}
{नरैर्दृतिसमारूढैः कृष्यमाणास्तुरङ्गमाः}%।। २८ ।।

\twolineshloka
{जग्मुः केचित्परं पारं बाहुभिर्मनुजेश्वर}
{जग्मुरन्ये परं पारं सङ्गृहीताश्च रश्मिभिः ।।}%।। २९ ।।

\twolineshloka
{नौस्थैरेव परैः शीघ्रं तुरगा नपबाहुभिः}
{महिषीणां च सङ्घानि गवां च यदुनन्दन} %॥३०॥

\twolineshloka
{जग्मुर्गोपालकैः सार्धं परं पारं च बाहुभिः}
{उष्ट्रगर्दभसङ्घानि तार्यमाणानि बाहुभिः}%।। ३१ ।।

\twolineshloka
{भूयोभूयो न्यवर्तन्त तत्र रावो महानभूत्}
{तीर्णस्य तार्यमाणस्य नरस्य यदुसत्तम}%।। ३२ ।।

\twolineshloka
{आसीत्कोलाहलो घोरस्तीरयोरुभयोरपि}
{भरतोऽपि महातेजाः स्नातो हुतहुताशनः}%।। ३३ ।।

\twolineshloka
{मत्स्यरूपधरं विष्णुं पटे सम्पूज्य यादव}
{स्वस्तिवाच्यांस्ततो विप्रान्गोभिर्वस्त्रैर्धनेन च}%।। ३४ ।।

\twolineshloka
{सम्पूज्य जाह्नवीं देवीं गन्धमाल्यानुलेपनैः}
{सम्पूजयामास तदा भूय एव द्विजोत्तमान्}%।। ३५ ।।

\twolineshloka
{गोभिरश्वैस्तथा निष्कैर्वस्त्रैर्गन्धैस्तथैव च}
{आरुरोह तदा नावं पाण्डुकम्बलसंवृताम्}%।।३६।।

\twolineshloka
{किङ्किणीजालविततां पताकाध्वजमालिनीम्}
{भरते तु समारूढे कर्णधारस्फिगाहता}%।। ३७ ।।

जगाम सा परं पारं नौर्विमानोपमा नृप ।।

\twolineshloka
{सम्प्राप्य स परं पारं सम्पूर्णकटकस्तदा}
{उवास राजंस्तत्रैव दिवि देवेश्वरो यथा}%।। ३८ ।।

\twolineshloka
{देवीं महादेवजटातटस्थां शशाङ्कसंसर्गविवृद्ध शीताम्}
{उत्तीर्य राजा भरतः स चैनां निन्ये क्षयं शक्रसमप्रभावः}%।। ३९ ।।

॥इति श्रीविष्णुधर्मोत्तरे प्रथमखण्डे मार्कण्डेयवज्रसंवादे गङ्गावतरणवर्णनं नाम षडुत्तरद्विशततमोऽध्यायः॥२०६॥

\sect{राजगृहगमनो --- सप्तोत्तरद्विशततमोऽध्यायः}

\uvacha{मार्कण्डेय उवाच}

\twolineshloka
{भरते तु समुत्तीर्णे गङ्गां गगनमेखलाम्}
{विवेश कटकं तस्य तमन्यः कौरवो नृप}%।। १ ।।

\twolineshloka
{बलेन चतुरङ्गेण समुद्राभेन संयुतः}
{ईश्वरश्च किरातानां दमनो नाम पार्थिवः}%।। २ ।।

\twolineshloka
{गन्धिनां बहुसाहस्रैर्बलैर्युक्तो व्यदृश्यत}
{यथार्हं पूजयित्वा तं भरतो धर्मवत्सलः}%।। ३।।

\twolineshloka
{आससादार्कतनयां यमुनां पापनाशिनीम्}
{यमस्य भगिनीं पुण्यां नीलमालां मनोहराम्}%।। ४ ।।

\twolineshloka
{यत्र क्वचन नद्यां हि कृत्वा श्राद्धन्नराधिप}
{अक्षयं फलमाप्नोति नाकपृष्ठे च मोदते}%।।५।।

\twolineshloka
{यत्र कृष्णचतुर्दश्यां स्नातः सम्पूज्य भानुजाम्}
{मुच्यते पातकैः सर्वैर्नाकलोकं स गच्छति}%।। ६ ।।

\twolineshloka
{अनर्काभ्युदिते काले माघकृष्णचतुर्दशीम्}
{यस्यां स्नातस्तु सम्पूज्य धर्मराजं तिलाम्भसा}%।। ७ ।।

\twolineshloka
{न दुर्गतिमवाप्नोति कुलं चैव समुद्धरेत्}
{यत्र क्वचन नद्यां हि माघकृष्णचतुर्दशीम्}%।। ८ ।।

\twolineshloka
{नादेयाम्भसि सर्वस्मिन्स्नातः पापैर्विमुच्यते}
{यमुना तु विशेषेण यमस्य भगिनीति सा}%।। ९ ।।

\twolineshloka
{स्नातश्च यामुने तोये सन्तर्प्य पतृदेवताः}
{न दुर्गतिमवाप्नोति नाकलोकं च गच्छति} %॥१०॥

\twolineshloka
{तां समासाद्य यमुनां यथा गङ्गाजलं तथा}
{विहृत्य भरतस्तद्वदुत्ततार महानदीम्}%।। ११ ।।

\twolineshloka
{समुत्तीर्णस्य यमुनां भरतस्य महात्मनः}
{विवेश कटकं राजा मत्स्यानां सुरथस्तदा}%।। १२ ।।

\twolineshloka
{गोरसेनश्च साल्वानां शिबीनां च प्रभद्रकः}
{स तैर्नृपतिभिः सार्धं कुरुक्षेत्रमुपाययौ}%।। १३ ।।

\twolineshloka
{यदर्थमेषा चरति लोके गाथा पुरातनी}
{पांसवोऽपि कुरुक्षेत्रे वायुना सममीरिताः}%।। १४ ।।

\twolineshloka
{अपि दुष्कृतकर्माणो नयन्ति परमां गतिम्}
{समन्तपञ्चके पुण्ये ये मृता मनुजेश्वर}%।। १५ ।।

\twolineshloka
{ते सर्वे नाकमासाद्य राजन्ते दिवि देववत्}
{सन्नीतिर्यत्र राजेन्द्र तीर्थं त्रैलोक्यविश्रुतम्}%।। १६ ।।

\twolineshloka
{तीर्थसन्नयनादेव सन्नीतिरिति विश्रुतम्}
{पृथिव्यां यानि तीर्थानि आसमुद्रसरांसि च}%।। १७ ।।

\twolineshloka
{मासान्ते सततं तत्र नित्यमायान्ति यादव}
{तत्र श्राद्धं तु यः कुर्याद्राहुग्रस्ते दिवाकरे}%।। १८ ।।

\twolineshloka
{अश्वमेधशतस्याग्रं फलं विन्दति मानवः}
{भरतस्तु समासाद्य तत्रोवास सुखी तदा}%।।१९।।

\twolineshloka
{पप्रच्छ ब्राह्मणांस्तत्र तीर्थसन्नीतकारणम्}
{पृष्टस्तु भरतेनाथ ब्राह्मणस्तु घटोदरः} %॥२०॥

\onelineshloka*
{उवाच भरतं तत्र कथां पापप्रणाशिनीम्}

\uvacha{घटोदर उवाच}
\onelineshloka
{शक्रे वृत्रवधाक्रान्ते त्रैलोक्ये दैत्यसाद्गते}%।। २१ ।।

\twolineshloka
{सब्रह्मकाः सुराः सर्वे विष्णुं शरणमाययुः}
{तानुवाच हरिर्देवश्च्यवनस्यात्मसम्भवः}%।। २२ ।।

\twolineshloka
{भार्गवो ब्राह्मणः श्रीमान्दधीच इति विश्रुतः}
{अस्थिभिः क्रियतां तस्य देवेन्द्रस्य वरायुधम्}%।। २३ ।।

\twolineshloka
{प्रविश्य देवांस्तत्राहं हन्ता वृत्रमसंशयम्}
{एवमुक्ताः सुराः सर्वे दधीचस्याश्रमं ययुः}%।। २४ ।।

\twolineshloka
{ददृशुश्च महाभागं दधीचं तपसां निधिम्}
{पूजयित्वा महाभागं तमूचुः संहताः सुरा.}%।। २५ ।।

\twolineshloka
{त्वदस्थिभिः करिष्यामो वज्रं दैत्यनिबर्हणम्}
{अन्यानि च तथास्त्राणि सुरकार्यार्थसिद्धये}%।। २६ ।।

\twolineshloka
{तत्र त्वं देवकार्यार्थं सन्न्यासं द्विज रोचय}
{एवमुक्तो दधीचस्तु प्रत्युवाच स तान्सुरान्}%।। २७ ।।

\twolineshloka
{तीर्थयात्रा प्रतिज्ञाता सर्वतीर्थेषु वै मया}
{तां तु कृत्वा करिष्यामि देहन्यासं सुरोत्तमाः}%।। २८ ।।

\uvacha{देवा ऊचुः}

\twolineshloka
{शक्तस्त्वं सर्वतीर्थानामाह्वाने द्विजपुङ्गव}
{इहाद्यैव समायान्तु सर्वतीर्थानि तेऽनघ}%।। २९ ।।

\onelineshloka*
{तेजसा च त्वदीयेन तथास्माकं च भार्गव}

\uvacha{घटोदर उवाच}
\onelineshloka
{एवमुक्तः सुरैः सर्वैस्तीर्थानि सरितस्तथा}%।।३० ।।

\twolineshloka
{सरांसि च समुद्राश्च तत्राजग्मुर्नराधिप}
{काम्येन महता राजन्योगेन परमेण च}%।। ३१ ।।

\twolineshloka
{ज्ञात्वा च तेषां सान्निध्यं तत्र स्नातो द्विजोत्तमः}
{तर्पणं च तथा कृत्वा सुराणां पितृभिः सह}%।। ३२ ।।

\onelineshloka
{उवाच देवांस्तत्रस्थानिदं वचनमर्थवत्}

\uvacha{दधीच उवाच}
\onelineshloka*
{अद्य प्रभृति मासान्ते भवद्भिः सततं सुराः}%।। ३३ ।।

\onelineshloka*
{सान्निध्यमिह कर्तव्यं तीर्थैश्चैव यथागतैः}

\uvacha{घटोदर उवाच}
\onelineshloka
{एवमुक्तैस्तथेत्युक्तो देवैस्तीर्थैश्च स त्वथ}%।। ३४ ।।

\twolineshloka
{त्यक्त्वा देहं दिवं यातो दधीचः स्वेन तेजसा}
{विश्वकर्मा च तस्यास्थ्नां भागैर्वज्रमथा करोत्}%।। ३५ ।।

\twolineshloka
{आयुधानि च देवानां तथैव च पृथक्पृथक्}
{तेन वज्रेण महता वृत्रं हत्वा महासुरम्}%।। ३६ ।।

\twolineshloka
{जघान दैत्यमुख्यानां नवतिर्नवतिस्तथा}
{ततः प्रभृति मासान्ते नित्यमेव रघूद्वह}%।। ३७ ।।

\onelineshloka*
{सान्निध्यमिह तीर्थानां देवतानां च कल्पितम्} 

\uvacha{मार्कण्डेय उवाच}
\onelineshloka
{एतच्छ्रुत्वा महातेजा दत्त्वा दानानि राघवः}%।। ३८ ।।

\twolineshloka
{ययौ सैन्येन महता भरतोऽमरकण्टकम्}
{तत्र विष्णुपदं प्राप्य पूजयित्वा च राघवः}%।।३९।।

\twolineshloka
{आससाद नदीं गौरीं पुण्यां पापयापहाम्}
{विश्वामित्राज्ञया रक्षः प्रविवेश पुरा नृपम्} %॥४०॥

\twolineshloka
{कशाहतेन मार्गस्थं शप्तं भूपाल शक्तिना}
{सौदासं स प्रविष्टस्तु भक्षयामास शक्तिनम्}%।। ४१ ।।

\twolineshloka
{ततः पुत्रशतं राजन्वसिष्ठस्यैव सत्वरः}
{हृते पुत्रशते दुखाद्वसिष्ठो भगवानृषिः}%।। ४२ ।।

\twolineshloka
{विवेश निम्नगां गौरीं प्राणत्यागचिकीर्षया}
{विप्रस्य भूप यातस्य विद्रुता सा तदाभवत्}%।। ४३ ।।

\twolineshloka
{ततः प्रभृति लोकेऽस्मिञ्छतद्रुरिति शब्दिता ।। शतद्रु}
{सर्वपापप्रशमनी सर्वकल्याणकारिणी}%।।४४।।

\twolineshloka
{स्नातानां च तथा राजन्दशधेनुफलप्रदा}
{तां समासाद्य तत्रापि दत्त्वा दानं स राघवः}%।।४५।।

\twolineshloka
{उत्तीर्य तां ययौ तत्र सा चान्त्या यत्र निम्नगा}
{यस्यां स्नात्वा विमुच्यन्ते सर्वपापभयैर्नराः}%।।४६।।

\twolineshloka
{यस्यां स्नानादवाप्नोति दशगोदानजं फलम्}
{आषाढे च तथा कृत्वा गोसहस्रफलं लभेत्}%।। ४७ ।।

\twolineshloka
{गौरीतोयाद्विनिर्मुक्तो वसिष्ठो भगवानृषिः}
{पाशबन्धभरैर्यस्यां पपात सहसा नृप}%।।४८।।

\twolineshloka
{विपाशश्च तथा देव्या कृत्वा तीरे विसर्जितः ।। विपाशा}
{शक्तिपुत्रं ततो दृष्ट्वा वसिष्ठोऽपि पराशरम्}%।। ४९ ।।

\twolineshloka
{अस्ति सन्तानमित्युक्त्वा मरणाद्विनवर्तत}
{सा चान्त्या च तथा लोके विपाशेत्यभिधीयते} %॥५०॥

\twolineshloka
{विपाशां च समुत्तीर्णे भरते धर्मवत्सले}
{विवेश कटकं तस्य कुणिदेशो महोदयः}%।। ५१ ।।

\twolineshloka
{त्रैगर्तो वसुधानश्च कुलूताधिपतिर्जयः}
{दाशेरकस्तथा राजा गोवाशन इति श्रुतः}%।। ५२ ।।

\twolineshloka
{इरावतीं शीघ्रगमां भरतोऽपि तदा ययौ}
{दशधेनुफलं यत्र स्नात एव समश्नुते}%।। ५३ ।।

\twolineshloka
{षष्टितीर्थसहस्राणि वहत्येका इरावती}
{अष्टम्यां तु विशेषेण यत्र युज्येत रेवती}%।। ५४ ।।

\twolineshloka
{तत्र दत्त्वा बहुविधं दानं रघुकुलोद्वहः}
{उत्तीर्य तां ययौ शीघ्रं देविकां पापनाशिनीम्}%।। ५९ ।।

\twolineshloka
{दृष्टमात्रैव या देवी सर्वकल्मषनाशिनी}
{शरीरबहुला सा तु हरस्य दयिता उमा}%।। ५६ ।।

\twolineshloka
{तत्रापि दत्त्वा दानानि पूजयामास शङ्करम्}
{भरते चाथ तत्रस्थे विविशुः पञ्च पार्थिवाः}%।। ५७ ।।

\twolineshloka
{पार्वतीया महाराज पदातिगणसङ्कुलाः}
{कुमारः श्रेणिमाञ्छूरो बलबन्धुः सुयोधनः}%।। ५८ ।।

\twolineshloka
{मद्रराजोंऽशुमांश्चैव तथैव च महाबलः}
{पूजितो मद्रराजेन शाकलेन नरोत्तमः}%।। ५९ ।।

\twolineshloka
{त्रिरात्रमुषितः श्रीमांश्चन्द्रभागां दीं ययौ}
{चन्द्रांशुशीतलजलां सर्वपापप्रणाशिनीम्} %॥६०॥

\twolineshloka
{बहुतीर्थसमायुक्तां स्नानात्सर्वप्रदां नृणाम् ।। ।}
{विशेषेण महाराज माघपुष्यत्रयोदशीम्}%।। ६१ ।।

\twolineshloka
{भरते तां समुत्तीर्णे विवेश कटकं ततः}
{एतच्छ्रुतञ्जयः श्रीमानभिचारः कृतञ्जयः}%।। ६२ ।। ।

\twolineshloka
{काश्मीरकश्च धर्मात्मा सुबाहुरिति विश्रुतः}
{आससाद स तैः सार्द्धं वितस्तां तु महानदीम्}%।।६३।।

\twolineshloka
{स्वर्गलोकप्रदां स्नाने सर्वकल्मषनाशिनीम्}
{प्रोष्ठपादस्य मासस्य शुक्लपक्षत्रयोदशीम्}%।। ६४ ।।

\twolineshloka
{विशेषेण महाराज पुण्यां परमपावनीम्}
{भरतस्तां समुत्तीर्य सुदामां चैव निम्नगाम्}%।। ६५ ।।

\twolineshloka
{आससाद महातेजाः कैकेयान्यदुनन्दन}
{बलेन चतुरङ्गेण युधाजित्कैकयाधिपः}%।। ६६ ।।

\twolineshloka
{निर्ययौ भरतं प्राप्तं श्रुत्वा भरतमातुलः}
{नागराश्च तथा मुख्या राजगृहनिवासिनः}%।। ६७ ।।

\twolineshloka
{ब्राह्मणाः क्षत्त्रिया वैश्या ये च वर्णवरा जनाः}
{यानैरुच्चावचैः सर्वै नगरात्तु विनिर्गताः}%।। ६८ ।।

\twolineshloka
{वादित्रान्पुरतः कृत्वा गणिकाश्च विनिर्गताः}
{प्रतिग्रहनिमित्तं तु राघवस्य महात्मनः}%।। ६९ ।।

\twolineshloka
{भरतोऽपि महातेजाः स समेत्ययुधाजिता}
{अभ्यधावत तत्प्रीत्या मातुलं कैकयाधिपम्} %॥७०॥

\twolineshloka
{कण्ठे गृहीत्वा भरतं मूर्ध्न्युपाघ्राय चासकृत्}
{भरतोऽपि समादाय राजगेहमुपागमत्}%।।७१।। ।।

\twolineshloka
{समे मनोरमे देशे प्रभूतयवसेन्धने}
{शिबिरं भरतश्चक्रे नगरस्याविदूरतः}%।। ७२ ।।

\twolineshloka
{पृथक्पृथक्तदा चक्रुः स्कन्दावारनिवेशनम्}
{मनःप्रियेषु देशेषु नानादेश्या नराधिपाः}%।। ७३ ।।

\twolineshloka
{कृत्वा निवेशान्मनुजेश्वराणामादिश्य भोगान्सकलान्स तेषाम्}
{विवेश नागेन स मातुलस्य पुरं प्रहृष्टो रघुवंशचन्द्रः}%।। ७४ ।।

॥इति श्रीविष्णुधर्मोत्तरे प्रथमखण्डे मार्कण्डेयवज्रसंवादे राजगृहगमनो नाम सप्तोत्तरद्विशततमोऽध्यायः॥२०७॥

\sect{भरतस्य राजगृहप्रवेशवर्णनम् --- अष्टोत्तरद्विशततमोऽध्यायः}

\uvacha{मार्कण्डेय उवाच}

\twolineshloka
{ततः स्वल्पपरीवारो भरतो धर्मवत्सलः}
{अभ्यधाय ततः प्रीत्या मातुलः केकयाधिपः}%।। १ ।।

\twolineshloka
{कण्ठे गृहीत्वा भरतं मूर्ध्न्युपाघ्राय चासकृत्}
{भरतोऽपि समादाय राजन् गृहमुपागतम्}%।।२ ।।

\twolineshloka
{प्रविवेश पुरं हृष्टो मातुलस्य युधाजितः}
{पताकाध्वजसम्बाधं सिक्तं चन्दनवारिणा}%।।३ ।।

\twolineshloka
{धूपेनाऽगुरुसाराणां समन्तादाकुलीकृतम्}
{कृतोपहारं सर्वत्र कुसुमैः पञ्चवर्णकैः}%।। ४ ।।

\twolineshloka
{सर्वपण्यविसंयुक्तं समालङ्कृतबालकम्}
{खमुल्लिखद्भिः प्रासादैः श्वेतमेघगणैर्युतम्}%।।५ ।।

\twolineshloka
{विन्यस्तबलिपूजञ्च देवतायतनेषु च}
{महावादित्रघोषेण समन्तात्कृतनिःस्वनम्}%।।६।।

\twolineshloka
{पौरैर्दिदृश्रुभिः सवैर्राकीर्णापणवीथिकम्}
{राजमार्गे महाराज पिण्डीकृतमहाजनम्}%।।७।।)

\twolineshloka
{राजमार्गादपाकृष्टशून्यसर्वनिवेशनम्}
{योषिद्वृन्दसमाक्रान्त राजमार्गमहागृहम्}%।। ८ ।।

\twolineshloka
{प्रविवेश पुरं श्रीमान्भरतो नागधूर्गतः}
{तस्य कामाभवपुषः प्रवेशे भरतस्य तुं}%।। ९ ।।

\twolineshloka
{गृहकार्याणि सन्त्यज्य ययुर्नार्यो गवक्षकान्}
{काश्चिदर्धानुलिप्ताङ्ग्यः काश्चिदेकाञ्जितेक्षणाः} %॥१०॥

\twolineshloka
{केशैः संयमितैकार्धैः काश्चिदर्धनिवेशितैः}
{एकस्मिंश्चरणे काश्चिद्गृहीत्वा काष्ठपादुकाः}%।।११।।

\twolineshloka
{त्वरिताशा ययुर्नार्यो द्वितीये चर्मपादुकाः}
{तथा पराः समाक्षिप्य पूर्वाक्रान्तगवाक्षकाः}%।।१२।।

\twolineshloka
{व्रजन्तीषु तथान्यासु काश्चिन्नार्यो गवाक्षकात्}
{वेगवत्यो ययुः शीघ्रं सन्त्यक्त्वा चर्मपादुकाः}%।। १३ ।।

\twolineshloka
{नार्यः स्ववदनैश्चक्रुः सुवक्त्राँस्तान्गवाक्षकान्}
{अर्धप्रविष्टसत्कम्बुपाणिवारिजकुड्मलाः}%।। १४ ।।

\twolineshloka
{द्वितीयपाणिसन्दर्शसमाक्रान्तैर्ययुः करैः}
{नीवीबन्धनविश्लेषसमाकुलितचेतनाः}%।। १५ ।।

\twolineshloka
{ययुरेवापरास्तत्र पाणिसंश्लिष्टनीवयः}
{कुसुमप्रकरं काश्चिदूहमानाः शिरोगतैः}%।। १६ ।।

\twolineshloka
{ययुरेवांशुकैदीर्घैस्त्वरमाणा गवाक्षकान्}
{सितासितेन रामाणां रमणीयेन राघवः}%।। १७ ।।

\twolineshloka
{ययौ दृष्टिनिपातेन रज्यमान इवांशुमान्}
{स चकर्ष तदा तासां पतितैर्नेत्ररश्मिभिः}%।।१८ ।।

\twolineshloka
{हृदयान्निगृहीत्वेव गच्छमानः स राघवः}
{स तु दृग्विषये यासां यासां तस्मात्परागतः}%।।१९।।

\twolineshloka
{न ता बुबुधिरे काञ्चित्क्रियां चित्रता इव}
{भरते दूरयातेऽपि निश्चेष्टाः काश्चिदेव ताः} %॥२०॥

\twolineshloka
{आकृष्टास्ता ययुः क्षोणीं प्रेर्यमाणैः सखीजनैः}
{प्रद्युम्नांशसमुत्पन्नः प्रद्युम्नसमदर्शनः}%।। २१ ।।

\twolineshloka
{आदाय तासां चेतांसि ययौ गजगृहं द्रुतम्}
{तस्मिन्राजगृहे राजन्राघवो राजमन्दिरम्}%।। २२ ।।

\twolineshloka
{आससाद महातेजाः कैलासमिव चापरम्}
{प्रविश्य स गृहं मुख्यं प्रेषयामास भूभुजम्}%।। २३ ।।

पानभोजनवासांसि सैनिकानां च यादव ।।

\twolineshloka
{उवास स सुखं तत्र पूजितश्च युधाजिता}
{जगाम चास्तं सविता जपापुष्पोत्करप्रभः}%।। २४ ।।

\twolineshloka
{जाम्बूनदे रत्नसहस्रचित्रे स राजपुत्रः शयने विचित्रे}
{सुष्वाप रात्रौ भवने विचित्रे निदाघरात्रिं पवने विचित्रे}%।। २५ ।।

॥इति श्रीविष्णुधर्मोत्तरे प्रथमखण्डे मार्कण्डेयवज्रसंवादे भरतस्य राजगृहप्रवेशवर्णनं नामाष्टोत्तरद्विशततमोऽध्यायः॥२०८॥

\sect{युद्धप्रसङ्गवर्णनम् --- नवोत्तरद्विशततमोऽध्यायः}

\uvacha{मार्कण्डेय उवाच}

\twolineshloka
{तस्मिन्रात्र्यवसाने तु भूमिपानां पृथक्पृथक्}
{कटकेष्वभ्यहन्यन्त संज्ञातूर्याणि यादव}%।। १ ।।।

\twolineshloka
{बहूनां बलमुख्यानां नानालिङ्गानि भागशः}
{तेषां कोलाहलः शब्दो बभूव गगनङ्गमः}%।। २ ।।

\twolineshloka
{विविशुर्भरतस्यापि ततः प्राबोधिका जनाः}
{वेणिका गायना मुख्या वंशवाद्यविदश्च ये}%।। ३ ।।

\twolineshloka
{मार्दङ्गिका पाणविकाः शङ्खवादनकाश्च ये}
{रक्तकण्ठाः सुमधुरा ये च मङ्गलपाठकाः}%।। ४ ।।

\twolineshloka
{सूतमागधमुख्याश्च बन्दिनश्च नराधिप}
{तुष्टुवुर्भरतं वीरं सुखसुप्तं महामतिम्}%।। ५ ।।

\twolineshloka
{सुप्रभातं समुत्तिष्ठ प्रभाता रजनी शुभा}
{दिक्प्राची रघुशार्दूल वर्ततेऽरुणरञ्जिता}%।। ६ ।।

\twolineshloka
{नूनमस्यां हि वेलायां मातलिस्त्रिदशाधिपम्}
{विबोधयति राजेन्द्र सुखाय जगतां विभुम्}%।। ७ ।।

\twolineshloka
{त्वयि सुप्ते जगत्सुप्तं विबुद्धे च सुखान्वितम्}
{तस्मादुत्तिष्ठ लोकानां शिवाय रघुनन्दन}%।। ८ ।।

\twolineshloka
{त्वं हि सर्वगुणारामो यथा रामो महीपतिः}
{गुणैः शशाङ्करश्म्याभैस्त्वया वै रञ्जितं जगत्}%।।९।।

\twolineshloka
{अत्याश्चर्यं महाबाहो यशसा सुसतेन ते}
{मुखान्यरातिवृन्दानां क्रियन्ते मलिनानि यत्} %॥१०॥

\twolineshloka
{कृपाणधारापानीयं दृष्ट्वाऽरातिगणस्तव}
{तृष्णयैव भवत्याशु रञ्जितानिलयो भयात्}%।। ११ ।।

\twolineshloka
{अक्षोभ्यश्चातिगम्भीरो भवान्रत्नाकरस्तथा ।।।}
{अग्राह्यत्वात्समुद्रेण न समः प्रतिभाति नः}%।।१२।।

\twolineshloka
{सौम्यः कलावाँल्लक्ष्मीवान्नयनानन्दकारकः}
{(दोषाकरेण क्षयिणा नौपम्यं ते नराधिप)}%।।१३।।

\twolineshloka
{बाहुभोज्यातिविस्तीर्णा सर्वाश्रयवती दृढा}
{नौपम्यं याति ते सम्यक् क्षोणी विन्ध्येन कम्पिना}%।। १४ ।।

\twolineshloka
{यत्तेऽस्ति तदवश्यं त्वं ददासि रिपुसूदन}
{अविद्यमाना भीर्दत्ता भवतारिगणे कुतः}%।।१५ ।।

\twolineshloka
{निमेषमपि यो दृष्टस्त्वयापाङ्गनिरीक्षणैः}
{समग्रदृष्ट्या य दृष्टो नित्यमेवेक्षणैः श्रियः ।}%।। १६ ।।

\twolineshloka
{क्षुरपर्यन्तधारेण चक्रेणारिगणस्य ते}
{शिरांस्यपहरत्याजौ देवदेवो जनार्दनः}%।। १७ ।।

\twolineshloka
{कपालमाली खट्वाङ्गी शशाङ्ककृतभूषणः ।}
{वामार्धदयिताकारः शङ्करः शं करोतु ते}%।।१८।।

\twolineshloka
{पद्मासनः पद्मजन्मा सर्वलोकपितामहः}
{ऋद्धिं मेधां धृतिं लक्ष्मीं बलं च विदधातु ते}%।। १९।।

\twolineshloka
{नभश्चरोऽम्बुजो देवो दिग्वधूरपूरकः}
{ब्रह्माण्डमण्डपे दीपः सुप्रभातं करोतु ते} %॥२०॥

\twolineshloka
{केशयक्षेशदेवेशप्रेतेश्वरनिशाचराः}
{सर्वदेवगणैः सार्धं सुप्रभातं दिशन्तु ते}%।। २१ ।।

\twolineshloka
{ऋषयः सरितः शैलाः सागराश्च दिशो दश}
{कालस्यावयवाश्चैव सुप्रभातं दिशन्तु ते}%।। २२ ।।

\twolineshloka
{इति शृण्वन्गिरं पुण्यामेषां मङ्गलवादिनाम्}
{महापुण्याहघोषेण तत्याज शयनं तदा}%।।२३।।

\twolineshloka
{आयव्ययं स शुश्राव लेखकैर्गणकैः सह}
{वेगोत्सर्गं ततः कृत्वा ययौ स्नानगृहं शुभम्}%।। २४ ।।

\twolineshloka
{स्नानारम्भं ततश्चक्रे दन्तधावनपूर्वकम्}
{उत्सादितः कषायेण बलवद्भिर्नरैस्तदा}%।। २५ ।।

\twolineshloka
{सुखं भद्रासनासीनः सूक्ष्माम्बरधरः प्रभुः}
{सस्नौ स विविधैस्तोयैर्नदीसागरजैः शुभैः}%।। २६ ।।

\twolineshloka
{कुम्भैः सुवर्णमाहेयैस्ताम्रै रौप्यमयैस्तथा}
{शताधिकैर्महाराज सर्वौषधि समन्वितैः}%।। २७ ।।

\twolineshloka
{क्षीरप्रवाहसंयुक्तैर्माल्यकण्ठैः सुपूजितैः}
{आवर्जितैर्महीपाल सुस्नातालङ्कृतैर्नरैः}%।। २८ ।।

\twolineshloka
{चन्दनस्रावसम्पूर्णैर्द्विजमन्त्रानुमन्त्रितैः}
{बभार वसनं चारु स्नानतोयौघसङ्कुलम्}%।। २९ ।।

\twolineshloka
{शशाङ्कमण्डलं पूर्णं तन्वभ्रैरिवसंवृतम्}
{वाद्यपुण्याहघोषेण तथा गीतस्वनेन च} %॥३०॥

\twolineshloka
{स्नात्वोपस्पृश्य विधिवत्पूर्वां सन्ध्यां समाहितः}
{ददर्श वदनं चारु दर्पणे चाथ सर्पिषि}%।। ३१ ।।

\twolineshloka
{स सुवर्णे महाराज दैवज्ञेनाभिमन्त्रिते}
{दिनेशं तिथिनक्षत्रं ततः शुश्राव राघवः}%।। ३२ ।।

\twolineshloka
{सांवत्सरमुखोद्गीर्णं कलिदुःस्वप्ननाशनम्}
{चकाराभ्यर्चनं चाथ देवदेवस्य चक्रिणः}%।। ३३ ।।

\twolineshloka
{गन्धमाल्यनमस्कारधूपूदीपान्नसम्पदा}
{स्तवैर्बलिप्रदानैश्च गीतवाद्यस्वनेन च}%।। ३४ ।।

\twolineshloka
{सम्पूज्य देवदेवेशं विवेशाग्निगृहं शुभम्}
{तत्राग्निं समुपस्थाय हुतं पूर्वं पुरोधसा}%।। ३५ ।।

\twolineshloka
{दुःस्वप्ननाशनं कर्म सिद्धिवृद्धिजयप्रदम्}
{प्रागेव तस्य कृतवान्विद्वान्नृपपुरोहितः}%।। ३६ ।।

\twolineshloka
{ततस्त्वौपसमे वह्नौ भरतः प्रीतमानसः}
{श्रीसूक्तं पौरुषं सूक्तं जुहाव प्रयतस्तदा}%।। ३७ ।।

\twolineshloka
{आज्येन मन्त्रपूतेन विधिना सुसमाहितः}
{दत्त्वा पूर्णाहुतिं चाग्नौ कृतजप्यो महामतिः}%।। ३८ ।।

\twolineshloka
{उदकेनार्चनं चक्रे देवानां पितृभिः सह}
{ततः स निर्ययौ तस्माद्भरतो वह्निवेश्मनः}%।। ३९ ।।

\twolineshloka
{निष्क्रम्य पूजयामास ब्राह्मणान्वसना तदा}
{गोभिरश्वैः सुवर्णेन दधिपुष्पफलान्वितैः} %॥४०॥

\twolineshloka
{मोदकैश्च तथा रत्नैर्वस्त्रैश्च रघुनन्दनः}
{पूजितानां द्विजेन्द्राणां भरतेन महात्मना}%।। ४१ ।।

\twolineshloka
{पुण्याहघोषस्त्रिदिवं जगाम मधुरस्वरः}
{स पूजयित्वा विप्रेन्द्रान्प्रविश्य च तथा गृहम्}%।। ४२ ।।

\twolineshloka
{नित्यकर्म च कृत्वेदं चन्दनेन सुगन्धिना}
{सूक्ष्मशुक्लपरीधानो वरधूपेन धूपितः}%।। ४३ ।।

\twolineshloka
{आभूष्य सर्वगात्राणां भूषणानि रघूद्वहः}
{शुक्लं सुगन्धि माल्यं च स्रजश्च विविधास्तथा}%।। ४४ ।।

\twolineshloka
{मङ्गलालम्भनं कृत्वा निश्चक्राम सभागृहम्}
{क्लृप्तं शय्यासनं तत्र वररत्नविभूषितम्}%।। ४५ ।।

\twolineshloka
{महार्घतोरणोपेतं सोत्तरच्छदमृद्धिमत्}
{वितानञ्च तथा दत्तं तस्योपरि महर्द्धिमत्}%।। ४६ ।।

\twolineshloka
{आसीनमासने तस्न्निभरतं सत्यसङ्गरम्}
{प्रांशवो बद्धनिस्त्रिंशाः कवचोत्तमभूषिताः}%।। ४७ ।।

\twolineshloka
{रक्ताम्बरधरा वीरा ररक्षुः पृष्ठसंस्थिताः}
{तथैवोभयपार्श्वस्थाः पूर्णचन्द्रनिभाननाः}%।। ४८ ।।

\twolineshloka
{वारमुख्याः सुवेशास्तमुपासां चक्रिरे तदा}
{बालव्यजनधारिण्यस्तालवृन्तकराः पराः}%।। ४९।।

\twolineshloka
{ताम्बूलभाण्डधारिण्यो नीलनीजलोचनाः}
{कुण्डली बद्धनिस्त्रिंशो दण्डपाणिः सुवेशवान्} %॥५०॥

\twolineshloka
{उवाच भरतं क्षत्ता भूमिविन्यस्तजानुकः}
{दिदृक्षवस्ते सम्प्राप्ता ब्राह्मणाः संशयच्छिदः}%।। ५१ ।।

\twolineshloka
{श्रेणीमहत्तरा ये च बलमुख्यास्तथैव च}
{प्रवेशयैनानित्युक्तो द्वास्थान्क्षत्ता ततोऽब्रवीत}%।। ५२ ।।

\twolineshloka
{ब्राह्मणान्बलमुख्यांश्च प्रवेशयत सत्वरम्}
{आशीर्भिरभिनन्द्यैनं सम्प्रविष्टा द्विजोत्तमाः}%।। ५३ ।।

\twolineshloka
{बृसीषु दन्तपीठेषु विविशुश्च यथासुखम्}
{ततस्तु बलमुख्यानां नमतां भरतं तदा}%।। ५४ ।।

\twolineshloka
{क्षत्ता जग्राह नामानि कण्ठारक्तस्वरान्वितः}
{ततस्तेषूपविष्टेषु द्वारेषु विवृतेषु च}%।। ५५ ।।

\twolineshloka
{प्रविवेश जनः सर्वो न न्यवार्येत कश्चन}
{एतस्मिन्नेव काले तु शुश्रुवे तुमुलो ध्वनिः}%।। ५६ ।।।

\twolineshloka
{भरतं द्रष्टुकामानां भूमिपानां महात्मनाम्}
{ह्रादेन गजघण्टानां बृंहितेन तथैव च}%।। ५७ ।।

\twolineshloka
{ह्रेषितेन तुरङ्गाणां रथनेमिस्वनेन च}
{नामभिः कीर्त्यमानैश्च बन्दिभिः पृथिवीक्षिताम्}%।। ५८ ।।

\twolineshloka
{शङ्खवादित्रघोषेण पटहानां स्वनेन च}
{( आजग्मुर्भरतं द्रष्टुं नरेन्द्राः प्रियदर्शनाः}%।। ५९ ।।

\twolineshloka
{तोरणादवतीर्यैव वाहनेभ्यो हीक्षितः}
{सर्वे स्वल्पपरीवारा विविशुस्ते सभां शुभाम्} %॥६०॥

\twolineshloka
{शिरःकम्पेन भरतं नमस्कृत्य निवेदिताः}
{प्रतीहारेण दक्षेण तेन चकुर्वरासनम्}%।। ६१ ।।

\twolineshloka
{सिंहासनस्थान्नृपतीन्भरतस्त्वनुरूपयन्}
{गिरा पप्रच्छ कुशलं पूजयामास चाप्यथ}%।। ६२ ।।

\twolineshloka
{तुष्टुवुर्वन्दिनस्तत्र नानादेश्यान्नराधिपान्}
{निवेदयन्तः स्तुत्यन्ते भरताय महात्मने}%।।६३।।

\twolineshloka
{स्तुवतां भरतं तत्र बन्दिनां स महास्वनः}
{प्रासादभोगसंरुद्धो विपुलः समपद्यत}%।। ६४ ।। 

\twolineshloka
{क्ष्मापालमौलिमाणिक्यमरीचिविकटोज्ज्वलम्}
{बालातपांशुच्छुरितं बभूव च सभागृहम्}%।। ६५ ।।

\twolineshloka
{ततः स भरतः श्रीमान्विससर्ज नराधिपान्}
{स्वहस्तदत्तताम्बूलान्प्रतीहारनिवेदितान् ।।}%।। ६६ ।।

\twolineshloka
{ततः समुत्थाय ययौ द्वितीयगृहमुत्तमम्}
{तत्र चक्रे तदा मन्त्रं मातुलेन युधाजिता}%।। ६७ ।।

\onelineshloka*
{पुरोधसा च गार्ग्येण स्वेन कालविदा तथा}

{युधाजिदुवाच}
\onelineshloka
{काले त्वमीप्सिते प्राप्तो गन्धर्वाणां वधेच्छया}%।। ६८ ।।

\twolineshloka
{तत्र यावन्न जानन्ति गन्धर्वास्ते त्वदागमम्}
{अवस्कन्देन तानद्य तावद्रात्रौ जहि प्रभो}%।। ६९ ।।

\onelineshloka*
{अवस्कन्देन निधनं सुखं तेषां भविष्यति}

\uvacha{कालविदुवाच}

\onelineshloka
{सुप्ते प्रमत्ते विश्वस्ते तथा राजञ्च्छ्रमान्विते}%।।७०।।

\twolineshloka
{अवतीर्णे बले चैव सरितं वहतीं तथा}
{रात्रौ जागरणश्रान्ते छद्मयुद्धं विधीयते}%।। ७१ ।।

\twolineshloka
{रात्रौ विहारशीलास्ते गन्धर्वाः सततं प्रभो}
{न तेऽवस्कन्दमर्हन्ति रात्रौ कैकेयिनन्दन}%।। ७२ ।।

\onelineshloka
{विजयश्च दिवा युद्धे भरतस्य प्रदृश्यते}


\uvacha{गार्ग्य उवाच}
\onelineshloka*
{आदौ दूतेन वक्तव्यं गन्धर्वाणां प्रयोजनम्}%।। ७३ ।।

\twolineshloka
{यथा देशमिमं त्यक्त्वा व्रजध्वं तुहिनाचलम्}
{गन्धर्वाणां निवासस्तु हिमवत्यचलोत्तमे}%।। ७४ ।।

\twolineshloka
{पूर्वमेव कृतस्तेन येनेदं निर्मितं जगत्}
{स्थानमेतन्मनुष्याणां त्यक्तुमर्हथ मा चिरम्}%।। ७५ ।।

\twolineshloka
{ते च दूतवचः श्रुत्वा स्थानं दद्युर्नराधिप}
{असंशयमुपारुह्य चैतन्मम मतं भवेत्}%।। ७६ ।।

\uvacha{भरत उवाच}

\twolineshloka
{राघवाः सत्यसन्धास्तु कूटयुद्धं न शिक्षिताः}
{तस्मात्तेषां वधः कार्यः सुयुद्धेन मया नृप}%।। ७७ ।।

\twolineshloka
{गार्ग्यवाक्यं तथा बुद्ध्या रोचतेऽतिशयेन मे}
{प्रयातु तेषां दौत्येन गार्ग्य एव महायशाः}%।। ७८ ।।

\uvacha{युधाजिदुवाच}

\twolineshloka
{गच्छ गार्ग्य महाभाग गन्धर्वाधिपतिं प्रति}
{तं च श्रावय वाक्यानि त्वयोक्तानीह यानि मे}%।। ७९ ।।

\twolineshloka
{अक्रियायां तथा तेषां श्रावयाऽग्र्याणि म चिरम्}
{अस्यैव नाम्ना धर्मज्ञ भरतस्य महात्मनः} %॥८०॥

\uvacha{मार्कण्डेय उवाच}

\twolineshloka
{एवमस्त्वित्यथोक्तोऽसौ ययौ गार्ग्यो महायशाः}
{रथेन काञ्चनाङ्गेन गन्धर्वनगरं प्रति}%।। ८१ ।।

\twolineshloka
{गते पुरोहिते गार्ग्ये भरतोऽपि महाशयाः}
{हस्तिपृष्ठे रथे चाश्वे शस्त्रे शास्त्रे तथैव च}%।। ८२ ।।

\twolineshloka
{व्यायामं च तथा चक्रे नियुद्धे च यदूत्तम}
{उत्सारितस्तथा पदभ्यां धावद्भिः कुशलैर्जनैः}%।।८३।।

\twolineshloka
{स्नातैः सम्पूजितो विष्णुर्विधिवत्सात्त्वतोत्तमैः}
{नमस्कृत्य तथैवाग्निं हुतं सुष्ठु पुरोधसा}%।। ८४ ।।

\twolineshloka
{तथा भुक्तवतां श्रुत्वा पूजितानां द्विजन्मनाम्}
{पुण्याहघोषं वित्तेन भूमिपाल विसर्ज्य तान्}%।। ८५ ।।

\twolineshloka
{परार्ध्यचन्दनाक्ताङ्गस्तनुचारुसिताम्बरः}
{सर्वालङ्करणोपेतः स्रग्वी धूपेन धूपितः}%।। ८६ ।।

\twolineshloka
{आसीनस्त्वासने दिव्ये बुभुजे स्वजनैर्युतः}
{मातुलस्य महार्घाणि शुचीनि गुणवन्ति च}%।। ८७ ।।

\twolineshloka
{अपरीक्षितपूर्वाणि पुरुषैराप्तकारिभिः}
{नरपक्षिमृगाणान्तु लिङ्गैर्वह्नौ तथैव च}%।। ८८ ।।

\twolineshloka
{भक्ष्यं भोज्यं च लेह्यं च पेयं चोष्यं तथैव च}
{पात्रेषु रुक्मरौप्येषु तथा मणिमयेषु च}%।। ६९ ।।

\twolineshloka
{भुक्त्वान्नं गीतशब्देन चाप्तैः कतियैः सह}
{तथोपस्पृश्य धर्मात्मा दन्तधावनपूर्वकम्} %॥९०॥

\twolineshloka
{चक्रस्य शयनं भेजे वामपार्श्वेन शत्रुहा}
{इतिहासं स शुश्राव तत्रस्थः पुरुषोत्तमः}%।। ९१ ।।

\twolineshloka
{ततः स शयनं त्यक्त्वा शास्त्राभ्यासं महायशाः}
{चकार रघुशार्दूलः सतां मार्गमनुव्रजन्}%।। ९२ ।।

\twolineshloka
{एतस्मिन्नेव काले तु सह गार्ग्यो युधाजिता}
{विवेश भरतं द्रष्टुं रथरेणुपरिप्लुतः}%।। ९३ ।।

\onelineshloka*
{सुखासीनश्च भरतं वाक्यमेतत्ततोऽब्रवीत्}

\uvacha{गार्ग्य उवाच}
\onelineshloka
{ततो वाक्यानि सर्वाणि शैलूषः श्रावितो मया}%।। ९४ ।।

\twolineshloka
{न तानि तस्य रोचन्ते सङ्ग्रामस्तस्य रोचते}
{भरतेन समागम्य श्वोभूते द्विजपुङ्गव}%।। ९५ ।।

\twolineshloka
{भरतं नाशयिष्यामि नीहारं चन्द्रमा यथा}
{इत्युक्त्वा स तु मां राजा प्रेषयामास सत्वरः}%।। ९६ ।।

\twolineshloka
{आह्वानदुन्दुभिस्तत्र निष्क्रान्ते मयि चाहतः}
{एतज्ज्ञात्वा स युद्धाय प्रातः सज्जो भवेत्तव}%।। ९७ ।।

\uvacha{मार्कण्डेय उवाच}

\twolineshloka
{इति गार्ग्यवचः श्रुत्वा भरतो गार्ग्यमब्रवीत्}
{गच्छ शीघ्रं गृहं ब्रह्मञ्छ्रान्तो रथबलाध्वतः}%।।९८।।

\twolineshloka
{अहमाज्ञापयिष्यामि सर्वं साङ्ग्रामिकं विधिम्}
{एतावदुक्त्वा विजयं सेनाध्यक्षमथाऽब्रवीत्}%।।९९।।

\twolineshloka
{ममोष्ट्रवाहिभिः शीघ्रं शिबिरेषु महीक्षिताम्}
{योधानाज्ञापयत्वद्य श्वोभूताय रणाय वै} %॥१००॥

\twolineshloka
{युद्धावधानिकं सर्वं कर्तव्यं च तथा त्वया}
{एवमाज्ञाप्य नागानां तुरगाणां तथैव च}%।। १०१ ।।

\twolineshloka
{प्रत्यावेक्षां ततः कृत्वा सन्ध्यामन्वास्य पश्चिमाम्}
{रहोगतः स शुश्राव नराणां मूढभाषितम्}%।।१०२ ।।

\twolineshloka
{आरुरोह तदा श्रुत्वा प्रासादं हिमपाण्डुरम्}
{कैलासशिखराकारं निर्वातं रजनीमुखे ।।}%।। १०३ ।।

\twolineshloka
{ततस्तु सैनिकः कश्चिच्छिबिरे भरतस्य तु}
{तलं तलेनाभ्यहनत्पवनार्थं यदृच्छया}%।। १०४ ।।

\twolineshloka
{ततस्तु सैनिकैः सर्वैस्तलतालैर्महास्वनम्}
{चक्रिरे पुरुषव्याघ्र तस्मिन्काले दिवं गतम्}%।। १०५ ।।

\twolineshloka
{प्रववौ च तदानीतो वायुर्मनुजपुङ्गव}
{एतस्मिन्नेव काले तु मद्रराजस्तदांशुमान्}%।। १०६ ।।

\twolineshloka
{प्रासादवरमारूढो ज्ञातवान्भरतं तदा}
{दीपालोकेन लक्ष्म्या च प्रासादस्य विवृद्धया}%।। १०७ ।।।

\twolineshloka
{स जगाम तदा राजा प्रहसन्सैनिकं जनम्}
{भरतस्य प्रदास्यामि युक्त्यैव बलदर्शनम्}%।। १०८ ।।

\twolineshloka
{तृणमुष्टिमुपादीप्य सर्वोऽपि कटके जनः}
{पाणावादाय मुदितः क्ष्वेडाशब्दं करोतु वै}%।। १०९ ।।

\twolineshloka
{पार्थिवेनैवमुक्ते तु कटके तस् धीमतः}
{सोल्काहस्तो जनः सर्वः क्षणेन समपद्यत} %॥११०॥

\twolineshloka
{शिबिरं मद्रराज्ञस्तु द्वितीयमिव चाम्बरम्}
{बभूव तारकाचित्रमुल्काहस्तैस्तदा नरैः}%।। १११ ।।

\twolineshloka
{तद्बलौघमपर्यन्तं सोल्काहस्तैर्जनैर्वृतम्}
{दृष्ट्वा जगाम धर्मात्मा परां प्रीतिं रघूद्वहः}%।। ११२ ।।

\twolineshloka
{क्ष्वेडाः किलकिलाश्चैव श्रुत्वा हर्षमुपागतः}
{तस्मिन्प्रशान्ते ज्वलने मन्दीभूते च निस्वने}%।। ११३ ।।

\twolineshloka
{अभ्यहन्यन्त भूपानां शिबिरेषु पृथक्पृथक्}
{संज्ञातूर्याणि रम्याणि नानालिङ्गानि चाप्यथ}%।। ११४ ।।

\twolineshloka
{सुबहूनि महाराज तेन कोलाहलं महत्}
{बभूव प्रीतिजननं भरतस्य महात्मनः}%।। ११५ ।।

मन्त्रयित्वा ततः श्रीमान्क्षणमात्रं युधाजिता।।

\twolineshloka
{तेनैव सह भुक्त्वा च सुष्वाप शयनोत्तमे}
{मधुरेण सुगीतेन वीणावेणुरवेण च}%।। ११६ ।।

\twolineshloka
{हिमावदाते वरमाल्यचित्रे वितानकाधोविहिते मनोज्ञे}
{सुष्वाप रात्रिं स महानुभावो भोगीन्द्रभोगे मधुजिद्यथैव}%।। ११७ ।।

॥इति श्रीविष्णुधर्मोत्तरे प्रथमखण्डे मार्कण्डेयवज्रसंवादे युद्धप्रसङ्गवर्णनं नाम नवोत्तरद्विशततमोऽध्यायः॥२०९॥
