\sect{रामेश्वरक्षेत्रमाहात्म्यवर्णनम्}

\src{स्कन्दपुराणम्}{खण्डः ७ (प्रभासखण्डः)}{प्रभासक्षेत्र माहात्म्यम्}{अध्यायः १११}
\vakta{}
\shrota{}
\tags{}
\notes{}
\textlink{https://sa.wikisource.org/wiki/स्कन्दपुराणम्/खण्डः_७_(प्रभासखण्डः)/प्रभासक्षेत्र_माहात्म्यम्/अध्यायः_१११}
\translink{https://www.wisdomlib.org/hinduism/book/the-skanda-purana/d/doc626899.html}

\storymeta




\uvacha{ईश्वर उवाच}

\twolineshloka
{ततो गच्छेन्महादेवि पुष्करारण्यमुत्तमम्}
{तस्मादीशानकोणस्थं धनुषां षष्टिभिः स्थितम्}%॥ १ ॥

\twolineshloka
{तत्र कुण्डं महादेवि ह्यष्टपुष्करसंज्ञितम्}
{सर्व पापहरं देवि दुष्प्राप्यमकृतात्मभिः}%॥ २ ॥

\twolineshloka
{तत्र कुण्डसमीपे तु पुरा रामेशधीमता}
{स्थापितं तन्महालिङ्गं रामेश्वर इति स्मृतम्}%॥ ३

\onelineshloka
{तस्य पूजनमात्रेण मुच्यते ब्रह्महत्यया}%॥ ४ ॥

\uvacha{श्रीदेव्युवाच}

\twolineshloka
{भगवन्विस्तराद्ब्रूहि रामेश्वरसमुद्भवम्}
{कथं तत्रागमद्रामः ससीतश्च सलक्ष्मणः}%॥ ५ ॥

\twolineshloka
{कथं प्रतिष्ठितं लिङ्गं पुष्करे पापतस्करे}
{एतद्विस्तरतो ब्रूहि फलं माहात्म्यसंयुतम्}%॥ ६ ॥

\uvacha{ईश्वर उवाच}

\twolineshloka
{चतुर्विंशयुगे रामो वसिष्ठेन पुरोधसा}
{पुरा रावणनाशार्थं जज्ञे दशरथात्मजः}%॥ ७ ॥

\twolineshloka
{ततः कालान्तरे देवि ऋषिशापान्महातपाः}
{ययौ दाशरथी रामः ससीतः सहलक्ष्मणः}%॥ ८ ॥

\twolineshloka
{वनवासाय निष्क्रान्तो दिव्यैर्ब्रह्मर्षिभिर्वृतः}
{ततो यात्राप्रसङ्गेन प्रभासं क्षेत्रमागतः}%॥ ९ ॥

\twolineshloka
{तं देशं तु समासाद्य सुश्रान्तो निषसाद ह}
{अस्तं गते ततः सूर्ये पर्णान्यास्तीर्य भूतले}%॥ १० ॥

\twolineshloka
{सुष्वापाथ निशाशेषे ददृशे पितरं स्वकम्}
{स्वप्ने दशरथं देवि सौम्यरूपं महाप्रभम्}%॥ ११ ॥

\twolineshloka
{प्रातरुत्थाय तत्सर्वं ब्राह्मणेभ्यो न्यवेदयत्}
{यथा दशरथः स्वप्ने दृष्टस्तेन महात्मना}%॥ १२ ॥

\uvacha{ब्राह्मणा ऊचुः}

\twolineshloka
{वृद्धिकामाश्च पितरो वरदास्तव राघव}
{दर्शनं हि प्रयच्छन्ति स्वप्नान्ते हि स्ववंशजे}%॥ १३ ॥

\twolineshloka
{एतत्तीर्थं महापुण्यं सुगुप्तं शार्ङ्गधन्वनः}
{पुष्करेति समाख्यातं श्राद्धमत्र प्रदीयताम्}%॥ १४ ॥

\twolineshloka
{नूनं दशरथो राजा तीर्थे चास्मिन्समीहते}
{त्वया दत्तं शुभं पिण्डं ततः स दर्शनं गतः}%॥ १५ ॥

\uvacha{ईश्वर उवाच}

\twolineshloka
{तेषां तद्वचनं श्रुत्वा रामो राजीवलोचनः}
{निमन्त्रयामास तदा श्राद्धार्हान्ब्राह्मणाञ्छुभान्}%॥ १६ ॥

\twolineshloka
{अब्रवील्लक्ष्मणं पार्श्वे स्थितं विनतकन्धरम्}
{फलार्थं व्रज सौमित्रे श्राद्धार्थं त्वरयाऽन्वितः}%॥ १७ ॥

\twolineshloka
{स तथेति प्रतिज्ञाय जगाम रघुनन्दनः}
{आनयामास शीघ्रं स फलानि विविधानि च}%॥ १८ ॥

\twolineshloka
{बिल्वानि च कपित्थानि तिन्दुकानि च भूरिशः}
{बदराणि करीराणि करमर्दानि च प्रिये}%॥ १९ ॥

\twolineshloka
{चिर्भटानि परूषाणि मातुलिङ्गानि वै तथा}
{नालिकेराणि शुभ्राणि इङ्गुदीसम्भवानि च}%॥ २० ॥

\twolineshloka
{अथैतानि पपाचाशु सीता जनकनन्दिनी}
{ततस्तु कुतपे काले स्नात्वा वल्कलभृच्छुचिः}%॥ २१ ॥

\twolineshloka
{ब्राह्मणानानयामास श्राद्धार्हान्द्विजसत्तमान्}
{गालवो देवलो रैभ्यो यवक्रीतोऽथ पर्वतः}%॥ २२ ॥

\twolineshloka
{भरद्वाजो वसिष्ठश्च जावालिर्गौतमो भृगुः}
{एते चान्ये च बहवो ब्राह्मणा वेदपारगाः}%॥ २३ ॥

\twolineshloka
{श्राद्धार्थं तस्य सम्प्राप्ता रामस्याक्लिष्टकर्मणः}
{एतस्मिन्नेव काले तु रामः सीतामभाषत}%॥ २४ ॥

\twolineshloka
{एहि वैदेहि विप्राणां देहि पादावनेजनम्}
{एतच्छ्रुत्वाऽथ सा सीता प्रविष्टा वृक्षमध्यतः}%॥ २५ ॥

\twolineshloka
{गुल्मैराच्छाद्य चात्मानं रामस्यादर्शने स्थिता}
{मुहुर्मुहुर्यदा रामः सीतासीतामभाषत}%॥ २६ ॥

\twolineshloka
{ज्ञात्वा तां लक्ष्मणो नष्टां कोपाविष्टं च राघवम्}
{स्वयमेव तदा चक्रे ब्राह्मणार्ह प्रतिक्रियाम्}%॥ २७ ॥

\twolineshloka
{अथ भुक्तेषु विप्रेषु कृत पिण्डप्रदानके}
{आगता जानकी सीता यत्र रामो व्यवस्थितः}%॥ २८ ॥

\threelineshloka
{तां दृष्ट्वा परुषैर्वाक्यैर्भर्त्सयामास राघवः}
{धिग्धिक्पापे द्विजांस्त्यक्त्वा पितृकृत्यमहोदयम्}
{क्व गताऽसि च मां हित्वा श्राद्धकाले ह्युपस्थिते}%॥ २९ ॥

\uvacha{ईश्वर उवाच}

\onelineshloka
{तस्य तद्वचनं श्रुत्वा भयभीता च जानकी}%॥ ३० ॥

\twolineshloka
{कृताञ्जलिपुटा भूत्वा वेपमाना ह्यभाषत}
{मा कोपं कुरु कल्याण मा मां निर्भर्त्सय प्रभो}%॥ ३१ ॥

\twolineshloka
{शृणु यस्माद्विभोऽन्यत्र गता त्यक्त्वा तवान्तिकम्}
{दृष्टस्तत्र पिता मेऽद्य तथा चैव पितामहः}%॥ ३२ ॥

\twolineshloka
{तस्य पूर्वतरश्चापि तथा मातामहादयः}
{अङ्गेषु ब्राह्मणेन्द्राणामाक्रान्तास्ते पृथक्पृथक्}%॥ ३३ ॥

\twolineshloka
{ततो लज्जा समभवत्तत्र मे रघुनन्दन}
{पित्रा तत्र महाबाहो मनोज्ञानि शुभानि च}%॥ ३४ ॥

\threelineshloka
{भक्ष्याणि भक्षितान्येव यानि वै गुणवन्ति च}
{स कथं सुकषायाणि क्षाराणि कटुकानि च}
{भक्षयिष्यति राजेन्द्र ततो मे दुःखमाविशत्}%॥ ३६ ॥

\twolineshloka
{एतस्मात्कारणान्नष्टा लज्जयाऽहं रघूद्वह}
{दृष्ट्वा श्वशुरवर्गं स्वं तस्मात्कोपं परित्यज}%॥ ३६ ॥

\twolineshloka
{तस्यास्तद्वचनं श्रुत्वा विस्मितो राघवोऽभवत्}
{विशेषेण ददौ तस्मिञ्छ्राद्धं तीर्थे तु पुष्करे}%॥ ३७ ॥

\twolineshloka
{तत्र पुष्करसान्निध्ये दक्षिणे धनुषां त्रये}
{लिङ्गं प्रतिष्ठयामास रामेश्वरमिति श्रुतम्}%॥ ३५ ॥

\twolineshloka
{यस्तं पूजयते भक्त्या गन्धपुष्पादिभिः क्रमात्}
{स प्राप्नोति परं स्थानं य्रत्र देवो जनार्दनः}%॥ ३९ ॥

\twolineshloka
{किमत्र बहुनोक्तेन द्वादश्यां यत्प्रदापयेत्}
{न तत्र परिसङ्ख्यानं त्रिषु लोकेषु विद्यते}%॥ ४० ॥

\twolineshloka
{शुक्राङ्गारकसंयुक्ता चतुर्थी या भवेत्क्वचित्}
{षष्ठी वात्र वरारोहे तत्र श्राद्धे महत्फलम्}%॥ ४१ ॥

\twolineshloka
{यावद्द्वादशवर्षाणि पितरश्च पितामहाः}
{तर्पिता नान्यमिच्छन्ति पुष्करे स्वकुलोद्भवे}%॥ ४२ ॥

\twolineshloka
{तत्र यो वाजिनं दद्यात्सम्यग्भक्तिसमन्वितः}
{अश्वमेधस्य यज्ञस्य फलं प्राप्नोति मानवः}%॥ ४३ ॥

\twolineshloka
{इति ते कथितं सम्यङ्माहात्म्यं पापनाशनम्}
{रामेश्वरस्य देवस्य पुष्करस्य च भामिनि}%॥ ४४ ॥

॥इति श्रीस्कान्दे महापुराण एकाशीतिसाहस्र्यां संहितायां सप्तमे प्रभासखण्डे प्रथमे प्रभासक्षेत्रमाहात्म्ये पुष्करमाहात्म्ये रामेश्वरक्षेत्रमाहात्म्यवर्णनं नामैकादशोत्तरशततमोऽध्यायः॥१११॥