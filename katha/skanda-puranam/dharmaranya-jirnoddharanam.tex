\sect{धर्मारण्यतीर्थक्षेत्रजीर्णोद्धारवर्णनम्}


\src{स्कन्दपुराणम्}{खण्डः ३ (ब्रह्मखण्डः)}{धर्मारण्य खण्डः}{अध्यायाः ३१--३५}
\vakta{}
\shrota{}
\tags{}
\notes{These five chapters describe Rāma’s pilgrimage to Dharmāraṇya, the establishment of Satyamandira, Rāma's Return to Ayodhyā, Rāma’s copper-plate grant to Brāhmaṇas and ultimately the repair of the ruins of Dharmāraṇya.}
\textlink{https://sa.wikisource.org/wiki/स्कन्दपुराणम्/खण्डः_३_(ब्रह्मखण्डः)/धर्मारण्य_खण्डः/अध्यायः_३१}
\translink{https://www.wisdomlib.org/hinduism/book/the-skanda-purana/d/doc423652.html}

\storymeta

\dnsub{एकत्रिंशोऽध्यायः --- दूतागमनम्}\resetShloka

\uvacha{श्रीराम उवाच}

\twolineshloka
{भगवन्यानि तीर्थानि सेवितानि त्वया विभो}
{एतेषां परमं तीर्थं तन्ममाचक्ष्व मानद}%॥ १ ॥

\twolineshloka
{मया तु सीताहरणे निहता ब्रह्मराक्षसाः}
{तत्पापस्य विशुदयर्थं वद तीर्थोत्तमोत्तमम्}%॥ २ ॥

\uvacha{वसिष्ठ उवाच}

\twolineshloka
{गङ्गा च नर्मदा तापी यमुना च सरस्वती}
{गण्डकी गोमती पूर्णा एता नद्यः सुपावनाः}%॥ ३ ॥

\twolineshloka
{एतासां नर्मदा श्रेष्ठा गङ्गा त्रिपथगामिनी}
{दहते किल्बिषं सर्वं दर्शनादेव राघव}%॥ ४ ॥

\twolineshloka
{दृष्ट्वा जन्मशतं पापं गत्वा जन्मशतत्रयम्}
{स्नात्वा जन्मसहस्रं च हन्ति रेवा कलौ युगे}%॥ ५ ॥

\twolineshloka
{नर्मदातीरमाश्रित्य शाकमूलफलैरपि}
{एकस्मिन्भोजिते विप्रे कोटि भोजफलं लभेत}%॥ ६ ॥

\twolineshloka
{गङ्गा गङ्गेति यो ब्रूयाद्योजनानां शतैरपि}
{मुच्यते सर्वपापेभ्यो विष्णुलोकं स गच्छति}%॥ ७ ॥

\twolineshloka
{फाल्गुनान्ते कुहूं प्राप्य तथा प्रौष्ठपदेऽसिते}
{पक्षे गङ्गामधि प्राप्य स्नानं च पितृतर्पणम्}%॥ ८ ॥

\twolineshloka
{कुरुते पिण्डदानानि सोऽक्षयं फलमश्नुते}
{शुचौ मासे च सम्प्राप्ते स्नानं वाप्यां करोति यः}%॥ ९ ॥

\twolineshloka
{चतुरशीतिनरकान्न पश्यति नरो नृप}
{तपत्याः स्मरणे राम महापातकिनामपि}%॥ १० ॥

\twolineshloka
{उद्धरेत्सप्तगोत्राणि कुलमेकोत्तरं शतम्}
{यमुनायां नरः स्नात्वा सर्वपापैः प्रमुच्यते}%॥ ११ ॥

\twolineshloka
{महापातकयुक्तोऽपि स गच्छेत्परमां गतिम्}
{कार्त्तिक्यां कृत्तिकायोगे सरस्वत्यां निमज्जयेत्}%॥ १२ ॥

\twolineshloka
{गच्छेत्स गरुडारूढः स्तूयमानः सुरोत्तमैः}
{स्नात्वा यः कार्तिके मासि यत्र प्राची सरस्वती}%॥ १३ ॥

\twolineshloka
{प्राचीं माधवमास्तूय स गच्छेत्परमां गतिम्}
{गण्डकीपुण्यतीर्थे हि स्नानं यः कुरुते नरः}%॥ १४ ॥

\twolineshloka
{शालग्रामशिलामर्च्य न भूयः स्तनपो भवेत्}
{गोमतीजलकल्लोलैर्मज्जयेत्कृष्णसन्निधौ}%॥ १५ ॥

\twolineshloka
{चतुर्भुजो नरो भूत्वा वैकुण्ठे मोदते चिरम्}
{चर्मण्वतीं नमस्कृत्य अपः स्पृशति यो नरः}%॥ १६ ॥

\twolineshloka
{स तारयति पूर्वजान्दश पूर्वान्दशापरान्}
{द्वयोश्च सङ्गमं दृष्ट्वा श्रुत्वा वा सागरध्वनिम्}%॥ १७ ॥

\twolineshloka
{ब्रह्महत्यायुतो वापि पूतो गच्छेत्परां गतिम्}
{माघमासे प्रयागे तु मज्जनं कुरुते नरः}%॥ १८ ॥

\twolineshloka
{इह लोके सुखं भुक्त्वा अन्ते विष्णुपदं व्रजेत्}
{प्रभासे ये नरा राम त्रिरात्रं ब्रह्मचारिणः}%॥ १९ ॥

\twolineshloka
{यमलोकं न पश्येयुः कुम्भीपाकादिकं तथा}
{नैमिषारण्यवासी यो नरो देवत्वमाप्नुयात्}%॥ २० ॥

\twolineshloka
{देवानामालयं यस्मात्तदेव भुवि दुर्लभम्}
{कुरुक्षेत्रे नरो राम ग्रहणे चन्द्रसूर्ययोः}%॥ २१ ॥

\twolineshloka
{हेमदानाच्च राजेन्द्र न भूयः स्तनपो भवेत्}
{श्रीस्थले दर्शनं कृत्वा नरः पापात्प्रमुच्यते}%॥ २२ ॥

\twolineshloka
{सर्वदुःखविनाशे च विष्णुलोके महीयते}
{काश्यपीं स्पर्शयेद्यो गां मानवो भुवि राघव}%॥ २३ ॥

\twolineshloka
{सर्वकामदुघावासमृषिलोकं स गच्छति}
{उज्जयिन्यां तु वैशाखे शिप्रायां स्नानमाचरेत्}%॥ २४ ॥

\twolineshloka
{मोचयेद्रौरवाद् घोरात्पूर्वजांश्च सहस्रशः}
{सिन्धुस्नानं नरो राम प्रकरोति दिनत्रयम्}%॥ २५ ॥

\twolineshloka
{सर्वपापविशुद्धात्मा कैलासे मोदते नरः}
{कोटितीर्थे नरः स्नात्वा दृष्ट्वा कोटीश्वरं शिवम्}%॥ २६ ॥

\twolineshloka
{ब्रह्महत्यादिभिः पापैर्लिप्यते न च स क्वचित्}
{अज्ञानामपि जन्तूनां महाऽमेध्ये तु गच्छताम्}%॥ २७ ॥

\twolineshloka
{पादोद्भूतं पयः पीत्वा सर्वपापं प्रणश्यति}
{वेदवत्यां नरो यस्तु स्नाति सूर्योदये शुभे}%॥ २८ ॥

\twolineshloka
{सर्वरोगात्प्रमुच्येत परं सुखमवाप्नुयात्}
{तीर्थानि राम सर्वत्र स्नानपानावगाहनैः}%॥ २९ ॥

\twolineshloka
{नाशयन्ति मनुष्याणां सर्वपापानि लीलया}
{तीर्थानां परमं तीर्थं धर्मारण्यं प्रचक्षते}%॥ ३० ॥

\twolineshloka
{ब्रह्मविष्णुशिवाद्यैर्यदादौ संस्थापितं पुरा}
{अरण्यानां च सर्वेषां तीर्थानां च विशेषतः}%॥ ३१ ॥

\twolineshloka
{धर्मारण्यात्परं नास्ति भुक्तिमुक्तिप्रदायकम्}
{स्वर्गे देवाः प्रशंसन्ति धर्मारण्यनिवासिनः}%॥ ३२ ॥

\twolineshloka
{ते पुण्यास्ते पुण्यकृतो ये वसन्ति कलौ नराः}
{धर्मारण्ये रामदेव सर्वकिल्बिषनाशने}%॥ ३३ ॥

\twolineshloka
{ब्रह्महत्यादिपापानि सर्वस्तेयकृतानि च}
{परदारप्रसङ्गादि अभक्ष्यभक्षणादि वै}%॥ ३४ ॥

\twolineshloka
{अगम्यागमना यानि अस्पर्शस्पर्शनादि च}
{भस्मीभवन्ति लोकानां धर्मारण्यावगाहनात्}%॥ ३५ ॥

\twolineshloka
{ब्रह्मघ्नश्च कृतघ्नश्च बालघ्नोऽनृतभाषणः}
{स्त्रीगोघ्नश्चैव ग्रामघ्रो धर्मारण्ये विमुच्यते}%॥ ३६ ॥

\twolineshloka
{नातः परं पावनं हि पापिनां प्राणिनां भुवि}
{स्वर्ग्यं यशस्यमायुष्यं वाञ्छितार्थप्रदं शुभम्}%॥ ३७ ॥

\twolineshloka
{कामिनां कामदं क्षेत्रं यतीनां मुक्तिदायकम्}
{सिद्धानां सिद्धिदं प्रोक्तं धर्मारण्यं युगेयुगे}%॥ ३८ ॥

\uvacha{ब्रह्मोवाच}

\twolineshloka
{वसिष्ठवचनं श्रुत्वा रामो धर्मभृतां वरः}
{परं हर्षमनुप्राप्य हृदयानन्दकारकम्}%॥ ३९ ॥

\twolineshloka
{प्रोत्फुल्लहृदयो रामो रोमाचिन्ततनूरुहः}
{गमनाय मतिं चक्रे धर्मारण्ये शुभव्रतः}%॥ ४० ॥

\twolineshloka
{यस्मिन्कीटपतङ्गादिमानुषाः पशवस्तथा}
{त्रिरात्रसेवनेनैव मुच्यन्ते सर्वपातकैः}%॥ ४१ ॥

\twolineshloka
{कुशस्थली यथा काशी शूलपाणिश्च भैरवः}
{यथा वै मुक्तिदो राम धर्मारण्यं तथोत्तमम्}%॥ ४२ ॥

\twolineshloka
{ततो रामो महेष्वासो मुदा परमया युतः}
{प्रस्थितस्तीर्थयात्रायां सीतया भ्रातृभिः सह}%॥ ४३ ॥

\twolineshloka
{अनुजग्मुस्तदा रामं हनुमांश्च कपीश्वरः}
{कौशल्या च सुमित्रा च कैकेयी च मुदान्विता}%॥ ४४ ॥

\twolineshloka
{लक्ष्मणो लक्षणोपेतो भरतश्च महामतिः}
{शत्रुघ्नः सैन्यसहितोप्ययोध्यावासिनस्तथा}%॥ ४५ ॥

\twolineshloka
{प्रकृतयो नरव्याघ्र धर्मारण्ये विनिर्ययुः}
{अनुजग्मुस्तदा रामं मुदा परमया युताः}%॥ ४६ ॥

\twolineshloka
{तीर्थयात्राविधिं कर्तुं गृहात्प्रचलितो नृपः}
{वसिष्ठं स्वकुलाचार्यमिदमाह महीपते}%॥ ४७ ॥

\uvacha{श्रीराम उवाच}

\twolineshloka
{एतदाश्चर्यमतुलं किमादि द्वारकाभवत्}
{कियत्कालसमुत्पन्ना वसिष्ठेदं वदस्व मे}%॥ ४८ ॥

\uvacha{वसिष्ठ उवाच}

\twolineshloka
{न जानामि महाराज कियत्कालादभूदिदम्}
{लोमशो जाम्बवांश्चैव जानातीति च कारणम्}%॥ ४९ ॥

\twolineshloka
{शरीरे यत्कृतं पापं नानाजन्मान्तरेष्वपि}
{प्रायश्चितं हि सर्वेषामेतत्क्षेत्र परं स्मृतम्}%॥ ५० ॥

\twolineshloka
{श्रुत्वेति वचनं तस्य रामं ज्ञानवतां वरः}
{गन्तुं कृतमतिस्तीर्थं यात्राविधिमथाचरत्}%॥ ५१ ॥

\twolineshloka
{वसिष्ठं चाग्रतः कृत्वा महामाण्डलिकैर्नृपैः}
{पुनश्चरविधिं कृत्वा प्रस्थितश्चोत्तरां दिशम्}%॥ ५२ ॥

\twolineshloka
{वसिष्ठं चाग्रतः कृत्वा प्रतस्थे पश्चिमां दिशम्}
{ग्रामाद्ग्राममतिक्रम्य देशाद्देशं वनाद्वनम्}%॥ ५३ ॥

\twolineshloka
{विमुच्य निर्ययौ रामः ससैन्यः सपरिच्छदः}
{गजवाजिसहस्रौघै रथैर्यानैश्च कोटिभिः}%॥ ५४ ॥

\twolineshloka
{शिबिकाभिश्चासङ्ख्याभिः प्रययौ राघवस्तदा}
{गजारूढः प्रपश्यंश्च देशान्विविधसौहृदान्}%॥ ५५ ॥

\twolineshloka
{श्वेतातपत्रं विधृत्य चामरेण शुभेन च}
{वीजितश्च जनौघेन रामस्तत्र समभ्यगात्}%॥ ५६ ॥

\twolineshloka
{वादित्राणां स्वनैघोरैर्नृत्यगीतपुरःसरैः}
{स्तूयमानोपि सूतैश्च ययौ रामो मुदान्वितः}%॥ ५७ ॥

\twolineshloka
{दशमेऽहनि सम्प्राप्तं धर्मारण्यमनुत्तमम्}
{अदूरे हि ततो रामो दृष्ट्वा माण्डलिकं पुरम्}%॥ ५८ ॥

\twolineshloka
{तत्र स्थित्वा ससैन्यस्तु उवास निशि तां पुरीम्}
{श्रुत्वा तु निर्जनं क्षेत्रमुद्वसं च भयानकम्}%॥ ५९ ॥

\threelineshloka
{व्याघ्रसिंहाकुलं तत्र यक्षराक्षससेवितम्}
{श्रुत्वा जनमुखाद्रामो धर्मारण्यमरण्यकम्}
{तच्छ्रुत्वा रामदेवस्तु न चिन्ता क्रियतामिति}%॥ ६०

\onelineshloka
{तत्रस्थान्वणिजः शूरान्दक्षान्स्वव्यवसायके}%॥ ६१ ॥

\twolineshloka
{समर्थान्हि महाकायान्महाबलपराक्रमान्}
{समाहूय तदा काले वाक्यमेतदथाब्रवीत्}%॥ ६२ ॥

\twolineshloka
{शिबिकां सुसुवणां मे शीघ्रं वाहयताचिरम्}
{यथा क्षणेन चैकेन धर्मरण्यं व्रजाम्यहम्}%॥ ६३ ॥

\twolineshloka
{तत्र स्नात्वा च पीत्वा च सर्वपापात्प्रमुच्यते}
{एवं ते वणिजः सर्वै रामेण प्रेरितास्तदा}%॥ ६४ ॥

\twolineshloka
{तथेत्युक्त्वा च ते सर्वे ऊहुस्तच्छिबिकां तदा}
{क्षेत्रमध्ये यदा रामः प्रविष्टः सहसैनिकः}%॥ ६५ ॥

\twolineshloka
{तद्यानस्य गतिर्मन्दा सञ्जाता किल भारत}
{मन्दशब्दानि वाद्यानि मातङ्गा मन्दगामिनः}%॥ ६६ ॥

\twolineshloka
{हयाश्च तादृशा जाता रामो विस्मय मागतः}
{गुरुं पप्रच्छ विनयाद्वशिष्ठं मुनिपुङ्गवम्}%॥ ६७ ॥

\twolineshloka
{किमेतन्मन्दगतयश्चित्रं हृदि मुनीश्वर}
{त्रिकालज्ञो मुनिः प्राह धर्मक्षेत्रमुपागतम्}%॥ ६८ ॥

\twolineshloka
{तीर्थे पुरातने राम पादचारेण गम्यते}
{एवं कृते ततः पश्चात्सैन्यसौख्यं भविष्यति}%॥ ६९ ॥

\twolineshloka
{पादचारी ततौ रामः सैन्येन सह संयुतः}
{मधुवासनके ग्रामे प्राप्तः परमभावनः}%॥ ७० ॥

\twolineshloka
{गुरुणा चोक्तमार्गेण मातॄणां पूजनं कृतम्}
{नानोपहारैर्विविधैः प्रतिष्ठाविधिपूर्वकम्}%॥ ७३ ॥

\twolineshloka
{ततो रामो हरिक्षेत्रं सुवर्णादक्षिणे तटे}
{निरीक्ष्य यज्ञयोग्याश्च भूमीर्वै बहुशस्तथा}%॥ ७२ ॥

\twolineshloka
{कृतकृत्यं तदात्मानं मेने रामो रघूद्वहः}
{धर्मस्थानं निरीक्ष्याथ सुवर्णाक्षोत्तरे तटे}%॥ ७३ ॥

\twolineshloka
{सैन्यसङ्घं समुत्तीर्य्य बभ्राम क्षेत्रमध्यतः}
{तत्र तीर्थेषु सर्वेषु देवतायतनेषु च}%॥ ७४ ॥

\twolineshloka
{यथोक्तानि च कर्माणि रामश्चक्रे विधानतः}
{श्राद्धानि विधिवच्चक्रे श्रद्धया परया युतः}%॥ ७५ ॥

\twolineshloka
{स्थापयामास रामेशं तथा कामेश्वरं पुनः}
{स्थानाद्वायुप्रदेशे तु सुवर्णो भयतस्तटे}%॥ ७६ ॥

\twolineshloka
{कृत्वैवं कृतकृत्योऽभूद्रामो दशरथात्मजः}
{कृत्वा सर्वविधिं चैव सभायां समुपाविशत्}%॥ ७७ ॥

\twolineshloka
{तां निशां स नदीतीरे सुष्वाप रघुनन्दनः}
{ततोऽर्द्धरात्रे सञ्जाते रामो राजीवलोचनः}%॥ ७८ ॥

\twolineshloka
{जागृतस्तु तदा काल एकाकी धर्मवत्सलः}
{अश्रौषीच्च क्षणे तस्मिन्रामो नारीविरोदनम्}%॥ ७९ ॥

\twolineshloka
{निशायां करुणैर्वाक्यै रुदन्तीं कुररीमिव}
{चारैर्विलोकयामास रामस्तामतिसम्भ्रमात्}%॥ ८० ॥

\twolineshloka
{दृष्ट्वातिविह्वलां नारीं क्रन्दन्तीं करुणैः स्वरैः}
{पृष्टा सा दुःखिता नारी रामदूतैस्तदानघ}%॥ ८१ ॥

\uvacha{दूता ऊचुः}

\twolineshloka
{कासि त्वं सुभगे नारि देवी वा दानवी नु किम्}
{केन वा त्रासितासि त्वं मुष्टं केन धनं तव}%॥ ८२ ॥

\twolineshloka
{विकला दारुणाञ्छब्दानुद्गिरन्ती मुहुर्मुहुः}
{कथयस्व यथातथ्यं रामो राजाभिपृच्छति}%॥ ८३ ॥

\twolineshloka
{तयोक्तं स्वामिनं दूताः प्रेषयध्वं ममान्तिकम्}
{यथाहं मानसं दुःखं शान्त्यै तस्मै निवेदये}%॥ ८४

\onelineshloka
{तथेत्युक्त्वा ततो दूता राममागत्य चाब्रुवन्}%॥ ८५ ॥

॥इति श्रीस्कान्दे महापुराण एकाशीतिसाहस्र्यां संहितायां तृतीये ब्रह्मखण्डे पूर्वभागे धर्मारण्यमाहात्म्ये दूतागमनं नामैकत्रिंशोऽध्यायः॥३१॥

\dnsub{द्वात्रिंशोऽध्यायः --- सत्यमन्दिरस्थापनम्}\resetShloka

\uvacha{व्यास उवाच}

\twolineshloka
{ततश्च रामदूतास्ते नत्वा राममथाब्रुवन्}
{रामराम महाबाहो वरनारी शुभानना}%॥ १ ॥

\twolineshloka
{सुवस्त्रभूषाभरणां मृदुवाक्यपरायणाम्}
{एकाकिनीं क्रदमानाम दृष्ट्वा तां विस्मिता वयम्}%॥ २ ॥

\twolineshloka
{समीपवर्तिनो भूत्वा पृष्टा सा सुरसुन्दरी}
{का त्वं देवि वरारोहे देवी वा दानवी नु किम्}%॥ ३ ॥

\twolineshloka
{रामः पृच्छति देवि त्वां ब्रूहि सर्वं यथातथम्}
{तच्छ्रुत्वा वचनं रामा सोवाच मधुरं वचः}%॥ ४

\onelineshloka
{रामं प्रेषयत भद्रं वो मम दुःखापहं परम्}%॥ ५ ॥

\threelineshloka
{तदाकर्ण्य ततो रामः सम्भ्रमात्त्वरितो ययौ}
{दृष्ट्वा तां दुःखसन्तप्तां स्वयं दुःखमवाप सः}
{उवाच वचनं रामः कृताञ्जलिपुटस्तदा}%॥ ६ ॥

\uvacha{श्रीराम उवाच}

\twolineshloka
{का त्वं शुभे कस्य परिग्रहो वा केनावधूता विजने निरस्ता}
{मुष्टं धनं केन च तावकीनमाचक्ष्व मातः सकलं ममाग्रे}%॥ ७ ॥

\twolineshloka
{इत्युक्त्वा चातिदुःखार्तो रामो मतिमतां वरः}
{प्रणामं दण्डवच्चक्रे चक्रपाणिरिवापरः} %॥ ८ ॥

\twolineshloka
{तयाभिवन्दितो रामः प्रगम्य च पुनःपुनः}
{तुष्टया परया प्रीत्या स्तुतो मधुरया गिरा}%॥ ९ ॥

\twolineshloka
{परमात्मन्परेशान दुःखहारिन्सनातन}
{यदर्थमवतारस्ते तच्च कार्यं त्वया कृतम्}%॥ १० ॥

\twolineshloka
{रावणः कुम्भकर्णश्च शक्रजित्प्रमुखास्तथा}
{खरदूषणत्रिशिरोमारीचाक्षकुमारकाः}%॥ ११

\onelineshloka
{असङ्ख्या निर्जिता रौद्रा राक्षसाः समराङ्गणे}%॥ १२ ॥

\twolineshloka
{किं वच्मि लोकेश सुकीर्त्तिमद्य ते वेधास्त्वदीयाङ्गजपद्मसम्भवः}
{विश्वं निविष्टं च ततो ददर्श वटस्य पत्रे हि यथो वटो मतः}%॥ १३ ॥

\twolineshloka
{धन्यो दशरथो लोके कौशल्या जननी तव}
{ययोर्जातोसि गोविन्द जगदीश परः पुमान्}%॥ १४ ॥

\twolineshloka
{धन्यं च तत्कुलं राम यत्र त्वमागतः स्वयम्}
{धन्याऽयोध्यापुरी राम धन्यो लोकस्त्वदाश्रयः}%॥ १५ ॥

\twolineshloka
{धन्यः सोऽपि हि वाल्मीकिर्येन रामायणं कृतम्}
{कविना विप्रमुख्येभ्य आत्मबुद्ध्या ह्यनागतम्}%॥ १६

\onelineshloka
{त्वत्तोऽभवत्कुलं चेदं त्वया देव सुपावितम्}%॥ १७ ॥

\twolineshloka
{नरपतिरिति लोकैः स्मर्यते वैष्णवांशः स्वयमसि रमणीयैस्त्वं गुणैर्विष्णुरेव}
{किमपि भुवनकार्यं यद्विचिन्त्यावतीर्य तदिह घटयतस्ते वत्स निर्विघ्नमस्तु}%॥ १८ ॥

\twolineshloka
{स्तुत्वा वाचाथ रामं हि त्वयि नाथे नु साम्प्रतम्}
{शून्या वर्ते चिरं कालं यथा दोषस्तथैव हि}%॥ १९ ॥

\twolineshloka
{धर्मारण्यस्य क्षेत्रस्य विद्धि मामधिदेवताम्}
{वर्षाणि द्वादशेहैव जातानि दुःखि तास्म्यहम्}%॥ २० ॥

\twolineshloka
{निर्जनत्वं ममाद्य त्वमुद्धरस्व महामते}
{लोहासुरभयाद्राम विप्राः सर्वे दिशो दश}%॥ २१ ॥

\twolineshloka
{गताश्च वणिजः सर्वे यथास्थानं सुदुःखिताः}
{स दैत्यो घातितो राम देवैः सुरभयङ्करः}%॥ २२ ॥

\twolineshloka
{आक्रम्यात्र महामायो दुराधर्षो दुरत्ययः}
{न ते जनाः समायान्ति तद्भयादति शङ्किताः}%॥ २३ ॥

\twolineshloka
{अद्य वै द्वादश समाः शून्यागारमनाथवत्}
{यस्माच्च दीर्घिकायां मे स्नानदानोद्यतो जनः}%॥ २४ ॥

\twolineshloka
{राम तस्यां दीर्घिकायां निपतन्ति च शूकराः}
{यत्राङ्गना भर्तृयुता जलक्रीडापरायणाः}%॥ २५ ॥

\twolineshloka
{चिक्रीडुस्तत्र महिषा निपतन्ति जलाशये}
{यत्र स्थाने सुपुष्पाणां प्रकरः प्रचुरोऽभवत्}%॥ २६ ॥

\twolineshloka
{तद्रुद्धं कण्टकैर्वृक्षैः सिंहव्याघ्रसमाकुलैः}
{सञ्चिक्रीडुः कुमाराश्च यस्यां भूमौ निरन्तरम्}%॥ २७ ॥

\twolineshloka
{कुमार्यश्चित्रकाणां च तत्र क्रीडं ति हर्षिताः}
{अकुर्वन्वाडवा यत्र वेदगानं तिरन्तरम्}%॥ २८ ॥

\twolineshloka
{शिवानां तत्र फेत्काराः श्रूयन्तेऽतिभयङ्कराः}
{यत्र धूमोऽग्निहोत्राणां दृश्यते वै गृहेगृहे}%॥ २९ ॥

\twolineshloka
{तत्र दावाः सधूमाश्च दृश्यन्तेऽत्युल्बणा भृशम्}
{नृत्यन्ते नर्त्तका यत्र हर्षिता हि द्विजाग्रतः}%॥ ३० ॥

\twolineshloka
{तत्रैव भूतवेताला प्रेताः नृत्यन्ति मोहिताः}
{नृपा यत्र सभायां तु न्यषीदन्मन्त्रतत्पराः}%॥ ३१ ॥

\twolineshloka
{तस्मिन्स्थाने निषीदन्ति गवया ऋक्षशल्लकाः}
{आवासा यत्र दृश्यन्ते द्विजानां वणिजां तथा}%॥ ३२ ॥

\twolineshloka
{कुट्टिमप्रतिमा राम दृश्यन्तेत्र बिलानि वै}
{कोटराणीह वृक्षाणां गवाक्षाणीह सर्वतः}%॥ ३३ ॥

\twolineshloka
{चतुष्का यज्ञवेदिर्हि सोच्छ्राया ह्यभवत्पुरा}
{तेऽत्र वल्मीकनिचयैर्दृश्यन्ते परिवेष्टिताः}%॥ ३४ ॥

\twolineshloka
{एवंविधं निवासं मे विद्धि राम नृपोत्तम}
{शून्यं तु सर्वतो यस्मान्निवासाय द्विजा गताः}%॥ ३५ ॥

\twolineshloka
{तेन मे सुमहद्दुःखं तस्मात्त्राहि नरेश्वर}
{एतच्छ्रुत्वा वचो राम उवाच वदतां वरः}%॥ ३६ ॥

\uvacha{श्रीराम उवाच}

\twolineshloka
{न जाने तावकान्विप्रांश्चतुर्दिक्षु समाश्रितान्}
{न तेषां वेद्म्यहं सङ्ख्यां नामगोत्रे द्विजन्मनाम्}%॥ ३७ ॥

\twolineshloka
{यथा ज्ञातिर्यथा गोत्रं याथातथ्यं निवेदय}
{तत आनीय तान्सर्वान्स्वस्थाने वासयाम्यहम्}%॥ ३८ ॥

\uvacha{श्रीमातोवाच}

\twolineshloka
{ब्रह्मविष्णुमहेशैश्च स्थापिता ये नरेश्वर}
{अष्टादश सहस्राणि ब्राह्मणा वेदपारगाः}%॥ ३९ ॥

\twolineshloka
{त्रयीविद्यासु विख्याता लोकेऽस्मिन्नमितद्युते}
{चतुष्षष्टिकगोत्राणां वाडवा ये प्रतिष्ठिताः}%॥ ४० ॥

\twolineshloka
{श्रीमातादात्त्रयीविद्यां लोके सर्वे द्विजोत्तमाः}
{षट्त्रिंशच्च सहस्राणि वैश्या धर्मपरायणाः}%॥ ४१ ॥

\twolineshloka
{आर्यवृत्तास्तु विज्ञेया द्विजशुश्रूषणे रताः}
{बहुलार्को नृपो यत्र संज्ञया सह राजते}%॥ ४२ ॥

\twolineshloka
{कुमारावश्विनौ देवौ धनदो व्ययपूरकः}
{अधिष्ठात्री त्वहं राम नाम्ना भट्टारिका स्मृता}%॥ ४३ ॥

\uvacha{श्रीसूत उवाच}

\twolineshloka
{स्थानाचाराश्च ये केचित्कुलाचारास्तथैव च}
{श्रीमात्रा कथितं सर्वं रामस्याग्रे पुरातनम्}%॥ ४४ ॥

\twolineshloka
{तस्यास्तु वचनं श्रुत्वा रामो मुदमवाप ह}
{सत्यंसत्यं पुनः सत्यं सत्यं हि भाषितं त्वया}%॥ ४५ ॥

\twolineshloka
{यस्मात्सत्यं त्वया प्रोक्तं तन्नाम्ना नगरं शुभम्}
{वासयामि जगन्मातः सत्यमन्दिरमेव च}%॥ ४६

\onelineshloka
{त्रैलोक्ये ख्यातिमाप्नोतु सत्यमन्दिरमु त्तमम्}%॥ ४७ ॥

\twolineshloka
{एतदुक्त्वा ततो रामः सहस्रशतसङ्ख्यया}
{स्वभृत्यान्प्रेषयामास विप्रानयनहेतवे}%॥ ४८ ॥

\twolineshloka
{यस्मिन्देशे प्रदेशे वा वने वा सरि तस्तटे}
{पर्यन्ते वा यथास्थाने ग्रामे वा तत्रतत्र च}%॥ ४९ ॥

\twolineshloka
{धर्मारण्यनिवासाश्च याता यत्र द्विजोत्तमाः}
{अर्घपाद्यैः पूजयित्वा शीघ्रमानयतात्र तान्}%॥ ५०

\onelineshloka
{अहमत्र तदा भोक्ष्ये यदा द्रक्ष्ये द्विजोत्तमान्}%॥ ५१ ॥

\twolineshloka
{विमान्य च द्विजानेतानागमिष्यति यो नरः}
{स मे वध्यश्च दण्ड्यश्च निर्वास्यो विषयाद्बहिः}%॥ ५२ ॥

\twolineshloka
{तच्छ्रुत्वा दारुणं वाक्यं दुःसहं दुःप्रधर्षणम्}
{रामाज्ञाकारिणो दूता गताः सर्वे दिशो दश}%॥ ५३ ॥

\twolineshloka
{शोधिता वाडवाः सर्वे लब्धाः सर्वे सुहर्षिताः}
{यथोक्तेन विधानेन अर्घपाद्यैरपूजयन्}%॥ ५४ ॥

\twolineshloka
{स्तुतिं चक्रुश्च विधिवद्विनयाचारपूर्वकम्}
{आमन्त्र्य च द्विजान्सर्वान्रामवाक्यं प्रकाशयन्}%॥ ५५ ॥

\twolineshloka
{ततस्ते वाडवाः सर्वे द्विजाः सेवकसंयुताः}
{गमनायोद्यताः सर्वे वेदशास्त्रपरायणाः}%॥ ५६ ॥

\twolineshloka
{आगता रामपार्श्वं च बहुमानपुरःसराः}
{समागतान्द्विजान्दृष्ट्वा रोमाञ्चिततनूरुहः}%॥ ५७ ॥

\twolineshloka
{कृतकृत्यमिवात्मानं मेने दाशरथिर्नृपः}
{स सम्भ्रमात्समुत्थाय पदातिः प्रययौ पुरः}%॥ ५८ ॥

\twolineshloka
{करसम्पुटकं कृत्वा हर्षाश्रु प्रतिमुञ्चयन्}
{जानुभ्यामवनिं गत्वा इदं वचनमब्रवीत्}%॥ ५९ ॥

\twolineshloka
{विप्रप्रसादात्कमलावरोऽहं विप्रप्रसादाद्धरणीधरोऽहम्}
{विप्रप्रसादाज्जगतीपतिश्च विप्रप्रसादान्मम रामनाम}%॥ ६० ॥

\twolineshloka
{इत्येवमुक्ता रामेण वाड वास्ते प्रहर्षिताः}
{जयाशीर्भिः प्रपूज्याथ दीर्घायुरिति चाब्रुवन्}%॥ ६१ ॥

\twolineshloka
{आवर्जितास्ते रामेण पाद्यार्घ्यविष्टरादिभिः}
{स्तुतिं चकार विप्राणां दण्डवत्प्रणिपत्य च}%॥ ६२ ॥

\twolineshloka
{कृताञ्जलिपुटः स्थित्वा चक्रे पादाभिवन्दनम्}
{आसनानि विचित्राणि हैमान्याभरणानि च}%॥ ६३ ॥

\twolineshloka
{समर्पयामास ततो रामो दशरथात्मजः}
{अङ्गुलीयकवासांसि उपवीतानि कर्णकान्}%॥ ६४ ॥

\twolineshloka
{प्रददौ विप्रमुख्येभ्यो नानावर्णाश्च धेनवः}
{एकैकशत सङ्ख्याका घटोध्नीश्च सवत्सकाः}%॥ ६५ ॥

\twolineshloka
{सवस्त्रा बद्धघण्टाश्च हेमशृङ्गविभूषिताः}
{रूप्यखुरास्ताम्रपृष्ठीः कांस्यपात्रसमन्विताः}%॥ ६६ ॥

॥इति श्रीस्कान्दे महापुराण एकाशीतिसाहस्र्यां संहितायां तृतीये ब्रह्मखण्डे पूर्वभागे धर्मारण्यमाहात्म्ये ब्रह्मनारदसंवादे सत्यमन्दिरस्थापन वर्णनोनाम द्वात्रिंशोऽध्यायः॥३२॥

\dnsub{त्रयस्त्रिंशोऽध्यायः --- श्रीरामचन्द्रस्य पुरप्रत्यागमनवर्णनम्}\resetShloka

\uvacha{राम उवाच}

\twolineshloka
{जीर्णोद्धारं करिष्यामि श्रीमातुर्वचनादहम्}
{आज्ञा प्रदीयतां मह्यं यथादानं ददामि वः}%॥ १ ॥

\twolineshloka
{पात्रे दानं प्रदातव्यं कृत्वा यज्ञवरं द्विजाः}
{नापात्रे दीयते किञ्चिद्दत्तं न तु सुखावहम्}%॥ २ ॥

\twolineshloka
{सुपात्रं नौरिव सदा तारयेदुभयोरपि}
{लोहपिण्डोपमं ज्ञेयं कुपात्रं भञ्जनात्मकम्}%॥ ३ ॥

\twolineshloka
{जातिमात्रेण विप्रत्वं जायते न हि भो द्विजाः}
{क्रिया बलवती लोके क्रियाहीने कुतः फलम्}%॥ ४ ॥

\twolineshloka
{पूज्यास्तस्मात्पूज्यतमा ब्राह्मणाः सत्यवादिनः}
{यज्ञकार्ये समुत्पन्ने कृपां कुर्वन्तु सर्वदा}%॥ ५ ॥

\uvacha{ब्रह्मोवाच}

\twolineshloka
{ततस्तु मिलिताः सर्वे विमृश्य च परस्परम्}
{केचिदूचुस्तदा रामं वयं शिलोञ्छजीविकाः}%॥ ६ ॥

\twolineshloka
{सन्तोषं परमास्थाय स्थिता धर्मपरायणाः}
{प्रतिग्रहप्रयोगेण न चास्माकं प्रयोजनम्}%॥ ७ ॥

\twolineshloka
{दशसूनासमश्चक्री दशचक्रिसमो ध्वजः}
{दशध्वजसमा वेश्या दशवेश्यासमो नृपः}%॥ ८ ॥

\twolineshloka
{राजप्रतिग्रहो घोरो राम सत्यं न संशयः}
{तस्माद्वयं न चेच्छामः प्रतिग्रहं भया वहम्}%॥ ९ ॥

\twolineshloka
{एकाहिका द्विजाः केचित्केचित्स्वामृतवृत्तयः}
{कुम्भीधान्या द्विजाः केचित्केचित्षट्कर्मतत्पराः}%॥ १० ॥

\twolineshloka
{त्रिमूर्तिस्थापिताः सर्वे पृथग्भावाः पृथग्गुणाः}
{केचिदेवं वदन्ति स्म त्रिमूर्त्याज्ञां विना वयम्}%॥ ११ ॥

\twolineshloka
{प्रतिग्रहस्य स्वीकारं कथं कुर्याम ह द्विजाः}
{न ताम्बूलं स्त्रीकृतं नो ह्यद्मो दानेन भषितम्}%॥ १२ ॥

\twolineshloka
{विमृश्य स तदा रामो वसिष्ठेन महात्मना}
{ब्रह्मविष्णुशिवादीनां सस्मार गुरुणा सह}

\twolineshloka
{स्मृतमात्रास्ततो देवास्तं देशं समुपागमन्}
{सूर्यकोटिप्रतीकाशीवमानावलिसंवृताः}%॥ १४}

\twolineshloka
{रामेण ते यथान्यायं पूजिताः परया मुदा}%॥
{निवेदितं तु तत्सर्वं रामेणातिसुबुद्धिना}%॥ १५ ॥

\twolineshloka
{अधिदेव्या वचनतो जीर्णोद्धारं करोम्यहम्}
{धर्मारण्ये हरिक्षेत्रे धर्मकूपसमीपतः}%॥ १६ ॥

\twolineshloka
{ततस्ते वाडवाः सर्वे त्रिमूर्त्तीः प्रणिपत्य च}
{महता हर्षवृन्देन पूर्णाः प्राप्तमनोरथाः}%॥ १७ ॥

\twolineshloka
{अर्घ्यपाद्यादिविधिना श्रद्धया तानपूजयन्}
{क्षणं विश्रम्य ते देवा ब्रह्मविष्णुशिवादयः}%॥ १८

\onelineshloka
{ऊचू रामं महाशक्तिं विनयात्कृतसम्पुटम्}%॥ १९ ॥

\uvacha{देवा ऊचुः}

\twolineshloka
{देवद्रुहस्त्वया राम ये हता रावणादयः}
{तेन तुष्टा वयं सर्वे भानुवंशविभूषण}%॥ २०

\onelineshloka
{उद्धरस्व महास्थानं महतीं कीर्तिमाप्नुहि}%॥ २१ ॥

\twolineshloka
{लब्ध्वा स तेषामाज्ञां तु प्रीतो दशरथात्मजः}
{जीर्णोद्धारेऽनन्तगुणं फलमिच्छन्निलापतिः}%॥ २२ ॥

\twolineshloka
{देवानां सन्निधौ तेषां कार्यारम्भमथाकरोत्}
{स्थण्डिलं पूर्वतः कृत्वा महागिरि समं शुभम्}%॥ २३ ॥

\twolineshloka
{तस्योपरि बहिःशाला गृहशाला ह्यनेकशः}
{ब्रह्मशालाश्च बहुशो निर्ममे शोभनाकृतीः}%॥ २४ ॥

\twolineshloka
{निधानैश्च समायुक्ता गृहोपकरणै र्वृताः}
{सुवर्णकोटिसम्पूर्णा रसवस्त्रादिपूरिताः}%॥ २५ ॥

\twolineshloka
{धनधान्यसमृद्धाश्च सर्वधातुयुतास्तथा}
{एतत्सर्वं कारयित्वा ब्राह्मणेभ्यस्तदा ददौ}%॥ २६ ॥

\twolineshloka
{एकैकशो दशदश ददौ धेनूः पयस्विनीः}
{चत्वारिंशच्छतं प्रादाद्ग्रामाणां चतुराधिकम्}%॥ २७ ॥

\twolineshloka
{त्रैविद्यद्विजविप्रेभ्यो रामो दशरथात्मजः}
{काजेशेन त्रयेणैव स्थापिता द्विजसत्तमाः}%॥ २८ ॥

\twolineshloka
{तस्मात्त्रयीविद्य इति ख्यातिर्लोके बभूव ह}
{एवंविधं द्विजेभ्यः स दत्त्वा दानं महाद्भुतम्}%॥ २९ ॥

\twolineshloka
{आत्मानं चापि मेने स कृतकृत्यं नरेश्वरः}
{ब्रह्मणा स्थापिताः पूर्वं विष्णुना शङ्करेण ये}%॥ ३० ॥

\twolineshloka
{ते पूजिता राघवेण जीर्णोद्धारे कृते सति}
{षट्त्रिंशच्च सहस्राणि गोभुजा ये वणिग्वराः}%॥ ३१ ॥

\twolineshloka
{शुश्रूषार्थं प्रदत्ता वै देवैर्हरिहरादिभिः}
{सन्तुष्टेन तु शर्वेण तेभ्यो दत्तं तु चेत नम्}%॥ ३२ ॥

\twolineshloka
{श्वेताश्वचामरौ दत्तौ खङ्गं दत्तं सुनिर्मलम्}
{तदा प्रबोधितास्ते च द्विजशुश्रूषणाय वै}%॥ ३३ ॥

\twolineshloka
{विवाहादौ सदा भाव्यं चामरै मङ्गलं वरम्}
{खङ्गं शुभं तदा धार्य्यं मम चिह्नं करे स्थितम्}%॥ ३४ ॥

\twolineshloka
{गुरुपूजा सदा कार्या कुलदेव्याः पुनःपुनः}
{वृद्ध्यागमेषु प्राप्तेषु वृद्धि दायकदक्षिणा}%॥ ३५ ॥

\twolineshloka
{एकादश्यां शनेर्वारे दानं देयं द्विजन्मने}
{प्रदेयं बालवृद्धेभ्यो मम रामस्य शासनात्}%॥ ३६ ॥

\twolineshloka
{मण्डलेषु च ये शुद्धा वणिग्वृत्तिरताः पराः}
{सपादलक्षास्ते दत्ता रामशासनपालकाः}%॥ ३७ ॥

\twolineshloka
{माण्डलीकास्तु ते ज्ञेया राजानो मण्डलेश्वराः}
{द्विज शुश्रूषणे दत्ता रामेण वणिजां वराः}%॥ ३८ ॥

\twolineshloka
{चामरद्वितयं रामो दत्तवान्खड्गमेव च}
{कुलस्य स्वामिनं सूर्यं प्रतिष्ठाविधिपूर्वकम्}%॥ ३९ ॥

\twolineshloka
{ब्रह्माणं स्थापयामास चतुर्वेदसमन्वितम्}
{श्रीमातरं महाशक्तिं शून्यस्वामिहरिं तथा}%॥ ४० ॥

\twolineshloka
{विघ्नापध्वंसनार्थाय दक्षिणद्वारसंस्थितम्}
{गणं संस्थापयामास तथान्याश्चैव देवताः}%॥ ४१ ॥

\twolineshloka
{कारितास्तेन वीरेण प्रासादाः सप्तभूमिकाः}
{यत्किं चित्कुरुते कार्यं शुभं माङ्गल्यरूपकम्}%॥ ४२ ॥

\twolineshloka
{पुत्रे जाते जातके वान्नाशने मुण्डनेऽपि वा}
{लक्षहोमे कोटिहोमे तथा यज्ञक्रियासु च}%॥ ४३ ॥

\twolineshloka
{वास्तुपूजाग्रहशान्त्योः प्राप्ते चैव महोत्सवे}
{यत्किञ्चित्कुरुते दानं द्रव्यं वा धान्यमुत्तमम्}%॥ ४४ ॥

\twolineshloka
{वस्त्रं वा धेनवो नाथ हेम रूप्यं तथैव च}
{विप्राणामथ शूद्राणां दीनानाथान्धकेषु च}%॥ ४५ ॥

\twolineshloka
{प्रथमं बकुलार्कस्य श्रीमातुश्चैव मानवः}
{भागं दद्याच्च निर्विघ्नकार्यसिद्ध्यै निरन्तरम्}%॥ ४६ ॥

\twolineshloka
{वचनं मे समुल्लङ्घ्य कुरुते योऽन्यथा नरः}
{तस्य तत्कर्मणो विघ्नं भविष्यति न संशयः}%॥ ४७ ॥

\twolineshloka
{एवमुक्त्वा ततो रामः प्रहृष्टेनान्तरात्मना}
{देवानामथ वापीश्च प्राकारांस्तु सुशोभनान्}%॥ ४८ ॥

\twolineshloka
{दुर्गोपकरणैर्युक्तान्प्रतोलीश्च सुविस्तृताः}
{निर्ममे चैव कुण्डानि सरांसि सरसीस्तथा}%॥ ४९ ॥

\twolineshloka
{धर्मवापीश्च कूपांश्च तथान्यान्देवनिर्मितान्}
{एतत्सर्वं च विस्तार्य धर्मारण्ये मनोरमे}%॥ ५० ॥

\twolineshloka
{ददौ त्रैविद्यमुख्येभ्यः श्रद्धया परया पुनः}
{ताम्रपट्टस्थितं रामशासनं लोपयेत्तु यः}%॥ ५१ ॥

\twolineshloka
{पूर्वजास्तस्य नरके पतन्त्यग्रे न सन्ततिः}
{वायुपुत्रं समाहूय ततो रामोऽब्रवीद्वचः}%॥ ५२ ॥

\twolineshloka
{वायुपुत्र महावीर तव पूजा भविष्यति}
{अस्य क्षेत्रस्य रक्षायै त्वमत्र स्थितिमाचर}%॥ ५३ ॥

\twolineshloka
{आञ्जनेयस्तु तद्वाक्यं प्रणम्य शिरसादधौ}
{जीर्णोद्धारं तदा कृत्वा कृतकृत्यो बभूव ह}%॥ ५४ ॥

\twolineshloka
{श्रीमातरं तदाभ्यर्च्य प्रसन्नेनान्तरात्मना}
{श्रीमातरं नमस्कृत्य तीर्थान्यन्यानि राघवः}%॥ ५५

\onelineshloka
{तेऽपि देवाः स्वकं स्थानं ययुर्बह्मपुरोगमाः}%॥ ५६ ॥

\twolineshloka
{दत्त्वाशिषं तु रामाय वाञ्छितं ते भविष्यति}
{रम्यं कृतं त्वया राम विप्राणां स्थापनादिकम्}%॥ ५७ ॥

\twolineshloka
{अस्माकमपि वात्सल्यं कृतं पुण्यवता त्वया}
{इति स्तुवन्तस्ते देवाः स्वानि स्थानानि भेजिरे}%॥ ५८ ॥
॥इति श्रीस्कान्दे महापुराण एकाशीतिसाहस्र्यां संहितायां तृतीये ब्रह्मखण्डे पूर्वार्धे धर्मारण्यमाहात्म्ये श्रीरामचन्द्रस्य पुरप्रत्यागमनवर्णनं नाम त्रयस्त्रिंशोऽध्यायः॥३३॥

\dnsub{चतुस्त्रिंशोऽध्यायः --- श्रीरामेण ब्राह्मणेभ्यः शासनपट्टप्रदानवर्णनम्}\resetShloka

\uvacha{व्यास उवाच}

\twolineshloka
{एवं रामेण धर्मज्ञ जीर्णोद्धारः पुरा कृतः}
{द्विजानां च हितार्थाय श्रीमातुर्वचनेन च}%॥ १ ॥

\uvacha{युधिष्ठिर उवाच}

\twolineshloka
{कीदृशं शासनं ब्रह्मन्रामेण लिखितं पुरा}
{कथयस्व प्रसादेन त्रेतायां सत्यमन्दिरे}%॥ २ ॥

\uvacha{व्यास उवाच}

\twolineshloka
{धर्मारण्ये वरे दिव्ये बकुलार्के स्वधिष्ठिते}
{शून्यस्वामिनि विप्रेन्द्र स्थिते नारायणे प्रभौ}%॥ ३ ॥

\twolineshloka
{रक्षणाधिपतौ देवे सर्वज्ञे गुणनायके}
{भवसागर मग्नानां तारिणी यत्र योगिनी}%॥ ४ ॥

\twolineshloka
{शासनं तत्र रामस्य राघवस्य च नामतः}
{शृणु ताम्राश्रयं तत्र लिखितं धर्मशास्त्रतः}%॥ ५ ॥

\twolineshloka
{महाश्चर्यकरं तच्च ह्यनेकयुगसंस्थितम्}
{सर्वो धातुः क्षयं याति सुवर्णं क्षयमेति च}%॥ ६ ॥

\twolineshloka
{प्रत्यक्षं दृश्यते पुत्र द्विजशासनमक्षयम्}
{अविनाशो हि ताम्रस्य कारणं तत्र विद्यते}%॥ ७ ॥

\twolineshloka
{वेदोक्तं सकलं यस्माद्विष्णुरेव हि कथ्यते}
{पुराणेषु च वेदेषु धर्मशास्त्रेषु भारत}%॥ ८ ॥

\twolineshloka
{सर्वत्र गीयते विष्णुर्नाना भावसमाश्रयः}
{नानादेशेषु धर्मेषु नानाधर्मनिषेविभिः}%॥ ९ ॥

\twolineshloka
{नानाभेदैस्तु सर्वत्र विष्णुरेवेति चिन्त्यते}
{अवतीर्णः स वै साक्षात्पुराणपुरुषो त्तमः}%॥ १० ॥

\twolineshloka
{देववैरिविनाशाय धर्मसंरक्षणाय च}
{तेनेदं शासनं दत्तमविनाशात्मकं सुत}%॥ ११ ॥

\twolineshloka
{यस्य प्रतापादृषद(य)स्तारिता जलमध्यतः}
{वानरैर्वेष्टिता लङ्का हेलया राक्षसा हताः}%॥ १२ ॥

\twolineshloka
{मुनिपुत्रं मृतं रामो यमलोकादुपानयत्}
{दुन्दुभिर्निहतो येन कबन्धोऽभिहतस्तथा}%॥ १३ ॥

\twolineshloka
{निहता ताडका चैव सप्तताला विभेदिताः}
{खरश्च दूषणश्चैव त्रिशिराश्च महासुरः}%॥ १४ ॥

\twolineshloka
{चतुर्दशसहस्राणि जवेन निहता रणे}
{तेनेदं शासनं दत्तमक्षयं न कथं भवेत्}%॥ १५ ॥

\twolineshloka
{स्ववंशवर्णनं तत्र लिखित्वा स्वयमेव तु}
{देशकालादिकं सर्वं लिलेख विधिपूर्वकम्}%॥ १६ ॥

\twolineshloka
{स्वमुद्राचिह्नितं तत्र त्रैविद्येभ्यस्तथा ददौ}
{चतुश्चत्वारिंशवर्षो रामो दशरथात्मजः}%॥ १७ ॥

\twolineshloka
{तस्मिन्काले महाश्चर्यं सन्दत्तं किल भारत}
{तत्र स्वर्णोपमं चापि रौप्योपमम थापि च}%॥ १८ ॥

\twolineshloka
{उवाह सलिलं तीर्थे देवर्षिपितृतृप्तिदम्}
{स्ववंशनायकस्याग्रे सूर्येण कृतमेव तत्}%॥ १९ ॥

\twolineshloka
{तद्दृष्ट्वा महदाश्चर्यं रामो विष्णुं प्रपूज्य च}
{रामलेखविचित्रैस्तु लिखितं धर्मशासनम्}%॥ २० ॥

\twolineshloka
{यद्दृष्ट्वाथ द्विजाः सर्वे संसारभयबन्धनम्}
{कुर्वते नैव यस्माच्च तस्मान्निखिलरक्षकम्}%॥ २१ ॥

\twolineshloka
{ये पापिष्ठा दुराचारा मित्रद्रोहरताश्च ये}
{तेषां प्रबोधनार्थाय प्रसिद्धिमकरोत्पुरा}%॥ २२ ॥

\twolineshloka
{रामलेखविचित्रैस्तु विचित्रे ताम्रपट्टके}
{वाक्यानीमानि श्रूयन्ते शासने किल नारद}%॥ २३ ॥

\twolineshloka
{आस्फोटयन्ति पितरः कथयन्ति पितामहाः}
{भूमिदोऽस्मत्कुले जातः सोऽस्मान्सन्तारयिष्यति}%॥ २४ ॥

\twolineshloka
{बहुभिर्बहुधा भुक्ता राजभिः पृथिवी त्वियम्}
{यस्ययस्य यदा भूमिस्तस्यतस्य तदा फलम्}%॥ २५ ॥

\twolineshloka
{षष्टिवर्षसहस्राणि स्वर्गे वसति भूमिदः}
{आच्छेत्ता चानुमन्ता च तान्येव नरकं व्रजेत्}%॥ २६ ॥

\twolineshloka
{सन्दंशैस्तुद्यमानस्तु मुद्गरैर्विनिहत्य च}
{पाशैः सुबध्यमानस्तु रोरवीति महास्वरम्}%॥ २७ ॥

\twolineshloka
{ताड्यमानः शिरे दण्डैः समालिङ्ग्य विभावसुम्}
{क्षुरिकया छिद्यमानो रोरवीति महास्वनम्}%॥ २८ ॥

\twolineshloka
{यमदूतैर्महाघोरैर्ब्रह्मवृत्तिविलोपकः}
{एवंविधैर्महादुष्टैः पीड्यन्ते ते महागणैः}%॥ २९ ॥

\twolineshloka
{ततस्तिर्यक्त्वमाप्नोति योनिं वा राक्षसीं शुनीम्}
{व्यालीं शृगालीं पैशाचीं महाभूतभयङ्करीम्}%॥ ३० ॥

\twolineshloka
{भूमेरङ्गुलहर्ता हि स कथं पापमाचरेत्}
{भूमेरङ्गुलदाता च स कथं पुण्यमाचरेत्}%॥ ३१ ॥

\twolineshloka
{अश्वमेधसहस्राणां राजसूयशतस्य च}
{कन्याशतप्रदानस्य फलं प्राप्नोति भूमिदः}%॥ ३२ ॥

\twolineshloka
{आयुर्यशः सुखं प्रज्ञा धर्मो धान्यं धनं जयः}
{सन्तानं वर्द्धते नित्यं भूमिदः सुखमश्मुते}%॥ ३३ ॥

\threelineshloka
{भूमेरङ्गुलमेकं तु ये हरन्ति खला नराः}
{वन्ध्याटवीष्वतोयासु शुष्ककोटरवासिनः}
{कृष्णसर्पाः प्रजायन्ते दत्तदायापहारकाः}%॥ ३४ ॥

\twolineshloka
{तडागानां सहस्रेण अश्वमेधशतेन वा}
{गवां कोटिप्रदानेन भूमिहर्त्ता विशुध्यति}%॥ ३५ ॥

\twolineshloka
{यानीह दत्तानि पुनर्धनानि दानानि धर्मार्थयशस्कराणि}
{औदार्यतो विप्रनिवेदितानि को नाम साधुः पुनराददीत}%॥ ३६ ॥

\twolineshloka
{चलदलदललीलाचञ्चले जीवलोके तृणलवलघुसारे सर्वसंसारसौख्ये}
{अपहरति दुराशः शासनं ब्राह्मणानां नरकगहनगर्त्तावर्तपातोत्सुको यः}%॥ ३७ ॥

\twolineshloka
{ये पास्यन्ति महीभुजः क्षितिमिमां यास्यन्ति भुक्त्वाखिलां नो याता न तु याति यास्यति न वा केनापि सार्द्धं धरा}
{यत्किञ्चिद्भुवि तद्विनाशि सकलं कीर्तिः परं स्थायिनी त्वेवं वै वसुधापि यैरुपकृता लोप्या न सत्कीर्तयः}%॥ ३८ ॥

\twolineshloka
{एकैव भगिनी लोके सर्वेषामेव भूभुजाम्}
{न भोज्या न करग्राह्या विप्रदत्ता वसुन्धरा}%॥ ३९ ॥

\twolineshloka
{दत्त्वा भूमिं भाविनः पार्थिवेशान्भूयोभूयो याचते रामचन्द्रः}
{सामान्योऽयं धर्मसेतुर्नृपाणां स्वे स्वे काले पालनीयो भवद्भिः}%॥ ४० ॥

\twolineshloka
{अस्मिन्वंशे क्षितौ कोपि राजा यदि भविष्यति}
{तस्याहं करलग्नोस्मि मद्दत्तं यदि पाल्यते}%॥ ४१ ॥

\twolineshloka
{लिखित्वा शासनं रामश्चातुर्वेद्यद्विजोत्तमान्}
{सम्पूज्य प्रददौ धीमान्वसिष्ठस्य च सन्निधौ}%॥ ४२ ॥

\twolineshloka
{ते वाडवा गृहीत्वा तं पट्टं रामाज्ञया शुभम्}
{ताम्रं हैमाक्षरयुतं धर्म्यं धर्मविभूषणम्}%॥ ४३ ॥

\twolineshloka
{पूजार्थं भक्तिकामार्थास्तद्रक्षणमकुर्वत}
{चन्दनेन च दिव्येन पुष्पेण च सुगन्धिना}%॥ ४४ ॥

\twolineshloka
{तथा सुवर्णपुष्पेण रूप्यपुष्पेण वा पुनः}
{अहन्यहनि पूजां ते कुर्वते वाडवाः शुभाम्}%॥ ४५ ॥

\twolineshloka
{तदग्रे दीपकं चैव घृतेन विमलेन हि}
{सप्तवर्तियुतं राजन्नर्घ्यं प्रकुर्वते द्विजाः}%॥ ४६ ॥

\twolineshloka
{नैवेद्यं कुर्वते नित्यं भक्तिपूर्वं द्विजोत्तमाः}
{रामरामेति रामेति मन्त्रमप्युच्चरन्ति हि}%॥ ४७ ॥

\twolineshloka
{अशने शयने पाने गमने चोपवेशने}
{सुखे वाप्यथवा दुःखे राममन्त्रं समुच्चरेत्}%॥ ४८ ॥

\twolineshloka
{न तस्य दुःखदौर्भाग्यं नाधिव्याधिभयं भवेत्}
{आयुः श्रियं बलं तस्य वर्द्धयन्ति दिने दिने}%॥ ४९ ॥

\twolineshloka
{रामेति नाम्ना मुच्येत पापाद्वै दारुणादपि}
{नरकं नहि गच्छेत गतिं प्राप्नोति शाश्वतीम्}%॥ ५० ॥

\uvacha{व्यास उवाच}

\twolineshloka
{इति कृत्वा ततो रामः कृतकृत्यममन्यत}
{प्रदक्षिणीकृत्य तदा प्रणम्य च द्विजान्बहून्}%॥ ५१ ॥

\twolineshloka
{दत्त्वा दानं भूरितरं गवाश्वमहिषीरथम्}
{ततः सर्वान्निजांस्तांश्च वाक्यमेतदुवाच ह}%॥ ५२ ॥

\twolineshloka
{अत्रैव स्थीयतां सर्वैर्यावच्चन्द्रदिवाकरौ}
{यावन्मेरुर्महीपृष्ठे सागराः सप्त एव च}%॥ ५३ ॥

\twolineshloka
{तावदत्रैव स्थातव्यं भवद्भिर्हि न संशयः}
{यदा हि शासनं विप्रा न मन्यन्ते नृपा भुवि}%॥ ५४ ॥

\twolineshloka
{अथवा वणिजः शूरा मदमायाविमोहिताः}
{मदाज्ञां न प्रकुर्वन्ति मन्यन्ते वा न ते जनाः}%॥ ५५ ॥

\twolineshloka
{तदा वै वायुपुत्रस्य स्मरणं क्रियतां द्विजाः}
{स्मृतमात्रो हनूमान्वै समागत्य करिष्यति}%॥ ५६ ॥

\twolineshloka
{सहसा भस्म तान्सत्यं वचनान्मे न संशयः}
{य इदं शासनं रम्यं पालयिष्यति भूपतिः}%॥ ५७ ॥

\twolineshloka
{वायुपुत्रः सदा तस्य सौख्यमृद्धिं प्रदास्यति}
{ददाति पुत्रान्पौत्रांश्च साध्वीं पत्नीं यशो जयम्}%॥ ५८ ॥

\twolineshloka
{इत्येवं कथयित्वा च हनुमन्तं प्रबोध्य च}
{निवर्तितो रामदेवः ससैन्यः सपरिच्छदः}%॥ ५९ ॥

\threelineshloka
{वादित्राणां स्वनैर्विष्वक्सूच्यमानशुभागमः}
{श्वेतातपत्रयुक्तोऽसौ चामरैर्वी जितो नरैः}
{अयोध्यां नगरीं प्राप्य चिरं राज्यं चकार ह}%॥ ६० ॥
॥इति श्रीस्कान्दे महापुराण एकाशीतिसाहस्र्यां संहितायां तृतीये ब्रह्मखण्डे पूर्वभागे धर्मारण्यमाहात्म्ये ब्रह्मनारदसंवादे श्रीरामेण ब्राह्मणेभ्यः शासनपट्टप्रदानवर्णनं नाम चतुस्त्रिंशोऽध्यायः॥३४॥

\dnsub{पञ्चत्रिंशोऽध्यायः --- श्रीरामरुद्रकृतधर्मारण्यतीर्थक्षेत्रजीर्णोद्धारवर्णनम्}\resetShloka

\uvacha{नारद उवाच}

\twolineshloka
{भगवन्देवदेवेश सृष्टिसंहारकारक}
{गुणातीतो गुणैर्युक्तो मुक्तीनां साधनं परम्}%॥ १ ॥

\twolineshloka
{संस्थाप्य वेदभवनं विधिवद्द्विज सत्तमान्}
{किं चक्रे रघुनाथस्तु भूयोऽयोध्यां गतस्तदा}%॥ २ ॥

\uvacha{ब्रह्मोवाच}

\twolineshloka
{स्वस्थाने ब्राह्मणास्तत्र कानि कर्माणि चक्रिरे}
{इष्टापूर्तरताः शान्ताः प्रतिग्रहपराङ्मुखाः}%॥ ३ ॥

\twolineshloka
{राज्यं चक्रुर्वनस्यास्य पुरोधा द्विजसत्तमः}
{उवाच रामपुरतस्तीर्थमाहात्म्यमुत्तमम्}%॥ ४ ॥

\twolineshloka
{प्रयागस्य च माहात्म्यं त्रिवेणीफलमुत्तमम्}
{प्रयागतीर्थमहिमा शुक्लतीर्थस्य चैव हि}%॥ ५ ॥

\twolineshloka
{सिद्धक्षेत्रस्य काश्याश्च गङ्गाया महिमा तथा}
{वसिष्ठः कथया मास तीर्थान्यन्यानि नारद}%॥ ६ ॥

\twolineshloka
{धर्मारण्यसुवर्णाया हरिक्षेत्रस्य तस्य च}
{स्नानदानादिकं सर्वं वाराणस्या यवाधिकम्}%॥ ७ ॥

\twolineshloka
{एतच्छ्रुत्वा रामदेवः स चमत्कृतमानसः}
{धर्मारण्ये पुनर्यात्रां कर्त्तुकामः समभ्यगात्}%॥ ८ ॥

\twolineshloka
{सीतया सह धर्मज्ञो गुरुसैन्यपुरःसरः}
{लक्ष्मणेन सह भ्रात्रा भरतेन सहायवान्}%॥ ९ ॥

\twolineshloka
{शत्रुघ्नेन परिवृतो गतो मोहेरके पुरे}
{तत्र गत्वा वसिष्ठं तु पृच्छतेऽसौ महामनाः}%॥ १० ॥

\uvacha{राम उवाच}

\twolineshloka
{धर्मारण्ये महाक्षेत्रे किं कर्त्तव्यं द्विजोत्तम}
{दानं वा नियमो वाथ स्नानं वा तप उत्तमम्}%॥ ११ ॥

\twolineshloka
{ध्यानं वाथ क्रतुं वाथ होमं वा जपमुत्तमम्}
{दानं वा नियमं वाथ स्नानं वा तप उत्तमम्}%॥ १२ ॥

\twolineshloka
{येन वै क्रियमाणेन तीर्थेऽस्मिन्द्विजसत्तम}
{ब्रह्महत्यादिपापेभ्यो मुच्यते तद्ब्रवीहि मे}%॥ १३ ॥

\uvacha{वसिष्ठ उवाच}

\twolineshloka
{यज्ञं कुरु महाभाग धर्मारण्ये त्वमुत्तमम्}
{दिनेदिने कोटिगुणं यावद्वर्षशतं भवेत्}%॥ १४ ॥

\twolineshloka
{तच्छ्रुत्वा चैव गुरुतो यज्ञारम्भं चकार सः}
{तस्मिन्नवसरे सीता रामं व्यज्ञापयन्मुदा}%॥ १५ ॥

\twolineshloka
{स्वामिन्पूर्वं त्वया विप्रा वृता ये वेदपारगाः}
{ब्रह्मविष्णुमहेशेन निर्मिता ये पुरा द्विजाः}%॥ १६ ॥

\twolineshloka
{कृते त्रेतायुगे चैव धर्मारण्यनिवासिनः}
{विप्रांस्तान्वै वृणुष्व त्वं तैरेव साधकोऽध्वरः}%॥ १७ ॥

\twolineshloka
{तच्छ्रुत्वा रामदेवेन आहूता ब्राह्मणास्तदा}
{स्थापिताश्च यथापूर्वमस्मिन्मोहे रके पुरे}%॥ १८ ॥

\twolineshloka
{तैस्त्वष्टादशसङ्ख्याकैस्त्रैविद्यैर्मेहिवाडवैः}
{यज्ञं चकार विधिवत्तैरेवायतबुद्धिभिः}%॥ १९ ॥

\twolineshloka
{कुशिकः कौशिको वत्स उपमन्युश्च काश्यपः}
{कृष्णात्रेयो भरद्वाजो धारिणः शौनको वरः}%॥ २० ॥

\twolineshloka
{माण्डव्यो भार्गवः पैङ्ग्यो वात्स्यो लौगाक्ष एव च}
{गाङ्गायनोथ गाङ्गेयः शुनकः शौनकस्तथा}%॥ २१ ॥

\uvacha{ब्रह्मोवाच}

\twolineshloka
{एभिर्विप्रैः क्रतुं रामः समाप्य विधिवन्नृपः}
{चकारावभृथं रामो विप्रान्सम्पूज्य भक्तितः}%॥ २२ ॥

\twolineshloka
{यज्ञान्ते सीतया रामो विज्ञप्तः सुविनीतया}
{अस्याध्वरस्य सम्पत्ती दक्षिणां देहि सुव्रत}%॥ २३ ॥

\twolineshloka
{मन्नाम्ना च पुरं तत्र स्थाप्यतां शीघ्रमेव च}
{सीताया वचनं श्रुत्वा तथा चक्रे नृपोत्तमः}%॥ २४ ॥

\twolineshloka
{तेषां च ब्राह्मणानां च स्थानमेकं सुनिर्भयम्}
{दत्तं रामेण सीतायाः सन्तोषाय महीभृता}%॥ २५ ॥

\twolineshloka
{सीतापुरमिति ख्यातं नाम चक्रे तदा किल}
{तस्याधिदेव्यौ वर्त्तेते शान्ता चैव सुमङ्गला}%॥ २६ ॥

\twolineshloka
{मोहेरकस्य पुरतो ग्रामद्वादशकं पुरः}
{ददौ विप्राय विदुषे समुत्थाय प्रहर्षितः}%॥ २७ ॥

\twolineshloka
{तीर्थान्तरं जगामाशु काश्यपीसरितस्तटे}
{वाडवाः केऽपि नीतास्ते रामेण सह धर्मवित्}%॥ २८ ॥

\twolineshloka
{धर्मालये गतः सद्यो यत्र माला कमण्डलुः}
{पुरा धर्मेण सुमहत्कृतं यत्र तपो मुने}%॥ २९ ॥

\twolineshloka
{तदारभ्य सुविख्यातं धर्मालयमिति श्रुतम्}
{ददौ दाशरथिस्तत्र महादानानि षोडश}%॥ ३० ॥

\twolineshloka
{पञ्चाशत्तदा ग्रामाः सीतापुरसमन्विताः}
{सत्यमन्दिरपर्यन्ता रघुना थेन वै तदा}%॥ ३१ ॥

\twolineshloka
{सीताया वचनात्तत्र गुरुवाक्येन चैव हि}
{आत्मनो वंशवृद्ध्यर्थं द्विजेभ्योऽदाद्रघूत्तमः}%॥ ३२ ॥

\twolineshloka
{अष्टादशसहस्राणां द्विजानामभवत्कुलम्}
{वात्स्यायन उपमन्युर्जातूकर्ण्योऽथ पिङ्गलः}%॥ ३३ ॥

\twolineshloka
{भारद्वाजस्तथा वत्सः कौशिकः कुश एव च}
{शाण्डिल्यः कश्यपश्चैव गौतमश्छान्धनस्तथा}%॥ ३४ ॥

\twolineshloka
{कृष्णात्रेयस्तथा वत्सो वसिष्ठो धारणस्तथा}
{भाण्डिलश्चैव विज्ञेयो यौवनाश्वस्ततः परम्}%॥ ३५ ॥

\twolineshloka
{कृष्णायनोपमन्यू च गार्ग्यमुद्गलमौखकाः}
{पुशिः पराशरश्चैव कौण्डिन्यश्च ततः परम्}%॥ ३६ ॥

\twolineshloka
{पञ्चपञ्चाशद्ग्रामाणां नामान्येवं यथाक्रमम्}
{सीतापुरं श्रीक्षेत्रं च मुशली मुद्गली तथा}%॥ ३७ ॥

\twolineshloka
{ज्येष्ठला श्रेयस्थानं च दन्ताली वटपत्रका}
{राज्ञः पुरं कृष्णवाटं देहं लोहं चनस्थनम्}%॥ ३८ ॥

\twolineshloka
{कोहेचं चन्दनक्षेत्रं थलं च हस्तिनापुरम्}
{कर्पटं कन्नजह्नवी वनोडफनफावली}%॥ ३९ ॥

\twolineshloka
{मोहोधं शमोहोरली गोविन्दणं थलत्यजम्}
{चारणसिद्धं सोद्गीत्राभाज्यजं वटमालिका}%॥ ४० ॥

\twolineshloka
{गोधरं मारणजं चैव मात्रमध्यं च मातरम्}
{बलवती गन्धवती ईआम्ली च राज्यजम्}%॥ ४१ ॥

\twolineshloka
{रूपावली बहुधनं छत्रीटं वंशञ्जं तथा}
{जायासंरणं गोतिकी च चित्रलेखं तथैव च}%॥ ४२ ॥

\twolineshloka
{दुग्धावली हंसावली च वैहोलं चैल्लजं तथा}
{नालावली आसावली सुहाली कामतः परम्}%॥ ४३ ॥

\twolineshloka
{रामेण पञ्चपञ्चाशद्ग्रामाणि वसनाय च}
{स्वयं निर्माय दत्तानि द्विजेभ्यस्तेभ्य एव च}%॥ ४४ ॥

\twolineshloka
{तेषां शुश्रूषणार्थाय वैश्यान्रामो न्यवे दयत्}
{षट्त्रिंशच्च सहस्राणि शूद्रास्तेभ्यश्चतुर्गुणान्}%॥ ४५ ॥

\twolineshloka
{तेभ्यो दत्तानि दानानि गवाश्ववसनानि च}
{हिरण्यं रजतं ताम्रं श्रद्धया परया मुदा}%॥ ४६ ॥

\uvacha{नारद उवाच}

\threelineshloka
{अष्टादशसहस्रास्ते ब्राह्मणा वेदपारगाः}
{कथं ते व्यभजन्ग्रामान्द्रामोत्पन्नं तथा वसु}
{वस्त्राद्यं भूषणाद्यं च तन्मे कथय सुव्र तम्}%॥ ४७ ॥

\uvacha{ब्रह्मोवाच}

\twolineshloka
{यज्ञान्ते दक्षिणा यावत्सर्त्विग्भिः स्वीकृता सुत}
{महादानादिकं सर्वं तेभ्य एव समर्पितम्}%॥ ४८ ॥

\twolineshloka
{ग्रामाः साधारणा दत्ता महास्थानानि वै तदा}
{ये वसन्ति च यत्रैव तानि तेषां भवन्त्विति}%॥ ४९ ॥

\twolineshloka
{वशिष्ठवचनात्तत्र ग्रामास्ते विप्रसात्कृताः}
{रघूद्वहेन धीरेण नोद्व सन्ति यथा द्विजाः}%॥ ५० ॥

\twolineshloka
{धान्यं तेषां प्रदत्तं हि विप्राणां चामितं वसु}
{कृताञ्जलिस्ततो रामो ब्राह्मणानिदमब्रवीत्}%॥ ५१ ॥

\twolineshloka
{यथा कृतयुगे विप्रास्त्रेतायां च यथा पुरा}
{तथा चाद्यैव वर्त्तव्यं मम राज्ये न संशयः}%॥ ५२ ॥

\twolineshloka
{यत्किञ्चिद्धनधान्यं वा यानं वा वसनानि वा}
{मणयः काञ्चनादींश्च हेमादींश्च तथा वसु}%॥ ५३ ॥

\twolineshloka
{ताम्राद्यं रजतादींश्च प्रार्थयध्वं ममाधुना}
{अधुना वा भविष्ये वाभ्यर्थनीयं यथोचितम्}%॥ ५४ ॥

\twolineshloka
{प्रेषणीयं वाचिकं मे सर्वदा द्विजसत्तमाः}
{यंयं कामं प्रार्थयध्वं तं तं दास्याम्यहं विभो}%॥ ५५ ॥

\twolineshloka
{ततो रामः सेवकादीनादरात्प्रत्यभाषत}
{विप्राज्ञा नोल्लङ्घनीया सेव नीया प्रयत्नतः}%॥ ५६ ॥

\twolineshloka
{यंयं कामं प्रार्थयन्ते कारयध्वं ततस्ततः}
{एवं नत्वा च विप्राणां सेवनं कुरुते तु यः}%॥ ५७ ॥

\twolineshloka
{स शूद्रः स्वर्गमाप्नोति धनवान्पुत्रवान्भवेत्}
{अन्यथा निर्धनत्वं हि लभते नात्र संशयः}%॥ ५८ ॥

\twolineshloka
{यवनो म्लेच्छजातीयो दैत्यो वा राक्षसोपि वा}
{योत्र विघ्नं करोत्येव भस्मीभवति तत्क्षणात्}%॥ ५९ ॥

\uvacha{ब्रह्मोवाच}

\twolineshloka
{ततः प्रदक्षिणीकृत्य द्विजान्रामोऽतिहर्षितः}
{प्रस्थानाभिमुखो विप्रैराशीर्भिरभिनन्दितः}%॥ ६० ॥

\twolineshloka
{आसीमान्तमनुव्रज्य स्नेहव्याकुललोचनाः}
{द्विजाः सर्वे विनिर्वृत्ता धर्मारण्ये विमोहिताः}%॥ ६१ ॥

\twolineshloka
{एवं कृत्वा ततो रामः प्रतस्थे स्वां पुरीं प्रति}
{काश्यपाश्चैव गर्गाश्च कृतकृत्या दृढव्रताः}%॥ ६२ ॥

\twolineshloka
{गुर्वासनसमाविष्टाः सभार्या ससुहृत्सुताः}
{राजधानीं तदा प्राप रामोऽयोध्यां गुणान्विताम्}%॥ ६३ ॥

\twolineshloka
{दृष्ट्वा प्रमुदिताः सर्वे लोकाः श्रीरघुनन्दनम्}
{ततो रामः स धर्मात्मा प्रजापालनतत्परः}%॥ ६४ ॥

\twolineshloka
{सीतया सह धर्मात्मा राज्यं कुर्वंस्तदा सुधीः}
{जानक्या गर्भमाधत्त रविवंशोद्भवाय च}%॥ ६५ ॥

॥इति श्रीस्कान्दे महापुराण एकाशीतिसाहस्र्यां संहितायां तृतीये ब्रह्मखण्डे पूर्वभागे धर्मारण्यमाहात्म्ये श्रीरामरुद्रकृतधर्मारण्यतीर्थक्षेत्रजीर्णोद्धारवर्णनं नाम पञ्चत्रिंशोऽध्यायः॥३५॥