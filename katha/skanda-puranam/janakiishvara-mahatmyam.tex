\sect{जानकीश्वरमाहात्म्यवर्णनम्}

\src{स्कन्दपुराणम्}{खण्डः ७ (प्रभासखण्डः)}{प्रभासक्षेत्र माहात्म्यम्}{अध्यायः ११३}
\vakta{}
\shrota{}
\tags{}
\notes{}
\textlink{https://sa.wikisource.org/wiki/स्कन्दपुराणम्/खण्डः_७_(प्रभासखण्डः)/प्रभासक्षेत्र_माहात्म्यम्/अध्यायः_११३}
\translink{https://www.wisdomlib.org/hinduism/book/the-skanda-purana/d/doc626901.html}

\storymeta




\uvacha{ईश्वर उवाच}

\twolineshloka
{ततो गच्छेन्महादेवि जानकीश्वरमुत्तमम्}
{रामेशान्नैऋते भागे धनुस्त्रिंशकसंस्थितम्}%॥ १ ॥

\twolineshloka
{पापघ्नं सर्वजन्तूनां जानक्याऽऽराधितं पुरा}
{प्रतिष्ठितं विशेषेण सम्यगाराध्यशङ्करम्}%॥ २ ॥

\twolineshloka
{पूर्वं तस्यैव लिङ्गस्य वसिष्ठेशेति नाम वै}
{तत्पश्चाज्जानकीशेति त्रेतायां प्रथितं क्षितौ}%॥ ३ ॥

\twolineshloka
{ततः षष्टिसहस्राणि वालखिल्या महर्षयः}
{तत्र सिद्धिमनुप्राप्तास्तेन सिद्धेश्वरेति च}%॥ ४ ॥

\twolineshloka
{ख्यातं कलौ महादेवि युगलिङ्गं महाप्रभम्}
{तद्दृष्ट्वा मुच्यते पापैर्दुःखदौर्भाग्यसम्भवैः}%॥ ५ ॥

\twolineshloka
{यस्तं पूजयते भक्त्या नारी वा पुरुषोऽपि वा}
{संस्नाप्य विधिवद्भक्त्या स मुक्तः पातकैर्भवेत्}%॥ ६ ॥

\twolineshloka
{स्नात्वा च पुष्करे तीर्थे यस्तल्लिगं प्रपूजयेत्}
{नियतो नियताहारो मासमेकं निरन्तरम्}%॥ ७ ॥

\threelineshloka
{दिनेदिने भवेत्तस्य वाजिमेधाधिकं फलम्}
{माघे मासि तृतीयायां या नारी तं प्रपूजयेत्}
{तदन्वयेऽपि दौर्भाग्यं दुःखं शोकश्च नो भवेत्}%॥ ८ ॥

\twolineshloka
{इति ते कथितं देवि माहात्म्यं पापनाशनम्}
{श्रुतं हरति पापानि सौभाग्यं सम्प्रयच्छति}%॥ ९ ॥

॥इति श्रीस्कान्दे महापुराण एकाशीतिसाहस्र्यां संहितायां सप्तमे प्रभासखण्डे प्रथमे प्रभासक्षेत्रमाहात्म्ये जानकीश्वरमाहात्म्यवर्णनं नाम त्रयोदशोत्तरशततमोऽध्यायः॥११३॥