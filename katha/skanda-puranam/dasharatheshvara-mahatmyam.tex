\sect{एकसप्तत्युत्तरशततमोऽध्यायः --- दशरथेश्वरमाहात्म्यवर्णनम्}

\src{स्कन्दपुराणम्}{खण्डः ७ (प्रभासखण्डः)}{प्रभासक्षेत्र माहात्म्यम्}{अध्यायः १७१}
\vakta{}
\shrota{}
\tags{}
\notes{}
\textlink{https://sa.wikisource.org/wiki/स्कन्दपुराणम्/खण्डः_७_(प्रभासखण्डः)/प्रभासक्षेत्र_माहात्म्यम्/अध्यायः_१७१}
\translink{https://www.wisdomlib.org/hinduism/book/the-skanda-purana/d/doc626959.html}

\storymeta




\uvacha{ईश्वर उवाच}

\twolineshloka
{ततो गच्छेन्महादेवि देवीमेकल्लवीरिकाम्}
{एकल्लवीरायाम्ये तु नातिदूरे व्यवस्थिताम्}%॥ १ ॥

\twolineshloka
{पूर्वं दशरथो योऽसौ सूर्यवंशविभूषणः}
{प्रभासं क्षेत्रमासाद्य तपश्चक्रे सुदुश्चरम्}%॥ २ ॥

\twolineshloka
{लिङ्गं तत्र प्रतिष्ठाप्य तोषयामास शाङ्करम्}
{स देवं प्रार्थयामास पुत्रं चैवामितौजसम्}%॥ ३ ॥

\twolineshloka
{ददौ तस्य तदा पुत्रं देवं त्रैलोक्यपूजितम्}
{रामेति नाम यस्यासीत्त्रैलोक्ये प्रथितं यशः}%॥ ४ ॥

\twolineshloka
{यस्याद्यापीह गायन्ति भूर्भुवःस्वर्नि वासिनः}
{देवदैत्यासुराः सर्वे वाल्मीक्याद्या महर्षयः}%॥ ५ ॥

\threelineshloka
{तल्लिङ्गस्य प्रभावेन प्राप्तं राज्ञा महद्यशः}
{कार्तिक्यां कार्तिके मासि विधिना यस्तमर्चयेत्}
{दीपपूजोपहारेण यशस्वी सोऽपि जायते}%॥ ६ ॥

॥इति श्रीस्कान्दे महापुराण एकाशीतिसाहस्र्यां संहितायां सप्तमे प्रभासखण्डे प्रथमे प्रभासक्षेत्रमाहात्म्ये दशरथेश्वरमाहात्म्यवर्णनं नामैकसप्तत्युत्तरशततमोऽध्यायः॥१७१॥