\sect{एकोनाशीतितमोऽध्यायः --- हनुमत्केश्वरमाहात्म्यवर्णनम्}

\src{स्कन्दपुराणम्}{खण्डः ५ (अवन्तीखण्डः)}{अवन्तीस्थचतुरशीतिलिङ्गमाहात्म्यम्}{अध्यायः ७९}
\vakta{}
\shrota{}
\tags{}
\notes{}
\textlink{https://sa.wikisource.org/wiki/स्कन्दपुराणम्/खण्डः_५_(अवन्तीखण्डः)/अवन्तीस्थचतुरशीतिलिङ्गमाहात्म्यम्/अध्यायः_७९}
\translink{https://www.wisdomlib.org/hinduism/book/the-skanda-purana/d/doc425723.html}

\storymeta




\uvacha{श्रीमहादेव उवाच}

\twolineshloka
{एकोनाशीतिकं विद्धि हनुमत्केश्वरं प्रिये}
{यस्य दर्शनमात्रेण समीहितफलं लभेत्}%॥ १ ॥

\twolineshloka
{प्राप्तराज्यस्य रामस्य राक्षसानां वधे कृते}
{आगता मुनयो देवि राघवं प्रतिनन्दितुम्}%॥ २ ॥

\twolineshloka
{रामेण पूजिताः सर्वे ह्यगस्तिप्रमुखा द्विजाः}
{प्रहृष्टमनसो विप्रा रामं वचनमब्रुवन्}%॥ ३ ॥

\twolineshloka
{दिष्ट्या तु निहतो राम रावणः पुत्रपौत्रवान्}
{दिष्ट्या विजयिनं त्वाऽद्य पश्यामः सह भार्यया}%॥ ४ ॥

\twolineshloka
{हनूमता च सहितं वानरेण महात्मना}
{दिष्ट्या पवनपुत्रेण राक्षसान्तकरेण च}%॥ ५ ॥

\twolineshloka
{चिरं जीवतु दीर्घायुर्वानरो हनुमान्सदा}
{अञ्जनीगर्भसम्भूतो रुद्रांशो हि धरातले}%॥ ६ ॥

\threelineshloka
{आखण्डलोऽग्निर्भगवान्यमो वै निऋतिस्तथा}
{वरुणः पवनश्चैव धनाध्यक्षस्तथा शिवः}
{ब्रह्मणा सहिताश्चैव दिक्पालाः पातु सर्वदा}%॥ ७ ॥

\twolineshloka
{श्रुत्वा तेषां तु वचनं मुनीनां भावितात्मनाम्}
{विस्मयं परमं गत्वा रामः प्राञ्जलिरब्रवीत्}%॥ ८ ॥

\twolineshloka
{किमर्थं लक्ष्मणं त्यक्त्वा वानरोऽयं प्रशंसितः}
{कीदृशः किम्प्रभावो वा किंवीर्यः किम्पराक्रमः}%॥ ९ ॥

\twolineshloka
{अथोचुः सत्यमेवैतत्कारणं वानरोत्तमे}
{न त्वस्य सदृशो वीर्ये विद्यते भुवनत्रये}%॥ १० ॥

\twolineshloka
{एष देव महाप्राज्ञो योजनानां शतं प्लुतः}
{धर्षयित्वा पुरीं लङ्कां रावणान्तःपुरं गतः}%॥ ११ ॥

\twolineshloka
{प्रादेशमात्रप्रतिमं कृतं रूपमनेन वै}
{दृष्टा सम्भाषिता सीता पृष्टा विश्वासिता तथा}%॥ १२ ॥

\twolineshloka
{सेनाग्रगा मन्त्रिपुत्राः किङ्करा रावणात्मजाः}
{हता हनुमता तत्र ताडिता रावणालये}%॥ १३ ॥

\twolineshloka
{भूयो बन्धविमुक्तेन सम्भाष्य तु दशाननम्}
{लङ्का भस्मीकृता तेन पातकेनेव मेदिनी}%॥ १४ ॥

\twolineshloka
{न कालस्य न शक्रस्य न विष्णोर्वेधसोऽपि वा}
{श्रूयन्ते तानि कर्माणि यादृशानि हनूमतः}%॥ १५ ॥

\uvacha{राम उवाच}

\twolineshloka
{एतस्य बाहुवीर्येण लङ्का सीता च लक्ष्मणः}
{प्राप्तो मम जयश्चैव राज्यं मित्राणि बान्धवाः}%॥ १६ ॥

\twolineshloka
{सखायं वानरपतिर्मुक्त्वैनं हरिपुङ्गवम्}
{प्रवृत्तिमपि को वेत्तुं जानक्याः शक्तिमान्भवेत्}%॥ १७ ॥

\twolineshloka
{वाली किमर्थमेतेन सुग्रीवप्रियकाम्यया}
{तदा वैरे समुत्पन्ने न दग्धस्तृणवत्कथम्}%॥ १८ ॥

\twolineshloka
{नायं विदितवान्मन्ये हनुमानात्मनो बलम्}
{उपेक्षितः क्लिश्यिमाने किमर्थं वानराधिपे}%॥ १९ ॥

\twolineshloka
{एवं ब्रुवाणं रामं तु मुनयो वाक्यमब्रुवन्}
{सत्यमेतद्रघुश्रेष्ठ यद्ब्रवीषि हनूमतः}%॥ २० ॥

\twolineshloka
{न बले विद्यते तुल्यो न गतौ न मतावपि}
{अमोघवाक्यैः शापस्तु दत्तोऽस्य मुनिभिः पुरा}%॥ २१ ॥

\twolineshloka
{न ज्ञातं हि बलं येन बलिना वालिमर्दने}
{बाल्येऽप्यनेन यत्कर्म कृतं नाम महात्मना}%॥ २२ ॥

\twolineshloka
{तन्न वर्णयितुं शक्यमेतस्य तु बलं महत्}
{यदि श्रोतुं तवेच्छास्ति निशामय वदामहे}%॥ २३ ॥

\twolineshloka
{असौ हि जातमात्रोऽपि बालार्क इव मूर्त्तिमान्}
{ग्रहीतुकामो बालार्कं पुप्लावाम्बरमध्यतः}%॥ २४ ॥

\twolineshloka
{तूर्णमाधावतो राम शक्रेण विदितात्मना}
{हनुस्तेनास्य सहसा कुलिशेनैव ताडितः}%॥ २५ ॥

\threelineshloka
{ततो गिरौ पपातैष शक्रवज्राभिताडितः}
{पततोस्य महावेगाद्वामो हनुरभज्यत}
{अस्मिंस्तु पतिते बाले मृतकल्पेऽशनिक्षतात्}%॥ २६ ॥

\twolineshloka
{ततो वायुः समादाय महा कालवनं गतः}
{लिङ्गमाराधयामास पुत्रार्थं पवनस्तदा}%॥ २७ ॥

\twolineshloka
{स्पृष्टमात्रस्तु लिङ्गेन समुत्तस्थौ प्लवङ्गमः}
{जलसिक्तं यथा सस्यं पुनर्जीवि तमाप्तवान्}%॥ २८ ॥

\twolineshloka
{प्राणवन्तमिमं दृष्ट्वा पवनो हर्षितस्तदा}
{प्रत्युवाच प्रसन्नात्मा पुत्रमादाय सत्वरम्}%॥ २९ ॥

\twolineshloka
{स्पर्शनादस्य लिङ्गस्य मम पुत्रः समुत्थितः}
{हनुमत्केश्वरो देवो विख्यातोऽयं भविष्यति}%॥ ३० ॥

\twolineshloka
{एतस्मिन्नन्तरे शक्रः समायातः सुरैर्वृतः}
{नीलोत्पलमयीं मालां सम्प्रगृह्येदमब्रवीत्}%॥ ३१ ॥

\twolineshloka
{मत्करोत्सृष्टवज्रेण यस्मादस्यहनुर्हतः}
{तदेष कपिशार्दूलो हनुमांस्तु भविष्यति}%॥ ३२ ॥

\twolineshloka
{वरुणोऽस्व वरं प्रादान्नास्य मृत्युर्भविष्यति}
{यमो दण्डादवध्यत्वमारोग्यं धनदो ददौ}%॥ ३३ ॥

\twolineshloka
{सूर्येण च प्रभा दत्ता पवनेन गतिर्द्रुता}
{लिङ्गेन च वरो दत्तो देवानां सन्निधौ तदा}%॥ ३४ ॥

\twolineshloka
{आयुधानां हि सर्वेषामवध्योऽयं भविष्यति}
{अजरश्चामरश्चैव भविष्यति न संशयः}%॥ ३५ ॥

\twolineshloka
{अमित्रभयदो ह्येष मित्राणामभय प्रदः}
{अजेयो भविता युद्धे लिङ्गेनोक्तं पुनःपुनः}%॥ ३६ ॥

\twolineshloka
{शत्रोर्बलोत्सादनाय राघवप्रीतये सदा}
{कियत्कालं बलं स्वीयं न स्मरिष्यति शापतः}%॥ ३७ ॥

\twolineshloka
{हते तु रावणे भूयो रामस्यानुमते स्थितः}
{विभीषणं प्रार्थयित्वा मामत्र स्थापयिष्यति}%॥ ३८ ॥

\twolineshloka
{ततो मां त्रिदशाः सर्वे पूजयिष्यन्ति भाविताः}
{तेनैव नाम्ना विख्यातिं पुनर्यास्यामि भूतले}%॥ ३९ ॥

\twolineshloka
{अथ गन्धवहः पुत्रं प्रगृह्य गृहमानयत्}
{अञ्जनायै तदाचख्यौ वरलब्धिं च लिङ्गतः}%॥ ४० ॥

\twolineshloka
{एवं लिङ्गप्रभावाच्च बलवान्मारुतात्मजः}
{स जातस्त्रिषुलोकेषु राम तस्मात्प्रशस्यते}%॥ ४१ ॥

\twolineshloka
{पराक्रमोत्साहमति प्रतापैः सौशील्यमाधुर्यनयादिकैश्च}
{गाम्भीर्यचातुर्यसुवीर्यधैर्यैर्हनूमतः कोऽभ्यधिकोऽस्ति लोके}%॥ ४२ ॥

\twolineshloka
{ममेव विक्षोभितसागरस्य लोकान्दि धक्षोरिव पावकस्य}
{प्रजा जिहीर्षोरिव चातकस्य हनूमतः स्थास्यति कः पुरस्तात्}%॥ ४३ ॥

\twolineshloka
{एतद्वै कथित तुभ्यं यन्मां त्वं परि पृच्छसि}
{हनूमतोऽस्य बालस्य कर्माण्यद्भुतविक्रम}%॥ ४४ ॥

\threelineshloka
{दृष्टः सभाजितश्चापि राम गच्छामहे वयम्}
{एवमुक्त्वा गताः सर्वे मुनयोऽवन्तिमण्डलम्}
{पूजयामासुरीशानं हनुमत्केश्वरं शिवम्}

\twolineshloka
{समर्चयन्ति ये भक्त्या लिङ्गं त्रिदशपूजितम्}
{हनुमत्केश्वरं देवं ते कृतार्थाः कलौ युगे}%॥ ४६ ॥

\twolineshloka
{व्रजन्त्येव सुदुष्प्राप्यं ब्रह्मसायुज्यमव्ययम्}
{सम्प्राप्य तु पुनर्जन्म लभन्ते मोक्षमव्ययम्}%॥ ४७ ॥

\twolineshloka
{यः पश्यति नरो लिङ्गं हनुमत्केश्वरॆ प्रिय}
{सोऽधिकं फलमाप्नोति सर्वदुःखविवर्जितः}%॥ ४८ ॥

\twolineshloka
{सर्वलोकेषु तस्यैव गतिर्न प्रतिहन्यते}
{दिव्येनैश्वर्ययोगेन युज्यते नात्र संशयः}%॥ ४९ ॥

\twolineshloka
{बालसूर्यप्रतीकाशविमानेन सुवर्चसा}
{वृतः स्त्रीणां सहस्रैस्तु स्वच्छदगमनागमः}%॥ ५० ॥

\twolineshloka
{विचरत्यविचारेण सर्वलोकान्दिवौकसाम्}
{स्पृहणीयतमः पुंसां सर्ववर्णोत्तमोधुना}%॥ ५१ ॥

\twolineshloka
{स्वर्गाच्च्युतः प्रजायेत कुले महति रूपवान्}
{धर्मज्ञो रुद्रभक्तश्च सर्वविद्यार्थपारगः}%॥ ५२ ॥

\twolineshloka
{राजा वा राजतुल्यो वा दर्शनादस्य जायते}
{स्पर्शनात्परमं पुण्यं यजनात्परमं पदम्}%॥ ५३ ॥

\twolineshloka
{एष ते कथितो देवि प्रभावः पापनाशनः}
{हनुमत्केश्वरेशस्य स्वप्नेश्वरमथो शृणु}%॥ ५४ ॥

॥इति श्रीस्कान्दे महापुराण एकाशीतिसाहस्र्यां संहितायां पञ्चम आवन्त्यखण्डे चतुरशीतिलिङ्गमाहात्म्य उमामहेश्वरसंवादे हनुमत्केश्वरमाहात्म्यवर्णनं नामैकोनाशीतितमोऽध्यायः॥७९॥