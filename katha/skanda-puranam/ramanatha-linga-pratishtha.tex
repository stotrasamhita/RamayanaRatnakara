\sect{रामनाथलिङ्गप्रतिष्ठाविधिवर्णनम्}

\src{स्कन्दपुराणम्}{खण्डः ३ (ब्रह्मखण्डः)}{सेतुखण्डः}{अध्यायः ४४}
\vakta{}
\shrota{}
\tags{}
\notes{Notably includes a short stotram by the Rishis, praising the glory of the Bhagavan Rāma.}
\textlink{https://sa.wikisource.org/wiki/स्कन्दपुराणम्/खण्डः_३_(ब्रह्मखण्डः)/सेतुखण्डः/अध्यायः_४४}
\translink{https://www.wisdomlib.org/hinduism/book/the-skanda-purana/d/doc423612.html}

\storymeta




\uvacha{ऋषय ऊचुः}

\twolineshloka
{सर्ववेदार्थतत्त्वज्ञ पुराणार्णवपारग}
{व्यासपादाम्बुजद्वन्द्वनमस्कारहृताशुभ}%॥ १ ॥

\twolineshloka
{पुराणार्थोपदेशेन सर्वप्राण्युपका रक}
{त्वया ह्यनुगृहीताः स्म पुराणकथनाद्वयम्}%॥ २ ॥

\twolineshloka
{अधुना सेतुमाहात्म्यकथनात्सुतरां मुने}
{वयं कृतार्थाः सञ्जाता व्यासशिष्य महामते}%॥ ३ ॥

\twolineshloka
{यथा प्रातिष्ठिपल्लिङ्गं रामो दशरथात्मजः}
{तच्छ्रोतुं वयमिच्छामस्त्वमिदानीं वदस्व नः}%॥ ४ ॥

\uvacha{श्रीसूत उवाच}

\twolineshloka
{यदर्थं स्थापितं लिङ्गं गन्धमादनपर्वते}
{रामचन्द्रेण विप्रेन्द्र तदिदानीं ब्रवीमि वः}%॥ ५ ॥

\twolineshloka
{हृतभार्यो वनाद्रामो रावणेन बलीयसा}
{कपिसेनायुतो धीरः ससौमि त्रिर्महाबलः}%॥६॥

\twolineshloka
{महेन्द्रं गिरिमासाद्य व्यलोकयत वारिधिम्}
{तस्मिन्नपारे जलधौ कृत्वा सेतुं रघूद्वहः}%॥७॥

\twolineshloka
{तेन गत्वा पुरीं लङ्कां रावणेनाभिरक्षि ताम्}
{अस्तङ्गते सहस्रांशौ पौर्णमास्यां निशामुखे}%॥८॥

\twolineshloka
{रामः ससैनिको विप्राः सुवेलगिरिमारुहत्}
{ततः सौधस्थितं रात्रौ दृष्ट्वा लङ्केश्वरं बली}%॥ ९ ॥

\twolineshloka
{सूर्यपुत्रोऽस्य मुकुटं पातयास भूतले}
{राक्षसो भग्नमुकुटः प्रविवेश गृहोदरम्}%॥ १० ॥

\twolineshloka
{गृहं प्रविष्टे लङ्केशे रामः सुग्रीवसंयुतः}
{सानुजः सेनया सार्द्धमवरुह्य गिरेस्तटात्}%॥ ११ ॥

\twolineshloka
{सेनां न्यवेशयद्वीरो रामो लङ्कासमीपतः}
{ततो निवेशमानांस्तान्वानरान्रावणानुगाः}%॥ १२ ॥

\twolineshloka
{अभिजग्मुर्महाकायाः सायुधाः सहसैनिकाः}
{पर्वणः पूतनो जृम्भः खरः क्रोधवशो हरिः}%॥ १३ ॥

\twolineshloka
{प्रारुजश्चारुजश्चैव प्रहस्तश्चेतरे तथा}
{ततोऽभिपततां तेषामदृश्यानां दुरात्मनाम्}%॥ १४ ॥

\twolineshloka
{अन्तर्धानवधं तत्र चकार स्म विभीषणः}
{ते दृश्यमाना बलिभिर्हरिभिर्दूरपातिभिः}%॥ १५ ॥

\twolineshloka
{निहताः सर्वतश्चैते न्यपतन्वै गतासवः}
{अमृष्यमाणः सबलो रावणो निर्ययावथ}%॥ १६ ॥

\twolineshloka
{व्यूह्य तान्वानरान्सर्वान्न्यवारयत सायकैः}
{राघवस्त्वथ निर्याय व्यूढानीको दशाननम्}%॥ १७ ॥

\twolineshloka
{प्रत्ययुध्यत वेगेन द्वन्द्वयुद्धमभूत्तदा}
{युयुधे लक्ष्मणेनाथ इन्द्रजिद्रावणात्मजः}%॥ १८ ॥

\twolineshloka
{विरूपाक्षेण सुग्रीवस्तारेयेणापि खर्वटः}
{पौण्ड्रेण च नलस्तत्र पुटेशः पनसेन च}%॥ १९ ॥

\twolineshloka
{अन्येपि कपयो वीरा राक्षसैर्द्वन्द्वमेत्य तु}
{चक्रुर्युद्धं सुतुमुलं भीरूणां भयवर्द्धनम्}%॥ २० ॥

\twolineshloka
{अथ रक्षांसि भिन्नानि वानरैर्भीमविक्रमैः}
{प्रदुद्रुवू रणादाशु लङ्कां रावणपालिताम्}%॥ २१ ॥

\twolineshloka
{भग्नेषु सर्वसैन्येषु रावणप्रेरितेन वै}
{पुत्रेणेन्द्रजिता युद्धे नागास्त्रैरतिदारुणैः}%॥ २२ ॥

\twolineshloka
{विद्धौ दाशरथी विप्रा उभौ तौ रामलक्ष्मणौ}
{मोचितौ वैनतेयेन गरुडेन महात्मना}%॥ २३ ॥

\twolineshloka
{तत्र प्रहस्तस्तरसा समभ्येत्य विभीषणम्}
{गदया ताडयामास विनद्य रणकर्कशः}%॥ २४ ॥

\twolineshloka
{स तयाभिहतो धीमान्गदया भामिवेगया}
{नाकम्पत महाबाहुर्हिमवानिव सुस्थितः}%॥ २५ ॥

\twolineshloka
{ततः प्रगृह्य विपुलामष्टघण्टां विभीषणः}
{अभिमन्त्र्य महाशक्तिं चिक्षे पास्य शिरः प्रति}%॥ २६ ॥

\twolineshloka
{पतन्त्या स तया वेगाद्राक्षसोऽशनिना यथा}
{हृतोत्तमाङ्गो ददृशे वातरुग्ण इव द्रुमः}%॥ २७ ॥

\twolineshloka
{तं दृष्ट्वा निहतं सङ्ख्ये प्रहस्तं क्षणदाचरम्}
{अभिदुद्राव धूम्राक्षो वेगेन महता कपीन्}%॥ २८ ॥

\twolineshloka
{कपिसैन्यं समालोक्य विद्रुतं पवनात्मजः}
{धूम्राक्षमाजघानाशु शरेण रणमूर्धनि}%॥ २९ ॥

\twolineshloka
{धूम्राक्षं निहतं दृष्ट्वा हतशेषा निशाचराः}
{सर्वं राज्ञे यथावृत्तं रावणाय न्यवेदयन्}%॥ ३० ॥

\twolineshloka
{ततः शयानं लङ्केशः कुम्भकर्णमबोधयत्}
{प्रबुद्धं प्रेषयामास युद्धाय स च रावणः}%॥ ३१ ॥

\twolineshloka
{आगतं कुम्भकर्णं तं ब्रह्मास्त्रेण तु लक्ष्मणः}
{जघान समरे क्रुद्धो गतासुर्न्यपतच्च सः}%॥ ३२ ॥

\twolineshloka
{दूषणस्यानुजौ तत्र वत्रवेगप्रमाथिनौ}
{हनुमन्नीलनिहतौ रावणप्रतिमौ रणे}%॥ ३३ ॥

\twolineshloka
{वज्रदंष्ट्रं समवधीद्विश्वकर्मसुतो नलः}
{अकम्पनं च न्यहनत्कुमुदो वानरर्षभः}%॥ ३४ ॥

\twolineshloka
{षष्ठ्यां पराजितो राजा प्राविशच्च पुरीं ततः}
{अतिकायो लक्ष्मणेन हतश्च त्रिशिरास्तथा}%॥ ३५ ॥

\twolineshloka
{सुग्रीवेण हतौ युद्धे देवान्त कनरान्तकौ}
{हनूमता हतौ युद्धे कुम्भकर्णसुतावुभौ}%॥ ३६ ॥

\twolineshloka
{विभीषणेन निहतो मकराक्षः खरात्मजः}
{तत इन्द्रजितं पुत्रं चोदयामास रावणः}%॥ ३७ ॥

\twolineshloka
{इन्द्रजिन्मोहयित्वा तौ भ्रातरौ रामलक्षमणौ}
{घोरैः शरैरङ्गदेन हतवाहो दिवि स्थितः}%॥ ३८ ॥

\twolineshloka
{कुमुदाङ्गदसुग्रीवनलजाम्बवदादिभिः}
{सहिता वानराः सर्वे न्यपतंस्तेन घातिताः}%॥ ३९ ॥

\twolineshloka
{एवं निहत्य समरे ससैन्यौ रामलक्ष्मणौ}
{अन्तर्दधे तदा व्योम्नि मेघनादो महाबलः}%॥ ४० ॥

\twolineshloka
{ततो विभीषणो राममिक्ष्वाकुकुलभूषणम्}
{उवाच प्राञ्जलिर्वाक्यं प्रणम्य च पुनःपुनः}%॥ ४१ ॥

\twolineshloka
{अयमम्भो गृहीत्वा तु राजराजस्य शासनात्}
{गुह्यकोऽभ्यागतो राम त्वत्सकाशमरिन्दम}%॥ ४२ ॥

\twolineshloka
{इदमम्भः कुबेरस्ते महाराज प्रयच्छति}
{अन्तर्हितानां भूतानां दर्शनार्थं परं तप}%॥ ४३ ॥

\twolineshloka
{अनेन स्पृष्टनयनो भूतान्यन्तर्हितान्यपि}
{भवान्द्रक्ष्यति यस्मै वा भवानेतत्प्रदास्यति}%॥ ४४ ॥

\twolineshloka
{सोऽपि द्रक्ष्यति भूतानि वियत्त्यन्तर्हितानि वै}
{तथेति रामस्तद्वारि प्रतिगृह्याथ सत्कृतम्}%॥ ४५ ॥

\twolineshloka
{चकार नेत्रयोः शौचं लक्ष्मणश्च महाबलः}
{सुग्रीवजाम्बवन्तौ च हनुमानङ्गदस्तथा}%॥ ४६ ॥

\twolineshloka
{मैन्दद्विविदनीलाश्च ये चान्ये वानरास्तथा}
{ते सर्वे रामदत्तेन वारिणा शुद्धचक्षुषः}%॥ ४७ ॥

\twolineshloka
{आकाशेन्तर्हितं वीरमपश्यन्रावणा त्मजम्}
{ततस्तमभिदुद्राव सौमित्रिर्दृष्टिगोचरम्}%॥ ४८ ॥

\twolineshloka
{ततो जघान सङ्कुद्धो लक्ष्मणः कृतलक्षणः}
{कुवेरप्रेषितजलैः पवित्रीकृतलोचनः}%॥ ४९ ॥

\twolineshloka
{ततः समभवद्युद्धं लक्ष्मणेन्द्रजितोर्महत्}
{अतीव चित्रमाश्चर्यं शक्रप्रह्लादयोरिव}%॥ ५० ॥

\twolineshloka
{ततस्तृतीयदिवसे यत्नेन महता द्विजाः}
{इन्द्रजिन्निहतो युद्धे लक्ष्मणेन बलीयसा}%॥ ५१ ॥

\twolineshloka
{ततो मूलबलं सर्वं हतं रामेण धीमता}
{अथ क्रुद्धो दशग्रीवः प्रियपुत्रे निपातिते}%॥ ५२ ॥

\twolineshloka
{निर्ययौ रथमास्थाय नगराद्बहुसैनिकः}
{रावणो जानकीं हन्तुमुद्युक्तो विन्ध्यवारितः}%॥ ५३ ॥

\twolineshloka
{ततो हर्यश्वयुक्तेन रथेनादित्यवर्चसा}
{उपतस्थे रणे रामं मातलिः शक्रसारथिः}%॥ ५४ ॥

\twolineshloka
{ऐन्द्रं रथं समारुह्य रामो धर्मभृतां वरः}
{शिरांसि राक्षसेन्द्रस्य ब्रह्मास्त्रेणावधीद्रणे}%॥ ५५ ॥

\twolineshloka
{ततो हतदशग्रीवं रामं दशरथात्मजम्}
{आशीर्भिर्जययुक्ताभिर्देवाः सर्षिपुरोगमाः}%॥ ५६ ॥

\twolineshloka
{तुष्टुवुः परिसन्तुष्टाः सिद्धविद्याधरास्तथा}
{रामं कमलपत्राक्षं पुष्प वर्षेरवाकिरन्}%॥ ५७ ॥

\twolineshloka
{रामस्तैः सुरसङ्घातैः सहितः सैनिकैर्वृतः}
{सीतासौमित्रिसहितः समारुह्य च पुष्पकम्}%॥ ५८ ॥

\twolineshloka
{तथाभिषिच्य राजानं लङ्कायां च विभीषणम्}
{कपिसेनावृतो रामो गन्धमादनमन्वगात्}%॥ ५९ ॥

\twolineshloka
{परिशोध्य च वैदेहीं गन्धमादनपर्वते}
{रामं कमलपत्राक्षं स्थितवानर संवृतम्}%॥ ६० ॥

\twolineshloka
{हतलङ्केश्वरं वीरं सानुजं सविभीषणम्}
{सभार्यं देववृन्दैश्च सेवितं मुनिपुङ्गवैः}%॥ ६१ ॥

\twolineshloka
{मुनयोऽभ्यागता द्रष्टुं दण्डकारण्य वासिनः}
{अगस्त्यं ते पुरस्कृत्य तुष्टुवुर्मैथिलीपतिम्}%॥ ६२ ॥

\uvacha{मुनय ऊचुः}

\twolineshloka
{नमस्ते रामचन्द्राय लोकानुग्रहकारिणे}
{अरावणं जगत्कर्तुमवतीर्णाय भूतले}%॥ ६३ ॥

\twolineshloka
{ताटिकादेहसंहर्त्रे गाधिजाध्वररक्षिणे}
{नमस्ते जितमारीच सुबाहुप्राणहारिणे}%॥ ६४ ॥

\twolineshloka
{अहल्यामुक्तिसन्दायिपादपङ्कजरेणवे}
{नमस्ते हरकोदण्डलीलाभञ्जनकारिणे}%॥ ६५ ॥

\twolineshloka
{नमस्ते मैथिलीपाणिग्रहणोत्सवशालिने}
{नमस्ते रेणुकापुत्रपराजयविधायिने}%॥ ६६ ॥

\twolineshloka
{सहलक्ष्मणसीताभ्यां कैकेय्यास्तु वरद्वयात्}
{सत्यं पितृवचः कर्तुं नमो वनमुपेयुषे}%॥ ६७ ॥

\twolineshloka
{भरतप्रार्थनादत्तपादुकायुगुलाय ते}
{नमस्ते शरभङ्गस्य स्वर्गप्राप्त्यैकहेतवे}%॥ ६८ ॥

\twolineshloka
{नमो विराधसंहर्त्रे गृधराजसखाय ते}
{मायामृगमहाक्रूरमारीचाङ्गविदारिणे}%॥ ६९ ॥

\twolineshloka
{सीतापहारिलोकेशयुद्धत्यक्तकलेवरम्}
{जटायुषं तु सन्दह्य तत्कैवल्यप्रदायिने}%॥ ७० ॥

\twolineshloka
{नमः कबन्धसंहर्त्रे शबरीपूजिताङ्घ्रये}
{प्राप्तसुग्रीवसख्याय कृतवालिवधाय ते}%॥ ७१ ॥

\twolineshloka
{नमः कृतवते सेतुं समुद्रे वरुणालये}
{सर्वराक्षससंहर्त्रे रावणप्राणहारिणे}%॥ ७२ ॥

\twolineshloka
{संसाराम्बुधिसन्तारपोतपादाम्बुजाय ते}
{नमो भक्तार्तिसंहर्त्रे सच्चिदानन्दरूपिणे}%॥ ७३ ॥

\twolineshloka
{नमस्ते रामभद्राय जगतामृद्धिहेतवे}
{रामादिपुण्यनामानि जपतां पापहारिणे}%॥ ७४ ॥

\twolineshloka
{नमस्ते सर्वलोकानां सृष्टिस्थित्यन्तकारिणे}
{नमस्ते करुणामूर्ते भक्तरक्षणदीक्षित}%॥ ७८५ ॥

\twolineshloka
{ससीताय नमस्तुभ्यं विभीषणसुखप्रद}
{लङ्केश्वरवधाद्राम पालितं हि जगत्त्वया}%॥ ७६ ॥

\twolineshloka
{रक्ष रक्ष जगन्नाथ पाह्यस्माञ्जानकीपते}
{स्तुत्वैवं मुनयः सर्वे तूष्णीं तस्थुर्द्विजोत्तमाः}%॥ ७७ ॥

\uvacha{श्रीसूत उवाच}

\twolineshloka
{य इदं रामचन्द्रस्य स्तोत्रं मुनिभिरीरितम्}
{त्रिसन्ध्यं पठते भक्त्या भुक्तिं मुक्तिं च विन्दति}%॥ ७८ ॥

\twolineshloka
{प्रयाणकाले पठतो न् भीतिरुपजायते}
{एतत्स्तोत्रस्य पठनाद्भूतवेतालकादयः}%॥ ७९ ॥

\twolineshloka
{नश्यन्ति रोगा नश्यन्ति नश्यते पापसञ्चयः}
{पुत्रकामो लभेत्पुत्रं कन्या विन्दति सत्पतिम्}%॥ ८० ॥

\twolineshloka
{मोक्षकामो लभेन्मोक्षं धनकामो धनं लभेत्}
{सर्वान्कामानवाप्नोति पठन्भक्त्या त्विमं स्तवम्}%॥ ८१ ॥

\twolineshloka
{ततो रामो मुनीन्प्राह प्रणम्य च कृताञ्जलिः}
{अहं विशुद्धये प्राप्यः सकलैरपि मानवैः}%॥ ८२ ॥

\twolineshloka
{मद्दृष्टिगोचरो जन्तुर्नित्यमोक्षस्य भाजनम्}
{तथाऽपि मुनयो नित्यं भक्तियुक्तेन चेतसा}%॥ ८३ ॥

\twolineshloka
{स्वात्मलाभेन सन्तुष्टान्साधून्भूतसुहृत्तमान्}
{निरहङ्कारिणः शान्तान्नमस्याम्यूर्ध्वरेतसः}%॥ ८४ ॥

\twolineshloka
{यस्माद्ब्रह्मण्यदेवोऽहमतो विप्रान्भजे सदा}
{युष्मान्पृच्छाम्यहं किञ्चित्तद्वदध्वं विचार्य तु}%॥ ८५ ॥

\threelineshloka
{रावणस्य वधाद्विप्रा यत्पापं मम वर्तते}
{तस्य मे निष्कृतिं ब्रूत पौलस्त्यवधजस्य हि}
{यत्कृत्वा तेन पापे न मुच्येऽहं मुनिपुङ्गवाः}%॥ ८६ ॥


\uvacha{मुनय ऊचुः}

\onelineshloka
{सत्यव्रत जगन्नाथ जगद्रक्षाधुरन्धर}%॥ ८७ ॥

\twolineshloka
{सर्वलोकोपकारार्थं कुरु राम शिवार्चनम्}
{गन्धमादनशृङ्गेऽस्मिन्महापुण्ये विमुक्तिदे}%॥ ८८ ॥

\twolineshloka
{शिवलिङ्गप्रतिष्ठां त्वं लोकसङ्ग्रहकाम्यया}
{कुरु राम दशग्रीववधदोषापनुत्तये}%॥ ८९ ॥


\twolineshloka
{लिङ्गस्थापनजं पुण्यं चतुर्वक्त्रोऽपि भाषितुम्॥}
{न शक्नोति ततो वक्तुं किं पुनर्मनुजेश्वर}%॥९॥

\twolineshloka
{यत्त्वया स्थाप्यते लिगं गन्धमादनपर्वते}
{अस्य सन्दर्शनं पुंसां काशीलिङ्गावलोकनात्}%॥ ९१ ॥

\twolineshloka
{अधिकं कोटिगुणितं फलवत्स्यान्न संशयः}
{तव नाम्ना त्विदं लिङ्गं लोके ख्यातिं समश्नुताम्}%॥ ९२ ॥

\twolineshloka
{नाशकं पुण्यपापाख्यकाष्ठानां दहनोपमम्}
{इदं रामेश्वरं लिङ्गं ख्यातं लोके भविष्यति}%॥ ९३ ॥

\twolineshloka
{मा विलम्बं कुरुष्वातो लिङ्गस्थापनकर्मणि}
{रामचन्द्र महाभाग करुणापूर्णविग्रह}%॥ ९४ ॥

\uvacha{श्रीसूत उवाच}

\twolineshloka
{इति श्रुत्वा वचो रामो मुनीनां तं मुनीश्वराः}
{पुण्यकालं विचार्याथ द्विमुहूर्तं जगत्पतिः}%॥ ९५ ॥

\twolineshloka
{कैलासं प्रेषयामास हनुमन्तं शिवालयम्}
{शिवलिङ्गं समानेतुं स्थापनार्थं रघूद्वहः}%॥ ९६ ॥

\uvacha{राम उवाच}

\twolineshloka
{हनूमन्नञ्जनीसूनो वायुपुत्र महाबल}
{कैलासं त्वरितो गत्वा लिङ्गमानय मा चिरम्}%॥ ९७ ॥

\twolineshloka
{इत्याज्ञप्तस्स रामेण भुजावास्फाल्य वीर्यवान्}
{मुहूर्तद्वितयं ज्ञात्वा पुण्यकालं कपीश्वरः}%॥ ९८ ॥

\twolineshloka
{पश्यतां सर्वदेवानामृषीणां च महात्मनाम्}
{उत्पपात महावेगश्चालयन्गन्धमादनम्}%॥ ९९ ॥

\twolineshloka
{लङ्घयन्स वियन्मार्गं कैलासं पर्वतं ययौ}
{न ददर्श महादेवं लिङ्गरूपधरं कपिः}%॥ १०० ॥

\twolineshloka
{कैलासे पर्वते तस्मिन्पुण्ये शङ्करपालिते}
{आञ्जनेयस्तपस्तेपे लिङ्गप्राप्त्यर्थमादरात्}%॥ १ ॥

\twolineshloka
{प्रागग्रेषु समासीनः कुशेषु मुनिपुङ्गवाः}
{ऊर्ध्वबाहुर्निरालम्बो निरुच्छ्वासो जितेन्द्रियः}%॥ २ ॥

\twolineshloka
{प्रसादयन्महादेवं लिङ्गं लेभे स मारुतिः}
{एतस्मिन्नन्तरे विप्रा मुनिभिस्तत्त्वदर्शिभिः}%॥ ३ ॥

\twolineshloka
{अनागतं हनूमन्तं कालं स्वल्पावशेषितम्}
{ज्ञात्वा प्रकथितं तत्र रामं प्रति महामतिम्}%॥ ४ ॥

\twolineshloka
{रामराम महाबाहो कालो ह्यत्येति साम्प्रतम्}
{जानक्या यत्कृतं लिङ्गं सैकतं लीलया विभो}%॥ ५ ॥

\twolineshloka
{तल्लिङ्गं स्थापयस्वाद्य महालिङ्गमनुत्तमम्}
{श्रुत्वैतद्वचनं रामो जानक्या सह सत्वरम्}%॥ ६ ॥

\twolineshloka
{मुनिभिः सहितः प्रीत्या कृतकौतुकमङ्गलः}
{ज्येष्ठे मासे सिते पक्षे दशम्यां बुधहस्तयोः}%॥ ७ ॥

\twolineshloka
{गरानन्दे व्यतीपाते कन्या चन्द्रे वृषे रवौ}
{दशयोगे महापुण्ये गन्धमादनपर्वते}%॥ ८ ॥

\twolineshloka
{सेतुमध्ये महादेवं लिङ्गरूपधरं हरम्}
{ईशानं कृत्तिवसनं गङ्गाचन्द्रकलाधरम्}%॥ ९ ॥

\twolineshloka
{रामो वै स्थापयामास शिवलिङ्गमनुत्तमम्}
{लिङ्गस्थं पूजयामास राघवः साम्बमीश्वरम्}%॥ ११० ॥

\twolineshloka
{लिङ्गस्थः स महादेवः पार्वत्या सह शङ्करः}
{प्रत्यक्षमेव भगवान्दत्तवान्वरमुत्तमम्}%॥ ११ ॥

\twolineshloka
{सर्वलोकशरण्याय राघवाय महात्मने}
{त्वयात्र स्थापितं लिङ्गं ये पश्यन्ति रघूद्वह}%॥ १२ ॥

\twolineshloka
{महापातकयुक्ताश्च तेषां पापं प्रणश्यति}
{सर्वाण्यपि हि पापानि धनुष्कोटौ निमज्जनात्}%॥ १३ ॥

\twolineshloka
{दर्शनाद्रामलिङ्गस्य पातकानि महान्त्यपि}
{विलयं यान्ति राजेन्द्र रामचन्द्र न संशयः}%॥ १४ ॥

\twolineshloka
{प्रादादेवं हि रामाय वरं देवोंऽबिकापतिः}
{तदग्रे नन्दिकेशं च स्थापयामास राघवः}%॥ १५ ॥

\twolineshloka
{ईश्वरस्याभिषेकार्थं धनुष्कोट्याथ राघवः}
{एकं कूपं धरां भित्त्वा जनयामास वै द्विजाः}%॥ १६ ॥

\twolineshloka
{तस्माज्जलमुपादाय स्नापयामास शङ्करम्}
{कोटितीर्थमिति प्रोक्तं तत्तीर्थं पुण्यमुत्तमम्}%॥ १७ ॥

\threelineshloka
{उक्तं तद्वैभवं पूर्वमस्माभिर्मुनिपुङ्गवाः}
{देवाश्च मुनयो नागा गन्धर्वाप्स रसां गणाः}
{सर्वेपि वानरा लिङ्गमेकैकं चक्रुरादरात्}%॥ १८ ॥


\uvacha{श्रीसूत उवाच}

\onelineshloka
{एवं वः कथितं विप्रा यथा रामेण धीमत}%॥ ५९ ॥

\twolineshloka
{स्थापितं शिवलिङ्गं वै भुक्तिमुक्तिप्रदायकम्}
{इमां लिङ्गप्रतिष्ठां यः शृणोति पठतेऽथवा}%॥ १२० ॥

\twolineshloka
{स रामेश्वरलिङ्गस्य सेवाफलमवाप्नुयात्}
{सायुज्यं च समाप्नोति रामनाथस्य वैभवात्}%॥ १२१ ॥

॥इति श्रीस्कान्दे महापुराण एकाशीतिसाहस्र्यां संहितायां तृतीये ब्रह्मखण्डे सेतुमाहात्म्ये रामनाथलिङ्गप्रतिष्ठाविधिवर्णनं नाम चतुश्चत्वारिंशोऽध्यायः॥४४॥

\sect{रामचन्द्रतत्त्वज्ञानोपदेशवर्णनम्}

\src{स्कन्दपुराणम्}{खण्डः ३ (ब्रह्मखण्डः)}{सेतुखण्डः}{अध्यायः ४५}
\vakta{}
\shrota{}
\tags{}
\notes{}
\textlink{https://sa.wikisource.org/wiki/स्कन्दपुराणम्/खण्डः_३_(ब्रह्मखण्डः)/सेतुखण्डः/अध्यायः_४५}
\translink{https://www.wisdomlib.org/hinduism/book/the-skanda-purana/d/doc423613.html}

\storymeta




\uvacha{श्रीसूत उवाच}

\twolineshloka
{एवं प्रतिष्ठिते लिङ्गे रामेणाक्लिष्टकारिणा}
{लिङ्गं वरं समादाय मारुतिः सहसाऽऽययौ}%॥ १ ॥

\twolineshloka
{रामं दाशरथिं वीरमभिवाद्य स मारुतिः}
{वैदेहीलक्ष्मणौ पश्चात्सुग्रीवं प्रणनाम च}%॥ २ ॥

\twolineshloka
{सीता सैकतलिङ्गं तत्पूजयन्तं रघूद्वहम्}
{दृष्ट्वाथ मुनिभिः सार्द्धं चुकोप पवनात्मजः}%॥ ३ ॥

\twolineshloka
{अत्यन्तं खेदखिन्नः सन्वृथाकृतपरिश्रमः}
{उवाच रामं धर्मज्ञं हनूमानञ्जनात्मजः}%॥ ४ ॥

\uvacha{हनूमानुवाच}

\twolineshloka
{दुर्जातोऽहं वृथा राम लोके क्लेशाय केवलम्}
{खिन्नोऽस्मि बहुशो देव राक्षसैः क्रूरकर्मभिः}%॥ ५ ॥

\twolineshloka
{मा स्म सीमन्तिनी काचिज्जनयेन्मादृशं सुतम्}
{यतोऽनुभूयते दुःखमनन्तं भवसागरे}%॥ ६ ॥

\twolineshloka
{खिन्नोऽस्मि सेवया पूर्वं युद्धेनापि ततोधिकम्}
{अनन्तं दुःखमधुना यतो मामवमन्यसे}%॥ ७ ॥

\twolineshloka
{सुग्रीवेण च भार्यार्थं राज्यार्थं राक्षसेन च}
{रावणावरजेन त्वं सेवितो ऽसि रघूद्वह}%॥ ८ ॥

\twolineshloka
{मया निर्हेतुकं राम सेवितोऽसि महामते}
{वानराणामनेकेषु त्वयाज्ञप्तोऽहमद्य वै}%॥ ९ ॥

\twolineshloka
{शिवलिङ्गं समानेतुं कैलासात्पर्वतो त्तमात्}
{कैलासं त्वरितो गत्वा न चापश्यं पिनाकिनम्}%॥ १० ॥

\twolineshloka
{तपसा प्रीणयित्वा तं साम्बं वृषभवाहनम्}
{प्राप्तलिङ्गो रघुपते त्वरितः समु पागतः}%॥ ११ ॥

\twolineshloka
{अन्यलिङ्गं त्वमधुना प्रतिष्ठाप्य तु सैकतम्}
{मुनिभिर्देवगन्धर्वैः साकं पूजयसे विभो}%॥ १२ ॥

\twolineshloka
{मयानीतमिदं लिङ्गं कैलासा त्पर्वताद्वृथा}
{अहो भाराय मे देहो मन्दभाग्यस्यजायते}%॥ १३ ॥

\twolineshloka
{भूतलस्य महाराज जानकीरमण प्रभो}
{इदं दुःखमहं सोढुं न शक्नोमि रघूद्वह}%॥ १४ ॥

\twolineshloka
{किं करिष्यामि कुत्राहं गमिष्यामि न मे गतिः}
{अतः शरीरं त्यक्ष्यामि त्वयाहमवमानितः}%॥ १५ ॥

\uvacha{श्रीसूत उवाच}

\twolineshloka
{एवं स बहुशो विप्राः क्रुशित्वा पवनात्मजः}
{दण्डवत्प्रणतो भूमौ क्रोधशोकाकुलोऽभवत्}%॥ १६ ॥

\threelineshloka
{तं दृष्ट्वा रघुनाथोऽपि प्रहसन्निदमब्रवीत्}
{पश्यतां सवदेवानां मुनीनां कपिरक्षसाम्}
{सान्त्वयन्मारुतिं तत्र दुःखं चास्य प्रमार्जयन्}%॥ १७ ॥

\uvacha{श्रीराम उवाच}

\onelineshloka
{सर्वं जानाम्यहं कार्यमात्मनोऽपि परस्य च}%॥ १८ ॥

\twolineshloka
{जातस्य जायमानस्य मृतस्यापि सदा कपे}
{जायते म्रियते जन्तुरेक एव स्वकर्मणा} % १९ ॥

\twolineshloka
{प्रयाति नरकं चापि परमात्मा तु निर्गुणः}
{एवं तत्त्वं विनिश्चित्य शोकं मा कुरु वानर}%॥ २० ॥

\twolineshloka
{लिङ्गत्रयविनिर्मुक्तं ज्योतिरेकं निरञ्जनम्}
{निराश्रयं निर्विकारमात्मानं पश्य नित्यशः}%॥ २१ ॥

\twolineshloka
{किमर्थं कुरुषे शोकं तत्त्वज्ञानस्य बाधकम्}
{तत्त्वज्ञाने सदा निष्ठां कुरु वानरसत्तम}%॥ २२ ॥

\twolineshloka
{स्वयम्प्रकाशमात्मानं ध्यायस्व सततं कपे}
{देहादौ ममतां मुञ्च तत्त्वज्ञानविरोधिनीम्}%॥ २३ ॥

\twolineshloka
{धर्मं भजस्व सततं प्राणिहिंसां परित्यज}
{सेवस्व साधुपुरुषाञ्जहि सर्वेन्द्रियाणि च}%॥ २४ ॥

\twolineshloka
{परित्यजस्व सततमन्येषां दोषकीर्तनम्}
{शिवविष्ण्वादिदेवानामर्चां कुरु सदा कपे}%॥ २५ ॥

\twolineshloka
{सत्यं वदस्व सततं परित्यज शुचं कपे}
{प्रत्यग्ब्रह्मैकताज्ञानं मोहवस्तुसमुद्गतम्}%॥ २६ ॥

\twolineshloka
{शोभनाशोभना भ्रान्तिः कल्पि तास्मिन्यथार्थवत्}
{अध्यास्ते शोभनत्वेन पदार्थे मोहवैभवात्}%॥ २७ ॥

\twolineshloka
{रोगो विजायते नृणां भ्रान्तानां कपिसत्तम}
{रागद्वेषबलाद्बद्धा धर्मा धर्मवशङ्गताः}%॥ २८ ॥

\twolineshloka
{देवतिर्यङ्मनुष्याद्या निरयं यान्ति मानवाः}
{चन्दनागरुकर्पूरप्रमुखा अतिशोभनाः}%॥ २९ ॥

\twolineshloka
{मलं भवन्ति यत्स्पर्शात्तच्छरीरं कथं सुखम्}
{भक्ष्यभोज्यादयः सर्वे पदार्था अतिशोभनाः}%॥ ३० ॥

\twolineshloka
{विष्ठा भवन्ति यत्सङ्गात्तच्छरीरं कथं सुखम्}
{सुगन्धि शीतलं तोयं मूत्रं यत्सङ्गमाद्भवेत्}%॥ ३१ ॥

\twolineshloka
{तत्कथं शोभनं पिण्डं भवेद्ब्रूहि कपेऽधुना}
{अतीव धवलाः शुद्धाः पटा यत्सङ्गमेनहि}%॥ ३२ ॥

\twolineshloka
{भवन्ति मलिनाः स्वेदात्तत्कथं शोभनं भवेत}
{श्रूयतां परमार्थो मे हनूमन्वायुनन्दन}%॥ ३३ ॥

\twolineshloka
{अस्मिन्संसारगर्ते तु किञ्चित्सौख्यं न विद्यते}
{प्रथमं जन्तुराप्नोति जन्म बाल्यं ततः परम्}%॥ ३४ ॥

\twolineshloka
{पश्चाद्यौवनमाप्नोति ततो वार्धक्यमश्नुते}
{पश्चान्मृत्युमवाप्नोति पुनर्जन्म तदश्नुते}%॥ ३५ ॥

\twolineshloka
{अज्ञानवैभवादेव दुःखमाप्नोति मानवः}
{तदज्ञान निवृत्तौ तु प्राप्नोति सुखमुत्तमम्}%॥ ३६ ॥

\twolineshloka
{अज्ञानस्य निवृत्तिस्तु ज्ञानादेव न कर्मणा}
{ज्ञानं नाम परं ब्रह्म ज्ञानं वेदान्तवाक्यजम्}%॥ ३७ ॥

\twolineshloka
{तज्ज्ञानं च विरक्तस्य जायते नेतरस्य हि}
{मुख्याधिकारिणः सत्यमाचार्यस्य प्रसादतः}%॥ ३९ ॥

\twolineshloka
{यदा सर्वे प्रमुच्यन्ते कामा येऽस्य हृदि स्थिताः}
{तदा मर्त्योऽमृतोऽत्रैव परं ब्रह्म समश्नुते}%॥ ३९ ॥

\twolineshloka
{जाग्रतं च स्वपन्तं च भुञ्जन्तं च स्थितं तथा}
{इमं जनं सदा क्रूरः कृतान्तः परिकर्षति}%॥ ४० ॥

\twolineshloka
{सर्वे क्षयान्ता निचयाः पतनान्ताः समुच्छ्रयाः}
{संयोगा विप्रयोगान्ता मरणान्तं च जीवितम्}%॥ ४१ ॥

\twolineshloka
{यथा फलानां पक्वानां नान्यत्र पतनाद्भयम्}
{यथा नराणां जातानां नान्यत्र पतनाद्भयम्}%॥ ४२ ॥

\twolineshloka
{यथा गृहं दृढस्तम्भं जीर्णं काले विनश्यति}
{एवं विनश्यन्ति नरा जरामृत्युवशङ्गताः}%॥ ४३ ॥

\twolineshloka
{अहोरात्रस्य गमनान्नृणामायुर्विनश्यति}
{आत्मानमनुशोच त्वं किमन्यमनुशोचसि}%॥ ४४ ॥

\twolineshloka
{नश्यत्यायुः स्थितस्यापि धावतोऽपि कपीश्वर}
{सहैव मृत्युर्व्रजति सह मृत्युर्निषीदति}%॥ ४५ ॥

\twolineshloka
{चरित्वा दूरदेशं च सह मृत्युर्निवर्तते}
{शरीरे वलयः प्राप्ताः श्वेता जाताः शिरोरुहाः}%॥ ४६ ॥

\twolineshloka
{जीर्यते जरया देहः श्वासकासादिना तथा}
{यथा काष्ठं च काष्ठं च समेयातां महोदधौ}%॥ ४७ ॥

\twolineshloka
{समेत्य च व्यपेयातां कालयोगेन वानर}
{एवं भार्या च पुत्रश्च वधुक्षेत्रधनानि च}%॥ ४८ ॥

\twolineshloka
{क्वचित्सम्भूय गच्छन्ति पुनरन्यत्र वानर}
{यथा हि पान्थं गच्छन्तं पथि कश्चित्पथि स्थितः}%॥ ४९ ॥

\twolineshloka
{अहमप्या गमिष्यामि भवद्भिः साकमित्यथ}
{कञ्चित्कालं समेतौ तौ पुनरन्यत्र गच्छतः}%॥ ५० ॥

\twolineshloka
{एवं भार्यासुतादीनां सङ्गमो नश्वरः कपे}
{शरीरजन्मना साकं मृत्युः सञ्जायते ध्रुवम्}%॥ ५१ ॥

\twolineshloka
{अवश्यम्भाविमरणे न हि जातु प्रतिक्रिया}
{एतच्छरीरपाते तु देही कर्मगतिं गतः}%॥ ५२ ॥

\twolineshloka
{प्राप्य पिण्डान्तरं वत्स पूर्वपिण्डं त्यजत्यसौ}
{प्राणिनां न सदैकत्र वासो भवति वानर}%॥ ५३ ॥

\twolineshloka
{स्वस्वकर्मवशात्सर्वे वियुज्यन्ते पृथक्पृथक्}
{यथा प्राणिशरीराणि नश्यन्ति च भवन्ति च}%॥ ५४ ॥

\twolineshloka
{आत्मनो जन्ममरणे नैव स्तः कपिसत्तम}
{अतस्त्वमञ्जनासूनो विशोकं ज्ञानमद्वयम्}%॥ ५५ ॥

\twolineshloka
{सद्रूपममलं ब्रह्म चिन्तयस्व दिवानिशम्}
{त्वत्कृतं मत्कृतं कर्म मत्कृतं त्वाकृतं तथा}%॥ ५६ ॥

\twolineshloka
{मल्लिङ्गस्थापनं तस्मात्त्वल्लिङ्ग स्थापनं कपे}
{मुहूर्तातिक्रमाल्लिङ्गं सैकतं सीतया कृतम्}%॥ ५७ ॥

\twolineshloka
{मयात्र स्थापितं तस्मात्कोपं दुःखं च मा कुरु}
{कैलासादागतं लिङ्गं स्थापयास्मिच्छुभे दिने}%॥ ५८ ॥

\twolineshloka
{तव नाम्ना त्विदं लिङ्गं यातु लोकत्रये प्रथाम्}
{हनूमदीश्वरं दृष्ट्वा द्रष्टव्यो राघवेश्वरः}%॥ ५९ ॥

\twolineshloka
{ब्रह्मराक्षसयूथानि हतानि भवता कपे}
{अतः स्वनाम्ना लिङ्गस्य स्थापनात्त्वं प्रमोक्ष्यसे}%॥ ६० ॥

\twolineshloka
{स्वयं हरेण दत्तं तु हनूमन्नामकं शिवम्}
{सम्पश्यन्रामनाथं च कृतकृत्यो भवेन्नरः}%॥ ६१ ॥

\twolineshloka
{योजनानां सहस्रेऽपि स्मृत्वा लिङ्गं हनूमतः}
{रामनाथेश्वरं चापि स्मृत्वा सायुज्यमाप्नुयात्}%॥ ६२ ॥

\twolineshloka
{तेनेष्टं सर्वयज्ञैश्च तपश्चाकारि कृत्स्नशः}
{येन दृष्टौ महादेवौ हनूमद्राघवेश्वरौ}%॥ ६३ ॥

\twolineshloka
{हनूमता कृतं लिङ्गं यच्च लिङ्गं मया कृतम्}
{जानकीयं च यल्लिङ्गं यल्लिङ्गं लक्ष्मणेश्वरम्}%॥ ६४ ॥

\twolineshloka
{सुग्रीवेण कृतं यच्च सेतुकर्त्रा नलेन च}
{अङ्गदेन च नीलेन तथा जाम्बवता कृतम्}%॥ ६५ ॥

\twolineshloka
{विभीषणेन यच्चापि रत्नलिङ्गं प्रतिष्ठितम्}
{इन्द्राद्यैश्च कृतं लिङ्गं यच्छेषाद्यैः प्रतिष्ठितम्}%॥ ६६ ॥

\twolineshloka
{इत्येकादशरूपोऽयं शिवः साक्षाद्विभासते}
{सदा ह्येतेषु लिङ्गेषु सन्निधत्ते महेश्वरः}%॥ ६७ ॥

\twolineshloka
{तत्स्वपापौघशुद्ध्यर्थं स्थापयस्व महेश्वरम्}
{अथ चेत्त्वं महाभाग लिङ्गमुत्सादयिष्यसि}%॥ ६८ ॥

\twolineshloka
{मयात्र स्थापितं वत्स सीतया सैकतं कृतम्}
{स्थापयिष्यामि च ततो लिङ्गमेतत्त्वया कृतम्}%॥ ६९ ॥

\twolineshloka
{पातालं सुतलं प्राप्य वितलं च रसातलम्}
{तलातलं च तदिदं भेदयित्वा तु तिष्ठति}%॥ ७० ॥

\twolineshloka
{प्रतिष्ठितं मया लिङ्गं भेत्तुं कस्य बलं भवेत्}
{उत्तिष्ठ लिङ्गमुद्वास्य मयैतत्स्थापितं कपे}%॥ ७१ ॥

\twolineshloka
{त्वया समाहृतं लिङ्गं स्थापयस्वाशु मा शुचः}
{इत्युक्तस्तं प्रणम्याथाज्ञातसत्त्वोऽथ वानरः}%॥ ७२ ॥

\twolineshloka
{उद्वासयामि वेगेन सैकतं लिङ्गमुत्त मम्}
{संस्थापयामि कैलासादानीतं लिङ्गमादरात्}%॥ ७३ ॥

\twolineshloka
{उद्वासने सैकतस्य कियान्भारो भवेन्मम}
{चेतसैवं विचार्यायं हनूमान्मारुता त्मजः}%॥ ७४ ॥

\twolineshloka
{पश्यतां सर्वदेवानां मुनीनां कपिरक्षसाम्}
{पश्यतो रामचन्द्रस्य लक्ष्मणस्यापि पश्यतः}%॥ ७५ ॥

\twolineshloka
{पश्यन्त्या अपि वैदेह्या लिङ्गं तत्सैकतं बलात्}
{पाणिना सर्वयत्नेन जग्राह तरसा बली}%॥ ७६ ॥

\twolineshloka
{यत्नेन महता चायं चालयन्नपि मारुतिः}
{नालं चालयितुं ह्यासीत्सैकतं लिङ्गमोजसा}%॥ ७७ ॥

\twolineshloka
{ततः किलकिलाशब्दं कुर्वन्वानरपुङ्गवः}
{पुच्छमुद्यम्य पाणिभ्यां निरास्थत्तन्निजौजसा}%॥ ७८ ॥

\twolineshloka
{इत्यनेकप्रकारेण चाल यन्नपि वानरः}
{नैव चालयितुं शक्तो बभूव पवनात्मजः}%॥ ७९ ॥

\twolineshloka
{तद्वेष्टयित्वा पुच्छेन पाणिभ्यां धरणीं स्पृशन्}
{उत्पपाताथ तरसा व्योम्नि वायुसुतः कपिः}%॥ ८० ॥

\twolineshloka
{कम्पयन्स धरां सर्वां सप्तद्वीपां सपर्वतम्}
{लिङ्गस्य क्रोशमात्रे तु मूर्च्छितो रुधिरं वमन्}%॥ ८१ ॥

\twolineshloka
{पपात हनुमान्विप्राः कम्पिताङ्गो धरातले}
{पततो वायुपुत्रस्य वक्त्राच्च नयनद्वयात्}%॥ ८२ ॥

\twolineshloka
{नासापुटाच्छ्रोत्ररन्ध्रादपानाच्च द्विजोत्तमाः}
{रुधिरौघः प्रसुस्राव रक्तकुण्ड मभूच्च तत्}%॥ ८३ ॥

\twolineshloka
{ततो हाहाकृतं सर्वं सदेवासुरमानुषम्}
{धावन्तौ कपिभिः सार्द्धमुभौ तौ रामलक्ष्मणौ}%॥ ८४ ॥

\twolineshloka
{जानकीसहितौ विप्रा ह्यास्तां शोकाकुलौ तदा}
{सीतया सहितौ वीरौ वानरैश्च महाबलौ}%॥ ८५ ॥

\twolineshloka
{रुरुचाते तदा विप्रा गन्धमादनपर्वते}
{यथा तारागणयुतौ रजन्यां शशि भास्करौ}%॥ ८६ ॥

\twolineshloka
{ददर्शतुर्हनूमन्तं चूर्णीकृतकलेवरम्}
{मूर्च्छितं पतितं भूमौ वमन्तं रुधिरं मुखात्}%॥ ८७ ॥

\twolineshloka
{विलोक्य कपयः सर्वे हाहाकृत्वाऽपतन्भुवि}
{कराभ्यां सदयं सीता हनूमन्तं मरुत्सुतम्}%॥ ८८ ॥

\twolineshloka
{ताततातेति पस्पर्श पतितं धरणीतले}
{रामोऽपि दृष्ट्वा पतितं हनूमन्तं कपीश्वरम्}%॥ ८९ ॥

\twolineshloka
{आरोप्याङ्कं स्वपाणिभ्यामाममर्श कलेवरम्}
{विमुञ्चन्नेत्रजं वारि वायुजं चाव्रवीद्द्विजाः}%॥ ९० ॥
॥इति श्रीस्कान्दे महापुराण एकाशीतिसाहस्र्यां संहितायां तृतीये ब्रह्मखण्डे सेतुमाहात्म्ये रामचन्द्रतत्त्वज्ञानोपदेशवर्णनं नाम पञ्चचत्वारिंशोऽध्यायः॥४५॥

\sect{षट्चत्वारिंशोऽध्यायः --- रामनाथलिङ्गप्रतिष्ठाकारणवर्णनम्}

\src{स्कन्दपुराणम्}{खण्डः ३ (ब्रह्मखण्डः)}{सेतुखण्डः}{अध्यायः ४६}
\vakta{}
\shrota{}
\tags{}
\notes{}
\textlink{https://sa.wikisource.org/wiki/स्कन्दपुराणम्/खण्डः_३_(ब्रह्मखण्डः)/सेतुखण्डः/अध्यायः_४६}
\translink{https://www.wisdomlib.org/hinduism/book/the-skanda-purana/d/doc423614.html}

\storymeta




\uvacha{श्रीराम उवाच}


\twolineshloka
{पम्पारण्ये वयं दीनास्त्वया वानरपुङ्गव॥}
{आश्वासिताः कारयित्वा}%सख्यमादित्यसूनुना॥१॥

\twolineshloka
{त्वां दृष्ट्वा पितरं बन्धून्कौसल्यां जननीमपि}
{न स्मरामो वयं सर्वान्मे त्वयोपकृतं बहु}%॥ २ ॥

\twolineshloka
{मदर्थं सागरस्तीर्णो भवता बहु योजनः}
{तलप्रहाराभिहतो मैनाकोऽपि नगोत्तमः}%॥ ३ ॥

\twolineshloka
{नागमाता च सुरसा मदर्थं भवता जिता}
{छायाग्रहां महाक्रूराम वधीद्राक्षसीं भवान्}%॥ ४ ॥

\twolineshloka
{सायं सुवेलमासाद्य लङ्कामाहत्य पाणिना}
{अयासी रावणगृहं मदर्थं त्वं महाकपे}%॥ ५ ॥

\twolineshloka
{सीतामन्विष्य लङ्कायां रात्रौ गतभयो भवान्}
{अदृष्ट्वा जानकीं पश्चादशोकवनिकां ययौ}%॥ ६ ॥

\twolineshloka
{नमस्कृत्य च वैदेहीमभिज्ञानं प्रदाय च}
{चूडामणिं समादाय मदर्थं जानकीकरात्}%॥ ७ ॥

\twolineshloka
{अशोकवनिकावृक्षानभाङ्क्षीस्त्वं महाकपे}
{ततस्त्वशीतिसाहस्रान्किङ्करान्नाम राक्षसान्}%॥ ८ ॥

\twolineshloka
{रावणप्रतिमान्युद्धे पत्यश्वेभरथाकुलान्}
{अवधीस्त्वं मदर्थे वै महाबलपराक्रमान्}%॥ ९ ॥

\twolineshloka
{ततः प्रहस्ततनयं जम्बुमालिनमागतम्}
{अवधीन्मन्त्रितनयान्सप्त सप्तार्चिवर्चसः}%॥ १० ॥

\twolineshloka
{पञ्च सेनापतीन्पश्चादनयस्त्वं यमालयम्}
{कुमारमक्षमवधीस्ततस्त्वं रणमूर्धनि}%॥ ११ ॥

\twolineshloka
{तत इन्द्रजिता नीतो राक्षसेन्द्र सभां शुभाम्}
{तत्र लङ्केश्वरं वाचा तृणीकृत्यावमन्य च}%॥ १२ ॥

\twolineshloka
{अभाङ्क्षीस्त्वं पुरीं लङ्कां मदर्थं वायुनन्दन}
{पुनः प्रतिनिवृत्तस्त्वमृष्यमूकं महागिरिम्}%॥ १३ ॥

\twolineshloka
{एवमादि महादुःखं मदर्थं प्राप्तवानसि}
{त्वमत्र भूतले शेषे मम शोकमुदीरयन्}%॥ १४ ॥

\twolineshloka
{अहं प्राणान्परित्यक्ष्ये मृतोऽसि यदि वायुज}
{सीतया मम किं कार्यं लक्ष्मणेनानुजेन वा}%॥ १५ ॥

\twolineshloka
{भरतेनापि किं कार्यं शत्रुघ्नेन श्रियापि वा}
{राज्येनापि न मे कार्यं परेतस्त्वं कपे यदि}%॥ १६ ॥

\twolineshloka
{उत्तिष्ठ हनुमन्वत्स किं शेषेऽद्य महीतले}
{शय्यां कुरु महाबाहो निद्रार्थं मम वानर}%॥ १७ ॥

\twolineshloka
{कन्दमूलफलानि त्वमाहारार्थं ममाहर}
{स्नातुमद्य गमिष्यामि शीघ्रं कलशमानय}%॥ १८ ॥

\twolineshloka
{अजिनानि च वासांसि दर्भांश्च समुपाहर}
{ब्रह्मास्त्रेणावबद्धोऽहं मोचितश्च त्वया हरे}%॥ १९ ॥

\twolineshloka
{लक्ष्मणेन सह भ्रात्रा ह्यौषधानयनेन वै}
{लक्ष्मणप्राणदाता त्वं पौलस्त्यमदनाशनः}%॥ २० ॥

\twolineshloka
{सहायेन त्वया युद्धे राक्षसा न्रावणादिकान्}
{निहत्यातिबलान्वीरानवापं मैथिलीमहम्}%॥ २१ ॥

\twolineshloka
{हनूमन्नञ्जनासूनो सीताशोकविनाशन}
{कथमेवं परित्यज्य लक्ष्मणं मां च जानकीम्}%॥ २२ ॥

\twolineshloka
{अप्रापयित्वायोध्यां त्वं किमर्थं गतवानसि}
{क्व गतोसि महावीर महाराक्षसकण्टक}%॥ २३ ॥

\twolineshloka
{इति पश्यन्मुखं तस्य निर्वाक्यं रघुनन्दनः}
{प्ररुदन्नश्रुजालेन सेचयामास वायुजम्}%॥ २४ ॥

\twolineshloka
{वायुपुत्रस्ततो मूर्च्छामपहाय शनैर्द्विजाः}
{पौलस्त्यभयसन्त्रस्तलोकरक्षार्थमागतम्}%॥ २५ ॥

\twolineshloka
{आश्रित्य मानुषं भावं नारायणमजं विभुम्}
{जानकीलक्ष्मणयुतं कपिभिः परिवारितम्}%॥ २६ ॥

\twolineshloka
{कालाम्भोधरसङ्काशं रणधूलिसमुक्षितम्}
{जटामण्डलशोभाढ्यं पुण्डरीकायतेक्षणम्}%॥ २७ ॥

\twolineshloka
{खिन्नं च बहुशो युद्धे ददर्श रघुनन्दनम्}
{स्तूयमानममित्रघ्नं देवर्षिपितृकिन्नरैः}%॥ २८ ॥

\twolineshloka
{दृष्ट्वा दाशरथिं रामं कृपाबहुलचेतसम्}
{रघुनाथकरस्पर्शपूर्णगात्रः स वानरः}%॥ २९ ॥

\twolineshloka
{पतित्वा दण्डवद्भूमौ कृताञ्जलिपुटो द्विजाः}
{अस्तौषीज्जानकीनाथं स्तोत्रैः श्रुतिमनोहरैः}%॥ ३० ॥

\uvacha{हनूमानुवाच}

\twolineshloka
{नमो रामाय हरये विष्णवे प्रभविष्णवे}
{आदिदेवाय देवाय पुराणाय गदाभृते}%॥ ३१ ॥

\twolineshloka
{विष्टरे पुष्पकं नित्यं निविष्टाय महात्मने}
{प्रहृष्टवानरानीकजुष्टपादाम्बुजाय ते}%॥ ३२ ॥

\twolineshloka
{निष्पिष्ट राक्षसेन्द्राय जगदिष्टविधायिने}
{नमः सहस्रशिरसे सहस्रचरणाय च}%॥ ३३ ॥

\twolineshloka
{सहस्राक्षाय शुद्धाय राघवाय च विष्णवे}
{भक्तार्तिहारिणे तुभ्यं सीतायाः पतये नमः}%॥ ३४ ॥

\twolineshloka
{हरये नारसिंहाय दैत्यराजविदारिणे}
{नमस्तुभ्यं वराहाय दंष्ट्रोद्धृतवसुन्धर}%॥ ३५ ॥

\twolineshloka
{त्रिविक्रमाय भवते बलियज्ञ विभेदिने}
{नमो वामनरूपाय नमो मन्दरधारिणे}%॥ ३६ ॥

\twolineshloka
{नमस्ते मत्स्यरूपाय त्रयीपालनकारिणे}
{नमः परशुरामाय क्षत्रियान्तकराय ते}%॥ ३७ ॥

\twolineshloka
{नमस्ते राक्षसघ्नाय नमो राघवरूपिणे}
{महादेव महाभीम महाकोदण्डभेदिने}%॥ ३८ ॥

\twolineshloka
{क्षत्रियान्तकरक्रूरभार्गवत्रासकारिणे}
{नमोऽस्त्वहिल्या सन्तापहारिणे चापहारिणे}%॥ ३९ ॥

\twolineshloka
{नागायुतवलोपेतताटकादेहहारिणे}
{शिलाकठिनविस्तारवालिवक्षोविभेदिने}%॥ ४० ॥

\twolineshloka
{नमो माया मृगोन्माथकारिणेऽज्ञानहारिणे}
{दशस्यन्दनदुःखाब्धिशोषणागस्त्यरूपिणे}%॥ ४१ ॥

\twolineshloka
{अनेकोर्मिसमाधूतसमुद्रमदहारिणे}
{मैथिलीमानसां भोजभानवे लोकसाक्षिणे}%॥ ४२ ॥

\twolineshloka
{राजेन्द्राय नमस्तुभ्यं जानकीपतये हरे}
{तारकब्रह्मणे तुभ्यं नमो राजीवलोचन}%॥ ४३ ॥

\twolineshloka
{रामाय रामचन्द्राय वरेण्याय सुखात्मने}
{विश्वामित्रप्रियायेदं नमः खरविदारिणे}%॥ ४४ ॥

\twolineshloka
{प्रसीद देवदेवेश भक्तानामभयप्रद}
{रक्ष मां करु णासिन्धो रामचन्द्र नमोऽस्तु ते}%॥ ४५ ॥

\twolineshloka
{रक्ष मां वेदवचसामप्यगोचर राघव}
{पाहि मां कृपया राम शरणं त्वामुपैम्यहम्}%॥ ४६ ॥

\twolineshloka
{रघुवीर महामोहमपाकुरु ममाधुना}
{स्नाने चाचमने भुक्तो जाग्रत्स्वप्नसुषुप्तिषु}%॥ ४७ ॥

\twolineshloka
{सर्वावस्थासु सर्वत्र पाहि मां रघुनन्दन}
{महिमानं तव स्तोतुं कः समर्थो जगत्त्रये}%॥ ४८ ॥

\twolineshloka
{त्वमेव त्वन्महत्त्वं वै जानासि रघुनन्दन}
{इति स्तुत्वा वायुपुत्रो रामचन्द्रं घृणानिधिम्}%॥ ४९ ॥

\twolineshloka
{सीतामप्यभितुष्टाव भक्तियुक्तेन चेतसा}
{जानकि त्वां नमस्यामि सर्वपापप्रणाशिनीम्}%॥ ५० ॥

\twolineshloka
{दारिद्र्यरणसंहर्त्रीं भक्तानामिष्टदायिनीम्}
{विदेहराजतनयां राघवानन्दकारिणीम्}%॥ ५१ ॥

\twolineshloka
{भूमेर्दुहितरं विद्यां नमामि प्रकृतिं शिवाम्}
{पौलस्त्यैश्वर्यसंहर्त्रीं भक्ताभीष्टां सरस्वतीम्}%॥ ५२ ॥

\twolineshloka
{पतिव्रताधुरीणां त्वां नमामि जनकात्मजाम्}
{अनुग्रहपरामृद्धिमनघां हरिवल्लभाम्}%॥ ५३ ॥

\twolineshloka
{आत्मविद्यां त्रयीरूपामुमारूपां नमाम्य हम्}
{प्रसादाभिमुखीं लक्ष्मीं क्षीराब्धितनयां शुभाम्}%॥ ५४ ॥

\twolineshloka
{नमामि चन्द्रभगिनीं सीतां सर्वाङ्गसुन्दरीम्}
{नमामि धर्मनिलयां करुणां वेदमातरम्}%॥ ५५ ॥

\twolineshloka
{पद्मालयां पद्महस्तां विष्णुवक्षस्थलालयाम्}
{नमामि चन्द्रनिलयां सीतां चन्द्रनिभाननाम्}%॥ ५६ ॥

\threelineshloka
{आह्लादरूपिणीं सिद्धिं शिवां शिवकरीं सतीम्}
{नमामि विश्वजननीं रामचन्द्रेष्टवल्लभाम्}
{सीतां सर्वानवद्याङ्गीं भजामि सततं हृदा}%॥ ५७ ॥


\uvacha{श्रीसूत उवाच}

\onelineshloka
{स्तुत्वैवं हनुमान्सीतारामचन्द्रौ सभक्तिकम्}%॥ ५८ ॥

\twolineshloka
{आनन्दाश्रुपरिक्लिन्नस्तूष्णीमास्ते द्विजोत्तमाः}
{य इदं वायुपुत्रेण कथितं पापनाशनम्}%॥ ५९ ॥

\twolineshloka
{स्तोत्रं श्रीरामचन्द्रस्य सीतायाः पठतेऽन्वहम्}
{स नरो महदैश्वर्यमश्नुते वाञ्छितं स दा}%॥ ६० ॥

\twolineshloka
{अनेकक्षेत्रधान्यानि गाश्च दोग्ध्रीः पयस्विनीः}
{आयुर्विद्याश्च पुत्रांश्च भार्यामपि मनोरमाम्}%॥ ६१ ॥

\twolineshloka
{एतत्स्तोत्रं सकृ द्विप्राः पठन्नाप्नोत्यसंशयः}
{एतत्स्तोत्रस्य पाठेन नरकं नैव यास्यति}%॥ ६२ ॥

\twolineshloka
{ब्रह्महत्यादिपापानि नश्यन्ति सुमहान्त्यपि}
{सर्वपापविनिर्मुक्तो देहान्ते मुक्तिमाप्नुयात्}%॥ ६३ ॥

\twolineshloka
{इति स्तुतो जगन्नाथो वायुपुत्रेण राघवः}
{सीतया सहितो विप्रा हनूमन्तमथाब्रवीत्}%॥ ६४ ॥

\uvacha{श्रीराम उवाच}

\twolineshloka
{अज्ञानाद्वा नरश्रेष्ठ त्वयेदं साहसं कृतम्}
{ब्रह्मणा विष्णुना वापि शक्रादित्रिदशैरपि}%॥ ६५ ॥

\twolineshloka
{नेदं लिङ्गं समुद्धर्तुं शक्यते स्थापितं मया}
{महादेवापराधेन पतितोऽस्यद्य मूर्च्छितः}%॥ ६६ ॥

\twolineshloka
{इतः परं मा क्रियतां द्रोहः साम्बस्य शूलिनः}
{अद्यारभ्य त्विदं कुण्डं तव नाम्ना जगत्त्रये}%॥ ६७ ॥

\twolineshloka
{ख्यातिं प्रयातु यत्र त्वं पतितो वानरोत्तम}
{महापातकसङ्घानां नाशः स्यादत्र मज्जनात्}%॥ ६८ ॥

\twolineshloka
{महादेवजटाजाता गौतमी सरितां वरा}
{अश्वमेधसहस्रस्य फलदा स्नायिनां नृणाम्}%॥ ६९ ॥

\twolineshloka
{ततः शतगुणा गङ्गा यमुना च सरस्वती}
{एतन्नदीत्रयं यत्र स्थले प्रवहते कपे}%॥ ७० ॥

\twolineshloka
{मिलित्वा तत्र तु स्नानं सहस्रगुणितं स्मृतम्}
{नदीष्वेतासु यत्स्नानात्फलं पुंसां भवेत्कपे}%॥ ७१ ॥

\twolineshloka
{तत्फलं तव कुण्डेऽस्मिन्स्नानात्प्राप्नोत्यसंशयम्}
{दुर्लभं प्राप्य मानुष्यं हनूमत्कुण्डतीरतः}%॥ ७२ ॥

\twolineshloka
{श्राद्धं न कुरुते यस्तु भक्तियुक्तेन चेतसा}
{निराशास्तस्य पितरः प्रयान्ति कुपिताः कपे}%॥ ७३ ॥

\twolineshloka
{कुप्यन्ति मुनयोऽप्यस्मै देवाः सेन्द्राः सचारणाः}
{न दत्तं न हुतं येन हनूमत्कुण्डतीरतः}%॥ ७४ ॥

\threelineshloka
{वृथाजीवित एवासाविहामुत्र च दुःखभाक्}
{हनूमत्कुण्डसविधे येन दत्तं तिलोदकम्}
{मोदन्ते पितरस्तस्य घृतकुल्याः पिबन्ति च}%॥ ७५ ॥


\uvacha{श्रीसूत उवाच}

\onelineshloka
{श्रुत्वैतद्वचनं विप्रा रामेणोक्तं स वायुजः}%॥ ७६ ॥

\twolineshloka
{उत्तरे रामनाथस्य लिङ्गं स्वेनाहृतं मुदा}
{आज्ञया रामचन्द्रस्य स्थापयामास वायुजः}%॥ ७७ ॥

\threelineshloka
{प्रत्यक्षमेव सर्वेषां कपिलाङ्गूलवेष्टितम्}
{हरोपि तत्पुच्छजा तं बिभर्ति च वलित्रयम्}
{तदुत्तरायां ककुभि गौरीं संस्थापयन्मुदा}%॥ ७८ ॥

\uvacha{श्रीसूत उवाच}

\twolineshloka
{एवं वः कथितं विप्रा यदर्थं राघवेण तु}
{लिङ्गं प्रतिष्ठितं सेतौ भुक्तिमुक्तिप्रदं नृणाम्}%॥ ७९ ॥

\twolineshloka
{यः पठेदिममध्यायं शृणुयाद्वा समाहितः}
{स विधूयेह पापानि शिवलोके महीयते}%॥ ८० ॥
॥इति श्रीस्कान्दे महापुराण एकाशीतिसाहस्र्यां संहितायां तृतीये ब्रह्मखण्डे सेतुमाहात्म्ये रामनाथलिङ्गप्रतिष्ठाकारणवर्णनं नाम षट्चत्वारिंशोऽध्यायः॥४६॥


\sect{सप्तचत्वारिंशोऽध्यायः --- रामस्य ब्रह्महत्योत्पत्तिहेतुनिरूपणम्}

\src{स्कन्दपुराणम्}{खण्डः ३ (ब्रह्मखण्डः)}{सेतुखण्डः}{अध्यायः ४७}
\vakta{}
\shrota{}
\tags{}
\notes{}
\textlink{https://sa.wikisource.org/wiki/स्कन्दपुराणम्/खण्डः_३_(ब्रह्मखण्डः)/सेतुखण्डः/अध्यायः_४७}
\translink{https://www.wisdomlib.org/hinduism/book/the-skanda-purana/d/doc423615.html}

\storymeta




\uvacha{ऋषय ऊचुः}

\twolineshloka
{राक्षसस्य वधात्सूत रावणस्य महामुने}
{ब्रह्महत्या कथमभूद्राघवस्य महात्मनः}%॥ १ ॥

\twolineshloka
{ब्राह्मणस्य वधात्सूत ब्रह्महत्याभिजायते}
{न ब्राह्मणो दशग्रीवः कथं तद्वद नो मुने}%॥ २ ॥

\twolineshloka
{ब्रह्महत्या भवेत्क्रूरा रामचन्द्रस्य धीमतः}
{एतन्नः श्रद्दधानानां वद कारुण्यतोऽधुना}%॥ ३ ॥

\twolineshloka
{इति पृष्टस्ततः सूतो नैमिषारण्यवासिभिः}
{वक्तुं प्रचक्रमे तेषां प्रश्नस्योत्तरमुत्तमम्}%॥ ४ ॥

\uvacha{श्रीसूत उवाच}

\twolineshloka
{ब्रह्मपुत्रो महातेजाः पुलस्त्योनाम वै द्विजाः}
{बभूव तस्य पुत्रोऽभूद्विश्रवा इति विश्रुतः}%॥ ५ ॥

\twolineshloka
{तस्य पुत्रः पुलस्त्य स्य विश्रवा मुनिपुङ्गवाः}
{चिरकालं तपस्तेपे देवैरपि सुदुष्करम्}%॥ ६ ॥

\twolineshloka
{तपः कुर्वति तस्मिंस्तु सुमाली नाम राक्षसः}
{पाताललोकाद्भूलोकं सर्वं वै विचचार ह}%॥ ७ ॥

\twolineshloka
{हेमनिष्काङ्गदधरः कालमेघनिभच्छविः}
{समादाय सुतां कन्यां पद्महीनामिव श्रियम्}%॥ ८ ॥

\twolineshloka
{विचरन्स महीपृष्ठे कदाचित्पुष्पकस्थितम्}
{दृष्ट्वा विश्रवसः पुत्रं कुबेरं वै धनेश्वरम्}%॥ ९ ॥

\twolineshloka
{चिन्तयामास विप्रेन्द्राः सुमाली स तु राक्षसः}
{कुबेरसदृशः पुत्रो यद्यस्माकं भविष्यति}%॥ १० ॥

\twolineshloka
{वयं वर्द्धामहे सर्वे राक्षसा ह्यकुतोभयाः}
{विचार्यैवं निजसुतामब्रवीद्राक्षसेश्वरः}%॥ ११ ॥

\twolineshloka
{सुते प्रदानकालोऽद्य तव कैकसि शोभने}
{अद्य ते यौवनं प्राप्तं तद्देया त्वं वराय हि}%॥ १२ ॥

\twolineshloka
{अप्रदानेन पुत्रीणां पितरो दुःखमाप्नुयुः}
{किं च सर्वगुणोत्कृष्टा लक्ष्मीरिव सुते शुभे}%॥ १३ ॥

\twolineshloka
{प्रत्याख्यानभयात्पुम्भिर्न च त्वं प्रार्थ्यसे शुभे}
{कन्यापितृत्वं दुःखाय सर्वेषां मानकाङ्क्षिणाम्}%॥ १४ ॥

\twolineshloka
{न जानेऽहं वरः को वा वरयेदिति कन्यके}
{सा त्वं पुलस्त्यतनयं मुनिं विश्रवसं द्विजम्}%॥ १५ ॥

\twolineshloka
{पितामहकुलोद्भूतं वरयस्व स्वयङ्गता}
{कुबेरतुल्यास्तनया भवेयुस्ते न संशयः}%॥ १६ ॥

\twolineshloka
{कैकसी तद्वचः श्रुत्वा सा कन्या पितृगौरवात्}
{अङ्गीचकार तद्वाक्यं तथास्त्विति शुचिस्मिता}%॥ १७ ॥

\twolineshloka
{पर्णशालां मुनिश्रेष्ठा गत्वा विश्रवसो मुनेः}
{अतिष्ठदन्तिके तस्य लज्जमाना ह्यधोमुखी}%॥ १८ ॥

\twolineshloka
{तस्मिन्नवसरे विप्राः पुलस्त्यतनयः सुधीः}
{अग्निहोत्रमुपास्ते स्म ज्वलत्पावकसन्निभः}%॥ १९ ॥

\twolineshloka
{सन्ध्याकालमतिक्रूरमविचिन्त्य तु कैकसी}
{अभ्येत्य तं मुनिं सुभ्रूः पितुर्वचनगौरवात्}%॥ २० ॥

\threelineshloka
{तस्थावधोमुखी भूमिं लिखत्यङ्गुष्ठकोटिना}
{विश्रवास्तां विलोक्याथ कैकसीं तनुमध्यमाम्}
{उवाच सस्मितो विप्राः पूर्णचन्द्रनिभाननाम्}%॥ २१ ॥

\uvacha{विश्रवा उवाच}

\onelineshloka
{शोभने कस्य पुत्री त्वं कुतो वा त्वमिहागता}%॥ २२ ॥

\twolineshloka
{कार्यं किं वा त्वमुद्दिश्य वर्तसेऽत्र शुचिस्मिते}
{यथार्थतो वदस्वाद्य मम सर्वमनिन्दिते}%॥ २३ ॥

\twolineshloka
{इतीरिता कैकसी सा कन्या बद्धाञ्जलिर्द्विजाः}
{उवाच तं मुनिं प्रह्वा विनयेन समन्विता}%॥ २४ ॥

\twolineshloka
{तपः प्रभावेन मुने मदभिप्रायमद्य तु}
{वेत्तुमर्हसि सम्यक्त्वं पुलस्त्यकुलदीपन}%॥ २५ ॥

\twolineshloka
{अहं तु कैकसी नाम सुमालिदुहिता मुने}
{मत्तातस्याज्ञया ब्रह्मंस्तवान्तिकमुपागता}%॥ २६ ॥

\twolineshloka
{शेष त्वं ज्ञानदृष्ट्याद्य ज्ञातुमर्हस्यसंशयः}
{क्षणं ध्यात्वा मुनिः प्राह विश्रवाः स तु कैकसीम्}%॥ २७ ॥

\twolineshloka
{मया ते विदितं सुभ्रूर्मनोगतमभीप्सितम्}
{पुत्राभिलाषिणी सा त्वं मामगाः साम्प्रतं शुभे}%॥ २८ ॥

\twolineshloka
{सायङ्कालेऽधुना क्रूरे यस्मान्मां त्वमुपागता}
{पुत्राभिलाषिणी भूत्वा तस्मात्त्वां प्रब्रवीम्यहम्}%॥ २९ ॥

\twolineshloka
{शृणुष्वावहिता रामे कैकसी त्वमनिन्दिते}
{दारुणान्दारुणाकारान्दारुणाभिजनप्रियान्}%॥ ३० ॥

\twolineshloka
{जनयिष्यसि पुत्रांस्त्वं राक्षसान्क्रूरकर्मणः}
{श्रुत्वा तद्वचनं सा तु कैकसी प्रणिपत्य तम्}%॥ ३१ ॥

\twolineshloka
{पुलस्त्यतनयं प्राह कृताञ्जलिपुटा द्विजाः}
{भगवदीदृशाः पुत्रास्त्वत्तः प्राप्तुं न युज्यते}%॥ ३२ ॥

\twolineshloka
{इत्युक्तः स मुनिः प्राह कैकसीं तां सुमध्यमाम्}
{मद्वंशानुगुणः पुत्रः पश्चिमस्ते भविष्यति}%॥ ३३ ॥

\twolineshloka
{धार्मिकः शास्त्रविच्छान्तो न तु राक्षसचेष्टितः}
{इत्युक्ता कैकसी विप्राः काले कतिपये गते}%॥ ३४ ॥

\twolineshloka
{सुषुवे तनयं क्रूरं रक्षोरूपं भयङ्करम्}
{द्विपञ्चशीर्षं कुमतिं विंशद्बाहुं भयानकम्}%॥ ३५ ॥

\twolineshloka
{ताम्रोष्ठं कृष्णवदनं रक्तश्मश्रु शिरोरुहम्}
{महादंष्ट्रं महाकायं लोकत्रासकरं सदा}%॥ ३६ ॥

\twolineshloka
{दशग्रीवाभिधः सोऽभूत्तथा रावण नामवान्}
{रावणानन्तरं जातः कुम्भकर्णाभिधः सुतः}%॥ ३७ ॥

\twolineshloka
{ततः शूर्पणखा नाम्ना क्रूरा जज्ञे च राक्षसी}
{ततो बभूव कैकस्या विभीषण इति श्रुतः}%॥ ३८ ॥

\twolineshloka
{पश्चिमस्तनयो धीमान्धार्मिको वेदशास्त्रवित्}
{एते विश्रवसः पुत्रा दशग्रीवादयो द्विजाः}%॥ ३९ ॥

\twolineshloka
{अतो दशग्रीववधात्कुम्भकर्णवधादपि}
{ब्रह्महत्या समभवद्रामस्याक्लिष्टकर्मणः}%॥ ४० ॥

\twolineshloka
{अतस्तच्छान्तये रामो लिङ्गं रामेश्वराभिधम्}
{स्थापयामास विधिना वैदिकेन द्विजोत्तमाः}%॥ ४१ ॥

\twolineshloka
{एवं रावणघातेन ब्रह्महत्यासमुद्भवः}
{समभूद्रामचन्द्रस्य लोककान्तस्य धीमतः}%॥ ४२ ॥

\twolineshloka
{तत्सहैतुकमाख्यातं भवतां ब्रह्मघातजम्}
{पापं यच्छान्तये रामो लिङ्गं प्रातिष्ठिपत्स्वयम्}%॥ ४३ ॥

\twolineshloka
{एवं लिङ्गं प्रतिष्ठाप्य रामचन्द्रोऽतिधार्मिकः}
{मेने कृतार्थमात्मानं ससीता वरजो द्विजाः}%॥ ४४ ॥

\twolineshloka
{ब्रह्महत्या गता यत्र रामचन्द्रस्य भूपतेः}
{तत्र तीर्थमभूत्किञ्चिद्ब्रह्महत्याविमोचनम्}%॥ ४५ ॥

\twolineshloka
{तत्र स्नानं महापुण्यं ब्रह्महत्याविनाशनम्}
{दृश्यते रावणोऽद्यापि छायारूपेण तत्र वै}%॥ ४६ ॥

\twolineshloka
{तदग्रे नागलोकस्य बिलमस्ति महत्तरम्}
{दशग्रीववधोत्पन्नां ब्रह्महत्यां बलीयसीम्}%॥ ४७ ॥

\twolineshloka
{तद्बिलं प्रापयामास जानकीरमणो द्विजाः}
{तस्योपरि बिलस्याथ कृत्वा मण्डपमुत्तमम्}%॥ ४८ ॥

\twolineshloka
{भैरवं स्थापयामास रक्षार्थं तत्र राघवः}
{भैरवाज्ञापरित्रस्ता ब्रह्महत्या भयङ्करी}%॥ ४९ ॥

\twolineshloka
{नाशक्नोत्तद्बिलादूर्ध्वं निर्गन्तुं द्विजसत्तमाः}
{तस्मिन्नेव बिले तस्थौ ब्रह्महत्या निरुद्यमा}%॥ ५० ॥

\twolineshloka
{रामनाथमहालिङ्गं दक्षिणे गिरिजा मुदा}
{वर्तते परमानन्दशिवस्यार्धशरीरिणी}%॥ ५१ ॥

\twolineshloka
{आदित्यसोमौ वर्तेते पार्श्वयोस्तत्र शूलिनः}
{देवस्य पुरतो वह्नी रामनाथस्य वर्तते}%॥ ५२ ॥

\twolineshloka
{आस्ते शतक्रतुः प्राच्यामाग्नेयां च तथाऽनलः}
{आस्ते यमो दक्षिणस्यां रामनाथस्य सेवकः}%॥ ५३ ॥

\twolineshloka
{नैर्ऋते निर्ऋतिर्विप्रा वर्तते शङ्करस्य तु}
{वारुण्यां वरुणो भक्त्या सेवते राघवेश्वरम्}%॥ ५४ ॥

\twolineshloka
{वायव्ये तु दिशो भागे वायुरास्ते शिवस्य तु}
{उत्तरस्यां च धनदो रामनाथस्य वर्तते}%॥ ५५ ॥

\twolineshloka
{ईशान्येऽस्य च दिग्भागे महेशो वर्तते द्विजाः}
{विनायक कुमारौ च महादेवसुतावुभौ}%॥ ५६ ॥

\twolineshloka
{यथाप्रदेशं वर्तेते रामनाथालयेऽधुना}
{वीरभद्रादयः सर्वे महेश्वरगणेश्वराः}%॥ ५७ ॥

\twolineshloka
{यथाप्रदेशं वर्तन्ते रामनाथालये सदा}
{मुनयः पन्नगाः सिद्धा गन्धर्वाप्सरसां गणाः}%॥ ५८ ॥

\twolineshloka
{सन्तुष्यमाणहृदया यथेष्टं शिवसन्निधौ}
{वर्तन्ते रामनाथस्य सेवार्थं भक्तिपूर्वकम्}%॥ ५९ ॥

\twolineshloka
{रामनाथस्य पूजार्थं श्रोत्रियान्ब्राह्मणान्बहून्}
{रामेश्वरे रघुपतिः स्थापयामास पूजकान्}%॥ ६० ॥

\twolineshloka
{रामप्रतिष्ठितान्विप्रान्हव्यकव्यादिनार्चयेत्}
{तुष्टास्ते तोषिताः सर्वा पितृभिः सहदेवताः}%॥ ६१ ॥

\twolineshloka
{तेभ्यो बहुधनान्ग्रामान्प्रददौ जानकीपतिः}
{रामनाथमहादेव नैवेद्यार्थमपि द्विजाः}%॥ ६२ ॥

\twolineshloka
{बहून्ग्रामान्बहुधनं प्रददौ लक्ष्मणाग्रजः}
{हारकेयूरकटकनिष्काद्याभरणानि च}%॥ ६३ ॥

\twolineshloka
{अनेकपट्ट वस्त्राणि क्षौमाणि विविधानि च}
{रामनाथाय देवाय ददौ दशरथात्मजः}%॥ ६४ ॥

\twolineshloka
{गङ्गा च यमुना पुण्या सरयूश्च सरस्वती}
{सेतौ रामेश्वरं देवं भजन्ते स्वाघशान्तये}%॥ ६५ ॥

\twolineshloka
{एतदध्यायपठनाच्छ्रवणादपि मानवः}
{विमुक्तः सर्वपापेभ्यः सायुज्यं लभते हरेः}%॥ ६६ ॥
॥इति श्रीस्कान्दे महापुराण एकाशीतिसाहस्र्यां संहितायां तृतीये ब्रह्मखण्डे सेतुमाहात्म्ये रामस्य ब्रह्महत्योत्पत्तिहेतुनिरूपणं नाम सप्तचत्वारिंशोऽध्यायः॥४७॥