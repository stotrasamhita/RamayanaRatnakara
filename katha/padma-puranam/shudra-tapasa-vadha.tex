\sect{शूद्रतापसवधः}

\src{पद्म-पुराणम्}{सृष्टिखण्डम्}{अध्यायः ३६}{१--१८५}
% \tags{concise, complete}
\notes{In this chapter, Rama is approached by a grieving brāhmaṇa whose son has died prematurely; learning that the death was due to a śūdra performing severe penance, Rama locates and kills the ascetic Śambūka, sending him to heaven, and the boy comes back to life at the very moment.}
\textlink{https://sa.wikisource.org/wiki/पद्मपुराणम्/खण्डः_१_(सृष्टिखण्डम्)/अध्यायः_३५}
\translink{https://www.wisdomlib.org/hinduism/book/the-padma-purana/d/doc364158.html}

\storymeta



\uvacha{भीष्म उवाच}

\twolineshloka
{उक्तं भगवता सर्वं पुराणाश्रयसंयुतम्}
{तथा श्वेतेन ब्रह्माण्डं गुरवे प्रतिपादितम्}% १

\twolineshloka
{श्रुत्वैतत्कौतुकं जातं यथा तेनास्थिलेहनम्}
{कृतं क्षुधापनोदार्थे अन्नदानाद्विना द्विज}% २

\twolineshloka
{तदहं श्रोतुमिच्छामि पृथिव्यां ये च पार्थिवाः}
{अन्नदानाद्दिवं प्राप्ताः क्रतवश्चान्नमूलकाः}% ३

\twolineshloka
{कथं तस्य मतिर्नष्टा श्वेतस्य च महात्मनः}
{न दत्तं तेनान्नदानमृषिभिर्वा न दर्शितम्}% ४

\twolineshloka
{अहो माहात्म्यमन्नस्य इह दत्तस्य यत्फलम्}
{परत्र भुज्यते पुम्भिः स्वर्गश्चाक्षयतां व्रजेत्}% ५

\twolineshloka
{अन्नदानं परं विप्राः कीर्तयन्ति सदोत्थिताः}
{अन्नदानात्सुरेद्रेण त्रैलोक्यमिह भुज्यते}% ६

\twolineshloka
{शतक्रतुरिति प्रोक्तः सर्वैरेव द्विजोत्तमैः}
{तेनावस्थां तत्सदृशीं प्राप्तवांस्त्रिदशेश्वरः}% ७

\twolineshloka
{दानदेवगतः स्वर्गं त्वत्तः सर्वं श्रुतं मया}
{अपरं च पुरावृत्तं निवृत्तं यदि कर्हिचित्}% ८

\onelineshloka*
{भूयोपि श्रोतुमिच्छामि तन्मे वद महामते}

\uvacha{पुलस्त्य उवाच}

\onelineshloka
{एतदाख्यानकं पूर्वमगस्त्येन महात्मना}% ९

\onelineshloka*
{रामाय कथितं राजंस्तत्ते वक्ष्यामि साम्प्रतम्}

\uvacha{भीष्म उवाच}

\onelineshloka
{कस्मिन्वंशे समुत्पन्नो रामोऽसौ नृपसत्तमः}% १०

\onelineshloka*
{यस्यागस्त्येन कथितश्चेतिहासः पुरातनः}

\uvacha{पुलस्त्य उवाच}

\onelineshloka
{रघुवंशे समुत्पन्नो रामो नाम महाबलः}% ११

\twolineshloka
{देवकार्यं कृतं तेन लङ्कायां रावणो हतः}
{पृथिवीं राज्यसंस्थस्य ऋषयोऽभ्यागता गृहे}% १२

\twolineshloka
{प्राप्तास्ते तु महात्मानो राघवस्य निवेशनम्}
{प्रतीहारस्ततो राममगस्त्यवचनाद्द्रुतम्}% १३

\twolineshloka
{आवेदयामास ऋषीन्प्राप्तास्तांश्च त्वरान्वितः}
{दृष्ट्वा रामं द्वारपालः पूर्णचन्द्रमिवोदितम्}% १४

\twolineshloka
{कौसल्यासुत भद्रं ते सुप्रभाताद्य शर्वरी}
{द्रष्टुमभ्युदयं तेद्य सम्प्राप्तो रघुनन्दन}% १५

\twolineshloka
{अगस्त्यो मुनिभिः सार्धं द्वारि तिष्ठति ते नृप}
{श्रुत्वा प्राप्तान्मुनीन्रामस्तान्भास्करसमद्युतीन्}% १६

\twolineshloka
{प्राह वाक्यं तदा द्वास्थं प्रवेशय त्वरान्वितः}
{किमर्थं तु त्वया द्वारि निरुद्धा मुनिसत्तमाः}% १७

\twolineshloka
{रामवाक्यान्मुनींस्तांस्तु प्रावेशयद्यथासुखम्}
{दृष्ट्वा तु तान्मुनीन्न्प्राप्तान्प्रत्युवाच कृताञ्जलि}% १८

\twolineshloka
{रामोऽभिवाद्य प्रणत आसनेषु न्यवेशयत्}
{ते तु काञ्चनचित्रेषु स्वास्तीर्णेषु सुखेषु च}% १९

\twolineshloka
{कुशोत्तरेषु चासीनाः समन्तान्मुनिपुङ्गवाः}
{पाद्यमाचमनीयं च ददौ चार्घ्यं पुरोहितः}% २०

\twolineshloka
{रामेण कुशलं पृष्टा ऋषयः सर्व एव ते}
{महर्षयो वेदविद इदं वचनमब्रुवन्}% २१

\twolineshloka
{कुशलं ते महाबाहो सर्वत्र रघुनन्दन}
{त्वां तु दिष्ट्या कुशलिनं पश्यामो हतविद्विषम्}% २२

\twolineshloka
{हृता सीतातिपापेन रावणेन दुरात्मना}
{पत्नी ते रघुशार्दूल तस्या एवौजसा हतः}% २३

\twolineshloka
{असहायेन चैकेन त्वया राम रणे हतः}
{यादृशं ते कृतं कर्म तस्य कर्ता न विद्यते}% २४

\twolineshloka
{इह सम्भाषितुं प्राप्ता दृष्ट्वा पूताः स्म साम्प्रतम्}
{दर्शनात्तव राजेन्द्र सर्वे जातास्तपस्विनः}% २५

\twolineshloka
{रावणस्य वधात्तेद्य कृतमश्रुप्रमार्जनम्}
{दत्वा पुण्यामिमां वीर जगत्यभयदक्षिणाम्}% २६

\twolineshloka
{दिष्ट्या वर्धसि काकुत्स्थ जयेनामितविक्रम}
{दृष्टस्सम्भाषितश्चासि यास्यामश्चाश्रमान्स्वकान्}% २७

\twolineshloka
{अरण्यं ते प्रविष्टस्य मया चेन्द्रशरासनम्}
{अर्पितं चाक्षयौ तूणौ कवचं च परन्तप}% २८

\twolineshloka
{भूयोप्यागमनं कार्यमाश्रमे मे रघूद्वह}
{एवमुक्त्वा तु ते सर्वे मुनयोन्तर्हिताऽभवन्}% २९

\twolineshloka
{गतेषु मुनिमुख्येषु रामो धर्मभृतां वरः}
{चिन्तयामास तत्कार्यं किं स्यान्मे मुनिनोदितम्}% ३०

\twolineshloka
{भूयोप्यागमनं कार्यमाश्रमे रघुनन्दन}
{अवश्यमेव गन्तव्यं मयाऽगस्त्यस्य सन्निधौ}% ३१

\twolineshloka
{श्रोतव्यं देवगुह्यं तु कार्यमन्यच्च यद्वदेत्}
{एवं चिन्तयतस्तस्य रामस्यामिततेजसः}% ३२

\twolineshloka
{करिष्ये नियतं धर्मं धर्मो हि परमा गतिः}
{सुतवर्षसहस्राणि दश राज्यमकारयत्}% ३३

\twolineshloka
{ददतो जुह्वतश्चैव जग्मुस्तान्येकवर्षवत्}
{प्रजाः पालयतस्तस्य राघवस्य महात्मनः}% ३४

\twolineshloka
{एतस्मिन्नेव दिवसे वृद्धो जानपदो द्विजः}
{मृतं पुत्रमुपादाय रामद्वारमुपागतः}% ३५

\twolineshloka
{उवाच विविधं वाक्यं स्नेहाक्षरसमन्वितम्}
{दुष्कृतं किन्तु मे पुत्र पूर्वदेहान्तरे कृतम्}% ३६

\twolineshloka
{त्वामेकपुत्रं यदहं पश्यामि निधनं गतम्}
{अप्राप्तयौवनं बालं पञ्चवर्षं गतायुषम्}% ३७

\twolineshloka
{अकाले कालमापन्नं दुःखाय मम पुत्रक}
{अकृत्वा पितृकार्याणि गतो वैवस्वतक्षयम्}% ३८

\twolineshloka
{रामस्य दुष्कृतं व्यक्तं येन ते मृत्युरागतः}
{बालवध्या ब्रह्मवध्या स्त्रीवध्या चैव राघवम्}% ३९

\twolineshloka
{प्रवेक्ष्यति न सन्देहः सभार्ये तु मृते मयि}
{शुश्राव राघवः सर्वं दुःखशोकसमन्वितम्}% ४०

\twolineshloka
{निवार्य तं द्विजं रामो वसिष्ठं वाक्यमब्रवीत्}
{किं मयाद्य च कर्तव्यं कार्यमेवं विधे स्थिते}% ४१

\twolineshloka
{प्राणानहं जुहोम्यग्नौ पर्वताद्वा पतेह्यहम्}
{कथं शुद्धिमहं यामि श्रुत्वा ब्राह्मणभाषितम्}% ४२

\twolineshloka
{वसिष्ठस्याग्रतः स्थित्वा राज्ञो दीनस्य नारदः}
{प्रत्युवाच श्रुतं वाक्यमृषीणां सन्निधौ तदा}% ४३

\twolineshloka
{शृणु राम यथाकालं प्राप्तो वै बालसङ्क्षयः}
{पुरा कृतयुगे राम सर्वत्र ब्राह्मणोत्तरम्}% ४४

\twolineshloka
{अब्राह्मणो न वै कश्चित्तपस्तपति राघव}
{अमृत्यवस्तदा सर्वे जायन्ते चिरजीविनः}% ४५

\twolineshloka
{त्रेतायुगे पुनः प्राप्ते ब्रह्मक्षत्रमनुत्तमम्}
{अधर्मो द्वापरे तेषां वैश्यान्शूद्रांस्तथाविशत्}% ४६

\twolineshloka
{एवं निरन्तरं जुष्टमुद्भूतमनृतं पुनः}
{अधर्मस्य त्रयः पादा एको धर्मस्य चागतः}% ४७

\twolineshloka
{ततः पूर्वे भृशं त्रस्ता वर्णा ब्राह्मणपूर्वकाः}
{भूयः पादस्तु धर्मस्य द्वितीयः समपद्यत}% ४८

\twolineshloka
{तस्मिन्द्वापरसंज्ञे तु तपो वैश्यं समाविशत्}
{युगत्रयस्य वैधर्म्यं धर्मस्य प्रतितिष्ठति}% ४९

\twolineshloka
{कलिसंज्ञे ततः प्राप्ते वर्तमाने युगेन्तिमे}
{अधर्मश्चानृतं चैव ववृधाते नरर्षभ1.35.}% ५०

\twolineshloka
{भविता शूद्रयोन्यां तु तपश्चर्या कलौ युगे}
{स ते विषयपर्यन्ते राजन्नुग्रतरं तपः}% ५१

\twolineshloka
{शूद्रस्तपति दुर्बुद्धिस्तेन बालवधः कृतः}
{यस्याधर्ममकार्यं वा विषये पार्थिवस्य हि}% ५२

\twolineshloka
{पुरे वा राजशार्दूल कुरुते दुर्मतिर्नरः}
{क्षिप्रं स नरकं याति यावदाभूतसम्प्लवम्}% ५३

\twolineshloka
{चतुर्थं तस्य पापस्य भागमश्नाति पार्थिवः}
{सत्त्वं पुरुषशार्दूल गच्छस्व विषयं स्वकम्}% ५४

\twolineshloka
{दुष्कृतं यत्र पश्येथास्तत्र यत्नं समाचर}
{एवं ते धर्मवृद्धिश्च बलस्य वर्धनं तथा}% ५५

\twolineshloka
{भविष्यति नरश्रेष्ठ बालस्यास्य च जीवनम्}
{नारदेनैवमुक्तस्तु साश्चर्यो रघुनन्दनः}% ५६

\twolineshloka
{प्रहर्षमतुलं लेभे लक्ष्मणं चेदमब्रवीत्}
{गच्छ सौम्य द्विजश्रेष्ठं समाश्वासय लक्ष्मण}% ५७

\twolineshloka
{बालस्य च शरीरं त्वं तैलद्रोण्यां निधापय}
{गन्धैश्च परमोदारैस्तैलैश्चैव सुगन्धिभिः}% ५८

\twolineshloka
{यथा न शीर्यते बालस्तथा सौम्य विधीयताम्}
{यथा शरीरं गुप्तं स्याद्बालस्याक्लिष्टकर्मणः}% ५९

\twolineshloka
{विपत्तिः परिभेदो वा न भवेत्तत्तथा कुरु}
{तथा सन्दिश्य सौमित्रं लक्ष्मणं शुभलक्षणम्}% ६०

\twolineshloka
{मनसा पुष्पकं दध्यावागच्छेति महायशाः}
{इङ्गितं तत्तु विज्ञाय कामगं हेमभूषितम्}% ६१

\twolineshloka
{आजगाम मुहूर्तात्तु समीपं राघवस्य हि}
{सोब्रवीत्प्राञ्जलिर्वाक्यमहमस्मि नराधिप}% ६२

\twolineshloka
{अग्रे तव महाबाहो किङ्करः समुपस्थितः}
{भाषितं सुचिरं श्रुत्वा पुष्पकस्य नराधिप}% ६३

\twolineshloka
{अभिवाद्य महर्षींस्तान्विमानं सोध्यरोहत}
{धनुर्गृहीत्वा तूणौ च खड्गं चापि महाप्रभम्}% ६४

\twolineshloka
{निक्षिप्य नगरे वीरौ सौमित्रि भरतावुभौ}
{प्रायात्प्रतीचीं त्वरितो विचिन्वन्सुसमाहितः}% ६५

\twolineshloka
{उत्तरामगमत्पश्चाद्दिशं हिमवदाश्रिताम्}
{पूर्वामपि दिशां गत्वा तथाऽपश्यन्नराधिपः}% ६६

\twolineshloka
{सर्वां शुद्धसमाचारामादर्शमिव निर्मलाम्}
{ततो दिशं समाक्रामद्दक्षिणां रघुनन्दनः}% ६७

\twolineshloka
{शैलस्य उत्तरे पार्श्वे ददर्श सुमहत्सरः}
{तस्मिन्सरसि तप्यन्तं तापसं सुमहत्तपः}% ६८

\twolineshloka
{ददर्श राघवो भीमं लम्बमानमधोमुखम्}
{तमुपागम्य काकुत्स्थस्तप्यमानं तु तापसम्}% ६९

\twolineshloka
{उवाच राघवो वाक्यं धन्यस्त्वममरप्रभ}
{कस्यां योनौ तपोवृद्धिर्वर्तते दृढनिश्चय}% ७०

\twolineshloka
{अहं दाशरथी रामः पृच्छामि त्वां कुतूहलात्}
{कोर्थो व्यवसितस्तुभ्यं स्वर्गलोकोथ वेतरः}% ७१

\twolineshloka
{किमर्थं तप्यसे वा त्वं श्रोतुमिच्छामि तापस}
{ब्राह्मणो वासि भद्रं ते क्षत्रियो वाथ दुर्जयः}% ७२

\twolineshloka
{वैश्यस्तृतीयवर्णो वा शूद्रो वा सत्यमुच्यताम्}
{तपः सत्यात्मकं नित्यं स्वर्गलोकपरिग्रहे}% ७३

\twolineshloka
{सात्विकं राजसं चैव तच्च सत्यात्मकं तपः}
{जगदुपकारहेतुर्हि सृष्टं तद्वै विरिञ्चिना}% ७४

\twolineshloka
{रौद्रं क्षत्रियतेजोजं तत्तु राजसमुच्यते}
{परस्योत्सादनार्थाय तच्चासुरमुदाहृतम्}% ७५

\twolineshloka
{अङ्गानि निह्नुते यो वा असृग्दिग्धानि भागशः}
{पञ्चाग्निंसाधयेद्वापि सिद्धिं वा मृत्युमेव वा}% ७६

\twolineshloka
{आसुरो ह्येष ते भावो न च मे त्वं द्विजो मतः}
{सत्यं ते वदतः सिद्धिरनृते नास्ति जीवितम्}% ७७

\twolineshloka
{तस्य तद्भाषितं श्रुत्वा रामस्याक्लिष्टकर्मणः}
{अवाक्शिरास्तथा भूतो वाक्यमेतदुवाच ह}% ७८

\twolineshloka
{स्वागतं ते नृपश्रेष्ठ चिराद्दृष्टोसि राघव}
{पुत्रभूतोस्मि ते चाहं पितृभूतोसि मेनघ}% ७९

\twolineshloka
{अथवा नैतदेवं हि सर्वेषां नृपतिः पिता}
{सत्वमर्च्योऽसि भो राजन्वयं ते विषये तपः}% ८०

\twolineshloka
{चरामस्तत्रभागोस्ति पूर्वं सृष्टः स्वयम्भुवा}
{न धन्याः स्मो वयं राम धन्यस्त्वमसि पार्थिव}% ८१

\twolineshloka
{यस्य ते विषये ह्येवं सिद्धिमिच्छन्ति तापसाः}
{तपसा त्वं मदीयेन सिद्धिमाप्नुहि राघव}% ८२

\twolineshloka
{यदेतद्भवता प्रोक्तं योनौ कस्यां तु ते तपः}
{शूद्रयोनिप्रसूतोहं तप उग्रं समास्थितः}% ८३

\twolineshloka
{देवत्वं प्रार्थये राम स्वशरीरेण सुव्रत}
{न मिथ्याहं वदे भूप देवलोकजिगीषया}% ८४

\twolineshloka
{शूद्रं मां विद्धि काकुत्स्थ शम्बूकं नाम नामतः}
{भाषतस्तस्य काकुत्स्थः खड्गं तु रुचिरप्रभम्}% ८५

\twolineshloka
{निष्कृष्य कोशाद्विमलं शिरश्चिच्छेद राघवः}
{तस्मिन्शूद्रे हते देवाः सेन्द्राश्चाग्निपुरोगमाः}% ८६

\twolineshloka
{साधुसाध्विति काकुत्स्थं प्रशशंसुर्मुहुर्मुहुः}
{पुष्पवृष्टिश्च महती देवानां सुसुगन्धिनी}% ८७

\twolineshloka
{आकाशाद्विप्रमुक्ता तु राघवं सर्वतोकिरत्}
{सुप्रीताश्चाब्रुवन्देवा रामं वाक्यविदांवरम्}% ८८

\twolineshloka
{सुरकार्यमिदं सौम्य कृतं ते रघुनन्दन}
{गृहाण च वरं राम यमिच्छसि महाव्रत}% ८९

\twolineshloka
{त्वत्कृतेन हि शूद्रोऽयं सशरीरोऽभ्यगाद्दिवम्}
{देवानां भाषितं श्रुत्वा राघवः सुसमाहितः}% ९०

\twolineshloka
{उवाच प्राञ्जलिर्वाक्यं सहस्राक्षं पुरन्दरम्}
{यदि देवाः प्रसन्ना मे वरार्हो यदि वाप्यहम्}% ९१

\twolineshloka
{कर्मणा यदि मे प्रीता द्विजपुत्रः स जीवतु}
{वरमेतद्धि भवतां काङ्क्षितं परमं हि मे}% ९२

\twolineshloka
{ममापराधाद्बालोऽसौ ब्राह्मणस्यैकपुत्रकः}
{अप्राप्तकालः कालेन नीतो वैवस्वत क्षयम्}% ९३

\twolineshloka
{तं जीवयत भद्रं वो नानृती स्यामहं गुरोः}
{द्विजस्य संश्रुतो ह्यर्थो जीवयिष्यामि ते सुतम्}% ९४

\twolineshloka
{मदीयेनायुषा बालं पादेनार्द्धेन वा सुराः}
{जीवेदयं वरो मह्यं वरकोट्यधिको वृतः}% ९५

\twolineshloka
{राघवस्य तु तद्वाक्यं श्रुत्वा विबुधसत्तमाः}
{प्रत्यूचुस्ते महात्मानं प्रीताः प्रीतिसमन्विताः}% ९६

\twolineshloka
{निर्वृतो भव काकुत्स्थ ब्राह्मणस्यैकपुत्रकः}
{जीवितं प्राप्तवान्भूयः समेतश्चापि बन्धुभिः}% ९७

\twolineshloka
{यस्मिन्मुहूर्ते काकुत्स्थ शूद्रोयं विनिपातितः}
{तस्मिन्मुहूर्ते सहसा जीवेन समयुज्यत}% ९८

\twolineshloka
{स्वस्ति प्राप्नुहि भद्रं ते साधयामः परन्तपः}
{अगस्त्यस्याश्रमपदे द्रष्टारः स्म महामुनिम्}% ९९

\twolineshloka
{स तथेति प्रतिज्ञाय देवानां रघुनन्दनः}
{आरुरोह विमानं तं पुष्पकं हेमभूषितम्}% १००

{॥इति श्रीपाद्मपुराणे प्रथमे सृष्टिखण्डे शूद्रतापसवधो नाम पञ्चत्रिंशोऽध्यायः॥३५॥}
