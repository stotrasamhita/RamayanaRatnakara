\sect{रामागस्त्यसंवादः}

\src{पद्म-पुराणम्}{सृष्टिखण्डम्}{अध्यायः ३६}{१--१८५}
% \tags{concise, complete}
\notes{This chapter describes the conversation between Rama and Agastya. It narrates how Rama, after defeating Ravana, meets Agastya in the forest. Agastya explains the significance of the divine ornaments given to Rama.}
\textlink{https://sa.wikisource.org/wiki/पद्मपुराणम्/खण्डः_१_(सृष्टिखण्डम्)/अध्यायः_३६}
\translink{https://www.wisdomlib.org/hinduism/book/the-padma-purana/d/doc364155.html}

\storymeta


\uvacha{पुलस्त्य उवाच}

\twolineshloka
{ततो देवाः प्रयातास्ते विमानैर्बहुभिस्तदा}
{रामोप्यनुजगामाशु कुम्भयोनेस्तपोवनम्}% १

\twolineshloka
{उक्तं भगवता तेन भूयोप्यागमनं क्रियाः}
{पूर्वमेव सभायां च यो मां द्रष्टुं समागतः}% २

\twolineshloka
{तदहं देवतादेशात्तत्कार्यार्थे महामुनिम्}
{पश्यामि तं मुनिं गत्वा देवदानवपूजितम्}% ३

\twolineshloka
{उपदेशं च मे तुष्टः स्वयं दास्यति सत्तमः}
{दुःखी येन पुनर्मर्त्ये न भवामि कदाचन}% ४

\twolineshloka
{पिता दशरथो मह्यं कौसल्या जननी तथा}
{सूर्यवंशे समुत्पन्नस्तथाप्येवं सुदुःखितः}% ५

\twolineshloka
{राज्यकाले वने वासो भार्यया चानुजेन च}
{हरणं चापि भार्याया रावणेन कृतं मम}% ६

\twolineshloka
{असहायेन तु मया तीर्त्वा सागरमुत्तमम्}
{रुद्ध्वा तु तां पुरीं सर्वां कृत्वा तस्य कुलक्षयम्}% ७

\twolineshloka
{दृष्टा सीता मया त्यक्ता देवानां तु पुरस्तदा}
{शुद्धां तां मां तथोचुस्ते मया सीता तथा गृहम्}% ८

\twolineshloka
{समानीता प्रीतिमता लोकवाक्याद्विसर्जिता}
{वने वसति सा देवी पुरे चाहं वसामि वै}% ९

\twolineshloka
{जातोहमुत्तमे वंशे उत्तमोहं धनुष्मताम्}
{उत्तमं दुःखमापन्नो हृदयं नैव भिद्यते}% १०

\twolineshloka
{वज्रसारस्य सारेण धात्राहं निर्मितो ध्रुवम्}
{इदानीं ब्राह्मणादेशाद्भ्रमामि धरणीतले}% ११

\twolineshloka
{तपः स्थितस्तु शूद्रोसौ मया पापो निपातितः}
{देववाक्यात्तु मे भूयः प्राणो मे हृदि संस्थितः}% १२

\twolineshloka
{पश्यामि तं मुनिं वन्द्यं जगतोस्य हिते रतम्}
{दृष्टेन मे तथा दुःखं नाशमेष्यति सत्वरम्}% १३

\twolineshloka
{उदयेन सहस्रांशोर्हिमं यद्वद्विलीयते}
{तद्वन्मे दुःखसम्प्राप्तिः सर्वथा नाशमेष्यति}% १४

\twolineshloka
{दृष्ट्वा च देवान्सम्प्राप्तानगस्त्यो भगवानृषिः}
{अर्घ्यमादाय सुप्रीतः सर्वांस्तानभ्यपूजयत्}% १५

\twolineshloka
{ते तु गृह्य ततः पूजां सम्भाष्य च महामुनिम्}
{जग्मुस्तेन तदा हृष्टा नाकपृष्ठं सहानुगाः}% १६

\twolineshloka
{गतेषु तेषु काकुत्स्थः पुष्पकादवरुह्य च}
{अभिवादयितुं प्राप्तः सोगस्त्यमृषिमुत्तमम्}% १७

\uvacha{राजोवाच}

\twolineshloka
{सुतो दशरथस्याहं भवन्तमभिवादितुम्}
{आगतो वै मुनिश्रेष्ठ सौम्येनेक्षस्व चक्षुषा}% १८

\twolineshloka
{निर्धूतपापस्त्वां दृष्ट्वा भवामीह न संशयः}
{एतावदुक्त्वा स मुनिमभिवाद्य पुनः पुनः}% १९

\twolineshloka
{कुशलं भृत्यवर्गस्य मृगाणां तनयस्य च}
{भगवद्दर्शनाकाङ्क्षी शूद्रं हत्वा त्विहागतः}% २०

\uvacha{अगस्त्य उवाच}

\twolineshloka
{स्वागतं ते रघुश्रेष्ठ जगद्वन्द्य सनातन}
{दर्शनात्तव काकुत्स्थ पूतोहं मुनिभिः सह}% २१

\twolineshloka
{त्वत्कृते रघुशार्दूल गृहाणार्घं महाद्युते}
{स्वागतं नरशार्दूल दिष्ट्या प्राप्तोसि शत्रुहन्}% २२

\twolineshloka
{त्वं हि नित्यं बहुमतो गुणैर्बहुभिरुत्तमैः}
{अतस्त्वं पूजनीयो वै मम नित्यं हृदिस्थितः}% २३

\twolineshloka
{सुरा हि कथयन्ति त्वां शूद्रघातिनमागतम्}
{ब्राह्मणस्य च धर्मेण त्वया वै जीवितः सुतः}% २४

\twolineshloka
{उष्यतां चेह भगवः सकाशे मम राघव}
{प्रभाते पुष्पकेणासि गन्तायोध्यां महामते}% २५

\twolineshloka
{इदं चाभरणं सौम्य सुकृतं विश्वकर्मणा}
{दिव्यं दिव्येनवपुषा दीप्यमानं स्वतेजसा}% २६

\twolineshloka
{प्रतिगृह्णीष्व राजेन्द्र मत्प्रियं कुरु राघव}
{लब्धस्य हि पुनर्द्दाने सुमहत्फलमुच्यते}% २७

\twolineshloka
{त्वं हि शक्तः परित्रातुं सेन्द्रानपि सुरोत्तमान्}
{तस्मात्प्रदास्ये विधिवत्प्रतीच्छस्व नरर्षभ}% २८

\twolineshloka
{अथोवाच महाबाहुरिक्ष्वाकूणां महारथः}
{कृताञ्जलिर्मुनिश्रेष्ठं स्वं च धर्ममनुस्मरन्}% २९

\twolineshloka
{प्रतिग्रहो वै भगवंस्तव मेऽत्र विगर्हितः}
{क्षत्रियेण कथं विप्र प्रतिग्राह्यं विजानता}% ३०

\twolineshloka
{ब्राह्मणेन तु यद्दत्तं तन्मे त्वं वक्तुमर्हसि}
{सपुत्रो गृहवानस्मि समर्थोस्मि महामुने}% ३१

\twolineshloka
{आपदा चन चाक्रान्तः कथं ग्राह्यः प्रतिग्रहः}
{भार्या मे सुचिरं नष्टा न चान्या मम विद्यते}% ३२

\twolineshloka
{केवलं दोषभागी च भवामीह न संशयः}
{कष्टां चैव दशां प्राप्य क्षत्रियोपि प्रतिग्रही}% ३३

\twolineshloka
{कुर्वन्न दोषमाप्नोति मनुरेवात्र कारणम्}
{वृद्धौ च मातापितरौ साध्वी भार्या शिशुः सुतः}% ३४

\twolineshloka
{अप्यकार्यशतं कृत्वा भर्तव्या मनुरब्रवीत्}
{नाहं प्रतीच्छे विप्रर्षे त्वया दत्तं प्रतिग्रहम्}% ३५

\onelineshloka
{न च मे भवता कोपः कार्यो वै सुरपूजित}% ३६

\uvacha{अगस्त्य उवाच}

\twolineshloka
{न च प्रतिग्रहे दोषो गृहीते पार्थिवैर्नृप}
{भवान्वै तारणे शक्तस्त्रैलोक्यस्यापि राघव}% ३७

\twolineshloka
{तारय ब्राह्मणं राम विशेषेण तपस्विनम्}
{तस्मात्प्रदास्ये विधिवत्प्रतीच्छस्व नराधिप}% ३८

\uvacha{राम उवाच}

\twolineshloka
{क्षत्रियेण कथं विप्र प्रतिग्राह्यं विजानता}
{ब्राह्मणेन तु यद्दत्तं तन्मे त्वं वक्तुमर्हसि}% ३९

\uvacha{अगस्त्य उवाच}

\twolineshloka
{आसीत्कृतयुगे राम ब्रह्मपूते पुरातने}
{अपार्थिवाः प्रजाः सर्वाः सुराणां च शतक्रतुः}% ४०

\twolineshloka
{ताः प्रजा देवदेवेशं राजार्थं समुपागमन्}
{सुराणां विद्यते राजा देवदेवः शतक्रतुः}% ४१

\twolineshloka
{श्रेयसेस्मासु लोकेश पार्थिवं कुरु साम्प्रतम्}
{यस्मिन्पूजां प्रयुञ्जानाः पुरुषा भुञ्जते महीम्}% ४२

\twolineshloka
{ततो ब्रह्मा सुरश्रेष्ठो लोकपालान्सवासवान्}
{समाहूयाब्रवीत्सर्वांस्तेजोभागोऽत्र युज्यताम्}% ४३

\twolineshloka
{ततो ददुर्लोकपालाश्चतुर्भागं स्वतेजसा}
{अक्षयश्च ततो ब्रह्मा यतो जातोऽक्षयो नृपः}% ४४

\twolineshloka
{तं ब्रह्मा लोकपालानामंशं पुंसामयोजयत्}
{ततो नृपस्तदा तासां प्रजानां क्षेमपण्डितः}% ४५

\twolineshloka
{तत्रैन्द्रेण तु भागेन सर्वानाज्ञापयेन्नृपः}
{वारुणेन च भागेन सर्वान्पुष्णाति देहिनः}% ४६

\twolineshloka
{कौबेरेण तथांशेन त्वर्थान्दिशति पार्थिवः}
{यश्च याम्यो नृपे भागस्तेन शास्ति च वै प्रजाः}% ४७

\twolineshloka
{तत्र चैन्द्रेण भागेन नरेन्द्रोसि रघूत्तम}
{प्रतिगृह्णीष्वाभरणं तारणार्थे मम प्रभो}% ४८

\twolineshloka
{ततो रामः प्रजग्राह मुनेर्हस्तान्महात्मनः}
{दिव्यमाभरणं चित्रं प्रदीप्तमिव भास्करम्}% ४९

\twolineshloka
{प्रतिगृह्य ततोगस्त्याद्राघवः परवीरहा}
{निरीक्ष्य सुचिरं कालं विचार्य च पुनः पुनः1.36.}% ५०

\twolineshloka
{मौक्तिकानि विचित्राणि धात्रीफलसमानि च}
{जाम्बूनदनिबद्धानि वज्रविद्रुमनीलकैः}% ५१

\twolineshloka
{पद्मरागैः सगोमेधैर्वैडूर्यैः पुष्परागकैः}
{सुनिबद्धं सुविभक्तं सुकृतं विश्वकर्मणा}% ५२

\twolineshloka
{दृष्ट्वा प्रीतिसमायुक्तो भूयश्चेदं व्यचिन्तयत्}
{नेदृशानि च रत्नानि मया दृष्टानि कानिचित्}% ५३

\twolineshloka
{उपशोभानि बद्धानि पृथ्वीमूल्यसमानि च}
{विभीषणस्य लङ्कायां न दृष्टानि मया पुरा}% ५४

\twolineshloka
{इति सञ्चित्य मनसा राघवस्तमृषिं पुनः}
{आगमं तस्य दिव्यस्य प्रष्टुं समुपचक्रमे}% ५५

\twolineshloka
{अत्यद्भुतमिदं ब्रह्मन्न प्राप्यं च महीक्षिताम्}
{कथं भगवता प्राप्तं कुतो वा केन निर्मितम्}% ५६

\twolineshloka
{कुतूहलवशाच्चैव पृच्छामि त्वां महामते}
{करतलेस्थिते रत्ने करमध्यं प्रकाशते}% ५७

\twolineshloka
{अधमं तद्विजानीयात्सर्वशास्त्रेषु गर्हितम्}
{दिशः प्रकाशयेद्यत्तन्मध्यमं मुनिसत्तम}% ५८

\twolineshloka
{ऊर्ध्वगं त्रिशिखं यत्स्यादुत्तमं तदुदाहृतम्}
{एतान्युत्तमजातीनि ऋषिभिः कीर्तितानि तु}% ५९

\twolineshloka
{आश्चर्याणां बहूनां हि दिव्यानां भगवान्निधिः}
{एवं वदति काकुत्स्थे मुनिर्वाक्यमथाब्रवीत्}% ६०

\uvacha{अगस्त्य उवाच}

\twolineshloka
{शृणु राम पुरावृत्तं पुरा त्रेतायुगे महत्}
{द्वापरे समनुप्राप्ते वने यद्दृष्टवानहम्}% ६१

\twolineshloka
{आश्चर्यं सुमहाबाहो निबोध रघुनन्दन}
{पुरा त्रेतायुगे ह्यासीदरण्यं बहुविस्तरम्}% ६२

\twolineshloka
{समन्ताद्योजनशतं मृगव्याघ्रविवर्जितम्}
{तस्मिन्निष्पुरुषेऽरण्ये चिकीर्षुस्तप उत्तमम्}% ६३

\twolineshloka
{अहमाक्रमितुं सौम्य तदरण्यमुपागतः}
{तस्यारण्यस्य मध्यं तु युक्तं मूलफलैः सदा}% ६४

\twolineshloka
{शाकैर्बहुविधाकारैर्नानारूपैः सुकाननैः}
{तस्यारण्यस्य मध्ये तु पञ्चयोजनमायतम्}% ६५

\twolineshloka
{हंसकारण्डवाकीर्णं चक्रवाकोपशोभितम्}
{तत्राश्चर्यं मया दृष्टं सरः परमशोभितम्}% ६६

\twolineshloka
{विसारिकच्छपाकीर्णं बकपङ्क्तिगणैर्युतम्}
{समीपे तस्य सरसस्तपस्तप्तुं गतः पुरा}% ६७

\twolineshloka
{देशं पुण्यमुपेत्यैवं सर्वहिंसाविवर्जितम्}
{तत्राहमवसं रात्रिं नैदाघीं पुरुषर्षभ}% ६८

\twolineshloka
{प्रभाते पुरुत्थाय सरस्तदुपचक्रमे}
{अथापश्यं शवमहमस्पृष्टजरसं क्वचित्}% ६९

\twolineshloka
{तिष्ठन्तं परया लक्ष्म्या सरसो नातिदूरतः}
{तदर्थं चिन्तयानोहं मुहूर्तमिव राघव}% ७०

\twolineshloka
{अस्य तीरे न वै प्राणी को वाप्येष सुरर्षभः}
{मुनिर्वा पार्थिवो वापि क्व मुनिः पार्थिवोपि वा}% ७१

\twolineshloka
{अथवा पार्थिवसुतस्तस्यैवं सम्भवः कृतः}
{अतीतेहनि रात्रौ वा प्रातर्वापि मृतो यदि}% ७२

\twolineshloka
{अवश्यं तु मया ज्ञेया सरसोस्य विनिष्क्रिया}
{यावदेवं स्थितश्चाहं चिन्तयानो रघूत्तम}% ७३

\twolineshloka
{अथापश्यं मूहूर्तात्तु दिव्यमद्भुतदर्शनम्}
{विमानं परमोदारं हंसयुक्तं मनोजवम्}% ७४

\twolineshloka
{पुरस्तत्र सहस्रं तु विमानेप्सरसां नृप}
{गन्धर्वाश्चैव तत्सङ्ख्या रमयन्ति वरं नरम्}% ७५

\twolineshloka
{गायन्ति दिव्यगेयानि वादयन्ति तथा परे}
{अथापश्यं नरं तस्माद्विमानादवरोहितम्}% ७६

\twolineshloka
{शवमांसं भक्षयन्तं च स्नात्वा रघुकुलोद्वह}
{ततो भुक्त्वा यथाकामं स मांसं बहुपीवरम्}% ७७

\twolineshloka
{अवतीर्य सरः शीघ्रमारुरोह दिवं पुनः}
{तमहं देवसङ्काशं श्रिया परमयान्वितम्}% ७८

\twolineshloka
{भो भो स्वर्गिन्महाभाग पृच्छामि त्वां कथं त्विदम्}
{जुगुप्सितस्तवाहारो गतिश्चेयं तवोत्तमा}% ७९

\twolineshloka
{यदि गुह्यं न चैतत्ते कथय त्वद्य मे भवान्}
{कामतः श्रोतुमिच्छामि किमेतत्परमं वचः}% ८०

\twolineshloka
{को भवान्वद सन्देहमाहारश्च विगर्हितः}
{त्वयेदं भुज्यते सौम्य किमर्थं क्व च वर्तसे}% ८१

\twolineshloka
{कस्यायमैश्वरोभावः शवत्वेन विनिर्मितः}
{आहारं च कथं निन्द्यं श्रोतुमिच्छामि तत्त्वतः}% ८२

\twolineshloka
{श्रुत्वा च भाषितं तत्र मम राम सतां वर}
{प्राञ्जलिः प्रत्युवाचेदं स स्वर्गी रघुनन्दन}% ८३

\twolineshloka
{शृणुष्वाद्य यथावृत्तं ममेदं सुखदुःखजम्}
{कामो हि दुरितक्रम्यः शृणु यत्पृच्छसे द्विज}% ८४

\twolineshloka
{पुरा वैदर्भको राजा पिता मे हि महायशाः}
{वासुदेव इति ख्यातस्त्रिषु लोकेषु धार्मिकः}% ८५

\twolineshloka
{तस्य पुत्रद्वयं ब्रह्मन्द्वाभ्यां स्त्रीभ्यामजायत}
{अहं श्वेत इति ख्यातो यवीयान्सुरथोऽभवत्}% ८६

\twolineshloka
{पितर्युपरते तस्मिन्पौरा मामभ्यषेचयन्}
{तत्राहङ्कारयन्राज्यं धर्मे चासं समाहितः}% ८७

\twolineshloka
{एवं वर्षसहस्राणि बहूनि समुपाव्रजन्}
{मम राज्यं कारयतः परिपालयतः प्रजाः}% ८८

\twolineshloka
{सोहं निमित्ते कस्मिंश्चिद्वैराग्येण द्विजोत्तम}
{मरणं हृदये कृत्वा तपोवनमुपागमम्}% ८९

\twolineshloka
{सोहं वनमिदं रम्यं भृशं पक्षिविवर्जितम्}
{प्रविष्टस्तप आस्थातुमस्यैव सरसोन्तिके}% ९०

\twolineshloka
{राज्येऽभिषिच्य सुरथं भ्रातरं तं नराधिपम्}
{इदं सरः समासाद्य तपस्तप्तं सुदारुणम्}% ९१

\twolineshloka
{दशवर्षसहस्राणि तपस्तप्त्वा महावने}
{शुभं तु भवनं प्राप्तो ब्रह्मलोकमनामयम्}% ९२

\twolineshloka
{स्वर्गस्थमपि मां ब्रह्मन्क्षुत्पिपासे द्विजोत्तम}
{अबाधेतां भृशं चाहमभवं व्यथितेन्द्रियः}% ९३

\twolineshloka
{ततस्त्रिभुवनश्रेष्ठमवोचं वै पितामहम्}
{भगवन्स्वर्गलोकोऽयं क्षुत्पिपासा विवर्जितः}% ९४

\twolineshloka
{कस्येयं कर्मणः पक्तिः क्षुत्पिपासे यतो हि मे}
{आहारः कश्च मे देव ब्रूहि त्वं श्रीपितामह}% ९५

\twolineshloka
{ततः पितामहः सम्यक्चिरं ध्यात्वा महामुने}
{मामुवाच ततो वाक्यं नास्ति भोज्यं स्वदेहजम्}% ९६

\twolineshloka
{ॠते ते स्वानि मांसानि भक्षय त्वं तु हि नित्यशः}
{स्वशरीरं त्वया पुष्टं कुर्वता तप उत्तमम्}% ९७

\twolineshloka
{नादत्तं जायते तात श्वेत पश्य महीतले}
{आग्रहाद्भिक्षमाणाय भिक्षापि प्राणिने पुरा}% ९८

\twolineshloka
{न हि दत्ता गृहे भ्रान्त्या मोहादतिथये तदा}
{तेन स्वर्गगतस्यापि क्षुत्पिपासे तवाधुना}% ९९

\twolineshloka
{स त्वं प्रपुष्टमाहारैः स्वशरीरमनुत्तमम्}
{भक्षयस्व च राजेन्द्र सा ते तृप्तिर्भविष्यति1.36.}% १००

\twolineshloka
{एवमुक्तस्ततो देवं ब्रह्माणमहमुक्तवान्}
{भक्षिते च स्वके देहे पुनरन्यन्न मे विभो}% १०१

\twolineshloka
{क्षुधानिवारणं नैव देहस्यास्य विनौदनम्}
{खादामि ह्यक्षयं देव प्रियं मे न हि जायते}% १०२

\twolineshloka
{ततोब्रवीत्पुनर्ब्रह्मा तव देहोऽक्षयः कृतः}
{दिनेदिने ते पुष्टात्मा शवः श्वेत भविष्यति}% १०३

\twolineshloka
{यावद्वर्षशतं पूर्णं स्वमांसं खाद भो नृप}
{यदागच्छति चागस्त्यः श्वेतारण्यं महातपाः}% १०४

\twolineshloka
{भगवानतिदुर्धर्षस्तदा कृच्छ्राद्विमोक्ष्यसे}
{स हि तारयितुं शक्तः सेन्द्रानपि सुरासुरान्}% १०५

\twolineshloka
{आहारं कुत्सितं चेमं राजर्षे किं पुनस्तव}
{सुरकार्यं महत्तेन सुकृतं तु महात्मना}% १०६

\twolineshloka
{उदधिं निर्जलं कृत्वा दानवाश्च निपातिताः}
{विन्ध्यश्चादित्यविद्वेषाद्वर्धमानो निवारितः}% १०७

\twolineshloka
{लम्बमाना मही चैषा गुरुत्वेनाधिवासिता}
{दक्षिणा दिग्दिवं याता त्रैलाक्यं विषमस्थितम्}% १०८

\twolineshloka
{मया गत्वा सुरैः सार्द्धं प्रेषितो दक्षिणां दिशम्}
{समां कुरु महाभाग गुरुत्वेन जगत्समम्}% १०९

\twolineshloka
{एवं च तेन मुनिना स्थित्वा सर्वा धरा समा}
{कृता राजेन्द्र मुनिना एवमद्यापि दृश्यते}% ११०

\twolineshloka
{सोहं भगवत श्रुत्वा देवदेवस्य भाषितम्}
{भुञ्जे च कुत्सिताहारं स्वशरीरमनुत्तमम्}% १११

\twolineshloka
{पूर्णं वर्षशतं चाद्य भोजनं कुत्सितं च मे}
{क्षयं नाभ्येति तद्विप्र तृप्तिश्चापि ममोत्तमा}% ११२

\twolineshloka
{तं मुनिं कृच्छ्रसन्तप्तश्चिन्तयामि दिवानिशम्}
{कदा वै दर्शनं मह्यं स मुनिर्दास्यते वने}% ११३

\twolineshloka
{एवं मे चिन्तयानस्य गतं वर्षशतन्त्विह}
{सोगस्त्यो हि गतिर्ब्रह्मन्मुनिर्मे भविता ध्रुवम्}% ११४

\twolineshloka
{न गतिर्भविता मह्यं कुम्भयोनिमृते द्विजम्}
{श्रुत्वेत्थं भाषितं राम दृष्ट्वाहारं च कुत्सितम्}% ११५

\twolineshloka
{कृपया परया युक्तस्तं नृपं स्वर्गगामिनम्}
{करोम्यहं सुधाभोज्यं नाशयामि च कुत्सितम्}% ११६

\twolineshloka
{चिन्तयन्नित्यवोचं तमगस्त्यः किं करिष्यति}
{अहमेतत्कुत्सितं ते नाशयामि महामते}% ११७

\twolineshloka
{ईप्सितं प्रार्थयस्वास्मान्मनः प्रीतिकरं परम्}
{स स्वर्गी मां ततः प्राह कथं ब्रह्मवचोन्यथा}% ११८

\twolineshloka
{कर्तुं मुने मया शक्यं न चान्यस्तारयिष्यति}
{ॠते वै कुम्भयोनिं तं मैत्रावरुणसम्भवम्}% ११९

\twolineshloka
{अपृष्टोपि मया ब्रह्मन्नेवमूचे पितामहः}
{एवं ब्रुवाणं तं श्वेतमुक्तवानहमस्मि सः}% १२०

\twolineshloka
{आगतस्तव भाग्येन दृष्टोहं नात्र संशयः}
{ततः स्वर्गी स मां ज्ञात्वा दण्डवत्पतितो भुवि}% १२१

\onelineshloka*
{तमुत्थाप्य ततो रामाब्रवं किं ते करोम्यहम्}

\uvacha{राजोवाच}

\onelineshloka
{आहारात्कुत्सिताद्ब्रह्मंस्तारयस्वाद्य दुष्कृतात्}% १२२

\twolineshloka
{येन लोकोऽक्षयः स्वर्गो भविता त्वत्कृतेन मे}
{ततः प्रतिग्रहो दत्तो जगद्वन्द्य नृपेण हि}% १२३

%%% CHECK FOR MISSING SHLOKAS!?

\onelineshloka
{भवान्मामनुगृह्णातु प्रतीच्छस्व प्रतिग्रहम्}% १२७

\twolineshloka
{कृता मतिस्तारणाय न लोभाद्रघुनन्दन}
{गृहीते भूषणे राम मम हस्तगते तदा}% १२८

\twolineshloka
{मानुषः पौर्विको देहस्तदा नष्टोस्य भूपते}
{प्रणष्टे तु शरीरे च राजर्षिः परया मुदा}% १२९

\twolineshloka
{मयोक्तोसौ विमानेन जगाम त्रिदिवं पुनः}
{तेन मे शक्रतुल्येन दत्तमाभरणं शुभम्}% १३०

\twolineshloka
{तस्मिन्निमित्ते काकुत्स्थ दत्तमद्भुतकर्मणा}
{श्वेतो वैदर्भको राजा तदाभूद्गतकल्मषः}% १३१

{॥इति श्रीपाद्मपुराणे प्रथमे सृष्टिखण्डे रामागस्त्यसंवादो नाम षट्त्रिंशोऽध्यायः॥३६॥}
