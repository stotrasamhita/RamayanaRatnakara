\chapt{रामहृदये रामचरितम्}

\src{अध्यात्म-रामायणम्}{बालकाण्डः}{अध्यायः १}{श्लोकाः ३२--४३}
\vakta{व्यासः}
\shrota{जनमेजयः}
\notes{}
\textlink{https://sa.wikisource.org/wiki/देवीभागवतपुराणम्/स्कन्धः_०३/अध्यायः_३०}
\translink{}

\storymeta


\uvacha{सीतोवाच}

\addtocounter{shlokacount}{31}

\twolineshloka
{रामं विद्धि परं ब्रह्म सच्चिदानन्दमद्वयम्}
{सर्वोपाधिविनिर्मुक्तं सत्तामात्रमगोचरम्} %1-32

\twolineshloka
{आनन्दं निर्मलं शान्तं निर्विकारं निरञ्जनम्}
{सर्वव्यापिनमात्मानं स्वप्रकाशमकल्मषम्} %1-33

\twolineshloka
{मां विद्धि मूलप्रकृतिं सर्गस्थित्यन्तकारिणीम्}
{तस्य सन्निधिमात्रेण सृजामीदमतन्द्रिता} %1-34

\twolineshloka
{तत्सान्निध्यान्मया सृष्टं तस्मिन्नारोप्यतेऽबुधैः}
{अयोध्यानगरे जन्म रघुवंशेऽतिनिर्मले} %1-35

\twolineshloka
{विश्वामित्रसहायत्वं मखसंरक्षणं ततः}
{अहल्याशापशमनं चापभङ्गो महेशितुः} %1-36

\twolineshloka
{मत्पाणिग्रहणं पश्चाद्भार्गवस्य मदक्षयः}
{अयोध्यानगरे वासो मया द्वादशवार्षिकः} %1-37

\twolineshloka
{दण्डकारण्यगमनं विराधवध एव च}
{मायामारीचमरणं मायासीताहृतिस्तथा} %1-38

\twolineshloka
{जटायुषो मोक्षलाभः कबन्धस्य तथैव च}
{शबर्याः पूजनं पश्चात्सुग्रीवेण समागमः} %1-39

\twolineshloka
{वालिनश्च वधः पश्चात्सीतान्वेषणमेव च}
{सेतुबन्धश्च जलधौ लङ्कायाश्च निरोधनम्} %1-40

\twolineshloka
{रावणस्य वधो युद्धे सपुत्रस्य दुरात्मनः}
{विभीषणे राज्यदानं पुष्पकेण मया सह} %1-41

\threelineshloka
{अयोध्यागमनं पश्चाद्राज्ये रामाभिषेचनम्}
{एवमादीनि कर्माणि मयैवाचरितान्यपि}
{आरोपयन्ति रामेऽस्मिन्निर्विकारेऽखिलात्मनि} %1-42

\fourlineindentedshloka
{रामो न गच्छति न तिष्ठति नानुशोच-}
{त्याकाङ्क्षते त्यजति नो न करोति किञ्चित्}
{आनन्दमूर्तिरचलः परिणामहीनो}
{मायागुणाननुगतो हि तथा विभाति} %1-43

॥इति श्रीमदध्यात्मरामायणे उमामहेश्वरसंवादे बालकाण्डे रामहृदयं नाम प्रथमे सर्गे रामचरितं सम्पूर्णम्॥

\closesection