\sect{रामकथाकथनम्}

\src{अध्यात्म-रामायणम्}{सुन्दरकाण्डः}{अध्यायः ३}{श्लोकाः १--३६}
\vakta{हनुमान्}
\shrota{सीता}
\notes{}
\textlink{}
\translink{}

\storymeta

\uvacha{श्री-महादेव उवाच}

\twolineshloka
{उद्बन्धनेन वा मोक्ष्ये शरीरं राघवं विना}
{जीवितेन फलं किं स्यान्मम रक्षोऽधिमध्यतः} %3-1

\twolineshloka
{दीर्घा वेणी ममात्यर्थमुद्बन्धाय भविष्यति}
{एवं निश्चितबुद्धिं तां मरणायाथ जानकीम्} %3-2

\twolineshloka
{विलोक्य हनुमान् किञ्चिद्विचार्यैतदभाषत}
{शनैः शनैः सूक्ष्मरूपो जानक्याः श्रोत्रगं वचः} %3-3

\twolineshloka
{इक्ष्वाकुवंशसम्भूतो राजा दशरथो महान्}
{अयोध्याधिपतिस्तस्य चत्वारो लोकविश्रुताः} %3-4

\twolineshloka
{पुत्रा देवसमाः सर्वे लक्षणैरुपलक्षिताः}
{रामश्च लक्ष्मणश्चैव भरतश्चैव शत्रुहा} %3-5

\twolineshloka
{ज्येष्ठो रामः पितुर्वाक्याद्दण्डकारण्यमागतः}
{लक्ष्मणेन सह भ्रात्रा सीतया भार्यया सह} %3-6

\twolineshloka
{उवास गौतमीतीरे पञ्चवट्यां महामनाः}
{तत्र नीता महाभागा सीता जनकनन्दिनी} %3-7

\twolineshloka
{रहिते रामचन्द्रेण रावणेन दुरात्मना}
{ततो रामोऽतिदुःखार्तो मार्गमाणोऽथ जानकीम्} %3-8

\twolineshloka
{जटायुषं पक्षिराजमपश्यत्पतितं भुवि}
{तस्मै दत्त्वा दिवं शीघ्रमृष्यमूकमुपागमत्} %3-9

\twolineshloka
{सुग्रीवेण कृता मैत्री रामस्य विदितात्मनः}
{तद्भार्याहारिणं हत्वा वालिनं रघुनन्दनः} %3-10

\twolineshloka
{राज्येऽभिषिच्य सुग्रीवं मित्रकार्यं चकार सः}
{सुग्रीवस्तु समानाय्य वानरान् वानरप्रभुः} %3-11

\twolineshloka
{प्रेषयामास परितो वानरान् परिमार्गणे}
{सीतायास्तत्र चैकोऽहं सुग्रीवसचिवो हरिः} %3-12

\twolineshloka
{सम्पातिवचनाच्छीघ्रमुल्लङ्घ्य शतयोजनम्}
{समुद्रं नगरीं लङ्कां विचिन्वन् जानकीं शुभाम्} %3-13

\twolineshloka
{शनैरशोकवनिकां विचिन्वन् शिंशपातरुम्}
{अद्राक्षं जानकीमत्र शोचन्तीं दुःखसम्प्लुताम्} %3-14

\twolineshloka
{रामस्य महिषीं देवीं कृतकृत्योऽहमागतः}
{इत्युक्त्वोपररामाथ मारुतिर्बुद्धिमत्तरः} %3-15

\twolineshloka
{सीता क्रमेण तत्सर्वं श्रुत्वा विस्मयमाययौ}
{किमिदं मे श्रुतं व्योम्नि वायुना समुदीरितम्} %3-16

\twolineshloka
{स्वप्नो वा मे मनोभ्रान्तिर्यदि वा सत्यमेव तत्}
{निद्रा मे नास्ति दुःखेन जानाम्येतत्कुतो भ्रमः} %3-17

\twolineshloka
{येन मे कर्णपीयुषं वचनं समुदीरितम्}
{स दृश्यतां महाभागः प्रियवादी ममाग्रतः} %3-18

\twolineshloka
{श्रुत्वा तज्जानकीवाक्यं हनुमान् पत्रखण्डतः}
{अवतीर्य शनैः सीतापुरतः समवस्थितः} %3-19

\twolineshloka
{कलविङ्कप्रमाणाङ्गो रक्तास्यः पीतवानरः}
{ननाम शनकैः सीतां प्राञ्जलिः पुरतः स्थितः} %3-20

\twolineshloka
{दृष्ट्वा तं जानकी भीता रावणोऽयमुपागतः}
{मां मोहयितुमायातो मायया वानराकृतिः} %3-21

\twolineshloka
{इत्येवं चिन्तयित्वा सा तूष्णीमासीदधोमुखी}
{पुनरप्याह तां सीतां देवि यत्त्वं विशङ्कसे} %3-22

\twolineshloka
{नाहं तथाविधो मातस्त्यज शङ्कां मयि स्थिताम्}
{दासोऽहं कोसलेन्द्रस्य रामस्य परमात्मनः} %3-23

\twolineshloka
{सचिवोऽहं हरीन्द्रस्य सुग्रीवस्य शुभप्रदे}
{वायोः पुत्रोऽहमखिलप्राणभूतस्य शोभने} %3-24

\twolineshloka
{तच्छ्रुत्वा जानकी प्राह हनूमन्तं कृताञ्जलिम्}
{वानराणां मनुष्याणां सङ्गतिर्घटते कथम्} %3-25

\twolineshloka
{यथा त्वं रामचन्द्रस्य दासोऽहमिति भाषसे}
{तामाह मारुतिः प्रीतो जानकीं पुरतः स्थितः} %3-26

\twolineshloka
{ऋष्यमूकमगाद्रामः शबर्या नोदितः सुधीः}
{सुग्रीवो ऋष्यमूकस्थो दृष्टवान् रामलक्ष्मणौ} %3-27

\twolineshloka
{भीतो मां प्रेषयामास ज्ञातुं रामस्य हृद्गतम्}
{ब्रह्मचारिवपुर्धृत्वा गतोऽहं रामसन्निधिम्} %3-28

\twolineshloka
{ज्ञात्वा रामस्य सद्भावं स्कन्धोपरि निधाय तौ}
{नीत्वा सुग्रीवसामीप्यं सख्यं चाकरवं तयोः} %3-29

\twolineshloka
{सुग्रीवस्य हृता भार्या वालिना तं रघूत्तमः}
{जघानैकेन बाणेन ततो राज्येऽभ्यषेचयत्} %3-30

\twolineshloka
{सुग्रीवं वानराणां स प्रेषयामास वानरान्}
{दिग्भ्यो महाबलान् वीरान् भवत्याः परिमार्गणे} %3-31

\onelineshloka
{गच्छन्तं राघवो दृष्ट्वा मामभाषत सादरम्} %3-32

\twolineshloka
{त्वयि कार्यमशेषं मे स्थितं मारुतनन्दन}
{ब्रूहि मे कुशलं सर्वं सीतायै लक्ष्मणस्य च} %3-33

\twolineshloka
{अङ्गुलीयकमेतन्मे परिज्ञानार्थमुत्तमम्}
{सीतायै दीयतां साधु मन्नामाक्षरमुद्रितम्} %3-34

\twolineshloka
{इत्युक्त्वा प्रददौ मह्यं कराग्रादङ्गुलीयकम्}
{प्रयत्नेन मयाऽऽनीतं देवि पश्याङ्गुलीयकम्} %3-35

\twolineshloka
{इत्युक्त्वा प्रददौ देव्यै मुद्रिकां मारुतात्मजः}
{नमस्कृत्य स्थितो दूराद्बद्धाञ्जलिपुटो हरिः} %3-36

॥इति श्रीमदध्यात्मरामायणे उमामहेश्वरसंवादे सुन्दरकाण्डे तृतीये सर्गे रामकथाकथनं सम्पूर्णम्॥

\closesection