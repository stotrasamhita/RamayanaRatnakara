\chapt{श्रीमन्नारायणीयम् --- श्रीरामचरितवर्णनम्}

\sect{दशकं ३४}

\src{श्रीमन्नारायणीयम्}{चतुस्त्रिंश-दशकं}{}{श्लोकाः १---10}
\vakta{शुकः}
\shrota{परीक्षितः}
\tags{concise, complete}
\notes{This chapter recounts the appearance of Lord Rāmacandra in the lineage of Mahārāja Khaṭvāṅga and details His divine exploits, including the slaying of Rāvaṇa and His triumphant return to Ayodhyā.}
\textlink{http://stotrasamhita.net/wiki/Narayaniyam/Dashaka_34}
\translink{}

\storymeta

\fourlineindentedshloka
{गीर्वाणैरर्थ्यमानो दशमुखनिधनं कोसलेऽष्वृश्यषृङ्गे}
{पुत्रीयामिष्टिमिष्ट्वा ददुषि दशरथक्ष्माभृते पायसाग्र्यम्}
{तद्भुक्त्या तत्पुरन्ध्रीष्वपि तिसृषु समं जातगर्भासु जातो}
{रामस्त्वं लक्ष्मणेन स्वयमथ भरतेनापि शत्रुघ्ननाम्ना} % ॥१॥

\fourlineindentedshloka
{कोदण्डी कौशिकस्य क्रतुवरमवितुं लक्ष्मणेनानुयातो}
{यातोऽभूस्तातवाचा मुनिकथितमनुद्वन्द्वशान्ताध्वखेदः}
{नॄणां त्राणाय बाणैर्मुनिवचनबलात्ताटकां पाटयित्वा}
{लब्ध्वास्मादस्त्रजालं मुनिवनमगमो देव सिद्धाश्रमाख्यम्} % ॥२॥

\fourlineindentedshloka
{मारीचं द्रावयित्वा मखशिरसि शरैरन्यरक्षांसि निघ्नन्}
{कल्यां कुर्वन्नहल्यां पथि पदरजसा प्राप्य वैदेहगेहम्}
{भिन्दानश्चान्द्रचूडं धनुरवनिसुतामिन्दिरामेव लब्ध्वा}
{राज्यं प्रातिष्ठथास्त्वं त्रिभिरपि च समं भ्रातृवीरैः सदारैः} % ॥३॥

\fourlineindentedshloka
{आरुन्धाने रुषान्धे भृगुकुलतिलके सङ्क्रमय्य स्वतेजो}
{याते यातोऽस्ययोध्यां सुखमिह निवसन्कान्तया कान्तमूर्ते}
{शत्रुघ्नेनैकदाथो गतवति भरते मातुलस्याधिवासम्}
{तातारब्धोऽभिषेकस्तव किल विहतः केकयाधीशपुत्र्या} % ॥४॥

\fourlineindentedshloka
{तातोक्त्या यातुकामो वनमनुजवधूसंयुतश्चापधारः}
{पौरानारूध्य मार्गे गुहनिलयगतस्त्वं जटाचीरधारी}
{नावा सन्तीर्य गङ्गामधिपदवि पुनस्तं भरद्वाजमारा-}
{न्नत्वा तद्वाक्यहेतोरतिसुखमवसश्चित्रकूटे गिरीन्द्रे} % ॥५॥

\fourlineindentedshloka
{श्रुत्वा पुत्रार्तिखिन्नं खलु भरतमुखात् स्वर्गयातं स्वतातम्}
{तप्तो दत्त्वाम्बु तस्मै निदधिथ भरते पादुकां मेदिनीं च}
{अत्रिं नत्वाथ गत्वा वनमतिविपुलां दण्डकां चण्डकायम्}
{हत्वा दैत्यं विराधं सुगतिमकलयश्चारु भोः शारभङ्गीम्} % ॥६॥

\fourlineindentedshloka
{नत्वाऽगस्त्यं समस्ताशरनिकरसपत्राकृतिं तापसेभ्यः}
{प्रत्यश्रौषीः प्रियैषी तदनु च मुनिना वैष्णवे दिव्यचापे}
{ब्रह्मास्त्रे चापि दत्ते पथि पितृसुहृदं दीक्ष्य जटायुम्}
{मोदाद्गोदातटान्ते परिरमसि पुरा पञ्चवट्यां वधूट्या} % ॥७॥

\fourlineindentedshloka
{प्राप्तायाः शूर्पणख्या मदनचलधृतेरर्थनैर्निस्सहात्मा}
{तां सौमित्रौ विसृज्य प्रबलतमरुषा तेन निर्लुननासाम्}
{दृष्ट्वैनां रुष्टचित्तं खरमभिपतितं दुषणं च त्रिमूर्धम्}
{व्याहिंसीराशरानप्ययुतसमधिकांस्तत्क्षणादक्षतोष्मा} % ॥८॥

\fourlineindentedshloka
{सोदर्याप्रोक्तवार्ताविवशदशमुखादिष्टमारीचमाया-}
{सारङ्गं सारसाक्ष्या स्पृहितमनुगतः प्रावधीर्बाणघातम्}
{तन्मायाक्रन्दनिर्यापितभवदनुजां रावणस्तामहार्षीत्}
{तेनार्तोऽपि त्वमन्तः किमपि मुदमधास्तद्वधोपायायलाभात्} % ॥९॥

\fourlineindentedshloka
{भूयस्तन्वीं विचिन्वन्नहृत दशमुखस्त्वद्वधूं मद्वधेने-}
{त्युक्त्वा याते जटायौ दिवमथ सुहृदः प्रातनोः प्रेतकार्यम्}
{गृह्णानं तं कबन्धं जघनिथ शबरीं प्रेक्ष्य पम्पातटे त्वम्}
{सम्प्राप्तो वातसूनुं भृशमुदितमनाः पाहि वातालयेश} % ॥१०॥

॥इति श्रीमन्नारायणीये श्रीरामचरितवर्णनं नाम चतुस्त्रिंश-दशकं सम्पूर्णम्॥

\closesection