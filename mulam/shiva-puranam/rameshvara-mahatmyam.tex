\chapt{शिव-पुराणम्}

\sect{रामेश्वरमाहात्म्यम्}

\src{शिव-पुराणम्}{पूर्वखण्डः}{अध्यायः ३१}{श्लोकाः १---४५}
\tags{concise, complete}
\notes{Summary of Ramayana.}
\textlink{}
\translink{}

\storymeta

\uvacha{सूत उवाच}

\twolineshloka
{अतः परं प्रवक्ष्यामि लिङ्गं रामेश्वराभिधम्} 
{उत्पन्नं च यथा पूर्वमृषयश्शृणुतादरात्}

\onelineshloka
{पुरा विष्णुः पृथिव्यां चावततार सतां प्रियः} %॥२॥

\twolineshloka
{तत्र सीता हृता विप्रा रावणेनोरुमायिना} 
{प्रापिता स्वगृहं सा हि लङ्कायां जनकात्मजा} %॥३॥

\twolineshloka
{अन्वेषणपरस्तस्याः किष्किन्धाख्यां पुरीमगात्} 
{सुग्रीवहितकृद्भूत्वा वालिनं सञ्जघान ह} %॥४॥

\twolineshloka
{तत्र स्थित्वा कियत्कालं तदन्वेषणतत्परः} 
{सुग्रीवाद्यैर्लक्ष्मणेन विचारं कृतवान्स वै} %॥५॥

\twolineshloka
{कपीन्सम्प्रेषयामास चतुर्दिक्षु नृपात्मजः} 
{हनुमत्प्रमुखान्रामस्तदन्वेषणहेतवे} %॥६॥

\twolineshloka
{अथ ज्ञात्वा गतां लङ्कां सीतां कपिवराननात्} 
{सीताचूडामणिं प्राप्य मुमुदे सोऽति राघवः} %॥७॥

\twolineshloka
{सकपीशस्तदा रामो लक्ष्मणेन युतो द्विजाः} 
{सुग्रीवप्रमुखैः पुण्यैर्वानरैर्बलवत्तरैः} %॥८॥

\twolineshloka
{पद्मैरष्टादशाख्यैश्च ययौ तीरं पयोनिधेः}
{दक्षिणे सागरे यो वै दृश्यते लवणाकरः}

\twolineshloka
{तत्रागत्य स्वयं रामो वेलायां संस्थितो हि सः} 
{वानरैस्सेव्यमानस्तु लक्ष्मणेन शिवप्रियः} %॥4॥

\twolineshloka
{हा जानकि कुतो याता कदा चेयं मिलिष्यति} 
{अगाधस्सागरश्चैवातार्या सेना च वानरी} %॥११॥

\twolineshloka
{राक्षसो गिरिधर्त्ता च महाबलपराक्रमः}
{लङ्काख्यो दुर्गमो दुर्ग इन्द्रजित्तनयोस्य वै} %॥१२॥

\twolineshloka
{इत्येवं स विचार्यैव तटे स्थित्वा सलक्ष्मणः}
{आश्वासितो वनौकोभिरङ्गदादिपुरस्सरैः} %॥१३॥

\twolineshloka
{एतस्मिन्नन्तरे तत्र राघवश्शैवसत्तमः}
{उवाच भ्रातरं प्रीत्या जलार्थी लक्ष्मणाभिधम्} %॥१४॥

\uvacha{राम उवाच}

\twolineshloka
{भ्रातर्लक्ष्मण वीरेशाहं जलार्थी पिपासितः}
{तदानय द्रुतं पाथो वानरैः कैश्चिदेव हि} %॥१५॥

\uvacha{सूत उवाच}

\twolineshloka
{तच्छ्रुत्वा वानरास्तत्र ह्यधावन्त दिशो दश} 
{नीत्वा जलं च ते प्रोचुः प्रणिपत्य पुरः स्थिताः} %॥१६॥

\uvacha{वानरा ऊचुः}

\twolineshloka
{जलं च गृह्यतां स्वामिन्नानीतं तत्त्वदाज्ञया}
{महोत्तमं च सुस्वादु शीतलं प्राणतर्पणम्} %॥१७॥

\uvacha{सूत उवाच}

\twolineshloka
{सुप्रसन्नतरो भूत्वा कृपादृष्ट्या विलोक्य तान् }
{तच्छ्रुत्वा रामचन्द्रोऽसौ स्वयं जग्राह तज्जलम्} %॥१८॥

\twolineshloka
{स शैवस्तज्जलं नीत्वा पातुमारब्धवान्यदा} 
{तदा च स्मरणं जातमित्थमस्य शिवेच्छया} %॥१९॥

\twolineshloka
{न कृतं दर्शनं शम्भोर्गृह्यते च जलं कथम्}
{स्वस्वामिनः परेशस्य सर्वानन्दप्रदस्य वै} %॥२०॥

\twolineshloka
{इत्युक्त्वा च जलं पीतं तदा रघुवरेण च}
{पश्चाच्च पार्थिवीं पूजां चकार रघुनन्दनः} %॥२१॥

\twolineshloka
{आवाहनादिकांश्चैव ह्युपचारान्प्रकल्प्य वै}
{विधिवत्षोडश प्रीत्या देवमानर्च शङ्करम्} %॥२२॥

\twolineshloka
{प्रणिपातैस्स्तवैर्दिव्यैश्शिवं सन्तोष्य यत्नतः}
{प्रार्थयामास सद्भक्त्या स रामश्शङ्करं मुदा} %॥२३॥

\uvacha{राम उवाच}

\twolineshloka
{स्वामिञ्छम्भो महादेव सर्वदा भक्तवत्सल} 
{पाहि मां शरणापन्नं त्वद्भक्तं दीनमानसम्} %॥२४॥

\twolineshloka
{एतज्जलमगाधं च वारिधेर्भवतारण}
{रावणाख्यो महावीरो राक्षसो बलवत्तरः} %॥२५॥

\twolineshloka
{वानराणां बलं ह्येतच्चञ्चलं युद्धसाधनम्}
{ममकार्यं कथं सिद्धं भविष्यति प्रियाप्तये} %॥२६॥

\twolineshloka
{तस्मिन्देव त्वया कार्यं साहाय्यं मम सुव्रत} 
{साहाय्यं ते विना नाथ मम कार्य्यं हि दुर्लभम्} %॥२७॥

\twolineshloka
{त्वदीयो रावणोऽपीह दुर्ज्जयस्सर्वथाखिलैः} 
{त्वद्दत्तवरदृप्तश्च महावीरस्त्रिलोकजित्} %॥२८॥

\twolineshloka
{अप्यहं तव दासोऽस्मि त्वदधीनश्च सर्वथा} 
{विचार्येति त्वया कार्यः पक्षपातस्सदाशिव} %॥२९॥

\uvacha{सूत उवाच}

\twolineshloka
{इत्येवं स च सम्प्रार्थ्य नमस्कृत्य पुनःपुनः} 
{तदा जयजयेत्युच्चैरुद्धोषैश्शङ्करेति च} %॥4॥

\twolineshloka
{इति स्तुत्वा शिवं तत्र मन्त्रध्यानपरायणः}
{पुनः पूजां ततः कृत्वा स्वाम्यग्रे स ननर्त ह} %॥३१॥

\twolineshloka
{प्रेमी विक्लिन्नहृदयो गल्लनादं यदाकरोत्} 
{तदा च शङ्करो देवस्सुप्रसन्नो बभूव ह} %॥३२॥

\twolineshloka
{साङ्गस्सपरिवारश्च ज्योतीरूपो महेश्वरः}
{यथोक्तरूपममलं कृत्वाविरभवद्द्रुतम्} %॥३३॥

\twolineshloka
{ततस्सन्तुष्टहृदयो रामभक्त्या महेश्वरः} 
{शिवमस्तु वरं ब्रूहि रामेति स तदाब्रवीत्} %॥३४॥

\twolineshloka
{तद्रूपं च तदा दृष्ट्वा सर्वे पूतास्ततस्स्वयम्} 
{कृतवान्राघवः पूजां शिवधर्मपरायणः} %॥३५॥

\twolineshloka
{स्तुतिं च विविधां कृत्वा प्रणिपत्य शिवं मुदा} 
{जयं च प्रार्थयामास रावणाजौ तदात्मनः} %॥३६॥

\twolineshloka
{ततः प्रसन्नहृदयो रामभक्त्या महेश्वरः} 
{जयोस्तु ते महाराज प्रीत्या स पुनरब्रवीत्} %॥३७॥

\twolineshloka
{शिवदत्तं जयं प्राप्य ह्यनुज्ञां समवाप्य च} 
{पुनश्च प्रार्थयामास साञ्जलिर्नतमस्तकः} %॥३८॥

\uvacha{राम उवाच}

\twolineshloka
{त्वया स्थेयमिह स्वामिंल्लोकानां पावनाय च} 
{परेषामुपकारार्थं यदि तुष्टोऽसि शङ्कर} %॥३९॥

\uvacha{सूत उवाच}

\twolineshloka
{इत्युक्तस्तु शिवस्तत्र लिङ्गरूपोऽभवत्तदा} 
{रामेश्वरश्च नाम्ना वै प्रसिद्धो जगतीतले} %॥४०॥

\twolineshloka
{रामस्तु तत्प्रभावाद्वै सिन्धुमुत्तीर्य चाञ्जसा}
{रावणादीन्निहत्याशु राक्षसान्प्राप तां प्रियाम्} %॥४१॥

\twolineshloka
{रामेश्वरस्य महिमाद्भुतोऽभूद्भुवि चातुलः} 
{भुक्तिमुक्तिप्रदश्चैव सर्वदा भक्तकामदः} %॥४२॥

\twolineshloka
{दिव्यगङ्गाजलेनैव स्नापयिष्यति यश्शिवम्} 
{रामेश्वरं च सद्भक्त्या स जीवन्मुक्त एव हि} %॥४३॥

\twolineshloka
{इह भुक्त्वाखिलान्भोगान्देवानां दुर्लभानपि} 
{अन्ते प्राप्य परं ज्ञानं कैवल्यं प्राप्नुयाद्ध्रुवम्} %॥४४॥

\twolineshloka
{इति वश्च समाख्यातं ज्योतिर्लिगं शिवस्य तु}
{रामेश्वराभिधं दिव्यं शृण्वतां पापहारकम्} %॥४५।

॥इति श्रीशिवमहापुराणे चतुर्थ्यां कोटिरुद्रसन्तायां रामेश्वरमाहात्म्यवर्णनं नामैकत्रिंशोऽध्यायः॥

\closesection