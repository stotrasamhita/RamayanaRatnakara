\chapt{शिव-पुराणम्}

\sect{रामपरीक्षा-वर्णनम् --- चतुर्विंशोऽध्यायः}

\src{शिव-पुराणम्}{द्वितीयायां रुद्रसंहितायां}{द्वितीये सतीखण्डे}{अध्यायः २४--२५}
\tags{concise, complete}
\notes{While wandering with Satī, Śiva bows to the grief-stricken Rāma in the forest, telling Her He is Viṣṇu incarnate. Still doubtful, Satī takes the form of Sītā to test him, but Rāma immediately recognises her as Satī, proving His divinity and dispelling Her doubts.}
\textlink{}
\translink{https://www.wisdomlib.org/hinduism/book/shiva-purana-english/d/doc226044.html}

\storymeta

\uvacha{नारद उवाच}

\twolineshloka
{ब्रह्मन् विधे प्रजानाथ महाप्राज्ञ कृपाकर}
{श्रावितं शङ्करयशस्सतीशङ्करयोः शुभम्} %।। १ ।।

\twolineshloka
{इदानीं ब्रूहि सत्प्रीत्या परं तद्यश उत्तमम्}
{किमकार्ष्टां हि तत्स्थौ वै चरितं दम्पती शिवौ} %।। २ ।।

\uvacha{ब्रह्मोवाच}

\twolineshloka
{सतीशिवचरित्रं च शृणु मे प्रेमतो मुने}
{लौकिकीं गतिमाश्रित्य चिक्रीडाते सदान्वहम्} %।। ३ ।।

\twolineshloka
{ततस्सती महादेवी वियोगमलभन्मुने}
{स्वपतश्शङ्करस्येति वदन्त्येके सुबुद्धयः} %।। ४ ।।

\twolineshloka
{वागर्थाविव सम्पृक्तौ शक्तोशौ सर्वदा चितौ}
{कथं घटेत च तयोर्वियोगस्तत्त्वतो मुने} %।।५।।

\twolineshloka
{लीलारुचित्वादथ वा सङ्घटेताऽखिलं च तत्}
{कुरुते यद्यदीशश्च सती च भवरीतिगौ} %।। ६ ।।

\twolineshloka
{सा त्यक्ता दक्षजा दृष्ट्वा पतिना जनकाध्वरे}
{शम्भोरनादरात्तत्र देहं तत्याज सङ्गता} %।। ७ ।।

\twolineshloka
{पुनर्हिमालये सैवाविर्भूता नामतस्सती}
{पार्वतीति शिवं प्राप तप्त्वा भूरि विवाहतः} %।। ८ ।।

\uvacha{सूत उवाच}

\twolineshloka
{इत्याकर्ण्य वचस्तस्य ब्रह्मणस्स तु नारदः}
{पप्रच्छ च विधातारं शिवाशिवमहद्यशः} %।।९।।

\uvacha{नारद उवाच}

\twolineshloka
{विष्णुशिष्य महाभाग विधे मे वद विस्तरात्}
{शिवाशिवचरित्रं तद्भवाचारपरानुगम्} %।। १० ।।

\twolineshloka
{किमर्थं शङ्करो जायां तत्याज प्राणतः प्रियाम्}
{तस्मादाचक्ष्व मे तात विचित्रमिति मन्महे} %।। ११ ।।

\twolineshloka
{कुतोऽह्यध्वरजः पुत्रां नादरोभूच्छिवस्य ते}
{कथं तत्याज सा देहं गत्वा तत्र पितृक्रतौ} %।। १२ ।।

\twolineshloka
{ततः किमभवत्तत्र किमकार्षीन्महेश्वरः}
{तत्सर्वं मे समाचक्ष्व श्रद्धायुक् तच्छुतावहम्} %।। १३ ।।

\uvacha{ब्रह्मोवाच}

\twolineshloka
{शृणु तात परप्रीत्या मुनिभिस्सह नारद}
{सुतवर्य महाप्राज्ञ चरितं शशिमौलिनः} %।। १४ ।।

\twolineshloka
{नमस्कृत्य महेशानं हर्यादिसुरसेवितम्}
{परब्रह्म प्रवक्ष्यामि तच्चरित्रं महाद्भुतम्} %।। १५ ।।

\twolineshloka
{सर्वेयं शिवलीला हि बहुलीलाकरः प्रभुः}
{स्वतन्त्रो निर्विकारी च सती सापि हि तद्विधा} %।। १६ ।।

\twolineshloka
{अन्यथा कस्समर्थो हि तत्कर्मकरणे मुने}
{परमात्मा परब्रह्म स एव परमेश्वरः} %।। १७ ।।

\twolineshloka
{यं सदा भजते श्रीशोऽहं चापि सकलाः सुराः}
{मुनयश्च महात्मानः सिद्धाश्च सनकादयः} %।। १८ ।।

\twolineshloka
{शेषस्सदा यशो यस्य मुदा गायति नित्यशः}
{पारं न लभते तात स प्रभुश्शङ्करः शिवः} %।। १९ ।।

\twolineshloka
{तस्यैव लीलया सर्वोयमिति तत्त्वविभ्रमः}
{तत्र दोषो न कस्यापि सर्वव्यापी स प्रेरकः} %।। २० ।।

\twolineshloka
{एकस्मिन्समये रुद्रस्सत्या त्रिभुवने भवः}
{वृषमारुह्य पर्याटद्रसां लीलाविशारदः} %।। २१ ।।

\twolineshloka
{आगत्य दण्डकारण्यं पर्यटन् सागराम्बराम्}
{दर्शयन् तत्र गां शोभां सत्यै सत्यपणः प्रभुः} %।। २२ ।।

\twolineshloka
{तत्र रामं ददर्शासौ लक्ष्मणेनान्वितं हरः}
{अन्विष्यन्तं प्रियां सीतां रावणेन हृता छलात्} %।। २३ ।।

\twolineshloka
{हा सीतेति प्रोच्चरन्तं विरहाविष्टमानसम्}
{यतस्ततश्च पश्यन्तं रुदन्तं हि मुहुर्मुहुः} %।। २४ ।।

\twolineshloka
{समिच्छन्तं च तत्प्राप्तिं पृच्छन्तं तद्गतिं हृदा}
{कुजादिभ्यो नष्टधियमत्रपं शोकविह्वलम्} %।।२५।।

\twolineshloka
{सूर्यवंशोद्भवं वीरं भूपं दशरथात्मजम्}
{भरताग्रजमानन्दरहितं विगतप्रभम्} %।।२६।।

\twolineshloka
{पूर्णकामो वराधीनं प्राणमत्स्म मुदा हरः}
{रामं भ्रमन्तं विपिने सलक्ष्मणमुदारधीः} %।।२७।।

\twolineshloka
{जयेत्युक्त्वाऽन्यतो गच्छन्नदात्तस्मै स्वदर्शनम्}
{रामाय विपिने तस्मिच्छङ्करो भक्तवत्सलः} %।।२८।।

\twolineshloka
{इतीदृशीं सतीं दृष्ट्वा शिवलीलां विमोहनीम्}
{सुविस्मिता शिवं प्राह शिवमायाविमोहिता} %।।२९।।

\uvacha{सत्युवाच}

\twolineshloka
{देव देव परब्रह्म सर्वेश परमेश्वर}
{सेवन्ते त्वां सदा सर्वे हरिब्रह्मादयस्सुराः} % ।।2.2.24.३०।।

\twolineshloka
{त्वं प्रणम्यो हि सर्वेषां सेव्यो ध्येयश्च सर्वदा}
{वेदान्तवेद्यो यत्नेन निर्विकारी परप्रभुः} %।।३१।।

\twolineshloka
{काविमौ पुरुषौ नाथ विरहव्याकुलाकृती}
{विचरन्तौ वने क्लिष्टौ दीनौ वीरौ धनुर्धरौ} %।।३२।।

\twolineshloka
{तयोर्ज्येष्ठं कञ्जश्यामं दृष्ट्वा वै केन हेतुना}
{सुदितस्सुप्रसन्नात्माऽभवो भक्त इवाऽधुना} %।।३३।।

\twolineshloka
{इति मे संशयं स्वामिञ्शङ्कर छेत्तुमर्हसि}
{सेव्यस्य सेवकेनैव घटते प्रणतिः प्रभो} %।।३४।।

\uvacha{ब्रह्मोवाच}

\twolineshloka
{आदिशक्तिस्सती देवी शिवा सा परमेश्वरी}
{शिवमायावशीभूत्वा पप्रच्छेत्थं शिवं प्रभुम्} %।। ३५ ।।

\twolineshloka
{तदाकर्ण्य वचस्सत्याश्शङ्करः परमेश्वरः}
{तदा विहस्य स प्राह सतीं लीलाविशारदः} %।।३६

\uvacha{परमेश्वर उवाच}

\twolineshloka
{शृणु देवि सति प्रीत्या यथार्थं वच्मि नच्छलम्}
{वरदानप्रभावात्तु प्रणामं चैवमादरात्} %।।३७।।

\twolineshloka
{रामलक्ष्मणनामानौ भ्रातरौ वीरसम्मतौ}
{सूर्यवंशोद्भवौ देवि प्राज्ञौ दशरथात्मजौ} %।।३८।।

\twolineshloka
{गौरवर्णौ लघुर्बन्धुश्शेषेशो लक्ष्मणाभिधः}
{ज्येष्ठो रामाभिधो विष्णुः पूर्णांशो निरुपद्रवः} %।।३९।।

\twolineshloka
{अवतीर्णं क्षितौ साधुरक्षणाय भवाय नः}
{इत्युक्त्वा विररामाऽसौ शम्भुस्मृतिकरः प्रभुः} %।। ४० ।।

\twolineshloka
{श्रुत्वापीत्थं वचश्शम्भोर्न विशश्वास तन्मनः}
{शिवमाया बलवती सैव त्रैलोक्यमोहिनी} %।। ४१ ।।

\twolineshloka
{अविश्वस्तं मनो ज्ञात्वा तस्याश्शम्भुस्सनातनः}
{अवोचद्वचनं चेति प्रभुलीलाविशारदः} %।। ४२ ।।

\uvacha{शिव उवाच}

\twolineshloka
{शृणु मद्वचनं देवि न विश्वसिति चेन्मनः}
{तव रामपरिक्षां हि कुरु तत्र स्वया धिया} %।। ४३ ।।

\twolineshloka
{विनश्यति यथा मोहस्तत्कुरु त्वं सति प्रिये}
{गत्वा तत्र स्थितस्तावद्वटे भव परीक्षिका}% । ४४ ।।

\uvacha{ब्रह्मोवाच}

\twolineshloka
{शिवाज्ञया सती तत्र गत्वाचिन्तयदीश्वरी}
{कुर्यां परीक्षां च कथं रामस्य वनचारिणः} %।।४५।।

\twolineshloka
{सीतारूपमहं धृत्वा गच्छेयं रामसन्निधौ}
{यदि रामो हरिस्सर्वं विज्ञास्यति न चान्यथा} %।।४६।।

\twolineshloka
{इत्थं विचार्य सीता सा भूत्वा रामसमीपतः}
{आगमत्तत्परीक्षार्थं सती मोहपरायणा} %।। ४७ ।।

\twolineshloka
{सीतारूपां सतीं दृष्ट्वा जपन्नाम शिवेति च}
{विहस्य तत्प्रविज्ञाय नत्वावोचद्रघूद्वहः} %।।४८।।

\uvacha{राम उवाच}

\twolineshloka
{प्रेमतस्त्वं सति ब्रूहि क्व शम्भुस्ते नमोगतः}
{एका हि विपिने कस्मादागता पतिना विना} %।। ४९ ।।

\twolineshloka
{त्यक्त्वा स्वरूपं कस्मात्ते धृतं रूपमिदं सति}
{ब्रूहि तत्कारणं देवि कृपां कृत्वा ममोपरि}%।। 2.2.24.५०।।

\uvacha{ब्रह्मोवाच}

\twolineshloka
{इति रामवचः श्रुत्वा चकितासीत्सती तदा}
{स्मृत्वा शिवोक्तं मत्वा चावितथं लज्जिता भृशम्} %।। ५१ ।।

\twolineshloka
{रामं विज्ञाय विष्णुं तं स्वरूपं संविधाय च}
{स्मृत्वा शिवपदं चित्ते सत्युवाच प्रसन्नधीः} %।। ५२ ।।

\twolineshloka
{शिवो मया गणैश्चैव पर्यटन् वसुधां प्रभुः}
{इहागच्छच्च विपिने स्वतन्त्रः परमेश्वरः} %।। ५३ ।।

\twolineshloka
{अपश्यदत्र स त्वां हि सीतान्वेषणतत्परम्}
{सलक्ष्मणं विरहिणं सीतया श्लिष्टमानसम्} %।। ५४ ।।

\twolineshloka
{नत्वा त्वां स गतो मूले वटस्य स्थित एव हि}
{प्रशंसन् महिमानं ते वैष्णवं परमं मुदा} %।। ५५ ।।

\twolineshloka
{चतुर्भुजं हरिं त्वां नो दृष्ट्वेव मुदितोऽभवत्}
{यथेदं रूपममलं पश्यन्नानन्दमाप्तवान्} %।।५६।।

\twolineshloka
{तच्छ्रुत्वा वचनं शम्भौर्भ्रममानीय चेतसि}
{तदाज्ञया परीक्षां ते कृतवत्य स्मि राघव} %।। ५७ ।।

\twolineshloka
{ज्ञातं मे राम विष्णुस्त्वं दृष्टा ते प्रभुताऽखिला}
{निःसशंया तदापि तच्छृणु त्वं च महामते} %।। ५८ ।।

\twolineshloka
{कथं प्रणम्यस्त्वं तस्य सत्यं ब्रूहि ममाग्रतः}
{कुरु निस्संशयां त्वं मां शमलं प्राप्नुहि द्रुतम्} %।।५९।।

\uvacha{ब्रह्मोवाच}

\twolineshloka
{इत्याकर्ण्य वचस्तस्या रामश्चोत्फुल्ललोचनः}
{अस्मरत्स्वं प्रभुं शम्भुं प्रेमाभूद्धृदि चाधिकम्} %।।2.2.24.६०।।

\twolineshloka
{सत्या विनाज्ञया शम्भुसमीपं नागमन्मुने}
{संवर्ण्य महिमानं च प्रावोचद्राघवस्सतीम्} %।। ६१ ।।

॥इति श्रीशिवमहापुराणे द्वितीयायां रुद्रसंहितायां द्वितीये सतीखण्डे रामपरीक्षावर्णनं नाम चतुर्विंशोऽध्यायः॥२४॥


\sect{सतीवियोगः --- पञ्चविंशोऽध्यायः}

\uvacha{राम उवाच}

\twolineshloka
{एकदा हि पुरा देवि शम्भुः परमसूतिकृत्}
{विश्वकर्माणमाहूय स्वलोके परतः परे} %।।१।।

\twolineshloka
{स्वधेनुशालायां रम्यं कारयामास तेन च}
{भवनं विस्तृतं सम्यक् तत्र सिंहासनं वरम्} %।। २ ।।

\twolineshloka
{तत्रच्छत्रं महादिव्यं सर्वदाद्भुत मुत्तमम्}
{कारयामास विघ्नार्थं शङ्करो विश्वकर्मणा} %।। ३ ।।

\twolineshloka
{शक्रादीनां जुहावाशु समस्तान्देवतागणान्}
{सिद्धगन्धर्वनागानुपदे शांश्च कृत्स्नशः} %।।४।।

\twolineshloka
{देवान् सर्वानागमांश्च विधिं पुत्रैर्मुनीनपि}
{देवीः सर्वा अप्सरोभिर्नानावस्तुसमन्विताः} %।। ५ ।।

\twolineshloka
{देवानां च तथर्षीणां सिद्धानां फणिनामपि}
{आनयन्मङ्गलकराः कन्याः षोडशषोडश} %।।६।।

\twolineshloka
{वीणामृदङ्गप्रमुखवाद्यान्नानाविधान्मुने}
{उत्सवं कारयामास वादयित्वा सुगायनैः} %।।७।।

\twolineshloka
{राजाभिषेकयोग्यानि द्रव्याणि सकलौषधैः}
{प्रत्यक्षतीर्थपाथोभिः पञ्चकुभांश्च पूरितान्} %।।८।।

\twolineshloka
{तथान्यास्संविधा दिव्या आनयत्स्वगणैस्तदा}
{ब्रह्मघोषं महारावं कारयामास शङ्करः} %।।९।।

\twolineshloka
{अथो हरिं समाहूय वैकुण्ठात्प्रीतमानसः}
{तद्भक्त्या पूर्णया देवि मोदतिस्म महेश्वरः} %।। १० ।।

\twolineshloka
{सुमुहूर्ते महादेवस्तत्र सिंहासने वरे}
{उपवेश्य हरिं प्रीत्या भूषयामास सर्वशः} %।।११।।

\twolineshloka
{आबद्धरम्यमुकुटं कृतकौतुकमङ्गलम्}
{अभ्यषिञ्चन्महेशस्तु स्वयं ब्रह्माण्डमण्डपे} %।। १२ ।।

\twolineshloka
{दत्तवान्निखिलैश्वर्यं यन्नैजं नान्यगामि यत्}
{ततस्तुष्टाव तं शम्भुस्स्वतन्त्रो भक्तवत्सलः} %।। १३ ।।

\twolineshloka
{ब्रह्माणं लोककर्तारमवोचद्वचनं त्विदम्}
{व्यापयन्स्वं वराधीनं स्वतन्त्रं भक्तवत्सलः} %।। १४।।

\uvacha{महेश उवाच}

\twolineshloka
{अतः प्रभृति लोकेश मन्निदेशादयं हरिः}
{मम वन्द्य स्वयं विष्णुर्जातस्सर्वश्शृणोति हि} %।। १५ ।।

\twolineshloka
{सर्वैर्देवादिभिस्तात प्रणमत्वममुं हरिम्}
{वर्णयन्तु हरिं वेदा ममैते मामिवाज्ञया}%।। १६ ।

\uvacha{राम उवाच}

\twolineshloka
{इत्युक्त्वाथ स्वयं रुद्रोऽनमद्वै गरुडध्वजम्}
{विष्णुभक्तिप्रसन्नात्मा वरदो भक्तवत्सलः} %।। १७।।

\twolineshloka
{ततो ब्रह्मादिभिर्देवैः सर्वरूपसुरैस्तथा}
{मुनिसिद्धादिभिश्चैवं वन्दितोभूद्धरिस्तदा} %।। १८ ।।

\twolineshloka
{ततो महेशो हरयेशंसद्दिविषदां तदा}
{महावरान् सुप्रसन्नो धृतवान्भक्तवत्सलः} %।। १९ ।।

\uvacha{महेश उवाच}

\twolineshloka
{त्वं कर्ता सर्वलोकानां भर्ता हर्ता मदाज्ञया}
{दाता धर्मार्थकामानां शास्ता दुर्नयकारिणाम्} %।। २० ।।

\twolineshloka
{जगदीशो जगत्पूज्यो महाबलपराक्रमः}
{अजेयस्त्वं रणे क्वापि ममापि हि भविष्यसि} %।। २१ ।।

\twolineshloka
{शक्तित्रयं गृहाण त्वमिच्छादि प्रापितं मया}
{नानालीलाप्रभावत्वं स्वतन्त्रत्वं भवत्रये} %।। २२ ।।

\twolineshloka
{त्वद्द्वेष्टारो हरे नूनं मया शास्याः प्रयत्नतः}
{त्वद्भक्तानां मया विष्णो देयं निर्वाणमुत्तमम्} %।। २३ ।।

\twolineshloka
{मायां चापि गृहाणेमां दुःप्रणोद्यां सुरादिभिः}
{यया सम्मोहितं विश्वमचिद्रूपं भविष्यति} %।। २४ ।।

\twolineshloka
{मम बाहुर्मदीयस्तं दक्षिणोऽसौ विधिर्हरे}
{अस्यापि हि विधेः पाता जनितापि भविष्यसि} %।। २५ ।।

\twolineshloka
{हृदयं मम यो रुद्रस्स एवाहं न संशयः}
{पूज्यस्तव सदा सोपि ब्रह्मादीनामपि ध्रुवम्} %।। २६ ।।

\twolineshloka
{अत्र स्थित्वा जगत्सर्वं पालय त्वं विशेषतः}
{नानावतारभेदैश्च सदा नानोति कर्तृभिः} %।। २७ ।।

\twolineshloka
{मम लोके तवेदं व स्थानं च परमर्द्धिमत्}
{गोलोक इति विख्यातं भविष्यति महोज्ज्वलम्} %।। २८ ।।

\twolineshloka
{भविष्यन्ति हरे ये तेऽवतारा भुवि रक्षकाः}
{मद्भक्तास्तान् ध्रुवं द्रक्ष्ये प्रीतानथ निजाद्वरात} %।। २९ ।।

\uvacha{राम उवाच}

\twolineshloka
{अखण्डैश्वर्यमासाद्य हरेरित्थं हरः स्वयम्}
{कैलासे स्वगणैस्तस्मिन् स्वैरं क्रीडत्युमापतिः} %।। ३० ।।

\twolineshloka
{तदाप्रभृति लक्ष्मीशो गोपवेषोभवत्तथा}
{अयासीत्तत्र सुप्रीत्या गोपगोपोगवां पतिः} %।। ३१ ।।

\twolineshloka
{सोपि विष्णुः प्रसन्नात्मा जुगोप निखिलं जगत्}
{नानावतारस्सन्धर्ता वनकर्ता शिवाज्ञया} %।। ३२ ।।

\twolineshloka
{इदानीं स चतुर्द्धात्रावातरच्छङ्कराज्ञया}
{रामोहं तत्र भरतो लक्ष्मणश्शत्रुहेति च} %।। ३३ ।।

\twolineshloka
{अथ पित्राज्ञया देवि ससीतालक्ष्मणस्सति}
{आगतोहं वने चाद्य दुःखितौ दैवतो ऽभवम्} %।। ३४ ।।

\twolineshloka
{निशाचरेण मे जाया हृता सीतेति केनचित्}
{अन्वेष्यामि प्रियां चात्र विरही बन्धुना वने} %।। ३५ ।।

\twolineshloka
{दर्शनं ते यदि प्राप्तं सर्वथा कुशलं मम}
{भविष्यति न सन्देहो मातस्ते कृपया सति} %।। ३६ ।।

\twolineshloka
{सीताप्राप्तिवरो देवि भविष्यति न संशयः}
{तं हत्वा दुःखदं पापं राक्षसं त्वदनुग्रहात्} %।। ३७ ।।

\twolineshloka
{महद्भाग्यं ममाद्यैव यद्यकार्ष्टां कृपां युवाम्}
{यस्मिन् सकरुणौ स्यातां स धन्यः पुरुषो वरः} %।। ३८ ।।

\twolineshloka
{इत्थमाभाष्य बहुधा सुप्रणम्य सतीं शिवाम्}
{तदाज्ञया वने तस्मिन् विचचार रघूद्वहः} %।। ३९ ।।

\twolineshloka
{अथाकर्ण्य सती वाक्यं रामस्य प्रयतात्मनः}
{हृष्टाभूत्सा प्रशंसन्ती शिवभक्तिरतं हृदि} %।। ४० ।।

\twolineshloka
{स्मृत्वा स्वकर्म मनसाकार्षीच्छोकं सुविस्तरम्}
{प्रत्यागच्छदुदासीना विवर्णा शिवसन्निधौ} %।।४१।।

\twolineshloka
{अचिन्तयत्पथि सा देवी सञ्चलन्ती पुनः पुनः}
{नाङ्गीकृतं शिवोक्तं मे रामं प्रति कुधीः कृता} %।।४२।।

\twolineshloka
{किमुत्तरमहं दास्ये गत्वा शङ्करसन्निधौ}
{इति सञ्चिन्त्य बहुधा पश्चात्तापोऽभवत्तदा} %।।४३।।

\twolineshloka
{गत्वा शम्भुसमीपं च प्रणनाम शिवं हृदा}
{विषण्णवदना शोकव्याकुला विगतप्रभा} %।।४४।।

\twolineshloka
{अथ तां दुःखितां दृष्ट्वा पप्रच्छ कुशलं हरः}
{प्रोवाच वचनं प्रीत्या तत्परीक्षा कृता कथम्} %।। ४५ ।।

\twolineshloka
{श्रुत्वा शिववचो नाहं किमपि प्रणतानना}
{सती शोकविषण्णा सा तस्थौ तत्र समीपतः} %।। ४६ ।।

\twolineshloka
{अथ ध्यात्वा महेशस्तु बुबोध चरितं हृदा}
{दक्षजाया महायोगी नानालीला विशारदः} %।। ४७ ।।

\twolineshloka
{सस्मार स्वपणं पूर्वं यत्कृतं हरिकोपतः}
{तत्प्रार्थितोथ रुद्रोसौ मर्यादा प्रतिपालकः} %।। ४८ ।।

\twolineshloka
{विषादोभूत्प्रभोस्तत्र मनस्येवमुवाच ह}
{धर्मवक्ता धर्मकर्त्ता धर्मावनकरस्सदा} %।। ४९ ।।

\uvacha{शिव उवाच}

\twolineshloka
{कुर्यां चेद्दक्षजायां हि स्नेहं पूर्वं यथा महान्}
{नश्येन्मम पणः शुद्धो लोकलीलानुसारिणः} %।। ५० ।।

\uvacha{ब्रह्मोवाच}

\twolineshloka
{इत्थं विचार्य बहुधा हृदा तामत्यजत्सतीम्}
{पणं न नाशयामास वेदधर्मप्रपालकः} %।। ५१ ।।

\twolineshloka
{ततो विहाय मनसा सतीं तां परमेश्वरः}
{जगाम स्वगिरि भेदं जगावद्धा स हि प्रभुः} %।। ५२ ।।

\twolineshloka
{चलन्तं पथि तं व्योमवाण्युवाच महेश्वरम्}
{सर्वान् संश्रावयन् तत्र दक्षजां च विशेषतः} %।। ५३ ।।

\uvacha{व्योमवाण्युवाच}

\twolineshloka
{धन्यस्त्वं परमेशान त्वत्त्समोद्य तथा पणः}
{न कोप्यन्यस्त्रिलोकेस्मिन् महायोगी महाप्रभुः} %।। ५४ ।।

\uvacha{ब्रह्मोवाच}

\twolineshloka
{श्रुत्वा व्योमवचो देवी शिवं पप्रच्छ विप्रभा}
{कं पणं कृतवान्नाथ ब्रूहि मे परमेश्वर} %।। ५५ ।।

\twolineshloka
{इति पृष्टोपि गिरिशस्सत्या हितकरः प्रभुः}
{नोद्वाहे स्वपणं तस्यै कहर्यग्रेऽकरोत्पुरा} %।।५६।।

\twolineshloka
{तदा सती शिवं ध्यात्वा स्वपतिं प्राणवल्लभम्}
{सर्वं बुबोध हेतुं तं प्रियत्यागमयं मुने} %।।५७।।

\twolineshloka
{ततोऽतीव शुशोचाशु बुध्वा सा त्यागमात्मनः}
{शम्भुना दक्षजा तस्मान्निश्वसन्ती मुहुर्मुहुः} %।।५८।।

\twolineshloka
{शिवस्तस्याः समाज्ञाय गुप्तं चक्रे मनोभवम्}
{सत्ये पणं स्वकीयं हि कथा बह्वीर्वदन्प्रभुः} %।।५९।।

\twolineshloka
{सत्या प्राप स कैलासं कथयन् विविधाः कथा}
{वरे स्थित्वा निजं रूपं दधौ योगी समाधिभृत्} %।।2.2.25.६०।।

\twolineshloka
{तत्र तस्थौ सती धाम्नि महाविषण्णमानसा}
{न बुबोध चरित्रं तत्कश्चिच्च शिवयोर्मुने} %।।६१।।

\twolineshloka
{महान्कालो व्यतीयाय तयोरित्थं महामुने}
{स्वोपात्तदेहयोः प्रभ्वोर्लोकलीलानुसारिणोः} %।। ६२ ।।

\twolineshloka
{ध्यानं तत्याज गिरिशस्ततस्स परमार्तिहृत्}
{तज्ज्ञात्वा जगदम्बा हि सती तत्राजगाम सा} %।। ६३ ।।

\twolineshloka
{ननामाथ शिवं देवी हृदयेन विदूयता}
{आसनं दत्तवाञ्शम्भुः स्वसन्मुख उदारधीः} %।। ६४ ।।

\twolineshloka
{कथयामास सुप्रीत्या कथा बह्वीर्मनोरमाः}
{निश्शोका कृतवान्सद्यो लीलां कृत्वा च तादृशीम्} %।।६५।।

\twolineshloka
{पूर्ववत्सा सुखं लेभे तत्याज स्वपणं न सः}
{नेत्याश्चर्यं शिवे तात मन्तव्यं परमेश्वरे} %।।६६।।

\twolineshloka
{इत्थं शिवाशिवकथां वदन्ति मुनयो मुने}
{किल केचिदविद्वांसो वियोगश्च कथं तयोः} %।।६७।।

\twolineshloka
{शिवाशिवचरित्रं को जानाति परमार्थतः}
{स्वेच्छया क्रीडतस्तो हि चरितं कुरुतस्सदा} %।। ६८ ।।

\twolineshloka
{वागर्थाविव सम्पृक्तौ सदा खलु सतीशिवौ}
{तयोर्वियोगस्सम्भाव्यस्सम्भवेदिच्छया तयोः} %।।६९।।

॥इति श्रीशिवमहापुराणे द्द्वितीयायां रुद्रसंहितायां द्वितीये सतीखण्डे सतीवियोगो नाम पञ्चविंशोऽध्यायः॥२५॥

\closesection