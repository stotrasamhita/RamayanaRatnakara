\chapt{सौरपुराणम्}

\src{सौरपुराणम्}{}{अध्यायः ३०}{श्लोकाः ४८--६९}
\vakta{}
\shrota{}
\notes{This chapter briefly recounts the life of Lord Rama---His divine birth, marriage to Sita, exile, Sita’s abduction by Ravana, the alliance with Hanuman and Sugriva, the war in Lanka, and His triumphant return and coronation. It concludes with a lineage of Rama's descendants from Lava/Kuśa.}
\textlink{https://archive.org/details/saurapurana1924compl/page/97/mode/2up}
\translink{}

\storymeta


\sect{त्रिंशोऽध्यायः --- इक्ष्वाकुकुलसम्भवनृपमालिका-कथनम्}

\addtocounter{shlokacount}{47}

\twolineshloka
{दीर्घबाहुस्ततो जज्ञे रघुस्तस्याभवत्सुतः}
{रघोरजस्तु विख्यातो राजा दशरथस्ततः} %॥४८॥

\twolineshloka
{तस्य पुत्राश्च चत्वारो धर्मज्ञा लोकविश्रुताः}
{रामोऽथ भरतश्चैव तृतीयो लक्ष्मणः स्मृतः} %॥४९॥

\twolineshloka
{चतुर्थश्चैव शत्रुघ्नो रामो नारायणः स्वयम्}
{धर्मज्ञः सत्यसङ्कल्पो महादेवपरायणः} %॥५०॥

\twolineshloka
{सीता तस्याभवद्भार्या पार्वत्यंशसमुद्भवा}
{जनकेन पुरा गौरी तपसा तोषिता यतः} %॥५१॥

\twolineshloka
{जनकाय ददौ शम्भुः प्रीतो धनुरनुत्तमम्}
{तद्धनुर्भञ्जयामास जनकस्य गृहे स्थितम्} %॥५२॥

\twolineshloka
{दृष्ट्वा पराक्रमं तस्य रामस्य गुणशालिनः}
{जनकः प्रददौ तस्मै सीतां ब्रह्मविदां वरः} %॥५३॥

\twolineshloka
{पित्रा कृतोऽभिषेकार्थं रामो राज्यस्य वै यदा}
{वारयामास कैकेयी तदा राज्ञः प्रिया वधूः} %॥५४॥

\twolineshloka
{राजंस्त्वया वरो दत्तः पूर्वमेव यतः प्रभो}
{राजानं मत्सुतं तस्माद्भरतं कर्तुमर्हसि} %॥५५॥

\twolineshloka
{इति तस्या वचः श्रुत्वा राज्ये तमभिषिच्य सः}
{प्रेषयामास तं रामं वनं प्रति सलक्ष्मणम्} %॥५६॥

\twolineshloka
{वनं गत्वा निवसतो भार्यां दृष्ट्वाऽथ राक्षसः}
{रावणो नाम पौलस्त्यो नीत्वा लङ्कां पुनर्ययौ} %॥५७॥

\twolineshloka
{अदृष्ट्वा तां ततः सीतां दुःखितौ रामलक्ष्मणौ}
{सख्यं वानरराजेन गत्वा दाशरथी द्विजाः} %॥५८॥

\twolineshloka
{सुग्रीवस्य सखा वीरो हनुमान्नाम वानरः}
{गत्वाऽथ रावणपुरीमपश्यज्जनकात्मजाम्} %॥५९॥

\twolineshloka
{अश्रुपूर्णेक्षणां सीतामिन्दीवरनिभाननाम्}
{विश्वासार्थं ददौ तस्यै रामस्यैवाङ्गुलीयकम्} %॥६०॥

\twolineshloka
{दृष्ट्वाऽङ्गुलीयकं सीता प्रहृष्टा च तदाऽभवत्}
{समाश्वास्य ततः सीतां प्रययौ राघवान्तिकम्} %॥६१॥

\twolineshloka
{रामस्तमागतं दृष्ट्वा प्रहर्षोत्फुल्ललोचनः}
{श्रुत्वा तद्वचनाद्वृत्तं युद्धाय कृतनिश्चयः} %॥६२॥

\twolineshloka
{सेतुं कृत्वाऽथ रक्षोभिर्युद्धं कृत्वा महामनाः}
{निहत्य रावणं रामो भ्रातृभिः सह सुव्रतः} %॥६३॥

\twolineshloka
{आनयामास तां सीतामशोकवनमध्यगाम्}
{प्रतिष्ठाप्य महादेवं सेतुमध्येऽथ राघवः} %॥६४॥

\twolineshloka
{लब्धवान्परमां भक्तिं शिवे शिवपराक्रमः}
{रामेश्वर इति ख्यातो महादेवः पिनाकधृक्} %॥६५॥

\twolineshloka
{तस्य दर्शनमात्रेण ब्रह्महत्यां व्यपोहति}
{अभिषिक्तस्ततो राज्ये रामो राजीवलोचनः} %॥६६॥

\twolineshloka
{पालयन्पृथिवीं सर्वां धर्मेण मुनिपुंगवाः}
{अयजद्देवदेवेशमश्वमेधेन शङ्करम्} %॥६७॥

\twolineshloka
{तस्य प्रसादात्स्वपदं प्राप्तवानथ राघवः}
{एवं सङ्क्षेपतः प्रोक्तं रामस्य चरितं मया} %॥६८॥

\twolineshloka
{इदं विस्तरतो विप्राः प्रोक्तं वाल्मीकिना पुनः}
{कुशश्चैको लवश्वान्यः पुत्रौ रामस्य सुव्रतौ} %॥६९॥

\threelineshloka
{सत्यसन्धौ महावीर्यौ महादेवपरायणौ}
{अतिथिश्च कुशाज्जज्ञे निषधस्तत्सुतोऽभवत्}
{नलस्तस्याभवत्पुत्रो नभस्तस्याभवत्सुतः} %॥७०॥

\twolineshloka
{ततश्चन्द्रावलोकश्च तारापीडस्ततोऽभवत्}
{ततश्चन्द्रगिरिर्नाम भानुजित्तत्सुतोऽभवत्} %॥७१॥

\twolineshloka
{एते सर्वे नृपाः प्रोक्ता इक्ष्वाकुकुलसम्भवाः}
{धर्मात्मानो महासत्त्वाः कीर्तिमन्तो दृढव्रताः} %॥७२॥

\twolineshloka
{इमं यः पठते नित्यमिक्ष्वाकोर्वंशमुत्तमम्}
{सर्वपापविनिर्मुक्तः सूर्यलोके महीयते} %॥७३॥

॥इति श्रीब्रह्मपुराणोपपुराणे श्रीसौरे सुतशौनकसंवादे प्रह्लादराज्यारोहणादीक्ष्वाकुकुलसम्भवनृपमालिकान्तकथनं नाम त्रिंशोऽध्यायः॥३०॥

\closesection