\chapt{देवी-भागवतम्}

\src{देवी-भागवतम्}{नवमः स्कन्धः}{अध्यायाः १६}{श्लोकाः १--६४}
% \vakta{व्यासः}
% \shrota{जनमेजयः}
% \tags{concise, complete}
\notes{This passage reveals that the real Sita was actually replaced by a shadow/illusory Sita before her abduction - the Fire God (Agni) had taken the real Sita for safekeeping and given Rama a magical duplicate, which was the one Ravana actually kidnapped, and during Sita's trial by fire after Ravana's defeat, Agni returned the real Sita while the shadow Sita was sent to practice austerities and later became Draupadi (wife of the five Pandavas) in her next incarnation.}
\textlink{https://sa.wikisource.org/wiki/देवीभागवतपुराणम्/स्कन्धः_०९/अध्यायः_१६}
\translink{}

\storymeta

\sect{महालक्षम्या वेदवतीरूपेण राजगृहे जन्मवर्णनम्}

\uvacha{श्रीनारायण उवाच}


\twolineshloka
{लक्ष्मीं तौ च समाराध्य चोग्रेण तपसा मुने}
{वरमिष्टं च प्रत्येकं सम्प्रापतुरभीप्सितम्} %॥ १ ॥

\twolineshloka
{महालक्ष्मीवरेणैव तौ पृथ्वीशौ बभूवतुः}
{पुण्यवन्तौ पुत्रवन्तौ धर्मध्वजकुशध्वजौ} %॥ २ ॥

\twolineshloka
{कुशध्वजस्य पत्‍नी च देवी मालावती सती}
{सा सुषाव च कालेन कमलांशां सुतां सतीम्} %॥ ३ ॥

\twolineshloka
{सा च भूयिष्ठकालेन ज्ञानयुक्ता बभूव ह}
{कृत्वा वेदध्वनिं स्पष्टमुत्तस्थौ सूतिकागृहात्} %॥ ४ ॥

\twolineshloka
{वेदध्वनिं सा चकार जातमात्रेण कन्यका}
{तस्मात्तां च वेदवतीं प्रवदन्ति मनीषिणः} %॥ ५ ॥

\twolineshloka
{जातमात्रेण सुस्नाता जगाम तपसे वनम्}
{सर्वैर्निषिद्धा यत्‍नेन नारायणपरायणा} %॥ ६ ॥

\twolineshloka
{एकमन्वन्तरं चैव पुष्करे च तपस्विनी}
{अत्युग्रां च तपस्यां च लीलया हि चकार सा} %॥ ७ ॥

\twolineshloka
{तथापि पुष्टा न क्लिष्टा नवयौवनसंयुता}
{सुश्राव सा च सहसा सुवाचमशरीरिणीम्} %॥ ८ ॥

\twolineshloka
{जन्मान्तरे च ते भर्ता भविष्यति हरिः स्वयम्}
{ब्रह्मादिभिर्दुराराध्यं पतिं लप्स्यसि सुन्दरि} %॥ ९ ॥

\twolineshloka
{इति श्रुत्वा च सा हृष्टा चकार ह पुनस्तपः}
{अतीव निर्जनस्थाने पर्वते गन्धमादने} %॥ १० ॥

\twolineshloka
{तत्रैव सुचिरं तप्त्वा विश्वस्य समुवास सा}
{ददर्श पुरतस्तत्र रावणं दुर्निवारणम्} %॥ ११ ॥

\twolineshloka
{दृष्ट्वा सातिथिभक्त्या च पाद्यं तस्मै ददौ किल}
{सुस्वादुभूतं च फलं जलं चापि सुशीतलम्} %॥ १२ ॥

\twolineshloka
{तच्च भुक्त्वा स पापिष्ठश्चोवास तत्समीपतः}
{चकार प्रश्नमिति तां का त्वं कल्याणि वर्तसे} %॥ १३ ॥

\twolineshloka
{तां दृष्ट्वा स वरारोहां पीनश्रोणिपयोधराम्}
{शरत्पद्मोत्सवास्यां च सस्मितां सुदतीं सतीम्} %॥ १४ ॥

\twolineshloka
{मूर्च्छामवाप कृपणः कामबाणप्रपीडितः}
{स करेण समाकृष्य शृङ्‌गारं कर्तुमुद्यतः} %॥ १५ ॥

\twolineshloka
{सती चुकोप दृष्ट्वा तं स्तम्भितं च चकार ह}
{स जडो हस्तपादैश्च किञ्चिद्वक्तुं न च क्षमः} %॥ १६ ॥

\twolineshloka
{तुष्टाव मनसा देवीं प्रययौ पद्मलोचनाम्}
{सा तुष्टा तस्य स्तवनं सुकृतं च चकार ह} %॥ १७ ॥

\twolineshloka
{सा शशाप मदर्थे त्वं विनङ्क्ष्यसि सबान्धवः}
{स्पृष्टाहं च त्वया कामाद्‌ बलं चाप्यवलोकय} %॥ १८ ॥

\twolineshloka
{इत्युक्त्वा सा च योगेन देहत्यागं चकार ह}
{गङ्‌गायां तां च सन्न्यस्य स्वगृहं रावणो ययौ} %॥ १९ ॥

\twolineshloka
{अहो किमद्‍भुतं दृष्टं किं कृतं वानयाधुना}
{इति सञ्चिन्त्य सञ्चिन्त्य विललाप पुनः पुनः} %॥ २० ॥

\twolineshloka
{सा च कालान्तरे साध्वी बभूव जनकात्मजा}
{सीतादेवीति विख्याता यदर्थे रावणो हतः} %॥ २१ ॥

\twolineshloka
{महातपस्विनी सा च तपसा पूर्वजन्मतः}
{लेभे रामं च भर्तारं परिपूर्णतमं हरिम्} %॥ २२ ॥

\twolineshloka
{सम्प्राप तपसाऽऽराध्य दुराराध्यं जगत्पतिम्}
{सा रमा सुचिरं रेमे रामेण सह सुन्दरी} %॥ २३ ॥

\twolineshloka
{जातिस्मरा न स्मरति तपसश्च क्लमं पुरा}
{सुखेन तज्जहौ सर्वं दुःखं चापि सुखं फले} %॥ २४ ॥

\twolineshloka
{नानाप्रकारविभवं चकार सुचिरं सती}
{सम्प्राप्य सुकुमारं तमतीव नवयौवना} %॥ २५ ॥

\twolineshloka
{गुणिनं रसिकं शान्तं कान्तं देवमनुत्तमम्}
{स्त्रीणां मनोज्ञं रुचिरं तथा लेभे यथेप्सितम्} %॥ २६ ॥

\twolineshloka
{पितुः सत्यपालनार्थं सत्यसन्धो रघूद्वहः}
{जगाम काननं पश्चात्कालेन च बलीयसा} %॥ २७ ॥

\twolineshloka
{तस्थौ समुद्रनिकटे सीतया लक्ष्मणेन च}
{ददर्श तत्र वह्निं च विप्ररूपधरं हरिः} %॥ २८ ॥

\twolineshloka
{रामं च दुःखितं दृष्ट्वा स च दुःखी बभूव ह}
{उवाच किञ्चित्सत्येष्टं सत्यं सत्यपरायणः} %॥ २९ ॥
\uvacha{द्विज उवाच}


\twolineshloka
{भगवच्छ्रूयतां राम कालोऽयं यदुपस्थितः}
{सीताहरणकालोऽयं तवैव समुपस्थितः} %॥ ३० ॥

\twolineshloka
{दैवं च दुर्निवार्यं च न च दैवात्परो बली}
{जगत्प्रसूं मयि न्यस्य छायां रक्षान्तिकेऽधुना} %॥ ३१ ॥

\twolineshloka
{दास्यामि सीतां तुभ्यं च परीक्षासमये पुनः}
{देवैः प्रस्थापितोऽहं च न च विप्रो हुताशनः} %॥ ३२ ॥

\twolineshloka
{रामस्तद्वचनं श्रुत्वा न प्रकाश्य च लक्ष्मणम्}
{स्वीकारं वचसश्चक्रे हृदयेन विदूयता} %॥ ३३ ॥

\twolineshloka
{वह्निर्योगेन सीताया मायासीतां चकार ह}
{तत्तुल्यगुणसर्वाङ्‌गां ददौ रामाय नारद} %॥ ३४ ॥

\twolineshloka
{सीतां गृहीत्वा स ययौ गोप्यं वक्तुं निषिध्य च}
{लक्ष्मणो नैव बुबुधे गोप्यमन्यस्य का कथा} %॥ ३५ ॥

\twolineshloka
{एतस्मिन्नन्तरे रामो ददर्श कानकं मृगम्}
{सीता तं प्रेरयामास तदर्थे यत्‍नपूर्वकम्} %॥ ३६ ॥

\twolineshloka
{सन्न्यस्य लक्ष्मणं रामो जानक्या रक्षणे वने}
{स्वयं जगाम तूर्णं तं विव्याध सायकेन च} %॥ ३७ ॥

\twolineshloka
{लक्ष्मणेति च शब्दं स कृत्वा च मायया मृगः}
{प्राणांस्तत्याज सहसा पुरो दृष्ट्वा हरिं स्मरन्} %॥ ३८ ॥

\twolineshloka
{मृगदेहं परित्यज्य दिव्यरूपं विधाय च}
{रत्‍ननिर्माणयानेन वैकुण्ठं स जगाम ह} %॥ ३९ ॥

\twolineshloka
{वैकुण्ठलोकद्वार्यासीत्किङ्‌करो द्वारपालयोः}
{पुनर्जगाम तद्द्वारमादेशाद्‌ द्वारपालयोः} %॥ ४० ॥

\twolineshloka
{अथ शब्दं च सा श्रुत्वा लक्ष्मणेति च विक्लवम्}
{तं हि सा प्रेरयामास लक्ष्मणं रामसन्निधौ} %॥ ४१ ॥

\twolineshloka
{गते च लक्ष्मणे रामं रावणो दुर्निवारणः}
{सीतां गृहीत्वा प्रययौ लङ्‌कामेव स्वलीलया} %॥ ४२ ॥

\twolineshloka
{विषसाद च रामश्च वने दृष्ट्वा च लक्ष्मणम्}
{तूर्णं च स्वाश्रमं गत्वा सीतां नैव ददर्श सः} %॥ ४३ ॥

\twolineshloka
{मूर्च्छां सम्प्राप सुचिरं विललाप भृशं पुनः}
{पुनः पुनश्च बभ्राम तदन्वेषणपूर्वकम्} %॥ ४४ ॥

\twolineshloka
{कालेन प्राप्य तद्वार्तां गोदावरीनदीतटे}
{सहायान्वानरात्कृत्वा बबन्ध सागरं हरिः} %॥ ४५ ॥

\twolineshloka
{लङ्‌कां गत्वा रघुश्रेष्ठो जघान सायकेन च}
{कालेन प्राप्य तं हत्वा रावणं बान्धवैः सह} %॥ ४६ ॥

\twolineshloka
{तां च वह्निपरीक्षां च कारयामास सत्वरम्}
{हुताशस्तत्र काले तु वास्तवीं जानकीं ददौ} %॥ ४७ ॥

\twolineshloka
{उवाच छाया वह्निं च रामं च विनयान्विता}
{करिष्यामीति किमहं तदुपायं वदस्व मे} %॥ ४८ ॥

\uvacha{श्रीरामाग्नी ऊचतुः}

\twolineshloka
{त्वं गच्छ तपसे देवि पुष्करं च सुपुण्यदम्}
{कृत्वा तपस्या तत्रैव स्वर्गलक्ष्मीर्भविष्यसि} %॥ ४९ ॥

\twolineshloka
{सा च तद्वचनं श्रुत्वा प्रतप्य पुष्करे तपः}
{दिव्यं त्रिलक्षवर्षं च स्वर्गलक्ष्मीर्बभूव ह} %॥ ५० ॥

\twolineshloka
{सा च कालेन तपसा यज्ञकुण्डसमुद्‍भवा}
{कामिनी पाण्डवानां च द्रौपदी द्रुपदात्मजा} %॥ ५१ ॥

\twolineshloka
{कृते युगे वेदवती कुशध्वजसुता शुभा}
{त्रेतायां रामपत्‍नी च सीतेति जनकात्मजा} %॥ ५२ ॥

\twolineshloka
{तच्छाया द्रौपदी देवी द्वापरे द्रुपदात्मजा}
{त्रिहायणी च सा प्रोक्ता विद्यमाना युगत्रये} %॥ ५३ ॥
\uvacha{नारद उवाच}


\twolineshloka
{प्रियाः पञ्च कथं तस्या बभूवुर्मुनिपुङ्‌गव}
{इति मच्चित्तसन्देहं भञ्ज सन्देहभञ्जन} %॥ ५४ ॥

\uvacha{श्रीनारायण उवाच}


\twolineshloka
{लङ्‌कायां वास्तवी सीता रामं सम्प्राप नारद}
{रूपयौवनसम्पन्ना छाया च बहुचिन्तया} %॥ ५५ ॥

\twolineshloka
{रामाग्न्योराज्ञया तप्तुमुपास्ते शङ्‌करं परम्}
{कामातुरा पतिव्यग्रा प्रार्थयन्ती पुनः पुनः} %॥ ५६ ॥

\twolineshloka
{पतिं देहि पतिं देहि पतिं देहि त्रिलोचन}
{पतिं देहि पतिं देहि पञ्चवारं चकार सा} %॥ ५७ ॥

\twolineshloka
{शिवस्तत्प्रार्थनां श्रुत्वा प्रहस्य रसिकेश्वरः}
{प्रिये तव प्रियाः पञ्च भविष्यन्ति वरं ददौ} %॥ ५८ ॥

\twolineshloka
{तेन सा पाण्डवानां च बभूव कामिनी प्रिया}
{इति ते कथितं सर्वं प्रस्तावं वास्तवं शृणु} %॥ ५९ ॥

\twolineshloka
{अथ सम्प्राप्य लङ्‌कायां सीतां रामो मनोहराम्}
{विभीषणाय तां लङ्‌कां दत्त्वायोध्यां ययौ पुनः} %॥ ६० ॥

\twolineshloka
{एकादशसहस्राब्दं कृत्वा राज्यं च भारते}
{जगाम सर्वैर्लोकैश्च सार्धं वैकुण्ठमेव च} %॥ ६१ ॥

\twolineshloka
{कमलांशा वेदवती कमलायां विवेश सा}
{कथितं पुण्यमाख्यानं पुण्यदं पापनाशनम्} %॥ ६२ ॥

\twolineshloka
{सततं मूर्तिमन्तश्च वेदाश्चत्वार एव च}
{सन्ति यस्याश्च जिह्वाग्रे सा च वेदवती श्रुता} %॥ ६३ ॥

\onelineshloka
{धर्मध्वजसुताख्यानं निबोध कथयामि ते}% ॥ ६४ ॥

॥इति श्रीमद्देवीभागवते महापुराणेऽष्टादशसाहस्र्यां संहितायां नवमस्कन्धे महालक्षम्या वेदवतीरूपेण राजगृहे जन्मवर्णनं नाम षोडशोऽध्यायः॥

\closesection