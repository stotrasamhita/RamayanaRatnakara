\chapt{रामोपाख्यान-पर्व}

\src{श्रीमन्महाभारतम्}{वन-पर्व}{श्रीरामोपाख्यानपर्व}{अध्यायाः २७५--२९३}
\vakta{मार्कण्डेयः}
\shrota{युधिष्ठिरः}
\tags{concise, complete}
\notes{This chapter recounts the appearance of Lord Rāmacandra in the lineage of Mahārāja Khaṭvāṅga and details His divine exploits, including the slaying of Rāvaṇa and His triumphant return to Ayodhyā.}
\textlink{https://sa.wikisource.org/wiki/देवीभागवतपुराणम्/स्कन्धः_०३/अध्यायः_२८}
\translink{}

रामचरित्रवर्णनम्

\uvacha{जनमेजय उवाच}
कथं रामेण तच्चीर्णं व्रतं देव्याः सुखप्रदम् ।
राज्यभ्रष्टः कथं सोऽथ कथं सीता हृता पुनः ॥ १ ॥
\uvacha{व्यास उवाच}
राजा दशरथः श्रीमानयोध्याधिपतिः पुरा ।
सूर्यवंशधरश्चासीद्देवब्राह्मणपूजकः ॥ २ ॥
चत्वारो जज्ञिरे तस्य पुत्रा लोकेषु विश्रुताः ।
रामलक्ष्मणशत्रुघ्ना भरतश्चेति नामतः ॥ ३ ॥
राज्ञः प्रियकराः सर्वे सदृशा गुणरूपतः ।
कौसल्यायाः सुतो रामः कैकेय्या भरतः स्मृतः ॥ ४ ॥
सुमित्रातनयौ जातौ यमलौ द्वौ मनोहरौ ।
ते जाता वै किशोराश्च धनुर्बाणधराः किल ॥ ५ ॥
सूनवः कृतसंस्कारा भूपतेः सुखवर्धकाः ।
कौशिकेन तदाऽऽगत्य प्रार्थितो रघुनन्दनः ॥ ६ ॥
राघवं मखरक्षार्थं सूनुं षोडशवार्षिकम् ।
तस्मै सोऽयं ददौ रामं कौशिकाय सलक्ष्मणम् ॥ ७ ॥
तौ समेत्य मुनिं मार्गे जग्मतुश्चारुदर्शनौ ।
ताटका निहता मार्गे राक्षसी घोरदर्शना ॥ ८ ॥
रामेणैकेन बाणेन मुनीनां दुःखदा सदा ।
यज्ञरक्षा कृता तत्र सुबाहुर्निहतः शठः ॥ ९ ॥
मारीचोऽथ मृतप्रायो निक्षिप्तो बाणवेगतः ।
एवं कृत्वा महत्कर्म यज्ञस्य परिरक्षणम् ॥ १० ॥
गतास्ते मिथिलां सर्वे रामलक्ष्मणकौशिकाः ।
अहल्या मोचिता शापान्निष्पापा सा कृताऽबला ॥ ११ ॥
विदेहनगरे तौ तु जग्मतुर्मुनिना सह ।
बभञ्ज शिवचापञ्च जनकेन पणीकृतम् ॥ १२ ॥
उपयेमे ततः सीतां जानकीञ्च रमांशजाम् ।
लक्ष्मणाय ददौ राजा पुत्रीमेकां तथोर्मिलाम् ॥ १३ ॥
कुशध्वजसुते कन्ये प्रापतुर्भ्रातरावुभौ ।
तथा भरतशत्रुघ्नौ सुशिलौ शुभलक्षणौ ॥ १४ ॥
एवं दारक्रियास्तेषां भ्रातॄणां चाभवन्नृप ।
चतुर्णां मिथिलायां तु यथाविधि विधानतः ॥ १५ ॥
राज्ययोग्यं सुतं दृष्ट्वा राजा दशरथस्तदा ।
राघवाय धुरं दातुं मनश्चक्रे निजाय वै ॥ १६ ॥
सम्भारं विहितं दृष्ट्वा कैकेयी पूर्वकल्पितौ ।
वरौ सम्प्रार्थयामास भर्तारं वशवर्तिनम् ॥ १७ ॥
राज्यं सुताय चैकेन भरताय महात्मने ।
रामाय वनवासञ्च चतुर्दशसमास्तथा ॥ १८ ॥
रामस्तु वचनात्तस्याः सीतालक्ष्मणसंयुतः ।
जगाम दण्डकारण्यं राक्षसैरुपसेवितम् ॥ १९ ॥
राजा दशरथः पुत्रविरहेण प्रपीडितः ।
जहौ प्राणानमेयात्मा पूर्वशापमनुस्मरन् ॥ २० ॥
भरतः पितरं दृष्ट्वा मृतं मातृकृतेन वै ।
राज्यमृद्धं न जग्राह भ्रातुः प्रियचिकीर्षया ॥ २१ ॥
पञ्चवट्यां वसन् रामो रावणावरजां वने ।
शूर्पणखां विरूपां वै चकारातिस्मरातुराम् ॥ २२ ॥
खरादयस्तु तां दृष्ट्वा छिन्ननासां निशाचराः ।
चक्रुः सङ्ग्राममतुलं रामेणामिततेजसा ॥ २३ ॥
स जघान खरादींश्च दैत्यानतिबलान्वितान् ।
मुनीनां हितमन्विच्छन् रामः सत्यपराक्रमः ॥ २४ ॥
गत्वा शूर्पणखा लङ्कां खरदूषणघातनम् ।
दूषिता कथयामास रावणाय च राघवात् ॥ २५ ॥
सोऽपि श्रुत्वा विनाशं तं जातः क्रोधवशः खलः ।
जगाम रथमारुह्य मारीचस्याश्रमं तदा ॥ २६ ॥
कृत्वा हेममृगं नेतुं प्रेषयामास रावणः ।
सीताप्रलोभनार्थाय मायाविनमसम्भवम् ॥ २७ ॥
सोऽथ हेममृगो भूत्वा सीतादृष्टिपथं गतः ।
मायावी चातिचित्राङ्गश्चरन्प्रबलमन्तिके ॥ २८ ॥
तं दृष्ट्वा जानकी प्राह राघवं दैवनोदिता ।
चर्मानयस्व कान्तेति स्वाधीनपतिका यथा ॥ २९ ॥
अविचार्याथ रामोऽपि तत्र संस्थाप्य लक्ष्मणम् ।
सशरं धनुरादाय ययौ मृगपदानुगः ॥ ३० ॥
सारङ्गोऽपि हरिं दृष्ट्वा मायाकोटिविशारदः ।
दृश्यादृश्यो बभूवाथ जगाम च वनान्तरम् ॥ ३१ ॥
मत्वा हस्तगतं रामः क्रोधाकृष्टधनुः पुनः ।
जघान चातितीक्ष्णेन शरेण कृत्रिमं मृगम् ॥ ३२ ॥
स हतोऽतिबलात्तेन चुक्रोश भृशदुःखितः ।
हा लक्ष्मण हतोऽस्मीति मायावी नश्वरः खलः ॥ ३३ ॥
स शब्दस्तुमुलस्तावज्जानक्या संश्रुतस्तदा ।
राघवस्येति सा मत्वा दीना देवरमब्रवीत् ॥ ३४ ॥
गच्छ लक्ष्मण तूर्णं त्वं हतोऽसौ रघुनन्दनः ।
त्वामाह्वयति सौ‌मित्रे साहाय्यं कुरु सत्वरम् ॥ ३५ ॥
तत्राह लक्ष्मणः सीतामम्ब रामवधादपि ।
नाहं गच्छेऽद्य मुक्त्वा त्वामसहायामिहाश्रमे ॥ ३६ ॥
आज्ञा मे राघवस्यात्र तिष्ठेति जनकात्मजे ।
तदतिक्रमभीतोऽहं न त्यजामि तवान्तिकम् ॥ ३७ ॥
दूरं वै राघवं दृष्ट्वा वने मायाविना किल ।
त्यक्त्वा त्वां नाधिगच्छामि पदमेकं शुचिस्मिते ॥ ३८ ॥
कृरु धैर्यं न मन्येऽद्य रामं हन्तुं क्षमं क्षिप्तौ ।
नाहं त्यक्त्वा गमिष्यामि विलंघ्य रामभाषितम् ॥ ।३९ ॥
\uvacha{व्यास उवाच}
रुदती सुदती प्राह ते तदा विधिनोदिता ।
अक्रूरा वचनं क्रूरं लक्ष्मणं शुभलक्षणम् ॥ ४० ॥
अहं जानामि सौ‌मित्रे सानुरागं च मां प्रति ।
प्रेरितं भरतेनैव मदर्थमिह सङ्गतम् ॥ ४१ ॥
नाहं तथाविधा नारी स्वैरिणी कुहकाधम ।
मृते रामे पतिं त्वां न कर्तुमिच्छामि कामतः ॥ ४२ ॥
नागमिष्यति चेद्रामो जीवितं सन्त्यजाम्यहम् ।
विना तेन न जीवामि विधुरा दुःखिता भृशम् ॥ ४३ ॥
गच्छ वा तिष्ठ सौ‍मित्रे न जानेऽहं तवेप्सितम् ।
क्व गतं तेऽद्य सौहार्दं ज्येष्ठे धर्मरते किल ॥ ४४ ॥
तच्छ्रुत्वा वचनं तस्या लक्ष्मणो दीनमानसः ।
प्रोवाच रुद्धकण्ठस्तु तां तदा जनकात्मजाम् ॥ ४५ ॥
किमात्थ क्षितिजे वाक्यं मयि क्रूरतरं किल ।
किं वदस्यत्यनिष्टं ते भावि जाने धिया ह्यहम् ॥ ४६ ॥
इत्युक्त्वा निर्ययौ वीरस्तां त्यक्त्वा प्ररुदन्भृशम् ।
अग्रजस्य ययौ पश्यञ्छोकार्तः पृथिवीपते ॥ ४७ ॥
गतेऽथ लक्ष्मणे तत्र रावणः कपटाकृतिः ।
भिक्षुवेषं ततः कृत्वा प्रविवेश तदाश्रमे ॥ ४८ ॥
जानकी तं यतिं मत्वा दत्त्वार्घ्यं वन्यमादरात् ।
भैक्ष्यं समर्पयामास रावणाय दुरात्मने ॥ ४९ ॥
तां पप्रच्छ स दुष्टात्मा नम्रपूर्वं मृदुस्वरम् ।
काऽसि पद्मपलाशाक्षि वने चैकाकिनी प्रिये ॥ ५० ॥
पिता कस्तेऽथ वामोरु भ्राता कः कः पतिस्तव ।
मूढेवैकाकिनी चात्र स्थिताऽसि वरवर्णिनि ॥ ५१ ॥
निर्जने विपिने किं त्वं सौधार्हा त्वमसि प्रिये ।
उटजे मुनिपत्‍नीवद्देवकन्यासमप्रभा ॥ ५२ ॥
\uvacha{व्यास उवाच}
इति तद्वचनं श्रुत्वा प्रत्युवाच विदेहजा ।
दिव्यं दिष्ट्या यतिं ज्ञात्वा मन्दोदर्याः पतिं तदा ॥ ५३ ॥
राजा दशरथः श्रीमांश्चत्वारस्तस्य वै सुताः ।
तेषां ज्येष्ठः पतिर्मेऽस्ति रामनामेति विश्रुतः ॥ ५४ ॥
विवासितोऽथ कैकेय्या कृते भूपतिना वरे ।
चतुर्दश समा रामो वसतेऽत्र सलक्ष्मणः ॥ ५५ ॥
जनकस्य सुता चाहं सीतानाम्नीति विश्रुता ।
भङ्क्त्वा शैवं धनुः कामं रामेणाहं विवाहिता ॥ ५६ ॥
रामबाहुबलेनात्र वसामो निर्भया वने ।
काञ्चनं मृगमालोक्य हन्तुं मे निर्गतः पतिः ॥ ५७ ॥
लक्ष्मणोऽपि पुनः श्रुत्वा रवं भ्रातुर्गतोऽधुना ।
तयोर्बाहुबलादत्र निर्भयाऽहं वसामि वै ॥ ५८ ॥
मयेदं कथितं सर्वं वृत्तान्तं वनवासके ।
तेऽत्रागत्यार्हणां ते वै करिष्यन्ति यथाविधि ॥ ५९ ॥
यतिर्विष्णुस्वरूपोऽसि तस्मात्त्वं पूजितो मया ।
आश्रमो विपिने घोरे कृतोऽस्ति रक्षसां कुले ॥ ६० ॥
तस्मात्त्वां परिपृच्छामि सत्यं ब्रूहि ममाग्रतः ।
कोऽसि त्रिदण्डिरूपेण विपिने त्वं समागतः ॥ ६१ ॥

\uvacha{रावण उवाच}
लङ्केशोऽहं मरालाक्षि श्रीमान्मन्दोदरीपतिः ।
त्वत्कृते तु कृतं रूपं मयेत्थं शोभनाकृते ॥ ६२ ॥
आगतोऽहं वरारोहे भगिन्या प्रेरितोऽत्र वै ।
जनस्थाने हतौ श्रुत्वा भ्रातरौ खरदूषणौ ॥ ६३ ॥
अङ्गीकुरु नृपं मां त्वं त्यक्त्वा तं मानुषं पतिम् ।
हृतराज्यं गतश्रीकं निर्बलं वनवासिनम् ॥ ६४ ॥
पट्टराज्ञी भव त्वं मे मन्दोदर्युपरि स्फुटम् ।
दासोऽस्मि तव तन्वङ्‌गि स्वामिनी भव भामिनि ॥ ६५ ॥
जेताऽहं लोकपालानां पतामि तव पादयोः ।
करं गृहाण मेऽद्य त्वं सनाथं कुरु जानकि ॥ ६६ ॥
पिता ते याचितः पूर्वं मया वै त्वत्कृतेऽबले ।
जनको मामुवाचेत्थं पणबन्धो मया कृतः ॥ ६७ ॥
रुद्रचापभयान्नाहं सम्प्राप्तस्तु स्वयंवरे ।
मनो मे संस्थितं तावन्निमग्नं विरहातुरम् ॥ ६८ ॥
वनेऽत्र संस्थितां श्रुत्वा पूर्वानुरागमोहितः ।
आगतोऽस्म्यसितापाङ्‌गि सफलं कुरु मे श्रमम् ॥ ६९ ॥


इति श्रीदेवीभागवते महापुराणेऽष्टादशसाहस्र्यां संहितायां तृतीयस्कन्धे रामचरित्रवर्णनं नाम अष्टाविंशोऽध्यायः ॥ २८ ॥


लक्ष्मणकृतरामशोकसान्त्वनम्

\uvacha{व्यास उवाच}
तदाकर्ण्य वचो दुष्टं जानकी भयविह्वला ।
वेपमाना स्थिरं कृत्वा मनो वाचमुवाच ह ॥ १ ॥
पौलस्त्य किमसद्वाक्यं त्वमात्थ स्मरमोहितः ।
नाहं वै स्वैरिणी किन्तु जनकस्य कुलोद्‌भवा ॥ २ ॥
गच्छ लङ्कां दशास्य त्वं राम त्वां वै हनिष्यति ।
मत्कृते मरणं तत्र भविष्यति न संशयः ॥ ३ ॥
इत्युक्त्वा पर्णशालायां गता सा वह्निसन्निधौ ।
गच्छ गच्छेति वदती रावणं लोकरावणम् ॥ ४ ॥
सोऽथ कृत्वा निजं रूपं जगामोटजमन्तिकम् ।
बलाज्जग्राह तां बालां रुदती भयविह्वलाम् ॥ ५ ॥
रामरामेति क्रन्दन्ती लक्ष्मणेति मुहुर्मुहुः ।
गृहीत्वा निर्गतः पापो रथमारोप्य सत्वरः ॥ ६ ॥
गच्छन्नरुणपुत्रेण मार्गे रुद्धो जटायुषा ।
सङ्ग्रामोऽभून्महारौद्रस्तयोस्तत्र वनान्तरे ॥ ७ ॥
हत्वा तं तां गृहीत्वा च गतोऽसौ राक्षसाधिपः ।
लङ्कायां क्रन्दती तात कुररीव दुरात्मनः ॥ ८ ॥
अशोकवनिकायां सा स्थापिता राक्षसीयुता ।
स्ववृत्तान्नैव चलिता सामदानादिभिः किल ॥ ९ ॥
रामोऽपि तं मृगं हत्वा जगामादाय निर्वृतः ।
आयान्तं लक्ष्मणं वीक्ष्य किं कृतं तेऽनुजासमम् ॥ १० ॥
एकाकिनीं प्रियां हित्वा किमर्थं त्वमिहागतः ।
श्रुत्वा स्वनं तु पापस्य राघवस्त्वब्रवीदिदम् ॥ ११ ॥
सौ‌मित्रिस्त्वब्रवीद्वाक्यं सीतावाग्बाणपीडितः ।
प्रभोऽत्राहं समायातः कालयोगान्न संशयः ॥ १२ ॥
तदा तौ पर्णशालायां गत्वा वीक्ष्यातिदुःखितौ ।
जानक्यन्वेषणे यत्‍नमुभौ कर्तुं समुद्यतौ ॥ १३ ॥
मार्गमाणौ तु सम्प्राप्तौ यत्रासौ पतितः खगः ।
जटायुः प्राणशेषस्तु पतितः पृथिवीतले ॥ १४ ॥
तेनोक्तं रावणेनाद्य हृता‍‍ऽसौ जनकात्मजा ।
मया निरुद्धः पापात्मा पातितोऽहं मृधे पुनः ॥ १५ ॥
इत्युक्त्वाऽसौ गतप्राणः संस्कृतो राघवेण वै ।
कृत्वौर्घ्वदैहिकं रामलक्ष्मणौ निर्गतौ ततः ॥ १६ ॥
कबन्धं घातयित्वासौ शापाच्चामोचयत्प्रभुः ।
वचनात्तस्य हरिणा सख्यं चक्रेऽथ राघवः ॥ १७ ॥
हत्वा च वालिनं वीरं किष्किन्धाराज्यमुत्तमम् ।
सुग्रीवाय ददौ रामः कृतसख्याय कार्यतः ॥ १८ ॥
तत्रैव वार्षिकान्मासांस्तस्थौ लक्ष्मणसंयुतः ।
चिन्तयञ्जानकीं चित्ते दशाननहृतां प्रियाम् ॥ १९ ॥
लक्ष्मणं प्राह रामस्तु सीताविरहपीडितः ।
सौ‌मित्रे कैकयसुता जाता पूर्णमनोरथा ॥ २० ॥
न प्राप्ता जानकी नूनं नाहं जीवामि तां विना ।
नागमिष्याम्ययोध्यायामृते जनकनन्दिनीम् ॥ २१ ॥
गतं राज्यं वने वासो मृतस्तातो हृता प्रिया ।
पीडयन्मां स दुष्टात्मा दैवो‍ऽग्रे किं करिष्यति ॥ २२ ॥
दुर्ज्ञेयं भवितव्यं हि प्राणिनां भरतानुज ।
आवयोः का गतिस्तात भविष्यति सुदुःखदा ॥ २३ ॥
प्राप्य जन्म मनोर्वंशे राजपुत्रावुभौ किल ।
वनेऽतिदुःखभोक्तारौ जातौ पूर्वकृतेन च ॥ २४ ॥
त्यक्त्वा त्वमपि भोगांस्तु मया सह विनिर्गतः ।
दैवयोगाच्च सौ‌मित्रे भुंक्ष्व दुःखं दुरत्ययम् ॥ २५ ॥
न कोऽप्यस्मत्कुले पूर्वं मत्समो दुःखभाङ्नरः ।
अकिञ्चनोऽक्षमः क्लिष्टो न भूतो न भविष्यति ॥ २६ ॥
किं करोम्यद्य सौ‌मित्रे मग्नोऽस्मि दुःखसागरे ।
न चास्ति तरणोपायो ह्यसहायस्य मे किल ॥ २७ ॥
न वित्तं न बलं वीर त्वमेकः सहचारकः ।
कोपं कस्मिन्करोम्यद्य भोगेस्मिन्स्वकृतेऽनुज ॥ २८ ॥
गतं हस्तगतं राज्यं क्षणादिन्द्रासनोपमम् ।
वने वासस्तु सम्प्राप्तः को वेद विधिनिर्मितम् ॥ २९ ॥
बालभावाच्च वैदेही चलिता चावयोः सह ।
नीता दैवेन दुष्टेन श्यामा दुःखतरां दशाम् ॥ ३० ॥
लङ्केशस्य गृहे श्यामा कथं दुःखं भविष्यति ।
पतिव्रता सुशीला च मयि प्रीतियुता भृशम् ॥ ३१ ॥
न च लक्ष्मण वैदेही सा तस्य वशगा भवेत् ।
स्वैरिणीव वरारोहा कथं स्याज्जनकात्मजा ॥ ३२ ॥
त्यजेत्प्राणान्नियन्तृत्वे मैथिली भरतानुज ।
न रावणस्य वशगा भवेदिति सुनिश्चितम् ॥ ३३ ॥
मृता चेज्जानकी वीर प्राणांस्त्यक्ष्याम्यसंशयम् ।
मृता चेदसितापाङ्गीं किं मे देहेन लक्ष्मण ॥ ३४ ॥
एवं विलपमानं तं रामं कमललोचनम् ।
लक्ष्मणः प्राह धर्मात्मा सान्त्वयन्नृतया गिरा ॥ ३५ ॥
धैर्यं कुरु महाबाहो त्यक्त्वा कातरतामिह ।
आनयिष्यामि वैदेहीं हत्वा तं राक्षसाधमम् ॥ ३६ ॥
आपदि सम्पदि तुल्या धैर्याद्‌भवन्ति ते धीराः ।
अल्पधियस्तु निमग्नाः कष्टे भवन्ति विभवेऽपि ॥ ३७ ॥
संयोगो विप्रयोगश्च दैवाधीनावुभावपि ।
शोकस्तु कीदृशस्तत्र देहेनात्मनि च क्वचित् ॥ ३८ ॥
राज्याद्यथा वने वासो वैदेह्या हरणं यथा ।
तथा काले समीचीने संयोगोऽपि भविष्यति ॥ ३९ ॥
प्राप्तव्यं सुखदुःखानां भोगान्निर्वर्तनं क्वचित् ।
नान्यथा जानकीजाने तस्माच्छोकं त्यजाधुना ॥ ४० ॥
वानराः सन्ति भूयांसो गमिष्यन्ति चतुर्दिशम् ।
शुद्धिं जनकनन्दिन्या आनयिष्यन्ति ते किल ॥ ४१ ॥
ज्ञात्वा मार्गस्थितिं तत्र गत्वा कृत्वा पराक्रमम् ।
हत्वा तं पापकर्माणमानयिष्यामि मैथिलीम् ॥ ४२ ॥
ससैन्यं भरतं वाऽपि समाहूय सहानुजम् ।
हनिष्यामो वयं शत्रुं किं शोचसि वृथाग्रज ॥ ४३ ॥
रघुणैकरथेनैव जिताः सर्वा दिशः पुरा ।
तद्वंशजः कथं शोकं कर्तुमर्हसि राघव ॥ ४४ ॥
एकोऽहं सकलाञ्जेतुं समर्थोऽस्मि सुरासुरान् ।
किं पुनः ससहायो वै रावणं कुलपांसनम् ॥ ४५ ॥
जनकं वा समानीय साहाय्ये रघुनन्दन ।
हनिष्यामि दुराचारं रावणं सुरकण्टकम् ॥ ४६ ॥
सुखस्यानन्तरं दुःखं दुःखस्यानन्तरं सुखम् ।
चक्रनेमिरिवैकं यन्न भवेद्‌रघुनन्दन ॥ ४७ ॥
मनोऽतिकातरं यस्य सुखदुःखसमुद्‌भवे ।
स शोकसागरे मग्नो न सुखी स्यात्कदाचन ॥ ४८ ॥
इन्द्रेण व्यसनं प्राप्तं पुरा वै रघुनन्दन ।
नहुषः स्थापितो देवैः सर्वैर्मघवतः पदे ॥ ४९ ॥
स्थितः पङ्कजमध्ये च बहुवर्षगणानपि ।
अज्ञातवासं मघवा भीतस्त्यक्त्वा निजं पदम् ॥ ५० ॥
पुनः प्राप्तं निजस्थानं काले विपरिवर्तिते ।
नहुषः पतितो भूमौ शापादजगराकृतिः ॥ ५१ ॥
इन्द्राणीं कामयानस्तु ब्राह्मणानवमन्य च ।
अगस्तिकोपात्सञ्जातः सर्पदेहो महीपतिः ॥ ५२ ॥
तस्माच्छोको न कर्तव्यो व्यसने सति राघव ।
उद्यमे चित्तमास्थाय स्थातव्यं वै विपश्चिता ॥ ५३ ॥
सर्वज्ञोऽसि महाभाग समर्थोऽसि जगत्पते ।
किं प्राकृत इवात्यर्थं कुरुषे शोकमात्मनि ॥ ५४ ॥
\uvacha{व्यास उवाच}
इति लक्ष्मणवाक्येन बोधितो रघुनन्दनः ।
त्यक्त्वा शोकं तथात्यर्थं बभूव विगतज्वरः ॥ ५५ ॥


इति श्रीदेवीभागवते महापुराणेऽष्टादशसाहस्र्यां संहितायां तृतीयस्कन्धे लक्ष्मणकृतरामशोकसान्त्वनं नामैकोनत्रिंशोऽध्यायः ॥ २९ ॥


रामाय देवीवरदानम्

\uvacha{व्यास उवाच}
एवं तौ संविदं कृत्वा यावत्तूष्णीं बभूवतुः ।
आजगाम तदाऽऽकाशान्नारदो भगवानृषिः ॥ १ ॥
रणयन्महतीं वीणां स्वरग्रामविभूषिताम् ।
गायन्बृहद्रथं साम तदा तमुपतस्थिवान् ॥ २ ॥
दृष्ट्वा तं राम उत्थाय ददावथ वृषं शुभम् ।
आसनं चार्घ्यपाद्यञ्च कृतवानमितद्युतिः ॥ ३ ॥
पूजां परमिकां कृत्वा कृताञ्जलिरुपस्थितः ।
उपविष्टः समीपे तु कृताज्ञो मुनिना हरिः ॥ ४ ॥
उपविष्टं तदा रामं सानुजं दुःखमानसम् ।
पप्रच्छ नारदः प्रीत्या कुशलं मुनिसत्तमः ॥ ५ ॥
कथं राघव शोकार्तो यथा वै प्राकृतो नरः ।
हृतां सीतां च जानामि रावणेन दुरात्मना ॥ ६ ॥
सुरसद्मगतश्चाहं श्रुतवाञ्जनकात्मजाम् ।
पौलस्त्येन हृतां मोहान्मरणं स्वमजानता ॥ ७ ॥
तव जन्म च काकुत्स्थ पौलस्त्यनिधनाय वै ।
मैथिलीहरणं जातमेतदर्थं नराधिप ॥ ८ ॥
पूर्वजन्मनि वैदेही मुनिपुत्री तपस्विनी ।
रावणेन वने दृष्टा तपस्यन्ती शुचिस्मिता ॥ ९ ॥
प्रार्थिता रावणेनासौ भव भार्येति राघव ।
तिरस्कृतस्तयाऽसौ वै जग्राह कबरं बलात् ॥ १० ॥
शशाप तत्क्षणं राम रावणं तापसी भृशम् ।
कुपिता त्यक्तुमिच्छन्ती देहं संस्पर्शदूषितम् ॥ ११ ॥
दुरात्मंस्तव नाशार्थं भविष्यामि धरातले ।
अयोनिजा वरा नारी त्यक्त्वा देहं जहावपि ॥ १२ ॥
सेयं रमांशसम्भूता गृहीता तेन रक्षसा ।
विनाशार्थं कुलस्यैव व्याली स्रगिव सम्भ्रमात् ॥ १३ ॥
तव जन्म च काकुत्स्थ तस्य नाशाय चामरैः ।
प्रार्थितस्य हरेरंशादजवंशेऽप्यजन्मनः ॥ १४ ॥
कुरु धैर्यं महाबाहो तत्र सा वर्ततेऽवशा ।
सती धर्मरता सीता त्वां ध्यायन्ती दिवानिशम् ॥ १५ ॥
कामधेनुपयः पात्रे कृत्वा मघवता स्वयम् ।
पानार्थं प्रेषितं तस्याः पीतं चैवामृतं यथा ॥ १६ ॥
सुरभीदुग्धपानात्सा क्षुत्तुड्‌दुःखविवर्जिता ।
जाता कमलपत्राक्षी वर्तते वीक्षिता मया ॥ १७ ॥
उपायं कथयाम्यद्य तस्य नाशाय राघव ।
व्रतं कुरुष्व श्रद्धावानाश्विने मासि साम्प्रतम् ॥ १८ ॥
नवरात्रोपवासञ्च भगवत्याः प्रपूजनम् ।
सर्वसिद्धिकरं राम जपहोमविधानतः ॥ १९ ॥
मेघ्यैश्च पशुभिर्देव्या बलिं दत्त्वा विशंसितैः ।
दशांशं हवनं कृत्वा सशक्तस्त्वं भविष्यसि ॥ २० ॥
विष्णुना चरितं पूर्वं महादेवेन ब्रह्मणा ।
तथा मघवता चीर्णं स्वर्गमध्यस्थितेन वै ॥ २१ ॥
सुखिना राम कर्तव्यं नवरात्रव्रतं शुभम् ।
विशेषेण च कर्तव्यं पुंसा कष्टगतेन वै ॥ २२ ॥
विश्वामित्रेण काकुत्स्थ कृतमेतन्न संशयः ।
भृगुणाऽथ वसिष्ठेन कश्यपेन तथैव च ॥ २३ ॥
गुरुणा हृतदारेण कृतमेतन्महाव्रतम् ।
तस्मात्त्वं कुरु राजेन्द्र रावणस्य वधाय च ॥ २४ ॥
इन्द्रेण वृत्रनाशाय कृतं व्रतमनुत्तमम् ।
त्रिपुरस्य विनाशाय शिवेनापि पुरा कृतम् ॥ २५ ॥
हरिणा मधुनाशाय कृतं मेरौ महामते ।
विधिवत्कुरु काकुत्स्थ व्रतमेतदतन्द्रितः ॥ २६ ॥
\uvacha{श्रीराम उवाच}
का देवी किं प्रभावा सा कुतो जाता किमाह्वया ।
व्रतं किं विधिवद्‌ब्रूहि सर्वज्ञोऽसि दयानिधे ॥ २७ ॥
\uvacha{नारद उवाच}
शृणु राम सदा नित्या शक्तिराद्या सनातनी ।
सर्वकामप्रदा देवी पूजिता दुःखनाशिनी ॥ २८ ॥
कारणं सर्वजन्तूनां ब्रह्मादीनां रघूद्वह ।
तस्याः शक्तिं विना कोऽपि स्पन्दितुं न क्षमो भवेत् ॥ २९ ॥
विष्णोः पालनशक्तिः सा कर्तृशक्तिः पितुर्मम ।
रुद्रस्य नाशशक्तिः सा त्वन्याशक्तिः परा शिवा ॥ ३० ॥
यच्च किञ्चित्क्वचिद्वस्तु सदसद्‌भुवनत्रये ।
तस्य सर्वस्य या शक्तिस्तदुत्पत्तिः कुतो भवेत् ॥ ३१ ॥
न ब्रह्मा न यदा विष्णुर्न रुद्रो न दिवाकरः ।
न चेन्द्राद्याः सुराः सर्वे न धरा न धराधराः ॥ ३२ ॥
तदा सा प्रकृतिः पूर्णा पुरुषेण परेण वै ।
संयुता विहरत्येव युगादौ निर्गुणा शिवा ॥ ३३ ॥
सा भूत्वा सगुणा पश्चात्करोति भुवनत्रयम् ।
पूर्वं संसृज्य ब्रह्मादीन्दत्त्वा शक्तीश्च सर्वशः ॥ ३४ ॥
तां ज्ञात्वा मुच्यते जन्तुर्जन्मसंसारबन्धनात् ।
सा विद्या परमा ज्ञेया वेदाद्या वेदकारिणी ॥ ३५ ॥
असंख्यातानि नामानि तस्या ब्रह्मादिभिः किल ।
गुणकर्मविधानैस्तु कल्पितानि च किं ब्रुवे ॥ ३६ ॥
अकारादिक्षकारान्तैः स्वरैर्वर्णैस्तु योजितैः ।
असंख्येयानि नामानि भवन्ति रघुनन्दन ॥ ३७ ॥

\uvacha{राम उवाच}
विधिं मे ब्रूहि विप्रर्षे व्रतस्यास्य समासतः ।
करोम्यद्यैव श्रद्धावाञ्छ्रीदेव्याः पूजनं तथा ॥ ३८ ॥
\uvacha{नारद उवाच}
पीठं कृत्वा समे स्थाने संस्थाप्य जगदम्बिकाम् ।
उपवासान्नवैव त्वं कुरु राम विधानतः ॥ ३९ ॥
आचार्योऽहं भविष्यामि कर्मण्यस्मिन्महीपते ।
देवकार्यविधानार्थमुत्साहं प्रकरोम्यहम् ॥ ४० ॥
\uvacha{व्यास उवाच}
तच्छ्रुत्वा वचनं सत्यं मत्वा रामः प्रतापवान् ।
कारयित्वा शुभं पीठं स्थापयित्वाम्बिकां शिवाम् ॥ ४१ ॥
विधिवत्पूजनं तस्याश्चकार व्रतवान् हरिः ।
सम्प्राप्ते चाश्विने मासि तस्मिन्गिरिवरे तदा ॥ ४२ ॥
उपवासपरो रामः कृतवान्व्रतमुत्तमम् ।
होमञ्च विधिवत्तत्र बलिदानञ्च पूजनम् ॥ ४३ ॥
भ्रातरौ चक्रतुः प्रेम्णा व्रतं नारदसम्मतम् ।
अष्टम्यां मध्यरात्रे तु देवी भगवती हि सा ॥ ४४ ॥
सिंहारूढा ददौ तत्र दर्शनं प्रतिपूजिता ।
गिरिशृङ्गे स्थितोवाच राघवं सानुजं गिरा ॥ ४५ ॥
मेघगम्भीरया चेदं भक्तिभावेन तोषिता ।

देव्युवाच
राम राम महाबाहो तुष्टाऽस्म्यद्म व्रतेन ते ॥ ४६ ॥
प्रार्थयस्व वरं कामं यत्ते मनसि वर्तते ।
नारायणांशसम्भूतस्त्वं वंशे मानवेऽनघे ॥ ४७ ॥
रावणस्य वधायैव प्रार्थितस्त्वमरैरसि ।
पुरा मत्स्यतनुं कृत्वा हत्वा घोरञ्च राक्षसम् ॥ ४८ ॥
त्वया वै रक्षिता वेदाः सुराणां हितमिच्छता ।
भूत्वा कच्छपरूपस्तु धृतवान्मन्दरं गिरिम् ॥ ४९ ॥
अकूपारं प्रमन्थानं कृत्वा देवानपोषयः ।
कोलरूपं परं कृत्वा दशनाग्रेण मेदिनीम् ॥ ५० ॥
धृतवानसि यद्‌राम हिरण्याक्षं जघान च ।
नारसिंहीं तनुं कृत्वा हिरण्यकशिपुं पुरा ॥ ५१ ॥
प्रह्लादं राम रक्षित्वा हतवानसि राघव ।
वामनं वपुरास्थाय पुरा छलितवान्बलिम् ॥ ५२ ॥
भूत्वेन्द्रस्यानुजः कामं देवकार्यप्रसाधकः ।
जमदग्निसुतस्त्वं मे विष्णोरंशेन सङ्गतः ॥ ५३ ॥
कृत्वान्तं क्षत्रियाणां तु दानं भूमेरदाद्‌द्विजे ।
तथेदानीं तु काकुत्स्थ जातो दशरथात्मज ॥ ५४ ॥
प्रार्थितस्तु सुरैः सर्वै रावणेनातिपीडितैः ।
कपयस्ते सहाया वै देवांशा बलवत्तराः ॥ ५५ ॥
भविष्यन्ति नरव्याघ्र मच्छक्तिसंयुता ह्यमी ।
शेषांशोऽप्यनुजस्तेऽयं रावणात्मजनाशकः ॥ ५६ ॥
भविष्यति न सन्देहः कर्तव्योऽत्र त्वयाऽनघ ।
वसन्ते सेवनं कार्यं त्वया तत्रातिश्रद्धया ॥ ५७ ॥
हत्वाऽथ रावणं पापं कुरु राज्यं यथासुखम् ।
एकादश सहस्राणि वर्षाणि पृथिवीतले ॥ ५८ ॥
कृत्वा राज्यं रघुश्रेष्ठ गन्ताऽसि त्रिदिवं पुनः ।
\uvacha{व्यास उवाच}
इत्युक्त्वान्तर्दधे देवी रामस्तु प्रीतमानसः ॥ ५९ ॥
समाप्य तद्‌व्रतं चक्रे प्रयाणं दशमीदिने ।
विजयापूजनं कृत्वा दत्त्वा दानान्यनेकशः ॥ ६० ॥
कपिपतिबलयुक्तः सानुजः श्रीपतिश्च
     प्रकटपरमशक्त्या प्रेरितः पूर्णकामः ।
उदधितटगतोऽसौ सेतुबन्धं विधाया-
     प्यहनदमरशत्रुं रावणं गीतकीर्तिः ॥ ६१ ॥
यः शृणोति नरो भक्त्या देव्याश्चरितमुत्तमम् ।
स भुक्त्वा विपुलान्भोगान्प्राप्नोति परमं पदम् ॥ ६२ ॥
सन्त्यन्यानि पुराणानि विस्तराणि बहूनि च ।
श्रीमद्‌भागवतस्यास्य न तुल्यानीति मे मतिः ॥ ६३ ॥

इति श्रीदेवीभागवते महापुराणेऽष्टादशसाहस्र्यां संहितायां तृतीयस्कन्धे रामाय देवीवरदानं नाम त्रिंशोऽध्यायः ॥ ३० ॥