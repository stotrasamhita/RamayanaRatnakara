\sect{लक्ष्मणकृतरामशोकसान्त्वनम्}

\src{देवी-भागवतम्}{तृतीयः स्कन्धः}{अध्यायः २८}{श्लोकाः १--५५}
\vakta{व्यासः}
\shrota{जनमेजयः}
\tags{concise, complete}
\notes{This chapter describes how Sita rejected Ravana's advances and was forcibly abducted to Lanka despite Jatayu's heroic attempt to stop him, followed by Rama's discovery of Her disappearance and His profound grief, being consoled by Lakshman's encouraging words about the cyclical nature of fortune and Their ability to rescue Her with the help of Their vānara allies.}
\textlink{https://sa.wikisource.org/wiki/देवीभागवतपुराणम्/स्कन्धः_०३/अध्यायः_२९}
\translink{}

\storymeta

\uvacha{व्यास उवाच}


\twolineshloka
{तदाकर्ण्य वचो दुष्टं जानकी भयविह्वला}
{वेपमाना स्थिरं कृत्वा मनो वाचमुवाच ह}% ॥ १ ॥

\twolineshloka
{पौलस्त्य किमसद्वाक्यं त्वमात्थ स्मरमोहितः}
{नाहं वै स्वैरिणी किन्तु जनकस्य कुलोद्‌भवा}% ॥ २ ॥

\twolineshloka
{गच्छ लङ्कां दशास्य त्वं राम त्वां वै हनिष्यति}
{मत्कृते मरणं तत्र भविष्यति न संशयः}% ॥ ३ ॥

\twolineshloka
{इत्युक्त्वा पर्णशालायां गता सा वह्निसन्निधौ}
{गच्छ गच्छेति वदती रावणं लोकरावणम्}% ॥ ४ ॥

\twolineshloka
{सोऽथ कृत्वा निजं रूपं जगामोटजमन्तिकम्}
{बलाज्जग्राह तां बालां रुदती भयविह्वलाम्}% ॥ ५ ॥

\twolineshloka
{रामरामेति क्रन्दन्ती लक्ष्मणेति मुहुर्मुहुः}
{गृहीत्वा निर्गतः पापो रथमारोप्य सत्वरः}% ॥ ६ ॥

\twolineshloka
{गच्छन्नरुणपुत्रेण मार्गे रुद्धो जटायुषा}
{सङ्ग्रामोऽभून्महारौद्रस्तयोस्तत्र वनान्तरे}% ॥ ७ ॥

\twolineshloka
{हत्वा तं तां गृहीत्वा च गतोऽसौ राक्षसाधिपः}
{लङ्कायां क्रन्दती तात कुररीव दुरात्मनः}% ॥ ८ ॥

\twolineshloka
{अशोकवनिकायां सा स्थापिता राक्षसीयुता}
{स्ववृत्तान्नैव चलिता सामदानादिभिः किल}% ॥ ९ ॥

\twolineshloka
{रामोऽपि तं मृगं हत्वा जगामादाय निर्वृतः}
{आयान्तं लक्ष्मणं वीक्ष्य किं कृतं तेऽनुजासमम्}% ॥ १० ॥

\twolineshloka
{एकाकिनीं प्रियां हित्वा किमर्थं त्वमिहागतः}
{श्रुत्वा स्वनं तु पापस्य राघवस्त्वब्रवीदिदम्}% ॥ ११ ॥

\twolineshloka
{सौ‌मित्रिस्त्वब्रवीद्वाक्यं सीतावाग्बाणपीडितः}
{प्रभोऽत्राहं समायातः कालयोगान्न संशयः}% ॥ १२ ॥

\twolineshloka
{तदा तौ पर्णशालायां गत्वा वीक्ष्यातिदुःखितौ}
{जानक्यन्वेषणे यत्‍नमुभौ कर्तुं समुद्यतौ}% ॥ १३ ॥

\twolineshloka
{मार्गमाणौ तु सम्प्राप्तौ यत्रासौ पतितः खगः}
{जटायुः प्राणशेषस्तु पतितः पृथिवीतले}% ॥ १४ ॥

\twolineshloka
{तेनोक्तं रावणेनाद्य हृता‍‍ऽसौ जनकात्मजा}
{मया निरुद्धः पापात्मा पातितोऽहं मृधे पुनः}% ॥ १५ ॥

\twolineshloka
{इत्युक्त्वाऽसौ गतप्राणः संस्कृतो राघवेण वै}
{कृत्वौर्घ्वदैहिकं रामलक्ष्मणौ निर्गतौ ततः}% ॥ १६ ॥

\twolineshloka
{कबन्धं घातयित्वासौ शापाच्चामोचयत्प्रभुः}
{वचनात्तस्य हरिणा सख्यं चक्रेऽथ राघवः}% ॥ १७ ॥

\twolineshloka
{हत्वा च वालिनं वीरं किष्किन्धाराज्यमुत्तमम्}
{सुग्रीवाय ददौ रामः कृतसख्याय कार्यतः}% ॥ १८ ॥

\twolineshloka
{तत्रैव वार्षिकान्मासांस्तस्थौ लक्ष्मणसंयुतः}
{चिन्तयञ्जानकीं चित्ते दशाननहृतां प्रियाम्}% ॥ १९ ॥

\twolineshloka
{लक्ष्मणं प्राह रामस्तु सीताविरहपीडितः}
{सौ‌मित्रे कैकयसुता जाता पूर्णमनोरथा}% ॥ २० ॥

\twolineshloka
{न प्राप्ता जानकी नूनं नाहं जीवामि तां विना}
{नागमिष्याम्ययोध्यायामृते जनकनन्दिनीम्}% ॥ २१ ॥

\twolineshloka
{गतं राज्यं वने वासो मृतस्तातो हृता प्रिया}
{पीडयन्मां स दुष्टात्मा दैवो‍ऽग्रे किं करिष्यति}% ॥ २२ ॥

\twolineshloka
{दुर्ज्ञेयं भवितव्यं हि प्राणिनां भरतानुज}
{आवयोः का गतिस्तात भविष्यति सुदुःखदा}% ॥ २३ ॥

\twolineshloka
{प्राप्य जन्म मनोर्वंशे राजपुत्रावुभौ किल}
{वनेऽतिदुःखभोक्तारौ जातौ पूर्वकृतेन च}% ॥ २४ ॥

\twolineshloka
{त्यक्त्वा त्वमपि भोगांस्तु मया सह विनिर्गतः}
{दैवयोगाच्च सौ‌मित्रे भुंक्ष्व दुःखं दुरत्ययम्}% ॥ २५ ॥

\twolineshloka
{न कोऽप्यस्मत्कुले पूर्वं मत्समो दुःखभाङ्नरः}
{अकिञ्चनोऽक्षमः क्लिष्टो न भूतो न भविष्यति}% ॥ २६ ॥

\twolineshloka
{किं करोम्यद्य सौ‌मित्रे मग्नोऽस्मि दुःखसागरे}
{न चास्ति तरणोपायो ह्यसहायस्य मे किल}% ॥ २७ ॥

\twolineshloka
{न वित्तं न बलं वीर त्वमेकः सहचारकः}
{कोपं कस्मिन्करोम्यद्य भोगेस्मिन्स्वकृतेऽनुज}% ॥ २८ ॥

\twolineshloka
{गतं हस्तगतं राज्यं क्षणादिन्द्रासनोपमम्}
{वने वासस्तु सम्प्राप्तः को वेद विधिनिर्मितम्}% ॥ २९ ॥

\twolineshloka
{बालभावाच्च वैदेही चलिता चावयोः सह}
{नीता दैवेन दुष्टेन श्यामा दुःखतरां दशाम्}% ॥ ३० ॥

\twolineshloka
{लङ्केशस्य गृहे श्यामा कथं दुःखं भविष्यति}
{पतिव्रता सुशीला च मयि प्रीतियुता भृशम्}% ॥ ३१ ॥

\twolineshloka
{न च लक्ष्मण वैदेही सा तस्य वशगा भवेत्}
{स्वैरिणीव वरारोहा कथं स्याज्जनकात्मजा}% ॥ ३२ ॥

\twolineshloka
{त्यजेत्प्राणान्नियन्तृत्वे मैथिली भरतानुज}
{न रावणस्य वशगा भवेदिति सुनिश्चितम्}% ॥ ३३ ॥

\twolineshloka
{मृता चेज्जानकी वीर प्राणांस्त्यक्ष्याम्यसंशयम्}
{मृता चेदसितापाङ्गीं किं मे देहेन लक्ष्मण}% ॥ ३४ ॥

\twolineshloka
{एवं विलपमानं तं रामं कमललोचनम्}
{लक्ष्मणः प्राह धर्मात्मा सान्त्वयन्नृतया गिरा}% ॥ ३५ ॥

\twolineshloka
{धैर्यं कुरु महाबाहो त्यक्त्वा कातरतामिह}
{आनयिष्यामि वैदेहीं हत्वा तं राक्षसाधमम्}% ॥ ३६ ॥

\twolineshloka
{आपदि सम्पदि तुल्या धैर्याद्‌भवन्ति ते धीराः}
{अल्पधियस्तु निमग्नाः कष्टे भवन्ति विभवेऽपि}% ॥ ३७ ॥

\twolineshloka
{संयोगो विप्रयोगश्च दैवाधीनावुभावपि}
{शोकस्तु कीदृशस्तत्र देहेनात्मनि च क्वचित्}% ॥ ३८ ॥

\twolineshloka
{राज्याद्यथा वने वासो वैदेह्या हरणं यथा}
{तथा काले समीचीने संयोगोऽपि भविष्यति}% ॥ ३९ ॥

\twolineshloka
{प्राप्तव्यं सुखदुःखानां भोगान्निर्वर्तनं क्वचित्}
{नान्यथा जानकीजाने तस्माच्छोकं त्यजाधुना}% ॥ ४० ॥

\twolineshloka
{वानराः सन्ति भूयांसो गमिष्यन्ति चतुर्दिशम्}
{शुद्धिं जनकनन्दिन्या आनयिष्यन्ति ते किल}% ॥ ४१ ॥

\twolineshloka
{ज्ञात्वा मार्गस्थितिं तत्र गत्वा कृत्वा पराक्रमम्}
{हत्वा तं पापकर्माणमानयिष्यामि मैथिलीम्}% ॥ ४२ ॥

\twolineshloka
{ससैन्यं भरतं वाऽपि समाहूय सहानुजम्}
{हनिष्यामो वयं शत्रुं किं शोचसि वृथाग्रज}% ॥ ४३ ॥

\twolineshloka
{रघुणैकरथेनैव जिताः सर्वा दिशः पुरा}
{तद्वंशजः कथं शोकं कर्तुमर्हसि राघव}% ॥ ४४ ॥

\twolineshloka
{एकोऽहं सकलाञ्जेतुं समर्थोऽस्मि सुरासुरान्}
{किं पुनः ससहायो वै रावणं कुलपांसनम्}% ॥ ४५ ॥

\twolineshloka
{जनकं वा समानीय साहाय्ये रघुनन्दन}
{हनिष्यामि दुराचारं रावणं सुरकण्टकम्}% ॥ ४६ ॥

\twolineshloka
{सुखस्यानन्तरं दुःखं दुःखस्यानन्तरं सुखम्}
{चक्रनेमिरिवैकं यन्न भवेद्‌रघुनन्दन}% ॥ ४७ ॥

\twolineshloka
{मनोऽतिकातरं यस्य सुखदुःखसमुद्‌भवे}
{स शोकसागरे मग्नो न सुखी स्यात्कदाचन}% ॥ ४८ ॥

\twolineshloka
{इन्द्रेण व्यसनं प्राप्तं पुरा वै रघुनन्दन}
{नहुषः स्थापितो देवैः सर्वैर्मघवतः पदे}% ॥ ४९ ॥

\twolineshloka
{स्थितः पङ्कजमध्ये च बहुवर्षगणानपि}
{अज्ञातवासं मघवा भीतस्त्यक्त्वा निजं पदम्}% ॥ ५० ॥

\twolineshloka
{पुनः प्राप्तं निजस्थानं काले विपरिवर्तिते}
{नहुषः पतितो भूमौ शापादजगराकृतिः}% ॥ ५१ ॥

\twolineshloka
{इन्द्राणीं कामयानस्तु ब्राह्मणानवमन्य च}
{अगस्तिकोपात्सञ्जातः सर्पदेहो महीपतिः}% ॥ ५२ ॥

\twolineshloka
{तस्माच्छोको न कर्तव्यो व्यसने सति राघव}
{उद्यमे चित्तमास्थाय स्थातव्यं वै विपश्चिता}% ॥ ५३ ॥

\twolineshloka
{सर्वज्ञोऽसि महाभाग समर्थोऽसि जगत्पते}
{किं प्राकृत इवात्यर्थं कुरुषे शोकमात्मनि}% ॥ ५४ ॥

\uvacha{व्यास उवाच}


\twolineshloka
{इति लक्ष्मणवाक्येन बोधितो रघुनन्दनः}
{त्यक्त्वा शोकं तथात्यर्थं बभूव विगतज्वरः}% ॥ ५५ ॥


॥इति श्रीदेवीभागवते महापुराणेऽष्टादशसाहस्र्यां संहितायां तृतीयस्कन्धे लक्ष्मणकृतरामशोकसान्त्वनं नामैकोनत्रिंशोऽध्यायः॥
