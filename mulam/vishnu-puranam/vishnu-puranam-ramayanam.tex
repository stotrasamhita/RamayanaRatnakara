\begin{flushleft}
\uvacha{श्रीपराशर उवाच}

काश्यपदुहिता मुमतिर्विदर्भराजतनया केशिनी च द्वे भार्ये सगरस्यास्ताम्॥१॥

ताभ्यां चापत्यार्थमौर्वः परमेण समाधिनाराधितो वरमदात्॥२॥

एका वंशकरमेकं पुत्रमपरा षष्टिं पुत्रसहस्राणां जनयिष्यतीति यस्या यदभिमतं तदिच्छया गृह्यतामित्युक्ते केशिन्येकं वरयामास॥३॥

सुमतिः पुत्रसहस्राणि षष्टिं वव्रे॥४॥

तथेत्युक्ते अल्पैरहोभिः केशिनी पुत्रमेकमसमञ्जसनामानं वंशकरमसूत॥५॥

काश्यपतनयायास्तु सुमत्याः षष्टिं पुत्रसहस्राण्यभवन्॥६॥

तस्मादसमञ्जसादंशुमान्नाम कुमारो जज्ञे॥७॥

स त्वससञ्जसो बालो बाल्यादेवासद्वृत्तोभूत्॥८॥

पिता चास्याचिन्तयदयमतीतबाल्यः सुबुद्धिमान् भविष्यतीति॥९॥

अथ तत्रापि च वयस्यतीते असच्चारीतमेनं पिता तत्याज॥१०॥

तान्यपि षष्टिः पुत्रसहस्राण्यसमञ्जसचारितमेवानुचक्रुः॥११॥

ततश्चाससमजसचरितानुकारिभिः सागरैरपध्वस्तयज्ञैः सन्मार्गे जगति देवाःसकलविद्यामयमसंस्पृष्टमशेषदोषैर्भगवतः पुरुषोत्तमस्यांशभूत कपिलं प्रणम्य तदर्थमूचुः॥१२॥

भगवन्नेभिः सगरतनयैरसमञ्जसचरितमनुगम्यते॥१३॥

कथमेभिर् असद्वृत्तमनुसरद्भिर्जगद्भविष्यतीति॥१४॥

अत्यार्त जगत्परित्राणाय च भगवतोत्र शरीरग्रहणमित्याकर्ण्य भगवानाहाल्पैरेव दिनैर्विनङ्क्ष्यन्तीति॥१५॥

अत्रान्तरे च सगरो हयमेधमारभत॥१६॥

तस्य च पुत्रैरधिष्ठितमस्याश्वं कोप्यपहृत्य भुवो बिलं प्रविवेश॥१७॥

ततस्तत्तनयाश्चाश्वखुरगतिनिर्वन्धेनावनीमेकैको योजनं चख्नुः॥१८॥

पाताले चाश्वं परिभ्रमन्तं तमवनीपतितनयास्ते ददृशुः॥१९॥

नातिदूरेऽवस्थितं च भगवन्तमपघने शरत्कालेर्कमिव तेजोभिरवनतमूर्धमधश्चाशेषदिशश्चोद्भासयमानं हयहर्तारं कपिलर्षिमपश्यन्॥२०॥

ततश्चोद्यतायुधा दुरात्मानोऽयमस्मदपकारी यज्ञविध्नकारी हन्यतां हयहर्ता हन्यतामीत्यवोच्न्नभ्यधावंश्च॥२१॥

ततस्तेनापि भगवता किञ्चिदीषत्परिवर्तितलोचनेनावलोकिताःस्वशरीरसमुत्थेनाग्निनादह्यमाना विनेशुः॥२२॥

सगरोप्यवगम्याश्वानुसारि तत्पुत्रबलमशेषं परमर्षिणा कपिलेन तेजसा दग्धं ततोंऽशुमन्तमसमञ्जसपुत्रमश्वानयनाययुयोज॥२३॥

स तु सगरकतयखातमार्गेण कपिलमुपगम्य भक्तिनम्रस्तदा तृष्टाव॥२४॥

अथैनं भगवानाह॥२५॥

गच्छैनं पितामहायाश्वं प्रापय वरं वृणीष्व च पुत्रक पौत्राश्च चते स्वर्गाद्गङ्गां भुवमानेष्यन्त इति॥२६॥

अथांशुमानपि स्वर्यातानां ब्रह्मदण्डहतानामस्मत्पितॄणामस्वर्गयोग्यानां स्वर्गप्राप्तिकरं वरमस्माकं प्रयच्छेति प्रत्याह॥२७॥

तदाकर्ण्य तं च भगवानाह उक्तमेवैतन्मयाद्य पौत्रस्ते त्रिदिवाद्गङ्गाम्भुवमानयिष्यतीति॥२८॥

तदम्भसा च संस्पृष्टेष्वस्थिभस्मासु एते च स्वर्गमारोक्ष्यन्ति॥२९॥

भगवद्विष्णुपादाङ्गुष्ठनिर्गतस्य हि जलस्यैतन्माहात्म्यम्॥३०॥

यन्न केवलमभिसन्धिपूर्वकं स्नानाद्युपभोगेषूपकारकमनभिसान्धितमप्यस्यां प्रेतप्राणस्यास्थिचर्मस्नायुकेशाद्युपस्पृष्टं शरीरजमपि पतितं सद्यः शरीरिणं स्वर्गं नयतीत्युक्तः प्रणम्य भवगतेऽश्वमादाय पितामहायज्ञमाजगाम्॥३१॥

सगरोप्यश्वमासाद्य तं यज्ञं समापयामास॥३२॥

सागरं चात्मजप्रीत्या पुत्रत्वे कल्पितवान्॥३३॥

तस्यांशुमतो दिलीपः पुत्रोभवत्॥३४॥

दिलीपस्य भगीरथः योऽसौ गङ्गां स्वर्गादिहानीय भगीरथीसंज्ञां चकार॥३५॥

भगीरथात्सुहोत्रःसुहोत्राच्छुतः तस्यापि नाभागः ततोम्बरीषः तत्पुत्रःसिन्धुद्वीपः सिन्धुद्वीपादयुतायुः॥३६॥

तत्पुत्रश्च ऋतुपर्णः योऽसौ नलसहायोक्षहृदयज्ञोभूत्॥३७॥

ऋतुपर्णपुत्रःसर्वकामः॥३८॥

तत्तनयःसुदासः॥३९॥

सुदासात्सौदासो मित्र सहनामा॥४०॥

स चाटव्यां मृगयार्थो पर्यटन् व्याग्रद्वयमपश्यत्॥४१॥

ताभ्यां तद्वनमपमृगं कृतं मत्वैकं तयोर्बाणेन जघान॥४२॥

म्रियमाणश्चासावतिभीषणाकृतिरतिकरालवदनो राक्षसोऽभूत्॥४३॥

द्वितीयोपि प्रतिक्रियां ते करिष्यामीत्युक्त्वान्तर्धानं जगाम॥४४॥

कालेन गच्छता सौदासो यज्ञमयजत्॥४५॥

परिनिष्ठितयज्ञे आचार्ये वसिष्ठे निष्क्रान्ते तद्रक्षो वसिष्ठरूपमास्थाय यज्ञावसाने मम नरमांसभोजनं देयमिति तत्संस्क्रियतां क्षणादागमिष्यामीत्युक्त्वा निष्क्रान्तः॥४६॥

भूयश्च सूदवेषं कृत्वा राजाज्ञया मानुषं मांसं संस्कृत्य राज्ञे न्यवेदयत्॥४७॥

सावपि हिरण्यपात्रे मांसमादाय वसिष्ठागमनप्रतीक्षकोऽभवत्॥४८॥

आगताय वसिष्ठाय निवेदितवान्॥४९॥

स चाप्यचिन्तयदहोस्य राज्ञो दौःशील्यं येनैतन्मांसमस्माकं प्रयच्छति किमेतद्द्रव्यजातमिति ध्यानपरोभवत्॥५०॥

अपश्यच्च तन्मांसम्मानुषम्॥५१॥

अतः क्रोधकलुषीकृतचेता राजनि सापमुत्ससर्ज॥५२॥

यस्मादभोज्यमेतदस्मद्विधानां तपस्विनामवगच्छन्नपि वान्मह्यं ददाति तस्मात्तवैवात्र लोलुपता भविष्यतीति॥५३॥

अनन्तरं च तेनापि भगवतैवाभिहितोस्मीत्युक्ते किं मयाभहितमिति मुनिः पुनरपि समाधौ तस्थौ॥५४॥

समाधिविज्ञानविगतार्थश्चानुग्रहं तस्मै चकार नात्यन्तिकमेतद्द्वादशाब्दं तव भोजनं भविष्यतीति॥५५॥

असावपि प्रतिगृह्योदकाञ्जलिं मुनिशापप्रदानायोद्यतो भगवन्नयमस्मद्गुरुर्नार्हस्येनं कुलदेवताभूतमाचार्यं शप्तुमिति मदयन्त्या स्वपत्न्या प्रसादितःसस्याम्बुदरक्षणार्थं तच्छापांवु नोर्व्यां न चाकासे चिक्षेप किं तु तेनैव स्वपदौ सिषेच॥५६॥

तेन च क्वोधाश्रितेनां चबुना दग्धच्छायौ तत्पादौ कल्माषतामुपगतौ ततःस कल्माषपादसंज्ञामवाप॥५७॥

वसिष्ठशापाच्च षष्ठेषष्ठे काले राक्षसस्वभावमेत्या टव्यां पर्यटन्ननेकशो मानुषानभक्षयत्॥५८॥

एकदा तु कञ्चिन्मुनिमृतुकाले भार्यासङ्गतं ददर्श॥५९॥

तयोश्च तमतिभीषणं राक्षसस्वरूपमवलोक्य त्रासाद्दम्पत्योः प्रधावितयोर्ब्रह्मणं जग्रह॥६०॥

ततःसा ब्रह्मणी बहुशस्तमभियाचितवती॥६१॥

प्रसीदेक्ष्वाकुकुलतिलकभूतस्त्वं महाराजो मियत्रसहो न राक्षसः॥६२॥

नार्हसि स्त्रीधर्मसुखाभिज्ञो मय्यकृतार्थायामस्मद्भर्तारं हन्तुमित्येवं बहुप्रकारं विलपन्त्यां व्याघ्रः पशुमिवारण्येऽभिमतं तं ब्राह्मणमभक्षयत्॥६३॥

ततश्चातिकोपसमन्विता ब्राह्मणी तं राजानं शशाप॥६४॥

यस्मादेवं मय्यतृषप्तायां त्वयायं मत्पतिर्भक्षितः तस्मात्त्वमपि कामोपभोगप्रवृत्तोन्तं प्राप्स्यसीति॥६५॥

शप्त्वा चैव साग्निं प्रविवेश॥६६॥

ततस्तस्य द्वादशाब्दपर्यये विमुक्तशापस्य स्त्रीविषयाभिलाषिणो मदयन्ती तं स्मारयामास॥६७॥

ततः परमसौ स्त्रीभोगं तत्याज॥६८॥

वसिष्ठश्चापुत्रेण राज्ञा पुत्रार्थमभ्यर्थितो मदयन्त्यां गर्भाधानं चकार॥६९॥

यदा च सप्तवर्षाण्यसौ गर्भेण जज्ञे ततस्तं गर्भमश्मना सा देवी जघान॥७०॥

पुत्रश्चाजायत॥७१॥

तस्य चाश्मक इत्येव नामाभवत्॥७२॥

अश्मस्य मूलको नाम पुत्रोऽभवत्॥७३॥

योऽसौ निःक्षत्रे क्ष्मातलेस्मिन् क्रियमाणे स्त्रीभिर्विवस्त्रभिः परिवार्य रक्षितः ततस्तं नारीकवचमुदाहरन्ति॥७४॥

मूलकाद्दशरथस्तस्मादिलिविलस्ततश्च विश्वसहः॥७५॥

तस्माच्च खट्वाङ्गः योसौ देवासुरसङ्ग्रामे देवैरभ्यर्थितोऽसुराञ्जघान॥७६॥

स्वर्गे च कृतप्रियैर्देवैर्वरग्रहणाय चोदितः प्राह॥७७॥

यद्यवश्यं वरो ग्राह्यः तन्ममायुः कथ्यतामिति॥७८॥

अनन्तरं च तैरुक्तं मुहूर्तमेकं प्रमाणं तवायुरित्युक्तोथाःस्वलितगतिना विमानेन लघिमादिगुणो मर्त्यलोकमागम्येदमाह॥७९॥

यथा न ब्राह्मणेभ्यः सकाशादात्मापि मे प्रियतरः न च स्व धर्मोल्लङ्घनं मया कदटचिदप्यनुष्ठितं न च सकलदेवमानुषपशुपक्षिवृक्षादिकेष्वच्युतव्यतिरेकवती दृष्टिर्ममाभूत्तथा तमेवं मुनिजनानुस्मृतं भगवन्तमस्खलितगतिः प्रापयेयमित्यशेषदेवगुरौ भगवत्यनिर्देश्यवपुषि सत्तामात्रात्मन्यात्मानं परमात्मनि वासुदेवाख्ये युयोज तत्रैव च लयमवाप॥८०॥

अत्रापि श्रूयते श्लोको गीतःसप्तर्षिभिः पुरा।
खट्वाङ्गेन समो नान्यः कश्चिदुर्व्यां भविष्यति॥८१॥

येन स्वर्गादिहागम्य मुहूर्तं प्राप्य जीवितम् ।
त्रयोतिसन्धिता लोका बुद्ध्या सत्येन चैव हि॥८२॥

खट्वाङ्गाद्दीर्घबाहुः पुत्रोऽभवत्॥८३॥

ततो रघुरभवत्॥८४॥

तस्मादप्यजः॥८५॥

अजाद्दशरथः॥८६॥

तस्यापि भगवानब्जनाभो जगतः स्थित्यर्थमात्मांशेन रामलक्ष्मणभरतशत्रुघ्नरूपेण चतुर्धा
पुत्रत्वमायासीत्॥८७॥

रामोपि बाल एव विश्वमित्रयागरक्षणाय गच्छंस्ताटकां जघान॥८८॥

यज्ञे च मारीचमिषुवाताहतं समुद्रे चिक्षेप॥८९॥

सुबाहुप्रमुखांश्च क्षयमनयत्॥९०॥

दर्शनमात्रेणाहल्यामपापां चकार॥९१॥

जनकगृहे च माहेश्वरं चापमनायासेन बभञ्ज॥९२॥

सीतामयोनिजां जनकराजतनयां वीर्यशुल्कां लेभे॥९३॥

सकलक्षत्रियक्षयकारिणमशेषहैहयकुलधूमकेतुभूतं च परशुराममपास्तवीर्यबलावलेपं चकार॥९४॥

पितृवचनाच्चागणितराज्याभिलाषो भ्रातृभार्यासमेतो वनं प्रविवेश॥९५॥

विराधखरदूषणादीन् कबन्धवालिनौ च निजघान॥९६॥

बद्धा चाम्भोनिधिमशेषराक्षसकुलक्षयं कृत्वा दशाननापहृतां भार्यं तद्वधादपहृतकलङ्कामप्यनलप्रवेशशुद्धामशेषदेवसङ्घैः स्तूयमानशीलां जनकराजकन्यामयोध्यामानिन्ये॥९७॥

ततश्चाभिषेकमङ्गलं मैत्रेय वर्षशतेनापि वक्तुं न शक्यते सङ्क्षेपेण श्रूयताम्॥९८॥

लक्ष्मणभरतशत्रुघ्नविभीषणसुग्रीवाङ्गदजाम्बवद्धनुमत्प्रभृतिभिः समुत्फुल्लवदनैश्छत्रचामरादियुतैः सेव्यमानो दशरथिर्ब्रह्मेन्द्राग्निजरृतिवरुणवायुकुबेरेशानप्रभृतिभिः सर्वामरैर्वसिष्ठ\-वामदेव\-वाल्मीकि\-मार्कण्डेय\-विश्वामित्र\-भरद्वाजागस्तय\-प्रभृतिभिर्मुनिवरैः ऋग्यजुसामाथर्वैः संस्तूयमानो नृत्यगीतवाद्याद्यखिललोकमङ्गलवाद्यौर्वीणावेशुमृदङ्गभेरी\-पटह\-शङ्ख\-काहल\-गोमुख\-प्रभृतिभिः सुनादैः समस्तभूभटतां मध्ये सकललोकरक्षार्थं यथोचितमभिषिक्तो दाशरथिः कोसलेन्द्रो रघुकुलतिलको जानकीप्रियो भ्रातृत्रयप्रियःसिंहासनगत एकादसाब्दसहस्रं राज्यमकरोत्॥९९॥

भरतोपि गन्धर्वविषयसाधनाय गच्छन् सङ्ग्रामे गन्धर्वकोटीस्तिस्त्रो जघान॥१००॥

शत्रुघ्नेनाप्यमितबलपराक्रमो मधुपुत्रो लवणो नाम राभसो निहातो मथुरा च निवेशिता॥१०१॥

इत्येवमाद्यतिबलपराक्रमविक्रमणेरदितुष्टसंहारिणोशेषस्य जगतो निष्पादितस्थितयो रामलक्षमण भरतशत्रुघ्नाः पुनरपि दिवमारूढाः॥१०२॥

येऽपि तेषु भगवदंशेष्वनुरागिणः कोसलनगरजानपदास्तेपि तन्मनसस्तत्सालोक्यता मवापुः॥१०३॥

अतिदुष्टसंहारिणो रामस्य कुशलवो द्वौ पुत्रौ लक्ष्मणस्याङ्गदचन्द्रकेतू तक्षपुष्कलौ भरतस्य सुबाहुशुरसेनौ शत्रुघ्नस्य॥१०४॥

कुशस्यातिथिरतिथेरपि निषधः पुत्रोऽभूत्॥१०५॥

निषधस्याप्यनलस्तस्मादपि नभः नभसः पुण्डरीकस्तत्तनयः क्षेमधन्वा तस्य च देवानीकस्तस्याप्यहीनकोऽहीनकस्यापि रुरुस्तस्य च पारियात्रकः पारियात्राद्देवलो देवलाद्वच्चलः तस्याप्युत्कः उत्काच्च वज्रनाभस्त स्माच्छङ्खणस्तमाद्युषितास्वस्ततश्च विश्वसहो जज्ञे॥१०६॥

तस्माद्धिरण्यनाभः यो महायोगीस्वराज्जैमिनेः शष्याद्याज्ञवःक्याद्योगमवाय॥१०७॥

हिरण्यनाभस्य पुत्रः पुष्यस्तस्माद्ध्रुवसन्धिस्ततःसुदर्शनस्तस्मादग्निवर्मस्ततः शीघ्रगस्तस्मादपि मरुः पुत्रोऽभवत्॥१०८॥

योसौ योगमास्थायाद्यापि कलापग्राममश्रित्य तिष्ठति॥१०९॥

आगामियुगे सूर्यशक्षत्रव्रत आवर्तयिता भविष्यति॥११०॥

तस्यात्मजः प्रशुश्रुकस्तस्यपि सुसन्धिस्ततश्चाप्यमर्षस्तस्य च सहस्वांस्ततश्च विश्वभवः॥१११॥

तस्य बृहद्बलः योर्जुनतनयेनाभिमन्युनाभारतयुद्धे क्षयमनीयत॥११२॥

एते इक्ष्वाकुभूपालाः प्राधान्येन मयेरिताः।\\
एतेषां चरितं शृण्वन् सर्वपापैः प्रमुच्यते॥११३॥

\end{flushleft}

॥इति श्रीविष्णुमहापुराणे चतुर्थंशे चतुर्थोऽध्यायः॥४॥
