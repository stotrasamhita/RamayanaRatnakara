% !TeX root = ./ramayana-sangraha-mulam
%! Tex program = latexmk -xelatex
\documentclass[twoside,12pt]{book}
\usepackage{emptypage}
\usepackage{etoolbox}
\newbool{kindle}
\newbool{print}
\setbool{kindle}{false}
\setbool{print}{false}
\ifbool{kindle}{\usepackage[paperwidth=126mm,paperheight=168mm,left=5mm,right=5mm,top=15mm,bottom=20mm]{geometry}}{\usepackage[top=1.5cm, bottom=1.8cm, left=1.5cm, right=1.5cm,paperwidth=148mm,paperheight=210mm]{geometry}}

% !TeX program = XeLaTeX
% !TeX root = ../ramayana-ratnakara-mulam.tex
\usepackage{shloka}
\usepackage{wallpaper}
\usepackage{charter,fbb}

\setmainfont[Script=Devanagari]{Sanskrit 2003}
\setromanfont{fbb}
\setsansfont{fbb}

%%% HEADERS and FOOTERS %%%
\usepackage{fancyhdr}
\pagestyle{fancyplain}
\setlength{\headheight}{28pt}
\lhead[\fancyplain{\rightmark}{\pagenumfont\large\thepage}]
   {\fancyplain{\rightmark}{\leftmark}}
\rhead[\fancyplain{\rightmark}{\leftmark}]
   {\fancyplain{\rightmark}{\pagenumfont\large\thepage}}
\cfoot{}

\fancypagestyle{fancyplain}{ %
\fancyhf{} % remove everything
\renewcommand{\headrulewidth}{0pt} % remove lines as well
\renewcommand{\footrulewidth}{0pt}
\cfoot{\pagenumfont\large\thepage}}

%%% SECTIONS and CHAPTERS %%%
\makeatletter
\renewcommand\section{\resetShloka\@startsection {section}{1}{\z@}%
%{2.3ex \@plus.2ex}%
%{-3.5ex \@plus -1ex \@minus -.2ex}%
%{2.3ex \@plus.2ex}%
{10pt}
{2pt}
{\normalfont\LARGE\bfseries}}

\renewcommand\chapter{\resetShloka\@startsection {chapter}{1}{\z@}%
{10pt}
{2pt}
{\normalfont\LARGE\bfseries}}
\makeatother

\setcounter{secnumdepth}{-1}
%for weird reasons this does not bookmark the section start, but the start of text in the section!!!
%\renewcommand\thesection{}
\renewcommand{\sectionmark}[1]{%
\markboth{\large #1}{\rightmark}
}
\renewcommand{\subsectionmark}[1]{%
\markboth{\large #1}{\rightmark}
}
\renewcommand{\chaptermark}[1]{%
\markboth{\large #1}{\rightmark}
}

\addtolength{\parskip}{4pt}
%\addtolength{\headsep}{10pt}
\setlength{\columnseprule}{1pt}
\setlength{\columnsep}{30pt}

%%% HYPERLINKS %%%
\usepackage[bookmarks=true,bookmarksopen=true,xetex,colorlinks=true,
linkcolor=black,					% colour of internal links
citecolor=cyan,					% colour of links to bibliography
filecolor=magenta,			% colour of file links
urlcolor=black					% colour of external links
]{hyperref}

%%% MISCELLANEOUS %%%
\hbadness=10000
\vbadness=10000
\hfuzz=6pt
%\listfiles

%% MACROS
\usepackage{fontawesome5} % Requires xelatex or lualatex
\usepackage{tcolorbox}
\tcbuselibrary{breakable, skins}

\newcommand{\src}[4]{%
  \def\sourceText{#1}%
  \def\sourceKanda{#2}%
  \def\sourceChapter{#3}%
  \def\sourceVerses{#4}%
  \def\fullSource{\sourceText}%
  %
  \if\relax\detokenize{#2}\relax
    % empty Kanda → do nothing
  \else
    \edef\fullSource{\fullSource / \sourceKanda}%
  \fi
  %
  \if\relax\detokenize{#3}\relax
    % empty Chapter → do nothing
  \else
    \edef\fullSource{\fullSource / \sourceChapter}%
  \fi
  %
  \if\relax\detokenize{#4}\relax
    % empty Verses → do nothing
  \else
    \edef\fullSource{\fullSource / \sourceVerses}%
  \fi
}

\renewcommand{\src}[4]{%
  \def\fullSource{}%
  \def\maybeadd##1{%
    \if\relax\detokenize{##1}\relax
      % empty → do nothing
    \else
      \ifx\fullSource\empty
        \edef\fullSource{##1}%
      \else
        \edef\fullSource{\fullSource\space/\space##1}%
      \fi
    \fi
  }%
  \maybeadd{#1}%
  \maybeadd{#2}%
  \maybeadd{#3}%
  \maybeadd{#4}%
}



\newcommand{\vakta}[1]{\def\vaktaName{#1}}
\newcommand{\shrota}[1]{\def\shrotaName{#1}}
\newcommand{\tags}[1]{\def\tagList{#1}}
\newcommand{\notes}[1]{\def\storyNotes{#1}}
\newcommand{\textlink}[1]{\def\textURL{#1}}
\newcommand{\translink}[1]{\def\transURL{#1}}


\newtcolorbox{StoryMetadataBox}{
  enhanced,
  drop small lifted shadow=red,
  breakable,
  colback=gray!5,
  colframe=gray!5,
  fonttitle=\bfseries,
  % title=\faBookOpen\quad Story Metadata,
  % coltitle=black,
  % sharp corners,
  % boxrule=0.5pt,
  left=1em, right=1em, top=0.5em, bottom=0.5em,
}

\def\extractdomain#1://#2/#3\enddomain{#2}

\def\domainfromurl#1{%
  \expandafter\extractdomain#1/\enddomain
}

\newcommand{\storymeta}{
\begin{StoryMetadataBox}
  \small
\begin{description}
  \item[\faBook]\textbf{\fullSource}
  \ifx\vaktaName\empty \else \item[\faUser\ ~वक्ता ---] \vaktaName\fi
\hspace{2em}\ifx\shrotaName\empty \else \faUserFriends\ \textbf{श्रोता} --- \shrotaName\fi
  % \item[\faCommentDots\ \sffamily\small\footnotesize~Notes:] \textsf{\footnotesize\storyNotes}
  \item[\faCommentDots\sffamily\small\footnotesize]\textsf{\footnotesize\storyNotes}
  % \item[\faTags]\hspace{-0.4ex}\tagList
  \ifx\textURL\empty \else\item[\faLink] \href{\textURL}{Source Text: \domainfromurl{\textURL}} \fi
  \ifx\transURL\empty \else\item[\faGlobe] \href{\transURL}{Translation: \domainfromurl{\transURL}} \fi
\end{description}
\end{StoryMetadataBox}
}

\def\sourceText{}
\def\sourceKanda{}
\def\sourceChapter{}
\def\sourceVerses{}
\def\fullSource{}
\def\vaktaName{}
\def\shrotaName{}
\def\tagList{}
\def\storyNotes{}
\def\textURL{}
\def\transURL{}


\begin{document}
% !TeX program = XeLaTeX
% !TeX root = ./ramayana-sangraha-mulam.tex
\pagenumbering{Roman}
\thispagestyle{empty}
%\begin{titlepage}
%\vspace*{12.5cm}\hspace{-2em}\centerline{\font\x="Sanskrit 2003:script=deva,mapping=tex-text" at 52pt \x श्रीमद्भगवद्गीता}
%\end{titlepage}
%\thispagestyle{empty}\clearpage
%Separate page will be made
\begin{titlepage}
\vspace*{6.5cm}\centerline{\font\x="Sanskrit 2003:script=deva,mapping=tex-text" at 48pt \x रामायण-रत्नाकरः}
\end{titlepage}
\clearemptydoublepage
%Uncomment these for producing PDF for electronic circulation
%\begin{center}
%
%\vspace*{8cm}
%
%{\large
%\scshape{For Personal Use Only\\
% Not For Commercial Printing/Distribution}
%}
%\end{center}
%
%
%\clearemptydoublepage
\setmainfont{Sanskrit 2003:script=deva}
\setcounter{page}{0}
\pagenumbering{roman}
\pdfbookmark[2]{Contents}{Contents}
%\thispagestyle{empty}
\renewcommand{\cftchapleader}{\cftdotfill{\cftdotsep}}
\begin{center}
\tableofcontents
\end{center}

\mbox{}
\clearpage
\thispagestyle{empty}
\clearemptydoublepage

% !TeX program = XeLaTeX
% !TeX root = ./ramayana-sangraha-mulam.tex

\begingroup
\fontspec[Script=Devanagari]{Adobe Devanagari}
\fontsize{12pt}{14.4pt}\selectfont
\centerline{\large{ॐ}}
\centerline{॥श्री-गणेशाय नमः॥}
\centerline{॥श्री-गुरुभ्यो नमः॥}
\centerline{॥श्री-सीता-लक्ष्मण-भरत-शत्रुघ्न-हनुमत्-समेत-श्री-रामचन्द्राय नमः॥}

\thispagestyle{empty}

\begin{center}
\chapter*{{प्रस्तावना}}
\end{center}

\twolineshloka*{सदाशिवसमारम्भां शङ्कराचार्यमध्यमाम्}
{अस्मदाचार्यपर्यन्तां वन्दे गुरुपरम्पराम्}

\twolineshloka*
{एष सेतुर्विधरणो लोकासम्भेदहेतवे}
{कोदण्डेन च दण्डेन रामेण गुरुणा कृतः}

रामायण-श्रोतॄणां कदापि तृप्तिर्न जायते! यथा महर्षिः वाल्मीकिः वदति--- ``रामो रामो राम इति प्रजानामभवन् कथाः'', तद्वत् इतिहासपुराणेष्वपि श्रीरामचन्द्रस्य बहवः कथाः लभ्यन्ते। तेषाम् एकत्र प्रस्तुतिं कर्तुम् एषः प्रयासः। सीतादेवी अपि अध्यात्मरामायणे रामस्य अरण्यगमनप्रसङ्गे वदति---

\centerline{``रामायणानि बहुशः श्रुतानि बहुभिर्द्विजैः॥२-४-७७॥''} 

अस्मिन् ग्रन्थे अनेकरामकथाः प्रस्तुताः सन्ति। यद्यपि बहवः कथाः श्रीमाद्वाल्मीकिरामायण\-मनुसृत्य एव वर्तन्ते, काश्चित् कथाः तदतिक्रम्य अपि अन्यकल्पेषु ये केचित्विचित्राः कथाप्रसङ्गाः सन्ति तान् वर्णयन्ति (यथा पद्मपुराणे श्रीरामचन्द्रः स्वयं महादेवं पृच्छति!)। काश्चित् कथाः श्रीमद्वाल्मीकि\-रामायणस्य अन्तर्गत-घट्टानां विस्तृतप्रस्तुतिं कुर्वन्ति। महाभारतेऽपि भीष्मः हनुमान् (स्वानुजं भीमं प्रति) नारदः च विभिन्नेषु प्रसङ्गेषु रामकथां कथयन्ति। विशेषतः वनपर्वणि मारकण्डेयमहर्षिः रामोपाख्यानपर्वणि रामकथां विस्तरेण वर्णयति।

एतासां कथानां वक्तॄन् व्यासं वाल्मीकिं च नमस्कृत्य एतस्य ग्रन्थस्य पारायणम् आरभामहे। रामे अनन्यभक्तिः सदा भवतु नः। 

\twolineshloka*
{नारायणं नमस्कृत्य नरं चैव नरोत्तमम्}
{देवीं सरस्वतीं चैव ततो जयमुदीरयेत्}

\twolineshloka*
{कूजन्तं राम रामेति मधुरं मधुराक्षरम्}
{आरुह्य कविताशाखां वन्दे वाल्मीकिकोकिलम्}


\twolineshloka*
{यत्र यत्र रघुनाथकीर्तनं तत्र तत्र कृतमस्तकाञ्जलिम्}
{बाष्पवारिपरिपूर्णलोचनं मारुतिं नमत राक्षसान्तकम्}

\twolineshloka*
{रामं रामानुजं सीतां भरतं भरतानुजम्}
{सुग्रीवं वायुसूनुं च प्रणमामि पुनः पुनः}

\twolineshloka*
{नमोऽस्तु रामाय सलक्ष्मणाय देव्यै च तस्यै जनकात्मजायै}
{नमोऽस्तु रुद्रेन्द्रयमानिलेभ्यो नमोऽस्तु चन्द्रार्कमरुद्गणेभ्यः}

यथा श्रीमद्भगवद्\-गीतायां भगवान् श्रीकृष्ण आह, ``कथयन्तश्च मां नित्यं तुष्यन्ति च रमन्ति च॥'' तथा वयं सर्वेऽपि रामकथामृतं श्रुत्वा परस्परं च कथयित्वा रामस्य अनन्तकल्याणगुणान् अनुभूय तुष्टिं प्राप्नुयामः! बलं विष्णोः प्रवर्धताम्!\\



\centerline{सर्वम् श्री-सीतारामचन्द्रार्पणमस्तु॥}
\endgroup
\medskip
\noindent{आषाढ-कृष्ण-द्वादशी} \hfill कार्तिकः रामसूनूः\\
विश्वावसु-संवत्सरः ५१२७ कर्कटकः ७ \hfill सर्वज्ञात्म-प्रतिष्ठानम्\\
July 22, 2025


\vfill

\centerline{\large \textbf{Acknowledgments}}

\textsf{\scriptsize Really grateful to all the selfless volunteers who have scanned various old texts, performed OCR, uploaded on wikisource and other repositories. What I present here is a compilation of several such texts. Also very grateful to wisdomlib.org for some fantastic indices (and translations) to the Puranas, which enabled an easier search. %This is an ongoing work, and I will continue to add more texts (incl. Skanda Purana, Brahmavaivarta Purana, etc.) as I find time to typeset them.
 I initially began proofreading the texts as I inserted them, but could not keep pace, especially as I encountered some of the massive texts like Padma Puranam --- in the fullness of time, and with the grace of Bhagavan Rama, hope that this project will see fruition at some point. But it's always important to get the version 0.1 going. Jaya Shri Rama!}

\bigskip

\centerline{\textit{\scriptsize Last updated: \textbf{\today}}}
\cleardoublepage
\thispagestyle{empty}
\ifbool{print}{%Margin changes for print
\addtolength{\evensidemargin}{-0.5cm}
\addtolength{\oddsidemargin}{0.5cm}}{}
\setmainfont[Script=Devanagari]{Adobe Devanagari}

\setcounter{page}{0}
\pagenumbering{arabic}
\sectionmark{\mbox{}}
\clearpage
% \fontsize{14.4pt}{18pt}\selectfont
\fontsize{16pt}{19.2pt}\selectfont
% \Large
\begin{center}
    \part{राम-चरितानि}
    \chapt{वाल्मीकि-रामायणम्}

\sect{सङ्क्षेपरामायणम् --- नारदवाक्यम्}

\src{श्रीमद्-वाल्मीकि-रामायणम्}{बालकाण्डः}{अध्यायः १}{श्लोकाः १---१००}
\vakta{नारदः}
\shrota{वाल्मीकिः}
\tags{concise, complete}
\notes{Origin of the Ramayana from Narada's narration to Valmiki.}
\textlink{}
\translink{}

\storymeta


\twolineshloka
{तपः स्वाध्यायनिरतं तपस्वी वाग्विदां वरम्}
{नारदं परिपप्रच्छ वाल्मीकिर्मुनिपुङ्गवम्}%1

\twolineshloka
{को न्वस्मिन् साम्प्रतं लोके गुणवान् कश्च वीर्यवान्}
{धर्मज्ञश्च कृतज्ञश्च सत्यवाक्यो दृढव्रतः}%2

\twolineshloka
{चारित्रेण च को युक्तः सर्वभूतेषु को हितः}
{विद्वान् कः कः समर्थश्च कश्चैकप्रियदर्शनः}%3

\twolineshloka
{आत्मवान् को जितक्रोधो मतिमान् कोऽनसूयकः}
{कस्य बिभ्यति देवाश्च जातरोषस्य संयुगे}%4

\twolineshloka
{एतदिच्छाम्यहं श्रोतुं परं कौतूहलं हि मे}
{महर्षे त्वं समर्थोऽसि ज्ञातुमेवंविधं नरम्}%5

\twolineshloka
{श्रुत्वा चैतत्त्रिलोकज्ञो वाल्मीकेर्नारदो वचः}
{श्रूयतामिति चाऽऽमन्त्र्य प्रहृष्टो वाक्यमब्रवीत्}%6

\twolineshloka
{बहवो दुर्लभाश्चैव ये त्वया कीर्तिता गुणाः}
{मुने वक्ष्याम्यहं बुद्‌ध्वा तैर्युक्तः श्रूयतां नरः}%7

\twolineshloka
{इक्ष्वाकुवंशप्रभवो रामो नाम जनैः श्रुतः}
{नियतात्मा महावीर्यो द्युतिमान् धृतिमान् वशी}%8

\twolineshloka
{बुद्धिमान् नीतिमान् वाग्मी श्रीमाञ्छत्रुनिबर्हणः}
{विपुलांसो महाबाहुः कम्बुग्रीवो महाहनुः}%9

\twolineshloka
{महोरस्को महेष्वासो गूढजत्रुररिन्दमः}
{आजानुबाहुः सुशिराः सुललाटः सुविक्रमः}%10

\twolineshloka
{समः समविभक्ताङ्गः स्निग्धवर्णः प्रतापवान्}
{पीनवक्षा विशालाक्षो लक्ष्मीवाञ्छुभलक्षणः}%11

\twolineshloka
{धर्मज्ञः सत्यसन्धश्च प्रजानां च हिते रतः}
{यशस्वी ज्ञानसम्पन्नः शुचिर्वश्यः समाधिमान्}%12

\twolineshloka
{प्रजापतिसमः श्रीमान् धाता रिपुनिषूदनः}
{रक्षिता जीवलोकस्य धर्मस्य परिरक्षिता}%13

\twolineshloka
{रक्षिता स्वस्य धर्मस्य स्वजनस्य च रक्षिता}
{वेदवेदाङ्गतत्त्वज्ञो धनुर्वेदे च निष्ठितः}%14

\twolineshloka
{सर्वशास्त्रार्थतत्त्वज्ञो स्मृतिमान् प्रतिभानवान्}
{सर्वलोकप्रियः साधुरदीनात्मा विचक्षणः}%15

\twolineshloka
{सर्वदाऽभिगतः सद्भिः समुद्र इव सिन्धुभिः}
{आर्यः सर्वसमश्चैव सदैव प्रियदर्शनः}%16

\twolineshloka
{स च सर्वगुणोपेतः कौसल्यानन्दवर्धनः}
{समुद्र इव गाम्भीर्ये धैर्येण हिमवानिव}%17

\twolineshloka
{विष्णुना सदृशो वीर्ये सोमवत् प्रियदर्शनः}
{कालाग्निसदृशः क्रोधे क्षमया पृथिवीसमः}%18

\twolineshloka
{धनदेन समस्त्यागे सत्ये धर्म इवापरः}
{तमेवङ्गुणसम्पन्नं रामं सत्यपराक्रमम्}%19

\twolineshloka
{ज्येष्ठं श्रेष्ठगुणैर्युक्तं प्रियं दशरथः सुतम्}
{प्रकृतीनां हितैर्युक्तं प्रकृतिप्रियकाम्यया}%20

\twolineshloka
{यौवराज्येन संयोक्तुम् ऐच्छत् प्रीत्या महीपतिः}
{तस्याभिषेकसम्भारान् दृष्ट्वा भार्याऽथ कैकयी}%21

\twolineshloka
{पूर्वं दत्तवरा देवी वरमेनमयाचत}
{विवासनं च रामस्य भरतस्याभिषेचनम्}%22

\twolineshloka
{स सत्यवचनाद्राजा धर्मपाशेन संयतः}
{विवासयामास सुतं रामं दशरथः प्रियम्}%23

\twolineshloka
{स जगाम वनं वीरः प्रतिज्ञामनुपालयन्}
{पितुर्वचननिर्देशात् कैकेय्याः प्रियकारणात्}%24

\twolineshloka
{तं व्रजन्तं प्रियो भ्राता लक्ष्मणोऽनुजगाम ह}
{स्नेहाद्विनयसम्पन्नः सुमित्रानन्दवर्धनः}%25

\twolineshloka
{भ्रातरं दयितो भ्रातुः सौभ्रात्रमनुदर्शयन्}
{रामस्य दयिता भार्या नित्यं प्राणसमा हिता}%26

\twolineshloka
{जनकस्य कुले जाता देवमायेव निर्मिता}
{सर्वलक्षणसम्पन्ना नारीणामुत्तमा वधूः}%27

\twolineshloka
{सीताऽप्यनुगता रामं शशिनं रोहिणी यथा}
{पौरैरनुगतो दूरं पित्रा दशरथेन च}%28

\twolineshloka
{शृङ्गवेरपुरे सूतं गङ्गाकूले व्यसर्जयत्}
{गुहमासाद्य धर्मात्मा निषादाधिपतिं प्रियम्}%29

\twolineshloka
{गुहेन सहितो रामो लक्ष्मणेन च सीतया}
{ते वनेन वनं गत्वा नदीस्तीर्त्वा बहूदकाः}%30

\twolineshloka
{चित्रकूटमनुप्राप्य भरद्वाजस्य शासनात्}
{रम्यमावसथं कृत्वा रममाणा वने त्रयः}%31

\twolineshloka
{देवगन्धर्वसङ्काशास्तत्र ते न्यवसन् सुखम्}
{चित्रकूटं गते रामे पुत्रशोकातुरस्तथा}%32

\twolineshloka
{राजा दशरथः स्वर्गं जगाम विलपन् सुतम्}
{मृते तु तस्मिन् भरतो वसिष्ठप्रमुखैर्द्विजैः}%33

\twolineshloka
{नियुज्यमानो राज्याय नैच्छद्राज्यं महाबलः}
{स जगाम वनं वीरो रामपादप्रसादकः}%34

\twolineshloka
{गत्वा तु स महात्मानं रामं सत्यपराक्रमम्}
{अयाचद्भ्रातरं रामम् आर्यभावपुरस्कृतः}%35

\twolineshloka
{त्वमेव राजा धर्मज्ञ इति रामं वचोऽब्रवीत्}
{रामोऽपि परमोदारः सुमुखः सुमहायशाः}%36

\twolineshloka
{न चेच्छत् पितुरादेशाद्राज्यं रामो महाबलः}
{पादुके चास्य राज्याय न्यासं दत्त्वा पुनः पुनः}%37

\twolineshloka
{निवर्तयामास ततो भरतं भरताग्रजः}
{स काममनवाप्यैव रामपादावुपस्पृशन्}%38

\twolineshloka
{नन्दिग्रामेऽकरोद्राज्यं रामागमनकाङ्क्षया}
{गते तु भरते श्रीमान् सत्यसन्धो जितेन्द्रियः}%39

\twolineshloka
{रामस्तु पुनरालक्ष्य नागरस्य जनस्य च}
{तत्राऽऽगमनमेकाग्रो दण्डकान् प्रविवेश ह}%40

\twolineshloka
{प्रविश्य तु महारण्यं रामो राजीवलोचनः}
{विराधं राक्षसं हत्वा शरभङ्गं ददर्श ह}%41

\twolineshloka
{सुतीक्ष्णं चाप्यगस्त्यं च अगस्त्यभ्रातरं तथा}
{अगस्त्यवचनाच्चैव जग्राहैन्द्रं शरासनम्}%42

\twolineshloka
{खड्गं च परमप्रीतस्तूणी चाक्षयसायकौ}
{वसतस्तस्य रामस्य वने वनचरैः सह}%43

\twolineshloka
{ऋषयोऽभ्यागमन् सर्वे वधायासुररक्षसाम्}
{स तेषां प्रति शुश्राव राक्षसानां तदा वने}%44

\twolineshloka
{प्रतिज्ञातश्च रामेण वधः संयति रक्षसाम्}
{ऋषीणामग्निकल्पानां दण्डकारण्यवासिनाम्}%45

\twolineshloka
{तेन तत्रैव वसता जनस्थाननिवासिनी}
{विरूपिता शूर्पणखा राक्षसी कामरूपिणी}%46

\twolineshloka
{ततः शूर्पणखावाक्यादुद्युक्तान् सर्वराक्षसान्}
{खरं त्रिशिरसं चैव दूषणं चैव राक्षसम्}%47

\twolineshloka
{निजघान रणे रामस्तेषां चैव पदानुगान्}
{वने तस्मिन् निवसता जनस्थाननिवासिनाम्}%48

\twolineshloka
{रक्षसां निहतान्यासन् सहस्राणि चतुर्दश}
{ततो ज्ञातिवधं श्रुत्वा रावणः क्रोधमूर्च्छितः}%49

\twolineshloka
{सहायं वरयामास मारीचं नाम राक्षसम्}
{वार्यमाणः सुबहुशो मारीचेन स रावणः}%50

\twolineshloka
{न विरोधो बलवता क्षमो रावण तेन ते}
{अनादृत्य तु तद्वाक्यं रावणः कालचोदितः}%51

\twolineshloka
{जगाम सहमारीचस्तस्याऽऽश्रमपदं तदा}
{तेन मायाविना दूरमपवाह्य नृपात्मजौ}%52

\twolineshloka
{जहार भार्यां रामस्य गृध्रं हत्वा जटायुषम्}
{गृध्रं च निहतं दृष्ट्वा हृतां श्रुत्वा च मैथिलीम्}%53

\twolineshloka
{राघवः शोकसन्तप्तो विललापाऽऽकुलेन्द्रियः}
{ततस्तेनैव शोकेन गृध्रं दग्ध्वा जटायुषम्}%54

\twolineshloka
{मार्गमाणो वने सीतां राक्षसं सन्ददर्श ह}
{कबन्धं नाम रूपेण विकृतं घोरदर्शनम्}%55

\twolineshloka
{तं निहत्य महाबाहुर्ददाह स्वर्गतश्च सः}
{स चास्य कथयामास शबरीं धर्मचारिणीम्}%56

\twolineshloka
{श्रमणीं धर्मनिपुणाम् अभिगच्छेति राघव}
{सोऽभ्यगच्छन् महातेजाः शबरीं शत्रुसूदनः}%57

\twolineshloka
{शबर्या पूजितः सम्यग्रामो दशरथात्मजः}
{पम्पातीरे हनुमता सङ्गतो वानरेण ह}%58

\twolineshloka
{हनुमद्वचनाच्चैव सुग्रीवेण समागतः}
{सुग्रीवाय च तत्सर्वं शंसद्रामो महाबलः}%59

\twolineshloka
{आदितस्तद्यथा वृत्तं सीतायाश्च विशेषतः}
{सुग्रीवश्चापि तत्सर्वं श्रुत्वा रामस्य वानरः}%60

\twolineshloka
{चकार सख्यं रामेण प्रीतश्चैवाग्निसाक्षिकम्}
{ततो वानरराजेन वैरानुकथनं प्रति}%61

\twolineshloka
{रामायाऽऽवेदितं सर्वं प्रणयाद्दुःखितेन च}
{प्रतिज्ञातं च रामेण तदा वालिवधं प्रति}%62

\twolineshloka
{वालिनश्च बलं तत्र कथयामास वानरः}
{सुग्रीवः शङ्कितश्चासीन्नित्यं वीर्येण राघवे}%63

\twolineshloka
{राघवः प्रत्ययार्थं तु दुन्दुभेः कायमुत्तमम्}
{दर्शयामास सुग्रीवो महापर्वतसन्निभम्}%64

\twolineshloka
{उत्स्मयित्वा महाबाहुः प्रेक्ष्य चास्थि महाबलः}
{पादाङ्गुष्ठेन चिक्षेप सम्पूर्णं दशयोजनम्}%65

\twolineshloka
{बिभेद च पुनः सालान् सप्तैकेन महेषुणा}
{गिरिं रसातलं चैव जनयन् प्रत्ययं तदा}%66

\twolineshloka
{ततः प्रीतमनास्तेन विश्वस्तः स महाकपिः}
{किष्किन्धां रामसहितो जगाम च गुहां तदा}%67

\twolineshloka
{ततोऽगर्जद्धरिवरः सुग्रीवो हेमपिङ्गलः}
{तेन नादेन महता निर्जगाम हरीश्वरः}%68

\twolineshloka
{अनुमान्य तदा तारां सुग्रीवेण समागतः}
{निजघान च तत्रैनं शरेणैकेन राघवः}%69

\twolineshloka
{ततः सुग्रीववचनाद्धत्वा वालिनमाहवे}
{सुग्रीवमेव तद्राज्ये राघवः प्रत्यपादयत्}%70

\twolineshloka
{स च सर्वान् समानीय वानरान् वानरर्षभः}
{दिशः प्रस्थापयामास दिदृक्षुर्जनकात्मजाम्}%71

\twolineshloka
{ततो गृध्रस्य वचनात्सम्पातेर्हनुमान् बली}
{शतयोजनविस्तीर्णं पुप्लुवे लवणार्णवम्}%72

\twolineshloka
{तत्र लङ्कां समासाद्य पुरीं रावणपालिताम्}
{ददर्श सीतां ध्यायन्तीमशोकवनिकां गताम्}%73

\twolineshloka
{निवेदयित्वाऽभिज्ञानं प्रवृत्तिं विनिवेद्य च}
{समाश्वास्य च वैदेहीं मर्दयामास तोरणम्}%74

\twolineshloka
{पञ्च सेनाग्रगान् हत्वा सप्त मन्त्रिसुतानपि}
{शूरमक्षं च निष्पिष्य ग्रहणं समुपागमत्}%75

\twolineshloka
{अस्त्रेणोन्मुक्तमात्मानं ज्ञात्वा पैतामहाद्वरात्}
{मर्षयन् राक्षसान् वीरो यन्त्रिणस्तान् यदृच्छया}%76

\twolineshloka
{ततो दग्ध्वा पुरीं लङ्काम् ऋते सीतां च मैथिलीम्}
{रामाय प्रियमाख्यातुं पुनरायान् महाकपिः}%77

\twolineshloka
{सोऽभिगम्य महात्मानं कृत्वा रामं प्रदक्षिणम्}
{न्यवेदयदमेयात्मा दृष्टा सीतेति तत्त्वतः}%78

\twolineshloka
{ततः सुग्रीवसहितो गत्वा तीरं महोदधेः}
{समुद्रं क्षोभयामास शरैरादित्यसन्निभैः}%79

\twolineshloka
{दर्शयामास चाऽऽत्मानं समुद्रः सरितां पतिः}
{समुद्रवचनाच्चैव नलं सेतुमकारयत्}%80

\twolineshloka
{तेन गत्वा पुरीं लङ्कां हत्वा रावणमाहवे}
{रामः सीतामनुप्राप्य परां व्रीडामुपागमत्}%81

\twolineshloka
{तामुवाच ततो रामः परुषं जनसंसदि}
{अमृष्यमाणा सा सीता विवेश ज्वलनं सती}%82

\twolineshloka
{ततोऽग्निवचनात् सीतां ज्ञात्वा विगतकल्मषाम्}
{कर्मणा तेन महता त्रैलोक्यं सचराचरम्}%83

\twolineshloka
{सदेवर्षिगणं तुष्टं राघवस्य महात्मनः}
{बभौ रामः सम्प्रहृष्टः पूजितः सर्वदैवतैः}%84

\twolineshloka
{अभिषिच्य च लङ्कायां राक्षसेन्द्रं विभीषणम्}
{कृतकृत्यस्तदा रामो विज्वरः प्रमुमोद ह}%85

\twolineshloka
{देवताभ्यो वरान् प्राप्य समुत्थाप्य च वानरान्}
{अयोध्यां प्रस्थितो रामः पुष्पकेण सुहृद्-वृतः}%86

\twolineshloka
{भरद्वाजाश्रमं गत्वा रामः सत्यपराक्रमः}
{भरतस्यान्तिके रामो हनूमन्तं व्यसर्जयत्}%87

\twolineshloka
{पुनराख्यायिकां जल्पन् सुग्रीवसहितस्तदा}
{पुष्पकं तत् समारुह्य नन्दिग्रामं ययौ तदा}%88

\twolineshloka
{नन्दिग्रामे जटां हित्वा भ्रातृभिः सहितोऽनघः}
{रामः सीतामनुप्राप्य राज्यं पुनरवाप्तवान्}%89

\twolineshloka
{प्रहृष्टमुदितो लोकस्तुष्टः पुष्टः सुधार्मिकः}
{निरामयो ह्यरोगश्च दुर्भिक्षभयवर्जितः}%90

\twolineshloka
{न पुत्रमरणं केचिद्-द्रक्ष्यन्ति पुरुषाः क्वचित्}
{नार्यश्चाविधवा नित्यं भविष्यन्ति पतिव्रताः}%91

\twolineshloka
{न चाग्निजं भयं किञ्चिन्नाप्सु मज्जन्ति जन्तवः}
{न वातजं भयं किञ्चिन्नापि ज्वरकृतं तथा}%92

\twolineshloka
{न चापि क्षुद्भयं तत्र न तस्करभयं तथा}
{नगराणि च राष्ट्राणि धनधान्ययुतानि च}%93

\twolineshloka
{नित्यं प्रमुदिताः सर्वे यथा कृतयुगे तथा}
{अश्वमेधशतैरिष्ट्वा तथा बहुसुवर्णकैः}%94

\twolineshloka
{गवां कोट्ययुतं दत्त्वा विद्वद्‌भ्यो विधिपूर्वकम्}
{असङ्ख्येयं धनं दत्त्वा ब्राह्मणेभ्यो महायशाः}%95

\twolineshloka
{राजवंशाञ्छतगुणान् स्थापयिष्यति राघवः}
{चातुर्वर्ण्यं च लोकेऽस्मिन् स्वे स्वे धर्मे नियोक्ष्यति}%96

\twolineshloka
{दशवर्षसहस्राणि दशवर्षशतानि च}
{रामो राज्यमुपासित्वा ब्रह्मलोकं गमिष्यति}%97

\twolineshloka
{इदं पवित्रं पापघ्नं पुण्यं वेदैश्च सम्मितम्}
{यः पठेद्रामचरितं सर्वपापैः प्रमुच्यते}%98

\twolineshloka
{एतदाख्यानमायुष्यं पठन् रामायणं नरः}
{सपुत्रपौत्रः सगणः प्रेत्य स्वर्गे महीयते}%99

\fourlineindentedshloka
{पठन् द्विजो वागृषभत्वमीयात्}
{स्यात् क्षत्रियो भूमिपतित्वमीयात्}
{वणिग्जनः पण्यफलत्वमीयात्}
{जनश्च शूद्रोऽपि महत्त्वमीयात्}%100
{॥इत्यार्षे श्रीमद्रामायणे वाल्मीकीये आदिकाव्ये बालकाण्डे प्रथमः सर्गः॥}

\closesection
    \sect{काव्य-सङ्क्षेपः}

\src{श्रीमद्-वाल्मीकि-रामायणम्}{बालकाण्डः}{अध्यायः ३}{श्लोकाः १---२९}
% \vakta{वाल्मीकिः}
\tags{concise, complete}
\notes{Summary of the Kavya by Valmiki.}
\textlink{}
\translink{}

\storymeta

\twolineshloka
{श्रुत्वा वस्तु समग्रं तद्धर्मात्मा धर्मसंहितम्}
{व्यक्तमन्वेषते भूयो यद्वृत्तं तस्य धीमतः} %||1-3-1||

\twolineshloka
{उपस्पृश्योदकं संयन्मुनिः स्थित्वा कृताञ्जलिः}
{प्राचीनाग्रेषु दर्भेषु धर्मेणान्वेषते गतिम्} %||1-3-2||

\twolineshloka
{जन्म रामस्य सुमहद्वीर्यं सर्वानुकूलताम्}
{लोकस्य प्रियतां क्षान्तिं सौम्यतां सत्यशीलताम्} %||1-3-3||

\twolineshloka
{नानाचित्राः कथाश्चान्या विश्वामित्रसहायने}
{जानक्याश्च विवाहं च धनुषश्च विभेदनम्} %||1-3-4||

\twolineshloka
{रामरामविवादं च गुणान्दाशरथेस्तथा}
{तथाभिषेकं रामस्य कैकेय्या दुष्टभावताम्} %||1-3-5||

\twolineshloka
{व्याघातं चाभिषेकस्य रामस्य च विवासनम्}
{राज्ञः शोकं विलापं च परलोकस्य चाश्रयम्} %||1-3-6||

\twolineshloka
{प्रकृतीनां विषादं च प्रकृतीनां विसर्जनम्}
{निषादाधिपसंवादं सूतोपावर्तनं तथा} %||1-3-7||

\twolineshloka
{गङ्गायाश्चाभिसन्तारं भरद्वाजस्य दर्शनम्}
{भरद्वाजाभ्यनुज्ञानाच्चित्रकूटस्य दर्शनम्} %||1-3-8||

\twolineshloka
{वास्तुकर्मनिवेशं च भरतागमनं तथा}
{प्रसादनं च रामस्य पितुश्च सलिलक्रियाम्} %||1-3-9||

\twolineshloka
{पादुकाग्र्याभिषेकं च नन्दिग्राम निवासनम्}
{दण्डकारण्यगमनं सुतीक्ष्णेन समागमम्} %||1-3-10||

\twolineshloka
{अनसूयासमस्यां च अङ्गरागस्य चार्पणम्}
{शूर्पणख्याश्च संवादं विरूपकरणं तथा} %||1-3-11||

\twolineshloka
{वधं खरत्रिशिरसोरुत्थानं रावणस्य च}
{मारीचस्य वधं चैव वैदेह्या हरणं तथा} %||1-3-12||

\twolineshloka
{राघवस्य विलापं च गृध्रराजनिबर्हणम्}
{कबन्धदर्शनं चैव पम्पायाश्चापि दर्शनम्} %||1-3-13||

\twolineshloka
{शबर्या दर्शनं चैव हनूमद्दर्शनं तथा}
{विलापं चैव पम्पायां राघवस्य महात्मनः} %||1-3-14||

\twolineshloka
{ऋष्यमूकस्य गमनं सुग्रीवेण समागमम्}
{प्रत्ययोत्पादनं सख्यं वालिसुग्रीवविग्रहम्} %||1-3-15||

\twolineshloka
{वालिप्रमथनं चैव सुग्रीवप्रतिपादनम्}
{ताराविलापसमयं वर्षरात्रिनिवासनम्} %||1-3-16||

\twolineshloka
{कोपं राघवसिंहस्य बलानामुपसङ्ग्रहम्}
{दिशः प्रस्थापनं चैव पृथिव्याश्च निवेदनम्} %||1-3-17||

\twolineshloka
{अङ्गुलीयकदानं च ऋक्षस्य बिलदर्शनम्}
{प्रायोपवेशनं चैव सम्पातेश्चापि दर्शनम्} %||1-3-18||

\twolineshloka
{पर्वतारोहणं चैव सागरस्य च लङ्घनम्}
{रात्रौ लङ्काप्रवेशं च एकस्यापि विचिन्तनम्} %||1-3-19||

\twolineshloka
{आपानभूमिगमनमवरोधस्य दर्शनम्}
{अशोकवनिकायानं सीतायाश्चापि दर्शनम्} %||1-3-20||

\twolineshloka
{अभिज्ञानप्रदानं च सीतायाश्चापि भाषणम्}
{राक्षसीतर्जनं चैव त्रिजटास्वप्नदर्शनम्} %||1-3-21||

\twolineshloka
{मणिप्रदानं सीताया वृक्षभङ्गं तथैव च}
{राक्षसीविद्रवं चैव किङ्कराणां निबर्हणम्} %||1-3-22||

\twolineshloka
{ग्रहणं वायुसूनोश्च लङ्कादाहाभिगर्जनम्}
{प्रतिप्लवनमेवाथ मधूनां हरणं तथा} %||1-3-23||

\twolineshloka
{राघवाश्वासनं चैव मणिनिर्यातनं तथा}
{सङ्गमं च समुद्रस्य नलसेतोश्च बन्धनम्} %||1-3-24||

\twolineshloka
{प्रतारं च समुद्रस्य रात्रौ लङ्कावरोधनम्}
{विभीषणेन संसर्गं वधोपायनिवेदनम्} %||1-3-25||

\twolineshloka
{कुम्भकर्णस्य निधनं मेघनादनिबर्हणम्}
{रावणस्य विनाशं च सीतावाप्तिमरेः पुरे} %||1-3-26||

\twolineshloka
{विभीषणाभिषेकं च पुष्पकस्य च दर्शनम्}
{अयोध्यायाश्च गमनं भरतेन समागमम्} %||1-3-27||

\twolineshloka
{रामाभिषेकाभ्युदयं सर्वसैन्यविसर्जनम्}
{स्वराष्ट्ररञ्जनं चैव वैदेह्याश्च विसर्जनम्} %||1-3-28||

\twolineshloka
{अनागतं च यत्किञ्चिद्रामस्य वसुधातले}
{तच्चकारोत्तरे काव्ये वाल्मीकिर्भगवानृषिः} %||1-4-29||


{॥इत्यार्षे श्रीमद्रामायणे वाल्मीकीये आदिकाव्ये बालकाण्डे तृतीयः सर्गः॥}

\closesection
    \sect{रामवृत्तसंश्रवः}

\src{श्रीमद्-वाल्मीकि-रामायणम्}{सुन्दरकाण्डः}{अध्यायः ३१}{श्लोकाः १---१९}
\vakta{हनुमान्}
\shrota{सीता}
\tags{concise, complete}
\notes{Narration of Rama's story to Sita by Hanuman.}
\textlink{}
\translink{}

\storymeta


\twolineshloka
{एवं बहुविधां चिन्तां चिन्तयित्वा महाकपिः}
{संश्रवे मधुरं वाक्यं वैदेह्या व्याजहार ह}

\twolineshloka
{राजा दशरथो नाम स्थकुञ्जरवाजिमान्}
{पुण्यशीलो महाकीर्तिर्ऋजुरासीन्महायशाः}

\twolineshloka
{राजर्षीणां गुणश्रेष्ठस्तपसा चर्षिभिः समः}
{चक्रवर्तिकुले जातः पुरन्दरसमो बले}

\twolineshloka
{अहिंसारतिरक्षुद्रो घृणी सत्यपराक्रमः}
{मुख्यश्चेक्ष्वाकुवंशस्य लक्ष्मीवाँल्लक्ष्मिवर्धनः}

\twolineshloka
{पार्थिवव्यञ्जनैर्युक्तः पृथुश्रीः पार्थिवर्षभः}
{पृथिव्यां चतुरान्तायां विश्रुतः सुखदः सुखी}

\twolineshloka
{तस्य पुत्रः प्रियो ज्येष्ठस्ताराधिपनिभाननः}
{रामो नाम विशेषज्ञः श्रेष्ठः सर्वधनुष्मताम्}

\twolineshloka
{रक्षिता स्वस्य वृत्तस्य स्वजनस्यापि रक्षिता}
{रक्षिता जीवलोकस्य धर्मस्य च परन्तपः}

\twolineshloka
{तस्य सत्याभिसन्धस्य वृद्धस्य वचनात् पितुः}
{सभार्यः सह च भ्रात्रा वीरः प्रव्राजितो वनम्}

\twolineshloka
{तेन तत्र महारण्ये मृगयां परिधावता}
{राक्षसा निहताः शूरा बहवः कामरूपिणः}

\twolineshloka
{जनस्थानवधं श्रुत्वा हतौ च खरदूषणौ}
{ततस्त्वमर्षापहृता जानकी रावणेन तु}

\twolineshloka
{वञ्चयित्वा वने रामं मृगरूपेण मायया}
{स मार्गमाणस्तां देवीं रामः सीतामनिन्दिताम्}

\twolineshloka
{आससाद वने मित्रं सुग्रीवं नाम वानरम्}
{ततः स वालिनं हत्वा रामः परपुरञ्जयः}

\twolineshloka
{प्रायच्छत् कपिराज्यं तत् सुग्रीवाय महाबलः}
{सुग्रीवेणापि सन्दिष्टा हरयः कामरूपिणः}

\twolineshloka
{दिक्षु सर्वासु तां देवीं विचिन्वन्ति सहस्रशः}
{अहं सम्पातिवचनाच्छतयोजनमायतम्}

\twolineshloka
{अस्या हेतोर्विशालाक्ष्याः सागरं वेगवान् प्लुतः}
{यथारूपां यथावर्णां यथालक्ष्मीं च निश्चिताम्}

\twolineshloka
{अश्रौषं राघवस्याहं सेयमासादिता मया}
{विररामैवमुक्त्वासौ वाचं वानरपुङ्गवः}

\threelineshloka
{जानकी चापि तच्छ्रुत्वा विस्मयं परमं गता}
{ततः सा वक्रकेशान्ता सुकेशी केशसंवृतम्}
{उन्नम्य वदनं भीरुः शिंशपामन्ववैक्षत}

\fourlineindentedshloka
{निशम्य सीता वचनं कपेश्च}
{दिशश्च सर्वाः प्रदिशश्च वीक्ष्य}
{स्वयं प्रहर्षं परमं जगाम}
{सर्वात्मना राममनुस्मरन्ती}

\fourlineindentedshloka
{सा तिर्यगूर्ध्वं च तथाप्यधस्तान्-}
{निरीक्षमाणा तमचिन्त्यबुद्धिम्}
{ददर्श पिङ्गाधिपतेरमात्यं}
{वातात्मजं सूर्यमिवोदयस्थम्}

इत्यार्षे श्रीमद्रामायणे वाल्मीकीये आदिकाव्ये चतुर्विंशतिसहस्रिकायां संहितायां सुन्दरकाण्डे रामवृत्तसंश्रवो नाम एकत्रिंशः सर्गः॥


\closesection
    \chapt{हनूमज्जानकीसंवादोपक्रमः}

\src{श्रीमद्-वाल्मीकि-रामायणम्}{सुन्दरकाण्डः}{अध्यायः १२९}{श्लोकाः १६---३२}
\vakta{सीता}
\shrota{हनुमान्}
\tags{concise, part}
\notes{Narration of Rama's story until Sita Haranam, by Sita Herself to Hanuman.}
\textlink{}
\translink{}

\storymeta


\twolineshloka
{सोऽवतीर्य द्रुमात्तस्माद्विद्रुमप्रतिमाननः}
{विनीतवेषः कृपणः प्रणिपत्योपसृत्य च}

\twolineshloka
{तामब्रवीन्महातेजा हनूमान् मारुतात्मजः}
{शिरस्यञ्जलिमाधाय सीतां मधुरया गिरा}

\twolineshloka
{का नु पद्मपलाशाक्षि क्लिष्टकौशेयवासिनि}
{द्रुमस्य शाखामालम्ब्य तिष्ठसि त्वमनिन्दिते}

\twolineshloka
{किमर्थं तव नेत्राभ्यां वारि स्रवति शोकजम्}
{पुण्डरीकपलाशाभ्यां विप्रकीर्णमिवोदकम्}

\twolineshloka
{सुराणामसुराणां वा नागगन्धर्वरक्षसाम्}
{यक्षाणां किन्नराणां वा का त्वं भवसि शोभने}

\twolineshloka
{का त्वं भवसि रुद्राणां मरुतां वा वरानने}
{वसूनां वा वरारोहे देवता प्रतिभासि मे}

\twolineshloka
{किं नु चन्द्रमसा हीना पतिता विबुधालयात्}
{रोहिणी ज्योतिषां श्रेष्ठा श्रेष्ठा सर्वगुणान्विता}

\twolineshloka
{का त्वं भवसि कल्याणि त्वमनिन्दितलोचने}
{कोपाद्वा यदि वा मोहाद्भर्तारमसितेक्षणे}

\twolineshloka
{वसिष्ठं कोपयित्वा त्वं वासि कल्याण्यरुन्धती}
{को नु पुत्रः पिता भ्राता भर्ता वा ते सुमध्यमे}

\twolineshloka
{अस्माल्लोकादमुं लोकं गतं त्वमनुशोचसि}
{रोदनादतिनिःश्वासाद्भूमिसंस्पर्शनादपि}

\twolineshloka
{न त्वां देवीमहं मन्ये राज्ञः संज्ञावधारणात्}
{व्यञ्जनानि च ते यानि लक्षणानि च लक्षये}

\twolineshloka
{महिषी भूमिपालस्य राजकन्यासि मे मता}
{रावणेन जनस्थानाद्बलादपहृता यदि}

\twolineshloka
{सीता त्वमसि भद्रं ते तन्ममाचक्ष्व पृच्छतः}
{यथा हि तव वै दैन्यं रूपं चाप्यतिमानुषम्}

\twolineshloka
{तपसा चान्वितो वेषस्त्वं राममहिषी धुवम्}
{सा तस्य वचनं श्रुत्वा रामकीर्तनहर्षिता}

\twolineshloka
{उवाच वाक्यं वैदेही हनुमन्तं द्रुमाश्रितम्}
{पृथिव्यां राजसिंहानां मुख्यस्य विदितात्मनः}

\twolineshloka
{स्नुषा दशस्थस्याहं शत्रुसैन्यप्रमाथिनः}
{दुहिता जनकस्याहं वैदेहस्य महात्मनः}

\twolineshloka
{सीता च नाम नाम्नाहं भार्या रामस्य धीमतः}
{समा द्वादश तत्राहं राघवस्य निवेशने}

\twolineshloka
{भुआना मानुषान् भोगान् सर्वकामसमृद्धिनी}
{ततस्त्रयोदशे वर्षे राज्ये चेक्ष्वाकुनन्दनम्}

\twolineshloka
{अभिषेचयितुं राजा सोपाध्यायः प्रचक्रमे}
{तस्मिन् सम्भ्रियमाणे तु राघवस्याभिषेचने}

\twolineshloka
{कैकेयी नाम भर्तारमिदं वचनमब्रवीत्}
{न पिबेयं न खादेयं प्रत्यहं मम भोजनम्}

\twolineshloka
{एष मे जीवितस्यान्तो रामो यद्यभिषिच्यते}
{यत्तदुक्तं त्वया वाक्यं प्रीत्या नृपतिसत्तम}

\twolineshloka
{तच्चेन्न वितथं कार्यं वनं गच्छतु राघवः}
{स राजा सत्यवाग्देव्या वरदानमनुस्मरन्}

\twolineshloka
{मुमोह वचनं श्रुत्वा कैकेय्याः क्रूरमप्रियम्}
{ततस्तु स्थविरो राजा सत्ये धर्मे व्यवस्थितः}

\twolineshloka
{ज्येष्ठं यशस्विनं पुत्रं रुदन् राज्यमयाचत}
{स पितुर्वचनं श्रीमानभिषेकात् परं प्रियम्}

\twolineshloka
{मनसा पूर्वमासाद्य वाचा प्रतिगृहीतवान्}
{दद्यान्न प्रतिगृह्णीयान्न ब्रूयात् किञ्चिदप्रियम्}

\twolineshloka
{अपि जीवितहेतोर्वा रामः सत्यपराक्रमः}
{स विहायोत्तरीयाणि महार्हाणि महायशाः}

\twolineshloka
{विसृज्य मनसा राज्यं जनन्यै मां समादिशत्}
{साहं तस्याग्रतस्तूर्णं प्रस्थिता वनचारिणी}

\twolineshloka
{न हि मे तेन हीनाया वासः स्वर्गेऽपि रोचते}
{प्रागेव तु महाभागः सौमित्रिर्मित्रनन्दनः}

\twolineshloka
{पूर्वजस्यानुयात्रार्थे द्रुमचीरैरलङ्कृतः}
{ते वयं भर्तुरादेशं बहुमान्य दृढव्रताः}

\twolineshloka
{प्रविष्टाः स्म पुरादृष्टं वनं गम्भीरदर्शनम्}
{वसतो दण्डकारण्ये तस्याहममितौजसः}

\threelineshloka
{रक्षसापहृता भार्या रावणेन दुरात्मना}
{द्वौ मासौ तेन मे कालो जीवितानुग्रहः कृतः}
{ऊर्ध्वं द्वाभ्यां तु मासाभ्यां ततस्त्यक्ष्यामि जीवितम्}

इत्यार्षे श्रीमद्रामायणे वाल्मीकीये आदिकाव्ये चतुर्विंशतिसहस्रिकायां संहितायाम्
सुन्दरकाण्डे हनूमज्जानकीसंवादो नाम त्रयस्त्रिंशः सर्गः॥

\closesection
    \chapt{हनूमदुपदेशः}

\src{श्रीमद्-वाल्मीकि-रामायणम्}{सुन्दरकाण्डः}{अध्यायः ५१}{श्लोकाः १---४६}
\vakta{हनुमान्}
\shrota{रावणादयः}
\tags{concise, part}
\notes{Narration of Rama's story and prowess by Hanuman, in Ravana's court.}
\textlink{}
\translink{}

\storymeta

\twolineshloka
{तं समीक्ष्य महासत्त्वं सत्त्ववान् हरिसत्तमः}
{वाक्यमर्थवदव्यग्रस्तमुवाच दशाननम्}

\twolineshloka
{अहं सुग्रीवसन्देशादिह प्राप्तस्तवालयम्}
{राक्षसेन्द्र हरीशस्त्वां भ्राता कुशलमब्रवीत्}

\twolineshloka
{भ्रातुः शृणु समादेशं सुग्रीवस्य महात्मनः}
{धर्मार्थोपहितं वाक्यमिह चामुत्र च क्षमम्}

\twolineshloka
{राजा दशरथो नाम रथकुञ्जरवाजिमान्}
{पितेव बन्धुर्लोकस्य सुरेश्वरसमद्युतिः}

\twolineshloka
{ज्येष्ठस्तस्य महाबाहुः पुत्रः प्रियकरः प्रभुः}
{पितुर्निदेशान्निष्क्रान्तः प्रविष्टो दण्डकावनम्}

\twolineshloka
{लक्ष्मणेन सह भ्रात्रा सीतया चापि भार्यया}
{रामो नाम महातेजा धर्म्यं पन्थानमाश्रितः}

\twolineshloka
{तस्य भार्या वने नष्टा सीता पतिमनुव्रता}
{वैदेहस्य सुता राज्ञो जनकस्य महात्मनः} 

\twolineshloka
{स मार्गमाणस्तां देवीं राजपुत्रः सहानुजः}
{ऋश्यमूकमनुप्राप्तः सुग्रीवेण समागतः}

\twolineshloka
{तस्य तेन प्रतिज्ञातं सीतायाः परिमार्गणम्}
{सुग्रीवस्यापि रामेण हरिराज्यं निवेदितम्}

\twolineshloka
{ततस्तेन मृधे हत्वा राजपुत्रेण वालिनम्}
{सुग्रीवः स्थापितो राज्ये हर्यृक्षाणां गणेश्वरः}

\twolineshloka
{त्वया विज्ञातपूर्वश्च वाली वानरपुङ्गवः}
{रामेण निहतः सङ्ख्ये शरेणैकेन वानरः}

\twolineshloka
{स सीतामार्गणे व्यग्रः सुग्रीवः सत्यसङ्गरः}
{हरीन् सम्प्रेषयामास दिशः सर्वा हरीश्वरः}

\twolineshloka
{तां हरीणां सहस्राणि शतानि नियुतानि च}
{दिक्षु सर्वासु मार्गन्ते ह्यधश्चोपरि चाम्बरे}

\twolineshloka
{वैनतेयसमाः केचित् केचित्तत्रानिलोपमाः}
{असङ्गगतयः शीघ्रा हरिवीरा महाबलाः}

\twolineshloka
{अहं तु हनुमान्नाम मारुतस्यौरसः सुतः}
{सीतायास्तु कृते तूर्णं शतयोजनमायतम्}

\twolineshloka
{समुद्रं लङ्घयित्वैव तां दिदृक्षुरिहागतः}
{भ्रमता च मया दृष्टा गृहे ते जनकात्मजा}

\twolineshloka
{तद्भवान् दृष्टधर्मार्थस्तपः कृतपरिग्रहः}
{परदारान् महाप्राज्ञ नोपरोद्धुं त्वमर्हसि}

\twolineshloka
{न हि धर्मविरुद्धेषु बह्वपायेषु कर्मसु}
{मूलघातिषु सज्जन्ते बुद्धिमन्तो भवद्विधाः}

\twolineshloka
{कश्च लक्ष्मणमुक्तानां रामकोपानुवर्तिनाम्}
{शराणामग्रतः स्थातुं शक्तो देवासुरेष्वपि}

\twolineshloka
{न चापि त्रिषु लोकेषु राजन् विद्येत कश्चन}
{राघवस्य व्यलीकं यः कृत्वा सुखमवाप्नुयात्}

\twolineshloka
{तत् त्रिकालहितं वाक्यं धर्म्यमर्थानुबन्धि च}
{मन्यस्व नरदेवाय जानकी प्रतिदीयताम्}

\twolineshloka
{दृष्टा हीयं मया देवी लब्धं यदिह दुर्लभम्}
{उत्तरं कर्म यच्छेषं निमित्तं तत्र राघवः}

\twolineshloka
{लक्षितेयं मया सीता तथा शोकपरायणा}
{गृह्य यां नाभिजानासि पञ्चास्यामिव पन्नगीम्}

\twolineshloka
{नेयं जरयितुं शक्या सासुरैरमरैरपि}
{विषसंसृष्टमत्यर्थं भुक्तमन्नमिवौजसा}

\twolineshloka
{तपःसन्तापलब्धस्ते योऽयं धर्मपरिग्रहः}
{न स नाशयितुं न्याय्य आत्मप्राणपरिग्रहः}

\twolineshloka
{अवध्यतां तपोभिर्यां भवान् समनुपश्यति}
{आत्मनः सासुरैर्देवैर्हेतुस्तत्राप्ययं महान्}

\twolineshloka
{सुग्रीवो न हि देवोऽयं नासुरो न च राक्षसः}
{न दानवो न गन्धर्वो न यक्षो न च पन्नगः}

\twolineshloka
{तस्मात् प्राणपरित्राणं कथं राजन् करिष्यसि}
{ननु धर्मोपसंहारमधर्मफलसंहितम्}

\twolineshloka
{तदेव फलमन्वेति धर्मश्चाधर्मनाशनः}
{प्राप्तं धर्मफलं तावद्भवता नात्र संशयः}

\twolineshloka
{फलमस्याप्यधर्मस्य क्षिप्रमेव प्रपत्स्यसे}
{जनस्थानवधं बुद्ध्वा बुद्ध्वा वालिवधं तथा}

\twolineshloka
{रामसुग्रीवसख्यं च बुध्यस्व हितमात्मनः}
{कामं खल्वहमप्येकः सवाजिरथकुञ्जराम्}

\twolineshloka
{लङ्कां नाशयितुं शक्तस्तस्यैष तु न निश्चयः}
{रामेण हि प्रतिज्ञातं हर्यृक्षगणसन्निधौ}

\twolineshloka
{उत्सादनममित्राणां सीता यैस्तु प्रधर्षिता}
{अपकुर्वन् हि रामस्य साक्षादपि पुरन्दरः}

\twolineshloka
{न सुखं प्राप्नुयादन्यः किं पुनस्त्वद्विधो जनः}
{यां सीतेत्यभिजानासि येयं तिष्ठति ते वशे}

\twolineshloka
{कालरात्रीति तां विद्धि सर्वलङ्काविनाशिनीम्}
{तदलं कालपाशेन सीताविग्रहरूपिणा}

\twolineshloka
{स्वयं स्कन्धावसक्तेन क्षममात्मनि चिन्त्यताम्}
{सीतायास्तेजसा दग्धां रामकोपप्रपीडिताम्}

\twolineshloka
{दह्यमानामिमां पश्य पुरीं साट्टप्रतोलिकाम्}
{स्वानि मित्राणि मन्त्रींश्च ज्ञातीन्भ्रातॄन्सुतान्हितान्}

\twolineshloka
{भोगान् दारांश्च लङ्कां च मा विनाशमुपानय}
{सत्यं राक्षसराजेन्द्र शृणुष्व वचनं मम}

\twolineshloka
{रामदासस्य दूतस्य वानरस्य विशेषतः}
{सर्वाँल्लोकान् सुसंहृत्य सभूतान् सचराचरान्}

\twolineshloka
{पुनरेव तदा स्रष्टुं शक्तो रामो महायशाः}
{देवासुरनरेन्द्रेषु यक्षरक्षोगणेषु च}

\twolineshloka
{विद्याधरेषु सर्वेषु गन्धर्वेषूरगेषु च}
{सिद्धेषु किन्नरेन्द्रेषु पतत्रिषु च सर्वतः}

\twolineshloka
{सर्वभूतेषु सर्वत्र सर्वकालेषु नास्ति सः}
{यो रामं प्रतियुध्येत विष्णुतुल्यपराक्रमम्}

\twolineshloka
{सर्वलोकेश्वरस्यैवं कृत्वा विप्रियमीदृशम्}
{रामस्य राजसिंहस्य दुर्लभं तव जीवितम्}

\fourlineindentedshloka
{देवाश्च दैत्याश्च निशाचरेन्द्र}
{गन्धर्वविद्याधरनागयक्षाः}
{रामस्य लोकत्रयनायकस्य}
{स्थातुं न शक्ताः समरेषु सर्वे}

\fourlineindentedshloka
{ब्रह्मा स्वयम्भूश्चतुराननो वा}
{रुद्रस्त्रिनेत्रस्त्रिपुरान्तको वा}
{इन्द्रो महेन्द्रः सुरनायको वा}
{त्रातुं न शक्ता युधि रामवध्यम्}

\fourlineindentedshloka
{स सौष्ठवोपेतमदीनवादिनः}
{कपेर्निशम्याप्रतिमोऽप्रियं वचः}
{दशाननः कोपविवृत्तलोचनः}
{समादिशत्तस्य वधं महाकपेः}

इत्यार्षे श्रीमद्रामायणे वाल्मीकीये आदिकाव्ये चतुर्विंशतिसहस्रिकायां संहितायाम् सुन्दरकाण्डे हनूमदुपदेशो नाम एकपञ्चाशः सर्गः॥

\closesection
    \sect{हनूमद्भरतसम्भाषणम्}

\src{श्रीमद्-वाल्मीकि-रामायणम्}{युद्धकाण्डः}{अध्यायः १२९}{श्लोकाः १---५४}
\vakta{हनुमान्}
\shrota{भरतः}
\tags{concise, complete}
\notes{Narration of Rama's story, from Bharata's departing Chitrakuta to the retrieval of Sita by conquering Lanka, by Hanuman.}
\textlink{}
\translink{}

\storymeta


\twolineshloka
{बहूनि नाम वर्षाणि गतस्य सुमहद्धनम्}
{शृणोम्यहं प्रीतिकरं मम नाथस्य कीर्तनम्}

\twolineshloka
{कल्याणी बत गाथेयं लौकिकी प्रतिभाति मे}
{एति जीवन्तमानन्दो नरं वर्षशतादपि}

\twolineshloka
{राघवस्य हरीणां च कथमासीत् समागमः}
{कस्मिन् देशे किमाश्रित्य तत्त्वमाख्याहि पृच्छतः}

\twolineshloka
{स पृष्टो राजपुत्रेण बृस्यां समुपवेशितः}
{आचचक्षे ततः सर्वं रामस्य चरितं वने}

\twolineshloka
{यथा प्रव्राजितो रामो मातुर्दत्तो वरस्तव}
{यथा च पुत्रशोकेन राजा दशस्थो मृतः}

\twolineshloka
{यथा दूतैस्त्वमानीतस्तूर्णं राजगृहात् प्रभो}
{त्वयायोध्यां प्रविष्टेन यथा राज्यं न चेप्सितम्}

\twolineshloka
{चित्रकूटं गिरिं गत्वा राज्येनामित्रकर्शन}
{निमन्त्रितस्त्वया भ्राता धर्ममाचरता सताम्}

\twolineshloka
{स्थितेन राज्ञो वचने यथा राज्यं विसर्जितम्}
{आर्यस्य पादुके गृह्य यथासि पुनरागतः}

\twolineshloka
{सर्वमेतन्महाबाहो यथावद्विदितं तव}
{त्वयि प्रतिप्रयाते तु यद्वृत्तं तन्निबोध मे}


\twolineshloka
{अपयाते त्वयि तदा समुद्भ्रान्तमृगद्विजम्}
{परिद्यूनमिवात्यर्थं तद्वनं समपद्यत}

\twolineshloka
{तद्धस्तिमृदितं घोरं सिंहव्याघ्रमृगायुतम्}
{प्रविवेशाथ विजनं सुमहद्दण्डकावनम्}

\twolineshloka
{तेषां पुरस्ताद्बलवान् गच्छतां गहने वने}
{निनदन् सुमहानादं विराधः प्रत्यदृश्यत}

\twolineshloka
{तमुत्क्षिप्य महानादमूर्ध्वबाहुमधोमुखम्}
{निखाते प्रक्षिपन्ति स्म नदन्तमिव कुञ्जरम्}

\twolineshloka
{तत् कृत्वा दुष्करं कर्म भ्रातरौ रामलक्ष्मणौ}
{सायाह्ने शरभङ्गस्य रम्यमाश्रममीयतुः}

\twolineshloka
{शरभङ्गे दिवं प्राप्ते रामः सत्यपराक्रमः}
{अभिवाद्य मुनीन् सर्वाञ्जनस्थानमुपागमत्}

\twolineshloka
{ततः पश्चाच्छूर्पणखा रामपार्श्वमुपागता}
{ततो रामेण सन्दिष्टो लक्ष्मणः सहसोत्थितः}

\twolineshloka
{प्रगृह्य खड्गं चिच्छेद कर्णनासं महाबलः}
{चतुर्दश सहस्राणि रक्षसां भीमकर्मणाम्}

\twolineshloka
{हतानि वसता तत्र राघवेण महात्मना}
{एकेन सह सङ्गम्य रणे रामेण सङ्गताः}

\twolineshloka
{अतुर्थभागेन निःशेषा राक्षसाः कृताः}
{महाबला महावीर्यास्तपसो विघ्नकारिणः}

\twolineshloka
{निहता राघवेणाजौ दण्डकारण्यवासिनः}
{राक्षसाश्च विनिष्पिष्टाः खरश्च निहतो रणे}

\twolineshloka
{ततस्तेनार्दिता बाला रावणं समुपागता}
{रावणानुचरो घोरो मारीचो नाम राक्षसः}

\twolineshloka
{लोभयामास वैदेहीं भूत्वा रत्नमयो मृगः}
{अथैनमब्रवीद्रामं वैदेही गृह्यतामिति}

\twolineshloka
{अहो मनोहरः कान्त आश्रमो नो भविष्यति}
{ततो रामो धनुष्पाणिर्धावन्तमनुधावति}

\twolineshloka
{स तं जघान धावन्तं शरेणानतपर्वणा}
{अथ सौम्य दशग्रीवो मृगं याते तु राघवे}

\twolineshloka
{लक्ष्मणे चापि निष्क्रान्ते प्रविवेशाश्रमं तदा}
{जग्राह तरसा सीतां ग्रहः खे रोहिणीमिव}

\twolineshloka
{त्रातुकामं ततो युद्धे हत्वा गृध्रं जटायुषम्}
{प्रगृह्य सीतां सहसा जगामाशु स रावणः}

\twolineshloka
{ततस्त्वद्भुतसङ्काशाः स्थिताः पर्वतमूर्धनि}
{सीतां गृहीत्वा गच्छन्तं वानराः पर्वतोपमाः}

\twolineshloka
{दशुर्विस्मितास्तत्र रावणं राक्षसाधिपम्}
{प्रविवेश ततो लङ्कां रावणो लोकरावणः}

\twolineshloka
{तां सुवर्णपरिक्रान्ते शुभे महति वेश्मनि}
{प्रवेश्य मैथिलीं वाक्यैः सान्त्वयामास रावणः}

\twolineshloka
{तृणकद्भाषितं तस्य तं च नैर्ऋतपुङ्गवम्}
{अचिन्तयन्ती वैदेही अशोकवनिकां गता}

\twolineshloka
{न्यवर्तत ततो रामो मृगं हत्वा महावने}
{निवर्तमानः काकुत्स्थो दृष्ट्वा गृधं प्रविव्यथे}

\twolineshloka
{गृद्धं हतं ततो दग्ध्वा रामः प्रियसखं पितुः}
{मार्गमाणस्तु वैदेहीं राघवः सहलक्ष्मणः}

\twolineshloka
{गोदावरीमन्वचरद्वनोद्देशांश्च पुष्पितान्}
{आसेदतुर्महारण्ये कबन्धं नाम राक्षसम्}

\twolineshloka
{ततः कबन्धवचनाद्रामः सत्यपराक्रमः}
{ऋश्यमूकं गिरिं गत्वा सुग्रीवेण समागतः}

\twolineshloka
{तयोः समागमः पूर्वं प्रीत्या हार्दो व्यजायत}
{भ्रात्रा निरस्तः क्रुद्धेन सुग्रीवो वालिना पुरा}

\twolineshloka
{इतरेतरसंवादात् प्रगाढः प्रणयस्तयोः}
{रामस्य बाहुवीर्येण स्वराज्यं प्रत्यपादयत्}


\twolineshloka
{वालिनं समरे हत्वा महाकायं महाबलम्}
{सुग्रीवः स्थापितो राज्ये सहितः सर्ववानरैः}

\twolineshloka
{रामाय प्रतिजानीते राजपुत्र्याश्च मार्गणम्}
{आदिष्टा वानरेन्द्रेण सुग्रीवेण महात्मना}

\twolineshloka
{दश कोट्यः प्लवङ्गानां सर्वाः प्रस्थापिता दिशः}
{तेषां नो विप्रकृष्टानां विन्ध्ये पर्वतसत्तमे}

\twolineshloka
{भृशं शोकाभितप्तानां महान् कालोऽत्यवर्तत}
{भ्राता तु गृधराजस्य सम्पातिर्नाम वीर्यवान्}

\twolineshloka
{समाख्याति स्म वसतिं सीताया रावणालये}
{सोऽहं शोकपरीतानां दुःखं तज्ज्ञातिनां नुदन्}

\twolineshloka
{आत्मवीर्यं समास्थाय योजनानां शतं प्लुतः}
{तत्राहमेकामद्राक्षमशोकवनिकां गताम्}

\twolineshloka
{कौशेयवस्त्रां मलिनां निरानन्दां दृढव्रताम्}
{तया समेत्य विधिवत् पृष्ट्वा सर्वमनिन्दिताम्}

\twolineshloka
{अभिज्ञानं च मे दत्तमर्चिष्मान् स महामणिः}
{अभिज्ञानं मणि लब्ध्वा चरितार्थोऽहमागतः}

\twolineshloka
{मया च पुनरागम्य रामस्याकिष्टकर्मणः}
{अभिज्ञानं मया दत्तमर्चिष्मान् स महामणिः}

\twolineshloka
{श्रुत्वा तु मैथिली हृष्टस्त्वाशश॑से॒ च जीवितम्}
{जीवितान्तमनुप्राप्तः पीत्वामृतमिवातुरः}

\twolineshloka
{उद्योजयिष्यन्नुद्योगं दधे कामं वधे मनः}
{जिघांसुरिव लोकान्ते सर्वाल्लोकान् विभावसुः}

\twolineshloka
{ततः समुद्रमासाद्य नलं सेतुमकारयत्}
{अतरत् कपिवीराणां वाहिनी तेन सेतुना}

\twolineshloka
{प्रहस्तमवधीन्नीलः कुम्भकर्णं तु राघवः}
{लक्ष्मणो रावणसुतं स्वयं रामस्तु रावणम्}

\twolineshloka
{स शक्रेण समागम्य यमेन वरुणेन च}
{महेश्वरः स्वयं भूम्यां तथा दशरथेन च}

\twolineshloka
{तैश्च दत्तवरः श्रीमानृषिभिश्च समागतः}
{सुरर्षिभिश्च काकुत्स्थो वरौलेमे परन्तपः}

\twolineshloka
{स तु दत्तवरः प्रीत्या वानरैश्च समागतः}
{पुष्पकेण विमानेन किष्किन्धामभ्युपागमत्}

\twolineshloka
{तं गङ्गां पुनरासाद्य वसन्तं मुनिसन्निधौ}
{अविघ्नं पुष्ययोगेन श्वो रामं द्रष्टुमर्हसि}

\fourlineindentedshloka
{ततस्तु सत्यं हनुमद्वचो महन्-}
{निशम्य हृष्टो भरतः कृताञ्जलिः}
{उवाच वाणीं मनसः प्रहर्षिणीं}
{चिरस्य पूर्णः खलु मे मनोरथः} 


इत्यार्षे श्रीमद्रामायणे वाल्मीकीये आदिकाव्ये चतुर्विंशतिसहत्रिकायां संहितायाम् युद्धकाण्डे हनूमद्भरतसम्भाषणं नाम एकोनत्रिंशदुत्तरशततमः सर्गः॥

\closesection
    \chapt{गायत्री रामयाणम्}

\annotwolineshloka
{तपः स्वाध्यायनिरतं तपस्वी वाग्विदां वरम्}
{नारदं परिपप्रच्छ वाल्मीकिर्मुनिपुङ्गवम्}{१-१-१}

\annotwolineshloka
{स हत्वा राक्षसान् सर्वान् यज्ञघ्नान् रघुनन्दनः}
{ऋषिभिः पूजितः सम्यक् यथेन्द्रो विजये पुरा}{१-३०-२३}

\annotwolineshloka
{विश्वामित्रस्तु धर्मात्मा श्रुत्वा जनकभाषितम्}
{वत्स राम धनुः पश्य इति राघवमब्रवीत्}{१-६७-१२}

\annotwolineshloka
{तुष्टावास्य तदा वंशं  प्रविश्य च विशाम्पतेः}
{शयनीयं नरेन्द्रस्य तदासाद्य व्यतिष्ठत}{२-१५-२०}

\annotwolineshloka
{वनवासं हि सङ्ख्याय वासांस्याभरणानि च}
{भर्तारमनुगच्छन्त्यै सीतायै श्वशुरो ददौ}{२-४०-१५}

\annotwolineshloka
{राजा सत्यं च धर्मं च  राजा कुलवतां कुलम्}
{राजा माता पिता चैव राजा हितकरो नृणाम्}{२-६७-३४}

\annotwolineshloka
{निरीक्ष्य स मुहूर्तं तु ददर्श भरतो गुरुम्}
{उटजे राममासीनं जटामण्डलधारिणम्}{२-९९-२५}

\annotwolineshloka
{यदि बुद्धिः कृता द्रष्टुम् अगस्त्यं तं महामुनिम्}
{अद्यैव गमने बुद्धिं रोचयस्व महायशाः}{३-११-४४}

\annotwolineshloka
{भरतस्यार्यपुत्रस्य श्वश्रूणां मम च प्रभो}
{मृगरूपमिदं व्यक्तं विस्मयं जनयिष्यति}{३-४३-१७}

\annotwolineshloka
{गच्छ शीघ्रमितो राम सुग्रीवं तं महाबलम्}
{वयस्यं तं कुरु क्षिप्रमितो गत्वाऽद्य राघव}{३-७२-१७}

\annotwolineshloka
{देशकालौ प्रतीक्षस्व क्षममाणः प्रियाप्रिये}
{सुखदुःखसहः काले  सुग्रीववशगो भव}{४-२२-२०}

\annotwolineshloka
{वन्द्यास्ते तु तपः सिद्धास्तपसा वीतकल्मषाः}
{प्रष्टव्याश्चापि सीतायाः प्रवृत्तिं विनयान्वितैः}{४-४३-३४}

\annotwolineshloka
{स निर्जित्य पुरीं श्रेष्ठां लङ्कां तां कामरूपिणीम्}
{विक्रमेण महातेजा हनूमान्मारुतात्मजः}{५-४-१}

\annotwolineshloka
{धन्या देवाः सगन्धर्वाः सिद्धाश्च परमर्षयः}
{मम पश्यन्ति ये नाथं रामं राजीवलोचनम्}{५-२६-४१}

\annotwolineshloka
{मङ्गलाभिमुखी तस्य सा तदासीन्महाकपेः}
{उपतस्थे विशालाक्षी प्रयता हव्यवाहनम्}{५-५३-२६}

\annofourlineindentedshloka
{हितं महार्थं मृदु हेतुसंहितम्}
{व्यतीतकालायतिसम्प्रतिक्षमम्}
{निशम्य तद्वाक्यमुपस्थितज्वरः}
{प्रसङ्गवानुत्तरमेतदब्रवीत्}{६-१०-२७}

\annotwolineshloka
{धर्मात्मा रक्षसां श्रेष्ठः सम्प्राप्तोऽयं विभीषणः}
{लङ्कैश्वर्यं ध्रुवं श्रीमानयं प्राप्नोत्यकण्टकम्}{६-४१-६८}

\annofourlineindentedshloka
{यो वज्रपाताशनिसन्निपातान्}{न चुक्षुभे नापि चचाल राजा}
{स रामबाणाभिहतो भृशार्तः}{चचाल चापं च मुमोच वीरः}{६-५९-१४०}

\annotwolineshloka
{यस्य विक्रममासाद्य राक्षसा निधनं गताः}
{तं मन्ये राघवं वीरं नारायणमनामयम्}{६-७२-११}

\annotwolineshloka
{न ते ददर्शिरे रामं दहन्तमरिवाहिनीम्}
{मोहिताः परमास्त्रेण गान्धर्वेण महात्मना}{६-९४-२६}

\annotwolineshloka
{प्रणम्य देवताभ्यश्च ब्राह्मणेभ्यश्च मैथिली}
{बद्धाञ्जलिपुटा चेदमुवाचाग्निसमीपतः}{६-११९-२३}

\annotwolineshloka
{चलनात्पर्वतेन्द्रस्य गणा देवाश्च कम्पिताः}
{चचाल पार्वती चापि तदाऽऽश्लिष्टा महेश्वरम्}{७-१६-२६}

\annotwolineshloka
{दाराः पुत्राः पुरं राष्ट्रं भोगाच्छादनभोजनम्}
{सर्वमेवाविभक्तं नौ भविष्यति हरीश्वर}{७-३४-४१}

\annotwolineshloka
{यामेव रात्रिं शत्रुघ्नः पर्णशालां समाविशत्}
{तामेव रात्रिं सीताऽपि प्रसूता दारकद्वयम्}{७-६६-१}

\twolineshloka*
{इदं रामायणं कृत्स्नं गायत्रीबीजसंयुतम्}
{त्रिसन्ध्यं यः पठेन्नित्यं सर्वपापैः प्रमुच्यते}

॥इति श्री-गायत्री रामायणं सम्पूर्णम्॥

\closesection
    \chapt{अध्यात्म-रामायणम्}

\sect{रामहृदये रामचरितम्}

\src{अध्यात्म-रामायणम्}{बालकाण्डः}{अध्यायः १}{श्लोकाः ३२--४३}
\vakta{सीता}
\shrota{हनुमान्}
\notes{In the opening chapter of Adhyatma Ramayana, Sita describes the divine nature of Rama and the various events of His life, which She attributes to Her own divine presence. This chapter is often referred to as ``Ramahṛdayam'' or ``The Heart of Rama.''}
\textlink{}
\translink{}

\storymeta


\uvacha{सीतोवाच}

\addtocounter{shlokacount}{31}

\twolineshloka
{रामं विद्धि परं ब्रह्म सच्चिदानन्दमद्वयम्}
{सर्वोपाधिविनिर्मुक्तं सत्तामात्रमगोचरम्} %1-32

\twolineshloka
{आनन्दं निर्मलं शान्तं निर्विकारं निरञ्जनम्}
{सर्वव्यापिनमात्मानं स्वप्रकाशमकल्मषम्} %1-33

\twolineshloka
{मां विद्धि मूलप्रकृतिं सर्गस्थित्यन्तकारिणीम्}
{तस्य सन्निधिमात्रेण सृजामीदमतन्द्रिता} %1-34

\twolineshloka
{तत्सान्निध्यान्मया सृष्टं तस्मिन्नारोप्यतेऽबुधैः}
{अयोध्यानगरे जन्म रघुवंशेऽतिनिर्मले} %1-35

\twolineshloka
{विश्वामित्रसहायत्वं मखसंरक्षणं ततः}
{अहल्याशापशमनं चापभङ्गो महेशितुः} %1-36

\twolineshloka
{मत्पाणिग्रहणं पश्चाद्भार्गवस्य मदक्षयः}
{अयोध्यानगरे वासो मया द्वादशवार्षिकः} %1-37

\twolineshloka
{दण्डकारण्यगमनं विराधवध एव च}
{मायामारीचमरणं मायासीताहृतिस्तथा} %1-38

\twolineshloka
{जटायुषो मोक्षलाभः कबन्धस्य तथैव च}
{शबर्याः पूजनं पश्चात्सुग्रीवेण समागमः} %1-39

\twolineshloka
{वालिनश्च वधः पश्चात्सीतान्वेषणमेव च}
{सेतुबन्धश्च जलधौ लङ्कायाश्च निरोधनम्} %1-40

\twolineshloka
{रावणस्य वधो युद्धे सपुत्रस्य दुरात्मनः}
{विभीषणे राज्यदानं पुष्पकेण मया सह} %1-41

\threelineshloka
{अयोध्यागमनं पश्चाद्राज्ये रामाभिषेचनम्}
{एवमादीनि कर्माणि मयैवाचरितान्यपि}
{आरोपयन्ति रामेऽस्मिन्निर्विकारेऽखिलात्मनि} %1-42

\fourlineindentedshloka
{रामो न गच्छति न तिष्ठति नानुशोच-}
{त्याकाङ्क्षते त्यजति नो न करोति किञ्चित्}
{आनन्दमूर्तिरचलः परिणामहीनो}
{मायागुणाननुगतो हि तथा विभाति} %1-43

॥इति श्रीमदध्यात्मरामायणे उमामहेश्वरसंवादे बालकाण्डे रामहृदयं नाम प्रथमे सर्गे रामचरितं सम्पूर्णम्॥

\closesection
    \chapt{रामकथाकथनम्}

\src{अध्यात्म-रामायणम्}{सुन्दरकाण्डः}{अध्यायः ३}{श्लोकाः १--३६}
\vakta{हनुमान्}
\shrota{सीता}
\notes{}
\textlink{}
\translink{}

\storymeta

\uvacha{श्री-महादेव उवाच}

\twolineshloka
{उद्बन्धनेन वा मोक्ष्ये शरीरं राघवं विना}
{जीवितेन फलं किं स्यान्मम रक्षोऽधिमध्यतः} %3-1

\twolineshloka
{दीर्घा वेणी ममात्यर्थमुद्बन्धाय भविष्यति}
{एवं निश्चितबुद्धिं तां मरणायाथ जानकीम्} %3-2

\twolineshloka
{विलोक्य हनुमान् किञ्चिद्विचार्यैतदभाषत}
{शनैः शनैः सूक्ष्मरूपो जानक्याः श्रोत्रगं वचः} %3-3

\twolineshloka
{इक्ष्वाकुवंशसम्भूतो राजा दशरथो महान्}
{अयोध्याधिपतिस्तस्य चत्वारो लोकविश्रुताः} %3-4

\twolineshloka
{पुत्रा देवसमाः सर्वे लक्षणैरुपलक्षिताः}
{रामश्च लक्ष्मणश्चैव भरतश्चैव शत्रुहा} %3-5

\twolineshloka
{ज्येष्ठो रामः पितुर्वाक्याद्दण्डकारण्यमागतः}
{लक्ष्मणेन सह भ्रात्रा सीतया भार्यया सह} %3-6

\twolineshloka
{उवास गौतमीतीरे पञ्चवट्यां महामनाः}
{तत्र नीता महाभागा सीता जनकनन्दिनी} %3-7

\twolineshloka
{रहिते रामचन्द्रेण रावणेन दुरात्मना}
{ततो रामोऽतिदुःखार्तो मार्गमाणोऽथ जानकीम्} %3-8

\twolineshloka
{जटायुषं पक्षिराजमपश्यत्पतितं भुवि}
{तस्मै दत्त्वा दिवं शीघ्रमृष्यमूकमुपागमत्} %3-9

\twolineshloka
{सुग्रीवेण कृता मैत्री रामस्य विदितात्मनः}
{तद्भार्याहारिणं हत्वा वालिनं रघुनन्दनः} %3-10

\twolineshloka
{राज्येऽभिषिच्य सुग्रीवं मित्रकार्यं चकार सः}
{सुग्रीवस्तु समानाय्य वानरान् वानरप्रभुः} %3-11

\twolineshloka
{प्रेषयामास परितो वानरान् परिमार्गणे}
{सीतायास्तत्र चैकोऽहं सुग्रीवसचिवो हरिः} %3-12

\twolineshloka
{सम्पातिवचनाच्छीघ्रमुल्लङ्घ्य शतयोजनम्}
{समुद्रं नगरीं लङ्कां विचिन्वन् जानकीं शुभाम्} %3-13

\twolineshloka
{शनैरशोकवनिकां विचिन्वन् शिंशपातरुम्}
{अद्राक्षं जानकीमत्र शोचन्तीं दुःखसम्प्लुताम्} %3-14

\twolineshloka
{रामस्य महिषीं देवीं कृतकृत्योऽहमागतः}
{इत्युक्त्वोपररामाथ मारुतिर्बुद्धिमत्तरः} %3-15

\twolineshloka
{सीता क्रमेण तत्सर्वं श्रुत्वा विस्मयमाययौ}
{किमिदं मे श्रुतं व्योम्नि वायुना समुदीरितम्} %3-16

\twolineshloka
{स्वप्नो वा मे मनोभ्रान्तिर्यदि वा सत्यमेव तत्}
{निद्रा मे नास्ति दुःखेन जानाम्येतत्कुतो भ्रमः} %3-17

\twolineshloka
{येन मे कर्णपीयुषं वचनं समुदीरितम्}
{स दृश्यतां महाभागः प्रियवादी ममाग्रतः} %3-18

\twolineshloka
{श्रुत्वा तज्जानकीवाक्यं हनुमान् पत्रखण्डतः}
{अवतीर्य शनैः सीतापुरतः समवस्थितः} %3-19

\twolineshloka
{कलविङ्कप्रमाणाङ्गो रक्तास्यः पीतवानरः}
{ननाम शनकैः सीतां प्राञ्जलिः पुरतः स्थितः} %3-20

\twolineshloka
{दृष्ट्वा तं जानकी भीता रावणोऽयमुपागतः}
{मां मोहयितुमायातो मायया वानराकृतिः} %3-21

\twolineshloka
{इत्येवं चिन्तयित्वा सा तूष्णीमासीदधोमुखी}
{पुनरप्याह तां सीतां देवि यत्त्वं विशङ्कसे} %3-22

\twolineshloka
{नाहं तथाविधो मातस्त्यज शङ्कां मयि स्थिताम्}
{दासोऽहं कोसलेन्द्रस्य रामस्य परमात्मनः} %3-23

\twolineshloka
{सचिवोऽहं हरीन्द्रस्य सुग्रीवस्य शुभप्रदे}
{वायोः पुत्रोऽहमखिलप्राणभूतस्य शोभने} %3-24

\twolineshloka
{तच्छ्रुत्वा जानकी प्राह हनूमन्तं कृताञ्जलिम्}
{वानराणां मनुष्याणां सङ्गतिर्घटते कथम्} %3-25

\twolineshloka
{यथा त्वं रामचन्द्रस्य दासोऽहमिति भाषसे}
{तामाह मारुतिः प्रीतो जानकीं पुरतः स्थितः} %3-26

\twolineshloka
{ऋष्यमूकमगाद्रामः शबर्या नोदितः सुधीः}
{सुग्रीवो ऋष्यमूकस्थो दृष्टवान् रामलक्ष्मणौ} %3-27

\twolineshloka
{भीतो मां प्रेषयामास ज्ञातुं रामस्य हृद्गतम्}
{ब्रह्मचारिवपुर्धृत्वा गतोऽहं रामसन्निधिम्} %3-28

\twolineshloka
{ज्ञात्वा रामस्य सद्भावं स्कन्धोपरि निधाय तौ}
{नीत्वा सुग्रीवसामीप्यं सख्यं चाकरवं तयोः} %3-29

\twolineshloka
{सुग्रीवस्य हृता भार्या वालिना तं रघूत्तमः}
{जघानैकेन बाणेन ततो राज्येऽभ्यषेचयत्} %3-30

\twolineshloka
{सुग्रीवं वानराणां स प्रेषयामास वानरान्}
{दिग्भ्यो महाबलान् वीरान् भवत्याः परिमार्गणे} %3-31

\onelineshloka
{गच्छन्तं राघवो दृष्ट्वा मामभाषत सादरम्} %3-32

\twolineshloka
{त्वयि कार्यमशेषं मे स्थितं मारुतनन्दन}
{ब्रूहि मे कुशलं सर्वं सीतायै लक्ष्मणस्य च} %3-33

\twolineshloka
{अङ्गुलीयकमेतन्मे परिज्ञानार्थमुत्तमम्}
{सीतायै दीयतां साधु मन्नामाक्षरमुद्रितम्} %3-34

\twolineshloka
{इत्युक्त्वा प्रददौ मह्यं कराग्रादङ्गुलीयकम्}
{प्रयत्नेन मयाऽऽनीतं देवि पश्याङ्गुलीयकम्} %3-35

\twolineshloka
{इत्युक्त्वा प्रददौ देव्यै मुद्रिकां मारुतात्मजः}
{नमस्कृत्य स्थितो दूराद्बद्धाञ्जलिपुटो हरिः} %3-36

॥इति श्रीमदध्यात्मरामायणे उमामहेश्वरसंवादे सुन्दरकाण्डे तृतीये सर्गे रामकथाकथनं सम्पूर्णम्॥

\closesection
    \chapt{श्रीमद्-भागवतम्}

\sect{श्रीरामचरितम्}

\src{श्रीमद्-भागवतम्}{नवमः स्कन्धः}{अध्यायः १०}{श्लोकाः १---५६}
\vakta{शुकः}
\shrota{परीक्षितः}
\tags{concise, complete}
\notes{This chapter recounts the appearance of Lord Rāmachandra in the lineage of Mahārāja Khaṭvāṅga and details His divine exploits, including the slaying of Rāvaṇa and His triumphant return to Ayodhyā.}
\textlink{http://stotrasamhita.net/wiki/Bhagavatam/Skandha_09/Adhyaya_10}
\translink{https://www.wisdomlib.org/hinduism/book/the-bhagavata-purana/d/doc1128849.html}

\storymeta


\uvacha{श्रीशुक उवाच}

\twolineshloka
{खट्वाङ्गाद्दीर्घबाहुश्च रघुस्तस्मात्पृथुश्रवाः}
{अजस्ततो महाराजस्तस्माद्दशरथोऽभवत्} %1

\threelineshloka
{तस्यापि भगवानेष साक्षाद्ब्रह्ममयो हरि}
{अंशांशेन चतुर्धागात्पुत्रत्वं प्रार्थितः सुरै}
{रामलक्ष्मणभरत शत्रुघ्ना इति संज्ञया} %2

\twolineshloka
{तस्यानुचरितं राजन्नृषिभिस्तत्त्वदर्शिभिः}
{श्रुतं हि वर्णितं भूरि त्वया सीतापतेर्मुहुः} %3

\fourlineindentedshloka
{गुर्वर्थे त्यक्तराज्यो व्यचरदनुवनं पद्मपद्भ्यां प्रियायाः}
{पाणिस्पर्शाक्षमाभ्यां मृजितपथरुजो यो हरीन्द्रानुजाभ्याम्}
{वैरूप्याच्छूर्पणख्याः प्रियविरहरुषारोपितभ्रूविजृम्भ}
{त्रस्ताब्धिर्बद्धसेतुः खलदवदहनः कोसलेन्द्रोऽवतान्नः} %4

\twolineshloka
{विश्वामित्राध्वरे येन मारीचाद्या निशाचराः}
{पश्यतो लक्ष्मणस्यैव हता नैरृतपुङ्गवाः} %5

\fourlineindentedshloka
{यो लोकवीरसमितौ धनुरैशमुग्रं}
{सीतास्वयंवरगृहे त्रिशतोपनीतम्}
{आदाय बालगजलील इवेक्षुयष्टिं}
{सज्ज्यीकृतं नृप विकृष्य बभञ्ज मध्ये} %6

\fourlineindentedshloka
{जित्वानुरूपगुणशीलवयोऽङ्गरूपां}
{सीताभिधां श्रियमुरस्यभिलब्धमानाम्}
{मार्गे व्रजन्भृगुपतेर्व्यनयत्प्ररूढं}
{दर्पं महीमकृत यस्त्रिरराजबीजाम्} %7

\fourlineindentedshloka
{यः सत्यपाशपरिवीतपितुर्निदेशं}
{स्त्रैणस्य चापि शिरसा जगृहे सभार्यः}
{राज्यं श्रियं प्रणयिनः सुहृदो निवासं}
{त्यक्त्वा ययौ वनमसूनिव मुक्तसङ्गः} %8

\fourlineindentedshloka
{रक्षःस्वसुर्व्यकृत रूपमशुद्धबुद्धेस्}
{तस्याः खरत्रिशिरदूषणमुख्यबन्धून्}
{जघ्ने चतुर्दशसहस्रमपारणीय}
{कोदण्डपाणिरटमान उवास कृच्छ्रम्} %9

\fourlineindentedshloka
{सीताकथाश्रवणदीपितहृच्छयेन}
{सृष्टं विलोक्य नृपते दशकन्धरेण}
{जघ्नेऽद्भुतैणवपुषाश्रमतोऽपकृष्टो}
{मारीचमाशु विशिखेन यथा कमुग्रः} %10

\fourlineindentedshloka
{रक्षोऽधमेन वृकवद्विपिनेऽसमक्षं}
{वैदेहराजदुहितर्यपयापितायाम्}
{भ्रात्रा वने कृपणवत्प्रियया वियुक्तः}
{स्त्रीसङ्गिनां गतिमिति प्रथयंश्चचार} %11

\fourlineindentedshloka
{दग्ध्वात्मकृत्यहतकृत्यमहन्कबन्धं}
{सख्यं विधाय कपिभिर्दयितागतिं तैः}
{बुद्ध्वाथ वालिनि हते प्लवगेन्द्रसैन्यैर्}
{वेलामगात्स मनुजोऽजभवार्चिताङ्घ्रिः} %12

\fourlineindentedshloka
{यद्रोषविभ्रमविवृत्तकटाक्षपात}
{सम्भ्रान्तनक्रमकरो भयगीर्णघोषः}
{सिन्धुः शिरस्यर्हणं परिगृह्य रूपी}
{पादारविन्दमुपगम्य बभाष एतत्} %13

\fourlineindentedshloka
{न त्वां वयं जडधियो नु विदाम भूमन्}
{कूटस्थमादिपुरुषं जगतामधीशम्}
{यत्सत्त्वतः सुरगणा रजसः प्रजेशा}
{मन्योश्च भूतपतयः स भवान्गुणेशः} %14

\fourlineindentedshloka
{कामं प्रयाहि जहि विश्रवसोऽवमेहं}
{त्रैलोक्यरावणमवाप्नुहि वीर पत्नीम्}
{बध्नीहि सेतुमिह ते यशसो वितत्यै}
{गायन्ति दिग्विजयिनो यमुपेत्य भूपाः} %15

\fourlineindentedshloka
{बद्ध्वोदधौ रघुपतिर्विविधाद्रिकूटैः}
{सेतुं कपीन्द्रकरकम्पितभूरुहाङ्गैः}
{सुग्रीवनीलहनुमत्प्रमुखैरनीकैर्}
{लङ्कां विभीषणदृशाविशदग्रदग्धाम्} %16

\fourlineindentedshloka
{सा वानरेन्द्रबलरुद्धविहारकोष्ठ}
{श्रीद्वारगोपुरसदोवलभीविटङ्का}
{निर्भज्यमानधिषणध्वजहेमकुम्भ}
{शृङ्गाटका गजकुलैर्ह्रदिनीव घूर्णा} %17

\fourlineindentedshloka
{रक्षःपतिस्तदवलोक्य निकुम्भकुम्भ}
{धूम्राक्षदुर्मुखसुरान्तकनरान्तकादीन्}
{पुत्रं प्रहस्तमतिकायविकम्पनादीन्}
{सर्वानुगान्समहिनोदथ कुम्भकर्णम्} %18

\fourlineindentedshloka
{तां यातुधानपृतनामसिशूलचाप}
{प्रासर्ष्टिशक्तिशरतोमरखड्गदुर्गाम्}
{सुग्रीवलक्ष्मणमरुत्सुतगन्धमाद}
{नीलाङ्गदर्क्षपनसादिभिरन्वितोऽगात्} %19

\fourlineindentedshloka
{तेऽनीकपा रघुपतेरभिपत्य सर्वे}
{द्वन्द्वं वरूथमिभपत्तिरथाश्वयोधैः}
{जघ्नुर्द्रुमैर्गिरिगदेषुभिरङ्गदाद्याः}
{सीताभिमर्षहतमङ्गलरावणेशान्} %20

\fourlineindentedshloka
{रक्षःपतिः स्वबलनष्टिमवेक्ष्य रुष्ट}
{आरुह्य यानकमथाभिससार रामम्}
{स्वःस्यन्दने द्युमति मातलिनोपनीते}
{विभ्राजमानमहनन्निशितैः क्षुरप्रैः} %21

\fourlineindentedshloka
{रामस्तमाह पुरुषादपुरीष यन्नः}
{कान्तासमक्षमसतापहृता श्ववत्ते}
{त्यक्तत्रपस्य फलमद्य जुगुप्सितस्य}
{यच्छामि काल इव कर्तुरलङ्घ्यवीर्यः} %22

\fourlineindentedshloka
{एवं क्षिपन्धनुषि सन्धितमुत्ससर्ज}
{बाणं स वज्रमिव तद्धृदयं बिभेद}
{सोऽसृग्वमन्दशमुखैर्न्यपतद्विमानाद्}
{धाहेति जल्पति जने सुकृतीव रिक्तः} %23

\twolineshloka
{ततो निष्क्रम्य लङ्काया यातुधान्यः सहस्रशः}
{मन्दोदर्या समं तत्र प्ररुदन्त्य उपाद्रवन्} %24

\twolineshloka
{स्वान्स्वान्बन्धून्परिष्वज्य लक्ष्मणेषुभिरर्दितान्}
{रुरुदुः सुस्वरं दीना घ्नन्त्य आत्मानमात्मना} %25

\twolineshloka
{हा हताः स्म वयं नाथ लोकरावण रावण}
{कं यायाच्छरणं लङ्का त्वद्विहीना परार्दिता} %26

\twolineshloka
{न वै वेद महाभाग भवान्कामवशं गतः}
{तेजोऽनुभावं सीताया येन नीतो दशामिमाम्} %27

\twolineshloka
{कृतैषा विधवा लङ्का वयं च कुलनन्दन}
{देहः कृतोऽन्नं गृध्राणामात्मा नरकहेतवे} %28

\uvacha{श्रीशुक उवाच}

\twolineshloka
{स्वानां विभीषणश्चक्रे कोसलेन्द्रानुमोदितः}
{पितृमेधविधानेन यदुक्तं साम्परायिकम्} %29

\twolineshloka
{ततो ददर्श भगवानशोकवनिकाश्रमे}
{क्षामां स्वविरहव्याधिं शिंशपामूलमाश्रिताम्} %30

\twolineshloka
{रामः प्रियतमां भार्यां दीनां वीक्ष्यान्वकम्पत}
{आत्मसन्दर्शनाह्लाद विकसन्मुखपङ्कजाम्} %31

\twolineshloka
{आरोप्यारुरुहे यानं भ्रातृभ्यां हनुमद्युतः}
{विभीषणाय भगवान्दत्त्वा रक्षोगणेशताम्} %32

\twolineshloka
{लङ्कामायुश्च कल्पान्तं ययौ चीर्णव्रतः पुरीम्}
{अवकीर्यमाणः सुकुसुमैर्लोकपालार्पितैः पथि} %33

\twolineshloka
{उपगीयमानचरितः शतधृत्यादिभिर्मुदा}
{गोमूत्रयावकं श्रुत्वा भ्रातरं वल्कलाम्बरम्} %34

\twolineshloka
{महाकारुणिकोऽतप्यज्जटिलं स्थण्डिलेशयम्}
{भरतः प्राप्तमाकर्ण्य पौरामात्यपुरोहितैः} %35

\twolineshloka
{पादुके शिरसि न्यस्य रामं प्रत्युद्यतोऽग्रजम्}
{नन्दिग्रामात्स्वशिबिराद्गीतवादित्रनिःस्वनैः} %36

\twolineshloka
{ब्रह्मघोषेण च मुहुः पठद्भिर्ब्रह्मवादिभिः}
{स्वर्णकक्षपताकाभिर्हैमैश्चित्रध्वजै रथैः} %37

\twolineshloka
{सदश्वै रुक्मसन्नाहैर्भटैः पुरटवर्मभिः}
{श्रेणीभिर्वारमुख्याभिर्भृत्यैश्चैव पदानुगैः} %38

\twolineshloka
{पारमेष्ठ्यान्युपादाय पण्यान्युच्चावचानि च}
{पादयोर्न्यपतत्प्रेम्णा प्रक्लिन्नहृदयेक्षणः} %39

\twolineshloka
{पादुके न्यस्य पुरतः प्राञ्जलिर्बाष्पलोचनः}
{तमाश्लिष्य चिरं दोर्भ्यां स्नापयन्नेत्रजैर्जलैः} %40

\twolineshloka
{रामो लक्ष्मणसीताभ्यां विप्रेभ्यो येऽर्हसत्तमाः}
{तेभ्यः स्वयं नमश्चक्रे प्रजाभिश्च नमस्कृतः} %41

\twolineshloka
{धुन्वन्त उत्तरासङ्गान्पतिं वीक्ष्य चिरागतम्}
{उत्तराः कोसला माल्यैः किरन्तो ननृतुर्मुदा} %42

\twolineshloka
{पादुके भरतोऽगृह्णाच्चामरव्यजनोत्तमे}
{विभीषणः ससुग्रीवः श्वेतच्छत्रं मरुत्सुतः} %43

\twolineshloka
{धनुर्निषङ्गान्छत्रुघ्नः सीता तीर्थकमण्डलुम्}
{अबिभ्रदङ्गदः खड्गं हैमं चर्मर्क्षराण्नृप} %44

\twolineshloka
{पुष्पकस्थो नुतः स्त्रीभिः स्तूयमानश्च वन्दिभिः}
{विरेजे भगवान्राजन्ग्रहैश्चन्द्र इवोदितः} %45

\twolineshloka
{भ्रात्राभिनन्दितः सोऽथ सोत्सवां प्राविशत्पुरीम्}
{प्रविश्य राजभवनं गुरुपत्नीः स्वमातरम्} %46

\twolineshloka
{गुरून्वयस्यावरजान्पूजितः प्रत्यपूजयत्}
{वैदेही लक्ष्मणश्चैव यथावत्समुपेयतुः} %47

\twolineshloka
{पुत्रान्स्वमातरस्तास्तु प्राणांस्तन्व इवोत्थिताः}
{आरोप्याङ्केऽभिषिञ्चन्त्यो बाष्पौघैर्विजहुः शुचः} %48

\twolineshloka
{जटा निर्मुच्य विधिवत्कुलवृद्धैः समं गुरुः}
{अभ्यषिञ्चद्यथैवेन्द्रं चतुःसिन्धुजलादिभिः} %49

\twolineshloka
{एवं कृतशिरःस्नानः सुवासाः स्रग्व्यलङ्कृतः}
{स्वलङ्कृतैः सुवासोभिर्भ्रातृभिर्भार्यया बभौ} %50

\threelineshloka
{अग्रहीदासनं भ्रात्रा प्रणिपत्य प्रसादित}
{प्रजाः स्वधर्मनिरता वर्णाश्रमगुणान्विता}
{जुगोप पितृवद्रामो मेनिरे पितरं च तम्} %51

\twolineshloka
{त्रेतायां वर्तमानायां कालः कृतसमोऽभवत्}
{रामे राजनि धर्मज्ञे सर्वभूतसुखावहे} %52

\twolineshloka
{वनानि नद्यो गिरयो वर्षाणि द्वीपसिन्धवः}
{सर्वे कामदुघा आसन्प्रजानां भरतर्षभ} %53

\twolineshloka
{नाधिव्याधिजराग्लानि दुःखशोकभयक्लमाः}
{मृत्युश्चानिच्छतां नासीद्रामे राजन्यधोक्षजे} %54

\twolineshloka
{एकपत्नीव्रतधरो राजर्षिचरितः शुचिः}
{स्वधर्मं गृहमेधीयं शिक्षयन्स्वयमाचरत्} %55

\twolineshloka
{प्रेम्णाऽनुवृत्त्या शीलेन प्रश्रयावनता सती}
{भिया ह्रिया च भावज्ञा भर्तुः सीताऽहरन्मनः} %56

॥इति श्रीमद्भागवते महापुराणे पारमहंस्यां संहितायां नवमस्कन्धे रामचरिते दशमोऽध्यायः॥

\closesection
    \sect{एकादशोऽध्यायः --- श्रीरामोपाख्यानम्}

\src{श्रीमद्-भागवतम्}{नवमः स्कन्धः}{अध्यायः ११}{श्लोकाः १---३६}
\vakta{शुकः}
\shrota{परीक्षितः}
\tags{concise, complete}
\notes{This chapter summarises how Lord Rāmachandra exemplified supreme dharma through performing extensive yajñas, generous gifts to brāhmaṇas, deep love for His subjects, and painful renunciation of Sītādevī , ultimately culminating in His departure to Vaikuntham after establishing His sons in the kingdom.}
\textlink{http://stotrasamhita.net/wiki/Bhagavatam/Skandha_09/Adhyaya_11}
\translink{}

\storymeta

\uvacha{श्रीशुक उवाच}

\twolineshloka
{भगवानात्मनात्मानं राम उत्तमकल्पकैः}
{सर्वदेवमयं देवमीजेऽथाचार्यवान्मखैः} %1

\twolineshloka
{होत्रेऽददाद्दिशं प्राचीं ब्रह्मणे दक्षिणां प्रभुः}
{अध्वर्यवे प्रतीचीं वा उत्तरां सामगाय सः} %2

\twolineshloka
{आचार्याय ददौ शेषां यावती भूस्तदन्तरा}
{अन्यमान इदं कृत्स्नं ब्राह्मणोऽर्हति निःस्पृहः} %3

\twolineshloka
{इत्ययं तदलङ्कार वासोभ्यामवशेषितः}
{तथा राज्ञ्यपि वैदेही सौमङ्गल्यावशेषिता} %4

\twolineshloka
{ते तु ब्राह्मणदेवस्य वात्सल्यं वीक्ष्य संस्तुतम्}
{प्रीताः क्लिन्नधियस्तस्मै प्रत्यर्प्येदं बभाषिरे} %5

\twolineshloka
{अप्रत्तं नस्त्वया किं नु भगवन्भुवनेश्वर}
{यन्नोऽन्तर्हृदयं विश्य तमो हंसि स्वरोचिषा} %6

\twolineshloka
{नमो ब्रह्मण्यदेवाय रामायाकुण्ठमेधसे}
{उत्तमश्लोकधुर्याय न्यस्तदण्डार्पिताङ्घ्रये} %7

\twolineshloka
{कदाचिल्लोकजिज्ञासुर्गूढो रात्र्यामलक्षितः}
{चरन्वाचोऽशृणोद्रामो भार्यामुद्दिश्य कस्यचित्} %8

\twolineshloka
{नाहं बिभर्मि त्वां दुष्टामसतीं परवेश्मगाम्}
{स्त्रैणो हि बिभृयात्सीतां रामो नाहं भजे पुनः} %9

\twolineshloka
{इति लोकाद्बहुमुखाद्दुराराध्यादसंविदः}
{पत्या भीतेन सा त्यक्ता प्राप्ता प्राचेतसाश्रमम्} %10

\twolineshloka
{अन्तर्वत्न्यागते काले यमौ सा सुषुवे सुतौ}
{कुशो लव इति ख्यातौ तयोश्चक्रे क्रिया मुनिः} %11

\twolineshloka
{अङ्गदश्चित्रकेतुश्च लक्ष्मणस्यात्मजौ स्मृतौ}
{तक्षः पुष्कल इत्यास्तां भरतस्य महीपते} %12

\twolineshloka
{सुबाहुः श्रुतसेनश्च शत्रुघ्नस्य बभूवतुः}
{गन्धर्वान्कोटिशो जघ्ने भरतो विजये दिशाम्} %13

\threelineshloka
{तदीयं धनमानीय सर्वं राज्ञे न्यवेदय}
{शत्रुघ्नश्च मधोः पुत्रं लवणं नाम राक्षस}
{हत्वा मधुवने चक्रे मथुरां नाम वै पुरीम्॥१४} %14

\twolineshloka
{मुनौ निक्षिप्य तनयौ सीता भर्त्रा विवासिता}
{ध्यायन्ती रामचरणौ विवरं प्रविवेश ह} %15

\twolineshloka
{तच्छ्रुत्वा भगवान्रामो रुन्धन्नपि धिया शुचः}
{स्मरंस्तस्या गुणांस्तांस्तान्नाशक्नोद्रोद्धुमीश्वरः} %16

\twolineshloka
{स्त्रीपुम्प्रसङ्ग एतादृक्सर्वत्र त्रासमावहः}
{अपीश्वराणां किमुत ग्राम्यस्य गृहचेतसः} %17

\twolineshloka
{तत ऊर्ध्वं ब्रह्मचर्यं धार्यन्नजुहोत्प्रभुः}
{त्रयोदशाब्दसाहस्रमग्निहोत्रमखण्डितम्} %18

\twolineshloka
{स्मरतां हृदि विन्यस्य विद्धं दण्डककण्टकैः}
{स्वपादपल्लवं राम आत्मज्योतिरगात्ततः} %19

\fourlineindentedshloka
{नेदं यशो रघुपतेः सुरयाच्ञयात्त}
{लीलातनोरधिकसाम्यविमुक्तधाम्नः}
{रक्षोवधो जलधिबन्धनमस्त्रपूगैः}
{किं तस्य शत्रुहनने कपयः सहायाः} %20

\fourlineindentedshloka
{यस्यामलं नृपसदःसु यशोऽधुनापि}
{गायन्त्यघघ्नमृषयो दिगिभेन्द्रपट्टम्}
{तं नाकपालवसुपालकिरीटजुष्ट}
{पादाम्बुजं रघुपतिं शरणं प्रपद्ये} %21

\twolineshloka
{स यैः स्पृष्टोऽभिदृष्टो वा संविष्टोऽनुगतोऽपि वा}
{कोसलास्ते ययुः स्थानं यत्र गच्छन्ति योगिनः} %22

\twolineshloka
{पुरुषो रामचरितं श्रवणैरुपधारयन्}
{आनृशंस्यपरो राजन्कर्मबन्धैर्विमुच्यते} %23

\uvacha{श्रीराजोवाच}


\twolineshloka
{कथं स भगवान्रामो भ्रात्न्वा स्वयमात्मनः}
{तस्मिन्वा तेऽन्ववर्तन्त प्रजाः पौराश्च ईश्वरे} %24

\uvacha{श्रीबादरायणिरुवाच}


\twolineshloka
{अथादिशद्दिग्विजये भ्रात्ंस्त्रिभुवनेश्वरः}
{आत्मानं दर्शयन्स्वानां पुरीमैक्षत सानुगः} %25

\twolineshloka
{आसिक्तमार्गां गन्धोदैः करिणां मदशीकरैः}
{स्वामिनं प्राप्तमालोक्य मत्तां वा सुतरामिव} %26

\twolineshloka
{प्रासादगोपुरसभा चैत्यदेवगृहादिषु}
{विन्यस्तहेमकलशैः पताकाभिश्च मण्डिताम्} %27

\twolineshloka
{पूगैः सवृन्तै रम्भाभिः पट्टिकाभिः सुवाससाम्}
{आदर्शैरंशुकैः स्रग्भिः कृतकौतुकतोरणाम्} %28

\twolineshloka
{तमुपेयुस्तत्र तत्र पौरा अर्हणपाणयः}
{आशिषो युयुजुर्देव पाहीमां प्राक्त्वयोद्धृताम्} %29

\fourlineindentedshloka
{ततः प्रजा वीक्ष्य पतिं चिरागतं}
{दिदृक्षयोत्सृष्टगृहाः स्त्रियो नराः}
{आरुह्य हर्म्याण्यरविन्दलोचनम्}
{अतृप्तनेत्राः कुसुमैरवाकिरन्} %30

\twolineshloka
{अथ प्रविष्टः स्वगृहं जुष्टं स्वैः पूर्वराजभिः}
{अनन्ताखिलकोषाढ्यमनर्घ्योरुपरिच्छदम्} %31

\twolineshloka
{विद्रुमोदुम्बरद्वारैर्वैदूर्यस्तम्भपङ्क्तिभिः}
{स्थलैर्मारकतैः स्वच्छैर्भ्राजत्स्फटिकभित्तिभिः} %32

\twolineshloka
{चित्रस्रग्भिः पट्टिकाभिर्वासोमणिगणांशुकैः}
{मुक्ताफलैश्चिदुल्लासैः कान्तकामोपपत्तिभिः} %33

\twolineshloka
{धूपदीपैः सुरभिभिर्मण्डितं पुष्पमण्डनैः}
{स्त्रीपुम्भिः सुरसङ्काशैर्जुष्टं भूषणभूषणैः} %34

\twolineshloka
{तस्मिन्स भगवान्रामः स्निग्धया प्रिययेष्टया}
{रेमे स्वारामधीराणामृषभः सीतया किल} %35

\twolineshloka
{बुभुजे च यथाकालं कामान्धर्ममपीडयन्}
{वर्षपूगान्बहून्नॄणामभिध्याताङ्घ्रिपल्लवः} % 36


॥इति श्रीमद्भागवते महापुराणे पारमहंस्यां संहितायां नवमस्कन्धे श्रीरामोपाख्याने एकादशोऽध्यायः॥


\closesection
    \chapt{श्रीमन्नारायणीयम्}

\sect{दशकं ३४ --- श्रीरामचरितवर्णनम्}

\src{श्रीमन्नारायणीयम्}{चतुस्त्रिंश-दशकं}{}{श्लोकाः १---10}
\vakta{शुकः}
\shrota{परीक्षितः}
\tags{concise, complete}
\notes{This chapter describes the events following the birth of Rāma, including His early life in Ayodhyā, the exile of Rāma, Sītā, and Lakṣmaṇa to the forest, their encounters with various Rishis and Rakshasas, and the abduction of Sita.}
\textlink{http://stotrasamhita.net/wiki/Narayaniyam/Dashaka_34}
\translink{}

\storymeta

\fourlineindentedshloka
{गीर्वाणैरर्थ्यमानो दशमुखनिधनं कोसलेऽष्वृश्यषृङ्गे}
{पुत्रीयामिष्टिमिष्ट्वा ददुषि दशरथक्ष्माभृते पायसाग्र्यम्}
{तद्भुक्त्या तत्पुरन्ध्रीष्वपि तिसृषु समं जातगर्भासु जातो}
{रामस्त्वं लक्ष्मणेन स्वयमथ भरतेनापि शत्रुघ्ननाम्ना} % ॥१॥

\fourlineindentedshloka
{कोदण्डी कौशिकस्य क्रतुवरमवितुं लक्ष्मणेनानुयातो}
{यातोऽभूस्तातवाचा मुनिकथितमनुद्वन्द्वशान्ताध्वखेदः}
{नॄणां त्राणाय बाणैर्मुनिवचनबलात्ताटकां पाटयित्वा}
{लब्ध्वास्मादस्त्रजालं मुनिवनमगमो देव सिद्धाश्रमाख्यम्} % ॥२॥

\fourlineindentedshloka
{मारीचं द्रावयित्वा मखशिरसि शरैरन्यरक्षांसि निघ्नन्}
{कल्यां कुर्वन्नहल्यां पथि पदरजसा प्राप्य वैदेहगेहम्}
{भिन्दानश्चान्द्रचूडं धनुरवनिसुतामिन्दिरामेव लब्ध्वा}
{राज्यं प्रातिष्ठथास्त्वं त्रिभिरपि च समं भ्रातृवीरैः सदारैः} % ॥३॥

\fourlineindentedshloka
{आरुन्धाने रुषान्धे भृगुकुलतिलके सङ्क्रमय्य स्वतेजो}
{याते यातोऽस्ययोध्यां सुखमिह निवसन्कान्तया कान्तमूर्ते}
{शत्रुघ्नेनैकदाथो गतवति भरते मातुलस्याधिवासम्}
{तातारब्धोऽभिषेकस्तव किल विहतः केकयाधीशपुत्र्या} % ॥४॥

\fourlineindentedshloka
{तातोक्त्या यातुकामो वनमनुजवधूसंयुतश्चापधारः}
{पौरानारूध्य मार्गे गुहनिलयगतस्त्वं जटाचीरधारी}
{नावा सन्तीर्य गङ्गामधिपदवि पुनस्तं भरद्वाजमारा-}
{न्नत्वा तद्वाक्यहेतोरतिसुखमवसश्चित्रकूटे गिरीन्द्रे} % ॥५॥

\fourlineindentedshloka
{श्रुत्वा पुत्रार्तिखिन्नं खलु भरतमुखात् स्वर्गयातं स्वतातम्}
{तप्तो दत्त्वाम्बु तस्मै निदधिथ भरते पादुकां मेदिनीं च}
{अत्रिं नत्वाथ गत्वा वनमतिविपुलां दण्डकां चण्डकायम्}
{हत्वा दैत्यं विराधं सुगतिमकलयश्चारु भोः शारभङ्गीम्} % ॥६॥

\fourlineindentedshloka
{नत्वाऽगस्त्यं समस्ताशरनिकरसपत्राकृतिं तापसेभ्यः}
{प्रत्यश्रौषीः प्रियैषी तदनु च मुनिना वैष्णवे दिव्यचापे}
{ब्रह्मास्त्रे चापि दत्ते पथि पितृसुहृदं दीक्ष्य जटायुम्}
{मोदाद्गोदातटान्ते परिरमसि पुरा पञ्चवट्यां वधूट्या} % ॥७॥

\fourlineindentedshloka
{प्राप्तायाः शूर्पणख्या मदनचलधृतेरर्थनैर्निस्सहात्मा}
{तां सौमित्रौ विसृज्य प्रबलतमरुषा तेन निर्लुननासाम्}
{दृष्ट्वैनां रुष्टचित्तं खरमभिपतितं दुषणं च त्रिमूर्धम्}
{व्याहिंसीराशरानप्ययुतसमधिकांस्तत्क्षणादक्षतोष्मा} % ॥८॥

\fourlineindentedshloka
{सोदर्याप्रोक्तवार्ताविवशदशमुखादिष्टमारीचमाया-}
{सारङ्गं सारसाक्ष्या स्पृहितमनुगतः प्रावधीर्बाणघातम्}
{तन्मायाक्रन्दनिर्यापितभवदनुजां रावणस्तामहार्षीत्}
{तेनार्तोऽपि त्वमन्तः किमपि मुदमधास्तद्वधोपायायलाभात्} % ॥९॥

\fourlineindentedshloka
{भूयस्तन्वीं विचिन्वन्नहृत दशमुखस्त्वद्वधूं मद्वधेने-}
{त्युक्त्वा याते जटायौ दिवमथ सुहृदः प्रातनोः प्रेतकार्यम्}
{गृह्णानं तं कबन्धं जघनिथ शबरीं प्रेक्ष्य पम्पातटे त्वम्}
{सम्प्राप्तो वातसूनुं भृशमुदितमनाः पाहि वातालयेश} % ॥१०॥

॥इति श्रीमन्नारायणीये श्रीरामचरितवर्णनं नाम चतुस्त्रिंश-दशकं सम्पूर्णम्॥

\closesection
    \sect{श्रीरामचरितवर्णनम् - २}

\src{श्रीमन्नारायणीयम्}{पञ्चत्रिंश-दशकं}{}{श्लोकाः १--१०}
\vakta{शुकः}
\shrota{परीक्षितः}
\tags{concise, complete}
\notes{This chapter summarises the events following the death of Vāli, including the alliance with Sugrīva, the search for Sītā, Setubandhanam, the eventual victory over Rāvaṇa, and the establishment of Rāma's rule in Ayodhyā.}
\textlink{http://stotrasamhita.net/wiki/Narayaniyam/Dashaka_35}
\translink{}

\storymeta


\fourlineindentedshloka
{नीतस्सुग्रीवमैत्रीं तदनु दुन्दुभेः कायमुच्चैः}
{क्षिप्त्वाङ्गुष्ठेन भूयो लुलविथ युगपत्पत्रिणा सप्त सालान्}
{हत्वा सुग्रीवघातोद्यतमतुलबलं वालिनं व्याजवृत्त्या}
{वर्षावेलामनैषीर्विरहतरळितस्त्वं मतङ्गाश्रमान्ते} %॥१॥

\fourlineindentedshloka
{सुग्रीवेणानुजोक्त्या सभयमभियता व्यूहितां वाहिनीं ता-}
{मृक्षाणां वीक्ष्य दिक्षु द्रुतमथ दयितामार्गणायावनम्राम्}
{सन्देशं चान्गुलीयं पवनसुतकरे प्रादिशो मोदशाली}
{मार्गे मार्गे ममार्गे कपिभिरपि तदी त्वत्प्रिया सप्रयासैः} %॥२॥

\fourlineindentedshloka
{त्वद्वार्ताकर्णनोद्यद्गरुदुरुजवसम्पातिसम्पातिवाक्य-}
{प्रोत्तीर्णार्णोधिरन्तर्नगरि जनकजां वीक्ष्य दत्त्वाऽङ्गुलीयम्}
{प्रक्षुद्योद्यानमक्षक्षपणचणरणः सोढबन्धो दशास्यम्}
{दृष्ट्वा प्लुष्ट्वा च लङ्कां झटिति स हनुमान्मौलिरत्नं ददौ ते} %॥३॥

\fourlineindentedshloka
{त्वं सुग्रीवाङ्गदादिप्रबलकपिचमूचक्रविक्रान्तभूमी-}
{चक्रोऽभिक्रम्य पारेजलधि निशिचरेन्द्रानुजाश्रीयमाणः}
{तत्प्रोक्तां शत्रुवार्तां रहसि निशमयन्प्रार्थनापार्थ्यरोष-}
{प्रास्ताग्नेयास्त्रतेजस्त्रसदुदधिगिरा लब्धवान्मध्यमार्गम्} %॥४॥

\fourlineindentedshloka
{कीशैराशान्तरोपाहृतगिरिनिकरैः सेतुमाधाप्य यातो}
{यातून्यामर्द्य दंष्ट्रानखशिखरिशिलासालशस्त्रैः स्वसैन्यैः}
{व्याकुर्वन्सानुजस्त्वं समरभुवि परं विक्रमं शक्रजेत्रा}
{वेगान्नागास्त्रबद्धः पतगपतिगरुन्मारुतैर्मोचितोऽभूः} %॥५॥

\fourlineindentedshloka
{सौमित्रिस्त्वत्र शक्तिप्रहृतिगळदसुर्वातजानीतशैल-}
{घ्राणात्प्रणानुपेतो व्यकृणुत कुसृतिश्लाघिनं मेघनादम्}
{मायाक्षोभेषु वैभीषणवचनहृतस्तम्भनः कुम्भकर्णम्}
{सम्प्राप्तं कम्पितोर्वीतलमखिलचमूभक्षिणं व्यक्षिणोस्त्वम्} %॥६॥

\fourlineindentedshloka
{गृह्णन् जम्भारिसम्प्रेषितरथकवचौ रावणेनाभियुध्यन्}
{ब्रह्मास्त्रेणास्य भिन्दन् गळततिमबलामग्निशुद्धां प्रगृह्णन्}
{देव श्रेणीवरोज्जीवितसमरमृतैरक्षतैऱ्क्षसङ्घैर्-}
{लङ्काभर्त्रा च साकं निजनगरमगाः सप्रियः पुष्पकेण} %॥७॥

\fourlineindentedshloka
{प्रीतो दिव्याभिषेकैरयुतसमधिकान्वत्सरान्पर्यरंसी-}
{र्मैथिल्यां पापवाचा शिव शिव किल तां गर्भिणीमभ्यहासीः}
{शत्रुघ्नेनार्दयित्वा लवणनिशिचरं प्रार्दयः शूद्रपाशम्}
{तावद्वाल्मीकिगेहे कृतवसतिरुपासूत सीता सुतौ ते} %॥८॥

\fourlineindentedshloka
{वाल्मीकेस्त्वत्सुतोद्गापितमधुरकृतेराज्ञया यज्ञवाटे}
{सीतां त्वय्याप्तुकामे क्षितिमविशदसौ त्वं च कालार्थितोऽभूः}
{हेतोः सौमित्रिघाती स्वयमथ सरयूमग्ननिश्शेषभृत्यैः}
{साकं नाकं प्रयातो निजपदमगमो देव वैकुण्ठमाद्यम्} %॥९॥

\fourlineindentedshloka
{सोऽयं मर्त्यावतारस्तव खलु नियतं मर्त्यशिक्षार्थमेवम्}
{विश्लेषार्तिर्निरागस्त्यजनमपि भवेत्कामधर्मातिसक्त्या}
{नो चेत्स्वात्मानुभूतेः क्वनु तव मनसो विक्रिया चक्रपाणे}
{स त्वं सत्त्वैकमूर्ते पवनपुरपते व्याधुनु व्याधितापान्} %॥१०॥

॥इति श्रीमन्नारायणीये श्रीरामचरितवर्णनं नाम पञ्चत्रिंश-दशकं सम्पूर्णम्॥

\closesection
    \chapt{भीष्मेण रामावतारकथनम्}

\src{श्रीमन्महाभारतम्}{सभा-पर्व}{अर्घाहरणपर्व}{अध्यायः ५०}
\vakta{भीष्मः}
\shrota{युधिष्ठिरादयः}
\tags{concise, complete}
\notes{Bhishma narrates the story of Rama, while recounting the greatness of Vishnu. He goes on to then talk about Krishna, and subsequently, Krishnavatara too.}
% \textlink{http://stotrasamhita.net/wiki/Narayaniyam/Dashaka_34}
\translink{}

\storymeta

\sect{अध्यायः ५०}

\uvacha{भीष्म उवाच}

\twolineshloka
{शृणु राजंस्ततो विष्णोः प्रादुर्भावं महात्मनः}
{अष्टाविंशे युगे चापि मार्कण्डेयपुरः सरः}


\twolineshloka
{तिथौ नावमिके जज्ञे तथा दशरथादपि}
{कृत्वाऽऽत्मानं महाबाहुश्चतुर्धा विष्णुरव्ययः}


\twolineshloka
{लोके राम इति ख्यातस्तेजसा भास्करोपमः}
{प्रसादनार्थं लोकस्य विष्णुस्तत्र सनातनः}


\twolineshloka
{धर्मार्थमेव कौन्तेय जज्ञे तत्र महायशाः}
{तमप्याहुर्मनुष्येन्द्रं सर्वभूतपतेस्तनुम्}


\twolineshloka
{यज्ञविघ्नकरस्तत्र विश्वामित्रस्य भारत}
{सुबाहुर्निहतस्तेन मारीचस्ताडितो भृशम्}


\twolineshloka
{तस्मै दत्तानि चास्राणि विश्वमित्रेण धीमता}
{वधार्थं सर्वशत्रूणां दुर्वाराणि सुरैरपि}


\twolineshloka
{वर्तमाने महायज्ञे जनकस्य महात्मनः}
{भग्नं माहेश्वरं चापं क्रीडता लीलया भृशम्}


\twolineshloka
{ततस्तु सीतां जग्राह भार्यार्थे जानकीं विभुः}
{नगरीं पुनरासाद्य मुमुदे तत्र सीतया}


\twolineshloka
{कस्यचित्त्वथ कालस्य पित्रा तत्राभिचोदितः}
{कैकेय्याः प्रियमन्विच्छन्वनमभ्यवपद्यत}


\twolineshloka
{यः समाः सर्वधर्मज्ञश्चतुर्दश वने वसन्}
{लक्ष्मणानुचरो रामः सर्वभूतहिते रतः}


\twolineshloka
{चतुर्दश वने तीर्त्वा तदा वर्षाणि भारत}
{रूपिणी यस्य पार्श्वस्था सीतेत्यभिहिता जनैः}


\twolineshloka
{पूर्वोचितत्वात्सा लक्ष्मीर्भर्तारमनुशोचति}
{जनस्थाने वसन्कार्यं त्रिदशानां चकार सः}


\twolineshloka
{मारीचं दूषणं हत्वा खरं त्रिशिरसं तथा}
{चतुर्दश सहस्राणि रक्षसां घोरकर्मणाम्}


\twolineshloka
{जघान रामो धर्मात्मा प्रजानां हितकाम्यया}
{विराधं च कबन्धं च राक्षसौ घोरकर्मिणौ}


\twolineshloka
{जघान च तदा रामो गन्धर्वौ शापविक्षतौ}
{स रावणस्य भगिनी नासाच्छेदमकारयत्}


\twolineshloka
{भार्यावियोगं तं प्राप्य मृगयन्व्यचरद्वनम्}
{स तस्मादृश्यमूकं तु गत्वा पम्पामतीत्य च}


\twolineshloka
{सुग्रीवं मारुतिं दृष्ट्वा चक्रे मैत्रीं तयोः स वै}
{अथ गत्वा स किष्किन्धां सुग्रीवेण तदा सह}


\twolineshloka
{निहत्य वालिनं युद्धे वानरेन्द्रं महाबलम्}
{अभ्यषिञ्चत्तदा रामः सुग्रीवं वानरेश्वरम्}


\twolineshloka
{ततः स वीर्यवान्राजंस्त्वरया वै समुत्सुकः}
{विचित्य वायुपुत्रेण लङ्कादेशं निवेदितः}


\twolineshloka
{सेतुं बद्ध्वा समुद्रस्य वानरैः स समुत्सुकः}
{सीतायाः पदमन्विच्छन्रामो लङ्कां विवेश वै}


\twolineshloka
{देवोरगगणानां हि यक्षराक्षसपक्षिणाम्}
{तत्रावद्यं राक्षसेन्द्रं रावणं युधि दुर्जयम्}


\twolineshloka
{युक्तं राक्षसकोटीभिर्भिन्नाञ्जनचयोपमम्}
{दुर्निरीक्ष्यं सुरगणैर्वरदानेन दर्पितम्}


\onelineshloka
{जघान सचिवैः सार्धं सान्वयं रावणं रणे}


\twolineshloka
{त्रैलोक्यकण्टकं वीरं महाकायं महाबलम्}
{रावणं सगणं हत्वा रामो भूतपतिः पुरा}


\twolineshloka
{लङ्कायां तं महात्मानं राक्षसेन्द्रं विभीषणम्}
{अभिषिच्य ततो राम अमरत्वं ददौ तदा}


\twolineshloka
{आरुह्य पुष्पकं रामः सीतामादाय पाण्डव}
{सबलं स्वपुरं गत्वा धर्मराज्यमपालयत्}


\twolineshloka
{दानवो लवणो नाम मधोः पुत्रो महाबलः}
{शत्रुघ्नेन हतो राजंस्तदा रामस्य शासनात्}


\twolineshloka
{एवं बहूनि कर्माणि कृत्वा लोकहिताय सः}
{राजं चकार विधिवद्रामो धर्मभृतां वरः}


\twolineshloka
{शताश्वमेधानाजह्रे ज्योतिरुक्थ्यान्निरर्गलान्}
{नाश्रूयन्ताशुभा वाचो नात्ययः प्राणिनां तदा}


\twolineshloka
{न दस्युजं भयं चासीद्रामे राज्यं प्रशसति}
{ऋषीणां देवतानां च मनुष्याणां तथैव च}


\twolineshloka
{पृथिव्यां धार्मिकाः सर्वे रामे राज्यं प्रशासति}
{नाधर्मिष्ठो नरः कश्चिद्बभूव प्राणिनां क्वचित्}


\twolineshloka
{प्राणापानौ समौ ह्यास्तां रामे राज्यं प्रशासति}
{गाधामप्यत्र गायन्ति ये पुराणविदो जनाः}


\twolineshloka
{श्यामो युवा लोहिताक्षो मातङ्गानामिवर्षभः}
{आजानुबाहुः सुमुखः सिंहस्कन्धो महाबलः}


\twolineshloka
{दशवर्षसहस्राणि दशवर्षशतानि च}
{राज्यं भोगं च सम्प्राप्य शशास पृथिवीमिमाम्}


\twolineshloka
{रामो रामो राम इति प्राजानामभवन्कथाः}
{रामभूतं जगदिदं रामे राज्यं प्रशासति}


\twolineshloka
{ऋग्यजुः सामहीनाश्च न तदाऽसन्द्विजायः}
{उषित्वा दण्डके कार्यं त्रिदशानां चकार सः}


\twolineshloka
{पूर्वापकारिणं तं तु पौलस्त्यं मनुजर्षभम्}
{देवगन्धर्वनागानामरिं स निजघान ह}


\twolineshloka
{सत्ववान्गुणसम्पन्नो दीप्यमानः स्वतेजसा}
{एवमेव महाबाहुरिक्ष्वाकुकुलवर्धनः}


\twolineshloka
{रावणं सगणं हत्वा दिवमाक्रमताभिभूः}
{इति दाशरथेः ख्यातः प्रादुर्भावो महात्मनः}


\twolineshloka
{ततः कृष्णो महाबाहुर्भीतानामभयङ्करः}
{अष्टाविंशे युगे राजञ्जज्ञे श्रीवत्सलक्षणः}


\twolineshloka
{पेशलश्च वदान्यश्चलोके बहुमतो नृषु}
{स्मृतिमान्देशकालज्ञः शङ्खचक्रगदासिभृत्}


\twolineshloka
{वासुदेव इति ख्यातो लोकानां हितकृत्सदा}
{वृष्णीनां च कुले जातो भूमेः प्रियचिकीर्षया}


\twolineshloka
{शत्रूणां भयकृद्दाता मधुहेति स विश्रुतः}
{शकटार्जुनरामाणां कीलस्थानान्यसूदयत्}


\twolineshloka
{कंसादीन्निजघानाऽऽजौ दैत्यान्मानुषविग्रहान्}
{अयं लोकहितार्थाय प्रादुर्भावो महात्मनः}


\twolineshloka
{कल्की विष्णुयशा नाम भूयश्चोत्पत्स्यते हरिः}
{लेर्युगान्ते सम्प्राप्ते धर्मे शिथिलतां गते}


\twolineshloka
{पाषण्डिनां गणानां हि वधार्थं भरतर्षभ}
{धर्मस्य च विवृद्ध्यर्थं विप्राणां हितकाम्यया}


\twolineshloka
{एते चान्ये च बहवो विष्णोर्देवगणैर्युताः}
{प्रादुर्भावाः पुराणेषु गीयन्ते ब्रह्मवादिभिः}


॥इति श्रीमन्महाभारते सभापर्वणि अर्घाहरण-पर्वणि पञ्चाशोऽध्यायः॥६०॥

\closesection
    \sect{रामोपाख्यान-पर्व}

\src{श्रीमन्महाभारतम्}{वन-पर्व}{श्रीरामोपाख्यानपर्व}{अध्यायाः २७५--२९३}
\vakta{मार्कण्डेयः}
\shrota{युधिष्ठिरः}
\tags{concise, complete}
\notes{From this chapter begins the detailed account of Rama, as narrated by by Rishi Markandeya to Yudhishthira in the ``Ramopakhyana parva'', spanning 19 chapters and 750+ shlokas in the Vanaparva of the Mahabharata. It narrates the birth of Rama, his early life, exile, encounters with various beings, and the ultimate victory over Ravana.}
% \textlink{http://stotrasamhita.net/wiki/Narayaniyam/Dashaka_34}
\translink{}

\storymeta

\dnsub{अध्यायः २७५}\resetShloka

\uvacha{मार्कण्डेय उवाच}

\twolineshloka
{प्राप्तमप्रतिमं दुःखं रामेण भरतर्षभ}
{रक्षसा जानकी तस्य हृता भार्या बलीयसा}


\twolineshloka
{आश्रमाद्राक्षसेन्द्रेण रावणेन दुरात्मना}
{मायामास्थाय तरसा हत्वा गृध्रं जटायुषम्}


\twolineshloka
{प्रत्याजहार तां रामः सुग्रीवबलमाश्रितः}
{बद्ध्वा सेतुं समुद्रस्य दग्ध्वा लङ्कां शितैः शरैः}

\uvacha{युधिष्ठिर उवाच}


\twolineshloka
{कस्मिन् रामः कुले जातः किंवीर्यः किम्पराक्रमः}
{रावणः कस्य पुत्रो वा किं वैरं तस्य तेन ह}


\threelineshloka
{एतन्मे भगवन्सर्वं सम्यगाख्यातुमर्हसि}
{त्वया प्रत्यक्षतो दृष्टं यथासर्वमशेषतः}
{श्रोतुमिच्छामि चरितं रामस्याक्लिष्टकर्मणः}

\uvacha{मार्कण्डेय उवाच}


\twolineshloka
{अजो नामाभवद्राजा महानिक्ष्वाकुवंशजः}
{तस्य पुत्रो दशरथः शश्वत्स्वाध्यायवाञ्छुचिः}


\twolineshloka
{अभवंस्तस्य चत्वारः पुत्रा धर्मार्थकोविदाः}
{रामलक्ष्मणशत्रुघ्ना भरतश्च महाबलः}


\twolineshloka
{रामस्य माता कौसल्या कैकेयी भरतस्य तु}
{सुतौ लक्ष्मणशत्रुघ्नौ सुमित्रायाः परन्तपौ}


\twolineshloka
{विदेहराजो जनकः सीता तस्यात्मजा विमो}
{यां चकार स्वयं त्वष्टा रामस्य महिषीं प्रियाम्}


\twolineshloka
{एतद्रामस्य ते जन्म सीतायाश्च प्रकीर्तितम्}
{रावणस्यापि ते जन्म व्याख्यास्यामि जनेश्वर}


\twolineshloka
{पितामहो रावणस्य साक्षाद्देवः प्रजापतिः}
{स्वयभूः सर्वलोकानां प्रभुः स्रष्टा महातपाः}


\twolineshloka
{पुलस्त्यो नाम तस्यासीन्मानसो दयितः सुतः}
{तस्य वैश्रवणो नाम गवि पुत्रोऽभवत्प्रभुः}


\twolineshloka
{पितरं स समुत्सृज्य पितामहमुपस्थितः}
{तस्य कोपात्पिता राजन्ससर्जात्मानमात्मना}


\twolineshloka
{स जज्ञे विश्रवा नाम तस्यात्मार्धेन वै द्विजः}
{प्रतीकाराय सक्रोधस्ततो वैश्रवणस्य वै}


\twolineshloka
{पितामहस्तु प्रीतात्मा ददौ वैश्रवणस्य ह}
{अमरत्वं धनेशत्वं लोकपालत्वमेव च}


\twolineshloka
{ईशानन तथा सख्यं पुत्रं च नलकूवरम्}
{राजधानीनिवेसं च लङ्कां रक्षोगणान्विताम्}


\twolineshloka
{विमानं पुष्पकं नाम कामगं च ददौ प्रभुः}
{यक्षाणामाधिपत्यञ्च राजराजत्वमेव च}


॥इति श्रीमन्महाभारते अरण्यपर्वणि रामोपाख्यान-पर्वणि त्रिशततमोऽध्यायः॥२७५॥

\storymeta

\dnsub{अध्यायः २७६}\resetShloka
\uvacha{मार्कण्डेय उवाच}


\twolineshloka
{पुलस्त्यस्य तु यः क्रोधादर्धदेहोऽभवन्मुनिः}
{विश्रवानाम सक्रोधं पितरं राक्षसेश्वरः}


\twolineshloka
{बुबुधे तं तु सक्रोधं पितरं राक्षसेश्वरः}
{कुबेरस्तत्प्रसादार्थं यतते स्म सदा नृप}


\twolineshloka
{स राजराजो लङ्कायां न्यवसन्नरवाहनः}
{राक्षसीः प्रददौ तिस्रः पितुर्वै परिचारिकाः}


\twolineshloka
{ताः सदा तं महात्मानं सन्तोषयितुमुद्यताः}
{ऋषिं भरतशार्दूल नृत्यगीतविशारदाः}


\twolineshloka
{पुष्पोत्कटा च राका च मालिनी च विशापते}
{अन्योन्यस्पर्धयाराजञ्श्रेयस्कामाः सुमध्यमाः}


\twolineshloka
{स तासां भगवांस्तुष्टो महात्मा प्रददौ वरान्}
{लोकपालोपमान्पुत्रानकैकस्या यथेप्सितान्}


\twolineshloka
{पुष्पोत्कटायां जज्ञाते द्वौ पुत्रौ राक्षसेश्वरौ}
{कुम्भकर्णदशग्रीवौ बलेनाप्रतिमौ भुवि}


\twolineshloka
{मालिन जनयामास पुत्रमेकं विभीषणम्}
{राकार्या मिथुनं जज्ञे खरः शूर्पणखा तथा}


\twolineshloka
{विभीषणस्तु रूपेण सर्वेभ्योऽभ्यधिकोऽभवत्}
{स बभूव महाभागो धर्मगोप्ता क्रियारतिः}


\twolineshloka
{दशग्रीवस्तु सर्वेषां श्रेष्ठो राक्षसपुङ्गवः}
{महोत्साहो महावीर्यो महासत्वपराक्रमः}


\twolineshloka
{कुम्भकर्णो बलेनासीत्सर्वेभ्योऽभ्यधिको युधि}
{मायावी रणशौण्डश्च रौद्रश्च रजनीचरः}


\twolineshloka
{खरो धनुषि विक्रान्तो ब्रह्मद्विट् पिशिताशनः}
{सिद्धविघ्नकरी चापि रौद्री शूर्पणखा तदा}


\twolineshloka
{सर्वे वेदविदः शूराः सर्वेसुचरितव्रताः}
{ऊषुः पित्रा सह रता गन्धमादनपर्वते}


\twolineshloka
{ततो वैश्रवणं तत्र ददृशुर्नरवाहनम्}
{पित्रा सार्धं समासीनमृद्ध्या परमया युतम्}


\twolineshloka
{जातामर्षास्ततस्ते तु तपसे धृतनिश्चयाः}
{ब्रह्माणं तोषयामासुर्घोरेण तपसा तदा}


\twolineshloka
{अतिष्ठदेकपादेन सहस्रं परिवत्सरान्}
{वायुभक्षो दशग्रीवः पञ्चाग्निः सुसमाहितः}


\twolineshloka
{अधःशायी कुम्भकर्णो यताहारो यतव्रतः}
{विभीषणः शीर्णपर्णमेकमभ्यवहारयन्}


\twolineshloka
{उपवासरतिर्धीमान्सदा जप्यपरायणः}
{तमेव कालमातिष्ठत्तीव्रं तप उदारधीः}


\twolineshloka
{स्वरः शूर्पणखा चैव तेषां वै तप्यतां तपः}
{परिचर्यां च रक्षां च चक्रतुर्हष्टमानसौ}


\twolineshloka
{पूर्णे वर्षसहस्रेतु शिरश्छित्त्वा दशाननः}
{जुहोत्यग्नौ दुराधर्षस्तेनातुष्यज्जगत्प्रभुः}


\twolineshloka
{ततो ब्रह्मा स्वयं गत्वा तपसस्तान्न्यवारयत्}
{प्रलोभ्यवरदानेन सर्वानेवपृथक्पृथक्}

\uvacha{ब्राह्मोवाच}


\twolineshloka
{प्रीतोऽस्मि वो निवर्तध्वं वरान्वृणुत पुत्रकाः}
{यद्यदिष्टमृते त्वेकममरत्वं तथाऽस्तु तत्}


\twolineshloka
{यद्यदग्नौ हुतं सर्वं शिरस्ते महदीप्सया}
{तथैव तानि ते देहे भविष्यन्ति यथेप्सया}


\twolineshloka
{वैरूप्यं च न ते देहे कामरूपधरस्तथा}
{भविष्यसि रणेऽरीणां विजेता न च संशयः}

\uvacha{रावण उवाच}


\twolineshloka
{गन्धर्वदेवासुरतो यक्षराक्षसतस्तथा}
{सर्पकिन्नरभूतेभ्यो न मे भूयात्पराभवः}

\uvacha{ब्रह्मोवाच}


\twolineshloka
{य एते कीर्तिताः सर्वे न तेभ्योऽस्ति भयं तव}
{ऋते मनुष्याद्भद्रं ते तथा तद्विहितं मया}

\uvacha{मार्कण्डेय उवाच}


\twolineshloka
{एवमुक्तो दशग्रीवस्तुष्टः समभवत्तदा}
{अवमेने हि दुर्बुद्धिर्मनुष्यान्पुरुषादकः}


\threelineshloka
{कुम्भकर्णमथोवाच तथैव प्रपितामहः}
{वरं वृणीष्व भद्रं ते प्रीतोस्मीति पुनःपुनः}
{स वव्रे महतीं निद्रां तमसा ग्रस्तचेतनः}


\twolineshloka
{तथाभविष्यतीत्युक्त्वा विभीषणमुवाच ह}
{वरं वृणीष्व पुत्र त्वं प्रीतोऽस्मीति पुनःपुनः}

\uvacha{विभीषण उवाच}


\twolineshloka
{परमापद्गतस्यापि नाधर्मे मे मतिर्भवेत्}
{अशिक्षितं च भगवन्ब्रह्मास्त्रं प्रतिभातु मे}

\uvacha{ब्रह्मोवाच}


\twolineshloka
{यस्माद्राक्षसयोनौ ते जातस्यामित्रकर्शन}
{नाधर्मे धीयते बुद्धिरमरत्वं ददानि ते}

\uvacha{मार्कण्डेय उवाच}


\twolineshloka
{राक्षसस्तु वरंलब्ध्वा दशग्रीवो विशापते}
{लङ्कायाश्च्यावयामास युधि जित्वा धनेश्वरम्}


\twolineshloka
{हित्वास भगवाँल्लङ्कामाविशद्गन्धमादनम्}
{गन्धर्वयक्षानुगतो रक्षःकिपुरुषैः सह}


\twolineshloka
{विमानं पुष्पकं तस्य जहाराक्रम्य रावणः}
{शशाप तं वैश्रवणो न त्वामेतद्वहिष्यति}


\twolineshloka
{यस्तु त्वां समरे हन्ता तमेवैतद्वहिष्यति}
{अवमत्य गुरुं मां च क्षिप्रं त्वन्न भविष्यसि}


\twolineshloka
{विभीषणस्तु धर्मात्मा सतां मार्गमनुस्मरन्}
{अन्वगच्छन्महाराज श्रिया परमया युतः}


\twolineshloka
{तस्मै स भगवांस्तुष्टो भ्राता भ्रात्रे धनेश्वरः}
{सैनापत्यं ददौ धीमान्यक्षराक्षससेनयोः}


\twolineshloka
{राक्षसाः पुरुषादाश्च पिशाचाश्च महाबलाः}
{सर्वे समेत्य राजानमभ्यषिञ्चन्दशाननम्}


\twolineshloka
{दशग्रीवश्चदैत्यानां दानवानां बलोत्कटः}
{आक्रम्य रत्नान्यहरत्कामरूपी विहङ्गम}


\twolineshloka
{रावयामास लोकान्यत्तस्माद्रावण उच्यते}
{दशग्रीवः कामबलो देवानां भयमादधत्}


॥इति श्रीमन्महाभारते अरण्यपर्वणि रामोपाख्यान-पर्वणि त्रिशततमोऽध्यायः॥२७६॥

\storymeta

\dnsub{अध्यायः २७७}\resetShloka

\uvacha{मार्कण्डेय उवाच}


\twolineshloka
{ततो ब्रह्मर्षयः सर्वे सिद्धा देवर्षयस्तथा}
{हव्यवाहं पुरस्कृत्य ब्रह्माणं शरणं गताः}

\uvacha{अग्निरुवाच}


\twolineshloka
{योसौ विश्रवसः पुत्रो दशग्रीवो महाबलः}
{अवध्यो वरदानेन कृतो भगवता पुरा}


\twolineshloka
{स बाधते प्रजाः सर्वा विप्रकारैर्महाबलः}
{ततो नस्त्रातु भगवान्नान्यस्त्राता हि विद्यते}

\uvacha{ब्रह्मोवाच}


\twolineshloka
{न स देवासुरैः शक्यो युद्धे जेतुं विभावसो}
{विहितं तत्रयत्कार्यमभितस्तस्य निग्रहः}


\twolineshloka
{तदर्थमवतीर्णोऽसौ मन्नियोगाच्चतुर्भुजः}
{विष्णुः प्रहरतां श्रेष्ठः स तत्कर्म करिष्यति}

\uvacha{मार्कण्डेय उवाच}


\twolineshloka
{पितामहस्ततस्तेषां सन्निधौ शक्रमब्रवीत्}
{सर्वैर्देवगणैः सार्धं सभव त्वं महीतले}


\twolineshloka
{विष्णोः सहायानृक्षीषु वानरीषु च सर्वशः}
{जनयध्वं सुतान्वीरान्कामरूपबलान्वितान्}


\twolineshloka
{ते यथोक्ता भगवता तत्प्रतिश्रुत्य शासनम्}
{ससृजुर्देवगन्धर्वाः पुत्रान्वानररूपिणः}


\twolineshloka
{ततो भागानुभागेन देवगन्धर्वदानवाः}
{अवतर्तुं महीं सर्वे मन्त्रयामासुरञ्जसा}


\threelineshloka
{अवतेरुर्महीं स्वर्गादंशैश्च सहिताः सुराः}
{ऋषयश्च महात्मानः सिद्धाश्च सह किन्नरैः}
{चारणाश्चासृजन्घोरान्वानरान्वनचारिणः}


\twolineshloka
{यस्य देवस्य यद्रूपं वेषस्तेजश्च यद्विधम्}
{अजायन्त समास्तेन तस्य तस्य सुतास्तदा}


\twolineshloka
{तेषां समक्षं गन्धर्वी दुन्दुभीं नाम नामतः}
{शशास वरदो देवो गच्छ कार्यार्थसिद्धये}


\twolineshloka
{पितामहवचः श्रुत्वा गन्धर्वी दुन्दुभी ततः}
{मन्थरा मानुषे लोके कुब्जा समभवत्तदा}


\twolineshloka
{शक्रप्रभृतयश्चैव सर्वे ते सुरसत्तमाः}
{वानरर्क्षवरस्त्रीषु जनयामासुरात्मजान्}


\twolineshloka
{तेऽन्ववर्तन्पितॄन्सर्वे यशसा च बलेन च}
{भेत्तारो गिरिशृङ्गाणां सालतालशिलायुधाः}


\twolineshloka
{वज्चसंहननाः सर्वेसर्वेऽमोघवलास्तथा}
{कामवीर्यबलाश्चैवसर्वे बुद्धिविशारदाः}


\twolineshloka
{नागायुतसमप्राणा वायुवेगसमा जवे}
{यथेच्छविनिपाताश्च केचिदत्र वनौकसः}


\twolineshloka
{एवं विधाय तत्सर्वं भगवाँल्लोकभावनः}
{मन्थरां बोधयामास यद्यत्कार्यं त्वया तथा}


\twolineshloka
{सा तद्वच समाज्ञाय तथा चक्रे मनोजवा}
{इतश्चेतश्च गच्छन्ती वैरसन्धुक्षणे रता}


॥इति श्रीमन्महाभारते अरण्यपर्वणि रामोपाख्यान-पर्वणि त्रिशततमोऽध्यायः॥२७७॥

\storymeta

\dnsub{अध्यायः २७८}\resetShloka

\uvacha{युधिष्ठिर उवाच}

\twolineshloka
{उक्तं भगवता जन्म रामादीनां पृथक्पृथक्}
{प्रस्थानकारणं ब्रह्मञ्श्रोतुमिच्छामि कथ्यताम्}


\twolineshloka
{कथं दाशरथी वीरौ भ्रातरौ रामलक्ष्मणौ}
{प्रस्तापितौ वने ब्रह्मन्मैथिली च यशस्विनी}

\uvacha{मार्कण्डेय उवाच}


\twolineshloka
{जातपुत्रो दशरथः प्रीतिमानभवन्नृप}
{क्रियारतिर्धर्मरतः सततं वृद्धसेविता}


\twolineshloka
{क्रमेण चास्य ते पुत्रा व्यवर्धन्त महौजसः}
{वेदेषु सरहस्येषु धनुर्वेदेषु पारगाः}


\twolineshloka
{चरितब्रह्मचर्यास्ते कृतदाराश्च पार्थिव}
{दृष्ट्वा रामं दशरथः प्रीतिमानभवत्सुखी}


\twolineshloka
{ज्येष्ठो रामोऽभवत्तेषां रमयामास हि प्रजाः}
{मनोहरतया धीमान्पितुर्हृदयनन्दनः}


\twolineshloka
{ततः स राजा मतिमान्मत्वाऽऽत्मानं वयोधिकम्}
{मन्त्रयामास सचिवैर्मन्त्रज्ञैश्च पुरोहितैः}


\twolineshloka
{अभिषेकाय रामस्य यावैराज्येन भारत}
{प्राप्तकालं च ते सर्वे मेनिरे मन्त्रिसत्तमाः}


\twolineshloka
{लोहिताक्षं महाबाहुं मत्तमातङ्गगामिनम्}
{कम्बुग्रीवं महोरस्कं नीलकुञ्चितमूर्धजम्}


\twolineshloka
{दीप्यमानं श्रिया वीरं शक्रादनवरं बले}
{पारगं सर्वधर्माणां बृहस्पतिसमं मतौ}


\twolineshloka
{सर्वानुरक्तप्रकृतिं सर्वविद्याविशारदम्}
{जितेन्द्रियममित्राणामपि दृष्टिमनोहरम्}


\twolineshloka
{नियन्तारमसाधूनां गोप्तारं धर्मचारिणाम्}
{धृतिमन्तमनाधृष्यं जेतारमपराजितम्}


\twolineshloka
{पुत्रं राजा दशरथः कौसल्यानन्दवर्धनम्}
{सन्दृश्यपरमां प्रीतिमगच्छत्कुलनन्दनम्}


\twolineshloka
{चिन्तयंश्च महातेजा गुणान्रामस्य वीर्यवान्}
{अभ्यभाषत भद्रं ते प्रीयमाणः पुरोहितम्}


\twolineshloka
{अद्य पुष्यो निशि ब्रह्मन्पुण्यं योगमुपैष्यति}
{सभाराः सभ्रियन्तां मे रामश्चोपनिमन्त्र्यताम्}


\twolineshloka
{श्व एवपुष्यो भविता यत्ररामः सुतो मया}
{यौवराज्येऽभिषेक्तव्यः पौरेषु सहमन्त्रिभिः}


\twolineshloka
{इति तद्राजवचनं प्रतिश्रुत्याथ मन्थरा}
{कैकेयीमभिगम्येदं काले वचनमब्रवीत्}


\twolineshloka
{अद्य कैकेयि दौर्भाग्यं राज्ञा ते ख्यापितं महत्}
{आशीविषस्त्वां सङ्क्रुद्धश्छन्नो दशति दुर्भगे}


\twolineshloka
{सुभगा खलु कौसल्या यस्याः पुत्रोऽभिषेक्ष्यते}
{कुतो हि तव सौभाग्यं यस्याः पुत्रो न राज्यभाक्}


\twolineshloka
{सा तद्वचनमाज्ञाय सर्वाभरणभूषिता}
{वेदी विलग्नमध्येन बिभ्रती रूपमुत्तमम्}


\twolineshloka
{वविक्ते पतिमासाद्य हसन्तीव शुचिस्मिता}
{राजानं तर्जयन्तीव मधुरं वाक्यमब्रवीत्}


\threelineshloka
{सत्यप्रतिज्ञ यन्मे त्वं काममेकं विसृष्टवान्}
{उपाकुरुष्व तद्राजंस्तस्मान्मुञ्चस्व सङ्कटात्}
{तदद्य कुरु सत्यं मे वरं वरद भूपते}

\uvacha{राजोवाच}


\twolineshloka
{वरं ददानि ते हन्त तद्गृहाण यदिच्छसि}
{अवध्यो वध्यतां कोद्य वध्यः कोऽद्य विमुच्यताम्}


\twolineshloka
{धनं ददानि कस्याद् ह्रियतां कस्यरवापुनः}
{ब्राह्मणस्वादिहान्यत्रयत्किञ्चिद्वित्तमस्ति मे}


\twolineshloka
{पृथिव्यां राजराजोऽस्मि चातुर्वर्ण्यस्य रक्षिता}
{यस्तेऽभिलषितः कामो ब्रूहि कल्याणि माचिरम्}


\twolineshloka
{सातद्वचनमाज्ञाय परिगृह्य नराधिपम्}
{आत्मनो बलमाज्ञाय तत एनमुवाच ह}


\twolineshloka
{आभिषेचनिकं यत्ते रामार्थमुपकल्पितम्}
{भरतस्तदवाप्नोतु वनं गच्छतु राघवः}


\twolineshloka
{नव पञ्च च वर्षाणि दण्डकारण्यमाश्रितः}
{चीराजिनजटाधारी रामो भवतु तापसः}


\twolineshloka
{स तं राजा वरं श्रुत्वा विप्रियं दारुणोदयम्}
{दुःखार्तो भरतश्रेष्ठ न किञ्चिद्व्याजहार ह}


\twolineshloka
{ततस्तथोक्तं पितरं रामो विज्ञाय वीर्यवान्}
{वनं प्रतस्थे धर्मात्मा राजा सत्यो भवत्विति}


\twolineshloka
{तमन्वगच्छल्लक्ष्मीवान्धनुष्माँल्लक्ष्मणस्तदा}
{सीता च भार्या भद्रं ते वैदेही जनकात्मजा}


\twolineshloka
{ततो वनं गतेरामे राजा दशरथस्तदा}
{समयुज्यत देहस्य कालपर्यायधर्मणा}


\twolineshloka
{रामं तु गतमाज्ञाय राजानं च तथागतम्}
{अनार्या भरतं देवी कैकेयी वाक्यमब्रवीत्}


\twolineshloka
{गतोदशरथः स्वर्गं वनस्थौ रामलक्ष्मणौ}
{गृहाण राज्यंविपुलं क्षेमं निहतकण्टकम्}


\twolineshloka
{तामुवाच स धर्मात्मा नृशंसं बत ते कृतम्}
{पतिं हत्वाकुलं चेदमुत्साद्य धनलुब्धया}


\twolineshloka
{अयशः पातयित्वा मे मूर्ध्नि त्वं कुलपांसने}
{सकामा भव मे मातरित्युक्त्वा प्ररुरोद ह}


\twolineshloka
{स चारित्रं विशोध्याथ सर्वप्रकृतिसन्निधौ}
{अन्वयाद्धातरं रामं विनिवर्तनलालसः}


\twolineshloka
{कौसल्यां च सुमित्रां च कैकेयीं च सुदुःखितः}
{अग्रे प्रस्थाप्य यानैः स शत्रुघ्नसहितो ययौ}


\twolineshloka
{वसिष्ठवामदेवाभ्यां विप्रैश्चान्यैः सहस्रशः}
{पौरजानपदैः सार्धं रामानयनकाङ्क्षया}


\twolineshloka
{ददर्श चित्रकूटस्थं स रामं सहलक्ष्मणम्}
{तापसानामलङ्कारं धारयन्तं धनुर्धरम्}


\twolineshloka
{उवाच प्राञ्जलिर्भूत्वाप्रणिपत्य रघूत्तमम्}
{शशंस मरणं राज्ञः सोऽनाथांश्चापि कोसलान्}


% Check verse!
नाथ त्वं प्रतिपद्यस्व स्वराज्यमिति चोक्तवान्
\twolineshloka
{स तस्य वचनं श्रुत्वा रामः परमदुःखितः}
{चकार देवकल्पस्य पितुः स्नात्वोदकक्रियाम्}


\threelineshloka
{अब्रवीत्स तदारामो भ्रातरं भ्रातृवत्सलम्}
{पादुके मे भविष्येते राज्यगोप्त्र्यौ परन्तप}
{एवमस्त्विति तं प्राह भरतः प्रणतस्तदा}


\twolineshloka
{विसर्जितः स रामेण पितुर्वचनकारिणा}
{नन्दिग्रामेऽकरोद्राज्यं पुरस्कृत्यास्य पादुके}


\twolineshloka
{रामस्तु पुनराशङ्क्य पौरजानपदागमम्}
{प्रविवेश महारण्यं शरभङ्गाश्रमं प्रति}


\twolineshloka
{सत्कृत्य शरभङ्गं स दण्डकारण्यमाश्रितः}
{नदीं गोदावरीं रम्यामाश्रित्य न्यवसत्तदा}


\twolineshloka
{वसतस्तस्य रामस्य ततः शूर्पणखाकृतम्}
{खरेणासीन्महद्वैरं जनस्थाननिवासिना}


\twolineshloka
{रक्षार्थं तापसानां तु राघवो धर्मवत्सलः}
{चतुर्दशसहस्राणि जघान भुवि राक्षसान्}


\twolineshloka
{दूषणं च स्वरं चैवनिहत्य सुमहाबलौ}
{चक्रे क्षेमं पुनर्धीमान्धर्मारण्यं स राघवः}


\twolineshloka
{हतेषु तेषु रक्षःसु ततः शूर्पणखा पुनः}
{ययौ निकृत्तनासोष्ठी लङ्कां भ्रातुर्निवेशनम्}


\twolineshloka
{ततो रावणमभ्येत्य राक्षसी दुःखमूर्च्छिता}
{पपात पादयोर्भ्रातुः संशुष्करुधिरानना}


\twolineshloka
{तां तथा विकृतां दृष्ट्वा रावणः क्रोधमूर्च्छितः}
{उत्पपातासनात्क्रुद्धो दन्तैर्दन्तानुपस्पृशन्}


\twolineshloka
{स्वानमात्यान्विसृज्याथ विविक्ते तामुवाच सः}
{केनास्येवं कृता भद्रे मामचिन्त्यावमत्य च}


\twolineshloka
{कः शूलं तीक्ष्णमासाद्य सर्वगात्रेषु सेवते}
{कः शिरस्यग्निमाधाय विश्वस्तः स्वपते सुखम्}


\twolineshloka
{आशीविषं घोरतरं पादेन स्पृशतीह कः}
{सिंहं केसरिणं मत्तः स्पृष्ट्वा दंष्ट्रासु तिष्ठति}


\twolineshloka
{इत्येवं ब्रुवतस्तस्य नेत्रेभ्यस्तेजसोऽर्चिषः}
{निश्चेरुर्दह्यतो रात्रौ वृक्षस्येव स्वरन्ध्रतः}


\twolineshloka
{तस्य तत्सर्वमाचख्यौ भगिनी रामविक्रमम्}
{खरदूषणसंयुक्तं राक्षसानां पराभवम्}


\twolineshloka
{ततो ज्ञातिवधं श्रुत्वा रावणः कालचोदितः}
{रामस्य वधमाकाङ्क्षन्मारीचं मनसागमत्}


\twolineshloka
{स निश्चित्यततः कृत्यं सागरं लवणाकरम्}
{ऊर्ध्वमाचक्रमे राजा विधाय नगरे विधिम्}


\twolineshloka
{त्रिकूटं समतिक्रम्य कालपर्वतमेव च}
{ददर्श मकरावासं गम्भीरोदं महोदधिम्}


\twolineshloka
{तमतीत्याथ गोकर्णमभ्यगच्छद्दशाननः}
{दयितं स्तानमव्यग्रं शूलपाणेर्महात्मनः}


\twolineshloka
{तत्राभ्यगच्छन्मारीचं पूर्वामात्यं दशाननः}
{पुरा रामभयादेव तापसं प्रियजीवितम्}


॥इति श्रीमन्महाभारते अरण्यपर्वणि रामोपाख्यान-पर्वणि त्रिशततमोऽध्यायः॥२७८॥

\storymeta

\dnsub{अध्यायः २७९}\resetShloka

\uvacha{मार्कण्डेय उवाच}


\twolineshloka
{मारीचस्त्वथ सभ्रान्तो दृष्ट्वा रावणमागतम्}
{पूजयामास सत्कारैः फलमूलादिभिस्ततः}


\twolineshloka
{विश्रान्तं चैनमासीनमन्वासीनः स राक्षसः}
{उवाच प्रश्रितं वाक्यं वाक्यज्ञो वाक्यकोविदम्}


\twolineshloka
{न ते प्रकृतिमान्वर्णः कच्चित्क्षेमं पुरे तव}
{कच्चित्प्रकृतयः सर्वा भजन्ते त्वां यथा पुरा}


\twolineshloka
{किमिहागमने चापि कार्यं ते राक्षसेश्वर}
{कृतमित्येव तद्विद्धि यद्यपि स्यात्सुदुष्करम्}


\twolineshloka
{शशंस रावणस्तस्मै तत्सर्वं रामचेष्टितम्}
{समासेनैव कार्याणि क्रोधामर्षसमन्वितः}


\twolineshloka
{मारीचस्त्वब्रवीच्छ्रत्वा समासेनैव रावणम्}
{अलं ते राममासाद्य वीर्यज्ञो ह्यस्मि तस्य वै}


\threelineshloka
{बाणवेगं हि कस्तस्य शक्तः सोढुं महात्मनः}
{प्रव्रज्यायां हि मे हेतुः स एव पुरुषर्षभः}
{विनाशमुखमेतत्ते केनाख्यातं दुरात्मना}


\twolineshloka
{तमुवाचाथ सक्रोधो रावणः परिभर्त्सयन्}
{अकुर्वतोऽस्मद्वचनं स्यान्मृत्युरपि ते ध्रुवम्}


\twolineshloka
{मरीचश्चिन्तयामास विशिष्टान्मरणं वरम्}
{अवश्यं मरणे प्राप्ते करिष्याम्यस्य यन्मतम्}


\twolineshloka
{ततस्तं प्रत्युवाचाथ मारीचो रक्षसांवरम्}
{किं ते साह्यां मया कार्यं करिष्याम्यवशोपि तत्}


\twolineshloka
{तमब्रवीद्दशग्रीवो गच्छ सीतां प्रलोभय}
{रत्नशृङ्गो मृगो भूत्वा रत्नचित्रतनूरुहः}


\twolineshloka
{ध्रुवं सीता समालक्ष्यत्वां रामं चोदयिष्यति}
{अपक्रान्ते च काकुत्स्थे सीता वश्या भविष्यति}


\twolineshloka
{तामादायापनेष्यामि ततः स नभविष्यति}
{भार्यावियोगाद्दुर्बुद्धिरेतत्साह्यं कुरुष्व मे}


\twolineshloka
{इत्येवमुक्तोमारीचः कृत्वोदकमथात्मनः}
{रावणं पुरतो यान्तमन्वगच्छत्सुदुःखितः}


\twolineshloka
{ततस्तस्याश्रमं गत्वारामस्याक्लिष्टकर्मणः}
{चक्रतुस्तद्यथा सर्वमुभौ यत्पूर्वमन्त्रितम्}


\twolineshloka
{रावणस्तु यतिर्भूत्वा मुण्डः कुण्डीत्रिदण्डधृत्}
{मृगश्चभूत्वामारीचस्तं देशमुपजग्मतुः}


\twolineshloka
{दर्शयामास मारीचो वैदेहीं मृगरूपधृत्}
{चोदयामास तस्यार्थे सा रामं विधिचोदिता}


\twolineshloka
{रामस्तस्याः प्रियं कुर्वन्धनुरादाय सत्वरः}
{रक्षार्थे लक्ष्मणं न्यस्य प्रययौ मृगलिप्सया}


\twolineshloka
{स धन्वी बद्धतूणीरः खङ्गगोधाङ्गुलित्रवान्}
{अन्वधावन्मृगं रामो रुद्रस्तारामृगं यथा}


\twolineshloka
{सोऽन्तर्हितः पुनस्तस्य दर्शनं राक्षसो व्रजन्}
{चकर्ष महदध्वानं रामस्तं वुबुधे ततः}


\twolineshloka
{निशाचरं विदित्वा तं राघवः प्रतिभानवान्}
{अमोघं शरमादाय जघान मृगरूपिणम्}


\twolineshloka
{स रामवाणाभिहतः कृत्वा रामस्वरं तदा}
{हा सीते लक्ष्मणेत्येवं चुक्रोशार्तस्वरेण ह}


\twolineshloka
{शुश्राव तस्य वैदेही ततस्तां करुणां गिरम्}
{साप्रापतत्ततः सीता तामुवाचाथ लक्ष्मणः}


\twolineshloka
{अलं ते शङ्कया भीरु को रामं प्रहरिष्यति}
{मुहूर्ताद्द्रक्ष्यसे रामं भर्तारं त्वं शुचिस्मितम्}


\twolineshloka
{इत्युक्ता सा प्ररुदती पर्यशङ्कत लक्ष्मणम्}
{हता वै स्त्रीस्वभावेन शुद्धचारित्रभूषणा}


\twolineshloka
{सा तं परुषमारब्धा वक्तुं साध्वी पतिव्रता}
{नैष कामो भवेन्मूढ यं त्वं प्रार्थयसे हृदा}


\twolineshloka
{अप्यहंशस्त्रमादाय हन्यामात्मानमात्मना}
{पतेयं गिरिशृङ्गाद्वा विशेयं वा हुताशनम्}


\twolineshloka
{रामं भर्तारमुत्सुज्यन त्वहं त्वां कथञ्चन}
{निहीनमुपतिष्ठेयं शार्दूली क्रोष्टुकं यथा}


\twolineshloka
{एतादृशं वचः श्रुत्वा लक्ष्मणः प्रियराघ्नव}
{पिधायकर्णौ सद्वृत्तः प्रस्थितो येन राघवः}


\twolineshloka
{स रामस्य पदं गृह्य प्रससार धनुर्धरः}
{अवीक्षमाणो विम्बोष्ठीं प्रययौ लक्ष्मणस्तदा}


\twolineshloka
{एतस्मिन्नन्तरे रक्षो रावणः प्रत्यदृश्यत}
{अभव्यो भव्यरूपेण भस्मच्छन्न इवानलः}


\twolineshloka
{यतिवेपप्रतिच्छन्नो जिहीर्षुस्तामनिन्दिताम्}
{उपागच्छत्स वैदेहीं रावणः पापनिश्चयः}


\twolineshloka
{सा तमालक्ष्यसप्राप्तं धर्मज्ञा जनकात्मजा}
{निमन्त्रयामास तदा फलमूलाशनादिभिः}


\twolineshloka
{अवमत्यततः सर्वं स्वं रूपं प्रत्यपद्यत}
{सान्त्वयामास वैदेहीं कामी राक्षसपुङ्गवः}


\twolineshloka
{सीते राक्षसराजोऽहंरावणो नाम विश्रुतः}
{मम लङ्कापुरी नाम्ना रम्या पारे महोदधेः}


\twolineshloka
{तत्र त्वं नरनारीषु शोभिष्यसि मया सह}
{भार्या मे भव सुश्रोणि तापसं त्यज राघवम्}


\twolineshloka
{एवमादीनि वाक्यानि श्रुत्वा तस्याथ जानकी}
{पिधाय कर्णौ सुश्रोणी मैवमित्यब्रवीद्वचः}


\twolineshloka
{प्रपतेद्द्यौः सनक्षत्रा पृथिवी शकलीभवेत्}
{शुष्येत्तोयनिधौ तोयं चन्द्रः शीतांशुतां त्यजेत्}


\twolineshloka
{उष्णांशुत्वमथो जह्यादादित्यो वह्निरुष्णताम्}
{त्यक्त्वाशैत्यं भजेन्नाहं त्यजेयंरघुनन्दनम्}


\twolineshloka
{कथं हि भिन्नकरटं पद्मिनं वनगोचरम्}
{उपस्थाय महानागं करेणुः सूकरं स्पृशेत्}


\twolineshloka
{कथं हि पीत्वा माध्वीकं पीत्वा च मधुमाधवीम्}
{लोभं सौवीरके कुर्यान्नारी काचिदिति स्मरेः}


\twolineshloka
{इति सा तं समाभाष्य प्रविवेशाश्रमं ततः}
{क्रोधात्प्रस्फुरमाणौष्ठी विधुन्वाना करौ मुहुः}


% Check verse!
तामधिद्रुत्य सुश्रोणीं रावणः प्रत्यषेधयत्
\twolineshloka
{भर्त्सयित्वातु रूक्षेण स्वरेण गतचेतनाम्}
{मूर्धजेषु निजग्राह ऊर्ध्वमाचक्रमे ततः}


\twolineshloka
{तां ददर्श ततो गृध्रो जटायुर्गिरिगोचरः}
{रुदतीं रामरामेति हियमाणां तपस्विनीम्}


॥इति श्रीमन्महाभारते अरण्यपर्वणि रामोपाख्यान-पर्वणि त्रिशततमोऽध्यायः॥२७९॥

\storymeta

\dnsub{अध्यायः २८०}\resetShloka

\uvacha{गार्कण्डेय उवाच}


\twolineshloka
{सखा दशरथस्यासीज्जटायुररुणात्मजः}
{गृध्रराजो महावीरः सम्पातिर्यस्य सोदरः}


\twolineshloka
{स ददर्श तदा सीतां रावणाङ्कगतां स्नुषाम्}
{सक्रोधोऽभ्यद्रवत्पक्षी रावणं राक्षसेश्वरम्}


\twolineshloka
{अथैनमब्रवीद्गृध्रो मुञ्चमुञ्चेति मैथिलीम्}
{ध्रियमाणे मयि कथं हरिष्यसि निशाचर}


\twolineshloka
{न हिमे मोक्ष्यसे जीवन्यदि नोत्सृजसे वधूम्}
{उक्त्वैवं राक्षसेन्द्रं तं चकर्त नखरैर्भृशम्}


\twolineshloka
{पक्षतुण्डप्रहारैश्च शतशो जर्जरीकृतम्}
{चक्षार रुधिरं भूरि गिरिः प्रस्रवणैरिव}


\twolineshloka
{स वध्यमानो गृध्रेण रामप्रियहितैषिणा}
{खङ्गमादाय चिच्छेद भुजौ तस्य पतत्रिणः}


\twolineshloka
{निहत्य गृध्रराजं सभिन्नाभ्रशिखरोपमम्}
{ऊर्ध्वमाचक्रमे सीतां गृहीत्वाऽङ्केन राक्षसः}


\twolineshloka
{यत्रयत्रतु वैदेही पश्यत्याश्रममण्डलम्}
{सरोवा सरितो वाऽपि तत्र मुञ्चति भूषणम्}


\twolineshloka
{सा ददर्श गिरिप्रस्थे पञ्च वानरपुङ्गवान्}
{तत्र वासो महद्दिव्यमुत्ससर्ज मनस्विनी}


\twolineshloka
{तत्तेषां वानरेन्द्राणां पपात पवनोद्धतम्}
{मध्ये सुपीतं पञ्चानां विद्युन्मेघान्तरे यथा}


\twolineshloka
{अचिरेणातिचक्राम खेचरः खे चरन्निव}
{ददर्शाथ पुरीं रम्यां बहुद्वारां मनोरमाम्}


\twolineshloka
{प्राकारवप्रसबाधां निर्मितां विश्वकर्मणा}
{प्रविवेशपुरीं लङ्कां ससीतो राक्षसेश्वरः}


\twolineshloka
{एवं हृतायां वैदेह्यां रामो हत्वा महामृगम्}
{निवृत्तो ददृशे दूराद्भ्रातरं लक्ष्मणं तदा}


\twolineshloka
{कथमुत्सृज्य वैदेहीं वने राक्षससेविते}
{इति तं भ्रातरं दृष्ट्वा प्राप्तोऽसीति व्यगर्हयत्}


\twolineshloka
{मृगरूपधरेणाथ रक्षसासोपकर्षणम्}
{भ्रातुरागमनं चैवचिन्तयन्पर्यतप्यत}


\twolineshloka
{गर्हयन्नेव रामस्तु त्वरितस्तं समासदत्}
{अपि जीवति वैदेहीमिति पश्यामि लक्ष्मण}


\twolineshloka
{तस्य तत्सर्वमाचख्यौ सीताया लक्ष्मणो वचः}
{यदुक्तवत्यसदृशं वैदेही पश्चिमं वचः}


\twolineshloka
{दह्यमानेन तु हृदा रामोऽभ्यपतदाश्रमम्}
{स ददर्श तदा गृध्रं निहतं पर्वतोपमम्}


\twolineshloka
{राक्षसं शङ्कमानस्तं विकृष्य बलवद्धनुः}
{अभ्यधावत काकुत्स्थस्ततस्तं सहलक्ष्मणः}


\twolineshloka
{स तावुवाच तेजस्वी सहितौ रामलक्ष्मणौ}
{गृध्रराजेस्मि भद्रंवां सखा दशरथस्य वै}


\twolineshloka
{तस्य तद्वचनं श्रुत्वा सङ्गृह्य धनुषी शुभे}
{कोयं पितरमस्माकं नाम्नाऽऽहेत्यूचतुश्च तौ}


\twolineshloka
{ततो ददृशतुस्तौ तं छिन्नपक्षद्वयं खगम्}
{तयोः शशंस गृध्रस्तु सीतार्थे रावणाद्वधम्}


\twolineshloka
{अपृच्छद्राघवो गृध्रं रावणः कां दिशं गतः}
{तस्य गृध्रः शिरःकम्पैराचचक्षे ममार च}


\twolineshloka
{दक्षिणामिति काकुत्स्थो विदित्वाऽस्य तदिङ्गितम्}
{संस्कारं लम्भयामास सखायं पूजयन्पितुः}


\twolineshloka
{ततो दृष्ट्वाऽऽश्रमपदं व्यपविद्धबृसीकटम्}
{विध्वस्तकलशं शून्यं गोमायुशतसङ्कुलम्}


\twolineshloka
{दुःखशोकसमाविष्टौ वैदेहीहरणार्दितौ}
{जग्मतुर्दण्डकारण्यं दक्षिणेन परन्तपौ}


\twolineshloka
{वने महति तस्मिंस्तु रामः सौमित्रिणा सह}
{ददर्श मृगयूथनि द्रवमाणानि सर्वशः}


\twolineshloka
{शब्दं च घोरं सत्वानां दावाग्नरिववर्धतः}
{अपश्येतां मुहूर्ताच्च कबन्धं घोरदर्शनम्}


\twolineshloka
{मेघपर्वतसङ्काशं सालस्कन्धं महाभूजम्}
{उरोगतविशालाक्षं महोदरमहामुखम्}


\twolineshloka
{यदृच्छयाथ तद्रक्षः करे जग्राह लक्ष्मणम्}
{विषादमगमत्सद्यः सौमित्रिरथ भारत}


\twolineshloka
{स राममभिसप्रेक्ष्य कृष्यते येन तन्मुखम्}
{विषण्णश्चाब्रवीद्रामं पश्यावस्थामिमां मम}


\twolineshloka
{हरणं चैववैदेह्या मम चायमुपप्लवः}
{राज्यभ्रंशश्च भवतस्तातस्य मरणं तथा}


\twolineshloka
{नाहं त्वां मह वैदेह्या समेतं कोसलागतम्}
{द्रक्ष्यामि प्रथिते राज्येपितृपैतामहे स्थितम्}


\twolineshloka
{द्रक्ष्यन्त्यार्यस्य धन्या ये कुशलाजशमीदलैः}
{अभिषिक्तस्य वदनं सोमं शान्तघनं यथा}


\twolineshloka
{एवं बहुविधं धीमान्विललाप स लक्ष्मणः}
{तमुवाचाथकाकुत्स्थः सभ्रमेष्वप्यसभ्रमः}


\threelineshloka
{मा विषीद नरव्याघ्र नैष कश्चिन्मयि स्थिते}
{शक्तो धर्षयितुं वीर सुमित्रानन्दवर्धन}
{छिन्ध्यस्य दक्षिणं बाहुं छिन्नः सव्यो मया भुजः}


\twolineshloka
{इत्येवं वदता तस्य भुजो रामेण पातितः}
{खङ्गेन भृशतीक्ष्णेन निकृत्तस्तिलकाण्डवत्}


\twolineshloka
{ततोऽस्य दक्षिणं बाहुं स्वङ्गेनाजघ्निवान्बली}
{सौमित्रिरपि सप्रेक्ष्यभ्रातरं राघवं स्थितम्}


\twolineshloka
{पुनर्जघान पार्श्वे वै तद्रक्षो लक्ष्मणो भृशम्}
{गतासुरपतद्भूमौ कबन्धः सुमहांस्ततः}


\twolineshloka
{तस्य देहाद्विनिःसृत्य पुरुषो दिव्यदर्शनः}
{ददृशे दिवमास्थाय दिवि सूर्य इव ज्वलन्}


\twolineshloka
{पप्रच्छ रामस्तं वाग्मी कस्त्वं प्रब्रूहि पृच्छतः}
{कामया किमिदं चित्रमाश्चर्यं प्रतिभाति मे}


\twolineshloka
{तस्याचचक्षेगन्धर्वोविश्वावसुरहं नृप}
{प्राप्तो ब्राह्मणशपेन योनिं राक्षससेविताम्}


\twolineshloka
{रावणेन हृतासीता लङ्कायां सन्निवेशिता}
{सुग्रीवमभिगच्छस्वस ते साह्यं करिष्यति}


\twolineshloka
{एषा पम्पा शिवजला हंसकारण्डवायुता}
{ऋश्यमूकस्य शैलस्य सन्निकर्षे तटाकिनी}


\twolineshloka
{वसते तत्रसुग्रीवश्चतुर्भिः सचिवैः सह}
{भ्राता वानरराजस्य वालिनो हेममालिनः}


\twolineshloka
{तेन त्वं सहसङ्गम्य दुःखमूलं निवेदय}
{समानशीलो भवतः साहाय्यं स करिष्यति}


\twolineshloka
{एतावच्छक्यमस्माभिर्वक्तुं द्रष्टासि जानकीम्}
{ध्रुवं वानरराजस्य विदितो रावणालयः}


\twolineshloka
{इत्युक्त्वाऽन्तर्हितो दिव्यः पुरुषः स महाप्रभः}
{विस्मयं जग्मतुश्चोभौ प्रवीरौ रामलक्ष्मणौ}


॥इति श्रीमन्महाभारते अरण्यपर्वणि रामोपाख्यान-पर्वणि त्रिशततमोऽध्यायः॥२८०॥

\storymeta

\dnsub{अध्यायः २८१}\resetShloka

\uvacha{मार्कण्डेय उवाच}


\twolineshloka
{ततोऽविदूरे नलिनीं प्रभूतकमलोत्पलाम्}
{सीताहरणदुःखार्तः पम्पां रामः समासदत्}


\twolineshloka
{मारुतेन सुशीतेन सुखेनामृतगन्धिना}
{सेव्यमानो वने तस्मिञ्जगाम मनसा प्रियाम्}


\twolineshloka
{विललाप स राजेन्द्रस्तत्र कान्तामनुस्मरन्}
{कामबाणाभिसन्तप्तं सौमित्रिस्तमथाब्रवीत्}


\twolineshloka
{न त्वामेवंविधो भावः स्प्रष्टुमर्हति मानद}
{आत्मवन्तमिव व्याधिः पुरुषंवृद्धसेविनम्}


\twolineshloka
{प्रवृत्तिरुपलब्धा ते वैदेह्या रावणस्य च}
{तां त्वं पुरुषकारेण बुद्ध्या चैवोपपादय}


\twolineshloka
{अभिगच्छाव सुग्रीवं शैलस्थं हरिपुङ्गवम्}
{मयि शिष्ये च भृत्ये च सहाये च समाश्वस}


\twolineshloka
{एवं बहुविधैर्वाक्यैर्लक्ष्मणेन स राघवः}
{उक्तः प्रकृतिमापेदे कार्ये चानन्तरोऽभवत्}


\twolineshloka
{निषेव्य वारि पम्पायास्तर्पयित्वा पितृनपि}
{प्रतस्थतुरभौ वीरौ भ्रातरौ रामलक्ष्मणौ}


\twolineshloka
{तावृश्यमूकमभ्येत्य बहुमूलफलद्रुमम्}
{गिर्यग्रे वानरान्पञ्च वीरौ ददृशतुस्तदा}


\twolineshloka
{सुग्रीवः प्रेषयामास सचिवं वानरं तयोः}
{बुद्धिमन्तं हनूमन्तं हिमवन्तमिव स्थितम्}


\twolineshloka
{तेन सम्भाष्य पूर्वं तौ सुग्रीवमभिजग्मतुः}
{सख्यं वानरराजेन चक्रे रामस्तदा नृप}


\twolineshloka
{ततः सीतां हृतां श्रुत्वा सुग्रीवो वालिना कृतम्}
{दुःखमाख्यातवान्सर्वं रामायामिततेजसे}


\twolineshloka
{तद्वासो दर्शयामास तस्य कार्ये निवेदिते}
{वानराणां तु यत्सीता ह्रियमाणा व्यपासृजत्}


\twolineshloka
{तत्प्रत्ययकरं लब्ध्वा सुग्रीवं प्लवगाधिपम्}
{पृथिव्यां वानरैश्वर्ये स्वयंरामोऽभ्यषेचयत्}


\twolineshloka
{प्रतिजज्ञे चकाकुत्स्थः समरे वालिनो वधम्}
{सुग्रीवश्चापि वैदेह्याः पुनरानयनं नृप}


\twolineshloka
{इत्येवं समयं कृत्वाविश्वास्य च परस्परम्}
{अभ्येत्य सर्वकिष्किन्धां तस्थुर्युद्धाभिकाङ्क्षिणः}


\twolineshloka
{सुग्रीवः प्राप्यकिष्किन्धां ननादौघनिभस्वनः}
{नसाय् तन्ममृषे वाली तारा तं प्रत्यषेधयत्}


\twolineshloka
{यथानदतिसुग्रीवो बलवानेष वानरः}
{मन्ये चाश्रयवान्प्राप्तो न त्वं निष्क्रान्तुमर्हसि}


\twolineshloka
{हेममाली ततो वाली तारां ताराधिपाननाम्}
{प्रोवाच वचनं वाग्मी तां वानरपतिः पतिः}


% Check verse!
\twolineshloka
{सर्वभूतरुतज्ञा त्वं शृणु सर्वं कपीश्वर}
{केन चाश्रयवान्प्राप्तो ममैष भ्रातृगन्धिकः}

\twolineshloka
{चिन्तयित्वा मुहूर्तं तु तारा ताराधिपप्रभा}
{पतिमित्यब्रवीत्प्राज्ञा शृणु सर्वं कपीश्वर}


\twolineshloka
{हृतदारो महासत्वोरामो दशरथात्मजः}
{तुल्यारिमित्रतां प्राप्तः सुग्रीवेण धनुर्धरः}


\twolineshloka
{भ्राता चास्य महाबाहुः सौमित्रिरपराजितः}
{लक्ष्मणो नाम मेधावी स्थितः कार्यार्थसिद्धये}


\twolineshloka
{मैन्दश्च द्विविदश्चापि हनूमांश्चानिलात्मजः}
{जाम्बवानृक्षराजश्च सुग्रीवसचिवाः स्थिताः}


\twolineshloka
{सर्व एते महात्मानो बुद्धिमन्तो महाबलाः}
{अलं तव विनाशाय रामवीर्यव्यपाश्रयाः}


\twolineshloka
{तस्यास्तदाक्षिप्य वचो हितमुक्तं कपीश्वरः}
{पर्यशङ्कत तामीर्षुः सुग्रीवगतमानसाम्}


\twolineshloka
{तारां परुषमुक्त्वा तु निर्जगाम गुहामुखात्}
{स्थितं माल्यवतोऽभ्याशे सुग्रीवं सोभ्यभाषत}


\twolineshloka
{असकृत्त्वं मया क्लीव निर्जितो जीवितप्रियः}
{मुक्तो गच्छसि दुर्बुद्धे कथङ्कारं रणे पुनः}


\twolineshloka
{इत्युक्तः प्राहसुग्रीवो भ्रातरं हेतुमद्वचः}
{प्राप्तकालममित्रघ्नं रामं सम्बोधयन्निव}


\twolineshloka
{हृतराज्यस्य मे राजन्हृतदारस्य च त्वया}
{किं मे जीवितसामर्थ्यमिति विद्धि समागतम्}


\twolineshloka
{एवमुक्त्वाबहुविधं ततस्तौ सन्निपेततुः}
{समरे वालिसुग्रीवौ सालतालशिलायुधौ}


\twolineshloka
{उभौ जघ्नतुरन्योन्यमुभौ भूमौ निपेततुः}
{उभौ ववल्गतुश्चित्रं मुष्टिभिश्च निजघ्नतुः}


\twolineshloka
{उभौ रुधिरसंसिक्तौ नखदन्तपरिक्षतौ}
{शुशुभाते तदा वीरौ पुष्पिताविव किंशुकौ}


\twolineshloka
{न विशेषस्तयोर्युद्धे यदा कश्चन दृश्यते}
{सुग्रीवस्य तदा मालां हनुमान्कण्ठ आसजत्}


\twolineshloka
{स मालया तदा वीरः शुशुभे कण्ठसक्तया}
{श्रीमानिव महाशैलो मलयो मेघमालया}


\twolineshloka
{कृतचिह्नं तु सुग्रीवं रामो दृष्ट्वा महाधनुः}
{विचकर्ष धनुःश्रेष्ठं वालिमुद्दिश्य लक्षयन्}


\twolineshloka
{विष्फारस्तस्य धनुषो यन्त्रस्येव तदा बभौ}
{वितत्रास तदा वाली शरेणाभिहतो हृदि}


\twolineshloka
{स भिन्नहृदयो वाली वक्राच्छोणितमुद्वमन्}
{ददर्शावस्थितं रामं ततः सौमित्रिणा सह}


\twolineshloka
{गर्हयित्वास काकुत्स्थं पपात भुवि मूर्च्छितः}
{तारा ददर्श तं भूमौ तारापतिमिव च्युतम्}


\twolineshloka
{हते वालिनि सुग्रीवः किष्किन्धां प्रत्यपद्यत}
{तां च तारापतिमुखीं तारां निपतितेश्वराम्}


\twolineshloka
{रामस्तु चतुरो मासान्पृष्ठे माल्यवतः शुभे}
{निवासमकरोद्धीमान्सुग्रीवेणाभ्युपस्थितः}


\twolineshloka
{रावणोऽपिपुरीं गत्वालङ्कां कामबलात्कृतः}
{सीतां निवेशयामास भवने नन्दनोपमे}


\twolineshloka
{अशोकवनिकाभ्यासे तापसास्रमसन्निभे}
{भर्तृस्मरणतन्वङ्गी तापसीवेषधारिणी}


\twolineshloka
{उपवासतपःशीला ततः सा पृथुलेक्षणा}
{उवास दुःखवसतिं फलमूलकृताशना}


\twolineshloka
{दिदेश राक्षसीस्तत्ररक्षणे राक्षसाधिपः}
{प्रासासिशूलपरशुमुद्गरालातधारिणीः}


\twolineshloka
{द्व्यक्षीं त्र्यक्षीं ललाटक्षीं दीर्घजिह्वामजिह्विकाम्}
{त्रिस्तनीमेकपादां च त्रिजटामेकलोचनाम्}


\twolineshloka
{एताश्चान्याश्च दीप्ताक्ष्यः करभोत्कटमूर्धजाः}
{परिवार्यासते सीतां दिवारात्रमतन्द्रिताः}


\twolineshloka
{तास्तु तामायतापाङ्गीं पिशाच्यो दारुणस्वराः}
{तर्जयन्ति सदा रौद्राः परुषव्यञ्जनस्वराः}


\twolineshloka
{खादाम पाटयामैनां तिलशः प्रविभज्यताम्}
{येयं भर्तारमस्माकमवमत्येह जीवति}


\twolineshloka
{इत्येवं परिभर्त्सन्तीस्त्रासयानाः पुनः पुनः}
{भर्तृशोकसमाविष्टा निःश्वस्येदमुवाच ताः}


\twolineshloka
{आर्याः खादत मां शीघ्रं न मे लोभोस्ति जीविते}
{विना तं पुण्डरीकाक्षं नीलकुञ्चितमूर्धजम्}


\twolineshloka
{अद्यैवाहं निराहारा जीवितप्रियवर्जिता}
{शोषयिष्यामि गात्राणि बल्ली तलगता यथा}


\twolineshloka
{न त्वन्यमभिगच्छेयं पुमांसं राघवादृते}
{इति जानीत सत्यं मेक्रियतां यदनन्तरम्}


\twolineshloka
{तस्यास्तद्वचनं श्रुत्वा राक्षस्यस्ताः खरस्वनाः}
{आख्यातुं राक्षसेन्द्राय जन्मुस्तत्सर्वमादितः}


\twolineshloka
{गतासु तासु सर्वासु त्रिजटा नाम राक्षसी}
{सान्त्वयामास वैदेहीं धर्मज्ञा प्रियवादिनी}


\twolineshloka
{सीते वक्ष्यामि ते किञ्चिद्विश्वासं करु मे सखि}
{भयं त्वं त्यज वामोरु शृणु चेदं वचो मम}


\twolineshloka
{अविन्ध्यो नाम मेधावी वृद्धो राक्षसपुङ्गवः}
{स रामस्य हितान्वेषी त्वदर्थे मामचूचुदत्}


\twolineshloka
{सीता मद्वचनाद्वाच्या समाश्वास्य प्रसाद्य च}
{भर्ता तेकुशली रामोलक्ष्मणानुगतो बली}


\twolineshloka
{सख्यं वानरराजेन शक्रप्रतिमतेजसा}
{कृतवान्राघवः श्रीमांस्त्वदर्थे च समुद्यतः}


\twolineshloka
{मा च ते भूद्भयं भीरु रावणाल्लोकगर्हितात्}
{नलकूबरशापेन रक्षिता ह्यसि नन्दिनि}


\twolineshloka
{शप्तो ह्येष पुरा पापो वधूं रम्भां परामृशन्}
{न शक्रोत्यवशां नारीमुपैतुमजितेन्द्रियः}


\twolineshloka
{क्षिप्रमेष्यति ते भर्ता सुग्रीवेणाभिरक्षितः}
{सौमित्रिसहितो धीमांस्त्वां चेतो मोक्षयिष्यति}


\twolineshloka
{स्वप्ना हि सुमहाघोरा दृष्टा मेऽनिष्टदर्शनाः}
{विनाशायास्य दुर्बुद्धेः पौलस्त्यस्य कुलस्य च}


\twolineshloka
{दारुणो ह्येष दुष्टात्मा क्षुद्रकर्मा निशाचरः}
{स्वभावाच्छीलदोषेण सर्वेषां भयवर्धनः}


\twolineshloka
{स्पर्धते सर्वदेवैर्यः कालोपहतचेतनः}
{मया विनासलिङ्गानि स्वप्ने दृष्टानि तस्य वै}


\twolineshloka
{तैलाभिषिक्तो विकचो मज्जनप्के दशाननः}
{असकृत्स्वरयुक्ते तु रथे नृत्यन्निव स्थितः}


\twolineshloka
{कुम्भकर्णादयश्चेमे नग्नाः पतितमूर्धजाः}
{गच्छन्ति दक्षिणामाशां रक्तमाल्यानुलेपनाः}


\twolineshloka
{श्वेतातपत्रः सोष्णीषः शुक्लमाल्यानुलेपनः}
{श्वेतपर्वतमारूढ एक एव विभीषणः}


\twolineshloka
{सचिवाश्चास्य चत्वारः शुक्लमाल्यानुलेपनाः}
{श्वेतपर्वतमारूढा मोक्ष्यन्तेऽस्मान्महाभयात्}


\twolineshloka
{रामस्यास्त्रेण पृथिवी परिक्षिप्ता ससागरा}
{यशसा पृथिवीं कृत्स्नां पूरयिष्यति ते पतिः}


\twolineshloka
{हस्तिसक्थिसमारूढो भुञ्जानो मधुपायसम्}
{लक्ष्मणश्च मया दृष्टो दिधक्षुः सर्वतो दिशम्}


\twolineshloka
{रुदती रुधिरार्द्राङ्गी व्याघ्रेण परिरक्षिता}
{असकृत्त्वं मया दृष्टा गच्छन्ती दिशमुत्तराम्}


\twolineshloka
{हर्षमेष्यसि वैदेहि क्षिप्रं भर्त्रा समन्विता}
{राघवेण सहभ्रात्रा सीते त्वमचिरादिव}


\twolineshloka
{इत्येतन्मृगशावाक्षी तच्छ्रुत्वा त्रिजटावचः}
{बभूवाशावती बाला पुनर्भर्तृसमागमे}


\twolineshloka
{तावदभ्यागता रौद्र्यः पिशाच्यस्ताःसुदारुणाः}
{ददृशुस्तां त्रिजटया सहासीनां यथापुरम्}


॥इति श्रीमन्महाभारते अरण्यपर्वणि रामोपाख्यान-पर्वणि त्रिशततमोऽध्यायः॥२८१॥

\storymeta

\dnsub{अध्यायः २८२}\resetShloka

\uvacha{मार्कण्डेय उवाच}


\twolineshloka
{ततस्तां भर्तृशोकार्तां दीनां मलिनवाससम्}
{मणिशेषाभ्यलङ्कारां रुदतीं च पतिव्रताम्}


\twolineshloka
{राक्षसीभिरुपास्यन्तीं समासीनां शिलातले}
{रावणःकामबाणार्तो ददर्शोपससर्प च}


\twolineshloka
{देवदानवगन्धर्वयक्षकिपुरुषैर्युधि}
{अजितोशोकवनिकां ययौ कन्दर्पपीडितः}


\twolineshloka
{दिव्याम्बरधरः श्रीमन्सुमृष्टमणिकुण्डलः}
{विचित्रमाल्यमुकुटो वसन्त इव मूर्तिमान्}


\twolineshloka
{न कल्पवृक्षसदृशोयत्नादपि विभूषितः}
{श्मशानचैत्यद्रुमवद्भूषितोऽपि भयङ्करः}


\twolineshloka
{स तस्यास्तनुमध्यायाः समीपे रजनीचरः}
{ददृशे रोहिणीमेत्य शनैश्चर इव ग्रैहः}


\twolineshloka
{स तामामन्त्र्य सुश्रोणीं पुष्पकेतुशराहतः}
{इदमित्यब्रवीद्वाक्यं त्रस्तां रौहीमिवाबलाम्}


\twolineshloka
{सीते पर्याप्तमेतावत्कृतोभर्तुरनुग्रहः}
{प्रसादं कुरु तन्वङ्गि क्रियतां परिकर्म ते}


\twolineshloka
{भजस्वमां वरारोहे महार्हाभरणाम्बरा}
{भवमे सर्वनारीणामुत्तमा वरवर्णिनी}


\threelineshloka
{सन्ति मे देवकन्याश्च गन्धर्वाणां च योषितः}
{सन्ति दानवकन्याश्च दैत्यानां चापि योषितः}
{तासामद्य विशालाक्षि सर्वासां मे भवोत्तमा}


\twolineshloka
{चतुर्दश पिशाचीनां कोट्यो मे वचने स्थिताः}
{द्विस्तावत्पुरुषादानां रक्षसां भीमकर्मणाम्}


\twolineshloka
{ततो मे त्रिगुणा यक्षा ये मद्वचनकारिणः}
{केचिदेव धनाध्यक्षं भ्रातरं मे समाश्रिताः}


\twolineshloka
{गन्दर्वाप्सरसो भद्रे मामापानगतं सदा}
{उपतिष्ठन्ति वामोरु यथैव भ्रातरं मम}


\twolineshloka
{पुत्रोऽहमपि विप्रर्षेः साक्षाद्विश्रवसो मुनेः}
{पञ्चमो लोकपालानामिति मे प्रथितं यशः}


\twolineshloka
{दिव्यानि भक्ष्यभोज्यानि पानानि विविधानि च}
{यथैव त्रिदशेशस्यतथैव मम भामिनि}


\twolineshloka
{क्षीयतां दुष्कृतं कर्म वनवासकृतं तव}
{भार्या मे भवसुश्रोणि यथा मण्डोदरीतथा}


\twolineshloka
{इत्युक्ता तेन वैदेही परिवृत्य सुभानना}
{तृणमन्तरतः कृत्वा तमुवाच निशाचरम्}


\twolineshloka
{अशिवेनातिवामोरूरजस्रं नेत्रवारिणा}
{स्तनावपतितौ बाला संहतावभिवर्षती}


\twolineshloka
{व्यवस्थाप्यकथञ्चित्सा विषादादतिमोहिता}
{उवाच वाक्यं तं क्षुद्रं वैदेही पतिदेवता}


\threelineshloka
{असकृद्वदतो वाक्यमीदृशं राक्षसेश्वर}
{विषादयुक्तमेतत्ते मया श्रुतमभाग्यया}
{तद्भद्रमुख भद्रं ते मानसं विनिवर्त्यताम्}


\twolineshloka
{परदाराऽस्म्यलभ्या च सततं च पतिव्रता}
{न चैवौपयिकी भार्य मानुषी तव राक्षस}


\twolineshloka
{विवशां धर्षयित्वच कां त्वं प्रीतिमवाप्स्यसि}
{न च पालयसे धर्मं लोकपालसमः कथम्}


\twolineshloka
{भ्रातरं राजराजं तं महेश्वरसस्वं प्रभुम्}
{धनेश्वरं व्यपदिशन्कथं त्विह न लज्जसे}


\twolineshloka
{इत्युक्त्वा प्रारुदत्सीता कम्पयन्ती पयोधरौ}
{शिरोधरां च तन्वङ्गी मुस्वं प्रच्छाद्यवाससा}


\twolineshloka
{तस्य रुदत्या भामित्या दीर्घा वेणी सुसयता}
{ददृशे स्वसिता स्निग्धा काली व्यालीव मूर्धनि}


\twolineshloka
{श्रुत्वा तद्रावणो वाक्यं सीतयोक्तं सुनिषुरम्}
{प्रत्याख्यातोऽपिदुर्मेधाः पुनरेवाब्रवीद्वचः}


\twolineshloka
{काममङ्गनि मे सीते दुनोतु मकरध्वजः}
{नत्वामकामां सुश्रोणीं समेप्ये चारुहासिनीम्}


\twolineshloka
{किन्नु शक्यं मया कर्तुं यत्त्वमद्यापिमानुषम्}
{आहारभूतमस्माकं राममेवानुरुध्यसे}


\twolineshloka
{इत्युक्त्वा तामनिन्द्याङ्गीं स राक्षसमहेश्वरः}
{तत्रैवान्तर्हितो भूत्वा जगामाभिमतां दिशम्}


\twolineshloka
{राक्षसीभिः परिवृतावैदेही शोककशिन्ता}
{सेव्यमाना त्रिजटया तत्रैव न्यवसत्तदा}


॥इति श्रीमन्महाभारते अरण्यपर्वणि रामोपाख्यान-पर्वणि त्रिशततमोऽध्यायः॥२८२॥

\storymeta

\dnsub{अध्यायः २८३}\resetShloka

\uvacha{मार्कण्डेय उवाच}


\twolineshloka
{राघवः सहसौमित्रिः सुग्रीवेणाभिपालितः}
{वसन्माल्यवतः पृष्ठे ददर्श विमलं नभः}


\twolineshloka
{सदृष्ट्वाविमले व्योम्नि निर्मलं शसलक्षणम्}
{ग्रहनक्षत्रताराभिरनुयान्तममित्रहा}


\twolineshloka
{कुमुदोत्पलपद्मानां गन्धमादाय वायुना}
{महीधरस्थः शीतेन सहसाप्रतिबोधितः}


\twolineshloka
{प्रभाते लक्ष्मणं वीरमभ्यभाषत दुर्मनाः}
{सीतां संस्मृत् यधर्मात्मा रुद्धां राक्षसवेश्मनि}


\twolineshloka
{गच्छ लक्ष्मण जानीहि किष्किन्दायां कपीश्वरम्}
{प्रमत्तं ग्राम्यधर्मेषु कृतघ्नं स्वार्थपण्डितम्}


\twolineshloka
{योसौ कुलाधमो मूढो मया राज्येऽभिषेचितः}
{सर्ववानरगोपुच्छा यमृक्षाश्च भजन्ति वै}


\twolineshloka
{यदर्थं निहतो बाली मया रघुकुलोद्वह}
{त्वया सहमहाबाहो किष्किन्धोपवने तदा}


\twolineshloka
{कृतघ्नं तमहं मन्ये वानरापशदं भुवि}
{यो मामेवङ्गतो मूढो न जानीतेऽद्य लक्ष्मण}


\twolineshloka
{असौ मन्ये न जानीते समयप्रतिपालनम्}
{कृतोपकारं मां नूनमवमत्याल्पया धिया}


\twolineshloka
{यदितावदनुद्युक्तः शेते कामसुखात्मकः}
{नेतव्यो वालिमार्गेण सर्वभूतगतिं त्वया}


\twolineshloka
{अथापि घटतेऽस्माकमर्ते वानरपुङ्गवः}
{तमादायैव काकुत्स्थ त्वरावान्भव माचिरम्}


\twolineshloka
{इत्युक्तो लक्ष्मणो भ्रात्रा गुरुवाक्यहिते रतः}
{प्रतस्थे रुचिरं गृह्य समार्गणगुणं धनुः}


\twolineshloka
{किष्किन्धाद्वारमासाद्यप्रविवेशानिवारितः}
{सक्रोध इतितं मत्वाराजा प्रत्युद्ययौ हरिः}


\twolineshloka
{तं सदारोविनीतात्मा सुग्रीवः प्लवगाधिपः}
{पूजया प्रतिजग्राह प्रीयमाणस्तदर्हया}


\twolineshloka
{तमब्रवीद्रामवचः सौमित्रिरकुतोभयः}
{स तत्सर्वमशेषेण श्रुत्वा प्रह्वः कृताञ्जलिः}


\twolineshloka
{सभृत्यदारो राजेन्द्रसुग्रीवो वानराधिपः}
{इदमाह वचः प्रीतो लक्ष्मणं नरकुञ्जरम्}


\twolineshloka
{नास्मि लक्ष्मण दुर्मेधा नाकृतज्ञो न निर्घृणः}
{श्रूयतां यः प्रयत्नो मे सीतापर्येषणे कृतः}


\twolineshloka
{दिशः प्रस्थापिताः सर्वेविनीता हरयो मया}
{सर्वेषां च कृतः कालो मासेऽभ्यागमने पुनः}


\twolineshloka
{यैरियं सवना साद्रिः सपुरा सागराम्बरा}
{विचेतव्या मही वीर सग्रामनगराकरा}


\twolineshloka
{स मासः पञ्चरात्रेण पूर्णो भवितुमर्हति}
{ततः श्रोष्यसि रामेण सहितः सुमहत्प्रियम्}


\twolineshloka
{इत्युक्तो लक्ष्मणस्तेन वानरेन्द्रेण धीमता}
{त्यक्त्वा रोषमदीनात्मा सुग्रीवं प्रत्यपूजयत्}


\twolineshloka
{सरामं सहसुग्रीवो माल्यवत्पुष्ठमास्थितम्}
{अभिगम्योदयं तस्य कार्यस्य प्रत्यवेदयत्}


\twolineshloka
{इत्येवं वानरेन्द्रास्ते समाजग्मुः सहस्रशः}
{दिशस्तिस्रो विचित्याथ न तु ये दक्षिणां गताः}


\twolineshloka
{आचख्युस्तत्र रामाय महीं सागरमेखलाम्}
{विचितां न तु वैदेह्या दर्शनं रावणस्य वा}


\twolineshloka
{गतास्तु दक्षिणामाशां ये वै वानरपुङ्गवाः}
{आशावांस्तेषु काकुत्स्थः प्राणानार्तोऽभ्यधारयत्}


\twolineshloka
{द्विमासोपरमे काले व्यतीते प्लवगास्ततः}
{सुग्रीवमभिगम्येदं त्वरिता वाक्यमब्रुवन्}


\twolineshloka
{रक्षितंवालिना यत्तत्स्फीतं मधुवनं महत्}
{त्वया च प्लवगश्रेष्ठ तद्भुङ्क्ते पवनात्मजः}


\twolineshloka
{वालिपुत्रोऽङ्गदश्चैव ये चान्ये प्लवगर्षभाः}
{विचेतुं दक्षिणामाशां राजन्प्रस्थापितास्त्वया}


\twolineshloka
{तेषामपनयं श्रुत्वा मेने सकृतकृत्यताम्}
{कृतार्थानां हि भृत्यानामेतद्भवति चेष्टितम्}


\twolineshloka
{स तद्रामाय मेधावी शशंस प्लवगर्षभः}
{रामश्चाप्यनुमानेन मेने दृष्टां तु मैथिलीम्}


\twolineshloka
{हनुमत्प्रमुखाश्चापि विश्रान्तास्ते प्लवङ्गमाः}
{अभिजग्मुर्हरीन्द्रं तं रामलक्ष्मणसन्निधौ}


\twolineshloka
{गतिं च मुखवर्णं च दृष्ट्वारामो हनूमतः}
{अगमत्प्रत्ययं भूयो दृष्टा सीतेति भारत}


\twolineshloka
{हनूमत्प्रमुखास्ते तु वानराः पूर्णमानसाः}
{प्रणेमुर्विधिवद्रामं सुग्रीवं लक्ष्मणं तथा}


\twolineshloka
{तानुवाचानतान्रामः प्रगृह्य सशरं धनुः}
{अपि मां जीवयिष्यध्वमपि वः कृतकृत्यता}


\twolineshloka
{अपि राज्यमयोध्यायां कारयिष्याम्यहं पुनः}
{निहत्यसमरे शत्रूनाहृत्यजनकात्मजाम्}


\twolineshloka
{अमोक्षयित्वावैदेहीमहत्वा च रणे रिपून्}
{हृतदारोऽवधूतश्चनाहं जीवितुमुत्सहे}


\twolineshloka
{इत्युक्तवचनं रामं प्रत्युवाचानिलात्मजः}
{प्रियमाख्यामि ते राम दृष्टा सा जानकी मया}


\twolineshloka
{विचित्य दक्षिणामाशां सपर्वतवनाकराम्}
{श्रान्ताः काले व्यतीते स्म दृष्टवन्तो महागुहाम्}


\twolineshloka
{प्रविशामो वयं तां तु बहुयोजनमायताम्}
{अन्धकारां सुविपिनां गहनां कीटसेविताम्}


\twolineshloka
{गत्वा सुमहदध्वानमादित्यस्य प्रभां ततः}
{दृष्टवन्तः स्म तत्रैवभवनं दिव्यमन्तरा}


\twolineshloka
{मयस्य किल दैत्यस्य तदा सद्वेश्म राघव}
{तत्र प्रभावती नाम तपोऽतप्यत तापसी}


\twolineshloka
{तया दत्तानि भोज्यानि पानानि विविधानि च}
{भुक्त्वा लब्धबलाः सन्तस्तयोक्तेन पथा ततः}


\twolineshloka
{निर्याय तस्मादुद्देशात्पश्यामो लवणाम्भसः}
{समीपे सह्यमलयौ दर्दुरं च महागिरिम्}


\twolineshloka
{ततो मलयमारुह्य पश्यन्तो वरुणालयम्}
{विषण्णा व्यथिताः खिन्ना निराशा जीविते भृशम्}


\twolineshloka
{अनेकशतविस्तीर्णं योजनानां महोदधिम्}
{तिमिनक्रझषावासं चिन्तयन्तः सुदुःखिताः}


\twolineshloka
{तत्रानशनसङ्कल्पं कृत्वाऽऽसीना वयं तदा}
{ततः कथान्ते गृध्रस्य जटायोरभवत्कथा}


\twolineshloka
{ततः पर्वतशृङ्गाभं घोररूपं भयावहम्}
{पक्षिणं दृष्टवन्तः स्म वैनतियेमिवापरम्}


\twolineshloka
{सोऽस्मानतर्कयद्भोक्तुमथाभ्येत्य वचोऽब्रवीत्}
{भोः क एष मम भ्रातुर्जटायोः कुरुते कथाम्}


\twolineshloka
{सपातिर्नाम तस्याहं ज्येष्ठो भ्राता खगाधिपः}
{अन्योन्यस्पर्धया रूढावावामदित्यसत्पदम्}


\twolineshloka
{ततो दग्धाविमौ पक्षौ न दग्धौ तु जटायुषः}
{तस्मान्मे चिरदृष्टः स भ्राता गृध्रपतः प्रियः}


\twolineshloka
{निर्दग्धपक्षः पतितो ह्यहमस्मिन्महागिरौ}
{द्रष्टुं वीरं न शक्नोमि भ्रातरं वै जटायुषम्}


\twolineshloka
{तस्यैवं वदतोऽस्माभिर्हतो भ्राता निवेदितः}
{व्यसनं भवतश्चेदं सङ्क्षेपाद्वै निवेदितम्}


\twolineshloka
{स सम्पातिस्तदा राजञ्श्रुत्वासुमहदप्रियम्}
{विषण्णचेताः पप्रच्छ पुनरस्मानरिन्दम}


\twolineshloka
{कः सरामः कथं सीता जटायुश्च कथं हतः}
{इच्छामि सर्वमेवैतच्छ्रोतुं प्लवगसत्तमाः}


\twolineshloka
{तस्याहं सर्वमेवैतद्भवतो व्यसनागमम्}
{प्रायोपवेशने चैवहेतुं विस्तरशोऽब्रुवम्}


\twolineshloka
{सोऽस्मानाश्वासयामास वाक्येनानेन पक्षिराट्}
{रावणो विदितो मह्यं लङ्का चास्य महापुरी}


\twolineshloka
{दृष्टापारे समुद्रस्य त्रिकूटगिरिकन्दरे}
{भवित्री तत्र वैदेही न मेऽस्त्यत्र विचारणा}


\twolineshloka
{इतितस्य वचः श्रुत्वा वयमुत्थाय सत्वराः}
{सागरक्रमणे मन्त्रं मन्त्रयामः परन्तप}


\threelineshloka
{नाध्यवास्यद्यदा कश्चित्सागरस्य विलङ्घनम्}
{ततः पितरमाविश्य पुप्लुवेऽहमहार्णवम्}
{शतयोजनविस्तीर्णं निहत्य जलराक्षसीम्}


\twolineshloka
{उपवासतपःशीला भर्तृदर्शनलालसा}
{जटिला मलदिग्धाङ्गीकृश दीना तपस्विन}


\twolineshloka
{निमित्तैस्तामहं सीतामुपलभ्य पृथग्विधैः}
{उपसृत्याब्रवं चार्यामभिगम्य रहोगताम्}


\twolineshloka
{सीते रामस्य दूतोऽहंवानरोमारुतात्मजः}
{त्वद्दर्शनमभिप्रप्सुरिह प्राप्तो विहायसा}


\twolineshloka
{राजपुत्रौ कुशलिनौ भ्रातरौ रामलक्ष्मणौ}
{सर्वशाखामृगेन्द्रेण सुग्रीवेणाभिपालितौ}


\twolineshloka
{कुशलन्त्वाब्रवीद्रामःसीते सौमित्रिणा सह}
{सखिभावाच्च सुग्रीवः कुशलं त्वाऽनुपृच्छति}


\twolineshloka
{क्षिप्रमेष्यति ते भर्ता सर्वशाखामृगैः सह}
{प्रत्ययं कुरु मे देवि वानरोऽस्मि न राक्षसः}


\twolineshloka
{मुहूर्तमिवच ध्यात्वा सीता मां प्रत्युवाच ह}
{अवैमि त्वांहनूमन्तमविन्ध्यवचनादहम्}


\twolineshloka
{अविन्ध्यो हि महाबाहो राक्षसो वृद्धसमतः}
{कथितस्तेन सुग्रीवस्त्वद्विधैः सचिवैर्वृतः}


\twolineshloka
{गम्यतामिति चोक्त्वा मां सीता पादादिमं मणिम्}
{घारिता येन वैदेही कालमेतमनिन्दिता}


\twolineshloka
{प्रत्ययार्थं कथां चेमां कथयामास जानकी}
{क्षिप्तामिषीकां काकाय चित्रकूटे महागिरौ}


\twolineshloka
{भवता पुरुषव्याघ्र प्रत्यभिज्ञानकारणात्}
{एकाक्षिविकलः काकः सुदुष्टात्मा कृतश्चवै}


\twolineshloka
{ग्राहयित्वाऽहमात्मानं ततो दग्ध्वाच तां पुरीम्}
{सप्राप्त इतितं रामः प्रियवादिनमार्चयत्}


॥इति श्रीमन्महाभारते अरण्यपर्वणि रामोपाख्यान-पर्वणि त्रिशततमोऽध्यायः॥२८३॥

\storymeta

\dnsub{अध्यायः २८४}\resetShloka

\uvacha{मार्कण्डेय उवाच}


\twolineshloka
{ततस्तत्रैवरामस्य समासीनस्य तैः सह}
{समाजग्मुः कपिश्रेष्ठाः सुग्रीववचनात्तदा}


\twolineshloka
{वृतः कोटिसहस्रेण वानराणां तरस्विनाम्}
{श्वशुरो वालिनः श्रीमान्सुषेणो राममभ्ययात्}


\twolineshloka
{कोटीशतवृतोवाऽपिगजो गवय एव च}
{वानरेन्द्रौ महावीर्यौ पृथक्पृथगदृश्यताम्}


\twolineshloka
{षष्टिकोटिसहस्राणि प्रकर्षन्प्रत्यदृश्यत}
{गोलाङ्गूलो महाराज गवाक्षो भीमदर्शनः}


\twolineshloka
{गन्धमादनवासी तु प्रथितो गन्धमादनः}
{कोटीशतसहस्राणि हरीणां समकर्षत}


\twolineshloka
{पनसो नाम मेधावी वानरःसुमहाबलः}
{कोटीर्दश द्वादश च त्रिंशत्पञ्च प्रकर्षति}


\twolineshloka
{श्रीमान्दधिमुखो नाम हरिवृद्धोऽतिवीर्यवान्}
{प्रचकर्ष महासैन्यं हरीणां भीमतेजसाम्}


\twolineshloka
{कृष्णानां मुखपुण्ड्राणामृक्षाणां भीमकर्मणाम्}
{कोटीर्दश द्वादश च त्रिंशत्पञ्च प्रकर्षति}


\twolineshloka
{एते चान्ये च बहवो हरियूथपयूथपाः}
{असङ्ख्येया महाराज समीयू रामकारणात्}


\twolineshloka
{गिरिकूटनिभाङ्गानां सिंहानामिव गर्जताम्}
{श्रूयते तुमुलः शब्दस्तत्रतत्रप्रधावताम्}


\twolineshloka
{गिरिकूटनिभाः क्नचित्केचिन्महिषसन्निभाः}
{शरदभ्रप्रतीकाशाः केचिद्धिङ्गुलकाननाः}


\twolineshloka
{उत्पतन्तः पतन्तश्च प्लवमानाश्च वानराः}
{उद्धुन्वन्तोऽपरे रेणून्समाजग्मुः समन्ततः}


\twolineshloka
{सवानरमहासैन्यः पूर्णसागरसन्निभः}
{निवेशमकरोत्तत्रसुग्रीवानुमते तदा}


\twolineshloka
{ततस्तेषु हरीन्द्रेषु समावृत्तेषु सर्वशः}
{तिथौ प्रशस्ते नक्षत्रे मुहूर्ते चाभिपूजिते}


\twolineshloka
{तेन व्यूढेन सैन्येन लोकानुद्वर्तयन्निव}
{प्रययौ राघवः श्रीमान्सुग्रीवसहितस्तदा}


\twolineshloka
{मुखमासीत्तु सैन्यस्य हनूमान्मारुतात्मजः}
{जघनं पालयामास सौमित्रिरकुतोभयः}


\twolineshloka
{बद्धगोधाङ्गुलित्रणौ राघवौ तत्रजग्मतुः}
{वृतौ हरिमहामात्रैश्चन्द्रसूर्यौ ग्रहैरिव}


\twolineshloka
{प्रबभौ हरिसैन्यं तत्सालतालशिलायुधम्}
{सुमहच्छालिभवनं यथा सूर्योदयं प्रति}


\twolineshloka
{नलनीलाङ्गदक्राथमैन्दद्विविदपालिता}
{ययौ सुमहती सेना राघवस्यार्थसिद्धये}


\twolineshloka
{विविधेषु प्रशस्तेषु बहुमूलफलेषु च}
{प्रभूतमधुमांसेषु वारिमत्सु विवेषु च}


\twolineshloka
{निवसन्ती निराबाधा तथैवगिरिसानुषु}
{उपायाद्धिरिसेना सा क्षारोदमथ मागरम्}


\twolineshloka
{द्वितीयसागरनिमं तद्बलबहुलध्वजम्}
{वेलावनं समासाद्य निवासमकरोत्तदा}


\twolineshloka
{ततो दाशरथिः श्रीमान्सुग्रीवं प्रत्यभाषत}
{मध्ये वानरमुख्यानां प्राप्तकालमिदं वचः}


\twolineshloka
{उपायः कोनु भवतां मतः सागरलङ्घने}
{इयं हि महती सेना सागरश्चातिदुस्तरः}


\twolineshloka
{तत्रान्ये व्याहरन्ति स्म वानराः पटुमानिनः}
{समर्था लङ्घने सिन्दोर्न तत्कृत्स्नस्य वानराः}


\twolineshloka
{केचिन्नौभिर्व्यवस्यन्ति केचिच्च विविधैः प्लवैः}
{नेति रामस्तु तान्सर्वान्सान्त्वयन्प्रत्यभाषत}


\twolineshloka
{शतयोजनविस्तारं न शक्ताः सर्ववानराः}
{क्रान्तुं तोयनिधिं वीरानैषा वो नैष्ठिकी मतिः}


\twolineshloka
{नावो न सन्ति सेनाया बह्व्यस्तारयितुं तथा}
{वणिजामुपघातं च कथमस्मद्विधश्चरेत्}


\twolineshloka
{विस्तीर्णं चैव नः सैन्यं हन्याच्छिद्रेण वै परः}
{प्लवोडुपप्रतारश्च नैवात्र मम रोचते}


\twolineshloka
{अहं त्विमं जलनिधिं समारप्स्याम्युपायतः}
{प्रतिशेष्याम्युपवसन्दर्शयिष्यति मां ततः}


\twolineshloka
{न चेद्दर्शयिता मार्गं धक्ष्याम्यनमहं ततः}
{महास्त्रैरप्रतिहतैरत्यग्निपवनोज्ज्वलैः}


\twolineshloka
{इत्युक्त्वा सहसौमित्रिरुपस्पृश्याथ राघवः}
{प्रतिशिस्ये जलनिधं विधिवत्कुशसंस्तरे}


\twolineshloka
{सागरस्तु ततः स्वप्ने दर्शयामास राघवम्}
{देवो नदनदीमर्ता श्रीमान्यादोगणैर्वृतः}


\twolineshloka
{कौसल्यामातरित्येवमाभाष्य मधुरं वचः}
{इदमित्याह रत्नानामाकरैः शतशो वृतः}


\threelineshloka
{ब्रूहि किं ते करोम्यत्र साहाय्यं पुरुषर्षभ}
{ऐक्ष्वाको ह्यस्मि ते ज्ञाती राम सत्यपराक्रम}
{एवमुक्तः समुद्रेण रामो वाक्यमथाब्रवीत्}


\threelineshloka
{मार्गमिच्छामि सैन्यस्य दत्तं नदनदीपते}
{येन गत्वा दशग्रीवं हन्यामि कुलपांसनम्}
{राक्षसं सानुबन्धं तं मम भार्यापहारिणम्}


\twolineshloka
{यद्येवं याचतो मार्गं न प्रदास्यति मे भवान्}
{शरैस्त्वां शोषयिष्यामि दिव्यास्त्रयतिमन्त्रितैः}


\twolineshloka
{इत्येवं ब्रुवतः श्रुत्वा रामस्य वरुणालयः}
{उवाचव्यथितोवाक्यमितिबद्धाञ्जलिःस्थितः}


\twolineshloka
{नेच्छामि प्रतिघातं ते नास्मि विघ्नकरस्तव}
{शृणु चेदं वचोराम श्रुत्वा कर्तव्यमाचर}


\twolineshloka
{यदि दास्यामि ते मार्गं सैन्यस्य व्रजतोऽऽज्ञया}
{अन्येऽप्याज्ञापयिष्यन्ति मामेवं धनुषोबलात्}


\twolineshloka
{अस्तित्वत्रनलो नाम वानरः शिल्पिसमतः}
{त्वष्टुः काकुत्स्थ तनयो बलवान्विश्वकर्मणः}


\twolineshloka
{स यत्काष्ठं तृणं वाऽपिशिलां वा क्षेप्स्यते मयि}
{सर्वं तद्धारयिष्यामि स ते सेतुर्भविष्यति}


\twolineshloka
{इत्युक्त्वाऽन्तर्हिते तस्मिन्रामो नलमुवाच ह}
{कुरु सेतुं समुद्रे त्वंशक्तो ह्यसि मतो मम}


\twolineshloka
{तेनोपायेन काकुत्स्थः सतुबन्धमकारयत्}
{दशयोजनविस्तारमायतं शतयोजनम्}


\twolineshloka
{नलसेतुरिति ख्यातो योऽद्यापि प्रथितो भुवि}
{रामस्याज्ञां पुरस्कृत्य धार्यते गिरिसन्निभः}


\twolineshloka
{तत्रस्थं स तु धर्मात्मा समागच्चद्विभीषणः}
{भ्राता वै राक्षसेन्द्रस्य चतुर्भिः सचिवैः सह}


\twolineshloka
{प्रतिजग्राह रामस्तं स्वागतेन महामनाः}
{सुग्रीवस्य तु शङ्काऽभूत्प्रणिधिः स्यादिति स्मह}


\twolineshloka
{राघवः सत्यचेष्टाभिः सम्यक्व चरितेङ्गितैः}
{यदा तत्त्वेन तुष्टोऽभूत्तत एनमपूजयत्}


\twolineshloka
{सर्वराक्षसराज्येचाप्यभ्यपिञ्चद्विभीषणम्}
{चक्रे च मन्त्रसचिवं सहृदं लक्ष्मणस्य च}


\twolineshloka
{विभीषणमते चैव सोऽत्यक्रामन्महार्णवम्}
{ससैन्यः सेतुना तेन मार्गेणैव नराधिपः}


\twolineshloka
{ततो गत्वासमासाद्य लङ्कोद्यानान्यनेकशः}
{भेदयामास कपिभिर्महान्ति च बहूनि च}


\twolineshloka
{तत्रास्तां रावणामात्यौ राक्षसौ शुकसारणौ}
{चरौ वानररूपेण तौ जग्राह विभीषणः}


\twolineshloka
{प्रतिपन्नौ यदा रूपं राक्षसं तौ निशाचरौ}
{दर्शयित्वा ततः सैन्यं रामः पश्चादवासृजत्}


\twolineshloka
{निवेश्योपवने सैन्यं स शूरः प्राज्यवानरम्}
{प्रेषयामास दुत्येन रावणस्य ततोऽङ्गदम्}


॥इति श्रीमन्महाभारते अरण्यपर्वणि रामोपाख्यान-पर्वणि त्रिशततमोऽध्यायः॥२८४॥

\storymeta

\dnsub{अध्यायः २८५}\resetShloka

\uvacha{मार्कण्डेय उवाच}


\twolineshloka
{प्रभूतान्नोदकेतस्मिन्बहुमूलफले वने}
{सेनां निवेश्य काकुत्स्थो विधिवत्पर्यरक्षत}


\twolineshloka
{रावणः संविधं चक्रे लङ्कायां शास्त्रनिर्मिताम्}
{प्रकृत्यैवदुराधर्षा दृढप्राकारतोरणा}


\twolineshloka
{अगाधतोयाः परिखा मीननक्रसमाकुलाः}
{बभूवुः सप्त दुर्धर्षाः स्वादिरैः शङ्कुभिश्चिताः}


\twolineshloka
{कर्णाटयन्त्रा दुर्धर्षा बभूवुः सहुडोपलाः}
{साशीविषघटायोधाः ससर्जरसपांसवः}


\twolineshloka
{मुसलालातनाराचतोमरासिपरश्वथैः}
{अन्विताश्चशतघ्नीभिः समधूच्छिष्टमुद्गराः}


\twolineshloka
{पुरद्वारेषु सर्वेषु गुल्माः स्थावरजङ्गमाः}
{बभूवुः पत्तिबहुलाः प्रभूतगजवाजिनः}


\twolineshloka
{अङ्गदस्त्वथ लङ्कायां द्वारदेशमुपागतः}
{विदितो राक्षसेन्द्रस्य प्रविवेशगतव्यथः}


\twolineshloka
{मध्ये राक्षसकोटीनां बह्वीनां सुमहाबलः}
{शुशुभे मेघमालाभिरादित्य इव संवृतः}


\twolineshloka
{ससमासाद्य पौलस्त्यममात्यैरभिसंवृतम्}
{रामसन्देशमामन्त्र्य वाग्मी वक्तुं प्रचक्रमे}


\twolineshloka
{आह त्वां राघवो राजन्कोसलेन्द्रो महायशाः}
{प्राप्तकालमिदं वाक्यं तदादत्स्व सुदुर्मते}


\twolineshloka
{अकृतात्मानमासाद्य राजानमनये रतम्}
{विनश्यन्त्यनयाविष्टा देशाश्च नगराणि च}


\twolineshloka
{त्वयैकेनापराद्धं मे सीतामाहरता बलात्}
{वधायानपराद्धानामन्येषां तद्भविष्यति}


\twolineshloka
{ये त्वया बलदर्पाभ्यामाविष्टेन वनेचराः}
{ऋषयोहिंसिताः पूर्वन्देवाश्चाप्यवमानिताः}


\twolineshloka
{राजर्षयश्च निहता रुदत्यश्चाहृताः स्त्रियः}
{तदिदं समनुप्राप्तं फलन्तस्यानयस्य ते}


\twolineshloka
{हन्तास्मि त्वां सहामात्यैर्युध्यस्व पुरुषो भव}
{पश्य मे धनुषो वीर्यं मानुषस्य निशाचर}


\twolineshloka
{मुच्यतां जानकी सीता न मे मोक्ष्यसि कर्हिचित्}
{अराक्षसमिमं लोकङ्कर्ताऽस्मि निशितैः शरैः}


\twolineshloka
{इति तस्य ब्रुवाणस्य दूतस्य परुषं वचः}
{श्रुत्वा न ममृषे राजा रावणः क्रोधमूर्च्छितः}


\twolineshloka
{इङ्गितज्ञास्ततो भर्तुश्चत्वारो रजनीचराः}
{चतुर्ष्वङ्गेषु जगृहुः शार्दूलमिव पक्षिणः}


\twolineshloka
{तांस्तथाङ्गेषु संसक्तानङ्गदो रजनीचरान्}
{आदायैव खमुत्पत्य प्रासादतलमाविशत्}


\twolineshloka
{वेगेनोत्पततस्तस्य पेतुस्ते रजनीचराः}
{भुवि सभिन्नहृदयाः प्रहारवरपीडिताः}


\twolineshloka
{संसक्तोहर्म्यशिखरात्तस्मात्पुनरवापतत्}
{लङ्घयित्वा पुरं लङ्कां सुवेलस्य समीपतः}


\twolineshloka
{कोसलेन्द्रमथागम्य सर्वमावेद्य वानरः}
{विशश्राम स तेजस्वी राघवेणाभिनन्दितः}


\twolineshloka
{ततः सर्वाभिसारेण हरीणां वातरंहसाम्}
{भेदयामास लङ्कायाः ग्राकारं रघुनन्दनः}


\twolineshloka
{विभीषणर्क्षाधिपती पुरस्कृत्याथ लक्ष्मणः}
{दक्षिणं नगरद्वारमवामृद्गाद्दुरासदम्}


\twolineshloka
{करभारुणगात्राणां हरीणां युद्धशालिनाम्}
{कोटीशतसहस्रेण लङ्कामभ्यपतत्तदा}


\twolineshloka
{प्रलम्बबाहूरुकरजङ्घान्तरविलम्बिनाम्}
{ऋक्षाणां धूम्रवर्णानां तिस्रः कोठ्यो व्यवस्थिताः}


\twolineshloka
{उत्पतद्भिः पतद्भिश्च निपतद्भिश्च वानरैः}
{नादृश्यत तदा सूर्यो रजसा नाशितप्रभः}


\twolineshloka
{शालिप्रसूनसदृशैः शिरीपकुसुमप्रभैः}
{तरुणादित्यसदृशैः शणगौरैश्च वैनरैः}


\twolineshloka
{प्राकारं ददृशुस्ते तु समन्तात्कपिलीकृतम्}
{राक्षसा विस्मिता राजन्सस्त्रीवृद्धाः समन्ततः}


\twolineshloka
{बिभिदुस्ते मणिस्तम्भान्कर्णाट्टशिखराणि च}
{भग्नोन्मथितशृङ्गाणि यन्त्राणि च विचिक्षिपुः}


\twolineshloka
{परिगृह्य शतघ्नीश्च सचक्राः सगुडोपलाः}
{चिक्षिपुर्भुजवेगेन लङ्कामध्येमहास्वनाः}


\twolineshloka
{प्राकारस्थाश्चये केचिन्निशाचरगणास्तथा}
{प्रदुद्रुवुस्ते शतशः कपिभिः समभिद्रुताः}


\twolineshloka
{ततस्तु राजवचनाद्राक्षसाः कामरूपिणः}
{निर्ययुर्विकृताकाराः सहस्रशतसङ्घशः}


\twolineshloka
{शखवर्षाणि वर्षन्तो द्रावयित्वा वनौकसः}
{प्राकारं शोभयन्तस्ते परं विस्मयमास्थिताः}


\twolineshloka
{स मापराशिसदृशैर्बभूव क्षणादाचरैः}
{कृतो निर्वानरो भूयः प्राकारो भीमदर्शनैः}


\twolineshloka
{पेतुः शलविभिन्नाङ्गा बहवो वानरर्पभाः}
{स्तम्भतोरणभग्नाश्चपेतुस्तत्रनिशाचराः}


\twolineshloka
{केशाकेश्यभवद्युद्धं रक्षसां वानरैः सह}
{नखैर्दन्तैश्च वीराणां खादतां वै परस्परम्}


\twolineshloka
{निष्टनन्तो ह्युभयतस्तत्र वानरराक्षसाः}
{हतानिपतिता भूमौ न मुञ्चन्ति परस्परम्}


\twolineshloka
{रामस्तु शरजालानिववर्ष जलदो यथा}
{तानिलङ्कां समासाद्य जघ्रुस्तान्रजनीचरान्}


\twolineshloka
{सौमित्रिरपि नाराचैर्दृढधन्वा जितक्लमः}
{आदिश्यादिश्य दुर्गस्थान्पातयामास राक्षसान्}


\twolineshloka
{ततः प्रत्यवहारोऽभूत्सैन्यानां राधवाज्ञया}
{कृते विमर्दे लङ्कायां लब्धलक्ष्योजयोत्तरः}


॥इति श्रीमन्महाभारते अरण्यपर्वणि रामोपाख्यान-पर्वणि त्रिशततमोऽध्यायः॥२८५॥

\storymeta

\dnsub{अध्यायः २८६}\resetShloka

\uvacha{मार्कण्डेय उवाच}


\twolineshloka
{ततो निविशमानांस्तान्सैनिकान्रावणानुगाः}
{अभिजग्मुर्गणाऽनके पिशाचक्षुद्ररक्षसाम्}


\twolineshloka
{पर्वणः पतनो जम्भः खरः क्रोधवशो हरिः}
{प्ररुजश्चारुजश्चैव प्रघसश्चैवमादयः}


\twolineshloka
{ततोऽभिपततां तेषामदृश्यानां दुरात्मनाम्}
{अन्तर्धानवधं तज्ज्ञश्चकार स विभीषणः}


\twolineshloka
{ते दृश्यमाना हरिभिर्बलिभिर्दूरपातिभिः}
{निहताः सर्वशो राजन्महीं जग्मुर्गतासवः}


\twolineshloka
{अमृष्यमाणः सबलो रावणो निर्ययावथ}
{राक्षसानां बलैर्घोरैः पिशाचानाञ्च संवृतः}


\twolineshloka
{युद्धशास्त्रविधानज्ञ उशना इव चापरः}
{व्यूह्यचौशनसं व्यूहं हरीनभ्यवहारयत्}


\twolineshloka
{राघवस्तु विनिर्यान्तं व्यूढानीकं दशाननम्}
{बार्हस्पत्यं विधं कृत्वा प्रतिव्यूह्य ह्यदृश्यत}


\twolineshloka
{समेत्य युयुधे तत्र ततो रामेण रावणः}
{युयुधे लक्ष्मणश्चापि तथैवेन्द्रजिता सह}


\twolineshloka
{विरूपाक्षेण सुग्रीवस्तारेण च निस्वर्वटः}
{पौण्ड्रेण च नलस्तत्र पदुशः पनसेन च}


\twolineshloka
{विषह्यं यं हि यो मेने स स तेन समेयिवान्}
{युयुधे युद्धवेलायां स्वबाहुबलमाश्रितः}


\twolineshloka
{स सप्रहारो ववृधे भीरूणां भयवर्धनः}
{रोमसंहर्षणो घोरः पुरा देवासुरे यथा}


\twolineshloka
{रावणो राममानर्च्छच्छक्तिशूलासिवृष्टिभिः}
{निशितैरायसैस्तीक्ष्णै रावणं चापि राघवः}


\twolineshloka
{तथैवेन्द्रजितं यत्तं लक्ष्मणो मर्मभेदिभिः}
{इन्द्रजिच्चापि सौमित्रिं बिभेद बहुभिः शरैः}


\twolineshloka
{विभीषणः प्रहस्तं च प्रहस्तश्च विभीषणम्}
{खगपत्रैः शरैस्तीक्ष्णैरभ्यवर्षद्गतव्यथः}


\twolineshloka
{तेषां बलवतामासीन्महास्त्राणां समागमः}
{विव्यथुः सकला येन त्रयो लोकाश्चराचराः}


॥इति श्रीमन्महाभारते अरण्यपर्वणि रामोपाख्यान-पर्वणि त्रिशततमोऽध्यायः॥२८६॥

\storymeta

\dnsub{अध्यायः २८७}\resetShloka

\uvacha{मार्कण्डेय उवाच}


\twolineshloka
{ततः प्रहस्तः सहसा समभ्येत्य विभीषणम्}
{गदया ताडयामास विनद्य रणकर्कशम्}


\twolineshloka
{स तयाऽभिहतो धीमान्गदया भीमवेगया}
{नाकम्पत महाबाहुर्हिमवानिव सुस्थिरः}


\twolineshloka
{ततः प्रगृह्यविपुलां शतघण्टां विभीषणः}
{अनुमन्त्र्य महाशक्तिं चिक्षेपास्य शिरः प्रति}


\twolineshloka
{पतन्त्या स तया वेगाद्राक्षसोऽशनिवेगया}
{हृतोत्तामङ्गो ददृशे वातरुग्ण इव द्रुमः}


\twolineshloka
{तं दृष्ट्वा निहतं सङ्ख्ये प्रहस्तं क्षणदाचरम्}
{अभिदुद्राव धूम्राक्षो वेगेन महता कपीन्}


\twolineshloka
{तस्य मेघोपमं सैन्यमापतद्भीमदर्शनम्}
{दृष्ट्वैव सहसा दीर्णा रणे वानरपुङ्गवाः}


\twolineshloka
{ततस्तान्सहसा दीर्णान्दृष्ट्वा वानरपुङ्गवान्}
{निर्ययौ कपिशार्दूलो हनूमान्मारुतात्मजः}


\twolineshloka
{तं दृष्ट्वाऽवस्थितं सङ्ख्ये हरयः पवनात्मजम्}
{महत्या त्वरया राजत्सन्न्यवर्तन्त सर्वशः}


\twolineshloka
{ततः शब्दो महानासीत्तुमुलो रोमहर्षणः}
{रामरावणसैन्यानामन्योन्यमभिधावताम्}


\twolineshloka
{तस्मिन्प्रवृत्ते सङ्ग्रामे घोरे रुधिरकर्दमे}
{क्षूम्राक्षः कपिसैन्यं तद्द्रावयामास पत्रिभिः}


\twolineshloka
{तं स रक्षोमहामात्रमापतन्तं सपत्नजित्}
{प्रतिजग्राह हनुमांस्तरसा पवनात्मजः}


\twolineshloka
{तयोर्युद्धमभूद्घोरं हरिराक्षसवीरयोः}
{जिगीषतोर्युधाऽन्योन्यमिन्द्रप्रह्लादयोरिव}


\twolineshloka
{गदाभिः परिघैश्चैव राक्षसो जघ्निवान्कपिम्}
{कपिश्च जघ्निवान्रक्षः सस्कन्धविटपैर्द्रुमैः}


\twolineshloka
{ततस्तमतिकोपेन साश्वं सरथसारथिम्}
{धूम्राक्षमवधीत्क्रुद्धो हनूमान्मारुतात्मजः}


\twolineshloka
{ततस्तं निहतं दृष्ट्वा धूम्राक्षं राक्षसोत्तमम्}
{हरयो जातविश्रम्भा जघ्नुरन्ये च सैनिकान्}


\twolineshloka
{ते वध्यमाना हरिभिर्बलिभिर्जितकाशिभिः}
{राक्षसा भग्नसङ्कल्पा लङ्कामभ्यपतन्भयात्}


\twolineshloka
{तेऽभिपत्य पुरं भग्ना हतशेषा निशाचराः}
{सर्वं राज्ञे यथावृत्तं रावणाय न्यवेदयन्}


\twolineshloka
{श्रुत्वा तु रावणस्तेभ्यः प्रहस्तं निहतं युधि}
{धूम्राक्षं च महेष्वासं ससैन्यं सहराक्षसैः}


\twolineshloka
{सुदीर्घमिव निःश्वस्य समुत्पत्य वरासनात्}
{उवाच कुम्भकर्णस्य कर्मकालोऽयमागतः}


\twolineshloka
{इत्येवमुक्त्वा विविधैर्वादित्रैः सुमहास्वनैः}
{शयानमतिनिद्रालुं कुम्भकर्णमबोधयत्}


\threelineshloka
{प्रबोध्य महता चैनं यत्नेनाऽऽगतसाध्वसः}
{स्वस्थमासीनमव्यग्रं विनिद्रं राक्षसाधिपः}
{ततोऽब्रवीद्दशग्रीवः कुम्भकर्णं महाबलम्}


\twolineshloka
{धन्योसि यस्य ते निद्रा कुम्भकर्णेयमीदृशी}
{य इदं दारुणं कालं न जानीषे महाभयम्}


\twolineshloka
{एष तीर्त्वाऽर्णवं रामः सेतुना हरिभिः सह}
{अवमत्येह नः सर्वान्करोति कदनं महत्}


\twolineshloka
{मया त्वपहृता भार्या सीता नामास्य जानकी}
{तां नेतुं स इहायातो बद्ध्वा सेतुं महार्णवे}


\twolineshloka
{तेन चैव प्रहस्तादिर्महान्नः स्वजनो हतः}
{तस्य नान्यो निहन्ताऽस्ति त्वामृतेशत्रुकर्शन}


\twolineshloka
{सदंशितोऽभिनिर्याहि त्वमद्य बलिनांवर}
{रामादीन्समरे सर्वाञ्जहि शत्रूनरिन्दम}


\twolineshloka
{दूषणावरजौ चैव वज्रवेगप्रमाथिनौ}
{तौ त्वां बलेन महता सहितावनुयास्यतः}


\twolineshloka
{इत्युक्त्वा राक्षुसपतिः कुम्भकर्णं तरस्विनम्}
{सन्दिदेशेतिकर्तव्ये वज्रवेगप्रमाथिनौ}


\twolineshloka
{तथेत्युक्त्वा युतौ वीरौ रावणं दूषाणानुजौ}
{कुम्भकर्णं पुरस्कृत्य तूर्णं निर्ययतुः पुरात्}


॥इति श्रीमन्महाभारते अरण्यपर्वणि रामोपाख्यान-पर्वणि त्रिशततमोऽध्यायः॥२८७॥

\storymeta

\dnsub{अध्यायः २८८}\resetShloka

\uvacha{मार्कण्डेय उवाच}


\twolineshloka
{ततो निर्याय स्वपुरात्कुम्भकर्णः सहानुगः}
{अपश्यत्कपिसैन्यं तज्जितकाश्यग्रतः स्थितम्}


\twolineshloka
{स वीक्षमाणस्तत्सैन्यं रामदर्शनकाङ्क्षया}
{अपश्यच्चापि सौमित्रिं धनुष्पाणिं व्यवस्थितम्}


\twolineshloka
{तमभ्येत्याशु हरयः परिवब्रुः समन्ततः}
{शैलवृक्षायुधा नादानमुञ्चन्भीषणास्ततः}


\twolineshloka
{अभ्यघ्नंश्च महाकायैर्बहुभिर्जगतीरुहैः}
{करजैरतुदंश्चान्ये विहाय भयमुत्तमम्}


\twolineshloka
{बहुधा युध्यमानास्ते युद्धमार्गैः प्लवङ्गमाः}
{नानाप्रहरणैर्भीमै राक्षसेन्द्रमताडयन्}


\twolineshloka
{स ताड्यमानः प्रहसन्भक्षयामास वानरान्}
{बलं चण्डबलाख्यं च वज्रबाहुं च वानरम्}


\twolineshloka
{तद्दृष्ट्वा व्यथनं कर्म कुम्भकर्णस्य रक्षसः}
{उदक्रोशन्परित्रस्तास्तारप्रभृतयस्तदा}


\twolineshloka
{तानुच्चैः क्रोशतः सैन्याञ्श्रुत्वा स हरियूथपान्}
{अभिदुद्राव सुग्रीवः कुम्भकर्णमपेतभीः}


\twolineshloka
{ततो निपत्य वेगेन कुम्भकर्णं महामना}
{सालेन जघ्निवान्मूर्ध्निं बलेन कपिकुञ्जरः}


\twolineshloka
{स महात्मा महावेगः कुम्भकर्णस्य मूर्धनि}
{बिभेद सालं सुग्रीवो न चैवाव्यथयत्कपिः}


\twolineshloka
{ततो विनद्यसहसा सालस्पर्शविबोधितः}
{दोर्भ्यामादाय सुग्रीवं कुम्भकर्णोऽहरद्बलात्}


\twolineshloka
{ह्रियमाणं तु सुग्रीवं कुम्भकर्णेन रक्षसा}
{अवेक्ष्याभ्यद्रवद्वीरः सौमित्रिर्मित्रनन्दनः}


\twolineshloka
{सोऽभिपत्य महर्वेगं रुक्मपुङ्खं महाशरम्}
{प्राहिणोत्कुम्भकर्णाय लक्ष्मणः परवीरहा}


\twolineshloka
{स तस्य देहावरणं भित्त्वा देहं च सायकः}
{जगाम दारयन्भूमिं रुधिरेण समुक्षितः}


\twolineshloka
{तथा स भिन्नहृदयः समुत्सृज्य कपीश्वरम्}
{वेगेन महताऽऽविष्टस्तिष्ठतिष्ठेति चाब्रवीत्}


\twolineshloka
{कुम्भकर्णो महेष्वासः प्रगृहीतशिलायुधः}
{अभिदुद्राव सौमित्रिमुद्यम्य महतीं शिलाम्}


\twolineshloka
{तस्याभिपततस्तूर्णं क्षुराभ्यामुच्छितौ करौ}
{चिच्छेद निशिताग्राभ्यां स बभूव चतुर्भुजः}


\twolineshloka
{तानप्यस्य भुजान्सर्वान्प्रगृहीतशिलायुधान्}
{क्षुरैश्चिच्छेद लघ्वस्त्रं सौमित्रिः प्रतिदर्शयन्}


\twolineshloka
{स बभूवातिकायश्च बहुपादशिरोभुजः}
{तं ब्रह्मास्त्रेण सौमित्रिर्ददाराद्रिचयोपमम्}


\twolineshloka
{स पपात महावीर्यो दिव्यास्त्राभिहतो रणे}
{महाशनिविनिर्दग्धः पादपोऽङ्कुरवानिव}


\twolineshloka
{तं दृष्ट्वा वृत्रसङ्काशं कुम्भकर्णं तरस्विनम्}
{गतासुं पतितं भूमौ राक्षसाः प्राद्रवन्भयात्}


\twolineshloka
{तथातान्द्रवतो योधान्दृष्ट्वा तौ दूषणानुजौ}
{अवस्थाप्याथ सौमित्रिं सङ्क्रुद्धावभ्यधावताम्}


\twolineshloka
{तावाद्रवन्तौ सङ्क्रुद्धौ वज्रवेगप्रमाथिनौ}
{अभिजग्राह सौमित्रिर्विनद्योभौ पतत्रिभिः}


\twolineshloka
{ततः सुतुमुलं युद्धमभवद्रोमहर्षणम्}
{दूषणानुजयोः पार्थ लक्ष्मणस्य च धीमतः}


\twolineshloka
{महता शरवर्षेण राक्षसौ सोऽभ्यवर्पत}
{तं चापिवीरौ सङ्क्रुद्धावुभौ तौ समवर्षताम्}


\twolineshloka
{मुहूर्तमेवमभवद्वज्रवेगप्रमाथिनोः}
{सौमित्रेश्च महाबाहोः सप्रहारः सुदारुणः}


\twolineshloka
{अथाद्रिशृङ्गमादाय हनुमान्मारुतात्मजः}
{अभिद्रुत्याददे प्राणान्वज्रवेगस्य रक्षसः}


\twolineshloka
{नीलश्च महता ग्राव्णा दूपणावरजं हरिः}
{प्रमाथिनमभिद्रुत्य प्रममाथ महाबलः}


\twolineshloka
{ततः प्रावर्तत पुनः सङ्ग्रामः कटुकोदयः}
{रामरावणसैन्यानामन्योन्यमभिधावताम्}


\twolineshloka
{शतसो नैर्ऋतान्वन्या जघ्नुर्वन्यांश्च नैर्ऋताः}
{नैर्ऋतास्तत्रवध्यन्ते प्रायेण न तु वानराः}


॥इति श्रीमन्महाभारते अरण्यपर्वणि रामोपाख्यान-पर्वणि त्रिशततमोऽध्यायः॥२८८॥

\storymeta

\dnsub{अध्यायः २८९}\resetShloka

\uvacha{मार्कण्डेय उवाच}


\twolineshloka
{ततः श्रुत्वाहतं सङ्ख्ये कुम्भकर्णं सहानुगम्}
{प्रहस्तं च महेष्वासं धूम्राक्षं चातितेजसम्}


\twolineshloka
{पुत्रमिन्द्रजितं वीरं रावणः प्रत्यभाषत}
{जहिरामममित्रघ्न सुग्रीवं च सलक्ष्मणम्}


\twolineshloka
{त्वया हि मम सत्पुत्र यशो दीप्तमुपार्जितम्}
{जित्वावज्रधरं सङ्ख्ये सहस्राक्षं शचीपतिम्}


\twolineshloka
{अन्तर्हितः प्रकाशो वा दिव्यैर्दत्तवरैः शरैः}
{जहि शत्रूनमित्रघ्न मम शस्त्रभृतांवर}


\twolineshloka
{रामलक्ष्मणसुग्रीवाः शरस्पर्शं न तेऽनघ}
{समर्थाः प्रतिसोढुं च कुतस्तदनुयायिनः}


\twolineshloka
{अगता या प्रहस्तेन कुम्भकर्णेन चानघ}
{खरस्यापचितिः सङ्ख्ये तां गच्छ त्वे महाभुज}


\twolineshloka
{त्वमद्य निशितैर्बाणैर्हत्वा शत्रून्ससैनिकान्}
{प्रतिनन्दय मां पुत्र पुरा जित्वेव वासवम्}


\twolineshloka
{इत्युक्तः स तथेत्युक्त्वा रथमास्थाय दंशिथः}
{प्रययाविन्द्रजिद्राजंस्तूर्णमायोधनं प्रति}


\twolineshloka
{ततो विश्राव्य विस्पष्टं नाम राक्षसपुङ्गवः}
{आह्वयामास समरे लक्ष्मणं शुभलक्षणम्}


\twolineshloka
{तं लक्ष्मणोऽभ्यधावच्च प्रगृह्य सशरं धनुः}
{त्रासयंस्तलघोषेण सिंहः क्षुद्रमृगं यथा}


\twolineshloka
{तयोः समभवद्युद्धं सुमहज्जयगृद्धिनोः}
{दिव्यास्त्रविदुपोस्तीव्रमन्योन्यस्पर्धिनोस्तदा}


\twolineshloka
{रावणिस्तु यदा नैनं विशेषयति सायकैः}
{ततो गुरुतरं यत्नमातिष्ठद्बलिनां वरः}


\twolineshloka
{तत एवं महावेगैरर्दयामास तोमरैः}
{तानागतान्स चिच्छेद सौमित्रिर्निशितैः शरैः}


\twolineshloka
{ते निकृत्ताः शरैस्तीक्ष्णैर्न्यपतन्धरणीतले}
{साधका रावणेराजौ शतशः शकलीकृताः}


\twolineshloka
{तमङ्गदो वालिसुतः श्रीमानुद्यम्य पादपम्}
{अभिद्रुत्य महावेगस्ताडयामास मूर्धनि}


\twolineshloka
{तस्येन्द्रजिदसभ्रान्तः प्रासेनोरसि वीर्यवान्}
{प्रहर्तुमैच्छत्तं चास्य प्रासं चिच्छेद लक्ष्मणः}


\twolineshloka
{तमभ्याशगतं वीरमङ्गदं रावणात्मजः}
{गदयाऽताडयत्सव्ये पार्श्वेवानरपुङ्गवम्}


\twolineshloka
{तमचिन्त्य प्रहारं स बलवान्वालिनः सुतः}
{ससर्जेन्द्रजितः क्रोधात्सालस्कन्धं तथाङ्गदः}


\twolineshloka
{सोऽङ्गदेन रुपोत्सृष्टो वधायेन्द्रजितस्तरुः}
{जघानेन्द्रजितः पार्थ रथं साश्वं ससारथिम्}


\twolineshloka
{ततो हताश्वात्प्रस्कन्द्य रथात्स हतसारथिः}
{तत्रैवान्तर्दधे राजन्मायया रावणात्मजः}


\twolineshloka
{अन्तर्हितं विदित्वा तं बहुमायं च राक्षसम्}
{रामस्तं देशमागम्य तत्सैन्यं पर्यरक्षत}


\twolineshloka
{स राममुद्दिश्य शरैस्ततो दत्तवरैस्तदा}
{विव्याध सर्वगात्रेषु लक्ष्मणं च महाबलम्}


\twolineshloka
{तमदृश्यंशरैः शूरौ माययाऽन्तर्हितं तदा}
{योधयामासतुरुभौ रावणिं रामलक्ष्मणौ}


\twolineshloka
{स रुषा सर्वगात्रेषु तयोः पुरुषसिंहयोः}
{व्यसृजत्सायकान्भूयः शतशोऽथ सहस्रशः}


\twolineshloka
{तमदृश्यं विचिन्वन्तः सृजन्तमनिशं शरान्}
{हरयो विविशुर्व्योम प्रगृह्य महतीः शिलाः}


\twolineshloka
{तांश्च तौ चाप्यदृश्यः सशरैर्विव्याध राक्षसः}
{स भृशं ताडयामास रावणिर्मायया वृतः}


\twolineshloka
{तौ शरैरर्दितौ वीरौ भ्रातरौ रामलक्ष्मणौ}
{पेततुर्गगनाद्भूमिं सूर्याचन्द्रमसाविव}


॥इति श्रीमन्महाभारते अरण्यपर्वणि रामोपाख्यान-पर्वणि त्रिशततमोऽध्यायः॥२८९॥

\storymeta

\dnsub{अध्यायः २९०}\resetShloka

\uvacha{मार्कण्डेय उवाच}


\twolineshloka
{तावुभौ पतितौ दृष्ट्वा भ्रातरौ रामलक्ष्मणौ}
{बबन्ध रावणिर्भूयः शरैर्दत्तवरैस्तदा}


\twolineshloka
{तौ वीरौ शरजालेन बद्धाविन्द्रजिता रणे}
{रेजतुः पुरुषव्याघ्रौ शकुन्ताविव पञ्जरे}


\twolineshloka
{दृष्ट्वा निपतितौ भूमौ सर्वाङ्गेषु शराचितौ}
{सुग्रीवः कपिभिः सार्धं परिवार्योपतस्तिवान्}


\twolineshloka
{सुषेणमैन्दद्विविदैः कुमुदेनाङ्गदेन च}
{हनुमम्नीलतारैश्च नलेन च कपीश्वरः}


\twolineshloka
{ततस्तं देशमागम्य कृतकर्मा विभीषणः}
{बोधयामास तौ वीरौ प्रज्ञास्त्रेण प्रमोहितौ}


\twolineshloka
{विशल्यौ चापि सुग्रीवः क्षणेनैतौ चकार ह}
{विशल्यया महौषध्या दिव्यमन्त्रप्रयुक्तया}


\twolineshloka
{तौ लब्धसंज्ञौ नृवरौ विशल्यावुदतिष्ठताम्}
{उभौ गतक्लमौ चाऽऽस्तां क्षणेनैतौ महारथौ}


\twolineshloka
{ततो विभीषणः पार्थ राममिक्ष्वाकुनन्दनम्}
{उवाच विज्वरं दृष्ट्वा कृताञ्जलिरिदं वचः}


\twolineshloka
{अयमम्भो गृहीत्वातु राजराजस्य शासनात्}
{गुह्यकोऽभ्यागतः श्वेतात्त्वत्सकाशमरिन्दम}


\twolineshloka
{इदमम्भः कुबेरस्ते महाराज प्रयच्छति}
{अन्तर्हितानां भूतानां दर्शनार्थं परन्तप}


\twolineshloka
{अनेन मृष्टनयनो भूतान्यन्तर्हितान्युत}
{भवान्द्रक्ष्यति यस्मै च भवानेतत्प्रदास्यति}


\twolineshloka
{तथेति रामस्तद्वारि प्रतिगृह्याभिसंस्कृतम्}
{चकार नेत्रयोः शौचं लक्ष्मणश्च महामनाः}


\twolineshloka
{सुग्रीवजाम्बवन्तौ चहनुमानङ्गदस्तथा}
{मैन्दद्विविदनीलाश्च प्रायः प्लवगसत्तमाः}


\twolineshloka
{तथासमभवच्चापि यदुवाच विभीषणः}
{क्षणेनातीन्द्रियाण्येषां चक्षुंष्यासन्युधिष्ठिर}


\twolineshloka
{इन्द्रजित्कृतकर्मा तु पित्रे कर्म तदाऽऽत्मनः}
{निवेद्य पुनरागच्छत्त्वरयाऽऽजिशिरःप्रति}


\twolineshloka
{तमागतं तु सङ्क्रुद्धं पुनरेव युयुत्सया}
{अभिदुद्राव सौमित्रिर्विभीषणमते स्थितः}


\twolineshloka
{अकृताह्निकमेवैनं जिघांसुर्जितकाशिनम्}
{शरैर्जघान सङ्क्रुद्धः कृतसंज्ञोऽथ लक्ष्मणः}


\twolineshloka
{तयोः समभवद्युद्धं तदाऽन्योन्यं जीगीषतोः}
{अतीव चित्रमाश्चर्यं शक्रप्रह्लादयोरिव}


\twolineshloka
{अविध्यदिन्द्रजित्तीक्ष्णैः सौमित्रिं मर्मभेदिभिः}
{सौमित्रिश्चानलस्पर्शैरविध्यद्रावणिं शरैः}


\twolineshloka
{सौमित्रिशरसंस्पर्शाद्रावणिः क्रोधमूर्च्छितः}
{असृजल्लक्ष्मणायाष्टौ शरानाशीविषोपमान्}


\twolineshloka
{तस्येषून्पावकस्पर्शैः सौमित्रिः पत्रिभिस्त्रिभिः}
{वारयामास नाराचैः सौमित्रिर्मित्रनन्दनः}


\twolineshloka
{असृजल्लक्ष्मणश्चाष्टौ राक्षसाय शरान्पुनः}
{तथा तं न्यहनद्वीरस्तन्मे निगदतः शृणु}


\twolineshloka
{एकेनास्य धनुष्मन्तं बाहुं देहादपातयत्}
{द्वितीयेन तु बाणेन भुजमन्यमपातयत्}


\twolineshloka
{तृतीयेन तु बाणेन शितधारेण भास्वता}
{जहार सुनसं चापि शिरो ज्वलितकुण्डलम्}


\twolineshloka
{विनिकृत्तभुजस्कन्धः कबन्धाकृतिदर्शनः}
{पपात वसुधायां तु छिन्नमूल इवद्रुमः}


\twolineshloka
{तं हत्वा सूतमप्यस्त्रैर्जघान बलिनां वरः}
{लङ्कां प्रवेशयामासुस्तं रथं वाजिनस्तदा}


\threelineshloka
{ददर्श रावणस्तं च रथं पुत्रविनाकृतम्}
{स पुत्रं निहतं श्रुत्वा त्रासात्सभ्रान्तमानसः}
{रावणः शोकमोहार्तो वैदेहीं हन्तुमुद्यतः}


\twolineshloka
{अशोकवनिकास्थां तां रामदर्शनलालसाम्}
{खड्गमादाय दुष्टात्मा जवेनाभिपपात ह}


\twolineshloka
{तं दृष्ट्वा तस्य दुर्बुद्धेरविन्ध्यः पापनिश्चयम्}
{शमयामास सङ्क्रुद्धं श्रूयतां येन हेतुना}


\twolineshloka
{महाराज्ये स्थितो दीप्ते न स्त्रियं हन्तुमर्हसि}
{हतैवैषा यदा स्त्री च बन्धनस्था च ते वशे}


\twolineshloka
{न चैषा दहभेदेन हतास्यादिति मे मतिः}
{जहि भर्तारमेवास्या हते तस्मिन्हता भवेत्}


\twolineshloka
{न हि ते विक्रमे तुल्यः साक्षादपि शतक्रतुः}
{असकृद्धि त्वया सन्द्रास्त्रासितास्त्रिदसा युधि}


\twolineshloka
{एवं बहुविधैर्वाक्यैरविन्ध्यो रावणं तदा}
{क्रुद्धं संशमयामास जगृहे च स तद्वचः}


\twolineshloka
{निर्याणे स मतिं कृत्वा नियन्तारं क्षपाचरः}
{आज्ञापयामास तदारथो मे कल्प्यतामिति}


॥इति श्रीमन्महाभारते अरण्यपर्वणि रामोपाख्यान-पर्वणि त्रिशततमोऽध्यायः॥२९०॥

\storymeta

\dnsub{अध्यायः २९१}\resetShloka

\uvacha{मार्कण्डेय उवाच}


\twolineshloka
{ततः क्रुद्धो दशग्रीवः प्रिये पुत्रे निपातिते}
{निर्ययौ रथमास्थाय हेमरत्नविभूषितम्}


\twolineshloka
{संवृतोराक्षसैर्घेरैर्विविधायुधपाणिभिः}
{अभिदुद्राव रामं स पोथयन्हरियूथपान्}


\twolineshloka
{तमाद्रवन्तं सङ्क्रुद्ध मैन्दनीलनलाङ्गदाः}
{हनुमाञ्जाम्बवांश्चैव ससैन्याः पर्यवारयन्}


\twolineshloka
{ते दशग्रीवसैन्यं तदृक्षवानरपुङ्गवाः}
{द्रुमैर्विध्वंसयाञ्चक्रुर्दशग्रीवस्य पश्यतः}


\twolineshloka
{ततः स्वसैन्यमालोक्य वध्यमानमरातिभिः}
{मायावी चासृजन्मायां रावणो राक्षसाधिपः}


\twolineshloka
{तस्य देहविनिष्क्रान्ताः शतशोऽथ सहस्रशः}
{राक्षसाः प्रत्यदृश्यन्त शरशक्त्यृष्टिपाणयः}


\twolineshloka
{तान्रामो जघ्निवान्सर्वान्दिव्येनास्त्रेण राक्षसान्}
{अथ भूयोपि मायां स व्यदधाद्राक्षसाधिपः}


\twolineshloka
{कृत्वा रामस्य रूपाणि लक्ष्मणस्य च भारत}
{अभिदुद्राव रामं च लक्ष्मणं च दशाननः}


\twolineshloka
{ततस्ते राममर्च्छन्तो लक्ष्मणं च क्षपाचराः}
{अभिपेतुस्तदा रामं प्रगृहीतशरासनाः}


\twolineshloka
{तां दृष्ट्वा राक्षसेन्द्रस्य मायामिक्ष्वाकुनन्दनः}
{उवाच रामः सौमित्रिमसभ्रान्तो बृहद्वचः}


\twolineshloka
{जहीमान्राक्षसान्पापानात्मनः प्रतिरूपकान्}
{इत्युक्त्वाऽभ्यहनद्रामो लक्ष्मणश्चात्मरूपकान्}


\twolineshloka
{ततो हर्यश्वयुक्तेन रथेनादित्यवर्चसा}
{उपतस्थे रणे रामं मातलिः शक्रसारथिः}

\uvacha{मातलिरुवाच}


\twolineshloka
{अयं हर्यश्वयुग्जैत्रो मघोनः स्यन्दनोत्तमः}
{त्वदर्थमिह सप्राप्तः सन्देशाद्वै शतक्रतोः}


\twolineshloka
{अनेन शक्रः काकुत्स्थ समरे दैत्यदानवान्}
{शतशः पुरुषव्याघ्र रथोदारेण जघ्निवान्}


\twolineshloka
{तदनन नरव्याघ्र मया यत्तेन संयुगे}
{स्यन्दनेन जहिक्षिप्रं रावणं मा चिरं कृथाः}


\twolineshloka
{इत्युक्तो राघवस्तथ्यं वचोऽशङ्कत मातलेः}
{मायैषाराक्षसस्येति तमुवाच विबीषणः}


\twolineshloka
{नेयं माया नरव्याघ्र रावणस्य दुरात्मनः}
{तदातिष्ठ रथंशीघ्रमिमसैन्द्रं महाद्युते}


\twolineshloka
{ततः प्रहृष्टः काकुत्स्थस्तथेत्युक्त्वा विभीषणम्}
{रथेनाभिपपाताथ दशग्रीवं रुषाऽन्वितः}


\twolineshloka
{हाहाकुतानि भूतानि रावणे समभिद्रुते}
{सिंहनादाः सपटहादिति दिव्यास्तथाऽनदन्}


\twolineshloka
{दशकन्धरराजसून्वोस्तथा युद्धमभून्महत्}
{अलब्धोपममन्यत्रतयोरेव तथाऽभवत्}


\twolineshloka
{सरामाय महाघोरं विससर्ज निशाचरः}
{शूलमिन्द्राशनिप्रख्यं ब्रह्मदण्डभिवोद्यतम्}


\twolineshloka
{तच्छूलं सत्वरं रामश्चच्छेद निशितैः शरैः}
{तद्दृष्ट्वा दुष्करं कर्म रावणं भयमाविशत्}


\twolineshloka
{ततः क्रुद्धः ससर्जाशु दशग्रीवः शिताञ्छरान्}
{सहस्रायुतशो रामे शस्त्राणि विविधानि च}


\twolineshloka
{ततो भुशुण्डीः शूलानि मुसलानि परश्वथान्}
{शक्तीश्च विविधाकाराः शतघ्नीश्च शितान्क्षुरान्}


\twolineshloka
{तां मायांविविधां दृष्ट्वा दशग्रीवस्य रक्षसः}
{भयात्प्रदुद्रुवुः सर्वे वानराः सर्वतोदिशम्}


\twolineshloka
{ततः सुपत्रं सुमुखंहेमपुङ्गं शरोत्तमम्}
{तूणादादाय काकुत्स्थो ब्रह्मास्त्रेण युयोज ह}


\twolineshloka
{तं प्रेक्ष्यबाणं रामेण ब्रह्मास्त्रेणानुमन्त्रितम्}
{जहृषुर्देवगन्धर्वा दृष्ट्वा शक्रपुरोगमाः}


\twolineshloka
{अल्पावशेषमायुश्च ततोऽमन्यन्त रक्षसः}
{ब्रह्मास्त्रोदीरणाच्छत्रोर्देवदानवकिन्नराः}


\twolineshloka
{ततः ससर्ज तं रामः शरमप्रतिमौजसम्}
{रावणान्तकरं घोरं ब्रह्मदण्डमिवोद्यतम्}


\threelineshloka
{मुक्तमात्रेण रामेण दूराकृष्टेन भारत}
{स तेन राक्षसश्रेष्ठः सरथः साश्वसारथिः}
{प्रजज्वाल महाज्वालेनाग्निनाभिपरिप्लुतः}


\twolineshloka
{ततः प्रहृष्टास्त्रिदशाः सहगन्धर्वचारणाः}
{निहतं रावणं दृष्ट्वा रामेणाक्लिष्टकर्मणा}


\twolineshloka
{तत्यजुस्तं महाभागं पञ्चभूतानि रावणम्}
{भ्रंशितः सर्वलोकेषु स हि ब्रह्मास्त्रतेजसा}


\twolineshloka
{शरीरधातवो ह्यस्य मासं रुधिरमेव च}
{नेशुर्ब्रह्मास्त्रनिर्दग्दा न च भस्माप्यदृश्यत}


॥इति श्रीमन्महाभारते अरण्यपर्वणि रामोपाख्यान-पर्वणि त्रिशततमोऽध्यायः॥२९१॥

\storymeta

\dnsub{अध्यायः २९२}\resetShloka

\uvacha{मार्कण्डेय उवाच}


\twolineshloka
{स हत्वा रावणं क्षुद्रं राक्षसेन्द्रं सुरद्विषम्}
{बभूव हृष्टः ससुहृद्रामः सौमित्रिणा सह}


\twolineshloka
{ततो हते दशग्रीवे देवाः सर्षिपुरोगमाः}
{आशीर्भिर्जययुक्ताभिरानर्चुस्तं महाभुजम्}


\twolineshloka
{रामं कमलपत्राक्षं तुष्टुवुः सर्वदेवताः}
{गन्धर्वाः पुष्पवर्षैश्च वाग्भिश्च त्रिदशालयाः}


\twolineshloka
{पूजयित्वा रणे रामं प्रतिजग्मुर्यथागतम्}
{तन्महोत्सवसङ्काशमासीदाकाशमच्युत}


\twolineshloka
{ततो हत्वा दशग्रीवं लङ्कां रामो महायशाः}
{विभीषणाय प्रददौ प्रभुः परपुरञ्जयः}


\twolineshloka
{ततः सीतां पुरस्कृत्य विभीषणपुरस्कृताम्}
{अविन्ध्यो नाम सुप्रज्ञो वृद्धामात्यो विनिर्ययौ}


\twolineshloka
{उवाच च महात्मानं काकुत्स्थं दैन्यमास्थितम्}
{प्रतीच्छ देवीं सद्वृत्तां महात्मञ्जानकीमिति}


\twolineshloka
{एतच्छ्रुत्वा वचस्तस्मादवतीर्य रथोत्तमात्}
{बाष्पेणापिहितां सीतां ददर्शेक्ष्वाकुनन्दनः}


\twolineshloka
{तां दृष्ट्वा चारुसर्वाङ्गीं यानस्थां शोककर्शिताम्}
{मलोपचितसर्वाङ्गीं जटिलां कृष्णवाससम्}


\twolineshloka
{उवाच रामो वैदेहीं परामर्शविशङ्कितः}
{लक्षयित्वेङ्गितं सर्वं प्रियं तस्यै निवेद्य सः}


\threelineshloka
{गच्छ वैदेहि मुक्ता त्वं यत्कार्यं तन्मया कृतम्}
{मामासाद्यपतिं भद्रे न त्वं राक्षसवेश्मनि}
{जरां व्रजेथा इति मे निहतोऽसौ निशाचरः}


\twolineshloka
{कथं ह्यस्मद्विधो जातु जानन् धर्मविनिश्चयम्}
{परहस्तगतां नारीं मुहूर्तमपि धारयेत्}


\twolineshloka
{सुवृत्तामसुवृत्तां वाऽप्यहं त्वामद्य मैथिलि}
{नोत्सहे परिभोगाय श्वावलीढं हविर्यथा}


\twolineshloka
{ततः सा सहसा बाला तच्छ्रुत्वा दारुणं वचः}
{पपात देवी व्यथिता निकृत्ता कदली यथा}


\twolineshloka
{योऽप्यस्या हर्षसम्भूतो मुखरागः पुराऽभवत्}
{क्षणेन सपुनर्नष्टो निःश्वासादिव दर्पणे}


\twolineshloka
{ततस्ते हरयः सर्वे तच्छ्रुत्वा रामभाषितम्}
{गतासुकल्पा निश्चेष्टा बभूवुः सहलक्ष्मणाः}


\twolineshloka
{ततो देवो विशुद्धात्मा विमानेन चतुर्मुखः}
{पद्मयोनिर्जगत्स्रष्टा दर्शयामास राघवम्}


\twolineshloka
{शक्रश्चाग्निश्च वायुश्च यमो वरुण एव च}
{यक्षाधिपश्च भगवांस्तथा सप्तर्षयोऽमलाः}


\twolineshloka
{राजा दशरथश्चैव दिव्यभास्वरमूर्तिमान्}
{विमानेन महार्हेण हंसयुक्तेन भास्वता}


\twolineshloka
{ततोऽन्तरिक्षं तत्सर्वं देवगन्धर्वसङ्कुलम्}
{शुशुभे तारकाचित्रं शरदीव नभस्तलम्}


\twolineshloka
{तत उत्थाय वैदेही तेषां मध्ये यशस्विनी}
{उवाच वाक्यं कल्याणी रामं पृथुलवक्षसम्}


\twolineshloka
{राजपुत्र न ते कोपं करोमि विदिताहि मे}
{गतिः स्त्रीणां नराणां च शृणु चेदं वचो मम}


\twolineshloka
{अन्तश्चरति भूतानां मातरिश्वा सदागतिः}
{स मे विमुञ्चतु प्राणान्यदि पापं चराम्यहम्}


\twolineshloka
{अग्निरापस्तथाऽऽकाशं पृथिवी वायुरेव च}
{विमुञ्चन्तु मम प्राणान्यदि पापं चराम्यहम्}


\twolineshloka
{यथाऽहं त्वदृते वीर नान्यं स्वप्नेऽप्यचिन्तयम्}
{तथा मे देव निर्दिष्टस्त्वमेव हि पतिर्भव}


\twolineshloka
{ततोऽन्तरिक्षे वागारीत्सुभगा लोकसाक्षिणी}
{पुण्यासंहर्षणी तेषां वानराणां महात्मनाम्}

\uvacha{वायुरुवाच}

\twolineshloka
{भो भो राघव सत्यं वै वायुरस्मि सदागतिः}
{अपापा मैथिली राजन् सङ्गच्छ सह भार्यया}

\uvacha{अग्निरुवाच}

\twolineshloka
{अहमन्तःशरीरस्थो भूतानां रघुनन्दन}
{सुसूक्ष्ममपि काकुत्स्थ मैथिली नापराध्यति}

\uvacha{वरुण उवाच}

\twolineshloka
{रसा वै मत्प्रसूता हि भूतदेहेषु राघव}
{अहंवै त्वां प्रब्रवीमि मैथिली प्रतिगृह्यताम्}

\uvacha{यम उवाच}

\twolineshloka
{धर्मोऽहमस्मि काकुत्स्थ साक्षी लोकस्य कर्मणाम्}
{शुभाशुभानां सीतेयमपापा प्रतिगृह्यताम्}

\uvacha{ब्रह्मोवाच}

\twolineshloka
{पुत्र नैतदिहाश्चर्यं त्वयि राजर्षिधर्मणि}
{साधो सद्वृत्त काकुत्स्थ शृणु चेदं वचो मम}

\twolineshloka
{शत्रुरेष त्वया वीर देवगन्धर्वभोगिनाम्}
{यक्षाणां दानवानां च महर्षीणां च पातितः}

\twolineshloka
{अवध्यः सर्वभूतानां मत्प्रसादात्पुराऽभवत्}
{कस्माच्चित्कारणात्पापः कञ्चित्कालमुपेक्षितः}


\twolineshloka
{वधार्थमात्मनस्तेन हृता सीता दुरात्मना}
{नलकूबरशापेन रक्षा चास्याः कृता मया}


\twolineshloka
{यदि ह्यकामामासेवेत्स्त्रियमन्यामपि ध्रुवम्}
{शतधाऽस्य फलेन्मूर्धा इत्युक्तः सोभवत्पुरा}


\twolineshloka
{नात्रशङ्का त्वया कार्या प्रतीच्छेमां महामते}
{कृतं त्वया महत्कार्यं देवानाममितप्रभ}

\uvacha{दशरथ उवाच}


\twolineshloka
{प्रीतोस्मि वत्स भद्रं ते पिता दशरथोस्मि ते}
{अनुजानामि राज्यं च प्रशाधि पुरुषोत्तम}

\uvacha{राम उवाच}


\twolineshloka
{अभिवादये त्वां राजेन्द्र यदि त्वं जनको मम}
{गमिष्यामि पुरीं रम्यामयोध्यां शासनात्तव}

\uvacha{मार्कण्डेय उवाच}


\threelineshloka
{तमुवाच पिता भूयः प्रहृष्टो भरतर्षभ}
{गच्छायोध्यां प्रशाधि त्वंराम रक्तान्तलोचन}
{सपूर्णानीहवर्षाणि चतुर्दश महाद्युते}


\twolineshloka
{ततो देवान्नमस्कृत्य मुहृद्भिरभिनन्दितः}
{महेन्द्रइव पौलोम्या भार्यया स समेयिवान्}


\twolineshloka
{ततो वरं ददौ तस्मै ह्यविन्ध्याय परन्तपः}
{त्रिजटां चार्थमानाभ्यां योजयामास राक्षसीम्}


\twolineshloka
{तमुवाच ततो ब्रह्मा देवैः शक्रषुरोगमैः}
{कौसल्यामातरिष्टांस्ते वरानद्य ददानि कान्}


\twolineshloka
{वव्रेरामः स्थितिं धर्मे शत्रुभिश्चापराजयम्}
{राक्षसैर्निहतानां च वानराणां समुद्भवम्}


\twolineshloka
{ततस्ते ब्रह्मणा प्रोक्ते तथेतिवचने तदा}
{समुत्तस्थुर्महाराज वानरा लब्धचेतसः}


\twolineshloka
{सीता चापि महाभागा वरं हनुमते ददौ}
{रामकीर्त्या समं पुत्र जीवितं ते भविष्यति}


\twolineshloka
{दिव्यास्त्वामुपभोगाश्च मत्प्रसादकृताः सदा}
{उपस्थास्यन्ति हनुमन्निति स्म हरिलोचन}


\twolineshloka
{ततस्ते प्रेक्षमाणानां तेपामक्लिष्टकर्मणाम्}
{अन्तर्धानं ययुर्देवाः सर्वे शक्रपुरोगमाः}


\twolineshloka
{दृष्ट्वा रामं तु जानक्या सङ्गतं शक्रसारथिः}
{उवाच परमप्रीतसुहृन्मध्य इदं वचः}


\twolineshloka
{देवगन्धर्वयक्षाणां मानुषासुरभोगिनाम्}
{अपनीतं त्वया दुःखमिदं सत्यपराक्रम}


\twolineshloka
{सदेवासुरगन्धर्वा यक्षराक्षसपन्नगाः}
{कथयिष्यन्ति लोकास्त्वां यावद्भूमिर्धरिष्यति}


\twolineshloka
{इत्येवमुक्त्वाऽनुज्ञाप्यरामं शस्त्रभृतांवरम्}
{सपूज्यापाक्रमत्तेन रथेनादित्यवर्चसा}


\twolineshloka
{ततःसीतां पुरस्कृत्य रामः सौमित्रिणा सह}
{सुग्रीवप्रमुखैश्चैव सहितः सर्ववानरैः}


\twolineshloka
{विधाय रक्षां लङ्कायां विभीषणपुरस्कृतः}
{सन्ततार पुनस्तेन सेतुना मकरालयम्}


\twolineshloka
{पुष्पकेण विमानेन खेचरेण विराजता}
{कामगेन यथामुख्यैरमात्यैः संवृतो वसी}


\twolineshloka
{ततस्तीरे समुद्रस्य यत्र शिश्ये स पार्थिवः}
{तत्रैवोवास धर्मात्मा सहितः सर्ववानरैः}


\twolineshloka
{अथैनान्राघवः काले समानीयाभिपूज्य च}
{विसर्जयामास तदा रत्नैः सन्तोष्य सर्वशः}


\twolineshloka
{गतेषु वानरेन्द्रेषु गोपुच्छर्क्षेषु तेषु च}
{सुग्रीवसहितो रामः किष्किन्दां पुनरागमत्}


\twolineshloka
{विभीषणेनानुगतः सुग्रीवसहितस्तदा}
{पुष्पकेण विमानेन वैदेह्या दर्शयन्वनम्}


\twolineshloka
{किष्किन्धां तु समासाद्यरामः प्रहरतांवरः}
{अङ्गदं कृतकर्माणं यौवराज्येऽभ्यषेचयत्}


\twolineshloka
{ततस्तैरेव सहितो रामः सौमित्रिणा सह}
{यथागतेन मार्गेण प्रययौ स्वपुरं प्रति}


\twolineshloka
{अयोध्यां स समासाद्यपुरीं राष्ट्रपतिस्ततः}
{भरताय हनूमन्तं दूतं प्रास्थापयद्द्रुतम्}


\twolineshloka
{लक्षयित्वेङ्गितं सर्वप्रियं तस्मै निवेद्य वै}
{वायुपुत्रे पुनः प्राप्ते नन्दिग्राममुपाविशत्}


\threelineshloka
{सतत्रमलदिग्धाङ्गं भरतं चीरवाससम्}
{नन्दिग्रामगतंरामः सशत्रुघ्नं सराघवः}
{अग्रतःपादुके कृत्वा ददर्शासीनमासने}


\twolineshloka
{समेत्यभरतेनाथ शत्रुघ्नेन च वीर्यवान्}
{राघवः सहसौमित्रिर्मुमुदे भरतर्षभ}


\twolineshloka
{ततो भरतशत्रुघ्नौ समेतौ गुरुणा तदा}
{वैदेह्या दर्शनेनोभौ प्रहर्षं समवापतुः}


\twolineshloka
{तस्मै तद्भरतो राज्यमागतायातिसत्कृतम्}
{न्यासं निर्यातयामास युक्तः परमया मुदा}


\twolineshloka
{ततस्तं वैष्णवे शूरं नक्षत्रेऽभिजितेऽहनि}
{वसिष्ठो वामदेवश्च सहितावभ्यषिञ्चताम्}


\twolineshloka
{सोभिषिक्तः कपिश्रेष्ठं सुग्रीवं ससुहृज्जनम्}
{विभीषणं च पौलस्त्यमन्वजानाद्गृहान्प्रति}


\twolineshloka
{अभ्यर्च्य विविधै रत्नैः प्रीतियुक्तौ मुदा युतौ}
{समाधायेतिकर्तव्यं दुःखेन विससर्ज ह}


\twolineshloka
{पुष्पकं च विमानं तत्पूजयित्वा स राघवः}
{प्रादाद्वैश्रवणायैव प्रीत्या स रघुनन्दनः}


\twolineshloka
{ततो देवर्षिसहितः सरितं गोमतीमनु}
{शताश्वमेधानाजह्रे जारूथ्यान्स निरर्गलान्}


॥इति श्रीमन्महाभारते अरण्यपर्वणि रामोपाख्यान-पर्वणि त्रिशततमोऽध्यायः॥२९२॥

\storymeta

\dnsub{अध्यायः २९३}\resetShloka

\uvacha{मार्कण्डेय उवाच}


\twolineshloka
{एवमेतन्महाबाहो रामेणामिततेजसा}
{प्राप्तं व्यसनमत्युग्रं वनवासकृतं पुरा}


\twolineshloka
{मा शुचः परुषव्याघ्र क्षत्रियोसि परन्तप}
{बाहुवीर्याश्रयेमार्गे वर्तसे दीप्तनिर्णये}


\twolineshloka
{न हि ते वृजिनं किञ्चिद्दृश्यते परमण्वपि}
{अस्मिन्मार्गे निपीदेयुः सेन्द्रा अपि सुरासुराः}


\twolineshloka
{संहत्य निहतोवृत्रो मरुद्भिर्वज्रपाणिना}
{नमुचिश्चैवदुर्धर्षो दीर्गजिह्वा चराक्षसी}


\twolineshloka
{सहायवति सर्वार्थाः सतिष्ठन्तीह सर्वशः}
{किन्नु तस्याजितं सङ्ख्ये यस्य भ्राता धनञ्जयः}


\twolineshloka
{अयं च बलिनांश्रेष्ठो भीमो भीमपराक्रमाः}
{युवानौ च महेष्वासौ वीरौ माद्रवतीसुतौ}


\twolineshloka
{एभिः सहायैः कस्मात्त्वं विषीदसि परन्तप}
{य इमे वज्रिणः सेनां जयेयुः समरुद्गणाम्}


\twolineshloka
{त्वमप्येभिर्महेष्वासैः सहायैर्देवरूपिभिः}
{विजेष्यसि रणे सर्वानमित्रान्भरतर्षभ}


\twolineshloka
{इतश्च त्वमिमां पश्यसैन्धवेन दुरात्मना}
{बलिना वीर्यमत्तेन हृतामेभिर्महात्मभिः}


\twolineshloka
{आनीतां द्रौपदीं कृष्णां कृत्वा कर्म सुदुष्करम्}
{जयद्रथं च राजानं विजितं वशमागतम्}


\twolineshloka
{असहायेन रामेण वैदेही पुनराहृता}
{हत्वासङ्ख्ये दशग्रीवं राक्षसं भीमविक्रमम्}


\twolineshloka
{यस्य शाखामृगामित्राण्यृक्षाः कालमुखास्तथा}
{जात्यन्तरगता राजन्नेतद्बुद्ध्याऽनुचिन्तय}


\twolineshloka
{तस्मात्सर्वं कुरुश्रेष्ठ मा शुचो भरतर्षभ}
{त्वद्विधा हि महात्मानो न शोचन्ति परन्तप}

\uvacha{वैशम्पायन उवाच}


\twolineshloka
{एवमाश्वासितो राजामार्कण्डेयेन धीमता}
{त्यक्त्वा दुःखमदीनात्मा पुनरप्येनमब्रवीत्}


॥इति श्रीमन्महाभारते अरण्यपर्वणि रामोपाख्यान-पर्वणि त्रिशततमोऽध्यायः॥२९३॥

रामोपाख्यान-पर्व समाप्तम्॥१८॥ 

\closesection
    \chapt{हनूमता रामकथाकथनम्}

\src{श्रीमन्महाभारतम्}{वन-पर्व}{तीर्थयात्रापर्व}{अध्यायाः १४९--१५०}
\vakta{हनुमान्}
\shrota{भीमः}
\tags{concise, complete}
\notes{After an interesting encounter between Hanuman and Bhima, at Bhima's request, Hanuman narrates Ramayana to Bhima.}
% \textlink{http://stotrasamhita.net/wiki/Narayaniyam/Dashaka_34}
\translink{}

\storymeta

\sect{अध्यायः १४९}

\uvacha{हनूमानुवाच}

\addtocounter{shlokacount}{25}

\twolineshloka
{यत्ते मम परिज्ञाने कौतूहलमरिन्दम}
{तत्सर्वमखिलेन त्वं शृणु पाण्डवनन्दन}


\twolineshloka
{अहं केसरिणः क्षेत्रे वायुना जगदायुषा}
{जातः कमलपत्राक्ष हनूमान्नाम वानरः}


\twolineshloka
{सूर्यपुत्रं च सुग्रीवं शक्रपुत्रं च वालिनम्}
{सर्ववानरराजानौ सर्ववानरयूथपाः}


\twolineshloka
{उपतस्थुर्महावीर्या मम चामित्रकर्शन}
{सुग्रीवेणाभवत्प्रीतिरनिलस्याग्निना यथा}


\twolineshloka
{निकृतः स ततो भ्रात्रा कस्मिंश्चित्कारणान्तरे}
{ऋश्यमूके मया सार्धं सुग्रीवो न्यवसच्चिरम्}


\twolineshloka
{अथ दाशरथिर्वीरो रामो नाम महाबलः}
{विष्णुर्मानुषरूपेण चचार वसुधातलम्}


\twolineshloka
{स पितुः प्रियमन्विच्छन्सहभार्यः सहानुजः}
{सधनुर्धन्विनां श्रेष्ठो दण्डकारण्यमाश्रितः}


\twolineshloka
{तस्य भार्या जनस्थानाच्छलेनापहृता बलात्}
{राक्षसेन्द्रेण बलिना रावणेन दुरात्मना}


\twolineshloka
{सुवर्णरत्नचित्रेण मृगरूपेण रक्षसा}
{वञ्चयित्वा नरव्याघ्रं मारीचेन तदाऽनघ}

॥इति श्रीमन्महाभारते अरण्यपर्वणि तीर्थयात्रा-पर्वणि एकोनपञ्चाशदधिकशततमोऽध्यायः॥१४९॥


\sect{अध्यायः १५०}

\uvacha{हनूमानुवाच}

\twolineshloka
{हृतदारः सह भ्रात्रा पत्नीं मार्गन्स राघवः}
{दृष्टवाञ्शैलशिखरे सुग्रीवं वानरर्षभम्}


\twolineshloka
{तेन तस्याभवत्सख्यं राघवस्य महात्मनः}
{स हत्वा वालिनं राज्ये सुग्रीवं प्रत्यपादयत्}


\twolineshloka
{स राज्यं प्राप्य सुग्रीवः सीतायाः परिमार्गणे}
{वानरान्प्रेषयामास शतशोऽथ सहस्रशः}


\twolineshloka
{ततो वानरकोटीभिः सहितोऽहं नरर्षभ}
{सीतां मार्गन्महाबाहो प्रस्थितो दक्षिणां दिशम्}


\twolineshloka
{ततः प्रवृत्तिः सीताया गृध्रेण सुमहात्मना}
{सम्पातिना समाख्याता रावणस्य निवेशने}


\twolineshloka
{ततोऽहं कार्यसिद्ध्यर्थं रामस्याक्लिष्टकर्मणः}
{शतयोजनविस्तारमर्णवं सहसा प्लुतः}


\twolineshloka
{अहं स्ववीर्यादुत्तीर्य सागरं मकरालयम्}
{सुतां जनकराजस्य सीतां सुररसुतोपमाम्}


\twolineshloka
{दृष्टवान्भरतश्रेष्ठ रावणस्य निवेशने}
{समेत्य तामहं देवीं वैदेहीं राघवप्रियाम्}


\twolineshloka
{दग्ध्वा लङ्कामशेषेण साट्टप्राकारतोरणाम्}
{प्रत्यागतश्चास्य पुनर्नाम तत्र प्रकाश्य वै}


\threelineshloka
{मद्वाक्यं चावधार्याशु रामो राजीवलोचनः}
{अबद्धपूर्वमन्यैश्च बद्ध्वा सेतुं महोदधौ}
{वृतो वानरकोटीभिः समुत्तीर्णो महार्णवम्}


\twolineshloka
{ततो रामेण वीर्येण हत्वा तान्सर्वराक्षसान्}
{रणे तु राक्षसगणं रावणं लोकरावणम्}


\twolineshloka
{निशाचरेनद्रं हत्वा तु सभ्रातृसुतबान्धवम्}
{राज्येऽभिषिच्य लङ्कायां राक्षसेन्द्रं विभीषणम्}


\twolineshloka
{धार्मिकं भक्तिमन्तं च भक्तानुगतवत्सलः}
{प्रत्याहृत्य ततः सीतां नष्टां वेदश्रुतिं यथा}


\threelineshloka
{तयैव सहितः साध्व्या पत्न्या रामो महायशाः}
{गत्वा ततोऽतित्वरितः स्वां पुरीं रघुनन्दनः}
{अध्यावसत्ततोऽयोध्यामयोध्यां द्विषतां प्रभुः}


\twolineshloka
{ततः प्रतिष्ठितो राज्ये रामो नृपतिसत्तमः}
{वरं मया याचितोऽसौ रामो राजीवलोचनः}


\twolineshloka
{यावद्रामकथेयं ते भवेल्लोकेषु शत्रुहन्}
{तावज्जीवेयमित्येवं तथाऽस्त्विति च सोब्ऽरवीत्}


\twolineshloka
{सीताप्रसादाच्च सदा मामिहस्थमरिन्दम}
{उपतिष्ठन्ति दिव्या हि भोगा भीम यथेप्सिताः}


\twolineshloka
{दशवर्षसहस्राणि दशवर्षशतानि च}
{राज्यं कारितवान्रामस्ततः स्वभवनं गतः}


\twolineshloka
{तदिहाप्सरसस्तात गन्धर्वाश्च सदाऽनघ}
{तस्य वीरस्य चरितं गायन्त्यो रमयन्ति माम्}


\twolineshloka
{अयं च मार्गो मर्त्यानामगम्यः कुरुनन्दन}
{ततोऽहं रुद्धवान्मार्गं तवेमं देवसेवितम्}


\twolineshloka
{त्वामनेन पथा यान्तं यक्षो वा राक्षसोऽपि वा}
{धर्षयेद्वा शपेद्वाऽपि मा कश्चिदिति भारत}


\twolineshloka
{दिव्यो देवपथो ह्येष नात्र गच्छन्ति मानुषाः}
{यदर्थमागतश्चासि अत एव सरश्च तत्}

॥इति श्रीमन्महाभारते अरण्यपर्वणि तीर्थयात्रा-पर्वणि पञ्चाशदधिकशततमोऽध्यायः॥१५०॥

\closesection
    \sect{षोडशराजकीये रामचरितम्}

\src{श्रीमन्महाभारतम्}{द्रोण-पर्व}{अभिमन्युवधपर्व}{अध्यायाः ५९}
\vakta{नारदः}
\shrota{सृञ्जयः}
\tags{concise, complete}
\notes{Narada narrates the story of 16 great kings who no longer existed, to emphasise the impermanence of life. Narada extols Rama in this brief outline of Rama's life and kingship.}
% \textlink{http://stotrasamhita.net/wiki/Narayaniyam/Dashaka_34}
\translink{}

\storymeta

\dnsub{अध्यायः ५९}

\uvacha{नारद उवाच}

\threelineshloka
{रामं दाशरथिं चैव मृतं सृञ्जय शुश्रुम}
{यं प्रजा अन्वमोदन्त पिता पुत्रमिवौरसम्}
{असङ्ख्येया गुणा यस्मिन्नासन्नमिततेजसि}


\twolineshloka
{यश्चतुर्दशवर्षाणि निदेशात्पितुरच्युतः}
{वने वनितया सार्धमवसल्लक्ष्मणाग्रजः}


\twolineshloka
{जघान च जनस्थाने राक्षसान्मनुजर्षभः}
{तपस्विनां रक्षणार्थं सहस्राणि चतुर्दश}


\twolineshloka
{तत्रैव वसतस्तस्य रावणो नाम राक्षसः}
{जहार भार्यां वैदेहीं सम्मोह्यैनं सहानुजम्}


\twolineshloka
{`रामो हृतां राक्षसेन भार्यां श्रुत्वा जटायुषः}
{आतुरः शोकसन्तप्तो रामोऽगच्छद्धरीश्वरम्}


\twolineshloka
{तेन रामः सुसङ्गम्य वानरैश्च महाबलैः}
{आजगामोदधेः पारं सेतुं कृत्वा महार्णवे}


\twolineshloka
{तत्र हत्वा तु पौलस्त्यान्ससुहृद्गणबान्धवान्}
{मायाविनं महाघोरं रावणं लोककण्टकम्'}


\twolineshloka
{सुरासुरैरवध्यं तं देवब्राह्मणकण्टकम्}
{जघान स महाबाहुः पौलस्त्यं सगणं रणे}


\twolineshloka
{`हत्वा तत्र रिपुं सङ्ख्ये भार्यया सह सङ्गतः}
{स च लङ्केश्वरं चक्रे धर्मात्मानं विभीषणम्}


\twolineshloka
{भार्यया सह संयुक्तस्ततो वानरसेनया}
{अयोध्यामागतो वीरः पुष्पकेण विराजता}


\twolineshloka
{तत्र राजन्प्रविष्टः सन्नयोध्यायां महायशाः}
{मातॄर्वयस्यान्सचिवानृत्विजः सपुरोहितान्}


\twolineshloka
{शुश्रूषमाणः सततं मन्त्रिभिश्चाभिषेचितः}
{विसृज्य हरिराजानं हनुमन्तं सहाङ्गदम्}


\twolineshloka
{भ्रातरं भरतं वीरं शत्रुघ्नं चैव लक्ष्मणम्}
{पूजयन्परया प्रीत्या वैदेह्या चाभिपूजितः}


\twolineshloka
{दशवर्षसहस्राणि दशवर्षशतानि च}
{चतुःसगारपर्यन्तां पृथिवीमन्वशासत}


\twolineshloka
{अश्वमेधशतैरीजं क्रतुभिर्भूरिदक्षिणैः}
{यश्च विप्रप्रसादेन सर्वकामानवाप्य च'}


\twolineshloka
{सम्प्राप्य विधिवद्राज्यं सर्वभूतानुकम्पनः}
{`सर्वद्वीपानवष्टभ्य प्रजा धर्मेण पालयन्'}


\threelineshloka
{स निर्गलं मुख्यतममश्वमेधशतं प्रभुः}
{आजहार सुरेशस्य हविषा मुदमाहरन्}
{अन्यैश्च विविधैर्यज्ञैरीजे बहुगुणैर्नृपः}


\twolineshloka
{क्षुत्पिपासेऽजयद्रामः सर्वरोगांश्च देहिनाम्}
{सततं गुणसम्पन्नो दीप्यमानः स्वतेजसा}


\threelineshloka
{अतिसर्वाणि भूतानि रामो दाशरथिर्बभौ}
{ऋषीणां देवतानां च मानुषाणां च सर्वशः}
{पृथिव्यां सह वासोऽभूद्रामे राज्यं प्रशासति}


\twolineshloka
{नाहीयत तदा प्रामः प्राणिनां न तदा व्यथा}
{प्राणापानौ समावास्तां रामे राज्यं प्रशासति}


\twolineshloka
{पर्यदीप्यन्त तेजांसि तदाऽनर्थाश्च नाभवन्}
{दीर्घायुषः प्रजाः सर्वा युवा न म्रियते तदा}


\twolineshloka
{वेदैश्चतुर्भिः सुप्रीताः प्राप्नुवन्ति दिवौकसः}
{हव्यं कव्यं च विविधं निष्पूर्तं हुतमेव च}


\twolineshloka
{अदंशमशका देशा नष्टव्यालसरीसृपाः}
{नाप्यु प्राणभृतां मृत्युर्नाकाले ज्वलनोऽदहत्}


\twolineshloka
{अधर्मरुचयो लुब्धा मूर्खा वा नाभवंस्तदा}
{शिष्टेष्टप्राज्ञकर्माणः सर्वे वर्णास्तदाऽभवन्}


\twolineshloka
{स्वधां पूजां च रक्षोभिर्जनस्थाने प्रणाशिताम्}
{प्रादान्निहत्य रक्षांसि पितृदेवेभ्य ईश्वरः}


\twolineshloka
{सहस्रपुत्राः पुरुषा दशवर्षशतायुषः}
{न च ज्येष्ठाः कनिष्ठेभ्यस्तदा श्राद्धानि कुर्वते}


\twolineshloka
{श्यामो युवा लोहिताक्षो मत्तमातङ्गविक्रमः}
{आजानुबाहुः सुभुजः सिंहस्कन्धो महाबलः}


\twolineshloka
{दशवर्षसहस्राणि दशवर्षशतानि च}
{सर्वभूतमनःकान्तो रामो राज्यमकारयत्}


\twolineshloka
{रामो रामो राम इति प्रजानामभवत्कथा}
{रामभूतं जगदभूद्रामे राज्यं प्रशासति}


\twolineshloka
{चतुर्विधाः प्रजा रामः स्वर्गं नीत्वा दिवं गतः}
{आत्मेच्छया प्रतिष्ठाप्य राजवंशमिहाष्टधा}


\threelineshloka
{स चेन्ममार सृञ्जय चतुर्भद्रतरस्त्वया}
{पुत्रात्पुण्यतरस्तुभ्यं मा पुत्रमनुतप्यथाः}
{अयज्वानमदक्षिण्यमभि श्वैत्येत्युदाहरत्}


॥इति श्रीमन्महाभारते द्रोणपर्वणि अभिमन्युवध-पर्वणि एकोनषष्टितमोऽध्यायः॥६०॥

\closesection
    \chapt{रामचरित्रम्}

\src{देवी-भागवतम्}{तृतीयः स्कन्धः}{अध्यायाः २८}{श्लोकाः १--६९}
\vakta{व्यासः}
\shrota{जनमेजयः}
\tags{concise, complete}
\notes{This passage from the Devi Bhagavatam describes how Rama and Lakshmana accompanied sage Vishvamitra, killed demons including Tadaka, led to Rama winning Sita by breaking Shiva's bow, their exile to the forest due to Queen Kaikeyi's demands, and culminates with Ravana disguising himself to approach Sita while She was alone after tricking Rama and Lakshman away.}
\textlink{https://sa.wikisource.org/wiki/देवीभागवतपुराणम्/स्कन्धः_०३/अध्यायः_२८}
\translink{}

\storymeta

\sect{रामचरित्रवर्णनम्}


\uvacha{जनमेजय उवाच}


\twolineshloka
{कथं रामेण तच्चीर्णं व्रतं देव्याः सुखप्रदम्}
{राज्यभ्रष्टः कथं सोऽथ कथं सीता हृता पुनः}% ॥ १ ॥

\uvacha{व्यास उवाच}


\twolineshloka
{राजा दशरथः श्रीमानयोध्याधिपतिः पुरा}
{सूर्यवंशधरश्चासीद्देवब्राह्मणपूजकः}% ॥ २ ॥

\twolineshloka
{चत्वारो जज्ञिरे तस्य पुत्रा लोकेषु विश्रुताः}
{रामलक्ष्मणशत्रुघ्ना भरतश्चेति नामतः}% ॥ ३ ॥

\twolineshloka
{राज्ञः प्रियकराः सर्वे सदृशा गुणरूपतः}
{कौसल्यायाः सुतो रामः कैकेय्या भरतः स्मृतः}% ॥ ४ ॥

\twolineshloka
{सुमित्रातनयौ जातौ यमलौ द्वौ मनोहरौ}
{ते जाता वै किशोराश्च धनुर्बाणधराः किल}% ॥ ५ ॥

\twolineshloka
{सूनवः कृतसंस्कारा भूपतेः सुखवर्धकाः}
{कौशिकेन तदाऽऽगत्य प्रार्थितो रघुनन्दनः}% ॥ ६ ॥

\twolineshloka
{राघवं मखरक्षार्थं सूनुं षोडशवार्षिकम्}
{तस्मै सोऽयं ददौ रामं कौशिकाय सलक्ष्मणम्}% ॥ ७ ॥

\twolineshloka
{तौ समेत्य मुनिं मार्गे जग्मतुश्चारुदर्शनौ}
{ताटका निहता मार्गे राक्षसी घोरदर्शना}% ॥ ८ ॥

\twolineshloka
{रामेणैकेन बाणेन मुनीनां दुःखदा सदा}
{यज्ञरक्षा कृता तत्र सुबाहुर्निहतः शठः}% ॥ ९ ॥

\twolineshloka
{मारीचोऽथ मृतप्रायो निक्षिप्तो बाणवेगतः}
{एवं कृत्वा महत्कर्म यज्ञस्य परिरक्षणम्}% ॥ १० ॥

\twolineshloka
{गतास्ते मिथिलां सर्वे रामलक्ष्मणकौशिकाः}
{अहल्या मोचिता शापान्निष्पापा सा कृताऽबला}% ॥ ११ ॥

\twolineshloka
{विदेहनगरे तौ तु जग्मतुर्मुनिना सह}
{बभञ्ज शिवचापञ्च जनकेन पणीकृतम्}% ॥ १२ ॥

\twolineshloka
{उपयेमे ततः सीतां जानकीञ्च रमांशजाम्}
{लक्ष्मणाय ददौ राजा पुत्रीमेकां तथोर्मिलाम्}% ॥ १३ ॥

\twolineshloka
{कुशध्वजसुते कन्ये प्रापतुर्भ्रातरावुभौ}
{तथा भरतशत्रुघ्नौ सुशिलौ शुभलक्षणौ}% ॥ १४ ॥

\twolineshloka
{एवं दारक्रियास्तेषां भ्रातॄणां चाभवन्नृप}
{चतुर्णां मिथिलायां तु यथाविधि विधानतः}% ॥ १५ ॥

\twolineshloka
{राज्ययोग्यं सुतं दृष्ट्वा राजा दशरथस्तदा}
{राघवाय धुरं दातुं मनश्चक्रे निजाय वै}% ॥ १६ ॥

\twolineshloka
{सम्भारं विहितं दृष्ट्वा कैकेयी पूर्वकल्पितौ}
{वरौ सम्प्रार्थयामास भर्तारं वशवर्तिनम्}% ॥ १७ ॥

\twolineshloka
{राज्यं सुताय चैकेन भरताय महात्मने}
{रामाय वनवासञ्च चतुर्दशसमास्तथा}% ॥ १८ ॥

\twolineshloka
{रामस्तु वचनात्तस्याः सीतालक्ष्मणसंयुतः}
{जगाम दण्डकारण्यं राक्षसैरुपसेवितम्}% ॥ १९ ॥

\twolineshloka
{राजा दशरथः पुत्रविरहेण प्रपीडितः}
{जहौ प्राणानमेयात्मा पूर्वशापमनुस्मरन्}% ॥ २० ॥

\twolineshloka
{भरतः पितरं दृष्ट्वा मृतं मातृकृतेन वै}
{राज्यमृद्धं न जग्राह भ्रातुः प्रियचिकीर्षया}% ॥ २१ ॥

\twolineshloka
{पञ्चवट्यां वसन् रामो रावणावरजां वने}
{शूर्पणखां विरूपां वै चकारातिस्मरातुराम्}% ॥ २२ ॥

\twolineshloka
{खरादयस्तु तां दृष्ट्वा छिन्ननासां निशाचराः}
{चक्रुः सङ्ग्राममतुलं रामेणामिततेजसा}% ॥ २३ ॥

\twolineshloka
{स जघान खरादींश्च दैत्यानतिबलान्वितान्}
{मुनीनां हितमन्विच्छन् रामः सत्यपराक्रमः}% ॥ २४ ॥

\twolineshloka
{गत्वा शूर्पणखा लङ्कां खरदूषणघातनम्}
{दूषिता कथयामास रावणाय च राघवात्}% ॥ २५ ॥

\twolineshloka
{सोऽपि श्रुत्वा विनाशं तं जातः क्रोधवशः खलः}
{जगाम रथमारुह्य मारीचस्याश्रमं तदा}% ॥ २६ ॥

\twolineshloka
{कृत्वा हेममृगं नेतुं प्रेषयामास रावणः}
{सीताप्रलोभनार्थाय मायाविनमसम्भवम्}% ॥ २७ ॥

\twolineshloka
{सोऽथ हेममृगो भूत्वा सीतादृष्टिपथं गतः}
{मायावी चातिचित्राङ्गश्चरन्प्रबलमन्तिके}% ॥ २८ ॥

\twolineshloka
{तं दृष्ट्वा जानकी प्राह राघवं दैवनोदिता}
{चर्मानयस्व कान्तेति स्वाधीनपतिका यथा}% ॥ २९ ॥

\twolineshloka
{अविचार्याथ रामोऽपि तत्र संस्थाप्य लक्ष्मणम्}
{सशरं धनुरादाय ययौ मृगपदानुगः}% ॥ ३० ॥

\twolineshloka
{सारङ्गोऽपि हरिं दृष्ट्वा मायाकोटिविशारदः}
{दृश्यादृश्यो बभूवाथ जगाम च वनान्तरम्}% ॥ ३१ ॥

\twolineshloka
{मत्वा हस्तगतं रामः क्रोधाकृष्टधनुः पुनः}
{जघान चातितीक्ष्णेन शरेण कृत्रिमं मृगम्}% ॥ ३२ ॥

\twolineshloka
{स हतोऽतिबलात्तेन चुक्रोश भृशदुःखितः}
{हा लक्ष्मण हतोऽस्मीति मायावी नश्वरः खलः}% ॥ ३३ ॥

\twolineshloka
{स शब्दस्तुमुलस्तावज्जानक्या संश्रुतस्तदा}
{राघवस्येति सा मत्वा दीना देवरमब्रवीत्}% ॥ ३४ ॥

\twolineshloka
{गच्छ लक्ष्मण तूर्णं त्वं हतोऽसौ रघुनन्दनः}
{त्वामाह्वयति सौ‌मित्रे साहाय्यं कुरु सत्वरम्}% ॥ ३५ ॥

\twolineshloka
{तत्राह लक्ष्मणः सीतामम्ब रामवधादपि}
{नाहं गच्छेऽद्य मुक्त्वा त्वामसहायामिहाश्रमे}% ॥ ३६ ॥

\twolineshloka
{आज्ञा मे राघवस्यात्र तिष्ठेति जनकात्मजे}
{तदतिक्रमभीतोऽहं न त्यजामि तवान्तिकम्}% ॥ ३७ ॥

\twolineshloka
{दूरं वै राघवं दृष्ट्वा वने मायाविना किल}
{त्यक्त्वा त्वां नाधिगच्छामि पदमेकं शुचिस्मिते}% ॥ ३८ ॥

\twolineshloka
{कृरु धैर्यं न मन्येऽद्य रामं हन्तुं क्षमं क्षिप्तौ}
{नाहं त्यक्त्वा गमिष्यामि विलंघ्य रामभाषितम्}% ॥ ।३९ ॥

\uvacha{व्यास उवाच}


\twolineshloka
{रुदती सुदती प्राह ते तदा विधिनोदिता}
{अक्रूरा वचनं क्रूरं लक्ष्मणं शुभलक्षणम्}% ॥ ४० ॥

\twolineshloka
{अहं जानामि सौ‌मित्रे सानुरागं च मां प्रति}
{प्रेरितं भरतेनैव मदर्थमिह सङ्गतम्}% ॥ ४१ ॥

\twolineshloka
{नाहं तथाविधा नारी स्वैरिणी कुहकाधम}
{मृते रामे पतिं त्वां न कर्तुमिच्छामि कामतः}% ॥ ४२ ॥

\twolineshloka
{नागमिष्यति चेद्रामो जीवितं सन्त्यजाम्यहम्}
{विना तेन न जीवामि विधुरा दुःखिता भृशम्}% ॥ ४३ ॥

\twolineshloka
{गच्छ वा तिष्ठ सौ‍मित्रे न जानेऽहं तवेप्सितम्}
{क्व गतं तेऽद्य सौहार्दं ज्येष्ठे धर्मरते किल}% ॥ ४४ ॥

\twolineshloka
{तच्छ्रुत्वा वचनं तस्या लक्ष्मणो दीनमानसः}
{प्रोवाच रुद्धकण्ठस्तु तां तदा जनकात्मजाम्}% ॥ ४५ ॥

\twolineshloka
{किमात्थ क्षितिजे वाक्यं मयि क्रूरतरं किल}
{किं वदस्यत्यनिष्टं ते भावि जाने धिया ह्यहम्}% ॥ ४६ ॥

\twolineshloka
{इत्युक्त्वा निर्ययौ वीरस्तां त्यक्त्वा प्ररुदन्भृशम्}
{अग्रजस्य ययौ पश्यञ्छोकार्तः पृथिवीपते}% ॥ ४७ ॥

\twolineshloka
{गतेऽथ लक्ष्मणे तत्र रावणः कपटाकृतिः}
{भिक्षुवेषं ततः कृत्वा प्रविवेश तदाश्रमे}% ॥ ४८ ॥

\twolineshloka
{जानकी तं यतिं मत्वा दत्त्वार्घ्यं वन्यमादरात्}
{भैक्ष्यं समर्पयामास रावणाय दुरात्मने}% ॥ ४९ ॥

\twolineshloka
{तां पप्रच्छ स दुष्टात्मा नम्रपूर्वं मृदुस्वरम्}
{काऽसि पद्मपलाशाक्षि वने चैकाकिनी प्रिये}% ॥ ५० ॥

\twolineshloka
{पिता कस्तेऽथ वामोरु भ्राता कः कः पतिस्तव}
{मूढेवैकाकिनी चात्र स्थिताऽसि वरवर्णिनि}% ॥ ५१ ॥

\twolineshloka
{निर्जने विपिने किं त्वं सौधार्हा त्वमसि प्रिये}
{उटजे मुनिपत्‍नीवद्देवकन्यासमप्रभा}% ॥ ५२ ॥

\uvacha{व्यास उवाच}


\twolineshloka
{इति तद्वचनं श्रुत्वा प्रत्युवाच विदेहजा}
{दिव्यं दिष्ट्या यतिं ज्ञात्वा मन्दोदर्याः पतिं तदा}% ॥ ५३ ॥

\twolineshloka
{राजा दशरथः श्रीमांश्चत्वारस्तस्य वै सुताः}
{तेषां ज्येष्ठः पतिर्मेऽस्ति रामनामेति विश्रुतः}% ॥ ५४ ॥

\twolineshloka
{विवासितोऽथ कैकेय्या कृते भूपतिना वरे}
{चतुर्दश समा रामो वसतेऽत्र सलक्ष्मणः}% ॥ ५५ ॥

\twolineshloka
{जनकस्य सुता चाहं सीतानाम्नीति विश्रुता}
{भङ्क्त्वा शैवं धनुः कामं रामेणाहं विवाहिता}% ॥ ५६ ॥

\twolineshloka
{रामबाहुबलेनात्र वसामो निर्भया वने}
{काञ्चनं मृगमालोक्य हन्तुं मे निर्गतः पतिः}% ॥ ५७ ॥

\twolineshloka
{लक्ष्मणोऽपि पुनः श्रुत्वा रवं भ्रातुर्गतोऽधुना}
{तयोर्बाहुबलादत्र निर्भयाऽहं वसामि वै}% ॥ ५८ ॥

\twolineshloka
{मयेदं कथितं सर्वं वृत्तान्तं वनवासके}
{तेऽत्रागत्यार्हणां ते वै करिष्यन्ति यथाविधि}% ॥ ५९ ॥

\twolineshloka
{यतिर्विष्णुस्वरूपोऽसि तस्मात्त्वं पूजितो मया}
{आश्रमो विपिने घोरे कृतोऽस्ति रक्षसां कुले}% ॥ ६० ॥

\twolineshloka
{तस्मात्त्वां परिपृच्छामि सत्यं ब्रूहि ममाग्रतः}
{कोऽसि त्रिदण्डिरूपेण विपिने त्वं समागतः}% ॥ ६१ ॥


\uvacha{रावण उवाच}


\twolineshloka
{लङ्केशोऽहं मरालाक्षि श्रीमान्मन्दोदरीपतिः}
{त्वत्कृते तु कृतं रूपं मयेत्थं शोभनाकृते}% ॥ ६२ ॥

\twolineshloka
{आगतोऽहं वरारोहे भगिन्या प्रेरितोऽत्र वै}
{जनस्थाने हतौ श्रुत्वा भ्रातरौ खरदूषणौ}% ॥ ६३ ॥

\twolineshloka
{अङ्गीकुरु नृपं मां त्वं त्यक्त्वा तं मानुषं पतिम्}
{हृतराज्यं गतश्रीकं निर्बलं वनवासिनम्}% ॥ ६४ ॥

\twolineshloka
{पट्टराज्ञी भव त्वं मे मन्दोदर्युपरि स्फुटम्}
{दासोऽस्मि तव तन्वङ्‌गि स्वामिनी भव भामिनि}% ॥ ६५ ॥

\twolineshloka
{जेताऽहं लोकपालानां पतामि तव पादयोः}
{करं गृहाण मेऽद्य त्वं सनाथं कुरु जानकि}% ॥ ६६ ॥

\twolineshloka
{पिता ते याचितः पूर्वं मया वै त्वत्कृतेऽबले}
{जनको मामुवाचेत्थं पणबन्धो मया कृतः}% ॥ ६७ ॥

\twolineshloka
{रुद्रचापभयान्नाहं सम्प्राप्तस्तु स्वयंवरे}
{मनो मे संस्थितं तावन्निमग्नं विरहातुरम्}% ॥ ६८ ॥

\twolineshloka
{वनेऽत्र संस्थितां श्रुत्वा पूर्वानुरागमोहितः}
{आगतोऽस्म्यसितापाङ्‌गि सफलं कुरु मे श्रमम्}% ॥ ६९ ॥


॥इति श्रीदेवीभागवते महापुराणेऽष्टादशसाहस्र्यां संहितायां तृतीयस्कन्धे रामचरित्रवर्णनं नाम अष्टाविंशोऽध्यायः॥

    \sect{लक्ष्मणकृतरामशोकसान्त्वनम्}

\src{देवी-भागवतम्}{तृतीयः स्कन्धः}{अध्यायः २८}{श्लोकाः १--५५}
\vakta{व्यासः}
\shrota{जनमेजयः}
\tags{concise, complete}
\notes{This chapter describes how Sita rejected Ravana's advances and was forcibly abducted to Lanka despite Jatayu's heroic attempt to stop him, followed by Rama's discovery of Her disappearance and His profound grief, being consoled by Lakshman's encouraging words about the cyclical nature of fortune and Their ability to rescue Her with the help of Their vānara allies.}
\textlink{https://sa.wikisource.org/wiki/देवीभागवतपुराणम्/स्कन्धः_०३/अध्यायः_२९}
\translink{}

\storymeta

\uvacha{व्यास उवाच}


\twolineshloka
{तदाकर्ण्य वचो दुष्टं जानकी भयविह्वला}
{वेपमाना स्थिरं कृत्वा मनो वाचमुवाच ह}% ॥ १ ॥

\twolineshloka
{पौलस्त्य किमसद्वाक्यं त्वमात्थ स्मरमोहितः}
{नाहं वै स्वैरिणी किन्तु जनकस्य कुलोद्‌भवा}% ॥ २ ॥

\twolineshloka
{गच्छ लङ्कां दशास्य त्वं राम त्वां वै हनिष्यति}
{मत्कृते मरणं तत्र भविष्यति न संशयः}% ॥ ३ ॥

\twolineshloka
{इत्युक्त्वा पर्णशालायां गता सा वह्निसन्निधौ}
{गच्छ गच्छेति वदती रावणं लोकरावणम्}% ॥ ४ ॥

\twolineshloka
{सोऽथ कृत्वा निजं रूपं जगामोटजमन्तिकम्}
{बलाज्जग्राह तां बालां रुदती भयविह्वलाम्}% ॥ ५ ॥

\twolineshloka
{रामरामेति क्रन्दन्ती लक्ष्मणेति मुहुर्मुहुः}
{गृहीत्वा निर्गतः पापो रथमारोप्य सत्वरः}% ॥ ६ ॥

\twolineshloka
{गच्छन्नरुणपुत्रेण मार्गे रुद्धो जटायुषा}
{सङ्ग्रामोऽभून्महारौद्रस्तयोस्तत्र वनान्तरे}% ॥ ७ ॥

\twolineshloka
{हत्वा तं तां गृहीत्वा च गतोऽसौ राक्षसाधिपः}
{लङ्कायां क्रन्दती तात कुररीव दुरात्मनः}% ॥ ८ ॥

\twolineshloka
{अशोकवनिकायां सा स्थापिता राक्षसीयुता}
{स्ववृत्तान्नैव चलिता सामदानादिभिः किल}% ॥ ९ ॥

\twolineshloka
{रामोऽपि तं मृगं हत्वा जगामादाय निर्वृतः}
{आयान्तं लक्ष्मणं वीक्ष्य किं कृतं तेऽनुजासमम्}% ॥ १० ॥

\twolineshloka
{एकाकिनीं प्रियां हित्वा किमर्थं त्वमिहागतः}
{श्रुत्वा स्वनं तु पापस्य राघवस्त्वब्रवीदिदम्}% ॥ ११ ॥

\twolineshloka
{सौ‌मित्रिस्त्वब्रवीद्वाक्यं सीतावाग्बाणपीडितः}
{प्रभोऽत्राहं समायातः कालयोगान्न संशयः}% ॥ १२ ॥

\twolineshloka
{तदा तौ पर्णशालायां गत्वा वीक्ष्यातिदुःखितौ}
{जानक्यन्वेषणे यत्‍नमुभौ कर्तुं समुद्यतौ}% ॥ १३ ॥

\twolineshloka
{मार्गमाणौ तु सम्प्राप्तौ यत्रासौ पतितः खगः}
{जटायुः प्राणशेषस्तु पतितः पृथिवीतले}% ॥ १४ ॥

\twolineshloka
{तेनोक्तं रावणेनाद्य हृता‍‍ऽसौ जनकात्मजा}
{मया निरुद्धः पापात्मा पातितोऽहं मृधे पुनः}% ॥ १५ ॥

\twolineshloka
{इत्युक्त्वाऽसौ गतप्राणः संस्कृतो राघवेण वै}
{कृत्वौर्घ्वदैहिकं रामलक्ष्मणौ निर्गतौ ततः}% ॥ १६ ॥

\twolineshloka
{कबन्धं घातयित्वासौ शापाच्चामोचयत्प्रभुः}
{वचनात्तस्य हरिणा सख्यं चक्रेऽथ राघवः}% ॥ १७ ॥

\twolineshloka
{हत्वा च वालिनं वीरं किष्किन्धाराज्यमुत्तमम्}
{सुग्रीवाय ददौ रामः कृतसख्याय कार्यतः}% ॥ १८ ॥

\twolineshloka
{तत्रैव वार्षिकान्मासांस्तस्थौ लक्ष्मणसंयुतः}
{चिन्तयञ्जानकीं चित्ते दशाननहृतां प्रियाम्}% ॥ १९ ॥

\twolineshloka
{लक्ष्मणं प्राह रामस्तु सीताविरहपीडितः}
{सौ‌मित्रे कैकयसुता जाता पूर्णमनोरथा}% ॥ २० ॥

\twolineshloka
{न प्राप्ता जानकी नूनं नाहं जीवामि तां विना}
{नागमिष्याम्ययोध्यायामृते जनकनन्दिनीम्}% ॥ २१ ॥

\twolineshloka
{गतं राज्यं वने वासो मृतस्तातो हृता प्रिया}
{पीडयन्मां स दुष्टात्मा दैवो‍ऽग्रे किं करिष्यति}% ॥ २२ ॥

\twolineshloka
{दुर्ज्ञेयं भवितव्यं हि प्राणिनां भरतानुज}
{आवयोः का गतिस्तात भविष्यति सुदुःखदा}% ॥ २३ ॥

\twolineshloka
{प्राप्य जन्म मनोर्वंशे राजपुत्रावुभौ किल}
{वनेऽतिदुःखभोक्तारौ जातौ पूर्वकृतेन च}% ॥ २४ ॥

\twolineshloka
{त्यक्त्वा त्वमपि भोगांस्तु मया सह विनिर्गतः}
{दैवयोगाच्च सौ‌मित्रे भुङ्क्ष्व दुःखं दुरत्ययम्}% ॥ २५ ॥

\twolineshloka
{न कोऽप्यस्मत्कुले पूर्वं मत्समो दुःखभाङ्नरः}
{अकिञ्चनोऽक्षमः क्लिष्टो न भूतो न भविष्यति}% ॥ २६ ॥

\twolineshloka
{किं करोम्यद्य सौ‌मित्रे मग्नोऽस्मि दुःखसागरे}
{न चास्ति तरणोपायो ह्यसहायस्य मे किल}% ॥ २७ ॥

\twolineshloka
{न वित्तं न बलं वीर त्वमेकः सहचारकः}
{कोपं कस्मिन्करोम्यद्य भोगेस्मिन्स्वकृतेऽनुज}% ॥ २८ ॥

\twolineshloka
{गतं हस्तगतं राज्यं क्षणादिन्द्रासनोपमम्}
{वने वासस्तु सम्प्राप्तः को वेद विधिनिर्मितम्}% ॥ २९ ॥

\twolineshloka
{बालभावाच्च वैदेही चलिता चावयोः सह}
{नीता दैवेन दुष्टेन श्यामा दुःखतरां दशाम्}% ॥ ३० ॥

\twolineshloka
{लङ्केशस्य गृहे श्यामा कथं दुःखं भविष्यति}
{पतिव्रता सुशीला च मयि प्रीतियुता भृशम्}% ॥ ३१ ॥

\twolineshloka
{न च लक्ष्मण वैदेही सा तस्य वशगा भवेत्}
{स्वैरिणीव वरारोहा कथं स्याज्जनकात्मजा}% ॥ ३२ ॥

\twolineshloka
{त्यजेत्प्राणान्नियन्तृत्वे मैथिली भरतानुज}
{न रावणस्य वशगा भवेदिति सुनिश्चितम्}% ॥ ३३ ॥

\twolineshloka
{मृता चेज्जानकी वीर प्राणांस्त्यक्ष्याम्यसंशयम्}
{मृता चेदसितापाङ्गीं किं मे देहेन लक्ष्मण}% ॥ ३४ ॥

\twolineshloka
{एवं विलपमानं तं रामं कमललोचनम्}
{लक्ष्मणः प्राह धर्मात्मा सान्त्वयन्नृतया गिरा}% ॥ ३५ ॥

\twolineshloka
{धैर्यं कुरु महाबाहो त्यक्त्वा कातरतामिह}
{आनयिष्यामि वैदेहीं हत्वा तं राक्षसाधमम्}% ॥ ३६ ॥

\twolineshloka
{आपदि सम्पदि तुल्या धैर्याद्‌भवन्ति ते धीराः}
{अल्पधियस्तु निमग्नाः कष्टे भवन्ति विभवेऽपि}% ॥ ३७ ॥

\twolineshloka
{संयोगो विप्रयोगश्च दैवाधीनावुभावपि}
{शोकस्तु कीदृशस्तत्र देहेनात्मनि च क्वचित्}% ॥ ३८ ॥

\twolineshloka
{राज्याद्यथा वने वासो वैदेह्या हरणं यथा}
{तथा काले समीचीने संयोगोऽपि भविष्यति}% ॥ ३९ ॥

\twolineshloka
{प्राप्तव्यं सुखदुःखानां भोगान्निर्वर्तनं क्वचित्}
{नान्यथा जानकीजाने तस्माच्छोकं त्यजाधुना}% ॥ ४० ॥

\twolineshloka
{वानराः सन्ति भूयांसो गमिष्यन्ति चतुर्दिशम्}
{शुद्धिं जनकनन्दिन्या आनयिष्यन्ति ते किल}% ॥ ४१ ॥

\twolineshloka
{ज्ञात्वा मार्गस्थितिं तत्र गत्वा कृत्वा पराक्रमम्}
{हत्वा तं पापकर्माणमानयिष्यामि मैथिलीम्}% ॥ ४२ ॥

\twolineshloka
{ससैन्यं भरतं वाऽपि समाहूय सहानुजम्}
{हनिष्यामो वयं शत्रुं किं शोचसि वृथाग्रज}% ॥ ४३ ॥

\twolineshloka
{रघुणैकरथेनैव जिताः सर्वा दिशः पुरा}
{तद्वंशजः कथं शोकं कर्तुमर्हसि राघव}% ॥ ४४ ॥

\twolineshloka
{एकोऽहं सकलाञ्जेतुं समर्थोऽस्मि सुरासुरान्}
{किं पुनः ससहायो वै रावणं कुलपांसनम्}% ॥ ४५ ॥

\twolineshloka
{जनकं वा समानीय साहाय्ये रघुनन्दन}
{हनिष्यामि दुराचारं रावणं सुरकण्टकम्}% ॥ ४६ ॥

\twolineshloka
{सुखस्यानन्तरं दुःखं दुःखस्यानन्तरं सुखम्}
{चक्रनेमिरिवैकं यन्न भवेद्‌रघुनन्दन}% ॥ ४७ ॥

\twolineshloka
{मनोऽतिकातरं यस्य सुखदुःखसमुद्‌भवे}
{स शोकसागरे मग्नो न सुखी स्यात्कदाचन}% ॥ ४८ ॥

\twolineshloka
{इन्द्रेण व्यसनं प्राप्तं पुरा वै रघुनन्दन}
{नहुषः स्थापितो देवैः सर्वैर्मघवतः पदे}% ॥ ४९ ॥

\twolineshloka
{स्थितः पङ्कजमध्ये च बहुवर्षगणानपि}
{अज्ञातवासं मघवा भीतस्त्यक्त्वा निजं पदम्}% ॥ ५० ॥

\twolineshloka
{पुनः प्राप्तं निजस्थानं काले विपरिवर्तिते}
{नहुषः पतितो भूमौ शापादजगराकृतिः}% ॥ ५१ ॥

\twolineshloka
{इन्द्राणीं कामयानस्तु ब्राह्मणानवमन्य च}
{अगस्तिकोपात्सञ्जातः सर्पदेहो महीपतिः}% ॥ ५२ ॥

\twolineshloka
{तस्माच्छोको न कर्तव्यो व्यसने सति राघव}
{उद्यमे चित्तमास्थाय स्थातव्यं वै विपश्चिता}% ॥ ५३ ॥

\twolineshloka
{सर्वज्ञोऽसि महाभाग समर्थोऽसि जगत्पते}
{किं प्राकृत इवात्यर्थं कुरुषे शोकमात्मनि}% ॥ ५४ ॥

\uvacha{व्यास उवाच}


\twolineshloka
{इति लक्ष्मणवाक्येन बोधितो रघुनन्दनः}
{त्यक्त्वा शोकं तथात्यर्थं बभूव विगतज्वरः}% ॥ ५५ ॥


॥इति श्रीदेवीभागवते महापुराणेऽष्टादशसाहस्र्यां संहितायां तृतीयस्कन्धे लक्ष्मणकृतरामशोकसान्त्वनं नामैकोनत्रिंशोऽध्यायः॥

    \sect{रामाय देवीवरदानम्}

\src{देवी-भागवतम्}{तृतीयः स्कन्धः}{अध्यायाः २८}{श्लोकाः १--६३}
\vakta{व्यासः}
\shrota{जनमेजयः}
\notes{This passage describes how the sage Narada appeared to console the grieving Rama, revealed that Sita's abduction was destined for Ravana's destruction, instructed Rama to perform the nine-day Devi worship (Navaratri) in the month of Ashvin, during which the Goddess appeared and blessed Rama with the assurance that he would defeat Ravana with the help of the vānaras, after which Rama successfully built the bridge across the ocean and killed Ravana.}
\textlink{https://sa.wikisource.org/wiki/देवीभागवतपुराणम्/स्कन्धः_०३/अध्यायः_३०}
\translink{}

\storymeta

\uvacha{व्यास उवाच}

\twolineshloka
{एवं तौ संविदं कृत्वा यावत्तूष्णीं बभूवतुः}
{आजगाम तदाऽऽकाशान्नारदो भगवानृषिः}% ॥ १ ॥

\twolineshloka
{रणयन्महतीं वीणां स्वरग्रामविभूषिताम्}
{गायन्बृहद्रथं साम तदा तमुपतस्थिवान्}% ॥ २ ॥

\twolineshloka
{दृष्ट्वा तं राम उत्थाय ददावथ वृषं शुभम्}
{आसनं चार्घ्यपाद्यञ्च कृतवानमितद्युतिः}% ॥ ३ ॥

\twolineshloka
{पूजां परमिकां कृत्वा कृताञ्जलिरुपस्थितः}
{उपविष्टः समीपे तु कृताज्ञो मुनिना हरिः}% ॥ ४ ॥

\twolineshloka
{उपविष्टं तदा रामं सानुजं दुःखमानसम्}
{पप्रच्छ नारदः प्रीत्या कुशलं मुनिसत्तमः}% ॥ ५ ॥

\twolineshloka
{कथं राघव शोकार्तो यथा वै प्राकृतो नरः}
{हृतां सीतां च जानामि रावणेन दुरात्मना}% ॥ ६ ॥

\twolineshloka
{सुरसद्मगतश्चाहं श्रुतवाञ्जनकात्मजाम्}
{पौलस्त्येन हृतां मोहान्मरणं स्वमजानता}% ॥ ७ ॥

\twolineshloka
{तव जन्म च काकुत्स्थ पौलस्त्यनिधनाय वै}
{मैथिलीहरणं जातमेतदर्थं नराधिप}% ॥ ८ ॥

\twolineshloka
{पूर्वजन्मनि वैदेही मुनिपुत्री तपस्विनी}
{रावणेन वने दृष्टा तपस्यन्ती शुचिस्मिता}% ॥ ९ ॥

\twolineshloka
{प्रार्थिता रावणेनासौ भव भार्येति राघव}
{तिरस्कृतस्तयाऽसौ वै जग्राह कबरं बलात्}% ॥ १० ॥

\twolineshloka
{शशाप तत्क्षणं राम रावणं तापसी भृशम्}
{कुपिता त्यक्तुमिच्छन्ती देहं संस्पर्शदूषितम्}% ॥ ११ ॥

\twolineshloka
{दुरात्मंस्तव नाशार्थं भविष्यामि धरातले}
{अयोनिजा वरा नारी त्यक्त्वा देहं जहावपि}% ॥ १२ ॥

\twolineshloka
{सेयं रमांशसम्भूता गृहीता तेन रक्षसा}
{विनाशार्थं कुलस्यैव व्याली स्रगिव सम्भ्रमात्}% ॥ १३ ॥

\twolineshloka
{तव जन्म च काकुत्स्थ तस्य नाशाय चामरैः}
{प्रार्थितस्य हरेरंशादजवंशेऽप्यजन्मनः}% ॥ १४ ॥

\twolineshloka
{कुरु धैर्यं महाबाहो तत्र सा वर्ततेऽवशा}
{सती धर्मरता सीता त्वां ध्यायन्ती दिवानिशम्}% ॥ १५ ॥

\twolineshloka
{कामधेनुपयः पात्रे कृत्वा मघवता स्वयम्}
{पानार्थं प्रेषितं तस्याः पीतं चैवामृतं यथा}% ॥ १६ ॥

\twolineshloka
{सुरभीदुग्धपानात्सा क्षुत्तुड्‌दुःखविवर्जिता}
{जाता कमलपत्राक्षी वर्तते वीक्षिता मया}% ॥ १७ ॥

\twolineshloka
{उपायं कथयाम्यद्य तस्य नाशाय राघव}
{व्रतं कुरुष्व श्रद्धावानाश्विने मासि साम्प्रतम्}% ॥ १८ ॥

\twolineshloka
{नवरात्रोपवासञ्च भगवत्याः प्रपूजनम्}
{सर्वसिद्धिकरं राम जपहोमविधानतः}% ॥ १९ ॥

\twolineshloka
{मेघ्यैश्च पशुभिर्देव्या बलिं दत्त्वा विशंसितैः}
{दशांशं हवनं कृत्वा सशक्तस्त्वं भविष्यसि}% ॥ २० ॥

\twolineshloka
{विष्णुना चरितं पूर्वं महादेवेन ब्रह्मणा}
{तथा मघवता चीर्णं स्वर्गमध्यस्थितेन वै}% ॥ २१ ॥

\twolineshloka
{सुखिना राम कर्तव्यं नवरात्रव्रतं शुभम्}
{विशेषेण च कर्तव्यं पुंसा कष्टगतेन वै}% ॥ २२ ॥

\twolineshloka
{विश्वामित्रेण काकुत्स्थ कृतमेतन्न संशयः}
{भृगुणाऽथ वसिष्ठेन कश्यपेन तथैव च}% ॥ २३ ॥

\twolineshloka
{गुरुणा हृतदारेण कृतमेतन्महाव्रतम्}
{तस्मात्त्वं कुरु राजेन्द्र रावणस्य वधाय च}% ॥ २४ ॥

\twolineshloka
{इन्द्रेण वृत्रनाशाय कृतं व्रतमनुत्तमम्}
{त्रिपुरस्य विनाशाय शिवेनापि पुरा कृतम्}% ॥ २५ ॥

\twolineshloka
{हरिणा मधुनाशाय कृतं मेरौ महामते}
{विधिवत्कुरु काकुत्स्थ व्रतमेतदतन्द्रितः}% ॥ २६ ॥

\uvacha{श्रीराम उवाच}


\twolineshloka
{का देवी किं प्रभावा सा कुतो जाता किमाह्वया}
{व्रतं किं विधिवद्‌ब्रूहि सर्वज्ञोऽसि दयानिधे}% ॥ २७ ॥

\uvacha{नारद उवाच}


\twolineshloka
{शृणु राम सदा नित्या शक्तिराद्या सनातनी}
{सर्वकामप्रदा देवी पूजिता दुःखनाशिनी}% ॥ २८ ॥

\twolineshloka
{कारणं सर्वजन्तूनां ब्रह्मादीनां रघूद्वह}
{तस्याः शक्तिं विना कोऽपि स्पन्दितुं न क्षमो भवेत्}% ॥ २९ ॥

\twolineshloka
{विष्णोः पालनशक्तिः सा कर्तृशक्तिः पितुर्मम}
{रुद्रस्य नाशशक्तिः सा त्वन्याशक्तिः परा शिवा}% ॥ ३० ॥

\twolineshloka
{यच्च किञ्चित्क्वचिद्वस्तु सदसद्‌भुवनत्रये}
{तस्य सर्वस्य या शक्तिस्तदुत्पत्तिः कुतो भवेत्}% ॥ ३१ ॥

\twolineshloka
{न ब्रह्मा न यदा विष्णुर्न रुद्रो न दिवाकरः}
{न चेन्द्राद्याः सुराः सर्वे न धरा न धराधराः}% ॥ ३२ ॥

\twolineshloka
{तदा सा प्रकृतिः पूर्णा पुरुषेण परेण वै}
{संयुता विहरत्येव युगादौ निर्गुणा शिवा}% ॥ ३३ ॥

\twolineshloka
{सा भूत्वा सगुणा पश्चात्करोति भुवनत्रयम्}
{पूर्वं संसृज्य ब्रह्मादीन्दत्त्वा शक्तीश्च सर्वशः}% ॥ ३४ ॥

\twolineshloka
{तां ज्ञात्वा मुच्यते जन्तुर्जन्मसंसारबन्धनात्}
{सा विद्या परमा ज्ञेया वेदाद्या वेदकारिणी}% ॥ ३५ ॥

\twolineshloka
{असंख्यातानि नामानि तस्या ब्रह्मादिभिः किल}
{गुणकर्मविधानैस्तु कल्पितानि च किं ब्रुवे}% ॥ ३६ ॥

\twolineshloka
{अकारादिक्षकारान्तैः स्वरैर्वर्णैस्तु योजितैः}
{असंख्येयानि नामानि भवन्ति रघुनन्दन}% ॥ ३७ ॥


\uvacha{राम उवाच}


\twolineshloka
{विधिं मे ब्रूहि विप्रर्षे व्रतस्यास्य समासतः}
{करोम्यद्यैव श्रद्धावाञ्छ्रीदेव्याः पूजनं तथा}% ॥ ३८ ॥

\uvacha{नारद उवाच}


\twolineshloka
{पीठं कृत्वा समे स्थाने संस्थाप्य जगदम्बिकाम्}
{उपवासान्नवैव त्वं कुरु राम विधानतः}% ॥ ३९ ॥

\twolineshloka
{आचार्योऽहं भविष्यामि कर्मण्यस्मिन्महीपते}
{देवकार्यविधानार्थमुत्साहं प्रकरोम्यहम्}% ॥ ४० ॥

\uvacha{व्यास उवाच}


\twolineshloka
{तच्छ्रुत्वा वचनं सत्यं मत्वा रामः प्रतापवान्}
{कारयित्वा शुभं पीठं स्थापयित्वाम्बिकां शिवाम्}% ॥ ४१ ॥

\twolineshloka
{विधिवत्पूजनं तस्याश्चकार व्रतवान् हरिः}
{सम्प्राप्ते चाश्विने मासि तस्मिन्गिरिवरे तदा}% ॥ ४२ ॥

\twolineshloka
{उपवासपरो रामः कृतवान्व्रतमुत्तमम्}
{होमञ्च विधिवत्तत्र बलिदानञ्च पूजनम्}% ॥ ४३ ॥

\twolineshloka
{भ्रातरौ चक्रतुः प्रेम्णा व्रतं नारदसम्मतम्}
{अष्टम्यां मध्यरात्रे तु देवी भगवती हि सा}% ॥ ४४ ॥

\twolineshloka
{सिंहारूढा ददौ तत्र दर्शनं प्रतिपूजिता}
{गिरिशृङ्गे स्थितोवाच राघवं सानुजं गिरा}% ॥ ४५ ॥

\onelineshloka*
{मेघगम्भीरया चेदं भक्तिभावेन तोषिता}

\uvacha{देव्युवाच}

\onelineshloka
{राम राम महाबाहो तुष्टाऽस्म्यद्म व्रतेन ते}% ॥ ४६ ॥

\twolineshloka
{प्रार्थयस्व वरं कामं यत्ते मनसि वर्तते}
{नारायणांशसम्भूतस्त्वं वंशे मानवेऽनघे}% ॥ ४७ ॥

\twolineshloka
{रावणस्य वधायैव प्रार्थितस्त्वमरैरसि}
{पुरा मत्स्यतनुं कृत्वा हत्वा घोरञ्च राक्षसम्}% ॥ ४८ ॥

\twolineshloka
{त्वया वै रक्षिता वेदाः सुराणां हितमिच्छता}
{भूत्वा कच्छपरूपस्तु धृतवान्मन्दरं गिरिम्}% ॥ ४९ ॥

\twolineshloka
{अकूपारं प्रमन्थानं कृत्वा देवानपोषयः}
{कोलरूपं परं कृत्वा दशनाग्रेण मेदिनीम्}% ॥ ५० ॥

\twolineshloka
{धृतवानसि यद्‌राम हिरण्याक्षं जघान च}
{नारसिंहीं तनुं कृत्वा हिरण्यकशिपुं पुरा}% ॥ ५१ ॥

\twolineshloka
{प्रह्लादं राम रक्षित्वा हतवानसि राघव}
{वामनं वपुरास्थाय पुरा छलितवान्बलिम्}% ॥ ५२ ॥

\twolineshloka
{भूत्वेन्द्रस्यानुजः कामं देवकार्यप्रसाधकः}
{जमदग्निसुतस्त्वं मे विष्णोरंशेन सङ्गतः}% ॥ ५३ ॥

\twolineshloka
{कृत्वान्तं क्षत्रियाणां तु दानं भूमेरदाद्‌द्विजे}
{तथेदानीं तु काकुत्स्थ जातो दशरथात्मज}% ॥ ५४ ॥

\twolineshloka
{प्रार्थितस्तु सुरैः सर्वै रावणेनातिपीडितैः}
{कपयस्ते सहाया वै देवांशा बलवत्तराः}% ॥ ५५ ॥

\twolineshloka
{भविष्यन्ति नरव्याघ्र मच्छक्तिसंयुता ह्यमी}
{शेषांशोऽप्यनुजस्तेऽयं रावणात्मजनाशकः}% ॥ ५६ ॥

\twolineshloka
{भविष्यति न सन्देहः कर्तव्योऽत्र त्वयाऽनघ}
{वसन्ते सेवनं कार्यं त्वया तत्रातिश्रद्धया}% ॥ ५७ ॥

\twolineshloka
{हत्वाऽथ रावणं पापं कुरु राज्यं यथासुखम्}
{एकादश सहस्राणि वर्षाणि पृथिवीतले}% ॥ ५८ ॥

\onelineshloka*
{कृत्वा राज्यं रघुश्रेष्ठ गन्ताऽसि त्रिदिवं पुनः}

\uvacha{व्यास उवाच}

\onelineshloka
{इत्युक्त्वान्तर्दधे देवी रामस्तु प्रीतमानसः}% ॥ ५९ ॥

\twolineshloka
{समाप्य तद्‌व्रतं चक्रे प्रयाणं दशमीदिने}
{विजयापूजनं कृत्वा दत्त्वा दानान्यनेकशः}% ॥ ६० ॥

\fourlineindentedshloka
{कपिपतिबलयुक्तः सानुजः श्रीपतिश्च}
{प्रकटपरमशक्त्या प्रेरितः पूर्णकामः}
{उदधितटगतोऽसौ सेतुबन्धं विधाया-}
{प्यहनदमरशत्रुं रावणं गीतकीर्तिः}% ॥ ६१ ॥

\twolineshloka
{यः शृणोति नरो भक्त्या देव्याश्चरितमुत्तमम्}
{स भुक्त्वा विपुलान्भोगान्प्राप्नोति परमं पदम्}% ॥ ६२ ॥

\twolineshloka
{सन्त्यन्यानि पुराणानि विस्तराणि बहूनि च}
{श्रीमद्‌भागवतस्यास्य न तुल्यानीति मे मतिः}% ॥ ६३ ॥

॥इति श्रीदेवीभागवते महापुराणेऽष्टादशसाहस्र्यां संहितायां तृतीयस्कन्धे रामाय देवीवरदानं नाम त्रिंशोऽध्यायः॥

\closesection
    \chapt{अग्नि-पुराणम्}

\src{अग्निपुराणम्}{अध्यायः ५}{}{श्लोकाः १--१४}
\vakta{}
\shrota{}
\notes{}
\textlink{https://sa.wikisource.org/wiki/अग्निपुराणम्/अध्यायः_५}
\translink{}

\storymeta

\sect{पञ्चमोऽध्यायः --- बाल-काण्ड-वर्णनम्}

\uvacha{अग्निरुवाच}
\twolineshloka
{रामायणमहं वक्ष्ये नारदेनोदितं पुरा}
{वाल्मीकये यथा तद्वत् पठितं भुक्तिमुक्तिदम्} %।। १ ।।

\uvacha{नारद उवाच}
\twolineshloka
{विष्णुनाभ्यव्जजो ब्रह्मा मरीचिर्ब्रह्मणः सुतः}
{मरीचेः कश्यपस्तस्मात् सूर्यो वैवस्वतो मनुः} %।। २ ।।

\twolineshloka
{ततस्तस्मात्तथेक्ष्वाकुस्तस्य वंशे ककुत्स्थकः}
{ककुत्स्थस्य रघुस्तस्मादजो दशरथस्ततः} %।। ३ ।।

\twolineshloka
{रावणादेर्वधार्थाय चतुर्द्धाभूत् स्वयं हरिः}
{राज्ञो दशरथाद्रामः कौशल्यायां बभूव ह} %।। ४ ।।

\twolineshloka
{कैकेय्यां भरतः पुत्रः सुमित्रायाञ्च लक्ष्मणः}
{शत्रुघ्नः ऋष्यश्रृङ्गेण तासु सन्दत्तपायसात्} %।। ५ ।।

\twolineshloka
{प्राशिताद्यज्ञसंसिद्धाद्रामाद्याश्च समाः पितुः}
{यज्ञविध्नविनाशाय विश्वामित्रार्थितो नृपः} %।। ६ ।।

\twolineshloka
{रामं सम्प्रेषयामास लक्ष्मणं मुनिना सह}
{रामो गतोऽस्त्रशस्त्राणि शिक्षितस्ताडकान्तकृत्} %।। ७ ।।

\twolineshloka
{मारीचं मानवास्त्रेण मोहितं दूरतोऽनयत्}
{सुबाहुं यज्ञहन्तारं सबलञ्चावधीद् बली} %।। ८ ।।

\twolineshloka
{सिद्धाश्रमनिवासी च विश्वामित्रादिभिः सह}
{गतः क्रतुं मैथिलस्य द्रष्टुञ्चापंसहानुजः} %।। ९ ।।

\twolineshloka
{शतानन्दनिमित्तेन विश्वामित्रप्रभावतः}
{रामाय कथितो राज्ञा समुनिः पूजितः क्रतौ} %।। १० ।।

\twolineshloka
{धनुरापूरयामास लीलया स बभञ्ज तत् }
{वीर्यशुल्कञ्च जनकः सीतां कन्यान्त्वयोनिजाम्} %।। ११ ।।

\twolineshloka
{ददौ रामाय रामोऽपि पित्रादौ हि समागते}
{उपयेमे जानकीन्तामुर्मिलां लक्ष्मणस्तथा} %।। १२ ।।

\twolineshloka
{श्रुतकीर्त्तिं माण्डवीञ्च कुशध्वजसुते तथा}
{जनकस्यानुजस्यैते शत्रुघ्नभरतावुभौ} %।। १३ ।।

\threelineshloka
{कन्ये द्वे उपयेमाते जनकेन सुपूजितः}
{रामोऽगात्सवशिष्ठाद्यैर्जामदग्न्यं विजित्य च}
{अयोध्यां भरतोभ्यागात् सशत्रुघ्नो युधाजितः} %।। १४ ।।

॥इत्यादिमहापुराणे आग्नेये रामायणे बालकाण्डवर्णनं नाम पञ्चमोऽध्यायः॥


\sect{षष्ठोऽध्यायः --- अयोध्या-काण्ड-वर्णनम्}

\uvacha{नारद उवाच}

\twolineshloka
{भरतेऽथ गते रामः पित्रादीनभ्यपूजयत्}
{राजा दशरथो राममुवाच शृणु राघव} % ०१

\twolineshloka
{गुणानुरागाद्राज्ये त्वं प्रजाभिरभिषेचितः}
{मनसाहं प्रभाते ते यौवराज्यं ददामि ह} % ०२

\twolineshloka
{रात्रौ त्वं सीतया सार्धं संयतः सुव्रतो भव}
{राज्ञश्च मन्त्रिणश्चाष्टौ सवसिष्ठास्तथाब्रुवन्} % ०३

\twolineshloka
{सृष्टिर्जयन्तो विजयः सिद्धार्थो राष्ट्रवर्धनः}
{अशोको धर्मपालश्च सुमन्त्रः सवसिष्ठकः} % ०४

\twolineshloka
{पित्रादिवचनं श्रुत्वा तथेत्युक्त्वा स राघवः}
{स्थितो देवार्चनं कृत्वा कौशल्यायै निवेद्य तत्} % ०५

\twolineshloka
{राजोवाच वसिष्ठादीन् रामराज्याभिषेचने}
{सम्भारान् सम्भवन्तु स्म इत्युक्त्वा कैकेयीङ्गतः} % ०६

\twolineshloka
{अयोध्यालङ्कृतिं दृष्ट्वा ज्ञात्वा रामाभिषेचनं}
{भविष्यतीत्याचचक्षे कैकेयीं मन्थरा सखी} % ०७

\twolineshloka
{पादौ गृहीत्वा रामेण कर्षिता सापराधतः}
{तेन वैरेण सा राम वनवासञ्च काङ्क्षति} % ०८

\twolineshloka
{कैकेयि त्वं समुत्तिष्ठ रामराज्याभिषेचनं}
{मरणं तव पुत्रस्य मम ते नात्र संशयः} % ०९

\twolineshloka
{कुब्जयोक्तञ्च तच्छ्रुत्वा एकमाभरणं ददौ}
{उवाच मे यथा रामस्तथा मे भरतः सुतः} % १०

\twolineshloka
{उपायन्तु न पश्यामि भरतो येन राज्यभाक्}
{कैकेयीमब्रवीत्क्रुद्धा हारं त्यक्त्वाथ मन्थरा} % ११

\twolineshloka
{बालिशे रक्ष भरतमात्मानं माञ्च राघवात्}
{भविता राघवो राजा राघवस्य ततः सुतः} % १२

\twolineshloka
{राजवंशस्तु कैकेयि भरतात्परिहास्यते}
{देवासुरे पुरा युद्धे शम्बरेण हताः सुराः} % १३

\twolineshloka
{रात्रौ भर्ता गतस्तत्र रक्षितो विद्यया त्वया}
{वरद्वयं तदा प्रादाद्याचेदानीं नृपं च तत्} % १४

\twolineshloka
{रामस्य च वने वासं नव वर्षाणि पञ्च च}
{यौवराज्यं च भरते तदिदानीं प्रदास्यति} % १५

\twolineshloka
{प्रोत्साहिता कुब्जया सा अनर्थे चार्थदर्शिनी}
{उवाच सदुपायं मे कच्चित्तं कारयिष्यति} % १६

\twolineshloka
{क्रोधागारं प्रविष्टाथ पतिता भुवि मूर्छिता}
{द्विजादीनर्चयित्वाऽथ राजा दशरथस्तदा} % १७

\twolineshloka
{ददर्श केकयीं रुष्टामुवाच कथमीदृशी}
{रोगार्ता किं भयोद्विग्ना किमिच्छसि करोमि तत्} % १८

\twolineshloka
{येन रामेण हि विना न जीवामि मुहूर्तकम्}
{शपामि तेन कुर्यां वै वाञ्छितं तव सुन्दरि} % १९

\twolineshloka
{सत्यं ब्रूहीति सोवाच नृपं मह्यं ददासि चेत्}
{वरद्वयं पूर्वदत्तं सत्यात्त्वं देहि मे नृप} % २०

\twolineshloka
{चतुर्दशसमा रामो वने वसतु संयतः}
{सम्भारैरेभिरद्यैव भरतोऽत्राभिषेच्यताम्} % २१

\twolineshloka
{विषं पीत्वा मरिष्यामि दास्यसि त्वं न चेन्नृप}
{तच्छ्रुत्वा मूर्छितो भूमौ वज्राहत इवापतत्} % २२

\twolineshloka
{मुहूर्ताच्चेतनां प्राप्य कैकेयीमिदमब्रवीत्}
{किं कृतं तव रामेण मया वा पापनिश्चये} % २३

\twolineshloka
{यन्मामेवं ब्रवीषि त्वं सर्वलोकाप्रियङ्करि}
{केवलं त्वत्प्रियं कृत्वा भविष्यामि सुनिन्दितः} % २४

\twolineshloka
{या त्वं भार्या कालरात्री भरतो नेदृशः सुतः}
{प्रशाधि विधवा राज्यं मृते मयि गते सुते} % २५

\twolineshloka
{सत्यपाशनिबद्धस्तु राममाहूय चाब्रवीत्}
{कैकेय्या वञ्चितो राम राज्यं कुरु निगृह्य माम्} % २६

\twolineshloka
{त्वया वने तु वस्तव्यं कैकेयीभरतो नृपः}
{पितरञ्चैव कैकेयीं नमस्कृत्य प्रदक्षिणं} % २७

\twolineshloka
{कृत्वा नत्वा च कौशल्यां समाश्वस्य सलक्ष्मणः}
{सीतया भार्यया सार्धं सरथः ससुमन्त्रकः} % २८

\twolineshloka
{दत्वा दानानि विप्रेभ्यो दीनानाथेभ्य एव सः}
{मातृभिश्चैव विप्राद्यैः शोकार्तैर्निर्गतः पुरात्} % २९

\twolineshloka
{उषित्वा तमसातीरे रात्रौ पौरान् विहाय च}
{प्रभाते तमपश्यन्तोऽयोध्यां ते पुनरागताः} % ३०

\twolineshloka
{रुदन् राजाऽपि कौशल्या गृहमागात्सुदुःखितः}
{पौरा जना स्त्रियः सर्वा रुरुदू राजयोषितः} % ३१

\twolineshloka
{रामो रथस्थश्चीराढ्यः शृङ्गवेरपुरं ययौ}
{गुहेन पूजितस्तत्र इङ्गुदीमूलमाश्रितः} % ३२

\twolineshloka
{लक्ष्मणः स गुहो रात्रौ चक्रतुर्जागरं हि तौ}
{सुमन्त्रं सरथं त्यक्त्वा प्रातर्नावाथ जाह्नवीम्} % ३३

\twolineshloka
{रामलक्ष्मणसीताश्च तीर्णा आपुः प्रयागकम्}
{भरद्वाजं नमस्कृत्य चित्रकूटं गिरिं ययुः} % ३४

\twolineshloka
{वास्तुपूजां ततः कृत्वा स्थिता मन्दाकिनीतटे}
{सीतायै दर्शयामास चित्रकूटं च राघवः} % ३५

\twolineshloka
{नखैर्विदारयन्तं तां काकं तच्चक्षुराक्षिपत्}
{ऐषिकास्त्रेण शरणं प्राप्तो देवान् विहायसः} % ३६

\twolineshloka
{रामे वनं गते राजा षष्ठेऽह्नि निशि चाब्रवीत्}
{कौशल्यां स कथां पौर्वां यदज्ञानाद्धतः पुरा} % ३७

\twolineshloka
{कौमारे सरयूतीरे यज्ञदत्तकुमारकः}
{शब्दभेदाच्च कुम्भेन शब्दं कुर्वंश्च तत्पिता} % ३८

\twolineshloka
{शशाप विलपन्मात्रा शोकं कृत्वा रुदन्मुहुः}
{पुत्रं विना मरिष्यावस्त्वं च शोकान्मरिष्यसि} % ३९

\twolineshloka
{पुत्रं विना स्मरन् शोकात्कौशल्ये मरणं मम}
{कथामुक्त्वाऽथ हा राममुक्त्वा राजा दिवङ्गतः} % ४०

\twolineshloka
{सुप्तं मत्त्वाऽथ कौशल्या सुप्ता शोकार्तमेव सा}
{सुप्रभाते गायनाश्च सूतमागधवन्दिनः} % ४१

\twolineshloka
{प्रबोधका बोधयन्ति न च बुध्यत्यसौ मृतः}
{कौशल्या तं मृतं ज्ञात्वा हा हताऽस्मीति चाब्रवीत्} % ४२

\twolineshloka
{नरा नार्योऽथ रुरुदुरानीतो भरतस्तदा}
{वशिष्ठाद्यैः सशत्रुघ्नः शीघ्रं राजगृहात्पुरीम्} % ४३

\twolineshloka
{दृष्ट्वा सशोकां कैकेयीं निन्दयामास दुःखितः}
{अकीर्तिः पातिता मूर्ध्नि कौशल्यां स प्रशस्य च} % ४४

\twolineshloka
{पितरं तैलद्रोणिस्थं संस्कृत्य सरयूतटे}
{वशिष्ठाद्यैर्जनैरुक्तो राज्यं कुर्विति सोऽब्रवीत्} % ४५

\twolineshloka
{व्रजामि राममानेतुं रामो राजा मतो बली}
{शृङ्गवेरं प्रयागं च भरद्वाजेन भोजितः} % ४६

\twolineshloka
{नमस्कृत्य भरद्वाजं रामं लक्ष्मणमागतः}
{पिता स्वर्गं गतो राम अयोध्यायां नृपो भव} % ४७

\twolineshloka
{अहं वनं प्रयास्यामि त्वदादेशप्रतीक्षकः}
{रामः श्रुत्वा जलं दत्वा गृहीत्वा पादुके व्रज} % ४८

\threelineshloka
{राज्यायाहन्नयास्यामि सत्याच्चीरजटाधरः}
{रामोक्तो भरतश्चायान्नन्दिग्रामे स्थितो बली}
{त्यक्त्वायोध्यां पादुके ते पूज्य राज्यमपालयत्} % ४९

॥इत्यादिमहापुराणे आग्नेये रामायणेऽयोध्याकाण्डवर्णनं नाम षष्ठोऽध्यायः॥


\sect{सप्तमोऽध्यायः --- अरण्य-काण्ड-वर्णनम्}

\uvacha{नारद उवाच}
\twolineshloka
{रामो वशिष्ठं मातॄश्च नत्वाऽत्रिञ्च प्रणम्य सः}
{अनसूयाञ्च तत्पत्नीं शरभङ्गं सुतीक्ष्णकम्}% ।। १ ।।

\twolineshloka
{अगस्त्य भ्रातरं नत्वा अगस्त्यन्तत्प्रसादतः}
{धनुः खङ्गञ्च सम्प्राप्य दण्डकारण्यमागतः}% ।। २ ।।

\twolineshloka
{जनस्थाने पञ्चवट्यां स्थितो गोदावरीं तटे}
{तत्र सूर्पणखायाता भक्षितुं तान् भयङ्करी}% ।। ३ ।।

\twolineshloka
{रामं सुरूपं दृष्ट्वा सा कामिनी वाक्यमब्रवीत्}
{कस्त्वं कस्मात्समायातो भर्त्ता मे भव चार्थितः}% ।। ४ ।।

\twolineshloka
{एतौ च भक्षयिष्यामि इत्युक्त्वा तं समुद्यता }
{तस्या नासाञ्च कर्णौ च रामोक्तो लक्ष्मणोऽच्छिनत्}% ।। ५।।

\twolineshloka
{रक्तं क्षरन्ती प्रययौ खरं भ्रातरमब्रवीत्}
{मरीष्यामि विनासाऽहं खर जीवामि वै तदा}% ।। ६ ।।

\twolineshloka
{रामस्य भार्य्या सीताऽसौ तस्यासील्लक्ष्मणोऽनुजः}
{तेषां यद्रुधिरं सोष्णं पाययिष्यसि मां यदि}% ।। ७ ।।

\twolineshloka
{खरस्तथेति तामुक्त्वा यतुर्दृशसहस्त्रकैः}
{रक्षसां दूषणेनागाद्योद्धु त्रिशिरसा सह}% ।। ८ ।।

\twolineshloka
{रामं रामोऽपि युयुधे शरैर्विव्याध राक्षसान्}
{हस्त्यश्वरथपादातं बलं निन्ये यमक्षयम्}% ।। ९ ।।

\twolineshloka
{त्रिशीर्षाणं खरं रौद्रं युध्यन्तञ्चौव दूषणम्}
{ययौ सूर्पणखा लङ्कां रावणाग्रेपतद् भुवि}% ।। १० ।।

\twolineshloka
{अब्रवीद्रावणं क्रुद्धा न त्वं राजा न रक्षकः}
{खरादिहन्तू रामस्य सीतां भार्यां हरस्व च}% ।। ११ ।।

\twolineshloka
{रामलक्ष्मणरक्तस्य पानाज्जीवामि नान्यथा}
{तथेत्याह च तच्छ्रुत्वा मारीचं प्राह वै व्रज}% ।। १२ ।।

\twolineshloka
{स्वर्णचित्रमृगो भूत्वा रामलक्ष्मणकर्षकः}
{सीताग्रे तां हरिष्यामि अन्यथा मरणं तव}% ।। १३ ।।

\twolineshloka
{मारीचो रावणं प्राह रामो मृत्युर्धनुर्धरः}
{रावणादपि मर्त्तव्यं मर्त्तव्यं राघवादपि}% ।। १४ ।।

\twolineshloka
{अवश्यं यदि मर्त्तव्यं वरं रामो न रावणः}
{इति मत्वा मृगो भूत्वा सीताग्रे व्यचरन्मुहुः}% ।। १५ ।।

\twolineshloka
{सीतया प्रेरितो रामः शरेणाथावधीच्च तम्}
{म्रियमाणो मृगः प्राह हा सीते लक्ष्मणेति च}% ।। १६ ।।

\twolineshloka
{सौमित्रिः सीतयोक्तोऽथ विरुद्धं राममागतः}
{रावणोऽप्यहरत् सीतां हत्वा गृध्रं जटायुषम्}% ।। १७ ।।

\twolineshloka
{जटायुषा स भिन्नाङ्गः अङ्केनादाय जानकीम्}
{गतो लङ्कामशोकाख्ये धारयामास चाब्रवीत्}% ।। १८ ।।

\twolineshloka
{भव भार्य्या ममाग्र्या त्वं राक्षस्यो रक्ष्यतामियम् }
{रामो हत्वा तु मारीचं दृष्ट्वा लक्ष्मणमब्रवीत्}% ।। १९ ।।

\twolineshloka
{मायामृगोऽसौ सौमित्रे यथा त्वमिह चागतः }
{तथा सीता हृता नूनं नापश्यत् स गतोऽथ ताम्}% ।। २० ।।

\twolineshloka
{शुशोच विललापार्त्तो मां त्यक्त्वा क्क गतासि वै}
{लक्ष्मणाश्वासितो रामो मार्गयामास जानकीम्}% ।। २१ ।।

\threelineshloka
{दृष्ट्वा जटायुस्तं प्राह रावणो हृतवांश्च ताम्}
{मृतोऽथ संस्कृतस्तेन कबन्धञ्चावधीत्ततः}
{शापमुक्तोऽब्रवीद्रामं स त्वं सुग्रीवमाव्रज} %।। २२ ।।

॥इत्यादिमहापुराणे अग्नेये रामायणे अरण्यकाण्डवर्णनं नाम सप्तमोऽध्यायः॥


\sect{अष्टमोऽध्यायः --- किष्किन्धा-काण्ड-वर्णनम्}

\uvacha{नारद उवाच}


\twolineshloka
{रामः पम्पासरो गत्वा शोचन् स शर्वरीं ततः}
{हनूमताऽथ सुग्रीवं नीतो मित्रं चकार ह}% ।। १ ।।

\twolineshloka
{सप्त तालन् विनिर्भिद्य शरेणैकेन पश्यतः}
{पादेन दुन्दुभेः कायञ्चिक्षेप दशयोजनम्}% ।। २ ।।

\twolineshloka
{तद्रिपुं बालिनं हत्वा भ्रातरं वैरसारिणम्}
{किष्किन्धां कपिरज्यञ्च रुमान्तारां समर्पयत्}% ।। ३ ।।

\twolineshloka
{ऋष्यमूकेहरीशायकिष्किन्धेशोऽब्रवीत्सच }
{सीतां त्वं प्राश्यसेयद्वत् तथा राम करोमिते}% ।। ४ ।।

\twolineshloka
{तछ्रुत्वा माल्यवत्पृष्ठे चातुर्मास्यं चकारसः}
{किष्किन्धायाञ्च सुग्रीवो यदा नायाति दर्शनम्}% ।। ५ ।।

\twolineshloka
{तदाऽब्रवीत्तं रामोक्तं लक्ष्मणो व्रज राघवम्}
{न स सङ्कुचितः पन्था येन बाली हतो गतः}% ।। ६ ।।

\twolineshloka
{समये तिष्ठ सुग्रीव मा बालिपथमन्वगः}
{सुग्रीव आह संसक्तो गतं कालं न बुद्धवान्}% ।। ७ ।।

\twolineshloka
{इत्युक्त्वा स गतो रामं नत्वोवाच हरीश्वरः}
{आनीता वानराः सर्वे सीतायाश्च गवेषणे}% ।। ८ ।।

\twolineshloka
{त्वन्मतात् प्रेषयिष्यामि विचिन्वन्तु च जानकीम् }
{पूर्वादौ मासमायान्तु मासादूर्ध्वं निहन्मि तान्}% ।। ९ ।।

\twolineshloka
{इत्युक्ता वानराः पूर्वपश्चमोत्तरमार्गगाः}
{जग्मू रामं ससुग्रीवमपश्यन्तस्तु जानकीम्}% ।। १० ।।

\twolineshloka
{रामाङ्गुलीयं सङ्गृह्य हनूमान् वानरैः सह}
{दक्षिणे मागयामास सुप्रभाया गुहान्तिके}% ।। ११ ।।

\twolineshloka
{मासादूर्ध्वञ्च विन्यस्ता अपश्यन्तस्तु जानकीम्}
{ऊचुर्वृथामरिष्यामो जटायुर्द्धन्य एव सः}% ।। १२ ।।

\twolineshloka
{सीतार्थे योऽत्यजत् प्राणान्रावणेन हतो रणे}
{तच्छ्रु त्वा प्राह सम्पातिर्विहाय कपिभक्षणम्}% ।। १३ ।।

\twolineshloka
{भ्राताऽसौ मे जटायुर्वै मयोड्डीनोऽर्कमण्डलम्}
{अर्क तापाद्रक्षितोऽगाद् दग्धपक्षोऽहमभ्रगः}% ।। १४ ।।

\twolineshloka
{रामवार्त्ताश्रवात् पक्षौ जातौ भूयोऽथ जानकीम्}
{पश्याम्यशोकवनिकागतां लङ्कागतां किल}% ।। १५ ।।

\twolineshloka
{शतयोजनविस्तीर्णे लवणाब्धौ त्रिकूटके}
{ज्ञात्वा रामं ससुग्रीवं वानराः कथयन्तु वै}% ।। १६ ।।

॥इत्यादिमहापुराणे आग्नेये रामायणे किष्किन्धाकाण्डर्णनं नाम अष्टमोऽध्यायः॥

\sect{नवमोऽध्यायः --- सुन्दरकाण्ड-वर्णनम्}

\uvacha{नारद उवाच}
\twolineshloka
{सम्पातिवचनं श्रुत्वा हनुमानङ्गदादयः}
{अब्धिं दृष्ट्वाऽब्रुवंस्तेऽब्धिं लङ्घयेत् को नु जीवयेत्}% ।। १ ।।

\twolineshloka
{कपीनां जीवनार्थाय रामकार्य्यप्रसिद्धये}
{शतयोजनविस्तीर्णं पुप्लुवेऽब्धिं स मारुतिः}% ।। २ ।।

\twolineshloka
{दृष्ट्वोत्थितञ्च मैनाकं सिंहिकां विनिपात्य च }
{लङ्कां दृष्ट्वा राक्षसानां गृहाणि वनितागृहे}% ।। ३ ।।

\twolineshloka
{दशग्रीवस्य कुम्भस्य कुम्भकर्णस्य रक्षसः}
{विभीषणस्येन्द्रजितो गृहेऽन्येषां च रक्षसाम्}% ।। ४ ।।

\twolineshloka
{नापश्यत् पानभूम्यादौ सीतां चिन्तापरायणः}
{अशोकवनिकां गत्वा दृष्टवाञ्छिंशपातले}% ।। ५ ।।

\twolineshloka
{राक्षसीरक्षितां सीतां भव भार्येति वादिनम्}
{रावणं शिशपास्थोऽथ नेति सीतान्तु वादिनीम्}% ।। ६ ।।

\twolineshloka
{भव भार्या रावणस्य राक्षसीर्वादिनीः कपिः}
{गते तु रावणे प्राह राजा दशरथोऽभवत्}% ।। ७ ।।

\twolineshloka
{रामोऽस्य लक्ष्ममः पुत्रौ वनवासङ्गतौ वरौ}
{रामपत्नी जानकी त्वं रावणेन हृता बलात्}% ।। ८ ।।

\twolineshloka
{रामः सुग्रीवमित्रस्त्वा मार्गयन् प्रैषयच्च माम् }
{साभिज्ञानं चागुलीयं रामदत्तं गृहाण वै}% ।। ९ ।।

\twolineshloka
{सीताऽङ्गुलीयं जग्रह साऽपश्यन्मारुतिन्तरौ}
{भूयोऽग्रे चोपविष्टं तमुवाच यदि जीवति}% ।। १० ।।

\twolineshloka
{रामः कथं न नयति शङ्कितामब्रवीत् कपिः}
{रामः सीते न जानीते ज्ञात्वा त्वां स नयिष्यति}% ।। ११ ।।

\twolineshloka
{रावणं राक्षसं हत्वा सबलं देविमाशुच}
{साभिज्ञानं देहि मे त्वं मणिं सीताऽददत्कपौ}% ।। १२ ।।

\twolineshloka
{उवाच मां यथा रामो नयेच्छीघ्रं तथा कुरु}
{काकाक्षिपातनकथाम्प्रतियाहि हि शोकह}% ।। १३।।

\twolineshloka
{मणिं कथां गृहीत्वाह हनूमान्नेष्यते पतिः }
{अथवा ते त्वारा काचित् पृष्ठमारुह मे शुभे}% ।। १४ ।।

\twolineshloka
{अद्य त्वां दर्शयिष्यामि ससुग्रीवञ्च राघवम् }
{सीताऽब्रवीद्धनूमन्तं नयतां मां हि राघवः}% ।। १५ ।।

\twolineshloka
{हनूमान् स दशग्रीवदर्शनोपायमाकरोत्}
{वनं बभञ्च तत्पालान् हत्वा दन्तनखादिभिः}% ।। १६ ।।

\twolineshloka
{हत्वा तु किङ्करान् सर्वान् सप्त मन्त्रिसुतानपि}
{पुत्रमक्षं कुमारञ्च शक्रजिच्चबबन्ध तम्}% ।। १७ ।।

\twolineshloka
{नागपाशेन पिङ्गाक्षं दर्शयामास रावणम्}
{उवाच रावणः कस्त्वं मारुतिः प्राह रावणम्}% ।। १८ ।।

\twolineshloka
{रामदूतो राघवाय सीतां देहि मरिष्यसि}
{रामबाणैर्हतः सार्धं लङ्कास्थै राक्षसैर्ध्रुवम्}% ।। १९ ।।

\twolineshloka
{रावणो हन्तुमुद्युक्तो विभीषणनिवारितः}
{दीपयामास लाङ्गूलं दीप्तपुच्छः स मारुतिः}% ।। २० ।।

\twolineshloka
{दग्ध्वा लङ्कां राक्षसाश्च दृष्ट्वा सीतां प्रणम्य ताम्}
{समुद्रपारमागम्य दृष्ट्वा सीतेति चाब्रवीत्}% ।। २१ ।।

\twolineshloka
{अङ्गदादीनङ्गदाद्यैः पीत्वा मधुवने मधु}
{जित्वा दधिमुखादींश्च दृष्ट्वा रामं च तेऽब्रुवन्}% ।। २२ ।।

\twolineshloka
{दृष्टा सीतेति रामोऽपि हृष्टः पप्रच्छ मारुतिम्}
{कथं दृष्टा त्वया सीता किमुवाच च मां प्रति}% ।। २३ ।।

\twolineshloka
{सीताकथामृतेनैव सिञ्च मां कामवह्निगम्}
{हनूमानब्रवीद्रामं लङ्घयित्वाऽब्धिमागतः}% ।। २४ ।।

\twolineshloka
{सीतां दृष्ठ्वा पुरीं दग्ध्वा सीतामणिं गृहाण वै}
{हत्वा त्वं रावणं सीतां प्रास्यसे राम मा शुचः}% ।। २५ ।।

\twolineshloka
{गृहीत्वा तं मणिं रामो रुरोद विरहातुरः }
{मणिं दृष्ट्वा जानकी मे दृष्टा सीता नयस्व माम्}% ।। २६ ।।

\twolineshloka
{तथा विना न जीवामि सुग्रीवाद्यैः प्रबोधितः}
{समुद्रतीरं गतवान् तत्र रामं विभीषणः}% ।। २७ ।।

\twolineshloka
{गतस्तिरस्कृतो भ्रात्रा रावणेन दुरात्मना}
{रामाय देहि सीतां त्वमित्युक्तेनासहायवान्}% ।। २८ ।।

\twolineshloka
{रामो विभीषणं मित्रं लङ्कैवर्येऽभ्यषेचयत्}
{समुद्रं प्रार्थयन्मार्गं यदा नायात्तदा शरैः}% ।। २९ ।।

\twolineshloka
{भेदयामास रामञ्च उवाचाब्धि समागतः}
{नलेन सेतुं बद्‌ध्वाब्धौ लङ्कां व्रज गभीरकः}% ।। ३० ।।

\threelineshloka
{अहं त्वया कृतः पूर्वं रामोऽपि नलसेतुना}
{कृतेन तरुशैलाद्यैर्गतः पारं महोदधेः}
{वानरैः स सुवेलस्थः सह लङ्कां ददर्श वै} %।। ३१ ।।

॥इत्यादिमहापुराणे आग्नये रामायणे सुन्दरकाण्डवर्णनं नाम नवमोऽध्यायः॥

\sect{अष्टमोऽध्यायः --- युद्ध-काण्ड-वर्णनम्}

\uvacha{नारद उवाच}
\twolineshloka
{रामोक्तञ्चाङ्गदौ गत्वा रावणं प्राह जानकी}
{दीयतां राघवायाशु अन्यथा त्वं मरिप्यसि}% ।। १ ।।

\twolineshloka
{रावणो हन्तुमुद्युक्तः सङ्ग्रामोद्धतराक्षसः}
{रामयाह दशग्रीवो युद्धमेकं तु मन्यते}% ।। २ ।।

\twolineshloka
{रामो युद्धाय तच्छ्रुत्वा लङ्कां सकपिराययौ}
{वानरो हनुमान् मैन्दो द्विविदौ जाम्बवान्नलः}% ।। ३ ।।

\twolineshloka
{नीलस्तारोङ्गदो धूभ्रः सुषेणः केशरी गयः}
{पनसो विनतो रम्भः शरभः कथनो बली}% ।। ४ ।।

\twolineshloka
{गवाक्षो दधिवक्त्रश्च गवयो गन्धमादनः}
{एते चान्ये च सुग्रीव एतैर्युक्तो ह्यसङ्ख्यकैः}% ।। ५ ।।

\twolineshloka
{रक्षसां वानराणाञ्च युद्धं सङ्कुलमाबभौ}
{राक्षसा वानराञ्जघ्नुः शरशक्तिगदादिभिः}% ।। ६ ।।

\twolineshloka
{वानरा राक्षसाञ् जघ्नुर्नखदन्तशिलादिभिः}
{हस्त्थश्वरथपादातं राक्षसानां बलं हतम्} %।।७ ।।

\twolineshloka
{हनूमान् गिरिऋङ्गेण धूम्राक्षमवधीद्रिपुम्}
{अकम्पनं प्रहस्तञ्च युध्यन्तं नील आवधीत्}% ।। ८ ।।

\twolineshloka
{इन्द्रजिच्छरबन्धाच्च विमुक्तौ रामलक्ष्मणौ}
{तार्क्ष्यसन्दर्शनाद् बाणैर्जघ्ननू राक्षसं बलम्}% ।। ९ ।।

\twolineshloka
{रामः शरैर्जर्जरितं रावणञ्चाकरोद्रणे}
{रावणः कुम्बकर्णञ्च बौधयामास दुः खितः}% ।। १० ।।

\twolineshloka
{कुम्भकर्णः प्रबुद्धोऽथ पीत्वा घटसहस्त्रकम्}
{मद्यस्य महिषादीनां भक्षयित्वाह रावणम्}% ।। ११ ।।

\twolineshloka
{सीताया हरणं पापं कृतन्त्वं हि गुरुर्यतः}
{अतो गच्छामि युद्धाय रामं हन्मि सवानरम्}% ।। १२ ।।

\twolineshloka
{इत्युक्त्वा वानरान् सर्वान् कुम्भकर्णो ममर्द्द ह}
{गृहीतस्तेन सुग्रीवः कर्णनासं चकर्त्त सः}% ।। १३ ।।

\twolineshloka
{कर्णनासाविहीनोऽसौ भक्षयामास वानरान्}
{अथ कुम्भो निकुम्भश्च मकराक्षश्च राक्षसः}% ।। १४ ।।

\twolineshloka
{ततः पादौ ततश्छित्त्वा शिरो भूमौ व्यपातयत् }
{अथ कुम्भो निकुम्भश्च मकराक्षश्च राक्षसः}% ।। १५ ।।

\twolineshloka
{महोदरो महापार्श्वो मत्त उन्तत्तराक्षसः}
{प्रघसो भासकर्णश्च विरूपाक्षस्छ संयुगे}% ।। १६ ।।

\twolineshloka
{देवान्तको नरान्तश्च त्रिशिराश्चातिकायकः}
{रामेण लक्ष्मणेनैते वानरैः सविभीषणैः}% ।। १७ ।।

\twolineshloka
{युध्यमानास्तथाह्यन्ये राक्षसाभुवि पातिताः}
{इन्द्रजिन्मायया युध्यन् रामादीन् सम्बबन्ध ह}% ।। १८ ।।

\twolineshloka
{वरदत्तैर्नागबाणै रोषध्या तौ विशल्यकौ}
{विशल्ययाव्रणौ कृत्वा मारुत्यानीतपर्वने}% ।। १९ ।।

\twolineshloka
{हनूमान् धारयामास तत्रागं यत्र संश्थितः}
{निकुम्भिलायां होमादि कुर्वन्तं तं हि लक्ष्मणः}% ।। २० ।।

\twolineshloka
{शरैरिन्द्रजितं वीरं युद्धे तं तु व्यशातयत्}
{रावणः शोकसन्तप्तः सीतां हन्तुं समुद्यतः}% ।। २१ ।।

\twolineshloka
{अविन्ध्यवारितो राजरथस्यः सबलौययौ}
{इन्द्रोक्तो मातलीरामं रथस्थं प्रचकार तम्}% ।। २२ ।।

\twolineshloka
{रामरावणयोर्युद्धं रामरावणयोरिव}
{रावणो वानरान् हन्ति मारुत्याद्याश्च रावणम्}% ।। २३ ।।

\twolineshloka
{रामः शस्त्रैस्तमस्त्रैश्च ववर्ष जलदो यथा}
{तस्य ध्वजं स चिच्छेद रथमश्वांश्च सारथिम्}% ।। २४ ।।

\twolineshloka
{धनुर्बाहूञ्छिरांस्येव उत्तिष्ठन्ति शिरांसि हि}
{पैतामहेन हृदयं भित्त्वा रामेण रावणः}% ।। २५ ।।

\twolineshloka
{भूतले पातितः सर्वै राक्षसै रुरुदुः स्त्रियः}
{आश्वास्य तञ्च संस्कृत्य रामज्ञप्तो विभीषणः}% ।। २६ ।।

\twolineshloka
{हनृमतानयद्रामः सीतां शुद्धां गृहीतवान्}
{रामो वह्नौ प्रविष्टान्तां शुद्धामिन्द्रादिभिः स्तुतः}% ।। २७ ।।

\twolineshloka
{ब्रह्मणा दशरथेन त्वं विष्ण् राक्षसमर्द्दनः}
{इन्द्रौर्च्चितोऽमृतवृष्ट्या जीवयामास वानरान्}% ।। २८ ।।

\twolineshloka
{रामेण पूजिता जग्मुर्युद्धं दृष्ट्वा दिवञ्च ते }
{रामो विभीषणायादाल्लङ्कामभ्यर्च्य वानरान्}% ।। २९ ।।

\twolineshloka
{ससीतः पुष्पके स्थित्वा गतमार्गेण वै गतः}
{दर्शयन् वनदुर्गाणि सीतायै हृष्टमानसः}% ।। ३० ।।

\twolineshloka
{भरद्वाजं नमस्कृत्य नन्दिग्रामं समागतः}
{भरतेन नतश्चागादयोध्यान्तत्र संश्थितः}% ।। ३१ ।।

\twolineshloka
{वसिष्ठादीन्नमस्कृत्य कौशल्याञ्चैव केकयीम् }
{सुमित्रां प्राप्तराज्योऽथ द्विजादीन् सोऽभ्यपूजयत्}% ।। ३२ ।।

\twolineshloka
{वासुदेवं स्वमात्मानमश्वमेधैरथायजत्}
{सर्वदानानि स ददौ पालयामास स प्रजाः}% ।। ३३।।

\threelineshloka
{पुत्रवद्धर्म्मकामादीन् दुष्टनिग्रहणे रतः}
{सर्वधर्म्मपरो लोकः सर्वशस्या च मेदिनी}
{नाकालमरणञ्चासीद्रामे राज्यं प्रशासति} %।। ३४ ।।

इत्यादिमहापुराणे आग्नेये रामायणे युद्धकाण्डवर्णनं नाम दशमोऽध्यायः ॥

\sect{एकादशोऽध्यायः --- उत्तर-काण्ड-वर्णनम्}

\uvacha{नारद उवाच}

\twolineshloka
{राज्यस्थं राघवं जग्मुरगस्त्याद्याः सुपूजिताः}
{धन्यस्त्वं विजयी यस्मादिन्द्रजिद्विनिपातितः}% ।। १ ।।

\twolineshloka
{ब्रह्मात्मजः पुलस्त्योभूद् विश्रवास्तस्यनैकषी}
{पुष्पोत्कटाभूत् प्रथमा तत्पुत्रोभूद्धनेश्वरः}% ।। २ ।।

\twolineshloka
{नैकष्यां रावणो जज्ञे विंशद्बाहुर्द्दशाननः}
{तपसा ब्रह्मदत्तेन वरेण जितदैवतः}% ।। ३ ।।

\twolineshloka
{कुम्भकर्णः सनिद्रोऽभूद्धर्म्मिष्ठोऽभूद्धिभीषणः}
{स्वसा शूर्पणखा तेषां रावणान्मेघनादकः}% ।। ४ ।।

\twolineshloka
{इन्द्रं जित्वेन्द्रजिच्चाभूद्रावणादधिको बली}
{हतस्त्वया लक्ष्मणेन देवादेः क्षेममिच्छता}% ।। ५ ।।

\twolineshloka
{इत्युक्त्वा ते गता विप्रा अगस्त्याद्या नमस्कृताः}
{देवप्रार्थितरामोक्तः शत्रुघ्नो लवणार्द्दनः}% ।। ६ ।।

\twolineshloka
{अभूत् पूर्म्मथुरा काचिद् रामोक्तो भरतोऽवधीत्}
{कोटित्रयञ्च शैलूषपुत्राणां निशितैः शरैः}% ।। ७ ।।

\twolineshloka
{शैलूषं दुप्टगन्धर्वं सिन्धुतीरनिवासिनम्}
{तक्षञ्च पुष्करं पुत्रं स्थापयित्वाथ देशयोः}% ।। ८ ।।

\twolineshloka
{भरतोगात्सशत्रुघ्नो राघवं पूजयन् स्थितः}
{रामो दुष्टान्निहत्याजौ शिष्टान् सम्पाल्य मानवः}% ।। ९ ।।

\twolineshloka
{पुत्रौ कुशलवौ जातौ वाल्मीकेराश्रमे वरौ}
{लोकापवादात्त्यक्तायां ज्ञातौ सुचरितश्रवात्}% ।। १० ।।

\twolineshloka
{राज्येभिषिच्य ब्रह्माहमस्मीति ध्यानतत्परः}
{दशवर्षसहस्त्राणि दशवर्षसतानि च}% ।। ११ ।।

\twolineshloka
{राज्यं कृत्वा क्रतून् कृत्वा स्वर्गं देवार्च्चितो ययौ}
{सपौरः सानुजः सीतापुत्रो जनपदान्वितः}% ।। १२ ।।
                     
\uvacha{अग्निरुवाच}
\twolineshloka
{वाल्मीकिर्नारदाच्छ्रु त्वा रामायणमकारयत्}
{सविस्तरं यदेतच्च श्रृणुयात्स दिवं व्रजेत्}% ।। १३ ।।

॥इत्यादिमहापुराणे आग्नेये रामायणे उत्तरकाण्डवर्णनं नाम एकादशोऽध्यायः॥

\closesection
    \chapt{इक्ष्वाकु-वंश-वर्णनम्}

\src{अग्निपुराणम्}{}{अध्यायः २१}{श्लोकाः १६--}
\vakta{}
\shrota{}
\notes{Brief story of Rama, in the context of Ikshvaku dynasty. Notable is the mention of Rameshvaram temple, and the Shiva linga installed by Rama.}
\textlink{https://sa.wikisource.org/wiki/कूर्मपुराणम्-पूर्वभागः/एकविंशतितमोऽध्यायः}
\translink{}

\storymeta

\uvacha{सूत उवाच}
\twolineshloka
{त्रिधन्वा राजपुत्रस्तु धर्मेणापालयन्महीम्}
{तस्य पुत्रोऽभवद् विद्वांस्त्रय्यारुण इति स्मृतः} %२१.१

\twolineshloka
{तस्य सत्यव्रतो नाम कुमारोऽभून्महाबलः}
{भार्या सत्यधना नाम हरिश्चन्द्रमजीजनत्} %२१.२

\twolineshloka
{हरिश्चन्द्रस्य पुत्रोऽभूद् रोहितो नाम वीर्यवान्}
{रोहितस्य वृकः पुत्रः तस्मात्बाहुरजायत} %२१.३

\threelineshloka*
{हरितो रोहितस्याथ धुन्धुस्तस्य सुतोऽभवत्}
{विजयश्च सुदेवश्च धुन्धुपुत्रौ बभूवतुः}
{विजयस्याभवत् पुत्रः कारुको नाम वीर्यवान्}

\twolineshloka
{सगरस्तस्य पुत्रौऽभूद् राजा परमधार्मिकः}
{द्वे भार्ये सगरस्यापि प्रभा भानुमती तथा} %२१.४

\twolineshloka
{ताभ्यामाराधितः वह्निः प्रादादौ वरमुत्तमम्}
{एकं भानुमती पुत्रमगृह्णादसमञ्जसम्} %२१.५

\twolineshloka
{प्रभा षष्टिसहस्त्रं तु पुत्राणां जगृहे शुभा}
{असमञ्सस्य तनयो ह्यंशुमान् नाम पार्थिवः} %२१.६

\twolineshloka
{तस्य पुत्रो दिलीपस्तु दिलीपात् तु भगीरथः}
{येन भागीरथी गङ्गा तपः कृत्वाऽवतारिता} %२१.७

\twolineshloka
{प्रसादाद् देवदेवस्य महादेवस्य धीमतः}
{भगीरथस्य तपसा देवः प्रीतमना हरः} %२१.८

\twolineshloka
{बभार शिरसा गङ्गां सोमान्ते सोमभूषणः}
{भगीरथसुतश्चापि श्रुतो नाम बभूव ह} %२१.९

\twolineshloka
{नाभागस्तस्य दायादः सिन्धुद्वीपस्ततोऽभवत्}
{अयुतायुः सुतस्तस्य ऋतुपर्णस्तु तत्सुतः} %२१.१०

\twolineshloka
{ऋतुपर्णस्य पुत्रोऽभूत् सुदासो नाम धार्मिकाः}
{सौदासस्तस्य तनयः ख्यातः कल्माषपादकः} %२१.११

\twolineshloka
{वसिष्ठस्तु महातेजाः क्षेत्रे कल्माषपादके}
{अश्मकं जनयामसा तमिक्ष्वाकुकुलध्वजम्} %२१.१२

\twolineshloka
{अश्मकस्योत्कलायां तु नकुलो नाम पार्थिवः}
{स हि रामभयाद् राजा वनं प्राप सुदुः खितः} %२१.१३

\twolineshloka
{विभ्रत् स नारीकवचं तस्माच्छतरथोऽभवत्}
{तस्माद् बिलिबिलिः श्रीमान्‌वृद्धशर्माचतत्सुतः} %२१.१४

\twolineshloka
{तस्माद् विश्वसहस्तस्मात् खट्वाङ्ग इति विश्रुतः}
{दीर्घबाहुः सुतस्तस्य रघुस्तस्मादजायत} %२१.१५

\twolineshloka
{रघोरजः समुत्पन्नो राजा दशरथस्ततः}
{रामो दाशरथिर्वोरो धर्मज्ञो लोकविश्रुतः} %२१.१६

\twolineshloka
{भरतो लक्ष्मणश्चैव शत्रुघ्नश्च महाबलः}
{सर्वे शक्रसमा युद्धे विष्णुशक्तिसमन्विताः} %२१.१७

\twolineshloka
{जज्ञे रावणनाशार्थं विष्णुरंशेन विश्वकृत्}
{रामस्य सुभगा भार्या जनकस्यात्मजा शुभा} %२१.१८

\twolineshloka
{सीता त्रिलोकविख्याता शीलौदार्यगुणान्विता}
{तपसा तोषिता देवी जनकेन गिरीन्द्रजा} %२१.१९

\twolineshloka
{प्रायच्छज्जानकीं सीतां राममेवाश्रितां पतिम्}
{प्रीतश्च भगवानीशस्त्रिशूली नीललोहितः} %२१.२०

\twolineshloka
{प्रददौ शत्रुनाशार्थं जनकायाद्‌भुतं धनुः}
{स राजा जनको विद्वान् दातुकामः सुतामिमाम्} %२१.२१

\twolineshloka
{अघोषयदमित्रघ्नो लोकेऽस्मिन् द्विजपुङ्गवाः}
{इदं धनुः समादातुं यः शक्नोति जगत्त्रये} %२१.२२

\twolineshloka
{देवो वा दानवो वाऽपि स सीतां लब्धुमर्हति}
{विज्ञाय रामो बलवान् जनकस्य गृहं प्रभुः} %२१.२३

\twolineshloka
{भञ्जयामास चादाय गत्वाऽसौ लीलयैव हि}
{उद्ववाह च तां कन्यां पार्वतीमिव शङ्करः} %२१.२४

\twolineshloka
{रामः परमधर्मात्मा सेनामिव च षण्मुखः}
{ततो बहुतिथे काले राजा दशरथः स्वयम्} %२१.२५

\twolineshloka
{रामं ज्येष्ठं सुतं वीरं राजानं कर्तुमारभत्}
{तस्याथ पत्नी सुभगा कैकेयी चारुभाषिणी} %२१.२६

\twolineshloka
{निवारयामास पतिं प्राह सम्भ्रान्तमानसा}
{मत्सुतं भरतं वीरं राजानं कर्त्तुमर्हसि} %२१.२७

\twolineshloka
{पूर्वमेव वरो यस्माद् दत्तो मे भवता यतः}
{स तस्या वचनं श्रुत्वा राजा दुःखितमानसः} %२१.२८

\twolineshloka
{बाढमित्यब्रवीद् वाक्यं तथा रामोऽपि धर्मवित्}
{प्रणम्याथ पितुः पादौ लक्ष्मणेन सहाच्युतः} %२१.२९

\twolineshloka
{ययौ वनं सपत्नीकः कृत्वा समयमात्मवान्}
{संवत्सराणां चत्वारि दश चैव महाबलः} %२१.३०

\twolineshloka
{उवास तत्र मतिमान् लक्ष्मणेन सह प्रभुः}
{कदाचिद् वसतोऽरण्ये रावणो नाम राक्षसः} %२१.३१

\twolineshloka
{परिव्राजकवेषेण सीतां हृत्वा ययौ पुरीम् ।।}
{अदृष्ट्वा लक्ष्मणो रामः सीतामाकुलितेन्द्रियौ} %२१.३२

\twolineshloka
{दुः खशोकाभिसन्तप्तौ बभूवतुररिन्दमौ}
{ततः कदाचित् कपिना सुग्रीवेण द्विजोत्तमाः} %२१.३३

\twolineshloka
{वानराणामभूत् सख्यं रामस्याक्लिष्टकर्मणः ।।}
{सुग्रीवस्यानुगो वीरो हनुमान्नाम वानरः} %२१.३४

\twolineshloka
{वायुपुत्रौ महातेजा रामस्यासीत् प्रियः सदा}
{स कृत्वा परमं धैर्यं रामाय कृतनिश्चयः} %२१.३५

\twolineshloka
{आनयिष्यामि तां सीतामित्युक्त्वा विचचार ह}
{महीं सागरपर्यन्तां सीतादर्शनतत्परः} %२१.३६

\twolineshloka
{जगाम रावणपुरीं लङ्कां सागरसंस्थिताम्}
{तत्राथ निर्जने देशे वृक्ष्मूले शुचिस्मिताम्} %२१.३७

\twolineshloka
{अपश्यदमलां सीतां राक्षसीभिः समावृताम्}
{अश्रुपूर्णेक्षणां हृद्यां संस्मरन्तीमनिन्दिताम्} %२१.३८

\twolineshloka
{राममिन्दीवरश्यामं लक्ष्मणं चात्मसंस्थितम्}
{निवेदयित्वा चात्मानं सीतायै रहसि स्वयम्} %२१.३९

\twolineshloka
{असंशयाय प्रददावस्यै रामाङ्‌गुलीयकम्}
{दृष्ट्वाऽङ्‌गुलीयकं सीता पत्युः परमशोभनम्} %२१.४०

\twolineshloka
{मेने समागतं रामं प्रीतिविस्फारितेक्षणा}
{समाश्वास्य तदा सीतां दृष्ट्वा रामस्य चान्तिकम्} %२१.४१

\twolineshloka
{नयिष्ये त्वां महाबाहुरुक्त्वा रामं ययौ पुनः}
{निवेदयित्वा रामाय सीतादर्शनमात्मवान्} %२१.४२

\twolineshloka
{तस्थौ रामेण पुरतो लक्ष्मणेन च पूजितः}
{ततः स रामो बलवान् सार्द्धं हनुमता स्वयम्} %२१.४३

\twolineshloka
{लक्ष्मणेन च युद्धाय बुद्धिं चक्रे हि रक्षसाम्}
{कृत्वाऽथ वानरशतैर्लङ्कामार्गं महोदधेः} %२१.४४

\twolineshloka
{सेतुं परमधर्मात्मा रावणं हतवान् प्रभुः}
{सपत्नीकं च ससुतं सभ्रातृकमरिन्दमः} %२१.४५

\twolineshloka
{आनयामास तां सीतां वायुपुत्रसहायवान्}
{सेतुमध्ये महादेवमीशानं कृत्तिवाससम्} %२१.४६

\twolineshloka
{स्थापयामास लिङ्गस्थं पूजयामास राघवः}
{तस्य देवो महादेवः पार्वत्या सह शङ्करः} %२१.४७

\twolineshloka
{प्रत्यक्षमेव भगवान् दत्तवान् वरमुत्तमम्}
{यत् त्वया स्थापितं लिङ्गं द्रक्ष्यन्तीह द्विजातयः} %२१४८

\twolineshloka
{महापातकसंयुक्तास्तेषां पापं विनङ्क्ष्यति}
{अन्यानि चैव पापानि स्नातस्यात्र महोदधौ} %२१.४९

\twolineshloka
{दर्शनादेव लिङ्गस्य नाशं यान्ति न संशयः}
{यावत् स्थास्यन्ति गिरयो यावदेषा च मेदिनी} %२१.५०

\twolineshloka
{यावत् सेतुश्च तावच्च स्थास्याम्यत्र तिरोहितः}
{स्नानं दानं जपः श्राद्धं भविष्यत्यक्षयं महत्} %२१.५१

\twolineshloka
{स्मरणादेव लिङ्गस्य दिनपापं प्रणश्यति}
{इत्युक्त्वा भगवाञ्छम्भुः परिष्वज्य तु राघवम्} %२१.५२

\twolineshloka
{सनन्दी सगणो रुद्रस्तत्रैवान्तरधीयत}
{रामोऽपि पालयामास राज्यं धर्मपरायणः} %२१.५३

\twolineshloka
{अभिषिक्तो महातेजा भरतेन महाबलः}
{विशेषाढ् ब्राह्मणान् सर्वान् पूजयामसचेश्वरम्} %२१.५४

\twolineshloka
{यज्ञेन यज्ञहन्तारमश्वमेधेन शङ्करम्}
{रामस्य तनयो जज्ञे कुश इत्यभिविश्रुतः} %२१.५५

\twolineshloka
{लवश्च सुमहाभागः सर्वतत्त्वार्थवित् सुधीः}
{अतिथिस्तु कुशाज्जज्ञे निषधस्तत्सुतोऽभवत्} %२१.५६

\twolineshloka
{नलस्तु निषधस्याभून्नभास्तमादजायत}
{नभसः पुण्डरीकाक्षः क्षेमधन्वा च तत्सुतः} %२१.५७

\twolineshloka
{तस्य पुत्रोऽभवद् वीरो देवानीकः प्रतापवान्}
{अहीनगुस्तस्य सुतो सहस्वांस्तत्सुतोऽभवत्} %२१.५८

\twolineshloka
{तस्माच्चन्द्रावलोकस्तु तारापीडस्तु तत्सुतः}
{तारापीडाच्चन्द्रगिरिर्भानुवित्तस्ततोऽभवत्} %२१.५९

\twolineshloka
{श्रुतायुरभवत् तस्मादेते इक्ष्वाकुवंशजाः}
{सर्वे प्राधान्यतः प्रोक्ताः समासेन द्विजोत्तमाः} %२१.६०

\twolineshloka
{य इमं श्रृणुयान्नित्यमिक्ष्वाकोर्वंशमुत्तमम्}
{सर्वपापविनिर्मुक्तो स्वर्गलोके महीयते} %२१.६१

॥इति श्रीकूर्मपुराणे षट्‌साहस्त्र्यां संहितायां पूर्वविभागे इक्ष्वाकुवंशवर्णनं नाम एकविंशोऽध्यायः॥
    \sect{द्विचत्वारिंशदधिक-द्विशततमोऽध्यायः --- रामस्यायोध्याप्रवेशः}

\src{पद्म-पुराणम्}{सृष्टिखण्डम्}{अध्यायः २४२--२४४}{}
% \tags{concise, complete}
\notes{}
\textlink{https://sa.wikisource.org/wiki/पद्मपुराणम्/खण्डः_५_(पातालखण्डः)/अध्यायः_००१}
\translink{https://www.wisdomlib.org/hinduism/book/the-padma-purana/d/doc365826.html}

\storymeta


\uvacha{रुद्र उवाच}

\twolineshloka
{स्वायम्भुवो मनुः पूर्वं द्वाशार्णं महामनुम्}
{जजाप गोमतीतीरे नैमिषे विमले शुभे}% १

\twolineshloka
{तेन वर्षसहस्रेण पूजितः कमलापतिः}
{मत्तो वरं वृणीष्वेति तं प्राह भगवान्हरिः}% २

\onelineshloka*
{ततः प्रोवाच हर्षेण मनुः स्वायम्भुवो हरिम्}

\uvacha{मनुरुवाच}
\onelineshloka
{पुत्रत्वं भज देवेश त्रीणि जन्मानि चाच्युत}% ३

\onelineshloka*
{त्वां पुत्रलालसत्वेन भजामि पुरुषोत्तमम्}

\uvacha{रुद्र उवाच}
\onelineshloka
{इत्युक्तस्तेन लक्ष्मीशः प्रोवाच सुमहागिरा}% ४

\uvacha{विष्णुरुवाच}

\twolineshloka
{भविष्यति नृपश्रेष्ठ यत्ते मनसि काङ्क्षितम्}
{ममैव च महत्प्रीतिस्तव पुत्रत्वहेतवे}% ५

\twolineshloka
{स्थितिप्रयोजने काले तत्र तत्र नृपोत्तम}
{त्वयि जाते त्वहमपि जातोस्मि तव सुव्रत}% ६

\twolineshloka
{परित्राणाय साधूनां विनाशाय च दुष्कृताम्}
{धर्म्मसंस्थापनार्थाय सम्भवामि तवानघ}% ७

\uvacha{रुद्र उवाच}

\twolineshloka
{एवं दत्वा वरं तस्मै तत्रैवान्तर्दधे हरिः}
{अस्याभूत्प्रथमं जन्म मनोः स्वायम्भुवस्य च}% ८

\twolineshloka
{रघूणामन्वये पूर्वं राजा दशरथो ह्यभूत्}
{द्वितीयो वसुदेवोऽभूद्वृष्णीनामन्वये विभुः}% ९

\twolineshloka
{कलेर्दिव्यसहस्राब्दप्रमाणस्यान्त्यपादयोः}
{शम्भलग्रामकं प्राप्य ब्राह्मणः सञ्जनिष्यति}% १०

\twolineshloka
{कौशल्या समभूत्पत्नी राज्ञो दशरथस्य हि}
{यदोर्वंशस्य सेवार्थं देवकी नाम विश्रुता}% ११

\twolineshloka
{हरिव्रतस्य विप्रस्य भार्य्या देवप्रभा पुनः}
{एवं मातृत्वमापन्ना त्रीणि जन्मानि शार्ङ्गिणः}% १२

\twolineshloka
{पूर्वं रामस्य चरितं वक्ष्यामि तव सुव्रते}
{यस्य स्मरणमात्रेण विमुक्तिः पापिनामपि}% १३

\twolineshloka
{हिरण्यकहिरण्याक्षौ द्वितीयं जन्मसंश्रितौ}
{कुम्भकर्ण दशग्रीवावजायेतां महाबलौ}% १४

\twolineshloka
{पुलस्त्यस्य सुतो विप्रो विश्रवा नाम धार्मिकः}
{तस्य पत्नी विशालाक्षी राक्षसेन्द्र सुताऽनघे}% १५

\twolineshloka
{सुकेशितनया सा स्यात्सुमाली दानवस्य च}
{केकसी नाम कन्यासीत्तस्य भार्या दृढव्रता}% १६

\twolineshloka
{कामोद्रिक्ता तु सा देवी सन्ध्याकाले महामुनिम्}
{रमयामास तन्वङ्गी यथेष्टं शुभदर्शना}% १७

\twolineshloka
{कामात्सन्ध्याभवाद्यत्वात्तस्यां जातौ महाबलौ}
{रावणः कुम्भकर्णश्च राक्षसौ लोकविश्रुतौ}% १८

\twolineshloka
{कन्या शूर्पणखा नाम जातातिविकृतानना}
{कस्यचित्त्वथ कालस्य तस्यां जातो विभीषणः}% १९

\twolineshloka
{सुशीलो भगवद्भक्तः सत्यवाग्धर्म्मवाञ्शुचिः}
{रावणः कुम्भकर्णश्च हिमवत्पर्वतोत्तमे}% २०

\twolineshloka
{महोग्रतपसा मां वै पूजयामासतुर्भृशम्}
{रावणस्त्वथ दुष्टात्मा स्वशिरःकमलैः शुभैः}% २१

\twolineshloka
{पूजयामास मां देवि दारुणेनैव कर्म्मणा}
{ततस्तमब्रुवं सुभ्रूः प्रहृष्टेनान्तरात्मना}% २२

\twolineshloka
{वरं वृणीष्व मे वत्स यत्ते मनसि वर्त्तते}
{ततः प्रोवाच दुष्टात्मा देवदानव रक्षसाम्}% २३

\twolineshloka
{अवध्यत्वं प्रदेहीति सर्वलोकजिगीषया}
{ततोऽहं दत्तवांस्तस्मै राक्षसाय दुरात्मने}% २४

\twolineshloka
{देवदानवयक्षाणामवध्यत्वं वरानने}
{राक्षसोऽसौ महावीर्यो वरदानात्तु गर्वितः}% २५

\twolineshloka
{त्रींल्लोकान्पीडयामास देवदानवमानुषान्}
{तेन सम्बाध्यमानाश्च देवा ब्रह्मपुरोगमाः}% २६

\twolineshloka
{भयार्त्ताः शरणं जग्मुरीश्वरं कमलापतिम्}
{ज्ञात्वाथ वेदनां तेषामभयाय सनातनः}% २७

\onelineshloka*
{उवाच त्रिदशान्सर्वान्ब्रह्मरुद्रपुरोगमान्}

\uvacha{श्रीभगवानुवाच}
\onelineshloka
{राज्ञो दशरथस्याहमुत्पत्स्यामि रघोः कुले}% २८

\twolineshloka
{वधिष्यामि दुरात्मानं रावणं सह बान्धवम्}
{मानुषं वपुरास्थाय हन्मि दैवतकण्टकम्}% २९

\twolineshloka
{नन्दिशापाद्भवन्तोऽपि वानरत्वमुपागताः}
{कुरुध्वं मम साहाय्यं गन्धर्वाप्सरसोत्तमाः}% ३०

\uvacha{रुद्र उवाच}

\twolineshloka
{इत्युक्ता देवतास्सर्वा देवदेवेन विष्णुना}
{वानरत्वमुपागम्य जज्ञिरे पृथिवीतले}% ३१

\twolineshloka
{भार्गवेण प्रदत्ता तु महीसागरमेखला}
{दत्ता महर्षिभिः पूर्वं रघूणां सुमहात्मनाम्}% ३२

\twolineshloka
{वैवस्वतमनोः पुत्रो राज्ञां श्रेष्ठो महाबलः}
{इक्ष्वाकुरिति विख्यातस्सर्वधर्म्मविदांवरः}% ३३

\twolineshloka
{तदन्वये महातेजा राजा दशरथो बली}
{अजस्य नृपतेः पुत्रः सत्यवान्शीलवान्शुचिः}% ३४

\twolineshloka
{स राजा पृथिवीं सर्वां पालयामास वीर्य्यतः}
{राज्येषु स्थापयामास सर्वान्पार्थिवसत्तमान्}% ३५

\twolineshloka
{कोशलस्य नृपस्याथ पुत्री सर्वाङ्गशोभना}
{कौशल्या नाम तां कन्यामुपयेमे स पार्थिवः}% ३६

\twolineshloka
{मागधस्य नृपस्याथ तनया च शुचिस्मिता}
{सुमित्रा नाम नाम्ना च द्वितीया तस्य भामिनी}% ३७

\twolineshloka
{तृतीया केकयस्याथ नृपतेर्दुहिता तथा}
{भार्य्याभूत्पद्मपत्राक्षी केकयी नाम नामतः}% ३८

\twolineshloka
{ताभिः स्म राजा भार्याभिस्तिसृभिर्धर्मसंयुतः}
{रमयामास काकुत्स्थः पृथिवीं चानुपालयन्}% ३९

\twolineshloka
{अयोध्या नाम नगरी सरयूतीर संस्थिता}
{सर्वरत्नसुसम्पूर्णा धनधान्यसमाकुला}% ४०

\twolineshloka
{प्राकारगोपुरैर्जुष्टा हेमप्राकारसङ्कुला}
{उत्तमैर्नागतुरगैर्महेन्द्रस्य यथा पुरी}% ४१

\twolineshloka
{तस्यां राजा स धर्मात्मा उवास मुनिसत्तमैः}
{पुरोहितेन विप्रेण वसिष्ठेन महात्मना}% ४२

\twolineshloka
{राज्यं चकारयामास सर्वं निहतकण्टकम्}
{यस्मादुत्पत्स्यते तस्यां भगवान्पुरुषोत्तमः}% ४३

\twolineshloka
{तस्मात्तु नगरी पुण्या साप्ययोध्येति कीर्तिता}
{नगरस्य परं धाम्नो नाम तस्याप्यभूच्छुभे}% ४४

\twolineshloka
{यत्रास्ते भगवान्विष्णुस्तदेव परमं पदम्}
{तत्र सद्यो भवेन्मोक्षः सर्वकर्म्मनिकृन्तनः}% ४५

\twolineshloka
{जाते तत्र महाविष्णौ नराः सर्वे मुदं ययुः}
{स राजा पृथिवीं सर्वां पालयित्वा शुभानने}% ४६

\twolineshloka
{अयजद्वैष्णवेष्ट्या च पुत्रार्थी हरिमच्युतम्}
{तेन सम्पूजितः श्रीशो राजा सर्वगतो हरिः}% ४७

\twolineshloka
{वैष्णवेन तु यज्ञेन वरदः प्राह केशवः}
{तस्मिन्नाविरभूदग्नौ यज्ञरूपो हरिस्तदा}% ४८

\twolineshloka
{शुद्धजाम्बूनदप्रख्यः शङ्खचक्रगदाधरः}
{शुक्लाम्बरधरः श्रीमान्सर्वभूषणभूषितः}% ४९

\twolineshloka
{श्रीवत्सकौस्तुभोरस्को वनमालाविभूषितः}
{पद्मपत्रविशालाक्षश्चतुर्बाहुरुदारधीः}% ५०

\twolineshloka
{सव्याङ्कस्थ श्रिया सार्द्धमाविरासीद्रमेश्वरः}
{वरदोस्मीति तं प्राह राजानं भक्तवत्सलः}% ५१

\twolineshloka
{तं दृष्ट्वा सर्वलोकेशं राजा हर्षसमाकुलः}
{ववन्दे भार्य्यया सार्द्धं प्रहृष्टेनान्तरात्मना}% ५२

\twolineshloka
{प्राञ्जलिः प्रणतो भूत्वा हर्षगद्गदया गिरा}
{पुत्रत्वं मे भजेत्याह देवदेवं जनार्दनम्}% ५३

\onelineshloka*
{ततः प्रसन्नो भगवान्प्राह राजानमच्युतः}

\uvacha{विष्णुरुवाच}
\onelineshloka
{उत्पत्स्येऽहं नृपश्रेष्ठ देवलोकहिताय वै}% ५४

\twolineshloka
{परित्राणाय साधूनां राक्षसानां वधाय च}
{मुक्तिं प्रदातुं लोकानां धर्म्मसंस्थापनाय च}% ५५

\uvacha{महादेव उवाच}

\twolineshloka
{इत्युक्त्वा पायसं दिव्यं हेमपात्रस्थितं शृतम्}
{लक्ष्म्याहस्तस्थितं शुभ्रं पार्थिवाय ददौ हरिः}% ५६

\uvacha{विष्णुरुवाच}

\twolineshloka
{इदं वै पायसं राजन्पत्नीभ्यस्तव सुव्रत}
{देहि ते तनयास्तासु उत्पत्स्यन्ते मदङ्गजाः}% ५७

\uvacha{महादेव उवाच}

\twolineshloka
{इत्युक्त्वा मुनिभिः सर्वैः स्तूयमानो जनार्दनः}
{स्वात्मानं दर्शयित्वाथ तथैवान्तरधीयत}% ५८

\twolineshloka
{स राजा तत्र दृष्ट्वा च पत्नीं ज्येष्ठां कनीयसीम्}
{विभज्य पायसं दिव्यं प्रददौ सुसमाहितः}% ५९

\twolineshloka
{एतस्मिन्नन्तरे पत्नी सुमित्रा तस्य मध्यमा}
{तत्समीपं प्रयाता सा पुत्रकामा सुलोचना}% ६०

\twolineshloka
{तां दृष्ट्वा तत्र कौशल्या कैकेयी च सुमध्यमा}
{अर्द्धमर्द्धं प्रददतुस्ते तस्यै पायसं स्वकम्}% ६१

\twolineshloka
{तत्प्राश्य पायसं दिव्यं राजपत्न्यः सुमध्यमाः}
{सम्पन्नगर्भाः सर्वास्ता विरेजुः शुभ्रवर्च्चसः}% ६२

\twolineshloka
{तासां स्वप्नेषु देवेशः पीतवासा जनार्दनः}
{शङ्खचक्रगदापाणिराविर्भूतस्तदा हरिः}% ६३

\twolineshloka
{अस्मिन्काले मनोरम्ये मधुमासि शुचिस्मिते}
{शुक्ले नवम्यां विमले नक्षत्रेऽदितिदैवते}% ६४

\twolineshloka
{मध्याह्नसमये लग्ने सर्वग्रहशुभान्विते}
{कौसल्या जनयामास पुत्रं लोकेश्वरं हरिम्}% ६५

\twolineshloka
{इन्दीवरदलश्यामं कोटिकन्दर्प्पसन्निभम्}
{पद्मपत्रविशालाक्षं सर्वाभरणशोभितम्}% ६६

\twolineshloka
{श्रीवत्सकौस्तुभोरस्कं सर्वाभरणभूषितम्}
{उद्यद्दिनकरप्रख्यकुण्डलाभ्यां विराजितम्}% ६७

\twolineshloka
{अनेकसूर्य्यसङ्काशं तेजसा महता वृतम्}
{परेशस्य तनो रम्यं दीपादुत्पन्नदीपवत्}% ६८

\twolineshloka
{ईशानं सर्वलोकानां योगिध्येयं सनातनम्}
{सर्वोपनिषदामर्थमनन्तं परमेश्वरम्}% ६९

\twolineshloka
{जगत्सर्गस्थितिलये हेतुभूतमनामयम्}
{शरण्यं सर्वभूतानां सर्वभूतमयं विभुम्}% ७०

\twolineshloka
{समुत्पन्ने जगन्नाथे देवदुन्दुभयो दिवि}
{विनेदुः पुष्पवर्षाणि ववर्षुः सुरसत्तमाः}% ७१

\twolineshloka
{प्रजापतिमुखा देवा विमानस्था नभस्तले}
{तुष्टुवुर्मुनिभिः सार्द्धं हर्षपूर्णाङ्गविह्वलाः}% ७२

\twolineshloka
{जगुर्गन्धर्वपतयो ननृतुश्चाप्सरोगणाः}
{ववुः पुण्यशिवा वाताः सुप्रभोभूद्दिवाकरः}% ७३

\twolineshloka
{जज्वलुश्चाग्नयः शान्ता विमलाश्च दिशोदश}
{ततस्स राजा हर्षेण पुत्रं दृष्ट्वा सनातनम्}% ७४

\twolineshloka
{पुरोधसा वसिष्ठेन जातकर्म्मतदाऽकरोत्}
{नाम चास्मै ददौ रम्यं वसिष्ठो भगवांस्तदा}% ७५

\twolineshloka
{श्रियः कमलवासिन्या रमणोऽयं महान्प्रभुः}
{तस्माच्छ्रीराम इत्यस्य नाम सिद्धं पुरातनम्}% ७६

\twolineshloka
{सहस्रनाम्नां श्रीशस्य तुल्यं मुक्तिप्रदं नृणाम्}
{विष्णुना स समुत्पन्नो विष्णुरित्यभिधीयते}% ७७

\twolineshloka
{एवं नामास्य दत्वाथ वसिष्ठो भगवानृषिः}
{परिणीय नमस्कृत्य स्तुत्वा स्तुतिभिरेव च}% ७८

\twolineshloka
{सङ्कीर्त्य नामसाहस्रं मङ्गलार्थं महात्मनः}
{विनिर्ययौ महातेजास्तस्मात्पुण्यतमाद्गृहात्}% ७९

\twolineshloka
{राजाथ विप्रमुख्येभ्यो ददौ बहुधनं मुदा}
{गवामयुतदानं च कारयामास धर्म्मतः}% ८०

\twolineshloka
{ग्रामाणां शतसाहस्रं ददौ रघुकुलोत्तमः}
{वस्त्रैराभरणैर्दिव्यैरसङ्ख्येयैर्धनैरपि}% ८१

\twolineshloka
{विष्णोस्सन्तुष्टये तत्र तर्प्पयामास भूसुरान्}
{कौसल्या च सुतं दृष्ट्वा रामं राजीवलोचनम्}% ८२

\twolineshloka
{फुल्लहस्तारविन्दाभं पद्महस्ताम्बुजान्वितम्}
{तस्य श्रीपादकमले पद्माब्जे च वरानने}% ८३

\twolineshloka
{शङ्खचक्रगदापद्मध्वजवस्त्रादिचिह्निते}
{दृष्ट्वा वक्षसि श्रीवत्सं कौस्तुभं वनमालया}% ८४

\twolineshloka
{तस्याङ्गे सा जगत्सर्वं सदेवासुरमानुषम्}
{स्मितवक्त्रे विशालाक्षी भुवनानि चतुर्दश}% ८५

\twolineshloka
{निश्वासे तस्य वेदांश्च सेतिहासान्महात्मनः}
{द्वीपानब्धीन्गिरींस्तस्य जघने वरवर्णिनि}% ८६

\twolineshloka
{नाभ्यां ब्रह्मशिवौ तस्य कर्णयोश्च दिशः शुभाः}
{नेत्रयोर्वह्निसूर्यौ च घ्राणे वायुं महाजवम्}% ८७


\threelineshloka
{सर्वोपनिषदामर्थं दृष्ट्वा तस्य विभूतयः}
{कृत्स्ना भीता वरारोहा प्रणम्य च पुनः पुनः}
{हर्षाश्रुपूर्णनयना प्राञ्जलिर्वाक्यमब्रवीत्}% ८८

\uvacha{कौशल्योवाच}

\twolineshloka
{धन्यास्मि देवदेवेश लब्ध्वा त्वां तनयं प्रभो}
{प्रसीद मे जगन्नाथ पुत्रस्नेहं प्रदर्शय}% ८९

\uvacha{ईश्वर उवाच}

\twolineshloka
{एवमुक्तो हृषीकेशो मात्रा सर्वगतो हरिः}
{मायामानुषतां प्राप्य शिशुभावाद्रुरोद सः}% ९०

\twolineshloka
{अथ प्रमुदिता देवी कौशल्या शुभलक्षणा}
{पुत्रमालिङ्ग्य हर्षेण स्तन्यं प्रादात्सुमध्यमा}% ९१

\twolineshloka
{तस्याः स्तन्यं पपौ देवो बालभावात्सनातनः}
{उवास मातुरुत्सङ्गे जगद्भर्ता महाविभुः}% ९२

\twolineshloka
{देशे तस्मिञ्छुशुभे रम्ये सर्वकामप्रदे नृणाम्}
{उत्सवं चक्रिरे पौरा हृष्टा जनपदा नराः}% ९३

\twolineshloka
{कैकेय्यां भरतो जज्ञे पाञ्चजन्यांशचोदितः}
{सुमित्रा जनयामास लक्ष्मणं शुभलक्षणम्}% ९४

\twolineshloka
{शत्रुघ्नं च महाभागा देवशत्रुप्रतापनम्}
{अनन्तांशेन सम्भूतो लक्ष्मणः परवीरहा}% ९५

\twolineshloka
{सुदर्शनांशाच्छत्रुघ्नः सञ्जज्ञेऽमितविक्रमः}
{ते सर्वे ववृधुस्तत्र वैवस्वतमनोः कुले}% ९६

\twolineshloka
{संस्कृतास्ते सुताः सम्यग्वसिष्ठेन महौजसा}
{अधीतवेदास्ते सर्वे श्रुतवन्तस्तथा नृपाः}% ९७

\twolineshloka
{सर्वशास्त्रार्थतत्वज्ञा धनुर्वेदे च निष्ठिताः}
{बभूवुः परमोदारा लोकानां हर्षवर्द्धनाः}% ९८

\twolineshloka
{युग्मं बभूवतुस्तत्र राजानौ रामलक्ष्मणौ}
{तथा भरतशत्रुघ्नौ तयोर्युग्मं बभूव ह}% ९९

\twolineshloka
{अथ लोकेश्वरी लक्ष्मीर्जनकस्य निवेशने}
{शुभक्षेत्रे हलोद्धाते सुनासीरे शुभेक्षणे}% १००

\twolineshloka
{बालार्ककोटिसङ्काशा रक्तोत्पलकराम्बुजा}
{सर्वलक्षणसम्पन्ना सर्वाभरणभूषिता}% १०१

\twolineshloka
{धृत्वा वक्षसि चार्वङ्गी मालामम्लानपङ्कजाम्}
{सीतामुखे समुत्पन्ना बालभावेन सुन्दरी}% १०२

\twolineshloka
{तां दृष्ट्वा जनको राजा कन्यां वेदमयीं शुभाम्}
{उद्धृत्यापत्यभावेन पुपोष मिथिलापतिः}% १०३

\twolineshloka
{जनकस्य गृहे रम्ये सर्वलोकेश्वरप्रिया}
{ववृधे सर्वलोकस्य रक्षणार्थं सुरेश्वरी}% १०४

\twolineshloka
{एतस्मिन्नन्तरे देवि कौशिको लोकविश्रुतः}
{सिद्धाश्रमे महापुण्ये भागीरथ्यास्तटे शुभे}% १०५

\twolineshloka
{क्रतुप्रवरमारेभे यष्टुं तत्र महामुनिः}
{वर्त्तमानस्य तस्यास्य यज्ञस्याथ द्विजन्मनः}% १०६

\twolineshloka
{क्रतुविध्वंसिनोऽभूवन्रावणस्य निशाचराः}
{कौशिकश्चिन्तयित्वाथ रघुवंशोद्भवं हरिम्}% १०७

\twolineshloka
{आनेतुमैच्छद्धर्मात्मा लोकानां हितकाम्यया}
{स गत्वा नगरीं रम्यामयोध्यां रघुपालिताम्}% १०८

\twolineshloka
{नृपश्रेष्ठं दशरथं ददर्श मुनिसत्तमः}
{राजापि कौशिकं दृष्ट्वा प्रत्युत्थाय कृताञ्जलिः}% १०९

\twolineshloka
{पुत्रैः सह महातेजा ववन्दे मुनिसत्तमम्}
{धन्योऽहमस्मीति वदन्हर्षेण रघुनन्दनम्}% ११०

\twolineshloka
{अर्चयामास विधिना निवेश्य परमासने}
{परिणीय नमस्कृत्य किं करोमीत्युवाच तम्}% १११

\onelineshloka*
{ततः प्रोवाच हृष्टात्मा विश्वामित्रो महातपाः}

\uvacha{विश्वामित्र उवाच}
\onelineshloka
{देहि मे राघवं राजन्रक्षणार्थं क्रतोर्मम}% ११२

\twolineshloka
{साफल्यमस्तु मे यज्ञे राघवस्य समीपतः}
{तस्माद्रामं रक्षणार्थं दातुमर्हसि भूपते}% ११३

\uvacha{ईश्वर उवाच}

\twolineshloka
{तच्छ्रुत्वा मुनिवर्य्यस्य वाक्यं सर्वविदां वरः}
{प्रददौ मुनिवर्य्याय राघवं सह लक्ष्मणम्}% ११४

\twolineshloka
{आदाय राघवं तत्र विश्वामित्रो महातपाः}
{स्वमाश्रममभिप्रीतः प्रययौ द्विजसत्तमः}% ११५

\twolineshloka
{ततः प्रहृष्टास्त्रिदशाः प्रयाते रघुसत्तमे}
{ववृषुः पुष्पवर्षाणि तुष्टुवुश्च महौजसः}% ११६

\twolineshloka
{अथाजगाम हृष्टात्मा वैनतेयो महाबलः}
{अदृश्यभूतो भूतानां सम्प्राप्य रघुसत्तमम्}% ११७

\twolineshloka
{ताभ्यां दिव्ये च धनुषी तूणौ चाक्षयसायकौ}
{दिव्यान्यस्त्राणि शस्त्राणि दत्वा च प्रययौ द्विजः}% ११८

\twolineshloka
{तौ रामलक्ष्मणौ वीरौ कौशिकेन महात्मना}
{गच्छन्ती ज्ञापितारण्ये राक्षसी घोरदर्शना}% ११९

\twolineshloka
{नाम्ना तु ताडका देवि भार्या सुन्दस्य रक्षसः}
{जघ्नतुस्तां महावीरौ बाणैर्दिव्यधनुश्च्युतैः}% १२०

\twolineshloka
{निहता राघवेणाथ राक्षसी घोरदर्शना}
{त्यक्त्वा तनुं घोररूपां दिव्यरूपा बभूव सा}% १२१

\twolineshloka
{जाज्वल्यमानावपुषा सर्वाभरणविभूषिता}
{प्रययौ वैष्णवं लोकं प्रणम्य च रघूत्तमौ}% १२२

\twolineshloka
{तां हत्वा राघवः श्रीमान्कौशिकस्याश्रमं शुभम्}
{प्रविवेश महातेजा लक्ष्मणेन महात्मना}% १२३

\twolineshloka
{ततः प्रहृष्टा मुनयः प्रत्युद्गम्य रघूत्तमम्}
{निवेश्य पूजयामासुरर्घाद्यैः परमात्मने}% १२४

\twolineshloka
{कौशिकः कृतदीक्षस्तु यंष्टुं यज्ञमनुत्तमम्}
{आरेभे मुनिभिः सार्द्धं विधिना मुनिसत्तमः}% १२५

\twolineshloka
{वर्त्तमाने महायज्ञे मारीचो नाम राक्षसः}
{भ्रात्रा सुबाहुना तत्र विघ्नं कर्तुमवस्थितः}% १२६

\twolineshloka
{दृष्ट्वा तौ राक्षसौ घोरौ राघवः परवीरहा}
{जघानैकेन बाणेन सुबाहुं राक्षसेश्वरम्}% १२७

\twolineshloka
{पवनास्त्रेण महता मारीचं स निशाचरम्}
{सागरे पातयामास शुष्कपर्णमिवानिलः}% १२८

\twolineshloka
{स रामस्य महावीर्य्यं दृष्ट्वा राक्षससत्तमः}
{न्यस्तशस्त्रस्तपस्तप्तुं प्रययौ महादाश्रमम्}% १२९

\twolineshloka
{विश्वामित्रो महातेजाः समाप्ते महति क्रतौ}
{प्रहृष्टमनसा तत्र पूजयामास राघवम्}% १३०

\twolineshloka
{समाश्लिष्य महात्मानं काकपक्षधरं हरिम्}
{नीलोत्पलदलश्यामं पद्मपत्रायतेक्षणम्}% १३१

\twolineshloka
{उपाघ्राय तदा मूर्ध्नि तुष्टाव मुनिसत्तमः}
{एतस्मिन्नन्तरे राजा मिथिलाया अधीश्वरः}% १३२

\twolineshloka
{वाजपेयं क्रतुं यष्टुमारेभे मुनिसत्तमैः}
{तं द्रष्टुं प्रययुस्सर्वे विश्वामित्रपुरोगमाः}% १३३

\twolineshloka
{मुनयो रघुशार्दूल सहिताः पुण्यचेतसः}
{गच्छतस्तस्य रामस्य पदाब्जेन महात्मनः}% १३४

\twolineshloka
{अभूत्सा पावनी भूमिः समाक्रान्ता महाशिला}
{सापि शप्ता पुरा भर्त्रा गौतमेन द्विजन्मना}% १३५

\twolineshloka
{अहल्या रघुनाथस्य पादस्पर्शाच्छुभाऽभवत्}
{अथ सम्प्राप्य नगरीं मिथिलां मुनिसत्तमाः}% १३६

\twolineshloka
{राघवाभ्यां तु सहिता बभूवुः प्रीतमानसाः}
{समागतान्महाभागान्दृष्ट्वा राजा महाबलः}% १३७

\twolineshloka
{प्रत्युद्गम्य प्रणम्याथ पूजयामास मैथिलः}
{रामं पद्मविशालाक्षमिन्दीवरदलप्रभम्}% १३८

\twolineshloka
{पीताम्बरधरं श्लक्ष्णं कोमलावयवोज्ज्वलम्}
{अवधीरित कन्दर्प्पकोटिलावण्यमुत्तमम्}% १३९

\twolineshloka
{सर्वलक्षणसम्पन्नं सर्वाभरणभूषितम्}
{स्वस्य हृत्पद्ममध्ये यः परेशस्य तनुर्हरिः}% १४०

\twolineshloka
{उत्पन्नो दीपवद्दीपात्सौशील्यादिगुणैः परैः}
{तं दृष्ट्वा रघुनाथं स जनको हृष्टमानसः}% १४१

\twolineshloka
{परेशमेव तं मेने रामं दशरथात्मजम्}
{पूजयामास काकुत्स्थं धन्योस्मीति ब्रुवन्नृपः}% १४२

\twolineshloka
{प्रसादं वासुदेवस्य विष्णोर्मेने नरेश्वरः}
{प्रदातुं दुहितां तस्मै मनसा चिन्तयन्प्रभुः}% १४३

\twolineshloka
{आत्मजौ रघुवंशस्य ज्ञात्वा तत्र नृपोत्तमः}
{पूजयामास धर्मेण वस्त्रैराभरणैः शुभैः}% १४४

\twolineshloka
{ऋषीन्समर्चयामास मधुपर्कादिपूजनैः}
{ततोऽवसाने यज्ञस्य रामो राजीवलोचनः}% १४५

\twolineshloka
{भङ्क्त्वा शैवं धनुर्दिव्यं जितवान्जनकात्मजाम्}
{अथासौ वीर्यशुल्केन महता परितोषितः}% १४६

\twolineshloka
{मुदा धरणिजां तस्मै प्रददौ मिथिलाधिपः}
{केशवाय श्रियमिव यथापूर्वं महार्णवः}% १४७

\twolineshloka
{स दूतं प्रेषयामास राघवं मिथिलाधिपः}
{पुत्राभ्यां सह धर्मात्मा मिथिलायां विवेश ह}% १४८

\twolineshloka
{वसिष्ठवामदेवाद्यैः प्रीतैः सह महीपतिः}
{उवास नगरे रम्ये जनकस्य रघूत्तमः}% १४९

\twolineshloka
{तस्मिन्नेव शुभे काले रामस्य धरणीसुताम्}
{विवाहमकरोद्राजा मिथिलेन समर्चितः}% १५०

\twolineshloka
{लक्ष्मणस्योर्मिलां नाम कन्यां जनकसम्भवाम्}
{जनकस्यानुजस्याथ तनये शुभवर्चसी}% १५१

\twolineshloka
{माण्डवी श्रुतकीर्त्तिश्च सर्वलक्षणलक्षिते}
{भरतस्य च सौमित्रेर्विवाहमकरोन्नृपः}% १५२

\twolineshloka
{निर्वर्त्यौद्वाहिकं तत्र राजा दशरथो बली}
{अयोध्यां प्रस्थितः श्रीमान्पौरैर्जनपदैर्वृतः}% १५३

\twolineshloka
{पारिबर्हं समादाय मैथिलेन च पूजितः}
{ससुतः सस्नुषः साश्वः सगजः सबलानुगः}% १५४

\twolineshloka
{तदध्वनि महावीर्य्यो जामदग्निः प्रतापवान्}
{गृहीत्वा परशुं चापं सङ्क्रुद्ध इव केसरी}% १५५

\twolineshloka
{अभ्यधावच्च काकुत्स्थं योद्धुकामो नृपान्तकः}
{सम्प्राप्य राघवं दृष्ट्वा वचनं प्राह भार्गवः}% १५६

\uvacha{परशुराम उवाच}

\twolineshloka
{रामराम महाबाहो शृणुष्व वचनं मम}
{बहुशः पार्थिवान्हत्वा संयुगे भूरिविक्रमान्}% १५७

\twolineshloka
{ब्राह्मणेभ्यो महीं दत्वा तपस्तप्तुमहं गतः}
{तव वीर्यबलं श्रुत्वा त्वया योद्धुमिहागतः}% १५८

\twolineshloka
{इक्ष्वाकवो न वध्या मे मातामहकुलोद्भवाः}
{वीर्य्यं क्षत्रबलं श्रुत्वा न शक्यं सहितुं मम}% १५९

\twolineshloka
{रौद्रं चापं दुराधर्षं भज्यमानां त्वया नृप}
{तस्माद्वदान्य युद्धं मे दीयतां रघुसत्तम}% १६०

\twolineshloka
{इदं तु वैष्णवं चापं तेन तुल्यमरिन्दम}
{आरोपय स्ववीर्येण निर्जितोस्मि त्वयैव हि}% १६१

\twolineshloka
{अथवा त्यज शस्त्राणि पुरस्ताद्बलिनो मम}
{शरणं भज काकुत्स्थ कातरोस्यथ चेतसी}% १६२

\uvacha{ईश्वर उवाच}

\twolineshloka
{एवमुक्तस्तु काकुत्स्थो भार्गवेण प्रतापवान्}
{तच्चापं तस्य जग्राह तच्छक्तिं वैष्णवीमपि}% १६३

\twolineshloka
{शक्त्या वियुक्तस्स तदा जामदग्निः प्रतापवान्}
{निर्वीर्यो नष्टतेजाभूत्कर्म्महीनो यथा द्विजः}% १६४

\twolineshloka
{विनष्टतेज सन्दृष्ट्वा भार्गवं नृपसत्तमाः}
{साधुसध्विति काकुत्स्थं प्रशशंसुर्मुहुर्मुहुः}% १६५

\twolineshloka
{काकुत्स्थस्तन्महच्चापं गृहीत्वारोप्य लीलया}
{सन्धाय बाणं तच्चापे भार्गवं प्राह विस्मितम्}% १६६

\uvacha{राम उवाच}

\twolineshloka
{अनेन शरमुख्येन किं कर्त्तव्यं तव द्विज}
{छेद्मि लोकमिमं चाधः स्वर्गं वा हन्मि ते तपः}% १६७

\uvacha{ईश्वर उवाच}

\twolineshloka
{तन्दृष्ट्वा घोरसङ्काशं बाणं रामस्य भार्गवः}
{ज्ञात्वा तं परमात्मानं प्रहृष्टो राममब्रवीत्}% १६८

\uvacha{परशुराम उवाच}

\twolineshloka
{रामराम महाबाहो न वेद्मि त्वां सनातनम्}
{जानाम्यद्यैव काकुत्स्थ तव वीर्य्यगुणादिभिः}% १६९

\twolineshloka
{त्वमादिपुरुषः साक्षात्परब्रह्मपरोऽव्ययः}
{त्वमनन्तो महाविष्णुर्वासुदेवः परात्परः}% १७०

\twolineshloka
{नारायणस्त्वं श्रीशस्त्वमीश्वरस्त्वं त्रयीमयः}
{त्वं कालस्त्वं जगत्सर्वमकाराख्यस्त्वमेव च}% १७१

\twolineshloka
{स्रष्टा धाता च संहर्त्ता त्वमेव परमेश्वरः}
{त्वमचिन्त्यो महद्भूतरूपस्त्वं तु मनुर्महान्}% १७२

\twolineshloka
{चतुःषट्पञ्चगुणवांस्त्वमेव पुरुषोत्तमः}
{त्वं यज्ञस्त्वं वषट्कारस्त्वमोङ्कारस्त्रयीमयः}% १७३

\twolineshloka
{व्यक्ताव्यक्तस्वरूपस्त्वं गुणभृन्निर्ग्गुणः परः}
{स्तोतुं त्वाहमशक्तश्च वेदानामप्यगोचरम्}% १७४

\twolineshloka
{यच्चापलत्वं कृतवांस्त्वां युयुत्सुतया प्रभो}
{तत्क्षन्तव्यं त्वया नाथ कृपया केवलेन तु}% १७५

\twolineshloka
{तव शक्त्या नृपान्सर्वाञ्जित्वा दत्वा महीं द्विजान्}
{त्वत्प्रसादवशादेव शान्तिमाप्नोति नैष्ठिकीम्}% १७६

\uvacha{ईश्वर उवाच}

\twolineshloka
{एवमुक्त्वा तु काकुत्स्थं जामदग्निर्महातपाः}
{परिणीय नमस्कृत्वा राघवं लोकरक्षकम्}% १७७

\twolineshloka
{शतक्रतुकृतं स्वर्गं तदस्त्राय न्यवेदयत्}
{राघवोऽथ महातेजा ववन्दे तं महामुनिम्}% १७८

\twolineshloka
{विधिवत्पूजयामास पाद्यार्घाचमनादिभिः}
{तेन सम्पूजितस्तत्र जामदग्निर्महातपाः}% १७९

\twolineshloka
{तपस्तप्तुं ययौ रम्यं नरनारायणाश्रमम्}
{राजा दशरथः सोऽथ पुत्रैर्दारसमन्वितैः}% १८०

\twolineshloka
{स्वां पुरीं सुमुहूर्त्तेन प्रविवेश महाबलः}
{राघवो लक्ष्मणश्चैव शत्रुघ्नो भरतस्तथा}% १८१

\twolineshloka
{स्वान्स्वान्दारानुपागम्य रेमिरे हृष्टमानसाः}
{तत्र द्वादश वर्षाणि सीतया सह राघवः}% १८२

\twolineshloka
{रमयामास धर्मात्मा नारायण इव श्रिया}
{तस्मिन्नेव तु राजाथ काले दशरथः सुतम्}% १८३

\twolineshloka
{ज्येष्ठं राज्येन संयोक्तुमैच्छत्प्रीत्या महीपतिः}
{तस्य भार्याथ कैकेयी पुरा दत्तवरा प्रिया}% १८४

\twolineshloka
{अयाचत नृपश्रेष्ठं भरतस्याभिषेचनम्}
{विवासनं च रामस्य वत्सराणि चतुर्दश}% १८५

\twolineshloka
{स राजा सत्यवचनाद्रामं राज्यादथोः सुतम्}
{विवासयामास तदा दुःखेन हतचेतनः}% १८६

\twolineshloka
{शक्तोऽपि राघवस्तस्मिन्राज्यं सन्त्यज्य धर्मतः}
{दशग्रीववधार्थाय पितुर्वचनहेतुना}% १८७

\twolineshloka
{वनं जगाम काकुत्स्थो लक्ष्मणेन च सीतया}
{राजा पुत्रवियोगार्त्तः शोकेन च ममार सः}% १८८

\twolineshloka
{नियुज्यमानो भरतस्तस्मिन्राज्ये समन्त्रिभिः}
{नैच्छद्राज्यं स धर्म्मात्मा सौभ्रात्रमनुदर्शयन्}% १८९

\twolineshloka
{वनमागम्य काकुत्स्थमयाचद्भ्रातरं ततः}
{रामस्तु पितुरादेशान्नैच्छद्राज्यमरिन्दमः}% १९०

\twolineshloka
{स्वपादुके ददौ तस्मै भक्त्या सोऽप्यग्रहीत्तथा}
{रामस्य पादुके राज्यमवाप्य भरतः शुभे}% १९१

\twolineshloka
{प्रत्यहं गन्धपुष्पैश्च पूजयन्कैकयीसुतः}
{तपश्चरणयुक्तेन तस्मिंस्तस्थौ नृपोत्तमः}% १९२

\twolineshloka
{यावदागमनं तस्य राघवस्य महात्मनः}
{तावद्व्रतपराः सर्वे बभूवुः पुरवासिनः}% १९३

\twolineshloka
{राघवश्चित्रकूटाद्रौ भरद्वाजाश्रमे शुभे}
{रमयामास वैदेह्या मन्दाकिन्या जले शुभे}% १९४

\twolineshloka
{कदाचिदङ्के वैदेह्याः शेते रामो महामनाः}
{ऐन्द्रिः काकस्समागम्य तस्मिन्नेव चचार ह}% १९५

\twolineshloka
{स दृष्ट्वा जानकीं तत्र कन्दर्प्पशरपीडितः}
{विददार नखैस्तीक्ष्णैः पीनोन्नतपयोधरम्}% १९६

\twolineshloka
{तं दृष्ट्वा वायसं रामः कुशं जग्राह पाणिना}
{ब्रह्मणास्त्रेण संयोज्य चिक्षेप धरणीधरः}% १९७

\twolineshloka
{तं तृणं घोरसङ्काशं ज्वालारचितविग्रहम्}
{दृष्ट्वा काकः प्रदुद्राव विमुञ्चन्कातरं स्वरम्}% १९८

\twolineshloka
{तं काकं प्रत्यनुययौ रामस्यास्त्रं सुदारुणम्}
{वायसस्त्रिषुलोकेषु बभ्राम भयपीडितः}% १९९

\twolineshloka
{यत्र यत्र ययौ काकः शरणार्थी स वायसः}
{तत्र तत्र तदस्त्रं तु प्रविवेश भयावहम्}% २००

\twolineshloka
{ब्रह्माणमिन्द्रं रुद्रं च यमं वरुणमेव च}
{शरणार्थी जगामाशु वायसः शस्त्रपीडितः}% २०१


\threelineshloka
{तं दृष्ट्वा वायसं सर्वे रुद्राद्या देव दानवाः}
{न शक्ताः स्म वयं त्रातुमिति प्राहुर्मनीषिणः}
{अथ प्रोवाच भगवान्ब्रह्मा त्रिभुवनेश्वरः}% २०२

\uvacha{ब्रह्मोवाच}

\twolineshloka
{भो भो बलिभुजां श्रेष्ठ तमेव शरणं व्रज}
{स एव रक्षकः श्रीमान्सर्वेषां करुणानिधिः}% २०३

\twolineshloka
{रक्षत्येव क्षमासारो वत्सलं शरणागतान्}
{ईश्वरः सर्वभूतानां सौशील्यादिगुणान्वितः}% २०४

\twolineshloka
{रक्षिता जीवलोकस्य पिता माता सखा सुहृत्}
{शरणं व्रज देवेशं नान्यत्र शरणं द्विज}% २०५

\uvacha{महादेव उवाच}

\twolineshloka
{इत्युक्तस्तेन बलिभुग्ब्रह्मणा रघुनन्दनम्}
{उपेत्य सहसा भूमौ निपपात भयातुरः}% २०६

\twolineshloka
{प्राणसंशयमापन्नं दृष्ट्वा सीताथ वायसम्}
{त्राहित्राहीति भर्तारमुवाच विनयाद्विभुम्}% २०७

\twolineshloka
{पुरतः पतितं देवी धरण्यां वायसं तदा}
{तच्छिरः पादयोस्तस्य योजयामास जानकी}% २०८

\twolineshloka
{समुत्थाप्य करेणाथ कृपापीयूषसागरः}
{ररक्ष रामो गुणवान् वायसं दययार्दितः}% २०९

\twolineshloka
{तमाह वायसं रामो मा भैरिति दयानिधिः}
{अभयं ते प्रदास्यामि गच्छ गच्छ यथासुखम्}% २१०

\twolineshloka
{प्रणम्य राघवायाथ सीतायै च मुहुर्मुहुः}
{स्वर्ल्लोकं प्रययावाशु राघवेण च रक्षितः}% २११

\twolineshloka
{ततो रामस्तु वैदेह्या लक्ष्मणेन च धीमता}
{उवास चित्रकूटाद्रौ स्तूयमानो महर्षिभिः}% २१२

\twolineshloka
{तस्मिन्सम्पूज्यमानस्तु भरद्वाजेन राघवः}
{जगामात्रेस्सुविपुलमाश्रमं रघुसत्तमः}% २१३

\twolineshloka
{समागतं रघुवरं दृष्ट्वा मुनिवरोत्तमः}
{भार्यया सह धर्म्मात्मा प्रत्युद्गम्य मुदा युतः}% २१४

\twolineshloka
{आसने सुशुभे मुख्ये निवेश्य सह सीतया}
{अर्घ्यपाद्याचमनीयं च वस्त्राणि विविधानि च}% २१५

\twolineshloka
{मधुपर्कन्ददौ प्रीत्या भूषणं चानुलेपनम्}
{तस्य पत्न्यनसूया तु दिव्याम्बरमनुत्तमम्}% २१६

\twolineshloka
{सीतायै प्रददौ प्रीत्या भूषणानि द्युमन्ति च}
{दिव्यान्नपानभक्षाद्यैर्भोजयामास राघवम्}% २१७

\twolineshloka
{तेन सम्पूजितस्तत्र भक्त्या परमया नृपः}
{उवास दिवसं तत्र प्रीत्या रामस्सलक्ष्मणः}% २१८

\twolineshloka
{प्रभाते विमले रामः समुत्थाय महामुनिम्}
{परिणीय प्रणम्याथ गमनायोपचक्रमे}% २१९

\twolineshloka
{अनुज्ञातस्ततस्तेन रामो राजीवलोचनः}
{प्रययौ दण्डकारण्यं महर्षिकुलसङ्कुलम्}% २२०

\twolineshloka
{तत्रातिभीषणं घोरं विराधं नाम राक्षसम्}
{हत्वाथ शरभङ्गस्य प्रविवेशाश्रमं शुभम्}% २२१

\twolineshloka
{स तु दृष्ट्वाथ काकुत्स्थं सद्यः सङ्क्षीणकल्मषः}
{प्रययौ ब्रह्मलोकं तु गन्धर्वाप्सरसान्वितम्}% २२२

\twolineshloka
{सुतीक्ष्णस्याप्यगस्त्यस्य ह्यगस्त्यभ्रातुरेव च}
{क्रमेण प्रययौ रामस्तैश्च सम्पूजितस्तथा}% २२३

\twolineshloka
{पञ्चवट्यां ततो रामो गोदावर्यास्तटे शुभे}
{उवास सुचिरं कालं सुखेन परमेण च}% २२४

\twolineshloka
{तत्र गत्वा मुनिश्रेष्ठास्तापसा धर्मचारिणः}
{पूजयामासुरात्मेशं रामं राजीवलोचनम्}% २२५

\twolineshloka
{भयं विज्ञापयामासुस्तं च रक्षोगणेरितम्}
{तानाश्वास्य तु काकुस्थो ददौ चाभयदक्षिणाम्}% २२६

\twolineshloka
{ते तु सम्पूजितास्तेन स्वाश्रमान्सम्प्रपेदिरे}
{तस्मिंस्त्रयोदशाब्दानि रामस्य परिनिर्य्ययुः}% २२७

\twolineshloka
{गोदावर्य्यास्तटे रम्ये पञ्चवट्यां मनोरमे}
{कस्यचित्त्वथ कालस्य राक्षसी घोररूपिणी}% २२८

\twolineshloka
{रावणस्य स्वसा तत्र प्रविवेश दुरासदा}
{सा तु दृष्ट्वा रघुवरं कोटिकन्दर्प्पसन्निभम्}% २२९

\twolineshloka
{इन्दीवरदलश्यामं पद्मपत्रायतेक्षणम्}
{प्रोन्नतांसं महाबाहुं कम्बुग्रीवं महाहनुम्}% २३०

\twolineshloka
{सम्पूर्णचन्द्रसदृशं सस्मिताननपङ्कजम्}
{भृङ्गावलिनिभैः स्निग्धैः कुटिलैः शीर्षजैर्वृतम्}% २३१

\twolineshloka
{रक्तारविन्दसदृशं पद्महस्ततलाङ्कितम्}
{निष्कलङ्केन्दुसदृशं नखपङ्क्तिविराजितम्}% २३२

\twolineshloka
{स्निग्धकोमलदूर्वाभं सौकुमार्य्यनिधिं शुभम्}
{पीतकौशेयवसनं सर्वाभरणभूषितम्}% २३३

\twolineshloka
{युवाकुमारवयसं जगन्मोहनविग्रहम्}
{दृष्ट्वा तं राक्षसी रामं कन्दर्प्पशरपीडिता}% २३४

\onelineshloka*
{अब्रवीत्समुपेत्याथ रामं कमललोचनम्}

\uvacha{राक्षस्युवाच}
\onelineshloka
{कस्त्वं तापसवेषेण वर्त्तसे दण्डके वने}% २३५

\twolineshloka
{आगतोऽसि किमर्थं च राक्षसानां दुरासदे}
{शीघ्रमाचक्ष्व तत्त्वेन नानृतं वक्तुमर्हसि}% २३६

\uvacha{महेश्वर उवाच}

\onelineshloka*
{इत्युक्तः स तदा रामः सम्प्रहस्याब्रवीद्वचः}

\uvacha{राम उवाच}

\twolineshloka
{राज्ञो दशरथस्याहं पुत्रो राम इतीरितः}
{असौ ममानुजो धन्वी लक्ष्मणो नाम चानघः}% २३७

\twolineshloka
{पत्नी चेयं च मे सीता जनकस्यात्मजा प्रिया}
{पितुर्वचननिर्देशादहं वनमिहागतः}% २३८

\twolineshloka
{विचरामो महारण्यमृषीणां हितकाम्यया}
{आगतासि किमर्थं त्वमाश्रमं मम सुन्दरि}% २३९

\onelineshloka*
{का त्वं कस्य कुले जाता सर्वं सत्यं वदस्व मे}

\uvacha{महेश्वर उवाच}
\onelineshloka
{इत्युक्ता सा तु रामेण प्राह वाक्यमशङ्किता}% २४०

\uvacha{राक्षस्युवाच}

\twolineshloka
{अहं विश्रवसः पुत्री रावणस्य स्वसा नृप}
{नाम्ना शूर्पणखा नाम त्रिषु लोकेषु विश्रुता}% २४१

\twolineshloka
{इदं च दण्डकारण्यं भ्रात्रा दत्तं मम प्रभो}
{भक्षयन्नृषिसङ्घान्वै विचरामि महावने}% २४२

\twolineshloka
{त्वां तु दृष्ट्वा मुनिवरं कन्दर्पशरपीडिता}
{रन्तुकामा त्वया सार्द्धमागतास्मि सुनिर्भया}% २४३

\twolineshloka
{मम त्वं नृपशार्दूल भर्ता भवितुमर्हसि}
{इमां तव सतीं सीतां ग्रसितुं भूप कामये}% २४४

\onelineshloka*
{वनेषु गिरिमुख्येषु रमयामि त्वया सह}

\uvacha{महेश्वर उवाच}

\onelineshloka
{इत्युक्त्वा राक्षसी सीतां ग्रसितुं वीक्ष्य चोद्यताम्}% २४५

\onelineshloka
{श्रीरामः खड्गमुद्यम्य नासाकर्णौ प्रचिच्छिदे}% २४६

\twolineshloka
{रुदन्ती सभयं शीघ्रं राक्षसी विकृतानना}
{खरालयं प्रविश्याह तस्य रामस्य चेष्टितम्}% २४७

\twolineshloka
{स तु राक्षससाहस्रैर्दूषणत्रिशिरो वृतः}
{आजगाम भृशं योद्धुं राघवं शत्रुसूदनः}% २४८

\twolineshloka
{तान्रामः कानने घोरे बाणः कालान्तकोपमैः}
{निजघान महाकायान्राक्षसांस्तत्र लीलया}% २४९

\twolineshloka
{खरं त्रिशिरसं चैव दूषणं तु महाबलम्}
{रणे निपातयामास बाणैराशीविषोपमैः}% २५०

\twolineshloka
{निहत्य राक्षसान्सर्वान्दण्डकारण्यवासिनः}
{पूजितः सुरसङ्घैश्च स्तूयमानो महर्षिभिः}% २५१

\twolineshloka
{उवास दण्डकारण्ये सीतया लक्ष्मणेन च}
{राक्षसानां वधं श्रुत्वा रावणः क्रोधमूर्च्छितः}% २५२

\twolineshloka
{आजगाम जनस्थानं मारीचेन दुरात्मना}
{सम्प्राप्य पञ्चवट्यां तु दशग्रीवः स राक्षसः}% २५३

\twolineshloka
{मायाविना मरीचेन मृगरूपेण रक्षसः}
{अपहृत्याश्रमाद्दूरे तौ तु दशरथात्मजौ}% २५४

\twolineshloka
{जहार सीतां रामस्य भार्यां स्ववधकाङ्क्षया}
{ह्रियमाणां तु तां दृष्ट्वा जटायुर्गृध्रराड्बली}% २५५

\twolineshloka
{रामस्य सौहृदात्तत्र युयुधे तेन रक्षसा}
{तं हत्वा बाहुवीर्येण रावणं शत्रुवारणः}% २५६

\twolineshloka
{प्रविवेश पुरीं लङ्कां राक्षसैर्बहुभिर्वृताम्}
{अशोकवनिकामध्ये निःक्षिप्य जनकात्मजाम्}% २५७

\twolineshloka
{निधनं रामबाणेन काङ्क्षयन्स्वगृहं विशत्}
{रामस्तु राक्षसं हत्वा मारीचं मृगरूपिणम्}% २५८

\twolineshloka
{पुनराविश्य तत्राथ भ्रात्रा सौमित्रिणा ततः}
{राक्षसापहृतां भार्यां ज्ञात्वा दशरथात्मजः}% २५९

\twolineshloka
{प्रभूतशोकसन्तप्तो विललाप महामतिः}
{मार्गमाणो वने सीतां पथि गृध्रं महाबलम्}% २६०

\twolineshloka
{विच्छिन्नपादपक्षं च पतितं धरणीतले}
{रुधिरापूर्णसर्वाङ्गं दृष्ट्वा विस्मयमागतः}% २६१

\twolineshloka
{पप्रच्छ राघवं श्रीमान्केन किं त्वं जिघांसितः}
{गृध्रस्तु राघवं दृष्ट्वा मन्दमन्दमुवाच ह}% २६२

\uvacha{गृध्र उवाच}

\twolineshloka
{रावणेन हृता राम तव भार्यां बलीयसा}
{तेन राक्षसमुख्येन सङ्ग्रामे निहतोस्म्यहम्}% २६३

\uvacha{महेश्वर उवाच}

\twolineshloka
{इत्युक्त्वा राघवस्याग्रे सहसा त्यक्तजीवितः}
{संस्कारमकरोद्रामस्तस्य ब्रह्मविधानतः}% २६४

\twolineshloka
{स्वपदं च ददौ तस्मै योगिगम्यं सनातनम्}
{राघवस्य प्रसादेन स गृध्रः परमं पदम्}% २६५

\twolineshloka
{हरेः सामान्यरूपेण मुक्तिं प्राप खगोत्तमः}
{माल्यवन्तं ततो गत्वा मतङ्गस्याश्रमे शुभे}% २६६

\twolineshloka
{अभिगम्य महाभागां शबरीं धर्मचारिणीम्}
{सा तु भागवतश्रेष्ठा दृष्ट्वा तौ रामलक्ष्मणौ}% २६७

\twolineshloka
{प्रत्युद्गम्य नमस्कृत्वा निवेश्य कुशविष्टरे}
{पादप्रक्षालनं कृत्वा वन्यैः पुष्पैः सुगन्धिभिः}% २६८

\twolineshloka
{अर्चयामास भक्त्या वै हर्षनिर्भरमानसा}
{फलानि च सुगन्धीनि मूलानि मधुराणि च}% २६९

\twolineshloka
{निवेदयामास तदा राघवाभ्यां दृढव्रता}
{फलान्यास्वाद्य काकुत्स्थस्तस्यै मुक्तिं ददौ पराम्}% २७०

\twolineshloka
{ततः पम्पासरो गत्वा राघवः शत्रुसूदनः}
{जघान राक्षसं तत्र कबन्धं घोररूपिणम्}% २७१

\twolineshloka
{तं निहत्य महावीर्यो ददाह स्वर्गतश्च सः}
{ततो गोदावरीं गत्वा रामो राजीवलोचनः}% २७२

\twolineshloka
{पप्रच्छ सीतां गङ्गे त्वं किं तां जानासि मे प्रियाम्}
{न शशंस तदा तस्मै सा गङ्गा तमसावृता}% २७३

\twolineshloka
{शशाप राघवः क्रोधाद्रक्ततोया भवेति ताम्}
{ततो भयात्समुद्विग्ना पुरस्कृत्य महामुनीन्}% २७४

\twolineshloka
{कृताञ्जलिपुटा दीना राघवं शरणं गता}
{ततो महर्षयस्सर्वे रामं प्राहुस्सनातनम्}% २७५

\uvacha{ऋषय ऊचुः}

\twolineshloka
{त्वत्पादकमलोद्भूता गङ्गा त्रैलोक्यपावनी}
{त्वमेव हि जगन्नाथ तां शापान्मोक्तुमर्हसि}% २७६

\uvacha{महेश्वर उवाच}

\onelineshloka*
{ततः प्रोवाच धर्मात्मा रामः शरणवत्सलः}

\uvacha{राम उवाच}

\twolineshloka
{शबर्याः स्नानमात्रेण सङ्गता शुभवारिणा}
{मुक्ता भवतु मच्छापाद्गङ्गेयं पापनाशिनी}% २७७

\twolineshloka
{एवमुक्त्वा तु काकुत्स्थः शबरीतीर्थमुत्तमम्}
{गङ्गा गयासमं चक्रे शार्ङ्गकोट्या महाबलः}% २७८

\twolineshloka
{महाभागवतानां च तीर्थं यस्योदकेऽभवत्}
{तच्छरीरं जगद्वन्द्यं भविष्यति न संशयः}% २७९

\twolineshloka
{एवमुक्त्वा तु काकुत्स्थ ऋष्यमूकं गिरिं ययौ}
{ततः पम्पासरस्तीरे वानरेण हनूमता}% २८०

\twolineshloka
{सङ्गतस्तस्य वचनात्सुग्रीवेण समागतः}
{सुग्रीववचनाद्धत्वा वालिनं वानरेश्वरम्}% २८१

\twolineshloka
{सुग्रीवमेव तद्राज्ये रामोसावभ्यषेचयत्}
{स तु सम्प्रेषयामास दिदृक्षुर्जनकात्मजाम्}% २८२

\twolineshloka
{हनुमत्प्रमुखान्वीरान्वानरान्वानराधिपः}
{स लङ्घयित्वा जलधिं हनूमान्मारुतात्मजः}% २८३

\twolineshloka
{प्रविश्य नगरीं लङ्कां दृष्ट्वा सीतां दृढव्रताम्}
{उपवासकृशां दीनां भृशं शोकपरायणाम्}% २८४

\twolineshloka
{मलपङ्केन दिग्धाङ्गीं मलिनाम्बरधारिणीम्}
{निवेदयित्वाऽभिज्ञानं प्रवृत्तिं च निवेद्य ताम्}% २८५

\twolineshloka
{सप्तमन्त्रिसुतांस्तत्र रावणस्य सुतं तथा}
{तोरणस्तम्भमुत्पाट्य निजघान स्वयं कपिः}% २८६

\twolineshloka
{समाश्वास्य च वैदेहीं बभञ्जोपवनं तदा}
{वनपालान्किङ्करांश्च पञ्चसेनाग्रनायकान्}% २८७

\twolineshloka
{रावणस्य सुतेनाथ निगृहीतो यदृच्छया}
{दृष्ट्वा च राक्षसेन्द्रं तु सम्भाषित्वा तथैव च}% २८८

\twolineshloka
{ददाह नगरीं लङ्कां स्वलाङ्गूलाग्निना कपिः}
{तया दत्तमभिज्ञानं गृहीत्वा पुनरागमत्}% २८९

\twolineshloka
{सोऽभिगम्य महातेजा रामं कमललोचनम्}
{न्यवेदयद्वानरेन्द्रो दृष्टा सीतेति तत्वतः}% २९०

\twolineshloka
{सुग्रीवसहितो रामो वानरैर्बहुभिर्वृतः}
{महोदधेस्तटं गत्वा तत्रानीकं न्यवेशयत्}% २९१

\twolineshloka
{रावणस्यानुजो भ्राता विभीषण इतीरितः}
{धर्मात्मा सत्यसन्धश्च महाभागवतोत्तमः}% २९२

\twolineshloka
{ज्ञात्वा समागतं रामं परित्यज्य स्वपूर्वजम्}
{राज्यं सुतांश्च दारांश्च राघवं शरणं ययौ}% २९३

\twolineshloka
{परिगृह्य च तं रामो मारुतेर्वचनात्प्रभुः}
{तस्मै दत्वाऽभयं सौम्यं रक्षो राज्येऽभ्यषेचयत्}% २९४

\twolineshloka
{ततस्समुद्रं काकुत्स्थस्तर्तुकामः प्रपद्य वै}
{सुप्रसन्नजलं तं तु दृष्ट्वा रामो महाबलः}% २९५

\twolineshloka
{शार्ङ्गमादाय बाणौघैः शोषयामास वारिधिम्}
{ततस्तु सरितामीशः काकुत्स्थं करुणानिधिम्}% २९६

\twolineshloka
{प्रपद्य शरणं देवमर्चयामास वारिधिः}
{पुनरापूर्य जलधिं वरुणास्त्रेण राघवः}% २९७

\twolineshloka
{उदधेर्वचनात्सेतुं सागरे मकरालये}
{गिरिभिर्वानरानीतैर्नलः सेतुमकारयत्}% २९८

\twolineshloka
{ततो गत्वा पुरीं लङ्कां सन्निवेश्य महाबलम्}
{सम्यगायोधनं चक्रे वानराणां च रक्षसाम्}% २९९

\twolineshloka
{ततो दशास्यतनयः शक्रजिद्राक्षसो बली}
{बबन्ध नागपाशैश्च तावुभौ रामलक्ष्मणौ}% ३००

\twolineshloka
{वैनतेयः समागत्य तान्यस्त्राणि प्रमोचयत्}
{राक्षसा निहतास्सर्वे वानरैश्च महाबलैः}% ३०१

\twolineshloka
{रावणस्यानुजं वीरं कुम्भकर्णं महाबलम्}
{निजघान रणे रामो बाणैरग्निशिखोपमैः}% ३०२

\twolineshloka
{ब्रह्मास्त्रेणेन्द्रजित्क्रुद्धः पातयामास वानरान्}
{हनूमता समानीतो महौषधि महीधरः}% ३०३

\twolineshloka
{तस्यानीतस्य च स्पर्शात्सर्व एव समुत्थिताः}
{ततो रामानुजो वीरः शक्रजेतारमाहवे}% ३०४

\twolineshloka
{निपातयामास शरैर्वृत्रं वज्रधरो यथा}
{निर्ययावथ पौलस्त्यो योद्धुं रामेण संयुगे}% ३०५

\twolineshloka
{चतुरङ्गबलैः सार्द्धं मन्त्रिभिश्च महाबलः}
{समन्ततोभवद्युद्धं वानराणां च रक्षसाम्}% ३०६

\twolineshloka
{रामरावणयोश्चैव तथा सौमित्रिणा सह}
{शक्त्या निपातयामास लक्ष्मणं राक्षसेश्वरः}% ३०७

\twolineshloka
{ततः क्रुद्धो महातेजा राघवो राक्षसान्तकः}
{जघान राक्षसान्वीराञ्शरैः कालान्तकोपमैः}% ३०८

\twolineshloka
{प्रदीप्तैर्बाणसाहस्रैः कालदण्डोपमैर्भृशम्}
{छादयामास काकुत्स्थो दशग्रीवं च राक्षसम्}% ३०९

\twolineshloka
{स तु निर्भिन्नसर्वाङ्गो राघवास्त्रैर्निशाचरः}
{भयात्प्रदुद्राव रणाल्लङ्कां प्रति निशाचरः}% ३१०

\twolineshloka
{जगद्राममयं पश्यन्निर्वेदाद्गृहमाविशत्}
{ततो हनूमता नीतो महौषधिमहागिरिः}% ३११

\twolineshloka
{तेन रामानुजस्तूर्णं लब्धसंज्ञोऽभवत्तदा}
{दशग्रीवस्ततो होममारेभे जयकाङ्क्षया}% ३१२

\twolineshloka
{ध्वंसितं वानरेन्द्रैस्तदभिचारात्मकं रिपोः}
{पुनर्युद्धाय पौलस्त्यो रामेण सह निर्ययौ}% ३१३

\twolineshloka
{दिव्यस्यन्दनमारुह्य राक्षसैर्बहुभिर्युतः}
{ततः शतमखो दिव्यं रथं हर्यश्वसंयुतम्}% ३१४

\twolineshloka
{राघवाय ससूतं हि प्रेषयामास बुद्धिमान्}
{रथं मातलिना नीतं समारुह्य रघूत्तमः}% ३१५

\twolineshloka
{स्तूयमानं सुरगणैर्युयुधे तेन रक्षसा}
{ततो युद्धमभूद्धोरं रामरावणयोर्महत्}% ३१६

\twolineshloka
{सप्ताह्निकमहोरात्रं शस्त्रास्त्रैरतिभीषणम्}
{विमानस्थाः सुरास्सर्वे ददृशुस्तत्र संयुगम्}% ३१७

\twolineshloka
{दशग्रीवस्य चिच्छेद शिरांसि रघुसत्तमः}
{समुत्थितानि बहुशो वरदानात्कपर्दिनः}% ३१८

\twolineshloka
{ब्राह्ममस्त्रं महारौद्रं वधायास्य दुरात्मनः}
{ससर्ज राघवस्तूर्णं कालाग्निसदृशप्रभम्}% ३१९

\twolineshloka
{तदस्त्रं राघवोत्सृष्टं रावणस्य स्तनान्तरम्}
{विदार्य धरणीं भित्त्वा रसातलतले गतम्}% ३२०

\twolineshloka
{सम्पूज्यमानं भुजगै राघवस्य करं ययौ}
{स गतासुर्महादैत्यः पपात च ममार च}% ३२१

\twolineshloka
{ततो देवगणास्सर्वे हर्षनिर्भरमानसाः}
{ववृषुः पुष्पवर्षाणि महात्मनि जगद्गुरौ}% ३२२

\twolineshloka
{जगुर्गन्धर्वपतयो ननृतुश्चाप्सरोगणाः}
{ववुः पुण्यास्तथा वाताः सुप्रभोऽभूद्दिवाकरः}% ३२३

\twolineshloka
{तुष्टुवुर्मुनयः सिद्धा देवगन्धर्वकिन्नराः}
{लङ्कायां राक्षसश्रेष्ठमभिषिच्य विभीषणम्}% ३२४

\twolineshloka
{कृतकृत्यमिवात्मानं मेने रघुकुलोत्तमः}
{रामस्तत्राब्रवीद्वाक्यमभिषिच्य विभीषणम्}% ३२५

\uvacha{राम उवाच}

\twolineshloka
{यावच्चन्द्रश्च सूर्यश्च यावत्तिष्ठति मेदिनी}
{यावन्ममकथालोके तावद्राज्यं विभीषणे}% ३२६

\twolineshloka
{गत्वा मम पदं दिव्यं योगिगम्यं सनातनम्}
{सपुत्रपौत्रः सगणः सम्प्राप्नुहि महाबलः}% ३२७

\uvacha{ईश्वर उवाच}

\twolineshloka
{एवं दत्वा वरं तस्मै राक्षसाय महाबलः}
{सम्प्राप्य मैथिलीं तत्र परुषं जनसंसदि}% ३२८

\twolineshloka
{उवाच राघवः सीतां गर्हितं वचनं बहु}
{सा तेन गर्हिता साध्वी विवेश चानलं महत्}% ३२९


\threelineshloka
{ततो देवगणास्सर्वे शिवब्रह्मपुरोगमाः}
{दृष्ट्वा तु मातरं वह्नौ प्रविशन्तीं भयातुराः}
{समागम्य रघुश्रेष्ठं सर्वे प्राञ्जलयोऽब्रुवन्}% ३३०

\uvacha{देवा ऊचुः}

\twolineshloka
{रामराम महाबाहो शृणु त्वं चातिविक्रम}
{सीतातिविमला साध्वी तव नीत्यानपायिनी}% ३३१

\twolineshloka
{अत्याज्या तु वृथा सा हि भास्करेण प्रभा यथा}
{सेयं लोकहितार्थाय समुत्पन्ना महीतले}% ३३२

\twolineshloka
{माता सर्वस्य जगतः समस्तजगदाश्रया}
{रावणः कुम्भकर्णश्च भृत्यौ पूर्वपरायणौ}% ३३३

\twolineshloka
{शापात्तौ सनकादीनां समुत्पन्नौ महीतले}
{तयोर्विमुक्त्यै वैदेही गृहीता दण्डके वने}% ३३४

\twolineshloka
{तावुभौ वै वधं प्राप्तौ त्वया राक्षसपुङ्गवौ}
{तौ विमुक्तौ दिवं यातौ पुत्रपौत्रसहानुगौ}% ३३५

\twolineshloka
{त्वं विष्णुस्त्वं परं ब्रह्म योगिध्येयः सनातनः}
{त्वमेव सर्वदेवानामनादिनिधनोऽव्ययः}% ३३६

\twolineshloka
{त्वं हि नारायणः श्रीमान्सीता लक्ष्मीः सनातनी}
{माता सा सर्वलोकानां पिता त्वं परमेश्वर}% ३३७

\twolineshloka
{नित्यैवैष जगन्माता तव नित्यानपायिनी}
{यथा सर्वगतस्त्वं हि तथा चेयं रघूत्तम}% ३३८

\twolineshloka
{तस्माच्छुद्धसमाचारां सीतां साध्वीं दृढव्रताम्}
{गृहाण सौम्य काकुत्स्थ क्षीराब्धेरिव मा चिरम्}% ३३९

\uvacha{ईश्वर उवाच}


\threelineshloka
{एतस्मिन्नन्तरे तत्र लोकसाक्षी स पावकः}
{आदाय सीतां रामाय प्रददौ सुरसन्निधौ}
{अब्रवीत्तत्र काकुत्स्थं वह्निः सर्वशरीरगः}% ३४०

\uvacha{वह्निरुवाच}

\twolineshloka
{इयं शुद्धसमाचारा सीता निष्कल्मषा विभो}
{गृहाण मा चिरं राम सत्यं सत्यं तवाब्रुवन्}% ३४१

\uvacha{ईश्वर उवाच}

\twolineshloka
{ततोऽग्निवचनात्सीतां परिगृह्य रघूद्वहः}
{बभूव रामः संहृष्टः पूज्यमानः सुरोत्तमैः}% ३४२

\twolineshloka
{राक्षसैर्निहता ये तु सङ्ग्रामे वानरोत्तमाः}
{पितामहवरात्तूर्णं जीवमानाः समुत्थिताः}% ३४३

\twolineshloka
{ततस्तु पुष्पकं नाम विमानं सूर्यसन्निभम्}
{भ्रात्रा गृहीतं सङ्ग्रामे कौबेरं राक्षसेश्वरः}% ३४४

\twolineshloka
{तद्राघवाय प्रददौ वस्त्राण्याभरणानि च}
{तेन सम्पूजितः श्रीमान्रामचन्द्रः प्रतापवान्}% ३४५

\twolineshloka
{आरुरोह विमानाग्र्यं वैदेह्या भार्यया सह}
{लक्ष्मणेन च शूरेण भ्रात्रा दशरथात्मजः}% ३४६

\twolineshloka
{ऋक्षवानरसङ्घातैः सुग्रीवेण महात्मना}
{विभीषणेन शूरेण राक्षसैश्च महाबलैः}% ३४७

\twolineshloka
{यथाविमाने वैकुण्ठे नित्यमुक्तैर्महात्मभिः}
{तथा सर्वे समारुह्य ऋक्षवानरराक्षसाः}% ३४८

\twolineshloka
{अयोध्यां प्रस्थितो रामः स्तूयमानः सुरोत्तमैः}
{भरद्वाजाश्रमं गत्वा रामः सत्यपराक्रमः}% ३४९

\twolineshloka
{भरतस्यान्तिके तत्र हनूमन्तं व्यसर्जयत्}
{स निषादालयं गत्वा गुहं दृष्ट्वाऽथ वैष्णवम्}% ३५०

\twolineshloka
{राघवागमनं तस्मै प्राह वानरपुङ्गवः}
{नन्दिग्रामं ततो गत्वा दृष्ट्वा तं राघवानुजम्}% ३५१

\twolineshloka
{न्यवेदयत्तथा तस्मै रामस्यागमनोत्सवम्}
{भरतश्चागतं श्रुत्वा वानरेण रघूत्तमम्}% ३५२

\twolineshloka
{प्रर्हर्षमतुलं लेभे सानुजः ससुहृज्जनः}
{पुनरागत्य काकुत्स्थं हनूमान्मारुतात्मजः}% ३५३

\twolineshloka
{सर्वं शशंस रामाय भरतस्य च वर्तितम्}
{राघवस्तु विमानाग्र्यादवरुह्य सहानुजः}% ३५४

\twolineshloka
{ववन्दे भार्यया सार्द्धं भारद्वाजं तपोनिधिम्}
{स तु सम्पूजयामास काकुत्स्थं सानुजं मुनिः}% ३५५

\twolineshloka
{पक्वान्नैः फलमूलाद्यैर्वस्त्रैराभरणैरपि}
{तेन सम्पूजितस्तत्र प्रणम्य मुनिसत्तमम्}% ३५६

\twolineshloka
{अनुज्ञातः समारुह्य पुष्पकं सानुगस्तदा}
{नन्दिग्रामं ययौ रामः पुष्पकेण सुहृद्वृतः}% ३५७

\twolineshloka
{मन्त्रिभिः पौरमुख्यैश्च सानुजः केकयीसुतः}
{प्रत्युद्ययौ नृपवरैः सबलैः पूर्वजं मुदा}% ३५८

\twolineshloka
{सम्प्राप्य रघुशार्दूलं ववन्दे सानुगैर्वृतः}
{पुष्पकादवरुह्याथ राघवः शत्रुतापनः}% ३५९

\twolineshloka
{भरतं चैव शत्रुघ्नमुपसम्परिषस्वजे}
{पुरोहितं वसिष्ठं च मातृवृद्धांश्च बान्धवान्}% ३६०

\twolineshloka
{प्रणनाम महातेजाः सीतया लक्ष्मणेन च}
{विभीषणं च सुग्रीवं जाम्बवन्तं तथाङ्गदम्}% ३६१

\twolineshloka
{हनुमन्तं सुषेणं च भरतः परिषस्वजे}
{भ्रातृभिः सानुगैस्तत्र मङ्गलस्नानपूर्वकम्}% ३६२

\twolineshloka
{दिव्यमाल्याम्बरधरो दिव्यगन्धानुलेपनः}
{आरुरोह रथं दिव्यं सुमन्त्राधिष्ठितं शुभम्}% ३६३

\twolineshloka
{संस्तूयमानस्त्रिदशैर्वैदेह्या लक्ष्मणेन च}
{भरतश्चैव सुग्रीवः शत्रुघ्नश्च विभीषणः}% ३६४

\twolineshloka
{अङ्गदश्च सुषेणश्च जाम्बवान्मारुतात्मजः}
{नीलो नलश्च सुभगः शरभो गन्धमादनः}% ३६५

\twolineshloka
{अन्ये च कपयः शूरा निषादाधिपतिर्गुहः}
{राक्षसाश्च महावीर्याः पार्थिवेन्द्रा महाबलाः}% ३६६

\twolineshloka
{गजानश्वानथो सम्यगारुह्य बहुशः शुभान्}
{नानामङ्गलवादित्रैः स्तुतिभिः पुष्कलैस्तथा}% ३६७

\twolineshloka
{ऋक्षवानररक्षोभिर्निषादवरसैनिकैः}
{प्रविवेश महातेजाः साकेतं पुरमव्ययम्}% २६८

\twolineshloka
{आलोक्य राजनगरीं पथि राजपुत्रो राजानमेव पितरं परिचिन्तयानः}
{सुग्रीवमारुतिविभीषणपुण्यपादसञ्चारपूतभवनं प्रविवेश रामः}% ३६९

{॥इति श्रीपाद्मे महापुराणे पञ्चपञ्चाशत्साहस्र्यां संहितायामुत्तरखण्डे उमामहेश्वरसंवाद रामस्यायोध्याप्रवेशो नाम द्विचत्वारिंशदधिकद्विशततमोऽध्यायः॥२४२॥}

\sect{त्रिचत्वारिंशदधिक-द्विशततमोऽध्यायः --- विश्वदर्शनम्}

\uvacha{शङ्कर उवाच}

\twolineshloka
{अथ तस्मिन्दिने पुण्ये शुभलग्ने शुभान्विते}
{मङ्गलस्याभिषेकार्थं मङ्गलं चक्रिरे जनाः}% १

\twolineshloka
{वसिष्ठो वामदेवश्च जाबालिरथ कश्यपः}
{मार्कण्डेयश्च मौद्गल्यः पर्वतो नारदस्तथा}% २

\twolineshloka
{एते महर्षयस्तत्र जपहोमपुरस्सरम्}
{अभिषेकं शुभं चक्रुर्मुनयो राजसत्तमम्}% ३

\twolineshloka
{नानारत्नमये दिव्ये हेमपीठे शुभान्विते}
{निवेश्य सीतया सार्द्धं श्रिया इव जनार्दनम्}% ४

\twolineshloka
{सौवर्णकलशैर्दिव्यैर्नानारत्नमयैः शुभैः}
{सर्वतीर्थोदकैः पुण्यैर्माङ्गल्यद्रव्यसंयुतैः}% ५

\twolineshloka
{दूर्वाग्रतुलसीपत्रपुष्पगन्धसमन्वितैः}
{मन्त्रपूतजलैः शुद्धैर्मुनयः संशितव्रताः}% ६

\twolineshloka
{अजपन्वैष्णवान्सूक्तान्चतुर्वेदमयान्शुभान्}
{अभिषेकं शुभं चक्रुः काकुत्स्थं जगतः पतिम्}% ७

\twolineshloka
{तस्मिन्शुभतमे लग्ने देवदुन्दुभयो दिवि}
{विनेदुः पुष्पवर्षाणि ववृषुश्च समन्ततः}% ८

\twolineshloka
{दिव्याम्बरैर्भूषणैश्च दिव्यगन्धानुलेपनैः}
{पुष्पैर्नानाविधैर्दिव्यैर्देव्या सह रघूद्वहः}% ९

\twolineshloka
{अलङ्कृतश्च शुशुभे मुनिभिर्वेदपारगैः}
{छत्रं च चामरं दिव्यं धृतवान्लक्ष्मणस्तदा}% १०

\twolineshloka
{पार्श्वे भरतशत्रुघ्नौ तालवृन्तौ विरेजतुः}
{दर्पणं प्रददौ श्रीमान्राक्षसेन्द्रो विभीषणः}% ११

\twolineshloka
{दधार पूर्णकलशं सुग्रीवो वानरेश्वरः}
{जाम्बवांश्च महातेजाः पुष्पमालां मनोहराम्}% १२

\twolineshloka
{वालिपुत्रस्तु ताम्बूलं सकर्पूरं ददौ हरेः}
{हनुमान्दीपकां दिव्यां सुषेणश्च ध्वजं शुभम्}% १३

\twolineshloka
{परिवार्य महात्मानं मन्त्रिणः समुपासिरे}
{सृष्टिर्जयन्तो विजयः सौराष्ट्रो राष्ट्रवर्द्धनः}% १४

\twolineshloka
{अकोपो धर्मपालश्च सुमन्त्रो मन्त्रिणः स्मृताः}
{राजानश्च नरव्याघ्रा नानाजनपदेश्वराः}% १५

\twolineshloka
{पौराश्च नैगमा वृद्धा राजानं पर्युपासत}
{ऋक्षैश्च वानरेन्द्रैश्च मन्त्रिभिः पृथिवीश्वरैः}% १६

\twolineshloka
{राक्षसैर्द्विजमुख्यैश्च किङ्करैश्च समावृतः}
{परे व्योम्नि यथा लीनो दैवतैः कमलापतिः}% १७

\twolineshloka
{तथा नृपवरः श्रीमान्साकेते शुशुभे तदा}
{इन्दीवरदलश्यामं पद्मपत्रनिभेक्षणम्}% १८

\twolineshloka
{आजानुबाहुं काकुत्स्थं पीतवस्त्रधरं हरिम्}
{कम्बुग्रीवं महोरस्कं विचित्राभरणैर्युतम्}% १९

\twolineshloka
{देव्या सह समासीनमभिषिक्तं रघूत्तमम्}
{विमानस्थाः सुरगणा हर्षनिर्भरमानसाः}% २०

\twolineshloka
{तुष्टुवुर्जयशब्देन गन्धर्वाप्सरसां गणाः}
{अभिषिक्तस्ततो रामो वसिष्ठाद्यैर्महर्षिभिः}% २१

\twolineshloka
{शुशुभे सीतया देव्या नारायण इव श्रिया}
{अतिमर्त्यतयाभीत उपासितुं पदाम्बुजम्}% २२

\threelineshloka
{दृष्ट्वा तुष्टाव हृष्टात्मा शङ्करो हृष्टमागतः}
{कृताञ्जलिपुटो भूत्वा सानन्दो गद्गदाकुलः}
{हर्षयन्सकलान्देवान्मुनीनपि च वानरान्}% २३

\uvacha{महादेव उवाच}

\twolineshloka
{नमो मूलप्रकृतये नित्याय परमात्मने}
{सच्चिदानन्दरूपाय विश्वरूपाय वेधसे}% २४

\twolineshloka
{नमो निरन्तरानन्द कन्दमूलाय विष्णवे}
{जगत्त्रयकृतानन्द मूर्त्तये दिव्यमूर्त्तये}% २५

\twolineshloka
{नमो ब्रह्मेन्द्रपूज्याय शङ्कराभयदाय च}
{नमो विष्णुस्वरूपाय सर्वरूपनमोनमः}% २६

\twolineshloka
{उत्पत्तिस्थितिसंहारकारिणे त्रिगुणात्मने}
{नमोस्तु निर्गतोपाधिस्वरूपाय महात्मने}% २७

\twolineshloka
{अनया विद्यया देव्या सीतयोपाधिकारिणे}
{नमः पुम्प्रकृतिभ्यां च युवाभ्यां जगतां कृते}% २८

\twolineshloka
{जगन्मातापितृभ्यां च जनन्यै राघवाय च}
{नमः प्रपञ्चरूपिण्यै निष्प्रपञ्चस्वरूपिणे}% २९

\twolineshloka
{नमो ध्यानस्वरूपिण्यै योगिध्येयात्ममूर्त्तये}
{परिणामापरीणामरिक्ताभ्यां च नमोनमः}% ३०

\twolineshloka
{कूटस्थबीजरूपिण्यै सीतायै राघवाय च}
{सीता लक्ष्मीर्भवान्विष्णुः सीता गौरी भवान्शिवः}% ३१

\twolineshloka
{सीता स्वयं हि सावित्रि भवान्ब्रह्मा चतुर्मुखः}
{सीता शची भवान्शक्रः सीता स्वाहानलो भवान्}% ३२

\twolineshloka
{सीता संहारिणी देवी यमरूपधरो भवान्}
{सीता हि सर्वसम्पत्तिः कुबेरस्त्वं रघूत्तम}% ३३

\twolineshloka
{सीता देवी च रुद्राणी भवान्रुद्रो महाबलः}
{सीता तु रोहिणी देवी चन्द्रस्त्वं लोकसौख्यदः}% ३४

\twolineshloka
{सीता संज्ञा भवान्सूर्यः सीता रात्रिर्दिवा भवान्}
{सीतादेवी महाकाली महाकालो भवान्सदा}% ३५

\twolineshloka
{स्त्रीलिङ्गेषु त्रिलोकेषु यत्तत्सर्वं हि जानकी}
{पुन्नाम लाञ्छितं यत्तु तत्सर्वं हि भवान्प्रभो}% ३६

\twolineshloka
{सर्वत्र सर्वदेवेश सीता सर्वत्र धारिणी}
{तदात्वमपिचत्रातुन्तच्छक्तिर्विश्वधारिणी}% ३७

\twolineshloka
{तस्मात्कोटिगुणं पुण्यं युवाभ्यां परिचिह्नितम्}
{चिह्नितं शिवशक्तिभ्यां चरितं तव शान्तिदम्}% ३८

\twolineshloka
{आवां राम जगत्पूज्यौ मम पूज्यौ सदा युवाम्}
{त्वन्नामजापिनी गौरी त्वन्मन्त्रजपवानहम्}% ३९

\twolineshloka
{मुमूर्षोर्मणिकर्ण्यां तु अर्द्धोदकनिवासिनः}
{अहं दिशामि ते मन्त्रं तारकं ब्रह्मदायकम्}% ४०

\twolineshloka
{अतस्त्वं जानकीनाथ परब्रह्मासि निश्चितम्}
{त्वन्मायामोहितास्सर्वे न त्वां जानन्ति तत्वतः}% ४१

\uvacha{ईश्वर उवाच}

\twolineshloka
{इत्युक्तः शम्भुना रामः प्रसादप्रवणोऽभवत्}
{दिव्यरूपधरः श्रीमानद्भुताद्भुतदर्शनः}% ४२

\twolineshloka
{तथा तं रूपमालोक्य नरवानरदेवताः}
{न द्रष्टुमपिशक्तास्ते तेजसं महदद्भुतम्}% ४३


\threelineshloka
{भयाद्वै त्रिदशश्रेष्ठाः प्रणेमुश्चातिभक्तितः}
{भीता विज्ञाय रामोऽपि नरवानरदेवताः}
{मायामानुषतां प्राप्य स देवानब्रवीत्पुनः}% ४४

\uvacha{रामचन्द्र उवाच}

\twolineshloka
{शृणुध्वं देवता यो मां प्रत्यहं संस्तुविष्यति}
{स्तवेन शम्भुनोक्तेन देवतुल्यो भवेन्नरः}% ४५

\twolineshloka
{विमुक्तः सर्वपापेभ्यो मत्स्वरूपं समश्नुते}
{रणे जयमवाप्नोति न क्वचित्प्रतिहन्यते}% ४६

\twolineshloka
{भूतवेतालकृत्याभिर्ग्रहैश्चापि न बाध्यते}
{अपुत्रो लभते पुत्रं पतिं विन्दति कन्यका}% ४७

\twolineshloka
{दरिद्रः श्रियमाप्नोति सत्ववाञ्शीलवान्भवेत्}
{आत्मतुल्यबलः श्रीमाञ्जायते नात्र संशयः}% ४८

\twolineshloka
{निर्विघ्नं सर्वकार्येषु सर्वारम्भेषु वै नृणाम्}
{यंयं कामयते मर्त्यः सुदुर्लभमनोरथम्}% ४९


\threelineshloka
{षण्मासात्सिद्धिमाप्नोति स्तवस्यास्य प्रसादतः}
{यत्पुण्यं सर्वतीर्थेषु सर्वयज्ञेषु यत्फलम्}
{तत्फलं कोटिगुणितं स्तवेनानेन लभ्यते}% ५०

\uvacha{ईश्वर उवाच}

\twolineshloka
{इत्युक्त्वा रामचन्द्रोऽसौ विससर्ज महेश्वरम्}
{ब्रह्मादि त्रिदशान्सर्वान्विससर्ज समागतान्}% ५१

\twolineshloka
{अर्चिता मानवाः सर्वे नरवानरदेवताः}
{विसृष्टा रामचन्द्रेण प्रीत्या परमया युताः}% ५२

\twolineshloka
{इत्थं विसृष्टाः खलु ते च सर्वे सुखं तदा जग्मुरतीवहृष्टाः}
{परं पठन्तः स्तवमीश्वरोक्तं रामं स्मरन्तो वरविश्वरूपम्}% ५३

{॥इति श्रीपाद्मे महापुराणे पञ्चपञ्चाशत्साहस्र्यां संहितायामुत्तरखण्डे उमामहेश्वर संवादे विश्वदर्शनं नाम त्रिचत्वारिंशदधिकद्विशततमोऽध्यायः॥२४३॥}

\sect{चतुश्चत्वारिंशदधिक-द्विशततमोऽध्यायः --- श्रीरामचरितकथनम्}

\uvacha{शङ्कर उवाच}

\twolineshloka
{अथ रामस्तु वैदेह्या राज्यभोगान्मनोरमान्}
{बुभुजे वर्षसाहस्रं पालयन्सर्वतोदिशः}% १

\twolineshloka
{अन्तःपुरजनास्सर्वे राक्षसस्य गृहे स्थिताम्}
{गर्हयन्ति स्म वैदेहीं तथा जानपदा जनाः}% २

\twolineshloka
{लोकापवादभीत्या च रामः शत्रुनिवारकः}
{दर्शयन्मानुषं धर्ममन्तर्वत्नीं नृपात्मजाम्}% ३

\twolineshloka
{वाल्मीकेराश्रमे पुण्ये गङ्गातीरे महावने}
{विससर्ज महातेजा गर्भिणीं मुनिसंसदि}% ४

\twolineshloka
{सा भर्तुः परतन्त्रा हि उवास मुनिवेश्मनि}
{अर्चिता मुनिपत्नीभिर्वाल्मीकमुनि रक्षिता}% ५

\twolineshloka
{तत्रैवासूत यमलौ नाम्ना कुशलवौ सुतौ}
{तौ च तत्रैव मुनिना संस्कृतौ च ववर्धतुः}% ६

\twolineshloka
{रामोऽपि भ्रातृभिस्सार्द्धं पालयामास मेदिनीम्}
{यमादिगुणसम्पन्नस्सर्वभोगविवर्जितः}% ७

\twolineshloka
{अर्चयन्सततं विष्णुमनादिनिधनं हरिम्}
{ब्रह्मचर्यपरो नित्यं शशास पृथिवीं नृपः}% ८

\twolineshloka
{शत्रुघ्नो लवणं हत्वा मथुरां देवनिर्मिताम्}
{पालयामास धर्मात्मा पुत्राभ्यां सह राघवः}% ९

\twolineshloka
{गन्धर्वान्भरतो हत्वा सिन्धोरुभयपार्श्वतः}
{स्वात्मजौ स्थापयामास तस्मिन्देशे महाबलौ}% १०

\twolineshloka
{पश्चिमे मद्रदेशे तु मद्रान्हत्वा च लक्ष्मणः}
{स्वसुतौ च महावीर्यौ अभिषिच्य महाबलः}% ११

\twolineshloka
{गत्वा पुनरयोध्यां तु रामपादावुपस्पृशत्}
{ब्राह्मणस्य मृतं बालं कालधर्ममुपागतम्}% १२

\twolineshloka
{जीवयामास काकुत्स्थः शूद्रं हत्वा च तापसम्}
{ततस्तु गौतमीतीरे नैमिषे जनसंसदि}% १३

\twolineshloka
{इयाज वाजिमेधं च राघवः परवीरहा}
{काञ्चनीं जानकीं कृत्वा तया सार्द्धं महाबलः}% १४

\twolineshloka
{चकार यज्ञान्बहुशो राघवः परमार्थवित्}
{अयुतान्यश्वमेधानि वाजपेयानि च प्रभुः}% १५

\twolineshloka
{अग्निष्टोमं विश्वजितं गोमेधं च शतक्रतुम्}
{चकार विविधान्यज्ञान्परिपूर्णसदक्षिणान्}% १६

\twolineshloka
{एतस्मिन्नन्तरे तत्र वाल्मीकिः सुमहातपाः}
{सीतामानीय काकुत्स्थमिदं वचनमब्रवीत्}% १७

\uvacha{वाल्मीकिरुवाच}


\threelineshloka
{अपापां मैथिलीं राम त्यक्तुं नार्हसि सुव्रत}
{इयं तु विरजा साध्वी भास्करस्य प्रभा यथा}
{अनन्या तव काकुत्स्थ कस्मात्त्यक्ता त्वयानघ}% १८

\uvacha{राम उवाच}

\twolineshloka
{अपापां मैथिलीं ब्रह्मन्जानामि वचनात्तव}
{रावणेन हृता साध्वी दण्डके विजने पुरा}% १९

\twolineshloka
{तं हत्वा समरे सीतां शुद्धामग्निमुखागताम्}
{पुनर्यातोस्म्ययोध्यायां सीतामादाय धर्मतः}% २०

\twolineshloka
{लोकापवादः सुमहानभूत्पौरजनेषु च}
{त्यक्ता मया शुभाचारा तद्भयात्तव सन्निधौ}% २१

\twolineshloka
{तस्माल्लोकस्य सन्तुष्ट्यै सीता मम परायणा}
{पार्थिवानां महर्षीणां प्रत्ययं कर्तुमर्हति}% २२

\uvacha{महेश्वर उवाच}

\twolineshloka
{एवमुक्ता तदा सीता मुनिपार्थिवसंसदि}
{चकारप्रत्ययं देवी लोकाश्चर्यकरं सती}% २३

\twolineshloka
{दर्शयंस्तस्य लोकस्य रामस्यानन्यतां सती}
{अब्रवीत्प्राञ्जलिः सीता सर्वेषां जनसंसदि}% २४

\uvacha{सीतोवाच}

\twolineshloka
{यथाऽहं राघवादन्यं मनसापि न चिन्तये}
{तथा मे धरणी देवी विवरन्दातुमर्हति}% २५

\twolineshloka
{यथैव सत्यमुक्तं मे वेद्मि रामात्परं न च}
{तथा स्वपुत्र्यां वैदेह्यां धरणी सहसा इयात्}% २६

\uvacha{महेश्वर उवाच}

\twolineshloka
{ततो रत्नमयं पीठं पृष्ठे धृत्वा खगेश्वरः}
{रसातलात्तदा वीरो विज्ञाय जननीं तदा}% २७

\twolineshloka
{ततस्तु धरणीदेवी हस्ताभ्यां गृह्य मैथिलीम्}
{स्वागतेनाभिनन्द्यैनामासने सन्न्यवेशयत्}% २८

\twolineshloka
{सीतां समागतां दृष्ट्वा दिवि देवगणा भृशम्}
{पुष्पवृष्टिमविच्छिन्नां दिव्यां सीतामवाकिरन्}% २९

\twolineshloka
{सापि दिव्याप्सरोभिस्तु पूज्यमाना सनातनी}
{वैनतेयं समारुह्य तस्मान्मार्गाद्दिवं ययौ}% ३०

\twolineshloka
{दासीगणैः पूर्वभागे संवृता जगदीश्वरी}
{सम्प्राप्य परमं धाम योगिगम्यं सनातनम्}% ३१

\twolineshloka
{रसातलप्रविष्टां तु तां दृष्ट्वा सर्वमानुषाः}
{साधुसाध्विति सीतेयमुच्चैः सर्वे प्रचुक्रुशुः}% ३२

\twolineshloka
{रामः शोकसमाविष्टः सङ्गृह्य तनयावुभौ}
{मुनिभिः पार्थिवेन्द्रैश्च साकेतं प्रविवेश ह}% ३३

\twolineshloka
{अथ कालेन महता मातरः संशितव्रताः}
{कालधर्मं समापन्ना भर्तुः स्वर्गं प्रपेदिरे}% ३४

\twolineshloka
{दशवर्षसहस्राणि दशवर्षशतानि च}
{चकार राज्यं धर्मेण राघवः संशितव्रतः}% ३५

\twolineshloka
{कस्यचित्त्वथकालस्य राघवस्य निवेशनम्}
{कालस्तापसरूपेण सम्प्राप्तो वाक्यमब्रवीत्}% ३६

\uvacha{काल उवाच}

\twolineshloka
{राम राम महाबाहो धात्रा सम्प्रेषितोऽस्म्यहम्}
{यद्ब्रवीमि रघुश्रेष्ठ तच्छृणुष्व महामते}% ३७

\twolineshloka
{द्वन्द्वमेव हि कार्यं स्यादावयोः परिभाषितम्}
{तदन्तरे प्रविष्टोयस्स वद्ध्यो हि भविष्यति}% ३८

\uvacha{महेश्वर उवाच}


\threelineshloka
{तथेति च प्रतिश्रुत्य रामो राजीवलोचनः}
{द्वास्थं कृत्वा तु सौमित्रिं कालो वाक्यमभाषत}
{वैवस्वतोऽब्रवीद्वाक्यं रामं दशरथात्मजम्}% ३९

\uvacha{काल उवाच}

\twolineshloka
{शृणु राम यथावृत्तं समागमनकारणात्}
{दशवर्षसहस्राणि दशवर्षशतानि च}% ४०

\twolineshloka
{वसामि मानुषे लोके हत्वा राक्षसपुङ्गवौ}
{एवमुक्तः सुरगणैरवतीर्णोसि भूतले}% ४१

\twolineshloka
{तदयं समयः प्राप्तः स्वर्लोकं गमितुं त्वया}
{सनाथा हि सुरास्सर्वे भवन्त्वद्य त्वयानघ}% ४२

\uvacha{महेश्वर उवाच}

\twolineshloka
{एवमस्त्विति काकुत्स्थो रामः प्राह महामुनिम्}
{एतस्मिन्नन्तरे तत्र दुर्वासास्तु महातपाः}% ४३

\onelineshloka*
{राजद्वारमुपागम्य लक्ष्मणं वाक्यमब्रवीत्}

\uvacha{दुर्वासा उवाच}
\onelineshloka
{मां निवेदय काकुत्स्थं शीघ्रं गत्वा नृपात्मज}% ४४

\uvacha{महेश्वर उवाच}

\twolineshloka
{तमब्रवील्लक्ष्मणस्तु असान्निध्यमिति द्विज}
{ततः क्रोधसमाविष्टः प्राह तं मुनिसत्तमः}% ४५

\uvacha{दुर्वासा उवाच}

\onelineshloka*
{शापं दास्यामि काकुत्स्थं रामं न यदि दर्शये}

\uvacha{महेश्वर उवाच}

\twolineshloka
{तस्माच्छापभयाद्विप्रं राघवाय न्यवेदयत्}
{तत्रैवान्तर्दधे कालः सर्वभूतभयावहः}% ४६

\twolineshloka
{पूजयामास तं प्राप्तमृषिं दुर्वाससं नृपः}
{अग्रजस्य प्रतिज्ञा तं विज्ञाय रघुसत्तमः}% ४७

\twolineshloka
{तत्याज मानुषं रूपं लक्ष्मणः सरयूजले}
{विसृज्य मानुषं रूपं प्रविवेश स्वकां तनुम्}% ४८

\twolineshloka
{फणासहस्रसंयुक्तः कोटीन्दुसमवर्चसः}
{दिव्यमाल्याम्बरधरो दिव्यगन्धानुलेपनः}% ४९

\twolineshloka
{नागकन्यासहस्रैस्तु संवृतः समलङ्कृतः}
{विमानं दिव्यमारुह्य प्रययौ वैष्णवं पदम्}% ५०

\twolineshloka
{लक्ष्मणस्य गतिं सर्वां विदित्वा रघुसत्तमः}
{स्वयमप्यथ काकुत्स्थः स्वर्गं गन्तुमभीप्सितः}% ५१

\twolineshloka
{अभिषिच्याथ काकुत्स्थः स्वात्मजौ च कुशीलवौ}
{विभज्य रथनागाश्वं सधनं प्रददौ तयोः}% ५२

\twolineshloka
{कुशवत्यां कुशं तं च शरवत्यां लवं तथा}
{स्थापयामास धर्मेण राज्ये स्वे रघुसत्तमः}% ५३

\twolineshloka
{अभिप्रायं तु विज्ञाय रामस्य विदितात्मनः}
{आजग्मुर्वानराः सर्वे राक्षसाः सुमहाबलाः}% ५४

\twolineshloka
{विभीषणोऽथ सुग्रीवो जाम्बवान्मारुतात्मजः}
{नीलो नलः सुषेणश्च निषादाधिपतिर्गुहः}% ५५

\twolineshloka
{अभिषिच्य सुतौ वीरौ शत्रुघ्नश्च महामनाः}
{सर्व एते समाजग्मुरयोध्यां रामपालिताम्}% ५६

\onelineshloka*
{ते प्रणम्य महात्मानमूचुः प्राञ्जलयस्तथा}

\uvacha{वानरप्रभृतय ऊचुः}

\onelineshloka
{स्वर्लोकं गन्तुमुद्युक्तं ज्ञात्वा त्वां रघुसत्तम}% ५७


\threelineshloka
{आगताः स्म वयं सर्वे तवानुगमनं प्रति}
{न शक्ताः स्म क्षणं राम जीवितुं त्वां विना प्रभो}
{तस्मात्त्वया विशालाक्ष गच्छामस्त्रिदशालयम्}% ५८

\uvacha{महेश्वर उवाच}

\twolineshloka
{तैरेवमुक्तः काकुत्स्थो बाढमित्यब्रवीत्ततः}
{अथोवाच महातेजा राक्षसेन्द्रं विभीषणम्}% ५९

\uvacha{राम उवाच}

\onelineshloka*
{राज्यं प्रशास धर्मेण मा प्रतिज्ञां वृथा कृथाः}

\twolineshloka
{यावच्चन्द्रश्च सूर्यश्च यावत्तिष्ठति मेदिनी}
{तावद्रमस्व सुप्रीतो काले मम पदं व्रज}% ६०

\uvacha{महेश्वर उवाच}

\twolineshloka
{इत्युक्त्वाथ स काकुत्स्थः स्वाड्गं विष्णुं सनातनम्}
{श्रीरङ्गशायिनं सौम्यमिक्ष्वाकुकुलदैवतम्}% ६१

\twolineshloka
{सम्प्रीत्या प्रददौ तस्मै रामो राजीवलोचनः}
{हनुमन्तमथोवाच राघवः शत्रुसूदनः}% ६२

\uvacha{राम उवाच}

\twolineshloka
{मत्कथाः प्रचरिष्यन्ति यावल्लोके हरीश्वर}
{तावत्त्वमास मेदिन्यां काले मां व्रज सुव्रत}% ६३

\uvacha{महेश्वर उवाच}

\onelineshloka*
{तमेवमुक्त्वा काकुत्स्थो जाम्बवन्तमथाब्रवीत्}

\uvacha{राम उवाच}
\onelineshloka
{द्वापरे समनुप्राप्ते यदूनामन्वये पुनः}% ६४

\twolineshloka
{भूभारस्य विनाशाय समुत्पत्स्याम्यहं भुवि}
{करिष्ये तत्र सङ्ग्रामं स्वयं भल्लूकसत्तम}% ६५

\uvacha{महेश्वर उवाच}

\twolineshloka
{तमेवमुक्त्वा काकुत्स्थः सर्वांस्तानृक्षवानरान्}
{उवाच वाचा गच्छध्वमिति रामो महाबलः}% ६६

\twolineshloka
{मन्त्रिणो नैगमाश्चैव भरतः कैकयीसुतः}
{राघवस्यानुगमने निश्चितास्ते समाययुः}% ६७

\twolineshloka
{ततः शुक्लाम्बरधरो ब्रह्मचारी ययौ परम्}
{कुशान्गृहीत्वा पाणिभ्यां संसक्तः प्रययौ परम्}% ६८

\twolineshloka
{रामस्य दक्षिणे पार्श्वे पद्महस्ता रमा गता}
{तथैव धरणीदेवी दक्षिणेतरगा तथा}% ६९

\twolineshloka
{वेदाः साङ्गाः पुराणानि सेतिहासानि सर्वतः}
{ॐकारोऽथ वषट्कारः सावित्री लोकपावनी}% ७०

\twolineshloka
{अस्त्रशस्त्राणि च तदा धनुराद्यानि पार्वति}
{अनुजग्मुस्तथा रामं सर्वे पुरुषविग्रहाः}% ७१

\twolineshloka
{भरतश्चैव शत्रुघ्नः सर्वे पुरनिवासिनः}
{सपुत्रदाराः काकुत्स्थमनुजग्मुः सहानुगाः}% ७२

\twolineshloka
{मन्त्रिणो भृत्यवर्गाश्च किङ्करा नैगमास्तथा}
{वानराश्चैव ऋक्षाश्च सुग्रीवसहितास्तदा}% ७३

\twolineshloka
{सपुत्रदाराः काकुत्स्थमन्वगच्छन्महामतिम्}
{पशवः पक्षिणश्चैव सर्वे स्थावरजङ्गमाः}% ७४

\twolineshloka
{अनुजग्मुर्महात्मानं समीपस्था नरोत्तमाः}
{ये च पश्यन्ति काकुत्स्थं स्वपथान्तर्गतं प्रभुम्}% ७५

\twolineshloka
{ते तथानुगता रामं निवर्त्तन्ते न केचन}
{अथ त्रियोजनं गत्वा नदीं पश्चान्मुखीं स्थिताम्}% ७६

\twolineshloka
{सरयूं पुण्यसलिलां प्रविवेश सहानुगः}
{ततः पितामहो ब्रह्मा सर्वदेवगणावृतः}% ७७

\twolineshloka
{तुष्टाव रघुशार्दूलमृषिभिः सार्द्धमक्षरैः}
{अब्रवीत्तत्र काकुत्स्थं प्रविष्टं सरयूजले}% ७८

\uvacha{ब्रह्मोवाच}

\twolineshloka
{आगच्छ विष्णो भद्रं ते दिष्ट्या प्राप्तोऽसि मानद}
{भ्रातृभिस्सहदेवाभैः प्रविशस्व निजां तनुम्}% ७९

\twolineshloka
{वैष्णवीं तां महातेजां देवाकारां सनातनीम्}
{त्वं हि लोकगतिर्देव न त्वां केचित्तु जानते}% ८०

\twolineshloka
{त्वामचिन्त्यं महात्मानमक्षरं सर्वसङ्ग्रहम्}
{यमिच्छसि महातेजस्तां तनुं प्रविशस्व भोः}% ८१

\uvacha{महेश्वर उवाच}

\twolineshloka
{तस्मिन्सूर्यकराकीर्णे पुष्पवृष्टिनिपातिते}
{उत्सृज्य मानुषं रूपं स्वां तनुं प्रविवेश ह}% ८२

\twolineshloka
{अंशाभ्यां शङ्खचक्राभ्यां शत्रुघ्नभरतावुभौ}
{तदा तेन महात्मानौ दिव्यतेजस्समन्वितौ}% ८३

\twolineshloka
{शङ्खचक्रगदाशार्ङ्गपद्महस्तश्चतुर्भुजः}
{दिव्याभरणसम्पन्नो दिव्यगन्धानुलेपनः}% ८४

\twolineshloka
{दिव्यपीताम्बरधरः पद्मपत्रनिभेक्षणः}
{युवा कुमारः सौम्याङ्गः कोमलावयवोज्ज्वलः}% ८५

\twolineshloka
{सुस्निग्धनीलकुटिलकुन्तलः शुभलक्षणः}
{नवदूर्वाङ्कुरः श्यामः पूर्णचन्द्र निभाननः}% ८६

\twolineshloka
{देवीभ्यां सहितः श्रीमान्विमानमधिरुह्य च}
{तस्मिन्सिंहासने दिव्ये मूले कल्पतरोः प्रभुः}% ८७

\twolineshloka
{निषसाद महातेजाः सर्वदेवैरभिष्टुतः}
{राघवानुगता ये च ऋक्षवानरमानुषाः}% ८८

\twolineshloka
{स्पृष्ट्वैव सरयूतोयं सुखेन त्यक्तजीविताः}
{रामप्रसादात्ते सर्वे दिव्यरूपधराः शुभाः}% ८९

\twolineshloka
{दिव्यमाल्याम्बरधरा दिव्यमङ्गलवर्चसः}
{आरुरोह विमानं तदसङ्ख्यास्तत्र देहिनः}% ९०

\twolineshloka
{सर्वैः परिवृतः श्रीमान्रामो राजीवलोचनः}
{पूजितः सुरसिद्धौघैर्मुनिभिस्तु महात्मभिः}% ९१

\twolineshloka
{आययौ शाश्वतं दिव्यमक्षरं स्वपदं विभुः}
{यः पठेद्रामचरितं श्लोकं श्लोकार्धमेव वा}% ९२

\twolineshloka
{शृणुयाद्वा तथा भक्त्या स्मरेद्वा शुभदर्शने}
{कोटिजन्मार्जितात्पापाज्ज्ञानतोऽज्ञानतः कृतात्}% ९३

\twolineshloka
{विमुक्तो वैष्णवं लोकं पुत्रदारसबान्धवैः}
{समाप्नुयाद्योगगम्यमनायासेन वै नरः}% ९४


\onelineshloka
{एतत्ते कथितं देवि रामस्य चरितं महत्}
{धन्योऽस्म्यहं त्वया देवि रामचन्द्रस्य कीर्त्तनात्}
{किमन्यच्छ्रोतुकामासि तद्ब्रवीमि वरानने}% ९५

{॥इति श्रीपाद्मे महापुराणे पञ्चपञ्चाशत्साहस्र्यां संहितायामुत्तरखण्डे उमामहेश्वर संवादे श्रीरामचरितकथनं नाम चतुश्चत्वारिंशदधिकद्विशततमोऽध्यायः॥२४४॥}



    \chapt{नरसिंह-पुराणम्}

\src{नरसिंह-पुराणम्}{अध्यायः २६}{}{}
\vakta{}
\shrota{}
\notes{}
\textlink{https://archive.org/details/narasimha-purana-english/page/171/mode/2up}
\translink{https://archive.org/details/narasimha-purana-english/page/171/mode/2up}

\storymeta

\sect{षडविंशोऽध्यायः --- सूर्यवंशानुचरितम्}

\addtocounter{shlokacount}{8}

\uvacha{सूत उवाच}

दीर्घबाहोरजोऽजाद्दशरथः। तस्य गृहे रावणविनाशार्थं साक्षान्नारायणोऽवतीर्णो रामः॥९॥

स तु पितृवचनाद भ्रातृभार्यासहितो दण्डकारण्यं प्राप्य तपश्चचार।

वने रावणापहतभार्यो भ्रात्रा सह दुःखितोऽनेककोटिवानरनायक सुग्रीवसहायो मदोदधौ

सेतुं निबध्य तैर्गत्वा लङ्कां रावणं देवकण्टकं सबान्धवं हत्वा सीतामादाय पुनरयोध्यां

प्राप्य भरताभिषिक्तो विभीषणाय लङ्काराज्यं विमानं वा दत्त्वा तं प्रेषयामास।

स तु परमेश्वरो विमानस्थो विभीषणेन नीयमानो लङ्कायामपि राक्षसपुर्यां वस्तुमनिच्छन् पुण्यारण्यं तत्र स्थापितवान्॥१०॥

तन्निरीक्ष्य तत्रैव महाहिभोगशयने भगवान् शेते। सोऽपि विभीषणस्ततस्तद्विमानं नेतुमसमर्थः, तद्वचनात् स्वां पुरीं जगाम॥११॥

नारायणसन्निधानान्महद्वैष्णवं क्षेत्रमभवदद्यापि दृश्यते। रामाल्लवो लवात्पद्यः पद्मादृतुपर्ण ऋतुपर्णादस्त्रपाणिः।

अस्त्रपाणेः शुद्धोदनः शुद्धोदनाद्वुधः। बुधाद्वंशो निवर्तते॥१२॥

\twolineshloka
{एते महीपा रविवंशजास्तव प्राधान्यतस्ते कथिता महाबलाः}
{पुरातनैर्यैर्वसुधा प्रपालिता यज्ञक्रियाभिश्च दिवौकसैर्नृपैः} % ॥१३॥

॥इति श्रीनरसिंहपुराणे सूर्यवंशानुचरितं नाम षडविंशोऽध्यायः ॥२६॥


\src{नरसिंह-पुराणम्}{अध्यायः ४७--५२}{}{}
\vakta{}
\shrota{}
\notes{Concise retelling of all the Kandas of Ramayana.}
\textlink{https://archive.org/details/narasimha-purana-english/page/171/mode/2up}
\translink{https://archive.org/details/narasimha-purana-english/page/171/mode/2up}

\storymeta


\sect{सप्तचत्वारिंशोऽध्यायः --- बाल-काण्डः}

\uvacha{मार्कण्डेय उवाच}

\twolineshloka
{श्रुणु राजन् प्रवक्ष्यामि प्रादुर्भावं हरेः शुभम्}
{निहतो रावणो येन सगणो देवकण्टकः} %॥१॥

\twolineshloka
{ब्रह्मणो मानसः पुत्रः पुनस्त्योऽभून्महामुनिः}
{तस्य वै विश्रवा नाम पुत्रोऽभूत्तस्य राक्षसः} %॥२॥

\twolineshloka
{तस्माज्जातो महावीरो रावणो लोकरावणः}
{तपसा महता युक्तः स तु लोकानुपाद्रवत्} %॥३॥

\twolineshloka
{सेन्द्रा देवा जितास्तेन गन्धर्वाः किन्नरास्तथा}
{यक्षाश्च दानवाश्चैव तेन राजन् विनिर्जिताः} %॥४॥

\twolineshloka
{स्त्रियश्चैव सुरुपिण्यो हतास्तेन दुरात्मना}
{देवादीनां नृपश्रेष्ठ रत्नानि विविधानि च} %॥५॥

\twolineshloka
{रणे कुबेरं निर्जित्य रावणो बलदर्पितः}
{तत्पुरीं जगृहे लङ्कां विमानं चापि पुष्पकम्} %॥६॥

\twolineshloka
{तस्यां पुर्यां दशग्रीवो रक्षसामधिपोऽभवत्}
{पुत्राश्च बहवस्तस्य बभूवुरमितौजसः} %॥७॥

\twolineshloka
{राक्षसाश्च तमाश्रित्य महाबलपराक्रमाः}
{अनेककोटयो राजन् लङ्कायां निवसन्ति ये} %॥८॥

\twolineshloka
{देवान् पितृन मनुष्यांश्च विद्याधरगणानपि}
{यक्षांश्चैव ततः सर्वे घातयन्ति दिवाशिनम्} %॥९॥

\twolineshloka
{सन्त्रस्तं तद्भयादेव जगदासीच्चराचरम्}
{दुःखाभिभूतमत्त्यर्थं सम्बभूव नराधिप} %॥१०॥

\twolineshloka
{एतस्मिन्ने व काले तु देवाः सेन्द्रा महर्षयः}
{सिद्धा विद्याधराश्चैव गन्धर्वाः किन्नरास्तथा} %॥११॥

\twolineshloka
{गुह्यका भुजगा यक्षा ये चान्ये स्वर्गवासिनः}
{ब्रह्माणमग्रतः कृत्वा शङ्करं च नराधिप} %॥१२॥

\twolineshloka
{ते ययुर्हतविक्रान्ताः क्षीराब्धेस्तटमुत्तमम्}
{तत्राराध्य हरिं देवतास्तस्थुः प्राञ्जलयस्तदा} %॥१३॥

\twolineshloka
{ब्रह्मा च विष्णुमाराध्य गन्धपुष्पादिभिः शुभैः}
{प्राञ्जलिः प्रणतो भूत्वा वासुदेवमथास्तुवत्} %॥१४॥

\uvacha{ब्रह्मोवाच}

\twolineshloka
{नमः क्षीराब्धिवासाय नागपर्यङ्कशायिने}
{नमः श्रीकरसंस्पृष्टदिव्यपादाय विष्णवे} %॥१५॥

\twolineshloka
{नमस्ते योगनिद्राय योगान्तर्भाविताय च}
{तार्क्ष्यासनाय देवाय गोविन्दाय नमो नमः} %॥१६॥

\twolineshloka
{नमः क्षीराब्धिकल्लोलस्पृष्टमात्राय शार्ङ्गिणे}
{नमोऽरविन्दपादाय पद्मनाभाय विष्णवे} %॥१७॥

\twolineshloka
{भक्तार्चितसुपादाय नमो योगाप्रियाय वै}
{शुभाङ्गाय सुनेत्राय माधवाय नमो नमः} %॥१८॥

\twolineshloka
{सुकेशाय सुनेत्राय सुललाटाय चक्रिणे}
{सुवक्त्राय सुकर्णाय श्रीधराय नमो नमः} %॥१९॥

\twolineshloka
{सुवक्षसे सुनाभाय पद्मनाभाय वै नमः}
{सुभ्रुवे चारुदेहाय चारुदन्ताय शार्ङ्गिणे} %॥२०॥

\twolineshloka
{चारुजङ्घाय दिव्याय केशवाय नमो नमः}
{सुनखाय सुशान्ताय सुविद्याय गदाभृते} %॥२१॥

\twolineshloka
{धर्माप्रियाय देवाय वामनाय नमो नमः}
{असुरघ्नाय चोग्राय रक्षोघ्नाय नमो नमः} %॥२२॥

\twolineshloka
{देवानामार्तिनाशाय भीमर्ककृते नमः}
{नमस्ते लोकनाथाय रावणान्तकृते नमः} %॥२३॥

\uvacha{मार्कण्डेय उवाच}

\twolineshloka
{इति स्तुतो हषीकेशस्तुतोष परमेष्ठिना}
{स्वरुपं दर्शयित्वा तु पितामहमुवाच ह} %॥२४॥

\twolineshloka
{किमर्थं तु सुरैः सार्धमागतस्त्वं पितामह}
{यत्कार्य ब्रूहि मे ब्रह्मन् यदर्थं संस्तुतस्त्वया} %॥२५॥

\twolineshloka
{इत्युक्तो देवदेवेन विष्णुना प्रभविष्णुना}
{सर्वदेवगणैः सार्धं ब्रह्मा प्राह जनार्दनम्} %॥२६॥

\uvacha{ब्रह्मोवाच}

\twolineshloka
{नाशितं तु जगत्सर्वं रावणेन दुरात्मना}
{सेन्द्राः पराजितास्तेन बहुशो रक्षसा विभो} %॥२७॥

\twolineshloka
{राक्षसैर्भक्षिता मर्त्या यज्ञाश्चापि विदूषिताः}
{देवकन्या हतास्तेन बलाच्छतसहस्त्रशः} %॥२८॥

\twolineshloka
{त्वामृते पुण्डरीकाक्ष रावणस्य वधं प्रति}
{न समर्था यतो देवास्त्वमतस्तद्वधं कुरु} %॥२९॥

\twolineshloka
{इत्युक्तो ब्रह्मणा विष्णुर्ब्रह्माणमिदमब्रवीत्}
{श्रृणुष्वावहितो ब्रह्मन् यद्वदामि हितं वचः} %॥३०॥

\twolineshloka
{सूर्यवंशोद्भवः श्रीमान् राजाऽऽसीद्भुवि वीर्यवान्}
{नाम्ना दशरथख्यातस्तस्य पुत्रो भवाम्यहम्} %॥३१॥

\twolineshloka
{रावणस्य वधार्थाय चतुर्धांशेन सत्तम}
{स्वांशैर्वानररुपेण सकला देवतागणाः} %॥३२॥

\twolineshloka
{वतार्यन्तां विश्वकर्तः स्यादेवं रावणक्षयः}
{इत्युक्तो देवदेवेन ब्रह्मा लोकपितामहः} %॥३३॥

\twolineshloka
{देवाश्च ते प्रणम्याथ मेरुपृष्ठं तदा ययुः}
{स्वांशैर्वानररुपेण अवतेरुश्च भूतले} %॥३४॥

\twolineshloka
{अथापुत्रो दशरथो मुनिभिर्वेदपारगैः}
{इष्टिं तु कारयामास पुत्रप्राप्तिकरी नृपः} %॥३५॥

\twolineshloka
{ततः सौवर्णपात्रस्थं हविरादाय पायसम्}
{वह्निः कुण्डात् समुत्तस्थौ नूनं देवेन नोदितः} %॥३६॥

\twolineshloka
{आदाय मुनयो मन्त्राच्चक्रुः पिण्डद्वयं शुभम्}
{दत्ते कौशल्यकैकेय्योर्द्वे पिण्डे मन्त्रमन्त्रिते} %॥३७॥

\twolineshloka
{ते पिण्डप्राशने काले सुमित्राया महामते}
{पिण्डाभ्यामल्पमल्पं तु सुभागिन्याः प्रयच्छतः} %॥३८॥

\twolineshloka
{ततस्ताः प्राशयामासू राजपत्न्यो यथाविधि}
{पिण्डान् देवकृतान् प्राश्य प्रापुर्गर्भाननिन्दितान्} %॥३९॥

\twolineshloka
{एवं विष्णुर्दशरथाज्जातस्तत्पत्निषु त्रिषु}
{स्वांशैर्लोकहितायैव चतुर्धा जगतीपते} %॥४०॥

\twolineshloka
{रामश्च लक्ष्मणश्चैव भरतः शत्रुघ्न एव च}
{जातकर्मादिकं प्राप्य संस्कारं मुनिसंस्कृतम्} %॥४१॥

\twolineshloka
{मन्त्रपिण्डवशाद्योगं प्राप्य चेरुर्यथार्भकाः}
{रामश्च लक्ष्मणश्चैव सह नित्यं विचेरतुः} %॥४२॥

\twolineshloka
{जन्मादिकृतसंस्कारौ पितुः प्रीतिकरौ नृप}
{ववृधाते महावीर्यौ श्रुतिशब्दातिलक्षणौ} %॥४३॥

\twolineshloka
{भरतः कैकयो राजन् भ्रात्रा सह गृहेऽवसत्}
{वेदशास्त्राणि बुबुधे शस्त्रशास्त्रं नृपोत्तम} %॥४४॥

\twolineshloka
{एतस्मिन्नेव काले तु विश्वामित्रो महातपाः}
{यागेन यष्टुमारेभे विधिना मधुसूदनम्} %॥४५॥

\twolineshloka
{स तु विघ्नेन यागोऽभूद्राक्षसैर्बहुशः पुरा}
{नेतुं स यागरक्षार्थं सम्प्रातो रामलक्ष्मणौ} %॥४६॥

\twolineshloka
{विश्वामित्रो नृपश्रेष्ठ तत्पितुर्मन्दिरं शुभम्}
{दशरथस्तु तं दृष्ट्वा प्रत्युत्थाय महामतिः} %॥४७॥

\twolineshloka
{अर्घ्यपाद्यादि विधिना विश्वामित्रमपूजयत्}
{स पूजितो मुनिः प्राह राजानं राजसन्निधौ} %॥४८॥

\twolineshloka
{श्रृणु राजन् दशरथ यदर्थमहमागतः}
{तत्कार्यं नृपशार्दूल कथयामि तवाग्रतः} %॥४९॥

\twolineshloka
{राक्षसैर्नाशितो यागो बहुशो मे दुरासदैः}
{यज्ञस्य रक्षणार्थं मे देहि त्वं रामक्ष्मणौ} %॥५०॥

\twolineshloka
{राजा दशरथः श्रुत्वा विश्वामित्रवचो नृप}
{विषण्णवदनो भूत्वा विश्वामित्रमुवाच ह} %॥५१॥

\twolineshloka
{बालाभ्यां मम पुत्राभ्यां किं ते कार्यं भविष्यति}
{अहं त्वया सहागत्य शक्त्या रक्षामि ते मखम्} %॥५२॥

\twolineshloka
{राज्ञस्तु वचनं श्रुत्वा राजानं मुनिरब्रवीत्}
{रामोऽपि शक्नुते नूनं सर्वान्नशयितुं नृप} %॥५३॥

\twolineshloka
{रामेणैव हि ते शक्या न त्वया राक्षसा नृप}
{अतो मे देहि रामं च न चिन्तां कर्तुमर्हसि} %॥५४॥

\twolineshloka
{इत्युक्तो मुनिना तेन विश्वामित्रेण धीमता}
{तूष्णीं स्थित्वा क्षणं राजा मुनिवर्यमुवाच ह} %॥५५॥

\twolineshloka
{यद्ववीमि मुनिश्रेष्ठ प्रसन्नस्त्वं निबोध मे}
{राजीवलोचनं राममहं दास्ये सहानुजम्} %॥५६॥

\twolineshloka
{किं त्वस्य जननी ब्रह्मन् अदृष्टैनं मरिष्यति}
{अतोऽहं चतुरङ्गेण बलेन सहितो मुने} %॥५७॥

\twolineshloka
{आगत्य राक्षसान् हन्मीत्येबं मे मनसि स्थितम्}
{विश्वामित्रः पुनः प्राह राजानममितौजसम्} %॥५८॥

\twolineshloka
{नाज्ञो रामो नृपश्रेष्ठ स सर्वज्ञः समः क्षमः}
{शेषनारायणावेतौ तव पुत्रौ न संशयः} %॥५९॥

\twolineshloka
{दुष्टानां निग्रहार्थाय शिष्टानां पालनाय च}
{अवतीर्णो न सन्देहो गृहे तव नराधिप} %॥६०॥

\twolineshloka
{न मात्रा न त्वया राजन् शोकः कार्योऽत्र चाण्वपि}
{निः क्षेपे च महाराज अर्पयिष्यामि ते सुतौ} %॥६१॥

\twolineshloka
{इत्युक्तो दशरथस्तेन विश्वामित्रेण धीमता}
{तच्छापभीतो मनसा नीयतामित्यभाषत्} %॥६२॥

\twolineshloka
{कृच्छ्रात्पित्रा विनिर्मुक्तं राममादाय सानुजम्}
{ततः सिद्धाश्रमं राजन् सम्प्रतस्थे स कौशिकः} %॥६३॥

\twolineshloka
{तं प्रस्थितमथालोक्य राजा दशरथस्तदा}
{अनुव्रज्याब्रवीदेतद् वचो दशरथस्तदा} %॥६४॥

\twolineshloka
{अपुत्रोऽहं पुरा ब्रह्मन् बहुभिः काम्यकर्मभिः}
{मुनिप्रसादादधुना पुत्रवानस्मि सत्तम} %॥६५॥

\twolineshloka
{मनसा तद्वियोगं तु न शक्ष्यामि विशेषतः}
{त्वमेव जानासि मुने नीत्वा शीघ्रं प्रयच्छ मे} %॥६६॥

\twolineshloka
{इत्येवमुक्तो राजानं विश्वामित्रोऽब्रवीत्पुनः}
{समाप्तयज्ञश्च पुनर्नेष्ये रामं च लक्ष्मणम्} %॥६७॥

\twolineshloka
{सत्यपूर्वं तु दास्यामि न चिन्तां कर्तुमर्हसि}
{इत्युक्तः प्रेषयामास रामं लक्ष्मणसंयुतम्} %॥६८॥

\twolineshloka
{अनिच्छन्नपि राजासौ मुनिशापभयान्नृपः}
{विश्वामित्रस्तु तौ गृह्य अयोध्याया ययौ शनैः} %॥६९॥

\twolineshloka
{सरय्वास्तीरमासाद्य गच्छन्नेव स कौशिकः}
{तयोः प्रीत्या स राजेन्द्र द्वे विद्ये प्रथमं ददौ} %॥७०॥

\twolineshloka
{बलामतिबलां चैव समन्त्रे च ससङ्ग्रहे}
{क्षुत्पिपासापनयने पुनश्चैव महामतिः} %॥७१॥

\twolineshloka
{अस्त्रग्राममशेषं तु शिक्षयित्वा तु तौ तदा}
{आश्रमाणि च दिव्यानि मुनीनां भावितात्मनाम्} %॥७२॥

\twolineshloka
{दर्शयित्वा उषित्वा च पुण्यस्थानेषु सत्तमः}
{गङ्गामुत्तीर्य शोणस्य तीरमासाद्य पश्चिमम्} %॥७३॥

\twolineshloka
{मुनिधार्मिकसिद्धांश्च पश्यन्तौ रामलक्ष्मणौ}
{ऋषिभ्यश्च वरान् प्राप्य तेन नीतौ नृपात्मजौ} %॥७४॥

\twolineshloka
{ताटकाया वनं घोरं मृत्योर्मुखमिवापरम्}
{गते तत्र नृपश्रेष्ठ विश्वामित्रो महातपाः} %॥७५॥

\twolineshloka
{राममक्लिष्टकर्माणमिदं वचनमब्रवीत्}
{राम राम महाबाहो ताटका नाम राक्षसी} %॥७६॥

\twolineshloka
{रावणस्य नियोगेन वसत्यस्मिन् महावने}
{तया मनुष्या बहवो मुनिपुत्रा मृगास्तथा} %॥७७॥

\twolineshloka
{निहता भक्षिताश्चैव तस्मात्तां वध सत्तम}
{इत्येवमुक्तो मुनिना रामस्तं मुनिमब्रवीत्} %॥७८॥

\twolineshloka
{कथं हि स्त्रीवधं कुर्यामहमद्य महामुने}
{स्त्रीवधे तु महापापं प्रवदन्ति मनीषिणः} %॥७९॥

\twolineshloka
{इति रामवचः श्रुत्वा विश्वामित्र उवाच तम्}
{तस्यास्तु निधनाद्राम जनाः सर्वे निराकुलाः} %॥८०॥

\twolineshloka
{भवन्ति सततं तस्मात् तस्याः पुण्यप्रदो वधः}
{इत्येवं वादिनि मुनौ विश्वामित्रे निशाचरी} %॥८१॥

\twolineshloka
{आगता सुमहाघोरा ताटका विवृतानना}
{मुनिना प्रेरितो रामस्तां दृष्ट्वा विवृताननाम्} %॥८२॥

\twolineshloka
{उद्यतैकभुजयष्टिमायतीं श्रोणिलम्बिपुरुषान्त्रमेखलाम्}
{तां विलोक्य वनितावधे घृणां पत्रिणा सह मुमोच राघवः} %॥८३॥

\twolineshloka
{शरं सन्धाय वेगेन तेन तस्या उरः स्थलम्}
{विपाटितं द्विधा राजन् सा पपात ममार च} %॥८४॥

\twolineshloka
{घातयित्वा तु तामेवं तावानीय मुनिस्तु तौ}
{प्रापयामास तं तत्र नानाऋषिनिषेवितम्} %॥८५॥

\twolineshloka
{नानाद्रुमलताकीर्णं नानापुष्पोपशोभितम्}
{नानानिर्झरतोयाढ्यं विन्ध्यशैलान्तरस्थितम्} %॥८६॥

\twolineshloka
{शकमूलफलोपेतं दिव्यं सिद्धाश्रमं स्वकम्}
{रक्षार्थं तावुभौ स्थाप्य शिक्षयित्वा विशेषतः} %॥८७॥

\twolineshloka
{ततश्चारब्धवान् यागं विश्वामित्रो महातपाः}
{दीक्षां प्रविष्टे च मुनौ विश्वामित्रे महात्मनि} %॥८८॥

\twolineshloka
{यज्ञे तु वितते तत्र कर्म कुर्वन्ति ऋत्विजः}
{मारीचश्च सुबाहुश्च बहवश्चान्यराक्षसाः} %॥८९॥

\twolineshloka
{आगता यागनाशाय रावणेन नियोजिताः}
{तानागतान् स विज्ञाय रामः कमललोचनः} %॥९०॥

\twolineshloka
{शरेण पातयामास सुबाहुं धरणीतले}
{असृक्प्रवाहं वर्षन्तं मारीचं भल्लकेन तु} %॥९१॥

\twolineshloka
{प्रताङ्य नीतवानब्धिं यथा पर्णं तु वायुना}
{शेषांस्तु हतवान् रामो लक्ष्मणश्च निशाचरान्} %॥९२॥

\twolineshloka
{रामेण रक्षितमखो विश्वामित्रो महायशाः}
{समाप्य यागं विधिवत् पूजयामास ऋत्विजान्} %॥९३॥

\twolineshloka
{सदस्यानपि सम्पूज्य यथार्हं च ह्यरिन्दम}
{रामं च लक्ष्मणं चैव पूजयामास भक्तितः} %॥९४॥

\twolineshloka
{ततो देवगणस्तुष्टो यज्ञभागेन सत्तम}
{ववर्ष पुष्पवर्षं तु रामदेवस्य मूर्धनि} %॥९५॥

\twolineshloka
{निवार्य राक्षसभयं कारयित्वा तु तन्मखम्}
{श्रुत्वा नानाकथाः पुण्या रामो भ्रातृसमन्वितः} %॥९६॥

\twolineshloka
{तेन नीतो विनीतात्मा अहल्या यत्र तिष्ठति}
{व्यभिचारान्महेन्द्रेण भर्त्रा शप्ता हि सा पुरा} %॥९७॥

\twolineshloka
{पाषाणभूता राजेन्द्र तस्य रामस्य दर्शनात्}
{अहल्या मुक्तशापा च जगाम गौतमं प्रति} %॥९८॥

\twolineshloka
{विश्वामित्रस्ततस्तत्र चिन्तयामास वै क्षणम्}
{कृतदारो मया नेयो रामः कमललोचनः} %॥९९॥

\twolineshloka
{इति सञ्चिन्त्य तौ गृह्य विश्वामित्रो महातपाः}
{शिष्यैः परिवृतोऽनेकैर्जगाम मिथिलां प्रति} %॥१००॥

\twolineshloka
{नानादेशादथायाता जनकस्य निवेशनम्}
{राजपुत्रा महावीर्याः पूर्वं सीताभिकाङ्क्षिणः} %॥१०१॥

\twolineshloka
{तान् दृष्ट्वा पूजयित्वा तु जनकश्च यथार्हतः}
{यत्सीतायाः समुत्पन्नं धनुर्माहेश्वरं महत्} %॥१०२॥

\twolineshloka
{अर्चितं गन्धमालाभी रम्यशोभासमन्विते}
{रङ्गे महति विस्तीर्णे स्थापयामास तद्धनुः} %॥१०३॥

\twolineshloka
{उवाच च नृपान् सर्वांस्तदोच्चैर्जनको नृपः}
{आकर्षणादिदं येन धनुर्भग्नं नृपात्मजाः} %॥१०४॥

\twolineshloka
{तस्येयं धर्मतो भार्या सीता सर्वाङ्गशोभना}
{इत्येवं श्राविते तेन जनकेन महात्मना} %॥१०५॥

\twolineshloka
{क्रमादादाय ते तत्तु सज्यीकर्तुमथाभवन्}
{धनुषा ताडिताः सर्वे क्रमात्तेन महीपते} %॥१०६॥

\twolineshloka
{विधूय पतिता राजन् विलजास्तत्र पार्थिवाः}
{तेषु भग्नेषु जनकस्तद्धनुस्त्र्यम्बकं नृप} %॥१०७॥

\twolineshloka
{संस्थाप्य स्थितवान् वीरो रामागमनकाङ्क्षया}
{विश्वामित्रस्ततः प्राप्तो मिथिलाधिपतेर्गृहम्} %॥१०८॥

\twolineshloka
{जनकोऽपि च तं दृष्टवा विश्वामित्रं गृहागतम्}
{रामलक्ष्मणसंयुक्तं शिष्यैश्चाभिगतं तदा} %॥१०९॥

\twolineshloka
{तं पूजयित्वा विधिवत्प्राज्ञं विप्रानुयायिनम्}
{रामं रघुपतिं चापि लावण्यादिगुणैर्युतम्} %॥११०॥

\twolineshloka
{शीलाचारगुणोपेतं लक्ष्मणं च महामतिम्}
{पूजयित्वा यथान्यायं जनकः प्रीतमानसः} %॥१११॥

\twolineshloka
{हेमपीठे सुखासीनं शिष्यैः पूर्वापरैर्वृतम्}
{विश्वामित्रमुवाचाथ किं कर्तव्यं मयेति सः} %॥११२॥

\uvacha{मार्कण्डेय उवाच}

\twolineshloka
{इति श्रुत्वा वचस्तस्य मुनिः प्राह महीपतिम्}
{एष रामो महाराज विष्णुः साक्षान्महीपतिः} %॥११३॥

\twolineshloka
{रक्षार्थं विष्टपानां तु जातो दशरथात्मजः}
{अस्मै सीतां प्रयच्छ त्वं देवकन्यामिव स्थिताम्} %॥११४॥

\twolineshloka
{अस्या विवाहे राजेन्द्र धनुर्भङ्गमुदीरितम्}
{तदानय भवधनुरर्चयस्व जनाधिप} %॥११५॥

\twolineshloka
{तथेत्युक्त्वा च राजा हि भवचापं तदद्भुतम्}
{अनेक भूभुजां भङ्गि स्थापयामास पूर्ववत्} %॥११६॥

\twolineshloka
{ततो दशरथसुतो विश्वामित्रेण चोदितः}
{तेषां मध्यात्समुत्थाय रामः कमललोचनः} %॥११७॥

\twolineshloka
{प्रणम्य विप्रान् देवांश्च धनुरादाय तत्तदा}
{सज्यं कृत्वा महाबाहुर्ज्याघोषमकरोत्तदा} %॥११८॥

\twolineshloka
{आकृष्यमाणं तु बलात्तेन भग्नं महद्धनुः}
{सीता च मालामादाय शुभां रामस्य मूर्धनि} %॥११९॥

\twolineshloka
{क्षिप्त्वा संवरयामास सर्वक्षत्रियसन्निधौ}
{ततस्ते क्षत्रियाः क्रुद्धा राममासाद्य सर्वतः} %॥१२०॥

\twolineshloka
{मुमुचुः शरजालानि गर्जयन्तो महाबलाः}
{तान्निरीक्ष्य ततो रामो धनुरादाय वेगवान्} %॥१२१॥

\twolineshloka
{ज्याघोषतलघोषेण कम्पयामास तान्नृपान्}
{चिच्छेद शरजालानि तेषां स्वास्त्रै रथांस्ततः} %॥१२२॥

\twolineshloka
{धनूंषि च पताकाश्च रामश्चिच्छेद लीलया}
{सन्नह्य स्वबलं सर्वं मिथिलाधिपतिस्ततः} %॥१२३॥

\twolineshloka
{जामातरं रणे रक्षन् पार्ष्णिग्राहो बभूव ह}
{लक्ष्मणश्च महावीरो विद्राव्य युधि तान्नृपान्} %॥१२४॥

\twolineshloka
{हस्त्यश्वाञ्जगृहे तेषां स्यन्दनानि बहूनि च}
{वाहनानि परित्यज्य पलायनपरान्नृपान्} %॥१२५॥

\twolineshloka
{तान्निहन्तुं च धावत्स पृष्ठतो लक्ष्मणस्तदा}
{मिथिलाधिपतिस्तं च वारयामास कौशिकः} %॥१२६॥

\twolineshloka
{जितसेनं महावीरं रामं भ्रात्रा समन्वितम्}
{आदाय प्रविवेशाथ जनकः स्वगृहं शुभम्} %॥१२७॥

\twolineshloka
{दूतं च प्रेषयामास तदा दशरथाय सः}
{श्रुत्वा दूतमुखात् सर्वं विदितार्थः स पार्थिवः} %॥१२८॥

\twolineshloka
{सभार्यः ससुतः श्रीमान् हस्त्यश्वरथवाहनः}
{मिथिलामाजगामाशु स्वबलेन समन्वितः} %॥१२९॥

\twolineshloka
{जनकोऽप्यस्य सत्कारं कृत्वा स्वां च सुतां ततः}
{विधिवत्कृतशुल्कां तां ददौ रामाय पार्थिव} %॥१३०॥

\twolineshloka
{अपराश्च सुतास्तिस्त्रो रुपवत्यः स्वलडकृताः}
{त्रिभ्यस्तु लक्ष्मणादिभ्यः स्वकन्या विधिवद्ददौ} %॥१३१॥

\twolineshloka
{एवं कृतविवाहोऽसौ रामः कमललोचनः}
{भ्रातृभिर्मातृभिः सार्धं पित्रा बलवता सह} %॥१३२॥

\threelineshloka
{दिनानि कतिचित्तत्र स्थितो विविधभोजनैः}
{ततोऽयोध्यापुरीं गन्तुमुत्सुकं ससुतं नृपम्}
{दृष्ट्वा दशरथं राजा सीतायाः प्रददौ वसु} %॥१३३॥

\fourlineindentedshloka
{रत्नानि दिव्यानि बहूनि दत्त्वा}
{रामाय वस्त्राण्यतिशोभनानि}
{हस्त्यश्वदासानपि कर्मयोग्यान्}
{दासीजनांश्च प्रवराः स्त्रियश्च} %॥१३४॥

\fourlineindentedshloka
{सीतां सुशीलां बहुरत्नभूषितां}
{रथं समारोप्य सुतां सुरुपाम्}
{वेदादिघोषैर्बहुमङ्गलैश्च}
{सम्प्रेषयामास स पार्थिवो बली} %॥१३५॥

\twolineshloka
{प्रेषयित्वा सुतां दिव्यां नत्वा दशरथं नृपम्}
{विश्वामित्रं नमस्कृत्य जनकः सन्निवृत्तवान्} %॥१३६॥

\twolineshloka
{तस्य पल्यो महाभागाः शिक्षयित्वा सुतां तदा}
{भर्तृभक्तिं कुरु शुभे श्वश्रूणां श्वशुरस्य च} %॥१३७॥

\twolineshloka
{श्वश्रूणामर्पयित्वा तां निवृत्ता विविशुः पुरम्}
{ततस्तु रामं गच्छन्तमयोध्यां प्रबलान्वितम्} %॥१३८॥

\twolineshloka
{श्रुत्वा परशुरामो वै पन्थानं संरुरोध ह}
{तं दृष्ट्वा राजपुरुषाः सर्वे ते दीनमानसाः} %॥१३९॥

\twolineshloka
{आसीद्दशरथश्चापि दुःखशोकपरिप्लुतः}
{सभार्यः सपरीवारो भार्गवस्य भयान्नृप} %॥१४०॥

\twolineshloka
{ततोऽब्रवीज्जनान् सर्वान् राजानं च सुदुः खितम्}
{वसिष्ठश्चोर्जिततपा ब्रह्मचारी महामुनिः} %॥१४१॥

\uvacha{वसिष्ठ उवाच}

\onelineshloka
{युष्माभिरत्र रामार्थं न कार्य दुःखमण्वपि} %॥१४२॥

\twolineshloka
{पित्रा वा मातृभिर्वापि अन्यैर्भृत्यजनैरपि}
{अयं हि नृपते रामः साक्षाद्विष्णुस्तु ते गृहे} %॥१४३॥

\twolineshloka
{जगतः पालनार्थाय जन्मप्राप्तो न संशयः}
{यस्य सकीर्त्य नामपि भवभीतिः प्रणश्चति} %॥१४४॥

\twolineshloka
{ब्रह्म मूर्तं स्वयं यत्र भयादेस्तत्र का कथा}
{यत्र सकीर्त्यते रामकथामात्रमपि प्रभो} %॥१४५॥

\twolineshloka
{नोपसर्गभयं तत्र नाकालमरणं नृणाम्}
{इत्युक्ते भार्गवो रामो राममाहाग्रतः स्थितम्} %॥१४६॥

\twolineshloka
{त्यज त्वं रामसज्ञां तु मया वा सगरं कुरु}
{इत्युक्ते राघवः प्राह भार्गवं तं पथि स्थितम्} %॥१४७॥

\twolineshloka
{रामसज्ञां कुतस्त्यक्ष्ये त्वया योत्स्ये स्थिरो भव}
{इत्युक्त्वा तं पृथक् स्थित्वा रामो राजीवलोचनः} %॥१४८॥

\twolineshloka
{ज्याघोषमकरोद्वीरो वीरस्यैवाग्रतस्तदा}
{ततः परशुरामस्य देहान्निष्क्रम्य वैष्णवम्} %॥१४९॥

\twolineshloka
{पश्यतां सर्वभूतानां तेजो राममुखेऽविशत्}
{दृष्ट्वा तं भार्गवो रामः प्रसन्नवदनोऽब्रवीत्} %॥१५०॥

\twolineshloka
{राम राम महाबाहो रामस्त्वं नात्र संशयः}
{विष्णुरेव भवाञ्जातो ज्ञातोऽस्यद्य मया विभो} %॥१५१॥

\twolineshloka
{गच्छ वीर यथाकामं देवकार्यं च वै कुरु}
{दुष्टानां निधनं कृत्वा शिष्टांश्च परिपालय} %॥१५२॥

\twolineshloka
{याहि त्वं स्वेच्छया राम अहं गच्छे तपोवनम्}
{इत्युक्त्वा पूजितस्तैस्तु मुनिभावेन भार्गवः} %॥१५३॥

\twolineshloka
{महेन्द्राद्रिं जगामाथ तपसे धृतमानसः}
{ततस्तु जातहर्षास्ते जना दशरथश्च ह} %॥१५४॥

\twolineshloka
{पुरीमयोध्यां सम्प्राप्य रामेण सह पार्थिवः}
{दिव्यशोभां पुरीं कृत्वा सर्वतो भद्रशालिनीम्} %॥१५५॥

\twolineshloka
{प्रत्युत्थाय ततः पौराः शङ्खतूर्यादिभिः स्वनैः}
{विशन्तं राममागत्य कृतदारं रणेऽजितम्} %॥१५६॥

\twolineshloka
{तं वीक्ष्य हर्षिताः सन्तो विविशुस्तेन वै पुरीम्}
{तौ दृष्ट्वा स मुनिः प्राप्तौ रामं लक्ष्मणमन्तिके} %॥१५७॥

\threelineshloka
{दशरथाय तत्पित्रे मातृभ्यश्च विशेषतः}
{तौ समर्प्य मुनिश्रेष्ठस्तेन राज्ञा च पूजितः}
{विश्वामित्रश्च सहसा प्रतिगन्तुं मनो दधे} %॥१५८॥

\fourlineindentedshloka
{समर्प्य राम स मुनिः सहानुजं}
{सभार्यमग्ने पितुरेकवल्लभम्}
{पुनः पुनः श्राव्य हसन्महामतिर्-}
{जगाम सिद्धाश्रममेवमात्मनः} %॥१५९॥

॥इति श्रीनरसिंहपुराणे रामप्रादुर्भावे सप्तचत्वारिंशोऽध्यायः॥४७॥

\sect{अष्टचत्वारिंशोऽध्यायः --- अयोध्या-काण्डः}

\uvacha{मार्कण्डेय उवाच}

\twolineshloka
{कृतदारो महातेजा रामः कमललोचनः}
{पित्रे सुमहतीं प्रीतिं जनानामुपपादयन्} %॥१॥

\twolineshloka
{अयोध्यायां स्थितो रामः सर्वभोगसमन्वितः}
{प्रीत्या नन्दत्ययोध्यायां रामे रघुपतौ नृप} %॥२॥

\twolineshloka
{भ्राता शत्रुघ्नसहितो भरतो मातुलं ययौ}
{ततो दशरथो राजा प्रसमीक्ष्य सुशोभनम्} %॥३॥

\twolineshloka
{युवानं बलिनं योग्यं भूपसिद्ध्यै सुतं कविम्}
{अभिषिच्य राज्यभारं रामे संस्थाप्य वैष्णवम्} %॥४॥

\twolineshloka
{पदं प्राप्तुं महद्यत्नं करिष्यामीत्यचिन्तयत्}
{सचिन्त्य तत्परो राजा सर्वदिक्षु समादिशत्} %॥५॥

\twolineshloka
{प्राज्ञान् भृत्यान महीपालान्मन्त्रिणश्च त्वरान्वितः}
{रामाभिषेकद्रव्याणि ऋषिप्रोक्तानि यानि वै} %॥६॥

\twolineshloka
{तानि भृत्याः समाहत्य शीघ्रमागन्तुमर्हथ}
{दूतामात्याः समादेशात्सर्वदिक्षु नराधिपान्} %॥७॥

\twolineshloka
{आहूय तान् समाहत्य शीघ्रमागन्तुमर्हथ}
{अयोध्यापुरमत्यर्थं सर्वशोभासमन्वितम्} %॥८॥

\twolineshloka
{जनाः कुरुत सर्वत्र नृत्यगीतादिनन्दितम्}
{पुरवासिजनानन्दं देशवासिमनः प्रियम्} %॥९॥

\twolineshloka
{रामाभिषेकं विपुलं श्वो भविष्यति जानथ}
{श्रुत्वेत्थं मन्त्रिणः प्राहुस्तं नृपं प्रणिपत्य च} %॥१०॥

\twolineshloka
{शोभनं ते मतं राजन् यदिदं परिभाषितम्}
{रामाभिषेकमस्माकं सर्वेषां च प्रियकरम्} %॥११॥

\twolineshloka
{इत्युक्तो दशरथस्तैस्तान् सर्वान् पुनरब्रवीत्}
{आनीयन्तां द्रुतं सर्वे सम्भारा मम शासनात्} %॥१२॥

\twolineshloka
{सर्वतः सारभूता च पुरी चेयं समन्ततः}
{अद्य शोभान्विता कार्या कर्तव्यं यागमण्डलम्} %॥१३॥

\twolineshloka
{इत्येवमुक्ता राज्ञा ते मन्त्रिणः शीघ्रकारिणः}
{तथैव चक्रुस्ते सर्वे पुनः पुनरुदीरिताः} %॥१४॥

\twolineshloka
{प्राप्तहर्षः स राजा च शुभं दिनमुदीक्षयन्}
{कौशल्या लक्ष्मणश्चैव सुमित्रा नागरो जनः} %॥१५॥

\twolineshloka
{रामाभिषेकमाकर्ण्य मुदं प्राप्यातिहर्षितः}
{श्वश्रूश्वशुरयोः सम्यक् शुश्रूषपणपरा तु सा} %॥१६॥

\twolineshloka
{मुदान्विता सिता सीता भर्तुराकर्ण्य शोभनम्}
{श्वोभाविन्यभिषेके तु रामस्य विदितात्मनः} %॥१७॥

\twolineshloka
{दासी तु मन्थरानाम्नी कैकेय्याः कुब्जरुपिणी}
{स्वां स्वामिनीं तु कैकेयीमिदं वचनमब्रवीत्} %॥१८॥

\twolineshloka
{श्रृणु राज्ञि महाभागे वचनं मम शोभनम्}
{त्वत्पतिस्तु महाराजस्तव नाशाय चोद्यतः} %॥१९॥

\twolineshloka
{रामोऽसौ कौसलीपुत्रः श्वो भविष्यति भूपतिः}
{वसुवाहनकोशादि राज्यं च सकलं शुभे} %॥२०॥

\twolineshloka
{भविष्यत्यद्य रामस्य भरतस्य न किचन}
{भरतोऽपि गतो दूरं मातुलस्य गृहं प्रति} %॥२१॥

\twolineshloka
{हा कष्टं मन्दभाग्यासि सापल्याद्दुःखिता भृशम्}
{सैवमाकर्ण्य कैकेयी कुब्जामिदमथाब्रवीत्} %॥२२॥

\twolineshloka
{पश्य मे दक्षतां कुब्जे अद्यैव त्वं विचक्षणे}
{यथा तु सकलं राज्यं भरतस्य भविष्यति} %॥२३॥

\twolineshloka
{रामस्य वनवासश्च तथा यत्नं करोम्यहम्}
{इत्युक्त्वा मन्थरां सा तु उन्मुच्य स्वाङ्गभूषणम्} %॥२४॥

\twolineshloka
{वस्त्रं पुष्पाणि चोन्मुच्य स्थूलवासोधराभवत्}
{निर्माल्यपुष्पधृक्कष्टा कश्मलाङ्गी विरुपिणी} %॥२५॥

\twolineshloka
{भस्मधूल्यादिनिर्दिग्धा भस्मधूल्या तथा श्रिते}
{भूभागे शान्तदीपे सा सन्ध्याकाले सुदुःखिता} %॥२६॥

\twolineshloka
{ललाटे श्वेतचैलं तु बद्ध्वा सुष्वाप भामिनी}
{मन्त्रिभिः सह कार्याणि सम्मन्त्र्य सकलानि तु} %॥२७॥

\twolineshloka
{पुण्याहः स्वस्तिमाङ्गल्यैः स्थाप्य रामं तु मण्डले}
{ऋषिभस्तु वसिष्ठाद्यैः सार्धं सम्भारमण्डपे} %॥२८॥

\twolineshloka
{वृद्धिजागरणीयैश्च सर्वतस्तूर्यनादिते}
{गीतनृत्यसमाकीर्णे शङ्खकाहलनिः स्वनैः} %॥२९॥

\twolineshloka
{स्वयं दशरथस्तत्र स्थित्वा प्रत्यागतः पुनः}
{कैकेया वेश्मनो द्वारं जरद्भिः परिरक्षितम्} %॥३०॥

\twolineshloka
{रामाभिषेकं कैकेयीं वक्तुकामः स पार्थिवः}
{कैकेयीभवनं वीक्ष्य सान्धकारमथाब्रवीत्} %॥३१॥

\twolineshloka
{अन्धकारमिदं कस्मादद्य ते मन्दिरे प्रिये}
{रामाभिषेकं हर्षाय अन्त्यजा अपि मेनिरे} %॥३२॥

\twolineshloka
{गृहालकरणं कुर्वन्त्यद्य लोका मनोहरम्}
{त्वयाद्य न कृतं कस्मादित्युक्त्वा च महीपतिः} %॥३३॥

\twolineshloka
{ज्वालायित्वा गृहे दीपान् प्रविवेश गृहं नृपः}
{अशोभनाङ्गीं कैकेयीं स्वपन्तीं पतितां भुवि} %॥३४॥

\twolineshloka
{दृष्ट्वा दशरथः प्राह तस्याः प्रियमिदं त्विति}
{आश्लिष्योत्थाय तां राजा श्रृणु मे परमं वचः} %॥३५॥

\twolineshloka
{स्वमातुरधिकां नित्यं यस्ते भक्तिं करोति वै}
{तस्याभिषेकं रामस्य श्वो भविष्यति शोभने} %॥३६॥

\twolineshloka
{इत्युक्ता पार्थिवेनापि किचिन्नोवाच सा शुभा}
{मुञ्चन्ती दीर्घमुष्णं च रोषोस्च्छ्वासं मुहुर्मुहुः} %॥३७॥

\twolineshloka
{तस्थावाश्लिष्य हस्ताभ्यां पार्थिवः प्राह रोषिताम्}
{किं ते कैकेयि दुःखस्य कारणं वद शोभने} %॥३८॥

\twolineshloka
{वस्त्राभरणरत्नादि यद्यदिच्छसि शोभने}
{तत्त्वं गृह्णीष्व निश्शङ्कं भाण्डारात् सुखिनी भव} %॥३९॥

\twolineshloka
{भाण्डारेण मम शुभे श्वोऽर्थसिद्धिर्भविष्यति}
{यदाभिषेकं सम्प्राप्ते रामे राजीवलोचने} %॥४०॥

\twolineshloka
{भाण्डागारस्य मे द्वारं मया मुक्तं निरर्गलम्}
{भविष्यति पुनः पूर्णं रामे राज्यं प्रशासति} %॥४१॥

\twolineshloka
{बहु मानय रामस्य अभिषेकं महात्मनः}
{इत्युक्ता राजवर्य्येण कैकेयी पापलक्षणा} %॥४२॥

\twolineshloka
{कुमतिर्नर्घुणा दुष्टा कुब्जया शिक्षिताब्रवीत्}
{राजानं स्वपतिं वाक्यं क्रूरमत्यन्तनिष्ठुरम्} %॥४३॥

\twolineshloka
{रत्नादि सकलं यत्ते तन्ममैव न संशयः}
{देवासुरमहायुद्धे प्रीत्या यन्मे वरद्वयम्} %॥४४॥

\twolineshloka
{पुरा दत्तं त्वया राजंस्तदिदानीं प्रयच्छ मे}
{इत्युक्तः पार्थिवः प्राह कैकेयीमशुभां तदा} %॥४५॥

\twolineshloka
{अदत्तमप्यहं दास्ये तव नान्यस्य वा शुभे}
{किं मे प्रतिश्रुतं पूर्वं दत्तमेव मया तव} %॥४६॥

\twolineshloka
{शुभाङ्गी भव कल्याणि त्यज कोपमनर्थकम्}
{रामाभिषेकजं हर्षं भजोत्तिष्ठ सुखी भव} %॥४७॥

\twolineshloka
{इत्युक्ता राजवर्येण कैकेयी कलहप्रिया}
{उवाच परुषं वाक्यं राज्ञो मरणकारणम्} %॥४८॥

\twolineshloka
{वरद्वयं पूर्वदत्तं यदि दास्यसि मे विभो}
{श्वोभूते गच्छतु वनं रामोऽयं कोशलात्मजः} %॥४९॥

\twolineshloka
{द्वादशाब्दं निवसतु त्वद्वाक्याद्दण्डके वने}
{अभिषेकं च राज्यं च भरतस्य भविष्यति} %॥५०॥

\twolineshloka
{इत्याकर्ण्य स कैकेया वचनं घोरमप्रियम्}
{पपात भुवि निस्सज्ञो राजा सापि विभूषिता} %॥५१॥

\twolineshloka
{रात्रिशेषं नयित्वा तु प्रभाते सा मुदावती}
{दूतं सुमन्त्रमाहैवं राम आनीयतामिति} %॥५२॥

\twolineshloka
{रामस्तु कृतपुण्याहः कृतस्वस्त्ययनो द्विजैः}
{यागमण्डपमध्यस्थः शङ्खतूर्यरवान्वितः} %॥५३॥

\twolineshloka
{तमासाद्य ततो दूतः प्रणिपत्य पुरः स्थितः}
{राम राम महाबाहो आज्ञापयति ते पिता} %॥५४॥

\twolineshloka
{द्रुतमुत्तिष्ठ गच्छ त्वं यत्र तिष्ठति ते पिता}
{इत्युक्तस्तेन दूतेन शीघ्रमुत्थाय राघवः} %॥५५॥

\twolineshloka
{अनुज्ञाप्य द्विजान् प्राप्तः कैकेय्या भवनं प्रति}
{प्रविशन्तं गृहं रामं कैकेयी प्राह निर्घृणा} %॥५६॥

\twolineshloka
{पितुस्तव मतं वत्स इदं ते प्रब्रवीम्यहम्}
{वने वस महाबाहो गत्वा त्वं द्वादशाब्दकम्} %॥५७॥

\twolineshloka
{अद्यैव गम्यतां वीर तपसे धृतमानसः}
{न चिन्त्यमन्यथा वत्स आदरात् कुरु मे वचः} %॥५८॥

\twolineshloka
{एतच्छुत्वा पितुर्वाक्यं रामः कमललोचनः}
{तथेत्याज्ञां गृहीत्वासौ नमस्कृत्य च तावुभौ} %॥५९॥

\twolineshloka
{निष्क्रम्य तदगृहाद्रामो धनुरादाय वेश्मतः}
{कौशल्यां च नमस्कृत्य सुमित्रां गन्तुमुद्यतः} %॥६०॥

\twolineshloka
{तच्छुत्वा तु ततः पौरा दुःखशोकपरिप्लुताः}
{विव्यथुश्चाथ सौमित्रिः कैकेयीं प्रति रोषितः} %॥६१॥

\twolineshloka
{ततस्तं राघवो दृष्ट्वा लक्ष्मणं रक्तलोचनम्}
{बारयामास धर्मज्ञो धर्मवाग्भिर्महामतिः} %॥६२॥

\twolineshloka
{ततस्तु तत्र ये वृद्धास्तान प्रणम्य मुनींश्च सः}
{रामो रथं खिन्नसूतं प्रस्थानायारुरोह वै} %॥६३॥

\twolineshloka
{आत्मीयं सकलं द्रव्यं ब्राह्मणेभ्यो नृपात्मजः}
{श्रद्धया परया दत्त्वा वस्त्राणि विविधानि च} %॥६४॥

\twolineshloka
{तिस्त्रः श्वश्रूः समामन्त्र्य श्वशुरं च विसज्ञितम्}
{मुञ्चन्तमश्रुधाराणि नेत्रयोः शोकजानि च} %॥६५॥

\twolineshloka
{पश्यती सर्वतः सीता चारुरोह तथा रथम्}
{रथमारुह्य गच्छन्तं सीतया सह राघवम्} %॥६६॥

\twolineshloka
{दृष्ट्वा सुमित्रा वचनं लक्ष्मणं चाह दुःखिता}
{रामं दशरथं विद्धि मां विद्धि जनकात्मजाम्} %॥६७॥

\twolineshloka
{अयोध्यामटर्वी विद्धि व्रज ताभ्यां गुणाकर}
{मात्रैवमुक्तो धर्मात्मा स्तनक्षीरार्द्रदेहया} %॥६८॥

\twolineshloka
{तां नत्वा चारुयानं तमारुरोह स लक्ष्मनः}
{गच्छतो लक्ष्मणो भ्राता सीता चैव पतिव्रताः} %॥६९॥

\twolineshloka
{रामस्य पृष्ठतो यातौ पुराद्धीरौ महामते}
{विधिच्छिन्नाभिषेकं तं रामं राजीवलोचनम्} %॥७०॥

\twolineshloka
{अयोध्याया विनिष्क्रान्तमनुयाताः पुरोहिताः}
{मन्त्रिणः पौरमुख्याश्च दुःखेन महतान्विताः} %॥७१॥

\twolineshloka
{तं च प्राप्य हि गच्छन्तं राममूचुरिदं वचः}
{राम राम महाबाहो गन्तुं नार्हसि शोभन} %॥७२॥

\twolineshloka
{राजन्नत्र निवर्तस्व विहायास्मान् क्व गच्छसि}
{इत्युक्तो राघवस्तैस्तु तानुवाच दृढव्रतः} %॥७३॥

\twolineshloka
{गच्छध्वं मन्त्रिणः पौरा गच्छध्वं च पुरोधसः}
{पित्रादेशं मया कार्यमभियास्यामि वै वनम्} %॥७४॥

\twolineshloka
{द्वादशाब्दं व्रतं चैतन्नीत्वाहं दण्डके वने}
{आगच्छामि पितुः पादं मातृणां द्रष्टुमञ्जसा} %॥७५॥

\twolineshloka
{इत्युक्त्वा ताञ्जगामाथ रामः सत्यपरायणः}
{तं गच्छन्तं पुनर्याताः पृष्ठतो दुःखिता जनाः} %॥७६॥

\twolineshloka
{पुनः प्राह स काकुत्स्थो गच्छध्वं नगरीमिमाम्}
{मातृश्च पितरं चैव शत्रुघ्नं नगरीमिमाम्} %॥७७॥

\twolineshloka
{प्रजाः समस्तास्त्रत्रस्था राज्यं भरतमेव च}
{पालयध्वं महाभागास्तपसे याम्यहं वनम्} %॥७८॥

\twolineshloka
{अथ लक्ष्मणमाहेदं वचनं राघवस्तदा}
{सीतामर्पय राजानं जनकं मिथिलेश्वरम्} %॥७९॥

\twolineshloka
{पितृमातृवशे तिष्ठ गच्छ लक्ष्मण याम्यहम्}
{इत्युक्तः प्राह धर्मात्मा लक्ष्मणो भ्रातृवत्सलः} %॥८०॥

\twolineshloka
{मैवामाज्ञापाय विभो मामद्य करुणाकर}
{गन्तुमिच्छसि यत्र त्वमवश्यं तत्र याम्यहम्} %॥८१॥

\twolineshloka
{इत्युक्तो लक्ष्मणेनासौ सीतां तामाह राघवः}
{सीते गच्छ ममादेशात् पितरं प्रति शोभने} %॥८२॥

\twolineshloka
{सुमित्राया गृहे चापि कौशल्यायाः सुमध्यमे}
{निवर्तस्व हि तावत्त्वं यावदागमनं मम} %॥८३॥

\twolineshloka
{इत्युक्ता राघवेनापि सीता प्राह कृताञ्जलिः}
{यत्र गत्वा वने वासं त्वं करोषि महाभुज} %॥८४॥

\twolineshloka
{तत्र गत्वा त्वया सार्धं वसाम्यहमरिन्दम}
{वियोगं नो सहे राजंस्त्वया सत्यवता क्वचित्} %॥८५॥

\twolineshloka
{अतस्त्वां प्रार्थयिष्यामि दयां कुरु मम प्रभो}
{गन्तुमिच्छसि यत्र त्वमवश्यं तत्र याम्यहम्} %॥८६॥

\twolineshloka
{नानायानैरुपगताञ्जनान् वीक्ष्य स पृष्ठतः}
{योषितां च गणान् रामो वारयामास धर्मवित्} %॥८७॥

\twolineshloka
{निवृत्त्य स्थीयतां स्वैरमयोध्यायां जनाः स्त्रियः}
{गत्वाहं दण्डकारण्यं तपसे धृतमानसः} %॥८८॥

\twolineshloka
{कतिपयाब्दादायास्ये नान्यथा सत्यमीरितम्}
{लक्ष्मणेन सह भ्रात्रा वैदेह्या च स्वभार्यया} %॥८९॥

\twolineshloka
{जनान्निवर्त्य रामोऽसौ जगाम च गुहाश्रमम्}
{गुहस्तु रामभक्तोऽसौ स्वभावादेव वैष्णवः} %॥९०॥

\twolineshloka
{कृताञ्जलिपुटो भूत्वा किं कर्तव्यमिति स्थितः}
{महता तपसाऽऽनीता गुरुणा या हि वः पुरा} %॥९१॥

\twolineshloka
{भगीरथेन या भूमिं सर्वपापहरा शुभा}
{नानामुनिजनैर्जुष्टा कूर्ममत्स्यसमाकुला} %॥९२॥

\twolineshloka
{गङ्गा तुङ्गोर्मिमालाढ्या स्फटिकाभजलावहा}
{गुहोपनीतनावा तु तां गङ्गां स महाद्युतिः} %॥९३॥

\twolineshloka
{उत्तीर्य भगवान् रामो भरद्वाजाश्रमं शुभम्}
{प्रयागे तु ततस्तस्मिन् स्त्रात्वा तीर्थे यथाविधि} %॥९४॥

\twolineshloka
{लक्ष्मणेन सह भ्रात्रा राघवः सीतया सह}
{भरद्वाजाश्रमे तत्र विश्रान्तस्तेन पूजितः} %॥९५॥

\twolineshloka
{ततः प्रभाते विमले तमनुज्ञाप्य राघवः}
{भरद्वाजोक्तमार्गेण चित्रकूटं शनैर्ययौ} %॥९६॥

\twolineshloka
{नानाद्रुमलताकीर्णं पुण्यतीर्थमनुत्तमम्}
{तापसं वेषमास्थाय जह्नुकन्यामतीत्य वै} %॥९७॥

\twolineshloka
{गते रामे सभार्ये तु सह भ्रात्रा ससारथौ}
{अयोध्यामवसन् भूप नष्टशोभां सुदुःखिताः} %॥९८॥

\twolineshloka
{नष्टसज्ञो दशरथः श्रुत्वा वचनमप्रियम्}
{रामप्रवासजननं कैकेय्या मुखनिस्सृतम्} %॥९९॥

\twolineshloka
{लब्धसज्ञः क्षणाद्राजा रामरामेति चुक्रुशे}
{कैकेय्युवाच भूपालं भरतं चाभिषेचय} %॥१००॥

\twolineshloka
{सीतालक्ष्मणसंयुक्तो रामचन्द्रो वनं गतः}
{पुत्रशोकाभिसन्तप्तो राजा दशरथस्तदा} %॥१०१॥

\twolineshloka
{विहाय देहं दुःखेन देवलोकं गतस्तदा}
{ततस्तस्य महापुर्य्यामयोध्यायामरिन्दम} %॥१०२॥

\twolineshloka
{रुरुदुर्दुःखशोकार्त्ता जनाः सर्वे च योषितः}
{कौशल्या च सुमित्रा च कैकेयी कष्टकारिणी} %॥१०३॥

\twolineshloka
{परिवार्य मृतं तत्र रुरुदुस्ताः पतिं ततः}
{ततः पुरोहितस्तत्र वसिष्ठः सर्वधर्मवित्} %॥१०४॥

\twolineshloka
{तैलद्रोण्यां विनिक्षिप्य मृतं राजकलेवरम्}
{दूत वैं प्रेषयामास सहमन्त्रिगणैः स्थितः} %॥१०५॥

\twolineshloka
{स गत्वा यत्र भरतः शत्रुघ्नेन सह स्थितः}
{तत्र प्राप्य तथा वार्तां सन्निवर्त्य नृपात्मजौ} %॥१०६॥

\twolineshloka
{तावानीय ततः शीघ्रमयोध्यां पुनरागतः}
{क्रूराणि दृष्ट्वा भरतो निमित्तानि च वै पथि} %॥१०७॥

\twolineshloka
{विपरीतं त्वयोध्यामिति मेने स पार्थिवः}
{निश्शोभां निर्गतश्रीकां दुःखशोकान्वितां पुरीम्} %॥१०८॥

\twolineshloka
{कैकेय्याग्निविनिर्दग्धामयोध्यां प्रविवेश सः}
{दुःखान्विता जनाः सर्वे तौ दृष्ट्वा रुरुदुर्भृशम्} %॥१०९॥

\twolineshloka
{हा तात राम हा सीते लक्ष्मणेति पुनः पुनः}
{रुरोद भरतस्तत्र शत्रुघ्नश्च सुदुःखितः} %॥११०॥

\twolineshloka
{कैकेय्यास्तत्क्षणाच्छुत्वा चुक्रोध भरतस्तदा}
{दुष्टा त्वं दुष्टचित्ता च यया रामः प्रवासितः} %॥१११॥

\twolineshloka
{लक्ष्मणेन सह भ्रात्रा राघवः सीतया वनम्}
{साहसं किं कृतं दुष्टे त्वया सद्यो‍ऽल्पभाग्यया} %॥११२॥

\twolineshloka
{उद्वास्य सीतया रामं लक्ष्मणेन महात्मना}
{ममैव पुत्रं राजानं करोत्विति मतिस्तव} %॥११३॥

\twolineshloka
{दुष्टाया नष्टभाग्यायाः पुत्रोऽहं भाग्यवर्जितः}
{भ्रात्रा रामेण रहितो नाहं राज्यं करोमि वै} %॥११४॥

\twolineshloka
{यत्र रामो नरव्याध्रः पद्यपत्रायतेक्षणः}
{धर्मज्ञः सर्वशास्त्राज्ञो मतिमान् बन्धुवत्सलः} %॥११५॥

\twolineshloka
{सीता च यत्र वैदेही नियमव्रतचारिणी}
{पतिव्रता महाभागा सर्वलक्षणसंयुता} %॥११६॥

\twolineshloka
{लक्ष्मणश्च महावीर्यो गुणवान् भ्रातृवत्सलः}
{तत्र यास्यामि कैकेयि महत्पापं त्वया कृतम्} %॥११७॥

\twolineshloka
{राम एव मम भ्राता ज्येष्ठो मतिमतां वरः}
{स एव राजा दुष्टात्मे भृत्यो‍ऽहं तस्य वै सदा} %॥११८॥

\twolineshloka
{इत्युक्त्वा मातरं तत्र रुरोद भृशदुःखितः}
{हा राजन् पृथिवीपाल मां विहाय सुदुःखितम्} %॥११९॥

\twolineshloka
{क्व गतोऽस्यद्य वै तात किं करोमीह तद्वद}
{भ्राता पित्रा समः क्वास्ते ज्येष्ठो मे करुणाकरः} %॥१२०॥

\twolineshloka
{सीता च मातृतुल्या मे क्व गतो लक्ष्मणश्चह}
{इत्येवं विलपन्तं तं भरतं मन्त्रिभिः सह} %॥१२१॥

\twolineshloka
{वसिष्ठो भगवानाह कालकर्मविभागवित्}
{उत्तिष्ठोत्तिष्ठ वत्स त्वं न शोकं कर्तुमर्हसि} %॥१२२॥

\twolineshloka
{कर्मकालवशादेव पिता ते स्वर्गमास्थितः}
{तस्य संस्कारकार्याणि कर्माणि कुरु शोभन} %॥१२३॥

\twolineshloka
{रामोऽपि दुष्टनाशाय शिष्टानां पालनाय च}
{अवतीर्णो जगत्स्वामी स्वांशेन भुवि माधवः} %॥१२४॥

\twolineshloka
{प्रायस्तत्रास्ति रामेण कर्तव्यं लक्ष्मणेन च}
{यत्रासौ भगवान् वीरः कर्मणा तेन चोदितः} %॥१२५॥

\twolineshloka
{तत्कृत्वा पुनरायाति रामः कमललोचनः}
{इत्युक्तो भरतस्तेन वसिष्ठेन महात्मना} %॥१२६॥

\twolineshloka
{संस्कारं लम्भयामास विधिदृष्टेन कर्मणा}
{अग्निहोत्राग्निना दग्ध्वा पितुर्देहं विधानतः} %॥१२७॥

\twolineshloka
{स्नात्वा सरय्वाः सलिले कृत्वा तस्योदकक्रियाम्}
{शत्रुघ्नेन सह श्रीमान्तातृभिर्बान्धवैः सह} %॥१२८॥

\twolineshloka
{तस्यौर्ध्वदेहिकं कृत्वा मन्त्रिणा मन्त्रिनायकः}
{हस्त्यश्वरथपत्तीभिः सह प्रायान्महामतिः} %॥१२९॥

\twolineshloka
{भरतो राममन्वेष्टुं राममार्गेण सत्तमः}
{तमायान्तं महासेनं रामस्यानुविरोधिनम्} %॥१३०॥

\twolineshloka
{मत्वा तं भरतं शत्रुं रामभक्तो गुहस्तदा}
{स्वं सैन्यं वर्तुलं कृत्वा सन्नद्धः कवची रथी} %॥१३१॥

\onelineshloka
{महाबलपरीवारो रुरोध भरतं पथि} %॥१३२॥

\twolineshloka
{सभ्रातृकं सभा र्यं मे राम स्वामिनमुत्तमम्}
{प्रापयस्त्वं वनं दुष्टं साम्प्रतं हन्तुमिच्छसि} %॥१३३॥

\twolineshloka
{गमिष्यसि दुरात्मंस्त्वं सेनया सह दुर्मते}
{इत्युक्तो भरतस्तत्र गुहेन नृपनन्दनः} %॥१३४॥

\twolineshloka
{तमुवाच विनीतात्मा रामायाथ कृताञ्जलिः}
{यथा त्वं रामभक्तोऽमि तथाहमपि भक्तिमान्} %॥१३५॥

\twolineshloka
{प्रोषिते मयि कैकेय्या कृतमेतन्महामते}
{रामस्यानयनार्थाय व्रजाम्यद्य महामते} %॥१३६॥

\twolineshloka
{सत्यपूर्वं गमिष्यामि पन्थानं देहि मे गुह}
{इति विश्वासमानीय जाह्नवीं तेन तारितः} %॥१३७॥

\twolineshloka
{नौकावृन्दैरनेकैस्तु स्त्रात्वासौ जाह्नवीजले}
{भरद्वाजाश्रमं प्राप्तो भरतस्तं महामुनिम्} %॥१३८॥

\twolineshloka
{प्रणम्य शिरसा तस्मै यथावृत्तमुवाच ह}
{भरद्वाजोऽपि तं प्राह कालेन कृतमीदृशम्} %॥१३९॥

\twolineshloka
{दुःखं न तावत् कर्तव्यं रामार्थेऽपि त्वयाधुना}
{वर्तते चित्रकूटेऽसौ रामः सत्यपराक्रमः} %॥१४०॥

\twolineshloka
{त्वयि तत्र गते वापि प्रायोऽसौ नागमिष्यति}
{तथापि तत्र गच्छ त्वं यदसौ वक्ति तत्कुरु} %॥१४१॥

\twolineshloka
{रामस्तु सीतया सार्धं वनखण्डे स्थितः शुभे}
{लक्ष्मणस्तु महावीर्यो दुष्टालोकनतत्परः} %॥१४२॥

\twolineshloka
{इत्युक्तो भरतस्तत्र भरद्वाजेन धीमता}
{उत्तीर्य यमुनां यातश्चित्रकूटं महानगम्} %॥१४३॥

\twolineshloka
{स्थितोऽसौ दृष्टवान्दूरात्सधूलीं चोत्तरां दिशम्}
{रामाय कथियित्वाऽऽस तदादेशात्तु लक्ष्मणः} %॥१४४॥

\twolineshloka
{वृक्षमारुह्य मेधावी वीक्षमाणः प्रयत्नतः}
{स ततो दृष्टवान् हष्टामायान्तीं महतीं चमूम्} %॥१४५॥

\twolineshloka
{हस्त्यश्वरथसंयुक्तां दृष्ट्वा राममथाब्रवीत्}
{हे भ्रातस्त्वं महाबाहो सीतापार्श्वे स्थिरो भव} %॥१४६॥

\twolineshloka
{भूपोऽस्ति बलवान् कश्चिद्धस्त्यश्वरथपत्तिभिः}
{इत्याकर्ण्य वचस्तस्य लक्ष्मणस्य महात्मनः} %॥१४७॥

\twolineshloka
{रामस्तब्रवीद्वीरो वीरं सत्यपराक्रमः}
{प्रायेण भरतोऽस्माकं द्रष्टुमायाति लक्ष्मण} %॥१४८॥

\twolineshloka
{इत्येवं वदतस्तस्य रामस्य विदितात्मनः}
{आरात्संस्थाप्य सेनां तां भरतो विनयान्वितः} %॥१४९॥

\twolineshloka
{ब्राह्मणैर्मन्त्रिभिः सार्धं रुदन्नागत्य पादयोः}
{रामस्य निपपाताथ वैदेह्या लक्ष्मणस्य च} %॥१५०॥

\twolineshloka
{मन्त्रिणो मातृवर्गश्च स्निग्धबन्धुसुहज्जनाः}
{परिवार्य ततो रामं रुरुदुः शोककातराः} %॥१५१॥

\twolineshloka
{स्वर्यातं पितरं ज्ञात्वा ततो रामो महामतिः}
{लक्ष्मणेन सह भ्रात्रा वैदोह्याथ समन्वितः} %॥१५२॥

\twolineshloka
{स्त्रात्वा मलापहे तीर्थे दत्त्वा च सलिलाञ्जलिम्}
{मात्रादीनभिवाद्याथ रामो दुःखसमन्वितः} %॥१५३॥

\twolineshloka
{उवाच भरतं राजन् दुःखेन महतान्वितम्}
{अयोध्यां गच्छ भरत इतः शीघ्रं महामते} %॥१५४॥

\twolineshloka
{राज्ञा विहीनां नगरीं अनाथां परिपालय}
{इत्युक्तो भरतः प्राह रामं राजीवलोलचनम्} %॥१५५॥

\twolineshloka
{त्वामृते पुरुषव्याघ्र न यास्येऽहमितो ध्रुवम्}
{यत्र त्वं तत्र यास्यामि वैदेही लक्ष्मणो यथा} %॥१५६॥

\twolineshloka
{इत्याकर्ण्य पुनः प्राह भरतं पुरतः स्थितम्}
{नृणां पितृसमो ज्येष्ठः स्वधर्ममनुवर्तिनाम्} %॥१५७॥

\twolineshloka
{यथा न लङ्ह्यं वचनं मया पितृमुखेरितम्}
{तथा त्वया न लङ्ह्यं स्याद्वचनं मम सत्तम} %॥१५८॥

\twolineshloka
{मत्समीपादितो गत्वा प्रजास्त्वं परिपालय}
{द्वादशाब्दिकमेतन्मे व्रतं पितृमुखेरितम्} %॥१५९॥

\twolineshloka
{तदरण्ये चरित्वा तु आगामिष्यामि तेऽन्तिकम्}
{गच्छ तिष्ठ ममादेशे न दुःखं कर्तुमर्हसि} %॥१६०॥

\twolineshloka
{इत्युक्तो भरतः प्राह बाष्पपर्याकुलेक्षणः}
{यथा पिता तथा त्वं मे नात्र कार्या विचारणा} %॥१६१॥

\twolineshloka
{तवादेशान्मया कार्यं देहि त्वं पादुके मम}
{नन्दिग्रामे वसिष्येऽहं पादुके द्वादशाब्दिकम्} %॥१६२॥

\twolineshloka
{त्वद्वेषमेव मद्वेषं त्वदव्रतं मे महाव्रतम्}
{त्वं द्वादशाब्दिकादूर्ध्वं यदि नायासि सत्तम} %॥१६३॥

\twolineshloka
{ततो हविर्यथा चाग्नौ प्रधक्ष्यामि कलेवरम्}
{इत्येवं शपथं कृत्वा भरतो हि सुदुःखितः} %॥१६४॥

\twolineshloka
{बहु प्रदक्षिणं कृत्वा नमस्कृत्य च राघवम्}
{पादुके शिरसा स्थाप्य भरतः प्रस्थितः शनैः} %॥१६५॥

\twolineshloka
{स कुर्वन् भ्रातुरादेशं नन्दिग्रामे स्थितो वशी}
{तपस्वी नियताहारः शाकमूलफलाशनः} %॥१६६॥

\fourlineindentedshloka
{जटाकलापं शिरसा च बिभ्रत्}
{त्वचश्च वाक्षीः किल वन्यभोजी}
{रामस्य वाक्यादरतो हदि स्थितं}
{बभार भूभारमनिन्दितात्मा} %॥१६७॥

॥इति श्रीनरसिंहपुराणे रामप्रादुर्भावे अष्टचत्वारिंशोऽध्यायः॥४८॥

\sect{एकोनपञ्चाशोऽध्यायः --- अरण्य-काण्डः}

\uvacha{मार्कण्डेय उवाच}

\twolineshloka
{गतेऽथ भरते तस्मिन् रामः कमललोचनः}
{लक्ष्मणेन सह भ्रात्रा भार्यया सीतया सह} %॥१॥

\twolineshloka
{शाकमूलफलाहारो विचचार महावने}
{कदाचिल्लक्ष्मणमृते रामदेवः प्रतापवान्} %॥२॥

\twolineshloka
{चित्रकूटवनोद्देशे वैदेह्युत्सङ्गमाश्रितः}
{सुष्वाप स मुहूर्तं तु ततः काको दुरात्मवान्} %॥३॥

\twolineshloka
{सीताभिमुखमभ्येत्य विददार स्तनान्तरम्}
{विदार्य वृक्षमारुह्य स्थितोऽसौ वायसाधमः} %॥४॥

\twolineshloka
{ततः प्रबुद्धो रामोऽसौ दृष्ट्वा रक्तं स्तनान्तरे}
{शोकाविष्टां तु सीतां तामुवाच कमलेक्षणः} %॥५॥

\twolineshloka
{वद स्तनान्तरे भद्रे तव रक्तस्य कारणम्}
{इत्युक्ता सा च तं प्राह भर्तारं विनयान्विता} %॥६॥

\twolineshloka
{पश्य राजेन्द्र वृक्षाग्रे वायसं दुष्टचेष्टितम्}
{अनेनैव कृतं कर्म सुप्ते त्वयि महामते} %॥७॥

\twolineshloka
{रामोऽपि दृष्टवान् काकं तस्मिन् क्रोधमथाकरोत्}
{इषीकास्त्रं समाधाय ब्रह्मास्त्रेणाभिमन्त्रितम्} %॥८॥

\twolineshloka
{काकमुद्दिश्य चिक्षेप सोऽप्यधावद्भयान्वितः}
{स त्विन्द्रस्य सुतो राजन्निन्द्रलोकं विवेश ह} %॥९॥

\twolineshloka
{रामास्त्रं प्रज्वलद्दीप्तं तस्यानु प्रविवेश वै}
{विदितार्थश्च देवेन्द्रो देवैः सह समन्वितः} %॥१०॥

\twolineshloka
{निष्क्रामयच्च तं दुष्टं राघवस्यापकारिणम्}
{ततोऽसौ सर्वदेवैस्तु देवलोकाद्वहिः कृतः} %॥११॥

\twolineshloka
{पुनः सोऽप्यपतद्रामं राजानं शरणं गतः}
{पाहि राम महाबाहो अज्ञानादपकारिणम्} %॥१२॥

\twolineshloka
{इति ब्रुवन्तं तं प्राह रामः कमललोचनः}
{अमोघं च ममैवास्त्रमङ्गमेकं प्रयच्छ वै} %॥१३॥

\twolineshloka
{ततो जीवसि दुष्ट त्वमपकारो महान् कृतः}
{इत्युक्तोऽसौ स्वकं नेत्रमेकमस्त्राय दत्तवान्} %॥१४॥

\twolineshloka
{अस्त्रं तन्नेत्रमेकं तु भस्मीकृत्य समाययौ}
{ततः प्रभृति काकानां सर्वेषामेकनेत्रता} %॥१५॥

\twolineshloka
{चक्षुषैकेन पश्यन्ति हेतुना तेन पार्थिव}
{उषित्वा तत्र सुचिरं चित्रकूटे स राघवः} %॥१६॥

\twolineshloka
{जगाम दण्डकारण्यं नानामुनिनिषेवितम्}
{सभ्रातृकः सभार्यश्च तापसं वेषमास्थितः} %॥१७॥

\twolineshloka
{धनुः पर्वसुपाणिश्च सेषुधिश्च महाबलः}
{ततो ददर्श तत्रस्थानम्बुभक्षान्महामुनीन्} %॥१८॥

\twolineshloka
{अश्मकुट्टाननेकांश्च दन्तोलूखलिनस्तथा}
{पञ्चाग्निमध्यगानन्यानन्यानुग्रतपश्चरान्} %॥१९॥

\twolineshloka
{तान् दृष्ट्वा प्रणिपत्योच्चै रामस्तैश्चाभिनन्दितः}
{ततोऽखिलं वनं दृष्ट्वा रामः साक्षाज्जनार्दनः} %॥२०॥

\twolineshloka
{भ्रातृभार्यासहायश्च सम्प्रतस्थे महामतिः}
{दर्शयित्वा तु सीतायै वनं कुसुमितं शुभम्} %॥२१॥

\twolineshloka
{नानाश्चर्यसमायुक्तं शनैर्गच्छन् स दृष्टवान्}
{कृष्णाङ्गं रक्तनेत्रं तु स्थूलशैलसमानकम्} %॥२२॥

\twolineshloka
{शुभ्रदंष्ट्रं महाबाहुं सन्ध्याघनशिरोरुहम्}
{मेघस्वनं सापराधं शरं सन्धाय राघवः} %॥२३॥

\twolineshloka
{विव्याध राक्षसं क्रोधाल्लक्ष्मणेन सह प्रभुः}
{अन्यैरवध्यं हत्वा तं गिरिगर्ते महातनुम्} %॥२४॥

\twolineshloka
{शिलाभिश्छाद्य गतवाज्शरभङ्गाश्रमं ततः}
{तं नत्वा तत्र विश्रम्य तत्कथातुष्टमानसः} %॥२५॥

\twolineshloka
{तीक्ष्णाश्रममुपागम्य दुष्टवांस्तं महामुनिम्}
{तेनादिष्टेन मार्गेण गत्वागस्त्यं ददर्श ह} %॥२६॥

\twolineshloka
{खङ्गं तु विमलं तस्मादवाप रघुनन्दनः}
{इषुधिं चाक्षयशरं चापं चैव तु वैष्णवम्} %॥२७॥

\twolineshloka
{ततोऽगस्त्याश्रमाद्रामो भ्रातृभार्यासमन्वितः}
{गोदावर्याः समीपे तु पञ्चवट्यामुवास सः} %॥२८॥

\twolineshloka
{ततो जटायुरभ्येत्य रामं कमललोचनम्}
{नत्वा स्वकुलमाख्याय स्थितवान् गृध्रनायकः} %॥२९॥

\twolineshloka
{रामोऽपि तत्र तं दृष्ट्वा आत्मवृत्तं विशेषतः}
{कथयित्वा तु तं प्राह सीतां रक्ष महामते} %॥३०॥

\twolineshloka
{इत्युक्तोऽसौ जतायुस्तु राममालिङग्य सादरम्}
{कार्यार्थं तु गते रामे भ्रात्रा सह वनान्तरम्} %॥३१॥

\twolineshloka
{अहं रक्ष्यामि ते भार्यां स्थीयतामत्र शोभन}
{इत्युक्त्वा गतवान्रामं गृध्रराजः स्वमाश्रमम्} %॥३२॥

\twolineshloka
{समीपे दक्षिणे भागे नानापक्षिनिषेविते}
{वसन्तं राघवं तत्र सीतया सह सुन्दरम्} %॥३३॥

\twolineshloka
{मन्मथाकारसदृशं कथयन्तं महाकथाः}
{कृत्वा मायामयं रुपं लावण्यगुणसंयुतम्} %॥३४॥

\twolineshloka
{मदनाक्रान्तहदया कदाचिद्रावणानुजा}
{गायन्ती सुस्वरं गीतं शनैरागत्य राक्षसी} %॥३५॥

\twolineshloka
{ददर्श राममासीनं कानने सीतया सह}
{अथ शूर्पणखा घोरा मायारुपधरा शुभा} %॥३६॥

\twolineshloka
{निश्शङ्का दुष्टचित्ता सा राघवं प्रत्यभाषत}
{भज मां कान्त कल्याणीं भजन्तीं कामिनीमिह} %॥३७॥

\twolineshloka
{भजमानां त्यजेद्यस्तु तस्य दोषो महान् भवेत्}
{इत्युक्तः शूर्पणखया रामस्तामाह पार्थिवः} %॥३८॥

\twolineshloka
{कलत्रवानहं बाले कनीयांसं भजस्व मे}
{इति श्रुत्वा ततः प्राह राक्षसी कामरुपिणी} %॥३९॥

\twolineshloka
{अतीव निपुणा चाहं रतिकर्मणि राघव}
{त्यक्त्वैनामनभिज्ञां त्वं सीतां मां भज शोभनाम्} %॥४०॥

\twolineshloka
{इत्याकर्ण्य वचः प्राह रामस्तां धर्मतत्परः}
{परस्त्रियं न गच्छेऽहं त्वमितो गच्छ लक्ष्मणम्} %॥४१॥

\twolineshloka
{तस्य नात्र वने भार्या त्वामसौ सग्रहीष्यति}
{इत्युक्ता सा पुनः प्राह रामं राजीवलोचनम्} %॥४२॥

\twolineshloka
{यथा स्याल्लक्ष्मणो भर्ता तथा त्वं देहि पत्रकम्}
{तथैवमुक्त्वा मतिमान् रामः कमललोचनः} %॥४३॥

\twolineshloka
{छिन्ध्यस्या नासिकामिति मोक्तव्या नात्र संशयः}
{इति रामो महाराजो लिख्य पत्रं प्रदत्तवान्} %॥४४॥

\twolineshloka
{सा गृहीत्वा तु तत्पत्रं गत्वा तस्मान्मुदान्विता}
{गत्वा दत्तवती तद्वल्लक्ष्मणाय महात्मने} %॥४५॥

\twolineshloka
{तां दृष्ट्वा लक्ष्मणः प्राह राक्षसीं कामरुपिणीम्}
{न लङ्घ्यं राघववचो मया तिष्ठात्मकश्मले} %॥४६॥

\twolineshloka
{तां प्रगृह्य ततः खङ्गमुद्यम्य विमलं सुधीः}
{तेन तत्कर्णनासां तु चिच्छेद तिलकाण्डवत्} %॥४७॥

\twolineshloka
{छिन्ननासा ततः सा तु रुरोद भृशदुः खिता}
{हा दशास्य मम भ्रातः सर्वदेवविमर्दक} %॥४८॥

\twolineshloka
{हा कष्टं कुम्भकर्णाद्यायाता मे चापदा परा}
{हा हा कष्टं गुणनिधे विभीषण महामते} %॥४९॥

\twolineshloka
{इत्येवमार्ता रुदती सा गत्वा खरदूषणौ}
{त्रिशिरसं च सा दृष्ट्वा निवेद्यात्मपराभवम्} %॥५०॥

\twolineshloka
{राममाह जनस्थाने भ्रात्रा सह महाबलम्}
{ज्ञात्वा ते राघवं क्रुद्धाः प्रेषयामासुरुर्जितान्} %॥५१॥

\twolineshloka
{चतुर्दशसहस्त्राणि राक्षसानां बलीयसाम्}
{अग्रे निजग्मुस्तेनैव रक्षसां नायकास्त्रयः} %॥५२॥

\twolineshloka
{रावणेन नियुक्तास्ते पुरैव तु महाबलाः}
{महाबलपरीवारा जनस्थानमुपागताः} %॥५३॥

\twolineshloka
{क्रोधेन महताऽऽविष्टा दृष्ट्वा तां छिन्ननासिकाम्}
{रुदतीमश्रुदिग्धाङ्गीं भगिनीं रावणस्य तु} %॥५४॥

\twolineshloka
{रामोऽपि तद्वलं दृष्ट्वा राक्षसानां बलीयसाम्}
{संस्थाप्य लक्ष्मणं तत्र सीताया रक्षणं प्रति} %॥५५॥

\twolineshloka
{गत्वा तु प्रहितैस्तत्र राक्षसैर्बलदर्पितैः}
{चतुर्दशसहस्त्रं तु राक्षसानां महाबलम्} %॥५६॥

\twolineshloka
{क्षणेन निहतं तेन शरैरग्निशिखोपमैः}
{खरश्च निहतस्तेन दूषणश्च महाबलः} %॥५७॥

\twolineshloka
{त्रिशिराश्च महारोषाद रणे रामेण पातितः}
{हत्वा तान् राक्षसान् दुष्टान् रामश्चाश्रममाविशत्} %॥५८॥

\twolineshloka
{शूर्पणखा च रुदती रावणान्तिकमागता}
{छिन्ननासां च तां दृष्ट्वा रावणो भगिनीं तदा} %॥५९॥

\twolineshloka
{मारीचं प्राह दुर्बुद्धिः सीताहरणकर्मणि}
{पुष्पकेण विमानेन गत्वाहं त्वं च मातुल} %॥६०॥

\twolineshloka
{जनस्थानसमीपे तु स्थित्वा तत्र ममाज्ञया}
{सौवर्णमृगरुपं त्वमास्थाय तु शनैः शनैः} %॥६१॥

\twolineshloka
{गच्छ त्वं तत्र कार्यार्थं यत्र सीता व्यवस्थिता}
{दृष्ट्वा सा मृगपोतं त्वां सौवर्णं त्वयि मातुल} %॥६२॥

\twolineshloka
{स्पृहां करिष्यते रामं प्रेषयिष्यति बन्धने}
{तद्वाक्यात्तत्र गच्छन्तं धावस्व गहने वने} %॥६३॥

\twolineshloka
{लक्ष्मणस्यापकर्षार्थं वक्तव्यं वागुदीरणम्}
{ततः पुष्पकमारुह्य मायारुपेण चाप्यहम्} %॥६४॥

\twolineshloka
{तां सीतामहमानेष्ये तस्यामासक्तमानसः}
{त्वमपि स्वेच्छया पश्चादागमिष्यसि शोभन} %॥६५॥

\twolineshloka
{इत्युक्ते रावणेनाथ मारीचो वाक्यमब्रवीत्}
{त्वमेव गच्छ पापिष्ठ नाहं गच्छामि तत्र वै} %॥६६॥

\twolineshloka
{पुरैवानेन रामेण व्यथितोऽहं मुनेर्मखे}
{इत्युक्तवति मारीचे रावणः क्रोधमूर्च्छितः} %॥६७॥

\twolineshloka
{मारीचं हन्तुमारेभे मारीचोऽप्याह रावणम्}
{तव हस्तवधाद्वीर रामेण मरणं वरम्} %॥६८॥

\twolineshloka
{अहं गमिष्यामि तत्र यत्र त्वं नेतुमिच्छसि}
{अथ पुष्पकमारुह्य जनस्थानमुपागतः} %॥६९॥

\twolineshloka
{मारीचस्तत्र सौवर्णं मृगमास्थाय चाग्रतः}
{जगाम यत्र सा सीता वर्तते जनकात्मजा} %॥७०॥

\twolineshloka
{सौवर्णं मृगपोतं तु दृष्ट्वा सीता यशस्विनी}
{भाविकर्मवशाद्रामुवाच पतिमात्मनः} %॥७१॥

\twolineshloka
{गृहीत्वा देहि सौवर्णं मृगपोतं नृपात्मज}
{अयोध्यायां तु मद्रेहे क्रीडनार्थमिदं मम} %॥७२॥

\twolineshloka
{तयैवमुक्तो रामस्तु लक्ष्मणं स्थाप्य तत्र वै}
{रक्षणार्थ तु सीताया गतोऽसौ मृगपृष्ठतः} %॥७३॥

\twolineshloka
{रामेण चानुयातोऽसौ अभ्यधावद्वने मृगः}
{ततः शरेण विव्याध रामस्तं मृगपोतकम्} %॥७४॥

\twolineshloka
{हा लक्ष्मणेति चोक्त्वासौ निपपात महीतले}
{मारीचः पर्वताकारस्तेन नष्टो बभूव सः} %॥७५॥

\twolineshloka
{आकर्ण्य रुदतः शब्दं सीता लक्ष्मणमब्रवीत्}
{गच्छ लक्ष्मण पुत्र त्वं यत्रायं शब्द उत्थितः} %॥७६॥

\twolineshloka
{भ्रातुर्ज्येष्ठस्य तत्त्वं वै रुदतः श्रूयते ध्वनिः}
{प्रायो रामस्य सन्देहं लक्षयेऽहं महात्मनः} %॥७७॥

\twolineshloka
{इत्युक्तः स तथा प्राह लक्ष्मणस्तामनिन्दिताम्}
{न हि रामस्य सन्देहो न भयं विद्यते क्वचित्} %॥७८॥

\twolineshloka
{इति ब्रुवाणं तं सीता भाविकर्मबलाद्भृतम्}
{लक्ष्मणं प्राह वैदेही विरुद्धवचनं तदा} %॥७९॥

\twolineshloka
{मृते रामे तु मामिच्छन्नतस्त्वं न गामिष्यसि}
{इत्युक्तः स विनीतात्म असहन्नप्रियं वचः} %॥८०॥

\twolineshloka
{जगाम राममन्वेष्टुं तदा पार्थिवनन्दनः}
{सन्न्यासवेषमास्थाय रावणोऽपि दुरात्मवान्} %॥८१॥

\twolineshloka
{स सीतापार्श्वमासाद्य वचनं चेदमुक्तवान्}
{आगतो भरतः श्रीमानयोध्याया महामतिः} %॥८२॥

\twolineshloka
{रामेण सह सम्भाष्य स्थितवांस्तत्र कानने}
{मां च प्रेषितवान् रामो विमानमिदमारुह} %॥८३॥

\twolineshloka
{अयोध्यां याति रामस्तु भरतेन प्रसादितः}
{मृगबालं तु वैदेहि क्रीडार्थं ते गृहीतवान्} %॥८४॥

\twolineshloka
{क्लेशितासि महारण्ये बहुकालं त्वमीदृशम्}
{सम्प्राप्तराज्यस्ते भर्ता रामः स रुचिराननः} %॥८५॥

\twolineshloka
{लक्ष्मणश्च विनीतात्मा विमानमिदमारुह}
{इत्युक्ता सा तथा गत्वा नीता तेन महात्मना} %॥८६॥

\twolineshloka
{आरुरोह विमानं तु छद्मना प्रेरिता सती}
{तज्जगाम ततः शीघ्रं विमानं दक्षिणां दिशम्} %॥८७॥

\twolineshloka
{ततः सीता सुदुःखार्ता विललाप सुदुःखिता}
{विमाने खेऽपि रोदन्त्याश्चक्रे स्पर्शं न राक्षसः} %॥८८॥

\twolineshloka
{रावणः स्वेन रुपेण बभूवाथ महातनुः}
{दशग्रीवं महाकायं दृष्ट्वा सीता सुदुःखिता} %॥८९॥

\twolineshloka
{हा राम वञ्चिताद्याहं केनापिच्छद्मरुपिणा}
{रक्षसा घोररुपेण त्रायस्वेति भयार्दिता} %॥९०॥

\twolineshloka
{हे लक्ष्मण महाबाहो मां हि दुष्टेन रक्षसा}
{द्रुतमागत्य रक्षस्व नीयमानामथाकुलाम्} %॥९१॥

\twolineshloka
{एवं प्रलपमानायाः सीतायास्तन्महस्त्स्वनम्}
{आकर्ण्य गृध्रराजस्तु जटायुस्तत्र चागतः} %॥९२॥

\twolineshloka
{तिष्ठ रावण दुष्टात्मन मुञ्च मुञ्चात्र मैथिलीम्}
{इत्युक्त्वा युयुधे तेन जटायुस्तत्र वीर्यवान्} %॥९३॥

\twolineshloka
{पक्षाभ्यां ताडयामास जटायुस्तस्य वक्षसि}
{ताडयन्तं तु तं मत्वा बलवानिति रावणः} %॥९४॥

\twolineshloka
{तुण्डचञ्जुप्रहारैस्तु भृशं तेन प्रपीडितः}
{तत उत्थाप्य वेगेन चन्द्रहासमसिं महत्} %॥९५॥

\twolineshloka
{जघान तेन दुष्टात्मा जटायुं धर्मचारिणम्}
{निपपात महीपृष्ठे जटायुः क्षीणचेतनः} %॥९६॥

\twolineshloka
{उवाच च दशग्रीवं दुष्टात्मन् न त्वया हतः}
{चन्द्रहासस्य वीर्येण हतोऽहं राक्षसाधम} %॥९७॥

\twolineshloka
{निरायुधं को हनेन्मूढ सायुधस्त्वामृते जनः}
{सीतापहरणं विद्धि मृत्युस्ते दुष्ट राक्षस} %॥९८॥

\twolineshloka
{दुष्ट रावण रामस्त्वां वधिष्यति न संशयः}
{रुदती दुःखशोकार्ता जटायुं प्राह मैथिली} %॥९९॥

\twolineshloka
{मत्कृते मरणं यस्मात्त्वया प्राप्तं द्विजोत्तम}
{तस्माद्रामप्रसादेन विष्णुलोकमवाप्स्यसि} %॥१००॥

\twolineshloka
{यावद्रामेण सङ्गस्ते भविष्यति महाद्विज}
{तावत्तिष्ठन्तु ते प्राणा इत्युक्त्वा तु खगोत्तमम्} %॥१०१॥

\twolineshloka
{ततस्तान्यर्पितान्यङ्गाद्भूषणानि विमुच्य सा}
{शीघ्रं निबध्य वस्त्रेण रामहस्तं गमिष्यथ} %॥१०२॥

\twolineshloka
{इत्युक्त्वा पातयामास भूमौ सीता सुदुःखिता}
{एवं हत्वा स सीतां तु जटायुं पात्य भूतले} %॥१०३॥

\twolineshloka
{पुष्पकेण गतः शीघ्रं लङ्कां दुष्टनिशाचरः}
{अशोकवनिकामध्ये स्थापयित्वा स मैथिलीम्} %॥१०४॥

\twolineshloka
{इमामत्रैव रक्षध्वं राक्षस्यो विकृताननाः}
{इत्यादिश्य गृहं यातो रावणो राक्षसेश्वरः} %॥१०५॥

\twolineshloka
{लङ्कानिवासिनश्चोचुरेकान्तं च परस्परम्}
{अस्याः पुर्या विनाशार्थं स्थापितेयं दुरात्मना} %॥१०६॥

\twolineshloka
{राक्षसीभिर्विरुपाभी रक्ष्यमाणा समन्ततः}
{सीता च दुःखिता तत्र स्मरन्ती राममेव सा} %॥१०७॥

\twolineshloka
{उवास सा सुदुःखार्ता दुःखिता रुदती भृशम्}
{यथा ज्ञानखले देवी हंसयाना सरस्वती} %॥१०८॥

\twolineshloka
{सुग्रीवभृत्या हरयश्चतुरश्च यदृच्छया}
{वस्त्रबद्धं तयोत्सृष्टं गृहीत्वा भूषणं द्रुतम्} %॥१०९॥

\twolineshloka
{स्वभर्त्रे विनिवेद्योचुः सुग्रीवाय महात्मने}
{अरण्येऽभून्महायुद्धं जटायो रावणस्य च} %॥११०॥

\twolineshloka
{अथ रामश्च तं हत्वा मारीचं माययाऽऽगतम्}
{निवृत्तो लक्ष्मणं दृष्ट्वा तेन गत्वा स्वमाश्रमम्} %॥१११॥

\twolineshloka
{सीतामपश्यन्दुः खार्तः प्ररुरोद स राघवः}
{लक्ष्मणश्च महातेजा रुरोद भृशदुःखितः} %॥११२॥

\twolineshloka
{बहुप्रकारमस्वस्थं रुदन्तं राघवं तदा}
{भूतले पतितं धीमानुत्थाप्याश्वास्य लक्ष्मणः} %॥११३॥

\twolineshloka
{उवाच वचनं प्राप्तं तदा यत्तच्छृणुष्व मे}
{अतिवेलं महाराज न शोकं कर्तुमर्हसि} %॥११४॥

\twolineshloka
{उत्तिष्ठोत्तिष्ठ शीघ्रं त्वं सीतां मृगयितुं प्रभो}
{इत्येवं वदता तेन लक्ष्मणेन महात्मना} %॥११५॥

\twolineshloka
{उत्थापितो नरपतिर्दुःखितो दुःखितेन तु}
{भ्रात्रा सह जगामाथ सीतां मृगयितुं वनम्} %॥११६॥

\fourlineindentedshloka
{वनानि सर्वाणि विशोध्य राघवो}
{गिरीन् समस्तान् गिरिसानुगोचरान्}
{तथा मुनीनामपि चाश्रमान् बहूंस्-}
{तृणादिवल्लीगहनेषु भूमिषु} %॥११७॥

\fourlineindentedshloka
{नदीतटे भूविवरे गुहायां}
{निरीक्षमाणोऽपि महानुभावः}
{प्रियामपश्यन् भृशदुःखितस्तदा}
{जटायुषं वीक्ष्य च घातितं नृपः} %॥११८॥

\fourlineindentedshloka
{अहो भवान् केन हतस्त्वमीदृशीं}
{दशामवाप्तोऽसि मृतोऽसि जीवसि}
{ममाद्य सर्वं समदुःखितस्य भोः}
{पत्नीवियोगादिह चागतस्य वै} %॥११९॥

\fourlineindentedshloka
{इत्युक्तमात्रे विहगोऽथ कृच्छ्रा-}
{दुवाच वाचं मधुरां तदानीम्}
{श्रृणुष्व राजन् मम वृत्तमत्र}
{वदामि दृष्टं च कृतं च सद्यः} %॥१२०॥

\fourlineindentedshloka
{दशाननस्तामपनीय मायया}
{सीतां समारोप्य विमानमुत्तमम्}
{जगाम खे दक्षिणदिङ्मुखोऽसौ}
{सीता च माता विललाप दुःखिता} %॥१२१॥

\fourlineindentedshloka
{आकर्ण्य सीतास्वनमागतोऽहं}
{सीतां विमोक्तुं स्वबलेन राघव}
{युद्धं च तेनाहमतीव कृत्वा}
{हतः पुनः खङ्गबलेन रक्षसा} %॥१२२॥

\fourlineindentedshloka
{वैदेहिवाक्यादिह जीवता मया}
{दृष्टो भवान् स्वर्गमितो गमिष्ये}
{मा राम शोकं कुरु भूमिपाल}
{जह्यद्य दुष्टं सगणं तु नैऋतम्} %॥१२३॥

\twolineshloka
{रामो जटायुषेत्युक्तः पुनस्तं चाह शोकतः}
{स्वस्त्यस्तु ते द्विजवर गतिस्तु परमास्तु ते} %॥१२४॥

\twolineshloka
{ततो जटायुः स्वं देहं विहाय गतवान्दिवन्}
{विमानेन तु रम्येण सेव्यमानोऽप्सरोगणैः} %॥१२५॥

\twolineshloka
{रामोऽपि दग्ध्वा तद्देहं स्नातो दत्त्वा जलाञ्जलिम्}
{भ्रात्रा सगच्छन् दुःखार्तो राक्षसी पथि दृष्टवान्} %॥१२६॥

\twolineshloka
{उद्वमन्तीं महोल्काभां विवृतास्यां भयकरीम्}
{क्षयं नयन्तीं जन्तून् वै पातयित्वा गतो रुषा} %॥१२७॥

\twolineshloka
{गच्छन् वनान्तरं रामः स कबन्धं ददर्श ह}
{विरुपं जठरमुखं दीर्घबाहुं घनस्तनम्} %॥१२८॥

\twolineshloka
{रुन्धानं राममार्गं तु दृष्ट्वा तं दग्धवाज्शनैः}
{दग्धोऽसौ दिव्यरुपी तु खस्थो राममभाषत} %॥१२९॥

\twolineshloka
{राम राम महाबाहो त्वया मम महामते}
{विरुपं नाशितं वीर मुनिशापाच्चिरागतम्} %॥१३०॥

\twolineshloka
{त्रिदिवं यामि धन्योऽस्मि त्वत्प्रसादान्न संशयः}
{त्वं सीताप्राप्तये सख्यं कुरु सूर्यसुतेन भोः} %॥१३१॥

\twolineshloka
{वानरेन्द्रेण गत्वा तु सुग्रीवे स्वं निवेद्य वै}
{भविष्यति नृपश्रेष्ठ ऋष्यमूकगिरि व्रज} %॥१३२॥

\twolineshloka
{इत्युक्त्वा तु गते तस्मिन् रामो लक्ष्मणसंयुतः}
{सिद्धैस्तु मुनिभिः शून्यमाश्रमं प्रविवेश ह} %॥१३३॥

\twolineshloka
{तत्रस्थां तापसीं दृष्ट्वा तया संलाप्य संस्थितः}
{शबरीं मुनिमुख्यानां सपर्याहतकल्मषाम्} %॥१३४॥

\twolineshloka
{तया सम्पूजितो रामो बदरादिभिरीश्वरः}
{साप्येनं पूजयित्वा तु स्वामवस्थां निवेद्य वै} %॥१३५॥

\twolineshloka
{सीतां त्वं प्राप्स्यसीत्युक्त्वा प्रविश्याग्निं दिवगता}
{दिवं प्रस्थाप्य तां चापि जगामान्यत्र राघवः} %॥१३६॥

\fourlineindentedshloka
{ततो विनीतेन गुणान्वितेन}
{भ्रात्रा समेतो जगदेकनाथः}
{प्रियावियोगेन सुदुःखितात्मा}
{जगाम याम्यां स तु रामदेवः} %॥१३७॥

॥इति श्रीनरसिंहपुराणे रामप्रादुर्भावे एकोनपञ्चाशोऽध्यायः॥४९॥

\sect{पञ्चाशत्तमोऽध्यायः --- किष्किन्धा-काण्डः}

\uvacha{मार्कण्डेय उवाच}

\twolineshloka
{बालिना कृतवैरोऽथ दुर्गवर्ती हरीश्वरः}
{सुग्रीवो दृष्टवान् दूराद् दृष्ट्वाऽऽह पवनात्मजम्} %॥१॥

\twolineshloka
{कस्येमौ सुधनुः पाणी चीरवल्कलधारिणौ}
{पश्यन्तौ सरसीं दिव्यां पद्मोत्पलसमावृताम्} %॥२॥

\twolineshloka
{नानारुपधरावेतौ तापसं वेषमास्थितौ}
{बालिदूताविह प्राप्ताविति निश्चित्य सूर्यजः} %॥३॥

\twolineshloka
{उत्पपात भयत्रस्तः ऋष्यमूकाद् वनान्तरम्}
{वानरैः सहितः सर्वैरगस्त्यश्रममुत्तमम्} %॥४॥

\twolineshloka
{तत्र स्थित्वा स सुग्रीवः प्राह वायुसुतं पुनः}
{हनूमन् पृच्छ शीघ्रं त्वं गच्छ तापसवेषधृक्} %॥५॥

\twolineshloka
{कौ हि कस्य सुतौ जातौ किमर्थं तत्र संस्थितौ}
{ज्ञात्वा सत्यं मम ब्रूहि वायुपुत्र महामते} %॥६॥

\twolineshloka
{इत्युक्तो हनुमान् गत्वा पम्पातटमनुत्तमम्}
{भिक्षुरुपी स तं प्राह रामं भ्रात्रा समन्वितम्} %॥७॥

\twolineshloka
{को भवानिह सम्प्राप्तस्तथ्यं ब्रूहि महामते}
{अरण्ये निर्जने घोरे कुतस्त्वं किं प्रयोजनम्} %॥८॥

\twolineshloka
{एवं वदन्तं तं प्राह लक्ष्मणो भ्रातुराज्ञया}
{प्रवक्ष्यामि निबोध त्वं रामवृत्तान्तमादितः} %॥९॥

\twolineshloka
{राजा दशरथो नाम बभूव भुवि विश्रुतः}
{तस्य पुत्रो महाबुद्धे रामो ज्येष्ठो ममाग्रजः} %॥१०॥

\twolineshloka
{अस्याभिषेक आरब्धः कैकेय्या तु निवारितः}
{पितुराज्ञामयं कुर्वन् रामो भ्राता ममाग्रजः} %॥११॥

\twolineshloka
{मया सह विनिष्क्रम्य सीतया सह भार्यया}
{प्रविष्टो दण्डकारण्यं नानामुनिसमाकुलम्} %॥१२॥

\twolineshloka
{जनस्थाने निवसतो रामस्यास्य महात्मनः}
{भार्या सीता तत्र वने केनापि पाप्पना हता} %॥१३॥

\twolineshloka
{सीतामन्वेषयन् वीरो रामः कमललोचनः}
{इहायातस्त्वया दृष्ट इति वृत्तान्तमीरितम्} %॥१४॥

\twolineshloka
{श्रुत्वा ततो वचस्तस्य लक्ष्मणस्य महात्मनः}
{अव्याञ्जितात्मा विश्वासाद्धनूमान् मारुतात्मजः} %॥१५॥

\twolineshloka
{त्वं मे स्वामी इति वदन् रामं रघुपतिं तदा}
{आश्वास्यानीय सुग्रीवं तयोः सख्यमकारयत्} %॥१६॥

\twolineshloka
{शिरस्यारोप्य पादाब्जं रामस्य विदितात्मनः}
{सुग्रीवो वानरेन्द्रस्तु उवाच मधुराक्षरम्} %॥१७॥

\twolineshloka
{अद्यप्रभृति राजेन्द्र त्वं मे स्वामी न संशयः}
{अहं तु तव भृत्यश्च वानरैः सहितः प्रभो} %॥१८॥

\twolineshloka
{त्वच्छत्रुर्मम शत्रुः स्यादद्यप्रभृति राघव}
{मित्रं ते मम सन्मित्रं त्वददुःखं तन्ममापि च} %॥१९॥

\twolineshloka
{त्वत्प्रीतिरेव मत्प्रीतिरित्युक्त्वा पुनराह तम्}
{वाली नाम मम ज्येष्ठो महाबलपराक्रमः} %॥२०॥

\twolineshloka
{भार्यापहारी दुष्टात्मा मदनासक्तमानसः}
{त्वामृते पुरुषव्याघ्र नास्ति हन्ताद्य वालिनम्} %॥२१॥

\twolineshloka
{युगपत्सप्ततालांस्तु तरुन् यो वै वधिष्यति}
{स तं वधिष्यतीत्युक्तं पुराणज्ञैर्नृपात्मज} %॥२२॥

\twolineshloka
{तत्प्रियार्थं हि रामोऽपि श्रीमांश्छित्त्वा महातरुन्}
{अर्धाकृष्टेन बाणेन युगप्रदघुनन्दनः} %॥२३॥

\twolineshloka
{विदध्वा महातरुन् रामः सुग्रीवं प्राह पार्थिवम्}
{वालिना गच्छ युध्यस्व कृतचिह्नो रवेः सुत} %॥२४॥

\twolineshloka
{इत्युक्तः कृतचिह्नोऽयं युद्धं चक्रेऽथ वालिना}
{रामोऽपि तत्र गत्वाथ शरेणैकेन वालिनम्} %॥२५॥

\twolineshloka
{विव्याध वीर्यवान् वाली पपात च ममार च}
{वित्रस्तं वालिपुत्रं तु अङ्गदं विनयान्वितम्} %॥२६॥

\twolineshloka
{रणशौण्डं यौवराज्ये नियुक्त्वा राघवस्तदा}
{तां च तारां तथा दत्त्वा रामश्च रविसूनवे} %॥२७॥

\twolineshloka
{सुग्रीवं प्राहं धर्मात्मा रामः कमललोचनः}
{राज्यमन्वेषय स्वं त्वं कपीनां पुनराव्रज} %॥२८॥

\twolineshloka
{त्वं सीतान्वेषणे यत्नं कुरु शीघ्रं हरीश्वर}
{इत्युक्तः प्राह सुग्रीवो रामं लक्ष्मणसंयुतम्} %॥२९॥

\twolineshloka
{प्रावृट्कालो महान् प्राप्तः साम्प्रतं रघुनन्दन}
{वानराणां गतिर्नास्ति वने वर्षति वासवे} %॥३०॥

\twolineshloka
{गते तमिंस्तु राजेन्द्र प्राप्ते शरदि निर्मले}
{चारान् सम्प्रेषयिष्यामि वानरान्दिक्षु राघव} %॥३१॥

\twolineshloka
{इत्युक्त्वा रामचन्द्रं स तं प्रणम्य कपीश्वरः}
{पम्पापुरं प्रविश्याथ रेमे तारासमन्वितः} %॥३२॥

\twolineshloka
{रामोऽपि विधिवदभ्रात्रा शैलसानौ महावने}
{निवासं कृतवान् शैले नीलकण्ठे महामतिः} %॥३३॥

\twolineshloka
{प्रावृट्काले गते कृच्छ्रात् प्राप्ते शरदि राघवः}
{सीतावियोगाद्व्यथितः सौमित्रिं प्राह लक्ष्मणम्} %॥३४॥

\twolineshloka
{उल्लङ्घितस्तु समयः सुग्रीवेण ततो रुषा}
{लक्ष्मणं प्राह काकुत्स्थो भ्रातरं भ्रातृवत्सलः} %॥३५॥

\twolineshloka
{गच्छ लक्ष्मण दुष्टोऽसौ नागतः कपिनायकः}
{गते तु वर्षाकालेऽहमागमिष्यामि तेऽन्तिकम्} %॥३६॥

\twolineshloka
{अनेकैर्वानरैः सार्धमित्युक्त्वासौ तदा गतः}
{तत्र गच्छ त्वरा युक्तो यत्रास्ते कपिनायकः} %॥३७॥

\twolineshloka
{तं दुष्टमग्रतः कृत्वा हरिसेनासमन्वितम्}
{रमन्तं तारया सार्धं शीघ्रमानय मां प्रति} %॥३८॥

\twolineshloka
{नात्रागच्छति सुग्रीवो यद्यसौ प्राप्तभूतिकः}
{तदा त्वयैवं वक्तव्यः सुग्रीवोऽनृतभाषकः} %॥३९॥

\twolineshloka
{वालिहन्ता शरो दुष्ट करे मेऽद्यापि तिष्ठति}
{स्मृत्वैतदाचर कपे रामवाक्यं हितं तव} %॥४०॥

\threelineshloka
{इत्युक्तस्तु तथेत्युक्त्वा रामं नत्वा च लक्ष्मणः}
{पम्पापुरं जगामाथ सुग्रीवो यत्र तिष्ठति}
{दृष्ट्वा स तत्र सुग्रीवं कपिराजं बभाष वै} %॥४१॥

\twolineshloka
{ताराभोगविषक्तस्त्वं रामकार्यपराङ्मुखः}
{किं त्वया विस्मृतं सर्वं रामाग्रे समयं कृतम्} %॥४२॥

\twolineshloka
{सीतामन्विष्य दास्यामि यत्र क्वापीति दुर्मते}
{हत्वा तु वालिनं राज्यं येन दत्तं पुरा तव} %॥४३॥

\twolineshloka
{त्वामृते कोऽवमन्येत कपीन्द्र पापचेतस}
{प्रतिश्रुत्य च रामस्य भार्याहीनस्य भूपते} %॥४४॥

\twolineshloka
{साहाय्यं ते करोमिति देवाग्निजलसन्निधौ}
{ये ये च शत्रवो राजंस्ते ते च मम शत्रवः} %॥४५॥

\twolineshloka
{मित्राणि यानि ते देव तानि मित्राणि मे सदा}
{सीतामन्वेषितुं राजन् वानरैर्बहुभिर्वृतः} %॥४६॥

\twolineshloka
{सत्यं यास्यामि ते पार्श्वमित्युक्त्वा कोऽन्यथाकरोत्}
{त्वामृते पापिनं दुष्टं रामदेवस्य सन्निधौ} %॥४७॥

\twolineshloka
{कारयित्वा तु तेनैवं स्वकार्यं दुष्टवानर}
{ऋषीणां सत्यवद्वाक्यं त्वयि दृष्टं मयाधुना} %॥४८॥

\twolineshloka
{सर्वस्य हि कृतार्थस्य मतिरन्या प्रवर्तते}
{वत्सः क्षीरक्षयं दृष्ट्वा परित्यजति मातरम्} %॥४९॥

\twolineshloka
{जनवृत्तविदां लोके सर्वज्ञानां महात्मनाम्}
{न तं पश्यामि लोकेऽस्मिन् कृतं प्रतिकरोति यः} %॥५०॥

\twolineshloka
{शास्त्रेषु निष्कृतिर्दृष्टा महापातकिनामपि}
{कृतघ्नस्य कपे दुष्ट न दृष्टा निष्कृतिः पुरा} %॥५१॥

\twolineshloka
{कृतघ्रना न कार्या ते त्वत्कृतं समयं स्मर}
{एह्येह्यागच्छ शरणं काकुत्स्थं हितपालकम्} %॥५२॥

\twolineshloka
{यदि नायासि च कपे रामवाक्यामिदं श्रृणु}
{नयिष्ये मृत्युसदनं सुग्रीवं वालिनं यथा} %॥५३॥

\twolineshloka
{स शरो विद्यतेऽस्माकं येन वाली हतः कपिः}
{लक्ष्मणेनैवमुक्तोऽसौ सुग्रीवः कपिनायकः} %॥५४॥

\twolineshloka
{निर्गत्य तु नमश्चके लक्ष्मणं मन्त्रिणोदितः}
{उवाच च महात्मानं लक्ष्मणं वानराधिपः} %॥५५॥

\twolineshloka
{अज्ञानकृतपापानामस्माकं क्षन्तुमर्हसि}
{समयः कृतो मया राज्ञा रामेणामिततेजसा} %॥५६॥

\twolineshloka
{यस्तदानीं महाभाग तमद्यापि न लङ्घये}
{यास्यामि निखिलरैद्य कपिभिर्नृपनन्दन} %॥५७॥

\twolineshloka
{त्वया सह महावीर रामपार्श्वं न संशयः}
{मां दृष्ट्वा तत्र काकुत्स्थो यद्वक्ष्यति च मां प्रति} %॥५८॥

\twolineshloka
{तत्सर्वं शिरसा गृह्य करिष्यासि न संशयः}
{सन्ति मे हरयः शूराः सीतान्वेषणकर्मणि} %॥५९॥

\twolineshloka
{तान्यहं प्रेषयिष्यामि दिक्षु सर्वासु पार्थिव}
{इत्युक्तः कपिराजेन सुग्रीवेण स लक्ष्मणः} %॥६०॥

\twolineshloka
{इहि शीघ्रं गमिष्यामो रामपार्श्वमितोऽधुना}
{सेना चाहूयतां वीर ऋक्षाणां हरिणामपि} %॥६१॥

\twolineshloka
{यां दृष्ट्वा प्रीतिमभ्येति राघवस्ते महामते}
{इत्युक्तो लक्ष्मणेनाथ सुग्रीवः स तु वीर्यवान्} %॥६२॥

\twolineshloka
{पार्श्वस्यं युवराजानमङ्गदं सज्ञयाब्रवीत्}
{सोऽपि निर्गत्य सेनानीमाह सेनापतिं तदा} %॥६३॥

\twolineshloka
{तेनाहूताः समागत्य ऋक्षवानरकोटयः}
{गुहास्थाश्च गिरिस्थाश्च वृक्षस्थाश्चैव वानराः} %॥६४॥

\twolineshloka
{तैः सार्धं पर्वताकारैर्वानरैर्भीमविक्रमैः}
{सुग्रीवः शीघ्रमागत्य ववन्दे राघवं तदा} %॥६५॥

\twolineshloka
{लक्ष्मणोऽपि नमस्कृत्य रामं भ्रातरमब्रवीत्}
{प्रसादं कुरु सुग्रीवे विनीते चाधुना नृप} %॥६६॥

\twolineshloka
{इत्युक्तो राघवस्तेन भ्रात्रा सुग्रीवमब्रवीत्}
{आगच्छात्र महावीर सुग्रीव कुशलं तव} %॥६७॥

\twolineshloka
{श्रुत्वेत्थं रामवचनं प्रसन्नं च नराधिपम}
{शिरस्यञ्जलिमाधाय सुग्रीवो राममब्रवीत्} %॥६८॥

\twolineshloka
{तदा मे कुशलं राजन् सीतादेवी तव प्रभो}
{अन्विष्य तु यदा दत्ता मया भवति नान्यथा} %॥६९॥

\twolineshloka
{इत्युक्ते वचने तेन हनूमान्मारुतात्मजः}
{नत्वा रामं बभाषैनं सुग्रीवं कपिनायकम्} %॥७०॥

\twolineshloka
{श्रृणु सुग्रीव मे वाक्यं राजायं दुःखितो भृशम्}
{सीतावियोगेन च सदा नाश्नाति च फलादिकम्} %॥७१॥

\twolineshloka
{अस्य दुःखेन सततं लक्ष्मणोऽयं सुदुःखितः}
{एतयोरत्र यावस्था तां श्रुत्वा भरतोऽनुजः} %॥७२॥

\twolineshloka
{दुःखी भवति तददुः खाददुः खं प्राप्नोति तज्जनः}
{यत एवमतो राजन् सीतान्वेषणमाचर} %॥७३॥

\twolineshloka
{इत्युक्ते वचने तत्र वायुपुत्रेण धीमता}
{जाम्बवानतितेजस्वी नत्वा रामं पुरः स्थितः} %॥७४॥

\twolineshloka
{स प्राह कपिराजं तं नीतिमान् नीतिमद्वचः}
{यदुक्तं वायुपुत्रेण तत्तथेत्यवगच्छ भोः} %॥७५॥

\twolineshloka
{यत्र क्वापि स्थिता सीता रामभार्या यशस्विनी}
{पतिव्रता महाभागा वैदेही जनकात्मजा} %॥७६॥

\twolineshloka
{अद्यापि वृत्तसम्पन्ना इति मे मनसि स्थितम्}
{न हि कल्याणचित्तायाः सीतायाः केनचिद्भुवि} %॥७७॥

\twolineshloka
{पराभवोऽस्ति सुग्रीव प्रेषयाद्यैव वानरान्}
{इत्युक्तस्तेन सुग्रीवः प्रीतामा कपिनायकः} %॥७८॥

\twolineshloka
{पश्चिमायां दिशि तदा प्रेषयामास तान् कपीन्}
{अन्वेष्टुं रामभार्यां तां महाबलपराक्रमः} %॥७९॥

\twolineshloka
{उत्तरस्यां दिशि तदा नियुतान् वानरानसौ}
{प्रेषयामास धर्मात्मा सीतान्वेषणकर्मणि} %॥८०॥

\twolineshloka
{पूर्वस्यां दिशि कर्पीश्च कपिराजः प्रतापवान्}
{प्रेषयामास रामस्य सुभार्यान्वेषणाय वै} %॥८१॥

\twolineshloka
{इति तान् प्रेषयामास वानरान् वानराधिपः}
{सुग्रीवो वालिपुत्रं तमङ्गदं प्राह बुद्धिमान्} %॥८२॥

\twolineshloka
{त्वं गच्छ दक्षिणं देशं सीतान्वेषणकर्मणि}
{जाम्बवांश्च हनूमांश्च मैन्दो द्विविद एव च} %॥८३॥

\twolineshloka
{नीलाद्याश्चैव हरयो महाबलपराक्रमाः}
{अनुयास्यन्ति गच्छन्तं त्वामद्य मम शासनात्} %॥८४॥

\twolineshloka
{अचिरादेव यूयं तां दृष्ट्वा सीतां यशस्विनीम्}
{स्थानतो रुपतश्चैव शीलतश्च विशेषतः} %॥८५॥

\twolineshloka
{केन नीता च कुत्रास्ते ज्ञात्वात्रागच्छ पुत्रक}
{इत्युक्तः कपिराजेन पितृव्येण महात्मना} %॥८६॥

\twolineshloka
{अङ्गदस्तूर्णमुत्थाय तस्याज्ञां शिरसा दधे}
{इत्युक्ते दूरतः स्थाप्य वानरानथ जाम्बवान्} %॥८७॥

\twolineshloka
{रामं च लक्ष्मणं चैव सुग्रीवं मारुतात्मजम्}
{एकतः स्थाप्य तानाह नीतिमान् नीतिमद्वचः} %॥८८॥

\twolineshloka
{श्रूयतां वचनं मेऽद्य सीतान्वेषणकर्मणि}
{श्रुत्वा च तदगृहाण त्वं रोचते यन्नृपात्मज} %॥८९॥

\twolineshloka
{रावणेन जनस्थानान्नीयमाना तपस्विनी}
{जटायुषा तु सा दृष्ट्वा शक्त्या युद्धं प्रकुर्वता} %॥९०॥

\twolineshloka
{भूषणानि च दृष्टानि तया क्षिप्तानि तेन वै}
{तान्यस्माभिः प्रदृष्टानि सुग्रीवायार्पितानि च} %॥९१॥

\twolineshloka
{जटायुवाक्याद्राजेन्द्र सत्यमित्यवधारय}
{एतस्मात्कारणात्सीता नीता तेनैव रक्षसा} %॥९२॥

\twolineshloka
{रावणेन महाबाहो लङ्कायां वर्तते तु सा}
{त्वां स्मरन्ती तु तत्रस्था त्वद्दुःखेन सुदुःखिता} %॥९३॥

\twolineshloka
{रक्षन्ती यत्नतो वृत्तं तत्रपि जनकात्मजा}
{त्वद्ध्यानेनैव स्वान् प्राणान्धारयन्ती शुभानना} %॥९४॥

\twolineshloka
{स्थिता प्रायेण ते देवी सीता दुःखपरायणा}
{हितमेव च ते राजन्नुदधेर्लङ्घने क्षमम्} %॥९५॥

\twolineshloka
{वायुपुत्रं हनूमन्तं त्वमत्रादोष्टुमर्हसि}
{त्वं चाप्यर्हसि सुग्रीव प्रेषितुं मारुतात्मजम्} %॥९६॥

\twolineshloka
{तमृते सागरं गन्तुं वानराणां न विद्यते}
{बलं कस्यापि वा वीर इति मे मनसि स्थितम्} %॥९७॥

\twolineshloka
{क्रियतां मव्दचः क्षिप्रं हितं पथ्यं च नः सदा}
{उक्ते जाम्बवतैवं तु नीतिस्वल्पाक्षरान्विते} %॥९८॥

\twolineshloka
{वाक्ये वानरराजोऽसौ शीघ्रमुत्थाय चासनात्}
{वायुपुत्रसमीपं तु तं गत्वा वाक्यमब्रवीत्} %॥९९॥

\twolineshloka
{श्रृणु मद्वचनं वीर हनुमन्मारुतात्मज}
{अयमिक्ष्वाकुतिलको राजा रामः प्रतापवान्} %॥१००॥

\twolineshloka
{पितुरादेशमादाय भ्रातृभार्यासमन्वितः}
{प्रविष्टो दण्डकारण्यं साक्षाद्धर्मपरायणः} %॥१०१॥

\twolineshloka
{सर्वात्मा सर्वलोकेशो विष्णुर्मानुषरुपवान्}
{अस्य भार्या हता तेन दुष्टेनापि दुरात्मना} %॥१०२॥

\twolineshloka
{तद्वियोगजदुःखार्तो विचिन्वंस्तां वने वने}
{त्वया दृष्टो नृपः पूर्वमयं वीरः प्रतापवान्} %॥१०३॥

\twolineshloka
{एतेन सह सगम्य समयं चापि कारितम्}
{अनेन निहतः शत्रुर्मम वालिर्महाबलः} %॥१०४॥

\twolineshloka
{अस्य प्रसादेन कपे राज्यं प्राप्तं मयाधुना}
{मया च तत्प्रतिज्ञातमस्य साहाय्यकर्मणि} %॥१०५॥

\twolineshloka
{तत्सत्यं कर्तुमिच्छामि त्वद्वलान्मारुतात्मज}
{उत्तीर्य सागरं वीर दृष्टा सीतामनिन्दिताम्} %॥१०६॥

\twolineshloka
{भूयस्तर्तुं बलं नास्ति वानराणां त्वया विना}
{अतस्त्वमेव जानासि स्वामिकार्यं महामते} %॥१०७॥

\twolineshloka
{बलवान्नीतिमांश्चैव दक्षस्त्वं दौत्यकर्मणि}
{तेनैवमुक्तो हनुमान् सुग्रीवेण महात्मना} %॥१०८॥

\twolineshloka
{स्वामिनोऽर्थं न किं कुर्यामीदृशं किं नु भाषसे}
{इत्युक्तो वायुपुत्रेण रामस्तं पुरतः स्थितम्} %॥१०९॥

\twolineshloka
{प्राह वाक्यं महाबाहुर्वाष्पसम्पूर्णलोचनः}
{सीतां स्मृत्वा सुदुःखार्तः कालयुक्तममित्रजित्} %॥११०॥

\twolineshloka
{त्वयि भारं समारोप्य समुद्रतरणादिकम्}
{सुग्रीवः स्थाप्यते ह्यत्र मया सार्धं महामते} %॥१११॥

\twolineshloka
{हनुमंस्तत्र गच्छ त्वं मत्प्रीत्यै कृतनिश्चयः}
{ज्ञातीनां च तथा प्रीत्यै सुग्रीवस्य विशेषतः} %॥११२॥

\twolineshloka
{प्रायेण रक्षसा नीता भार्या मे जनकात्मजा}
{तत्र गच्छ महावीर यत्र सीता व्यवस्थिता} %॥११३॥

\twolineshloka
{यदि पृच्छति सादृश्यं मदाकारमशेषतः}
{अतो निरीक्ष्य मां भूयो लक्ष्मणं च ममानुजम्} %॥११४॥

\twolineshloka
{ज्ञात्वा सर्वाङ्गगं लक्ष्म सकलं चावयोरिह}
{नान्यथा विश्वसेत्सीता इति मे मनसि स्थितम्} %॥११५॥

\twolineshloka
{इत्युक्तो रामदेवेन प्रभञ्जनसुतो बली}
{उत्थाय तत्पुरः स्थित्वा कृताञ्जलिरुवाच तम्} %॥११६॥

\twolineshloka
{जानामि लक्षणं सर्वं युवयोस्तु विशेषतः}
{गच्छामि कपिभिः सार्धं त्वं शोकं मा कुरुष्व वै} %॥११७॥

\twolineshloka
{अन्यच्च देह्यभिज्ञानं विश्वासो येन मे भवेत्}
{सीतायास्तव देव्यास्तु राजन् राजीवलोचन} %॥११८॥

\twolineshloka
{इत्युक्तो वायुपुत्रेण रामः कमललोचनः}
{अङ्गुलीयकमुन्मुच्य दत्तवान् रामचिह्नितम्} %॥११९॥

\twolineshloka
{तदगृहीत्वा तदा सोऽपि हनुमान्मारुतात्मजः}
{रामं प्रदक्षिणीकृत्य लक्ष्मणं च कपीश्वरम्} %॥१२०॥

\twolineshloka
{नत्वा ततो जगामाशु हनुमानञ्जनीसुतः}
{सुग्रीवोऽपि च ताञ्छुत्वा वानरान् गन्तुमुद्यतान्} %॥१२१॥

\twolineshloka
{आज्ञेयानाज्ञापयति वानरान् बलदर्पितान्}
{श्रृण्वन्तु वानराः सर्वे शासनं मम भाषितम्} %॥१२२॥

\twolineshloka
{विलम्बनं न कर्तव्यं युष्माभिः पर्वतादिषु}
{द्रुतं गत्वा तु तां वीक्ष्य आगन्तव्यमनिन्दिताम्} %॥१२३॥

\twolineshloka
{रामपत्नीं महाभागां स्थास्येऽहं रामसन्निधौ}
{कर्तनं वा करिष्यामि अन्यथा कर्णनासयोः} %॥१२४॥

\twolineshloka
{एवं तान् प्रेषयित्वा तु आज्ञापूर्वं कपीश्वरः}
{अथ ते वानरा याताः पश्चिमादिषु दिक्षु वै} %॥१२५॥

\twolineshloka
{ते सानुषु समस्तेषु गिरीणामपि मूर्धसु}
{नदीतीरेषु सर्वेषु मुनीनामाश्रमेषु च} %॥१२६॥

\twolineshloka
{कन्दरेषु च सर्वेषु वनेषूपवनेषु च}
{वृक्षेषु वृक्षगुल्मेषु गुहासु च शिलासु च} %॥१२७॥

\twolineshloka
{सह्यपर्वतपार्श्वेषु विन्ध्यसागरपार्श्वयोः}
{हिमवत्यपि शैले च तथा किम्पुरुषादिषु} %॥१२८॥

\twolineshloka
{मनुदेशेषु सर्वेषु सप्तपातालकेषु च}
{मध्यदेशेषु सर्वेषु कश्मीरेषु महाबलाः} %॥१२९॥

\twolineshloka
{पूर्वदेशेषु सर्वेषु कामरुपेषु कोशले}
{तीर्थस्थानेषु सर्वेषु सप्तकोङ्कणकेषु च} %॥१३०॥

\twolineshloka
{यत्र तत्रैव ते सीतामदृष्ट्वा पुनरागताः}
{आगत्य ते नमस्कृत्य रामलक्ष्मणपादयोः} %॥१३१॥

\twolineshloka
{सुग्रीवं च विशेषेण नास्माभिः कमलेक्षणा}
{दृष्टा सीता महाभागेत्युक्त्वा तांस्तत्र तस्थिरे} %॥१३२॥

\twolineshloka
{ततस्तं दुःखितं प्राह रामदेवं कपीश्वरः}
{सीता दक्षिणदिग्भागे स्थिता द्रष्टुं वने नृप} %॥१३३॥

\twolineshloka
{शक्या वानरसिंहेन वायुपुत्रेण धीमता}
{दृष्टा सीतामिहायाति हनुमान्नात्र संशयः} %॥१३४॥

\twolineshloka
{स्थिरो भव महाबाहो राम सत्यमिदं वचः}
{लक्ष्मणोऽप्याह शकुनं तत्र वाक्यमिदं तदा} %॥१३५॥

\twolineshloka
{सर्वथा दृष्टसीतस्तु हनुमानागमिष्यति}
{इत्याश्वास्य स्थितौ तत्र रामं सुग्रीवलक्ष्मणौ} %॥१३६॥

\twolineshloka
{अथाङ्गदं पुरस्कृत्य ये गता वानरोत्तमाः}
{यत्नादन्वेषणार्थाय रामपत्नीं यशस्विनीम्} %॥१३७॥

\twolineshloka
{अदृष्ट्वा श्रममापन्नाः कृच्छ्रभूतास्तदा वने}
{भक्षणेन विहीनास्ते क्षुधया च प्रपीडिताः} %॥१३८॥

\twolineshloka
{भ्रमद्भिर्गहनेऽरण्ये क्वापि दृष्ट्वा च सुप्रभा}
{गुहानिवासिनी सिद्धा ऋषिपत्नी ह्यनिन्दिता} %॥१३९॥

\twolineshloka
{सा च तानागतान्दृष्ट्वा स्वाश्रमं प्रति वानरान्}
{आगताः कस्य यूयं तु कुतः किं नु प्रयोजनम्} %॥१४०॥

\twolineshloka
{इत्युक्ते जाम्बवानाह तां सिद्धां सुमहामतिः}
{सुग्रीवस्य वयं भृत्या आगता ह्यत्र शोभने} %॥१४१॥

\twolineshloka
{रामभार्यार्थमनघे सीतान्वेषणकर्मणि}
{कां दिग्भूता निराहारा अदृष्टा जनकात्मजाम्} %॥१४२॥

\twolineshloka
{इत्युक्ते जाम्बवत्यत्र पुनस्तानाह सा शुभा}
{जानामि रामं सीतां च लक्ष्मणं च कपीश्वरम्} %॥१४३॥

\twolineshloka
{भुञ्जीध्वमत्र मे दत्तमाहारं च कपीश्वराः}
{रामकार्यागतास्त्वत्र यूयं रामसमा मम} %॥१४४॥

\twolineshloka
{इत्युक्त्वा चामृतं तेषां योगाद्दत्वा तपस्विनी}
{भोजयित्वा यथाकामं भूयस्तानाह तापसी} %॥१४५॥

\twolineshloka
{सीतास्थानं तु जानाति सम्पातिर्नाम पक्षिराट्}
{आस्थितो वै वने सोऽपि महेन्द्रे पर्वते द्विजः} %॥१४६॥

\twolineshloka
{मार्गेणानेन हरयस्तत्र यूयं गमिष्यथ}
{स वक्ति सीतां सम्पातिर्दूरदर्शी तु यः खगः} %॥१४७॥

\twolineshloka
{तेनादिष्टं तु पन्थानं पुनरासाद्य गच्छथ}
{अवश्यं जानकीं सीतां द्रक्ष्यते पवनात्मजः} %॥१४८॥

\twolineshloka
{तयैवमुक्ताः कपयः परां प्रीतिमुपागताः}
{ह्यष्टास्तेजनमापन्नास्तां प्रणम्य प्रतस्थिरे} %॥१४९॥

\twolineshloka
{महेन्द्राद्रिं गता वीरा वानरास्तद्दिदृक्षया}
{तत्र सम्पातिमासीनं दृष्टवन्तः कपीश्वराः} %॥१५०॥

\twolineshloka
{तानुवाचाथ सम्पातिर्वानरानागतान्द्विजः}
{के यूयमिति सम्प्राप्ताः कस्य वा ब्रूत मा चिरम्} %॥१५१॥

\twolineshloka
{इत्युक्ते वानरा ऊचुर्यथावृत्तमनुक्रमात्}
{रामदूता वयं सर्वे सीतान्वेषणकर्मणि} %॥१५२॥

\twolineshloka
{प्रेषिताः कपिराजेन सुग्रीवेण महात्मना}
{त्वां द्रष्टुमिह सम्प्राप्ताः सिद्धाया वचनादद्विज} %॥१५३॥

\twolineshloka
{सीतास्थानं महाभाग त्वं नो वद महामते}
{इत्युक्तो वानरैः श्येनो वीक्षाचक्रे सुदक्षिणाम्} %॥१५४॥

\twolineshloka
{सीतां दृष्ट्वा स लङ्कायामशोकाख्ये महावने}
{स्थितेति कथितं तेज जटायुस्तु मृतस्तव} %॥१५५॥

\twolineshloka
{भ्रातेति चोचुः स स्नात्वा दत्त्वा तस्योदकाञ्जलिम्}
{योगमास्थाय स्वं देहं विससर्ज महामतिः} %॥१५६॥

\twolineshloka
{ततस्तं वानरा दग्ध्वा दत्त्वा तस्योदकाञ्जलिम्}
{गत्वा महेन्द्रश्रृङ्गं ते तमारुह्य क्षणं स्थिताः} %॥१५७॥

\twolineshloka
{सागरं वीक्ष्य ते सर्वे परस्परमथाब्रुवन्}
{रावणेनैव भार्या सा नीता रामस्य निश्चितम्} %॥१५८॥

\twolineshloka
{सम्पातिवचनादद्य सज्ञातं सकलं हि तत्}
{वानराणां तु कश्चात्र उत्तीर्य लवणोदधिम्} %॥१५९॥

\twolineshloka
{लङ्कां प्रविश्य दृष्ट्वा तां रामपत्नीं यशस्विनीम्}
{पुनश्चोदधितरणे शक्तिं ब्रूत हि शोभनाः} %॥१६०॥

\twolineshloka
{इत्युक्तो जाम्बवान् प्राह सर्वे शक्तास्तु वानराः}
{सागरोत्तरणे किन्तु कार्यमन्यस्य सम्भवेत्} %॥१६१॥

\twolineshloka
{तत्र दक्षोऽयमेवात्र हनुमानिति मे मतिः}
{कालक्षेपो न कर्तव्यो मासार्धमधिकं गतम्} %॥१६२॥

\twolineshloka
{यद्यदृष्ट्वा तु गच्छामो वैदेहीं वानरर्षभाः}
{कर्णनासादि नः स्वाङ्गं निकृन्तति कपीश्वरः} %॥१६३॥

\twolineshloka
{तस्मात् प्रार्थ्यः स चास्माभिर्वायुपुत्रस्तु मे मतिः}
{इत्युक्तास्ते तथेत्यूचुर्वानरा वृद्धवानरम्} %॥१६४॥

\twolineshloka
{ततस्ते प्रार्थयामासुर्वानराः पवनात्मजम्}
{हनुमन्तं महाप्राज्ञं दक्षं कार्येषु चाधिकम्} %॥१६५॥

\threelineshloka
{गच्छ त्वं रामभृत्यस्त्वं रावणस्य भयाय च}
{रक्षस्व वानरकुलमस्माकमञ्जनीसुत}
{इत्युक्तस्तांस्तथेत्याह वानरान् पवनात्मजः} %॥१६६॥

\fourlineindentedshloka
{रामप्रयुक्तश्च पुनः स्वभर्तृणा}
{पुनर्महेन्द्रे कपिभिश्च नोदितः}
{गन्तुं प्रचक्रे मतिमञ्जनीसुतः}
{समुद्रमुत्तीर्य निशाचरालयम्} %॥१६७॥

॥इति श्रीनरसिंहपुराणे रामप्रादुर्भावे पञ्चाशत्तमोऽध्यायः॥५०॥

\sect{एकपञ्चाशत्तमोऽध्यायः --- सुन्दर-काण्डः}

\uvacha{मार्कण्डेय उवाच}

\twolineshloka
{स तु रावणनीतायाः सीतायाः परिमार्गणम्}
{इयेष पदमन्वेष्टुं चारणाचरिते पथि} %॥१॥

\twolineshloka
{अञ्जलिं प्राङ्मुखं कृत्वा सगणायात्मयोनये}
{मनसाऽऽवन्द्य रामं च लक्ष्मणं च महारथम्} %॥२॥

\twolineshloka
{सागरं सरितश्चैव प्रणम्य शिरसा कपिः}
{ज्ञातीश्चैव परिष्वज्य कृत्वा चैव प्रदक्षिणाम्} %॥३॥

\twolineshloka
{अरिष्टं गच्छ पन्थानं पुण्यवायुनिषेवितम्}
{पुनरागमनायेति वानरैरभिपूजितः} %॥४॥

\twolineshloka
{अञ्जसा स्वं तथा वीर्यमाविवेशाथ वीर्यवान्}
{मार्गमालोकयन् दूरादूर्ध्वं प्रणिहितेक्षणः} %॥५॥

\twolineshloka
{सम्पूर्णमिव चात्मानं भावयित्वा महाबलः}
{उत्पपात गिरेः श्रृङ्गान्निष्पीड्य गिरिमम्बरम्} %॥६॥

\twolineshloka
{पितुर्मार्गेण यातस्य वायुपुत्रस्य धीमतः}
{रामकार्यपरस्यास्य सागरेण प्रचोदितः} %॥७॥

\twolineshloka
{विश्रामार्थं समुत्तस्थौ मैनाको लवणोदधेः}
{तं निरीक्ष्य निपीड्याथ रयात्सम्भाष्य सादरम्} %॥८॥

\twolineshloka
{उत्पतंश्च वने वीरः सिंहिकास्यं महाकपिः}
{आस्यप्रान्तं प्रविश्याथ वेगेनान्तर्विनिस्सृतः} %॥९॥

\twolineshloka
{निस्सृत्य गतवाञ्शीघ्रं वायुपुत्रः प्रतापवान्}
{लङ्घयित्वा तु तं देशं सागरं पवनात्मजः} %॥१०॥

\twolineshloka
{त्रिकूटशिखरे रम्ये वृक्षाग्रे निपपात ह}
{तस्मिन् स पर्वतश्रेष्ठे दिनं नीत्वा दिनक्षये} %॥११॥

\twolineshloka
{सन्ध्यामुपास्य हनुमान् रात्रौ लङ्कां शनैर्निशि}
{लङ्काभिधां विनिर्जित्य देवतां प्रविवेश ह} %॥१२॥

\twolineshloka
{लङ्कामनेकरत्नाढ्यां बह्वाश्चर्यसमन्विताम्}
{राक्षसेषु प्रसुप्तेषु नीतिमान् पवनात्मजः} %॥१३॥

\twolineshloka
{रावणस्य ततो वेश्म प्रविवेशाथ ऋद्धिमत्}
{शयानं रावणं दृष्ट्वा तल्पे महति वानरः} %॥१४॥

\twolineshloka
{नासापुटैर्घोरकारैर्विशद्भिर्वायुमोचकैः}
{तथैव दशभिर्वक्त्रैर्दंष्टोपेतैस्तु संयुतम्} %॥१५॥

\twolineshloka
{स्त्रीसहस्थैस्तु दृष्ट्वा तं नानाभरणभूषितम्}
{तस्मिन् सीतामदृष्ट्वा तु रावणस्य गृहे शुभे} %॥१६॥

\twolineshloka
{तथा शयानं स्वगृहे राक्षसानां च नायकम्}
{दुःखितो वायुपुत्रस्तु सम्पातेर्वचनं स्मरन्} %॥१७॥

\twolineshloka
{अशोकवनिकां प्राप्तो नानापुष्पसमन्विताम्}
{जुष्टां मलयजातेन चन्दनेन सुगन्धिना} %॥१८॥

\twolineshloka
{प्रविश्य शिंशपावृक्षमाश्रितां जनकात्मजाम्}
{रामपत्नीं समद्राक्षीद राक्षसीभिः सुरक्षिताम्} %॥१९॥

\twolineshloka
{अशोकवृक्षमारुह्य पुष्पितं मधुपल्लवम्}
{आसाचक्रे हरिस्तत्र सेयं सीतेति संस्मरन्} %॥२०॥

\twolineshloka
{सीतां निरीक्ष्य वृक्षाग्रे यावदास्तेऽनिलात्मजः}
{स्त्रीभिः परिवृतस्तत्र रावणस्तावदागतः} %॥२१॥

\twolineshloka
{आगत्य सीतां प्राहाथ प्रिये मां भज कामुकम्}
{भूषिता भव वैदेहि त्यज रामगतं मनः} %॥२२॥

\twolineshloka
{इत्येवं भाषमाणं तमन्तर्धाय तृणं ततः}
{प्राह वाक्यं शनैः सीता कम्पमानाथ रावणम्} %॥२३॥

\twolineshloka
{गच्छ रावण दुष्ट त्वं परदारपरायण}
{अचिराद्रामबाणास्ते पिबन्तु रुधिरं रणे} %॥२४॥

\twolineshloka
{तथेत्यक्तो भर्त्सितश्च राक्षसीराह राक्षसः}
{द्विमासाभ्यन्तरे चैनां वशीकुरुत मानुषीम्} %॥२५॥

\twolineshloka
{यदि नेच्छति मां सीता ततः खादत मानुषीम्}
{इत्युक्त्वा गतवान् दुष्टो रावणः स्वं निकेतनम्} %॥२६॥

\twolineshloka
{ततो भयेन तां प्राहू राक्षस्यो जनकात्मजाम्}
{रावणं भज कल्याणी सधनं सुखिनी भव} %॥२७॥

\twolineshloka
{इत्युक्ता प्राह ताः सीता राघवोऽलघुविक्रमः}
{निहत्य रावणं युद्धे सगणं मां नयिष्यति} %॥२८॥

\twolineshloka
{नाहमन्यस्य भार्या स्यामृते रामं रघूत्तमम्}
{स ह्यागत्य दशग्रीवं हत्वा मां पालयिष्यति} %॥२९॥

\twolineshloka
{इत्याकर्ण्य वचस्तस्या राक्षस्यो ददृशुर्भयम्}
{हन्यतां हन्यतामेषा भक्ष्यतां भक्ष्यतामियम्} %॥३०॥

\twolineshloka
{ततस्तास्त्रिजटा प्राह स्वप्ने दृष्टमनिन्दिता}
{श्रृणुध्वं दुष्टराक्षस्यो रावणस्य विनाशनः} %॥३१॥

\twolineshloka
{रक्षोभिः सह सर्वेस्तु रावणस्य मृतिप्रदः}
{लक्ष्मणेन सह भ्रात्रा रामस्य विजयप्रदः} %॥३२॥

\twolineshloka
{स्वप्नः शुभो मया दृष्टः सीतायाश्च पतिप्रदः}
{त्रिजटावाक्यमाकर्ण्य सीतापार्श्वं विसृज्य ताः} %॥३३॥

\twolineshloka
{राक्षस्यस्ता ययुः सर्वाः सीतामाहाञ्जनीसुतः}
{कीर्तयन् रामवृत्तान्तं सकलं पवनात्मजः} %॥३४॥

\twolineshloka
{तस्यां विश्वासमानीय दत्त्वा रामाङ्गुलीयकम्}
{सम्भाष्य लक्षणं सर्वं रामलक्ष्मणयोस्ततः} %॥३५॥

\twolineshloka
{महत्या सेनया युक्तः सुग्रीवः कपिनायकः}
{तेन सार्धमिहागत्य रामस्तव पतिः प्रभुः} %॥३६॥

\twolineshloka
{लक्ष्मणश्च महावीरो देवरस्ते शुभानने}
{रावणं सगणं हत्वा त्वामितोऽऽदाय गच्छति} %॥३७॥

\twolineshloka
{इत्युक्ते सा तु विश्वस्ता वायुपुत्रमथाब्रवीत्}
{कथमत्रागतो वीर त्वमुत्तीर्य महोदधितम्} %॥३८॥

\twolineshloka
{इत्याकर्ण्य वचस्तस्याः पुनस्तामाह वानरः}
{गोष्पदवन्मयोत्तीर्णः समुद्रोऽयं वरानने} %॥३९॥

\twolineshloka
{जपतो रामरामेति सागरो गोष्पदायते}
{दुःखमग्नासि वैदेहि स्थिरा भव शुभानने} %॥४०॥

\twolineshloka
{क्षिप्रं पश्यसि रामं त्वं सत्यमेतदब्रवीमि ते}
{इत्याश्वास्य सतीं सीतां दुःखितां जनकात्मजाम्} %॥४१॥

\twolineshloka
{ततश्चूडामणिं प्राप्य श्रुत्वा काकपराभवम्}
{नत्वा तां प्रस्थितो वीरो गन्तुं कृतमतिः कपिः} %॥४२॥

\twolineshloka
{ततो विमृश्य तद्भड्क्त्वा क्रीडावनमशेषतः}
{तोरणस्थो ननादोच्चै रामो जयति वीर्यवान्} %॥४३॥

\twolineshloka
{अनेकान् राक्षसान् हत्वा सेनाः सेनापतींश्च सः}
{तदा त्वक्षकुमारं तु हत्वा रावणसैनिकम्} %॥४४॥

\twolineshloka
{साश्वं ससारथिं हत्वा इन्द्रजित्तं गृहीतवान्}
{रावणस्य पुरः स्थित्वा रामं सकीर्त्य लक्ष्मणम्} %॥४५॥

\twolineshloka
{सुग्रीवं च महावीर्यं दग्ध्वा लङ्कामशेषतः}
{निर्भर्त्य्स्य रावण दुष्टं पुनः सम्भाष्य जानकीम्} %॥४६॥

\twolineshloka
{भूयः सागरमुत्तीर्य ज्ञातीनासाद्य वीर्यवान्}
{सीतादर्शनमावेद्य हनूमांश्चैव पूजितः} %॥४७॥

\twolineshloka
{वानरैः सार्धमागत्य हनुमान्मधुवनं महत्}
{निहत्य रक्षपालांस्तु पाययित्वा च तन्मधु} %॥४८॥

\twolineshloka
{सर्वे दधिमुखं पात्य हर्षितो हरिभिः सह}
{खमुत्पत्य च सम्प्राप्य रामलक्ष्मणपादयोः} %॥४९॥

\twolineshloka
{नत्वा तु हनुमांस्तत्र सुग्रीवं च विशेषतः}
{आदितः सर्वमावेद्य समुद्रतरणादिकम्} %॥५०॥

\twolineshloka
{कथयामास रामाय सीता द्रुष्टा मयेति वै}
{अशोकवनिकामध्ये सीता देवी सुदुःखिता} %॥५१॥

\twolineshloka
{राक्षसीभिः परिवृत्ता त्वां स्मरन्ती च सर्वदा}
{अश्रुपूर्णमुखी दीना तव पत्नी वरानना} %॥५२॥

\twolineshloka
{शीलवृत्तसमायुक्ता तत्रापि जनकात्मजा}
{सर्वत्रान्वेषमाणेन मया दुष्टा पतिव्रता} %॥५३॥

\twolineshloka
{मया सम्भाषिता सीता विश्वस्ता रघुनन्दन}
{अलङ्कारश्च सुमणिस्तया ते प्रेषितः प्रभो} %॥५४॥

\twolineshloka
{इत्युक्त्वा दत्तवांस्तस्मै चूडामणिमनुत्तमम्}
{इदं च वचनं तुभ्यं पल्या सम्प्रेषितं श्रृणु} %॥५५॥

\twolineshloka
{चित्रकूटे मदङ्के तु सुप्ते त्वयि महाव्रत}
{वायसाभिभवं राजंस्तत्किल स्मर्तुमर्हसि} %॥५६॥

\twolineshloka
{अल्पापराधे राजेन्द्र त्वया बलिभुजि प्रभो}
{यत्कृतं तन्न कर्तुं च शक्यं देवासुरैरपि} %॥५७॥

\threelineshloka
{ब्रह्मास्त्रं तु तदोत्सृष्टं रावणं किं न जेष्यसि}
{इत्येवमादि बहुशः प्रोक्त्वा सीता रुरोद ह}
{एवं तु दुःखिता सीता तां मोक्तुं यत्नमाचर} %॥५८॥

\fourlineindentedshloka
{इत्येवमुक्ते पवनात्मजेन}
{सीतावचस्तच्छुभभूषणं च}
{श्रुत्वा च दृष्ट्वा च रुरोद रामः}
{कपिं समालिङ्य शनैः प्रतस्थे} %॥५९॥

॥इति श्रीनरसिंहपुराणे रामप्रादुर्भावे एकपञ्चाशत्तमोऽध्यायः॥५१॥

\sect{द्विपञ्चाशोऽध्यायः --- युद्ध-काण्डः}

\uvacha{मार्कण्डेय उवाच}

\twolineshloka
{इति श्रुत्वा प्रियावार्तां वायुपुत्रेण कीर्तिताम्}
{रामो गत्वा समुद्रान्तं वानरैः सह विस्तृतैः} %॥१॥

\twolineshloka
{सागरस्य तटे रम्ये तालीवनविराजिते}
{सुग्रीवो जाम्बवांश्चाथ वानरैरतिहर्षितैः} %॥२॥

\twolineshloka
{सख्यातीतैर्वृतः श्रीमान् नक्षत्रैरिव चन्द्रमाः}
{अनुजेन च धीरेण वीक्ष्य तस्थौ सरित्पतिम्} %॥३॥

\twolineshloka
{रावणेनाथ लङ्कायां स सूक्तौ भर्त्सितोऽनुजः}
{विभीषणो महाबुद्धिः शास्त्रज्ञैर्मन्त्रिभिः सह} %॥४॥

\twolineshloka
{नरसिंहे महादेवे श्रीधरे भक्तवत्सले}
{एवं रामेऽचलां भक्तिमागत्य विनयात्तदा} %॥५॥

\twolineshloka
{कृताञ्जलिरुवाचेदं राममक्लिष्टकारिणम्}
{राम राम महाबाहो देवदेव जनार्दन} %॥६॥

\twolineshloka
{विभीषणोऽस्मि मां रक्ष अहं ते शरणं गतः}
{इत्युक्त्वा निपपाताथ प्राञ्जली रामपादयोः} %॥७॥

\twolineshloka
{विदितार्थोऽथ रामस्तु तमुत्थाप्य महामतिम्}
{समुद्रतोयैस्तं वीरमभिषिच्य विभीषणम्} %॥८॥

\twolineshloka
{लङ्काराज्यं तवैवेति प्रोक्तः सम्भाष्य तस्थिवान्}
{ततो विभीषणेनोक्तं त्वं विष्णुर्भुवनेश्वरः} %॥९॥

\twolineshloka
{अब्धिर्ददातु मार्गं ते देव तं याचयामहे}
{इत्युक्तो वानरैः सार्धं शिश्ये तत्र स राघवः} %॥१०॥

\twolineshloka
{सुप्ते रामे गतं तत्र त्रिरात्रमतितद्युतौ}
{ततः क्रुद्धो जगन्नाथो रामो राजीवलोचनः} %॥११॥

\twolineshloka
{संशोषणमपां कर्तुमस्त्रमाग्नेयमाददे}
{तदोत्थाय वचः प्राह लक्ष्मणश्च रुषान्वितम्} %॥१२॥

\twolineshloka
{क्रोधस्ते लयकर्ता हि एनं जहि महामते}
{भूतानां रक्षणार्थाय अवतारस्त्वया कृतः} %॥१३॥

\twolineshloka
{क्षन्तव्यं देवदेवेश इत्युक्त्वा धृतवान् शरम्}
{ततो रात्रित्रये याते कुद्धं राममवेक्ष्य सः} %॥१४॥

\twolineshloka
{आग्नेयास्त्राच्च सन्त्रस्तः सागरोऽभ्येत्य मूर्तिमान्}
{आह रामं महादेवं रक्ष मामपकारिणम्} %॥१५॥

\twolineshloka
{मार्गो दत्तो मया तेऽद्य कुशलः सेतुकर्मणि}
{नलश्च कथितो वीरस्तेन कारय राघव} %॥१६॥

\twolineshloka
{यावदिष्टं तु विस्तीर्ण सेतुबन्धमुत्तमम्}
{ततो नलमुखैरन्यैर्वानैररमितौजसैः} %॥१७॥

\twolineshloka
{बन्धयित्वा महासेतुं तेन गत्वा स राघवः}
{सुवेलाख्यं गिरिं प्राप्तः स्थितोऽसौ वानरैर्वृतः} %॥१८॥

\twolineshloka
{हर्म्यस्थलास्थितं दुष्टं रावणं वीक्ष्य चाङ्गध}
{रामादेशादथोत्प्लुत्य दूतकर्मसु तत्परः} %॥१९॥

\twolineshloka
{प्रादात्पादप्रहारं तु रोषाद्रावणमूर्धनि}
{विस्मितं तैः सुरगणैर्वीक्षितः सोऽतिवीर्यवान्} %॥२०॥

\twolineshloka
{साधयित्वा प्रतिज्ञां तां सुवेलं पुनरागतः}
{ततो वानरसेनाभिः सख्यातिताभिरच्युतः} %॥२१॥

\twolineshloka
{रुरोध रावणपुरीं लङ्कां तत्र प्रतापवान्}
{रामः समन्तादालोक्य प्राह लक्ष्मणमन्तिके} %॥२२॥

\fourlineindentedshloka
{तीर्णोऽर्णवः कवलितेव कपीश्वरस्य}
{सेनाभटैर्झटिति राक्षसराजधानीम्}
{यत्पौरुषोचितामिहाङ्कुरितं मया तद्}
{दैवस्य वश्यमपरं धनुषोऽथ वास्य} %॥२३॥

\onelineshloka*
{लक्ष्मणः प्राह--- कातरजनमनोऽवलम्बिना किं दैवेन।}

\fourlineindentedshloka
{यावल्ललाटशिखरं भ्रुकुटिर्नयाति}
{यावन्न कार्मुकशिखामधिरोहति ज्या}
{तावन्निशाचरपतेः पटिमानमेतु}
{त्रैलोक्यमूलविभुजेषु दर्पः} %॥२४॥


तदा लक्ष्मणः रामस्य कर्णे लगित्वा पितृवधवैरस्मरणे अथ 
तद्भक्तिवीर्यपरीक्षणाय लक्षणविज्ञानायादिश्यतामङ्गदाय दूत्यम्
रामः साधु इति भणित्वा अङ्गदं सबहुमानमवलोक्य आदिशति॥२५॥

अङ्गद! पिता ते यद्वाली बलिनि दशकण्ठे
कलितवान्नशक्तास्तद्वक्तुं वयमपि मुदा तेन पुलकः
स एव त्वं व्यावर्त्तयसि तनुजत्वेन पितृतां 
ततः किं वक्तव्यं तिलकयति सृष्टार्थपदवीम्॥२६॥

अङ्गदो मौलिमण्डलमिलत्करयुगलेन प्रणम्य---\\
यदाज्ञापयति देवः। अवधार्यताम्॥२७॥

किं प्राकारविहारतोरणवतीं लङ्कामिहैवानये
किं वा सैन्यमहं द्रुतं रघुपते तत्रैव सम्पादये
अत्यल्पं कुलपर्वतैरविरलैर्बध्नामि वा सागरं
देवादेशय किं करोमि सकलं दोर्द्दण्डसाध्यं मम॥२८॥

श्रीरामस्तद्वचनमात्रेणैव तद्भक्तिं सामर्थ्य चावेक्ष्य वदति॥२९॥

अज्ञानादथवाधिपत्यरभसा वास्मत्परोक्षे ह्नता सीतेयं
प्रविमुच्यतामिति वचो गत्वा दशास्यं वद
नो चेल्लोक्ष्मणमुक्तमार्गणगणच्छेदोच्छलच्छोणित-
च्छत्रच्छन्नदिगन्तमन्तकपुरीं पुत्रैर्वृतो यास्यसि॥३०॥

अङ्गदः--- देव!॥३१॥

\addtocounter{shlokacount}{7}
\twolineshloka
{सन्धौ वा विग्रहे वापि मयि दूते दशाननी}
{अक्षता वाक्षता वापि क्षितिपीठे लुठिष्यति} %॥३२॥

\twolineshloka
{तदा श्रीरामचन्द्रेण प्रशस्य प्रहितोऽङ्गदः}
{उक्तिप्रत्युक्तिचात्यर्यैः पराजित्यागतो रिपुम्} %॥३३॥

\twolineshloka
{राघवस्य बलं ज्ञात्वा चारैस्तदनुजस्य च}
{वानराणां च भीतोऽपि निर्भीरिव दशाननः} %॥३४॥

\twolineshloka
{लङ्कापुरस्य रक्षार्थमादिदेश स राक्षसान्}
{आदिश्य सर्वतो दिक्षु पुत्रानाह दशाननः} %॥३५॥

\threelineshloka
{धूम्राक्षं धूम्रपानं च राक्षसा यात मे पुरीम्}
{पाशैर्बध्नीत तौ मर्यौ अमित्रान्तकवीर्यवान्}
{कुम्भकर्णोऽपि मदभ्राता तुर्यनादैः प्रबोधितः} %॥३६॥

\twolineshloka
{राक्षसाश्चैव सन्दिष्टा रावणेन महाबलाः}
{तस्याज्ञां शिरसाऽऽदाय युयुधुर्वानरैः सह} %॥३७॥

\twolineshloka
{युध्यमाना यथाशक्त्या कोटिसख्यास्तु राक्षसाः}
{वानरैर्निधनं प्राप्ताः पुनरन्यान् यथाऽऽदिशत्} %॥३८॥

\twolineshloka
{पूर्वद्वारे दशग्रीवो राक्षसानमितौजसः}
{ते चापि युध्य हरिभिर्नीलाद्यैर्निधनं गताः} %॥३९॥

\twolineshloka
{अथ दक्षिणदिग्भागे रावणेन नियोजिताः}
{ते सर्वे वानरवरैर्दारितास्तु यमं गताः} %॥४०॥

\twolineshloka
{पश्चिमेऽङ्गदमुख्यैश्च वानरैरतिगर्वितैः}
{राक्षसाः पर्वताकाराः प्रापिता यमसादनम्} %॥४१॥

\twolineshloka
{तदुत्तरे तु दिग्भागे रावणेन निवेशिताः}
{पेतुस्ते राक्षसाः क्रूरा मैन्दाद्यैर्वानरैर्हताः} %॥४२॥

\twolineshloka
{ततो वानरसङ्घास्तु लङ्काप्राकारमुच्छ्रितम्}
{उत्प्लुत्याभ्यन्तरस्थांश्च राक्षसान् बलदर्पितान्} %॥४३॥

\twolineshloka
{हत्वा शीघ्रं पुनः प्राप्ताः स्वसेनामेव वानराः}
{एवं हतेषु सर्वेषु राक्षसेषु दशाननः} %॥४४॥

\twolineshloka
{रोदमानासु तस्त्रीषु निर्गतः क्रोधमूर्च्छितः}
{द्वारे स पश्चिमे वीरो राक्षसैर्बहुभिर्वृतः} %॥४५॥

\twolineshloka
{क्वासौ रामेति च वदन् धनुष्पाणीः प्रतापवान्}
{रथस्थः शरवर्षं च विसृजन् वानरेषु सः} %॥४६॥

\twolineshloka
{ततस्तद्वाणछिन्नाङ्गा वानरा दुद्रुवुस्तदा}
{पलायमानांस्तान् दृष्ट्वा वानरान् राघवस्तदा} %॥४७॥

\twolineshloka
{कस्मात्तु वानरा भग्नाः किमेषां भयमागतम्}
{इति रामवचः श्रुत्वा प्राह वाक्यं विभीषणः} %॥४८॥

\twolineshloka
{श्रृणु राजन् महाबाहो रावणो निर्गतोऽधुना}
{तद्वाणाछिन्ना हरयः पलायन्ते महामते} %॥४९॥

\twolineshloka
{इत्युक्तो राघवस्तेन धनुरुद्यम्य रोषितः}
{ज्याघोषतलघोषाभ्यां पूरयामास खं दिशः} %॥५०॥

\twolineshloka
{युयुधे रावणेनाथ रामः कमललोचनः}
{सुग्रीवो जाम्बवांश्चैव हनूमानङ्गदस्तथा} %॥५१॥

\twolineshloka
{विभीषणो वानराश्च लक्ष्मणश्चापि वीर्यवान्}
{उपेत्य रावणीं सेनां वर्षन्तीं सर्वसायकान्} %॥५२॥

\twolineshloka
{हस्त्यश्वरथसंयुक्तां ते निजघ्नुर्महाबलाः}
{रामरावणयोर्युद्धमभूत् तत्रापि भीषणम्} %॥५३॥

\twolineshloka
{रावणेन विसृष्टानि शस्त्रास्त्राणि च यानि वै}
{तानि छित्त्वाथ शस्त्रैस्तु राघवश्च महाबलः} %॥५४॥

\twolineshloka
{शरेण सारथिं हत्वा दशभिश्च महाहयान्}
{रावणस्य धनुश्छित्त्वा भल्लेनैकेन राघवः} %॥५५॥

\twolineshloka
{मुकुटं पञ्चदशभिश्छित्त्वा तन्मस्तकं पुनः}
{सुवर्णपुङ्खैर्दशभिः शरैर्विव्याध वीर्यवान्} %॥५६॥

\twolineshloka
{तदा दशास्यो व्यथितो रामबाणैर्भृशं तदा}
{विवेश मन्त्रिभिर्नीतः स्वपुरीं देवमर्दकः} %॥५७॥

\twolineshloka
{बोधितस्तूर्यनादैस्तु गजयूथक्रमैः शनैः}
{पुनः प्राकारमुल्लङ्घ्य कुम्भकर्णो विनिर्गतः} %॥५८॥

\twolineshloka
{उत्तुङ्गस्थूलदेहोऽसौ भीमदृष्टिर्महाबलः}
{वानरान् भक्षयन् दुष्टो विचचार क्षुधान्वितः} %॥५९॥

\twolineshloka
{तं दृष्टोत्पत्य सुग्रीवः शूलेनोरस्यताडयत्}
{कर्णद्वयं कराभ्यां तुच्छित्त्वा वक्त्रेण नासिकाम्} %॥६०॥

\twolineshloka
{सर्वतो युध्यमानांश्च रक्षोनाथान् रणेऽधिकान्}
{राघवो घातयित्वा तु वानरेन्दैः समन्ततः} %॥६१॥

\twolineshloka
{चकर्त विशिखैस्तीक्ष्णैः कुम्भकर्णस्य कन्धराम्}
{विजित्येन्द्रजितं साक्षादगरुडेनागतेन सः} %॥६२॥

\twolineshloka
{रामो लक्ष्मणसंयुक्तः शुशुभे वानरैर्वृतः}
{व्यर्थं गते चेन्द्रजिति कुम्भकर्णे निपातिते} %॥६३॥

\twolineshloka
{लङ्कानाथस्ततः कुद्धः पुत्रं त्रिशिरसं पुनः}
{अतिकायमहाकायौ देवान्तकनरान्तकौ} %॥६४॥

\twolineshloka
{यूयं हत्वा तु पुत्राद्या तौ नरौ युधि निघ्रत}
{तान्नियुज्य दशग्रीवः पुत्रानेवं पुनर्ब्रवीत्} %॥६५॥

\twolineshloka
{महोदरमहापार्श्वो सार्धमेतैर्महाबलैः}
{सग्रामेऽस्मिन् रिपून हन्तुं युवां व्रजतमुद्यतौ} %॥६६॥

\twolineshloka
{दृष्टा तानागतांश्चैव युध्यमानान् रणे रिपून्}
{अनयल्लक्ष्मणः षड्भिः शरैस्तीक्ष्णैर्यमालयम्} %॥६७॥

\twolineshloka
{वानराणां समूहश्च शिष्टांश्च रजनीचरान्}
{सुग्रीवेण हतः कुम्भो राक्षसो बलदर्पितः} %॥६८॥

\twolineshloka
{निकुम्भो वायुपुत्रेण निहतो देवकण्टकः}
{विरुपाक्षं युध्यमानं गदया तु विभीषणः} %॥६९॥

\twolineshloka
{भीममैन्दौ च श्वपतिं वानरेन्दौ निजघ्रतुः}
{अङ्गदो जाम्बवांश्चाथ हरयोऽन्यान्निशाचरान्} %॥७०॥

\twolineshloka
{युध्यमानस्तु समरे महालक्षं महाचलम्}
{जघान रामोऽथ रणे बाणवृष्टिकरं नृप} %॥७१॥

\twolineshloka
{इन्द्रजिन्मन्त्रलब्धं तु रथमारुह्य वै पुनः}
{वानरेषु च सर्वेषु शरवर्षं ववर्ष सः} %॥७२॥

\twolineshloka
{रात्रौ तद्वाणाभिन्नं तु बलं सर्वं च राघवम्}
{निश्चेष्टमखिलं दृष्ट्वा जाम्बवत्प्रेरितस्तदा} %॥७३॥

\twolineshloka
{वीर्यादौषधमानीय हनुमान मारुतात्मजः}
{भूम्यां शयानमुत्थाप्य रामं हरिगणांस्तथा} %॥७४॥

\twolineshloka
{तैरेव वानरैः सार्धं ज्वलितोल्काकरैर्निशि}
{दाहयामास लङ्कां तां हस्त्यश्वरथरक्षसाम्} %॥७५॥

\twolineshloka
{वर्षन्तं शरजालानि सर्वदिक्षु घनो यथा}
{स भ्रात्रा मेघनादं तं घातयामास राघवः} %॥७६॥

\twolineshloka
{घातितेष्वथ रक्षस्सु पुत्रमित्रादिबन्धुषु}
{कारितेष्वथ विघ्नेषु होमजप्यादिकर्मणाम्} %॥७७॥

\twolineshloka
{ततः क्रुद्धो दशग्रीवो लङ्काद्वारे विनिर्गतः}
{क्वासौ राम इति ब्रूते मानुषस्तापसाकृतिः} %॥७८॥

\twolineshloka
{योद्धा कपिबलीत्युच्चैर्व्याहरद्राक्षसाधिपः}
{वेगवद्भिर्विनीतैश्च अश्वैश्चित्ररथे स्थितः} %॥७९॥

\twolineshloka
{अथायान्तं तु तं दृष्टा रामः प्राह दशाननम्}
{रामोऽहमत्र दुष्टात्मत्रेहि रावण मां प्रति} %॥८०॥

\twolineshloka
{इत्युक्ते लक्ष्मणः प्राह रामं राजीवलोचनम्}
{अनेन रक्षसा योत्स्ये त्वं तिष्ठेति महाबल} %॥८१॥

\twolineshloka
{ततस्तु लक्ष्मणो गत्वा रुरोध शरवृष्टिभिः}
{विंशद्वाहुविसृष्टैस्तु शस्त्रास्त्रैर्लक्ष्मणं युधि} %॥८२॥

\twolineshloka
{रुरोध स दशग्रीवः तयोर्युद्धमभून्महत्}
{देवा व्योम्नि विमानस्था वीक्ष्य तस्थुर्महाहवम्} %॥८३॥

\twolineshloka
{ततो रावणशस्त्राणिच्छित्वा स्वैस्तीक्ष्णसायकैः}
{लक्ष्मणः सारथिं हत्वा स्याश्वानपि भल्लकैः} %॥८४॥

\twolineshloka
{रावणस्य धनुश्छित्तआ ध्वजं च निशितैः शरैः}
{वक्षः स्थलं महावीर्यो विव्याध परवीरहा} %॥८५॥

\twolineshloka
{ततो रथान्निपत्याधः क्षिप्रं राक्षसनायकः}
{शक्तिं जग्राह कुपितो घण्टानादविनादिनीम्} %॥८६॥

\twolineshloka
{अग्निज्वालाज्वलज्जिह्वां महोल्कासदृशद्युतिम्}
{दृढमुष्ट्या तु निक्षिप्ता शक्तिः सा लक्ष्मणोरसि} %॥८७॥

\twolineshloka
{विदार्यान्तः प्रविष्टाथ देवास्त्रस्तास्ततोऽम्बरे}
{लक्ष्मणं पतितं दृष्ट्वा रुदद्भिर्वानरेश्वरैः} %॥८८॥

\twolineshloka
{दुःखितः शीघ्रमागम्य तत्पार्शं प्राह राघवः}
{क्व गतो हनुमान वीरो मित्रो मे पवनात्मजः} %॥८९॥

\twolineshloka
{यदि जीवति मे भ्राता कथचित्पतितो भुवि}
{इत्युक्ते हनुमान राजन् वीरो विख्यातपौरुषः} %॥९०॥

\twolineshloka
{बदध्वाञ्जलिं बभाषेदं देह्यनुज्ञां स्थितोऽस्मि भोः}
{रामः प्राह महावीर विशल्यकरणी मम} %॥९१॥

\twolineshloka
{अनुजं विरुजं शीघ्रं कुरु मित्र महाबल}
{ततो वेगात्समुत्पत्य गत्वा द्रोणागिरिं कपिः} %॥९२॥

\twolineshloka
{बदध्वा च शीघ्रमानीय लक्ष्मणं नीरुजं क्षणात्}
{चकार देवदेवेशां पश्यतां राघवस्य च} %॥९३॥

\twolineshloka
{ततः कुद्धो जगन्नाथो रामः कमललोचनः}
{रावण्यस्य बलं शिष्टं हस्त्यश्वरथराक्षसम्} %॥९४॥

\twolineshloka
{हत्वा क्षणेन रामस्तु तच्छरीरं तु सायकैः}
{तीक्ष्णैर्जर्जरित्म कृत्वा रस्थिवान् वानरैर्वृतः} %॥९५॥

\twolineshloka
{अस्तचेष्टो दशग्रीवः सज्ञां प्राप्य शनैः पुनः}
{उत्थाय रावणः कुद्धः सिंहनादं ननाद च} %॥९६॥

\twolineshloka
{तत्रादश्रवणैर्व्योनि वित्रस्तो देवतागणः}
{एतस्मिन्नेव काले तु रामं प्राप्य महामुनिः} %॥९७॥

\twolineshloka
{रावणे बद्धवैरस्तु अगस्त्यो वै जयप्रदम्}
{आदित्यहदयं नाम मन्त्रं प्रादाज्जयप्रदम्} %॥९८॥

\twolineshloka
{रामोऽपि जप्त्वा तन्मत्रमगस्त्योक्तं जयप्रदम्}
{तद्दत्तं वैष्णवं चापमतुलं सद्गुणं दृढम्} %॥९९॥

\twolineshloka
{पूजायित्वा तदादाय सज्यं कृत्वा महाबलः}
{सौवर्णपुङ्खैस्तीक्ष्णैस्तु शरैर्मर्मविदारणेः} %॥१००॥

\twolineshloka
{युयुधे राक्षसेन्द्रेण रघुनाथः प्रतापवान्}
{तयोस्तु युध्यतोस्तत्र भीमशक्त्योर्महामते} %॥१०१॥

\twolineshloka
{परस्परविसृष्टस्तु व्योम्नि संवर्द्धितोऽनलः}
{समुत्थितो नृपश्रेष्ठ रामरावणयोर्युधि} %॥१०२॥

\twolineshloka
{सगरे वर्तमाने तु रामो दाशरथिस्तदा}
{पदातिर्युयुधे वीरो रामोऽनुक्तपराक्रमः} %॥१०३॥

\twolineshloka
{सहस्त्राश्वयुतं दिव्यं रथं मातलिमेव च}
{प्रेषयामास देवेन्द्रो महान्तं लोकविश्रुतम्} %॥१०४॥

\twolineshloka
{रामस्तं रथमारुह्य पूज्यमानः सुरोत्तमैः}
{मातल्युक्तोपदेशस्तु रामचन्द्रः प्रतापवान्} %॥१०५॥

\twolineshloka
{ब्रह्मदत्तवरं दुष्टं ब्रह्मास्त्रेण दशाननम्}
{जघान वैरिणं क्रूरं रामदेवः प्रतापवान्} %॥१०६॥

\twolineshloka
{रामेण निहते तत्र रावणे सगणे रिपौ}
{इन्द्राद्या देवताः सर्वाः परस्परमथाबुवन्} %॥१०७॥

\twolineshloka
{रामो भूत्वा हरिर्यस्मादस्माकं वैरिणं रणे}
{अन्यैरवध्यमप्येनं जघान युधि रावणम्} %॥१०८॥

\twolineshloka
{तस्मात्तं रामनामानमनन्तमपराजितम्}
{पूजयामोऽवतीर्यैनमित्युक्त्वा ते दिवौकसः} %॥१०९॥

\twolineshloka
{नानाविमानैः श्रीमद्भिरवतीर्य महीतले}
{रुद्रेन्द्रवसुचन्द्राद्या विधातारं सनातनम्} %॥११०॥

\twolineshloka
{विष्णुं जिष्णुं जगन्मूर्तिं सानुजं राममव्ययम्}
{तं पूजयित्वा विधिवत्परिवार्योपतास्थिरे} %॥१११॥

\twolineshloka
{रामोऽयं दृश्यतां देवा लक्ष्मणोऽयं व्यवस्थितः}
{सुग्रीवो रविपुत्रोऽयं वायुपुत्रोऽयमास्थितः} %॥११२॥

\twolineshloka
{अङ्गदाद्या इमे सर्वे इत्यूचुस्ते दिवौकसः}
{गन्धामोदितदिक्चक्रा भ्रमरालिपदानुगा} %॥११३॥

\twolineshloka
{देवस्त्रीकरनिर्मुक्ता राममूर्धनि शोभिता}
{पपात पुष्पवृष्टिस्तु लक्ष्मणस्य च मूर्धनि} %॥११४॥

\twolineshloka
{ततो ब्रह्मा समागत्य हंसयानेन राघवम्}
{अमोघाख्येन स्तोत्रेण स्तुत्वा राममवोचत} %॥११५॥

\uvacha{ब्रह्मोवाच}

\twolineshloka
{त्वं विष्णुरादिर्भूतानामनन्तो ज्ञानदृक्प्रभुः}
{त्वमेव शाश्वतं ब्रह्म वेदान्ते विदितं परम्} %॥११६॥

\twolineshloka
{त्वया यदद्य निहतो रावणो लोकरावणः}
{तदाशु सर्वलोकानां देवानां कर्म साधितम्} %॥११७॥

\twolineshloka
{इत्युक्ते पद्मयोनौ तु शङ्करः प्रीतिमास्थितः}
{प्रणम्य रामं तस्मै तं भूयो दशरथं नृपम्} %॥११८॥

\twolineshloka
{दर्शयित्वा गतो देवः सीता शुद्धेति कीर्तयन्}
{ततो बाहुबलप्राप्तं विमानं पुष्पकं शुभम्} %॥११९॥

\twolineshloka
{पूतामारोप्य सीतां तामादिष्टः पवनात्मजः}
{ततस्तु जानकीं देवीं विशोकां भूषणान्विताम्} %॥१२०॥

\twolineshloka
{वन्दितां वानरेन्दैस्तु सार्धं भ्रात्रा महाबलः}
{प्रतिष्ठाप्य महादेवं सेतुमध्ये स राघवः} %॥१२१॥

\twolineshloka
{लब्धवान् परमां भक्तिं शिवे शम्भोरनुग्रहात्}
{रामेश्वर इति ख्यातो महादेवः पिनाकधृक्} %॥१२२॥

\twolineshloka
{तस्य दर्शनमात्रेण सर्वहत्यां व्यपोहति}
{रामस्तीर्णप्रतिज्ञोऽसौ भरतासक्तमानसः} %॥१२३॥

\threelineshloka
{ततोऽयोध्यां पुरीं दिव्यां गत्वा तस्यां द्विजोत्तमैः}
{अभिषिक्तो वसिष्ठाद्यैर्भरतेन प्रसादितः}
{अकरोद्धर्मतो राज्यं चिरं रामः प्रतापवान्} %॥१२४॥

\sixlineindentedshloka
{यज्ञादिकं कर्म निजं च कृत्वा}{पौरेस्तु रामो दिवमारुरोह}
{राजन्मया ते कथितं समासतो}{रामस्य भूम्यां चरितं महात्मनः}
{इदं सुभक्त्या पठतां च श्रृण्वतां}{ददाति रामः स्वपदं जगत्पतिः} %॥१२५॥

॥इति श्रीनरसिंहपुराणे रामप्रादुर्भावे द्विपञ्चाशोऽध्यायः॥५२॥

\closesection
    \chapt{पद्म-पुराणम्}
    \sect{पुराकल्पीयरामायणकथनम्}

\src{पद्म-पुराणम्}{सृष्टिखण्डम्}{अध्यायः ३३}{१--१८५}
% \tags{concise, complete}
\notes{This chapter describes the Ramayana of another Kalpa, as told by Shiva to Rama, at the request of the latter. A key difference is the absence of Setubandhanam; instead, the army crosses the ocean by meaans of Shiva's massive bow, the Ājagava!}
\textlink{https://sa.wikisource.org/wiki/पद्मपुराणम्/खण्डः_५_(पातालखण्डः)/अध्यायः_११६}
\translink{https://www.wisdomlib.org/hinduism/book/the-padma-purana/d/doc365826.html}

\storymeta

\begin{flushleft}
    
\uvacha{सूत उवाच}

सन्ध्यावन्दनकर्मक्रियतामिति\\
रामो मुनिमाचष्टायम्।\\
उष्णद्युतिरप्यस्तमुपैति
द्विजकुलमेतन्नीडमुपैति॥१॥

स्वयमपि सन्ध्यावन्दनकामो\\
ऽव्रजदुत्तरदिशमुज्झितयानः।\\
हाहाहूहूकृतसङ्गीतिर्\\
बन्दीप्रमुखप्रस्तुतकीर्तिः।॥२॥

गौतमीतटमुपेत्य राघवो\\
वायुनन्दनसुधौतपद्युगः।
जाम्बवत्कृतकरावलम्बनः\\
प्रापदुत्तमनदीं तु गौतमीम्॥३॥

करद्वये धृतकुशः स राघवः\\
प्रागमद्वरुणदिशामथोत्तमाम्।\\
दत्वा ततोऽर्घत्रितयं यथाविधि\\
प्रहृष्टरोमाथ जजाप सोऽन्तरे॥४॥

सम्प्रार्थयित्वा वरुणं यथाक्रमं\\
शम्भुं वसिष्ठं प्रणनाम राघवः।
ताभ्यां कृताशीरगमन्मनःपदं\\
हनूमता क्षालित-पादपङ्कजः॥\\
जुहाव वह्नीनथ बन्दिमागधैः\\
संस्तूयमानोऽथ विनिर्ययौ बहिः॥५॥

प्रहसच्चन्द्रकिरणैः सुधालिप्तमिवाम्बरम्।\\
प्रसन्नताराकुसुमं वितानमिव सर्वतः॥६॥

अथागच्छत्सौधतलं वृद्धामात्येन कल्पितम्।\\
नानासनसमोपेतं सभास्थानं ययौ नृपः॥७॥

अथ मुनिं ह्युपवेश्य स राघवः\\
स्वयमपि प्रथमासनमाभजत्।\\
कपिगणाः परितः पृथुविग्रहा\\
रचनयास्थितिमाप्रतिपेदिरे॥८॥

सुखस्थितं नृपमभिवीक्ष्य स द्विजो\\
वचस्तदा समुचितमाह शम्भुः।\\
इहस्थितो भवति समस्तपूजितः\\
कथं कथा नृपवर वर्तते गुहायाम्॥९॥

आकर्ण्याथ रघूद्वहो द्विजवचः शुश्रूषुरासीत्कथां\\
तत्रस्थो निपुणं निवार्यवचनं सर्वैः श्रुतं तत्क्षणात्।
शुश्रूषामि कथं महाद्भुततया स्वात्माश्रयामन्यथा\\
रक्षोबाधनवादिनीमथनृपः किन्त्वेतदित्याह च॥१०॥

कुम्भश्रोत्रवधः पुरा समजनि प्राप्तो दशास्यो वधं\\
पश्चादित्ययमन्यथा विरचितं रामायणं भाषते।
कोऽयं विप्रवरः समस्तजनता नास्तिक्य सम्पादको\\
राज्ञांस्थानमुपेत्य वक्ति समया दण्ड्योऽथ पूज्योऽथवा॥११॥

अथाह जाम्बवानमुं रघूत्तमं कथां प्रति\\
रामायणं न तावकं त्विदं हि कल्पितं मतम्।\\
समस्तमत्र विस्तराद्वदामि देव तच्छृणु\\
पङ्केरुहस्यसूनुतो मया श्रुतं पुराह्यभूत्॥१२॥

जाम्बवन्तं विज्ञाप्य रामचन्द्रो वचनमाह॥१३॥

\uvacha{श्रीराम उवाच}

कीर्तय पुराणं मे शुश्रूषुः कुतूहलादहं प्रणीतं तत्केन च विज्ञातम्॥१४॥

जाम्बवानथ बभाषे हि विधात्रे नमो नमस्तथैव विधुभूषणकेशवाभ्याम्॥१५॥

अथ पुरातनं रामायणं कथयामि॥१६॥

यस्य श्रवणेनाखिलजन्मसम्पादित पापक्षयो जायते॥१७॥

अथ तथापि दशरथो दशरथसमानरथी महीयसा बलेन सुमानसनामनगरजिगीषया पङ्केरुहसुतसुतं वसिष्ठमाहूय नमस्कृत्वा मुनिदत्तानुज्ञः शताक्षौहिणीसेनया सहारुह्य तुरङ्गमं चन्द्रसमानशरीरमतिरोषसमाविष्टो विष्टरश्रवसमाराध्य दण्डयात्रां चकार॥१८॥

साध्यो नाम स्वीयया सेनयावृतो दशरथाभिमुखमाययौ योद्धुं युद्धं चान्योन्यमभूत्॥१९॥

मासमेकं युद्धं कृत्वा दशरथस्तं साध्यं जग्राह॥२०॥

अथ साध्यसूनुर्भूषणोनामाल्पपरिवारो युयुधे दशरथेन॥२१॥

दशरथोऽपि साध्यसूनुं भुवोभूषणमवलोक्य योद्धुमेव नैच्छत्॥२२॥

कथमेतादृशं हन्मि चास्मिन्हतेऽस्य कथं पिता भविष्यति कथं तन्माता कथमप्रौढयौवनाप्रियार्या॥२३॥

अमुष्य हि देहे समालिङ्गनचुम्बनपरिवर्तन नवीनतरदलारविन्दपदानि कुसुमानीव दृश्यन्ते॥२४॥

एतत्समानवर्णवया एतादृशः सुभगः परमप्रीतिवर्धनो नाम पुत्रो भल्लूकभक्षितोमृतःस्मृतिपथं प्राप्यापि मां रक्षयितुमिच्छतीव मम हृदयमन्यथा करोति इति मनसा वितर्क्यातिबालकं ग्रहीतुमारभत्॥२५॥

स च साध्योपि पराधीनो बभूव॥२६॥

स च कुमारेण सह पराजय खेदमपि मत्वा सुखमध्युवास च॥२७॥

दशरथोऽपि तत्र मासं स्थित्वा तत्पुत्रसन्दर्शनसुखमवलोक्याचिन्तयत्॥२८॥

अहो सर्वदुःखापनोदनक्षममेतन्मुखावलोकनं पुत्रसंवर्धनं नाम सर्वराष्ट्रिको मम जयः पुत्रवियोगमनुस्मरतो दुःखाय केवलं भवति॥२९॥

तदस्य पृच्छां करोमि कथमीदृशो जायते पुत्र इति वितर्क्य तमपृच्छत्॥३०॥

साध्योऽपि सकलमोक्षमार्गं क्षितीशायादिशत्॥३१॥

हरीशानौ सहाराध्य सर्वैकादशीरुपोष्य द्वादशीषु ब्राह्मणानाराध्य तत्तत्कालभवं फलपूर्वमन्नाद्यं व्यञ्जनं पुष्पं च न्यायेन सम्पाद्य कपिलाघृतेन केशवं स्नापयित्वा मुद्गचूर्णेन संलिप्य स्वादूदकेन स्नापयित्वा सुरभिपाटीरं स्वयमुद्घृष्टं मृगनाभ्यागुरुसारेण वासमेतं देवाङ्गं सर्वमुपलिप्य तुलसीदलैर्यूथिकाकरवीरनीलोत्पलकमल कोकनदद्रो णकुसुम मरुवदमनकगिरिकर्णिकाकेतकीदलपूर्वैर्यथासम्भवमभ्यर्च्य द्वादशाक्षरेण पुरुषसूक्तेन वा नाम्ना षोडशोपचारेण वाराध्य प्रणम्य नृत्यं कृत्वा देवं क्षमापयेत्॥३२॥

तथा व्रतानि विचित्राणि नारायणप्रीणनाय कुर्यात्॥३३॥

प्रसन्नो भगवान्मुनिरीप्सितं पुत्रं यच्छति तदमुमाराधयस्वेति दशरथमुक्तवान्॥३४॥

स चापि साध्यं तत्र स्थाप्य गत्वायोध्यां तथा सर्वं कृतवान्॥३५॥

अथ पुत्रकामेष्टौ समाप्तायामाहवनीयाद्यज्ञमूर्तिः शङ्खचक्रगदापाणिरुदतिष्ठत्

राजानं च वरं वृणीष्वेत्युक्तवान्॥३६॥

स च राजा वव्रे पुत्रानतिधार्म्मिकान्दीर्घायुषश्चतुरोलोकोपकारकान्देहीति॥३७॥

अथ राजमहिष्यश्चतस्रः कौशल्या सुमित्रा सुरूपा सुवेषा चेति॥३८॥

राजानमब्रुवन्देव प्रतियोषमेकेन पुत्रेण भवितव्यम्॥३९॥

अथ कौशल्योवाच एष यदि प्रसन्नो देवस्तदायमुत्पद्यतां मम॥४०॥

\uvacha{राजोवाच}

मम यदिष्टं तदयं प्रार्थ्यते हरिः विष्णोप्रसीददेवेश कमलापते शङ्खचक्रगदाधरविभीषणसृष्टिसमस्तलोकपालादिपूजित पादयुगलशाश्वतहरे नमस्ते नमस्ते एवं स्तुतो भगवानथ राजानमाह॥४१॥

\uvacha{माधव उवाच}

तव पुत्रो भविष्यामि कौशल्यायाम् अथ चरुं प्राविशद्धरिः तं चरुं हि चतुर्धा विभज्य भार्याभ्यो दत्तवान्॥४२॥

अथ कौशल्यायां रामो लक्ष्मणः सुमित्रायां सुरूपायां भरतः सुवेषायां शत्रुघ्नो जज्ञे खात्पुष्पवृष्टिश्च पपात॥४३॥

अथ चतुराननः स्वयमुपेत्य जातकर्मादिकाः क्रियाश्चक्रे॥४४॥

त्रिभुवनाभिरामतया राम इति नाम चक्रे रूपशौर्यादिलक्ष्मीयोग्यतया लक्ष्मण इत्यपरस्य भुवं भारात्तारयतीति भरतः शत्रून्हन्तीति शत्रुघ्न इति नामानि कृत्वा ब्रह्मा स्वभवनं जगाम शिशवश्च वृद्धिमीयुः॥४५॥

अथ पादसञ्चारिणं बालचन्द्र सङ्काशदर्शनं बिम्बाधरमुन्नततिलप्रसूननासं पुरश्चूलिकालम्बमानरत्नपत्रकं श्रवणलोललम्बमानकुण्डलं वक्षःस्थलविचलित स्थूलमुक्ताहारं विलसत्कार्तस्वरो बह्वलयं सिञ्जन्मणिकङ्कणरत्नाङ्गुलीयकं हेममणिरचितश्रोणिसूत्रं सिञ्जन्ननूपुरोपशोभितपादमङ्गुलीयोपशोभितपादमध्याङ्गुलीयकं वज्राङ्कुशसरोजलाञ्छनशोभितोरुपादतलं तूणीरसदृश जङ्घं करिकरसदृशोरुं विस्तृतजघनं सूक्ष्ममध्यंवर्तुलावर्तकं गम्भीरनाभिमिन्द्रनीलशिलाविशालवक्षःस्थलं कम्बुग्रीवं चन्द्रबिम्बसदृशवदनमर्धचन्द्रसदृशललाटं नीलकुटिलकुन्तलं क्रोडासक्तन्धूलिभिरापाण्डुरं फुल्लपद्मदलारक्तविलोललोचनं महेश्वरमिवोद्धूलितभूतिं महेश्वरमिव दिगम्बरं रामं कुमारं राजा दशरथो दृष्ट्वा हर्षपरिपूर्णहृदयः पुत्रमालिङ्ग्य चुम्बित्वा वक्षस्यालिलिङ्ग दृढम्॥४६॥

अथ कुमारोपि पार्श्वेनास्याङ्कमारोप्य कलकलितलोचनो यत्किञ्चिदुवाच॥४७॥

याचमानमितस्ततो वीक्षमाणः तात गच्छेशये तात क्रीडामि तातेत्यादिपुत्रसुखमनुभूयानुभूय निर्वृतिं ययौ॥४८॥

अथ कदाचिद्भोक्तुमागते राजनि रामचन्द्रो बालक्रीडासक्तहृदयो बहुक्रीडनकरकमल उत्प्लुत्य धावमानो नरपतिपुरःस्थितमणिखचितसुवर्णभाजनस्थमन्नं वामकरेण गृहीत्वा राजनि चिक्षेप

इदमपि राजा सुखाय मेने एतादृशान्यन्यानि चकार रामचन्द्रः॥४९॥

अथ कदाचित्क्रीडमाने रामे वात्या राममपातयद्रामश्च रुदन्नपतत्॥५०॥

एतस्मिन्नन्तरे ब्रह्मराक्षसो राममगृह्णाद्रामश्च मूर्च्छामाप॥५१॥

अथ सहचरो बाल इतस्ततो रोरूयमाणो रामं तथाविधं राज्ञे व्यज्ञापयत्॥५२॥

अथ राजा राममादाय वसिष्ठमाह किमिदं रामस्येति पप्रच्छ॥५३॥

अथ वसिष्ठो भस्मादायाभिमन्त्र्य ब्रह्मराक्षसं मोचयामास॥५४॥

पप्रच्छ को भवानिति स चाहाहं वेदगर्वितो ब्राह्मणो बहुशः परधनमपहृत्य ब्रह्मराक्षसो जातो मे निष्कृतिं विचारय॥५५॥

\uvacha{वसिष्ठ उवाच}

इदानीमितः परमेकवर्षशतोपभोग्यं राक्षसत्वं नरकं भागीरथीस्नानमेकं शिवाय बिल्वपत्रशतं समर्प्य ततः स्नात्वा पापाद्विमुक्तो भवसीति॥५६॥

कदाचित्तादृशं कृतपुण्यं तव पदं प्रयच्छामि तदुपरिशिष्टां गतिं भजेति वसिष्ठवाक्यमाकर्ण्य ब्रह्मराक्षसो वसिष्ठोपदिष्टपुण्यवशाद्दिव्यशरीरो भूत्वा नमस्कृत्वा स्वर्गं जगाम॥५७॥

अथ रामं प्राप्तेकाले उपनीय वसिष्ठो वेदानध्यापयामास षडङ्गानि मीमांसाद्वयं नीतिशास्त्रं चाध्यापयामास॥५८॥

अथ धनुर्वेदमायुर्वेदं भरत गान्धर्व वास्तु शाकुन विविधयुद्धशास्त्राणि च॥५९॥

अथ विवाहं कर्तुकामेन राज्ञा दशरथे नानानादेशजनपतीन्प्रति दूताः प्रेरिताः॥६०॥

अथ कश्चिच्छीघ्रमागत्य राजानमिदमब्रवीत्॥६१॥

राजन्विदर्भदेशाधिपतिर्विदेहो नाम राजा तस्य पुत्री वैदेही होमलब्धारूपेण लक्ष्मीसमा सर्वलक्षणसम्पन्ना रामयोग्या विद्यते स च तां दातुं राजा रामायोद्यतः तद्गम्यतां शीघ्रमिति॥६२॥

अथ वसिष्ठादीन्प्रेषयामास ते च तत्र गत्वा तां च निरीक्ष्य लग्नं निश्चित्यायोध्यामेत्य राजानमुक्त्वा रामसहिताः पृथिवीपतिसमेताः शीघ्रं विविध करि तुरग शकट शिबिकान्दोलिकाभिरतिसुभगरूपभोगविलासक्रियानिपुणा विदितविविधचेष्टागन्धर्वाः कामशास्त्रसुकुशलाः मृदुकठिनपृथुपयोधरासन्न कण्ठाः स्थूलसूक्ष्मललाटबिम्बदशनच्छदमुखपङ्कजाः कुटिलकुन्तलदीर्घकेशधम्मिल्लाः कनकपत्रकर्णाः स्नानचेष्टयोत्थितरोमशोभित जपाकुसुमरक्तदशना विशदविस्फुरच्छफरीलोचनाः शुक्तिकासदृशश्रवणाः नक्षत्रसदृशस्थूलमुक्ताफलोपशोभितनासापुटा मुकुरसदृशकपोलास्तिलप्रसूननासिका आनम्रमध्यप्रदेशचूचुका इन्द्रगोपप्रतीकाशाधरपुटदशनक्षताः समदीर्घकाङ्गप्रदर्शनास्थितसर्वप्रदेशवर्तुला नातिमांसलाः पिण्डकाग्रन्थिनीव्योवलितबाहुमूला अनतिचिरकालोत्थितरोमतया हरिद्रा वर्णतया च कर्णिकारदलसदृशबाहुमूला मृदुस्निग्धवर्तुलसूक्ष्ममध्यप्रदेशाः कठिनस्थूलवर्तुलामग्नचूचुकपरस्परस्थानाक्रमणस्पर्धि पयोधरमध्यलब्धपदकपयोधरोपरिचञ्चलविविधमणिमय हारोपशोभित वक्षःस्थलाः पयोधरपरितो लब्धपदतया तरुणदृष्टिपरम्परतया असमानयानाभिकूपोपरितन रोमराज्योपशोभितोदरप्रदेशा भज्यमान मध्यस्थलीकरणएव वलीत्रयोपशोभित मुष्टिग्राह्यमध्याः करिकरोपमजघनप्रदेशा अरोमसदृशमृदुस्निग्धामला समजान्व्यः कदलीस्तम्भसन्निभोरुयुगला आमग्नजानुकृशकुशवर्तुलपिण्डिकारहितजङ्घा आमग्नगुल्फा आसूक्ष्मस्निग्धा दीर्घदीर्घाङ्गुली पादानूपुररवाह्वयमानमदना हंसमतङ्गजगमना दक्षिणाङ्गुष्ठस्पर्शिकच्छाग्रा उपरिकच्छं नीवीं कृत्वा करद्वययुता वस्त्रप्रदेशकण्ठमप्रावृत्या परवसनपरिभागा वृतस्तनवसना परभागे वामांस एव दक्षिणपार्श्वगतेन दशाभागेन नाभिप्रान्तेन प्रवेशितोपशोभितगात्रयष्टयो योषितो विवाहमङ्गलकर्मकरणायानेकश आगच्छन्॥६३॥

बालिकाश्च विद्युल्लतांशुशोधितगात्रयष्टय उद्भिन्नकुचकमलकुड्मलविविधहारोपशोभिवक्षसो यत्किञ्चिद्भाषिण्योऽतिचपलमृदुगतयो वृद्धवनिताश्चागच्छन्॥६४॥

अथ विदेहे पुरतः क्रोशमात्रे चूतवनिकायां विविधविटपविस्तरप्रदेशविविधविहङ्गकूजिताकर्णनदत्तकर्णवनहरिणशाववत्यां महारजतनिर्मितोच्चनीचप्रासादोपशोभितप्रदेशविविधविहङ्गायां हेमवल्कलसंवीतभसितोद्धूलितशरीरजटिलमुनिगणध्यानोपासनोपशोभितवृक्षमालायां विविधविद्यधरवधूपयोधरभाराभिभूतविचरिततरङ्गसरसीयुतायां सरस्तीरमिलितसैरन्ध्रीयुवतिभिराहूयमान तरुणजनायां नानावर्णकुसुमसौरभवासिताशेषप्रदेशायामितस्ततो रिरंसया प्रदर्शितस्फारशफरीविलोचनतरलचक्षुषा प्रभाविलसितशरीरवेश्याजनायां विविधाश्चर्ययुतायां दशरथः सामात्यपुरोहिताभिरामरामादिपुत्रसहितः सुखमुवास॥६५॥

अथ वैदेहोऽपि मिथिलां नानापताकोपशोभितां विविधप्रासादगोपुरोद्यानदेवतायतनोपशोभितामन्योन्यकेलिचतुरयुवतिजनानुकीर्णामुशीरविरचितमहाप्रपां सुकेलीजनोपशोभितविशिखां

विविधपण्योपशोभितरथ्यां तत्रतत्र ब्रह्मघोषशोभितमठां प्रतिमन्दिरं मीमांसादिव्याख्यानसम्पादि सामाध्ययनां सुपुण्यहविर्गन्धसामादिस्वरपदक्रमश्रुतिब्राह्मणवाटिकामनेकपरिवृढमन्दिरप्रवेशनिष्क्रीतागुरु-कुङ्कुमाध्वर्युवेषां मृदुलवसन ताम्बूल रक्तदन्तच्छदकामिनीमृदुवचनकठिनवचनकरसंज्ञानिर्धारित प्रतिवचनविविधोपायना हरणकरजनोपशोभितां मृदुधवलजघनपरिवीतवस्त्रोपरिभागेन स्निग्धवर्तुलपरस्परसङ्घर्षपयोधरमध्यप्रदेशोपशोभित वामांसकन्दोपशोभितवनितां विविधमुक्ताहारजपासङ्काशदशनच्छद मन्दहासमालाकारसहस्रोपशोभितां पुण्यासवसाधनमन्दिरां तत्र तत्र विचित्रतोरणां विशुद्धवीथिकां तत्रतत्र स्थापितकल्पपादपां रम्भाविभूषितद्वारां पुरीं शोभितां शोभयामास॥६६॥

अथाभिकलनार्थं विलासिन्यो निशादूर्वाक्षतामन्त्रमङ्गलकज्जलितकैशिकधमिल्ललताग्रन्थित-जटोपशोभित-सीमन्तशीर्षशोभित-नासामुख-विचित्राभरणार्हणहेमपात्रावस्थिताज्यगुग्गलु फलादिसौभाग्यद्रव्यमुद्वहन्तीभिः स्त्रीभिरन्यैरपिशोभितजनैः स राजा निर्जगाम॥६७॥

तदानीं मङ्गलतुर्यघोषादेव दुन्दुभिभेरीनिःसाणमर्दलशङ्खादिनादाः प्रादुर्बभूवुः॥६८॥

गायकाश्च मङ्गलानि जगुः॥६९॥

मङ्गलवेदवाक्यानुपाठेन वैदिकाः ब्राह्मणाः\\
कुलपाठका भेरीघोषेण कृत्स्नमाकाशमापूरयन्॥७०॥

अथान्योन्याक्षतपूर्वमङ्गीकुर्वन्तः सूतबन्दिजनादिभिः स्तूयमानाः पुरं प्रविविशुः॥७१॥

विदेहनगरात्पश्चिमभागे निर्मितं मन्दिरं दशरथः प्रविवेश॥७२॥

अवशिष्टाश्च यथायोग्यं भवनं विविशुः॥७३॥

अथ नारदो मिथिलां तदानीमेवागच्छत्॥७४॥

विदेहोऽपि देवर्षिमभिपूज्य स्वागतं दृष्ट्वा भोजनं कारयित्वा सुखासीनाय मुनये सघनसारं ताम्बूलं दत्वा व्यज्ञापयत्॥७५॥

श्वो विवाहे भवान्स्थातुमर्हति कारयितुं विवाहम्॥७६॥

नारद उवाच॥७७॥

श्वो हि नक्षत्रं सूर्यनक्षत्रदर्शनं तत्र विवाहो न कर्तव्य इति॥७८॥

अथ मौहूर्तिकं वृद्धगार्ग्यमाहूय राजा पप्रच्छ क्व विवाहमुहूर्तः॥७९॥

श्व इति गार्ग्य उवाच॥८०॥

राजा च नारदं गार्ग्यं चोदीक्ष्य भो इदमित्थमिति पप्रच्छ॥८१॥

अथ नारदो गार्ग्यमुवाच॥८२॥

कथमुक्तं लग्नं दास्यसि॥८३॥

अथ गार्ग्यो विषघटिकाश्च विहाय लग्नं दास्यामि इत्युवाच॥८४॥

नारदोपि ब्रह्मवचनानि किं न जानासीत्युक्तवान्गार्ग्यम्॥८५॥

गार्ग्यस्तुष्टस्तान्दोषानपठत्॥८६॥

उल्का च ब्रह्मदण्डश्च मोघः कम्पस्तथैव च

सर्वकार्यविनाशाय दृष्टा वै ब्रह्मणा पुरा॥८७॥

प्रतिष्ठासु विवाहेषु मौञ्जीबन्धाभिषेकतः

अन्येषु सर्वकार्येषु विषनाडीर्विवर्जयेत्॥८८॥

अतः परं तु कार्याणां करणेन च दोषभाक्।\\
विवाहादिषु कार्येषु दोषमेव वदाम्यतः॥८९॥

उल्का दहेत्कुलं सर्वं ब्रह्मदण्डो विनाशयेत्।\\
मोघस्तु मरणा यस्यात्कम्पः कम्पाय कर्म्मणः॥९०॥

इति नारदोक्तमाकर्ण्य गार्ग्यो मुनिर्मौनपरोऽभवत्॥९१॥

दध्यौ रविं ग्रहपतिं विहाय विषनाडीर्विवाहः क्रियतमिति॥९२॥

\uvacha{नारद उवाच}

कथं ब्रह्मवचनम्॥९३॥

\uvacha{सूर्य उवाच}

देशभेदेन व्यवस्थोदिता तदस्मिन्देशे विवाहो विषघटिकां विहाय कर्तव्य एव॥९४॥

नारदोप्यनुमेने॥९५॥

उवाच श्वः पराह्णे च क्षत्त्रविवाहश्च भवेदतः स्वयं वरार्थं नृपा आगच्छन्तु तन्नृपदूतान्प्रेषय॥९६॥

अथ राजा दशरथानुमतेन सर्वानेव नृपानागमय्याचिन्तयत्

कथं सर्वानेव तिरस्कृत्य वैदेही रामाय देयेति॥९७॥

अथ रात्रौ मुहुर्मुहुर्निःश्वस्य निद्रालुरपि न निद्रामाप॥९८॥

अथ मध्ये निशं राजा शुचिर्भूत्वा त्र्यम्बकं साम्बिकं मङ्गलदुकूलधारिणं कमलभवपुरुषोत्तमशक्र-प्रमुखनिखिलदेवैर्भृगुप्रमुखमुनिवरैर्हाहाप्रमुखगन्धर्वैस्तुम्बुरुप्रमुखैश्च श्रुतिस्मृतीतिहासपुराणैर्मूर्तिमद्भिश्च सिद्धविद्याधरादिमातृकागणैश्च नन्दीप्रमुखगणैश्च सेव्यमान-पादकमलं सर्वामङ्गलपरिमोचकमतिपुण्यसलिलया गङ्गया निष्कलं केन चन्द्रमसा सेव्यमान शिरोभागं वामाङ्कारूढया गिरिजया प्रदीयमानवीटिं सहासं सकामं सहावेक्षणमाददानं गोक्षीरसदृशं प्रतिकूलकस्तूरिकासदृशकण्ठं मृदुसूक्ष्मस्निग्धजटाभिर्विरचितकपर्दं विशुद्धकार्तस्वरकुण्डलोपशोभितगण्डभागं द्विरष्टवर्षवयसं गोक्षीरसदृश-स्थूलमुक्ताफलकौसुम्भवर्णाञ्चलेनावेष्टितशिरोभागं विविधरत्नविरचितकार्तस्वरभूषितवक्षःस्थलमतिधवलोपवीतेनोपशोभितशरीरमम्बिकानुलग्नकुङ्कुमारुणसुगन्धिशरीरमीक्षमाणं तर्जमानकाममार्गणं कोटिकन्दर्पसदृशं मनसाचिन्तयत्॥९९॥

जजाप शतरुद्रियं जुहाव च तेनैव कामाहुतीः प्रास्तुवच्च पुरुषसूक्तेन॥१००॥

अथ तादृश एव महेश्वरस्तत्र प्रादुरभूत्॥१०१॥

अथ राजा नमस्कृत्यास्तुवीत॥१०२॥

\uvacha{राजोवाच}

क्षितिसलिलगगनपवनदहनरविशशियजमानमूर्तिभिरष्टमूर्ते विश्वमूर्ते लोकमूर्ते त्रिभुवनमूर्ते वेदपुराणमूर्ते यज्ञमूर्ते स्तोत्रमूर्ते शास्त्रमूर्तेस्वधामूर्ते नारायणमूर्ते सर्वदेवतामूर्ते त्रयीमय त्रयीप्रमाण त्रयीनेत्र सामप्रिय वसुधाराप्रिय भक्तिप्रिय भक्तसुलभाभक्तविदूर स्तुतिप्रिय धूपप्रिय दीपप्रिय घृतक्षीरप्रिय द्रो णकरवीरप्रिय श्रीपत्रप्रिय कमलकह्लारप्रिय नन्द्यावर्तप्रिय बकुलप्रिय यूथिकाप्रिय कोकनदप्रिय ग्रीष्मजलावासप्रिय यमनियमप्रिय नियतेन्द्रियप्रिय जपप्रिय श्राद्धप्रिय गायनप्रिय गायत्रीप्रिय पञ्चब्रह्मप्रिय सदाचारप्रिय गोत्रोत्सादिकमलभवहरिहरनयनसमर्चित पादकमलजयप्रद हरिप्रार्थितजलोत्पादितचक्रप्रदर्शकृत्स्मृतियुक्तिप्रद स्मृतमङ्गलप्रद मृत्युञ्जय नमस्ते नमस्ते॥१०३॥

इति स्तोत्रमाकर्ण्य भगवान्भवो राजानमुवाच वरदोऽहं वरं वृणु॥१०४॥

राजोवाच श्रीमन्मम कन्या वैदेही रामाय दित्सिता स्वयंवरे कुलरूपबलोत्साह सम्पन्नानेकभूपराक्षसविप्रादिसर्वप्राणिसमागमे रामाधिकबलो यदि तामग्रहीष्यत्

तदा वचनमनृतं मम पापं च भविष्यति॥१०५॥

प्रत्युत दशरथोऽपि सर्वानेवागतान्विजेतुमलं क्षत्रकदनश्च रामो यद्यायास्यति तर्हि मम सुतां किं करिष्यति वा किं किं वा प्रेषयिष्यति कीदृशं कारयिष्यति मम किं वा करिष्यति सर्वथा हि प्रभूतबलवाहनो नरपतिरशेषमपि त्रिभुवनं हन्यात् किमुत मामल्पसत्वं किमुत बहुना भवानेव शरणं ममोपायं वद यथा विवाहे श्रेयो भविष्यति रामश्च जामाता भविष्यति॥१०६॥

शम्भुरपि तथा करोमीत्युवाच॥१०७॥

राम एव नाथः सीताया भविष्यति

रामं च कृत्वा स्वस्त्यद्यैव करिष्यामि गृहाणाजगवं धनुरिदम्॥१०८॥

\uvacha{राजोवाच}

किमेतेनाजगवेन धनुषा स्वयंवरे सीतां रामं प्रापय॥१०९॥

\uvacha{शङ्कर उवाच}

इदं धनुरसज्यं मे यस्तु सज्यं करिष्यति।\\
तस्मै देया मया सीता प्रतिज्ञामेवमाचर॥११०॥

इत्येवमुक्त्वा भगवान्गणैरन्तर्दधे हरः।\\
अथादातुं धनू राजा न शशाकातियत्नतः॥१११॥

अथोज्ज्वलं शतसहस्रगजबलं समाहूय गृहाणेत्युवाच॥११२॥

स चापि मातुलं नत्वाऽट्टहासं कृत्वोत्प्लुत्य धनुर्द्वाभ्यां कराभ्यामुद्दधार जानुपर्यन्तं मातुलो मारीचः श्रुत्वा एकाकी विप्रवेषं कृत्वा विदेहमयाचत वैश्वदेवान्ते प्राप्तमतिथिं मामवैहि॥११३॥

\uvacha{राजोवाच}

स्वागतं भो इदं ब्रह्मन्नासनं तत्र निषीदेति॥११४॥

स चातिथिस्तथेत्युक्त्वा निषसाद॥११५॥

अथ राजा जलमादाय पादौ प्रक्षाल्य गन्धपुष्पाक्षतैरभ्यर्च्य महाजं तस्मै निवेद्य भोजनाय प्रार्थयामास॥११६॥

स चापि तदन्नं षड्रसोपेतं सौवर्णभाजनगतमीक्षमाणइवेतस्ततो विलोकयामास॥११७॥

तस्मिन्नेवावसरे सीता पद्मकिञ्जल्कप्रभेषदरुणवसनं बिभ्रती नीलकुटिलकुन्तलैश्चलद्भिर्यूनां मनांस्याकर्षयद्भिः प्रेक्षमाणदृष्टिभग्नकलैरिव स्त्रीणां चित्तमीदृशमिति दर्शयद्भिरिवोपशोभित-ललाटानङ्गचापसुभ्रूपद्मपत्रारुणविलोचना तिलप्रसूननासामृदुस्निग्धरो मशकपोलानन्तरा रक्तोष्ठराक्तासनमाणिक्यनिभदाडिमीदशना जपाकुसुमारुणाधरातिशोभितचिबुकाशुक्ति-कर्णासमदीर्घकण्ठातिमांसलवक्षाः पीनोद्भिन्नकुचकुड्मलानेकहारोपशोभिता सुभगाकारानतिमांसलबाहुलता मुग्धायतसमानाङ्गुलिशिखापद्मारुणपल्लवाविविधबहुरत्नाङ्गुलिभूषणामुष्टिग्राह्यमध्यासु रोमराजिगम्भीरनाभिः पृथुजघनाकरिकरोरुस्तूणीरजङ्घासुपादकमलानूपुरादि पादविभूषणा पादाङ्गुलीभूषिता विकसितसौगन्धिकं विदधती भुञ्जानमारीचस्य पुरतश्चागता॥११८॥

वीक्ष्यासावचिन्तयदेनां कथमपहरामि कथमालिङ्गामि कथमन्यत्किञ्चित्करोमीत्येवमवसरमलभमानस्तूष्णीमेव विनिर्गतः॥११९॥

अथ देवा धनुःसज्जीकरणाय यतमाना अहं पूर्विकया विद्यमाना अन्योन्यतिरस्कारेण महेन्द्रः प्राप धनुरुत्तमं प्रान्तद्वयात्परं नावनमयितुं शशाक॥१२०॥

अथ सूर्यो धनुरादाय नमयन्नेव निपपात॥१२१॥

वायुर्बलवतां श्रेष्ठो जग्राहाजगवमथ स्वेनैव करेणोत्कर्षयन्नधः पपात धनुश्च वायोरुपरि पपात अहसंस्तदा सर्वे॥१२२॥

एतस्मिन्नन्तरे तुरगवरमारुह्य बाणासुरः सहस्रबाहुरनेकानेकशिरोभिर्दैत्यैः परिवृतः प्रह्लादसमेतो विदेहपुरीमाजगाम॥१२३॥

अथ स्वविभूषणोद्भासितां दिशं कुर्वन्स्वतेजसापयशसो देवताः कुर्वन्नानाविधगीतं शृण्वन्द्व्यङ्गुलमात्रेण शक्तो विरराम॥१२४॥

प्रह्लादो बलिश्चैव धावातेऽथ विरेमतुः॥१२५॥

अथ राक्षसेषु तूष्णीम्भूतेषु राजानोऽतिबलिनः समागता ज्याबन्धाशक्ता अपसृत्य तस्थुः॥१२६॥

अथ ब्राह्मणाः समागताः॥१२७॥

अथ विश्वामित्रो धनुरादाय एकाङ्गुलपर्यन्तं सज्यं कृत्वा विरराम

निवृत्ताश्चापरे॥१२८॥

अथ दिनमात्रे धनुषि तूष्णीम्भूतेषु राघवः सहानुजैरागत्य धनुर्निरीक्ष्यास्पृशत्॥१२९॥

अथ राजकुमाराः शतशः समागताः सर्वाभरणभूषिता धनुर्दृष्ट्वापस्पृशुर्न चालनक्षमाः॥१३०॥

अथ दाशरथिप्रमुखाःकुमाराः समागताः॥१३१॥

अथ वेत्रझर्झरपाणयः समागमन्सर्वानेवापसारयामासुः॥१३२॥

अथ रामो लक्ष्मरणहस्तं गृहीत्वा सर्वाभरणभूषितो धनुरासाद्य स्पृष्ट्वा नत्वा प्रदक्षिणीकृत्य धनुरादायोद्दधार॥१३३॥

तदादानसमये सर्व एवैत्य सहासमूचुः अत्र भग्ना महारथा इति॥१३४॥

अथ स रामो धनुर्ज्यास्थानमवनमय्य धनुषि जानुं कृत्वा सज्यमेककरेणोत्पादयन्कोट्यामनामयत्॥१३५॥

अथ सज्जीकृतं दृष्ट्वा सर्व एव नासाग्रन्यस्ताङ्गुलयोऽभवन्॥१३६॥

रामोऽपि ज्यामन्वनादयत् तेन नादेन सर्वेषां मनांसि क्षुभितान्यासन्॥१३७॥

रामेणसज्यितं धनुरिति सर्वत्र वादः सञ्जातः॥१३८॥

जनकोऽपि सीतां रामाय ददौ राजभिश्च युद्धं कृत्वा तान्निर्ज्जित्य स्वपुरीमागात्॥१३९॥

अथैकदा दशरथो रामं यौवराज्येभिषिच्य सुखी बभूव सर्वप्रजारञ्जनाच्च रामो राजानुमत इति सर्वप्रजावादोऽभूत्॥१४०॥

अथ कैकयदेशाधिपतितनया सुवेषा रामं राजानमसहमाना राजानमुवाच मम वरदानावसर इति राजा चिन्तयत्किं देयमिति॥१४१॥

\uvacha{देव्युवाच}

चतुर्दशवर्षाणि रामो वनं विशतु पालयतु राज्यं भरतः॥१४२॥

राजानृतवचनदोषभयात्कथङ्कथमपि स्वीचकार॥१४३॥

अथ वसिष्ठं भावितयावोचत रामो वनाय निर्गच्छति अस्य किं वा भवेदिति विचार्य शुभाशुभं ब्रूहि॥१४४॥

वसिष्ठो विचार्य सहर्षं राजानमुवाच॥१४५॥

गत्वा वनं निखिलदानववीरहन्ता\\
शम्भोरनेकविधपूजनमातनोति।\\
सीतावियोगरुषितः कपिसेनया च\\
तीर्त्वोदधिं दशमुखं च निहन्ति रामः॥१४६॥

आगम्य राज्यं रघुनन्दनोऽपि\\
बहूनि वर्षाणि समातनोति।\\
प्रशस्तकीर्तिर्निखिलेऽपि लोके\\
शर्वेण देवेन चिरं न्यवात्सीत्॥१४७॥

सुपुत्रयुक्तो बहुयज्ञयाजी\\
परिवृढः सर्वगुणाधिकश्च॥१४८॥

इति वसिष्ठवचनं श्रुत्वा दशरथो रामगुणाननुस्मरन्नित्युवाच श्रेयो मे मरणं रामस्य निर्गमने इति॥१४९॥

अथ रामो मातरं पितरं गुरुं च वसिष्ठं पितृपत्नीर्नमस्कृत्य वनाय जगाम॥१५०॥

अथोपवने दिनमेकं स्थित्वा जटाः कारयित्वा वल्कलं वासो धृत्वैकोपवीती कृतदन्तशुद्धिरेकेनोपवीतेन जटां बद्ध्वा भस्मोद्धूलितसर्वाङ्गो भसित निष्ठुरकायो मुक्ताफलदाममणिव्यत्यस्त-रुद्राक्षमालामुरसि दधानोऽल्पभूषणाधिभूषित सीतासहायो लक्ष्मणानुचरो विवेश वनान्तरम्॥१५१॥

अथानेक राक्षसांस्तस्मिन्निजघान भवानिव निखिलं चकार सीतापहरणादिनिखिलमपि भवता यथातथा स्याथ सुग्रीवाश्रममृष्यमूकपर्वतं रामो जगाम निबिडच्छायाचूतवृक्षमासाद्य लक्ष्मणसहायः परिश्रयमकल्पयत्

वृक्षे तु धनुषी आरोप्यासीनलक्ष्मणाङ्के शिरः कृत्वा हरिचर्मशय्याशयनोलक्षिताङ्गीतिं शृण्वन्वृक्षफलं निरीक्षमाणो वानरमेकं मणिकुण्डलं हेमपिङ्गलं सदृढबद्धमौञ्जीकौपीनमच्छोपवीतिनमतिचञ्चलं फलमादायात्मनिविक्षिपन्तं पुष्पमञ्जरीश्च किरन्तं गानमनुकुर्वन्तं व्यजनेन रामं वीजयन्तमारुह्य शाखामपि तथा वीजयन्तमाबद्धचूतफलमात्रं रामो वीक्ष्य लक्ष्मणमभाषत॥१५२॥

लक्ष्मण कोऽयं कपिरिति॥१५३॥

लक्ष्मणोऽपि न जान इत्युवाच

अथ रामः समाहूय कस्य त्वं किं नामत्येपृच्छत्॥१५४॥

स च सुग्रीवस्य हनूमानित्युवाच॥१५५॥

रामं नत्वा सुग्रीवमेत्य नत्वा देवनारायणइवापरः पुरुषो युवा मेघश्यामो जटी आजानुबाहुरतियशस्वी सूर्यसङ्काशेन सहापरेण नरेण इहास्ते॥१५६॥

अथ तरुच्छायाधः संस्थितौ सर्वलक्षणसम्पन्नौ राजपुत्रौ दृष्ट्वा उक्तश्च ताभ्यां सुग्रीवाय निवेदयेति तत्त्वयि निवेदितम्॥१५७॥

अथ सुग्रीवः सत्वरमुत्थाय पुष्पसलिलादिद्रव्यमादाय पादप्रक्षालनादिकं कृत्वा फलानि समर्प्य व्यज्ञापयत्॥१५८॥

कौ युवां किमर्थमागतौ राजपुत्रौ तपस्विनाविति सुग्रीववचनमाकर्ण्य लक्ष्मणेनाभाषत रामः॥१५९॥

दशरथतनयावावां रामलक्ष्मणौ दुष्टनिग्रह शिष्टपरिपालनाय वनं गताविति॥१६०॥

अथ सुग्रीव आह युवयोरुपकारमपकारं कार्यमस्तीति लक्ष्यते॥१६१॥

अन्यथा सेनासमेतावागमिष्यतः लक्ष्मण आह अस्ति कार्यान्तरम् अमुष्य भार्या केनापहृता न ज्ञायते तामन्वेष्टुमागतौ तदेवावयोः कार्यमन्यदानुषङ्गिकम्

तदर्थमपि जलधिं तराव अपि पातालं प्रविशाव अपि नाकं साधयावः अपि महेन्द्रं पातयावः अपि बलिनं हनावः किमपि कुर्वहे॥१६२॥

\uvacha{सुग्रीव उवाच}

रावणेनापहृतया कयाचिद्ध्रियमाणागतया विभूषणानि कानिचित्परित्यक्ता निगतानि मया सङ्गृहीतानि तानि दर्शयामीत्याभाष्य रामं मन्दिरमागमय्य दर्शयामास॥१६३॥

रामोऽपि निरीक्ष्य निश्चित्य प्ररुद्य क्व गतोऽसौ रावण इति पप्रच्छ स च दक्षिणामाशां गत इति बभाषे॥१६४॥

अथ रामस्तेन सख्यमकरोत् अपृच्छच्च किमर्थमिह भार्याहीनः स्थित इति॥१६५॥

\uvacha{सुग्रीव उवाच}

मम भ्राता वाली महाबलो ममभार्यां राज्यं चापहृत्य किष्किन्धायामास्ते युद्धेन चाहं पराजितः तद्वधाय सर्वथा मम चिन्ता यथासौ त्वया निहन्यते तथाहमपि सागरं बद्ध्वा परतीरे लङ्कायां स्थितां सीतां रावणेनापहृतां तव समर्पयामीत्याभाष्य शपथं कृत्वा सुग्रीवो वालिनातिबलिना युद्धायाहूतेन युयुधे॥१६६॥

रामोऽप्यनन्तरमनिश्चयाद्वालिनं नाहनत्॥१६७॥

अथ सुग्रीवः पलायितो राममिदमभाषत॥१६८॥

तव चित्तमविज्ञाय प्रवृत्तोऽहमरणाय॥१६९॥

रामोपि युवयोर्विशेषाज्ञानान्मया तूष्णीं भूतं चिह्नितं त्वा निरीक्ष्यतं हन्मि॥१७०॥

अथ सुग्रीवश्चिह्नं कृत्वा वालिनं युद्धायाहूय समतिष्ठत॥१७१॥

तारा बभाषे वालिनम्॥१७२॥

सहायवानिव लक्ष्यते सुग्रीवो नोचेदेवं नाह्वयति ज्ञातं मया रामलक्ष्मणौ दशरथतनयौ नारायणांशौ भूभारावतरणाय समागतौ तावस्य सहायभूतौ॥१७३॥

\uvacha{वाल्युवाच}

नीतिमान्राम इति मया श्रुतः नहि बलवन्तं विहाय दुर्बलं भजते तादृशः समायातु वा रामः प्रतिपन्नमधिकं कृत्वा बिभेति वीरो यदि रामः स्वयं युद्धाय यातस्तदा युद्धं कर्तव्यमित्याभाष्य तारां सम्भाव्य सुग्रीवयुद्धाय निर्यातः॥१७४॥

अथ मुष्टियुद्धमन्योन्यमभूत्॥१७५॥

रामोऽपि वालिनं जघान॥१७६॥

पपात च वाल्याह चाशस्त्रयुद्धे वा बाणघातोऽथ शोणितसर्वाङ्गो बभूव॥१७७॥

अथ तारा चाङ्गदश्च समागत्य व्यथितौ बभूवतुः॥१७८॥

अथ राघवं वानराः समायाता वाल्युपान्ते निपेतूरुरुदुश्च॥१७९॥

अथ तारा रामं बभाषे शास्त्रकुशलाः शूरा धार्मिका राघवाः पुरा चापि राम कथं पापमकार्षीः॥१८०॥

न क्षत्रधर्म्मं जानीषे राजगणसेवितम्॥१८१॥

अन्योन्यं युद्ध्यतोर्युद्धे जयो वा मरणं भवेत्।\\
अन्यो यदितयोर्हन्याद्ब्रह्महा स निगद्यते॥१८२॥

किं वैरेण वालिनमाहनः किं वानरमांसाशया॥१८३॥

अभोज्यं वानरं मांसं यद्यात्मनोऽप्रियात्सुखाभावादपरेषामपि तथाभावं मन्यसे अहो विमोहाद्यदिमामादातुमिदं कृतमेकपत्नीव्रतं तव॥१८४॥

यदि रावणहृतां सीतामानेतुं सुग्रीवसहायाय कृतमेवमेव हा महदन्तरं बलवृद्धेन महाबलेन वालिना सद्भावेन दिनकरावर्तितान्तरे सीतामानेतुं समर्थेन स्मरणागतरावणदानमर्थेन वानरराजेन पञ्चाशत्परार्धवानरभल्लूकसेनावता आत्मकार्येण सिद्ध्यत इति किं सुग्रीवेणाल्पवीर्येण सप्तपरार्धसेनापतिना कपिना किं सिद्ध्यति कार्यं वचनवतः॥१८५॥

अहो ज्ञातं सर्वदेव भद्रं यदुक्तोसि॥१८६॥

वक्ति च रामः पृथिवीपतिना मया दुष्टनिग्रहणं कार्यं शिष्टपरिपालनं च वालिना सुग्रीवमहिषी रुमापहृता राज्यं च अतश्च न तादृग्वधे दोषः॥१८७॥

\uvacha{तारोवाच}

सुग्रीवोऽपि तर्हि वध्यो दुन्दुभिना युद्ध्यता वालिना बिलेप्रविष्टेन वत्सरं तत्रोषितं तदन्तरे च मामपहृत्य राज्यं च कृतं सुग्रीवेण तं पूर्वमपि पश्चात्तं हन्तु॥१८८॥

\uvacha{राम उवाच}

कियत्कालात्पूर्वमिदं च वद॥१८९॥

\uvacha{तारोवाच}

षष्टिवर्षसहस्रादर्वाकशीतितमे वर्षे रक्षोयुद्धे सुग्रीवेण राज्यमपहृतम्॥१९०॥

पुनश्च वर्षान्तरे प्राप्तेन वालिना सुग्रीवः पलायितः॥१९१॥

अपहृता तस्य भार्या राज्यं चापहृतम्॥१९२॥

तस्मिन्नेव दिने भवतः पितुर्दशरथस्याभिषेकः॥१९३॥

\uvacha{राघव उवाच}

मया पितुरनुशासनाद्राज्यगतदुष्टनिग्रहणं कृतम् गुरुवचनस्यानुल्लङ्घनीयत्वात्तदपहरणवेलायां यो राजासनाचरत्॥१९४॥

अथवा स्वतन्त्रौ मृगौ मृगयोर्हतश्च वाली मृगाणामन्योन्यं दारणाद्य जगुप्सा च॥१९५॥

यतो मम मृगयावदाथवा मृगाणाम् चलितस्थितबद्धानां चलद्भ्रान्तपलायिनाम् अथावसृजतासङ्गमुज्झिता मृगया तथा॥१९६॥

मृगयाशास्त्रविधितो मृगयेयं मया कृता।\\ 
दर्शनादर्शनाभ्यां च धावताधावता तथा॥१९७॥

अवरोहात्परं स्थानं सात्विकानां प्रभिद्यते।\\
राज्ञश्च मृगयाधर्मो विना आमिषभोजनम्॥१९८॥

अथ रामवचनमाकर्ण्य सर्व एव प्राकम्पयञ्छिरांसि॥१९९॥
वाली बभाषे राममञ्जलिमस्तके निधाय नमस्ते राम शृणु वचनं मम॥२००॥

शङ्खचक्रगदापाणिः पीतवासा जगद्गुरुः।\\
नारायणः स्वयं साक्षाद्भवानिति मया श्रुतम्॥२०१॥

त्वां योगिनश्चिन्तयन्ति त्वां यजन्ति च यज्विनः।\\
हव्यकव्यभुगे कस्त्वं पितृदेवस्वरूपधृक्॥२०२॥

मरणे चिन्तयानस्य त्वां विमुक्तिरदूरतः।\\
सत्वं मे दर्शनं प्राप्तो राम मे पापसङ्क्षयः॥२०३॥

गृहाण बाणं काकुत्स्थ व्यथितो भृशमस्म्यहम्॥२०४॥

अथ रामस्तथेति बाणमादाय वालिनमुवाच किमिष्टं दीयतां वद॥२०५॥

\uvacha{कपिरुवाच}

यदि प्रसन्नो भगवान्मम सद्गतिं देहि॥२०६॥

अयं सुग्रीवस्तथा रक्षणीयोऽङ्गदोऽथ तारा च मया पापिनापराधः कृतस्तत्फलमनुभूतम्॥२०७॥

अथ रामं पश्यन्नेव वाली ममार स्वर्गं च गतः॥२०८॥

अथ सुग्रीवं राज्येऽभिषिच्य स्वयं वनं विवेश॥२०९॥

अथ तेन सहायेन जलधिसमीपं गत्वा क्व लङ्का क्व सीता क्व चारातिरिति सुग्रीवमाह रामः॥२१०॥

अथ हनुमानाह प्रविश्य लङ्कां विचित्य सीतां सर्वतत्त्वमवगत्य युद्धं सन्धिर्वा कर्तव्यः तदुदधिलङ्घनाय किञ्चित्समादिशतु भगवान्॥२११॥

अथ सुग्रीवमाह रामः कथमेतद्घटत इति॥२१२॥

\uvacha{कपिरुवाच}

मम वानरा भल्लूप्रमुखाः कोटिशः सन्ति॥२१३॥

एकं नियुज्य सर्वमाकलय्य यथायुक्तं तथा करणीयम्॥२१४॥

अथ जाम्बवानाह हनूमानेको गच्छतु बुध्यतु लङ्काम्॥२१५॥

अथ हनूमानगमल्लङ्कापुरं विचित्य सीतामशोकवनिकायामासीनां तथा च सम्भाष्य विश्वासं कृत्वा वनं बभञ्ज वनरक्षकांश्च॥२१६॥

बद्धो रक्षसा लङ्कां दग्ध्वा उत्तरकूलं गत्वा रामं दृष्ट्वा वृत्तान्तं कथयित्वा तूष्णीमतिष्ठत्॥२१७॥

अथ रामः सर्वैर्विचारयामास जाम्बवानुवाच रामेण लङ्का कपिभिर्विनश्यतीति नारदेन ममोक्तम्॥२१८॥

अथ सागरोत्तरणे यत्नतया स्थेयम्॥२१९॥

अथ रामः शङ्करमाराध्य सर्वं निवेदयित्वा त्वदुक्तं करोमीति वचनमुक्त्वा शिवमभ्यर्च्य प्रणतो भूत्वा व्यजिज्ञपत्॥२२०॥

हे महादेव महाभूतग्रास महाप्रलयकारण महाहिभूषण महारुद्र शङ्कर परमेश्वर विरूपाक्ष नागयज्ञोपवीतकरि कृत्तिवसन ब्रह्मशिरः कपालमालाभरभूषण नरकास्थिभूषणभसित परनारायणप्रिय शुभचरित पञ्चब्रह्मादिदेव पञ्चानन चतुर्वदन वेदवेद्य भक्तसुलभा भक्तदुर्लभ परमानन्दविज्ञान पर पूषदन्तपातन दक्षशिरश्छेदन ब्रह्मपञ्चमशिरोहरण पार्वतीवल्लभ नारदोपगीयमान शुभचरित शर्व त्रिनेत्र त्रिशूलधर पिनाकपाणे कपर्दिन्ननेकरूपधर वृषभवाहन शुद्धस्फटिकसङ्काश चतुर्भुज नानायुध दक्षिणामूर्ते ईश्वर देवपते गङ्गाधर त्रिपुरहर श्रीशैलनिवास काशीनाथ केदारेश्वर भूषणसिद्धेश्वर गोकर्णेश्वर कनखलेश्वर पर्वतेश्वर चक्रप्रद बाणचिन्तापादक मुरहर पूजितचरणकमल सोमसोमभूषण सर्वज्ञ ज्योतिर्मय जगन्मय नमस्ते नमस्ते॥२२१॥

एवं स्तुवतो रामस्य पुरतो लिङ्गमध्यकोपेतस्तेजोमयमूर्तिराविर्बभूव॥२२२॥

अभयवानथ पुनः पद्मासनासीनमुमाधिष्ठिताङ्कमीशमामुक्तसर्वाभरणं सुकान्तिकिरीटिनं हैमवतीकटिस्पर्शं करद्वयेनाभयवरप्रदं तरङ्गितानेकदिशाभिः

पूर्णतेजस्विनं हासमुखं प्रसन्नवदनं ददर्श रामः॥२२३॥

परमेशितारं ननाम बद्धाञ्जलिपुनश्च दण्डवत्पपात॥२२४॥

अथ रामं परमेश्वरोऽपि वरं वृणु त्वं वरदोऽहमित्युक्तवान्॥२२५॥

\uvacha{राम उवाच}

लङ्कां गमिष्यामि समुद्रतरणे उपायमेकं मम देहि शम्भो॥२२६॥

\uvacha{शम्भुरुवाच}

ममाजगवं धनुरस्ति तत्कालरूपमविकल्पं वा भवति

तदारुह्य समुद्रं तीर्त्वा लङ्कामाप्नुहि॥२२७॥

रामस्तथेति निश्चित्य सस्माराजगवम्॥२२८॥

आगतं धनुस्ततश्च रामोऽपूजयत्॥२२९॥

अथ हरो धनुरादाय रामाय दत्तवान्॥२३०॥

रामोऽपि जलधावपातयत्॥२३१॥

आरुरुहुः सर्वे वानरा रामलक्ष्मणौ च षष्टिपरार्धं तेषामसङ्ख्येषु वानरेषु धनुरारूढेषु निकामं ययौ॥२३२॥

धनुस्तटं वानराश्च ततस्ततो गत्वा निरीक्षयामासुः॥२३३॥

अथातिकायो नाम रक्षः कपिबलमालोक्य रावणायोक्तवान्॥२३४॥

रावणोऽपि किं कपिभिः शाखामृगैः किं वा मानुषाभ्यां रामलक्ष्मणाभ्यां किमायातं

दैवागतमसकं भोजनमित्युवाच॥२३५॥

अथ सुग्रीवः पश्चिमावलम्बिनि भास्वति हनूमज्जाम्बवदादिमहाबलैश्चातिकायैरसङ्ख्यातैर्लङ्कापार्श्वं गत्वा उपवनं प्रवश्यि नानाफलानि खादित्वा पयः पीत्वोपवनरक्षिराक्षसान्विद्राव्य सर्वविपिनमेकैकशो गृहीत्वा प्राद्र वन्लङ्कां गोपुरं च गत्वा समारुह्य प्रासादं च विशीर्यैकैकशः केचित्स्तम्भमादाय रक्षोभिर्युयुधुः॥२३६॥

एके च शालां बभञ्जुर्गृहाणि चूर्णयामासुर्बालवृद्धस्त्रीजनादिकं सर्वमेव निजघ्नुः॥२३७॥

अथैकं प्राकारं निर्जितमाज्ञाय रावण इन्द्रजितं सन्दिदेश॥२३८॥

इन्द्रजिता च युद्धं वानराः कृत्वा भीताः पलायिताश्च॥२३९॥

अथ हनूमानखिलं निर्गतमाज्ञाय रावणं ज्ञात्वा वानरानाहूय निर्भर्त्स्य सेनां महतीं कारयित्वा दशमुखं कल्पयित्वा मोदयामास॥२४०॥

अथ खस्थ एवेन्द्रजिद्युयुधे न च वानरास्तं दृष्टवन्तः॥२४१॥

अथ हनूमज्जाम्बवन्तौ खमुत्पत्य पर्वतशिखराभ्यामिन्द्रजितं निजघ्नतुः॥२४२॥

अथ भुविपपात तं लक्ष्मणश्च यमलोकगामिनं चकार॥२४३॥

अथातिकायमहाकायौ वानरसैन्यं बहुशो हत्वा लक्ष्मणं पीडयित्वा रामेण संयुध्य सुग्रीवं कृत्वा हनूमज्जाम्बवद्भ्यां युयुधाते पराजितौ गृहीत्वा तौ च योद्धारावादाय रामसमीपं गत्वा रामाय न्यवेदयताम्॥२४४॥

अतिकायमभाषत रामः॥२४५॥

रावणस्य मम युद्धं ब्रूहि सचिवानामन्येषां महाभयानां च॥२४६॥

\uvacha{अतिकाय उवाच}

निश्चितमिदं पुरास्माभिः कार्यं सेनां विभागशः कृत्वा विद्युन्माली नाम राक्षसो महाबलो विचित्रयोधी दर्शनादर्शनयोधी वानरैः सर्वैरेक एव युध्यते॥२४७॥

अपरे च बलिनो महान्तः शिक्षितास्त्राश्चागता आवां च युवाभ्यां युध्यावो रावणः पुष्पकमारुह्यापरभागेन त्वामेव निहनिष्यति॥२४८॥

अन्ये च राक्षसाः कुम्भकर्णमुखाश्चात्मरूपं कृत्वा त्वां परिवार्य गृहीत्वा सीतायै दर्शयित्वा तत्सन्निधावेव हनिष्यति॥२४९॥

रामः प्राह अहो बलवतां किमसाध्यमेवं भवति दैवगतिः कुटिला सुग्रीवोऽतिकोपनः सक्रोधं दृष्ट्वा राममुवाच॥२५०॥

वध्यावेतौ न मोचनीयौ॥२५१॥

रामः प्राहावध्यौ मोचनीयावेतौ वसनानि भूषणान्यानयेत्युक्तमात्रे हनूमता तान्यानीतानि रामस्ताभ्यां दत्तवान्॥२५२॥

नत्वा यदेतल्लङ्काद्वारे दृश्यते दारुपञ्चवक्त्रं शुक्रेणोक्तमेतेन च्छिन्नेन रावणो हन्यते॥२५३॥

अथ च दारुच्छेदनसमनन्तरं पातालं गन्तव्यमिति भार्गवभाषितं शासनं लिखितम्॥२५४॥

तस्मात्त्वमिदं दार्वेकप्रयत्नैकबाणनिपातेन पञ्चधाच्छिन्धि ततस्तव शक्तिं ज्ञात्वा युद्धमतिदृढं कुर्वीमहि॥२५५॥

अथ भार्गववचोविज्ञाय रामः पूर्वकोट्यां स्पर्शमात्रेण सज्यं कृत्वा धनुषि बाणं संयोज्य रक्षोभ्यां हनूमताश्रावयन्नेव बाणं मुमोच॥२५६॥

बाणं धनुषश्चलितं तौ राक्षसौ बाणमार्गं निरीक्षमाणौ दारु बाणेन पञ्चधा च्छिन्नं निरीक्ष्य रामं व्यज्ञापयतामावयोः शिशवो रक्षणीयास्त्वयेति तथेत्याह रामः राक्षसौ लङ्कां प्रविष्टौ॥२५७॥

अथ प्राकारयुद्धं कर्तुं वानरा गत्वा सर्वतो वरणमात्रं हि पार्ष्णिभिः पादैर्जानुभिः करैः पृष्ठैश्च तलसमं कृत्वा द्वितीयप्राकारं गतास्तदा च रावणः समागत्य सर्वानेवेषुभिर्द्रावयित्वा तदनुगच्छन्राममगात्॥२५८॥

अथ राममपि पञ्चभिर्बाणैर्विव्याध॥२५९॥

अथ रामो दशभिर्बाणै रावणं सव्रणं चकार॥२६०॥

अनयोरतिदारुणमन्योन्यं युद्धं बभूव॥२६१॥

रावणो दशभिर्बाणैर्विव्याध॥२६२॥

अथ रामबाणैश्च क्षतशरीरो राक्षसः पलायनपरोऽभवत्॥२६३॥

वानरा लक्ष्मणश्च कोटिकोटिराक्षसानघ्नन्॥२६४॥

अथपरस्मिन्नहनि विभीषणो रावणं विचार्येदमुवाच॥२६५॥

तृतीयोपायकालोऽयं चतुर्थं न विचारय।\\
चतुर्थो विपरीतो न शस्तः शस्तार्थकारिणः॥२६६॥

परस्य चाऽत्मनः शक्तिं विदित्वा चाऽत्मनोऽधिकाम्।\\
तदा युद्धं प्रशस्तं स्याद्विपरीतं विनाशकम्॥२६७॥

रामेण बलिनानैव युद्धं ते दुर्बलस्य च।\\
एकेषुवालिहन्ताऽसौ वालिर्ज्ञातस्त्वया पुरा॥२६८॥

मारीचमेकबाणेन भवानपि पलायितः।\\
निहता राक्षसाः शूरा इन्द्र जिच्च सुतो हतः॥२६९॥

वरेण्यत्रितयं भग्नं तेन युद्धं च नैव ते।\\
दासभावमथो वाऽपि दत्त्वा सीतामथाऽप्नुहि॥२७०॥

गोपुरस्थं तथा दारु पञ्चवक्त्रमथेषुणा।\\
चिच्छेद पञ्चधा तेन रामस्त्वां मारयिष्यति॥२७१॥

त्वदर्थं बहवो नष्टा नाशमेष्यन्ति चापरे।\\
एको न्यायः सुखार्थाय न च मौढ्यं सहोदर॥२७२॥

मानुषीं मृत्युसंयुक्तामनिच्छन्तीं पतिव्रताम्।\\
पत्नीं बलवतश्चापि पूजयित्वा विसर्जय॥२७३॥

अनिच्छन्त्याः समायोगे भवेद्दुःखपरम्परा।\\
दुर्गन्धमलसंयुक्तो नारीसङ्गो जुगुप्सितः॥२७४॥

विरक्तिरथ चेज्जाता दुःखायाकार्यवर्तनम्।\\
अनुरागोयदि भवेन्मरणं नरकं ततः॥२७५॥

आत्मनो मरणं व्यर्थं तस्याश्चाद्य समागमे।\\
त्यागो वा मरणं तात धर्मपत्न्यास्तथा भवेत्॥२७६॥

एवमादि तथाऽन्यच्च कश्मलं सम्भविष्यति।\\
अन्यदाख्यामि ते वाक्यं सर्वेषां च प्रियं हितम्॥२७७॥

गत्वा रामान्तिकं नत्वा स्तुत्वा विज्ञाप्य राघवम्।\\
क्षम राम महावीर शरणागतवत्सल॥२७८॥

तामसा राक्षसाः सर्वे वयमेते सुपापिनः।\\
सीतापहारजं दोषं त्यक्त्वा पुत्रानवेहि नः॥२७९॥

त्वदधीना वयं राम रक्ष वा मारयेच्छया।\\
इत्युदीर्य पुरस्तस्य राघवस्य स्थिता वयम्॥२८०॥

स्थिरायुषो भविष्यामः स्थिरराज्या दशानन।\\
अथाऽहं रावणो वाक्यमहो नो राक्षसो भवान्॥२८१॥

न शूरो राजधर्मं च न च जानासि शाश्वतम्।\\
परनारीपरद्र व्यपरराज्यनिषेवया॥२८२॥

शूराणामुत्तमो धर्मो न षण्ढानां भवादृशाम्।\\
शत्रुपक्षं समालिग्य निर्गच्छेच्छा हि चेन्नृप॥२८३॥

अथ बिभीषणो मन्दिरं गत्वा रामान्तिकं गत्वा तं शरणमभजत्॥२८४॥

अथ रावणः पुरान्निर्गत्य रामेण लक्ष्मणवानरै राक्षसा अपि युयुधिरे॥२८५॥

अथ रावणं महाबलं हन्तुमशक्तोरामो विभीषणमुखमवलोक्य तदुक्तचिह्नपदं बाणेन निर्भिद्यामारयत्॥२८६॥

अथ कुम्भकर्णो महागदामादाय सर्वं निष्पाद्य वानराननेकशो भक्षयित्वा रामोत्तमाङ्गं गदयाऽहन्॥२८७॥

अथ रामो निशितबाणशतेन तमहन्ममार कुम्भकर्णः॥२८८॥

अथ विभीषणेन रावणादेः श्राद्धादिकं कारयित्वा शिवालयं तन्नाम्ना कारयित्वा तमेव लङ्काराज्ये विभीषणमभिषिच्य सीतामग्निप्रवेशशुद्धामुमामहेश्वराभ्यां नमयित्वा पुरहरेण दत्ताखिलामृतबलायुष्यः सुपुष्पकमारुह्य जलधिमुत्तीर्य पारावारतटे सेनां समवस्थाप्य शिवप्रतिष्ठां तत्र कृत्वा मुनिभिर्देवैरभ्यर्चितोऽयोध्यामगमत्॥२८९॥

अथ भरतादिसमुपेतो नागरैर्वसिष्ठेन मुनिभिश्चाभ्यर्चितःस्वगृहमगमत्॥२९०॥

आत्मनाऽगतानिन्द्रा दिदेवानासनादिनाऽभ्यर्च्य वानरान्सम्पूज्य मुक्तजटोऽभिषिक्तो राज्ये रावणवधहर्षिता देवा राममूचुः॥२९१॥

त्वयाऽत्मराज्ये स्थापिता वयं नः सर्वदा परिपालय त्वमादिनारायणो देवो निखिलदुष्टनिग्रहार्थमवतीर्णो रावणं सबान्धवं हत्वा लोकत्रयरक्षकोऽसि श्रिया सह सुखी भवेत्युदीर्य स्वर्गं गताः॥२९२॥

अथायोध्यावासिनो रामं प्रहर्षिता ऊचुः॥२९३॥

हत्वा शत्रून्समायातो दृष्ट्वा प्राप्तोऽसि वै शिवम्।\\
दिष्ट्या त्वं राजसे राम दिष्ट्या पालयसे प्रजाः॥२९४॥

त्वया यज्ञाः करिष्यन्ते त्वया धर्मो विवर्धते।\\
इतिपौरवचःश्रुत्वारामो राजीवलोचनः॥२९५॥

वस्त्रादिभिरथोसर्वान्नागरान्समपूजयत्।\\
मुनीनुवाचधर्म्मात्मापूजयित्वाखिलैर्जनैः॥२९६॥

कच्चित्तपःसमृद्धंवःकच्चिद्यज्ञःस्वनुष्ठितः।\\
कच्चित्स्वदारनिरताःकच्चिदीशोभिपूज्यते॥२९७॥

कच्चित्सुप्रजसोभार्याःकच्चित्सर्वंसुखोत्तरम्॥२९८॥

\uvacha{मुनय ऊचुः} 

त्वयि राजनि काकुत्स्थ सर्वं स्वस्थं तपस्विनाम्।\\
गच्छामहे पदमितः किं वा त्वं मन्यसे नृप॥२९९॥

\uvacha{राम उवाच}

यस्य विप्राः प्रसीदन्ति तस्य शम्भुः प्रसीदति।\\
यस्य प्रसीदतीशानस्तस्य भद्रं भविष्यति॥३००॥

तत्कृत्वा भोजनमिह गन्तुमर्हा अनन्तरम्।\\
अथेत्युक्त्वा मुनिगणाः कृत्वा भोजनमुत्तमम्॥३०१॥

अभिवर्ध्यतमाशीर्भिर्हृष्टाः स्वं स्वं पदं ययुः।\\
रामोऽपि परमप्रीतः सभार्यश्च सहानुजः॥३०२॥

अकण्टकं स कृतवान्राज्यं सर्वजनप्रियः।\\
शृणोत्येतदुपाख्यानं यः कश्चिदपि पातकी॥३०३॥

सर्वपापविनिर्मुक्तः परं ब्रह्माधिगच्छति।\\
न दुर्गतिर्भवेत्तस्य यश्चेदं स्मरते नरः॥३०४॥

यश्चापि कीर्तयेत्तस्य एवमेतदुदीरितम्॥३०५॥

\end{flushleft}

इति श्रीपद्मपुराणे पातालखण्डे शिवराघवसंवादे पुराकल्पीयरामायणकथनं नाम षोडशोत्तरशततमोऽध्यायः॥११६॥


    \sect{द्विचत्वारिंशदधिक-द्विशततमोऽध्यायः --- रामस्यायोध्याप्रवेशः}

\src{पद्म-पुराणम्}{सृष्टिखण्डम्}{अध्यायः २४२--२४४}{}
% \tags{concise, complete}
\notes{}
\textlink{https://sa.wikisource.org/wiki/पद्मपुराणम्/खण्डः_५_(पातालखण्डः)/अध्यायः_००१}
\translink{https://www.wisdomlib.org/hinduism/book/the-padma-purana/d/doc365826.html}

\storymeta


\uvacha{रुद्र उवाच}

\twolineshloka
{स्वायम्भुवो मनुः पूर्वं द्वाशार्णं महामनुम्}
{जजाप गोमतीतीरे नैमिषे विमले शुभे}% १

\twolineshloka
{तेन वर्षसहस्रेण पूजितः कमलापतिः}
{मत्तो वरं वृणीष्वेति तं प्राह भगवान्हरिः}% २

\onelineshloka*
{ततः प्रोवाच हर्षेण मनुः स्वायम्भुवो हरिम्}

\uvacha{मनुरुवाच}
\onelineshloka
{पुत्रत्वं भज देवेश त्रीणि जन्मानि चाच्युत}% ३

\onelineshloka*
{त्वां पुत्रलालसत्वेन भजामि पुरुषोत्तमम्}

\uvacha{रुद्र उवाच}
\onelineshloka
{इत्युक्तस्तेन लक्ष्मीशः प्रोवाच सुमहागिरा}% ४

\uvacha{विष्णुरुवाच}

\twolineshloka
{भविष्यति नृपश्रेष्ठ यत्ते मनसि काङ्क्षितम्}
{ममैव च महत्प्रीतिस्तव पुत्रत्वहेतवे}% ५

\twolineshloka
{स्थितिप्रयोजने काले तत्र तत्र नृपोत्तम}
{त्वयि जाते त्वहमपि जातोस्मि तव सुव्रत}% ६

\twolineshloka
{परित्राणाय साधूनां विनाशाय च दुष्कृताम्}
{धर्म्मसंस्थापनार्थाय सम्भवामि तवानघ}% ७

\uvacha{रुद्र उवाच}

\twolineshloka
{एवं दत्वा वरं तस्मै तत्रैवान्तर्दधे हरिः}
{अस्याभूत्प्रथमं जन्म मनोः स्वायम्भुवस्य च}% ८

\twolineshloka
{रघूणामन्वये पूर्वं राजा दशरथो ह्यभूत्}
{द्वितीयो वसुदेवोऽभूद्वृष्णीनामन्वये विभुः}% ९

\twolineshloka
{कलेर्दिव्यसहस्राब्दप्रमाणस्यान्त्यपादयोः}
{शम्भलग्रामकं प्राप्य ब्राह्मणः सञ्जनिष्यति}% १०

\twolineshloka
{कौशल्या समभूत्पत्नी राज्ञो दशरथस्य हि}
{यदोर्वंशस्य सेवार्थं देवकी नाम विश्रुता}% ११

\twolineshloka
{हरिव्रतस्य विप्रस्य भार्य्या देवप्रभा पुनः}
{एवं मातृत्वमापन्ना त्रीणि जन्मानि शार्ङ्गिणः}% १२

\twolineshloka
{पूर्वं रामस्य चरितं वक्ष्यामि तव सुव्रते}
{यस्य स्मरणमात्रेण विमुक्तिः पापिनामपि}% १३

\twolineshloka
{हिरण्यकहिरण्याक्षौ द्वितीयं जन्मसंश्रितौ}
{कुम्भकर्ण दशग्रीवावजायेतां महाबलौ}% १४

\twolineshloka
{पुलस्त्यस्य सुतो विप्रो विश्रवा नाम धार्मिकः}
{तस्य पत्नी विशालाक्षी राक्षसेन्द्र सुताऽनघे}% १५

\twolineshloka
{सुकेशितनया सा स्यात्सुमाली दानवस्य च}
{केकसी नाम कन्यासीत्तस्य भार्या दृढव्रता}% १६

\twolineshloka
{कामोद्रिक्ता तु सा देवी सन्ध्याकाले महामुनिम्}
{रमयामास तन्वङ्गी यथेष्टं शुभदर्शना}% १७

\twolineshloka
{कामात्सन्ध्याभवाद्यत्वात्तस्यां जातौ महाबलौ}
{रावणः कुम्भकर्णश्च राक्षसौ लोकविश्रुतौ}% १८

\twolineshloka
{कन्या शूर्पणखा नाम जातातिविकृतानना}
{कस्यचित्त्वथ कालस्य तस्यां जातो विभीषणः}% १९

\twolineshloka
{सुशीलो भगवद्भक्तः सत्यवाग्धर्म्मवाञ्शुचिः}
{रावणः कुम्भकर्णश्च हिमवत्पर्वतोत्तमे}% २०

\twolineshloka
{महोग्रतपसा मां वै पूजयामासतुर्भृशम्}
{रावणस्त्वथ दुष्टात्मा स्वशिरःकमलैः शुभैः}% २१

\twolineshloka
{पूजयामास मां देवि दारुणेनैव कर्म्मणा}
{ततस्तमब्रुवं सुभ्रूः प्रहृष्टेनान्तरात्मना}% २२

\twolineshloka
{वरं वृणीष्व मे वत्स यत्ते मनसि वर्त्तते}
{ततः प्रोवाच दुष्टात्मा देवदानव रक्षसाम्}% २३

\twolineshloka
{अवध्यत्वं प्रदेहीति सर्वलोकजिगीषया}
{ततोऽहं दत्तवांस्तस्मै राक्षसाय दुरात्मने}% २४

\twolineshloka
{देवदानवयक्षाणामवध्यत्वं वरानने}
{राक्षसोऽसौ महावीर्यो वरदानात्तु गर्वितः}% २५

\twolineshloka
{त्रींल्लोकान्पीडयामास देवदानवमानुषान्}
{तेन सम्बाध्यमानाश्च देवा ब्रह्मपुरोगमाः}% २६

\twolineshloka
{भयार्त्ताः शरणं जग्मुरीश्वरं कमलापतिम्}
{ज्ञात्वाथ वेदनां तेषामभयाय सनातनः}% २७

\onelineshloka*
{उवाच त्रिदशान्सर्वान्ब्रह्मरुद्रपुरोगमान्}

\uvacha{श्रीभगवानुवाच}
\onelineshloka
{राज्ञो दशरथस्याहमुत्पत्स्यामि रघोः कुले}% २८

\twolineshloka
{वधिष्यामि दुरात्मानं रावणं सह बान्धवम्}
{मानुषं वपुरास्थाय हन्मि दैवतकण्टकम्}% २९

\twolineshloka
{नन्दिशापाद्भवन्तोऽपि वानरत्वमुपागताः}
{कुरुध्वं मम साहाय्यं गन्धर्वाप्सरसोत्तमाः}% ३०

\uvacha{रुद्र उवाच}

\twolineshloka
{इत्युक्ता देवतास्सर्वा देवदेवेन विष्णुना}
{वानरत्वमुपागम्य जज्ञिरे पृथिवीतले}% ३१

\twolineshloka
{भार्गवेण प्रदत्ता तु महीसागरमेखला}
{दत्ता महर्षिभिः पूर्वं रघूणां सुमहात्मनाम्}% ३२

\twolineshloka
{वैवस्वतमनोः पुत्रो राज्ञां श्रेष्ठो महाबलः}
{इक्ष्वाकुरिति विख्यातस्सर्वधर्म्मविदांवरः}% ३३

\twolineshloka
{तदन्वये महातेजा राजा दशरथो बली}
{अजस्य नृपतेः पुत्रः सत्यवान्शीलवान्शुचिः}% ३४

\twolineshloka
{स राजा पृथिवीं सर्वां पालयामास वीर्य्यतः}
{राज्येषु स्थापयामास सर्वान्पार्थिवसत्तमान्}% ३५

\twolineshloka
{कोशलस्य नृपस्याथ पुत्री सर्वाङ्गशोभना}
{कौशल्या नाम तां कन्यामुपयेमे स पार्थिवः}% ३६

\twolineshloka
{मागधस्य नृपस्याथ तनया च शुचिस्मिता}
{सुमित्रा नाम नाम्ना च द्वितीया तस्य भामिनी}% ३७

\twolineshloka
{तृतीया केकयस्याथ नृपतेर्दुहिता तथा}
{भार्य्याभूत्पद्मपत्राक्षी केकयी नाम नामतः}% ३८

\twolineshloka
{ताभिः स्म राजा भार्याभिस्तिसृभिर्धर्मसंयुतः}
{रमयामास काकुत्स्थः पृथिवीं चानुपालयन्}% ३९

\twolineshloka
{अयोध्या नाम नगरी सरयूतीर संस्थिता}
{सर्वरत्नसुसम्पूर्णा धनधान्यसमाकुला}% ४०

\twolineshloka
{प्राकारगोपुरैर्जुष्टा हेमप्राकारसङ्कुला}
{उत्तमैर्नागतुरगैर्महेन्द्रस्य यथा पुरी}% ४१

\twolineshloka
{तस्यां राजा स धर्मात्मा उवास मुनिसत्तमैः}
{पुरोहितेन विप्रेण वसिष्ठेन महात्मना}% ४२

\twolineshloka
{राज्यं चकारयामास सर्वं निहतकण्टकम्}
{यस्मादुत्पत्स्यते तस्यां भगवान्पुरुषोत्तमः}% ४३

\twolineshloka
{तस्मात्तु नगरी पुण्या साप्ययोध्येति कीर्तिता}
{नगरस्य परं धाम्नो नाम तस्याप्यभूच्छुभे}% ४४

\twolineshloka
{यत्रास्ते भगवान्विष्णुस्तदेव परमं पदम्}
{तत्र सद्यो भवेन्मोक्षः सर्वकर्म्मनिकृन्तनः}% ४५

\twolineshloka
{जाते तत्र महाविष्णौ नराः सर्वे मुदं ययुः}
{स राजा पृथिवीं सर्वां पालयित्वा शुभानने}% ४६

\twolineshloka
{अयजद्वैष्णवेष्ट्या च पुत्रार्थी हरिमच्युतम्}
{तेन सम्पूजितः श्रीशो राजा सर्वगतो हरिः}% ४७

\twolineshloka
{वैष्णवेन तु यज्ञेन वरदः प्राह केशवः}
{तस्मिन्नाविरभूदग्नौ यज्ञरूपो हरिस्तदा}% ४८

\twolineshloka
{शुद्धजाम्बूनदप्रख्यः शङ्खचक्रगदाधरः}
{शुक्लाम्बरधरः श्रीमान्सर्वभूषणभूषितः}% ४९

\twolineshloka
{श्रीवत्सकौस्तुभोरस्को वनमालाविभूषितः}
{पद्मपत्रविशालाक्षश्चतुर्बाहुरुदारधीः}% ५०

\twolineshloka
{सव्याङ्कस्थ श्रिया सार्द्धमाविरासीद्रमेश्वरः}
{वरदोस्मीति तं प्राह राजानं भक्तवत्सलः}% ५१

\twolineshloka
{तं दृष्ट्वा सर्वलोकेशं राजा हर्षसमाकुलः}
{ववन्दे भार्य्यया सार्द्धं प्रहृष्टेनान्तरात्मना}% ५२

\twolineshloka
{प्राञ्जलिः प्रणतो भूत्वा हर्षगद्गदया गिरा}
{पुत्रत्वं मे भजेत्याह देवदेवं जनार्दनम्}% ५३

\onelineshloka*
{ततः प्रसन्नो भगवान्प्राह राजानमच्युतः}

\uvacha{विष्णुरुवाच}
\onelineshloka
{उत्पत्स्येऽहं नृपश्रेष्ठ देवलोकहिताय वै}% ५४

\twolineshloka
{परित्राणाय साधूनां राक्षसानां वधाय च}
{मुक्तिं प्रदातुं लोकानां धर्म्मसंस्थापनाय च}% ५५

\uvacha{महादेव उवाच}

\twolineshloka
{इत्युक्त्वा पायसं दिव्यं हेमपात्रस्थितं शृतम्}
{लक्ष्म्याहस्तस्थितं शुभ्रं पार्थिवाय ददौ हरिः}% ५६

\uvacha{विष्णुरुवाच}

\twolineshloka
{इदं वै पायसं राजन्पत्नीभ्यस्तव सुव्रत}
{देहि ते तनयास्तासु उत्पत्स्यन्ते मदङ्गजाः}% ५७

\uvacha{महादेव उवाच}

\twolineshloka
{इत्युक्त्वा मुनिभिः सर्वैः स्तूयमानो जनार्दनः}
{स्वात्मानं दर्शयित्वाथ तथैवान्तरधीयत}% ५८

\twolineshloka
{स राजा तत्र दृष्ट्वा च पत्नीं ज्येष्ठां कनीयसीम्}
{विभज्य पायसं दिव्यं प्रददौ सुसमाहितः}% ५९

\twolineshloka
{एतस्मिन्नन्तरे पत्नी सुमित्रा तस्य मध्यमा}
{तत्समीपं प्रयाता सा पुत्रकामा सुलोचना}% ६०

\twolineshloka
{तां दृष्ट्वा तत्र कौशल्या कैकेयी च सुमध्यमा}
{अर्द्धमर्द्धं प्रददतुस्ते तस्यै पायसं स्वकम्}% ६१

\twolineshloka
{तत्प्राश्य पायसं दिव्यं राजपत्न्यः सुमध्यमाः}
{सम्पन्नगर्भाः सर्वास्ता विरेजुः शुभ्रवर्च्चसः}% ६२

\twolineshloka
{तासां स्वप्नेषु देवेशः पीतवासा जनार्दनः}
{शङ्खचक्रगदापाणिराविर्भूतस्तदा हरिः}% ६३

\twolineshloka
{अस्मिन्काले मनोरम्ये मधुमासि शुचिस्मिते}
{शुक्ले नवम्यां विमले नक्षत्रेऽदितिदैवते}% ६४

\twolineshloka
{मध्याह्नसमये लग्ने सर्वग्रहशुभान्विते}
{कौसल्या जनयामास पुत्रं लोकेश्वरं हरिम्}% ६५

\twolineshloka
{इन्दीवरदलश्यामं कोटिकन्दर्प्पसन्निभम्}
{पद्मपत्रविशालाक्षं सर्वाभरणशोभितम्}% ६६

\twolineshloka
{श्रीवत्सकौस्तुभोरस्कं सर्वाभरणभूषितम्}
{उद्यद्दिनकरप्रख्यकुण्डलाभ्यां विराजितम्}% ६७

\twolineshloka
{अनेकसूर्य्यसङ्काशं तेजसा महता वृतम्}
{परेशस्य तनो रम्यं दीपादुत्पन्नदीपवत्}% ६८

\twolineshloka
{ईशानं सर्वलोकानां योगिध्येयं सनातनम्}
{सर्वोपनिषदामर्थमनन्तं परमेश्वरम्}% ६९

\twolineshloka
{जगत्सर्गस्थितिलये हेतुभूतमनामयम्}
{शरण्यं सर्वभूतानां सर्वभूतमयं विभुम्}% ७०

\twolineshloka
{समुत्पन्ने जगन्नाथे देवदुन्दुभयो दिवि}
{विनेदुः पुष्पवर्षाणि ववर्षुः सुरसत्तमाः}% ७१

\twolineshloka
{प्रजापतिमुखा देवा विमानस्था नभस्तले}
{तुष्टुवुर्मुनिभिः सार्द्धं हर्षपूर्णाङ्गविह्वलाः}% ७२

\twolineshloka
{जगुर्गन्धर्वपतयो ननृतुश्चाप्सरोगणाः}
{ववुः पुण्यशिवा वाताः सुप्रभोभूद्दिवाकरः}% ७३

\twolineshloka
{जज्वलुश्चाग्नयः शान्ता विमलाश्च दिशोदश}
{ततस्स राजा हर्षेण पुत्रं दृष्ट्वा सनातनम्}% ७४

\twolineshloka
{पुरोधसा वसिष्ठेन जातकर्म्मतदाऽकरोत्}
{नाम चास्मै ददौ रम्यं वसिष्ठो भगवांस्तदा}% ७५

\twolineshloka
{श्रियः कमलवासिन्या रमणोऽयं महान्प्रभुः}
{तस्माच्छ्रीराम इत्यस्य नाम सिद्धं पुरातनम्}% ७६

\twolineshloka
{सहस्रनाम्नां श्रीशस्य तुल्यं मुक्तिप्रदं नृणाम्}
{विष्णुना स समुत्पन्नो विष्णुरित्यभिधीयते}% ७७

\twolineshloka
{एवं नामास्य दत्वाथ वसिष्ठो भगवानृषिः}
{परिणीय नमस्कृत्य स्तुत्वा स्तुतिभिरेव च}% ७८

\twolineshloka
{सङ्कीर्त्य नामसाहस्रं मङ्गलार्थं महात्मनः}
{विनिर्ययौ महातेजास्तस्मात्पुण्यतमाद्गृहात्}% ७९

\twolineshloka
{राजाथ विप्रमुख्येभ्यो ददौ बहुधनं मुदा}
{गवामयुतदानं च कारयामास धर्म्मतः}% ८०

\twolineshloka
{ग्रामाणां शतसाहस्रं ददौ रघुकुलोत्तमः}
{वस्त्रैराभरणैर्दिव्यैरसङ्ख्येयैर्धनैरपि}% ८१

\twolineshloka
{विष्णोस्सन्तुष्टये तत्र तर्प्पयामास भूसुरान्}
{कौसल्या च सुतं दृष्ट्वा रामं राजीवलोचनम्}% ८२

\twolineshloka
{फुल्लहस्तारविन्दाभं पद्महस्ताम्बुजान्वितम्}
{तस्य श्रीपादकमले पद्माब्जे च वरानने}% ८३

\twolineshloka
{शङ्खचक्रगदापद्मध्वजवस्त्रादिचिह्निते}
{दृष्ट्वा वक्षसि श्रीवत्सं कौस्तुभं वनमालया}% ८४

\twolineshloka
{तस्याङ्गे सा जगत्सर्वं सदेवासुरमानुषम्}
{स्मितवक्त्रे विशालाक्षी भुवनानि चतुर्दश}% ८५

\twolineshloka
{निश्वासे तस्य वेदांश्च सेतिहासान्महात्मनः}
{द्वीपानब्धीन्गिरींस्तस्य जघने वरवर्णिनि}% ८६

\twolineshloka
{नाभ्यां ब्रह्मशिवौ तस्य कर्णयोश्च दिशः शुभाः}
{नेत्रयोर्वह्निसूर्यौ च घ्राणे वायुं महाजवम्}% ८७


\threelineshloka
{सर्वोपनिषदामर्थं दृष्ट्वा तस्य विभूतयः}
{कृत्स्ना भीता वरारोहा प्रणम्य च पुनः पुनः}
{हर्षाश्रुपूर्णनयना प्राञ्जलिर्वाक्यमब्रवीत्}% ८८

\uvacha{कौशल्योवाच}

\twolineshloka
{धन्यास्मि देवदेवेश लब्ध्वा त्वां तनयं प्रभो}
{प्रसीद मे जगन्नाथ पुत्रस्नेहं प्रदर्शय}% ८९

\uvacha{ईश्वर उवाच}

\twolineshloka
{एवमुक्तो हृषीकेशो मात्रा सर्वगतो हरिः}
{मायामानुषतां प्राप्य शिशुभावाद्रुरोद सः}% ९०

\twolineshloka
{अथ प्रमुदिता देवी कौशल्या शुभलक्षणा}
{पुत्रमालिङ्ग्य हर्षेण स्तन्यं प्रादात्सुमध्यमा}% ९१

\twolineshloka
{तस्याः स्तन्यं पपौ देवो बालभावात्सनातनः}
{उवास मातुरुत्सङ्गे जगद्भर्ता महाविभुः}% ९२

\twolineshloka
{देशे तस्मिञ्छुशुभे रम्ये सर्वकामप्रदे नृणाम्}
{उत्सवं चक्रिरे पौरा हृष्टा जनपदा नराः}% ९३

\twolineshloka
{कैकेय्यां भरतो जज्ञे पाञ्चजन्यांशचोदितः}
{सुमित्रा जनयामास लक्ष्मणं शुभलक्षणम्}% ९४

\twolineshloka
{शत्रुघ्नं च महाभागा देवशत्रुप्रतापनम्}
{अनन्तांशेन सम्भूतो लक्ष्मणः परवीरहा}% ९५

\twolineshloka
{सुदर्शनांशाच्छत्रुघ्नः सञ्जज्ञेऽमितविक्रमः}
{ते सर्वे ववृधुस्तत्र वैवस्वतमनोः कुले}% ९६

\twolineshloka
{संस्कृतास्ते सुताः सम्यग्वसिष्ठेन महौजसा}
{अधीतवेदास्ते सर्वे श्रुतवन्तस्तथा नृपाः}% ९७

\twolineshloka
{सर्वशास्त्रार्थतत्वज्ञा धनुर्वेदे च निष्ठिताः}
{बभूवुः परमोदारा लोकानां हर्षवर्द्धनाः}% ९८

\twolineshloka
{युग्मं बभूवतुस्तत्र राजानौ रामलक्ष्मणौ}
{तथा भरतशत्रुघ्नौ तयोर्युग्मं बभूव ह}% ९९

\twolineshloka
{अथ लोकेश्वरी लक्ष्मीर्जनकस्य निवेशने}
{शुभक्षेत्रे हलोद्धाते सुनासीरे शुभेक्षणे}% १००

\twolineshloka
{बालार्ककोटिसङ्काशा रक्तोत्पलकराम्बुजा}
{सर्वलक्षणसम्पन्ना सर्वाभरणभूषिता}% १०१

\twolineshloka
{धृत्वा वक्षसि चार्वङ्गी मालामम्लानपङ्कजाम्}
{सीतामुखे समुत्पन्ना बालभावेन सुन्दरी}% १०२

\twolineshloka
{तां दृष्ट्वा जनको राजा कन्यां वेदमयीं शुभाम्}
{उद्धृत्यापत्यभावेन पुपोष मिथिलापतिः}% १०३

\twolineshloka
{जनकस्य गृहे रम्ये सर्वलोकेश्वरप्रिया}
{ववृधे सर्वलोकस्य रक्षणार्थं सुरेश्वरी}% १०४

\twolineshloka
{एतस्मिन्नन्तरे देवि कौशिको लोकविश्रुतः}
{सिद्धाश्रमे महापुण्ये भागीरथ्यास्तटे शुभे}% १०५

\twolineshloka
{क्रतुप्रवरमारेभे यष्टुं तत्र महामुनिः}
{वर्त्तमानस्य तस्यास्य यज्ञस्याथ द्विजन्मनः}% १०६

\twolineshloka
{क्रतुविध्वंसिनोऽभूवन्रावणस्य निशाचराः}
{कौशिकश्चिन्तयित्वाथ रघुवंशोद्भवं हरिम्}% १०७

\twolineshloka
{आनेतुमैच्छद्धर्मात्मा लोकानां हितकाम्यया}
{स गत्वा नगरीं रम्यामयोध्यां रघुपालिताम्}% १०८

\twolineshloka
{नृपश्रेष्ठं दशरथं ददर्श मुनिसत्तमः}
{राजापि कौशिकं दृष्ट्वा प्रत्युत्थाय कृताञ्जलिः}% १०९

\twolineshloka
{पुत्रैः सह महातेजा ववन्दे मुनिसत्तमम्}
{धन्योऽहमस्मीति वदन्हर्षेण रघुनन्दनम्}% ११०

\twolineshloka
{अर्चयामास विधिना निवेश्य परमासने}
{परिणीय नमस्कृत्य किं करोमीत्युवाच तम्}% १११

\onelineshloka*
{ततः प्रोवाच हृष्टात्मा विश्वामित्रो महातपाः}

\uvacha{विश्वामित्र उवाच}
\onelineshloka
{देहि मे राघवं राजन्रक्षणार्थं क्रतोर्मम}% ११२

\twolineshloka
{साफल्यमस्तु मे यज्ञे राघवस्य समीपतः}
{तस्माद्रामं रक्षणार्थं दातुमर्हसि भूपते}% ११३

\uvacha{ईश्वर उवाच}

\twolineshloka
{तच्छ्रुत्वा मुनिवर्य्यस्य वाक्यं सर्वविदां वरः}
{प्रददौ मुनिवर्य्याय राघवं सह लक्ष्मणम्}% ११४

\twolineshloka
{आदाय राघवं तत्र विश्वामित्रो महातपाः}
{स्वमाश्रममभिप्रीतः प्रययौ द्विजसत्तमः}% ११५

\twolineshloka
{ततः प्रहृष्टास्त्रिदशाः प्रयाते रघुसत्तमे}
{ववृषुः पुष्पवर्षाणि तुष्टुवुश्च महौजसः}% ११६

\twolineshloka
{अथाजगाम हृष्टात्मा वैनतेयो महाबलः}
{अदृश्यभूतो भूतानां सम्प्राप्य रघुसत्तमम्}% ११७

\twolineshloka
{ताभ्यां दिव्ये च धनुषी तूणौ चाक्षयसायकौ}
{दिव्यान्यस्त्राणि शस्त्राणि दत्वा च प्रययौ द्विजः}% ११८

\twolineshloka
{तौ रामलक्ष्मणौ वीरौ कौशिकेन महात्मना}
{गच्छन्ती ज्ञापितारण्ये राक्षसी घोरदर्शना}% ११९

\twolineshloka
{नाम्ना तु ताडका देवि भार्या सुन्दस्य रक्षसः}
{जघ्नतुस्तां महावीरौ बाणैर्दिव्यधनुश्च्युतैः}% १२०

\twolineshloka
{निहता राघवेणाथ राक्षसी घोरदर्शना}
{त्यक्त्वा तनुं घोररूपां दिव्यरूपा बभूव सा}% १२१

\twolineshloka
{जाज्वल्यमानावपुषा सर्वाभरणविभूषिता}
{प्रययौ वैष्णवं लोकं प्रणम्य च रघूत्तमौ}% १२२

\twolineshloka
{तां हत्वा राघवः श्रीमान्कौशिकस्याश्रमं शुभम्}
{प्रविवेश महातेजा लक्ष्मणेन महात्मना}% १२३

\twolineshloka
{ततः प्रहृष्टा मुनयः प्रत्युद्गम्य रघूत्तमम्}
{निवेश्य पूजयामासुरर्घाद्यैः परमात्मने}% १२४

\twolineshloka
{कौशिकः कृतदीक्षस्तु यंष्टुं यज्ञमनुत्तमम्}
{आरेभे मुनिभिः सार्द्धं विधिना मुनिसत्तमः}% १२५

\twolineshloka
{वर्त्तमाने महायज्ञे मारीचो नाम राक्षसः}
{भ्रात्रा सुबाहुना तत्र विघ्नं कर्तुमवस्थितः}% १२६

\twolineshloka
{दृष्ट्वा तौ राक्षसौ घोरौ राघवः परवीरहा}
{जघानैकेन बाणेन सुबाहुं राक्षसेश्वरम्}% १२७

\twolineshloka
{पवनास्त्रेण महता मारीचं स निशाचरम्}
{सागरे पातयामास शुष्कपर्णमिवानिलः}% १२८

\twolineshloka
{स रामस्य महावीर्य्यं दृष्ट्वा राक्षससत्तमः}
{न्यस्तशस्त्रस्तपस्तप्तुं प्रययौ महादाश्रमम्}% १२९

\twolineshloka
{विश्वामित्रो महातेजाः समाप्ते महति क्रतौ}
{प्रहृष्टमनसा तत्र पूजयामास राघवम्}% १३०

\twolineshloka
{समाश्लिष्य महात्मानं काकपक्षधरं हरिम्}
{नीलोत्पलदलश्यामं पद्मपत्रायतेक्षणम्}% १३१

\twolineshloka
{उपाघ्राय तदा मूर्ध्नि तुष्टाव मुनिसत्तमः}
{एतस्मिन्नन्तरे राजा मिथिलाया अधीश्वरः}% १३२

\twolineshloka
{वाजपेयं क्रतुं यष्टुमारेभे मुनिसत्तमैः}
{तं द्रष्टुं प्रययुस्सर्वे विश्वामित्रपुरोगमाः}% १३३

\twolineshloka
{मुनयो रघुशार्दूल सहिताः पुण्यचेतसः}
{गच्छतस्तस्य रामस्य पदाब्जेन महात्मनः}% १३४

\twolineshloka
{अभूत्सा पावनी भूमिः समाक्रान्ता महाशिला}
{सापि शप्ता पुरा भर्त्रा गौतमेन द्विजन्मना}% १३५

\twolineshloka
{अहल्या रघुनाथस्य पादस्पर्शाच्छुभाऽभवत्}
{अथ सम्प्राप्य नगरीं मिथिलां मुनिसत्तमाः}% १३६

\twolineshloka
{राघवाभ्यां तु सहिता बभूवुः प्रीतमानसाः}
{समागतान्महाभागान्दृष्ट्वा राजा महाबलः}% १३७

\twolineshloka
{प्रत्युद्गम्य प्रणम्याथ पूजयामास मैथिलः}
{रामं पद्मविशालाक्षमिन्दीवरदलप्रभम्}% १३८

\twolineshloka
{पीताम्बरधरं श्लक्ष्णं कोमलावयवोज्ज्वलम्}
{अवधीरित कन्दर्प्पकोटिलावण्यमुत्तमम्}% १३९

\twolineshloka
{सर्वलक्षणसम्पन्नं सर्वाभरणभूषितम्}
{स्वस्य हृत्पद्ममध्ये यः परेशस्य तनुर्हरिः}% १४०

\twolineshloka
{उत्पन्नो दीपवद्दीपात्सौशील्यादिगुणैः परैः}
{तं दृष्ट्वा रघुनाथं स जनको हृष्टमानसः}% १४१

\twolineshloka
{परेशमेव तं मेने रामं दशरथात्मजम्}
{पूजयामास काकुत्स्थं धन्योस्मीति ब्रुवन्नृपः}% १४२

\twolineshloka
{प्रसादं वासुदेवस्य विष्णोर्मेने नरेश्वरः}
{प्रदातुं दुहितां तस्मै मनसा चिन्तयन्प्रभुः}% १४३

\twolineshloka
{आत्मजौ रघुवंशस्य ज्ञात्वा तत्र नृपोत्तमः}
{पूजयामास धर्मेण वस्त्रैराभरणैः शुभैः}% १४४

\twolineshloka
{ऋषीन्समर्चयामास मधुपर्कादिपूजनैः}
{ततोऽवसाने यज्ञस्य रामो राजीवलोचनः}% १४५

\twolineshloka
{भङ्क्त्वा शैवं धनुर्दिव्यं जितवान्जनकात्मजाम्}
{अथासौ वीर्यशुल्केन महता परितोषितः}% १४६

\twolineshloka
{मुदा धरणिजां तस्मै प्रददौ मिथिलाधिपः}
{केशवाय श्रियमिव यथापूर्वं महार्णवः}% १४७

\twolineshloka
{स दूतं प्रेषयामास राघवं मिथिलाधिपः}
{पुत्राभ्यां सह धर्मात्मा मिथिलायां विवेश ह}% १४८

\twolineshloka
{वसिष्ठवामदेवाद्यैः प्रीतैः सह महीपतिः}
{उवास नगरे रम्ये जनकस्य रघूत्तमः}% १४९

\twolineshloka
{तस्मिन्नेव शुभे काले रामस्य धरणीसुताम्}
{विवाहमकरोद्राजा मिथिलेन समर्चितः}% १५०

\twolineshloka
{लक्ष्मणस्योर्मिलां नाम कन्यां जनकसम्भवाम्}
{जनकस्यानुजस्याथ तनये शुभवर्चसी}% १५१

\twolineshloka
{माण्डवी श्रुतकीर्त्तिश्च सर्वलक्षणलक्षिते}
{भरतस्य च सौमित्रेर्विवाहमकरोन्नृपः}% १५२

\twolineshloka
{निर्वर्त्यौद्वाहिकं तत्र राजा दशरथो बली}
{अयोध्यां प्रस्थितः श्रीमान्पौरैर्जनपदैर्वृतः}% १५३

\twolineshloka
{पारिबर्हं समादाय मैथिलेन च पूजितः}
{ससुतः सस्नुषः साश्वः सगजः सबलानुगः}% १५४

\twolineshloka
{तदध्वनि महावीर्य्यो जामदग्निः प्रतापवान्}
{गृहीत्वा परशुं चापं सङ्क्रुद्ध इव केसरी}% १५५

\twolineshloka
{अभ्यधावच्च काकुत्स्थं योद्धुकामो नृपान्तकः}
{सम्प्राप्य राघवं दृष्ट्वा वचनं प्राह भार्गवः}% १५६

\uvacha{परशुराम उवाच}

\twolineshloka
{रामराम महाबाहो शृणुष्व वचनं मम}
{बहुशः पार्थिवान्हत्वा संयुगे भूरिविक्रमान्}% १५७

\twolineshloka
{ब्राह्मणेभ्यो महीं दत्वा तपस्तप्तुमहं गतः}
{तव वीर्यबलं श्रुत्वा त्वया योद्धुमिहागतः}% १५८

\twolineshloka
{इक्ष्वाकवो न वध्या मे मातामहकुलोद्भवाः}
{वीर्य्यं क्षत्रबलं श्रुत्वा न शक्यं सहितुं मम}% १५९

\twolineshloka
{रौद्रं चापं दुराधर्षं भज्यमानां त्वया नृप}
{तस्माद्वदान्य युद्धं मे दीयतां रघुसत्तम}% १६०

\twolineshloka
{इदं तु वैष्णवं चापं तेन तुल्यमरिन्दम}
{आरोपय स्ववीर्येण निर्जितोस्मि त्वयैव हि}% १६१

\twolineshloka
{अथवा त्यज शस्त्राणि पुरस्ताद्बलिनो मम}
{शरणं भज काकुत्स्थ कातरोस्यथ चेतसी}% १६२

\uvacha{ईश्वर उवाच}

\twolineshloka
{एवमुक्तस्तु काकुत्स्थो भार्गवेण प्रतापवान्}
{तच्चापं तस्य जग्राह तच्छक्तिं वैष्णवीमपि}% १६३

\twolineshloka
{शक्त्या वियुक्तस्स तदा जामदग्निः प्रतापवान्}
{निर्वीर्यो नष्टतेजाभूत्कर्म्महीनो यथा द्विजः}% १६४

\twolineshloka
{विनष्टतेज सन्दृष्ट्वा भार्गवं नृपसत्तमाः}
{साधुसध्विति काकुत्स्थं प्रशशंसुर्मुहुर्मुहुः}% १६५

\twolineshloka
{काकुत्स्थस्तन्महच्चापं गृहीत्वारोप्य लीलया}
{सन्धाय बाणं तच्चापे भार्गवं प्राह विस्मितम्}% १६६

\uvacha{राम उवाच}

\twolineshloka
{अनेन शरमुख्येन किं कर्त्तव्यं तव द्विज}
{छेद्मि लोकमिमं चाधः स्वर्गं वा हन्मि ते तपः}% १६७

\uvacha{ईश्वर उवाच}

\twolineshloka
{तन्दृष्ट्वा घोरसङ्काशं बाणं रामस्य भार्गवः}
{ज्ञात्वा तं परमात्मानं प्रहृष्टो राममब्रवीत्}% १६८

\uvacha{परशुराम उवाच}

\twolineshloka
{रामराम महाबाहो न वेद्मि त्वां सनातनम्}
{जानाम्यद्यैव काकुत्स्थ तव वीर्य्यगुणादिभिः}% १६९

\twolineshloka
{त्वमादिपुरुषः साक्षात्परब्रह्मपरोऽव्ययः}
{त्वमनन्तो महाविष्णुर्वासुदेवः परात्परः}% १७०

\twolineshloka
{नारायणस्त्वं श्रीशस्त्वमीश्वरस्त्वं त्रयीमयः}
{त्वं कालस्त्वं जगत्सर्वमकाराख्यस्त्वमेव च}% १७१

\twolineshloka
{स्रष्टा धाता च संहर्त्ता त्वमेव परमेश्वरः}
{त्वमचिन्त्यो महद्भूतरूपस्त्वं तु मनुर्महान्}% १७२

\twolineshloka
{चतुःषट्पञ्चगुणवांस्त्वमेव पुरुषोत्तमः}
{त्वं यज्ञस्त्वं वषट्कारस्त्वमोङ्कारस्त्रयीमयः}% १७३

\twolineshloka
{व्यक्ताव्यक्तस्वरूपस्त्वं गुणभृन्निर्ग्गुणः परः}
{स्तोतुं त्वाहमशक्तश्च वेदानामप्यगोचरम्}% १७४

\twolineshloka
{यच्चापलत्वं कृतवांस्त्वां युयुत्सुतया प्रभो}
{तत्क्षन्तव्यं त्वया नाथ कृपया केवलेन तु}% १७५

\twolineshloka
{तव शक्त्या नृपान्सर्वाञ्जित्वा दत्वा महीं द्विजान्}
{त्वत्प्रसादवशादेव शान्तिमाप्नोति नैष्ठिकीम्}% १७६

\uvacha{ईश्वर उवाच}

\twolineshloka
{एवमुक्त्वा तु काकुत्स्थं जामदग्निर्महातपाः}
{परिणीय नमस्कृत्वा राघवं लोकरक्षकम्}% १७७

\twolineshloka
{शतक्रतुकृतं स्वर्गं तदस्त्राय न्यवेदयत्}
{राघवोऽथ महातेजा ववन्दे तं महामुनिम्}% १७८

\twolineshloka
{विधिवत्पूजयामास पाद्यार्घाचमनादिभिः}
{तेन सम्पूजितस्तत्र जामदग्निर्महातपाः}% १७९

\twolineshloka
{तपस्तप्तुं ययौ रम्यं नरनारायणाश्रमम्}
{राजा दशरथः सोऽथ पुत्रैर्दारसमन्वितैः}% १८०

\twolineshloka
{स्वां पुरीं सुमुहूर्त्तेन प्रविवेश महाबलः}
{राघवो लक्ष्मणश्चैव शत्रुघ्नो भरतस्तथा}% १८१

\twolineshloka
{स्वान्स्वान्दारानुपागम्य रेमिरे हृष्टमानसाः}
{तत्र द्वादश वर्षाणि सीतया सह राघवः}% १८२

\twolineshloka
{रमयामास धर्मात्मा नारायण इव श्रिया}
{तस्मिन्नेव तु राजाथ काले दशरथः सुतम्}% १८३

\twolineshloka
{ज्येष्ठं राज्येन संयोक्तुमैच्छत्प्रीत्या महीपतिः}
{तस्य भार्याथ कैकेयी पुरा दत्तवरा प्रिया}% १८४

\twolineshloka
{अयाचत नृपश्रेष्ठं भरतस्याभिषेचनम्}
{विवासनं च रामस्य वत्सराणि चतुर्दश}% १८५

\twolineshloka
{स राजा सत्यवचनाद्रामं राज्यादथोः सुतम्}
{विवासयामास तदा दुःखेन हतचेतनः}% १८६

\twolineshloka
{शक्तोऽपि राघवस्तस्मिन्राज्यं सन्त्यज्य धर्मतः}
{दशग्रीववधार्थाय पितुर्वचनहेतुना}% १८७

\twolineshloka
{वनं जगाम काकुत्स्थो लक्ष्मणेन च सीतया}
{राजा पुत्रवियोगार्त्तः शोकेन च ममार सः}% १८८

\twolineshloka
{नियुज्यमानो भरतस्तस्मिन्राज्ये समन्त्रिभिः}
{नैच्छद्राज्यं स धर्म्मात्मा सौभ्रात्रमनुदर्शयन्}% १८९

\twolineshloka
{वनमागम्य काकुत्स्थमयाचद्भ्रातरं ततः}
{रामस्तु पितुरादेशान्नैच्छद्राज्यमरिन्दमः}% १९०

\twolineshloka
{स्वपादुके ददौ तस्मै भक्त्या सोऽप्यग्रहीत्तथा}
{रामस्य पादुके राज्यमवाप्य भरतः शुभे}% १९१

\twolineshloka
{प्रत्यहं गन्धपुष्पैश्च पूजयन्कैकयीसुतः}
{तपश्चरणयुक्तेन तस्मिंस्तस्थौ नृपोत्तमः}% १९२

\twolineshloka
{यावदागमनं तस्य राघवस्य महात्मनः}
{तावद्व्रतपराः सर्वे बभूवुः पुरवासिनः}% १९३

\twolineshloka
{राघवश्चित्रकूटाद्रौ भरद्वाजाश्रमे शुभे}
{रमयामास वैदेह्या मन्दाकिन्या जले शुभे}% १९४

\twolineshloka
{कदाचिदङ्के वैदेह्याः शेते रामो महामनाः}
{ऐन्द्रिः काकस्समागम्य तस्मिन्नेव चचार ह}% १९५

\twolineshloka
{स दृष्ट्वा जानकीं तत्र कन्दर्प्पशरपीडितः}
{विददार नखैस्तीक्ष्णैः पीनोन्नतपयोधरम्}% १९६

\twolineshloka
{तं दृष्ट्वा वायसं रामः कुशं जग्राह पाणिना}
{ब्रह्मणास्त्रेण संयोज्य चिक्षेप धरणीधरः}% १९७

\twolineshloka
{तं तृणं घोरसङ्काशं ज्वालारचितविग्रहम्}
{दृष्ट्वा काकः प्रदुद्राव विमुञ्चन्कातरं स्वरम्}% १९८

\twolineshloka
{तं काकं प्रत्यनुययौ रामस्यास्त्रं सुदारुणम्}
{वायसस्त्रिषुलोकेषु बभ्राम भयपीडितः}% १९९

\twolineshloka
{यत्र यत्र ययौ काकः शरणार्थी स वायसः}
{तत्र तत्र तदस्त्रं तु प्रविवेश भयावहम्}% २००

\twolineshloka
{ब्रह्माणमिन्द्रं रुद्रं च यमं वरुणमेव च}
{शरणार्थी जगामाशु वायसः शस्त्रपीडितः}% २०१


\threelineshloka
{तं दृष्ट्वा वायसं सर्वे रुद्राद्या देव दानवाः}
{न शक्ताः स्म वयं त्रातुमिति प्राहुर्मनीषिणः}
{अथ प्रोवाच भगवान्ब्रह्मा त्रिभुवनेश्वरः}% २०२

\uvacha{ब्रह्मोवाच}

\twolineshloka
{भो भो बलिभुजां श्रेष्ठ तमेव शरणं व्रज}
{स एव रक्षकः श्रीमान्सर्वेषां करुणानिधिः}% २०३

\twolineshloka
{रक्षत्येव क्षमासारो वत्सलं शरणागतान्}
{ईश्वरः सर्वभूतानां सौशील्यादिगुणान्वितः}% २०४

\twolineshloka
{रक्षिता जीवलोकस्य पिता माता सखा सुहृत्}
{शरणं व्रज देवेशं नान्यत्र शरणं द्विज}% २०५

\uvacha{महादेव उवाच}

\twolineshloka
{इत्युक्तस्तेन बलिभुग्ब्रह्मणा रघुनन्दनम्}
{उपेत्य सहसा भूमौ निपपात भयातुरः}% २०६

\twolineshloka
{प्राणसंशयमापन्नं दृष्ट्वा सीताथ वायसम्}
{त्राहित्राहीति भर्तारमुवाच विनयाद्विभुम्}% २०७

\twolineshloka
{पुरतः पतितं देवी धरण्यां वायसं तदा}
{तच्छिरः पादयोस्तस्य योजयामास जानकी}% २०८

\twolineshloka
{समुत्थाप्य करेणाथ कृपापीयूषसागरः}
{ररक्ष रामो गुणवान् वायसं दययार्दितः}% २०९

\twolineshloka
{तमाह वायसं रामो मा भैरिति दयानिधिः}
{अभयं ते प्रदास्यामि गच्छ गच्छ यथासुखम्}% २१०

\twolineshloka
{प्रणम्य राघवायाथ सीतायै च मुहुर्मुहुः}
{स्वर्ल्लोकं प्रययावाशु राघवेण च रक्षितः}% २११

\twolineshloka
{ततो रामस्तु वैदेह्या लक्ष्मणेन च धीमता}
{उवास चित्रकूटाद्रौ स्तूयमानो महर्षिभिः}% २१२

\twolineshloka
{तस्मिन्सम्पूज्यमानस्तु भरद्वाजेन राघवः}
{जगामात्रेस्सुविपुलमाश्रमं रघुसत्तमः}% २१३

\twolineshloka
{समागतं रघुवरं दृष्ट्वा मुनिवरोत्तमः}
{भार्यया सह धर्म्मात्मा प्रत्युद्गम्य मुदा युतः}% २१४

\twolineshloka
{आसने सुशुभे मुख्ये निवेश्य सह सीतया}
{अर्घ्यपाद्याचमनीयं च वस्त्राणि विविधानि च}% २१५

\twolineshloka
{मधुपर्कन्ददौ प्रीत्या भूषणं चानुलेपनम्}
{तस्य पत्न्यनसूया तु दिव्याम्बरमनुत्तमम्}% २१६

\twolineshloka
{सीतायै प्रददौ प्रीत्या भूषणानि द्युमन्ति च}
{दिव्यान्नपानभक्षाद्यैर्भोजयामास राघवम्}% २१७

\twolineshloka
{तेन सम्पूजितस्तत्र भक्त्या परमया नृपः}
{उवास दिवसं तत्र प्रीत्या रामस्सलक्ष्मणः}% २१८

\twolineshloka
{प्रभाते विमले रामः समुत्थाय महामुनिम्}
{परिणीय प्रणम्याथ गमनायोपचक्रमे}% २१९

\twolineshloka
{अनुज्ञातस्ततस्तेन रामो राजीवलोचनः}
{प्रययौ दण्डकारण्यं महर्षिकुलसङ्कुलम्}% २२०

\twolineshloka
{तत्रातिभीषणं घोरं विराधं नाम राक्षसम्}
{हत्वाथ शरभङ्गस्य प्रविवेशाश्रमं शुभम्}% २२१

\twolineshloka
{स तु दृष्ट्वाथ काकुत्स्थं सद्यः सङ्क्षीणकल्मषः}
{प्रययौ ब्रह्मलोकं तु गन्धर्वाप्सरसान्वितम्}% २२२

\twolineshloka
{सुतीक्ष्णस्याप्यगस्त्यस्य ह्यगस्त्यभ्रातुरेव च}
{क्रमेण प्रययौ रामस्तैश्च सम्पूजितस्तथा}% २२३

\twolineshloka
{पञ्चवट्यां ततो रामो गोदावर्यास्तटे शुभे}
{उवास सुचिरं कालं सुखेन परमेण च}% २२४

\twolineshloka
{तत्र गत्वा मुनिश्रेष्ठास्तापसा धर्मचारिणः}
{पूजयामासुरात्मेशं रामं राजीवलोचनम्}% २२५

\twolineshloka
{भयं विज्ञापयामासुस्तं च रक्षोगणेरितम्}
{तानाश्वास्य तु काकुस्थो ददौ चाभयदक्षिणाम्}% २२६

\twolineshloka
{ते तु सम्पूजितास्तेन स्वाश्रमान्सम्प्रपेदिरे}
{तस्मिंस्त्रयोदशाब्दानि रामस्य परिनिर्य्ययुः}% २२७

\twolineshloka
{गोदावर्य्यास्तटे रम्ये पञ्चवट्यां मनोरमे}
{कस्यचित्त्वथ कालस्य राक्षसी घोररूपिणी}% २२८

\twolineshloka
{रावणस्य स्वसा तत्र प्रविवेश दुरासदा}
{सा तु दृष्ट्वा रघुवरं कोटिकन्दर्प्पसन्निभम्}% २२९

\twolineshloka
{इन्दीवरदलश्यामं पद्मपत्रायतेक्षणम्}
{प्रोन्नतांसं महाबाहुं कम्बुग्रीवं महाहनुम्}% २३०

\twolineshloka
{सम्पूर्णचन्द्रसदृशं सस्मिताननपङ्कजम्}
{भृङ्गावलिनिभैः स्निग्धैः कुटिलैः शीर्षजैर्वृतम्}% २३१

\twolineshloka
{रक्तारविन्दसदृशं पद्महस्ततलाङ्कितम्}
{निष्कलङ्केन्दुसदृशं नखपङ्क्तिविराजितम्}% २३२

\twolineshloka
{स्निग्धकोमलदूर्वाभं सौकुमार्य्यनिधिं शुभम्}
{पीतकौशेयवसनं सर्वाभरणभूषितम्}% २३३

\twolineshloka
{युवाकुमारवयसं जगन्मोहनविग्रहम्}
{दृष्ट्वा तं राक्षसी रामं कन्दर्प्पशरपीडिता}% २३४

\onelineshloka*
{अब्रवीत्समुपेत्याथ रामं कमललोचनम्}

\uvacha{राक्षस्युवाच}
\onelineshloka
{कस्त्वं तापसवेषेण वर्त्तसे दण्डके वने}% २३५

\twolineshloka
{आगतोऽसि किमर्थं च राक्षसानां दुरासदे}
{शीघ्रमाचक्ष्व तत्त्वेन नानृतं वक्तुमर्हसि}% २३६

\uvacha{महेश्वर उवाच}

\onelineshloka*
{इत्युक्तः स तदा रामः सम्प्रहस्याब्रवीद्वचः}

\uvacha{राम उवाच}

\twolineshloka
{राज्ञो दशरथस्याहं पुत्रो राम इतीरितः}
{असौ ममानुजो धन्वी लक्ष्मणो नाम चानघः}% २३७

\twolineshloka
{पत्नी चेयं च मे सीता जनकस्यात्मजा प्रिया}
{पितुर्वचननिर्देशादहं वनमिहागतः}% २३८

\twolineshloka
{विचरामो महारण्यमृषीणां हितकाम्यया}
{आगतासि किमर्थं त्वमाश्रमं मम सुन्दरि}% २३९

\onelineshloka*
{का त्वं कस्य कुले जाता सर्वं सत्यं वदस्व मे}

\uvacha{महेश्वर उवाच}
\onelineshloka
{इत्युक्ता सा तु रामेण प्राह वाक्यमशङ्किता}% २४०

\uvacha{राक्षस्युवाच}

\twolineshloka
{अहं विश्रवसः पुत्री रावणस्य स्वसा नृप}
{नाम्ना शूर्पणखा नाम त्रिषु लोकेषु विश्रुता}% २४१

\twolineshloka
{इदं च दण्डकारण्यं भ्रात्रा दत्तं मम प्रभो}
{भक्षयन्नृषिसङ्घान्वै विचरामि महावने}% २४२

\twolineshloka
{त्वां तु दृष्ट्वा मुनिवरं कन्दर्पशरपीडिता}
{रन्तुकामा त्वया सार्द्धमागतास्मि सुनिर्भया}% २४३

\twolineshloka
{मम त्वं नृपशार्दूल भर्ता भवितुमर्हसि}
{इमां तव सतीं सीतां ग्रसितुं भूप कामये}% २४४

\onelineshloka*
{वनेषु गिरिमुख्येषु रमयामि त्वया सह}

\uvacha{महेश्वर उवाच}

\onelineshloka
{इत्युक्त्वा राक्षसी सीतां ग्रसितुं वीक्ष्य चोद्यताम्}% २४५

\onelineshloka
{श्रीरामः खड्गमुद्यम्य नासाकर्णौ प्रचिच्छिदे}% २४६

\twolineshloka
{रुदन्ती सभयं शीघ्रं राक्षसी विकृतानना}
{खरालयं प्रविश्याह तस्य रामस्य चेष्टितम्}% २४७

\twolineshloka
{स तु राक्षससाहस्रैर्दूषणत्रिशिरो वृतः}
{आजगाम भृशं योद्धुं राघवं शत्रुसूदनः}% २४८

\twolineshloka
{तान्रामः कानने घोरे बाणः कालान्तकोपमैः}
{निजघान महाकायान्राक्षसांस्तत्र लीलया}% २४९

\twolineshloka
{खरं त्रिशिरसं चैव दूषणं तु महाबलम्}
{रणे निपातयामास बाणैराशीविषोपमैः}% २५०

\twolineshloka
{निहत्य राक्षसान्सर्वान्दण्डकारण्यवासिनः}
{पूजितः सुरसङ्घैश्च स्तूयमानो महर्षिभिः}% २५१

\twolineshloka
{उवास दण्डकारण्ये सीतया लक्ष्मणेन च}
{राक्षसानां वधं श्रुत्वा रावणः क्रोधमूर्च्छितः}% २५२

\twolineshloka
{आजगाम जनस्थानं मारीचेन दुरात्मना}
{सम्प्राप्य पञ्चवट्यां तु दशग्रीवः स राक्षसः}% २५३

\twolineshloka
{मायाविना मरीचेन मृगरूपेण रक्षसः}
{अपहृत्याश्रमाद्दूरे तौ तु दशरथात्मजौ}% २५४

\twolineshloka
{जहार सीतां रामस्य भार्यां स्ववधकाङ्क्षया}
{ह्रियमाणां तु तां दृष्ट्वा जटायुर्गृध्रराड्बली}% २५५

\twolineshloka
{रामस्य सौहृदात्तत्र युयुधे तेन रक्षसा}
{तं हत्वा बाहुवीर्येण रावणं शत्रुवारणः}% २५६

\twolineshloka
{प्रविवेश पुरीं लङ्कां राक्षसैर्बहुभिर्वृताम्}
{अशोकवनिकामध्ये निःक्षिप्य जनकात्मजाम्}% २५७

\twolineshloka
{निधनं रामबाणेन काङ्क्षयन्स्वगृहं विशत्}
{रामस्तु राक्षसं हत्वा मारीचं मृगरूपिणम्}% २५८

\twolineshloka
{पुनराविश्य तत्राथ भ्रात्रा सौमित्रिणा ततः}
{राक्षसापहृतां भार्यां ज्ञात्वा दशरथात्मजः}% २५९

\twolineshloka
{प्रभूतशोकसन्तप्तो विललाप महामतिः}
{मार्गमाणो वने सीतां पथि गृध्रं महाबलम्}% २६०

\twolineshloka
{विच्छिन्नपादपक्षं च पतितं धरणीतले}
{रुधिरापूर्णसर्वाङ्गं दृष्ट्वा विस्मयमागतः}% २६१

\twolineshloka
{पप्रच्छ राघवं श्रीमान्केन किं त्वं जिघांसितः}
{गृध्रस्तु राघवं दृष्ट्वा मन्दमन्दमुवाच ह}% २६२

\uvacha{गृध्र उवाच}

\twolineshloka
{रावणेन हृता राम तव भार्यां बलीयसा}
{तेन राक्षसमुख्येन सङ्ग्रामे निहतोस्म्यहम्}% २६३

\uvacha{महेश्वर उवाच}

\twolineshloka
{इत्युक्त्वा राघवस्याग्रे सहसा त्यक्तजीवितः}
{संस्कारमकरोद्रामस्तस्य ब्रह्मविधानतः}% २६४

\twolineshloka
{स्वपदं च ददौ तस्मै योगिगम्यं सनातनम्}
{राघवस्य प्रसादेन स गृध्रः परमं पदम्}% २६५

\twolineshloka
{हरेः सामान्यरूपेण मुक्तिं प्राप खगोत्तमः}
{माल्यवन्तं ततो गत्वा मतङ्गस्याश्रमे शुभे}% २६६

\twolineshloka
{अभिगम्य महाभागां शबरीं धर्मचारिणीम्}
{सा तु भागवतश्रेष्ठा दृष्ट्वा तौ रामलक्ष्मणौ}% २६७

\twolineshloka
{प्रत्युद्गम्य नमस्कृत्वा निवेश्य कुशविष्टरे}
{पादप्रक्षालनं कृत्वा वन्यैः पुष्पैः सुगन्धिभिः}% २६८

\twolineshloka
{अर्चयामास भक्त्या वै हर्षनिर्भरमानसा}
{फलानि च सुगन्धीनि मूलानि मधुराणि च}% २६९

\twolineshloka
{निवेदयामास तदा राघवाभ्यां दृढव्रता}
{फलान्यास्वाद्य काकुत्स्थस्तस्यै मुक्तिं ददौ पराम्}% २७०

\twolineshloka
{ततः पम्पासरो गत्वा राघवः शत्रुसूदनः}
{जघान राक्षसं तत्र कबन्धं घोररूपिणम्}% २७१

\twolineshloka
{तं निहत्य महावीर्यो ददाह स्वर्गतश्च सः}
{ततो गोदावरीं गत्वा रामो राजीवलोचनः}% २७२

\twolineshloka
{पप्रच्छ सीतां गङ्गे त्वं किं तां जानासि मे प्रियाम्}
{न शशंस तदा तस्मै सा गङ्गा तमसावृता}% २७३

\twolineshloka
{शशाप राघवः क्रोधाद्रक्ततोया भवेति ताम्}
{ततो भयात्समुद्विग्ना पुरस्कृत्य महामुनीन्}% २७४

\twolineshloka
{कृताञ्जलिपुटा दीना राघवं शरणं गता}
{ततो महर्षयस्सर्वे रामं प्राहुस्सनातनम्}% २७५

\uvacha{ऋषय ऊचुः}

\twolineshloka
{त्वत्पादकमलोद्भूता गङ्गा त्रैलोक्यपावनी}
{त्वमेव हि जगन्नाथ तां शापान्मोक्तुमर्हसि}% २७६

\uvacha{महेश्वर उवाच}

\onelineshloka*
{ततः प्रोवाच धर्मात्मा रामः शरणवत्सलः}

\uvacha{राम उवाच}

\twolineshloka
{शबर्याः स्नानमात्रेण सङ्गता शुभवारिणा}
{मुक्ता भवतु मच्छापाद्गङ्गेयं पापनाशिनी}% २७७

\twolineshloka
{एवमुक्त्वा तु काकुत्स्थः शबरीतीर्थमुत्तमम्}
{गङ्गा गयासमं चक्रे शार्ङ्गकोट्या महाबलः}% २७८

\twolineshloka
{महाभागवतानां च तीर्थं यस्योदकेऽभवत्}
{तच्छरीरं जगद्वन्द्यं भविष्यति न संशयः}% २७९

\twolineshloka
{एवमुक्त्वा तु काकुत्स्थ ऋष्यमूकं गिरिं ययौ}
{ततः पम्पासरस्तीरे वानरेण हनूमता}% २८०

\twolineshloka
{सङ्गतस्तस्य वचनात्सुग्रीवेण समागतः}
{सुग्रीववचनाद्धत्वा वालिनं वानरेश्वरम्}% २८१

\twolineshloka
{सुग्रीवमेव तद्राज्ये रामोसावभ्यषेचयत्}
{स तु सम्प्रेषयामास दिदृक्षुर्जनकात्मजाम्}% २८२

\twolineshloka
{हनुमत्प्रमुखान्वीरान्वानरान्वानराधिपः}
{स लङ्घयित्वा जलधिं हनूमान्मारुतात्मजः}% २८३

\twolineshloka
{प्रविश्य नगरीं लङ्कां दृष्ट्वा सीतां दृढव्रताम्}
{उपवासकृशां दीनां भृशं शोकपरायणाम्}% २८४

\twolineshloka
{मलपङ्केन दिग्धाङ्गीं मलिनाम्बरधारिणीम्}
{निवेदयित्वाऽभिज्ञानं प्रवृत्तिं च निवेद्य ताम्}% २८५

\twolineshloka
{सप्तमन्त्रिसुतांस्तत्र रावणस्य सुतं तथा}
{तोरणस्तम्भमुत्पाट्य निजघान स्वयं कपिः}% २८६

\twolineshloka
{समाश्वास्य च वैदेहीं बभञ्जोपवनं तदा}
{वनपालान्किङ्करांश्च पञ्चसेनाग्रनायकान्}% २८७

\twolineshloka
{रावणस्य सुतेनाथ निगृहीतो यदृच्छया}
{दृष्ट्वा च राक्षसेन्द्रं तु सम्भाषित्वा तथैव च}% २८८

\twolineshloka
{ददाह नगरीं लङ्कां स्वलाङ्गूलाग्निना कपिः}
{तया दत्तमभिज्ञानं गृहीत्वा पुनरागमत्}% २८९

\twolineshloka
{सोऽभिगम्य महातेजा रामं कमललोचनम्}
{न्यवेदयद्वानरेन्द्रो दृष्टा सीतेति तत्वतः}% २९०

\twolineshloka
{सुग्रीवसहितो रामो वानरैर्बहुभिर्वृतः}
{महोदधेस्तटं गत्वा तत्रानीकं न्यवेशयत्}% २९१

\twolineshloka
{रावणस्यानुजो भ्राता विभीषण इतीरितः}
{धर्मात्मा सत्यसन्धश्च महाभागवतोत्तमः}% २९२

\twolineshloka
{ज्ञात्वा समागतं रामं परित्यज्य स्वपूर्वजम्}
{राज्यं सुतांश्च दारांश्च राघवं शरणं ययौ}% २९३

\twolineshloka
{परिगृह्य च तं रामो मारुतेर्वचनात्प्रभुः}
{तस्मै दत्वाऽभयं सौम्यं रक्षो राज्येऽभ्यषेचयत्}% २९४

\twolineshloka
{ततस्समुद्रं काकुत्स्थस्तर्तुकामः प्रपद्य वै}
{सुप्रसन्नजलं तं तु दृष्ट्वा रामो महाबलः}% २९५

\twolineshloka
{शार्ङ्गमादाय बाणौघैः शोषयामास वारिधिम्}
{ततस्तु सरितामीशः काकुत्स्थं करुणानिधिम्}% २९६

\twolineshloka
{प्रपद्य शरणं देवमर्चयामास वारिधिः}
{पुनरापूर्य जलधिं वरुणास्त्रेण राघवः}% २९७

\twolineshloka
{उदधेर्वचनात्सेतुं सागरे मकरालये}
{गिरिभिर्वानरानीतैर्नलः सेतुमकारयत्}% २९८

\twolineshloka
{ततो गत्वा पुरीं लङ्कां सन्निवेश्य महाबलम्}
{सम्यगायोधनं चक्रे वानराणां च रक्षसाम्}% २९९

\twolineshloka
{ततो दशास्यतनयः शक्रजिद्राक्षसो बली}
{बबन्ध नागपाशैश्च तावुभौ रामलक्ष्मणौ}% ३००

\twolineshloka
{वैनतेयः समागत्य तान्यस्त्राणि प्रमोचयत्}
{राक्षसा निहतास्सर्वे वानरैश्च महाबलैः}% ३०१

\twolineshloka
{रावणस्यानुजं वीरं कुम्भकर्णं महाबलम्}
{निजघान रणे रामो बाणैरग्निशिखोपमैः}% ३०२

\twolineshloka
{ब्रह्मास्त्रेणेन्द्रजित्क्रुद्धः पातयामास वानरान्}
{हनूमता समानीतो महौषधि महीधरः}% ३०३

\twolineshloka
{तस्यानीतस्य च स्पर्शात्सर्व एव समुत्थिताः}
{ततो रामानुजो वीरः शक्रजेतारमाहवे}% ३०४

\twolineshloka
{निपातयामास शरैर्वृत्रं वज्रधरो यथा}
{निर्ययावथ पौलस्त्यो योद्धुं रामेण संयुगे}% ३०५

\twolineshloka
{चतुरङ्गबलैः सार्द्धं मन्त्रिभिश्च महाबलः}
{समन्ततोभवद्युद्धं वानराणां च रक्षसाम्}% ३०६

\twolineshloka
{रामरावणयोश्चैव तथा सौमित्रिणा सह}
{शक्त्या निपातयामास लक्ष्मणं राक्षसेश्वरः}% ३०७

\twolineshloka
{ततः क्रुद्धो महातेजा राघवो राक्षसान्तकः}
{जघान राक्षसान्वीराञ्शरैः कालान्तकोपमैः}% ३०८

\twolineshloka
{प्रदीप्तैर्बाणसाहस्रैः कालदण्डोपमैर्भृशम्}
{छादयामास काकुत्स्थो दशग्रीवं च राक्षसम्}% ३०९

\twolineshloka
{स तु निर्भिन्नसर्वाङ्गो राघवास्त्रैर्निशाचरः}
{भयात्प्रदुद्राव रणाल्लङ्कां प्रति निशाचरः}% ३१०

\twolineshloka
{जगद्राममयं पश्यन्निर्वेदाद्गृहमाविशत्}
{ततो हनूमता नीतो महौषधिमहागिरिः}% ३११

\twolineshloka
{तेन रामानुजस्तूर्णं लब्धसंज्ञोऽभवत्तदा}
{दशग्रीवस्ततो होममारेभे जयकाङ्क्षया}% ३१२

\twolineshloka
{ध्वंसितं वानरेन्द्रैस्तदभिचारात्मकं रिपोः}
{पुनर्युद्धाय पौलस्त्यो रामेण सह निर्ययौ}% ३१३

\twolineshloka
{दिव्यस्यन्दनमारुह्य राक्षसैर्बहुभिर्युतः}
{ततः शतमखो दिव्यं रथं हर्यश्वसंयुतम्}% ३१४

\twolineshloka
{राघवाय ससूतं हि प्रेषयामास बुद्धिमान्}
{रथं मातलिना नीतं समारुह्य रघूत्तमः}% ३१५

\twolineshloka
{स्तूयमानं सुरगणैर्युयुधे तेन रक्षसा}
{ततो युद्धमभूद्धोरं रामरावणयोर्महत्}% ३१६

\twolineshloka
{सप्ताह्निकमहोरात्रं शस्त्रास्त्रैरतिभीषणम्}
{विमानस्थाः सुरास्सर्वे ददृशुस्तत्र संयुगम्}% ३१७

\twolineshloka
{दशग्रीवस्य चिच्छेद शिरांसि रघुसत्तमः}
{समुत्थितानि बहुशो वरदानात्कपर्दिनः}% ३१८

\twolineshloka
{ब्राह्ममस्त्रं महारौद्रं वधायास्य दुरात्मनः}
{ससर्ज राघवस्तूर्णं कालाग्निसदृशप्रभम्}% ३१९

\twolineshloka
{तदस्त्रं राघवोत्सृष्टं रावणस्य स्तनान्तरम्}
{विदार्य धरणीं भित्त्वा रसातलतले गतम्}% ३२०

\twolineshloka
{सम्पूज्यमानं भुजगै राघवस्य करं ययौ}
{स गतासुर्महादैत्यः पपात च ममार च}% ३२१

\twolineshloka
{ततो देवगणास्सर्वे हर्षनिर्भरमानसाः}
{ववृषुः पुष्पवर्षाणि महात्मनि जगद्गुरौ}% ३२२

\twolineshloka
{जगुर्गन्धर्वपतयो ननृतुश्चाप्सरोगणाः}
{ववुः पुण्यास्तथा वाताः सुप्रभोऽभूद्दिवाकरः}% ३२३

\twolineshloka
{तुष्टुवुर्मुनयः सिद्धा देवगन्धर्वकिन्नराः}
{लङ्कायां राक्षसश्रेष्ठमभिषिच्य विभीषणम्}% ३२४

\twolineshloka
{कृतकृत्यमिवात्मानं मेने रघुकुलोत्तमः}
{रामस्तत्राब्रवीद्वाक्यमभिषिच्य विभीषणम्}% ३२५

\uvacha{राम उवाच}

\twolineshloka
{यावच्चन्द्रश्च सूर्यश्च यावत्तिष्ठति मेदिनी}
{यावन्ममकथालोके तावद्राज्यं विभीषणे}% ३२६

\twolineshloka
{गत्वा मम पदं दिव्यं योगिगम्यं सनातनम्}
{सपुत्रपौत्रः सगणः सम्प्राप्नुहि महाबलः}% ३२७

\uvacha{ईश्वर उवाच}

\twolineshloka
{एवं दत्वा वरं तस्मै राक्षसाय महाबलः}
{सम्प्राप्य मैथिलीं तत्र परुषं जनसंसदि}% ३२८

\twolineshloka
{उवाच राघवः सीतां गर्हितं वचनं बहु}
{सा तेन गर्हिता साध्वी विवेश चानलं महत्}% ३२९


\threelineshloka
{ततो देवगणास्सर्वे शिवब्रह्मपुरोगमाः}
{दृष्ट्वा तु मातरं वह्नौ प्रविशन्तीं भयातुराः}
{समागम्य रघुश्रेष्ठं सर्वे प्राञ्जलयोऽब्रुवन्}% ३३०

\uvacha{देवा ऊचुः}

\twolineshloka
{रामराम महाबाहो शृणु त्वं चातिविक्रम}
{सीतातिविमला साध्वी तव नीत्यानपायिनी}% ३३१

\twolineshloka
{अत्याज्या तु वृथा सा हि भास्करेण प्रभा यथा}
{सेयं लोकहितार्थाय समुत्पन्ना महीतले}% ३३२

\twolineshloka
{माता सर्वस्य जगतः समस्तजगदाश्रया}
{रावणः कुम्भकर्णश्च भृत्यौ पूर्वपरायणौ}% ३३३

\twolineshloka
{शापात्तौ सनकादीनां समुत्पन्नौ महीतले}
{तयोर्विमुक्त्यै वैदेही गृहीता दण्डके वने}% ३३४

\twolineshloka
{तावुभौ वै वधं प्राप्तौ त्वया राक्षसपुङ्गवौ}
{तौ विमुक्तौ दिवं यातौ पुत्रपौत्रसहानुगौ}% ३३५

\twolineshloka
{त्वं विष्णुस्त्वं परं ब्रह्म योगिध्येयः सनातनः}
{त्वमेव सर्वदेवानामनादिनिधनोऽव्ययः}% ३३६

\twolineshloka
{त्वं हि नारायणः श्रीमान्सीता लक्ष्मीः सनातनी}
{माता सा सर्वलोकानां पिता त्वं परमेश्वर}% ३३७

\twolineshloka
{नित्यैवैष जगन्माता तव नित्यानपायिनी}
{यथा सर्वगतस्त्वं हि तथा चेयं रघूत्तम}% ३३८

\twolineshloka
{तस्माच्छुद्धसमाचारां सीतां साध्वीं दृढव्रताम्}
{गृहाण सौम्य काकुत्स्थ क्षीराब्धेरिव मा चिरम्}% ३३९

\uvacha{ईश्वर उवाच}


\threelineshloka
{एतस्मिन्नन्तरे तत्र लोकसाक्षी स पावकः}
{आदाय सीतां रामाय प्रददौ सुरसन्निधौ}
{अब्रवीत्तत्र काकुत्स्थं वह्निः सर्वशरीरगः}% ३४०

\uvacha{वह्निरुवाच}

\twolineshloka
{इयं शुद्धसमाचारा सीता निष्कल्मषा विभो}
{गृहाण मा चिरं राम सत्यं सत्यं तवाब्रुवन्}% ३४१

\uvacha{ईश्वर उवाच}

\twolineshloka
{ततोऽग्निवचनात्सीतां परिगृह्य रघूद्वहः}
{बभूव रामः संहृष्टः पूज्यमानः सुरोत्तमैः}% ३४२

\twolineshloka
{राक्षसैर्निहता ये तु सङ्ग्रामे वानरोत्तमाः}
{पितामहवरात्तूर्णं जीवमानाः समुत्थिताः}% ३४३

\twolineshloka
{ततस्तु पुष्पकं नाम विमानं सूर्यसन्निभम्}
{भ्रात्रा गृहीतं सङ्ग्रामे कौबेरं राक्षसेश्वरः}% ३४४

\twolineshloka
{तद्राघवाय प्रददौ वस्त्राण्याभरणानि च}
{तेन सम्पूजितः श्रीमान्रामचन्द्रः प्रतापवान्}% ३४५

\twolineshloka
{आरुरोह विमानाग्र्यं वैदेह्या भार्यया सह}
{लक्ष्मणेन च शूरेण भ्रात्रा दशरथात्मजः}% ३४६

\twolineshloka
{ऋक्षवानरसङ्घातैः सुग्रीवेण महात्मना}
{विभीषणेन शूरेण राक्षसैश्च महाबलैः}% ३४७

\twolineshloka
{यथाविमाने वैकुण्ठे नित्यमुक्तैर्महात्मभिः}
{तथा सर्वे समारुह्य ऋक्षवानरराक्षसाः}% ३४८

\twolineshloka
{अयोध्यां प्रस्थितो रामः स्तूयमानः सुरोत्तमैः}
{भरद्वाजाश्रमं गत्वा रामः सत्यपराक्रमः}% ३४९

\twolineshloka
{भरतस्यान्तिके तत्र हनूमन्तं व्यसर्जयत्}
{स निषादालयं गत्वा गुहं दृष्ट्वाऽथ वैष्णवम्}% ३५०

\twolineshloka
{राघवागमनं तस्मै प्राह वानरपुङ्गवः}
{नन्दिग्रामं ततो गत्वा दृष्ट्वा तं राघवानुजम्}% ३५१

\twolineshloka
{न्यवेदयत्तथा तस्मै रामस्यागमनोत्सवम्}
{भरतश्चागतं श्रुत्वा वानरेण रघूत्तमम्}% ३५२

\twolineshloka
{प्रर्हर्षमतुलं लेभे सानुजः ससुहृज्जनः}
{पुनरागत्य काकुत्स्थं हनूमान्मारुतात्मजः}% ३५३

\twolineshloka
{सर्वं शशंस रामाय भरतस्य च वर्तितम्}
{राघवस्तु विमानाग्र्यादवरुह्य सहानुजः}% ३५४

\twolineshloka
{ववन्दे भार्यया सार्द्धं भारद्वाजं तपोनिधिम्}
{स तु सम्पूजयामास काकुत्स्थं सानुजं मुनिः}% ३५५

\twolineshloka
{पक्वान्नैः फलमूलाद्यैर्वस्त्रैराभरणैरपि}
{तेन सम्पूजितस्तत्र प्रणम्य मुनिसत्तमम्}% ३५६

\twolineshloka
{अनुज्ञातः समारुह्य पुष्पकं सानुगस्तदा}
{नन्दिग्रामं ययौ रामः पुष्पकेण सुहृद्वृतः}% ३५७

\twolineshloka
{मन्त्रिभिः पौरमुख्यैश्च सानुजः केकयीसुतः}
{प्रत्युद्ययौ नृपवरैः सबलैः पूर्वजं मुदा}% ३५८

\twolineshloka
{सम्प्राप्य रघुशार्दूलं ववन्दे सानुगैर्वृतः}
{पुष्पकादवरुह्याथ राघवः शत्रुतापनः}% ३५९

\twolineshloka
{भरतं चैव शत्रुघ्नमुपसम्परिषस्वजे}
{पुरोहितं वसिष्ठं च मातृवृद्धांश्च बान्धवान्}% ३६०

\twolineshloka
{प्रणनाम महातेजाः सीतया लक्ष्मणेन च}
{विभीषणं च सुग्रीवं जाम्बवन्तं तथाङ्गदम्}% ३६१

\twolineshloka
{हनुमन्तं सुषेणं च भरतः परिषस्वजे}
{भ्रातृभिः सानुगैस्तत्र मङ्गलस्नानपूर्वकम्}% ३६२

\twolineshloka
{दिव्यमाल्याम्बरधरो दिव्यगन्धानुलेपनः}
{आरुरोह रथं दिव्यं सुमन्त्राधिष्ठितं शुभम्}% ३६३

\twolineshloka
{संस्तूयमानस्त्रिदशैर्वैदेह्या लक्ष्मणेन च}
{भरतश्चैव सुग्रीवः शत्रुघ्नश्च विभीषणः}% ३६४

\twolineshloka
{अङ्गदश्च सुषेणश्च जाम्बवान्मारुतात्मजः}
{नीलो नलश्च सुभगः शरभो गन्धमादनः}% ३६५

\twolineshloka
{अन्ये च कपयः शूरा निषादाधिपतिर्गुहः}
{राक्षसाश्च महावीर्याः पार्थिवेन्द्रा महाबलाः}% ३६६

\twolineshloka
{गजानश्वानथो सम्यगारुह्य बहुशः शुभान्}
{नानामङ्गलवादित्रैः स्तुतिभिः पुष्कलैस्तथा}% ३६७

\twolineshloka
{ऋक्षवानररक्षोभिर्निषादवरसैनिकैः}
{प्रविवेश महातेजाः साकेतं पुरमव्ययम्}% २६८

\twolineshloka
{आलोक्य राजनगरीं पथि राजपुत्रो राजानमेव पितरं परिचिन्तयानः}
{सुग्रीवमारुतिविभीषणपुण्यपादसञ्चारपूतभवनं प्रविवेश रामः}% ३६९

{॥इति श्रीपाद्मे महापुराणे पञ्चपञ्चाशत्साहस्र्यां संहितायामुत्तरखण्डे उमामहेश्वरसंवाद रामस्यायोध्याप्रवेशो नाम द्विचत्वारिंशदधिकद्विशततमोऽध्यायः॥२४२॥}

\sect{त्रिचत्वारिंशदधिक-द्विशततमोऽध्यायः --- विश्वदर्शनम्}

\uvacha{शङ्कर उवाच}

\twolineshloka
{अथ तस्मिन्दिने पुण्ये शुभलग्ने शुभान्विते}
{मङ्गलस्याभिषेकार्थं मङ्गलं चक्रिरे जनाः}% १

\twolineshloka
{वसिष्ठो वामदेवश्च जाबालिरथ कश्यपः}
{मार्कण्डेयश्च मौद्गल्यः पर्वतो नारदस्तथा}% २

\twolineshloka
{एते महर्षयस्तत्र जपहोमपुरस्सरम्}
{अभिषेकं शुभं चक्रुर्मुनयो राजसत्तमम्}% ३

\twolineshloka
{नानारत्नमये दिव्ये हेमपीठे शुभान्विते}
{निवेश्य सीतया सार्द्धं श्रिया इव जनार्दनम्}% ४

\twolineshloka
{सौवर्णकलशैर्दिव्यैर्नानारत्नमयैः शुभैः}
{सर्वतीर्थोदकैः पुण्यैर्माङ्गल्यद्रव्यसंयुतैः}% ५

\twolineshloka
{दूर्वाग्रतुलसीपत्रपुष्पगन्धसमन्वितैः}
{मन्त्रपूतजलैः शुद्धैर्मुनयः संशितव्रताः}% ६

\twolineshloka
{अजपन्वैष्णवान्सूक्तान्चतुर्वेदमयान्शुभान्}
{अभिषेकं शुभं चक्रुः काकुत्स्थं जगतः पतिम्}% ७

\twolineshloka
{तस्मिन्शुभतमे लग्ने देवदुन्दुभयो दिवि}
{विनेदुः पुष्पवर्षाणि ववृषुश्च समन्ततः}% ८

\twolineshloka
{दिव्याम्बरैर्भूषणैश्च दिव्यगन्धानुलेपनैः}
{पुष्पैर्नानाविधैर्दिव्यैर्देव्या सह रघूद्वहः}% ९

\twolineshloka
{अलङ्कृतश्च शुशुभे मुनिभिर्वेदपारगैः}
{छत्रं च चामरं दिव्यं धृतवान्लक्ष्मणस्तदा}% १०

\twolineshloka
{पार्श्वे भरतशत्रुघ्नौ तालवृन्तौ विरेजतुः}
{दर्पणं प्रददौ श्रीमान्राक्षसेन्द्रो विभीषणः}% ११

\twolineshloka
{दधार पूर्णकलशं सुग्रीवो वानरेश्वरः}
{जाम्बवांश्च महातेजाः पुष्पमालां मनोहराम्}% १२

\twolineshloka
{वालिपुत्रस्तु ताम्बूलं सकर्पूरं ददौ हरेः}
{हनुमान्दीपकां दिव्यां सुषेणश्च ध्वजं शुभम्}% १३

\twolineshloka
{परिवार्य महात्मानं मन्त्रिणः समुपासिरे}
{सृष्टिर्जयन्तो विजयः सौराष्ट्रो राष्ट्रवर्द्धनः}% १४

\twolineshloka
{अकोपो धर्मपालश्च सुमन्त्रो मन्त्रिणः स्मृताः}
{राजानश्च नरव्याघ्रा नानाजनपदेश्वराः}% १५

\twolineshloka
{पौराश्च नैगमा वृद्धा राजानं पर्युपासत}
{ऋक्षैश्च वानरेन्द्रैश्च मन्त्रिभिः पृथिवीश्वरैः}% १६

\twolineshloka
{राक्षसैर्द्विजमुख्यैश्च किङ्करैश्च समावृतः}
{परे व्योम्नि यथा लीनो दैवतैः कमलापतिः}% १७

\twolineshloka
{तथा नृपवरः श्रीमान्साकेते शुशुभे तदा}
{इन्दीवरदलश्यामं पद्मपत्रनिभेक्षणम्}% १८

\twolineshloka
{आजानुबाहुं काकुत्स्थं पीतवस्त्रधरं हरिम्}
{कम्बुग्रीवं महोरस्कं विचित्राभरणैर्युतम्}% १९

\twolineshloka
{देव्या सह समासीनमभिषिक्तं रघूत्तमम्}
{विमानस्थाः सुरगणा हर्षनिर्भरमानसाः}% २०

\twolineshloka
{तुष्टुवुर्जयशब्देन गन्धर्वाप्सरसां गणाः}
{अभिषिक्तस्ततो रामो वसिष्ठाद्यैर्महर्षिभिः}% २१

\twolineshloka
{शुशुभे सीतया देव्या नारायण इव श्रिया}
{अतिमर्त्यतयाभीत उपासितुं पदाम्बुजम्}% २२

\threelineshloka
{दृष्ट्वा तुष्टाव हृष्टात्मा शङ्करो हृष्टमागतः}
{कृताञ्जलिपुटो भूत्वा सानन्दो गद्गदाकुलः}
{हर्षयन्सकलान्देवान्मुनीनपि च वानरान्}% २३

\uvacha{महादेव उवाच}

\twolineshloka
{नमो मूलप्रकृतये नित्याय परमात्मने}
{सच्चिदानन्दरूपाय विश्वरूपाय वेधसे}% २४

\twolineshloka
{नमो निरन्तरानन्द कन्दमूलाय विष्णवे}
{जगत्त्रयकृतानन्द मूर्त्तये दिव्यमूर्त्तये}% २५

\twolineshloka
{नमो ब्रह्मेन्द्रपूज्याय शङ्कराभयदाय च}
{नमो विष्णुस्वरूपाय सर्वरूपनमोनमः}% २६

\twolineshloka
{उत्पत्तिस्थितिसंहारकारिणे त्रिगुणात्मने}
{नमोस्तु निर्गतोपाधिस्वरूपाय महात्मने}% २७

\twolineshloka
{अनया विद्यया देव्या सीतयोपाधिकारिणे}
{नमः पुम्प्रकृतिभ्यां च युवाभ्यां जगतां कृते}% २८

\twolineshloka
{जगन्मातापितृभ्यां च जनन्यै राघवाय च}
{नमः प्रपञ्चरूपिण्यै निष्प्रपञ्चस्वरूपिणे}% २९

\twolineshloka
{नमो ध्यानस्वरूपिण्यै योगिध्येयात्ममूर्त्तये}
{परिणामापरीणामरिक्ताभ्यां च नमोनमः}% ३०

\twolineshloka
{कूटस्थबीजरूपिण्यै सीतायै राघवाय च}
{सीता लक्ष्मीर्भवान्विष्णुः सीता गौरी भवान्शिवः}% ३१

\twolineshloka
{सीता स्वयं हि सावित्रि भवान्ब्रह्मा चतुर्मुखः}
{सीता शची भवान्शक्रः सीता स्वाहानलो भवान्}% ३२

\twolineshloka
{सीता संहारिणी देवी यमरूपधरो भवान्}
{सीता हि सर्वसम्पत्तिः कुबेरस्त्वं रघूत्तम}% ३३

\twolineshloka
{सीता देवी च रुद्राणी भवान्रुद्रो महाबलः}
{सीता तु रोहिणी देवी चन्द्रस्त्वं लोकसौख्यदः}% ३४

\twolineshloka
{सीता संज्ञा भवान्सूर्यः सीता रात्रिर्दिवा भवान्}
{सीतादेवी महाकाली महाकालो भवान्सदा}% ३५

\twolineshloka
{स्त्रीलिङ्गेषु त्रिलोकेषु यत्तत्सर्वं हि जानकी}
{पुन्नाम लाञ्छितं यत्तु तत्सर्वं हि भवान्प्रभो}% ३६

\twolineshloka
{सर्वत्र सर्वदेवेश सीता सर्वत्र धारिणी}
{तदात्वमपिचत्रातुन्तच्छक्तिर्विश्वधारिणी}% ३७

\twolineshloka
{तस्मात्कोटिगुणं पुण्यं युवाभ्यां परिचिह्नितम्}
{चिह्नितं शिवशक्तिभ्यां चरितं तव शान्तिदम्}% ३८

\twolineshloka
{आवां राम जगत्पूज्यौ मम पूज्यौ सदा युवाम्}
{त्वन्नामजापिनी गौरी त्वन्मन्त्रजपवानहम्}% ३९

\twolineshloka
{मुमूर्षोर्मणिकर्ण्यां तु अर्द्धोदकनिवासिनः}
{अहं दिशामि ते मन्त्रं तारकं ब्रह्मदायकम्}% ४०

\twolineshloka
{अतस्त्वं जानकीनाथ परब्रह्मासि निश्चितम्}
{त्वन्मायामोहितास्सर्वे न त्वां जानन्ति तत्वतः}% ४१

\uvacha{ईश्वर उवाच}

\twolineshloka
{इत्युक्तः शम्भुना रामः प्रसादप्रवणोऽभवत्}
{दिव्यरूपधरः श्रीमानद्भुताद्भुतदर्शनः}% ४२

\twolineshloka
{तथा तं रूपमालोक्य नरवानरदेवताः}
{न द्रष्टुमपिशक्तास्ते तेजसं महदद्भुतम्}% ४३


\threelineshloka
{भयाद्वै त्रिदशश्रेष्ठाः प्रणेमुश्चातिभक्तितः}
{भीता विज्ञाय रामोऽपि नरवानरदेवताः}
{मायामानुषतां प्राप्य स देवानब्रवीत्पुनः}% ४४

\uvacha{रामचन्द्र उवाच}

\twolineshloka
{शृणुध्वं देवता यो मां प्रत्यहं संस्तुविष्यति}
{स्तवेन शम्भुनोक्तेन देवतुल्यो भवेन्नरः}% ४५

\twolineshloka
{विमुक्तः सर्वपापेभ्यो मत्स्वरूपं समश्नुते}
{रणे जयमवाप्नोति न क्वचित्प्रतिहन्यते}% ४६

\twolineshloka
{भूतवेतालकृत्याभिर्ग्रहैश्चापि न बाध्यते}
{अपुत्रो लभते पुत्रं पतिं विन्दति कन्यका}% ४७

\twolineshloka
{दरिद्रः श्रियमाप्नोति सत्ववाञ्शीलवान्भवेत्}
{आत्मतुल्यबलः श्रीमाञ्जायते नात्र संशयः}% ४८

\twolineshloka
{निर्विघ्नं सर्वकार्येषु सर्वारम्भेषु वै नृणाम्}
{यंयं कामयते मर्त्यः सुदुर्लभमनोरथम्}% ४९


\threelineshloka
{षण्मासात्सिद्धिमाप्नोति स्तवस्यास्य प्रसादतः}
{यत्पुण्यं सर्वतीर्थेषु सर्वयज्ञेषु यत्फलम्}
{तत्फलं कोटिगुणितं स्तवेनानेन लभ्यते}% ५०

\uvacha{ईश्वर उवाच}

\twolineshloka
{इत्युक्त्वा रामचन्द्रोऽसौ विससर्ज महेश्वरम्}
{ब्रह्मादि त्रिदशान्सर्वान्विससर्ज समागतान्}% ५१

\twolineshloka
{अर्चिता मानवाः सर्वे नरवानरदेवताः}
{विसृष्टा रामचन्द्रेण प्रीत्या परमया युताः}% ५२

\twolineshloka
{इत्थं विसृष्टाः खलु ते च सर्वे सुखं तदा जग्मुरतीवहृष्टाः}
{परं पठन्तः स्तवमीश्वरोक्तं रामं स्मरन्तो वरविश्वरूपम्}% ५३

{॥इति श्रीपाद्मे महापुराणे पञ्चपञ्चाशत्साहस्र्यां संहितायामुत्तरखण्डे उमामहेश्वर संवादे विश्वदर्शनं नाम त्रिचत्वारिंशदधिकद्विशततमोऽध्यायः॥२४३॥}

\sect{चतुश्चत्वारिंशदधिक-द्विशततमोऽध्यायः --- श्रीरामचरितकथनम्}

\uvacha{शङ्कर उवाच}

\twolineshloka
{अथ रामस्तु वैदेह्या राज्यभोगान्मनोरमान्}
{बुभुजे वर्षसाहस्रं पालयन्सर्वतोदिशः}% १

\twolineshloka
{अन्तःपुरजनास्सर्वे राक्षसस्य गृहे स्थिताम्}
{गर्हयन्ति स्म वैदेहीं तथा जानपदा जनाः}% २

\twolineshloka
{लोकापवादभीत्या च रामः शत्रुनिवारकः}
{दर्शयन्मानुषं धर्ममन्तर्वत्नीं नृपात्मजाम्}% ३

\twolineshloka
{वाल्मीकेराश्रमे पुण्ये गङ्गातीरे महावने}
{विससर्ज महातेजा गर्भिणीं मुनिसंसदि}% ४

\twolineshloka
{सा भर्तुः परतन्त्रा हि उवास मुनिवेश्मनि}
{अर्चिता मुनिपत्नीभिर्वाल्मीकमुनि रक्षिता}% ५

\twolineshloka
{तत्रैवासूत यमलौ नाम्ना कुशलवौ सुतौ}
{तौ च तत्रैव मुनिना संस्कृतौ च ववर्धतुः}% ६

\twolineshloka
{रामोऽपि भ्रातृभिस्सार्द्धं पालयामास मेदिनीम्}
{यमादिगुणसम्पन्नस्सर्वभोगविवर्जितः}% ७

\twolineshloka
{अर्चयन्सततं विष्णुमनादिनिधनं हरिम्}
{ब्रह्मचर्यपरो नित्यं शशास पृथिवीं नृपः}% ८

\twolineshloka
{शत्रुघ्नो लवणं हत्वा मथुरां देवनिर्मिताम्}
{पालयामास धर्मात्मा पुत्राभ्यां सह राघवः}% ९

\twolineshloka
{गन्धर्वान्भरतो हत्वा सिन्धोरुभयपार्श्वतः}
{स्वात्मजौ स्थापयामास तस्मिन्देशे महाबलौ}% १०

\twolineshloka
{पश्चिमे मद्रदेशे तु मद्रान्हत्वा च लक्ष्मणः}
{स्वसुतौ च महावीर्यौ अभिषिच्य महाबलः}% ११

\twolineshloka
{गत्वा पुनरयोध्यां तु रामपादावुपस्पृशत्}
{ब्राह्मणस्य मृतं बालं कालधर्ममुपागतम्}% १२

\twolineshloka
{जीवयामास काकुत्स्थः शूद्रं हत्वा च तापसम्}
{ततस्तु गौतमीतीरे नैमिषे जनसंसदि}% १३

\twolineshloka
{इयाज वाजिमेधं च राघवः परवीरहा}
{काञ्चनीं जानकीं कृत्वा तया सार्द्धं महाबलः}% १४

\twolineshloka
{चकार यज्ञान्बहुशो राघवः परमार्थवित्}
{अयुतान्यश्वमेधानि वाजपेयानि च प्रभुः}% १५

\twolineshloka
{अग्निष्टोमं विश्वजितं गोमेधं च शतक्रतुम्}
{चकार विविधान्यज्ञान्परिपूर्णसदक्षिणान्}% १६

\twolineshloka
{एतस्मिन्नन्तरे तत्र वाल्मीकिः सुमहातपाः}
{सीतामानीय काकुत्स्थमिदं वचनमब्रवीत्}% १७

\uvacha{वाल्मीकिरुवाच}


\threelineshloka
{अपापां मैथिलीं राम त्यक्तुं नार्हसि सुव्रत}
{इयं तु विरजा साध्वी भास्करस्य प्रभा यथा}
{अनन्या तव काकुत्स्थ कस्मात्त्यक्ता त्वयानघ}% १८

\uvacha{राम उवाच}

\twolineshloka
{अपापां मैथिलीं ब्रह्मन्जानामि वचनात्तव}
{रावणेन हृता साध्वी दण्डके विजने पुरा}% १९

\twolineshloka
{तं हत्वा समरे सीतां शुद्धामग्निमुखागताम्}
{पुनर्यातोस्म्ययोध्यायां सीतामादाय धर्मतः}% २०

\twolineshloka
{लोकापवादः सुमहानभूत्पौरजनेषु च}
{त्यक्ता मया शुभाचारा तद्भयात्तव सन्निधौ}% २१

\twolineshloka
{तस्माल्लोकस्य सन्तुष्ट्यै सीता मम परायणा}
{पार्थिवानां महर्षीणां प्रत्ययं कर्तुमर्हति}% २२

\uvacha{महेश्वर उवाच}

\twolineshloka
{एवमुक्ता तदा सीता मुनिपार्थिवसंसदि}
{चकारप्रत्ययं देवी लोकाश्चर्यकरं सती}% २३

\twolineshloka
{दर्शयंस्तस्य लोकस्य रामस्यानन्यतां सती}
{अब्रवीत्प्राञ्जलिः सीता सर्वेषां जनसंसदि}% २४

\uvacha{सीतोवाच}

\twolineshloka
{यथाऽहं राघवादन्यं मनसापि न चिन्तये}
{तथा मे धरणी देवी विवरन्दातुमर्हति}% २५

\twolineshloka
{यथैव सत्यमुक्तं मे वेद्मि रामात्परं न च}
{तथा स्वपुत्र्यां वैदेह्यां धरणी सहसा इयात्}% २६

\uvacha{महेश्वर उवाच}

\twolineshloka
{ततो रत्नमयं पीठं पृष्ठे धृत्वा खगेश्वरः}
{रसातलात्तदा वीरो विज्ञाय जननीं तदा}% २७

\twolineshloka
{ततस्तु धरणीदेवी हस्ताभ्यां गृह्य मैथिलीम्}
{स्वागतेनाभिनन्द्यैनामासने सन्न्यवेशयत्}% २८

\twolineshloka
{सीतां समागतां दृष्ट्वा दिवि देवगणा भृशम्}
{पुष्पवृष्टिमविच्छिन्नां दिव्यां सीतामवाकिरन्}% २९

\twolineshloka
{सापि दिव्याप्सरोभिस्तु पूज्यमाना सनातनी}
{वैनतेयं समारुह्य तस्मान्मार्गाद्दिवं ययौ}% ३०

\twolineshloka
{दासीगणैः पूर्वभागे संवृता जगदीश्वरी}
{सम्प्राप्य परमं धाम योगिगम्यं सनातनम्}% ३१

\twolineshloka
{रसातलप्रविष्टां तु तां दृष्ट्वा सर्वमानुषाः}
{साधुसाध्विति सीतेयमुच्चैः सर्वे प्रचुक्रुशुः}% ३२

\twolineshloka
{रामः शोकसमाविष्टः सङ्गृह्य तनयावुभौ}
{मुनिभिः पार्थिवेन्द्रैश्च साकेतं प्रविवेश ह}% ३३

\twolineshloka
{अथ कालेन महता मातरः संशितव्रताः}
{कालधर्मं समापन्ना भर्तुः स्वर्गं प्रपेदिरे}% ३४

\twolineshloka
{दशवर्षसहस्राणि दशवर्षशतानि च}
{चकार राज्यं धर्मेण राघवः संशितव्रतः}% ३५

\twolineshloka
{कस्यचित्त्वथकालस्य राघवस्य निवेशनम्}
{कालस्तापसरूपेण सम्प्राप्तो वाक्यमब्रवीत्}% ३६

\uvacha{काल उवाच}

\twolineshloka
{राम राम महाबाहो धात्रा सम्प्रेषितोऽस्म्यहम्}
{यद्ब्रवीमि रघुश्रेष्ठ तच्छृणुष्व महामते}% ३७

\twolineshloka
{द्वन्द्वमेव हि कार्यं स्यादावयोः परिभाषितम्}
{तदन्तरे प्रविष्टोयस्स वद्ध्यो हि भविष्यति}% ३८

\uvacha{महेश्वर उवाच}


\threelineshloka
{तथेति च प्रतिश्रुत्य रामो राजीवलोचनः}
{द्वास्थं कृत्वा तु सौमित्रिं कालो वाक्यमभाषत}
{वैवस्वतोऽब्रवीद्वाक्यं रामं दशरथात्मजम्}% ३९

\uvacha{काल उवाच}

\twolineshloka
{शृणु राम यथावृत्तं समागमनकारणात्}
{दशवर्षसहस्राणि दशवर्षशतानि च}% ४०

\twolineshloka
{वसामि मानुषे लोके हत्वा राक्षसपुङ्गवौ}
{एवमुक्तः सुरगणैरवतीर्णोसि भूतले}% ४१

\twolineshloka
{तदयं समयः प्राप्तः स्वर्लोकं गमितुं त्वया}
{सनाथा हि सुरास्सर्वे भवन्त्वद्य त्वयानघ}% ४२

\uvacha{महेश्वर उवाच}

\twolineshloka
{एवमस्त्विति काकुत्स्थो रामः प्राह महामुनिम्}
{एतस्मिन्नन्तरे तत्र दुर्वासास्तु महातपाः}% ४३

\onelineshloka*
{राजद्वारमुपागम्य लक्ष्मणं वाक्यमब्रवीत्}

\uvacha{दुर्वासा उवाच}
\onelineshloka
{मां निवेदय काकुत्स्थं शीघ्रं गत्वा नृपात्मज}% ४४

\uvacha{महेश्वर उवाच}

\twolineshloka
{तमब्रवील्लक्ष्मणस्तु असान्निध्यमिति द्विज}
{ततः क्रोधसमाविष्टः प्राह तं मुनिसत्तमः}% ४५

\uvacha{दुर्वासा उवाच}

\onelineshloka*
{शापं दास्यामि काकुत्स्थं रामं न यदि दर्शये}

\uvacha{महेश्वर उवाच}

\twolineshloka
{तस्माच्छापभयाद्विप्रं राघवाय न्यवेदयत्}
{तत्रैवान्तर्दधे कालः सर्वभूतभयावहः}% ४६

\twolineshloka
{पूजयामास तं प्राप्तमृषिं दुर्वाससं नृपः}
{अग्रजस्य प्रतिज्ञा तं विज्ञाय रघुसत्तमः}% ४७

\twolineshloka
{तत्याज मानुषं रूपं लक्ष्मणः सरयूजले}
{विसृज्य मानुषं रूपं प्रविवेश स्वकां तनुम्}% ४८

\twolineshloka
{फणासहस्रसंयुक्तः कोटीन्दुसमवर्चसः}
{दिव्यमाल्याम्बरधरो दिव्यगन्धानुलेपनः}% ४९

\twolineshloka
{नागकन्यासहस्रैस्तु संवृतः समलङ्कृतः}
{विमानं दिव्यमारुह्य प्रययौ वैष्णवं पदम्}% ५०

\twolineshloka
{लक्ष्मणस्य गतिं सर्वां विदित्वा रघुसत्तमः}
{स्वयमप्यथ काकुत्स्थः स्वर्गं गन्तुमभीप्सितः}% ५१

\twolineshloka
{अभिषिच्याथ काकुत्स्थः स्वात्मजौ च कुशीलवौ}
{विभज्य रथनागाश्वं सधनं प्रददौ तयोः}% ५२

\twolineshloka
{कुशवत्यां कुशं तं च शरवत्यां लवं तथा}
{स्थापयामास धर्मेण राज्ये स्वे रघुसत्तमः}% ५३

\twolineshloka
{अभिप्रायं तु विज्ञाय रामस्य विदितात्मनः}
{आजग्मुर्वानराः सर्वे राक्षसाः सुमहाबलाः}% ५४

\twolineshloka
{विभीषणोऽथ सुग्रीवो जाम्बवान्मारुतात्मजः}
{नीलो नलः सुषेणश्च निषादाधिपतिर्गुहः}% ५५

\twolineshloka
{अभिषिच्य सुतौ वीरौ शत्रुघ्नश्च महामनाः}
{सर्व एते समाजग्मुरयोध्यां रामपालिताम्}% ५६

\onelineshloka*
{ते प्रणम्य महात्मानमूचुः प्राञ्जलयस्तथा}

\uvacha{वानरप्रभृतय ऊचुः}

\onelineshloka
{स्वर्लोकं गन्तुमुद्युक्तं ज्ञात्वा त्वां रघुसत्तम}% ५७


\threelineshloka
{आगताः स्म वयं सर्वे तवानुगमनं प्रति}
{न शक्ताः स्म क्षणं राम जीवितुं त्वां विना प्रभो}
{तस्मात्त्वया विशालाक्ष गच्छामस्त्रिदशालयम्}% ५८

\uvacha{महेश्वर उवाच}

\twolineshloka
{तैरेवमुक्तः काकुत्स्थो बाढमित्यब्रवीत्ततः}
{अथोवाच महातेजा राक्षसेन्द्रं विभीषणम्}% ५९

\uvacha{राम उवाच}

\onelineshloka*
{राज्यं प्रशास धर्मेण मा प्रतिज्ञां वृथा कृथाः}

\twolineshloka
{यावच्चन्द्रश्च सूर्यश्च यावत्तिष्ठति मेदिनी}
{तावद्रमस्व सुप्रीतो काले मम पदं व्रज}% ६०

\uvacha{महेश्वर उवाच}

\twolineshloka
{इत्युक्त्वाथ स काकुत्स्थः स्वाड्गं विष्णुं सनातनम्}
{श्रीरङ्गशायिनं सौम्यमिक्ष्वाकुकुलदैवतम्}% ६१

\twolineshloka
{सम्प्रीत्या प्रददौ तस्मै रामो राजीवलोचनः}
{हनुमन्तमथोवाच राघवः शत्रुसूदनः}% ६२

\uvacha{राम उवाच}

\twolineshloka
{मत्कथाः प्रचरिष्यन्ति यावल्लोके हरीश्वर}
{तावत्त्वमास मेदिन्यां काले मां व्रज सुव्रत}% ६३

\uvacha{महेश्वर उवाच}

\onelineshloka*
{तमेवमुक्त्वा काकुत्स्थो जाम्बवन्तमथाब्रवीत्}

\uvacha{राम उवाच}
\onelineshloka
{द्वापरे समनुप्राप्ते यदूनामन्वये पुनः}% ६४

\twolineshloka
{भूभारस्य विनाशाय समुत्पत्स्याम्यहं भुवि}
{करिष्ये तत्र सङ्ग्रामं स्वयं भल्लूकसत्तम}% ६५

\uvacha{महेश्वर उवाच}

\twolineshloka
{तमेवमुक्त्वा काकुत्स्थः सर्वांस्तानृक्षवानरान्}
{उवाच वाचा गच्छध्वमिति रामो महाबलः}% ६६

\twolineshloka
{मन्त्रिणो नैगमाश्चैव भरतः कैकयीसुतः}
{राघवस्यानुगमने निश्चितास्ते समाययुः}% ६७

\twolineshloka
{ततः शुक्लाम्बरधरो ब्रह्मचारी ययौ परम्}
{कुशान्गृहीत्वा पाणिभ्यां संसक्तः प्रययौ परम्}% ६८

\twolineshloka
{रामस्य दक्षिणे पार्श्वे पद्महस्ता रमा गता}
{तथैव धरणीदेवी दक्षिणेतरगा तथा}% ६९

\twolineshloka
{वेदाः साङ्गाः पुराणानि सेतिहासानि सर्वतः}
{ॐकारोऽथ वषट्कारः सावित्री लोकपावनी}% ७०

\twolineshloka
{अस्त्रशस्त्राणि च तदा धनुराद्यानि पार्वति}
{अनुजग्मुस्तथा रामं सर्वे पुरुषविग्रहाः}% ७१

\twolineshloka
{भरतश्चैव शत्रुघ्नः सर्वे पुरनिवासिनः}
{सपुत्रदाराः काकुत्स्थमनुजग्मुः सहानुगाः}% ७२

\twolineshloka
{मन्त्रिणो भृत्यवर्गाश्च किङ्करा नैगमास्तथा}
{वानराश्चैव ऋक्षाश्च सुग्रीवसहितास्तदा}% ७३

\twolineshloka
{सपुत्रदाराः काकुत्स्थमन्वगच्छन्महामतिम्}
{पशवः पक्षिणश्चैव सर्वे स्थावरजङ्गमाः}% ७४

\twolineshloka
{अनुजग्मुर्महात्मानं समीपस्था नरोत्तमाः}
{ये च पश्यन्ति काकुत्स्थं स्वपथान्तर्गतं प्रभुम्}% ७५

\twolineshloka
{ते तथानुगता रामं निवर्त्तन्ते न केचन}
{अथ त्रियोजनं गत्वा नदीं पश्चान्मुखीं स्थिताम्}% ७६

\twolineshloka
{सरयूं पुण्यसलिलां प्रविवेश सहानुगः}
{ततः पितामहो ब्रह्मा सर्वदेवगणावृतः}% ७७

\twolineshloka
{तुष्टाव रघुशार्दूलमृषिभिः सार्द्धमक्षरैः}
{अब्रवीत्तत्र काकुत्स्थं प्रविष्टं सरयूजले}% ७८

\uvacha{ब्रह्मोवाच}

\twolineshloka
{आगच्छ विष्णो भद्रं ते दिष्ट्या प्राप्तोऽसि मानद}
{भ्रातृभिस्सहदेवाभैः प्रविशस्व निजां तनुम्}% ७९

\twolineshloka
{वैष्णवीं तां महातेजां देवाकारां सनातनीम्}
{त्वं हि लोकगतिर्देव न त्वां केचित्तु जानते}% ८०

\twolineshloka
{त्वामचिन्त्यं महात्मानमक्षरं सर्वसङ्ग्रहम्}
{यमिच्छसि महातेजस्तां तनुं प्रविशस्व भोः}% ८१

\uvacha{महेश्वर उवाच}

\twolineshloka
{तस्मिन्सूर्यकराकीर्णे पुष्पवृष्टिनिपातिते}
{उत्सृज्य मानुषं रूपं स्वां तनुं प्रविवेश ह}% ८२

\twolineshloka
{अंशाभ्यां शङ्खचक्राभ्यां शत्रुघ्नभरतावुभौ}
{तदा तेन महात्मानौ दिव्यतेजस्समन्वितौ}% ८३

\twolineshloka
{शङ्खचक्रगदाशार्ङ्गपद्महस्तश्चतुर्भुजः}
{दिव्याभरणसम्पन्नो दिव्यगन्धानुलेपनः}% ८४

\twolineshloka
{दिव्यपीताम्बरधरः पद्मपत्रनिभेक्षणः}
{युवा कुमारः सौम्याङ्गः कोमलावयवोज्ज्वलः}% ८५

\twolineshloka
{सुस्निग्धनीलकुटिलकुन्तलः शुभलक्षणः}
{नवदूर्वाङ्कुरः श्यामः पूर्णचन्द्र निभाननः}% ८६

\twolineshloka
{देवीभ्यां सहितः श्रीमान्विमानमधिरुह्य च}
{तस्मिन्सिंहासने दिव्ये मूले कल्पतरोः प्रभुः}% ८७

\twolineshloka
{निषसाद महातेजाः सर्वदेवैरभिष्टुतः}
{राघवानुगता ये च ऋक्षवानरमानुषाः}% ८८

\twolineshloka
{स्पृष्ट्वैव सरयूतोयं सुखेन त्यक्तजीविताः}
{रामप्रसादात्ते सर्वे दिव्यरूपधराः शुभाः}% ८९

\twolineshloka
{दिव्यमाल्याम्बरधरा दिव्यमङ्गलवर्चसः}
{आरुरोह विमानं तदसङ्ख्यास्तत्र देहिनः}% ९०

\twolineshloka
{सर्वैः परिवृतः श्रीमान्रामो राजीवलोचनः}
{पूजितः सुरसिद्धौघैर्मुनिभिस्तु महात्मभिः}% ९१

\twolineshloka
{आययौ शाश्वतं दिव्यमक्षरं स्वपदं विभुः}
{यः पठेद्रामचरितं श्लोकं श्लोकार्धमेव वा}% ९२

\twolineshloka
{शृणुयाद्वा तथा भक्त्या स्मरेद्वा शुभदर्शने}
{कोटिजन्मार्जितात्पापाज्ज्ञानतोऽज्ञानतः कृतात्}% ९३

\twolineshloka
{विमुक्तो वैष्णवं लोकं पुत्रदारसबान्धवैः}
{समाप्नुयाद्योगगम्यमनायासेन वै नरः}% ९४


\onelineshloka
{एतत्ते कथितं देवि रामस्य चरितं महत्}
{धन्योऽस्म्यहं त्वया देवि रामचन्द्रस्य कीर्त्तनात्}
{किमन्यच्छ्रोतुकामासि तद्ब्रवीमि वरानने}% ९५

{॥इति श्रीपाद्मे महापुराणे पञ्चपञ्चाशत्साहस्र्यां संहितायामुत्तरखण्डे उमामहेश्वर संवादे श्रीरामचरितकथनं नाम चतुश्चत्वारिंशदधिकद्विशततमोऽध्यायः॥२४४॥}



    \chapt{ब्रह्म-पुराणम्}

\sect{अनन्तवासुदेवमाहात्म्यवर्णनम्}

\src{ब्रह्म-पुराणम्}{पूर्वखण्डः}{अध्यायः १७६}{श्लोकाः ३७---५१}
\tags{concise, complete}
\notes{Summary of Ramayana, during the narration of AnantaVasudeva Mahatmyam.}
\textlink{https://sa.wikisource.org/wiki/ब्रह्मपुराणम्/अध्यायः_१७६}
\translink{}

\storymeta


\uvacha{मुनय ऊचुः}

\twolineshloka
{न हि नस्तृप्तिरस्तीह शृण्वतां भगवत्कथाम्}
{पुनरेव परं गुह्यं वक्तुमर्हस्यशेषतः} %॥१॥

\twolineshloka
{अनन्तवासुदेवस्य न सम्यग्वर्णितं त्वया}
{श्रोतुमिच्छामहे देव विस्तरेण वदस्व नः} %॥२॥

\uvacha{ब्रह्मोवाच}

\twolineshloka
{प्रवक्ष्यामि मुनिश्रेष्ठाः सारात्सारतरं परम्}
{अनन्तवासुदेवस्य माहात्म्यं भुवि दुर्लभम्} %॥३॥

\twolineshloka
{आदिकल्पे पुरा विप्रास्त्वहमव्यक्तजन्मवान्}
{विश्वकर्माणमाहूय वचनं प्रोक्तवानिदम्} %॥४॥

\twolineshloka
{वरिष्ठं देवशिल्पीन्द्रं विश्वकर्माग्रकर्मिणम्}
{प्रतिमां वासुदेवस्य कुरु शैलमयीं भुवि} %॥५॥

\twolineshloka
{यां प्रेक्ष्य विधिवद्‌भक्ताः सेन्द्रा वै मानुषादयः}
{येन दानवरक्षोभ्यो विज्ञाय सुमहद्‌भयम्} %॥६॥

\twolineshloka
{त्रिदिवं समनुप्राप्य सुमेरुशिखरं चिरम्}
{वासुदेवं समाराध्य निरातङ्का वसन्ति ते} %॥७॥

\twolineshloka
{मम तद्वचनं श्रुत्वा विश्वकर्मा तु तत्क्षणात्}
{चकार प्रतिमां शुद्धां शङ्खचक्रगदाधराम्} %॥८॥

\twolineshloka
{सर्वलक्षणसंयुक्तां पुण्डरीकायतेक्षणाम्}
{श्रीवत्सलक्ष्संयुक्तामत्युग्रां प्रतिमोत्तमाम्} %॥९॥

\twolineshloka
{वनमालावृतोरस्कां मुकुटाङ्गदधारिणीम्}
{पीतवस्त्रां सुपीनांसां कुण्डलाभ्यामलङ्कृताम्} %॥१०॥

\twolineshloka
{एवं सा प्रतिमा दिव्या गुह्यमन्त्रैस्तदा स्वयम्}
{प्रतिष्ठाकालमासाद्य मयाऽऽसौ निर्मिता पुरा} %॥११॥

\twolineshloka
{तस्मिन्काले तदा शक्रो देवराट्खेचरैः सह}
{जगाम ब्रह्मसदनमारुह्य गजमुत्तमम्} %॥१२॥

\twolineshloka
{प्रसाद्य प्रतिमां शक्रः स्नानदानैः पुनः पुनः}
{प्रतिमां तां समादाय स्वपुरं पुनरागमत्} %॥१३॥

\twolineshloka
{तां समाराध्य सुचिरं यतवाक्कायमानसः}
{वृत्राद्यानसुरान्क्रूरान्नमुचिप्रमुखान्स च} %॥१४॥

\twolineshloka
{निहत्य दानवान्भीमान्भूक्तवान्भुवनत्रयम्}
{द्वितीये च युगे प्राप्ते त्रेतायां राक्षसाधिपः} %॥१५॥

\twolineshloka
{बभूव सुमहावीर्यो दशग्रीवः प्रतापवान्}
{दश वर्षसहस्राणि निराहारो जितेन्द्रियः} %॥१६॥

\twolineshloka
{चचार व्रतमत्युग्रं तपः परमदुश्चरम्}
{तपसा तेन तुष्टोऽहं वरं तस्मै प्रदत्तवान्} %॥१७॥

\twolineshloka
{अवध्यः सर्वदेवानां स दैत्योरगयक्षसाम्}
{शापप्रहरणैरुग्रैरवध्यो यमकिङ्करैः} %॥१८॥

\twolineshloka
{वरं प्राप्य तदा रक्षो यक्षान्सर्वगणानिमान्}
{धनाध्यक्षं विनिर्जित्य शक्रं जेतुं समुद्यतः} %॥१९॥

\twolineshloka
{सङ्ग्रामं सुमहाघोरं कृत्वा देवैः स राक्षसः}
{देवराजं विनिर्जित्य तदा इन्द्रिजितेति वै} %॥२०॥

\twolineshloka
{राक्षसस्तत्सुरो नाम मेघनादः प्रलब्धवान्}
{अमरावतीं ततः प्राप्य देवराजगृहे शुभे} %॥२१॥

\twolineshloka
{ददर्शाञ्जनसङ्काशां रावणस्तु बलान्वितः}
{प्रतिमां वासुदेवस्य सर्वलक्षणसंयुताम्} %॥२२॥

\twolineshloka
{श्रीवत्सलक्ष्मसंयुक्तं पद्मपत्रायतेक्षणाम्}
{वनमालावृतोरस्कां सर्वकामफलप्रदाम्} %॥२३॥

\twolineshloka
{शङ्खचक्रगदाहस्तां पीतवस्त्रां चतुर्भुजाम्}
{सर्वाभरणसंयुक्तां सर्वकामफलप्रदाम्} %॥२४॥

\twolineshloka
{विहाय रत्नसङ्घांश्च प्रतिमां शुभलक्षणाम्}
{पुष्पकेण विमानेन लङ्कां प्रास्थापयद्‌द्रुतम्} %॥२५॥

\twolineshloka
{पुराध्यक्षः स्थितः श्रीमान्धर्मात्मा स विभीषणः}
{रावणस्यानुजो मन्त्री नारायणपरायणः} %॥२६॥

\twolineshloka
{दृष्ट्वा तां प्रतिमां दिव्यां देवेन्द्रभवनच्युताम्}
{रोमाञ्चिततनुर्भूत्वा विस्मयं समपद्यत} %॥२७॥

\twolineshloka
{प्रणम्य शिरसा देवं प्रहृष्टेनान्तरात्मना}
{अद्य मे सफलं जन्म अद्य मे सफलं तपः} %॥२८॥

\twolineshloka
{इत्युक्त्वा स तु धर्मात्मा प्रणिपत्य मुहुर्मुहुः }
{ज्येष्ठं भ्रातरमासाद्य कृताञ्जलिरभाषत} %॥२९॥

\twolineshloka
{राजन्प्रतिमया त्वं मे प्रसादं कर्तुमर्हसि}
{यामाराध्य जगन्नाथ निस्तरेयं भवार्णवम्} %॥३०॥

\twolineshloka
{भ्रातुर्वचनमाकर्ण्य रावणस्तं तदाऽब्रवीत्}
{गृहाण प्रतिमां वीर त्वनया किं करोम्यहम्} %॥३१॥

\twolineshloka
{स्वयम्भुवं समाराध्य त्रैलोक्यं विजये त्वहम्}
{नानाश्चर्यमयं देवं सर्वभूतभवोद्भवम्} %॥३२॥

\twolineshloka
{विभीषणो महाबुद्धिस्तदा तां प्रतिमां शुभाम्}
{शतमष्टोत्तरं चाब्दं समाराध्य जनार्दनम्} %॥३३॥

\twolineshloka
{अजरामरणं प्राप्तमणिमादिगुणैर्युतम्}
{राज्यं लङ्काधिपत्यं च भोगान्भुङ्क्ते यथेप्सितान्} %॥३४॥

\uvacha{मुनय ऊचुः}

\twolineshloka
{अहो नो विस्मयो जातः श्रुत्वेदं परमामृतम्}
{अनन्तवासुदेवस्य सम्भवं भुवि दुर्लभम्} %॥३५॥

\twolineshloka
{श्रोतुमिच्छामहे देव विस्तरेण यथातथम्}
{तस्य देवस्य माहात्म्यं वक्तुमर्हस्यशेषतः} %॥३६॥

\dnsub{रामकथासारः}

\uvacha{ब्रह्मोवाच}

\twolineshloka
{तदा स राक्षसः क्रूरो देवगन्धर्वकिन्नरान्}
{लोकपालान्समनुजान्मुनिसिद्धांश्च पापकृत्} %॥३७॥

\twolineshloka
{विजित्य समरे सर्वानाजहार तदङ्गनाः}
{संस्थाप्य नगरीं लङ्कां पुनः सीतार्थमोहितः} %॥३८॥

\twolineshloka
{शङ्कितो मृगरूपेण सौवर्णेन च रावणः}
{ततः क्रुद्धेन रामेण रणे सौमित्रिणा सह} %॥३९॥

\twolineshloka
{रावणस्य वधार्थाय हत्वा वालिं मनोजवम्}
{अभिषिक्तश्च सुग्रीवो युवराजोऽङ्गदस्तथा} %॥४०॥

\twolineshloka
{हनुमान्नलनीलश्च जाम्बवान्पनसस्तथा}
{गवयश्च गवाक्षश्च पाठीनः परमौजसः} %॥४१॥

\twolineshloka
{एतैश्चान्यैश्च बहुभिर्वानरैः समहाबलैः}
{समावृतो महाघोरै रामो राजीवलोचनः} %॥४२॥

\twolineshloka
{गिरीणां सर्वसङ्घातैः सेतुं बद्‌ध्वा महोदधौ}
{बलेन महता रामः समुत्तीर्य महोदधिम्} %॥४३॥

\twolineshloka
{सङ्ग्राममतुलं चक्रे रक्षोगणसमन्वितः}
{यमहस्तं प्रहस्तं च निकुम्भं कुम्भमेव च} %॥४४॥

\twolineshloka
{नरान्तकं महावीर्यं तथा चैव यमान्तकम्}
{मालाढ्यं मालिकाढ्यं च हत्वा रामस्तु वीर्यवान्} %॥४५॥

\twolineshloka
{पुनरिन्द्रजितं हत्वा कुम्भकर्णं सरावणम्}
{वैदेहीं चाग्निनाऽऽसोध्य दत्त्वा राज्यं विभिषणे} %॥४६॥

\twolineshloka
{वासुदेवं समादाय यानं पुष्पकमारुहत्}
{लीलया समनुप्रापदयोध्यां पूर्वपालिताम्} %॥४७॥

\twolineshloka
{कनिष्ठं भरतं स्नेहाच्छत्रुघ्नं भक्तवत्सलः}
{लीलया समनुप्रापदयोध्यां पूर्वपालिताम्} %॥४८॥

\twolineshloka
{पुरातनीं स्वमूर्तिं च समाराध्य ततो हरिः}
{दश वर्षसहस्राणि दश वर्षशतानि च} %॥४९॥

\twolineshloka
{भुक्ताव सागरपर्यन्तां मेदिनीं स तु राघवः}
{राज्यमासाद्य सुगतिं वैष्णवं पदमाविशत्} %॥५०॥

\twolineshloka
{तां चापि प्रतिमां रामः समुद्रेशाय दत्तवान्}
{धन्यो रक्षयितासि त्वं तोयरत्नसमन्वितः} %॥५१॥

\closesub

\twolineshloka
{द्वापरं युगमासाद्य यदा देवो जगत्पतिः}
{धरण्याश्चानुरोधेन भावशैथिल्यकारणात्} %॥५२॥

\twolineshloka
{अवतीर्णः स भगवान्वसुदेवकुले प्रभुः}
{कंसादीनां वधार्थाय सङ्कर्षणसहायवान्} %॥५३॥

\twolineshloka
{तदा तां प्रतिमां विप्राः सर्ववाञ्छाफलप्रदाम्}
{सर्वलोकहितार्थाय कस्यचित्कारणान्तरे} %॥५४॥

\twolineshloka
{तस्मिन्क्षेत्रवरे पुण्ये दुर्लभे पुरुषोत्तमे}
{उज्जहार स्वयं तोयात्समुद्रः सरितां पतिः} %॥५५॥

\twolineshloka
{तदा प्रभृति तत्रैव क्षेत्रे मुक्तिप्रदे द्विजाः}
{आस्ते स देवो देवानां सर्वकामफलप्रदः} %॥५६॥

\twolineshloka
{ये संश्रयन्ति चानन्तं भक्त्या सर्वेश्वरं प्रभुम्}
{वाङ्मनः कर्मभिर्नित्यं ते यान्ति परमं पदम्} %॥५७॥

\twolineshloka
{दृष्ट्वाऽनन्तं सकृद्‌भक्त्या सम्पूज्य प्रणिपत्य च}
{राजसूयाश्वमेधाभ्यां फलं दशगुणं लभेत्} %॥५८॥

\twolineshloka
{सर्वकामसमृद्धेन कामगेन सुवर्चसा}
{विमानेनार्कवर्णेन किङ्किणीजालमालिना} %॥५९॥

\twolineshloka
{त्रिःसप्तकुलमुद्धृत्य दिव्यस्त्रीगणसेवितः}
{उपगीयमानो गन्धर्वैर्नरो विष्णुपुरं व्रजेत्} %॥६०॥

\twolineshloka
{तत्र भुक्त्वा वरान्भोगाञ्जरामरणवर्जितः}
{दिव्यरूपधरः श्रीमान्यावदाभूतसम्प्लवम्} %॥६१॥

\twolineshloka
{पुण्यक्षयादिहाऽऽयातश्चतुर्वेदी द्विजोत्तमः}
{वैष्णवं योगमास्थाय ततो मोक्षमवाप्नुयात्} %॥६२॥

\twolineshloka
{एवं मया त्वनन्तोऽसौ कीर्तितो मुनिसत्तमाः}
{कः शक्नोति गुणान्वक्तुं तस्य वर्षशतैरपि} %॥६३ 

॥इति श्रीमहापुराणे आदिब्राह्मे स्वयम्भ्वृषिसंवादेऽनन्तवासुदेवमाहात्म्यनिरूपणं नाम षट्सप्तत्यधिकशततमोऽध्यायः॥१७६॥
    \sect{विष्णोर्प्रादुर्भावः --- रामावतारः}

\src{ब्रह्म-पुराणम्}{पूर्वखण्डः}{अध्यायः २१३}{श्लोकाः १२४---१५८}
\tags{concise, complete}
\notes{Summary of Ramayana, during the narration of various Vishnu avataras.}
\textlink{https://sa.wikisource.org/wiki/ब्रह्मपुराणम्/अध्यायः_२१३}
\translink{}

\storymeta

\uvacha{व्यास उवाच}

\addtocounter{shlokacount}{123}

\twolineshloka
{चतुर्विंशे युगे वाऽपि विश्वामित्रपुरःसरः}
{जज्ञे दशरथस्याथ पुत्रः पद्मयतेक्षणः} %॥१२४॥

\twolineshloka
{कृत्वाऽत्मानं महाबाहुश्चतुर्धा प्रभुरीश्वरः}
{लोके राम इति ख्यातस्तेजसा भास्करोपमः} %॥१२५॥

\twolineshloka
{प्रसादनार्थं लोकस्य रक्षसां निग्रहाय च}
{धर्मस्य च विवृद्ध्यर्थं जज्ञे तत्र महयशाः} %॥१२६॥

\twolineshloka
{तमप्याहुर्मनुष्येन्द्रं सर्वभूतहिते रतम्}
{यः समाः सर्वधर्मज्ञश्चतुर्दश वनेऽवसत्} %॥१२७॥

\twolineshloka
{लक्ष्मणानुचरो रामः सर्वभूतहिते रतः}
{चतुर्दश वने तप्त्वा तपो वर्षणि राघवः} %॥१२८॥

\twolineshloka
{रूपिणी तस्य पार्श्वस्था सीतेति प्रथिता जने}
{पूर्वोदिता तु या लक्ष्मीर्भर्तारमनुगच्छति} %॥१२९॥

\twolineshloka
{जनस्थाने वसन्कार्यं त्रिदशानां चकार सः}
{तस्यापकारिणं क्रूरं पौलस्त्यं मनुजर्षभः} %॥१३०॥

\twolineshloka
{सीतायाः पदमन्विच्छन्निजघान महायशाः}
{देवासुरगणानां च यक्षराक्षसभोगिनाम्} %॥१३१॥

\twolineshloka
{यत्रावध्यं राक्षसेन्द्रं रावणं युधि दुर्जयम्}
{युक्तं राक्षसकोटीभिर्नीलाञ्जनचयोपमम्} %॥१३२॥

\twolineshloka
{त्रैलाक्यद्रावणं क्रूरं रावणं राक्षसेश्वरम्}
{दुर्जयं दुर्धरं दृप्तं शार्दूलसमविक्रमम्} %॥१३३॥

\twolineshloka
{दुर्निरीक्ष्यं सुरगणैर्वरदानेन दर्पितम्}
{जघान सचिवैः सार्धं ससैन्यं रावणं युधि} %॥१३४॥

\twolineshloka
{महाभ्रगणसङ्काशं महाकायं महाबलम्}
{रावणं निजघानाऽऽशु रामो भूतपतिः पुरा} %॥१३५॥

\twolineshloka
{सुग्रीवस्य कृते येन वानरेन्द्रो महाबलः}
{वाली विनिहतः सङ्ख्ये सुग्रीवश्चाभिषेचितः} %॥१३६॥

\twolineshloka
{मधोश्च तनयो दृप्तो लवणो नाम दानवः}
{हतो मधुवने वीरो वरमत्तो महासुरः} %॥१३७॥

\twolineshloka
{यज्ञविघ्नकरौ येन मुनीनां भावितात्मनाम्}
{मारीचश्च सुबाहुश्च बलेन बलिनां वरौ} %॥१३८॥

\twolineshloka
{निहतौ च निराशौ च कृतौ तेन महात्मना}
{समरे युद्धशौण्डेन तथाऽन्ये चापि राक्षसाः} %॥१३९॥

\twolineshloka
{विराधश्च कबन्धश्च राक्षसौ भीमविक्रमौ}
{जघान पुरुषव्याघ्रो गन्धवौ शापमोहितौ} %॥१४०॥

\twolineshloka
{हुताशनार्कांशुतडिद्‌गुणाभैः प्रतप्तजाम्बूनदचित्रपुङ्खैः}
{महेन्द्रवज्राशनितुल्यसारै रिपून्स रामः समरे निजघ्ने} %॥१४१॥

\twolineshloka
{तस्मै दत्तानि शस्त्राणि विश्वामित्रेण धीमता}
{वधार्थं देवशत्रूणां दुर्धर्षाणां सुरैरपि} %॥१४२॥

\twolineshloka
{वर्तमाने मखे येन जनकस्य महात्मनः}
{भग्नं माहेश्वरं चापं क्रीडता लीलया पुरा} %॥१४३॥

\twolineshloka
{एतानि कृत्वा कर्माणि रामो धर्मभृतां वरः}
{दशाश्वमेधाञ्जारूथ्यानाजहार निरर्गलान्} %॥१४४॥

\twolineshloka
{नाश्रूयन्ताशुभा वाचो नाऽऽकुलं मारुतो ववौ}
{न वित्तहरणं चाऽऽसीद्रामे राज्यं प्रशासति} %॥१४५॥

\twolineshloka
{परिदेवन्ति विधवा नानर्थाश्च कदाचन}
{सर्वमासीच्छुभं तत्र रामे राज्यं प्रशासति} %॥१४६॥

\twolineshloka
{न प्राणिनां भयं चाऽऽसीज्जलाग्न्यनिलघातजम्}
{न चापि वृद्धा बालानां प्रेतकार्याणि चक्रिरे} %॥१४७॥

\twolineshloka
{ब्रह्मचर्यपरं क्षत्रं विशस्तु क्षत्रिये रताः}
{शूद्राश्चैव हि वर्णास्त्रीञ्शुश्रूषन्त्यनहङ्कृताः} %॥१४८॥

\twolineshloka
{नार्यो नात्यचरन्भर्तॄन्भार्यां नात्यचरत्पतिः}
{सर्वमासीज्जगद्दान्तं निर्दस्युरभवन्मही} %॥१४९॥

\twolineshloka
{राम एकोऽभवद्भर्ता रामः पालयिताऽभवत्}
{आसन्वर्षसहस्राणि तथा पुत्रसहस्रिणः} %॥१५०॥

\twolineshloka
{अरोगाः प्राणिनश्चाऽऽसन्रामे राज्यं प्रशासति}
{देवतानामृषीणां च मनुष्याणां च सर्वशः} %॥१५१॥

\twolineshloka
{पृथिव्यां समवायोऽभूद्रामे राज्यं प्रशासति}
{गाथामप्यत्र गायन्ति ये पुराणविदो जनाः} %॥१५२॥

\twolineshloka
{रामे निबद्धतत्त्वार्था माहात्म्यं तस्य धीमतः}
{श्यामो युवा लोहिताक्षो दीप्तास्यो मितभाषितः} %॥१५३॥

\twolineshloka
{आजानुबाहुः सुमुखः सिंहस्कन्धो महाभुजः}
{दश वर्षसहस्राणि रामो राज्यमकारयत्} %॥१५४॥

\twolineshloka
{ऋक्सामयजुषां घोषो ज्याघोषश्च महात्मनः}
{अव्युच्छिन्नोऽभवद्राष्ट्रे दीयतां भुज्यतामिति} %॥१५५॥

\twolineshloka
{सत्त्ववान्गुणसम्पन्नो दीप्यमानः स्वतेजसा}
{अतिचन्द्रं च सूर्यं च रामो दाशरथिर्बभौ} %॥१५६॥

\twolineshloka
{ईजे क्रतुशतैः पुण्यैः समाप्तवरदक्षिणैः}
{हित्वाऽयोध्यां दिवं यातो राघवो हि महाबलः} %॥१५७॥

\twolineshloka
{एवमेव महाबाहुरिक्ष्वाकुकुलनन्दनः}
{रावणं सगणं हत्वा दिवमाचक्रमे विभुः} %॥१५८॥

॥इति श्रीमहापुराणे आदिब्राह्मे विष्णोः प्रादुर्भावानुकीर्तनं नाम त्रयोदशाधिकद्विशततमोऽध्यायः॥२१३॥

    \chapt{शिव-पुराणम्}

\sect{रामेश्वरमाहात्म्यम्}

\src{शिव-पुराणम्}{पूर्वखण्डः}{अध्यायः ३१}{श्लोकाः १---४५}
\tags{concise, complete}
\notes{Summary of Ramayana.}
\textlink{}
\translink{}

\storymeta

\uvacha{सूत उवाच}

\twolineshloka
{अतः परं प्रवक्ष्यामि लिङ्गं रामेश्वराभिधम्} 
{उत्पन्नं च यथा पूर्वमृषयश्शृणुतादरात्}

\onelineshloka
{पुरा विष्णुः पृथिव्यां चावततार सतां प्रियः} %॥२॥

\twolineshloka
{तत्र सीता हृता विप्रा रावणेनोरुमायिना} 
{प्रापिता स्वगृहं सा हि लङ्कायां जनकात्मजा} %॥३॥

\twolineshloka
{अन्वेषणपरस्तस्याः किष्किन्धाख्यां पुरीमगात्} 
{सुग्रीवहितकृद्भूत्वा वालिनं सञ्जघान ह} %॥४॥

\twolineshloka
{तत्र स्थित्वा कियत्कालं तदन्वेषणतत्परः} 
{सुग्रीवाद्यैर्लक्ष्मणेन विचारं कृतवान्स वै} %॥५॥

\twolineshloka
{कपीन्सम्प्रेषयामास चतुर्दिक्षु नृपात्मजः} 
{हनुमत्प्रमुखान्रामस्तदन्वेषणहेतवे} %॥६॥

\twolineshloka
{अथ ज्ञात्वा गतां लङ्कां सीतां कपिवराननात्} 
{सीताचूडामणिं प्राप्य मुमुदे सोऽति राघवः} %॥७॥

\twolineshloka
{सकपीशस्तदा रामो लक्ष्मणेन युतो द्विजाः} 
{सुग्रीवप्रमुखैः पुण्यैर्वानरैर्बलवत्तरैः} %॥८॥

\twolineshloka
{पद्मैरष्टादशाख्यैश्च ययौ तीरं पयोनिधेः}
{दक्षिणे सागरे यो वै दृश्यते लवणाकरः}

\twolineshloka
{तत्रागत्य स्वयं रामो वेलायां संस्थितो हि सः} 
{वानरैस्सेव्यमानस्तु लक्ष्मणेन शिवप्रियः} %॥4॥

\twolineshloka
{हा जानकि कुतो याता कदा चेयं मिलिष्यति} 
{अगाधस्सागरश्चैवातार्या सेना च वानरी} %॥११॥

\twolineshloka
{राक्षसो गिरिधर्त्ता च महाबलपराक्रमः}
{लङ्काख्यो दुर्गमो दुर्ग इन्द्रजित्तनयोस्य वै} %॥१२॥

\twolineshloka
{इत्येवं स विचार्यैव तटे स्थित्वा सलक्ष्मणः}
{आश्वासितो वनौकोभिरङ्गदादिपुरस्सरैः} %॥१३॥

\twolineshloka
{एतस्मिन्नन्तरे तत्र राघवश्शैवसत्तमः}
{उवाच भ्रातरं प्रीत्या जलार्थी लक्ष्मणाभिधम्} %॥१४॥

\uvacha{राम उवाच}

\twolineshloka
{भ्रातर्लक्ष्मण वीरेशाहं जलार्थी पिपासितः}
{तदानय द्रुतं पाथो वानरैः कैश्चिदेव हि} %॥१५॥

\uvacha{सूत उवाच}

\twolineshloka
{तच्छ्रुत्वा वानरास्तत्र ह्यधावन्त दिशो दश} 
{नीत्वा जलं च ते प्रोचुः प्रणिपत्य पुरः स्थिताः} %॥१६॥

\uvacha{वानरा ऊचुः}

\twolineshloka
{जलं च गृह्यतां स्वामिन्नानीतं तत्त्वदाज्ञया}
{महोत्तमं च सुस्वादु शीतलं प्राणतर्पणम्} %॥१७॥

\uvacha{सूत उवाच}

\twolineshloka
{सुप्रसन्नतरो भूत्वा कृपादृष्ट्या विलोक्य तान् }
{तच्छ्रुत्वा रामचन्द्रोऽसौ स्वयं जग्राह तज्जलम्} %॥१८॥

\twolineshloka
{स शैवस्तज्जलं नीत्वा पातुमारब्धवान्यदा} 
{तदा च स्मरणं जातमित्थमस्य शिवेच्छया} %॥१९॥

\twolineshloka
{न कृतं दर्शनं शम्भोर्गृह्यते च जलं कथम्}
{स्वस्वामिनः परेशस्य सर्वानन्दप्रदस्य वै} %॥२०॥

\twolineshloka
{इत्युक्त्वा च जलं पीतं तदा रघुवरेण च}
{पश्चाच्च पार्थिवीं पूजां चकार रघुनन्दनः} %॥२१॥

\twolineshloka
{आवाहनादिकांश्चैव ह्युपचारान्प्रकल्प्य वै}
{विधिवत्षोडश प्रीत्या देवमानर्च शङ्करम्} %॥२२॥

\twolineshloka
{प्रणिपातैस्स्तवैर्दिव्यैश्शिवं सन्तोष्य यत्नतः}
{प्रार्थयामास सद्भक्त्या स रामश्शङ्करं मुदा} %॥२३॥

\uvacha{राम उवाच}

\twolineshloka
{स्वामिञ्छम्भो महादेव सर्वदा भक्तवत्सल} 
{पाहि मां शरणापन्नं त्वद्भक्तं दीनमानसम्} %॥२४॥

\twolineshloka
{एतज्जलमगाधं च वारिधेर्भवतारण}
{रावणाख्यो महावीरो राक्षसो बलवत्तरः} %॥२५॥

\twolineshloka
{वानराणां बलं ह्येतच्चञ्चलं युद्धसाधनम्}
{ममकार्यं कथं सिद्धं भविष्यति प्रियाप्तये} %॥२६॥

\twolineshloka
{तस्मिन्देव त्वया कार्यं साहाय्यं मम सुव्रत} 
{साहाय्यं ते विना नाथ मम कार्य्यं हि दुर्लभम्} %॥२७॥

\twolineshloka
{त्वदीयो रावणोऽपीह दुर्ज्जयस्सर्वथाखिलैः} 
{त्वद्दत्तवरदृप्तश्च महावीरस्त्रिलोकजित्} %॥२८॥

\twolineshloka
{अप्यहं तव दासोऽस्मि त्वदधीनश्च सर्वथा} 
{विचार्येति त्वया कार्यः पक्षपातस्सदाशिव} %॥२९॥

\uvacha{सूत उवाच}

\twolineshloka
{इत्येवं स च सम्प्रार्थ्य नमस्कृत्य पुनःपुनः} 
{तदा जयजयेत्युच्चैरुद्धोषैश्शङ्करेति च} %॥4॥

\twolineshloka
{इति स्तुत्वा शिवं तत्र मन्त्रध्यानपरायणः}
{पुनः पूजां ततः कृत्वा स्वाम्यग्रे स ननर्त ह} %॥३१॥

\twolineshloka
{प्रेमी विक्लिन्नहृदयो गल्लनादं यदाकरोत्} 
{तदा च शङ्करो देवस्सुप्रसन्नो बभूव ह} %॥३२॥

\twolineshloka
{साङ्गस्सपरिवारश्च ज्योतीरूपो महेश्वरः}
{यथोक्तरूपममलं कृत्वाविरभवद्द्रुतम्} %॥३३॥

\twolineshloka
{ततस्सन्तुष्टहृदयो रामभक्त्या महेश्वरः} 
{शिवमस्तु वरं ब्रूहि रामेति स तदाब्रवीत्} %॥३४॥

\twolineshloka
{तद्रूपं च तदा दृष्ट्वा सर्वे पूतास्ततस्स्वयम्} 
{कृतवान्राघवः पूजां शिवधर्मपरायणः} %॥३५॥

\twolineshloka
{स्तुतिं च विविधां कृत्वा प्रणिपत्य शिवं मुदा} 
{जयं च प्रार्थयामास रावणाजौ तदात्मनः} %॥३६॥

\twolineshloka
{ततः प्रसन्नहृदयो रामभक्त्या महेश्वरः} 
{जयोस्तु ते महाराज प्रीत्या स पुनरब्रवीत्} %॥३७॥

\twolineshloka
{शिवदत्तं जयं प्राप्य ह्यनुज्ञां समवाप्य च} 
{पुनश्च प्रार्थयामास साञ्जलिर्नतमस्तकः} %॥३८॥

\uvacha{राम उवाच}

\twolineshloka
{त्वया स्थेयमिह स्वामिंल्लोकानां पावनाय च} 
{परेषामुपकारार्थं यदि तुष्टोऽसि शङ्कर} %॥३९॥

\uvacha{सूत उवाच}

\twolineshloka
{इत्युक्तस्तु शिवस्तत्र लिङ्गरूपोऽभवत्तदा} 
{रामेश्वरश्च नाम्ना वै प्रसिद्धो जगतीतले} %॥४०॥

\twolineshloka
{रामस्तु तत्प्रभावाद्वै सिन्धुमुत्तीर्य चाञ्जसा}
{रावणादीन्निहत्याशु राक्षसान्प्राप तां प्रियाम्} %॥४१॥

\twolineshloka
{रामेश्वरस्य महिमाद्भुतोऽभूद्भुवि चातुलः} 
{भुक्तिमुक्तिप्रदश्चैव सर्वदा भक्तकामदः} %॥४२॥

\twolineshloka
{दिव्यगङ्गाजलेनैव स्नापयिष्यति यश्शिवम्} 
{रामेश्वरं च सद्भक्त्या स जीवन्मुक्त एव हि} %॥४३॥

\twolineshloka
{इह भुक्त्वाखिलान्भोगान्देवानां दुर्लभानपि} 
{अन्ते प्राप्य परं ज्ञानं कैवल्यं प्राप्नुयाद्ध्रुवम्} %॥४४॥

\twolineshloka
{इति वश्च समाख्यातं ज्योतिर्लिगं शिवस्य तु}
{रामेश्वराभिधं दिव्यं शृण्वतां पापहारकम्} %॥४५।

॥इति श्रीशिवमहापुराणे चतुर्थ्यां कोटिरुद्रसन्तायां रामेश्वरमाहात्म्यवर्णनं नामैकत्रिंशोऽध्यायः॥

\closesection
    \chapt{सौरपुराणम्}

\sect{इक्ष्वाकुकुलसम्भवनृपमालिका-कथनम्}

\src{सौरपुराणम्}{}{अध्यायः ३०}{श्लोकाः ४८--६९}
\vakta{}
\shrota{}
\notes{This chapter briefly recounts the life of Lord Rama---His divine birth, marriage to Sita, exile, Sita's abduction by Ravana, the alliance with Hanuman and Sugriva, the war in Lanka, and His triumphant return and coronation. It concludes with a lineage of Rama's descendants from Lava/Kuśa.}
\textlink{https://archive.org/details/saurapurana1924compl/page/97/mode/2up}
\translink{}

\storymeta



\addtocounter{shlokacount}{47}

\twolineshloka
{दीर्घबाहुस्ततो जज्ञे रघुस्तस्याभवत्सुतः}
{रघोरजस्तु विख्यातो राजा दशरथस्ततः} %॥४८॥

\twolineshloka
{तस्य पुत्राश्च चत्वारो धर्मज्ञा लोकविश्रुताः}
{रामोऽथ भरतश्चैव तृतीयो लक्ष्मणः स्मृतः} %॥४९॥

\twolineshloka
{चतुर्थश्चैव शत्रुघ्नो रामो नारायणः स्वयम्}
{धर्मज्ञः सत्यसङ्कल्पो महादेवपरायणः} %॥५०॥

\twolineshloka
{सीता तस्याभवद्भार्या पार्वत्यंशसमुद्भवा}
{जनकेन पुरा गौरी तपसा तोषिता यतः} %॥५१॥

\twolineshloka
{जनकाय ददौ शम्भुः प्रीतो धनुरनुत्तमम्}
{तद्धनुर्भञ्जयामास जनकस्य गृहे स्थितम्} %॥५२॥

\twolineshloka
{दृष्ट्वा पराक्रमं तस्य रामस्य गुणशालिनः}
{जनकः प्रददौ तस्मै सीतां ब्रह्मविदां वरः} %॥५३॥

\twolineshloka
{पित्रा कृतोऽभिषेकार्थं रामो राज्यस्य वै यदा}
{वारयामास कैकेयी तदा राज्ञः प्रिया वधूः} %॥५४॥

\twolineshloka
{राजंस्त्वया वरो दत्तः पूर्वमेव यतः प्रभो}
{राजानं मत्सुतं तस्माद्भरतं कर्तुमर्हसि} %॥५५॥

\twolineshloka
{इति तस्या वचः श्रुत्वा राज्ये तमभिषिच्य सः}
{प्रेषयामास तं रामं वनं प्रति सलक्ष्मणम्} %॥५६॥

\twolineshloka
{वनं गत्वा निवसतो भार्यां दृष्ट्वाऽथ राक्षसः}
{रावणो नाम पौलस्त्यो नीत्वा लङ्कां पुनर्ययौ} %॥५७॥

\twolineshloka
{अदृष्ट्वा तां ततः सीतां दुःखितौ रामलक्ष्मणौ}
{सख्यं वानरराजेन गत्वा दाशरथी द्विजाः} %॥५८॥

\twolineshloka
{सुग्रीवस्य सखा वीरो हनुमान्नाम वानरः}
{गत्वाऽथ रावणपुरीमपश्यज्जनकात्मजाम्} %॥५९॥

\twolineshloka
{अश्रुपूर्णेक्षणां सीतामिन्दीवरनिभाननाम्}
{विश्वासार्थं ददौ तस्यै रामस्यैवाङ्गुलीयकम्} %॥६०॥

\twolineshloka
{दृष्ट्वाऽङ्गुलीयकं सीता प्रहृष्टा च तदाऽभवत्}
{समाश्वास्य ततः सीतां प्रययौ राघवान्तिकम्} %॥६१॥

\twolineshloka
{रामस्तमागतं दृष्ट्वा प्रहर्षोत्फुल्ललोचनः}
{श्रुत्वा तद्वचनाद्वृत्तं युद्धाय कृतनिश्चयः} %॥६२॥

\twolineshloka
{सेतुं कृत्वाऽथ रक्षोभिर्युद्धं कृत्वा महामनाः}
{निहत्य रावणं रामो भ्रातृभिः सह सुव्रतः} %॥६३॥

\twolineshloka
{आनयामास तां सीतामशोकवनमध्यगाम्}
{प्रतिष्ठाप्य महादेवं सेतुमध्येऽथ राघवः} %॥६४॥

\twolineshloka
{लब्धवान्परमां भक्तिं शिवे शिवपराक्रमः}
{रामेश्वर इति ख्यातो महादेवः पिनाकधृक्} %॥६५॥

\twolineshloka
{तस्य दर्शनमात्रेण ब्रह्महत्यां व्यपोहति}
{अभिषिक्तस्ततो राज्ये रामो राजीवलोचनः} %॥६६॥

\twolineshloka
{पालयन्पृथिवीं सर्वां धर्मेण मुनिपुङ्गवाः}
{अयजद्देवदेवेशमश्वमेधेन शङ्करम्} %॥६७॥

\twolineshloka
{तस्य प्रसादात्स्वपदं प्राप्तवानथ राघवः}
{एवं सङ्क्षेपतः प्रोक्तं रामस्य चरितं मया} %॥६८॥

\twolineshloka
{इदं विस्तरतो विप्राः प्रोक्तं वाल्मीकिना पुनः}
{कुशश्चैको लवश्वान्यः पुत्रौ रामस्य सुव्रतौ} %॥६९॥

\threelineshloka
{सत्यसन्धौ महावीर्यौ महादेवपरायणौ}
{अतिथिश्च कुशाज्जज्ञे निषधस्तत्सुतोऽभवत्}
{नलस्तस्याभवत्पुत्रो नभस्तस्याभवत्सुतः} %॥७०॥

\twolineshloka
{ततश्चन्द्रावलोकश्च तारापीडस्ततोऽभवत्}
{ततश्चन्द्रगिरिर्नाम भानुजित्तत्सुतोऽभवत्} %॥७१॥

\twolineshloka
{एते सर्वे नृपाः प्रोक्ता इक्ष्वाकुकुलसम्भवाः}
{धर्मात्मानो महासत्त्वाः कीर्तिमन्तो दृढव्रताः} %॥७२॥

\twolineshloka
{इमं यः पठते नित्यमिक्ष्वाकोर्वंशमुत्तमम्}
{सर्वपापविनिर्मुक्तः सूर्यलोके महीयते} %॥७३॥

॥इति श्रीब्रह्मपुराणोपपुराणे श्रीसौरे सुतशौनकसंवादे प्रह्लादराज्यारोहणादीक्ष्वाकुकुलसम्भवनृपमालिकान्तकथनं नाम त्रिंशोऽध्यायः॥३०॥

\closesection
    
    \part{रामायणान्तर्गताः कथाः}
    \chapt{वेदवती-चरित्रम्}

\src{देवी-भागवतम्}{नवमः स्कन्धः}{अध्यायाः १६}{श्लोकाः १--६४}
% \vakta{व्यासः}
% \shrota{जनमेजयः}
% \tags{concise, complete}
\notes{This passage reveals that the real Sita was actually replaced by a shadow/illusory Sita before her abduction - the Fire God (Agni) had taken the real Sita for safekeeping and given Rama a magical duplicate, which was the one Ravana actually kidnapped, and during Sita's trial by fire after Ravana's defeat, Agni returned the real Sita while the shadow Sita was sent to practice austerities and later became Draupadi (wife of the five Pandavas) in her next incarnation.}
\textlink{https://sa.wikisource.org/wiki/देवीभागवतपुराणम्/स्कन्धः_०९/अध्यायः_१६}
\translink{}

\storymeta

\sect{महालक्षम्या वेदवतीरूपेण राजगृहे जन्मवर्णनम्}

\uvacha{श्रीनारायण उवाच}


\twolineshloka
{लक्ष्मीं तौ च समाराध्य चोग्रेण तपसा मुने}
{वरमिष्टं च प्रत्येकं सम्प्रापतुरभीप्सितम्} %॥ १ ॥

\twolineshloka
{महालक्ष्मीवरेणैव तौ पृथ्वीशौ बभूवतुः}
{पुण्यवन्तौ पुत्रवन्तौ धर्मध्वजकुशध्वजौ} %॥ २ ॥

\twolineshloka
{कुशध्वजस्य पत्‍नी च देवी मालावती सती}
{सा सुषाव च कालेन कमलांशां सुतां सतीम्} %॥ ३ ॥

\twolineshloka
{सा च भूयिष्ठकालेन ज्ञानयुक्ता बभूव ह}
{कृत्वा वेदध्वनिं स्पष्टमुत्तस्थौ सूतिकागृहात्} %॥ ४ ॥

\twolineshloka
{वेदध्वनिं सा चकार जातमात्रेण कन्यका}
{तस्मात्तां च वेदवतीं प्रवदन्ति मनीषिणः} %॥ ५ ॥

\twolineshloka
{जातमात्रेण सुस्नाता जगाम तपसे वनम्}
{सर्वैर्निषिद्धा यत्‍नेन नारायणपरायणा} %॥ ६ ॥

\twolineshloka
{एकमन्वन्तरं चैव पुष्करे च तपस्विनी}
{अत्युग्रां च तपस्यां च लीलया हि चकार सा} %॥ ७ ॥

\twolineshloka
{तथापि पुष्टा न क्लिष्टा नवयौवनसंयुता}
{सुश्राव सा च सहसा सुवाचमशरीरिणीम्} %॥ ८ ॥

\twolineshloka
{जन्मान्तरे च ते भर्ता भविष्यति हरिः स्वयम्}
{ब्रह्मादिभिर्दुराराध्यं पतिं लप्स्यसि सुन्दरि} %॥ ९ ॥

\twolineshloka
{इति श्रुत्वा च सा हृष्टा चकार ह पुनस्तपः}
{अतीव निर्जनस्थाने पर्वते गन्धमादने} %॥ १० ॥

\twolineshloka
{तत्रैव सुचिरं तप्त्वा विश्वस्य समुवास सा}
{ददर्श पुरतस्तत्र रावणं दुर्निवारणम्} %॥ ११ ॥

\twolineshloka
{दृष्ट्वा सातिथिभक्त्या च पाद्यं तस्मै ददौ किल}
{सुस्वादुभूतं च फलं जलं चापि सुशीतलम्} %॥ १२ ॥

\twolineshloka
{तच्च भुक्त्वा स पापिष्ठश्चोवास तत्समीपतः}
{चकार प्रश्नमिति तां का त्वं कल्याणि वर्तसे} %॥ १३ ॥

\twolineshloka
{तां दृष्ट्वा स वरारोहां पीनश्रोणिपयोधराम्}
{शरत्पद्मोत्सवास्यां च सस्मितां सुदतीं सतीम्} %॥ १४ ॥

\twolineshloka
{मूर्च्छामवाप कृपणः कामबाणप्रपीडितः}
{स करेण समाकृष्य शृङ्‌गारं कर्तुमुद्यतः} %॥ १५ ॥

\twolineshloka
{सती चुकोप दृष्ट्वा तं स्तम्भितं च चकार ह}
{स जडो हस्तपादैश्च किञ्चिद्वक्तुं न च क्षमः} %॥ १६ ॥

\twolineshloka
{तुष्टाव मनसा देवीं प्रययौ पद्मलोचनाम्}
{सा तुष्टा तस्य स्तवनं सुकृतं च चकार ह} %॥ १७ ॥

\twolineshloka
{सा शशाप मदर्थे त्वं विनंक्ष्यसि सबान्धवः}
{स्पृष्टाहं च त्वया कामाद्‌ बलं चाप्यवलोकय} %॥ १८ ॥

\twolineshloka
{इत्युक्त्वा सा च योगेन देहत्यागं चकार ह}
{गङ्‌गायां तां च संन्यस्य स्वगृहं रावणो ययौ} %॥ १९ ॥

\twolineshloka
{अहो किमद्‍भुतं दृष्टं किं कृतं वानयाधुना}
{इति सञ्चिन्त्य सञ्चिन्त्य विललाप पुनः पुनः} %॥ २० ॥

\twolineshloka
{सा च कालान्तरे साध्वी बभूव जनकात्मजा}
{सीतादेवीति विख्याता यदर्थे रावणो हतः} %॥ २१ ॥

\twolineshloka
{महातपस्विनी सा च तपसा पूर्वजन्मतः}
{लेभे रामं च भर्तारं परिपूर्णतमं हरिम्} %॥ २२ ॥

\twolineshloka
{सम्प्राप तपसाऽऽराध्य दुराराध्यं जगत्पतिम्}
{सा रमा सुचिरं रेमे रामेण सह सुन्दरी} %॥ २३ ॥

\twolineshloka
{जातिस्मरा न स्मरति तपसश्च क्लमं पुरा}
{सुखेन तज्जहौ सर्वं दुःखं चापि सुखं फले} %॥ २४ ॥

\twolineshloka
{नानाप्रकारविभवं चकार सुचिरं सती}
{सम्प्राप्य सुकुमारं तमतीव नवयौवना} %॥ २५ ॥

\twolineshloka
{गुणिनं रसिकं शान्तं कान्तं देवमनुत्तमम्}
{स्त्रीणां मनोज्ञं रुचिरं तथा लेभे यथेप्सितम्} %॥ २६ ॥

\twolineshloka
{पितुः सत्यपालनार्थं सत्यसन्धो रघूद्वहः}
{जगाम काननं पश्चात्कालेन च बलीयसा} %॥ २७ ॥

\twolineshloka
{तस्थौ समुद्रनिकटे सीतया लक्ष्मणेन च}
{ददर्श तत्र वह्निं च विप्ररूपधरं हरिः} %॥ २८ ॥

\twolineshloka
{रामं च दुःखितं दृष्ट्वा स च दुःखी बभूव ह}
{उवाच किञ्चित्सत्येष्टं सत्यं सत्यपरायणः} %॥ २९ ॥
\uvacha{द्विज उवाच}


\twolineshloka
{भगवच्छ्रूयतां राम कालोऽयं यदुपस्थितः}
{सीताहरणकालोऽयं तवैव समुपस्थितः} %॥ ३० ॥

\twolineshloka
{दैवं च दुर्निवार्यं च न च दैवात्परो बली}
{जगत्प्रसूं मयि न्यस्य छायां रक्षान्तिकेऽधुना} %॥ ३१ ॥

\twolineshloka
{दास्यामि सीतां तुभ्यं च परीक्षासमये पुनः}
{देवैः प्रस्थापितोऽहं च न च विप्रो हुताशनः} %॥ ३२ ॥

\twolineshloka
{रामस्तद्वचनं श्रुत्वा न प्रकाश्य च लक्ष्मणम्}
{स्वीकारं वचसश्चक्रे हृदयेन विदूयता} %॥ ३३ ॥

\twolineshloka
{वह्निर्योगेन सीताया मायासीतां चकार ह}
{तत्तुल्यगुणसर्वाङ्‌गां ददौ रामाय नारद} %॥ ३४ ॥

\twolineshloka
{सीतां गृहीत्वा स ययौ गोप्यं वक्तुं निषिध्य च}
{लक्ष्मणो नैव बुबुधे गोप्यमन्यस्य का कथा} %॥ ३५ ॥

\twolineshloka
{एतस्मिन्नन्तरे रामो ददर्श कानकं मृगम्}
{सीता तं प्रेरयामास तदर्थे यत्‍नपूर्वकम्} %॥ ३६ ॥

\twolineshloka
{संन्यस्य लक्ष्मणं रामो जानक्या रक्षणे वने}
{स्वयं जगाम तूर्णं तं विव्याध सायकेन च} %॥ ३७ ॥

\twolineshloka
{लक्ष्मणेति च शब्दं स कृत्वा च मायया मृगः}
{प्राणांस्तत्याज सहसा पुरो दृष्ट्वा हरिं स्मरन्} %॥ ३८ ॥

\twolineshloka
{मृगदेहं परित्यज्य दिव्यरूपं विधाय च}
{रत्‍ननिर्माणयानेन वैकुण्ठं स जगाम ह} %॥ ३९ ॥

\twolineshloka
{वैकुण्ठलोकद्वार्यासीत्किङ्‌करो द्वारपालयोः}
{पुनर्जगाम तद्द्वारमादेशाद्‌ द्वारपालयोः} %॥ ४० ॥

\twolineshloka
{अथ शब्दं च सा श्रुत्वा लक्ष्मणेति च विक्लवम्}
{तं हि सा प्रेरयामास लक्ष्मणं रामसन्निधौ} %॥ ४१ ॥

\twolineshloka
{गते च लक्ष्मणे रामं रावणो दुर्निवारणः}
{सीतां गृहीत्वा प्रययौ लङ्‌कामेव स्वलीलया} %॥ ४२ ॥

\twolineshloka
{विषसाद च रामश्च वने दृष्ट्वा च लक्ष्मणम्}
{तूर्णं च स्वाश्रमं गत्वा सीतां नैव ददर्श सः} %॥ ४३ ॥

\twolineshloka
{मूर्च्छां सम्प्राप सुचिरं विललाप भृशं पुनः}
{पुनः पुनश्च बभ्राम तदन्वेषणपूर्वकम्} %॥ ४४ ॥

\twolineshloka
{कालेन प्राप्य तद्वार्तां गोदावरीनदीतटे}
{सहायान्वानरात्कृत्वा बबन्ध सागरं हरिः} %॥ ४५ ॥

\twolineshloka
{लङ्‌कां गत्वा रघुश्रेष्ठो जघान सायकेन च}
{कालेन प्राप्य तं हत्वा रावणं बान्धवैः सह} %॥ ४६ ॥

\twolineshloka
{तां च वह्निपरीक्षां च कारयामास सत्वरम्}
{हुताशस्तत्र काले तु वास्तवीं जानकीं ददौ} %॥ ४७ ॥

\twolineshloka
{उवाच छाया वह्निं च रामं च विनयान्विता}
{करिष्यामीति किमहं तदुपायं वदस्व मे} %॥ ४८ ॥

\uvacha{श्रीरामाग्नी ऊचतुः}

\twolineshloka
{त्वं गच्छ तपसे देवि पुष्करं च सुपुण्यदम्}
{कृत्वा तपस्या तत्रैव स्वर्गलक्ष्मीर्भविष्यसि} %॥ ४९ ॥

\twolineshloka
{सा च तद्वचनं श्रुत्वा प्रतप्य पुष्करे तपः}
{दिव्यं त्रिलक्षवर्षं च स्वर्गलक्ष्मीर्बभूव ह} %॥ ५० ॥

\twolineshloka
{सा च कालेन तपसा यज्ञकुण्डसमुद्‍भवा}
{कामिनी पाण्डवानां च द्रौपदी द्रुपदात्मजा} %॥ ५१ ॥

\twolineshloka
{कृते युगे वेदवती कुशध्वजसुता शुभा}
{त्रेतायां रामपत्‍नी च सीतेति जनकात्मजा} %॥ ५२ ॥

\twolineshloka
{तच्छाया द्रौपदी देवी द्वापरे द्रुपदात्मजा}
{त्रिहायणी च सा प्रोक्ता विद्यमाना युगत्रये} %॥ ५३ ॥
\uvacha{नारद उवाच}


\twolineshloka
{प्रियाः पञ्च कथं तस्या बभूवुर्मुनिपुङ्‌गव}
{इति मच्चित्तसंदेहं भञ्ज संदेहभञ्जन} %॥ ५४ ॥

\uvacha{श्रीनारायण उवाच}


\twolineshloka
{लङ्‌कायां वास्तवी सीता रामं सम्प्राप नारद}
{रूपयौवनसम्पन्ना छाया च बहुचिन्तया} %॥ ५५ ॥

\twolineshloka
{रामाग्न्योराज्ञया तप्तुमुपास्ते शङ्‌करं परम्}
{कामातुरा पतिव्यग्रा प्रार्थयन्ती पुनः पुनः} %॥ ५६ ॥

\twolineshloka
{पतिं देहि पतिं देहि पतिं देहि त्रिलोचन}
{पतिं देहि पतिं देहि पञ्चवारं चकार सा} %॥ ५७ ॥

\twolineshloka
{शिवस्तत्प्रार्थनां श्रुत्वा प्रहस्य रसिकेश्वरः}
{प्रिये तव प्रियाः पञ्च भविष्यन्ति वरं ददौ} %॥ ५८ ॥

\twolineshloka
{तेन सा पाण्डवानां च बभूव कामिनी प्रिया}
{इति ते कथितं सर्वं प्रस्तावं वास्तवं शृणु} %॥ ५९ ॥

\twolineshloka
{अथ सम्प्राप्य लङ्‌कायां सीतां रामो मनोहराम्}
{विभीषणाय तां लङ्‌कां दत्त्वायोध्यां ययौ पुनः} %॥ ६० ॥

\twolineshloka
{एकादशसहस्राब्दं कृत्वा राज्यं च भारते}
{जगाम सर्वैर्लोकैश्च सार्धं वैकुण्ठमेव च} %॥ ६१ ॥

\twolineshloka
{कमलांशा वेदवती कमलायां विवेश सा}
{कथितं पुण्यमाख्यानं पुण्यदं पापनाशनम्} %॥ ६२ ॥

\twolineshloka
{सततं मूर्तिमन्तश्च वेदाश्चत्वार एव च}
{सन्ति यस्याश्च जिह्वाग्रे सा च वेदवती श्रुता} %॥ ६३ ॥

\onelineshloka
{धर्मध्वजसुताख्यानं निबोध कथयामि ते}% ॥ ६४ ॥

॥इति श्रीमद्देवीभागवते महापुराणेऽष्टादशसाहस्र्यां संहितायां नवमस्कन्धे महालक्षम्या वेदवतीरूपेण राजगृहे जन्मवर्णनं नाम षोडशोऽध्यायः॥

\closesection
    \chapt{ब्रह्म-पुराणम्}

\sect{रामतीर्थवर्णनम्}

\src{ब्रह्म-पुराणम्}{}{अध्यायः १२३}{}
% \tags{concise, complete}
\notes{This chapter describes the significance of Rama Tirtha, a sacred place associated with that helped Dasharatha expiate his sins.}
\textlink{https://sa.wikisource.org/wiki/ब्रह्मपुराणम्/अध्यायः_१२३}
\translink{}

\storymeta


\twolineshloka
{रामतीर्थमिति ख्यातं भ्रूणहत्याविनाशनम्}
{तस्य श्रवणमात्रेण सर्वपापैः प्रमुच्यते} %॥१॥

\twolineshloka
{इक्ष्वाकुवंशप्रभवः क्षत्रियो लोकविश्रुतः}
{बलवान्मतिमाञ्शूरो यथा शक्रः पुरन्दरः} %॥२॥

\twolineshloka
{पितृपैतामहं राज्यं कुर्वन्नास्ते यथा बलिः}
{तस्य तिस्रो महिष्यः स्यू राज्ञो दशरथस्य हि} %॥३॥

\twolineshloka
{कौशल्या च सुमित्रा च कैकेयी च महामते}
{एताः कुलीनाः सुभगा रूपलक्षणसंयुताः} %॥४॥

\twolineshloka
{तस्मिन् राजनि राज्ये तु स्थितेऽयोध्यापतौ मुने}
{वसिष्ठे ब्रह्मविच्छ्रेष्ठे पुरोधसि विशेषतः} %॥५॥

\twolineshloka
{न च व्याधिर्न दुर्भिक्षं न चावृष्टिर्न चाधयः}
{ब्रह्मक्षत्रविशां नित्यं शूद्राणां च विशेषतः} %॥६॥

\twolineshloka
{आश्रमाणां तु सर्वेषामानन्दोऽभूत्पृथक्पृथक्}
{तस्मिञ्शासति राजेन्द्र इक्ष्वाकूणां कुलोद्वहे} %॥७॥

\twolineshloka
{देवानां दानवानां तु राज्यार्थे विग्रहोऽभवत्}
{क्वापि तत्र जयं प्रापुर्देवाः क्वापि तथेतरे} %॥८॥

\twolineshloka
{एवं प्रवर्तमाने तु त्रैलोक्यमतिपीडितम्}
{अभून्नारद तत्राहमवदं दैत्यदानवान्} %॥९॥

\twolineshloka
{देवांश्चापि विशेषेण न कृतं तैर्मदीरितम्}
{पुनश्च सङ्गरस्तेषां बभूव सुमहान्मिथः} %॥१०॥

\twolineshloka
{विष्णुं गत्वा सुराः प्रोचुस्तथेशानं जगन्मयम्}
{तावूचतुरुभौ देवानसुरान् दैत्यदानवान्} %॥११॥

\twolineshloka
{तपसा बलिनो यान्तु पुनः कुर्वन्तु सङ्गरम्}
{तथेत्याहुर्ययुः सर्वे तपसे नियतव्रताः} %॥१२॥

\twolineshloka
{ययुस्तु राक्षसान् देवाः पुनस्ते मत्सरान्विताः}
{देवानां दानवानां च सङ्गरोऽभूत्सुदारुणः} %॥१३॥

\twolineshloka
{न तत्र देवा जेतारो नैव दैत्याश्च दानवाः}
{संयुगे वर्तमाने तु वागुवाचाशरीरिणी} %॥१४॥

\uvacha{आकाशवागुवाच}

\onelineshloka
{येषां दशरथो राजा ते जेतारो न चेतरे} %}%॥* १५॥

\uvacha{ब्रह्मोवाच}

\twolineshloka
{इति श्रुत्वा जयायोभौ जग्मतुर्देवदानवौ}
{तत्र वायुस्त्वरन् प्राप्तो राजानमवदत्तदा} %॥१६॥

\uvacha{वायुरुवाच}


\twolineshloka
{आगन्तव्यं त्वया राजन् देवदानवसङ्गरे}
{यत्र राजा दशरथो जयस्तत्रेति विश्रुतम्} %॥१७॥

\onelineshloka
{तस्मात्त्वं देवपक्षे स्या भवेयुर्जयिनः सुराः}%॥* १८॥

\uvacha{ब्रह्मोवाच}


\twolineshloka
{तद्वायुवचनं श्रुत्वा राजा दशरथो नृपः}
{आगम्यते मया सत्यं गच्छ वायो यथासुखम्} %॥१९॥

\twolineshloka
{गते वायौ तदा दैत्या आजग्मुर्भूपतिं प्रति}
{तेऽप्यूचुर्भगवन्नस्मत्साहाय्यं कर्तुमर्हसि} %॥२०॥

\twolineshloka
{राजन् दशरथ श्रीमन् विजयस्त्वयि संस्थितः}
{तस्मात्त्वं वै दैत्यपतेः साहाय्यं कर्तुमर्हसि} %॥२१॥

\twolineshloka
{ततः प्रोवाच नृपतिर्वायुना प्रार्थितः पुरा}
{प्रतिज्ञातं मया तच्च यान्तु दैत्याश्च दानवाः} %॥२२॥

\twolineshloka
{स तु राजा तथा चक्रे गत्वा चैव त्रिविष्टपम्}
{युद्धं चक्रे तथा दैत्यैर्दानवैः सह राक्षसैः} %॥२३॥

\twolineshloka
{पश्यत्सु देवसङ्घेषु नमुचेर्भ्रातरस्तदा}
{विविधुर्निशितैर्बाणैरथाक्षं नृपतेस्तथा} %॥२४॥

\twolineshloka
{भिन्नाक्षं तं रथं राजा न जानाति स सम्भ्रमात्}
{राजान्तिके स्थिता सुभ्रूः कैकेय्याज्ञायि नारद} %॥२५॥

\twolineshloka
{न ज्ञापितं तया राज्ञे स्वयमालोक्य सुव्रता}
{भग्नमक्षं समालक्ष्य चक्रे हस्तं तदा स्वकम्} %॥२६॥

\twolineshloka
{अक्षवन्मुनिशार्दूल तदेतन्महदद्भुतम्}
{रथेन रथिनां श्रेष्ठस्तया दत्तकरेण च} %॥२७॥

\twolineshloka
{जितवान् दैत्यदनुजान् देवैः प्राप्य वरान् बहून्}
{ततो देवैरनुज्ञातस्त्वयोध्यां पुनरभ्यगात्} %॥२८॥

\twolineshloka
{स तु मध्ये महाराजो मार्गे वीक्ष्य तदा प्रियाम्}
{कैकेय्याः कर्म तद्दृष्ट्वा विस्मयं परमं गतः} %॥२९॥

\twolineshloka
{ततस्तस्यै वरान् प्रादात्त्रींस्तु नारद सा अपि}
{अनुमान्य नृपप्रोक्तं कैकेयी वाक्यमब्रवीत्} %॥३०॥

\uvacha{कैकेय्युवाच}


\onelineshloka
{त्वयि तिष्ठन्तु राजेन्द्र त्वया दत्ता वरा अमी}%॥* ३१॥

\uvacha{ब्रह्मोवाच}


\twolineshloka
{विभूषणानि राजेन्द्रो दत्त्वा स प्रियया सह}
{रथेन विजयी राजा ययौ स्वनगरं सुखी} %॥३२॥

\twolineshloka
{योषितां किमदेयं हि प्रियाणामुचितागमे}
{स कदाचिद्दशरथो मृगयाशीलिभिर्वृतः} %॥३३॥

\twolineshloka
{अटन्नरण्ये शर्वर्यां वारिबन्धमथाकरोत्}
{सप्तव्यसनहीनेन भवितव्यं तु भूभुजा} %॥३४॥

\twolineshloka
{इति जानन्नपि च तच्चकार तु विधेर्वशात्}
{गर्तं प्रविश्य पानार्थमागतान्निशितैः शरैः} %॥३५॥

\twolineshloka
{मृगान् हन्ति महाबाहुः शृणु कालविपर्ययम्}
{गर्तं प्रविष्टे नृपतौ तस्मिन्नेव नगोत्तमे} %॥३६॥

\twolineshloka
{वृद्धो वैश्रवणो नाम न शृणोति न पश्यति}
{तस्य भार्या तथाभूता तावब्रूतां तदा सुतम्} %॥३७॥

\uvacha{मातापितरावूचतुः॒}


\twolineshloka
{आवां तृषार्तौ रात्रिश्च कृष्णा चापि प्रवर्तते}
{वृद्धानां जीवितं कृत्स्नं बालस्त्वमसि पुत्रक} %॥३८॥

\twolineshloka
{अन्धानां बधिराणां च वृद्धानां धिक्च जीवितम्}
{जराजर्जरदेहानां धिग्धिक्पुत्रक जीवितम्} %॥३९॥

\twolineshloka
{तावत्पुम्भिर्जीवितव्यं यावल्लक्ष्मीर्दृढं वपुः}
{यावदाज्ञाप्रतिहता तीर्थादावन्यथा मृतिः} %॥४०॥

\uvacha{ब्रह्मोवाच}

\twolineshloka
{इत्येतद्वचनं श्रुत्वा वृद्धयोर्गुरुवत्सलः}
{पुत्रः प्रोवाच तद्दुःखं गिरा मधुरया हरन्} %॥४१॥

\uvacha{पुत्र उवाच}

\twolineshloka
{मयि जीवति किं नाम युवयोर्दुःखमीदृशम्}
{न हरत्यात्मजः पित्रोर्यश्चरित्रैर्मनोरुजम्} %॥४२॥

\onelineshloka
{तेन किं तनुजेनेह कुलोद्वेगविधायिना}%॥* ४३॥

\uvacha{ब्रह्मोवाच}


\twolineshloka
{इत्युक्त्वा पितरौ नत्वा तावाश्वास्य महामनाः}
{तरुस्कन्धे समारोप्य वृद्धौ च पितरौ तदा} %॥४४॥

\twolineshloka
{हस्ते गृहीत्वा कलशं जगाम ऋषिपुत्रकः}
{स ऋषिर्न तु राजानं जानाति नृपतिर्द्विजम्} %॥४५॥

\twolineshloka
{उभौ सरभसौ तत्र द्विजो वारि समाविशत्}
{सत्वरं कलशे न्युब्जे वारि गृह्णन्तमाशुगैः} %॥४६॥

\twolineshloka
{द्विजं राजा द्विपं मत्वा विव्याध निशितैः शरैः}
{वनद्विपोऽपि भूपानामवध्यस्तद्विदन्नपि} %॥४७॥

\twolineshloka
{विव्याध तं नृपः कुर्यान्न किं किं विधिवञ्चितः}
{स विद्धो मर्मदेशे तु दुःखितो वाक्यमब्रवीत्} %॥४८॥

\uvacha{द्विज उवाच}


\twolineshloka
{केनेदं दुःखदं कर्म कृतं सद्ब्राह्मणस्य मे}
{मैत्रो ब्राह्मण इत्युक्तो नापराधोऽस्ति कश्चन} %॥४९॥

\uvacha{ब्रह्मोवाच}


\twolineshloka
{तदेतद्वचनं श्रुत्वा मुनेरार्तस्य भूपतिः}
{निश्चेष्टश्च निरुत्साहो शनैस्तं देशमभ्यगात्} %॥५०॥

\twolineshloka
{तं तु दृष्ट्वा द्विजवरं ज्वलन्तमिव तेजसा}
{असावप्यभवत्तत्र सशल्य इव मूर्च्छितः} %॥५१॥

\onelineshloka
{आत्मानमात्मना कृत्वा स्थिरं राजाब्रवीदिदम्}%॥* ५२॥

\uvacha{राजोवाच}


\twolineshloka
{को भवान् द्विजशार्दूल किमर्थमिह चागतः}
{वद पापकृते मह्यं वद मे निष्कृतिं पराम्} %॥५३॥

\twolineshloka
{ब्रह्महा वर्णिभिः किन्तु श्वपचैरपि जातुचित्}
{न स्प्रष्टव्यो महाबुद्धे द्रष्टव्यो न कदाचन} %॥५४॥

\uvacha{ब्रह्मोवाच}


\onelineshloka
{तद्राजवचनं श्रुत्वा मुनिपुत्रोऽब्रवीद्वचः}%॥* ५५॥

\uvacha{मुनिपुत्र उवाच}


\twolineshloka
{उत्क्रमिष्यन्ति मे प्राणा अतो वक्ष्यामि किञ्चन}
{स्वच्छन्दवृत्तिताज्ञाने विद्धि पाकं च कर्मणाम्} %॥५६॥

\twolineshloka
{आत्मार्थं तु न शोचामि वृद्धौ तु पितरौ मम}
{तयोः शुश्रूषकः कः स्यादन्धयोरेकपुत्रयोः} %॥५७॥

\twolineshloka
{विना मया महारण्ये कथं तौ जीवयिष्यतः}
{ममाभाग्यमहो कीदृक्पितृशुश्रूषणे क्षतिः} %॥५८॥

\twolineshloka
{जाता मेऽद्य विना प्राणैर्हा विधे किं कृतं त्वया}
{तथापि गच्छ तत्र त्वं गृहीतकलशस्त्वरन्} %॥५९॥

\onelineshloka
{ताभ्यां देह्युदपानं त्वं यथा तौ न मरिष्यतः}%॥* ६०॥

\uvacha{ब्रह्मोवाच}


\twolineshloka
{इत्येवं ब्रुवतस्तस्य गताः प्राणा महावने}
{विसृज्य सशरं चापमादाय कलशं नृपः} %॥६१॥

\twolineshloka
{तत्रागात्स तु वेगेन यत्र वृद्धौ महावने}
{वृद्धौ चापि तदा रात्रौ तावन्योन्यं समूचतुः} %॥६२॥

\uvacha{वृद्धावूचतुः॒}


\twolineshloka
{उद्विग्नः कुपितो वा स्यादथवा भक्षितः कथम्}
{न प्राप्तश्चावयोर्यष्टिः किं कुर्मः का गतिर्भवेत्} %॥६३॥

\twolineshloka
{न कोपि तादृशः पुत्रो विद्यते सचराचरे}
{यः पित्रोरन्यथा वाक्यं न करोत्यपि निन्दितः} %॥६४॥

\twolineshloka
{वज्रादपि कठोरं वा जीवितं तमपश्यतोः}
{शीघ्रं न यान्ति यत्प्राणास्तदेकायत्तजीवयोः} %॥६५॥

\uvacha{ब्रह्मोवाच}


\twolineshloka
{एवं बहुविधा वाचो वृद्धयोर्वदतोर्वने}
{तदा दशरथो राजा शनैस्तं देशमभ्यगात्} %॥६६॥

\onelineshloka
{पादसञ्चारशब्देन मेनाते सुतमागतम्}%॥* ६७॥

\uvacha{वृद्धावूचतुः॒}


\twolineshloka
{कुतो वत्स चिरात्प्राप्तस्त्वं दृष्टिस्त्वं परायणम्}
{न ब्रूषे किन्तु रुष्टोऽसि वृद्धयोरन्धयोः सुतः} %॥६८॥

\uvacha{ब्रह्मोवाच}


\twolineshloka
{सशल्य इव दुःखार्तः शोचन् दुष्कृतमात्मनः}
{स भीत इव राजेन्द्रस्तावुवाचाथ नारद} %॥६९॥

\twolineshloka
{उदपानं च कुरुतां तच्छ्रुत्वा नृपभाषितम्}
{नायं वक्ता सुतोऽस्माकं को भवांस्तत्पुरा वद} %॥७०॥

\onelineshloka
{पश्चात्पिबावः पानीयं ततो राजाब्रवीच्च तौ}%॥* ७१॥

\uvacha{राजोवाच}


\onelineshloka
{तत्र तिष्ठति वां पुत्रो यत्र वारिसमाश्रयः}%॥* ७२॥

\uvacha{ब्रह्मोवाच}


\twolineshloka
{तच्छ्रुत्वोचतुरार्तौ तौ सत्यं ब्रूहि न चान्यथा}
{आचचक्षे ततो राजा सर्वमेव यथातथम्} %॥७३॥

\twolineshloka
{ततस्तु पतितौ वृद्धौ तत्रावां नय मा स्पृश}
{ब्रह्मघ्नस्पर्शनं पापं न कदाचिद्विनश्यति} %॥७४॥

\twolineshloka
{निन्ये वै श्रवणं वृद्धं सभार्यं नृपसत्तमः}
{यत्रासौ पतितः पुत्रस्तं स्पृष्ट्वा तौ विलेपतुः} %॥७५॥

\uvacha{वृद्धावूचतुः॒}


\twolineshloka
{यथा पुत्रवियोगेन मृत्युर्नौ विहितस्तथा}
{त्वं चापि पाप पुत्रस्य वियोगान्मृत्युमाप्स्यसि} %॥७६॥

\uvacha{ब्रह्मोवाच}


\twolineshloka
{एवं तु जल्पतोर्ब्रह्मन् गताः प्राणास्ततो नृपः}
{अग्निना योजयामास वृद्धौ च ऋषिपुत्रकम्} %॥७७॥

\twolineshloka
{ततो जगाम नगरं दुःखितो नृपतिर्मुने}
{वसिष्ठाय च तत्सर्वं न्यवेदयदशेषतः} %॥७८॥

\twolineshloka
{नृपाणां सूर्यवंश्यानां वसिष्ठो हि परा गतिः}
{वसिष्ठोऽपि द्विजश्रेष्ठैः सम्मन्त्र्याह च निष्कृतिम्} %॥७९॥

\uvacha{वसिष्ठ उवाच}


\twolineshloka
{गालवं वामदेवं च जाबालिमथ कश्यपम्}
{एतानन्यान् समाहूय हयमेधाय यत्नतः} %॥८०॥

\onelineshloka
{यजस्व हयमेधैश्च बहुभिर्बहुदक्षिणैः}%॥* ८१॥

\uvacha{ब्रह्मोवाच}


\twolineshloka
{अकरोद्धयमेधांश्च राजा दशरथो द्विजैः}
{एतस्मिन्नन्तरे तत्र वागुवाचाशरीरिणी} %॥८२॥

\uvacha{आकाशवाण्युवाच}

पूतं शरीरमभवद्राज्ञो दशरथस्य हि।

\twolineshloka
{व्यवहार्यश्च भविता भविष्यन्ति तथा सुताः}
{ज्येष्ठपुत्रप्रसादेन राजापापो भविष्यति} %॥८३॥

\uvacha{ब्रह्मोवाच}


\twolineshloka
{ततो बहुतिथे काले ऋष्यशृङ्गान्मुनीश्वरात्}
{देवानां कार्यसिद्ध्यर्थं सुता आसन् सुरोपमाः} %॥८४॥

\twolineshloka
{कौशल्यायां तथा रामः सुमित्रायां च लक्ष्मणः}
{शत्रुघ्नश्चापि कैकेय्यां भरतो मतिमत्तरः} %॥८५॥

\twolineshloka
{ते सर्वे मतिमन्तश्च प्रिया राज्ञो वशे स्थिताः}
{तं राजानमृषिः प्राप्य विश्वामित्रः प्रजापतिः} %॥८६॥

\twolineshloka
{रामं च लक्ष्मणं चापि अयाचत महामते}
{यज्ञसंरक्षणार्थाय ज्ञाततन्महिमा मुनिः} %॥८७॥

\onelineshloka
{चिरप्राप्तसुतो वृद्धो राजा नैवेत्यभाषत}%॥* ८८॥

\uvacha{राजोवाच}


\twolineshloka
{महता दैवयोगेन कथञ्चिद्वार्धके मुने}
{जातावानन्दसन्दोह दायकौ मम बालकौ} %॥८९॥

\onelineshloka
{सशरीरमिदं राज्यं दास्ये नैव सुताविमौ}%॥* ९०॥

\uvacha{ब्रह्मोवाच}


\onelineshloka
{वसिष्ठेन तदा प्रोक्तो राजा दशरथस्त्विति}%॥* ९१॥

\uvacha{वसिष्ठ उवाच}


\onelineshloka
{रघवः प्रार्थनाभङ्गं न राजन् क्वापि शिक्षिताः}%॥* ९२॥

\uvacha{ब्रह्मोवाच}


\onelineshloka
{रामं च लक्ष्मणं चैव कथञ्चिदवदन्नृपः}%॥* ९३॥

\uvacha{राजोवाच}


\onelineshloka
{विश्वामित्रस्य ब्रह्मर्षेः कुरुतां यज्ञरक्षणम्}%॥* ९४॥

\uvacha{ब्रह्मोवाच}


\twolineshloka
{वदन्निति सुतौ सोष्णं निश्वसन् ग्लपिताधरः}
{पुत्रौ समर्पयामास विश्वामित्रस्य शास्त्रकृत्} %॥९५॥

\twolineshloka
{तथेत्युक्त्वा दशरथं नमस्य च पुनः पुनः}
{जग्मतू रक्षणार्थाय विश्वामित्रेण तौ मुदा} %॥९६॥

\twolineshloka
{ततः प्रहृष्टः स मुनिर्मुदा प्रादात्तदोभयोः}
{माहेश्वरीं महाविद्यां धनुर्विद्यापुरःसराम्} %॥९७॥

\twolineshloka
{शास्त्रीमास्त्रीं लौकिकीं च रथविद्यां गजोद्भवाम्}
{अश्वविद्यां गदाविद्यां मन्त्राह्वानविसर्जने} %॥९८॥

\twolineshloka
{सर्वविद्यामथावाप्य उभौ तौ रामलक्ष्मणौ}
{वनौकसां हितार्थाय जघ्नतुस्ताटकां वने} %॥९९॥

\twolineshloka
{अहल्यां शापनिर्मुक्तां पादस्पर्शाच्च चक्रतुः}
{यज्ञविध्वंसनायाताञ्जघ्नतुस्तत्र राक्षसान्} %॥१००॥

\twolineshloka
{कृतविद्यौ धनुष्पाणी चक्रतुर्यज्ञरक्षणम्}
{ततो महामखे वृत्ते विश्वामित्रो मुनीश्वरः} %॥१०१॥

\twolineshloka
{पुत्राभ्यां सहितो राज्ञो जनकं द्रष्टुमभ्यगात्}
{चित्रामदर्शयत्तत्र राजमध्ये नृपात्मजः} %॥१०२॥

\twolineshloka
{रामः सौमित्रिसहितो धनुर्विद्यां गुरोर्मताम्}
{तत्प्रीतो जनकः प्रादात्सीतां लक्ष्मीमयोनिजाम्} %॥१०३॥

\twolineshloka
{तथैव लक्ष्मणस्यापि भरतस्यानुजस्य च}
{शत्रुघ्नभरतादीनां वसिष्ठादिमते स्थितः} %॥१०४॥

\twolineshloka
{राजा दशरथः श्रीमान् विवाहमकरोन्मुने}
{ततो बहुतिथे काले राज्यं तस्य प्रयच्छति} %॥१०५॥

\twolineshloka
{नृपतौ सर्वलोकानामनुमत्या गुरोरपि}
{मन्थरात्मकदुर्दैव प्रेरिता मत्सराकुला} %॥१०६॥

\twolineshloka
{कैकेयी विघ्नमातस्थे वनप्रव्राजनं तथा}
{भरतस्य च तद्राज्यं राजा नैव च दत्तवान्} %॥१०७॥

\twolineshloka
{पितरं सत्यवाक्यं तं कुर्वन् रामो महावनम्}
{विवेश सीतया सार्धं तथा सौमित्रिणा सह} %॥१०८॥

\twolineshloka
{सतां च मानसं शुद्धं स विवेश स्वकैर्गुणैः}
{तस्मिन् विनिर्गते रामे वनवासाय दीक्षिते} %॥१०९॥

\twolineshloka
{समं लक्ष्मणसीताभ्यां राज्यतृष्णाविवर्जिते}
{तं रामं चापि सौमित्रिं सीतां च गुणशालिनीम्} %॥११०॥

\twolineshloka
{दुःखेन महताविष्टो ब्रह्मशापं च संस्मरन्}
{तदा दशरथो राजा प्राणांस्तत्याज दुःखितः} %॥१११॥

\twolineshloka
{कृतकर्मविपाकेन राजा नीतो यमानुगैः}
{तस्मै राज्ञे महाप्राज्ञ यावत्स्थावरजङ्गमे} %॥११२॥

\twolineshloka
{यमसद्मन्यनेकानि तामिस्रादीनि नारद}
{नरकाण्यथ घोराणि भीषणानि बहूनि च} %॥११३॥

\twolineshloka
{तत्र क्षिप्तस्तदा राजा नरकेषु पृथक्पृथक्}
{पच्यते छिद्यते राजा पिष्यते चूर्ण्यते तथा} %॥११४॥

\twolineshloka
{शोष्यते दश्यते भूयो दह्यते च निमज्ज्यते}
{एवमादिषु घोरेषु नरकेषु स पच्यते} %॥११५॥

\twolineshloka
{रामोऽपि गच्छन्नध्वानं चित्रकूटमथागमत्}
{तत्रैव त्रीणि वर्षाणि व्यतीतानि महामते} %॥११६॥

\twolineshloka
{पुनः स दक्षिणामाशामाक्रामद्दण्डकं वनम्}
{विख्यातं त्रिषु लोकेषु देशानां तद्धि पुण्यदम्} %॥११७॥

\twolineshloka
{प्राविशत्तन्महारण्यं भीषणं दैत्यसेवितम्}
{तद्भयादृषिभिस्त्यक्तं हत्वा दैत्यांस्तु राक्षसान्} %॥११८॥

\twolineshloka
{विचरन् दण्डकारण्ये ऋषिसेव्यमथाकरोत्}
{तत्रेदं वृत्तमाख्यास्ये शृणु नारद यत्नतः} %॥११९॥

\twolineshloka
{तावच्छनैस्त्वगाद्रामो यावद्योजनपञ्चकम्}
{गौतमीं समनुप्राप्तो राजापि नरके स्थितः} %॥१२०॥

\twolineshloka
{यमः स्वकिङ्करानाह रामो दशरथात्मजः}
{गौतमीमभितो याति पितरं तस्य धीमतः} %॥१२१॥

\twolineshloka
{आकर्षन्त्वथ राजानं नरकान्नात्र संशयः}
{उत्तीर्य गौतमीं याति यावद्योजनपञ्चकम्} %॥१२२॥

\twolineshloka
{रामस्तावत्तस्य पिता नरके नैव पच्यताम्}
{यदेतन्मद्वचः पुण्यं न कुर्युर्यदि दूतकाः} %॥१२३॥

\twolineshloka
{ततश्च नरके घोरे यूयं सर्वे निमज्जथ}
{या काप्युक्ता परा शक्तिः शिवस्य समवायिनी} %॥१२४॥

\twolineshloka
{तामेव गौतमीं सन्तो वदन्त्यम्भःस्वरूपिणीम्}
{हरिब्रह्ममहेशानां मान्या वन्द्या च सैव यत्} %॥१२५॥

\twolineshloka
{निस्तीर्यते न केनापि तदतिक्रमजं त्वघम्}
{पापिनोऽप्यात्मजः कश्चिद्यश्च गङ्गामनुस्मरेत्} %॥१२६॥

\twolineshloka
{सोऽनेकदुर्गनिरयान्निर्गतो मुक्ततां व्रजेत्}
{किं पुनस्तादृशः पुत्रो गौतमीनिकटे स्थितः} %॥१२७॥

\twolineshloka
{यस्यासौ नरके पक्तुं न कैरपि हि शक्यते}
{दक्षिणाशापतेर्वाक्यं निशम्य यमकिङ्कराः} %॥१२८॥

\twolineshloka
{नरके पच्यमानं तमयोध्याधिपतिं नृपम्}
{उत्तार्य घोरनरकाद्वचनं चेदमब्रुवन्} %॥१२९॥

\uvacha{यमकिङ्करा ऊचुः॒}


\twolineshloka
{धन्योऽसि नृपशार्दूल यस्य पुत्रः स तादृशः}
{इह चामुत्र विश्रान्तिः सुपुत्रः केन लभ्यते} %॥१३०॥

\uvacha{ब्रह्मोवाच}


\onelineshloka
{स विश्रान्तः शनै राजा किङ्करान् वाक्यमब्रवीत्}%॥* १३१॥

\uvacha{राजोवाच}


\twolineshloka
{नरकेष्वथ घोरेषु पच्यमानः पुनः पुनः}
{कथं त्वाकर्षितः शीघ्रं तन्मे वक्तुमिहार्हथ} %॥१३२॥

\uvacha{ब्रह्मोवाच}


\onelineshloka
{तत्र कश्चिच्छान्तमना राजानमिदमब्रवीत्}%॥* १३३॥

\uvacha{यमदूत उवाच}


\twolineshloka
{वेदशास्त्रपुराणादावेतद्गोप्यं प्रयत्नतः}
{प्रकाश्यते तदपि ते सामर्थ्यं पुत्रतीर्थयोः} %॥१३४॥

\twolineshloka
{रामस्तव सुतः श्रीमान् गौतमीतीरमागतः}
{तस्मात्त्वं नरकाद्घोरादाकृष्टोऽसि नरोत्तम} %॥१३५॥
यदि त्वां तत्र गौतम्यां स्मरेद्रामः सलक्ष्मणः।

\twolineshloka
{स्नानं कृत्वाथ पिण्डादि ते दद्यात्स नृपोत्तम}
{ततस्त्वं सर्वपापेभ्यो मुक्तो यासि त्रिविष्टपम्} %॥१३६॥

\uvacha{राजोवाच}


\twolineshloka
{तत्र गत्वा भवद्वाक्यमाख्यास्ये स्वसुतौ प्रति}
{भवन्त एव शरणमनुज्ञां दातुमर्हथ} %॥१३७॥

\uvacha{ब्रह्मोवाच}


\twolineshloka
{तद्राजवचनं श्रुत्वा कृपया यमकिङ्कराः}
{आज्ञां च प्रददुस्तस्मै राजा प्रागात्सुतौ प्रति} %॥१३८॥

\twolineshloka
{भीषणं यातनादेहमापन्नो निःश्वसन्मुहुः}
{निरीक्ष्य स्वं लज्जमानः कृतं कर्म च संस्मरन्} %॥१३९॥

\twolineshloka
{स्वेच्छया विहरन् गङ्गामाससाद च राघवः}
{गौतम्यास्तटमाश्रित्य रामो लक्ष्मण एव च} %॥१४०॥

\twolineshloka
{सीतया सह वैदेह्या सस्नौ चैव यथाविधि}
{नैव तत्राभवद्भोज्यं भक्ष्यं वा गौतमीतटे} %॥१४१॥

\twolineshloka
{तद्दिने तत्र वसतां गौतमीतीरवासिनाम्}
{तद्दृष्ट्वा दुःखितो भ्राता लक्ष्मणो राममब्रवीत्} %॥१४२॥

\uvacha{लक्ष्मण उवाच}


\twolineshloka
{पुत्रौ दशरथस्यावां तवापि बलमीदृशम्}
{नास्ति भोज्यमथास्माकं गङ्गातीरनिवासिनाम्} %॥१४३॥

\uvacha{राम उवाच}


\twolineshloka
{भ्रातर्यद्विहितं कर्म नैव तच्चान्यथा भवेत्}
{पृथिव्यामन्नपूर्णायां वयमन्नाभिलाषिणः} %॥१४४॥

\twolineshloka
{सौमित्रे नूनमस्माभिर्न ब्राह्मणमुखे हुतम्}
{अवज्ञया महीदेवांस्तर्पयन्त्यर्चयन्ति न} %॥१४५॥
ते ये लक्ष्मण जायन्ते सर्वदैव बुभुक्षिताः।

\twolineshloka
{स्नात्वा देवानथाभ्यर्च्य होतव्यश्च हुताशनः}
{ततः स्वसमये देवो विधास्यत्यशनं तु नौ} %॥१४६॥

\uvacha{ब्रह्मोवाच}


\twolineshloka
{भ्रात्रोः सञ्जल्पतोरेवं पश्यतोः कर्मणो गतिम्}
{शनैर्दशरथो राजा तं देशमुपजग्मिवान्} %॥१४७॥

\twolineshloka
{तं दृष्ट्वा लक्ष्मणः शीघ्रं तिष्ठ तिष्ठेति चाब्रवीत्}
{धनुराकृष्य कोपेन रक्षस्त्वं दानवोऽथवा} %॥१४८॥

\twolineshloka
{आसन्नं च पुनर्दृष्ट्वा याहि याह्यत्र पुण्यभाक्}
{रामो दाशरथी राजा धर्मभाक्पश्य वर्तते} %॥१४९॥

\twolineshloka
{गुरुभक्तः सत्यसन्धो देवब्राह्मणसेवकः}
{त्रैलोक्यरक्षादक्षोऽसौ वर्तते यत्र राघवः} %॥१५०॥

\twolineshloka
{न तत्र त्वादृशामस्ति प्रवेशः पापकर्मणाम्}
{यदि प्रविशसे पाप ततो वधमवाप्स्यसि} %॥१५१॥
तत्पुत्रवचनं श्रुत्वा शनैराहूय वाचया।

\twolineshloka
{उवाचाधोमुखो भूत्वा स्नुषां पुत्रौ कृताञ्जलिः}
{मुहुरन्तर्विनिध्यायन् गतिं दुष्कृतकर्मणः} %॥१५२॥

\uvacha{राजोवाच}

अहं दशरथो राजा पुत्रौ मे शृणुतं वचः।

\twolineshloka
{तिसृभिर्ब्रह्महत्याभिर्वृतोऽहं दुःखमागतः}
{छिन्नं पश्यत मे देहं नरकेषु च पातितम्} %॥१५३॥

\uvacha{ब्रह्मोवाच}


\twolineshloka
{ततः कृताञ्जली रामः सीतया लक्ष्मणेन च}
{भूमौ प्रणेमुस्ते सर्वे वचनं चैतदब्रुवन्} %॥१५४॥

\uvacha{सीतारामलक्ष्मणा ऊचुः॒}


\onelineshloka
{कस्येदं कर्मणस्तात फलं नृपतिसत्तम}%॥* १५५॥

\uvacha{ब्रह्मोवाच}


\onelineshloka
{स च प्राह यथावृत्तं ब्रह्महत्यात्रयं तथा}%॥* १५६॥

\uvacha{राजोवाच}


\onelineshloka
{निष्कृतिर्ब्रह्महन्तॄणां पुत्रौ क्वापि न विद्यते}%॥* १५७॥

\uvacha{ब्रह्मोवाच}


\twolineshloka
{ततो दुःखेन महता वृताः सर्वे भुवं गताः}
{राजानं वनवासं च मातरं पितरं तथा} %॥१५८॥
दुःखागमं कर्मगतिं नरके पातनं तथा।

\twolineshloka
{एवमाद्यथ संस्मृत्य मुमोह नृपतेः सुतः}
{विसञ्ज्ञं नृपतिं दृष्ट्वा सीता वाक्यमथाब्रवीत्} %॥१५९॥

\uvacha{सीतोवाच}


\twolineshloka
{न शोचन्ति महात्मानस्त्वादृशा व्यसनागमे}
{चिन्तयन्ति प्रतीकारं दैव्यमप्यथ मानुषम्} %॥१६०॥

\twolineshloka
{शोचद्भिर्युगसाहस्रं विपत्तिर्नैव तीर्यते}
{व्यामोहमाप्नुवन्तीह न कदाचिद्विचक्षणाः} %॥१६१॥

\twolineshloka
{किमनेनात्र दुःखेन निष्फलेन जनेश्वर}
{देहि हत्यां प्रथमतो या जाता ह्यतिभीषणा} %॥१६२॥

\twolineshloka
{पितृभक्तः पुण्यशीलो वेदवेदाङ्गपारगः}
{अनागा यो हतो विप्रस्तत्पापस्यात्र निष्कृतिम्} %॥१६३॥

\twolineshloka
{आचरामि यथाशास्त्रं मा शोकं कुरुतं युवाम्}
{द्वितीयां लक्ष्मणो हत्यां गृह्णातु त्वपरां भवान्} %॥१६४॥

\uvacha{ब्रह्मोवाच}


\twolineshloka
{एतद्धर्मयुतं वाक्यं सीतया भाषितं दृढम्}
{तथेति चाहतुरुभौ ततो दशरथोऽब्रवीत्} %॥१६५॥

\uvacha{दशरथ उवाच}


\twolineshloka
{त्वं हि ब्रह्मविदः कन्या जनकस्य त्वयोनिजा}
{भार्या रामस्य किं चित्रं यद्युक्तमनुभाषसे} %॥१६६॥

\twolineshloka
{न कोपि भवतां किन्तु श्रमः स्वल्पोऽपि विद्यते}
{गौतम्यां स्नानदानेन पिण्डनिर्वपणेन च} %॥१६७॥

\twolineshloka
{तिसृभिर्ब्रह्महत्याभिर्मुक्तो यामि त्रिविष्टपम्}
{त्वया जनकसम्भूते स्वकुलोचितमीरितम्} %॥१६८॥

\twolineshloka
{प्रापयन्ति परं पारं भवाब्धेः कुलयोषितः}
{गोदावर्याः प्रसादेन किं नामास्त्यत्र दुर्लभम्} %॥१६९॥

\uvacha{ब्रह्मोवाच}


\twolineshloka
{तथेति क्रियमाणे तु पिण्डदानाय शत्रुहा}
{नैवापश्यद्भक्ष्यभोज्यं ततो लक्ष्मणमब्रवीत्} %॥१७०॥

\twolineshloka
{लक्ष्मणः प्राह विनयादिङ्गुद्याश्च फलानि च}
{सन्ति तेषां च पिण्याकमानीतं तत्क्षणादिव} %॥१७१॥

\twolineshloka
{पिण्याकेनाथ गङ्गायां पिण्डं दातुं तथा पितुः}
{मनः कुर्वंस्ततो रामो मन्दोऽभूद्दुःखितस्तदा} %॥१७२॥

\twolineshloka
{दैवी वागभवत्तत्र दुःखं त्यज नृपात्मज}
{राज्यभ्रष्टो वनं प्राप्तः किं वै निष्किञ्चनो भवान्} %॥१७३॥

\twolineshloka
{अशठो धर्मनिरतो न शोचितुमिहार्हसि}
{वित्तशाठ्येन यो धर्मं करोति स तु पातकी} %॥१७४॥

\twolineshloka
{श्रूयते सर्वशास्त्रेषु यद्राम शृणु यत्नतः}
{यदन्नः पुरुषो राजंस्तदन्नास्तस्य देवताः} %॥१७५॥

\twolineshloka
{पिण्डे निपतिते भूमौ नापश्यत्पितरं तदा}
{शवं च पतितं यत्र शवतीर्थमनुत्तमम्} %॥१७६॥

\twolineshloka
{महापातकसङ्घात विघातकृदनुस्मृतिः}
{तत्रागच्छंल्लोकपाला रुद्रादित्यास्तथाश्विनौ} %॥१७७॥

\twolineshloka
{स्वं स्वं विमानमारूढास्तेषां मध्येऽतिदीप्तिमान्}
{विमानवरमारूढः स्तूयमानश्च किन्नरैः} %॥१७८॥

\twolineshloka
{आदित्यसदृशाकारस्तेषां मध्ये बभौ पिता}
{तमदृष्ट्वा स्वपितरं देवान् दृष्ट्वा विमानिनः} %॥१७९॥

\twolineshloka
{कृताञ्जलिपुटो रामः पिता मे क्वेत्यभाषत}
{इति दिव्याभवद्वाणी रामं सम्बोध्य सीतया} %॥१८०॥

\twolineshloka
{तिसृभिर्ब्रह्महत्याभिर्मुक्तो दशरथो नृपः}
{वृतं पश्य सुरैस्तात देवा अप्यूचिरे च तम्} %॥१८१॥

\uvacha{देवा ऊचुः॒}


\twolineshloka
{धन्योऽसि कृतकृत्योऽसि राम स्वर्गं गतः पिता}
{नानानिरयसङ्घातात्पूर्वजानुद्धरेत्तु यः} %॥१८२॥

\twolineshloka
{स धन्योऽलङ्कृतं तेन कृतिना भुवनत्रयम्}
{एनं पश्य महाबाहो मुक्तपापं रविप्रभम्} %॥१८३॥

\twolineshloka
{सर्वसम्पत्तियुक्तोऽपि पापी दग्धद्रुमोपमः}
{निष्किञ्चनोऽपि सुकृती दृश्यते चन्द्रमौलिवत्} %॥१८४॥

\uvacha{ब्रह्मोवाच}


\onelineshloka
{दृष्ट्वाब्रवीत्सुतं राजा आशीर्भिरभिनन्द्य च}%॥* १८५॥

\uvacha{राजोवाच}


\twolineshloka
{कृतकृत्योऽसि भद्रं ते तारितोऽहं त्वयानघ}
{धन्यः स पुत्रो लोकेऽस्मिन् पितॄणां यस्तु तारकः} %॥१८६॥

\uvacha{ब्रह्मोवाच}

\onelineshloka*
{ततः सुरगणाः प्रोचुर्देवानां कार्यसिद्धये}

\twolineshloka
{रामं च पुरुषश्रेष्ठं गच्छ तात यथासुखम्}
{ततस्तद्वचनं श्रुत्वा रामस्तानब्रवीत्सुरान्} %॥१८७॥

\uvacha{राम उवाच}


\onelineshloka
{गुरौ पितरि मे देवाः किं कृत्यमवशिष्यते}%॥* १८८॥

\uvacha{देवा ऊचुः॒}


\twolineshloka
{नदी न गङ्गया तुल्या न त्वया सदृशः सुतः}
{न शिवेन समो देवो न तारेण समो मनुः} %॥१८९॥
त्वया राम गुरूणां च कार्यं सर्वमनुष्ठितम्।

\twolineshloka
{तारिताः पितरो राम त्वया पुत्रेण मानद}
{गच्छन्तु सर्वे स्वस्थानं त्वं च गच्छ यथासुखम्} %॥१९०॥

\uvacha{ब्रह्मोवाच}


\twolineshloka
{तद्देववचनाद्धृष्टः सीतया लक्ष्मणाग्रजः}
{तद्दृष्ट्वा गङ्गामाहात्म्यं विस्मितो वाक्यमब्रवीत्} %॥१९१॥

\uvacha{राम उवाच}


\twolineshloka
{अहो गङ्गाप्रभावोऽयं त्रैलोक्ये नोपमीयते}
{वयं धन्या यतो गङ्गा दृष्टास्माभिस्त्रिपावनी} %॥१९२॥

\uvacha{ब्रह्मोवाच}


\twolineshloka
{हर्षेण महता युक्तो देवं स्थाप्य महेश्वरम्}
{तं षोडशभिरीशानमुपचारैः प्रयत्नतः} %॥१९३॥

\twolineshloka
{सम्पूज्यावरणैर्युक्तं षट्त्रिंशत्कलमीश्वरम्}
{कृताञ्जलिपुटो भूत्वा रामस्तुष्टाव शङ्करम्} %॥१९४॥

\uvacha{राम उवाच}


\fourlineindentedshloka
{नमामि शम्भुं पुरुषं पुराणं}
{नमामि सर्वज्ञमपारभावम्} 
{नमामि रुद्रं प्रभुमक्षयं तं} 
{नमामि शर्वं शिरसा नमामि}% १९५

\fourlineindentedshloka
{नमामि देवं परमव्ययं तम्} 
{उमापतिं लोकगुरुं नमामि}
{नमामि दारिद्र्यविदारणं तं}
{नमामि रोगापहरं नमामि}% १९६

\fourlineindentedshloka
{नमामि कल्याणमचिन्त्यरूपं}
{नमामि विश्वोद्भवबीजरूपम्}
{नमामि विश्वस्थितिकारणं तं}
{नमामि संहारकरं नमामि}% १९७

\fourlineindentedshloka
{नमामि गौरीप्रियमव्ययं तं}
{नमामि नित्यं क्षरमक्षरं तम्}
{नमामि चिद्रूपममेयभावं}
{त्रिलोचनं तं शिरसा नमामि}% १९८

\fourlineindentedshloka
{नमामि कारुण्यकरं भवस्य}
{भयङ्करं वापि सदा नमामि}
{नमामि दातारमभीप्सितानां}
{नमामि सोमेशमुमेशमादौ}% १९९

\fourlineindentedshloka
{नमामि वेदत्रयलोचनं तं}
{नमामि मूर्तित्रयवर्जितं तम्}
{नमामि पुण्यं सदसद्व्यतीतं}
{नमामि तं पापहरं नमामि}% २००

\fourlineindentedshloka
{नमामि विश्वस्य हिते रतं तं}
{नमामि रूपाणि बहूनि धत्ते}
{यो विश्वगोप्ता सदसत्प्रणेता}
{नमामि तं विश्वपतिं नमामि}% २०१

\fourlineindentedshloka
{यज्ञेश्वरं सम्प्रति हव्यकव्यं}
{तथा गतिं लोकसदाशिवो यः}
{आराधितो यश्च ददाति सर्वं}
{नमामि दानप्रियमिष्टदेवम्}% २०२

\fourlineindentedshloka
{नमामि सोमेश्वरमस्वतन्त्रम्}
{उमापतिं तं विजयं नमामि}
{नमामि विघ्नेश्वरनन्दिनाथं}
{पुत्रप्रियं तं शिरसा नमामि}% २०३

\fourlineindentedshloka
{नमामि देवं भवदुःखशोक}
{विनाशनं चन्द्रधरं नमामि}
{नमामि गङ्गाधरमीशमीड्यम्}
{उमाधवं देववरं नमामि}% २०४

\fourlineindentedshloka
{नमाम्यजादीशपुरन्दरादि}
{सुरासुरैरर्चितपादपद्मम्}
{नमामि देवीमुखवादनानाम्}
{ईक्षार्थमक्षित्रितयं य ऐच्छत्}% २०५

\fourlineindentedshloka
{पञ्चामृतैर्गन्धसुधूपदीपैर्}
{विचित्रपुष्पैर्विविधैश्च मन्त्रैः}
{अन्नप्रकारैः सकलोपचारैः}
{सम्पूजितं सोममहं नमामि}% २०६

\uvacha{ब्रह्मोवाच}


\twolineshloka
{ततः स भगवानाह रामं शम्भुः सलक्ष्मणम्}
{वरान् वृणीष्व भद्रं ते रामः प्राह वृषध्वजम्} %॥२०७॥

\uvacha{राम उवाच}


\twolineshloka
{स्तोत्रेणानेन ये भक्त्या तोष्यन्ति त्वां सुरोत्तम}
{तेषां सर्वाणि कार्याणि सिद्धिं यान्तु महेश्वर} %॥२०८॥

\twolineshloka
{येषां च पितरः शम्भो पतिता नरकार्णवे}
{तेषां पिण्डादिदानेन पूता यान्तु त्रिविष्टपम्} %॥२०९॥

\twolineshloka
{जन्मप्रभृति पापानि मनोवाक्कायिकं त्वघम्}
{अत्र तु स्नानमात्रेण तत्सद्यो नाशमाप्नुयात्} %॥२१०॥

\twolineshloka
{अत्र ये भक्तितः शम्भो ददत्यर्थिभ्य अण्वपि}
{सर्वं तदक्षयं शम्भो दातॄणां फलकृद्भवेत्} %॥२११॥

\uvacha{ब्रह्मोवाच}


\twolineshloka
{एवमस्त्विति तं रामं शङ्करो हृषितोऽब्रवीत्}
{गते तस्मिन् सुरश्रेष्ठे रामोऽप्यनुचरैः सह} %॥२१२॥

\twolineshloka
{गौतमी यत्र चोत्पन्ना शनैस्तं देशमभ्यगात्}
{ततः प्रभृति तत्तीर्थं रामतीर्थमुदाहृतम्} %॥२१३॥

\twolineshloka
{दयालोरपतत्तत्र लक्ष्मणस्य कराच्छरः}
{तद्बाणतीर्थमभवत्सर्वापद्विनिवारणम्} %॥२१४॥

\twolineshloka
{यत्र सौमित्रिणा स्नानं शङ्करस्यार्चनं कृतम्}
{तत्तीर्थं लक्ष्मणं जातं तथा सीतासमुद्भवम्} %॥२१५॥

\twolineshloka
{नानाविधाशेषपाप सङ्घनिर्मूलनक्षमम्}
{यदङ्घ्रिसङ्गादभवद्गङ्गा त्रैलोक्यपावनी} %॥२१६॥

\twolineshloka
{स यत्र स्नानमकरोत्तद्वैशिष्ट्यं किमुच्यते}
{तद्रामतीर्थसदृशं तीर्थं क्वापि न विद्यते} %॥२१७॥

॥इति श्रीमहापुराणे आदिब्राह्मे तीर्थमाहात्म्ये रामतीर्थवर्णनं नाम त्रयोविंशत्यधिकशततमोऽध्यायः॥१२३॥

    \sect{सहस्रकुण्डाख्यतीर्थवर्णनम्}

\src{ब्रह्म-पुराणम्}{गौतमीमाहात्म्यम्}{अध्यायः १५४}{}
% \tags{concise, complete}
\notes{This chapter describes the significance of the Sahasrakunda Tirtha, where Lord Rama performed rituals and established a sacred site after defeating Ravana. It also narrates the events leading to the establishment of this Tirtha, including Rama's return to Ayodhya with Sita.}
\textlink{https://sa.wikisource.org/wiki/ब्रह्मपुराणम्/अध्यायः_१५४}
\translink{}

\storymeta

\uvacha{ब्रह्मोवाच}

\twolineshloka
{सहस्रकुण्डमाख्यातं तीर्थं वेदविदो विदुः}
{यस्य स्मरणमात्रेण सुखी सम्पद्यते नरः} %॥१॥

\twolineshloka
{पुरा दाशरथी रामः सेतुं बद्‌ध्वा महार्णवे}
{लङ्कां दग्ध्वा रिपून्हत्वा रावणादीन्रणे शरैः} %॥२॥

\twolineshloka
{वैदेहीं च समासाद्य रामो वचनमब्रवीत्}
{पश्यत्सु लोकपालेषु तस्याऽऽचार्ये पुरः स्थिते} %॥३॥

\twolineshloka
{अग्नौ शुद्धिगतां सीतां रामो लक्ष्मणसन्निधौ}
{एहि वैदेहि शुद्धऽसि अङ्कमारोढुमर्हसि} %॥४॥

\twolineshloka
{नेत्युवाच तदा श्रीमानङ्गदो हनुमांस्तथा}
{अयोध्यायां तु वैदेहि सार्धं यामः सुहृज्जनैः} %॥५॥

\twolineshloka
{तत्र शुद्धिमवाप्याथ पुनर्भातृषु मातृषु}
{लौकिकेष्वपि पश्यत्सु ततः शुद्धा नृपात्मजा} %॥६॥

\twolineshloka
{अयोध्यायां सुपुण्येऽह्नि अङ्कमारोढुमर्हंसि}
{अस्याश्चरित्रविषये सन्देहः कस्य जायते} %॥७॥

\twolineshloka
{लोकापवादस्तदपि निरस्यः स्वजनेषु हि}
{तयोर्वाक्यमनादृत्य लक्ष्मणः सविभीषणः} %॥८॥

\twolineshloka
{रामश्च जाम्बवांश्चैव तामाह्वयन्नृपात्मजाम्}
{स्वस्तीत्युक्ता देवताभी राज्ञोङ्कं चाऽऽरुरोह सा} %॥९॥

\twolineshloka
{मुदतिस्ते ययुः शीघ्रं पुष्पकेण विराजता}
{अयोध्यां नगरीं प्राप्य तथा राज्यं स्वकं तु यत्} %॥१०॥

\twolineshloka
{मुदितास्तेऽभवन्सर्वे सदा रामानुवर्तिनः}
{ततः कतिपयाहेषु अनार्येभ्यो विरूपिकाम्} %॥११॥

\twolineshloka
{वाचं श्रुत्वा स तत्याज गुर्विणीं तामयोनिजाम्}
{मिथ्यापवादमपि हि न सहन्ते कुलोन्नताः} %॥१२॥

\twolineshloka
{वाल्मीकेर्मुनिमुख्यस्य आश्रमस्य समीपतः}
{तत्याज लक्ष्मणः सीतामदुष्टां रुदतीं रुदन्} %॥१३॥

\twolineshloka
{नोल्लङ्घ्याऽऽज्ञा गुरूणामित्यसौ तदकरोद्भिया}
{ततः कतिपयाहेतु व्यतीतेषु नृपात्मजः} %॥१४॥

\twolineshloka
{रामः सौमित्रिणा सार्धं हयमेधाय दीक्षितः}
{तत्रैवाऽऽजग्मतुरुभौ रामपुत्रौ यशस्विनौ} %॥१५॥

\twolineshloka
{लवः कुशश्च विख्यातौ नारदाविव गायकौ}
{रामायणं समग्रं तद्‌गन्धर्वाविव सुस्वरौ} %॥१६॥

\twolineshloka
{रामाय चरितं सर्वं गायमानौ समीयतुः}
{यज्ञवाटं राजसुतौ हेतुभिर्लक्षितौ तदा} %॥१७॥

\twolineshloka
{रामपुत्रावुभौ शूरौ वैदेह्यास्तनयाविति}
{तावानीय ततः पुत्रावभिषच्य यथाक्रमम्} %॥१८॥

\twolineshloka
{अङ्कारूढौ ततः कृत्वा सस्वजे तौ पुनः पुनः}
{संसारदुःखिन्नानामगतीनां शरीरिणाम्} %॥१९॥

\twolineshloka
{पुत्रालिङ्गनमेवात्र परं विश्रान्तिकारणम्}
{मुहुरालिङ्ग्य तौ पुत्रौ मुहुः स्वजति चुम्बति} %॥२०॥

\twolineshloka
{किमप्यन्तर्ध्याति च निःश्वसत्यपि वै मुहुः}
{एतस्मिन्नन्तरे प्राप्ता राक्षसा लङ्कवासिनः} %॥२१॥

\twolineshloka
{सुग्रीवो हनुमांश्चैव अङ्गदो जाम्बवांस्तथा}
{अन्ये च वानराः सर्वे विभीषणपुरः सराः} %॥२२॥

\twolineshloka
{ते चाऽऽगत्य नृपं प्राप्ताः सिंहासनमुपस्थितम्}
{सीतामदृष्ट्वा हनुमानङ्गदः कनकाङ्गदः} %॥२३॥

\twolineshloka
{क्व गताऽयोनिजा माता एको रामोऽत्र दृश्यते}
{रामेण सा परित्यक्ता इत्यूचुर्द्वारपालकाः} %॥२४॥

\twolineshloka
{पश्यत्सु लोकपालेषु आर्ये तत्र प्रवादिनि}
{अग्नौ शुद्धिगतां(ता)सीतां(ता)किन्तु राजा निरङ्कुशः} %॥२५॥

\twolineshloka
{उत्पन्नैर्लौकिकैर्वाक्यै रामस्त्यजति तां प्रियाम्}
{मरिष्याव इति ह्युक्त्वा गौतमीं पुनरीयतुः} %॥२६॥

\twolineshloka
{रामस्तौ पृष्ठतोऽभ्येत्य(?)अयोध्यावासिभिः सह}
{आगत्य गौतमीं तत्राकुर्वंस्त परमं तपः} %॥२७॥

\twolineshloka
{स्मारं स्मारं निश्वसन्तस्तां सीतां लोकमातरम्}
{संसारास्थाविरहिता गौतमीसेवनोत्सुकाः} %॥२८॥

\twolineshloka
{लोकत्रयपतिः साक्षाद्रामोऽनुजसमन्वितः}
{प्राप्तं स्नात्वा च गौतम्यां शिवाराधनतत्परः} %॥२९॥

\twolineshloka
{परितापं हजौ सर्वं सहस्रपरिवारितः}
{यत्र चाऽऽसीत्स वृत्तान्तः सहस्रकुण्डमुच्यते} %॥३०॥

\twolineshloka
{दशापराणि तीर्थानि तत्र सर्वार्थदानि च}
{तत्र स्नानं च दानं च सहस्रफलदायकम्} %॥३१॥

\twolineshloka
{यत्र श्रीगौतमीतीरे वसिष्ठादिमुनीश्वरैः}
{सर्वापत्तारकं होममकारयदघान्तकम्} %॥३२॥

\twolineshloka
{सहस्रसङ्ख्यायुक्तेषु कुण्डेषु वसुधारया}
{सर्वानपेक्षितान्कामानवापासौ महातपाः} %॥३३॥

\twolineshloka
{गौतम्याः सरिदम्बायाः प्रसादाद्राक्षसान्तकः}
{सहस्रकुण्डाभिधं तदभूत्तीर्थं महाफलम्} %॥३४॥

॥इति श्रीमहापुराणे आदिब्राह्मे तीर्थमाहात्म्ये सहस्रकुण्डादिदशतीर्थवर्णनं नाम चतुष्पञ्चाशदधिकशततमोऽध्यायः॥१५४॥

॥गौतमीमाहात्म्ये पञ्चाशीतितमोऽध्यायः॥८५॥
    \chapt{पद्म-पुराणम्}

\sect{मार्कण्डेयाश्रमदर्शनम्}

\src{पद्म-पुराणम्}{सृष्टिखण्डम्}{अध्यायः ३३}{१--१८५}
% \tags{concise, complete}
\notes{This chapter describes the visit of Rama to the Markandeya Ashrama, the story of Markandeya, and the significance of the Pushkara Tirtha. It also includes a most interesting episode of Rama performing Shrāddha, and Sita actually seeing Dasharatha descended in the Brāhmaṇas.}
\textlink{https://sa.wikisource.org/wiki/पद्मपुराणम्/खण्डः_१_(सृष्टिखण्डम्)/अध्यायः_३३}
\translink{https://www.wisdomlib.org/hinduism/book/the-padma-purana/d/doc364155.html}

\storymeta


\uvacha{भीष्म उवाच}

\twolineshloka
{मार्कण्डेयेन वै रामः कथमत्र प्रबोधितः}
{कथं समागमो भूतः कस्मिन्काले कदा मुने} %॥१।

\twolineshloka
{मार्कण्डेयः कस्य सुतः कथं जातो महातपाः}
{नाम्नोऽस्य निगमं ब्रूहि यथाभूतं महामुने} %॥२।
\uvacha{पुलस्त्य उवाच}

\twolineshloka
{अथ ते सम्प्रवक्ष्यामि मार्कण्डेयोद्भवं पुनः}
{पुराकल्पे मुनिः पूर्वं मृकण्डुर्नाम विश्रुतः} %॥३।

\twolineshloka
{भृगोः पुत्रो महाभागः सभार्यस्तप्तवांस्तपः}
{तस्य पुत्रस्तदा जातो वसतस्तु वनान्तरे} %॥४।

\twolineshloka
{सपञ्चवार्षिको भूतो बाल एव गुणाधिकः}
{ज्ञानिना स तदा दृष्टो भ्रमन्बालस्तदाङ्गणे} %॥५।

\twolineshloka
{स्थित्वा स सुचिरं कालं भाव्यर्थं प्रत्यबुध्यत}
{तस्य पित्रा स वै पृष्टः कियदायुः सुतस्य मे} %॥६।

\twolineshloka
{सङ्ख्यायाचक्ष्व वर्षाणि तस्याल्पान्यधिकानि वा}
{मृकण्डुनैवमुक्तस्तु स ज्ञानी वाक्यमब्रवीत्} %॥७।

\twolineshloka
{षण्मासमायुः पुत्रस्य धात्रा सृष्टं मुनीश्वर}
{नैव शोकस्त्वया कार्यः सत्यमेतदुदाहृतम्} %॥८।

\twolineshloka
{स तच्छ्रुत्वा वचो भीष्म ज्ञानिना यदुदाहृतम्}
{अथोपनयनं चक्रे बालकस्य पिता तदा} %॥९।

\twolineshloka
{आह चैनं पितापुत्रमृषींस्त्वमभिवादय}
{एवमुक्तः स वै पित्रा प्रहृष्टश्चाभिवादने} %॥१०।

\twolineshloka
{न वर्णा वर्णतां वेत्ति सर्ववर्णाभिवादनः}
{पञ्चमासास्त्वतिक्रान्ता दिवसाः पञ्चविंशतिः} %॥११।

\twolineshloka
{मार्गेणाथ समायाता ऋषयस्तत्र सप्त वै}
{बालेन तेन ते दृष्टाः सर्वे चाप्यभिवादिताः} %॥१२।

\twolineshloka
{आयुष्मान्भव तैरुक्तः स बालो दण्डमेखली}
{उक्त्वैवं ते पुनर्बालमपश्यन्क्षीणजीवितम्} %॥१३।

\twolineshloka
{दिनानि पञ्च तस्यायुर्ज्ञात्वा भीताश्च ते नृप}
{तं गृहीत्वा बालकं च गतास्ते ब्रह्मणोन्तिकम्} %॥१४।

\twolineshloka
{प्रतिमुच्य च तं राजन्प्रणिपेतुः पितामहम्}
{अयमावेदितस्तैस्तु तेन ब्रह्माभिवादितः} %॥१५।

\twolineshloka
{चिरायुर्ब्रह्मणा बालः प्रोक्तः स ऋषिसन्निधौ}
{ततस्ते मुनयः प्रीताः श्रुत्वा वाक्यं पितामहात्} %॥१६।

\twolineshloka
{पितामह ऋषीन्दृष्ट्वा प्रोवाच विस्मयान्वितः}
{कार्येण येन चायातः कोयं बालो निवेद्यताम्} %॥१७।

\twolineshloka
{ततस्त ऋषयो राजन्सर्वं तस्मै न्यवेदयन्}
{पुत्रो मृकण्डोः क्षीणायुः सायुषं कुरु बालकम्} %॥१८।

\twolineshloka
{अल्पायुषस्त्वस्य मुनिर्बध्वेमां चापि मेखलाम्}
{यज्ञोपवीतं दण्डं च दत्वा चैनमबोधयत्} %॥१९।

\twolineshloka
{यं कञ्चित्पश्यसे बाल भ्रमन्तं भूतले जनम्}
{तस्याभिवादः कर्तव्य एवमाह पिता वचः} %॥२०।

\twolineshloka
{अभिवादनशीलोयं क्षितौ दृष्टः परिभ्रमन्}
{तीर्थयात्राप्रसङ्गेन दैवयोगात्पितामह} %॥२१।

\twolineshloka
{चिरायुर्भव पुत्रेति प्रोक्तोसौ तत्र बालकः}
{कथं वचो भवेत्सत्यमस्माकं भवता सह} %॥२२।

\twolineshloka
{एवमुक्तस्तदा तैस्तु ब्रह्मा लोकपितामहः}
{ऋतवाक्यादियं भूमिः संस्थिता सर्वतोभया} %॥२३।


\uvacha{ब्रह्मोवाच}

\twolineshloka
{मत्समश्चायुषा बालो मार्कण्डेयो भविष्यति}
{कल्पस्यादौ तथाचान्ते मतो मे मुनिसत्तमः} %॥२४।

\twolineshloka
{एवं ते मुनयो बालं ब्रह्मलोके पितामहात्}
{संसाध्य प्रेषयामासुर्भूयोप्येनं धरातलम्} %॥२५।

\twolineshloka
{तीर्थयात्रां गता विप्रा मार्कण्डेयो निजं गृहम्}
{जगाम तेषु यातेषु पितरं स्वमथाब्रवीत्} %॥२६।

\twolineshloka
{ब्रह्मलोकमहं नीतो मुनिभिर्ब्रह्मवादिभिः}
{दीर्घायुश्च कृतश्चास्मि वरान्दत्वा विसर्जितः} %॥२७।

\twolineshloka
{एतदन्यच्च मे दत्तं गतं चिन्ताकरं तव}
{कल्पस्यादौ तथा चान्ते भविष्ये समनन्तरे} %॥२८।

\twolineshloka
{लोककर्तुर्ब्रह्मणोहं प्रसादात्तस्य वै पितः}
{पुष्करं वै गमिष्यामि तपस्तप्तुं समुद्यतः} %॥२९।

\twolineshloka
{तत्राहं देवदेवेशमुपासिष्ये पितामहम्}
{सर्वकामावाप्तिकरं सर्वारातिनिबर्हणम्} %॥३०।

\twolineshloka
{सर्वसौख्यप्रदं देवमिन्द्रादीनां परायणम्}
{ब्रह्माणं तोषयिष्यामि सर्वलोकपितामहम्} %॥३१।

\twolineshloka
{मार्कण्डेयवचः श्रुत्वा मृकण्डुर्मुनिसत्तमः}
{जगाम परमं हर्षं क्षणमेकं समुच्छ्वसन्} %॥३२।

\twolineshloka
{धैर्यं सुमनसा स्थाय इदं वचनमब्रवीत्}
{अद्य मे सफलं जन्म जीवितं च सुजीवितम्} %॥३३।

\twolineshloka
{सर्वस्य जगतां स्रष्टा येन दृष्टः पितामहः}
{त्वया दायादवानस्मि पुत्रेण वंशधारिणा} %॥३४।

\twolineshloka
{त्वं गच्छ पश्य देवेशं पुष्करस्थं पितामहम्}
{दृष्टे तस्मिन्जगन्नाथे न जरामृत्युरेव च} %॥३५।

\twolineshloka
{नृणां भवति सौख्यानि तथैश्वर्यं तपोऽक्षयम्}
{त्रीणि शृङ्गाणि शुभ्राणि त्रीणि प्रस्रवणानि च} %॥३६।

\twolineshloka
{पुष्कराणि तथा त्रीणि नविद्मस्तत्र कारणम्}
{कनीयांसं मध्यमं च तृतीयं ज्येष्ठपुष्करम्} %॥३७।

\twolineshloka
{शृङ्गशब्दाभिधानानि शुभप्रस्रवणानि च}
{ब्रह्माविष्णुस्तथा रुद्रो नित्यं सन्निहितास्त्रयः} %॥३८।

\twolineshloka
{पुष्करेषु महाराजा नातः पुण्यतमं भुवि}
{विरजं विमलं तोयं त्रिषु लोकेषु विश्रुतम्} %॥३९।

\twolineshloka
{ब्रह्मलोकस्य पन्थानं धन्याः पश्यन्ति पुष्करं}
{यस्तु वर्षशतं साग्रमग्निहोत्रमुपासते} %॥४०।

\twolineshloka
{कार्तिकीं वा वसेदेकां पुष्करे सममेव च}
{कर्तुम्मया न शकितं कर्मणा नैव साधितम्} %॥४१।

\twolineshloka
{तदयत्नात्त्वया तात मृत्युस्सर्वहरो जितः}
{तत्र दृष्टस्स देवेशो ब्रह्मा लोकपितामहः} %॥४२।

\twolineshloka
{नान्यो मर्त्यस्त्वया तुल्यो भविता जगतीतले}
{अहं वै तोषितो येन पञ्चवार्षिकजन्मना} %॥४३।

\twolineshloka
{वरेण त्वं मदीयेन उपमां चिरजीविनाम्}
{गमिष्यसि न सन्देहस्तथाशीर्वचनम्मम} %॥४४।

\twolineshloka
{एवं वदन्ति ते सर्वे व्रज लोकान्यथेप्सितान्}
{एवं लब्धप्रसादेन मृकण्डुतनयेन च} %॥४५।

\twolineshloka
{आश्रमः स्थापितस्तेन मार्कण्डाश्रम इत्युत}
{तत्र स्नात्वा शुचिर्भूत्वा वाजपेयफलं लभेत्} %॥४६।

\onelineshloka*
{सर्वपापविशुद्धात्मा चिरायुर्जायते नरः}


\uvacha{पुलस्त्य उवाच}

\onelineshloka
{तथान्यं ते प्रवक्ष्यामि इतिहासं पुरातनम्} %॥४७।

\twolineshloka
{यथा रामेण वै तीर्थं पुष्करं तु विनिर्मितम्}
{चित्रकूटात्पुरा रामो मैथिल्या लक्ष्मणेन च} %॥४८।

\onelineshloka*
{अत्रेराश्रममासाद्य पप्रच्छ मुनिसत्तमम्}


\uvacha{राम उवाच}

\onelineshloka
{कानि पुण्यानि तीर्थानि किं वा क्षेत्रं महामुने} %॥४९।

\twolineshloka
{यत्र गत्वा नरो योगिन्वियोगं सह बन्धुभिः}
{नैव प्राप्नोति भगवन्तन्ममाचक्ष्व सुव्रत} %॥५०।

\twolineshloka
{अनेन वनवासेन राज्ञस्तु मरणेन च}
{भरतस्य वियोगेन परितप्ये ह्यहं त्रिभिः} %॥५१।

\twolineshloka
{तद्वाक्यं राघवेणोक्तं श्रुत्वा विप्रर्षभस्तदा}
{ध्यात्वा च सुचिरं कालमिदं वचनमब्रवीत्} %॥५२।


\uvacha{अत्रिरुवाच}

\twolineshloka
{साधु पृष्टं त्वया वीर रघूणां वंशवर्धन}
{मम पित्रा कृतं तीर्थं पुष्करं नाम विश्रुतम्} %॥५३।

\twolineshloka
{पर्वतौ द्वौ च विख्यातौ मर्यादा यज्ञपर्वतौ}
{कुण्डत्रयं तयोर्मध्ये ज्येष्ठमध्यकनिष्ठकम्} %॥५४।

\twolineshloka
{तेषु गत्वा दशरथं पिण्डदानेन तर्पय}
{तीर्थानां प्रवरं तीर्थं क्षेत्राणामपि चोत्तमम्} %॥५५।

\twolineshloka
{अवियोगा च सुरसा वापी रघुकुलोद्वह}
{तथा सौभाग्यकूपोन्यः सुजलो रघुनन्दन} %॥५६।

\twolineshloka
{तेषु पिण्डप्रदानेन पितरो मोक्षमाप्नुयुः}
{आभूतसम्प्लवं कालमेतदाह पितामहः} %॥५७।

\twolineshloka
{तत्र राघव गच्छस्व भूयोप्यागमनं क्रियाः}
{तथेति चोक्त्वा रामोपि गमनाय मनो दधे} %॥५८।

\twolineshloka
{ऋक्षवन्तमभिक्रम्य नगरं वैदिशं तथा}
{चर्मण्वतीं समुत्तीर्य प्राप्तोसौ यज्ञपर्वतम्} %॥५९।

\twolineshloka
{तमतिक्रम्य वेगेन मध्यमे पुष्करे स्थितः}
{पितॄन्सन्तर्पयामास अद्भिर्देवांश्च सर्वशः} %॥६०।

\twolineshloka
{स्नानावसाने रामेण मार्कण्डो मुनिपुङ्गवः}
{आगच्छन्शिष्यसंयुक्तो दृष्टस्तत्रैव धीमता} %॥६१।

\twolineshloka
{गत्वा वै सम्मुखं तस्य प्रणिपत्य च सादरम्}
{पृष्टोऽवियोगदः कूपः कतमस्यां दिशि प्रभो} %॥६२।

\twolineshloka
{सुतो दशरथस्याहं रामो नाम जनैः स्मृतः}
{सौभाग्यवापीं तां द्रष्टुमहं प्राप्तोत्रिशासनात्} %॥६३।

\twolineshloka
{तत्स्थानं तौ च वै कूपौ भगवान्प्रब्रवीतु मे}
{एवमुक्तश्च रामेण मार्कण्डः प्रत्युवाच ह} %॥६४।
\uvacha{मार्कण्डेय उवाच}

\twolineshloka
{साधु राघव भद्रं ते सुकृतं भवता कृतम्}
{तीर्थयात्राप्रसङ्गेन यत्प्राप्तोसीह साम्प्रतम्} %॥६५।

\twolineshloka
{एह्यागच्छस्व पश्य स्ववापीं तामवियोगदाम्}
{अवियोगश्च सर्वैश्च कूप एवात्र जायते} %॥६६।

\twolineshloka
{आमुष्मिके चैहिके च जीवतोपि मृतस्य वा}
{एतद्वाक्यं मुनीन्द्रस्य श्रुत्वा लक्ष्मणपूर्वजः} %॥६७।

\twolineshloka
{सस्मार रामो राजानं तदा दशरथं नृप}
{भरतं सह शत्रुघ्न्रं भातॄनन्यांश्चनागरान्} %॥६८।

\twolineshloka
{एवं चिन्तयतस्तस्य सन्ध्याकालो व्यजायत}
{उपास्य पश्चिमां सन्ध्यां मुनिभिः सह राघवः} %॥६९।

\twolineshloka
{सुष्वाप तां निशां तत्र भ्रातृभार्यासमन्वितः}
{विभावर्यवसाने तु स्वप्नान्ते रघुनन्दनः} %॥७०।

\twolineshloka
{पित्रा मात्रा तथा चान्यैरयोध्यायां स्थितः किल}
{विवाहमङ्गले वृत्ते बहुभिर्बान्धवैः सह} %॥७१।

\twolineshloka
{समासीनः सभार्योऽसावृषिभिः परिवारितः}
{लक्ष्मणेनाप्येवमेव दृष्टोऽसौ सीतया तथा} %॥७२।

\twolineshloka
{प्रभाते तु मुनीनां तत्सर्वमेव प्रकीर्तितम्}
{ऋषिभिश्च तथेत्युक्तः सत्यमेतद्रघूत्तम} %॥७३।

\twolineshloka
{मृतस्य दर्शने श्राद्धं कार्यमावश्यकं स्मृतम्}
{वृद्धिकामास्तु पितरस्तथा चैवान्नकाङ्क्षिणः} %॥७४।

\twolineshloka
{ददन्ति दर्शनं स्वप्ने भक्तियुक्तस्य राघव}
{अवियोगस्तु ते भ्रात्रा पित्रा च भरतेन च} %॥७५।

\twolineshloka
{चतुर्दशानां वर्षाणां भविता राघव ध्रुवम्}
{कुरु श्राद्धं तथा वीर राज्ञो दशरथस्य च} %॥७६।

\twolineshloka
{अमी च ऋषयः सर्वे तव भक्ताः कृतक्षणाः}
{अहं च जमदग्निश्च भारद्वाजश्च लोमशः} %॥७७।

\twolineshloka
{देवरातः शमीकश्च षडेते वै द्विजोत्तमाः}
{श्राद्धे च ते महाबाहो सम्भारांस्त्वमुपाहर} %॥७८।

\twolineshloka
{मुख्यं चेङ्गुदिपिण्याकं बदरामलकैः सह}
{श्रीफलानि च पक्वानि मूलं चोच्चावचं बहु} %॥७९।

\twolineshloka
{मार्गेण चाथ मांसेन धान्येन विविधेन च}
{तृप्तिं प्रयच्छ विप्राणां श्राद्धदानेन सुव्रत} %॥८०।

\twolineshloka
{पुष्करारण्यमासाद्य नियतो नियताशनः}
{पितॄंस्तर्पयते यस्तु सोश्वमेधमवाप्नुयात्} %॥८१।

\twolineshloka
{स्नानार्थं तु वयं राम गच्छामो ज्येष्ठपुष्करम्}
{इत्युक्त्वा ते गताः सर्वे मुनयो राघवं नृप} %॥८२।

\twolineshloka
{लक्ष्मणं चाब्रवीद्रामो मेध्यमाहर मे मृगम्}
{शुद्धेक्षणं च शशकं कृष्णशाकं तथा मधु} %॥८३।

\twolineshloka
{जम्बीराणि च मुख्यानि मूलानि विविधानि च}
{पक्वानि च कपित्थानि फलान्यन्यानि यानि च} %॥८४।

\twolineshloka
{तान्याहरस्व वै श्राद्धे क्षिप्रमेवास्तु लक्ष्मण}
{तथा तत्कृतवान्सर्वं रामादेशाच्च राघवः} %॥८५।

\twolineshloka
{बदरेङ्गुदिशाकानि मूलानि विविधानि च}
{तत्राहृत्य च रामेण कूटाकारः कृतो महान्} %॥८६।

\twolineshloka
{परिपक्वं च जानक्या सिद्धं रामे निवेदितम्}
{स्नात्वा रामो योगवाप्यां मुनींस्ताननुपालयन्} %॥८७।

\twolineshloka
{मध्याह्नाच्चलिते सूर्ये काले कुतपके तथा}
{आयाता ऋषयः सर्वे ये रामेणानुमन्त्रिताः} %॥८८।

\twolineshloka
{तानागतान्मुनीन्दृष्ट्वा वैदेही जनकात्मजा}
{रामान्तिकं परित्यज्य व्रीडिताऽन्यत्र संस्थिता} %॥८९।

\twolineshloka
{विस्मयोत्फुल्लनयना चिन्तयाना च वेपती}
{ब्राह्मणा नेह जानन्ति श्राद्धकाले ह्युपस्थिताः} %॥९०।

\twolineshloka
{रामेण भोजिता विप्राः स्मृत्युक्तेन यथाविधि}
{वैदिक्यश्च कृतास्सर्वाः सत्क्रिया यास्समीरिताः} %॥९१।

\twolineshloka
{पुराणोक्तो विधिश्चैव वैश्वदेविकपूर्वकः}
{भुक्तवत्सु च विप्रेषु दत्वा पिण्डान्यथाक्रमम्} %॥९२।

\twolineshloka
{प्रेषितेषु यथाशक्ति दत्वा तेषु च दक्षिणाम्}
{गतेषु विप्रमुख्येषु प्रियां रामोऽब्रवीदिदम्} %॥९३।

\twolineshloka
{किमर्थं सुभ्रु नष्टासि मुनीन्दृष्ट्वा त्विहागतान्}
{तत्सर्वं त्वमिदं तत्वं कारणं वद माचिरम्} %॥९४।

\twolineshloka
{भवितव्यं कारणेन तच्च गोप्यं न मे कुरु}
{शापितासि मम प्राणैर्लक्ष्मणस्य शुचिस्मिते} %॥९५।

\twolineshloka
{एवमुक्ता तदा भर्त्रा त्रपयाऽवाङ्मुखी स्थिता}
{विमुञ्चन्ती साऽश्रुपातं राघवं वाक्यमब्रवीत्} %॥९६।

\twolineshloka
{शृणु त्वं नाथ यद्दृष्टमाश्चर्यमिह यादृशम्}
{राम त्वयाऽचिन्त्यमानो राजेन्द्रस्त्विह चागतः} %॥९७।

\twolineshloka
{सर्वाभरणसंयुक्तौ द्वौ चान्यौ च तथाविधौ}
{द्विजानां देहसंयुक्तास्त्रयस्ते रघुनन्दन} %॥९८।

\twolineshloka
{पितरस्तु मया दृष्टा ब्राह्मणाङ्गेषु राघव}
{दृष्ट्वा त्रपान्विता चाहमपक्रान्ता तवान्तिकात्} %॥९९।

\twolineshloka
{त्वया वै भोजिता विप्राः कृतं श्राद्धं यथाविधि}
{वल्कलाजिनसंवीता कथं राज्ञः पुरःसरा} %॥१००।

\twolineshloka
{भवामि रिपुवीरघ्न सत्यमेतदुदाहृतम्}
{कौशेयानि च वस्त्राणि कैकेय्यापहृतानि च} %॥१०१।

\twolineshloka
{ततः प्रभृति चैवाहं चीरिणी तु वनाश्रयम्}
{ज्ञात्वाहं न वदे किञ्चिन्मा ते दुःखं भवत्विति} %॥१०२।

\twolineshloka
{नाहं स्मरामि वै मातुर्न पितुश्च परन्तप}
{कदा भविष्यतीहान्तो वनवासस्य राघव} %॥१०३।

\twolineshloka
{एतदेवानिशं राम चिन्तयन्त्याः पुनः पुनः}
{व्रजन्ति दिवसा नाथ तव पद्भ्यां शपाम्यहम्} %॥१०४।

\twolineshloka
{स्वहस्तेन कथं राज्ञो दास्ये वै भोजनं त्विदम्}
{दासानामपि यो दासो नोपभुञ्जीतयत्क्वचित्} %॥१०५।

\twolineshloka
{एतादृशी कथं त्वस्मै सम्प्रदातुं समुत्सहे}
{याहं राज्ञा पुरा दृष्टा सर्वालङ्कारभूषिता} %॥१०६।

\twolineshloka
{बालव्यजनहस्ता च वीजयन्ती नराधिपम्}
{सा स्वेदमलदिग्धाङ्गी कथं पश्यामि भूमिपम्} %॥१०७।

\twolineshloka
{व्यक्तं त्रिविष्टपं प्राप्तस्त्वया पुत्रेण तारितः}
{दृष्ट्वा मां दुःखितां बालां वने क्लिष्टामनागसम्} %॥१०८।

\twolineshloka
{शोकः स्यात्पार्थिवस्यास्य तेन नष्टास्मि राघव}
{भवान्प्राणसमो राम न ते गोप्यं ममत्विह} %॥१०९।

\twolineshloka
{सत्येन तेन चैवाथ स्पृशामि चरणौ तव}
{तच्छ्रुत्वा राघवः प्रीतः प्रियां तां प्रियवादिनीम्} %॥११०।

\twolineshloka
{अङ्कमानीय सुदृढं परिष्वज्य च सादरम्}
{भुक्तौ भोज्यं तदा वीरौ पश्चाद्भुक्ता च जानकी} %॥१११।

\twolineshloka
{एवं स्थितौ तदा सा च तां रात्रिं तत्र राघवौ}
{उदिते च सहस्रांशौ गमनाय मनो दधुः} %॥११२।

\twolineshloka
{प्रत्यङ्मुखं गतः क्रोशं ज्येष्ठं यावच्च पुष्करम्}
{पूर्वभागे पुष्करस्य यावत्तिष्ठति राघवः} %॥११३।

\twolineshloka
{शुश्राव च ततो वाचं देवदूतेन भाषितम्}
{भो भो राघव भद्रं ते तीर्थमेतत्सुदुर्लभम्} %॥११४।

\twolineshloka
{अस्मिन्स्थाने स्थितो वीर आत्मनः पुण्यतां कुरु}
{देवकार्यं त्वया कार्यं हन्तव्या देवशत्रवः} %॥११५।

\twolineshloka
{ततो हृष्टमना वीरो ह्यब्रवील्लक्ष्मणं वचः}
{सौमित्रेऽनुगृहीतोहं देवदेवेन ब्रह्मणा} %॥११६।

\twolineshloka
{अत्राश्रमपदं कृत्वा मासमेकं च लक्ष्मण}
{व्रतं चरितुमिच्छामि कायशोधनमुत्तमम्} %॥११७।

\twolineshloka
{तथेति लक्ष्मणेनोक्ते व्रतं परिसमाप्यतु}
{पिण्डदानादिभिर्दानैः श्राद्धैश्चैव पितामहान्} %॥११८।

\twolineshloka
{पुष्करे तु तदा रामोऽतर्पयद्विधिवत्तदा}
{कनका सुप्रभा चैव नन्दा प्राची सरस्वती} %॥११९।

\twolineshloka
{पञ्चस्रोताः पुष्करेषु पितॄणां तुष्टिदायिनी}
{दैनन्दिनीं पितॄणां तु पूजां तां पितृपूर्विकाम्} %॥१२०।

\twolineshloka
{रचयित्वा तदा रामो लक्ष्मणं वाक्यमब्रवीत्}
{एहि लक्ष्मण शीघ्रं त्वं पुष्कराज्जलमानय} %॥१२१।

\twolineshloka
{पादप्रक्षालनं कृत्वा शयनं कुरु संस्तरे}
{विभावर्यां निवृत्तायां यास्यामो दक्षिणां दिशम्} %॥१२२।

\twolineshloka
{लक्ष्मणस्त्वब्रवीद्वाक्यं सीतयानीय तां पयः}
{नाहं राम सर्वकाले दासभावं करोमि ते} %॥१२३।

\twolineshloka
{इयम्पुष्टाचसुभृशम्पीवरीचममाप्युत}
{किं त्वं करिष्यस्यनया भार्यया वद साम्प्रतम्} %॥१२४।

\twolineshloka
{किं वा मृतस्य वै पृष्ठ इयं यास्यति ते प्रिया}
{रक्षसे त्वं सदा कालं सुपुष्टां चैव सर्वदा} %॥१२५।

\twolineshloka
{हृष्टा चैषा क्लेशयति सततं मां रघूत्तम}
{त्वं च क्लेशयसे राम परत्र जायते क्षतिः} %॥१२६।

\twolineshloka
{त्वत्कृते च सदा चाहं पिपासां क्षुधया सह}
{संसहामि न सन्देहः परत्र च निशामय} %॥१२७।

\twolineshloka
{मृतानां पृष्ठतः कश्चिद्गतो नैव च दृश्यते}
{भार्य्या पुत्रो धनं चापि एवमाहुर्मनीषिणः} %॥१२८।

\twolineshloka
{मृतश्च ते पिता राम त्यक्त्वा राज्यमकण्टकम्}
{विनिक्षिप्य वने त्वां च कैकेय्याः प्रियकाम्यया} %॥१२९।

\twolineshloka
{इहस्थिता सा कैकेयी धनं सर्वे च बान्धवाः}
{महाराजो दशरथ एक एव गतो गतिम्} %॥१३०।

\twolineshloka
{मन्येहं न त्वया सार्धं सीता यास्यति वै ध्रुवम्}
{करिष्यसे किमनया वद राघव साम्प्रतम्} %॥१३१।

\twolineshloka
{श्रुत्वा चाश्रुतपूर्वं हि वाक्यं लक्ष्मणभाषितम्}
{विमना राघवस्तस्थौ सीता चापि वरानना} %॥१३२।

\twolineshloka
{यदुक्तं लक्ष्मणेनाथ सीता सर्वं चकार ह}
{स्नात्वा भुक्त्वा ततो वीरौ पुष्करे पुष्करेक्षणौ} %॥१३३।

\twolineshloka
{नीत्वा विभावरीं तत्र गमनाय मनो दधुः}
{एह्युत्तिष्ठ च सौमित्रे व्रजामो दक्षिणां दिशम्} %॥१३४।

\twolineshloka
{सौमित्रिरब्रवीद्राम नाहं यास्ये कथञ्चन}
{व्रज त्वमनया सार्धं भार्यया कमलेक्षण} %॥१३५।

\twolineshloka
{नान्यद्वनं गमिष्यामि नैवायोध्यां च राघव}
{अस्मिन्वने वसिष्यामि वर्षाणीह चतुर्दश} %॥१३६।

\twolineshloka
{मया विना त्वयोध्यायां यदि त्वं न गमिष्यसि}
{अनेन वर्त्मना भूप आगन्तव्यं त्वया विभो} %॥१३७।

\twolineshloka
{यदि जीवामि तत्कालं पुनर्यास्ये पितुः पुरम्}
{तपस्सम्भावयिष्यामि मया त्वं किं करिष्यसि} %॥१३८।

\twolineshloka
{व्रज सौम्य शिवः पन्थामा च ते परिपन्थिनः}
{पश्यामि त्वां पुनः प्राप्तं सभार्यं कमलेक्षणम्} %॥१३९।

\twolineshloka
{पितृपैतामहं राज्यमयोध्यायां नराधिप}
{शत्रुघ्नभरतौ चोभौ त्वदाज्ञाकरणे स्थितौ} %॥१४०।

\twolineshloka
{अहं ते प्रतिकूलस्तु वनवासे विशेषतः}
{अनारतं दिवा चाहं रात्रौ चैव परन्तप} %॥१४१।

\twolineshloka
{कर्मकर्तुं न शक्रोमि व्रज सौम्य यथासुखम्}
{एवं ब्रुवाणं सौमित्रिमुवाच रघुनन्दनः} %॥१४२।

\twolineshloka
{कथं पूर्वमयोध्याया निर्गतोसि मया सह}
{वने वत्स्याम्यहं राम नववर्षाणि पञ्च च} %॥१४३।

\twolineshloka
{न तु त्वया विरहितः स्वर्गेपि निवसे क्वचित्}
{या गतिस्ते नरव्याघ्र मम सापि भविष्यति} %॥१४४।

\twolineshloka
{प्रसादः क्रियतां मह्यं नय मामपि राघव}
{इदानीमर्धमार्गे त्वं कथं स्थास्यसि शत्रुहन्} %॥१४५।

\twolineshloka
{लक्ष्मणस्त्वब्रवीद्रामं नाहं गन्ता वने पुनः}
{लक्ष्मणं संस्थितं ज्ञात्वा रामो वचनमब्रवीत्} %॥१४६।

\twolineshloka
{मामनुव्रज सौमित्र एको यास्यामि काननम्}
{द्वितीया मे त्वियं सीता रामेणोक्तस्तु लक्ष्मणः} %॥१४७।

\twolineshloka
{गृहीत्वाऽथ समुत्तस्थौ रामवाक्यं स लक्ष्मणः}
{मर्यादापर्वतं प्राप्तौ क्षेत्रसीमां परन्तपौ} %॥१४८।

\twolineshloka
{अजगन्धं च देवेशं देवदेवं पिनाकिनम्}
{अष्टाङ्गप्रणिपातेन नत्वा रामस्त्रिलोचनम्} %॥१४९।

\twolineshloka
{तुष्टाव प्रयतः स्थित्वा शङ्करं पार्वतीप्रियम्}
{कृताञ्जलिपुटो भूत्वा रोमाञ्चितशरीरकः} %॥१५०।

\twolineshloka
{सात्विकं भावमापन्नो विनिर्धूतरजस्तमाः}
{लोकानां कारणं देवं बुबुधे विबुधाधिपम्} %॥१५१।
\uvacha{राम उवाच}

\twolineshloka
{कृत्स्नस्य योऽस्य जगतः स चराचरस्य कर्ता कृतस्य च पुनः सुखदुःखदश्च}
{संहारहेतुरपि यः पुनरन्तकाले तं शङ्करं शरणदं शरणं व्रजामि} %॥१५२।

\twolineshloka
{योऽयं सकृद्विमलचारुविलोलतोयां गङ्गां महोर्मिविषमां गगनात्पतन्तीम्}
{मूर्ध्ना दधेऽस्रजमिव प्रविलोलपुष्पां तं शङ्करं शरणदं शरणं व्रजामि} %१५३ ।

\twolineshloka
{कैलासशैलशिखरं परिकम्प्यमानं कैलासशृङ्गसदृशेन दशाननेन}
{यत्पादपद्मविधृतं स्थिरतां दधार तं शङ्करं शरणदं शरणं व्रजामि} %॥१५४।

\twolineshloka
{येनासकृद्दनुसुताः समरे निरस्ता विद्याधरोरगगणाश्च वरैः समग्रैः}
{संयोजिता मुनिवराः फलमूलभक्षास्तं शङ्करं शरणदं शरणं व्रजामि} %॥१५५।

\twolineshloka
{दक्षाध्वरे च नयने च तथा भगस्य पूष्णस्तथा दशनपङ्क्तिमपातयच्च}
{तस्तम्भयः कुलिशयुक्तमथेन्द्रहस्तं तं शङ्करं शरणदं शरणं व्रजामि} %॥१५६।

\twolineshloka
{एनःकृतोपिविषयेष्वपिसक्तचित्ताज्ञानान्वयश्रुतगुणैरपिनैवयुक्ताः}
{यं संश्रिताः सुखभुजः पुरुषा भवन्ति तं शङ्करं शरणदं शरणं व्रजामि} %॥१५७।

\twolineshloka
{अत्रिप्रसूतिरविकोटिसमानतेजाः सन्त्रासनं विबुधदानवसत्तमानाम्}
{यः कालकूटमपिबत्प्रसभं सुदीप्तं तं शङ्करं शरणदं शरणं व्रजामि} %॥१५८।

\twolineshloka
{ब्रह्मेन्द्ररुद्रमरुतां च सषण्मुखानां दद्याद्वरं सुबहुशो भगवान्महेशः}
{नन्दिं च मृत्युवदनात्पुनरुज्जहार तं शङ्करं शरणदं शरणं व्रजामि} %॥१५९।

\twolineshloka
{आराधितः सुतपसा हिमवन्निकुञ्जे धूमव्रतेन मनसापि परैरगम्ये}
{सञ्जीवनीमकथयद्भृगवे महात्मा तं शङ्करं शरणदं शरणं व्रजामि} %॥१६०।

\twolineshloka
{नानाविधैर्गजबिडालसमानवक्त्रैर्दक्षाध्वरप्रमथनैर्बलिभिर्गणैन्द्रैः }
{योभ्यर्चितोमरगणैश्च सलोकपालैस्तं शङ्करं शरणदं शरणं व्रजामि} %॥१६१।

\twolineshloka
{शङ्खेन्दुकुन्दधवलं वृषभं प्रवीरमारुह्य यः क्षितिधरेन्द्रसुतानुयातः}
{यात्यम्बरं प्रलयमेघविभूषितं च तं शङ्करं शरणदं शरणं व्रजामि} %॥१६२।

\twolineshloka
{शान्तं मुनिं यमनियोगपरायणैस्तैर्भीमैर्महोग्रपुरुषैः प्रतिनीयमानम्}
{भक्त्यानतं स्तुतिपरं प्रसभं ररक्ष तं शङ्करं शरणदं शरणं व्रजामि} %॥१६३।

\twolineshloka
{यः सव्यपाणि कमलाग्रनखेन देवस्तत्पञ्चमं प्रसभमेव पुरस्सुराणाम्}
{ब्राह्मं शिरस्तरुणपद्मनिभं चकर्त्त तं शङ्करं शरणदं शरणं व्रजामि} %॥१६४।

\twolineshloka
{यस्य प्रणम्य चरणौ वरदस्य भक्त्या स्तुत्वा च वाग्भिरमलाभिरतन्द्रितात्मा}
{दीप्तस्तमांसि नुदते स्वकरैर्विवस्वांस्तं शङ्करं शरणदं शरणं व्रजामि} %॥१६५।

\twolineshloka
{ये त्वां सुरोत्तमगुरुं पुरुषा विमूढा जानन्ति नास्य जगतः सचराचरस्य}
{ऐश्वर्यमाननिगमानुशयेन पश्चात्ते यातनामनुभवन्त्यविशुद्धचित्ताः} %॥१६६।

\twolineshloka
{तस्यैवं स्तुवतोऽवोचच्छूलपाणिर्वृषध्वजः}
{उवाच वचनं हृष्टो राघवं तुष्टमानसः} %॥१६७।
\uvacha{रुद्र उवाच}

\twolineshloka
{राम हृष्टोस्मि भद्रं ते जातस्त्वं निर्मले कुले}
{त्वं चापि जगतां वन्द्यो देवो मानुषरूपधृत्} %॥१६८।

\twolineshloka
{त्वया नाथेन वै देवाः सुखिनः शाश्वतीः समा}
{सेविष्यन्ते चिरं कालं गते वर्षे चतुर्दशे} %॥१६९।

\twolineshloka
{अयोध्यामागतं त्वां ये द्रक्ष्यन्ति भुवि मानवाः}
{सुखं तेऽत्र भजिष्यन्ति स्वर्गे वासन्तथाक्षयम्} %॥१७०।

\twolineshloka
{देवकार्यं महत्कृत्वा आगच्छेथाः पुनः पुरीम्}
{राघवस्तु तथा देवं नत्वा शीघ्रं विनिर्गतः} %॥१७१।

\twolineshloka
{इन्द्रमार्गां नदीं प्राप्य जटाजूटं नियम्य च}
{अब्रवील्लक्ष्मणं राम इदमर्पय मे धनुः} %॥१७२।

\twolineshloka
{रामवाक्यं तु तच्छ्रुत्वा सीतां वै लक्ष्मणोऽब्रवीत्}
{किमर्थं देवि रामेण त्यक्तोहं कारणं विना} %॥१७३।

\twolineshloka
{अपराधं न जानामि कुपितो यन्महाभुजः}
{रामेणाहं परित्यक्तः प्राणांस्त्यक्ष्याम्यसंशयम्} %॥१७४।

\twolineshloka
{नैव मे जीवितेनार्थो धिग्धिङ्मां कुलपांसनम्}
{आर्यस्य येन वै मन्युर्जनितः पापकारिणा} %॥१७५।

\twolineshloka
{कांस्तु लोकान्गमिष्यामि अपध्यातो महात्मना}
{उभौ हस्तौ मुखे कृत्वा साश्रुकण्ठोऽब्रवीदिदम्} %॥१७६।

\twolineshloka
{नापराध्यामि रामस्य कर्मणा मनसा गिरा}
{स्पृष्टौ ते चरणौ देवि मम नान्या गतिर्भवेत्} %॥१७७।

\twolineshloka
{ततः सीताऽब्रवीद्रामं त्यक्तः किमनुजस्त्वया}
{वैषम्यं त्यज्यतां बाले लक्ष्मणे लक्ष्मिवर्धने} %॥१७८।

\twolineshloka
{राघवस्त्वब्रवीत्सीतां नाहं त्यक्ष्यामि लक्ष्मणम्}
{न कदाचिदपि स्वप्ने लक्ष्मणस्य मतं प्रिये} %॥१७९।

\twolineshloka
{श्रुतपूर्वं च सुश्रोणि क्षेत्रस्यास्य विचेष्टितम्}
{अत्र क्षेत्रे जनास्सत्यं सर्वे हि स्वार्थतत्पराः} %॥१८०।

\twolineshloka
{परस्परं न पश्यन्ति स्वात्मनश्च हितं वचः}
{न शृण्वन्ति पितुः पुत्राः पुत्राणां पितरस्तथा} %॥१८१।

\twolineshloka
{न शिष्या हि गुरोर्वाक्यं शिष्यस्यापि तथा गुरुः}
{अर्थानुबन्धिनीप्रीतिर्न कश्चित्कस्यचित्प्रियः} %॥१८२।

\twolineshloka
{इत्येवं कथयन्नेव प्राप्तो रेवां महानदीम्}
{चक्रेभिषेकं काकुत्स्थः सानुजः सह सीतया} %॥१८३।

\twolineshloka
{तर्पयित्वा च सलिलैः स्वान्पितॄन्दैवतान्यपि}
{उदीक्ष्य च मुहुः सूर्यं देवताश्च समाहितः} %॥१८४।

\twolineshloka
{कृताभिषेकस्तु रराज रामः सीता द्वितीयः सह लक्ष्मणेन}
{कृताभिषेकः सह शैलपुत्र्या गुहेन सार्धं भगवानिवेशः} %॥१८५।

॥इति श्रीपाद्मपुराणे प्रथमे सृष्टिखण्डे मार्कण्डेयाश्रमदर्शनं नाम त्र्यस्त्रिंशोऽध्यायः॥३३॥
    \sect{शूद्रतापसवधः}

\src{पद्म-पुराणम्}{सृष्टिखण्डम्}{अध्यायः ३६}{१--१८५}
% \tags{concise, complete}
\notes{In this chapter, Rama is approached by a grieving brāhmaṇa whose son has died prematurely; learning that the death was due to a śūdra performing severe penance, Rama locates and kills the ascetic Śambūka, sending him to heaven, and the boy comes back to life at the very moment.}
\textlink{https://sa.wikisource.org/wiki/पद्मपुराणम्/खण्डः_१_(सृष्टिखण्डम्)/अध्यायः_३५}
\translink{https://www.wisdomlib.org/hinduism/book/the-padma-purana/d/doc364158.html}

\storymeta



\uvacha{भीष्म उवाच}

\twolineshloka
{उक्तं भगवता सर्वं पुराणाश्रयसंयुतम्}
{तथा श्वेतेन ब्रह्माण्डं गुरवे प्रतिपादितम्}% १

\twolineshloka
{श्रुत्वैतत्कौतुकं जातं यथा तेनास्थिलेहनम्}
{कृतं क्षुधापनोदार्थे अन्नदानाद्विना द्विज}% २

\twolineshloka
{तदहं श्रोतुमिच्छामि पृथिव्यां ये च पार्थिवाः}
{अन्नदानाद्दिवं प्राप्ताः क्रतवश्चान्नमूलकाः}% ३

\twolineshloka
{कथं तस्य मतिर्नष्टा श्वेतस्य च महात्मनः}
{न दत्तं तेनान्नदानमृषिभिर्वा न दर्शितम्}% ४

\twolineshloka
{अहो माहात्म्यमन्नस्य इह दत्तस्य यत्फलम्}
{परत्र भुज्यते पुम्भिः स्वर्गश्चाक्षयतां व्रजेत्}% ५

\twolineshloka
{अन्नदानं परं विप्राः कीर्तयन्ति सदोत्थिताः}
{अन्नदानात्सुरेद्रेण त्रैलोक्यमिह भुज्यते}% ६

\twolineshloka
{शतक्रतुरिति प्रोक्तः सर्वैरेव द्विजोत्तमैः}
{तेनावस्थां तत्सदृशीं प्राप्तवांस्त्रिदशेश्वरः}% ७

\twolineshloka
{दानदेवगतः स्वर्गं त्वत्तः सर्वं श्रुतं मया}
{अपरं च पुरावृत्तं निवृत्तं यदि कर्हिचित्}% ८

\onelineshloka*
{भूयोपि श्रोतुमिच्छामि तन्मे वद महामते}

\uvacha{पुलस्त्य उवाच}

\onelineshloka
{एतदाख्यानकं पूर्वमगस्त्येन महात्मना}% ९

\onelineshloka*
{रामाय कथितं राजंस्तत्ते वक्ष्यामि साम्प्रतम्}

\uvacha{भीष्म उवाच}

\onelineshloka
{कस्मिन्वंशे समुत्पन्नो रामोऽसौ नृपसत्तमः}% १०

\onelineshloka*
{यस्यागस्त्येन कथितश्चेतिहासः पुरातनः}

\uvacha{पुलस्त्य उवाच}

\onelineshloka
{रघुवंशे समुत्पन्नो रामो नाम महाबलः}% ११

\twolineshloka
{देवकार्यं कृतं तेन लङ्कायां रावणो हतः}
{पृथिवीं राज्यसंस्थस्य ऋषयोऽभ्यागता गृहे}% १२

\twolineshloka
{प्राप्तास्ते तु महात्मानो राघवस्य निवेशनम्}
{प्रतीहारस्ततो राममगस्त्यवचनाद्द्रुतम्}% १३

\twolineshloka
{आवेदयामास ऋषीन्प्राप्तास्तांश्च त्वरान्वितः}
{दृष्ट्वा रामं द्वारपालः पूर्णचन्द्रमिवोदितम्}% १४

\twolineshloka
{कौसल्यासुत भद्रं ते सुप्रभाताद्य शर्वरी}
{द्रष्टुमभ्युदयं तेद्य सम्प्राप्तो रघुनन्दन}% १५

\twolineshloka
{अगस्त्यो मुनिभिः सार्धं द्वारि तिष्ठति ते नृप}
{श्रुत्वा प्राप्तान्मुनीन्रामस्तान्भास्करसमद्युतीन्}% १६

\twolineshloka
{प्राह वाक्यं तदा द्वास्थं प्रवेशय त्वरान्वितः}
{किमर्थं तु त्वया द्वारि निरुद्धा मुनिसत्तमाः}% १७

\twolineshloka
{रामवाक्यान्मुनींस्तांस्तु प्रावेशयद्यथासुखम्}
{दृष्ट्वा तु तान्मुनीन्न्प्राप्तान्प्रत्युवाच कृताञ्जलि}% १८

\twolineshloka
{रामोऽभिवाद्य प्रणत आसनेषु न्यवेशयत्}
{ते तु काञ्चनचित्रेषु स्वास्तीर्णेषु सुखेषु च}% १९

\twolineshloka
{कुशोत्तरेषु चासीनाः समन्तान्मुनिपुङ्गवाः}
{पाद्यमाचमनीयं च ददौ चार्घ्यं पुरोहितः}% २०

\twolineshloka
{रामेण कुशलं पृष्टा ऋषयः सर्व एव ते}
{महर्षयो वेदविद इदं वचनमब्रुवन्}% २१

\twolineshloka
{कुशलं ते महाबाहो सर्वत्र रघुनन्दन}
{त्वां तु दिष्ट्या कुशलिनं पश्यामो हतविद्विषम्}% २२

\twolineshloka
{हृता सीतातिपापेन रावणेन दुरात्मना}
{पत्नी ते रघुशार्दूल तस्या एवौजसा हतः}% २३

\twolineshloka
{असहायेन चैकेन त्वया राम रणे हतः}
{यादृशं ते कृतं कर्म तस्य कर्ता न विद्यते}% २४

\twolineshloka
{इह सम्भाषितुं प्राप्ता दृष्ट्वा पूताः स्म साम्प्रतम्}
{दर्शनात्तव राजेन्द्र सर्वे जातास्तपस्विनः}% २५

\twolineshloka
{रावणस्य वधात्तेद्य कृतमश्रुप्रमार्जनम्}
{दत्वा पुण्यामिमां वीर जगत्यभयदक्षिणाम्}% २६

\twolineshloka
{दिष्ट्या वर्धसि काकुत्स्थ जयेनामितविक्रम}
{दृष्टस्सम्भाषितश्चासि यास्यामश्चाश्रमान्स्वकान्}% २७

\twolineshloka
{अरण्यं ते प्रविष्टस्य मया चेन्द्रशरासनम्}
{अर्पितं चाक्षयौ तूणौ कवचं च परन्तप}% २८

\twolineshloka
{भूयोप्यागमनं कार्यमाश्रमे मे रघूद्वह}
{एवमुक्त्वा तु ते सर्वे मुनयोन्तर्हिताऽभवन्}% २९

\twolineshloka
{गतेषु मुनिमुख्येषु रामो धर्मभृतां वरः}
{चिन्तयामास तत्कार्यं किं स्यान्मे मुनिनोदितम्}% ३०

\twolineshloka
{भूयोप्यागमनं कार्यमाश्रमे रघुनन्दन}
{अवश्यमेव गन्तव्यं मयाऽगस्त्यस्य सन्निधौ}% ३१

\twolineshloka
{श्रोतव्यं देवगुह्यं तु कार्यमन्यच्च यद्वदेत्}
{एवं चिन्तयतस्तस्य रामस्यामिततेजसः}% ३२

\twolineshloka
{करिष्ये नियतं धर्मं धर्मो हि परमा गतिः}
{सुतवर्षसहस्राणि दश राज्यमकारयत्}% ३३

\twolineshloka
{ददतो जुह्वतश्चैव जग्मुस्तान्येकवर्षवत्}
{प्रजाः पालयतस्तस्य राघवस्य महात्मनः}% ३४

\twolineshloka
{एतस्मिन्नेव दिवसे वृद्धो जानपदो द्विजः}
{मृतं पुत्रमुपादाय रामद्वारमुपागतः}% ३५

\twolineshloka
{उवाच विविधं वाक्यं स्नेहाक्षरसमन्वितम्}
{दुष्कृतं किन्तु मे पुत्र पूर्वदेहान्तरे कृतम्}% ३६

\twolineshloka
{त्वामेकपुत्रं यदहं पश्यामि निधनं गतम्}
{अप्राप्तयौवनं बालं पञ्चवर्षं गतायुषम्}% ३७

\twolineshloka
{अकाले कालमापन्नं दुःखाय मम पुत्रक}
{अकृत्वा पितृकार्याणि गतो वैवस्वतक्षयम्}% ३८

\twolineshloka
{रामस्य दुष्कृतं व्यक्तं येन ते मृत्युरागतः}
{बालवध्या ब्रह्मवध्या स्त्रीवध्या चैव राघवम्}% ३९

\twolineshloka
{प्रवेक्ष्यति न सन्देहः सभार्ये तु मृते मयि}
{शुश्राव राघवः सर्वं दुःखशोकसमन्वितम्}% ४०

\twolineshloka
{निवार्य तं द्विजं रामो वसिष्ठं वाक्यमब्रवीत्}
{किं मयाद्य च कर्तव्यं कार्यमेवं विधे स्थिते}% ४१

\twolineshloka
{प्राणानहं जुहोम्यग्नौ पर्वताद्वा पतेह्यहम्}
{कथं शुद्धिमहं यामि श्रुत्वा ब्राह्मणभाषितम्}% ४२

\twolineshloka
{वसिष्ठस्याग्रतः स्थित्वा राज्ञो दीनस्य नारदः}
{प्रत्युवाच श्रुतं वाक्यमृषीणां सन्निधौ तदा}% ४३

\twolineshloka
{शृणु राम यथाकालं प्राप्तो वै बालसङ्क्षयः}
{पुरा कृतयुगे राम सर्वत्र ब्राह्मणोत्तरम्}% ४४

\twolineshloka
{अब्राह्मणो न वै कश्चित्तपस्तपति राघव}
{अमृत्यवस्तदा सर्वे जायन्ते चिरजीविनः}% ४५

\twolineshloka
{त्रेतायुगे पुनः प्राप्ते ब्रह्मक्षत्रमनुत्तमम्}
{अधर्मो द्वापरे तेषां वैश्यान्शूद्रांस्तथाविशत्}% ४६

\twolineshloka
{एवं निरन्तरं जुष्टमुद्भूतमनृतं पुनः}
{अधर्मस्य त्रयः पादा एको धर्मस्य चागतः}% ४७

\twolineshloka
{ततः पूर्वे भृशं त्रस्ता वर्णा ब्राह्मणपूर्वकाः}
{भूयः पादस्तु धर्मस्य द्वितीयः समपद्यत}% ४८

\twolineshloka
{तस्मिन्द्वापरसंज्ञे तु तपो वैश्यं समाविशत्}
{युगत्रयस्य वैधर्म्यं धर्मस्य प्रतितिष्ठति}% ४९

\twolineshloka
{कलिसंज्ञे ततः प्राप्ते वर्तमाने युगेन्तिमे}
{अधर्मश्चानृतं चैव ववृधाते नरर्षभ1.35.}% ५०

\twolineshloka
{भविता शूद्रयोन्यां तु तपश्चर्या कलौ युगे}
{स ते विषयपर्यन्ते राजन्नुग्रतरं तपः}% ५१

\twolineshloka
{शूद्रस्तपति दुर्बुद्धिस्तेन बालवधः कृतः}
{यस्याधर्ममकार्यं वा विषये पार्थिवस्य हि}% ५२

\twolineshloka
{पुरे वा राजशार्दूल कुरुते दुर्मतिर्नरः}
{क्षिप्रं स नरकं याति यावदाभूतसम्प्लवम्}% ५३

\twolineshloka
{चतुर्थं तस्य पापस्य भागमश्नाति पार्थिवः}
{सत्त्वं पुरुषशार्दूल गच्छस्व विषयं स्वकम्}% ५४

\twolineshloka
{दुष्कृतं यत्र पश्येथास्तत्र यत्नं समाचर}
{एवं ते धर्मवृद्धिश्च बलस्य वर्धनं तथा}% ५५

\twolineshloka
{भविष्यति नरश्रेष्ठ बालस्यास्य च जीवनम्}
{नारदेनैवमुक्तस्तु साश्चर्यो रघुनन्दनः}% ५६

\twolineshloka
{प्रहर्षमतुलं लेभे लक्ष्मणं चेदमब्रवीत्}
{गच्छ सौम्य द्विजश्रेष्ठं समाश्वासय लक्ष्मण}% ५७

\twolineshloka
{बालस्य च शरीरं त्वं तैलद्रोण्यां निधापय}
{गन्धैश्च परमोदारैस्तैलैश्चैव सुगन्धिभिः}% ५८

\twolineshloka
{यथा न शीर्यते बालस्तथा सौम्य विधीयताम्}
{यथा शरीरं गुप्तं स्याद्बालस्याक्लिष्टकर्मणः}% ५९

\twolineshloka
{विपत्तिः परिभेदो वा न भवेत्तत्तथा कुरु}
{तथा सन्दिश्य सौमित्रं लक्ष्मणं शुभलक्षणम्}% ६०

\twolineshloka
{मनसा पुष्पकं दध्यावागच्छेति महायशाः}
{इङ्गितं तत्तु विज्ञाय कामगं हेमभूषितम्}% ६१

\twolineshloka
{आजगाम मुहूर्तात्तु समीपं राघवस्य हि}
{सोब्रवीत्प्राञ्जलिर्वाक्यमहमस्मि नराधिप}% ६२

\twolineshloka
{अग्रे तव महाबाहो किङ्करः समुपस्थितः}
{भाषितं सुचिरं श्रुत्वा पुष्पकस्य नराधिप}% ६३

\twolineshloka
{अभिवाद्य महर्षींस्तान्विमानं सोध्यरोहत}
{धनुर्गृहीत्वा तूणौ च खड्गं चापि महाप्रभम्}% ६४

\twolineshloka
{निक्षिप्य नगरे वीरौ सौमित्रि भरतावुभौ}
{प्रायात्प्रतीचीं त्वरितो विचिन्वन्सुसमाहितः}% ६५

\twolineshloka
{उत्तरामगमत्पश्चाद्दिशं हिमवदाश्रिताम्}
{पूर्वामपि दिशां गत्वा तथाऽपश्यन्नराधिपः}% ६६

\twolineshloka
{सर्वां शुद्धसमाचारामादर्शमिव निर्मलाम्}
{ततो दिशं समाक्रामद्दक्षिणां रघुनन्दनः}% ६७

\twolineshloka
{शैलस्य उत्तरे पार्श्वे ददर्श सुमहत्सरः}
{तस्मिन्सरसि तप्यन्तं तापसं सुमहत्तपः}% ६८

\twolineshloka
{ददर्श राघवो भीमं लम्बमानमधोमुखम्}
{तमुपागम्य काकुत्स्थस्तप्यमानं तु तापसम्}% ६९

\twolineshloka
{उवाच राघवो वाक्यं धन्यस्त्वममरप्रभ}
{कस्यां योनौ तपोवृद्धिर्वर्तते दृढनिश्चय}% ७०

\twolineshloka
{अहं दाशरथी रामः पृच्छामि त्वां कुतूहलात्}
{कोर्थो व्यवसितस्तुभ्यं स्वर्गलोकोथ वेतरः}% ७१

\twolineshloka
{किमर्थं तप्यसे वा त्वं श्रोतुमिच्छामि तापस}
{ब्राह्मणो वासि भद्रं ते क्षत्रियो वाथ दुर्जयः}% ७२

\twolineshloka
{वैश्यस्तृतीयवर्णो वा शूद्रो वा सत्यमुच्यताम्}
{तपः सत्यात्मकं नित्यं स्वर्गलोकपरिग्रहे}% ७३

\twolineshloka
{सात्विकं राजसं चैव तच्च सत्यात्मकं तपः}
{जगदुपकारहेतुर्हि सृष्टं तद्वै विरिञ्चिना}% ७४

\twolineshloka
{रौद्रं क्षत्रियतेजोजं तत्तु राजसमुच्यते}
{परस्योत्सादनार्थाय तच्चासुरमुदाहृतम्}% ७५

\twolineshloka
{अङ्गानि निह्नुते यो वा असृग्दिग्धानि भागशः}
{पञ्चाग्निंसाधयेद्वापि सिद्धिं वा मृत्युमेव वा}% ७६

\twolineshloka
{आसुरो ह्येष ते भावो न च मे त्वं द्विजो मतः}
{सत्यं ते वदतः सिद्धिरनृते नास्ति जीवितम्}% ७७

\twolineshloka
{तस्य तद्भाषितं श्रुत्वा रामस्याक्लिष्टकर्मणः}
{अवाक्शिरास्तथा भूतो वाक्यमेतदुवाच ह}% ७८

\twolineshloka
{स्वागतं ते नृपश्रेष्ठ चिराद्दृष्टोसि राघव}
{पुत्रभूतोस्मि ते चाहं पितृभूतोसि मेनघ}% ७९

\twolineshloka
{अथवा नैतदेवं हि सर्वेषां नृपतिः पिता}
{सत्वमर्च्योऽसि भो राजन्वयं ते विषये तपः}% ८०

\twolineshloka
{चरामस्तत्रभागोस्ति पूर्वं सृष्टः स्वयम्भुवा}
{न धन्याः स्मो वयं राम धन्यस्त्वमसि पार्थिव}% ८१

\twolineshloka
{यस्य ते विषये ह्येवं सिद्धिमिच्छन्ति तापसाः}
{तपसा त्वं मदीयेन सिद्धिमाप्नुहि राघव}% ८२

\twolineshloka
{यदेतद्भवता प्रोक्तं योनौ कस्यां तु ते तपः}
{शूद्रयोनिप्रसूतोहं तप उग्रं समास्थितः}% ८३

\twolineshloka
{देवत्वं प्रार्थये राम स्वशरीरेण सुव्रत}
{न मिथ्याहं वदे भूप देवलोकजिगीषया}% ८४

\twolineshloka
{शूद्रं मां विद्धि काकुत्स्थ शम्बूकं नाम नामतः}
{भाषतस्तस्य काकुत्स्थः खड्गं तु रुचिरप्रभम्}% ८५

\twolineshloka
{निष्कृष्य कोशाद्विमलं शिरश्चिच्छेद राघवः}
{तस्मिन्शूद्रे हते देवाः सेन्द्राश्चाग्निपुरोगमाः}% ८६

\twolineshloka
{साधुसाध्विति काकुत्स्थं प्रशशंसुर्मुहुर्मुहुः}
{पुष्पवृष्टिश्च महती देवानां सुसुगन्धिनी}% ८७

\twolineshloka
{आकाशाद्विप्रमुक्ता तु राघवं सर्वतोकिरत्}
{सुप्रीताश्चाब्रुवन्देवा रामं वाक्यविदांवरम्}% ८८

\twolineshloka
{सुरकार्यमिदं सौम्य कृतं ते रघुनन्दन}
{गृहाण च वरं राम यमिच्छसि महाव्रत}% ८९

\twolineshloka
{त्वत्कृतेन हि शूद्रोऽयं सशरीरोऽभ्यगाद्दिवम्}
{देवानां भाषितं श्रुत्वा राघवः सुसमाहितः}% ९०

\twolineshloka
{उवाच प्राञ्जलिर्वाक्यं सहस्राक्षं पुरन्दरम्}
{यदि देवाः प्रसन्ना मे वरार्हो यदि वाप्यहम्}% ९१

\twolineshloka
{कर्मणा यदि मे प्रीता द्विजपुत्रः स जीवतु}
{वरमेतद्धि भवतां काङ्क्षितं परमं हि मे}% ९२

\twolineshloka
{ममापराधाद्बालोऽसौ ब्राह्मणस्यैकपुत्रकः}
{अप्राप्तकालः कालेन नीतो वैवस्वत क्षयम्}% ९३

\twolineshloka
{तं जीवयत भद्रं वो नानृती स्यामहं गुरोः}
{द्विजस्य संश्रुतो ह्यर्थो जीवयिष्यामि ते सुतम्}% ९४

\twolineshloka
{मदीयेनायुषा बालं पादेनार्द्धेन वा सुराः}
{जीवेदयं वरो मह्यं वरकोट्यधिको वृतः}% ९५

\twolineshloka
{राघवस्य तु तद्वाक्यं श्रुत्वा विबुधसत्तमाः}
{प्रत्यूचुस्ते महात्मानं प्रीताः प्रीतिसमन्विताः}% ९६

\twolineshloka
{निर्वृतो भव काकुत्स्थ ब्राह्मणस्यैकपुत्रकः}
{जीवितं प्राप्तवान्भूयः समेतश्चापि बन्धुभिः}% ९७

\twolineshloka
{यस्मिन्मुहूर्ते काकुत्स्थ शूद्रोयं विनिपातितः}
{तस्मिन्मुहूर्ते सहसा जीवेन समयुज्यत}% ९८

\twolineshloka
{स्वस्ति प्राप्नुहि भद्रं ते साधयामः परन्तपः}
{अगस्त्यस्याश्रमपदे द्रष्टारः स्म महामुनिम्}% ९९

\twolineshloka
{स तथेति प्रतिज्ञाय देवानां रघुनन्दनः}
{आरुरोह विमानं तं पुष्पकं हेमभूषितम्}% १००

{॥इति श्रीपाद्मपुराणे प्रथमे सृष्टिखण्डे शूद्रतापसवधो नाम पञ्चत्रिंशोऽध्यायः॥३५॥}

    \sect{षट्त्रिंशोऽध्यायः}

\uvacha{पुलस्त्य उवाच}

\twolineshloka
{ततो देवाः प्रयातास्ते विमानैर्बहुभिस्तदा}
{रामोप्यनुजगामाशु कुम्भयोनेस्तपोवनम्}% १

\twolineshloka
{उक्तं भगवता तेन भूयोप्यागमनं क्रियाः}
{पूर्वमेव सभायां च यो मां द्रष्टुं समागतः}% २

\twolineshloka
{तदहं देवतादेशात्तत्कार्यार्थे महामुनिम्}
{पश्यामि तं मुनिं गत्वा देवदानवपूजितम्}% ३

\twolineshloka
{उपदेशं च मे तुष्टः स्वयं दास्यति सत्तमः}
{दुःखी येन पुनर्मर्त्ये न भवामि कदाचन}% ४

\twolineshloka
{पिता दशरथो मह्यं कौसल्या जननी तथा}
{सूर्यवंशे समुत्पन्नस्तथाप्येवं सुदुःखितः}% ५

\twolineshloka
{राज्यकाले वने वासो भार्यया चानुजेन च}
{हरणं चापि भार्याया रावणेन कृतं मम}% ६

\twolineshloka
{असहायेन तु मया तीर्त्वा सागरमुत्तमम्}
{रुद्ध्वा तु तां पुरीं सर्वां कृत्वा तस्य कुलक्षयम्}% ७

\twolineshloka
{दृष्टा सीता मया त्यक्ता देवानां तु पुरस्तदा}
{शुद्धां तां मां तथोचुस्ते मया सीता तथा गृहम्}% ८

\twolineshloka
{समानीता प्रीतिमता लोकवाक्याद्विसर्जिता}
{वने वसति सा देवी पुरे चाहं वसामि वै}% ९

\twolineshloka
{जातोहमुत्तमे वंशे उत्तमोहं धनुष्मताम्}
{उत्तमं दुःखमापन्नो हृदयं नैव भिद्यते}% १०

\twolineshloka
{वज्रसारस्य सारेण धात्राहं निर्मितो ध्रुवम्}
{इदानीं ब्राह्मणादेशाद्भ्रमामि धरणीतले}% ११

\twolineshloka
{तपः स्थितस्तु शूद्रोसौ मया पापो निपातितः}
{देववाक्यात्तु मे भूयः प्राणो मे हृदि संस्थितः}% १२

\twolineshloka
{पश्यामि तं मुनिं वन्द्यं जगतोस्य हिते रतम्}
{दृष्टेन मे तथा दुःखं नाशमेष्यति सत्वरम्}% १३

\twolineshloka
{उदयेन सहस्रांशोर्हिमं यद्वद्विलीयते}
{तद्वन्मे दुःखसम्प्राप्तिः सर्वथा नाशमेष्यति}% १४

\twolineshloka
{दृष्ट्वा च देवान्सम्प्राप्तानगस्त्यो भगवानृषिः}
{अर्घ्यमादाय सुप्रीतः सर्वांस्तानभ्यपूजयत्}% १५

\twolineshloka
{ते तु गृह्य ततः पूजां सम्भाष्य च महामुनिम्}
{जग्मुस्तेन तदा हृष्टा नाकपृष्ठं सहानुगाः}% १६

\twolineshloka
{गतेषु तेषु काकुत्स्थः पुष्पकादवरुह्य च}
{अभिवादयितुं प्राप्तः सोगस्त्यमृषिमुत्तमम्}% १७

\uvacha{राजोवाच}

\twolineshloka
{सुतो दशरथस्याहं भवन्तमभिवादितुम्}
{आगतो वै मुनिश्रेष्ठ सौम्येनेक्षस्व चक्षुषा}% १८

\twolineshloka
{निर्धूतपापस्त्वां दृष्ट्वा भवामीह न संशयः}
{एतावदुक्त्वा स मुनिमभिवाद्य पुनः पुनः}% १९

\twolineshloka
{कुशलं भृत्यवर्गस्य मृगाणां तनयस्य च}
{भगवद्दर्शनाकाङ्क्षी शूद्रं हत्वा त्विहागतः}% २०

\uvacha{अगस्त्य उवाच}

\twolineshloka
{स्वागतं ते रघुश्रेष्ठ जगद्वन्द्य सनातन}
{दर्शनात्तव काकुत्स्थ पूतोहं मुनिभिः सह}% २१

\twolineshloka
{त्वत्कृते रघुशार्दूल गृहाणार्घं महाद्युते}
{स्वागतं नरशार्दूल दिष्ट्या प्राप्तोसि शत्रुहन्}% २२

\twolineshloka
{त्वं हि नित्यं बहुमतो गुणैर्बहुभिरुत्तमैः}
{अतस्त्वं पूजनीयो वै मम नित्यं हृदिस्थितः}% २३

\twolineshloka
{सुरा हि कथयन्ति त्वां शूद्रघातिनमागतम्}
{ब्राह्मणस्य च धर्मेण त्वया वै जीवितः सुतः}% २४

\twolineshloka
{उष्यतां चेह भगवः सकाशे मम राघव}
{प्रभाते पुष्पकेणासि गन्तायोध्यां महामते}% २५

\twolineshloka
{इदं चाभरणं सौम्य सुकृतं विश्वकर्मणा}
{दिव्यं दिव्येनवपुषा दीप्यमानं स्वतेजसा}% २६

\twolineshloka
{प्रतिगृह्णीष्व राजेन्द्र मत्प्रियं कुरु राघव}
{लब्धस्य हि पुनर्द्दाने सुमहत्फलमुच्यते}% २७

\twolineshloka
{त्वं हि शक्तः परित्रातुं सेन्द्रानपि सुरोत्तमान्}
{तस्मात्प्रदास्ये विधिवत्प्रतीच्छस्व नरर्षभ}% २८

\twolineshloka
{अथोवाच महाबाहुरिक्ष्वाकूणां महारथः}
{कृताञ्जलिर्मुनिश्रेष्ठं स्वं च धर्ममनुस्मरन्}% २९

\twolineshloka
{प्रतिग्रहो वै भगवंस्तव मेऽत्र विगर्हितः}
{क्षत्रियेण कथं विप्र प्रतिग्राह्यं विजानता}% ३०

\twolineshloka
{ब्राह्मणेन तु यद्दत्तं तन्मे त्वं वक्तुमर्हसि}
{सपुत्रो गृहवानस्मि समर्थोस्मि महामुने}% ३१

\twolineshloka
{आपदा चन चाक्रान्तः कथं ग्राह्यः प्रतिग्रहः}
{भार्या मे सुचिरं नष्टा न चान्या मम विद्यते}% ३२

\twolineshloka
{केवलं दोषभागी च भवामीह न संशयः}
{कष्टां चैव दशां प्राप्य क्षत्रियोपि प्रतिग्रही}% ३३

\twolineshloka
{कुर्वन्न दोषमाप्नोति मनुरेवात्र कारणम्}
{वृद्धौ च मातापितरौ साध्वी भार्या शिशुः सुतः}% ३४

\twolineshloka
{अप्यकार्यशतं कृत्वा भर्तव्या मनुरब्रवीत्}
{नाहं प्रतीच्छे विप्रर्षे त्वया दत्तं प्रतिग्रहम्}% ३५

\onelineshloka
{न च मे भवता कोपः कार्यो वै सुरपूजित}% ३६

\uvacha{अगस्त्य उवाच}

\twolineshloka
{न च प्रतिग्रहे दोषो गृहीते पार्थिवैर्नृप}
{भवान्वै तारणे शक्तस्त्रैलोक्यस्यापि राघव}% ३७

\twolineshloka
{तारय ब्राह्मणं राम विशेषेण तपस्विनम्}
{तस्मात्प्रदास्ये विधिवत्प्रतीच्छस्व नराधिप}% ३८

\uvacha{राम उवाच}

\twolineshloka
{क्षत्रियेण कथं विप्र प्रतिग्राह्यं विजानता}
{ब्राह्मणेन तु यद्दत्तं तन्मे त्वं वक्तुमर्हसि}% ३९

\uvacha{अगस्त्य उवाच}

\twolineshloka
{आसीत्कृतयुगे राम ब्रह्मपूते पुरातने}
{अपार्थिवाः प्रजाः सर्वाः सुराणां च शतक्रतुः}% ४०

\twolineshloka
{ताः प्रजा देवदेवेशं राजार्थं समुपागमन्}
{सुराणां विद्यते राजा देवदेवः शतक्रतुः}% ४१

\twolineshloka
{श्रेयसेस्मासु लोकेश पार्थिवं कुरु साम्प्रतम्}
{यस्मिन्पूजां प्रयुञ्जानाः पुरुषा भुञ्जते महीम्}% ४२

\twolineshloka
{ततो ब्रह्मा सुरश्रेष्ठो लोकपालान्सवासवान्}
{समाहूयाब्रवीत्सर्वांस्तेजोभागोऽत्र युज्यताम्}% ४३

\twolineshloka
{ततो ददुर्लोकपालाश्चतुर्भागं स्वतेजसा}
{अक्षयश्च ततो ब्रह्मा यतो जातोऽक्षयो नृपः}% ४४

\twolineshloka
{तं ब्रह्मा लोकपालानामंशं पुंसामयोजयत्}
{ततो नृपस्तदा तासां प्रजानां क्षेमपण्डितः}% ४५

\twolineshloka
{तत्रैन्द्रेण तु भागेन सर्वानाज्ञापयेन्नृपः}
{वारुणेन च भागेन सर्वान्पुष्णाति देहिनः}% ४६

\twolineshloka
{कौबेरेण तथांशेन त्वर्थान्दिशति पार्थिवः}
{यश्च याम्यो नृपे भागस्तेन शास्ति च वै प्रजाः}% ४७

\twolineshloka
{तत्र चैन्द्रेण भागेन नरेन्द्रोसि रघूत्तम}
{प्रतिगृह्णीष्वाभरणं तारणार्थे मम प्रभो}% ४८

\twolineshloka
{ततो रामः प्रजग्राह मुनेर्हस्तान्महात्मनः}
{दिव्यमाभरणं चित्रं प्रदीप्तमिव भास्करम्}% ४९

\twolineshloka
{प्रतिगृह्य ततोगस्त्याद्राघवः परवीरहा}
{निरीक्ष्य सुचिरं कालं विचार्य च पुनः पुनः1.36.}% ५०

\twolineshloka
{मौक्तिकानि विचित्राणि धात्रीफलसमानि च}
{जाम्बूनदनिबद्धानि वज्रविद्रुमनीलकैः}% ५१

\twolineshloka
{पद्मरागैः सगोमेधैर्वैडूर्यैः पुष्परागकैः}
{सुनिबद्धं सुविभक्तं सुकृतं विश्वकर्मणा}% ५२

\twolineshloka
{दृष्ट्वा प्रीतिसमायुक्तो भूयश्चेदं व्यचिन्तयत्}
{नेदृशानि च रत्नानि मया दृष्टानि कानिचित्}% ५३

\twolineshloka
{उपशोभानि बद्धानि पृथ्वीमूल्यसमानि च}
{विभीषणस्य लङ्कायां न दृष्टानि मया पुरा}% ५४

\twolineshloka
{इति सञ्चित्य मनसा राघवस्तमृषिं पुनः}
{आगमं तस्य दिव्यस्य प्रष्टुं समुपचक्रमे}% ५५

\twolineshloka
{अत्यद्भुतमिदं ब्रह्मन्न प्राप्यं च महीक्षिताम्}
{कथं भगवता प्राप्तं कुतो वा केन निर्मितम्}% ५६

\twolineshloka
{कुतूहलवशाच्चैव पृच्छामि त्वां महामते}
{करतलेस्थिते रत्ने करमध्यं प्रकाशते}% ५७

\twolineshloka
{अधमं तद्विजानीयात्सर्वशास्त्रेषु गर्हितम्}
{दिशः प्रकाशयेद्यत्तन्मध्यमं मुनिसत्तम}% ५८

\twolineshloka
{ऊर्ध्वगं त्रिशिखं यत्स्यादुत्तमं तदुदाहृतम्}
{एतान्युत्तमजातीनि ऋषिभिः कीर्तितानि तु}% ५९

\twolineshloka
{आश्चर्याणां बहूनां हि दिव्यानां भगवान्निधिः}
{एवं वदति काकुत्स्थे मुनिर्वाक्यमथाब्रवीत्}% ६०

\uvacha{अगस्त्य उवाच}

\twolineshloka
{शृणु राम पुरावृत्तं पुरा त्रेतायुगे महत्}
{द्वापरे समनुप्राप्ते वने यद्दृष्टवानहम्}% ६१

\twolineshloka
{आश्चर्यं सुमहाबाहो निबोध रघुनन्दन}
{पुरा त्रेतायुगे ह्यासीदरण्यं बहुविस्तरम्}% ६२

\twolineshloka
{समन्ताद्योजनशतं मृगव्याघ्रविवर्जितम्}
{तस्मिन्निष्पुरुषेऽरण्ये चिकीर्षुस्तप उत्तमम्}% ६३

\twolineshloka
{अहमाक्रमितुं सौम्य तदरण्यमुपागतः}
{तस्यारण्यस्य मध्यं तु युक्तं मूलफलैः सदा}% ६४

\twolineshloka
{शाकैर्बहुविधाकारैर्नानारूपैः सुकाननैः}
{तस्यारण्यस्य मध्ये तु पञ्चयोजनमायतम्}% ६५

\twolineshloka
{हंसकारण्डवाकीर्णं चक्रवाकोपशोभितम्}
{तत्राश्चर्यं मया दृष्टं सरः परमशोभितम्}% ६६

\twolineshloka
{विसारिकच्छपाकीर्णं बकपङ्क्तिगणैर्युतम्}
{समीपे तस्य सरसस्तपस्तप्तुं गतः पुरा}% ६७

\twolineshloka
{देशं पुण्यमुपेत्यैवं सर्वहिंसाविवर्जितम्}
{तत्राहमवसं रात्रिं नैदाघीं पुरुषर्षभ}% ६८

\twolineshloka
{प्रभाते पुरुत्थाय सरस्तदुपचक्रमे}
{अथापश्यं शवमहमस्पृष्टजरसं क्वचित्}% ६९

\twolineshloka
{तिष्ठन्तं परया लक्ष्म्या सरसो नातिदूरतः}
{तदर्थं चिन्तयानोहं मुहूर्तमिव राघव}% ७०

\twolineshloka
{अस्य तीरे न वै प्राणी को वाप्येष सुरर्षभः}
{मुनिर्वा पार्थिवो वापि क्व मुनिः पार्थिवोपि वा}% ७१

\twolineshloka
{अथवा पार्थिवसुतस्तस्यैवं सम्भवः कृतः}
{अतीतेहनि रात्रौ वा प्रातर्वापि मृतो यदि}% ७२

\twolineshloka
{अवश्यं तु मया ज्ञेया सरसोस्य विनिष्क्रिया}
{यावदेवं स्थितश्चाहं चिन्तयानो रघूत्तम}% ७३

\twolineshloka
{अथापश्यं मूहूर्तात्तु दिव्यमद्भुतदर्शनम्}
{विमानं परमोदारं हंसयुक्तं मनोजवम्}% ७४

\twolineshloka
{पुरस्तत्र सहस्रं तु विमानेप्सरसां नृप}
{गन्धर्वाश्चैव तत्सङ्ख्या रमयन्ति वरं नरम्}% ७५

\twolineshloka
{गायन्ति दिव्यगेयानि वादयन्ति तथा परे}
{अथापश्यं नरं तस्माद्विमानादवरोहितम्}% ७६

\twolineshloka
{शवमांसं भक्षयन्तं च स्नात्वा रघुकुलोद्वह}
{ततो भुक्त्वा यथाकामं स मांसं बहुपीवरम्}% ७७

\twolineshloka
{अवतीर्य सरः शीघ्रमारुरोह दिवं पुनः}
{तमहं देवसङ्काशं श्रिया परमयान्वितम्}% ७८

\twolineshloka
{भो भो स्वर्गिन्महाभाग पृच्छामि त्वां कथं त्विदम्}
{जुगुप्सितस्तवाहारो गतिश्चेयं तवोत्तमा}% ७९

\twolineshloka
{यदि गुह्यं न चैतत्ते कथय त्वद्य मे भवान्}
{कामतः श्रोतुमिच्छामि किमेतत्परमं वचः}% ८०

\twolineshloka
{को भवान्वद सन्देहमाहारश्च विगर्हितः}
{त्वयेदं भुज्यते सौम्य किमर्थं क्व च वर्तसे}% ८१

\twolineshloka
{कस्यायमैश्वरोभावः शवत्वेन विनिर्मितः}
{आहारं च कथं निन्द्यं श्रोतुमिच्छामि तत्त्वतः}% ८२

\twolineshloka
{श्रुत्वा च भाषितं तत्र मम राम सतां वर}
{प्राञ्जलिः प्रत्युवाचेदं स स्वर्गी रघुनन्दन}% ८३

\twolineshloka
{शृणुष्वाद्य यथावृत्तं ममेदं सुखदुःखजम्}
{कामो हि दुरितक्रम्यः शृणु यत्पृच्छसे द्विज}% ८४

\twolineshloka
{पुरा वैदर्भको राजा पिता मे हि महायशाः}
{वासुदेव इति ख्यातस्त्रिषु लोकेषु धार्मिकः}% ८५

\twolineshloka
{तस्य पुत्रद्वयं ब्रह्मन्द्वाभ्यां स्त्रीभ्यामजायत}
{अहं श्वेत इति ख्यातो यवीयान्सुरथोऽभवत्}% ८६

\twolineshloka
{पितर्युपरते तस्मिन्पौरा मामभ्यषेचयन्}
{तत्राहङ्कारयन्राज्यं धर्मे चासं समाहितः}% ८७

\twolineshloka
{एवं वर्षसहस्राणि बहूनि समुपाव्रजन्}
{मम राज्यं कारयतः परिपालयतः प्रजाः}% ८८

\twolineshloka
{सोहं निमित्ते कस्मिंश्चिद्वैराग्येण द्विजोत्तम}
{मरणं हृदये कृत्वा तपोवनमुपागमम्}% ८९

\twolineshloka
{सोहं वनमिदं रम्यं भृशं पक्षिविवर्जितम्}
{प्रविष्टस्तप आस्थातुमस्यैव सरसोन्तिके}% ९०

\twolineshloka
{राज्येऽभिषिच्य सुरथं भ्रातरं तं नराधिपम्}
{इदं सरः समासाद्य तपस्तप्तं सुदारुणम्}% ९१

\twolineshloka
{दशवर्षसहस्राणि तपस्तप्त्वा महावने}
{शुभं तु भवनं प्राप्तो ब्रह्मलोकमनामयम्}% ९२

\twolineshloka
{स्वर्गस्थमपि मां ब्रह्मन्क्षुत्पिपासे द्विजोत्तम}
{अबाधेतां भृशं चाहमभवं व्यथितेन्द्रियः}% ९३

\twolineshloka
{ततस्त्रिभुवनश्रेष्ठमवोचं वै पितामहम्}
{भगवन्स्वर्गलोकोऽयं क्षुत्पिपासा विवर्जितः}% ९४

\twolineshloka
{कस्येयं कर्मणः पक्तिः क्षुत्पिपासे यतो हि मे}
{आहारः कश्च मे देव ब्रूहि त्वं श्रीपितामह}% ९५

\twolineshloka
{ततः पितामहः सम्यक्चिरं ध्यात्वा महामुने}
{मामुवाच ततो वाक्यं नास्ति भोज्यं स्वदेहजम्}% ९६

\twolineshloka
{ॠते ते स्वानि मांसानि भक्षय त्वं तु हि नित्यशः}
{स्वशरीरं त्वया पुष्टं कुर्वता तप उत्तमम्}% ९७

\twolineshloka
{नादत्तं जायते तात श्वेत पश्य महीतले}
{आग्रहाद्भिक्षमाणाय भिक्षापि प्राणिने पुरा}% ९८

\twolineshloka
{न हि दत्ता गृहे भ्रान्त्या मोहादतिथये तदा}
{तेन स्वर्गगतस्यापि क्षुत्पिपासे तवाधुना}% ९९

\twolineshloka
{स त्वं प्रपुष्टमाहारैः स्वशरीरमनुत्तमम्}
{भक्षयस्व च राजेन्द्र सा ते तृप्तिर्भविष्यति1.36.}% १००

\twolineshloka
{एवमुक्तस्ततो देवं ब्रह्माणमहमुक्तवान्}
{भक्षिते च स्वके देहे पुनरन्यन्न मे विभो}% १०१

\twolineshloka
{क्षुधानिवारणं नैव देहस्यास्य विनौदनम्}
{खादामि ह्यक्षयं देव प्रियं मे न हि जायते}% १०२

\twolineshloka
{ततोब्रवीत्पुनर्ब्रह्मा तव देहोऽक्षयः कृतः}
{दिनेदिने ते पुष्टात्मा शवः श्वेत भविष्यति}% १०३

\twolineshloka
{यावद्वर्षशतं पूर्णं स्वमांसं खाद भो नृप}
{यदागच्छति चागस्त्यः श्वेतारण्यं महातपाः}% १०४

\twolineshloka
{भगवानतिदुर्धर्षस्तदा कृच्छ्राद्विमोक्ष्यसे}
{स हि तारयितुं शक्तः सेन्द्रानपि सुरासुरान्}% १०५

\twolineshloka
{आहारं कुत्सितं चेमं राजर्षे किं पुनस्तव}
{सुरकार्यं महत्तेन सुकृतं तु महात्मना}% १०६

\twolineshloka
{उदधिं निर्जलं कृत्वा दानवाश्च निपातिताः}
{विन्ध्यश्चादित्यविद्वेषाद्वर्धमानो निवारितः}% १०७

\twolineshloka
{लम्बमाना मही चैषा गुरुत्वेनाधिवासिता}
{दक्षिणा दिग्दिवं याता त्रैलाक्यं विषमस्थितम्}% १०८

\twolineshloka
{मया गत्वा सुरैः सार्द्धं प्रेषितो दक्षिणां दिशम्}
{समां कुरु महाभाग गुरुत्वेन जगत्समम्}% १०९

\twolineshloka
{एवं च तेन मुनिना स्थित्वा सर्वा धरा समा}
{कृता राजेन्द्र मुनिना एवमद्यापि दृश्यते}% ११०

\twolineshloka
{सोहं भगवत श्रुत्वा देवदेवस्य भाषितम्}
{भुञ्जे च कुत्सिताहारं स्वशरीरमनुत्तमम्}% १११

\twolineshloka
{पूर्णं वर्षशतं चाद्य भोजनं कुत्सितं च मे}
{क्षयं नाभ्येति तद्विप्र तृप्तिश्चापि ममोत्तमा}% ११२

\twolineshloka
{तं मुनिं कृच्छ्रसन्तप्तश्चिन्तयामि दिवानिशम्}
{कदा वै दर्शनं मह्यं स मुनिर्दास्यते वने}% ११३

\twolineshloka
{एवं मे चिन्तयानस्य गतं वर्षशतन्त्विह}
{सोगस्त्यो हि गतिर्ब्रह्मन्मुनिर्मे भविता ध्रुवम्}% ११४

\twolineshloka
{न गतिर्भविता मह्यं कुम्भयोनिमृते द्विजम्}
{श्रुत्वेत्थं भाषितं राम दृष्ट्वाहारं च कुत्सितम्}% ११५

\twolineshloka
{कृपया परया युक्तस्तं नृपं स्वर्गगामिनम्}
{करोम्यहं सुधाभोज्यं नाशयामि च कुत्सितम्}% ११६

\twolineshloka
{चिन्तयन्नित्यवोचं तमगस्त्यः किं करिष्यति}
{अहमेतत्कुत्सितं ते नाशयामि महामते}% ११७

\twolineshloka
{ईप्सितं प्रार्थयस्वास्मान्मनः प्रीतिकरं परम्}
{स स्वर्गी मां ततः प्राह कथं ब्रह्मवचोन्यथा}% ११८

\twolineshloka
{कर्तुं मुने मया शक्यं न चान्यस्तारयिष्यति}
{ॠते वै कुम्भयोनिं तं मैत्रावरुणसम्भवम्}% ११९

\twolineshloka
{अपृष्टोपि मया ब्रह्मन्नेवमूचे पितामहः}
{एवं ब्रुवाणं तं श्वेतमुक्तवानहमस्मि सः}% १२०

\twolineshloka
{आगतस्तव भाग्येन दृष्टोहं नात्र संशयः}
{ततः स्वर्गी स मां ज्ञात्वा दण्डवत्पतितो भुवि}% १२१

\onelineshloka*
{तमुत्थाप्य ततो रामाब्रवं किं ते करोम्यहम्}

\uvacha{राजोवाच}

\onelineshloka
{आहारात्कुत्सिताद्ब्रह्मंस्तारयस्वाद्य दुष्कृतात्}% १२२

\twolineshloka
{येन लोकोऽक्षयः स्वर्गो भविता त्वत्कृतेन मे}
{ततः प्रतिग्रहो दत्तो जगद्वन्द्य नृपेण हि}% १२३

%%% CHECK FOR MISSING SHLOKAS!?

\onelineshloka
{भवान्मामनुगृह्णातु प्रतीच्छस्व प्रतिग्रहम्}% १२७

\twolineshloka
{कृता मतिस्तारणाय न लोभाद्रघुनन्दन}
{गृहीते भूषणे राम मम हस्तगते तदा}% १२८

\twolineshloka
{मानुषः पौर्विको देहस्तदा नष्टोस्य भूपते}
{प्रणष्टे तु शरीरे च राजर्षिः परया मुदा}% १२९

\twolineshloka
{मयोक्तोसौ विमानेन जगाम त्रिदिवं पुनः}
{तेन मे शक्रतुल्येन दत्तमाभरणं शुभम्}% १३०

\twolineshloka
{तस्मिन्निमित्ते काकुत्स्थ दत्तमद्भुतकर्मणा}
{श्वेतो वैदर्भको राजा तदाभूद्गतकल्मषः}% १३१

{॥इति श्रीपाद्मपुराणे प्रथमे सृष्टिखण्डे रामागस्त्यसंवादो नाम षट्त्रिंशोऽध्यायः॥३६॥}

    \sect{सप्तत्रिंशोऽध्यायः --- यज्ञनिवारणम्}

\src{पद्म-पुराणम्}{सृष्टिखण्डम्}{अध्यायः ३७}{१--१७१}
% \tags{concise, complete}
\notes{This chapter describes Rāma’s Abstaining from the Performance of Rājasūya yajna.}
\textlink{https://sa.wikisource.org/wiki/पद्मपुराणम्/खण्डः_१_(सृष्टिखण्डम्)/अध्यायः_३७}
\translink{https://www.wisdomlib.org/hinduism/book/the-padma-purana/d/doc364160.html}

\storymeta


\uvacha{पुलस्त्य उवाच}

\twolineshloka
{तदद्भुततमं वाक्यं श्रुत्वा च रघुनन्दनः}
{गौरवाद्विस्मयाच्चापि भूयः प्रष्टुं प्रचक्रमे}% १

\uvacha{राम उवाच}

\twolineshloka
{भगवंस्तद्वनं घोरं यत्रासौ तप्तवांस्तपः}
{श्वेतो वैदर्भको राजा तदद्भुतमभूत्कथम्}% २

\twolineshloka
{विषमं तद्वनं राजा शून्यं मृगविवर्जितम्}
{प्रविष्टस्तप आस्थातुं कथं वद महामुने}% ३

\twolineshloka
{समन्ताद्योजनशतं निर्मनुष्यमभूत्कथम्}
{भवान्कथं प्रविष्टस्तद्येन कार्येण तद्वद}% ४

\uvacha{अगस्त्य उवाच}

\twolineshloka
{पुरा कृतयुगे राजा मनुर्दण्डधरः प्रभुः}
{तस्य पुत्रोथ नाम्नासीदिक्ष्वाकुरमितद्युतिः}% ५

\twolineshloka
{तं पुत्रं पूर्वजं राज्ये निक्षिप्य भुविसम्मतम्}
{पृथिव्यां राजवंशानां भव राजेत्युवाच ह}% ६

\twolineshloka
{तथेति च प्रतिज्ञातं पितुः पुत्रेण राघव}
{ततःपरमसंहृष्टः पुनस्तं प्रत्यभाषत}% ७

\twolineshloka
{प्रीतोस्मि परमोदार कर्मणा ते न संशयः}
{दण्डेन च प्रजा रक्ष न च दण्डमकारणम्}% ८

\twolineshloka
{अपराधिषु यो दण्डः पात्यते मानवैरिह}
{स दण्डो विधिवन्मुक्तः स्वर्गं नयति पार्थिवम्}% ९

\twolineshloka
{तस्माद्दण्डे महाबाहो यत्नवान्भव पुत्रक}
{धर्मस्ते परमो लोके कृत एवं भविष्यति}% १०

\twolineshloka
{इति तं बहुसन्दिश्य मनुः पुत्रं समाधिना}
{जगाम त्रिदिवं हृष्टो ब्रह्मलोकमनुत्तमम्}% ११

\twolineshloka
{जनयिष्ये कथं पुत्रानिति चिन्तापरोऽभवत्}
{कर्मभिर्बहुभिस्तैस्तैस्ससुतैस्संयुतोऽभवत्}% १२

\twolineshloka
{तोषयामास पुत्रैस्स पितॄन्देवसुतोपमैः}
{सर्वेषामुत्तमस्तेषां कनीयान्रघुनन्दन}% १३

\twolineshloka
{शूरश्च कृतविद्यश्च गुरुश्च जनपूजया}
{नाम तस्याथ दण्डेति पिता चक्रे स बुद्धिमान्}% १४

\twolineshloka
{भविष्यद्दण्डपतनं शरीरे तस्य वीक्ष्य च}
{सम्पश्यमानस्तं दोषं घोरं पुत्रस्य राघव}% १५

\twolineshloka
{स विन्ध्यनीलयोर्मध्ये राज्यमस्य ददौ प्रभुः}
{स दण्डस्तत्र राजाभूद्रम्ये पर्वतमूर्द्धनि}% १६

\twolineshloka
{पुरं चाप्रतिमं तेन निवेशाय तथा कृतम्}
{नाम तस्य पुरस्याथ मधुमत्तमिति स्वयम्}% १७

\twolineshloka
{तथादेशेन सम्पन्नः शूरो वासमथाकरोत्}
{एवं राजा स तद्राज्यं चकार सपुरोहितः}% १८

\twolineshloka
{प्रहृष्ट सुप्रजाकीर्णं देवराजो यथा दिवि}
{ततः स दण्डः काकुत्स्थ बहुवर्षगणायुतम्}% १९

\twolineshloka
{अकारयत्तु धर्मात्मा राज्यं निहतकण्टकम्}
{अथ काले तु कस्मिंश्चिद्राजा भार्गवमाश्रमम्}% २०

\twolineshloka
{रमणीयमुपाक्रामच्चैत्रमासे मनोरमे}
{तत्र भार्गवकन्यां तु रूपेणाप्रतिमां भुवि}% २१

\twolineshloka
{विचरन्तीं वनोद्देशे दण्डोऽपश्यदनुत्तमाम्}
{उत्तुङ्गपीवरीं श्यामां चन्द्राभवदनां शुभाम्}% २२

\twolineshloka
{सुनासां चारुसर्वाङ्गीं पीनोन्नतपयोधराम्}
{मध्ये क्षामां च विस्तीर्णां दृष्ट्वा तां कुरुते मुदम्}% २३

\twolineshloka
{एकवस्त्रां वने चैकां प्रथमे यौवने स्थिताम्}
{स तां दृष्ट्वात्वधर्मेण अनङ्गशरपीडितः}% २४

\twolineshloka
{अभिगम्य सुविश्रान्तां कन्यां वचनमब्रवीत्}
{कुतस्त्वमसि सुश्रोणि कस्य चासि सुशोभने}% २५

\twolineshloka
{पीडतोहमनङ्गेन पृच्छामि त्वां सुशोभने}
{त्वया मेऽपहृतं चित्तं दर्शनादेव सुन्दरि}% २६

\twolineshloka
{इदं ते वदनं रम्यं मुनीनां चित्तहारकम्}
{यद्यहं न लभे भोक्तुं मृतं मामवधारय}% २७

\twolineshloka
{त्वया हृता मम प्राणा मां जीवय सुलोचने}
{दासोस्मि ते वरारोहे भक्तं मां भज शोभने}% २८

\twolineshloka
{तस्यैवं तु ब्रुवाणस्य मदोन्मत्तस्य कामिनः}
{भार्गवी प्रत्युवाचेदं वचः सविनयं नृपम्}% २९

\twolineshloka
{भार्गवस्य सुतां विद्धि शुक्रस्याक्लिष्टकर्मणः}
{अरजां नाम राजेन्द्र ज्येष्ठामाश्रमवासिनः}% ३०

\twolineshloka
{शुक्रः पिता मे राजेन्द्र त्वं च शिष्यो महात्मनः}
{धर्मतो भगिनी चाहं भवामि नृपनन्दन}% ३१

\twolineshloka
{एवंविधं वचो वक्तुं न त्वमर्हसि पार्थिव}
{अन्येभ्योपि सुदुष्टेभ्यो रक्ष्या चाहं सदा त्वया}% ३२

\twolineshloka
{क्रोधनो मे पिता रौद्रो भस्मत्वं त्वां समानयेत्}
{अथवा राजधर्मेणासम्बन्धं कुरुषे बलात्}% ३३

\twolineshloka
{पितरं याचयस्व त्वं धर्मदृष्टेन कर्मणा}
{वरयस्व नृपश्रेष्ठ पितरं मे महाद्युतिम्}% ३४

\twolineshloka
{अन्यथा विपुलं दुःखं तव घोरं भवेद्ध्रुवम्}
{क्रुद्धो हि मे पिता सर्वं त्रैलोक्यमभिनिर्दहेत्}% ३५

\twolineshloka
{ततोऽशुभं महाघोरं श्रुत्वा दण्डः सुदारुणम्}
{प्रत्युवाच मदोन्मत्तः शिरसाभिनतः पुनः}% ३६

\twolineshloka
{प्रसादं कुरु सुश्रोणि कामोन्मत्तस्य कामिनि}
{त्वया रुद्धा मम प्राणा विशीर्यन्ति शुभानने}% ३७

\twolineshloka
{त्वां प्राप्य वैरं मेऽत्रास्तु वधो वापि महत्तरः}
{भक्तं भजस्व मां भीरु त्वयि भक्तिर्हि मे परा}% ३८

\twolineshloka
{एवमुक्त्वा तु तां कन्यां बलात्सङ्गृह्य बाहुना}
{अन्येन राज्ञा हस्तेन विवस्त्रा सा तथा कृता}% ३९

\twolineshloka
{अङ्गमङ्गे समाश्लेष्य मुखे चैव मुखं कृतम्}
{विस्फुरन्तीं यथाकामं मैथुनायोपचक्रमे}% ४०

\twolineshloka
{तमनर्थं महाघोरं दण्डः कृत्वा सुदारुणम्}
{नगरं स्वं जगामाशु मदोन्मत्त इव द्विपः}% ४१

\twolineshloka
{भार्गवी रुदती दीना आश्रमस्याविदूरतः}
{प्रत्यपालयदुद्विग्ना पितरं देवसम्मितम्}% ४२

\twolineshloka
{स मुहूर्तादुपस्पृश्य देवर्षिरमितद्युतिः}
{स्वमाश्रमं शिष्यवृतं क्षुधार्तः सन्यवर्तत}% ४३

\twolineshloka
{सोपश्यदरजां दीनां रजसा समभिप्लुताम्}
{चन्द्रस्य घनसंयुक्तां ज्योत्स्नामिव पराजिताम्}% ४४

\twolineshloka
{तस्य रोषः समभवत्क्षुधार्तस्य महात्मनः}
{निर्दहन्निव लोकांस्त्रींस्तान्शिष्यान्समुवाच ह}% ४५

\twolineshloka
{पश्यध्वं विपरीतस्य दण्डस्यादीर्घदर्शिनः}
{विपत्तिं घोरसङ्काशां दीप्तामग्निशिखामिव}% ४६

\twolineshloka
{यन्नाशं दुर्गतिं प्राप्तस्सानुगश्च न संशयः}
{यस्तु दीप्तहुताशस्य अर्चिः संस्पृष्टवानिह}% ४७

\twolineshloka
{यस्मात्स कृतवान्पापमीदृशं घोरसम्मितम्}
{तस्मात्प्राप्स्यति दुर्मेधाः पांसुवर्षमनुत्तमम्}% ४८

\twolineshloka
{कुराजा देशसंयुक्तः सभृत्यबलवाहनः}
{पापकर्मसमाचारो वधं प्राप्स्यति दुर्मतिः}% ४९

\twolineshloka
{समन्ताद्योजनशतं विषयं चास्य दुर्मतेः}
{धुनोतु पांसुवर्षेण महता पाकशासनः1.37.}% ५०

\twolineshloka
{सर्वसत्वानि यानीह जङ्गमस्थावराणि वै}
{सर्वेषां पांसुवर्षेण क्षयः क्षिप्रं भविष्यति}% ५१

\twolineshloka
{दण्डस्य विषयो यावत्तावत्सवनमाश्रमम्}
{पांसुवर्षमिवाकस्मात्सप्तरात्रं भविष्यति}% ५२

\twolineshloka
{इत्युक्त्वा क्रोधसन्तप्तस्तमाश्रमनिवासिनम्}
{जनं जनपदस्यान्ते स्थीयतामित्युवाच ह}% ५३

\twolineshloka
{उक्तमात्रे उशनसा आश्रमावसथो जनः}
{क्षिप्रं तु विषयात्तस्मात्स्थानं चक्रे च बाह्यतः}% ५४

\twolineshloka
{तं तथोक्त्वा मुनिजनमरजामिदमब्रवीत्}
{आश्रमे त्वं सुदुर्मेधे वस चेह समाहिता}% ५५

\twolineshloka
{इदं योजनपर्यन्तमाश्रमं रुचिरप्रभम्}
{अरजे विरजास्तिष्ठ कालमत्र समाश्शतम्}% ५६

\twolineshloka
{श्रुत्वा नियोगं विप्रर्षेररजा भार्गवी तदा}
{तथेति पितरं प्राह भार्गवं भृशदुःखिता}% ५७

\twolineshloka
{इत्युक्त्वा भार्गवो वासं तस्मादन्यमुपाक्रमत्}
{सप्ताहे भस्मसाद्भूतं यथोक्तं ब्रह्मवादिना}% ५८

\twolineshloka
{तस्माद्दण्डस्य विषयो विन्ध्यशैलस्य मानुष}
{शप्तो ह्युशनसा राम तदाभूद्धर्षणे कृते}% ५९

\twolineshloka
{ततःप्रभृति काकुत्स्थ दण्डकारण्यमुच्यते}
{एतत्ते सर्वमाख्यातं यन्मां पृच्छसि राघव}% ६०

\twolineshloka
{सन्ध्यामुपासितुं वीर समयो ह्यतिवर्तते}
{एते महर्षयो राम पूर्णकुम्भाः समन्ततः}% ६१

\twolineshloka
{कृतोदका नरव्याघ्र पूजयन्ति दिवाकरम्}
{सर्वैरॄषिभिरभ्यस्तैः स्तोत्रैर्ब्रह्मादिभिः कृतैः}% ६२

\twolineshloka
{रविरस्तङ्गतो राम गत्वोदकमुपस्पृश}
{ॠषेर्वचनमादाय रामः सन्ध्यामुपासितुम्}% ६३

\twolineshloka
{उपचक्राम तत्पुण्यं ससरोरघुनन्दनः}
{अथ तस्मिन्वनोद्देशे रम्ये पादपशोभिते}% ६४

\twolineshloka
{नदपुण्ये गिरिवरे कोकिलाशतमण्डिते}
{नानापक्षिरवोद्याने नानामृगसमाकुले}% ६५

\twolineshloka
{सिंहव्याघ्रसमाकीर्णे नानाद्विजसमावृते}
{गृध्रोलूकौ प्रवसितौ बहून्वर्षगणानपि}% ६६

\twolineshloka
{अथोलूकस्य भवनं गृध्रः पापविनिश्चयः}
{ममेदमिति कृत्वाऽसौ कलहं तेन चाकरोत्}% ६७

\twolineshloka
{राजा सर्वस्य लोकस्य रामो राजीवलोचनः}
{तं प्रपद्यावहै शीघ्रं कस्यैतद्भवनं भवेत्}% ६८

\twolineshloka
{गृध्रोलूकौ प्रपद्येतां जातकोपावमर्षिणौ}
{रामं प्रपद्यतौ शीघ्रं कलिव्याकुलचेतसौ}% ६९

\twolineshloka
{तौ परस्परविद्वेषौ स्पृशतश्चरणौ तथा}
{अथ दृष्ट्वा राघवेन्द्रं गृध्रो वचनमब्रवीत्}% ७०

\twolineshloka
{सुराणामसुराणां च त्वं प्रधानो मतो मम}
{बृहस्पतेश्च शुक्राच्च त्वं विशिष्टो महामतिः}% ७१

\twolineshloka
{परावरज्ञो भूतानां मर्त्ये शक्र इवापरः}
{दुर्निरीक्षो यथा सूर्यो हिमवानिव गौरवे}% ७२

\twolineshloka
{सागरश्चासि गाम्भीर्ये लोकपालो यमो ह्यसि}
{क्षान्त्या धरण्या तुल्योसि शीघ्रत्वे ह्यनिलोपमः}% ७३

\twolineshloka
{गुरुस्त्वं सर्वसम्पन्नो विष्णुरूपोसि राघव}
{अमर्षी दुर्जयो जेता सर्वास्त्रविधिपारगः}% ७४

\twolineshloka
{शृणु त्वं मम देवेश विज्ञाप्यं नरपुङ्गव}
{ममालयं पूर्वकृतं बाहुवीर्येण वै प्रभो}% ७५

\twolineshloka
{उलूको हरते राजंस्त्वत्समीपे विशेषतः}
{ईदृशोयं दुराचारस्त्वदाज्ञा लङ्घको नृप}% ७६

\twolineshloka
{प्राणान्तिकेन दण्डेन राम शासितुमर्हसि}
{एवमुक्ते तु गृध्रेण उलूको वाक्यमब्रवीत्}% ७७

\twolineshloka
{शृणु देव मम ज्ञाप्यमेकचित्तो नराधिप}
{सोमाच्छक्राच्च सूर्याच्च धनदाच्च यमात्तथा}% ७८

\twolineshloka
{जायते वै नृपो राम किञ्चिद्भवति मानुषः}
{त्वं तु सर्वमयो देवो नारायणपरायणः}% ७९

\twolineshloka
{प्रोच्यते सोमता राजन्सम्यक्कार्ये विचारिते}
{सम्यग्रक्षसि तापेभ्यस्तमोघ्नो हि यतो भवान्}% ८०

\twolineshloka
{दोषे दण्डात्प्रजानां त्वं यतः पापभयापहः}
{दाता प्रहर्ता गोप्ता च तेनेन्द्र इव नो भवान्}% ८१

\twolineshloka
{अधृष्यः सर्वभूतेषु तेजसा चानलो मतः}
{अभीक्ष्णं तपसे पापांस्तेन त्वं राम भास्करः}% ८२

\twolineshloka
{साक्षाद्वित्तेशतुल्यस्त्वमथवा धनदाधिकः}
{चित्तायत्ता तु पत्नीश्रीर्नित्यं ते राजसत्तम}% ८३

\twolineshloka
{धनदस्य तु कोशेन धनदस्तेन वैभवान्}
{समः सर्वेषु भूतेषु स्थावरेषु चरेषु च}% ८४

\twolineshloka
{शत्रौ मित्रे च ते दृष्टिः समन्ताद्याति राघव}
{धर्मेण शासनं नित्यं व्यवहारविधिक्रमैः}% ८५

\twolineshloka
{यस्य रुष्यसि वै राम मृत्युस्तस्याभिधीयते}
{गीयसे तेन वै राजन्यम इत्यभिविश्रुतः}% ८६

\twolineshloka
{यश्चासौ मानुषो भावो भवतो नृपसत्तम}
{आनृशंस्यपरो राजा सर्वेषु कृपयान्वितः}% ८७

\twolineshloka
{दुर्बलस्य त्वनाथस्य राजा भवति वै बलम्}
{अचक्षुषो भवेच्चक्षुरमतेषु मतिर्भवेत्}% ८८

\twolineshloka
{अस्माकमपि नाथस्त्वं श्रूयतां मम धार्मिक}
{भवता तत्र मन्तव्यं यथैते किल पक्षिणः}% ८९

\twolineshloka
{योस्मन्नाथः स पक्षीन्द्रो भवतो विनियोज्यकः}
{अस्वाम्यं देव नास्माकं सन्निधौ भवतः प्रभो}% ९०

\twolineshloka
{भवतैव कृतं पूर्वं भूतग्रामं चतुर्विधम्}
{ममालयप्रविष्टस्तु गृध्रो मां बाधते नृप}% ९१

\twolineshloka
{भवान्देवमनुष्येषु शास्ता वै नरपुङ्गव}
{एतच्छ्रुत्वा तु वै रामः सचिवानाह्वयत्स्वयम्}% ९२

\twolineshloka
{विष्टिर्जयन्तो विजयः सिद्धार्थो राष्ट्रवर्धनः}
{अशोको धर्मपालश्च सुमन्त्रश्च महाबलः}% ९३

\twolineshloka
{एते रामस्य सचिवा राज्ञो दशरथस्य च}
{नीतियुक्ता महात्मानः सर्वशास्त्रविशारदाः}% ९४

\twolineshloka
{सुशान्ताश्च कुलीनाश्च नये मन्त्रे च कोविदाः}
{तानाहूय स धर्मात्मा पुष्पकादवरुह्य च}% ९५

\twolineshloka
{गृध्रोलूकौ विवदन्तौ पृच्छति स्म रघूत्तमः}
{कति वर्षाणि भो गृध्र तवेदं निलयं कृतम्}% ९६

\twolineshloka
{एतन्मे कौतुकं ब्रूहि यदि जानासि तत्त्वतः}
{एतच्छ्रुत्वा वचो गृध्रो बभाषे राघवं स्थितम्}% ९७

\twolineshloka
{इयं वसुमती राम मानुषैर्बहुबाहुभिः}
{उच्छ्रितैराचिता सर्वा तदाप्रभृति मद्गृहम्}% ९८

\twolineshloka
{उलूकस्त्वब्रवीद्रामं पादपैरुपशोभिता}
{यदैव पृथिवी राजंस्तदाप्रभृति मे गृहम्}% ९९

\twolineshloka
{एतच्छ्रुत्वा तु रामो वै सभासद उवाचह}
{न सा सभा यत्र न सन्ति वृद्धा वृद्धा न ते ये न वदन्ति धर्मम्}% १००

\twolineshloka
{नासौ धर्मो यत्र न चास्ति सत्यं न तत्सत्यं यच्छलमभ्युपैति}
{ये तु सभ्याः सभां गत्वा तूष्णीं ध्यायन्त आसते}% १०१

\twolineshloka
{यथाप्राप्तं न ब्रुवते सर्वे तेऽनृतवादिनः}
{न वक्ति च श्रुतं यश्च कामात्क्रोधात्तथा भयात्}% १०२

\twolineshloka
{सहस्रं वारुणाः पाशाः प्रतिमुञ्चन्ति तं नरम्}
{तेषां संवत्सरे पूर्णे पाश एकः प्रमुच्यते}% १०३

\twolineshloka
{तस्मात्सत्यं तु वक्तव्यं जानता सत्यमञ्जसा}
{एतच्छ्रुत्वा तु सचिवा राममेवाब्रुवंस्तदा}% १०४

\twolineshloka
{उलूकः शोभते राजन्न तु गृध्रो महामते}
{त्वं प्रमाणं महाराज राजा हि परमा गतिः}% १०५

\twolineshloka
{राजमूलाः प्रजाः सर्वा राजा धर्मः सनातनः}
{शास्ता राजा नृणां येषां न ते गच्छन्ति दुर्गतिम्}% १०६

\twolineshloka
{वैवस्वतेन मुक्ताश्च भवन्ति पुरुषोत्तमाः}
{सचिवानां वचः श्रुत्वा रामो वचनमब्रवीत्}% १०७

\twolineshloka
{श्रूयतामभिधास्यामि पुराणं यदुदाहृतम्}
{द्यौः सचन्द्रार्कनक्षत्रा सपर्वतमहीद्रुमम्}% १०८

\twolineshloka
{सलिलार्णवसम्मग्नं त्रैलोक्यं सचराचरम्}
{एकमेव तदा ह्यासीत्सर्वमेकमिवाम्बरम्}% १०९

\twolineshloka
{पुनर्भूः सह लक्ष्म्या च विष्णोर्जठरमाविशत्}
{तां निगृह्य महातेजाः प्रविश्य सलिलार्णवम्}% ११०

\twolineshloka
{सुष्वाप हि कृतात्मा स बहुवर्षशतान्यपि}
{विष्णौ सुप्ते ततो ब्रह्मा विवेश जठरं ततः}% १११

\twolineshloka
{बहुस्रोतं च तं ज्ञात्वा महायोगी समाविशत्}
{नाभ्यां विष्णोः समुद्भूतं पद्मं हेमविभूषितम्}% ११२

\twolineshloka
{स तु निर्गम्य वै ब्रह्मा योगी भूत्वा महाप्रभुः}
{सिसृक्षुः पृथिवीं वायुं पर्वतांश्च महीरुहान्}% ११३

\twolineshloka
{तदन्तराः प्रजाः सर्वा मानुषांश्च सरीसृपान्}
{जरायुजाण्डजान्सर्वान्ससर्ज स महातपाः}% ११४

\twolineshloka
{तस्य गात्रसमुत्पन्नः कैटभो मधुना सह}
{दानवौ तौ महावीर्यौ घोरौ लब्धवरौ तदा}% ११५

\twolineshloka
{दृष्ट्वा प्रजापतिं तत्र क्रोधाविष्टावुभौ नृप}
{वेगेन महता भोक्तुं स्वयम्भुवमधावताम्}% ११६

\twolineshloka
{दृष्ट्वा सत्वानि सर्वाणि निस्सरन्ति पृथक्पृथक्}
{ब्रह्मणा संस्तुतो विष्णुर्हत्वा तौ मधुकैटभौ}% ११७

\twolineshloka
{पृथिवीं वर्धयामास स्थित्यर्थं मेदसा तयोः}
{मेदोगन्धा तु धरणी मेदिनीत्यभिधां गता}% ११८

\twolineshloka
{तस्माद्गृध्रस्त्वसत्यो वै पापकर्मापरालयम्}
{स्वीयं करोति पापात्मा दण्डनीयो न संशयः}% ११९

\twolineshloka
{ततोऽशरीरिणीवाणी अन्तरिक्षात्प्रभाषते}
{मा वधी राम गृध्रं त्वं पूर्वन्दग्धं तपोबलात्}% १२०

\twolineshloka
{पुरा गौतम दग्धोऽयं प्रजानाथो जनेश्वर}
{ब्रह्मदत्तस्तु नामैष शूरः सत्यव्रतः शुचिः}% १२१

\twolineshloka
{गृहमागत्य विप्रर्षेर्भोजनं प्रत्ययाचत}
{साग्रं वर्षशतं चैव भुक्तवान्नृपसत्तम}% १२२

\twolineshloka
{ब्रह्मदत्तस्य वै तस्य पाद्यमर्घ्यं स्वयं ततः}
{आत्मनैवाकरोत्सम्यग्भोजनार्थं महाद्युते}% १२३

\twolineshloka
{समाविश्य गृहं तस्य आहारे तु महात्मनः}
{नारीं पूर्णस्तनीं दृष्ट्वा हस्तेनाथ परामृशत्}% १२४

\twolineshloka
{अथ क्रुद्धेन मुनिना शापो दत्तः सुदारुणः}
{गृध्रत्वं गच्छ वै मूढ राजा मुनिमथाब्रवीत्}% १२५

\twolineshloka
{कृपां कुरु महाभाग शापोद्धारो भविष्यति}
{दयालुस्तद्वचः श्रुत्वा पुनराह नराधिप}% १२६

\twolineshloka
{उत्पत्स्यते रघुकुले रामो नाम महायशाः}
{इक्ष्वाकूणां महाभागो राजा राजीवलोचनः}% १२७

\twolineshloka
{तेन दृष्टो विपापस्त्वं भविता नरपुङ्गव}
{दृष्टो रामेण तच्छ्रुत्वा बभूव पृथिवीपतिः}% १२८

\twolineshloka
{गृध्रत्वं त्यज्य वै शीघ्रं दिव्यगन्धानुलेपनः}
{पुरुषो दिव्यरूपोऽसौ बभाषे तं नराधिपम्}% १२९

\twolineshloka
{साधु राघव धर्मज्ञ त्वत्प्रसादादहं विभो}
{विमुक्तो नरकाद्घोरादपापस्तु त्वया कृतः}% १३०

\twolineshloka
{विसर्जितं मया गार्ध्यं नररूपी महीपतिः}
{उलूकं प्राह धर्मज्ञ स्वगृहं विश कौशिक}% १३१

\twolineshloka
{अहं सन्ध्यामुपासित्वा गमिष्ये यत्र वै मुनिः}
{अथोदकमुपस्पृश्य सन्ध्यामन्वास्य पश्चिमाम्}% १३२

\twolineshloka
{आश्रमं प्राविशद्रामः कुम्भयोनेर्महात्मनः}
{तस्यागस्त्यो बहुगुणं फलमूलं च सादरम्}% १३३

\twolineshloka
{रसवन्ति च शाकानि भोजनार्थमुपाहरत्}
{सभुक्तवान्नरव्याघ्रस्तदन्नममृतोपमम्}% १३४

\twolineshloka
{प्रीतश्च परितुष्टश्च तां रात्रिं समुपावसत्}
{प्रभाते काल्यमुत्थाय कृत्वाह्निकमरिन्दम}% १३५

\twolineshloka
{ॠषिं समभिचक्राम गमनाय रघूत्तमः}
{अभिवाद्याब्रवीद्रामो महर्षिं कुम्भसम्भवम्}% १३६

\twolineshloka
{आपृच्छे साधये ब्रह्मन्ननुज्ञातुं त्वमर्हसि}
{धन्योस्म्यनुगृहीतोस्मि दर्शनेन महामुने}% १३७

\twolineshloka
{दिष्ट्या चाहं भविष्यामि पावनात्मा महात्मनः}
{एवं ब्रुवति काकुत्स्थे वाक्यमद्भुतदर्शनम्}% १३८

\twolineshloka
{उवाच परमप्रीतो बाष्पनेत्रस्तपोधनः}
{अत्यद्भुतमिदं वाक्यं तव राम शुभाक्षरम्}% १३९

\twolineshloka
{पावनं सर्वभूतानां त्वयोक्तं रघुनन्दन}
{मुहूर्तमपि राम त्वां मैत्रेणेक्षन्ति ये नराः}% १४०

\twolineshloka
{पावितास्सर्वसूक्तैस्ते कथ्यन्ते त्रिदिवौकसः}
{ये च त्वां घोरचक्षुर्भिरीक्षन्ते प्राणिनो भुवि}% १४१

\twolineshloka
{ते हता ब्रह्मदण्डेन सद्यो नरकगामिनः}
{ईदृशस्त्वं रघुश्रेष्ठ पावनः सर्वदेहिनाम्}% १४२

\twolineshloka
{कथयन्तश्च लोकास्त्वां सिद्धिमेष्यन्ति राघव}
{गच्छस्वानातुरोऽविघ्नं पन्थानमकुतोभयः}% १४३

\twolineshloka
{प्रशाधि राज्यं धर्मेण गतिस्तु जगतां भवान्}
{एवमुक्तस्तु मुनिना प्राञ्जलि प्रग्रहो नृपः}% १४४

\twolineshloka
{अभिवादयितुं चक्रे सोऽगस्त्यमृषिसत्तमम्}
{अभिवाद्य मुनिश्रेष्ठंस्तांश्च सर्वांस्तपोधिकान्}% १४५

\twolineshloka
{अथारोहत्तदाव्यग्रः पुष्पकं हेमभूषितम्}
{तं प्रयान्तं मुनिगणा आशीर्वादैस्समन्ततः}% १४६

\twolineshloka
{अपूपुजन्नरेन्द्रं तं सहस्राक्षमिवामराः}
{ततोऽर्धदिवसे प्राप्ते रामः सर्वार्थकोविदः}% १४७

\twolineshloka
{अयोध्यां प्राप्य काकुत्स्थः पद्भ्यां कक्षामवातरत्}
{ततो विसृज्य रुचिरं पुष्पकं कामवाहितम्}% १४८

\twolineshloka
{कक्षान्तराद्विनिष्क्रम्य द्वास्थान्राजाऽब्रवीदिदम्}
{लक्ष्मणं भरतं चैव गच्छध्वं लघुविक्रमाः}% १४९

\twolineshloka
{ममागमनमाख्याय समानयत मा चिरम्}
{श्रुत्वाथ भाषितं द्वास्था रामस्याक्लिष्टकर्मणः1.37.}% १५०

\twolineshloka
{गत्वा कुमारावाहूय राघवाय न्यवदेयन्}
{द्वास्थैः कुमारावानीतौ राघवस्य निदेशतः}% १५१

\twolineshloka
{दृष्ट्वा तु राघवः प्राप्तौ प्रियौ भरतलक्ष्मणौ}
{समालिङ्ग्य तु रामस्तौ वाक्यं चेदमुवाच ह}% १५२

\twolineshloka
{कृतं मया यथातथ्यं द्विजकार्यमनुत्तमम्}
{धर्महेतुमतो भूयः कर्तुमिच्छामि राघवौ}% १५३

\twolineshloka
{भवद्भ्यामात्मभूताभ्यां राजसूयं क्रतूत्तमम्}
{सहितो यष्टुमिच्छामि यत्र धर्मश्च शाश्वतः}% १५४

\twolineshloka
{पुष्करस्थेन वै पूर्वं ब्रह्मणा लोककारिणा}
{शतत्रयेण यज्ञानामिष्टं षष्ट्याधिकेन च}% १५५

\twolineshloka
{इष्ट्वा हि राजसूयेन सोमो धर्मेण धर्मवित्}
{प्राप्तः सर्वेषु लोकेषु कीर्तिस्थानमनुत्तमम्}% १५६

\twolineshloka
{इष्ट्वा हि राजसूयेन मित्रः शत्रुनिबर्हणः}
{मुहूर्तेन सुशुद्धेन वरुणत्वमुपागतः}% १५७

\onelineshloka*
{तस्माद्भवन्तौ सञ्चिन्त्य कार्येस्मिन्वदतं हि तत्}

\uvacha{भरत उवाच}

\onelineshloka
{त्वं धर्मः परमः साधो त्वयि सर्वा वसुन्धरा}% १५८

\twolineshloka
{प्रतिष्ठिता महाबाहो यशश्चामितविक्रम}
{महीपालाश्च सर्वे त्वां प्रजापतिमिवामराः}% १५९

\twolineshloka
{निरीक्षन्ते महात्मानो लोकनाथ तथा वयम्}
{प्रजाश्च पितृवद्राजन्पश्यन्ति त्वां महामते}% १६०

\twolineshloka
{पृथिव्यां गतिभूतोसि प्राणिनामिह राघव}
{सत्वमेवंविधं यज्ञं नाहर्त्तासि परन्तप}% १६१

\twolineshloka
{पृथिव्यां सर्वभूतानां विनाशो दृश्यते यतः}
{श्रूयते राजशार्दूल सोमस्य मनुजेश्वर}% १६२

\twolineshloka
{ज्योतिषां सुमहद्युद्धं सङ्ग्रामे तारकामये}
{तारा बृहस्पतेर्भार्या हृता सोमेनकामतः}% १६३

\twolineshloka
{तत्र युद्धं महद्वृत्तं देवदानवनाशनम्}
{वरुणस्य क्रतौ घोरे सङ्ग्रामे मत्स्यकच्छपाः}% १६४

\twolineshloka
{निवृत्ते राजशार्दूल सर्वे नष्टा जलेचराः}
{हरिश्चन्द्रस्य यज्ञान्ते राजसूयस्य राघव}% १६५

\twolineshloka
{आडीबकम्महद्युद्धं सर्वलोकविनाशनम्}
{पृथिव्यां यानि सत्वानि तिर्यग्योनिगतानि वै}% १६६

\twolineshloka
{दिव्यानां पार्थिवानां च राजसूये क्षयः श्रुतः}
{स त्वं पुरुषशार्दूल बुद्ध्या सञ्चिन्त्य पार्थिव}% १६७

\twolineshloka
{प्राणिनां च हितं सौम्यं पूर्णधर्मं समाचर}
{भरतस्य वचः श्रुत्वा राघवः प्राह सादरम्}% १६८

\twolineshloka
{प्रीतोस्मि तव धर्मज्ञ वाक्येनानेन शत्रुहन्}
{निवर्तिता राजसूयान्मतिर्मे धर्मवत्सल}% १६९

\twolineshloka
{पूर्णं धर्मं करिष्यामि कान्यकुब्जे च वामनम्}
{स्थापयिष्याम्यहं वीर सा मे ख्यातिर्दिवं गता}% १७०

\onelineshloka
{भविष्यति न सन्देहो यथा गङ्गा भगीरथात्}% १७१

{॥इति श्रीपाद्मपुराणे प्रथमे सृष्टिखण्डे यज्ञनिवारणं नाम सप्तत्रिंशोऽध्यायः॥३७॥}

    \sect{वामनप्रतिष्ठा}

\src{पद्म-पुराणम्}{सृष्टिखण्डम्}{अध्यायः ३८}{१--१९४}
% \tags{concise, complete}
\notes{This chapter describes the conversation between Rama and Agastya. It narrates how Rama, after defeating Ravana, meets Agastya in the forest. Agastya explains the significance of the divine ornaments given to Rama.}
\textlink{https://sa.wikisource.org/wiki/पद्मपुराणम्/खण्डः_१_(सृष्टिखण्डम्)/अध्यायः_३८}
\translink{https://www.wisdomlib.org/hinduism/book/the-padma-purana/d/doc364161.html}

\storymeta


\uvacha{भीष्म उवाच}

\twolineshloka
{कथं रामेण विप्रर्षे कान्यकुब्जे तु वामनः}
{स्थापितः क्व च लब्धोसौ विस्तरान्मम कीर्तय}% १

\twolineshloka
{तथा हि मधुरा चैषा या वाणी रामकीर्तने}
{कीर्तिता भगवन्मह्यं हृता कर्णसुखावह}% २

\twolineshloka
{अनुरागेण तं लोकाः स्नेहात्पश्यन्ति राघवम्}
{धर्मज्ञश्च कृतज्ञश्च बुद्ध्या च परिनिष्ठितः}% ३

\twolineshloka
{प्रशास्ति पृथिवीं सर्वां धर्मेण सुसमाहितः}
{तस्मिन्शासति वै राज्यं सर्वकामफलाद्रुमाः}% ४

\twolineshloka
{रसवन्तः प्रभूताश्च वासांसि विविधानि च}
{अकृष्टपच्या पृथिवी निःसपत्ना महात्मनः}% ५

\twolineshloka
{देवकार्यं कृतं तेन रावणो लोककण्टकः}
{सपुत्रोमात्यसहितो लीलयैव निपातितः}% ६

\twolineshloka
{तस्यबुद्धिस्समुत्पन्ना पूर्णे धर्मे द्विजोत्तम}
{तस्याहं चरितं सर्वं श्रोतुमिच्छामि वै मुने}% ७

\uvacha{पुलस्त्य उवाच}

\twolineshloka
{कस्यचित्त्वथ कालस्य रामो धर्मपथे स्थितः}
{यच्चकार महाबाहो शृणुष्वैकमना नृप}% ८

\twolineshloka
{सस्मार राक्षसेन्द्रं तं कथं राजा विभीषणः}
{लङ्कायां संस्थितो राज्यं करिष्यति च राक्षसः}% ९

\twolineshloka
{गीर्वाणेषु प्रातिकूल्यं विनाशस्य तु लक्षणम्}
{मया तस्य तु तद्दत्तं राज्यं चन्द्रार्ककालिकम्}% १०

\twolineshloka
{तस्याविनाशतः कीर्तिः स्थिरा मे शाश्वती भवेत्}
{रावणेन तपस्तप्तं विनाशायात्मनस्त्विह}% ११

\twolineshloka
{विध्वस्तः स च पापिष्ठो देवकार्ये मयाधुना}
{तदिदानीं मयान्वेष्यः स्वयं गत्वा विभीषणः}% १२

\twolineshloka
{सन्देष्टव्यं हितं तस्य येन तिष्ठेत्स शाश्वतम्}
{एवं चिन्तयतस्तस्य रामस्यामिततेजसः}% १३

\twolineshloka
{आजगामाथ भरतो रामं दृष्ट्वाब्रवीदिदम्}
{किं त्वं चिन्तयसे देव न रहस्यं वदस्व मे}% १४

\twolineshloka
{देवकार्ये धरायां वा स्वकार्ये वा नरोत्तम}
{एवं ब्रुवन्तं भरतं ध्यायमानमवस्थितम्}% १५

\twolineshloka
{अब्रवीद्राघवो वाक्यं रहस्यं तु न वै तव}
{भवान्बहिश्चरः प्राणो लक्ष्मणश्च महायशाः}% १६

\twolineshloka
{अवेद्यं भवतो नास्ति मम सत्यं विधारय}
{एषा मे महती चिन्ता कथं देवैर्विभीषणः}% १७

\twolineshloka
{वर्तते यद्धितार्थं वै दशग्रीवो निपातितः}
{गमिष्ये तदहं लङ्कां यत्र चासौ विभीषणः}% १८

\twolineshloka
{तं च दृष्ट्वा पुरीं तां तु कार्यमुक्त्वा च राक्षसम्}
{आलोक्य सर्ववसुधां सुग्रीवं वानरेश्वरम्}% १९

\twolineshloka
{महाराजं च शत्रुघ्नं भातृपुत्रांश्च सर्वशः}
{एवं वदति काकुत्स्थे भरतः पुरतः स्थितः}% २०

\twolineshloka
{उवाच राघवं वाक्यं गमिष्ये भवता सह}
{एवं कुरु महाबाहो सौमित्रिरिह तिष्ठतु}% २१

\twolineshloka
{इत्युक्त्वा भरतं रामः सौमित्रं चाह वै पुरे}
{रक्षाकार्या त्वया वीर यावदागमनं हि नौ}% २२

\twolineshloka
{एवं लक्ष्मणमादिश्य ध्यात्वा वै पुष्पकं नृप}
{आरुरोह स वै यानं कौसल्यानन्दवर्धनः}% २३

\twolineshloka
{पुष्पकं तु ततः प्राप्तं गान्धारविषयो यतः}
{भरतस्य सुतौ दृष्ट्वा जगन्नीतिं निरीक्ष्य च}% २४

\twolineshloka
{पूर्वां दिशं ततो गत्वा लक्ष्मणस्य सुतौ यतः}
{पुरेषु तेषु षड्रात्रमुषित्वा रघुनन्दनौ}% २५

\twolineshloka
{गतौ तेन विमानेन दक्षिणामभितो दिशम्}
{गङ्गायामुनसम्भेदं प्रयागमृषिसेवितम्}% २६

\twolineshloka
{अभिवाद्य भरद्वाजमत्रेराश्रममीयतुः}
{सम्भाष्य च मुनींस्तत्र जनस्थानमुपागतौ}% २७

\uvacha{राम उवाच}

\twolineshloka
{अत्र पूर्वं हृता सीता रावणेन दुरात्मना}
{हत्वा जटायुषं गृध्रं योसौ पितृसखो हि नौ}% २८

\twolineshloka
{अत्रास्माकं महद्युद्धं कबन्धेन कुबुद्धिना}
{हतेन तेन दग्धेन सीतास्ते रावणालये}% २९

\twolineshloka
{ॠष्यमूके गिरिवरे सुग्रीवो नाम वानरः}
{स ते करिष्यते साह्यं पम्पां व्रज सहानुजः}% ३०

\twolineshloka
{पम्पासरः समासाद्य शबरीं गच्छ तापसीम्}
{इत्युक्तो दुःखितो वीर निराशो जीविते स्थितः}% ३१

\twolineshloka
{इयं सा नलिनी वीर यस्यां वै लक्ष्मणोवदत्}
{मा कृथाः पुरुषव्याघ्र शोकं शत्रुविनाशन}% ३२

\twolineshloka
{आज्ञाकारिणि भृत्ये च मयि प्राप्स्यसि मैथिलीम्}
{अत्र मे वार्षिका मासा गता वर्षशतोपमाः}% ३३

\twolineshloka
{अत्रैव निहतो वाली सुग्रीवार्थे परन्तप}
{एषा सा दृश्यते नूनं किष्किन्धा वालिपालिता}% ३४

\twolineshloka
{यस्यां वै स हि धर्मात्मा सुग्रीवो वानरेश्वरः}
{वानरैः सहितो वीर तावदास्ते समाः शतम्}% ३५

\twolineshloka
{वानरैस्सह सुग्रीवो यावदास्ते सभां गतः}
{तावत्तत्रागतौ वीरौ पुर्यां भरतराघवौ}% ३६

\twolineshloka
{दृष्ट्वा स भ्रातरौ प्राप्तौ प्रणिपत्याब्रवीदिदम्}
{क्व युवां प्रस्थितौ वीरौ कार्यं किं नु करिष्यथः}% ३७

\twolineshloka
{विनिवेश्यासने तौ च ददावर्घ्ये स्वयं तदा}
{एवं सभास्थिते तत्र धर्मिष्टे रघुनन्दने}% ३८

\twolineshloka
{अङ्गदोथ हनूमांश्च नलो नीलश्च पाटलः}
{गजो गवाक्षो गवयः पनसश्च महायशाः}% ३९

\twolineshloka
{पुरोधसो मन्त्रिणश्च दैवज्ञो दधिवक्रकः}
{नीलश्शतबलिर्मैन्दो द्विविदो गन्धमादनः}% ४०

\twolineshloka
{वीरबाहुस्सुबाहुश्च वीरसेनो विनायकः}
{सूर्याभः कुमुदश्चैव सुषेणो हरियूथपः}% ४१

\twolineshloka
{ॠषभो विनतश्चैव गवाख्यो भीमविक्रमः}
{ॠक्षराजश्च धूम्रश्च सहसैन्यैरुपागताः}% ४२

\twolineshloka
{अन्तःपुराणि सर्वाणि रुमा तारा तथैव च}
{अवरोधोङ्गदस्यापि तथान्याः परिचारिकाः}% ४३

\twolineshloka
{प्रहर्षमतुलं प्राप्य साधुसाध्विति चाब्रुवन्}
{वानराश्च महात्मानः सुग्रीवसहितास्तदा}% ४४

\twolineshloka
{वानर्यश्च महाभागास्ताराद्यास्तत्र राघवम्}
{अभिप्रेक्ष्याश्रुकण्ठ्यश्च प्रणिपत्येदमब्रुवन्}% ४५

\twolineshloka
{क्व सा देवी त्वया देव या विनिर्जित्यरावणम्}
{शुद्धिं कृत्वा हि ते वह्नौ पितुरग्र उमापतेः}% ४६

\twolineshloka
{त्वयानीता पुरीं राम न तां पश्यामि तेग्रतः}
{न विना त्वं तया देव शोभसे रघुनन्दन}% ४७

\twolineshloka
{त्वया विनापि साध्वी सा क्व नु तिष्ठति जानकी}
{अन्यां भार्यां न ते वेद्मि भार्याहीनो न शोभसे}% ४८

\twolineshloka
{क्रौञ्चयुग्मं मिथो यद्वच्चक्रवाकयुगं यथा}
{एवं वदन्तीं तां तारां ताराधिपसमाननाम्}% ४९

\twolineshloka
{प्राह प्रवचसां श्रेष्ठो रामो राजीवलोचनः}
{चारुदंष्ट्रे विशालाक्षि कालो हि दुरतिक्रमः1.38.}% ५०

\twolineshloka
{सर्वं कालकृतं विद्धि जगदेतच्चराचरम्}
{विसृज्यताः स्त्रियः सर्वाः सुग्रीवोभिमुखः स्थितः}% ५१

\uvacha{सुग्रीव उवाच}

\twolineshloka
{भवन्तौ येन कार्येण इहायातौ नरेश्वरौ}
{तच्चापि कथ्यतां शीघ्रं कृत्यकालो हि वर्तते}% ५२


\threelineshloka
{ब्रुवाणमेवं सुग्रीवं भरतो रामचोदितः}
{आचचक्षे च गमनं लङ्कायां राघवस्य तु}
{तौ चाब्रवीच्च सुग्रीवो भवद्भ्यां सहितः पुरीम्}% ५३

\twolineshloka
{गमिष्ये राक्षसं देव द्रष्टुं तत्र विभीषणम्}
{सुग्रीवेणैवमुक्ते तु गच्छस्वेत्याह राघवः}% ५४

\twolineshloka
{सुग्रीवो राघवौ तौ च पुष्पके तु स्थितास्त्रयः}
{तावत्प्राप्तं विमानं तु समुद्रस्योत्तरं तटम्}% ५५

\twolineshloka
{अब्रवीद्भरतं रामो ह्यत्र मे राक्षसेश्वरः}
{चतुर्भिः सचिवैः सार्धं जीवितार्थे विभीषणः}% ५६

\twolineshloka
{प्राप्तस्ततो लक्ष्मणेन लङ्काराज्येभिषेचितः}
{अत्र चाहं समुद्रस्य परेपारे स्थितस्त्र्यहम्}% ५७

\twolineshloka
{दर्शनं दास्यते मेऽसौ ज्ञातिकार्यं भविष्यति}
{तावन्न दर्शनं मह्यं दत्तमेतेन शत्रुहन्}% ५८

\twolineshloka
{ततः कोपः सुमद्भूतश्चतुर्थेहनि राघव}
{धनुरायम्य वेगेन दिव्यमस्त्रं करे धृतम्}% ५९

\twolineshloka
{दृष्ट्वा मां शरणान्वेषी भीतो लक्ष्मणमाश्रितः}
{सुग्रीवेणानुनीतोऽस्मि क्षम्यतां राघव त्वया}% ६०

\twolineshloka
{ततो मयोत्क्षिप्तशरो मरुदेशे ह्यपाकृतः}
{ततस्समुद्रराजेन भृशं विनयशालिना}% ६१

\twolineshloka
{उक्तोहं सेतुबन्धेन लङ्कां त्वं व्रज राघव}
{लङ्घयित्वा नरव्याघ्र वारिपूर्णं महोदधिम्}% ६२

\twolineshloka
{एष सेतुर्मया बद्धः समुद्रे वरुणालये}
{त्रिभिर्दिनैः समाप्तिं वै नीतो वानरसत्तमैः}% ६३

\twolineshloka
{प्रथमे दिवसे बद्धो योजनानि चतुर्दश}
{द्वितीयेहनि षट्त्रिंशत्तृतीयेर्धशतं तथा}% ६४

\twolineshloka
{इयं सा दृश्यते लङ्का स्वर्णप्राकारतोरणा}
{अवरोधो महानत्र कृतो वानरसत्तमैः}% ६५

\twolineshloka
{अत्र युद्धं महद्वृत्तं चैत्राशुक्लचतुर्दशि}
{अष्टचत्वारिंशद्दिनं यत्रासौ रावणो हतः}% ६६

\twolineshloka
{अत्र प्रहस्तो नीलेन हतो राक्षसपुङ्गवः}
{हनूमता च धूम्राक्षो ह्यत्रैव विनिपातितः}% ६७

\twolineshloka
{महोदरातिकायौ च सुग्रीवेण महात्मना}
{अत्रैव मे कुम्भकर्णो लक्ष्मणेनेन्द्रजित्तथा}% ६८

\twolineshloka
{मया चात्र दशग्रीवो हतो राक्षसपुङ्गवः}
{अत्र सम्भाषितुं प्राप्तो ब्रह्मा लोकपितामहः}% ६९

\twolineshloka
{पार्वत्या सहितो देवः शूलपाणिर्वृषध्वजः}
{महेन्द्राद्याः सुरगणाः सगन्धर्वास्स किन्नराः}% ७०

\twolineshloka
{पिता मे च समायातो महाराजस्त्रिविष्टपात्}
{वृतश्चाप्सरसां सङ्घैर्विद्याधरगणैस्तथा}% ७१

\twolineshloka
{तेषां समक्षं सर्वेषां जानकी शुद्धिमिच्छता}
{उक्ता सीता हव्यवाहं प्रविष्टा शुद्धिमागता}% ७२

\twolineshloka
{लङ्काधिपैः सुरैर्दृष्टा गृहीता पितृशासनात्}
{अथाप्युक्तोथ राज्ञाहमयोध्यां गच्छ पुत्रकम्}% ७३

\twolineshloka
{न मे स्वर्गो बहुमतस्त्वया हीनस्य राघव}
{तारितोहं त्वया पुत्र प्राप्तोऽस्मीन्द्रसलोकताम्}% ७४

\twolineshloka
{लक्ष्मणं चाब्रवीद्राजा पुत्र पुण्यं त्वयार्जितम्}
{भ्रात्रासममथो दिव्यांल्लोकान्प्राप्स्यसि चोत्तमान्}% ७५

\twolineshloka
{आहूय जानकीं राजा वाक्यं चेदमुवाच ह}
{न च मन्युस्त्वया कार्यो भर्तारं प्रति सुव्रते}% ७६

\twolineshloka
{ख्यातिर्भविष्यत्येवाग्र्या भर्तुस्ते शुभलोचने}
{एवं वदति रामे तु पुष्पके च व्यवस्थिते}% ७७

\twolineshloka
{तत्र ये राक्षसवरास्ते गत्वाशु विभीषणम्}
{प्राप्तो रामः ससुग्रीवश्चारा इत्थं तदाऽवदन्}% ७८

\twolineshloka
{विभीषणस्तु तच्छ्रुत्वा रामागमनमन्तिके}
{चारांस्तान्पूजयामास सर्वकामधनादिभिः}% ७९

\twolineshloka
{अलङ्कृत्य पुरीं तां तु निष्क्रान्तः सचिवैः सह}
{दृष्ट्वा रामं विमानस्थं मेराविव दिवाकरम्}% ८०

\twolineshloka
{अष्टाङ्गप्रणिपातेन नत्वा राघवमब्रवीत्}
{अद्य मे सफलं जन्म प्राप्ताः सर्वे मनोरथाः}% ८१

\twolineshloka
{यद्दृष्टौ देवचरणौ जगद्वन्द्यावनिन्दितौ}
{कृतः श्लाघ्योस्म्यहं देव शक्रादीनां दिवौकसाम्}% ८२

\twolineshloka
{आत्मानमधिकं मन्ये त्रिदशेशात्पुरन्दरात्}
{रावणस्य गृहे दीप्ते सर्वरत्नोपशोभिते}% ८३

\twolineshloka
{उपविष्टे तु काकुत्स्थे अर्घं दत्वा विभीषणः}
{उवाच प्राञ्जलिर्भूत्वा सुग्रीवं भरतं तथा}% ८४

\twolineshloka
{इहागतस्य रामस्य यद्दास्ये न तदस्ति मे}
{इयं च लङ्का रामेण रिपुं त्रैलोक्यकण्टकम्}% ८५

\twolineshloka
{हत्वा तु पापकर्माणं दत्ता पूर्वं पुरी मम}
{इयं पुरी इमे दारा अमी पुत्रास्तथा ह्यहम्}% ८६

\twolineshloka
{सर्वमेतन्मया दत्तं सर्वमक्षयमस्तु ते}
{ततः प्रकृतयः सर्वा लङ्कावासिजनाश्च ये}% ८७

\twolineshloka
{आजग्मू राघवं द्रष्टुं कौतूहलसमन्विताः}
{उक्तो विभीषणस्तैस्तु रामं दर्शय नः प्रभो}% ८८

\twolineshloka
{विभीषणेन कथिता राघवाय महात्मने}
{तेषामुपायनं सर्वं भरतो रामचोदितः}% ८९

\twolineshloka
{जग्राह वानरेन्द्रश्च धनरत्नौघसञ्चयम्}
{एवं तत्र त्र्यहं रामो ह्यवसद्राक्षसालये}% ९०

\twolineshloka
{चतुर्थेहनि सम्प्राप्ते रामे चापि सभास्थिते}
{केकसी पुत्रमाहेदं रामं द्रक्ष्यामि पुत्रक}% ९१

\twolineshloka
{दृष्टे तस्मिन्महत्पुण्यं प्राप्यते मुनिसत्तमैः}
{विष्णुरेष महाभागश्चतुर्मूर्तिस्सनातनः}% ९२

\twolineshloka
{सीता लक्ष्मीर्महाभाग न बुद्धा साग्रजेन ते}
{पित्रा ते पूर्वमाख्यातं देवानां दिविसङ्गमे}% ९३

\twolineshloka
{कुले रघूणां वै विष्णुः पुत्रो दशरथस्य तु}
{भविष्यति विनाशाय दशग्रीवस्य रक्षसः}% ९४

\uvacha{विभीषण उवाच}

\twolineshloka
{एवं कुरुष्व वै मातर्गृहाण नवमं वरम्}
{पात्रं चन्दनसंयुक्तं दधिक्षौद्राक्षतैः सह}% ९५

\twolineshloka
{दूर्वयार्घं सह कुरु राजपुत्रस्य दर्शनम्}
{सरमामग्रतः कृत्वा याश्चान्या देवकन्यकाः}% ९६

\twolineshloka
{व्रजस्व राघवाभ्याशं तस्मादग्रे व्रजाम्यहम्}
{एवमुक्त्वा गतं रक्षो यत्र रामो व्यवस्थितः}% ९७

\twolineshloka
{उत्सार्य दानवान्सर्वान्रामं द्रष्टुं समागतान्}
{सभां तां विमलां कृत्वा रामं स्वाभिमुखे स्थितम्}% ९८

\uvacha{विभीषण उवाच}

\twolineshloka
{विज्ञाप्यं शृणु मे देव वदतश्च विशाम्पते}
{दशग्रीवं कुम्भकर्णं या च मां चाप्यजीजनत्}% ९९

\twolineshloka
{इयं सा देवमाता नः पादौ ते द्रष्टुमिच्छति}
{तस्यास्तु त्वं कृपां कृत्वा दर्शनं दातु मर्हसि1.38.}% १००

\uvacha{राम उवाच}

\twolineshloka
{अहं तस्याः समीपं तु मातृदर्शनकाङ्क्षया}
{गमिष्ये राक्षसेन्द्र त्वं शीघ्रं याहि ममाग्रतः}% १०१

\twolineshloka
{प्रतिज्ञाय तु तं वाक्यमुत्तस्थौ च वरासनात्}
{मूर्ध्नि चाञ्जलिमाधाय प्रणाममकरोद्विभुः}% १०२

\twolineshloka
{अभिवादयेहं भवतीं माता भवसि धर्मतः}
{महता तपसा चापि पुण्येन विविधेन च}% १०३

\twolineshloka
{इमौ ते चरणौ देवि मानवो यदि पश्यति}
{पूर्णस्स्यात्तदहं प्रीतो दृष्ट्वेमौ पुत्रवत्सले}% १०४

\twolineshloka
{कौसल्या मे यथा माता भवती च तथा मम}
{केकसी चाब्रवीद्रामं चिरं जीव सुखी भव}% १०५

\twolineshloka
{भर्त्रा मे कथितं वीर विष्णुर्मानुषरूपधृत्}
{अवतीर्णो रघुकुले हितार्थेत्र दिवौकसाम्}% १०६

\twolineshloka
{दशग्रीव विनाशाय भूतिं दातुं विभीषणे}
{वालिनो निधनं चैव सेतुबन्धं च सागरे}% १०७

\twolineshloka
{पुत्रो दशरथस्यैव सर्वं स च करिष्यति}
{इदानीं त्वं मया ज्ञातः स्मृत्वा तद्भर्तृभाषितम्}% १०८

\twolineshloka
{सीता लक्ष्मीर्भवान्विष्णुर्देवा वै वानरास्तथा}
{गृहं पुत्र गमिष्यामि स्थिरकीर्तिमवाप्नुहि}% १०९

\uvacha{सरमोवाच}

\twolineshloka
{इहैव वत्सरं पूर्णमशोकवनिकास्थिता}
{सेविता जानकी देव सुखं तिष्ठति ते प्रिया}% ११०

\twolineshloka
{नित्यं स्मरामि वै पादौ सीतायास्तु परन्तप}
{कदा द्रक्ष्यामि तां देवीं चिन्तयाना त्वहर्निशम्}% १११

\twolineshloka
{किमर्थं देवदेवेन नानीता जानकी त्विह}
{एकाकी नैव शोभेथा योषिता च तया विना}% ११२

\twolineshloka
{समीपे शोभते सीता त्वं च तस्याः परन्तप}
{एवं ब्रुवन्त्यां भरतः केयमित्यब्रवीद्वचः}% ११३

\twolineshloka
{ततश्चेङ्गितविद्रामो भरतं प्राह सत्वरम्}
{विभीषणस्य भार्या वै सरमा नाम नामतः}% ११४

\twolineshloka
{प्रिया सखी महाभागा सीतायास्सुदृढं मता}
{सर्वङ्कालकृतं पश्य न जाने किं करिष्यति}% ११५

\twolineshloka
{गच्छ त्वं सुभगे भर्तृगेहं पालय शोभने}
{मां त्यक्त्वा हि गता देवी भाग्यहीनं गतिर्यथा}% ११६

\twolineshloka
{तया विरहितः सुभ्रु रतिं विन्दे न कर्हिचित्}
{शून्या एव दिशः सर्वाः पश्यामीह पुनर्भ्रमन्}% ११७

\twolineshloka
{विसृज्यतां च सरमां सीतायास्तु प्रियां सखीम्}
{गतायामथ केकस्यां रामः प्राह विभीषणम्}% ११८

\twolineshloka
{दैवतेभ्यः प्रियं कार्यं नापराध्यास्त्वया सुराः}
{आज्ञया राजराजस्य वर्तितव्यं त्वयानघ}% ११९

\twolineshloka
{लङ्कायां मानुषो यो वै समागच्छेत्कथञ्चन}
{राक्षसैर्न च हन्तव्यो द्रष्टव्योसौ यथा त्वहम्}% १२०

\uvacha{विभीषण उवाच}

\twolineshloka
{आज्ञयाहं नरव्याघ्र करिष्ये सर्वमेव तु}
{विभीषणे हि वदति वायू राममुवाच ह}% १२१

\twolineshloka
{इहास्तिवैष्णवी मूर्तिः पूर्वं बद्धो बलिर्यया}
{तां नयस्व महाभाग कान्यकुब्जे प्रतिष्ठय}% १२२

\twolineshloka
{विदित्वा तदभिप्रायं वायुना समुदाहृतम्}
{विभीषणस्त्वलङ्कृत्य रत्नैः सर्वैश्च वामनम्}% १२३

\twolineshloka
{आनीय चार्पयद्रामे वाक्यं चेदमुवाच ह}
{यदा वै निर्जितः शक्रो मेघनादेन राघव}% १२४

\twolineshloka
{तदा वै वामनस्त्वेष आनीतो जलजेक्षण}
{नयस्व तमिमं देव देवदेवं प्रतिष्ठय}% १२५

\twolineshloka
{तथेति राघवः कृत्वा पुष्पकं च समारुहत्}
{धनं रत्नमसङ्ख्येयं वामनं च सुरोत्तमम्}% १२६

\twolineshloka
{गृह्य सुग्रीवभरतावारूढौ वामनादनु}
{व्रजन्नेवाम्बरे रामस्तिष्ठेत्याह विभीषणम्}% १२७

\twolineshloka
{राघवस्य वचः श्रुत्वा भूयोप्याह स राघवम्}
{करिष्ये सर्वमेतद्धि यदाज्ञप्तं विभो त्वया}% १२८

\twolineshloka
{सेतुनानेन राजेन्द्र पृथिव्यां सर्वमानवाः}
{आगत्य प्रतिबाधेरन्नाज्ञाभङ्गो भवेत्तव}% १२९

\twolineshloka
{कोत्र मे नियमो देव किन्नु कार्यं मया विभो}
{श्रुत्वैतद्राघवो वाक्यं राक्षसोत्तमभाषितम्}% १३०

\twolineshloka
{कार्मुकं गृह्य हस्तेन रामः सेतुं द्विधाच्छिनत्}
{त्रिर्विभज्य च वेगेन मध्ये वै दशयोजनम्}% १३१

\twolineshloka
{छित्वा तु योजनं चैकमेकं खण्डत्रयं कृतम्}
{वेलावनं समासाद्य रामः पूजां रमापतेः}% १३२

\twolineshloka
{कृत्वा रामेश्वरं नाम्ना देवदेवं जनार्दनम्}
{अभिषिच्याथ सङ्गृह्य वामनं रघुनन्दनः}% १३३

\twolineshloka
{दक्षिणादुदधेश्चैव निर्जगाम त्वरान्वितः}
{अन्तरिक्षादभूद्वाणी मेघगम्भीरनिःस्वना}% १३४

\uvacha{रुद्र उवाच}

\twolineshloka
{भो भो रामास्तु भद्रं ते स्थितोऽहमिह साम्प्रतम्}
{यावज्जगदिदं राम यावदेषा धरा स्थिता}% १३५

\twolineshloka
{तावदेव च ते सेतु तीर्थं स्थास्यति राघव}
{श्रुत्वैवं देवदेवस्य गिरं ताममृतोपमाम्}% १३६

\uvacha{राम उवाच}

\twolineshloka
{नमस्ते देवदेवेश भक्तानामभयङ्कर}
{गौरीकान्त नमस्तुभ्यं दक्षयज्ञविनाशन}% १३७

\twolineshloka
{नमो भवाय शर्वाय रुद्राय वरदाय च}
{पशूनाम्पतये नित्यं चोग्राय च कपर्दिने}% १३८

\twolineshloka
{महादेवाय भीमाय त्र्यम्बकाय दिशाम्पते}
{ईशानाय भगघ्नाय नमोस्त्वन्धकघातिने}% १३९

\twolineshloka
{नीलग्रीवाय घोराय वेधसे वेधसा स्तुत}
{कुमारशत्रुनिघ्नाय कुमारजननाय च}% १४०

\twolineshloka
{विलोहिताय धूम्राय शिवाय क्रथनाय च}
{नमो नीलशिखण्डाय शूलिने दैत्यनाशिने}% १४१

\twolineshloka
{उग्राय च त्रिनेत्राय हिरण्यवसुरेतसे}
{अनिन्द्यायाम्बिकाभर्त्रे सर्वदेवस्तुताय च}% १४२

\twolineshloka
{अभिगम्याय काम्याय सद्योजाताय वै नमः}
{वृषध्वजाय मुण्डाय जटिने ब्रह्मचारिणे}% १४३

\twolineshloka
{तप्यमानाय तप्याय ब्रह्मण्याय जयाय च}
{विश्वात्मने विश्वसृजे विश्वमावृत्य तिष्ठते}% १४४

\twolineshloka
{नमो नमोस्तु दिव्याय प्रपन्नार्तिहराय च}
{भक्तानुकम्पिने देव विश्वतेजो मनोगते}% १४५

\uvacha{पुलस्त्य उवाच}

\twolineshloka
{एवं संस्तूयमानस्तु देवदेवो हरो नृप}
{उवाच राघवं वाक्यं भक्तिनम्रं पुरास्थितम्}% १४६

\uvacha{रुद्र उवाच}

\twolineshloka
{भो भो राघव भद्रं ते ब्रूहि यत्ते मनोगतम्}
{भवान्नारायणो नूनं गूढो मानुषयोनिषु}% १४७

\twolineshloka
{अवतीर्णो देवकार्यं कृतं तच्चानघ त्वया}
{इदानीं स्वं व्रजस्थानं कृतकार्योसि शत्रुहन्}% १४८

\twolineshloka
{त्वया कृतं परं तीर्थं सेत्वाख्यं रघुनन्दन}
{आगत्य मानवा राजन्पश्येयुरिह सागरे}% १४९

\twolineshloka
{महापातकयुक्ता ये तेषां पापं विलीयते}
{ब्रह्मवध्यादिपापानि यानि कष्टानि कानिचित्1.38.}% १५०

\twolineshloka
{दर्शनादेव नश्यन्ति नात्र कार्या विचारणा}
{गच्छ त्वं वामनं स्थाप्य गङ्गातीरे रघूत्तम}% १५१

\twolineshloka
{पृथिव्यां सर्वशः कृत्वा भागानष्टौ परन्तप}
{श्वेतद्वीपं स्वकं स्थानं व्रज देव नमोस्तु ते}% १५२

\twolineshloka
{प्रणिपत्य ततो रामस्तीर्थं प्राप्तश्च पुष्करम्}
{विमानं तु न यात्यूर्ध्वं वेष्टितं तत्तु राघवः}% १५३

\twolineshloka
{किमिदं वेष्टितं यानं निरालम्बेऽम्बरे स्थितम्}
{भवितव्यं कारणेन पश्येत्याह स्म वानरम्}% १५४

\twolineshloka
{सुग्रीवो रामवचनादवतीर्य धरातले}
{स च पश्यति ब्रह्माणं सुरसिद्धसमन्वितम्}% १५५

\twolineshloka
{ब्रह्मर्षिसङ्घसहितं चतुर्वेदसमन्वितम्}
{दृष्ट्वाऽऽगत्याब्रवीद्रामं सर्वलोकपितामहः}% १५६

\twolineshloka
{सहितो लोकपालैश्च वस्वादित्यमरुद्गणैः}
{तं देवं पुष्पकं नैव लङ्घयेद्धि पितामहम्}% १५७

\twolineshloka
{अवतीर्य ततो रामः पुष्पकाद्धेमभूषितात्}
{नत्वा विरिञ्चनं देवं गायत्र्या सह संस्थितम्}% १५८

\twolineshloka
{अष्टाङ्गप्रणिपातेन पञ्चाङ्गालिङ्गितावनिः}
{तुष्टाव प्रणतो भूत्वा देवदेवं विरिञ्चनम्}% १५९

\uvacha{राम उवाच}

\twolineshloka
{नमामि लोककर्तारं प्रजापतिसुरार्चितम्}
{देवनाथं लोकनाथं प्रजानाथं जगत्पतिम्}% १६०

\twolineshloka
{नमस्ते देवदेवेश सुरासुरनमस्कृत}
{भूतभव्यभवन्नाथ हरिपिङ्गललोचन}% १६१

\twolineshloka
{बालस्त्वं वृद्धरूपी च मृगचर्मासनाम्बरः}
{तारणश्चासि देवस्त्वं त्रैलोक्यप्रभुरीश्वरः}% १६२

\twolineshloka
{हिरण्यगर्भः पद्मगर्भः वेदगर्भः स्मृतिप्रदः}
{महासिद्धो महापद्मी महादण्डी च मेखली}% १६३

\twolineshloka
{कालश्च कालरूपी च नीलग्रीवो विदांवरः}
{वेदकर्तार्भको नित्यः पशूनां पतिरव्ययः}% १६४

\twolineshloka
{दर्भपाणिर्हंसकेतुः कर्ता हर्ता हरो हरिः}
{जटी मुण्डी शिखी दण्डी लगुडी च महायशाः}% १६५

\twolineshloka
{भूतेश्वरः सुराध्यक्षः सर्वात्मा सर्वभावनः}
{सर्वगः सर्वहारी च स्रष्टा च गुरुरव्ययः}% १६६

\twolineshloka
{कमण्डलुधरो देवः स्रुक्स्रुवादिधरस्तथा}
{हवनीयोऽर्चनीयश्च ॐकारो ज्येष्ठसामगः}% १६७

\twolineshloka
{मृत्युश्चैवामृतश्चैव पारियात्रश्च सुव्रतः}
{ब्रह्मचारी व्रतधरो गुहावासी सुपङ्कजः}% १६८

\twolineshloka
{अमरो दर्शनीयश्च बालसूर्यनिभस्तथा}
{दक्षिणे वामतश्चापि पत्नीभ्यामुपसेवितः}% १६९

\twolineshloka
{भिक्षुश्च भिक्षुरूपश्च त्रिजटी लब्धनिश्चयः}
{चित्तवृत्तिकरः कामो मधुर्मधुकरस्तथा}% १७०

\twolineshloka
{वानप्रस्थो वनगत आश्रमी पूजितस्तथा}
{जगद्धाता च कर्त्ता च पुरुषः शाश्वतो ध्रुवः}% १७१

\twolineshloka
{धर्माध्यक्षो विरूपाक्षस्त्रिधर्मो भूतभावनः}
{त्रिवेदो बहुरूपश्च सूर्यायुतसमप्रभः}% १७२

\twolineshloka
{मोहकोवन्धकश्चैवदानवानांविशेषतः}
{देवदेवश्च पद्माङ्कस्त्रिनेत्रोऽब्जजटस्तथा}% १७३

\twolineshloka
{हरिश्मश्रुर्धनुर्धारी भीमो धर्मपराक्रमः}
{एवं स्तुतस्तु रामेण ब्रह्मा ब्रह्मविदांवरः}% १७४

\twolineshloka
{उवाच प्रणतं रामं करे गृह्य पितामहः}
{विष्णुस्त्वं मानुषे देहेऽवतीर्णो वसुधातले}% १७५

\twolineshloka
{कृतं तद्भवता सर्वं देवकार्यं महाविभो}
{संस्थाप्य वामनं देवं जाह्नव्या दक्षिणे तटे}% १७६

\twolineshloka
{अयोध्यां स्वपुरीं गत्वा सुरलोकं व्रजस्व च}
{विसृष्टो ब्रह्मणा रामः प्रणिपत्य पितामहम्}% १७७

\twolineshloka
{आरूढः पुष्पकं यानं सम्प्राप्तो मधुरां पुरीम्}
{समीक्ष्य पुत्रसहितं शत्रुघ्नं शत्रुघातिनम्}% १७८

\twolineshloka
{तुतोष राघवः श्रीमान्भरतः स हरीश्वरः}
{शत्रुघ्नो भ्रातरौ प्राप्तौ शक्रोपेन्द्राविवागतौ}% १७९

\twolineshloka
{प्रणिपत्य ततो मूर्ध्ना पञ्चाङ्गालिङ्गितावनिः}
{उत्थाप्य चाङ्कमारोप्य रामो भ्रातरमञ्जसा}% १८०

\twolineshloka
{भरतश्च ततः पश्चात्सुग्रीवस्तदनन्तरम्}
{उपविष्टोऽथ रामाय सोऽर्घमादाय सत्वरम्}% १८१

\twolineshloka
{राज्यं निवेदयामास चाष्टाङ्गं राघवे तदा}
{श्रुत्वा प्राप्तं ततो रामं सर्वो वै माथुरो जनः}% १८२

\twolineshloka
{वर्णा ब्राह्मणभूयिष्ठा द्रष्टुमेनं समागताः}
{सम्भाष्य प्रकृतीः सर्वा नैगमान्ब्राह्मणैः सह}% १८३

\twolineshloka
{दिनानि पञ्चोषित्वाऽत्र रामो गन्तुं मनो दधे}
{शत्रुघ्नश्च ततो रामे वाजिनोथ गजांस्तथा}% १८४

\twolineshloka
{कृताकृतं च कनकं तत्रोपायनमाहरत्}
{रामस्त्वाह ततः प्रीतः सर्वमेतन्मया तव}% १८५

\twolineshloka
{दत्तं पुत्रौ तेऽभिषिञ्च राजानौ माथुरे जने}
{एवमुक्त्वा ततो रामः प्राप्तो मध्यन्दिने रवौ}% १८६

\twolineshloka
{महोदयं समासाद्य गङ्गातीरे स वामनम्}
{प्रतिष्ठाप्य द्विजानाह भाविनः पार्थिवांस्तथा}% १८७

\twolineshloka
{मया कृतोऽयं धर्मस्य सेतुर्भूतिविवर्धनः}
{प्राप्ते काले पालनीयो न च लोप्यः कथञ्चन}% १८८

\twolineshloka
{प्रसारितकरेणैवं प्रार्थनैषा मया कृता}
{नृपाः कृते मयार्थित्वे यत्क्षेमं क्रियतामिह}% १८९

\twolineshloka
{नित्यं दैनन्दिनीपूजा कार्या सर्वैरतन्द्रितैः}
{ग्रामान्दत्वा धनं तच्च लङ्काया आहृतं च यत्}% १९०

\twolineshloka
{प्रेषयित्वा च किष्किन्धां सुग्रीवं वानरेश्वरम्}
{अयोध्यामागतो रामः पुष्पकं तमथाब्रवीत्}% १९१

\twolineshloka
{नागन्तव्यं त्वया भूयस्तिष्ठ यत्र धनेश्वरः}
{कृतकृत्यस्ततो रामः कर्तव्यं नाप्यमन्यत}% १९२

\uvacha{पुलस्त्य उवाच}

\twolineshloka
{एवन्ते भीष्म रामस्य कथायोगेन पार्थिव}
{उत्पत्तिर्वामनस्योक्ता किं भूयः श्रोतुमिच्छसि}% १९३

\twolineshloka
{कथयामि तु तत्सर्वं यत्र कौतूहलं नृप}
{सर्वं ते कीर्त्तयिष्यामि येनार्थी नृपनन्दन}% १९४

{॥इति श्रीपाद्मपुराणे प्रथमे सृष्टिखण्डे वामनप्रतिष्ठानामाष्टत्रिंशोऽध्यायः॥३८॥}


    \dnsub{प्रथमोऽध्यायः}%\resetShloka

\src{पद्म-पुराणम्}{पातालखण्डः}{अध्यायाः १--६८}{}
% \tags{concise, complete}
\notes{Rāma returns to Ayodhyā, reunites with Bharata, and is consecrated king. Sage Agastya visits and narrates Rāvaṇa’s story, advising Rāma to perform a horse sacrifice. Śatrughna is appointed to guard the sacrificial horse, leading to a campaign across various regions, marked by battles, pilgrimages, and moral episodes. Numerous kings are encountered and defeated, including Subāhu, Damana, and Vīramaṇi. Kuśa and Lava bind the horse, leading to confrontations with Rāma’s army. Eventually, the sacrifice is completed with the singing of the Rāmāyaṇa by the twins.}
\textlink{https://sa.wikisource.org/wiki/पद्मपुराणम्/खण्डः_५_(पातालखण्डः)/अध्यायः_००१}
\translink{https://www.wisdomlib.org/hinduism/book/the-padma-purana/d/doc365311.html}

\storymeta



\twolineshloka
{नारायणं नमस्कृत्य नरं चैव नरोत्तमम्}
{देवींसरस्वतीं व्यासं ततो जयमुदीरयेत्}% १

\uvacha{ऋषय ऊचुः}

\twolineshloka
{श्रुतं सर्वं महाभाग स्वर्गखण्डं मनोहरम्}
{त्वत्तोऽधुना वदायुष्मञ्छ्रीरामचरितं हि नः}% २

\uvacha{सूत उवाच}

\twolineshloka
{अथैकदा धराधारं पृष्टवान्भुजगेश्वरम्}
{वात्स्यायनो मुनिवरः कथामेतां सुनिर्मलाम्}% ३

\uvacha{श्रीवात्स्यायन उवाच}

\twolineshloka
{शेषाशेष कथास्त्वत्तो जगत्सर्गलयादिकाः}
{भूगोलश्च खगोलश्च ज्योतिश्चक्रविनिर्णयः}% ४

\twolineshloka
{महत्तत्त्वादिसृष्टीनां पृथक्तत्त्वविनिर्णयः}
{नानाराजचरित्राणि कथितानि त्वयानघ}% ५

\twolineshloka
{सूर्यवंशभवानां च राज्ञां चारित्रमद्भुतम्}
{तत्रानेकमहापापहरा रामकृता कथा}% ६

\twolineshloka
{तस्य वीरस्य रामस्य हयमेधकथा श्रुता}
{सङ्क्षेपतो मया त्वत्तस्तामिच्छामि सविस्तराम्}% ७

\twolineshloka
{या श्रुता संस्मृता चोक्ता महापातकहारिणी}
{चिन्तितार्थप्रदात्री च भक्तचित्तप्रतोषदा}% ८

\uvacha{शेष उवाच}

\twolineshloka
{धन्योसि द्विजवर्य त्वं यस्य ते मतिरीदृशी}
{रघुवीरपदद्वन्द्व मकरन्द स्पृहावती}% ९

\twolineshloka
{वदन्ति मुनयः सर्वे साधूनां सङ्गमं वरम्}
{यस्मात्पापक्षयकरी रघुनाथकथा भवेत्}% १०

\twolineshloka
{त्वया मेऽनुग्रहः सृष्टो यद्रामः स्मारितः पुनः}
{सुरासुरकिरीटौघ मणिनीराजिताङ्घ्रिकः}% ११

\twolineshloka
{रावणारिकथा वार्द्धौ मशको मादृशः कियान्}
{यत्र ब्रह्मादयो देवा मोहिता न विदन्त्यपि}% १२

\twolineshloka
{तथापि भो मया तुभ्यं वक्तव्यं स्वीयशक्तितः}
{पक्षिणः स्वगतिं श्रित्वा खे गच्छन्ति सुविस्तरे}% १३

\twolineshloka
{चरितं रघुनाथस्य शतकोटिप्रविस्तरम्}
{येषां वै यादृशी बुद्धिस्ते वदन्त्येव तादृशम्}% १४

\twolineshloka
{रघुनाथसतीकीर्तिर्मद्बुद्धिं निर्मलीमसाम्}
{करिष्यति स्वसम्पर्कात्कनकं त्वनलो यथा}% १५

\uvacha{सूत उवाच}

\twolineshloka
{इत्युक्त्वा तं मुनिवरं ध्यानस्तिमितलोचनः}
{ज्ञानेनालोकयाञ्चक्रे कथां लोकोत्तरां शुभाम्}% १६

\twolineshloka
{गद्गदस्वरसंयुक्तो महाहर्षाङ्किताङ्गकः}
{कथयामास विशदां कथां दाशरथेः पुनः}% १७

\uvacha{शेष उवाच}

\twolineshloka
{लङ्केश्वरे विनिहते देवदानवदुःखदे}
{अप्सरोगणवक्त्राब्जचन्द्रमः कान्तिहर्तरि}% १८

\twolineshloka
{सुराः सर्वे सुखं प्रापुरिन्द्र प्रभृतयस्तदा}
{सुखं प्राप्ताः स्तुतिं चक्रुर्दासवत्प्रणतिं गताः}% १९

\twolineshloka
{लङ्कायां च प्रतिष्ठाप्य धर्मयुक्तं बिभीषणम्}
{सीतयासहितो रामः पुष्पकं समुपाश्रितः}% २०

\twolineshloka
{सुग्रीवहनुमत्सीतालक्ष्मणैः संयुतस्तदा}
{बिभीषणोऽपि सचिवैरन्वगाद्विरहोत्सुकः}% २१

\twolineshloka
{लङ्कां स पश्यन्बहुधा भग्नप्राकारतोरणाम्}
{दृष्ट्वाऽशोकवनं तत्र सीतास्थानं मुमूर्च्छ ह}% २२

\twolineshloka
{शिंशपांस्तत्र वृक्षांश्च पुष्पितान्कोरकैर्युतान्}
{राक्षसीभिः समाकीर्णान्मृताभिर्हनुमद्भयात्}% २३

\twolineshloka
{इत्थं सर्वं विलोक्याशु रामः प्रायात्पुरीं प्रति}
{ब्रह्मादिदेवैः सहितः स्वीयस्वीयविमानकैः}% २४

\twolineshloka
{देवदुन्दुभिनिर्घोषाञ्छृण्वञ्छ्रोत्रसुखावहान्}
{तथैवाप्सरसां नृत्यैः पूज्यमानो रघूत्तमः}% २५

\twolineshloka
{सीतायै दर्शयन्मार्गे तीर्थान्याश्रमवन्ति च}
{मुनींश्च मुनिपुत्रांश्च मुनिपत्नीः पतिव्रताः}% २६

\twolineshloka
{यत्रयत्र कृतावासाः पूर्वं रामेण धीमता}
{तान्सर्वान्दर्शयामास लक्ष्मणेन समन्वितः}% २७

\twolineshloka
{इत्येवं दर्शयंस्तस्यै रामोऽद्राक्षीत्स्वकां पुरीम्}
{तस्याः पुनः समीपे तु नन्दिग्रामं ददर्श ह}% २८

\twolineshloka
{यत्र वै भरतो राजा पालयन्धर्ममास्थितः}
{भ्रातुर्वियोगजनितं दुःखचिह्नं वहन्बहु}% २९

\twolineshloka
{गर्तशायी ब्रह्मचारी जटावल्कलसंयुतः}
{कृशाङ्गयष्टिर्दुःखार्तः कुर्वन्रामकथां मुहुः}% ३०

\twolineshloka
{यवान्नमपि नो भुङ्क्ते जलं पिबति नो मुहुः}
{उद्यन्तं सवितारं यो नमस्कृत्य ब्रवीति च}% ३१

\twolineshloka
{जगन्नेत्रसुरस्वामिन्हर मे दुष्कृतं महत्}
{मदर्थे रामचन्द्रोऽपि जगत्पूज्यो वनं ययौ}% ३२

\twolineshloka
{सीतया सुकुमाराङ्ग्या सेव्यमानोऽटवीं गतः}
{या सीता पुष्पपर्यङ्के वृन्तमासाद्य दुःखिता}% ३३

\twolineshloka
{या सीता रविसन्तापं कदापि प्राप नो सती}
{मदर्थे जानकी सा च प्रत्यरण्यं भ्रमत्यहो}% ३४

\twolineshloka
{या सीता राजवृन्दैश्च न दृष्टा नयनैः कदा}
{सा सीता दृश्यते नूनं किरातैः कालरूपिभिः}% ३५

\twolineshloka
{या सीता मधुरं त्वन्नं भोजिता न बुभुक्षति}
{सा सीताद्य वनस्थानि फलानि प्रार्थयत्यहो}% ३६

\twolineshloka
{इत्येवमन्वहं सूर्यमुपस्थाय वदत्यदः}
{प्रातःप्रातर्महाराजो भरतो रामवल्लभः}% ३७

\twolineshloka
{यश्चोच्यमानः सचिवैः समदुःखसुखैर्बुधैः}
{नीतिज्ञैः शास्त्रनिपुणैरिति प्रोवाच तान्नृपः}% ३८

\twolineshloka
{अमात्या दुर्भगं मां किं प्रब्रूत पुरुषाधमम्}
{मदर्थे मेऽग्रजो रामो वनं प्राप्यावसीदति}% ३९

\twolineshloka
{दुर्भगस्य मम प्रस्वाः पापमार्जनमादरात्}
{करोमि रामचन्द्राङ्घ्रिं स्मारं स्मारं सुमन्त्रिणः}% ४०

\twolineshloka
{धन्या सुमित्रा सुतरां वीरसूः स्वपतिप्रिया}
{यस्यास्तनूजो रामस्य चरणौ सेवतेऽन्वहम्}% ४१

\twolineshloka
{यत्र ग्रामे स्थितो नूनं भरतो भ्रातृवत्सलः}
{विलापं प्रकरोत्युच्चैस्तं ग्रामं स ददर्श ह}% ४२

{॥इति श्रीपद्मपुराणे पातालखण्डे शेषवात्स्यायनसंवादे रामाश्वमेधे रघुनाथस्य भरतावासनन्दिग्रामदर्शनो नाम प्रथमोऽध्यायः॥१॥}

\dnsub{द्वितीयोऽध्यायः}%\resetShloka

\uvacha{शेष उवाच}

\twolineshloka
{अथ तद्दर्शनोत्कण्ठा विह्वलीकृतचेतसा}
{पुनः पुनः स्मृतो भ्राता भरतो धार्मिकाग्रणीः}% १

\twolineshloka
{उवाच च हनूमन्तं बलवन्तं समीरजम्}
{प्रस्फुरद्दशनव्याज चन्द्रकान्तिहतान्धकः}% २

\twolineshloka
{शृणु वीर हनूमंस्त्वं मद्गिरं भ्रातृनोदिताम्}
{चिरन्तनवियोगेन गद्गदीकृतविह्वलाम्}% ३

\twolineshloka
{गच्छ तं भ्रातरं वीर समीरणतनूद्भव}
{मद्वियोगकृशां यष्टिं वपुषो बिभ्रतं हठात्}% ४

\twolineshloka
{यो वल्कलं परीधत्ते जटां धत्ते शिरोरुहे}
{फलानां भक्षणमपि न कुर्याद्विरहातुरः}% ५

\twolineshloka
{परस्त्री यस्य मातेव लोष्टवत्काञ्चनं पुनः}
{प्रजाः पुत्रानिवोद्रक्षेद्बान्धवो मम धर्मवित्}% ६

\twolineshloka
{मद्वियोगजदुःखाग्निज्वालादग्धकलेवरम्}
{मदागमनसन्देश पयोवृष्ट्याशु सिञ्चतम्}% ७

\twolineshloka
{सीतया सहितं रामं लक्ष्मणेन समन्वितम्}
{सुग्रीवादिकपीन्द्रैश्च रक्षोभिः सबिभीषणैः}% ८

\twolineshloka
{प्राप्तं निवेदय सुखात्पुष्पकासनसंस्थितम्}
{येन मे सोऽनुजः शीघ्रं सुखमेति मदागमात्}% ९

\twolineshloka
{इति श्रुत्वा ततो वाक्यं रघुवीरस्य धीमतः}
{जगाम भरतावासं नन्दिग्रामं निदेशकृत्}% १०

\twolineshloka
{गत्वा स नन्दिग्रामं तु मन्त्रिवृद्धैः सुसंयतम्}
{भरतं भ्रातृविरहक्लिन्नं धीमान्ददर्श ह}% ११

\twolineshloka
{कथयन्तं मन्त्रिवृद्धान्रामचन्द्रकथानकम्}
{तदीय पदापाथोज मकरन्दसुनिर्भरः}% १२

\twolineshloka
{नमश्चकार भरतं धर्मं मूर्तियुतं किल}
{विधात्रा सकलांशेन सत्त्वेनैव विनिर्मितम्}% १३

\twolineshloka
{तं दृष्ट्वा भरतः शीघ्रं प्रत्युत्थाय कृताञ्जलि}
{स्वागतं चेति होवाच रामस्य कुशलं वद}% १४

\twolineshloka
{इत्येवं वदतस्तस्य भुजो दक्षिणतोऽस्फुरत्}
{हृदयाच्च गतः शोको हर्षास्रैः पूरिताननः}% १५

\twolineshloka
{विलोक्य तादृशं भूपं प्रत्युवाच कपीश्वरः}
{निकटे हि पुरः प्राप्तं विद्धि रामं सलक्ष्मणम्}% १६

\twolineshloka
{रामागमनसन्देशामृतसिक्तकलेवरः}
{प्रापयद्धर्षपूरं हि सहस्रास्यो न वेद्म्यहम्}% १७

\twolineshloka
{जगाद मम तन्नास्ति यत्तुभ्यं दीयते मया}
{दासोऽस्मि जन्मपर्यन्तं रामसन्देशहारकः}% १८

\twolineshloka
{वसिष्ठोऽपि गृहीत्वार्घ्यं मन्त्रिवृद्धाः सुहर्षिताः}
{जग्मुस्ते रामचन्द्रं च हनुमद्दर्शिताध्वना}% १९

\twolineshloka
{दृष्ट्वा दूरात्समायान्तं रामचन्द्रं मनोरमम्}
{पुष्पकासनमध्यस्थं ससीतं सहलक्ष्मणम्}% २०

\twolineshloka
{रामोऽपि दृष्ट्वा भरतं पादचारेण सङ्गतम्}
{जटावल्कलकौपीन परिधानसमन्वितम्}% २१

\twolineshloka
{अमात्यान्भ्रातृवेषेण समवेषाञ्जटाधरान्}
{नित्यं तपः क्लिष्टतया कृशरूपान्ददर्श ह}% २२

\twolineshloka
{रामोऽपि चिन्तयामास दृष्ट्वा वै तादृशं नृपम्}
{अहो दशरथस्यायं राजराजस्य धीमतः}% २३

\twolineshloka
{पुत्रः पदातिरायाति जटावल्कलवेषभृत्}
{न दुःखं तादृशं मेऽन्यद्वनमध्यगतस्य हि}% २४

\twolineshloka
{यादृशं मद्वियोगेन चैतस्य परिवर्त्तते}
{अहो पश्यत मे भ्राता प्राणात्प्रियतमः सखा}% २५

\twolineshloka
{श्रुत्वा मां निकटे प्राप्तं मन्त्रिवृद्धैः सुहर्षितैः}
{द्रष्टुं मां भरतोऽभ्येति वसिष्ठेन समन्वितः}% २६

\twolineshloka
{इति ब्रुवन्नरपतिः पुष्पकान्नभसोऽङ्गणात्}
{बिभीषणहनूमद्भ्यां लक्ष्मणेन कृतादरः}% २७

\twolineshloka
{यानादवतताराशु विरहात्क्लिन्नमानसः}
{भ्रातर्भ्रातः पुनर्भ्रातर्भ्रातर्भ्रातर्वदन्मुहुः}% २८

\threelineshloka
{दृष्ट्वा समुत्तीर्णमिमं रामचन्द्रं सुरैर्युतम्}
{भरतो भ्रातृविरहक्लिन्नं धीमान्ददर्श ह}
{हर्षाश्रूणि प्रमुञ्चंश्च दण्डवत्प्रणनाम ह}% २९

\twolineshloka
{रघुनाथोऽपि तं दृष्ट्वा दण्डवत्पतितं भुवि}
{उत्थाप्य जगृहे दोर्भ्यां हर्षालोकसमन्वितः}% ३०

\twolineshloka
{उत्थापितोऽपि हि भृशं नोदतिष्ठद्रुदन्मुहुः}
{रामचन्द्रपदाम्भोजग्रहणासक्तबाहुभृत्}% ३१

\uvacha{भरत उवाच}

\twolineshloka
{दुराचारस्य दुष्टस्य पापिनो मे कृपां कुरु}
{रामचन्द्र महाबाहो कारुण्यात्करुणानिधे}% ३२

\twolineshloka
{यस्ते विदेहजा पाणिस्पर्शं क्रूरममन्यत}
{स एव चरणो राम वने बभ्राम मत्कृते}% ३३

\twolineshloka
{इत्युक्त्वाश्रुमुखो दीनः परिरभ्य पुनः पुनः}
{प्राञ्जलिः पुरतस्तस्थौ हर्षविह्वलिताननः}% ३४

\twolineshloka
{रघुनाथस्तमनुजं परिष्वज्य कृपानिधिः}
{प्रणम्य च महामन्त्रिमुख्यानापृच्छ्य सादरम्}% ३५

\twolineshloka
{भरतेन समं भ्रात्रा पुष्पकासनमास्थितः}
{सीतां ददर्श भरतो भ्रातृपत्नीमनिन्दिताम्}% ३६

\twolineshloka
{अनसूयामिवात्रेः किं लोपामुद्रां घटोद्भुवः}
{पतिव्रतां जनकजाममन्यतननाम च}% ३७

\twolineshloka
{मातः क्षमस्व यदघं मया कृतमबुद्धिना}
{त्वत्सदृश्यः पतिपराः सर्वेषां साधुकारिकाः}% ३८

\twolineshloka
{जानक्यापि महाभागा देवरं वीक्ष्य सादरम्}
{आशीर्भिरभियुज्याथ समपृच्छदनामयम्}% ३९

\twolineshloka
{विमानवरमारूढास्ते सर्वे नभसोऽङ्गणे}
{क्षणादालोकयाञ्चक्रे निकटे स्वपितुः पुरीम्}% ४०

{॥इति श्रीपद्मपुराणे पातालखण्डे रामाश्वमेधे शेषवात्स्यायनसंवादे रामराजधानीदर्शनो नाम द्वितीयोऽध्यायः॥२॥}

\dnsub{तृतीयोऽध्यायः}%\resetShloka

\uvacha{शेष उवाच}

\twolineshloka
{दृष्ट्वा रामो राजधानीं निजलोकनिवासिनीम्}
{जहर्ष मतिमान्वीरश्चिरदर्शनलालसः}% १

\twolineshloka
{भरतोऽपि स्वकं मित्रं सुमुखं नगरं प्रति}
{प्रेषयामास सचिवं नगरोत्सवसिद्धये}% २

\uvacha{भरत उवाच}

\twolineshloka
{कुर्वन्तु लोकास्त्वरितं रघुनाथागमोत्सवम्}
{मन्दिरे मन्दिरे रम्यं कृतकौतुकचित्रकम्}% ३

\twolineshloka
{विपांसुका राजमार्गाश्चन्दनद्रवसिञ्चिताः}
{प्रसूनभरसङ्कॢप्ता हृष्टपुष्टजनावृताः}% ४

\twolineshloka
{विचित्रवर्णध्वजभा चित्रिताखिलस्वाङ्गणाः}
{मेघागमे धनुरिव पश्यन्त्वेव वलीमुखाः}% ५

\twolineshloka
{प्रतिगेहं तु लोकानां कारयन्त्वगरूक्षणम्}
{यद्धूमं वीक्ष्य शिखिनो नृत्यं कुर्वन्तु लीलया}% ६

\twolineshloka
{हस्तिनो मम शैलाभानाधोरणसुयन्त्रितान्}
{विचित्रयन्तु बहुशो गैरिकाद्युपधातुभिः}% ७

\twolineshloka
{वाजिनश्चित्रिता भूयः सुशोभन्तु मनोजवाः}
{यद्वेगवीक्षणादेव गर्वं त्यजति स्वर्हयः}% ८

\twolineshloka
{कन्याः सहस्रशो रम्याः सर्वाभरणभूषिताः}
{गजोपरि समारूढा मुक्ताभिर्विकिरन्तु च}% ९

\twolineshloka
{ब्राह्मण्यः पात्रहस्ताश्च दूर्वाहारिद्रसंयुताः}
{सुवासिन्यो महाराजं रामं नीराजयन्तु ताः}% १०

\twolineshloka
{कौसल्यापुत्रसंयोगसन्देश विधुरा सती}
{हर्षं प्राप्नोतु सुकृशा तदीक्षणसुलालसा}% ११

\twolineshloka
{इत्येवमादिरचनाः पुरशोभाविधायिकाः}
{करोतु जनता हृष्टा रामस्यागमनं प्रति}% १२

\uvacha{शेष उवाच}

\twolineshloka
{इति श्रुत्वा ततो वाक्यं सुमुखो मन्त्रवित्तमः}
{प्रययौ नगरीं कर्तुं कृतकौतुकतोरणाम्}% १३

\twolineshloka
{गत्वाथ नगरीं तां वै मन्त्री तु सुमुखाभिधः}
{ख्यापयामास लोकानां रामागममहोत्सवम्}% १४

\twolineshloka
{लोकाः श्रुत्वा पुरीं प्राप्तं रघुनाथं सुहर्षिताः}
{पूर्वं तदीय विरहत्यक्तभोगसुखादयः}% १५

\twolineshloka
{ब्राह्मणा वेदसम्पन्नाः पवित्रा दर्भपाणयः}
{धौतोत्तरीयवलिता जग्मुः श्रीरघुनायकम्}% १६

\twolineshloka
{क्षत्रिया ये शूरतमा धनुर्बाणधरा वराः}
{सङ्ग्रामे बहुशो वीरा जेतारो ययुरप्यमुम्}% १७

\twolineshloka
{वैश्या धनसमृद्धाश्च मुद्राशोभितपाणयः}
{शुभ्रवस्त्रपरीधाना अभिजग्मुर्नरेश्वरम्}% १८

\twolineshloka
{शूद्रा द्विजेषु ये भक्ताः स्वीयाचारसुनिष्ठिताः}
{वेदाचाररता ये वै तेऽपिजग्मुः पुरीपतिम्}% १९

\twolineshloka
{ये ये वृत्तिकरा लोकाः स्वे स्वे कर्मण्यधिष्ठिताः}
{स्वकं वस्तु समादाय ययुः श्रीरामभूपतिम्}% २०

\twolineshloka
{इत्थं भूपतिसन्देशात्प्रमोदाप्लवसंयुताः}
{नाना कौतुकसंयुक्ता आजग्मुर्मनुजेश्वरम्}% २१

\uvacha{शेष उवाच}

\twolineshloka
{रघुनाथोऽपि सकलैर्दैवतैः स्वस्वयानगैः}
{परीतः प्रविवेशोच्चैः पुरीं रचितमोहनाम्}% २२

\twolineshloka
{प्लवङ्गाः प्लवनैर्युक्ता आकाशपथचारिणः}
{स्वस्वशोभापरीताङ्गाश्चानुजग्मुः पुरोत्तमम्}% २३

\twolineshloka
{पुष्पकादवरुह्याशु नरयानमथारुहत्}
{सीतया सहितो रामः परिवारसमावृतः}% २४

\twolineshloka
{अयोध्यां प्रविवेशाथ कृतकौतुकतोरणाम्}
{हृष्टपुष्टजनाकीर्णामुत्सवैः परीभूषिताम्}% २५

\twolineshloka
{वीणापणवभेर्यादिवादित्रैराहतैर्भृशम्}
{शोभमानः स्तूयमानः सूतमागधबन्दिभिः}% २६

\twolineshloka
{जय राघवरामेति जय सूर्यकुलाङ्गद}
{जय दाशरथे देव जयताल्लोकनायकः}% २७

\twolineshloka
{इति शृण्वञ्छुभां वाचं पौराणां हर्षिताङ्गिनाम्}
{रामदर्शनसञ्जात पुलकोद्भेद शोभिनाम्}% २८

\twolineshloka
{प्रविवेश वरं मार्गं रथ्याचत्वरभूषितम्}
{चन्दनोदकसंसिक्तं पुष्पपल्लवसंयुतम्}% २९

\twolineshloka
{तदा पौराङ्गनाः काश्चिद्गवाक्षबिलमाश्रिताः}
{रघुनाथस्वरूपेक्षा जातकामा अथाब्रुवन्}% ३०

\uvacha{पौराङ्गना ऊचुः}

\twolineshloka
{धन्या अभूवन्बत भिल्लकन्या वनेषु या राममुखारविन्दम्}
{स्वलोचनेन्दीवरकैरथापिबन्स्वभाग्यसञ्जातमहोदया इमाः}% ३१

\twolineshloka
{धन्यं मुखं पश्यत वीरधाम्नः श्रीरामदेवस्य सरोजनेत्रम्}
{यद्दर्शनं धातृमुखाः सुरा अपि प्रापुर्महद्भाग्ययुता वयन्त्वहो}% ३२

\twolineshloka
{एतन्मुखं पश्यत चारुहासं किरीटसंशोभिनिजोत्तमाङ्गम्}
{बन्धूकधिक्कारलसच्छविप्रदं दन्तच्छदं बिभ्रतमुच्चनासम्}% ३३

\twolineshloka
{इति गदितवतीस्ताः स्नेहभारेण रामा नलिनदलसदृक्षैर्नेत्रपातैर्निरीक्ष्य}
{निखिलगुरुरनूनप्रेमभारं नृलोकं जननिगृहमियेष प्रोषिताङ्गेन हृष्टः}% ३४

{॥इति श्रीपद्मपुराणे पातालखण्डे शेषवात्स्यायनसंवादे रामाश्वमेधे रघुनाथस्य पुरप्रवेशनं नाम तृतीयोऽध्यायः॥३॥}

\dnsub{चतुर्थोऽध्यायः}%\resetShloka

\uvacha{वात्स्यायन उवाच}

\twolineshloka
{भुजगाधीश्वरेशान धराभारधरक्षम}
{शृण्वेकं संशयं मह्यं कृपया कथयस्व तम्}% १

\twolineshloka
{रघुनाथस्य गमनं वनं प्रति यदा ह्यभूत्}
{तदा प्रभृति देहेन स्थिता शून्येन चेतसा}% २

\twolineshloka
{तद्विप्रयोगविधुरा कृशदेहातिदुःखिता}
{सुमुखान्मन्त्रिणः श्रुत्वा रघुनाथं समागतम्}% ३

\twolineshloka
{कथं जहर्ष किमभूत्कीदृशं तत्र चिह्नितम्}
{रामचन्द्रस्य सन्देशहर्तारं किमुवाच सा}% ४

\twolineshloka
{एतन्मे संशयं छिन्धि रघुनाथगुणोदयम्}
{यथावच्छृण्वते मह्यं कथयस्व प्रसादतः}% ५

\uvacha{शेष उवाच}

\twolineshloka
{साधुपृष्टं महाभाग द्विजवर्यपुरस्कृत}
{तन्मे निगदतः साक्षाच्छृणुष्वैकमनाः किल}% ६

\twolineshloka
{सा वै तद्वदनाम्भोज च्युतं रामागमामृतम्}
{पीत्वा पीत्वा बभूवाहो स्थगिताङ्गेन विह्वला}% ७

\twolineshloka
{किं मे स्वप्नो विमूढायाः किं वा भ्रमकरं वचः}
{ममवै मन्दभाग्यायाः कथं रामेक्षणं पुनः}% ८

\twolineshloka
{बहुना तपसा कृत्वा प्राप्तोऽयं वै सुतः शिशुः}
{केनचिन्मम पापेन विप्रयोगं गतः पुनः}% ९

\twolineshloka
{सुमन्त्रिन्कुशली रामः सीतालक्ष्मणसंयुतः}
{कथं मां स्मरते वीरो वनचारी सुदुःखिताम्}% १०


\threelineshloka
{इति सा विललापोच्चै रघुनाथस्मृतिं गता}
{न निवेद निजं किञ्चित्परकीयं विमोहिता}
{सुमुखोऽपि तथा दृष्ट्वा दुःखितां मातरं भृशम्}% ११

\twolineshloka
{वीजयामास वासोग्रैः संज्ञामाप च सा पुनः}
{उवाच जननीं सौम्यं वचोहर्षकरं मुहुः}% १२

\twolineshloka
{रघुनाथागमस्मार हृष्टां तां व्यदधात्पुनः}
{मातर्विद्धि गृहं प्राप्तं रघुनाथं सलक्ष्मणम्}% १३

\twolineshloka
{सीतया सहितं पश्य चाशीर्भिरभियुङ्क्ष्व च}
{इति तथ्यं वचः श्रुत्वा सुमुखेन प्रभाषितम्}% १४

\twolineshloka
{यादृशं हर्षमापेदे तादृशं वेद्म्यहं नहि}
{उत्थाय चाजिरे प्राप्ता रोमाञ्चिततनूरुहा}% १५

\twolineshloka
{हर्षविह्वलिताङ्ग्यश्रु मुञ्चन्ती राममैक्षत}
{तावत्स रामो राजेन्द्रो नरयानमधिश्रितः}% १६

\twolineshloka
{प्राप्तः स्वमातुर्भवनं कैकेय्याः सुनयः पुरः}
{कैकेय्यपि त्रपाभारनम्रा रामं पुरःस्थितम्}% १७

\twolineshloka
{नोवाच किञ्चिन्महतीं चिन्तां प्राप्तवती मुहुः}
{सूर्यवंशध्वजो रामो मातरं वीक्ष्य लज्जिताम्}% १८

\onelineshloka*
{उवाच सान्त्वयंस्तां च वाक्यैर्विनयमिश्रितैः}

\uvacha{श्रीराम उवाच}

\onelineshloka
{मातर्मया वनं गत्वा सर्वमाचरितं तथा}% १९

\twolineshloka
{अधुना करवै किं वा त्वदाज्ञातो जनन्यहो}
{मया न्यूनं कृतं नास्ति कथं मां नेक्ष्यसे पुनः}% २०

\twolineshloka
{आशीर्भिरभिनन्द्यैनं भरतं मां च वीक्षय}
{इति श्रुत्वापि तद्वाक्यं सा नम्रवदनानघ}% २१

\twolineshloka
{शनैः शनैः प्रत्युवाच राम गच्छ स्वमालयम्}
{रामोऽपि श्रुत्वा वचनं जनन्याः पुरुषोत्तमः}% २२

\twolineshloka
{नमस्कृत्य ययौ गेहं सुमित्रायाः कृपानिधिः}
{सुमित्रा पुत्रसहितं रामं दृष्ट्वा महामनाः}% २३

\twolineshloka
{चिरञ्जीव चिरञ्जीव ह्याशीर्भिरिति चाभ्यधात्}
{मातुश्च रामभद्रोऽपि चरणौ प्रणिपत्य च}% २४

\twolineshloka
{परिष्वज्य मुदायुक्तो जगाद वचनं पुनः}
{रत्नगर्भे मम भ्रात्रा केनापि न कृतं तथा}% २५

\onelineshloka
{यथायमकरोद्धीमान्ममदुःखापनोदनम्}% २६

\twolineshloka
{रावणेन हृता सीता मया यत्प्राप्यते पुनः}
{मातस्तत्सर्वमाविद्धि लक्ष्मणस्य विचेष्टितम्}% २७

\twolineshloka
{दत्तामाशिषमागृह्य शिरसायं सुमित्रया}
{निजमातुश्च भवनं प्रययौ विबुधैर्वृतः}% २८

\twolineshloka
{मातरं वीक्ष्य हृषितां निजदर्शनलालसाम्}
{स्वयानादवरुह्याशु चरणावग्रहीद्धरिः}% २९

\twolineshloka
{माता तद्दर्शनोत्कण्ठा विह्वलीकृतमानसा}
{परिष्वज्य परिष्वज्य रामं मुदमवाप सा}% ३०

\twolineshloka
{शरीरे रोमहर्षोऽभूद्गद्गदा वागभूत्तदा}
{हर्षाश्रूणि तु सोष्णानि प्रवाहं प्रापुरापदात्}% ३१

\twolineshloka
{जननीं वीक्ष्य विनयी ताटङ्कद्वयवर्जिताम्}
{कराकल्प पदाकल्परहितां बिभ्रतीं तनुम्}% ३२

\twolineshloka
{किञ्चित्स्वदर्शनाद्धृष्टां कृशाङ्गीं तां स शोकभाक्}
{दुःखस्य समयो नायमिति मत्वा जगाद ताम्}% ३३

\uvacha{श्रीराम उवाच}

\twolineshloka
{मातर्मया त्वच्चरणौ चिरकालं न सेवितौ}
{ततः क्षमस्वापराधं भाग्यहीनस्य वै मम}% ३४

\twolineshloka
{ये पुत्रा मातापित्रोर्न शुश्रूषायां समुत्सुकाः}
{ते मन्तव्याः परा मातः कीटका रेतसो भवाः}% ३५

\twolineshloka
{किं कुर्वे जनकाज्ञातो गतो वै दण्डकं वनम्}
{तत्रापि त्वत्कृपापाङ्गात्तीर्णोऽस्मि दुःखसागरम्}% ३६

\twolineshloka
{रावणेन हृता सीता लङ्कायां गमिता पुनः}
{त्वत्कृपातो मया लब्धा तं हत्वा राक्षसेश्वरम्}% ३७

\twolineshloka
{सीतेयं त्वच्चरणयोः पतिता वै पतिव्रता}
{सम्भावयाशु चकितां त्वत्पादार्पितमानसाम्}% ३८

\twolineshloka
{इति श्रुत्वा तु तद्वाक्यं पादयोः पतितां स्नुषाम्}
{आशीर्भिरभियुज्यैनां बभाषे तां पतिव्रताम्}% ३९

\twolineshloka
{सीते स्वपतिना सार्द्धं चिरं विलस भामिनि}
{पुत्रौ प्रसूय च कुलं स्वकं पावय पावने}% ४०

\twolineshloka
{त्वत्सदृश्यः पतिपराः पतिदुःखसुखानुगाः}
{भवन्ति दुःखभागिन्यो न हि सत्यं जगत्त्रये}% ४१

\twolineshloka
{विदेहपुत्रि स्वकुलं त्वया पावितमात्मना}
{रामपादाब्जयुगलमनुयान्त्या महावनम्}% ४२

\twolineshloka
{किं चित्रं यत्पुमांसस्तु वैरिकोटिप्रभञ्जनाः}
{येषां गेहे सती भार्या स्वपतिप्रियवाञ्छिका}% ४३

\twolineshloka
{इत्युक्त्वा रघुनाथस्य भार्यामञ्चितलोचनाम्}
{तूष्णीं बभूव हृषिता प्रहृष्टस्वतनूरुहा}% ४४

\twolineshloka
{अथ भ्रातास्य भरतः पित्रा दत्तं निजं महत्}
{राज्यं निवेदयामास रामचन्द्राय धीमते}% ४५

\twolineshloka
{मन्त्रिणस्ते प्रहृष्टाङ्गा दैवज्ञान्मन्त्रकोविदान्}
{आहूय सुमुहूर्तन्ते पप्रच्छुः परमादरात्}% ४६

\twolineshloka
{शुभे मुहूर्ते सुदिने शुभनक्षत्रसंयुते}
{अभिषेकं महाराज्ये कारयामासुरुद्यताः}% ४७

\twolineshloka
{सप्तद्वीपवतीं पृथ्वीं व्याघ्रचर्मणि सुन्दरे}
{लिखित्वोपरि राजेन्द्रो महाराजोधितस्थिवान्}% ४८

\twolineshloka
{तद्दिनादेव साधूनां मनांसि प्रमुदं ययुः}
{दुष्टानां चेतसो ग्लानिरभवत्परतापिनाम्}% ४९

\twolineshloka
{स्त्रियस्तु पतिभक्त्या च पतिव्रतपरायणाः}
{मनसापि कदा पापं नाचरन्ति जना मुने}% ५०

\twolineshloka
{दैत्यादेवास्तथा नागा यक्षासुरमहोरगाः}
{सर्वे न्यायपथे स्थित्वा रामाज्ञां शिरसा दधुः}% ५१

\twolineshloka
{परोपकरणेयुक्ताः स्वधर्मसुखनिर्वृताः}
{विद्याविनोदगमिता दिनरात्रिक्षणाः शुभाः}% ५२

\twolineshloka
{वातोऽपि मार्गसंस्थानां बलान्नाहरते महान्}
{वासांस्यपि तु सूक्ष्माणि तत्र चौरकथा नहि}% ५३

\twolineshloka
{धनदो ह्यर्थिनां रामः कारुण्यश्च कृपानिधिः}
{भ्रातृभिः सहितो नित्यं गुरुदेवस्तुतिं व्यधात्}% ५४

{॥इति श्रीपद्मपुराणे पातालखण्डे शेषवात्स्यायनसंवादे रामाश्वमेधे रघुवरस्य राज्याभिषेको नाम चतुर्थोऽध्यायः॥४॥}

\dnsub{पञ्चमोऽध्यायः}%\resetShloka

\uvacha{शेष उवाच}

\twolineshloka
{अथाभिषिक्तं रामं तु तुष्टुवुः प्रणताः सुराः}
{रावणाभिधदैत्येन्द्र वधहर्षितमानसाः}% १

\uvacha{देवा ऊचुः}

\twolineshloka
{जय दाशरथे सुरार्तिहञ्जयजय दानववंशदाहक}
{जय देववराङ्गनागणग्रहणव्यग्रकरारिदारक}% २

\twolineshloka
{तवयद्दनुजेन्द्र नाशनं कवयो वर्णयितुं समुत्सुकाः}
{प्रलये जगतान्ततीः पुनर्ग्रससे त्वं भुवनेशलीलया}% ३

\twolineshloka
{जय जन्मजरादिदुःखकैः परिमुक्तप्रबलोद्धरोद्धर}
{जय धर्मकरान्वयाम्बुधौ कृतजन्मन्नजरामराच्युत}% ४

\twolineshloka
{तव देववरस्य नामभिर्बहुपापा अपि ते पवित्रिताः}
{किमु साधुद्विजवर्यपूर्वकाः सुतनुं मानुषतामुपागताः}% ५

\twolineshloka
{हरविरिञ्चिनुतं तव पादयोर्युगलमीप्सितकामसमृद्धिदम्}
{हृदि पवित्रयवादिकचिह्नितैः सुरचितं मनसा स्पृहयामहे}% ६

\twolineshloka
{यदि भवान्न दधात्यभयं भुवो मदनमूर्ति तिरस्करकान्तिभृत्}
{सुरगणा हि कथं सुखिनः पुनर्ननुभवन्ति घृणामय पावन}% ७

\twolineshloka
{यदा यदास्मान्दनुजाहि दुःखदास्तदा तदा त्वं भुवि जन्मभाग्भवेः}
{अजोऽव्ययोऽपीशवरोऽपि सन्विभो स्वभावमास्थाय निजं निजार्चितः}% ८

\twolineshloka
{मृतसुधासदृशैरघनाशनैः सुचरितैरवकीर्य महीतलम्}
{अमनुजैर्गुणशंसिभिरीडितः प्रविश चाशु पुनर्हि स्वकं पदम्}% ९

\twolineshloka
{अनादिराद्योजररूपधारी हारी किरीटी मकरध्वजाभः}
{जयं करोतु प्रसभं हतारिः स्मरारि संसेवितपादपद्मः}% १०

\twolineshloka
{इत्युक्त्वा ते सुराः सर्वे ब्रह्मेन्द्रप्रमुखा मुहुः}
{प्रणेमुररिनाशेन प्रीणिता रघुनायकम्}% ११

\twolineshloka
{इति स्तुत्यातिसंहृष्टो रघुनाथो महायशाः}
{प्रोवाच तान्सुरान्वीक्ष्य प्रणतान्नतकन्धरान्}% १२

\uvacha{श्रीराम उवाच}

\twolineshloka
{सुरा वृणुत मे यूयं वरं किञ्चित्सुदुर्ल्लभम्}
{यं कोऽपि देवो दनुजो न यक्षः प्राप सादरः}% १३

\uvacha{सुरा ऊचुः}

\twolineshloka
{स्वामिन्भगवतः सर्वं प्राप्तमस्माभिरुत्तमम्}
{यदयं निहतः शत्रुरस्माकं तु दशाननः}% १४

\twolineshloka
{यदायदाऽसुरोऽस्माकं बाधां परिदधाति भोः}
{तदा तदेति कर्तव्यमेतावद्वैरिनाशनम्}% १५

\onelineshloka*
{तथेत्युक्त्वा पुनर्वीरः प्रोवाच रघुनन्दनः}

\uvacha{श्रीराम उवाच}

\onelineshloka
{सुराः शृणुत मद्वाक्यमादरेण समन्विताः}% १६

\twolineshloka
{भवत्कृतं मदीयैर्वैगुणैर्ग्रथितमद्भुतम्}
{स्तोत्रं पठिष्यति मुहुः प्रातर्निशि सकृन्नरः}% १७

\twolineshloka
{तस्य वैरि पराभूतिर्न भविष्यति दारुणा}
{न च दारिद्र्यसंयोगो न च व्याधिपराभवौ}% १८

\twolineshloka
{मदीयचरणद्वन्द्वे भक्तिस्तेषां तु भूयसी}
{भविष्यति मुदायुक्ते स्वान्ते पुंसां तु पाठतः}% १९

\twolineshloka
{इत्युक्त्वा सोऽभवत्तूष्णीं नरदेवशिरोमणिः}
{सुराः सर्वे प्रहृष्टास्ते ययुर्लोकं स्वकं स्वकम्}% २०

\twolineshloka
{रघुनाथोऽपि भ्रातॄंस्तान्पालयंस्तातवद्बुधान्}
{प्रजाः पुत्रानिव स्वीयाल्लाँलयँल्लोकनायकः}% २१

\twolineshloka
{यस्मिञ्छासति लोकानां नाकालमरणं नृणाम्}
{न रोगादि पराभूतिर्गृहेषु च महीयसी}% २२

\twolineshloka
{नेतिः कदापि द्दश्येत वैरिजं भयमेव च}
{वृक्षाः सदैव फलिनो मही भूयिष्ठधान्यका}% २३

\twolineshloka
{पुत्रपौत्रपरीवार सनाथी कृतजीवनाः}
{कान्ता संयोगजसुखैर्निरस्तविरहक्लमाः}% २४

\twolineshloka
{नित्यं श्रीरघुनाथस्य पादपद्मकथोत्सुकाः}
{कदापि परनिन्दासु वाचस्तेषां भवन्ति न}% २५

\twolineshloka
{कारवोऽपि कदा पापं नाचरन्ति मनस्यहो}
{रघुनाथकराघातदुःखशङ्काभिशंसिनः}% २६

\twolineshloka
{सीतापतिमुखालोक निश्चलीभूतलोचनाः}
{लोका बभूवुः सततं कारुण्यपरिपूरिताः}% २७

\twolineshloka
{राज्यं प्राप्तमसापत्नं समृद्धबलवाहनम्}
{ऋषिभिर्हृष्टपुष्टैश्च रम्यं हाटकभूषणैः}% २८

\twolineshloka
{सम्पुष्टमिष्टापूर्तानां धर्माणां नित्यकर्तृभिः}
{सदा सम्पन्नसस्यं च सुवसुक्षेत्रसंयुतम्}% २९

\twolineshloka
{सुदेशं सुप्रजं स्वस्थं सुतृणं बहुगोधनम्}
{देवतायतनानां च राजिभिः परिराजितम्}% ३०

\twolineshloka
{सुपूर्णा यत्र वै ग्रामाः सुवित्तर्द्धिविराजिताः}
{सुपुष्पकृत्रिमोद्यानाः सुस्वादुफलपादपाः}% ३१

\twolineshloka
{सपद्मिनीककासारा यत्र राजन्ति भूमयः}
{सदम्भा निम्नगा यत्र न यत्र जनता क्वचित्}% ३२

\twolineshloka
{कुलान्येव कुलीनानां वर्णानां नाधनानि च}
{विभ्रमो यत्र नारीषु न विद्वत्सु च कर्हिचित्}% ३३

\twolineshloka
{नद्यः कुटिलगामिन्यो न यत्र विषये प्रजाः}
{तमोयुक्ताः क्षपा यत्र बहुलेषु न मानवाः}% ३४

\twolineshloka
{रजोयुजः स्त्रियो यत्र नाधर्मबहुला नराः}
{धनैरनन्धो यत्रास्ति जनो नैव च भोजने}% ३५

\twolineshloka
{अनयः स्यन्दनो यत्र न च वैराजपूरुषः}
{दण्डः परशुकुद्दालवालव्यजनराजिषु}% ३६

\twolineshloka
{आतपत्रेषु नान्यत्र क्वचित्क्रोधोपरोधजः}
{अन्यत्राक्षिकवृन्देभ्यः क्वचिन्न परिदेवनम्}% ३७

\twolineshloka
{आक्षिका एव दृश्यन्ते यत्र पाशकपाणयः}
{जाड्यवार्ता जलेष्वेव स्त्रीमध्या एव दुर्बलाः}% ३८

\twolineshloka
{कठोरहृदया यत्र सीमन्तिन्यो न मानवाः}
{औषधेष्वेव यत्रास्ति कुष्ठयोगो न मानवे}% ३९

\twolineshloka
{वेधो यत्र सुरत्नेषु शूलं मूर्तिकरेषु वै}
{कम्पः सात्विकभावोत्थो न भयात्क्वापि कस्यचित्}% ४०

\twolineshloka
{सञ्ज्वरः कामजो यत्र दारिद्र्यकलुषस्य च}
{दुर्ल्लभत्वं सदैवस्य सुकृतेन च वस्तुनः}% ४१

\twolineshloka
{इभा एव प्रमत्ता वै युद्धे वीच्यो जलाशये}
{दानहानिर्गजेष्वेव तीक्ष्णा एव हि कण्टकाः}% ४२

\twolineshloka
{बाणेषु गुणविश्लेषो बन्धोक्तिः पुस्तके दृढा}
{स्नेहत्यागः खलेष्वेव न च वै स्वजने जने}% ४३

\twolineshloka
{तं देशं पालयामास लालयँल्लालिताः प्रजाः}
{धर्मं संस्थापयन्देशे दुष्टे दण्डधरोपमः}% ४४

\twolineshloka
{एवं पालयतो देशं धर्मेण धरणीतलम्}
{सहस्रं च व्यतीयुर्वै वर्षाण्येकादश प्रभोः}% ४५

\twolineshloka
{तत्र नीचजनाच्छ्रुत्वा सीताया अपमानताम्}
{स्वां च निन्दां रजकतस्तां तत्याज रघूद्वहः}% ४६

\twolineshloka
{पृथ्वीं पालयमानस्य धर्मेण नृपतेस्तदा}
{सीतां विरहितामेकां निदेशेन सुरक्षिताम्}% ४७

\twolineshloka
{कदाचित्संसदो मध्ये ह्यासीनस्य महामतेः}
{आजगाम मुनिश्रेष्ठः कुम्भोत्पत्तिर्मुनिर्महान्}% ४८

\twolineshloka
{गृहीत्वार्घ्यं समुत्तस्थौ वसिष्ठेन समन्वितः}
{जनताभिर्महाराजो वार्धिशोषकमागतम्}% ४९

\twolineshloka
{स्वागतेन सुसम्भाव्य पप्रच्छ तमनामयम्}
{सुखोपविष्टं विश्रान्तं बभाषे रघुनन्दनः}% ५०

{॥इति श्रीपद्मपुराणे पातालखण्डे शेषवात्स्यायनसंवादे रामाश्वमेधे अगस्त्यसमागमो नाम पञ्चमोऽध्यायः॥५॥}

\dnsub{षष्ठोऽध्यायः}%\resetShloka

\uvacha{शेष उवाच}

\twolineshloka
{इत्थं स्वागतसन्तुष्टं ब्रह्मचर्यतपोनिधिम्}
{उवाच मतिमान्वीरः सर्वलोकगुरुर्मुनिम्}% १

\twolineshloka
{स्वागतं ते महाभाग कुम्भयोने तपोनिधे}
{त्वद्दर्शनेन सर्वे वै पाविताः सकुटुम्बकाः}% २

\twolineshloka
{कच्चिन्मतिस्ते वेदेषु शास्त्रेषु परिवर्तते}
{त्वत्तपोविघ्नकर्ता वै नास्ति भूमण्डले क्वचित्}% ३

\twolineshloka
{लोपामुद्रा महाभाग या च ते धर्मचारिणी}
{यस्याः पतिव्रता धर्मात्सर्वं भवति शोभनम्}% ४

\twolineshloka
{अपि शंस महाभाग धर्ममूर्ते कृपानिधे}
{अलोलुपस्य किं कार्यं करवाणि मुनीश्वर}% ५

\twolineshloka
{त्वत्तपोयोगतः सर्वं भवति स्वेच्छया बहु}
{तथापि मयि कृत्वैव कृपां शंश मुनीश्वरः}% ६

\uvacha{शेष उवाच}

\twolineshloka
{इत्युक्तो लोकगुरुणा राजराजेन धीमता}
{उवाच रामं लोकेशं विनीततरभाषया}% ७

\uvacha{अगस्त्य उवाच}

\twolineshloka
{स्वामिंस्तव सुदुर्दर्शं दर्शनं दैवतैरपि}
{मत्वा समागतं विद्धि राजराज कृपानिधे}% ८

\twolineshloka
{हतस्त्वया रावणाख्यस्त्वसुरो लोककण्टकः}
{दिष्ट्याद्य देवाः सुखिनो दिष्ट्या राजा बिभीषणः}% ९

\twolineshloka
{राम त्वद्दर्शनान्मेऽद्य गतं वै दुष्कृतं किल}
{सम्पूर्णो मे मनःकोश आनन्देन सुरोत्तम}% १०

\twolineshloka
{इत्युक्त्वा स बभूवाशु तूष्णीं कुम्भसमुद्भवः}
{रामसन्दर्शनाह्लादविह्वलीकृतमानसः}% ११

\twolineshloka
{रामः पप्रच्छ तं भूयो मुनिं ज्ञानविशारदम्}
{लोकातीतं भवद्भावि सर्वं जानासि सर्वतः}% १२

\twolineshloka
{मुने कथय मे सर्वं पृच्छतो हि सुविस्तरम्}
{कोऽसौ मया हतो यो हि रावणो विबुधार्दनः}% १३

\twolineshloka
{कुम्भकर्णोऽपि कस्त्वेष का जातिर्वै दुरात्मनः}
{देवो दैत्यः पिशाचो वा राक्षसो वा महामुने}% १४

\twolineshloka
{सर्वमाख्याहि सर्वज्ञ सर्वं जानासि विस्तरात्}
{अतः कथय मे सर्वं कृपां कृत्वा ममोपरि}% १५

\twolineshloka
{इति श्रुत्वा ततो वाक्यं कुम्भजन्मा तपोनिधिः}
{यत्पृष्टं रघुराजेन प्रवक्तुं तत्प्रचक्रमे}% १६

\twolineshloka
{राजन्सृष्टिकरो ब्रह्मा पुलस्त्यस्तत्सुतोऽभवत्}
{ततस्तु विश्रवा जज्ञे वेदविद्याविशारदः}% १७

\twolineshloka
{तस्य पत्नीद्वयं जातं पातिव्रत्यचरित्रभृत्}
{एका मन्दाकिनी नाम्नी द्वितीया कैकसी स्मृता}% १८

\twolineshloka
{पूर्वस्यां धनदो जज्ञे लोकपालविलासभृत्}
{योऽसौ शिवप्रसादेन लङ्कावासमचीकरत्}% १९

\twolineshloka
{विद्युन्मालिसुतायां तु पुत्रत्रयमभून्महत्}
{रावणः कुम्भकर्णश्च तथा पुण्यो बिभीषणः}% २०

\twolineshloka
{राक्षस्युदरजन्मत्वात्सन्ध्यासमयसम्भवात्}
{द्वयोरधर्मनिपुणा मतिरासीन्महामते}% २१

\twolineshloka
{एकदा तु विमानेन पुष्पकेण सुशोभिना}
{काञ्चनीयोपकल्पेन किङ्किणीजालमालिना}% २२

\twolineshloka
{आरुह्य पितरौ द्रष्टुं प्रायाच्छोभासमन्वितः}
{स्वगणैः संस्तुतो भूत्वा नानारत्नविभूषणैः}% २३

\twolineshloka
{आगत्य पित्रोश्चरणे पतित्वा चिरमात्मजः}
{हर्षविह्वलितात्मा च रोमाञ्चिततनूरुहः}% २४

\twolineshloka
{उवाच मेऽद्य सुदिनं महाभाग्यफलोदयः}
{यन्मे युष्मत्पदौ दृष्टौ महापुण्यददर्शनौ}% २५

\twolineshloka
{इत्यादिभिः स्तुतिपदैः स्तुत्वागान्मन्दिरं स्वकम्}
{पितरावपि संहृष्टौ पुत्रस्नेहाद्बभूवतुः}% २६

\twolineshloka
{तं दृष्ट्वा रावणो धीमाञ्जगाद निजमातरम्}
{कोऽयं पुमान्सुरो वाथ यक्षो वाथ नरोत्तमः}% २७

\twolineshloka
{योऽसौ मम पितुःपादौ सन्निषेव्य गतः पुनः}
{महाभाग्यनिधिः स्वीयैर्गणैः सुपरिवारितः}% २८

\twolineshloka
{केनेदं तपसा लब्धं विमानं वायुवेगधृक्}
{उद्यानारामलीलादि विलासस्थानमुत्तमम्}% २९

\uvacha{शेष उवाच}

\twolineshloka
{इति वाक्यं समाकर्ण्य जननी रोषविक्लवा}
{उवाच पुत्रं विमनाः किञ्चिन्नेत्रविकारिणी}% ३०

\twolineshloka
{रे पुत्र शृणु मद्वाक्यं बहुशिक्षासमन्वितम्}
{एतस्य जन्मकर्मादि विचारचतुराधिकम्}% ३१

\twolineshloka
{सपत्न्या मम कुक्षिस्थं विधानं समुपस्थितम्}
{येन स्वमातुर्विमलं कुलमुज्ज्वलितं महत्}% ३२

\twolineshloka
{त्वं तु मत्कुक्षिजः कीटः पापः स्वोदरपूरकः}
{यथा खरः स्वकं भारं जानाति न च तद्गुणम्}% ३३

\twolineshloka
{तथा त्वं लक्ष्यसेऽज्ञानी शयनासनभोगवान्}
{सुप्तो गतः क्वचिद्भ्रष्ट इत्येव तव सम्भवः}% ३४

\twolineshloka
{अनेन तपसा लब्धं शिवसन्तोषकारिणा}
{लङ्कावासो मनोवेगं विमानं राज्यसम्पदः}% ३५

\twolineshloka
{सुधन्या जननी त्वस्य सुभाग्या सुमहोदया}
{यस्याः पुत्रो निजगुणैर्लब्धवान्महतां पदम्}% ३६

\twolineshloka
{इति क्रुधा भाषितमार्तया तया मात्रा स्वयाऽकर्ण्य दुरात्मसत्तमः}
{रोषं विधायात्मगतं पुनर्वचो जगाद तां निश्चयभृत्तपः प्रति}% ३७

\uvacha{रावण उवाच}

\twolineshloka
{जनन्याकर्णय वचो मम गर्वसमन्वितम्}
{रत्नगर्भा त्वमेवासि यस्याः पुत्रास्त्रयो वयम्}% ३८

\twolineshloka
{कोऽसौ कीटः स धनदः क्व तपः स्वल्पकं पुनः}
{कालं का किन्तु तद्राज्यं स्वल्पसेवकसंयुतम्}% ३९

\twolineshloka
{मातः शृणु ममोत्साहात्प्रतिज्ञां करुणान्विते}
{न केनापि कृतां कर्त्रा महाभाग्ये हि कैकसि}% ४०

\twolineshloka
{यद्यहं भुवनं सर्वं वशेन स्थापयामि वै}
{तपोभिर्दुष्कृतैः कृत्वा ब्रह्मसन्तोषकारकैः}% ४१

\twolineshloka
{अन्नोदके सदा त्यक्त्वा निद्रां क्रीडां तथा पुनः}
{चेत्तदा पितृलोकस्य घातात्पापं भवेन्मम}% ४२

\twolineshloka
{कुम्भकर्णोऽपि कृतवान्विभीषणसमन्वितः}
{रावणेन सहभ्रात्रेत्युक्त्वागाद्गिरिकाननम्}% ४३

{॥इति श्रीपद्मपुराणे पातालखण्डे शेषवात्स्यायनसंवादे रामाश्वमेधे रावणोत्पत्तिर्नाम षष्ठोऽध्यायः॥६॥}

\dnsub{सप्तमोऽध्यायः}%\resetShloka

\uvacha{अगस्त्य उवाच}

\twolineshloka
{अथोग्रं स तपो दैत्यो दशवर्षसहस्रकम्}
{चकार भानुमक्ष्णा च पश्यन्नूर्ध्वपदे स्थितः}% १

\twolineshloka
{कुम्भकर्णोऽपि कृतवांस्तपः परमदुश्चरम्}
{विभीषणस्तु धर्मात्मा चचार परमं तपः}% २

\twolineshloka
{तदा प्रसन्नो भगवान्देवदेवः प्रजापतिः}
{देवदानवयक्षादि मुकुटैः परिसेवितः}% ३

\twolineshloka
{ददौ राज्यं च सुमहद्भुवनत्रयभास्वरम्}
{वपुश्च कृतवान्रम्यं देवदानवसेवितम्}% ४

\twolineshloka
{तदा सन्तापितो भ्राता धनदो धर्मबुद्धिमान्}
{विमानं तु ततो नीतं लङ्का च नगरी हठात्}% ५

\twolineshloka
{भुवनं तापितं सर्वं देवाश्चैव दिवो गताः}
{हतवान्ब्राह्मणकुलं मुनीनां मूलकृन्तनः}% ६

\twolineshloka
{तदातिदुःखिता देवाः सेन्द्रा ब्रह्माणमाययुः}
{स्तुतिं चक्रुर्महात्मानो दण्डवत्प्रणतिं गताः}% ७

\twolineshloka
{ते तुष्टुवुः सुराः सर्वे वाग्भिरर्थ्याभिरादृताः}
{ततः प्रसन्नो भगवान्किङ्करोमीति चाब्रवीत्}% ८

\twolineshloka
{ततो निवेदयाञ्चक्रुर्ब्रह्मणे विबुधाः पुरः}
{दशग्रीवाच्च सङ्कष्टं तथा निजपराभवम्}% ९

\twolineshloka
{क्षणं ध्यात्वा ययौ ब्रह्मा कैलासं त्रिदशैः सह}
{तस्य शैलस्य पार्श्वे तु वैचित्र्येण समाकुलाः}% १०

\twolineshloka
{स्थिताः सन्तुष्टुवुर्देवाः शम्भुं शक्रपुरोगमाः}
{नमो भवाय शर्वाय नीलग्रीवाय ते नमः}% ११

\twolineshloka
{नमः स्थूलाय सूक्ष्माय बहुरूपाय ते नमः}
{इति सर्वमुखेनोक्तां वाणीमाकर्ण्य शङ्करः}% १२

\twolineshloka
{प्रोवाच नन्दिनं देवा नानयेति ममान्तिकम्}
{एतस्मिन्नन्तरे देवा आहूता नन्दिना च ते}% १३

\twolineshloka
{प्रविश्यान्तःपुरे देवा ददृशुर्विस्मितेक्षणाः}
{ब्रह्मागत्य ददर्शाथ शङ्करं लोकशङ्करम्}% १४

\twolineshloka
{गणकोटिसहस्रैस्तु सेवितं मोदशालिभिः}
{नग्नैर्विरूपैः कुटिलैर्धूसरैर्विकटैस्तथा}% १५

\twolineshloka
{प्रणिपत्याग्रतः स्थित्वा सह देवैः पितामहः}
{उवाच देवदेवेशं पश्यावस्थां दिवौकसाम्}% १६

\twolineshloka
{कृपां कुरु महादेव शरणागतवत्सल}
{दुष्टदैत्यवधार्थं त्वं समुद्योगं विधेहि भोः}% १७

\twolineshloka
{सोऽपि तद्वचनं श्रुत्वा दैन्यशोकसमन्वितम्}
{त्रिदशैः सहितैः सर्वैराजगाम हरेः पदम्}% १८

\twolineshloka
{तुष्टुवुर्मुनयः सर्वे ससुरोरगकिन्नराः}
{जय माधव देवेश जय भक्तजनार्तिहन्}% १९

\twolineshloka
{विलोकय महादेव लोकयस्व स्वसेवकान्}
{इत्युच्चैर्जगदुः सर्वे देवाः शर्वपुरोगमाः}% २०

\twolineshloka
{इत्युक्तमाकर्ण्य सुराधिनाथो दृष्ट्वा सुरार्तिं परिचिन्त्य विष्णुः}
{जगाद देवाञ्जलदोच्चया गिरा दुःखं तु तेषां प्रशमं नयन्निव}% २१

\twolineshloka
{भो ब्रह्मशर्वेन्द्र पुरोगमामराः शृण्वन्तु वाचं भवतां हितेरताम्}
{जाने दशग्रीवकृतं भयं वस्तन्नाशयाम्यद्य कृतावतारः}% २२

\twolineshloka
{पुरी त्वयोध्या रविवंशजातैर्नृपैर्महादानमखादिसत्क्रियैः}
{प्रपालिता भूतलमण्डनीया विराजते राजतभूमिभागैः}% २३

\twolineshloka
{तस्यां दशरथो राजा निरपत्यः श्रियान्वितः}
{पालयत्यधुना राज्यं दिक्चक्रजयवान्विभुः}% २४

\twolineshloka
{स तु वन्द्यादृष्यशृङ्गात्प्रार्थितात्पुत्रकाम्यया}
{पुत्रेष्ट्यां विधिना यज्वा महाबलसमन्वितः}% २५

\twolineshloka
{ततोऽहं प्रार्थितः पूर्वं तपसा तेन भोः सुराः}
{पत्नीषु तिसृषु प्रीत्या चतुर्धापि भवत्कृते}% २६

\twolineshloka
{राम लक्ष्मण शत्रुघ्न भरताख्या समन्वितः}
{कर्तास्मि रावणोद्धारं समूल बलवाहनम्}% २७

\twolineshloka
{भवन्तोऽपि स्वकैरंशैरवतीर्य चरन्त्विह}
{ऋक्षवानररूपेण सर्वत्र पृथिवीतले}% २८

\twolineshloka
{इत्युक्त्वा विररामाशु नभसीरितवाङ्मुने}
{देवाः श्रुत्वा महद्वाक्यं सर्वे संहृष्टमानसाः}% २९

\twolineshloka
{ते चक्रुर्गदितं यादृग्देवदेवेन धीमता}
{स्वैःस्वैरंशैर्मही पूर्णा ऋक्षवानररूपिभिः}% ३०

\twolineshloka
{योऽसौ विष्णुर्महादेवो देवानां दुःखनाशकः}
{सत्वमेव महाराज भगवान्कृतविग्रहः}% ३१

\twolineshloka
{भरतोऽयं लक्ष्मणश्च शत्रुघ्नश्च महामते}
{तावकांशाद्दशग्रीवो जनितश्च सुरार्द्दनः}% ३२

\twolineshloka
{पूर्ववैरानुबन्धेन जानकीं हृतवान्पुनः}
{स त्वया निहतो दैत्यो ब्रह्मराक्षसजातिमान्}% ३३

\twolineshloka
{पुलस्त्यपुत्रो दैत्येन्द्र सर्वलोकैककण्टकः}
{पातितः पृथिवी सर्वा सुखमापमहेश्वर}% ३४

\twolineshloka
{ब्राह्मणानां सुखं त्वद्य मुनीनां तापसं बलम्}
{शिवानि सर्वतीर्थानि सर्वे यज्ञाः सुसंहिताः}% ३५

\twolineshloka
{त्वयि राज्ञि जगत्सर्वं सदेवासुरमानुषम्}
{सुखं प्रपेदे विश्वात्मञ्जगद्योने नरोत्तम}% ३६

\twolineshloka
{एतत्ते सर्वमाख्यातं यत्पृष्टोऽहं त्वयानघ}
{उत्पत्तिश्च विपत्तिश्च मया मत्यनुसारतः}% ३७

\twolineshloka
{इत्थं निशम्य दितिजेन्द्रकुलानुकारिवार्तां महापुरुष ईश्वरईशिता च}
{संरुद्धबाष्पगलदश्रुमुखारविन्दो भूमौ पपात सदसि प्रथितप्रभावः}% ३८

{॥इति श्रीपद्मपुराणे पातालखण्डे शेषवात्स्यायनसंवादे रामाश्वमेधे रावणोत्पत्तिविपत्तिकथनन्नामसप्तमोऽध्यायः॥७॥}

\dnsub{अष्टमोऽध्यायः}%\resetShloka

\uvacha{शेष उवाच}

\twolineshloka
{वात्स्यायनमुनिश्रेष्ठ कथा पापप्रणाशिनी}
{ब्रह्मण्यदेवदेवस्य सर्वधर्मैकरक्षितुः}% १

\twolineshloka
{राजानं मूर्च्छितं दृष्ट्वा कुम्भजन्मा तपोनिधिः}
{शनैःशनैः करेणाशु पस्पर्शाश्रु जगाद च}% २

\twolineshloka
{भो रामाश्वसिहि क्षिप्रं किमर्थमवसीदसि}
{भवान्दैत्यकुलच्छेत्ता महाविष्णुः सनातनः}% ३

\twolineshloka
{भूतं भव्यं भवच्चैव जगत्स्थास्नु चरिष्णु च}
{त्वदृते नास्ति सञ्चारी किमर्थमिह मूर्च्छितः}% ४

\twolineshloka
{श्रुत्वा वाक्यं महाराजः कुम्भजन्मसमीरितम्}
{उत्तस्थौ विगलन्नेत्र बाष्पपूरितसन्मुखः}% ५

\twolineshloka
{उवाच दीनदीनं च विस्पष्टाक्षरविस्तरम्}
{त्रपाभर नमन्मूर्तिर्ब्रह्मद्रोहपराङ्मुखः}% ६

\uvacha{श्रीराम उवाच}

\twolineshloka
{अहो मे पश्यता ज्ञानं विमूढस्य दुरात्मनः}
{यद्ब्राह्मणकुले रूढं हतवान्कामलोलुपः}% ७

\twolineshloka
{महिलार्थे त्वहं विप्रं वेदशास्त्रविवेकवान्}
{हतवान्वाडवकुलं बुद्धिहीनोति दुर्मतिः}% ८

\twolineshloka
{इक्ष्वाकूणां कुले जातु ब्राह्मणो न दुरुक्तिभाक्}
{ईदृशं कुर्वता कर्म मयैतत्सुकलङ्कितम्}% ९

\twolineshloka
{ये ब्राह्मणास्तु पूजार्हा दानसम्मानभोजनैः}
{ते मया निहता विप्राः शरसङ्घातसंहितैः}% १०

\twolineshloka
{काँल्लोकान्नु गमिष्यामि कुम्भीपाकोऽपि दुःसहः}
{न तादृशं तीर्थमस्ति यन्मां पावयितुं क्षमम्}% ११

\twolineshloka
{न यज्ञो न तपो दानं न वा चैव व्रतादिकम्}
{यत्तु वै ब्राह्मणद्रोग्धुर्ममपावनतारकम्}% १२

\twolineshloka
{यैः कोपितं ब्रह्मकुलं नरैर्निरयगामिभिः}
{ते नरा बहुशो दुःखं भोक्ष्यन्ति निरयं गताः}% १३

\twolineshloka
{वेदा मूलं तु धर्माणां वर्णाश्रमविवेकिनाम्}
{तन्मूलं ब्राह्मणकुलं सर्ववेदैकशाखिनः}% १४

\twolineshloka
{मूलच्छेत्तुर्ममौद्धत्यात्को लोकोनु भविष्यति}
{किमद्यकरणीयं वै येन मे हि शिवं भवेत्}% १५

\uvacha{शेष उवाच}

\twolineshloka
{विलपन्तं भृशं रामं राजेन्द्रं रघुपुङ्गवम्}
{मायामनुष्यवपुषं कुम्भजन्माब्रवीद्वचः}% १६

\uvacha{अगस्त्य उवाच}

\twolineshloka
{मा विषादं महाधीर कुरु राजन्महामते}
{न ते ब्राह्मणहत्या स्याद्दुष्टानां नाशमिच्छतः}% १७

\twolineshloka
{त्वं पुराणः पुमान्साक्षादीश्वरः प्रकृतेः परः}
{कर्ता हर्ताऽविता साक्षी निर्गुणः स्वेच्छया गुणी}% १८

\twolineshloka
{सुरापो ब्रह्महत्याकृत्स्वर्णस्तेयी महाघकृत्}
{सर्वे त्वन्नामवादेन पूताः शीघ्रं भवन्ति हि}% १९

\twolineshloka
{इयं देवी जनकजा महाविद्या महामते}
{यस्याः स्मरणमात्रेण मुक्ता यास्यन्ति सद्गतिम्}% २०

\twolineshloka
{रावणोऽपि न वै दैत्यो वैकुण्ठे तव सेवकः}
{ऋषीणां शापतोऽवाप्तो दैत्यत्वं दनुजान्तक}% २१

\twolineshloka
{तस्यानुग्रहकर्ता त्वं न तु हन्ता द्विजन्मनः}
{एवं सञ्चिन्त्य मा भूयो निजं शोचितुमर्हसि}% २२

\twolineshloka
{इति श्रुत्वा ततो वाक्यं रामः परपुरञ्जयः}
{उवाच मधुरं वाक्यं गद्गदस्वरभाषितम्}% २३

\uvacha{श्रीराम उवाच}

\twolineshloka
{पातकं द्विविधं प्रोक्तं ज्ञाताज्ञातविभेदतः}
{ज्ञातं यद्बुद्धिपूर्वं हि अज्ञातं तद्विवर्जितम्}% २४

\twolineshloka
{बुद्धिपूर्वं कृतं कर्म भोगेनैव विनश्यति}
{नश्येदनुशयादन्यदिदं शास्त्रविनिश्चितम्}% २५

\twolineshloka
{कुर्वतो बुद्धिपूर्वं मे ब्रह्महत्यां सुनिन्दिताम्}
{न मे दुःखापनोदाय साधुवादः सुसम्मतः}% २६

\twolineshloka
{प्रब्रूहि तादृशं मह्यं यादृशं पापदाहकम्}
{व्रतं दानं मखं किञ्चित्तीर्थमाराधनं महत्}% २७

\twolineshloka
{येन मे विमला कीर्तिर्लोकान्वै पावयिष्यति}
{पापाचाराप्तकालुष्यान्ब्रह्महत्याहतप्रभान्}% २८

\uvacha{शेष उवाच}

\twolineshloka
{इत्युक्तवन्तं तं रामं जगाद स तपोनिधिः}
{सुरासुरनमन्मौलि मणिनीराजिताङ्घ्रिकम्}% २९

\twolineshloka
{शृणु राम महावीर लोकानुग्रहकारक}
{विप्रहत्यापनोदाय तव यद्वचनं ब्रुवे}% ३०

\twolineshloka
{सर्वं स पापं तरति योऽश्वमेधं यजेत वै}
{तस्मात्त्वं यज विश्वात्मन्वाजिमेधेन शोभिना}% ३१

\twolineshloka
{सप्ततन्तुर्महीभर्त्रा त्वया साध्यो मनीषिणा}
{महासमृद्धियुक्तेन महाबलसुशालिना}% ३२

\twolineshloka
{स वाजिमेधो विप्राणां हत्यायाः पापनोदनः}
{कृतवान्यं महाराजो दिलीपस्तव पूर्वजः}% ३३

\twolineshloka
{शतक्रतुः शतं कृत्वा क्रतूनां पुरुषर्षभः}
{पदमापामरावत्यां देवदैत्यसुसेवितम्}% ३४

\twolineshloka
{मनुश्च सगरो राजा मरुत्तो नहुषात्मजः}
{एते ते पूर्वजाः सर्वे यज्ञं कृत्वा पदं गताः}% ३५

\twolineshloka
{तस्मात्त्वं कुरु राजेन्द्र समर्थोऽसि समन्ततः}
{भ्रातरो लोकपालाभा वर्तन्ते तव भावुकाः}% ३६

\twolineshloka
{इत्युक्तमाकर्ण्य मुनेः स भाग्यवान् रघूत्तमो ब्राह्मणघातभीतः}
{पप्रच्छ यागे सुमतिं चिकीर्षन्विधिं पुरावित्परिगीयमानः}% ३७

{॥इति श्रीपद्मपुराणे पातालखण्डे शेषवात्स्यायनसंवादे रामाश्वमेधे रघुनाथस्यागस्त्योपदेशोनामाष्टमोऽध्यायः॥८॥}

\dnsub{नवमोऽध्यायः}%\resetShloka

\uvacha{श्रीराम उवाच}

\twolineshloka
{कीदृशोऽश्वस्तत्र भाव्यः को विधिस्तत्र पूजने}
{कथं वा शक्यते कर्तुं के जेयास्तत्र वैरिणः}% १

\uvacha{अगस्त्य उवाच}

\twolineshloka
{गङ्गाजलसमानेन वर्णेन वपुषा शुभः}
{कर्णे श्यामो मुखे रक्तः पीतः पुच्छे सुलक्षितः}% २

\twolineshloka
{मनोवेगः सर्वगतिरुच्चैःश्रवस्समप्रभः}
{वाजिमेधे हयः प्रोक्तः शुभलक्षणलक्षितः}% ३

\twolineshloka
{वैशाखपूर्णमास्यां तु पूजयित्वा यथाविधि}
{पत्रं लिखित्वा भाले तु स्वनामबलचिह्नितम्}% ४

\twolineshloka
{मोचनीयः प्रयत्नेन रक्षकैः परिरक्षितः}
{यत्र गच्छति यज्ञाश्वस्तत्र गच्छेत्सुरक्षकः}% ५

\twolineshloka
{यस्तम्बलान्निबध्नाति स्ववीर्यबलदर्पितः}
{तस्मात्प्रसभमानेयः परिरक्षाकरैर्हयः}% ६

\twolineshloka
{कर्त्रा तावत्सुविधिना स्थातव्यं नियमादिह}
{मृगशृङ्गधरो भूत्वा ब्रह्मचर्यसमन्वितः}% ७

\twolineshloka
{व्रतं पालयमानस्य यावद्वर्षमतिक्रमेत्}
{तावद्दीनान्धकृपणाः परितोष्या धनादिभिः}% ८

\twolineshloka
{अन्नं तु बहुशो देयं धनं वा भूरि मारिष}
{यद्यत्प्रार्थयते धीमांस्तत्तदेव ददाति हि}% ९

\twolineshloka
{एवं प्रकुर्वतः कर्म यज्ञः सम्पूर्णतां गतः}
{करोति सर्वपापानां नाशनं रिपुनाशन}% १०

\twolineshloka
{तस्माद्भवान्समर्थोऽस्ति करणे पालनेऽर्चने}
{कृत्वा कीर्तिं सुविमलां पावयान्याञ्जनान्नृप}% ११

\uvacha{श्रीराम उवाच}

\twolineshloka
{विलोकय द्विजश्रेष्ठ वाजिशालां ममाधुना}
{तादृशाः सन्ति नो वाश्वाः शुभलक्षणलक्षिताः}% १२

\twolineshloka
{इति श्रुत्वा तु तद्वाक्यमगस्त्यः करुणाकरः}
{उत्तस्थौ वीक्षमाणोऽयं यागार्हान्वाजिनः शुभान्}% १३

\twolineshloka
{गत्वाथ तत्र शालायां रामचन्द्रसमन्वितः}
{ददर्शाश्वान्विचित्राङ्गान्मनोवेगान्महाबलान्}% १४

\twolineshloka
{अवनितलगताः किं वाजिराजस्य वंश्याः किमथ रघुपतीनामेकतः कीर्तिपिण्डाः}
{किमिदममृतराशिर्वाहरूपेण सिन्धोर्मुनिरिति मनसोन्तर्विस्मयं प्राप पश्यन्}% १५

\twolineshloka
{एकतः शोणदेहानां वाजिनां पङ्क्तिरुत्तमा}
{एकतः श्यामकर्णाश्च कस्तूरीकान्तिसप्रभाः}% १६

\twolineshloka
{एकतः कनकाभाश्च त्वन्यतो नीलवर्णिनः}
{एकतः शबलैर्वर्णैर्विशिष्टैर्वाजिभिर्वृताः}% १७

\twolineshloka
{एवं पश्यन्मुनिः सर्वान्कौतुकाविष्टमानसः}
{ययावन्यत्र तान्द्रष्टुं यागयोग्यान्हयान्मुनिः}% १८

\twolineshloka
{ददर्श तत्र शतशो बद्धांस्तादृशवर्णकान्}
{दृष्ट्वा विस्मयमापेदे स मुनिर्हर्षिताङ्गकः}% १९

\twolineshloka
{एकतः श्यामकर्णांश्च सर्वाङ्गैः क्षीरसन्निभान्}
{पीतपुच्छान्मुखे रक्ताञ्छुभलक्षणलक्षितान्}% २०

\twolineshloka
{निरीक्ष्य परितोऽनघान्विमलनीरधारानिभान्मनोजवनशोभितान्विमलकीर्तिपुञ्जप्रभान्}
{पयोनिधिविशोषको मुनिरुवाचसीतापतिं विचित्रहयदर्शनाद्धृषितनेत्रवक्त्रप्रभः}% २१

\uvacha{अगस्त्य उवाच}

\twolineshloka
{हयमेधक्रतौ योग्यान्वाहांस्ते बहुशः शुभान्}
{पश्यतो नेत्रयोर्मेऽद्य तृप्तिर्नास्ति रघूत्तम}% २२

\twolineshloka
{रामचन्द्र महाभाग सुरासुरनमस्कृत}
{यज्ञं कुरु महाराज हयमेधं सुविस्तरम्}% २३

\twolineshloka
{सुरपतिरिव सर्वान्यज्ञसङ्घान्करिष्यंस्तपन इव सुपर्वारातितोयं विशोष्यन्}
{हतरिपुगणमुख्यं साम्परायं विजित्य क्षितितलसुखभोगं कुर्विदं भूरिभाग}% २४

\twolineshloka
{इत्येवं वाक्यवादेन परितुष्टाखिलेन्द्रियः}
{सर्वान्वै यज्ञसम्भारानाजहार मनोहरान्}% २५

\twolineshloka
{मुन्यन्वितो महाराजः सरयूतीरमागतः}
{सुवर्णलाङ्गलैर्भूमिं विचकर्ष महीयसीम्}% २६

\twolineshloka
{विलिख्य भूमिं बहुशश्चतुर्योजनसम्मिताम्}
{मण्डपान्रचयामास यज्ञार्थं स नरोत्तमः}% २७

\twolineshloka
{कुण्डं तु विधिवत्कृत्वा योनिमेखलयान्वितम्}
{अनेकरत्नरचितं सर्वशोभासमन्वितम्}% २८

\twolineshloka
{मुनीश्वरो महाभागो वसिष्ठः सुमहातपाः}
{सर्वं तत्कारयामास वेदशास्त्रविधिश्रितम्}% २९

\twolineshloka
{प्रेषितास्तेन मुनिना शिष्या मुनिवराश्रमान्}
{कथयामासुरुद्युक्तं हयमेधे रघूत्तमम्}% ३०

\twolineshloka
{आकारितास्तदा सर्वे ऋषयस्तपतां वराः}
{आजग्मुः परमेशस्य दर्शने त्वतिलालसाः}% ३१

\twolineshloka
{नारदोसितनामा च पर्वतः कपिलो मुनिः}
{जातूकर्ण्योऽङ्गिरा व्यास आर्ष्टिषेणोऽत्रिरासुरिः}% ३२

\twolineshloka
{हारीतो याज्ञवल्क्यश्च संवर्तः शुकसंज्ञितः}
{इत्येवमादयो राम हयमेधवरं ययुः}% ३३

\twolineshloka
{तान्सर्वान्पूजयामास रघुराजो महामनाः}
{प्रत्युत्थानाभिवादाभ्यामर्घ्यविष्टरकादिभिः}% ३४

\twolineshloka
{गां हिरण्यं ददौ तेभ्यः प्रायशो दृष्टविक्रमः}
{महद्भाग्यं त्वद्यमेऽस्ति यद्यूयं दर्शनं गताः}% ३५

\uvacha{शेष उवाच}

\twolineshloka
{एवं समाकुले ब्रह्मन्नृषिवर्य समागमे}
{धर्मवार्ता बभूवाहो वर्णाश्रमसुसम्मता}% ३६

\uvacha{वात्स्यायन उवाच}

\twolineshloka
{का धर्मवार्ता तत्रासीत्किं वा कथितमद्भुतम्}
{साधवः सर्वलोकानां कारुण्यात्किमुताब्रुवन्}% ३७

\uvacha{शेष उवाच}

\twolineshloka
{तान्समेतान्मुनीन्दृष्ट्वा रामो दाशरथिर्महान्}
{पप्रच्छ सर्वधर्मांश्च सर्ववर्णाश्रमोचितान्}% ३८

\twolineshloka
{ते तु पृष्टा हि रामेण धर्मान्प्रोचुर्महागुणान्}
{तान्प्रवक्ष्यामि ते सर्वान्यथाविधि शृणुष्व तान्}% ३९

\uvacha{ऋषय ऊचुः}

\twolineshloka
{ब्राह्मणेन सदा कार्यं यजनाध्ययनादिकम्}
{वेदान्पठित्वा विरजो नैव गार्हस्थ्यमाविशेत्}% ४०

\twolineshloka
{ब्राह्मणेन सदा त्याज्यं नीचसेवानुजीवनम्}
{आपद्गतोऽपि जीवेत न श्ववृत्त्या कदाचन}% ४१

\twolineshloka
{ऋतुकालाभिगमनं धर्मोऽयं गृहिणः परः}
{स्त्रीणां वरमनुस्मृत्याऽपत्यकामोथवा भवेत्}% ४२

\twolineshloka
{दिवाभिगमनं पुंसामनायुष्यकरं मतम्}
{श्राद्धाहः सर्वपर्वाणि यतस्त्याज्यानि धीमता}% ४३

\twolineshloka
{तत्र गच्छेत्स्त्रियं मोहाद्धर्मात्प्रच्यवते परात्}
{ऋतुकालाभिगामी यः स्वदारनिरतश्च यः}% ४४

\twolineshloka
{सर्वदा ब्रह्मचारी ह विज्ञेयः स गृहाश्रमी}
{ऋतुः षोडशयामिन्यश्चतस्रस्ता सुगर्हिताः}% ४५

\twolineshloka
{पुत्रदास्तासु या युग्मा अयुग्माः कन्यकाप्रदाः}
{त्यक्त्वा चन्द्रमसं दुष्टं मघां मूलं विहाय च}% ४६

\twolineshloka
{शुचिः सन्निर्विशेत्पत्नीं पुन्नामर्क्षे विशेषतः}
{शुचिं पुत्रं प्रसूयेत पुरुषार्थप्रसाधनम्}% ४७

\twolineshloka
{आर्षे विवाहे गोद्वन्द्वं यदुक्तं तत्प्रशस्यते}
{शुल्कमण्वपि कन्यायाः कन्याक्रेतुस्तु पापकृत्}% ४८

\twolineshloka
{वाणिज्यं नृपतेः सेवा वेदानध्ययनं तथा}
{कुविवाहः क्रियालोपः कुलपातनहेतवः}% ४९

\twolineshloka
{अन्नोदक पयो मूलफलैर्वापि गृहाश्रमी}
{गोदानेन तु यत्पुण्यं पात्राय विधिपूर्वकम्}% ५०

\twolineshloka
{अनर्चितोऽतिथिर्गेहाद्भग्नाशो यस्य गच्छति}
{आजन्मसञ्चितात्पुण्यात्क्षणात्स हि बहिर्भवेत्}% ५१

\twolineshloka
{पितृदेवमनुष्येभ्यो दत्त्वाश्नीतामृतं गृही}
{स्वार्थं पचत्यघं भुङ्क्ते केवलं स्वोदरम्भरिः}% ५२

\twolineshloka
{षष्ठ्यष्टम्योर्विशेत्पापं तैले मांसे सदैव हि}
{चतुर्दश्यां तथामायां त्यजेत क्षुरमङ्गनाम्}% ५३

\twolineshloka
{रजस्वलां न सेवेत नाश्नीयात्सह भार्यया}
{एकवासा न भुञ्जीत न भुञ्जीतोत्कटासने}% ५४

\twolineshloka
{नाश्नन्तीं स्त्रियमीक्षेत तेजःकामो नरोत्तमः}
{मुखेनोपधमेन्नाग्निं नग्नां नेक्षेत योषितम्}% ५५

\twolineshloka
{नाङ्घ्री प्रतापयेदग्नौ न वस्त्वशुचि निक्षिपेत्}
{प्राणिहिंसां न कुर्वीत नाश्नीयात्सन्ध्ययोर्द्वयोः}% ५६

\twolineshloka
{नाचक्षीत धयन्तीं गां नेन्द्रचापं प्रदर्शयेत्}
{न दिवोद्गतसारं च भक्षयेद्दधिनो निशि}% ५७

\twolineshloka
{स्त्रीं धर्मिणीं नाभिवादेन्नाद्यादातृप्ति रात्रिषु}
{तौर्यत्रिकप्रियो न स्यात्कांस्ये पादौ न धावयेत्}% ५८

\twolineshloka
{न धारयेदन्यभुक्तं वासश्चोपानहावपि}
{न भिन्नभाजनेऽश्नीयान्नाश्नीतान्नं विदूषितम्}% ५९

\twolineshloka
{संविशेन्नार्द्रचरणो नोच्छिष्टः क्वचिदाव्रजेत्}
{शयानो वा न चाश्नीयान्नोच्छिष्टः संस्पृशेच्छिरः}% ६०

\twolineshloka
{न मनुष्यस्तुतिं कुर्यान्नात्मानमवमानयेत्}
{अभ्युद्यतं न प्रणमेत्परमर्माणि नो वदेत्}% ६१

\twolineshloka
{एवं गार्हस्थ्यमाश्रित्य वानप्रस्थाश्रमं व्रजेत्}
{सस्त्रीको वा गतस्त्रीको विरज्येत ततः परम्}% ६२

\twolineshloka
{इत्येवमादयो धर्मा गदिता ऋषिभिस्तदा}
{श्रुता रामेण महता सर्वलोकहितैषिणा}% ६३

{॥इति श्रीपद्मपुराणे पातालखण्डे शेषवात्स्यायनसंवादे रामाश्वमेधे सर्वधर्मोपदेशो नाम नवमोऽध्यायः॥९॥}

\dnsub{दशमोऽध्यायः}%\resetShloka

\uvacha{शेष उवाच}

\twolineshloka
{इत्थं संशृण्वतो धर्मान्वसन्तः समुपस्थितः}
{यत्र यज्ञ क्रियादीनां प्रारम्भः सुमहात्मनाम्}% १

\twolineshloka
{दृष्ट्वा तं समयं धीमान्वसिष्ठः कलशोद्भवः}
{रामचन्द्रं महाराजं प्रत्युवाच यथोचितम्}% २

\uvacha{वसिष्ठ उवाच}

\twolineshloka
{रामचन्द्र महाबाहो समयः पर्यभूत्तव}
{हयो यत्र प्रमुच्येत यज्ञार्थं परिपूजितः}% ३

\twolineshloka
{सामग्री क्रियतां तत्र आहूयन्तां द्विजोत्तमाः}
{करोतु पूजां भगवान्ब्राह्मणानां यथोचिताम्}% ४

\twolineshloka
{दीनान्धकृपणानां च दानं स्वान्ते समुत्थितम्}
{ददातु विधिवत्तेषां प्रतिपूज्याधिमान्य च}% ५

\twolineshloka
{भवान्कनकसत्पत्न्या दीक्षितोऽत्र व्रतं चर}
{भूमिशायी ब्रह्मचारी वसुभोगविवर्जितः}% ६

\twolineshloka
{मृगशृङ्गधरः कट्यां मेखलाजिनदण्डभृत्}
{करोतु यज्ञसम्भारं सर्वद्रव्यसमन्वितम्}% ७

\twolineshloka
{इति श्रुत्वा महद्वाक्यं वसिष्ठस्य यथार्थकम्}
{उवाच लक्ष्मणं धीमान्नानार्थपरिबृंहितम्}% ८

\uvacha{श्रीराम उवाच}

\twolineshloka
{शृणु लक्ष्मण मद्वाक्यं श्रुत्वा तत्कुरु सत्वरम्}
{हयमानय यत्नेन वाजिमेधक्रियोचितम्}% ९

\uvacha{शेष उवाच}

\twolineshloka
{श्रुत्वा वाक्यं रघुपतेः शत्रुजिल्लक्ष्मणस्तदा}
{सेनापतिमुवाचेदं वचो विविधवर्णनम्}% १०

\uvacha{लक्ष्मण उवाच}

\fourlineindentedshloka
{वीराकर्णय मे वचः सुमधुरं श्रुत्वा त्वरातः पुनः}
{कार्यं तत्क्षितिपालमौलिमुकुटैर्घृष्टाङ्घ्रि रामाज्ञया}
{सेनां कालबलप्रभञ्जनबलप्रोद्यत्समर्थाङ्गिनीं}
{सज्जां सद्रथहस्तिपत्तिसुहयारोहैर्विधे ह्यन्विताम्}% ११

\twolineshloka
{सज्जीयतां वायुजवास्तुरङ्गास्तरङ्गमाला ललिताङ्घ्रिपाताः}
{सदश्वचारैर्बहुशस्त्रधारिभिः संरोहिता वैरिबलप्रहारिभिः}% १२

\twolineshloka
{संलक्ष्यतां हस्तिनः पर्वताभा आधोरणैः प्रासकुन्ताग्रहस्तैः}
{शूरैः सास्त्रैर्भूरिदानोपहाराः क्षीबाणस्ते सर्वशस्त्रास्त्रपूर्णाः}% १३

\twolineshloka
{विततबहुसमृद्धिभ्राजमाना रथा मे पवनजवनवेगैर्वाजिभिर्युज्यमानाः}
{विविधरिपुविनाशस्मारकैरायुधास्त्रैर्भृतवलभिविभागानीयतां सूतवृन्दैः}% १४

\twolineshloka
{पत्तयः शतशो मह्यमायान्त्वस्त्राग्न्यपाणयः}
{हयमेधार्हवाहस्य रक्षणे विततोद्यमाः}% १५

\twolineshloka
{इत्याकर्ण्य वचस्तस्य लक्ष्मणस्य महात्मनः}
{सेनानी कालजिन्नामा कारयामास सज्जताम्}% १६

\twolineshloka
{दशध्रुवकमण्डितो लघुसुरोमशोभान्वितो विविक्तगलशुक्तिभृद्विततकण्ठको शेमणिः मुखे}
{विशदकान्तिधृत्त्वसितकान्तिभृत्कर्णयोर्व्यराजत तदाह यो धृतकराग्ररश्मिच्छटः}% १७

\twolineshloka
{कलासंशोभितमुखः स्फुरद्रत्नविशोभितः}
{मुक्ताफलानां मालाभिः शोभितो निर्ययौ हयः}% १८

\twolineshloka
{श्वेतातपत्ररचितः सितचामरशोभितः}
{बहुशोभापरीताङ्गो निर्ययौ हयराट्ततः}% १९

\twolineshloka
{अग्रतो मध्यतश्चैके पृष्ठतः सैनिकास्तथा}
{देवा हरिं यथापूर्वं सेवन्ते सेवनोचितम्}% २०

\twolineshloka
{अथ सैन्यं समाहूय सर्वमाज्ञापयत्तदा}
{हस्त्यश्वरथपादातवृन्दैः सुबहुसङ्कुलम्}% २१

\twolineshloka
{ततस्ततः समेतानां सैन्यानां श्रूयते ध्वनिः}
{ततो दुन्दुभिनादोऽभूत्तस्मिन्पुरवरे तदा}% २२

\twolineshloka
{तन्निनादेन शूराणां प्रियेण महता तदा}
{कम्पन्ति गिरिशृङ्गाणि प्रासादा विचलन्ति च}% २३

\twolineshloka
{हेषारवो महानासीद्वाजिनां मुह्यतां नृप}
{रथाङ्गघातसङ्घुष्टा धरा सञ्चलतीव सा}% २४

\twolineshloka
{चलितैर्गजयूथैश्च पृथ्वी रुद्धा समन्ततः}
{रजस्तु प्रचलत्तत्र जनान्तर्द्धानमादधात्}% २५

\twolineshloka
{निर्जगाम महासैन्यं छत्रैः सञ्छाद्य भास्करम्}
{सेनान्याकालजिन्नाम्ना प्रेरितं जनसङ्कुलम्}% २६

\twolineshloka
{गर्जन्तस्तलवीराग्र्याः कुर्वन्तो रणसम्भ्रमम्}
{रघुनाथस्य यागाय सज्जास्ते प्रययुर्मुदा}% २७

\twolineshloka
{मृगमदमयमङ्गेष्वङ्गरागं दधानाः कुसुमविमलमालाशोभितस्वोत्तमाङ्गाः}
{मुकुटकटकभूषाभूषिताङ्गाः समस्ताः प्रययुरवनिनाथप्रेरितास्तेऽपि सर्वे}% २८

\twolineshloka
{इत्येवं ते महाराजं ययुः सेनाचरा वराः}
{धनुर्धराः पाशधराः खड्गधाराः स्फुटक्रमाः}% २९

\twolineshloka
{एवं शनैःशनैः प्राप्तो मण्डपं यागचिह्नितम्}
{हयः खुरक्षततलां भूमिं कुर्वन्नभः प्लवन्}% ३०

\twolineshloka
{रामो दृष्ट्वा हरिं प्राप्तं बहुसन्तुष्टमानसः}
{वसिष्ठं प्रेरयामास क्रियाकर्तव्यतां प्रति}% ३१

\twolineshloka
{वसिष्ठो राममाहूय स्वर्णपत्नीसमन्वितम्}
{प्रयोगं कारयामास ब्रह्महत्यापनोदनम्}% ३२

\twolineshloka
{ब्रह्मचर्यव्रतधरो मृगशृङ्गपरिग्रहः}
{तत्कर्म कारयामास रामः परपुरञ्जयः}% ३३

\twolineshloka
{प्रारेभे यागकर्मार्थं कुण्डं मण्डपसम्मितम्}
{तत्राचार्योभवद्धीमान्वेदशास्त्रविचारवित्}% ३४

\twolineshloka
{वसिष्ठो रघुनाथस्य कुलपूर्वगुरुर्मुनिः}
{ब्रह्मंस्तत्राचरद्ब्रह्मकर्मागस्त्यस्तपोनिधिः}% ३५

\twolineshloka
{वाल्मीकिर्मुनिरध्वर्युर्मुनिः कण्वस्तु द्वारपः}
{अष्टौ द्वाराणि तत्रासन्सतोरण शुभानि वै}% ३६

\twolineshloka
{द्वारि द्वारि द्वयं विप्र ब्राह्मणस्याधिमन्त्रवित्}
{पूर्वद्वारि मुनिश्रेष्ठौ देवलासित संज्ञितौ}% ३७

\twolineshloka
{दक्षिणद्वारि भूमानौ कश्यपात्री तपोनिधी}
{पश्चिमद्वारि ऋषभौ जातूकर्ण्योऽथ जाजलिः}% ३८

\twolineshloka
{उत्तरद्वारि तु मुनी द्वौ द्वितैकत तापसौ}
{एवं द्वारविधिं कृत्वा वसिष्ठः कलशोद्भवः}% ३९

\twolineshloka
{हयवर्यस्य सत्पूजां कर्तुमारभत द्विज}
{सुवासिन्यः स्त्रियस्तत्र वासोलङ्कारभूषिताः}% ४०

\twolineshloka
{हरिद्राक्षतगन्धाद्यैः पूजयामासुरर्चितम्}
{नीराजनं ततः कृत्वा धूपयित्वागुरूक्षणैः}% ४१

\twolineshloka
{वर्धापनं ततो वेश्याश्चक्रुस्ता वाडवाज्ञया}
{एवं सम्पूज्य विमले भाले चन्दनचर्चिते}% ४२

\twolineshloka
{कुङ्कुमादिकगन्धाढ्ये सर्वशोभासमन्विते}
{बबन्ध भास्वरं पत्रं तप्तहाटकनिर्मितम्}% ४३

\twolineshloka
{तत्रालिखद्दाशरथेः प्रतापबलमूर्जितम्}
{सूर्यवंशध्वजो धन्वी धनुर्दीक्षा गुरुर्गुरुः}% ४४

\twolineshloka
{यं देवाः सासुराः सर्वे नमन्ति मणिमौलिभिः}
{तस्यात्मजो वीरबलदर्पहारी रघूद्वहः}% ४५

\twolineshloka
{रामचन्द्रो महाभागः सर्वशूरशिरोमणिः}
{तन्माता कोसलनृपपत्नीगर्भसमुद्भवा}% ४६

\twolineshloka
{तस्याः कुक्षिभवं रत्नं रामः शत्रुक्षयङ्करः}
{करोति हयमेधं वै ब्राह्मणेन सुशिक्षितः}% ४७

\twolineshloka
{रावणाभिधविप्रेन्द्र वधपापापनुत्तये}
{मोचितस्तेन वाहानां मुख्योऽसौ वाजिनां वरः}% ४८

\twolineshloka
{महाबलपरीवार परिखाभिः सुरक्षितः}
{तद्रक्षकोऽस्ति तद्भ्राता शत्रुघ्नो लवणान्तकः}% ४९

\twolineshloka
{हस्त्यश्वरथपादात सेनासङ्घसमन्वितः}
{यस्य राज्ञ इति श्रेष्ठो मानः स्यात्स्वबलोन्मदात्}% ५०

\twolineshloka
{वयं धनुर्धराः शूराः श्रेष्ठा वयमिहोत्कटाः}
{ते गृह्णन्तु बलाद्वाहं रत्नमालाविभूषितम्}% ५१

\twolineshloka
{मनोवेगं कामजवं सर्वगत्यधिभास्वरम्}
{ततो मोचयिता भ्राता शत्रुघ्नो लीलया हयम्}% ५२

\onelineshloka
{शरासनविनिर्मुक्त वत्सदन्तैः शिखाशितैः}% ५३

\fourlineindentedshloka
{इत्येवमादि विलिलेख महामुनीन्द्रः}
{श्रीरामचन्द्र भुजवीर्यलसत्प्रतापम्}
{शोभानिधानमतिचञ्चलवायुवेगं}
{पातालभूतलविशेषगतिं मुमोच}% ५४

\twolineshloka
{शत्रुघ्नमादिदेशाथ रामः शस्त्रभृतां वरः}
{याहि वाहस्य रक्षार्थं पृष्ठतः स्वैरगामिनः}% ५५

\twolineshloka
{शत्रुघ्न गच्छ वाहस्य मार्गं भद्रं भवेत्तव}
{भवेतां शत्रुविजयौ रिपुकर्षण ते भुजौ}% ५६

\fourlineindentedshloka
{ये योद्धारः प्रतिरणगतास्ते त्वया वारणीया-}
{वाहं रक्ष स्वकगुणगणैः संयुतः सन्महोर्व्याम्}
{सुप्तान्भ्रष्टान्विगतवसनान्भीतभीतांस्तु नम्रां-}
{स्तान्मा हन्याः सुकृतकृतिनो येन शंसन्ति कर्म}% ५७

\twolineshloka
{विरथा भयसन्त्रस्ता ये वदन्ति वयं तव}
{ते त्वया न हि हन्तव्याः शत्रुघ्न सुकृतैषिणा}% ५८

\twolineshloka
{यो हन्याद्विमदं मत्तं सुप्तं मग्नं भयातुरम्}
{तावकोऽहं ब्रुवाणं च स व्रजत्यधमां गतिम्}% ५९

\twolineshloka
{परस्वे चित्तवृत्तिं त्वं मा कृथाः पारदारिके}
{नीचे रतिं न कुर्वीथाः सर्वसद्गुणपूरितः}% ६०

\twolineshloka
{वृद्धानां प्रेरणं पूर्वं मा कुर्वीथा रणं जय}
{पूज्यपूजातिक्रमं त्वं मा विधेहि दयान्वितः}% ६१

\twolineshloka
{गां विप्रं च नमस्कुर्या वैष्णवं धर्मसंयुतम्}
{अभिवाद्य यतो गच्छेस्तत्र सिद्धिमवाप्नुयाः}% ६२

\twolineshloka
{विष्णुः सर्वेश्वरः साक्षी सर्वव्यापकदेहभृत्}
{ये तदीया महाबाहो तद्रूपा विचरन्ति हि}% ६३

\twolineshloka
{ये स्मरन्ति महाविष्णुं सर्वभूतहृदि स्थितम्}
{ते मन्तव्या महाविष्णु समरूपा रघूद्वह}% ६४

\twolineshloka
{यस्य स्वीयो न पारक्यो यस्य मित्रसमो रिपुः}
{ते वैष्णवाः क्षणादेव पापिनं पावयन्ति हि}% ६५

\twolineshloka
{येषां प्रियं भागवतं येषां वै ब्राह्मणाः प्रियाः}
{वैकुण्ठात्प्रेषितास्तेऽत्र लोकपावनहेतवे}% ६६

\twolineshloka
{येषां वक्त्रे हरेर्नाम हृदि विष्णुः सनातनः}
{उदरे विष्णुनैवेद्यः स श्वपाकोऽपि वैष्णवः}% ६७

\twolineshloka
{येषां वेदाः प्रियतमा न च संसारजं सुखम्}
{स्वधर्मनिरता ये च तान्नमस्कुर्विहान्वितान्}% ६८

\twolineshloka
{शिवे विष्णौ न वा भेदो न च ब्रह्ममहेशयोः}
{तेषां पादरजः पूतं वहाम्यघविनाशनम्}% ६९

\twolineshloka
{गौरी गङ्गा महालक्ष्मीर्यस्य नास्ति पृथक्तया}
{ते मन्तव्या नराः सर्वे स्वर्गलोकादिहागताः}% ७०

\twolineshloka
{शरणागतरक्षी च मानदानपरायणः}
{यथाशक्ति हरेः प्रीत्यै स ज्ञेयो वैष्णवोत्तमः}% ७१

\twolineshloka
{यस्य नाम महापापराशिं दहति सत्वरम्}
{तदीय चरणद्वन्द्वे भक्तिर्यस्य स वैष्णवः}% ७२

\twolineshloka
{इन्द्रियाणि वशे येषां मनोऽपि हरिचिन्तकम्}
{तान्नमस्कृत्य पूयात्सह्या जन्ममरणान्तिकात्}% ७३

\twolineshloka
{परस्त्रियं त्वं करवालवत्त्यजन्भवेर्यशोभूषणभूतिभूमिः}
{एवं ममादेशमथाचरंश्च लभेः परं धाम सुयोगमीड्यम्}% ७४

{॥इति श्रीपद्मपुराणे पातालखण्डे शेषवात्स्यायनसंवादे रामाश्वमेधे शत्रुघ्नशिक्षाकथनं नाम दशमोऽध्यायः॥१०॥}

\dnsub{एकादशोऽध्यायः}%\resetShloka

\uvacha{शेष उवाच}

\twolineshloka
{एवमाज्ञाप्य भगवान्रामश्चामित्रकर्षणः}
{वीरानालोकयन्भूयो जगाद शुभया गिरा}% १

\twolineshloka
{शत्रुघ्नस्य मम भ्रातुर्वाजिरक्षाकरस्य वै}
{को गन्ता पृष्ठतो रक्षंस्तन्निदेशप्रपालकः}% २

\twolineshloka
{यः सर्ववीरान्प्रतिमुख्यमागतान्विनिर्जयेन्मर्मभिदस्त्रसङ्घैः}
{गृह्णात्वसौ मे करवीटकं तद्भूमौ यशः स्वं प्रथयन्सुविस्तरम्}% ३

\twolineshloka
{इत्युक्तवति रामे तु पुष्कलो भरतात्मजः}
{जग्राह वीटकं तस्माद्रघुराजकराम्बुजात्}% ४

\twolineshloka
{स्वामिन्गच्छामि शत्रुघ्न पृष्ठरक्षाकरोऽन्वहम्}
{सन्नद्धः सर्वशस्त्रास्त्र चापबाणधरः प्रभो}% ५

\twolineshloka
{सर्वमद्य क्षितितलं त्वत्प्रतापो विजेष्यते}
{एते निमित्तभूता वै रामचन्द्र महामते}% ६

\twolineshloka
{भवत्कृपातः सकलं ससुरासुरमानुषम्}
{उपस्थितं प्रयुद्धाय तन्निषेधे क्षमो ह्यहम्}% ७

\twolineshloka
{सर्वं स्वामी ज्ञास्यति यन्ममविक्रम दर्शनात्}
{एष गन्तास्मि शत्रुघ्न पृष्ठरक्षाप्रकारकः}% ८

\twolineshloka
{एवं ब्रुवन्तं भरतात्मजं स प्रस्तूय साध्वित्यनुमोदमानः}
{शशंस सर्वान्कपिवीरमुख्यान्प्रभञ्जनोद्भूतमुखान्हरिः प्रभुः}% ९

\twolineshloka
{भो हनूमन्महावीर शृणु मद्वाक्यमादृतः}
{त्वत्प्रसादान्मया प्राप्तमिदं राज्यमकण्टकम्}% १०

\twolineshloka
{सीतया मम संयोगे यो भवाञ्जलधिं तरेः}
{चरितं तद्धरे वेद्मि सर्वं तव कपीश्वर}% ११

\twolineshloka
{त्वं गच्छ मम सैन्यस्य पालकः सन्ममाज्ञया}
{शत्रुघ्नः सोदरो मह्यं पालनीयस्त्वहं यथा}% १२

\twolineshloka
{यत्र यत्र मतिभ्रंशः शत्रुघ्नस्य प्रजायते}
{तत्र तत्र प्रबोद्धव्यो भ्राता मम महामते}% १३

\twolineshloka
{इति श्रुत्वा महद्वाक्यं रामचन्द्रस्य धीमतः}
{शिरसा तत्समाधाय प्रणाममकरोत्तदा}% १४

\twolineshloka
{अथादिशन्महाराजो जाम्बवन्तं कपीश्वरम्}
{रघुनाथस्य सेवायै कपिषूत्तमतेजसम्}% १५

\twolineshloka
{अङ्गदो गवयो मैन्दस्तथा दधिमुखः कपिः}
{सुग्रीवः प्लवगाधीशः शतवल्यक्षिकौ कपी}% १६

\twolineshloka
{नीलो नलो मनोवेगोऽधिगन्ता वानराङ्गजः}
{इत्येवमादयो यूयं सज्जीभूता भवन्तु भोः}% १७

\twolineshloka
{सर्वैर्गजैः सदश्वैश्च तप्तहाटकभूषणैः}
{कवचैः सशिरस्त्राणैर्भूषितायां तु सत्वराः}% १८

\uvacha{शेष उवाच}

\twolineshloka
{सुमन्त्रमाहूय सुमन्त्रिणं तदा जगाद रामो बलवीर्यशोभनः}
{अमात्यमौले वद केऽत्र योज्या नरा हयं पालयितुं समर्थाः}% १९

\twolineshloka
{तदुक्तमेवमाकर्ण्य जगाद परवीरहा}
{हयस्य रक्षणे योग्यान्बलिनोऽत्र नराधिपान्}% २०

\twolineshloka
{रघुनाथ शृणुष्वैतान्नववीरान्सुसंहितान्}
{धनुर्धरान्महाविद्यान्सर्वशस्त्रास्त्रकोविदान्}% २१

\twolineshloka
{प्रतापाग्र्यं नीलरत्नं तथा लक्ष्मीनिधिं नृपम्}
{रिपुतापं चोग्रहयं तथा शस्त्रविदं नृपम्}% २२

\twolineshloka
{राजन्योऽसौ नीलरत्नो महावीरो रथाग्रणीः}
{स एव लक्षं रक्षेत लक्षं युध्येत निर्भयः}% २३

\twolineshloka
{अक्षौहिणीभिर्दशभिर्यातु वाहस्य रक्षणे}
{दंशितैस्स शिरस्त्राणैर्मम बाहुभिरुद्धतैः}% २४

\twolineshloka
{प्रतापाग्र्यो यो ह्ययं च रिपुगर्वमशातयत्}
{सव्यापसव्यबाणानां मोक्ता सर्वास्त्रवित्तमः}% २५

\twolineshloka
{एषोऽक्षौहिणिविंशत्या यातु यज्ञहयावने}
{सन्नद्धो रिपुनाशाय युवाको दण्डदण्डभृत्}% २६

\twolineshloka
{तथा लक्ष्मीनिधिस्त्वेष यातु राजन्यसत्तमः}
{यस्तपोभिः शतधृतिं प्रसाद्यास्त्राणि चाभ्यसत्}% २७

\twolineshloka
{ब्रह्मास्त्रं पाशुपत्यास्त्रं गारुडं नागसंज्ञितम्}
{मायूरं नाकुलं रौद्रं वैष्णवं मेघसंज्ञितम्}% २८

\twolineshloka
{वज्रं पार्वतसंज्ञं च तथा वायव्यसंज्ञितम्}
{इत्यादिकानामस्त्राणां सम्प्रयोगविसर्गवित्}% २९

\twolineshloka
{स एष निजसैन्यानामक्षौहिण्यैकया युतः}
{प्रयातु शूरमुकुटः सर्ववैरिप्रभञ्जनः}% ३०

\twolineshloka
{रिपुतापोऽयमेवाद्य गच्छत्वग्र्यो धनुर्भृताम्}
{सर्वशस्त्रास्त्रकुशलो रिपुवंशदवानलः}% ३१

\twolineshloka
{गच्छतात्सेनया बह्व्या चतुरङ्गसमेतया}
{शत्रुघ्नाज्ञां शिरस्येते दधत्वद्य बलोत्कटाः}% ३२

\twolineshloka
{उग्राश्वोऽपि महाराजा तथा शस्त्रविदेष च}
{सर्वे यान्तु सुसन्नद्धास्तव वाहस्य पालकाः}% ३३

\twolineshloka
{इति भाषितमाकर्ण्य मन्त्रिणः प्रजहर्ष च}
{आज्ञापयामास च तान्सुमन्त्रकथितान्भटान्}% ३४

\twolineshloka
{तेऽनुज्ञां रघुनाथस्य प्राप्य मोदं प्रपेदिरे}
{चिरकालं साम्परायं वाञ्च्छन्तो युद्धदुर्मदाः}% ३५

\twolineshloka
{सन्नद्धाः कवचाद्यैश्च तथा शस्त्रास्त्रवर्तनैः}
{ययुः शत्रुघ्नसंवासं सीतापति प्रणोदिताः}% ३६

\uvacha{शेष उवाच}

\twolineshloka
{अथोक्त ऋषिणा रामो विधिना पूजयत्क्रमात्}
{आचार्यादीनृषीन्सर्वान्यथोक्तवरदक्षिणैः}% ३७

\twolineshloka
{आचार्याय ददौ रामो हस्तिनं षष्टिहायनम्}
{हयमेकं मनोवेगं रत्नमालाविभूषितम्}% ३८

\twolineshloka
{पौरटं रथमेकं च मणिरत्नविभूषितम्}
{चतुर्भिर्वाजिभिर्युक्तं सर्वोपस्करसंयुतम्}% ३९

\twolineshloka
{मणिलक्षं तु प्रत्यक्षं मुक्ताफलतुलाशतम्}
{विद्रुमस्य तुलानां तु सहस्रं स्फुटतेजसाम्}% ४०

\twolineshloka
{ग्राममेकं सुसम्पन्नं नानाजनसमाकुलम्}
{विचित्रसस्यनिष्पन्नं विविधैर्मन्दिरैर्वृतम्}% ४१

\twolineshloka
{ब्रह्मणेऽपि तथैवादाद्धोत्रेऽप्यध्वर्यवे ददौ}
{ऋत्विग्भ्यो भूरिशो दत्त्वा प्रणनाम रघूत्तमः}% ४२

\twolineshloka
{सर्वे ते विविधा वाग्भिराशीर्भिरभिपूजिताः}
{चिरञ्जीव महाराज रामचन्द्र रघूद्वह}% ४३

\twolineshloka
{कन्यादानं भूमिदानं गजदानं तथैव च}
{अश्वदानं स्वर्णदानं तिलदानं समौक्तिकम्}% ४४

\twolineshloka
{अन्नदानं पयोदानमभयं दानमुत्तमम्}
{रत्नदानानि सर्वाणि विप्रेभ्यश्चादिशन्महान्}% ४५

\twolineshloka
{देहि देहि धनं देहि मानेति ब्रूहि कस्यचित्}
{ददात्वन्नं ददात्वन्नं सर्वभोगसमन्वितम्}% ४६

\twolineshloka
{इत्थं प्रावर्तत मखो रघुनाथस्य धीमतः}
{सदक्षिणो द्विजवरैः पूर्णः सर्वशुभक्रियः}% ४७

\twolineshloka
{अथ रामानुजो गत्वा मातरं प्रणनाम ह}
{आज्ञापयाश्वरक्षार्थमेष गच्छामि शोभने}% ४८

\twolineshloka
{त्वत्कृपातो रिपुकुलं जित्वा शोभासमन्वितः}
{आयास्यामि महाराजैर्हयवर्यसमन्वितः}% ४९

\uvacha{मातोवाच}

\twolineshloka
{पुत्र गच्छ महावीर शिवाः पन्थान एव ते}
{सर्वान्रिपुगणाञ्जित्वा पुनरागच्छ सन्मते}% ५०

\twolineshloka
{पुष्कलं पालय निजभ्रातृजं धर्मवित्तमम्}
{महाबलिनमद्यापि बालकं लीलयायुतम्}% ५१

\twolineshloka
{पुत्रागच्छसि चेद्युक्तः पुष्कलेन शुभान्वितः}
{तदा मम प्रमोदः स्यादन्यथा शोकभागहम्}% ५२

\twolineshloka
{इति सम्भाष्यमाणां स्वां मातरं प्रत्युवाच सः}
{त्वदीयचरणद्वन्द्वं स्मरन्प्राप्स्यामि शोभनम्}% ५३

\twolineshloka
{पुष्कलं पालयित्वाहं निजाङ्गमिव शोभने}
{स्वनामसदृशं कृत्वा पुनरेष्यामि मोदवान्}% ५४

\twolineshloka
{इत्युक्त्वा प्रययौ वीरो रामं स मखमण्डपे}
{आसीनं मुनिवर्याग्र्यैर्यज्ञवेषधरं वरम्}% ५५

\twolineshloka
{उवाच मतिमान्वीरः सर्वशोभासमन्वितः}
{रामाज्ञापय रक्षार्थं हयस्यानुज्ञया तव}% ५६

\twolineshloka
{रघुनाथोऽपि तच्छ्रुत्वा भद्रमस्त्विति चाब्रवीत्}
{बालं स्त्रियं प्रमत्तं त्वं मा हन्याः शस्त्रवर्जितम्}% ५७

\twolineshloka
{तदा लक्ष्मीनिधिर्भ्राता जानक्या जनकात्मजः}
{प्रहस्य किञ्चिन्नयने नर्तयन्राममब्रवीत्}% ५८

\uvacha{लक्ष्मीनिधिरुवाच}

\twolineshloka
{रामचन्द्र महाबाहो सर्वधर्मपरायण}
{शत्रुघ्नं शिक्षय तथा यथा लोकोत्तरो भवेत्}% ५९

\twolineshloka
{कुलोचितं कर्म कुर्वन्नग्रजाचरितं तथा}
{गच्छेत्स परमं धाम तेजोबलसमन्वितम्}% ६०

\twolineshloka
{त्वया प्रोक्तं महाराज ब्राह्मणं नावमानयेत्}
{पित्रा तव हतो विप्रः पितृभक्तिपरायणः}% ६१

\twolineshloka
{त्वयापि सुमहत्कर्म कृतं लोकविगर्हितम्}
{अवध्यां महिलां यस्त्वं हतवान्नियतं ततः}% ६२

\twolineshloka
{अग्रजोऽस्य महाराज कृतवान्यं पराक्रमम्}
{सनकेन कृतः पूर्वं राक्षस्याः कर्णकर्तनम्}% ६३

\twolineshloka
{एवं करिष्यति नृपः शत्रुघ्नः शिक्षया तव}
{यदि नायं तथा कुर्यात्कुलस्यासदृशं भवेत्}% ६४

\twolineshloka
{इत्युक्तवन्तं तं रामः प्रत्युवाच हसन्निव}
{मेघगम्भीरया वाचा सर्ववाक्यविशारदः}% ६५

\twolineshloka
{शृण्वन्तु योगिनः शान्ताः समदुःखसुखाः पुनः}
{जानन्त्यपारसंसारनिस्तारतरणादिकम्}% ६६

\twolineshloka
{ये शूराः समहेष्वासाः सर्वशस्त्रास्त्रकोविदाः}
{ते च जानन्ति युद्धस्य वार्त्तां न तु भवादृशाः}% ६७

\twolineshloka
{परोपतापिनो ये वै ये चोत्पथविसारिणः}
{ते हन्तव्या नृपैः सर्वैः सर्वलोकहितैषिभिः}% ६८

\twolineshloka
{इत्युक्तमाकर्ण्य सभासदस्ते सर्वे स्मितं चक्रुररिन्दमस्य}
{कुम्भोद्भवः पूजितमेनमश्वं विमोचयामास सुशोभितं हि}% ६९

\twolineshloka
{इमं मन्त्रं समुच्चार्य वसिष्ठः कलशोद्भवः}
{कराग्रेण स्पृशन्नश्वं मुमोच जयकाङ्क्षया}% ७०

\twolineshloka
{वाजिन्गच्छ यथालीलं सर्वत्र धरणीतले}
{यागार्थे मोचितो येन पुनरागच्छ सत्वरः}% ७१

\twolineshloka
{अश्वस्तु मोचितः सर्वैर्भटैः शस्त्रास्त्रकोविदैः}
{परीतः प्रययौ प्राचीं दिशं वायुजवान्वितः}% ७२

\twolineshloka
{प्रचचार बलं सर्वं कम्पयद्धरणीतलम्}
{शेषोऽपि किञ्चिन्न तया फणया धृतवान्भुवम्}% ७३

\twolineshloka
{दिशः प्रसेदुः परितः क्ष्मातलं शोभयान्वितम्}
{वायवस्तं तु शत्रुघ्नं पृष्ठतो मन्दगामिनः}% ७४

\twolineshloka
{शत्रुघ्नस्य प्रयाणायाभ्युद्य तस्य भुजोऽस्फुरत्}
{दक्षिणः शुभमाशंसी जयाय च बभूव ह}% ७५

\twolineshloka
{पुष्कलः स्वगृहं रम्यं प्रविवेश समृद्धिमत्}
{वितर्दिभिर्वलक्षाभिः शोभितं रत्नवेदिकम्}% ७६

\twolineshloka
{तत्रापश्यन्निजां भार्यां पतिव्रतपरायणाम्}
{किञ्चित्स्वदर्शनाद्धृष्टां भर्तृदर्शनलालसाम्}% ७७

\twolineshloka
{मुखारविन्देन च नागवल्लीदलं सुकर्पूरयुतं च चर्वती}
{नासाफलं तोयभवं महाधनं बाह्वोर्मृणालीसदृशोः सुकङ्कणे}% ७८

\twolineshloka
{कुचौ तु मालूरफलोपमौ वरौ नितम्बबिम्बं वरनीवि शोभितम्}
{पादौ तुलाकोटिधरौ सुकोमलौ दधत्यहो एक्षत सत्पतिं स्वकम्}% ७९

\twolineshloka
{परिरभ्य प्रियां धीरो गद्गदस्वरभाषिणीम्}
{तदुरोजपरीरम्भनिर्भरीकृतदेहकाम्}% ८०

\twolineshloka
{उवाच भद्रे गच्छामि शत्रुघ्नपृष्ठरक्षकः}
{रामाज्ञया याज्ञमश्वं पालयन्रथसंयुतः}% ८१

\twolineshloka
{त्वया मे मातरः पूज्याः पादसंवाहनादिमिः}
{तदुच्छिष्टं हि भुञ्जाना तत्कर्मकरणादरा}% ८२

\twolineshloka
{सर्वाः पतिव्रता नार्यो लोपामुद्रादिकाः शुभाः}
{नावमान्यास्त्वया भीरु स्वतपोबलशोभिताः}% ८३

{॥इति श्रीपद्मपुराणे पातालखण्डे शेषवात्स्यायनसंवादे रामाश्वमेधे हयमोचनपुष्कलभार्यासमागमो नाम एकादशोऽध्यायः॥११॥}

\dnsub{द्वादशोऽध्यायः}%\resetShloka

\uvacha{शेष उवाच}

\twolineshloka
{इत्युक्तवन्तं स्वपतिं वीक्ष्य प्रेम्णा सुनिर्भरम्}
{प्रत्युवाच हसन्तीव किञ्चिद्गद्गदभाषिणी}% १

\twolineshloka
{नाथ ते विजयोभूयात्सर्वत्र रणमण्डले}
{शत्रुघ्नाज्ञा प्रकर्तव्या हयरक्षा यथा भवेत्}% २

\twolineshloka
{स्मरणीया हि सर्वत्र सेविका त्वत्पदानुगा}
{कदापि मानसं नाथ त्वत्तो नान्यत्र गच्छति}% ३

\twolineshloka
{परमायोधने कान्त स्मर्तव्याहं न जातुचित्}
{सत्यां मयि तव स्वान्ते युद्धे विजयसंशयः}% ४

\twolineshloka
{पद्मनेत्र तथा कार्यमूर्मिलाद्या यथा मम}
{हास्यं नैव प्रकुर्वन्ति मां वीक्ष्य करताडनैः}% ५

\twolineshloka
{इयं पत्नी महाभीरोः सङ्ग्रामे प्रपलायितुः}
{कातरा यर्हि युद्ध्यन्ति शूराणां समयः कुतः}% ६

\twolineshloka
{इत्येवं न हसन्त्युच्चैर्यथा मे देवराङ्गनाः}
{तथा कार्यं महाबाहो रामस्य हयरक्षणे}% ७

\twolineshloka
{योद्धा त्वमादौ सर्वत्र परे ये तव पृष्ठतः}
{धनुष्टङ्कारबधिराः क्रियन्तां बलिनः परे}% ८

\twolineshloka
{तवप्रोद्यत्कराम्भोज करवालभिया बलम्}
{परेषां भवतात्क्षिप्रमन्योन्य भयव्याकुलम्}% ९

\twolineshloka
{कुलं महदलं कार्यं परान्विजयता त्वया}
{गच्छ स्वामिन्महाबाहो तव श्रेयो भवत्विह}% १०

\twolineshloka
{इदं धनुर्गृहाणाशु महद्गुणविभूषितम्}
{यस्य गर्जितमाकर्ण्य वैरिवृन्दं भयातुरम्}% ११

\twolineshloka
{इमौ ते त्विषुधी वीर बध्येतां शं यथा भवेत्}
{वैरिकोटिविनिष्पेष बाणकोटि सुपूरितौ}% १२

\twolineshloka
{कवचं त्विदमाधेहि शरीरे कामसुन्दरे}
{वज्रप्रभा महादीप्ति हतसन्तमसन्दृढम्}% १३

\twolineshloka
{शिरस्त्राणं निजोत्तंसे कुरु कान्त मनोरमम्}
{इमेव तंसे विशदे मणिरत्नविभूषिते}% १४

\fourlineindentedshloka
{इति सुविमलवाचं वीरपुत्रीं प्रपश्यन्}
{नयनकमलदृष्ट्या वीक्षमाणस्तन्दङ्गम्}
{अधिगतपरिमोदो भारतिः शत्रुजेता}
{रणकरणसमर्थस्तां जगादातिधीरः}% १५

\uvacha{पुष्कल उवाच}

\twolineshloka
{कान्ते यत्त्वं वदसि मां तथा सर्वं चराम्यहम्}
{वीरपत्नी भवेत्कीर्तिस्तव कान्तिमतीप्सिता}% १६

\twolineshloka
{इति कान्तिमतीदत्तं कवचं मुकुटं वरम्}
{धनुर्महेषुधीखड्गं सर्वं जग्राह वीर्यवान्}% १७

\twolineshloka
{परिधाय च तत्सर्वं बहुशो भासमन्वितः}
{शुशुभेऽतीव सुभटः सर्वशस्त्रास्त्रकोविदः}% १८

\twolineshloka
{तमस्त्रशस्त्रशोभाढ्यं वीरमालाविभूषितम्}
{कुङ्कुमागुरुकस्तूरी चन्दनादिकचर्चितम्}% १९

\twolineshloka
{नानाकुसुममालाभिराजानुपरिशोभितम्}
{नीराजयामास मुहुस्तत्र कान्तिमती सती}% २०

\twolineshloka
{नीराजयित्वा बहुशः किरन्ती मौक्तिकैर्मुहुः}
{गलदश्रुचलन्नेत्रा परिरेभे पतिं निजम्}% २१

\twolineshloka
{दृढं सपरिरभ्यैतां चिरमाश्वासयत्तदा}
{वीरपत्नि कान्तिमति विरहं मा कृथा मम}% २२

\twolineshloka
{एष गच्छामि सविधे तव भामे पतिव्रते}
{इत्युक्त्वा तां निजां पत्नीं रथमारुरुहे वरम्}% २३

\twolineshloka
{तं प्रयान्तं पतिं श्रेष्ठं नयनैर्निमिषोज्झितैः}
{विलोकयामास तदा पतिव्रतपरायणा}% २४

\twolineshloka
{स ययौ जनकं द्रष्टुं जननीं प्रेमविह्वलाम्}
{गत्वा पितरमम्बां च ववन्दे शिरसा मुदा}% २५

\twolineshloka
{माता पुत्रं परिष्वज्य स्वाङ्कमारोपयत्तदा}
{मुञ्चन्ती बाष्पनिचयं स्वस्त्यस्त्विति जगाद सा}% २६

\twolineshloka
{पितरं प्राह भरतं रामो यज्ञकरः परः}
{पालनीयो लक्ष्मणेन भवद्भिश्च महात्मभिः}% २७

\twolineshloka
{आज्ञप्तोऽसौ जनन्या च पित्रा हृषितया गिरा}
{ययौ शत्रुघ्नकटकं महावीरविभूषितम्}% २८

\twolineshloka
{रथिभिः पत्तिभिर्वीरैः सदश्वैः सादिभिर्वृतः}
{ययौ मुदा रघूत्तंस महायज्ञहयाग्रणीः}% २९

\twolineshloka
{गच्छन्पाञ्चालदेशांश्च कुरूंश्चैवोत्तरान्कुरून्}
{दशार्णाञ्छ्रीविशालांश्च सर्वशोभासमन्वितः}% ३०

\twolineshloka
{तत्र तत्रोपशृण्वानो रघुवीरयशोऽखिलम्}
{रावणासुरघातेन भक्तरक्षाविधायकम्}% ३१

\twolineshloka
{पुनश्च हयमेधादि कार्यमारभ्य पावनम्}
{यशो वितन्वन्भुवने लोकान्रामोऽविता भयात्}% ३२

\twolineshloka
{तेभ्यस्तुष्टो ददौ हारान्रत्नानि विविधानि च}
{महाधनानि वासांसि शत्रुघ्नः प्रवरो महान्}% ३३

\twolineshloka
{सुमतिर्नाम तेजस्वी सर्वविद्याविशारदः}
{रघुनाथस्य सचिवः शत्रुघ्नानुचरो वरः}% ३४

\twolineshloka
{ययौ तेन महावीरो ग्रामाञ्जनपदान्बहून्}
{रघुनाथप्रतापेन न कोपि हृतवान्हयम्}% ३५

\twolineshloka
{देशाधिपाये बहवो महाबलपराक्रमाः}
{हस्त्यश्वरथपादात चतुरङ्गसमन्विताः}% ३६

\twolineshloka
{सम्पदो बहुशो नीत्वा मुक्तामाणिक्यसंयुताः}
{शत्रुघ्नं हयरक्षार्थमागतं प्रणता मुहुः}% ३७

\twolineshloka
{इदं राज्यं धनं सर्वं सपुत्रपशुबान्धवम्}
{रामचन्द्रस्य सर्वं हि न मदीयं रघूद्वह}% ३८

\twolineshloka
{एवं तदुक्तमाकर्ण्य शत्रुघ्नः परवीरहा}
{आज्ञां स्वां तत्र संज्ञाप्य ययौ तैः सहितः पथि}% ३९

\twolineshloka
{एवं क्रमेण सम्प्राप्तः शत्रुघ्नो हयसंयुतः}
{अहिच्छत्रां पुरीं ब्रह्मन्नानाजनसमाकुलाम्}% ४०

\twolineshloka
{ब्रह्मद्विजसमाकीर्णां नानारत्नविभूषिताम्}
{सौवर्णैः स्फाटिकैर्हर्म्यैर्गोपुरैः समलङ्कृताम्}% ४१

\twolineshloka
{यत्र रम्भा तिरस्कारकारिण्यः कमलाननाः}
{दृश्यन्ते सर्वहर्म्येषु ललना लीलयान्विताः}% ४२

\twolineshloka
{यत्र स्वाचारललिताः सर्वभोगैकभोगिनः}
{धनदानुचरायद्वत्तथा लीलासमन्विताः}% ४३

\twolineshloka
{यत्र वीरा धनुर्हस्ताःशरसन्धानकोविदाः}
{कुर्वन्ति तत्र राजानं सुहृष्टं सुमदाभिधम्}% ४४

\twolineshloka
{एवंविधं ददर्शासौ नगरं दूरतः प्रभुः}
{पार्श्वे तस्य पुरश्रेष्ठमुद्यानं शोभयान्वितम्}% ४५

\twolineshloka
{पुन्नागैर्नागचम्पैश्च तिलकैर्देवदारुभिः}
{अशोकैः पाटलैश्चूतैर्मन्दारैःकोविदारकैः}% ४६

\twolineshloka
{आम्रजम्बुकदम्बैश्च प्रियालपनसैस्तथा}
{शालैस्तालैस्तमालैश्च मल्लिकाजातियूथिभिः}% ४७

\twolineshloka
{नीपैः कदम्बैर्बकुलैश्चम्पकैर्मदनादिभिः}
{शोभितं सददर्शाथशत्रुघ्नःपरवीरहा}% ४८

\fourlineindentedshloka
{हयोगतस्तद्वनमध्यदेशे}
{तमालतालादि सुशोभिते वै}
{ययौ ततः पृष्ठत एव वीरो}
{धनुर्धरैः सेवितपादपद्मः}% ४९

\twolineshloka
{ददर्श त रचितं देवायतनमद्भुतम्}
{इन्द्रनीलैश्च वैडूर्यैस्तथा मारकतैरपि}% ५०

\twolineshloka
{शोभितं सुरसेवार्हं कैलासप्रस्थसन्निभम्}
{जातरूपमयैः स्तम्भैःशोभितं सद्मनां वरम्}% ५१

\twolineshloka
{दृष्ट्वातद्रघुनाथस्य भ्राता देवालयं वरम्}
{पप्रच्छ सुमतिं स्वीयं मन्त्रिणं वदतांवरम्}% ५२

\uvacha{शत्रुघ्न उवाच}

\twolineshloka
{वदामात्य वरेदं किं कस्यदेवस्य केतनम्}
{का देवता पूज्यतेऽत्र कस्य हेतोः स्थितानघ}% ५३

\onelineshloka
{एवमाकर्ण्य यथावदिहसर्वशः}% ५४

\twolineshloka
{कामाक्षायाः परं स्थानं विद्धि विश्वैकशर्मदम्}
{यस्या दर्शनमात्रेण सर्वसिद्धिः प्रजापते}% ५५

\twolineshloka
{देवासुरास्तु यां स्तुत्वा नत्वा प्राप्ताखिलां श्रियम्}
{धर्मार्थकाममोक्षाणां दात्री भक्तानुकम्पिनी}% ५६

\twolineshloka
{याचिता सुमदेनात्राहिच्छत्रा पतिना पुरा}
{स्थिता करोति सकलं भक्तानां दुःखहारिणी}% ५७

\twolineshloka
{तां नमस्कुरु शत्रुघ्न सर्ववीर शिरोमणे}
{नत्वाशु सिद्धिं प्राप्नोषि ससुरासुरदुर्ल्लभाम्}% ५८

\twolineshloka
{इति श्रुत्वाथ तद्वाक्यं शत्रुघ्नः शत्रुतापनः}
{पप्रच्छ सकलां वार्तां भवान्याः पुरुषर्षभः}% ५९

\uvacha{शत्रुघ्न उवाच}

\twolineshloka
{कोऽहिच्छत्रापती राजा सुमदः किं तपः कृतम्}
{येनेयं सर्वलोकानां माता तुष्टात्र संस्थिता}% ६०

\twolineshloka
{वद सर्वं महामात्य नानार्थपरिबृंहितम्}
{यथावत्त्वं हि जानासि तस्माद्वद महामते}% ६१

\uvacha{सुमतिरुवाच}

\twolineshloka
{हेमकूटो गिरिः पूतः सर्वदेवोपशोभितः}
{तत्रास्ति तीर्थं विमलमृषिवृन्दसुसेवितम्}% ६२

\twolineshloka
{सुमदो हि तपस्तेपे हतमातृपितृप्रजः}
{अरिभिः सर्वसामन्तैर्जगाम तपसे हि तम्}% ६३

\twolineshloka
{वर्षाणि त्रीणि सपदा त्वेकेन मनसा स्मरन्}
{जगतां मातरं दध्यौ नासाग्रस्तिमितेक्षणः}% ६४

\twolineshloka
{वर्षाणि त्रीणि शुष्काणां पर्णानां भक्षणं चरन्}
{चकार परमुग्रं स तपः परमदुश्चरम्}% ६५

\twolineshloka
{अब्दानि त्रीणि सलिले शीतकाले ममज्ज सः}
{ग्रीष्मे चचार पञ्चाग्नीन्प्रावृट्सु जलदोन्मुखः}% ६६

\twolineshloka
{त्रीणि वर्षाणि पवनं संरुध्य स्वान्तगोचरम्}
{भवानीं संस्मरन्धीरो न च किञ्चन पश्यति}% ६७

\twolineshloka
{वर्षे तु द्वादशेऽतीते दृष्ट्वैतत्परमं तपः}
{विभाव्य मनसातीव शक्रः पस्पर्ध तं भयात्}% ६८

\twolineshloka
{आदिदेश सकामं तु परिवारपरीवृतम्}
{अप्सरोभिः सुसंयुक्तं ब्रह्मेन्द्रादिजयोद्यतम्}% ६९

\twolineshloka
{गच्छ कामसखे मह्यं प्रियमाचर मोहन}
{सुमदस्य तपोविघ्नं समाचर यथा भवेत्}% ७०

\twolineshloka
{इति श्रुत्वा महद्वाक्यं तुरासाहः स्वयम्प्रभुः}
{उवाच विश्वविजये प्रौढगर्वो रघूद्वह}% ७१

\uvacha{काम उवाच}

\twolineshloka
{स्वामिन्कोऽसौ हि सुमदः किं तपः स्वल्पकं पुनः}
{ब्रह्मादीनां तपोभङ्गं करोम्यस्य तु का कथा}% ७२

\twolineshloka
{मद्बाणबलनिर्भिन्नश्चन्द्रस्तारां गतः पुरा}
{त्वमप्यहल्यां गतवान्विश्वामित्रस्तु मेनिकाम्}% ७३

\twolineshloka
{चिन्तां मा कुरु देवेन्द्र सेवके मयि संस्थिते}
{एष गच्छामि सुमदं देवान्पालय मारिष}% ७४

\twolineshloka
{एवमुक्त्वा कामदेवो हेमकूटं गिरिं ययौ}
{वसन्तेन युतः सख्या तथैवाप्सरसाङ्गणैः}% ७५

\twolineshloka
{वसन्तस्तत्र सकलान्वृक्षान्पुष्पफलैर्युतान्}
{कोकिलान्षट्पदश्रेण्या घुष्टानाशु चकार ह}% ७६

\twolineshloka
{वायुः सुशीतलो वाति दक्षिणां दिशमाश्रितः}
{कृतमालासरित्तीर लवङ्गकुसुमान्वितः}% ७७

\twolineshloka
{एवंविधे वने वृत्ते रम्भानामाप्सरोवरा}
{सखीभिः संवृता तत्र जगाम सुमदान्तिकम्}% ७८

\twolineshloka
{तत्रारभत गानं सा किन्नरस्वरशोभना}
{मृदङ्गपणवानेकवाद्यभेदविशारदा}% ७९

\fourlineindentedshloka
{तद्गानमाकर्ण्य नराधिपोऽसौ}
{वसन्तमालोक्य मनोहरं च}
{तथान्यपुष्टारटितं मनोरमं}
{चकार चक्षुः परिवर्तनं बुधः}% ८०

\twolineshloka
{तं प्रबुद्धं नृपं वीक्ष्य कामः पुष्पायुधस्त्वरन्}
{चकार सत्वरं सज्यं धनुस्तत्पृष्ठतोऽनघ}% ८१

\fourlineindentedshloka
{एकाप्सरास्तत्र नृपस्य पादयोः}
{संवाहनं नर्तितनेत्रपल्लवा}
{चकार चान्या तु कटाक्षमोक्षणं}
{चकार काचिद्भृशमङ्गचेष्टितम्}% ८२

\twolineshloka
{अप्सरोभिस्तथाकीर्णः कामविह्वलमानसः}
{चिन्तयामास मतिमाञ्जितेन्द्रियशिरोमणिः}% ८३

\twolineshloka
{एता मे तपसो विघ्नकारिण्योऽप्सरसां वराः}
{शक्रेण प्रेषिताः सर्वाः करिष्यन्ति यथातथम्}% ८४

\twolineshloka
{इति सञ्चिन्त्य सुतपास्ता उवाच वराङ्गनाः}
{का यूयं कुत्र संस्थाः किं भवतीनां चिकीर्षितम्}% ८५

\twolineshloka
{अत्यद्भुतं जातमहो यद्भवत्योऽक्षिगोचराः}
{यास्तपोभिः सुदुष्प्राप्यास्ता मे तपस आगताः}% ८६

{॥इति श्रीपद्मपुराणे पातालखण्डे शेषवात्स्यायनसंवादे रामाश्वमेधे कामाक्षोपाख्यानं नाम द्वादशोऽध्यायः॥१२॥}

\dnsub{त्रयोदशोऽध्यायः}%\resetShloka

\uvacha{शेष उवाच}

\twolineshloka
{इति वाक्यं समाकर्ण्य सुमदस्य तपोनिधेः}
{जगदुः कामसेनास्तं रम्भाद्यप्सरसो मुदा}% १

\twolineshloka
{त्वत्तपोभिर्वयं कान्त प्राप्ताः सर्ववराङ्गनाः}
{तासां यौवनसर्वस्वं भुङ्क्ष्व त्यज तपःफलम्}% २

\twolineshloka
{इयं घृताची सुभगा चम्पकाभशरीरभृत्}
{कर्पूरगन्धललितं भुनक्तु त्वन्मुखामृतम्}% ३

\fourlineindentedshloka
{एतां महाभाग सुशोभिविभ्रमां}
{मनोहराङ्गीं घनपीनसत्कुचाम्}
{कान्तोपभुङ्क्ष्वाशु निजोग्रपुण्यतः}
{प्राप्तां पुनस्त्वं त्यज दुःखजातम्}% ४

\fourlineindentedshloka
{मामप्यनर्घ्याभरणोपशोभितां}
{मन्दारमालापरिशोभिवक्षसम्}
{नानारताख्यानविचारचञ्चुरां}
{दृढं यथा स्यात्परिरम्भणं कुरु}% ५

\fourlineindentedshloka
{पिबामृतं मामकवक्त्रनिर्गतं}
{विमानमारुह्य वरं मया सह}
{सुमेरुशृङ्गं बहुपुण्यसेवितं}
{सम्प्राप्य भोगं कुरु सत्तपः फलम्}% ६

\fourlineindentedshloka
{तिलोत्तमा यौवनरूपशोभिता}
{गृह्णातु ते मूर्धनि तापवारणम्}
{सुचामरौ सन्ततधारयाङ्कितौ}
{गङ्गाप्रवाहाविव सुन्दरोत्तम}% ७

\fourlineindentedshloka
{शृणुष्व भोः कामकथां मनोहरां}
{पिबामृतं देवगणादिवाञ्छितम्}
{उद्यानमासाद्य च नन्दनाभिधं}
{वराङ्गनाभिर्विहरं कुरु प्रभो}% ८

\fourlineindentedshloka
{इत्युक्तमाकर्ण्य महामतिर्नृपो}
{विचारयामास कुतो ह्युपस्थिताः}
{मया सुसृष्टास्तपसा सुराङ्गनाः}
{प्रत्यूह एवात्र विधेयमेष किम्}% ९

\twolineshloka
{इति चिन्तातुरो राजा स्वान्ते सञ्चिन्तयन्सुधीः}
{जगाद मतिमान्वीरः सुमदो देवताङ्गनाः}% १०

\twolineshloka
{यूयं तु ममचित्तस्था जगन्मातृस्वरूपकाः}
{मया सञ्चिन्त्यते या हि सापि त्वद्रूपिणी मता}% ११

\twolineshloka
{इदं तुच्छं स्वर्गसुखं त्वयोक्तं सविकल्पकम्}
{मत्स्वामिनी मया भक्त्या सेविता दास्यते वरम्}% १२

\twolineshloka
{यत्कृपातो विधिः सत्यलोकं प्राप्तो महानभूत्}
{सा मे दास्यति सर्वं हि भक्तदुःखान्तकारिणी}% १३

\twolineshloka
{किं नन्दनं किं तु गिरिः कनकेन सुमण्डितः}
{किं सुधा स्वल्पपुण्येन प्राप्या दानवदुःखदा}% १४

\twolineshloka
{इति वाक्यं समाकर्ण्य कामस्तु विविधैः शरैः}
{प्राहरन्नरदेवस्य कर्तुं किञ्चिन्न वै प्रभुः}% १५

\twolineshloka
{कटाक्षैर्नूपुरारावैः परिरम्भैर्विलोकनैः}
{न तस्य चित्तं विभ्रान्तं कर्तुं शक्ता वराङ्गनाः}% १६

\twolineshloka
{गत्वा यथागतं शक्रं जगदुर्धीरधीर्नृपः}
{तच्छ्रुत्वा मघवा भीतः सेवामारभतात्मनः}% १७

\twolineshloka
{अथ निश्चितमालोक्य पादपद्मे स्वकेऽम्बिका}
{जितेन्द्रियं महाराजं प्रत्यक्षाभूत्सुतोषिता}% १८

\twolineshloka
{पञ्चास्यपृष्ठललिता पाशाङ्कुशधरावरा}
{धनुर्बाणधरा माता जगत्पावनपावनी}% १९

\twolineshloka
{तां वीक्ष्य मातरं धीमान्सूर्यकोटिसमप्रभाम्}
{धनुर्बाणसृणीपाशान्दधानां हर्षमाप्तवान्}% २०

\twolineshloka
{शिरसा बहुशो नत्वा मातरं भक्तिभाविताम्}
{हसन्तीं निजदेहेषु स्पृशन्तीं पाणिना मुहुः}% २१

\twolineshloka
{तुष्टाव भक्त्युत्कलितचित्तवृत्तिर्महामतिः}
{गद्गदस्वरसंयुक्तः कण्टकाङ्गोपशोभितः}% २२

\twolineshloka
{जय देवि महादेवि भक्तवृन्दैकसेविते}
{ब्रह्मरुद्रादिदेवेन्द्र सेविताङ्घ्रियुगेऽनघे}% २३

\twolineshloka
{मातस्तव कलाविद्धमेतद्भाति चराचरम्}
{त्वदृते नास्ति सर्वं तन्मातर्भद्रे नमोस्तु ते}% २४

\twolineshloka
{मही त्वयाऽधारशक्त्या स्थापिता चलतीह न}
{सपर्वतवनोद्यान दिग्गजैरुपशोभिता}% २५

\twolineshloka
{सूर्यस्तपति खे तीक्ष्णैरंशुभिः प्रतपन्महीम्}
{त्वच्छक्त्या वसुधासंस्थं रसं गृह्णन्विमुञ्चति}% २६

\twolineshloka
{अन्तर्बहिः स्थितो वह्निर्लोकानां प्रकरोतु शम्}
{त्वत्प्रतापान्महादेवि सुरासुरनमस्कृते}% २७

\twolineshloka
{त्वं विद्या त्वं महामाया विष्णोर्लोकैकपालिनः}
{स्वशक्त्या सृजसीदं त्वं पालयस्यपि मोहिनि}% २८

\twolineshloka
{त्वत्तः सर्वे सुराः प्राप्य सिद्धिं सुखमयन्ति वै}
{मां पालय कृपानाथे वन्दिते भक्तवल्लभे}% २९

\twolineshloka
{रक्ष मां सेवकं मातस्त्वदीयचरणारणम्}
{कुरु मे वाञ्छितां सिद्धिं महापुरुषपूर्वजे}% ३०

\uvacha{सुमतिरुवाच}

\twolineshloka
{एवं तुष्टा जगन्माता वृणीष्व वरमुत्तमम्}
{उवाच भक्तं सुमदं तपसा कृशदेहिनम्}% ३१

\twolineshloka
{इत्येतद्वाक्यमाकर्ण्य प्रहृष्टः सुमदो नृपः}
{वव्रे निजं हृतं राज्यं हतदुर्जनकण्टकम्}% ३२

\twolineshloka
{महेशीचरणद्वन्द्वे भक्तिमव्यभिचारिणीम्}
{प्रान्ते मुक्तिं तु संसारवारिधेस्तारिणीं पुनः}% ३३

\uvacha{कामाक्षोवाच}

\twolineshloka
{राज्यं प्राप्नुहि सुमद सर्वत्रहतकण्टकम्}
{महिलारत्नसञ्जुष्टपादपद्मद्वयो भव}% ३४

\twolineshloka
{ततवैरिपराभूतिर्माभूयात्सुमदाभिध}
{यदा तु रावणं हत्वा रघुनाथो महायशाः}% ३५

\twolineshloka
{करिष्यत्यश्वमेधं हि सर्वसम्भारशोभितम्}
{तस्य भ्राता महावीरः शत्रुघ्नः परवीरहा}% ३६

\twolineshloka
{पालयन्हयमायास्यत्यत्र वीरादिभिर्वृतः}
{तस्मै सर्वं समर्प्य त्वं राज्यमृद्धं धनादिकम्}% ३७

\twolineshloka
{पालयिष्यसि योधैः स्वैर्धनुर्धारिभिरुद्भटैः}
{ततः पृथिव्यां सर्वत्र भ्रमिष्यसि महामते}% ३८

\twolineshloka
{ततो रामं नमस्कृत्य ब्रह्मेन्द्रेशादिसेवितम्}
{मुक्तिं प्राप्स्यसि दुष्प्रापां योगिभिर्यमसाधनैः}% ३९

\twolineshloka
{तावत्कालमिहस्थास्ये यावद्रामहयागमः}
{पश्चात्त्वां तु समुद्धृत्य गन्तास्मि परमं पदम्}% ४०

\twolineshloka
{इत्युक्त्वान्तर्दधे देवी सुरासुरनमस्कृता}
{सुमदोऽप्यहिच्छत्रायां शत्रून्हत्वा नृपोऽभवत्}% ४१

\twolineshloka
{एष राजा समर्थोऽपि बलवाहनसंयुतः}
{न ग्रहीष्यति ते वाहं महामायासुशिक्षितः}% ४२

\twolineshloka
{श्रुत्वा प्राप्तं पुरी पार्श्वे हयमेधहयोत्तमम्}
{त्वां च सर्वैर्महाराजैः सेविताङ्घ्रिं महामतिम्}% ४३

\twolineshloka
{सर्वं दास्यति सर्वज्ञ राजा सुमदनामधृक्}
{अधुनातन्महाराज रामचन्द्र प्रतापतः}% ४४

\uvacha{शेष उवाच}

\twolineshloka
{इति वृत्तं समाकर्ण्य सुमदस्य महायशाः}
{साधुसाध्विति चोवाच जहर्ष मतिमान्बली}% ४५

\twolineshloka
{अहिच्छत्रापतिः सर्वैः स्वगणैः परिवारितः}
{सभायां सुखमास्ते यो बहुराजन्यसेवितः}% ४६

\twolineshloka
{ब्राह्मणा वेदविदुषो वैश्या धनसमृद्धयः}
{राजानं पर्युपासन्ते सुमदंशो भयान्वितम्}% ४७

\twolineshloka
{वेदविद्याविनोदेन न्यायिनो ब्राह्मणा वराः}
{आशीर्वदन्ति तं भूपं सर्वलोकैकरक्षकम्}% ४८

\twolineshloka
{एतस्मिन्समये कश्चिदागत्य नृपतिं जगौ}
{स्वामिन्न जाने कस्यास्ति हयः पत्रधरोऽन्तिके}% ४९

\twolineshloka
{तच्छ्रुत्वा सेवकं श्रेष्ठं प्रेषयामास सत्वरः}
{जानीहि कस्य राज्ञोऽयमश्वो मम पुरान्तिके}% ५०

\twolineshloka
{गत्वाथ सेवकस्तत्र ज्ञात्वा वृत्तान्तमादितः}
{निवेदयामास नृपं महाराजन्यसेवितम्}% ५१

\twolineshloka
{स श्रुत्वा रघुनाथस्य हयं नित्यमनुस्मरन्}
{आज्ञापयामास जनं सर्वं राजाविशारदः}% ५२

\twolineshloka
{लोका मदीयाः सर्वे ये धनधान्यसमाकुलाः}
{तोरणादीनि गेहेषु मङ्गलानि सृजन्त्विह}% ५३

\twolineshloka
{कन्याः सहस्रशो रम्याः सर्वाभरणभूषिताः}
{गजोपरिसमारूढा यान्तु शत्रुघ्नसम्मुखम्}% ५४

\twolineshloka
{इत्यादिसर्वमाज्ञाप्य ययौ राजा स्वयं ततः}
{पुत्रपौत्रमहिष्यादिपरिवारसमावृतः}% ५५

\twolineshloka
{शत्रुघ्नः सुमहामात्यैः सुभटैः पुष्कलादिभिः}
{संयुतो भूपतिं वीरं ददर्श सुमदाभिधम्}% ५६

\twolineshloka
{हस्तिभिः सादिसंयुक्तैः पत्तिभिः परतापनैः}
{वाजिभिर्भूषितैर्वीरैः संयुतं वीरशोभितम्}% ५७

\twolineshloka
{अथागत्य महाराजः शत्रुघ्नं नतवान्मुदा}
{धन्योऽस्मि कृतकृत्योऽस्मि सत्कृतं च कृतं वपुः}% ५८

\twolineshloka
{इदं राज्यं गृहाणाशु महाराजोपशोभितम्}
{महामाणिक्यमुक्तादि महाधनसुपूरितम्}% ५९

\twolineshloka
{स्वामिंश्चिरं प्रतीक्षेऽहं हयस्यागमनं प्रति}
{कामाक्षाकथितं पूर्वं जातं सम्प्रति तत्तथा}% ६०

\twolineshloka
{विलोकय पुरं मह्यं कृतार्थान्कुरु मानवान्}
{पावयास्मत्कुलं सर्वं रामानुज महीपते}% ६१

\twolineshloka
{इत्युक्त्वारोहयामास कुञ्जरं चन्द्रसुप्रभम्}
{पुष्कलं च महावीरं तथा स्वयमथारुहत्}% ६२

\twolineshloka
{भेरीपणवतूर्याणां वीणादीनां स्वनस्तदा}
{व्याप्नोति स्म महाराज सुमदेन प्रणोदितः}% ६३

\fourlineindentedshloka
{कन्याः समागत्य महानरेन्द्रं-}
{शत्रुघ्नमिन्द्रादिकसेविताङ्घ्रिम्}
{करिस्थिता मौक्तिकवृन्दसङ्घै-}
{र्वर्धापयामासुरिनप्रयुक्ताः}% ६४

\twolineshloka
{शनैःशनैः समागत्य पुरीमध्ये जनैर्मुदा}
{वर्धापितो गृहं प्राप तोरणादिकभूषितम्}% ६५

\twolineshloka
{हयरत्नेन संयुक्तस्तथा वीरैः सुशोभितः}
{राज्ञा पुरस्कृतो राजा शत्रुघ्नः प्राप मन्दिरम्}% ६६

\twolineshloka
{अर्घादिभिः पूजयित्वा रघुनाथानुजं तदा}
{सर्वं समर्पयामास रामचन्द्राय धीमते}% ६७

{॥इति श्रीपद्मपुराणे पातालखण्डे शेषवात्स्यायनसंवादे रामाश्वमेधे शत्रुघ्नाहिच्छत्रापुरीप्रवेशो नाम त्रयोदशोऽध्यायः॥१३॥}

\dnsub{चतुर्दशोऽध्यायः}%\resetShloka

\uvacha{शेष उवाच}

\twolineshloka
{अथ स्वागतसन्तुष्टं शत्रुघ्नं प्राह भूमिपः}
{रघुनाथकथां श्रेष्ठां शुश्रूषुः पुरुषर्षभः}% १

\uvacha{सुमद उवाच}

\twolineshloka
{कच्चिदास्ते सुखं रामः सर्वलोकशिरोमणिः}
{भक्तरक्षावतारोऽयं ममानुग्रहकारकः}% २

\twolineshloka
{धन्या लोका इमे पुर्यां रघुनाथमुखाम्बुजम्}
{ये पिबन्त्यनिशं चाक्षिपुटकैः परिमोदिताः}% ३

\twolineshloka
{अर्थजातं मदीयं च नितरां पुरुषर्षभ}
{कृतार्थं कुलभूम्यादि वस्तुजातं महामते}% ४

\twolineshloka
{कामाक्षया प्रसादो मे कृतः पूर्वं दयार्द्रया}
{रघुनाथमुखाम्भोजं द्रक्ष्येद्य सकुटुम्बकः}% ५

\twolineshloka
{इत्युक्तवति वीरे तु सुमदे पार्थिवोत्तमे}
{सर्वं तत्कथयामास रघुनाथगुणोदयम्}% ६

\twolineshloka
{त्रिरात्रं तत्र संस्थित्य रघुनाथानुजः परम्}
{गन्तुं चकार धिषणां राज्ञा सह महामतिः}% ७

\twolineshloka
{तज्ज्ञात्वा सुमदः शीघ्रं पुत्रं राज्येऽभ्यषेचयत्}
{शत्रुघ्नेन महाराज्ञा पुष्कलेनानुमोदितः}% ८

\twolineshloka
{वासांसि बहुरत्नानि धनानि विविधानि च}
{शत्रुघ्नसेवकेभ्योऽसौ प्रादात्तत्र महामतिः}% ९

\twolineshloka
{ततो गमनमारेभे मन्त्रिभिर्बहुवित्तमैः}
{पत्तिभिर्वाजिभिर्नागैः सदश्वैरथ कोटिभिः}% १०

\twolineshloka
{शत्रुघ्नः सहितस्तेन सुमदेन धनुर्भृता}
{जगाम मार्गे विहसन्रघुनाथप्रतापभृत्}% ११

\twolineshloka
{पयोष्णीतीरमासाद्य जगाम स हयोत्तमः}
{पृष्ठतोऽनुययुः सर्वे योधा वै हयरक्षिणः}% १२

\twolineshloka
{आश्रमान्विविधान्पश्यन्नृषीणां सुतपोभृताम्}
{तत्रतत्र विशृण्वानो रघुनाथगुणोदयम्}% १३

\twolineshloka
{एष धीमान्हरिर्याति हरिणा परिरक्षितः}
{हरिभिर्हरिभक्तैश्च हरिवर्यानुगैर्मुहुः}% १४

\twolineshloka
{इति शृण्वञ्छुभा वाचो मुनीनां परितः प्रभुः}
{तुतोष भक्त्युत्कलितचित्तवृत्तिभृतां महान्}% १५

\twolineshloka
{ददर्श चाश्रमं शुद्धं जनजन्तुसमाकुलम्}
{वेदध्वनिहताशेषा मङ्गलं शृण्वतां नृणाम्}% १६

\twolineshloka
{अग्निहोत्रहविर्धूम पवित्रितनभोखिलम्}
{मुनिवर्यकृतानेक यागयूपसुशोभितम्}% १७

\twolineshloka
{यत्र गावस्तु हरिणा पाल्यन्ते पालनोचिताः}
{मूषका न खनन्त्यस्मिन्बिडालस्य भयाद्बिलम्}% १८

\twolineshloka
{मयूरैर्नकुलैः सार्द्धं क्रीडन्ति फणिनोनिशम्}
{गजैः सिंहैर्नित्यमत्र स्थीयते मित्रतां गतैः}% १९

\twolineshloka
{एणास्तत्रत्य नीवारभक्षणेषु कृतादराः}
{न भयं कुर्वते कालाद्रक्षिता मुनिवृन्दकैः}% २०

\twolineshloka
{गावः कुम्भसमोधस्का नन्दिनी समविग्रहाः}
{कुर्वन्ति चरणोत्थेन रजसेलां पवित्रिताम्}% २१

\twolineshloka
{मुनिवर्याः समित्पाणि पद्मैर्धर्मक्रियोचिताम्}
{दृष्ट्वा पप्रच्छसुमतिं सर्वज्ञं राम मन्त्रिणम्}% २२

\uvacha{शत्रुघ्न उवाच}

\twolineshloka
{सुमते कस्य संस्थानं मुनेर्भाति पुरोगतम्}
{निर्वैरिजन्तु संसेव्यं मुनिवृन्दसमाकुलम्}% २३

\twolineshloka
{श्रोष्यामि मुनिवार्तां च विदधामि पवित्रताम्}
{निजं वपुस्तदीयाभिर्वार्ताभिर्वर्णनादिभिः}% २४

\twolineshloka
{इति श्रुत्वा महद्वाक्यं शत्रुघ्नस्य महात्मनः}
{कथयामास सचिवो रघुनाथस्य धीमतः}% २५

\uvacha{सुमतिरुवाच}

\twolineshloka
{च्यवनस्याश्रमं विद्धि महातापसशोभितम्}
{निर्वैरिजन्तुसङ्कीर्णं मुनिपत्नीभिरावृतम्}% २६

\twolineshloka
{योऽसौ महामुनिः स्वर्गवैद्ययोर्भागमादधात्}
{स्वायम्भुवमहायज्ञे शक्रमानविभेदनः}% २७

\twolineshloka
{महामुनेः प्रभावोऽयं न केनापि समाप्यते}
{तपोबलसमृद्धस्य वेदमूर्तिधरस्य ह}% २८

\twolineshloka
{श्रुत्वा रामानुजो वार्तां च्यवनस्य महामुनेः}
{सर्वं पप्रच्छ सुमतिं शक्रमानादिभञ्जनम्}% २९

\uvacha{शत्रुघ्न उवाच}

\twolineshloka
{कदासौ दस्रयोर्भागं चकार सुरपङ्क्तिषु}
{किं कृतं देवराजेन स्वायम्भुव महामखे}% ३०

\uvacha{सुमतिरुवाच}

\twolineshloka
{ब्रह्मवंशेऽतिविख्यातो मुनिर्भृगुरिति श्रुतः}
{कदाचिद्गतवान्सायं समिदाहरणं प्रति}% ३१

\twolineshloka
{तदा मखविनाशाय दमनो राक्षसो बली}
{आगत्योच्चैर्जगादेदं महाभयकरं वचः}% ३२

\twolineshloka
{कुत्रास्ति मुनिबन्धुः स कुत्र तन्महिलानघा}
{पुनः पुनरुवाचेदं वचो रोषसमाकुलः}% ३३

\twolineshloka
{तदाहुतवहो ज्ञात्वा राक्षसाद्भयमागतम्}
{दर्शयामास तज्जायामन्तर्वत्नीमनिन्दिताम्}% ३४

\twolineshloka
{जहार राक्षसस्तां तु रुदन्तीं कुररीमिव}
{भृगो रक्षपते रक्ष रक्ष नाथ तपोनिधे}% ३५

\twolineshloka
{एवं वदन्तीमार्तां तां गृहीत्वा निरगाद्बहिः}
{दुष्टो वाक्यप्रहारेण बोधयन्स भृगोः सतीम्}% ३६

\twolineshloka
{ततो महाभयत्रस्तो गर्भश्चोदरमध्यतः}
{पपात प्रज्वलन्नेत्रो वैश्वानर इवाङ्गजः}% ३७

\twolineshloka
{तेनोक्तं मा व्रजाशु त्वं भस्मी भव सुदुर्मते}
{न हि साध्वी परामर्शं कृत्वा श्रेयोऽधियास्यसि}% ३८

\twolineshloka
{इत्युक्तः स पपाताशु भस्मीभूतकलेवरः}
{माता तदार्भकं नीत्वा जगामाश्रममुन्मनाः}% ३९

\twolineshloka
{भृगुर्वह्निकृतं सर्वं ज्ञात्वा कोपसमाकुलः}
{शशाप सर्वभक्षस्त्वं भव दुष्टारिसूचक}% ४०

\twolineshloka
{तदा शप्तोऽतिदुःखार्तो जग्राहाङ्घ्र्याशुशुक्षणिः}
{कुरु मेऽनुग्रहं स्वामिन्कृपार्णव महामते}% ४१

\twolineshloka
{मयानृतं वचोभीत्या कथितं न गुरुद्रुहा}
{तस्मान्ममोपरि कृपां कुरु धर्मशिरोमणे}% ४२

\twolineshloka
{तदानुग्रहमाधाच्च सर्वभक्षो भवाञ्छुचिः}
{इत्युक्तवान्हुतभुजं दयार्द्रो मुनितापसः}% ४३

\twolineshloka
{गर्भाच्च्युतस्य पुत्रस्य जातकर्मादिकं शुचिः}
{चकार विधिवद्विप्रो दर्भपाणिः सुमङ्गलः}% ४४

\twolineshloka
{च्यवनाच्च्यवनं प्राहुः पुत्रं सर्वे तपस्विनः}
{शनैःशनैः स ववृधे शुक्ले प्रतिपदिन्दुवत्}% ४५

\twolineshloka
{स जगाम तपः कर्तुं रेवां लोकैकपावनीम्}
{शिष्यैः परिवृतः सर्वैस्तपोबलसमन्वितैः}% ४६

\twolineshloka
{गत्वा तत्र तपस्तेपे वर्षाणामयुतं महान्}
{अंसयोः किंशुकौ जातौ वल्मीकोपरिशोभितौ}% ४७

\twolineshloka
{मृगा आगत्य तस्याङ्गे कण्डूं विदधुरुत्सुकाः}
{न किञ्चित्स हि जानाति दुर्वारतपसावृतः}% ४८

\twolineshloka
{कदाचिन्मनुरुद्युक्तस्तीर्थयात्रां प्रति प्रभुः}
{सकुटुम्बो ययौ रेवां महाबलसमावृतः}% ४९

\twolineshloka
{तत्र स्नात्वा महानद्यां सन्तर्प्य पितृदेवताः}
{दानानि ब्राह्मणेभ्यश्च प्रादाद्विष्णुप्रतुष्टये}% ५०

\twolineshloka
{तत्कन्या विचरन्ती सा वनमध्ये इतस्ततः}
{सखीभिः सहिता रम्या तप्तहाटकभूषणा}% ५१

\twolineshloka
{तत्र दृष्ट्वाथ वल्मीकं महातरुसुशोभितम्}
{निमेषोन्मेषरहितं तेजः किञ्चिद्ददर्श सा}% ५२

\twolineshloka
{गत्वा तत्र शलाकाभिरतुदद्रुधिरं स्रवत्}
{दृष्ट्वा राज्ञाङ्गजा खेदं प्राप्तवत्यतिदुःखिता}% ५३

\twolineshloka
{न जनन्यै तथा पित्रे शशंसाघेन विप्लुता}
{स्वयमेवात्मनात्मानं सा शुशोच भयातुरा}% ५४

\twolineshloka
{तदा भूश्चलिता राजन्दिवश्चोल्का पपात ह}
{धूम्रा दिशो भवन्सर्वाः सूर्यश्च परिवेषितः}% ५५

\twolineshloka
{तदा राज्ञो हया नष्टा हस्तिनो बहवो मृताः}
{धनं नष्टं रत्नयुतं कलहोभून्मिथस्तदा}% ५६

\twolineshloka
{तदालोक्य नृपो भीतः किञ्चिदुद्विग्नमानसः}
{जनानपृच्छत्केनापि मुनये त्वपराधितम्}% ५७

\twolineshloka
{पारम्पर्येण तज्ज्ञात्वा स्वपुत्र्याः परिचेष्टितम्}
{ययौ सुदुःखितस्तत्र समृद्धबलवाहनः}% ५८

\twolineshloka
{तं वै तपोनिधिं वीक्ष्य महता तपसायुतम्}
{स्तुत्वा प्रसादयामास मुनिवर्य दयां कुरु}% ५९

\twolineshloka
{तस्मै तुष्टो जगादायं मुनिवर्यो महातपाः}
{तवात्मजाकृतं सर्वमुत्पाताद्यमवेहि तत्}% ६०

\twolineshloka
{तव पुत्र्या महाराज चक्षुर्विस्फोटनं कृतम्}
{बहुसुस्राव रुधिरं जानती त्वामुवाच न}% ६१

\twolineshloka
{तस्मादियं महाभूप मह्यं देया यथाविधि}
{ततश्चोत्पातशमनं भविष्यति न संशयः}% ६२

\twolineshloka
{तच्छ्रुत्वा दुःखितो राजा प्रज्ञाचाक्षुष आत्मजाम्}
{ददौ कुलवयोरूप शीललक्षणसंयुताम्}% ६३

\twolineshloka
{दत्ता यदा नृपेणेयं कन्या कमललोचना}
{तदोत्पाताः शमं याताः सर्वे मुनिरुषोद्गताः}% ६४

\twolineshloka
{राजा दत्त्वात्मजां तस्मै मुनये तपसान्निधे}
{प्राप स्वां नगरीं भूयो दुःखितोऽयं दयायुतः}% ६५

{॥इति श्रीपद्मपुराणे पातालखण्डे शेषवात्स्यायनसंवादे रामाश्वमेधे च्यवनोपाख्यानं नाम चतुर्दशोऽध्यायः॥१४॥}

\dnsub{पञ्चदशोऽध्यायः}%\resetShloka

\uvacha{सुमतिरुवाच}

\twolineshloka
{अथर्षिः स्वाश्रमं गत्वा मानव्या सह भार्यया}
{मुदं प्राप हताशेष पातको योगयुक्तया}% १

\fourlineindentedshloka
{सा मानवी तं वरमात्मनः पतिं}
{नेत्रेणहीनं जरसा गतौजसम्}
{सिषेव एनं हरिमेधसोत्तमं}
{निजेष्टदात्रीं कुलदेवतां यथा}% २

\fourlineindentedshloka
{शूश्रूषती स्वं पतिमिङ्गितज्ञा}
{महानुभावं तपसां निधिं प्रियम्}
{परां मुदं प्राप सती मनोहरा}
{शची यथा शक्रनिषेवणोद्यता}% ३

\twolineshloka
{चरणौ सेवते तन्वी सर्वलक्षणलक्षिता}
{राजपुत्री सुन्दराङ्गी फलमूलोदकाशना}% ४

\twolineshloka
{नित्यं तद्वाक्यकरणे तत्परा पूजने रता}
{कालक्षेपं प्रकुरुते सर्वभूतहिते रता}% ५

\twolineshloka
{विसृज्य कामं दम्भं च द्वेषं लोभमघं मदम्}
{अप्रमत्तोद्यता नित्यं च्यवनं समतोषयत्}% ६

\twolineshloka
{एवं तस्य प्रकुर्वाणा सेवां वाक्कायकर्मभिः}
{सहस्राब्दं महाराज सा च कामं मनस्यधात्}% ७

\twolineshloka
{कदाचिद्देवभिषजावागतावाश्रमे मुनेः}
{स्वागतेन सुसम्भाव्य तयोः पूजां चकार सा}% ८

\fourlineindentedshloka
{शर्यातिकन्याकृतपूजनार्घ-}
{पाद्यादिना तोषितचित्तवृत्ती}
{तावूचतुः स्नेहवशेन सुन्दरौ}
{वरं वृणुष्वेति मनोहराङ्गीम्}% ९

\twolineshloka
{तुष्टौ तौ वीक्ष्य भिषजौ देवानां वरयाचने}
{मतिं चकार नृपतेः पुत्री मतिमतां वरा}% १०

\twolineshloka
{पत्यभिप्रायमालक्ष्य वाचमूचे नृपात्मजा}
{दत्तं मे चक्षुषी पत्युर्यदि तुष्टौ युवां सुरौ}% ११

\twolineshloka
{इत्येतद्वचनं श्रुत्वा सुकन्या या मनोहरम्}
{सतीत्वं च विलोक्येदमूचतुर्भिषजां वरौ}% १२

\twolineshloka
{त्वत्पतिर्यदि देवानां भागं यज्ञे दधात्यसौ}
{आवयोरधुना कुर्वश्चक्षुषोः स्फुटदर्शनम्}% १३

\twolineshloka
{च्यवनोऽप्योमिति प्राह भागदाने वरौजसोः}
{तदा हृष्टावश्विनौ तमूचतुस्तपतां वरम्}% १४

\twolineshloka
{निमज्जतां भवानस्मिन्ह्रदे सिद्धविनिर्मिते}
{इत्युक्तो जरयाग्रस्त देहो धमनिसन्ततः}% १५

\twolineshloka
{ह्रदं प्रवेशितोऽश्विभ्यां स्वयं चामज्जतां ह्रदे}
{पुरुषास्त्रय उत्तस्थुरपीच्या वनिताप्रियाः}% १६

\twolineshloka
{रुक्मस्रजः कुण्डलिनस्तुल्यरूपाः सुवाससः}
{तान्निरीक्ष्य वरारोहा सुरूपान्सूर्यवर्चसः}% १७

\twolineshloka
{अजानती पतिं साध्वी ह्यश्विनौ शरणं ययौ}
{दर्शयित्वा पतिं तस्यै पातिव्रत्येन तोषितौ}% १८

\twolineshloka
{ऋषिमामन्त्र्य ययतुर्विमानेन त्रिविष्टपम्}
{यक्ष्यमाणे क्रतौ स्वीयभागकार्याशयायुतौ}% १९

\twolineshloka
{कालेन भूयसा क्षामां कर्शितां व्रतचर्यया}
{प्रेमगद्गदया वाचा पीडितः कृपयाब्रवीत्}% २०

\fourlineindentedshloka
{तुष्टोऽहमद्य तव भामिनि मानदायाः}
{शुश्रूषया परमया हृदि चैकभक्त्या}
{यो देहिनामयमतीव सुहृत्स्वदेहो}
{नावेक्षितः समुचितः क्षपितुं मदर्थे}% २१

\fourlineindentedshloka
{ये मे स्वधर्मनिरतस्य तपः समाधि}
{विद्यात्मयोगविजिता भगवत्प्रसादाः}
{तानेव ते मदनुसेवनयाऽविरुद्धान्}
{दृष्टिं प्रपश्य वितराम्यभयानशोकान्}% २२

\fourlineindentedshloka
{अन्ये पुनर्भगवतो भ्रुव उद्विजृम्भ-}
{विस्रंसितार्थरचनाः किमुरुक्रमस्य}
{सिद्धासि भुङ्क्ष्व विभवान्निजधर्मदोहान्}
{दिव्यान्नरैर्दुरधिगान्नृपविक्रियाभिः}% २३

\fourlineindentedshloka
{एवं ब्रुवाणमबलाखिलयोगमाया}
{विद्याविचक्षणमवेक्ष्य गताधिरासीत्}
{सम्प्रश्रयप्रणयविह्वलया गिरेषद्}
{व्रीडाविलोकविलसद्धसिताननाह}% २४

\uvacha{सुकन्योवाच}

\fourlineindentedshloka
{राद्धं बत द्विजवृषैतदमोघयोग-}
{मायाधिपे त्वयि विभो तदवैमि भर्तः}
{यस्तेऽभ्यधायि समयः सकृदङ्गसङ्गो}
{भूयाद्गरीयसि गुणः प्रसवः सतीनाम्}% २५

\fourlineindentedshloka
{तत्रेति कृत्यमुपशिक्ष्य यथोपदेशं}
{येनैष कर्शिततमोति रिरंसयात्मा}
{सिध्येत ते कृतमनोभव धर्षिताया}
{दीनस्तदीशभवनं सदृशं विचक्ष्व}% २६

\uvacha{सुमतिरुवाच}

\twolineshloka
{प्रियायाः प्रियमन्विच्छंश्च्यवनो योगमास्थितः}
{विमानं कामगं राजंस्तर्ह्येवाविरचीकरत्}% २७

\twolineshloka
{सर्वकामदुघं रम्यं सर्वरत्नसमन्वितम्}
{सर्वार्थोपचयोदर्कं मणिस्तम्भैरुपस्कृतम्}% २८

\twolineshloka
{दिव्योपस्तरणोपेतं सर्वकालसुखावहम्}
{पट्टिकाभिः पताकाभिर्विचित्राभिरलङ्कृतम्}% २९

\twolineshloka
{स्रग्भिर्विचित्रमालाभिर्मञ्जुसिञ्जत्षडङ्घ्रिभिः}
{दुकूलक्षौमकौशेयैर्नानावस्त्रैर्विराजितम्}% ३०

\twolineshloka
{उपर्युपरि विन्यस्तनिलयेषु पृथक्पृथक्}
{कॢप्तैः कशिपुभिः कान्तं पर्यङ्कव्यजनादिभिः}% ३१

\twolineshloka
{तत्रतत्र विनिक्षिप्त नानाशिल्पोपशोभितम्}
{महामरकतस्थल्या जुष्टं विद्रुमवेदिभिः}% ३२

\twolineshloka
{द्वाःसु विद्रुमदेहल्या भातं वज्रकपाटकम्}
{शिखरेष्विन्द्रनीलेषु हेमकुम्भैरधिश्रितम्}% ३३

\twolineshloka
{चक्षुष्मत्पद्मरागाग्र्यैर्वज्रभित्तिषु निर्मितैः}
{जुष्टं विचित्रवैतानैर्मुक्ताहारावलम्बितैः}% ३४

\twolineshloka
{हंसपारावतव्रातैस्तत्र तत्र निकूजितम्}
{कृत्रिमान्मन्यमानैस्तानधिरुह्याधिरुह्य च}% ३५

\twolineshloka
{विहारस्थानविश्राम संवेश प्राङ्गणाजिरैः}
{यथोपजोषं रचितैर्विस्मापनमिवात्मनः}% ३६

\twolineshloka
{एवं गृहं प्रपश्यन्तीं नातिप्रीतेन चेतसा}
{सर्वभूताशयाभिज्ञः स्वयं प्रोवाच तां प्रति}% ३७

\twolineshloka
{निमज्ज्यास्मिन्ह्रदे भीरु विमानमिदमारुह}
{सुभ्रूर्भर्तुः समादाय वचः कुवलयेक्षणा}% ३८

\twolineshloka
{सरजो बिभ्रती वासो वेणीभूतांश्च मूर्द्धजान्}
{अङ्गं च मलपङ्केन सञ्छन्नं शबलस्तनम्}% ३९

\twolineshloka
{आविवेश सरस्तत्र मुदा शिवजलाशयम्}
{सान्तःसरसि वेश्मस्थाः शतानि दशकन्यकाः}% ४०

\twolineshloka
{सर्वाः किशोरवयसो ददर्शोत्पलगन्धयः}
{तां दृष्ट्वा शीघ्रमुत्थाय प्रोचुः प्राञ्जलयः स्त्रियः}% ४१

\twolineshloka
{वयं कर्मकरीस्तुभ्यं शाधि नः करवाम किम्}
{स्नानेन ता महार्हेण स्नापयित्वा मनस्विनीम्}% ४२

\twolineshloka
{दुकूले निर्मले नूत्ने ददुरस्यै च मानद}
{भूषणानि परार्घ्यानि वरीयांसि द्युमन्ति च}% ४३

\twolineshloka
{अन्नं सर्वगुणोपेतं पानं चैवामृतासवम्}
{अथादर्शे स्वमात्मानं स्रग्विणं विरजोम्बरम्}% ४४

\twolineshloka
{ताभिः कृतस्वस्त्ययनं कन्याभिर्बहुमानितम्}
{हारेण च महार्हेण रुचकेन च भूषितम्}% ४५

\twolineshloka
{निष्कग्रीवं वलयिनं क्वणत्काञ्चननूपुरम्}
{श्रोण्योरध्यस्तया काञ्च्या काञ्चन्या बहुरत्नया}% ४६

\twolineshloka
{सुभ्रुवा सुदता शुक्लस्निग्धापाङ्गेन चक्षुषा}
{पद्मकोशस्पृधा नीलैरलकैश्च लसन्मुखम्}% ४७

\twolineshloka
{यदा सस्मार दयितमृषीणां वल्लभं पतिम्}
{तत्र चास्ते सहस्त्रीभिर्यत्रास्ते स मुनीश्वरः}% ४८

\twolineshloka
{भर्तुः पुरस्तादात्मानं स्त्रीसहस्रवृतं तदा}
{निशाम्य तद्योगगतिं संशयं प्रत्यपद्यत}% ४९

\twolineshloka
{सतां कृत मलस्नानां विभ्राजन्तीमपूर्ववत्}
{आत्मनो बिभ्रतीं रूपं संवीतरुचिरस्तनीम्}% ५०

\twolineshloka
{विद्याधरी सहस्रेण सेव्यमानां सुवाससम्}
{जातभावो विमानं तदारोहयदमित्रहन्}% ५१

\fourlineindentedshloka
{तस्मिन्नलुप्तमहिमा प्रिययानुषक्तो}
{विद्याधरीभिरुपचीर्णवपुर्विमाने}
{बभ्राज उत्कचकुमुद्गणवानपीच्य}
{स्ताराभिरावृत इवोडुपतिर्नभःस्थः}% ५२

\fourlineindentedshloka
{तेनाष्टलोकपविहारकुलाचलेन्द्र-}
{द्रोणीष्वनङ्गसखमारुतसौभगासु}
{सिद्धैर्नुतोद्युधुनिपातशिवस्वनासु}
{रेमे चिरं धनदवल्ललनावरूथी}% ५३

\twolineshloka
{वैश्रम्भके सुरवने नन्दने पुष्पभद्रके}
{मानसे चैत्ररथ्ये च सरे मे रामया रतः}% ५४

{॥इति श्रीपद्मपुराणे पातालखण्डे शेषवात्स्यायनसंवादे रामाश्वमेधे च्यवनस्य तपोभोगवर्णनं नाम पञ्चदशोऽध्यायः॥१५॥}

\dnsub{षोडशोऽध्यायः}%\resetShloka

\uvacha{सुमतिरुवाच}

\twolineshloka
{एवं तया क्रीडमानः सर्वत्र धरणीतले}
{नाबुध्यत गतानब्दाञ्छतसङ्ख्या परीमितान्}% १

\twolineshloka
{ततो ज्ञात्वाथव तद्विप्रः स्वकालपरिवर्तिनीम्}
{मनोरथैश्च सम्पूर्णां स्वस्यप्रियतमां वराम्}% २

\twolineshloka
{न्यवर्तताश्रमं श्रेष्ठं पयोष्णीतीरसंस्थितम्}
{निर्वैरजं तु जनतासङ्कुलं मृगसेवितम्}% ३

\twolineshloka
{तत्रावसत्स सुतपाः शिष्यैर्वेदसमन्वितैः}
{सेविताङ्घ्रियुगो नित्यं तताप परमं तपः}% ४

\twolineshloka
{कदाचिदथ शर्यातिर्यष्टुमैच्छत देवताः}
{तदा च्यवनमानेतुं प्रेषयामास सेवकान्}% ५

\twolineshloka
{तैराहूतो द्विजवरस्तत्रागच्छन्महातपाः}
{सुकन्यया धर्मपत्न्या स्वाचार परिनिष्ठया}% ६

\twolineshloka
{आगतं तं मुनिवरं पत्न्या पुत्र्या महायशाः}
{ददर्श दुहितुः पार्श्वे पुरुषं सूर्यवर्चसम्}% ७

\twolineshloka
{राजा दुहितरं प्राह कृतपादाभिवन्दनाम्}
{आशिषो न प्रयुञ्जानो नातिप्रीतमना इव}% ८

\fourlineindentedshloka
{चिकीर्षितं ते किमिदं पतिस्त्वया}
{प्रलम्भितो लोकनमस्कृतो मुनिः}
{त्वया जराग्रस्तमसम्मतं पतिं}
{विहाय जारं भजसेऽमुमध्वगम्}% ९

\twolineshloka
{कथं मतिस्तेऽवगतान्यथासतां कुलप्रसूतेः कुलदूषणं त्विदम्}
{बिभर्षि जारं यदपत्रपाकुलं पितुः स्वभर्तुश्च नयस्यधस्तमाम्}% १०

\twolineshloka
{एवं ब्रुवाणं पितरं स्मयमाना शुचिस्मिता}
{उवाच तात जामाता तवैष भृगुनन्दनः}% ११

\twolineshloka
{शशंस पित्रे तत्सर्वं वयोरूपाभिलम्भनम्}
{विस्मितः परमप्रीतस्तनयां परिषस्वजे}% १२

\twolineshloka
{सोमेनायाजयद्वीरं ग्रहं सोमस्य चाग्रहीत्}
{असोमपोरप्यश्विनोश्च्यवनः स्वेन तेजसा}% १३

\twolineshloka
{ग्रहं तु ग्राहयामास तपोबलसमन्वितः}
{वज्रं गृहीत्वा शक्रस्तु हन्तुं ब्राह्मणसत्तमम्}% १४

\twolineshloka
{अपङ्क्तिपावनौ देवौ कुर्वाणं पङ्क्तिगोचरौ}
{शक्रं वज्रधरं दृष्ट्वा मुनिः स्वहननोद्यतम्}% १५

\twolineshloka
{हुङ्कारमकरोद्धीमान्स्तम्भयामास तद्भुजम्}
{इन्द्रः स्तब्धभुजस्तत्र दृष्टः सर्वैश्च मानवैः}% १६

\twolineshloka
{कोपेन श्वसमानोऽहिर्यथा मन्त्रनियन्त्रितः}
{तुष्टाव स मुनिं शक्रः स्तब्धबाहुस्तपोनिधिम्}% १७

\twolineshloka
{अश्विभ्यां भागदानं च कुर्वन्तं निर्भयान्तरम्}
{कथयामास भोः स्वामिन्दीयतामश्विनोर्बलि}% १८

\twolineshloka
{मया न वार्यते तात क्षमस्वाघं महत्कृतम्}
{इत्युक्तः स मुनिः कोपं जहौ तूर्णं कृपानिधिः}% १९

\twolineshloka
{इन्द्रो मुक्तभुजोऽप्यासीत्तदानीं पुरुषर्षभ}
{एतद्वीक्ष्य जनाः सर्वे कौतुकाविष्टमानसाः}% २०

\twolineshloka
{शशंसुर्ब्राह्मणबलं ते तु देवादिदुर्ल्लभम्}
{ततो राजा बहुधनं ब्राह्मणेभ्योऽददन्महान्}% २१

\twolineshloka
{चक्रे चावभृथस्नानं यागान्ते शत्रुतापनः}
{त्वया पृष्टं यदाचक्ष्व च्यवनस्य महोदयम्}% २२

\twolineshloka
{स मया कथितः सर्वस्तपोयोगसमन्वितः}
{नमस्कृत्वा तपोमूर्तिमिमं प्राप्य जयाशिषः}% २३

\onelineshloka*
{प्रेषय त्वं सपत्नीकं रामयज्ञे मनोरमे}

\uvacha{शेष उवाच}

\onelineshloka
{एवं तु कुर्वतोर्वार्तां हयः प्रापाश्रमं प्रति}% २४

\twolineshloka
{विदधद्वायुवेगेन पृथ्वीं खुरविलक्षिताम्}
{दूर्वाङ्कुरान्मुखाग्रेण चरंस्तत्र महाश्रमे}% २५

\twolineshloka
{मुनयो यावदादाय दर्भान्स्नातुं गता नदीम्}
{शत्रुघ्नः शत्रुसेनायास्तापनः शूरसम्मतः}% २६

\twolineshloka
{तावत्प्राप मुनेर्वासं च्यवनस्यातिशोभितम्}
{गत्वा तदाश्रमे वीरो ददर्श च्यवनं मुनिम्}% २७

\twolineshloka
{सुकन्यायाः समीपस्थं तपोमूर्तिमिवस्थितम्}
{ववन्दे चरणौ तस्य स्वाभिधां समुदाहरन्}% २८

\onelineshloka
{शत्रुघ्नोहं रघुपतेर्भ्राता वाहस्य पालकः}% २९

\twolineshloka
{नमस्करोमि युष्मभ्यं महापापोपशान्तये}
{इति वाक्यं समाकर्ण्य जगाद मुनिसत्तमः}% ३०

\twolineshloka
{शत्रुघ्न तव कल्याणं भूयान्नरवरर्षभ}
{यज्ञं पालयमानस्य कीर्तिस्ते विपुला भवेत्}% ३१

\twolineshloka
{चित्रं पश्यत भो विप्रा रामोऽपि मखकारकः}
{यन्नामस्मरणादीनि कुर्वन्ति पापनाशनम्}% ३२

\twolineshloka
{महापातकसंयुक्ताः परदाररता नराः}
{यन्नामस्मरणोद्युक्ता मुक्ता यान्ति परां गतिम्}% ३३

\twolineshloka
{पादपद्मसमुत्थेन रेणुना ग्रावमूर्तिभृत्}
{तत्क्षणाद्गौतमार्धाङ्गी जाता मोहनरूपधृक्}% ३४

\twolineshloka
{मामकीयस्य रूपस्य ध्यानेन प्रेमनिर्भरा}
{सर्वपातकराशिं सा दग्ध्वा प्राप्ता सुरूपताम्}% ३५

\twolineshloka
{दैत्या यस्य मनोहारिरूपं प्रधनमण्डले}
{पश्यन्तः प्रापुरेतस्य रूपं विकृतिवर्जितम्}% ३६

\twolineshloka
{योगिनो ध्याननिष्ठा ये यं ध्यात्वा योगमास्थिताः}
{संसारभयनिर्मुक्ताः प्रयाताः परमं पदम्}% ३७

\twolineshloka
{धन्योऽहमद्य रामस्य मुखं द्रक्ष्यामि शोभनम्}
{पयोजदलनेत्रान्तं सुनसं सुभ्रुसून्नतम्}% ३८

\twolineshloka
{सा जिह्वा रघुनाथस्य नामकीर्तनमादरात्}
{करोति विपरीता या फणिनो रसना समा}% ३९

\twolineshloka
{अद्य प्राप्तं तपःपुण्यमद्य पूर्णा मनोरथाः}
{यद्द्रक्ष्ये रामचन्द्रस्य मुखं ब्रह्मादिदुर्ल्लभम्}% ४०

\twolineshloka
{तत्पादरेणुना स्वाङ्गं पवित्रं विदधाम्यहम्}
{विचित्रतरवार्ताभिः पावये रसनां स्वकाम्}% ४१

\fourlineindentedshloka
{इत्यादि रामचरणस्मरणप्रबुद्ध-}
{प्रेमव्रजप्रसृतगद्गदवागुदश्रुः}
{श्रीरामचन्द्र रघुपुङ्गवधर्ममूर्ते}
{भक्तानुकम्पकसमुद्धर संसृतेर्माम्}% ४२

\twolineshloka
{जल्पन्नश्रुकलापूर्णो मुनीनां पुरतस्तदा}
{नाज्ञासीत्तत्र पारक्यं निजं ध्यानेन संस्थितः}% ४३

\twolineshloka
{शत्रुघ्नस्तं मुनिं प्राह स्वामिन्नो मखसत्तमः}
{क्रियतां भवता पादरजसा सुपवित्रितः}% ४४

\twolineshloka
{महद्भाग्यं रघुपतेर्यद्युष्मन्मानसान्तरे}
{तिष्ठत्यसौ महाबाहुः सर्वलोकसुपूजितः}% ४५

\twolineshloka
{इत्युक्तः सपरीवारः सर्वाग्निपरिसंवृतः}
{जगाम च्यवनस्तत्र प्रमोदह्रदसम्प्लुतः}% ४६

\twolineshloka
{हनूमांस्तं पदायान्तं रामभक्तमवेक्ष्य ह}
{शत्रुघ्नं निजगादासौ वचो विनयसंयुतः}% ४७

\twolineshloka
{स्वामिन्कथयसि त्वं चेन्महापुरुषसुन्दरम्}
{रामभक्तं मुनिवरं नयामि स्वपुरीमहम्}% ४८

\twolineshloka
{इति श्रुत्वा महद्वाक्यं कपिवीरस्य शत्रुहा}
{आदिदेश हनूमन्तं गच्छ प्रापयतं मुनिम्}% ४९

\twolineshloka
{हनूमांस्तं मुनिं स्वीये पृष्ठ आरोप्य वेगवान्}
{सकुटुम्बं निनायाशु वायुः ख इव सर्वगः}% ५०

\twolineshloka
{आगतं तं मुनिं दृष्ट्वा रामो मतिमतां वरः}
{अर्घ्यपाद्यादिकं चक्रे प्रीतः प्रणयविह्वलः}% ५१

\twolineshloka
{धन्योऽस्मि मुनिवर्यस्य दर्शनेन तवाधुना}
{पवित्रितो मखो मह्यं सर्वसम्भारसम्भृतः}% ५२

\twolineshloka
{इति वाक्यं समाकर्ण्य च्यवनो मुनिसत्तमः}
{उवाच प्रेमनिर्भिन्न पुलकाङ्गोऽतिनिर्वृतः}% ५३

\twolineshloka
{स्वामिन्ब्रह्मण्यदेवस्य तव वाडवपूजनम्}
{युक्तमेव महाराज धर्ममार्गं प्ररक्षितुः}% ५४

{॥इति श्रीपद्मपुराणे पातालखण्डे शेषवात्स्यायनसंवादे रामाश्वमेधे च्यवनाश्रमे हयगमनो नाम षोडशोऽध्यायः॥१६॥}

\dnsub{सप्तदशोऽध्यायः}%\resetShloka

\uvacha{शेष उवाच}

\twolineshloka
{शत्रुघ्नश्च्यवनस्याथ दृष्ट्वाऽचिन्त्यं तपोबलम्}
{प्रशशंस तपो ब्राह्मं सर्वलोकैकवन्दितम्}% १

\twolineshloka
{अहो पश्यत योगस्य सिद्धिं ब्राह्मणसत्तमे}
{यः क्षणादेव दुष्प्रापं तद्विमानमचीकरत्}% २

\twolineshloka
{क्व भोगसिद्धिर्महती मुनीनाममलात्मनाम्}
{क्व तपोबलहीनानां भोगेच्छा मनुजात्मनाम्}% ३

\twolineshloka
{इति स्वगतमाशंसञ्छत्रुघ्नश्च्यवनाश्रमे}
{क्षणं स्थित्वा जलं पीत्वा सुखसम्भोगमाप्तवान्}% ४

\twolineshloka
{हयस्तस्याः पयोष्ण्याख्या नद्याः पुण्यजलात्मनः}
{पयः पीत्वा ययौ मार्गे वायुवेगगतिर्महान्}% ५

\twolineshloka
{योधास्तन्निर्गमं दृष्ट्वा पृष्ठतोऽनुययुस्तदा}
{हस्तिभिः पत्तिभिः केचिद्रथैः केचन वाजिभिः}% ६

\twolineshloka
{शत्रुघ्नोऽमात्यवर्येण सुमत्याख्येन संयुतः}
{पृष्ठतोऽनुजगामाशु रथेन हयशोभिना}% ७

\twolineshloka
{गच्छन्वाजीपुरं प्राप्तो विमलाख्यस्य भूपतेः}
{रत्नातटाख्यं च जनैर्हृष्टपुष्टैः समाकुलम्}% ८

\twolineshloka
{स सेवकादुपश्रुत्य रघुनाथ हयोत्तमम्}
{पुरोन्तिके हि सम्प्राप्तं सर्वयोधसमन्वितम्}% ९

\twolineshloka
{तदा गजानां सप्तत्या चन्द्रवर्णसमानया}
{अश्वानामयुतैः सार्धं रथानां काञ्चनत्विषाम्}% १०

\twolineshloka
{सहस्रेण च संयुक्तः शत्रुघ्नं प्रति जग्मिवान्}
{शत्रुघ्नं स नमस्कृत्य सर्वान्प्राप्तान्महारथान्}% ११

\twolineshloka
{वसुकोशं धनं सर्वं राज्यं तस्मै निवेद्य च}
{किं करोमीति राजा तं जगाद पुरतः स्थितः}% १२

\fourlineindentedshloka
{राजापि तं स्वीयपदे प्रणम्रं}
{दोर्भ्यां दृढं सम्परिषस्वजे महान्}
{जगाम साकं तनये स्वराज्यं}
{निक्षिप्य सर्वं बहुधन्विभिर्वृतः}% १३

\twolineshloka
{रामचन्द्राभिधां श्रुत्वा सर्वश्रुतिमनोहराम्}
{सर्वे प्रणम्य तं वाहं ददुर्वसुमहाधनम्}% १४

\twolineshloka
{राजानं पूजयित्वा तु शत्रुघ्नः परया मुदा}
{सेनया सहितोऽगच्छद्वाजिनः पृष्ठतस्तदा}% १५

\twolineshloka
{एवं स गच्छंस्तन्मार्गे पर्वताग्र्यं ददर्श ह}
{स्फाटिकैः कानकै रौप्यै राजितं प्रस्थराजिभिः}% १६

\twolineshloka
{जलनिर्झरसंह्रादं नानाधातुकभूतलम्}
{गैरिकादिकसद्धातु लाक्षारङ्गविराजितम्}% १७

\twolineshloka
{यत्र सिद्धाङ्गनाः सिद्धैः सङ्क्रीडन्त्यकुतोभयाः}
{गन्धर्वाप्सरसो नागा यत्र क्रीडन्ति लीलया}% १८

\twolineshloka
{गङ्गातरङ्गसंस्पर्श शीतवायुनिषेवितम्}
{वीणारणद्धंसशुकक्वणसुन्दरशोभितम्}% १९

\twolineshloka
{पर्वतं वीक्ष्य शत्रुघ्न उवाच सुमतिं त्विदम्}
{तद्दर्शनसमुद्भूत विस्मयाविष्टमानसः}% २०

\twolineshloka
{कोऽयं महागिरिवरो विस्मापयति मे मनः}
{महारजतसत्प्रस्थो मार्गे राजति मेऽद्भुतः}% २१

\twolineshloka
{अत्र किं देवतावासो देवानां क्रीडनस्थलम्}
{यदेतन्मनसः क्षोभं करोति श्रीसमुच्चयैः}% २२

\twolineshloka
{इति वाक्यं समाकर्ण्य जगाद सुमतिस्तदा}
{वक्ष्यमाणगुणागार रामचन्द्र पदाब्जधीः}% २३

\twolineshloka
{नीलोऽयं पर्वतो राजन्पुरतो भाति भूमिप}
{मनोहरैर्महाशृङ्गैः स्फाटिकाग्रैः समन्ततः}% २४

\twolineshloka
{एनं पश्यन्ति नो पापाः परदाररता नराः}
{विष्णोर्गुणगणान्ये वै न मन्यन्ते नराधमाः}% २५

\twolineshloka
{श्रुतिस्मृतिसमुत्थं ये धर्मं सद्भिः सुसाधितम्}
{न मन्यन्ते स्वबुद्धिस्थ हेतुवादविचारणाः}% २६

\twolineshloka
{नीलीविक्रयकर्तारो लाक्षाविक्रयकारकाः}
{यो ब्राह्मणो घृतादीनि विक्रीणाति सुरापकः}% २७

\twolineshloka
{कन्यां रूपेण सम्पन्नां न दद्यात्कुलशीलिने}
{विक्रीणाति द्रव्यलोभात्पिता पापेन मोहितः}% २८

\twolineshloka
{पत्नीं दूषयते यस्तु कुलशीलवतीं नरः}
{स्वयमेवात्ति मधुरं बन्धुभ्यो न ददाति यः}% २९

\twolineshloka
{भोजने ब्राह्मणार्थे च पाकभेदं करोति यः}
{कृसरं पायसं वापि नार्थिनं दापयेत्कुधीः}% ३०

\twolineshloka
{अतिथीनवमन्यन्ते सूर्यतापादितापितान्}
{अन्तरिक्षभुजो ये च ये च विश्वासघातकाः}% ३१

\twolineshloka
{न पश्यन्ति महाराज रघुनाथ पराङ्मुखाः}
{असौ पुण्यो गिरिवरः पुरुषोत्तम शोभितः}% ३२

\twolineshloka
{पवित्रयति सर्वान्नो दर्शनेन मनोहरः}
{अत्र तिष्ठति देवानां मुकुटैरर्चिताङ्घ्रिकः}% ३३

\twolineshloka
{पुण्यवद्भिर्दर्शनार्हः पुण्यदः पुरुषोत्तमः}
{श्रुतयो नेतिनेतीति ब्रुवाणा न विदन्ति यम्}% ३४

\twolineshloka
{यत्पादरज इन्द्रादिदेवैर्मृग्यं सुदुर्ल्लभम्}
{वेदान्तादिभिरन्यूनैर्वाक्यैर्विदन्ति यं बुधाः}% ३५

\twolineshloka
{सोऽत्र श्रीमान्नीलशैले वसते पुरुषोत्तमः}
{आरुह्य तं नमस्कृत्य सम्पूज्य सुकृतादिना}% ३६

\twolineshloka
{नैवेद्यं भक्षयित्वा वै भूप भूयाच्चतुर्भुजः}
{अत्राप्युदाहरन्तीममितिहासं पुरातनम्}% ३७

\twolineshloka
{तं शृणुष्व महाराज सर्वाश्चर्यसमन्वितम्}
{रत्नग्रीवस्य नृपतेर्यद्वृत्तं सकुटुम्बिनः}% ३८

\twolineshloka
{चतुर्भुजादिकं प्राप्तं देवदानवदुर्लभम्}
{आसीत्काञ्ची महाराज पुरी लोकेषु विश्रुता}% ३९

\twolineshloka
{महाजनपरीवारसमृद्धबलवाहना}
{यस्यां वसन्ति विप्राग्र्याः षट्कर्मनिरता भृशम्}% ४०

\twolineshloka
{सर्वभूतहिते युक्ता रामभक्तिषु लालसाः}
{क्षत्रिया रणकर्तारः सङ्ग्रामेऽप्यपलायिनः}% ४१

\twolineshloka
{परदार परद्रव्य परद्रोहपराङ्मुखाः}
{वैश्याः कुसीदकृष्यादिवाणिज्यशुभवृत्तयः}% ४२

\twolineshloka
{कुर्वन्ति रघुनाथस्य पदाम्भोजे रतिं सदा}
{शूद्रा ब्राह्मणसेवाभिर्गतरात्रिदिनान्तराः}% ४३

\twolineshloka
{कुर्वन्ति कथनं रामरामेति रसनाग्रतः}
{प्राकृताः केऽपि नो पापं कुर्वन्ति मनसात्र वै}% ४४

\twolineshloka
{दानं दया दमः सत्यं तत्र तिष्ठन्ति नित्यशः}
{वदते न पराबाधं वाक्यं कोऽपि नरोऽनघः}% ४५

\twolineshloka
{न पारक्ये धने लोभं कुर्वन्ति न हि पातकम्}
{एवं प्रजा महाराज रत्नग्रीवेण पाल्यते}% ४६

\twolineshloka
{षष्ठांशं तत्र गृह्णाति नान्यं लोभविवर्जितः}
{एवं पालयमानस्य प्रजाधर्मेण भूपतेः}% ४७

\twolineshloka
{गतानि बहुवर्षाणि सर्वभोगविलासिनः}
{विशालाक्षीं महाराज एकदा ह्यूचिवानिदम्}% ४८

\twolineshloka
{पतिव्रतां धर्मपत्नीं पतिव्रतपरायणाम्}
{पुत्रा जाता विशालाक्षि प्रजारक्षा धुरन्धराः}% ४९

\twolineshloka
{परीवारो महान्मह्यं वर्तते विगतज्वरः}
{हस्तिनो मम शैलाभा वाजिनः पवनोपमाः}% ५०

\twolineshloka
{रथाश्च सुहयैर्युक्ता वर्तन्ते मम नित्यशः}
{महाविष्णुप्रसादेन किञ्चिन्न्यूनं ममास्ति न}% ५१

\twolineshloka
{एवं मनोरथस्त्वेकस्तिष्ठते मानसे मम}
{परं तीर्थं मया नाद्य कृतं परमशोभने}% ५२

\twolineshloka
{गर्भवासविरामाय क्षमं गोविन्दशोभितम्}
{वृद्धो जातोऽस्म्यहं तावद्वलीपलितदेहवान्}% ५३

\twolineshloka
{करिष्यामि मनोहारि तीर्थसेवनमादृतः}
{यो नरो जन्मपर्यन्तं स्वोदरस्य प्रपूरकः}% ५४

\twolineshloka
{न करोति हरेः पूजां स नरो गोवृषः स्मृतः}
{तस्माद्गच्छामि भो भद्रे तीर्थयात्रां प्रति प्रिये}% ५५

\twolineshloka
{सकुटुम्बः सुते न्यस्य धुरं राज्यस्य निर्भृताम्}
{इति व्यवस्य सन्ध्यायां हरिं ध्यायन्निशान्तरे}% ५६

\twolineshloka
{अद्राक्षीत्स्वप्नमप्येकं ब्राह्मणं तापसं वरम्}
{प्रातरुत्थाय राजासौ कृत्वा सन्ध्यादिकाः क्रियाः}% ५७

\twolineshloka
{सभां मन्त्रिजनैः सार्द्धं सुखमासेदिवान्महान्}
{तावद्विप्रं ददर्शाथ तापसं कृशदेहिनम्}% ५८

\twolineshloka
{जटावल्कलकौपीनधारिणं दण्डपाणिनम्}
{अनेकतीर्थसेवाभिः कृतपुण्यकलेवरम्}% ५९

\twolineshloka
{राजा तं वीक्ष्य शिरसा प्रणनाम महाभुजः}
{अर्घ्यपाद्यादिकं चक्रे प्रहृष्टात्मा महीपतिः}% ६०

\twolineshloka
{सुखोपविष्टं विश्रान्तं पप्रच्छ विदितं द्विजम्}
{स्वामिंस्त्वद्दर्शनान्मेऽद्य गतं देहस्य पातकम्}% ६१

\twolineshloka
{महान्तः कृपणान्पातुं यान्ति तद्गेहमादरात्}
{तस्मात्कथय भो विप्र वृद्धस्य मम सम्प्रति}% ६२

\twolineshloka
{को देवो गर्भनाशाय किं तीर्थं च क्षमं भवेत्}
{यूयं सर्वगताः श्रेष्ठाः समाधिध्यानतत्पराः}% ६३

\twolineshloka
{सर्वतीर्थावगाहेन कृतपुण्यात्मनोऽमलाः}
{यथावच्छृण्वते मह्यं श्रद्दधानाय विस्तरात्}% ६४

\onelineshloka*
{कथयस्व प्रसादेन सर्वतीर्थविचक्षण}

\uvacha{ब्राह्मण उवाच}

\onelineshloka
{शृणु राजेन्द्र वक्ष्यामि यत्पृष्टं तीर्थसेवनम्}% ६५

\twolineshloka
{कस्य देवस्य कृपया गर्भनिर्वारणं भवेत्}
{सेव्यः श्रीरामचन्द्रोऽसौ संसारज्वरनाशकः}% ६६

\twolineshloka
{पूज्यः स एव भगवान्पुरुषोत्तमसंज्ञकः}
{नाना पुर्यो मया दृष्टाः सर्वपापक्षयङ्कराः}% ६७

\twolineshloka
{अयोध्या सरयूस्तापी तथा द्वारं हरेः परम्}
{अवन्ती विमला काञ्ची रेवा सागरगामिनी}% ६८

\twolineshloka
{गोकर्णं हाटकाख्यं च हत्याकोटिविनाशनम्}
{मल्लिकाख्यो महाशैलो मोक्षदः पश्यतां नृणाम्}% ६९

\twolineshloka
{यत्राङ्गेषु नृणां तोयं श्यामं वा निर्मलं भवेत्}
{पातकस्यापहारीदं मया दृष्टं तु तीर्थकम्}% ७०

\twolineshloka
{मया द्वारवती दृष्टा सुरासुर निषेविता}
{गोमती यत्र वहति साक्षाद्ब्रह्मजला शुभा}% ७१

\twolineshloka
{यत्र स्वापो लयः प्रोक्तो मृतिर्मोक्ष इति श्रुतिः}
{यस्यां संवसतां नॄणां न कलि प्रभवेत्क्वचित्}% ७२

\twolineshloka
{चक्राङ्का यत्र पाषाणा मानवा अपि चक्रिणः}
{पशवः कीटपक्ष्याद्याः सर्वे चक्रशरीरिणः}% ७३

\twolineshloka
{त्रिविक्रमो वसेद्यस्यां सर्वलोकैकपालकः}
{सा पुरी तु महापुण्यैर्मया दृग्गोचरीकृता}% ७४

\twolineshloka
{कुरुक्षेत्रं मया दृष्टं सर्वहत्यापनोदनम्}
{स्यमन्तपञ्चकं यत्र महापातकनाशनम्}% ७५

\twolineshloka
{वाराणसी मया दृष्टा विश्वनाथकृतालया}
{यत्रोपदिशते मन्त्रं तारकं ब्रह्मसंज्ञितम्}% ७६

\fourlineindentedshloka
{यस्यां मृताः कीटपतङ्गभृङ्गाः}
{पश्वादयो वा सुरयोनयो वा}
{स्वकर्मसम्भोगसुखं विहाय}
{गच्छन्ति कैलासमतीतदुःखाः}% ७७

\twolineshloka
{मणिकर्णिर्यत्र तीर्थं यस्यामुत्तरवाहिनी}
{करोति संसृतेर्बन्धच्छेदं पापकृतामपि}% ७८

\twolineshloka
{कपर्दिनः कुण्डलिनः सर्पभूषाधरावराः}
{गजचर्मपरीधाना वसन्ति गतदुःखकाः}% ७९

\twolineshloka
{कालभैरवनामात्र करोति यमशासनम्}
{न करोति नृणां वार्तां यमो दण्डधरः प्रभुः}% ८०

\twolineshloka
{एतादृशी मया दृष्टा काशी विश्वेश्वराङ्किता}
{अनेकान्यपि तीर्थानि मया दृष्टानि भूमिप}% ८१

\twolineshloka
{परमेकं महच्चित्रं यद्दृष्टं नीलपर्वते}
{पुरुषोत्तमसान्निध्ये तन्न क्वाप्यक्षिगोचरम्}% ८२

{॥इति श्रीपद्मपुराणे पातालखण्डे शेषवात्स्यायनसंवादे रामाश्वमेधे ब्राह्मणसमागमो नाम सप्तदशोऽध्यायः॥१७॥}

\dnsub{अष्टादशोऽध्यायः}%\resetShloka

\uvacha{ब्राह्मण उवाच}

\twolineshloka
{राजंस्त्वं शृणु यद्वृत्तं नीले पर्वतसत्तमे}
{यच्छ्रद्दधानाः पुरुषा यान्ति ब्रह्म सनातनम्}% १

\twolineshloka
{मया पर्यटता तत्र गतं नीलाभिधे गिरौ}
{गङ्गासागरतोयेन क्षालितप्राङ्गणे मुहुः}% २

\twolineshloka
{तत्र भिल्ला मया दृष्टाः पर्वताग्रे धनुर्भृतः}
{चतुर्भुजा मूलफलैर्भक्ष्यैर्निर्वाहितक्लमाः}% ३

\twolineshloka
{तदा मे मनसि क्षिप्रं संशयः सुमहानभूत्}
{चतुर्भुजाः किमेते वै धनुर्बाणधरा नराः}% ४

\twolineshloka
{वैकुण्ठवासिनां रूपं दृश्यते विजितात्मनाम्}
{कथमेतैरुपालब्धं ब्रह्माद्यैरपि दुर्ल्लभम्}% ५

\twolineshloka
{शङ्खचक्रगदाशार्ङ्गपद्मोल्लसितपाणयः}
{वनमालापरीताङ्गा विष्णुभक्ता इवान्तिके}% ६

\twolineshloka
{संशयाविष्टचित्तेन मया पृष्टं तदा नृप}
{यूयं के बत युष्माभिर्लब्धं चातुर्भुजं कथम्}% ७

\twolineshloka
{तदा तैर्बहु हास्यं तु कृत्वा मां प्रतिभाषितम्}
{ब्राह्मणोऽयं न जानाति पिण्डमाहात्म्यमद्भुतम्}% ८

\twolineshloka
{इति श्रुत्वाऽवदं चाहं कः पिण्डः कस्य दीयते}
{तन्मम ब्रूत धर्मिष्ठाश्चतुर्भुजशरीरिणः}% ९

\twolineshloka
{तदा मद्वाक्यमाकर्ण्य कथितं तैर्महात्मभिः}
{सर्वं तत्र तु यद्वृत्तं चतुर्भुजभवादिकम्}% १०

\uvacha{किराता ऊचुः}

\twolineshloka
{शृणु ब्राह्मण वृत्तान्तमस्माकं पृथुकः शिशुः}
{नित्यं जम्बूफलादीनि भक्षयन्क्रीडया चरन्}% ११

\twolineshloka
{एकदा रममाणस्तु गिरिशृङ्गं मनोरमम्}
{समारुरोह शिशुभिः समन्तात्परिवारितः}% १२

\twolineshloka
{तदा तत्र ददर्शाहं देवायतनमद्भुतम्}
{गारुत्मतादिमणिभिः खचितं स्वर्णभित्तिकम्}% १३

\twolineshloka
{स्वकान्त्यातिमिरश्रेणीं दारयद्रविवद्भृशम्}
{दृष्ट्वा विस्मयमापेदे किमिदं कस्य वै गृहम्}% १४

\twolineshloka
{गत्वा विलोकयामीति किमिदं महतां पदम्}
{इति सञ्चिन्त्य गेहान्तर्जगाम बहुभाग्यतः}% १५

\twolineshloka
{ददर्श तत्र देवेशं सुरासुरनमस्कृतम्}
{किरीटहारकेयूरग्रैवेयाद्यैर्विराजितम्}% १६

\twolineshloka
{मनोहरावतंसौ च धारयन्तं सुनिर्मलौ}
{पादपद्मे तुलसिका गन्धमत्तषडङ्घ्रिके}% १७

\twolineshloka
{शङ्खचक्रगदाशार्ङ्ग पद्माद्यैर्मूर्तिसंयुतैः}
{उपासिताङ्घ्रिं श्रीमूर्तिं नारदाद्यैः सुसेवितम्}% १८

\twolineshloka
{केचिद्गायन्ति नृत्यन्ति हसन्ति परमाद्भुतम्}
{प्रीणयन्ति महाराजं सर्वलोकैकवन्दितम्}% १९

\twolineshloka
{हरिं वीक्ष्य मदीयोर्भस्तत्र सञ्जग्मिवान्मुने}
{देवास्तत्र विधायोच्चैः पूजां धूपादिसंयताम्}% २०

\twolineshloka
{नैवेद्यं श्रीप्रियस्यार्थे कृत्वा नीराजनं ततः}
{जग्मुः स्वं स्वं गृहं राजन्कृपां पश्यन्त आदरात्}% २१

\twolineshloka
{महाभाग्यवशात्तेन प्राप्तं नैवेद्यसिक्थकम्}
{पतितं ब्रह्मदेवाद्यैर्दुर्ल्लभं सुरमानुषैः}% २२

\twolineshloka
{तद्भक्षणं च कृत्वाथो श्रीमूर्तिमवलोक्य च}
{चतुर्भुजत्वमाप्तं वै पृथुकेन सुशोभिना}% २३

\twolineshloka
{तदास्माभिर्गृहं प्राप्तो बालको वीक्षितो मुहुः}
{चतुर्भुजत्वं सम्प्राप्तः शङ्खचक्रादिधारकः}% २४

\twolineshloka
{अस्माभिः पृष्टमेतस्य किमेतज्जातमद्भुतम्}
{तदा प्रोवाच नः सर्वान्बालकः परमाद्भुतम्}% २५

\twolineshloka
{शिखराग्रे गतः पूर्वं तत्र दृष्टः सुरेश्वरः}
{तत्र नैवेद्यसिक्थं तु मया प्राप्तं मनोहरम्}% २६

\twolineshloka
{तस्य भक्षणमात्रेण कारणेन तु साम्प्रतम्}
{चतुर्भुजत्वं सम्प्राप्तो विस्मयेन समन्वितः}% २७

\twolineshloka
{तच्छ्रुत्वा तु वचस्तस्य सद्यः सम्प्राप्तविस्मयैः}
{अस्माभिरप्यसौ दृष्टो देवः परमदुर्ल्लभः}% २८



{अन्नादिकं तत्र भुक्तं सर्वस्वादसमन्वितम्}
{वयं चतुर्भुजा जाता देवस्य कृपया पुनः}
{गत्वा त्वमपि देवस्य दर्शनं कुरु सत्तम}% २९

\twolineshloka
{भुक्त्वा तत्रान्नसिक्थं तु भव विप्र चतुर्भुजः}
{त्वया पृष्टं यदाश्चर्यं तदुक्तं वाडवर्षभ}% ३०

{॥इति श्रीपद्मपुराणे पातालखण्डे शेषवात्स्यायनसंवादे रामाश्वमेधे ब्राह्मणोपदेशोनामाष्टादशोऽध्यायः॥१८॥}

\dnsub{ऊनविंशोऽध्यायः}%\resetShloka

\uvacha{ब्राह्मण उवाच}

\twolineshloka
{इति श्रुत्वा तु तद्वाक्यं भिल्लानामहमद्भुतम्}
{अत्याश्चर्यमिदं मत्वा प्रहृष्टोऽभवमित्युत}% १

\twolineshloka
{गङ्गासागरसंयोगे स्नात्वा पुण्यकलेवरः}
{शृङ्गमारुरुहे तत्र मणिमाणिक्यचित्रितम्}% २

\twolineshloka
{तत्रापश्यं महाराज देवं देवादिवन्दितम्}
{नमस्कृत्वा कृतार्थोऽहं जातोन्नप्राशनेन च}% ३

\twolineshloka
{चतुर्भुजत्वं सम्प्राप्तः शङ्खचक्रादिचिह्नितम्}
{पुरुषोत्तमदर्शनेन न पुनर्गर्भमाविशम्}% ४

\twolineshloka
{राजंस्त्वमपि तत्राशु गच्छ नीलाभिधं गिरिम्}
{कृतार्थं कुरु चात्मानं गर्भदुःखविवर्जितम्}% ५

\twolineshloka
{इत्याकर्ण्य वचस्तस्य वाडवाग्र्यस्य धीमतः}
{पप्रच्छ हृष्टगात्रस्तु तीर्थयात्राविधिं मुनिम्}% ६

\uvacha{राजोवाच}

\twolineshloka
{साधु विप्राग्र्य हे साधो त्वया प्रोक्तं ममानघ}
{पुरुषोत्तममाहात्म्यं शृण्वतां पापनाशनम्}% ७

\twolineshloka
{ब्रूहि तत्तीर्थयात्रायां विधिं श्रुतिसमन्वितम्}
{विधिना केन सम्पूर्ण फलप्राप्तिर्नृणां भवेत्}% ८

\uvacha{ब्राह्मण उवाच}

\twolineshloka
{शृणु राजन्प्रवक्ष्यामि तीर्थयात्राविधिं शुभम्}
{येन सम्प्राप्यते देवः सुरासुरनमस्कृतः}% ९

\twolineshloka
{वलीपलितदेहो वा यौवनेनान्वितोऽपि वा}
{ज्ञात्वा मृत्युमनिस्तीर्यं हरिं शरणमाव्रजेत्}% १०

\twolineshloka
{तत्कीर्तने तच्छ्रवणे वन्दने तस्य पूजने}
{मतिरेव प्रकर्तव्या नान्यत्र वनितादिषु}% ११

\twolineshloka
{सर्वं नश्वरमालोक्य क्षणस्थायि सुदुःखदम्}
{जन्ममृत्युजरातीतं भक्तिवल्लभमच्युतम्}% १२

\twolineshloka
{क्रोधात्कामाद्भयाद्द्वेषाल्लोभाद्दम्भान्नरः पुनः}
{यथाकथञ्चिद्विभजन्न स दुःखं समश्नुते}% १३

\twolineshloka
{स हरिर्जायते साधुसङ्गमात्पापवर्जितात्}
{येषां कृपातः पुरुषा भवन्त्यसुखवर्जिताः}% १४

\twolineshloka
{ते साधवः शान्तरागाः कामलोभविवर्जिताः}
{ब्रुवन्ति यन्महाराज तत्संसारनिवर्तकम्}% १५

\twolineshloka
{तीर्थेषु लभ्यते साधू रामचन्द्र परायणः}
{यद्दर्शनं नृणां पापराशिदाहाशुशुक्षणिः}% १६

\twolineshloka
{तस्मात्तीर्थेषु गन्तव्यं नरैः संसारभीरुभिः}
{पुण्योदकेषु सततं साधुश्रेणिविराजिषु}% १७

\twolineshloka
{तानि तीर्थानि विधिना दृष्टानि प्रहरन्त्यघम्}
{तं विधिं नृपशार्दूल कुरुष्व श्रुतिगोचरम्}% १८

\twolineshloka
{विरागं जनयेत्पूर्वं कलत्रादि कुटुम्बके}
{असत्यभूतं तज्ज्ञात्वा हरिं तु मनसा स्मरेत्}% १९

\twolineshloka
{क्रोशमात्रं ततो गत्वा रामरामेति च ब्रुवन्}
{तत्र तीर्थादिषु स्नात्वा क्षौरं कुर्याद्विधानवित्}% २०

\twolineshloka
{मनुष्याणां च पापानि तीर्थानि प्रति गच्छताम्}
{केशानाश्रित्य तिष्ठन्ति तस्माद्वपनमाचरेत्}% २१

\twolineshloka
{ततो दण्डं तु निर्ग्रन्थिं कमण्डलुमथाजिनम्}
{बिभृयाल्लोभनिर्मुक्तस्तीर्थवेषधरो नरः}% २२

\twolineshloka
{विधिना गच्छतां नॄणां फलावाप्तिर्विशेषतः}
{तस्मात्सर्वप्रयत्नेन तीर्थयात्राविधिं चरेत्}% २३

\twolineshloka
{यस्य हस्तौ च पादौ च मनश्चैव सुसंहितम्}
{विद्या तपश्च कीर्तिश्च स तीर्थफलमश्नुते}% २४

\twolineshloka
{हरेकृष्ण हरेकृष्ण भक्तवत्सल गोपते}
{शरण्य भगवन्विष्णो मां पाहि बहुसंसृतेः}% २५

\twolineshloka
{इति ब्रुवन्रसनया मनसा च हरिं स्मरन्}
{पादचारी गतिं कुर्यात्तीर्थं प्रति महोदयः}% २६

\twolineshloka
{यानेन गच्छन्पुरुषः समभागफलं लभेत्}
{उपानद्भ्यां चतुर्थांशं गोयाने गोवधादिकम्}% २७

\twolineshloka
{व्यवहर्ता तृतीयांशं सेवयाष्टमभागभाक्}
{अनिच्छया व्रजंस्तत्र तीर्थमर्धफलं लभेत्}% २८

\twolineshloka
{यथायथं प्रकर्तव्या तीर्थानामभियात्रिका}
{पापक्षयो भवत्येव विधिदृष्ट्या विशेषतः}% २९

\twolineshloka
{तत्र साधून्नमस्कुर्यात्पादवन्दनसेवनैः}
{तद्द्वारा हरिभक्तिर्हि प्राप्यते पुरुषोत्तमे}% ३०

\twolineshloka
{इति तीर्थविधिः प्रोक्तः समासेन न विस्तरात्}
{एवं विधिं समाश्रित्य गच्छ त्वं पुरुषोत्तमम्}% ३१

\twolineshloka
{तुभ्यं तुष्टो महाराज दास्यते भक्तिमच्युतः}
{यथा संसारनिर्वाहः क्षणादेव भविष्यति}% ३२

\twolineshloka
{तीर्थयात्राविधिं श्रुत्वा सर्वपातकनाशनम्}
{मुच्यते सर्वपापेभ्य उग्रेभ्यः पुरुषर्षभ}% ३३

\uvacha{सुमतिरुवाच}

\twolineshloka
{इति वाक्यं समाकर्ण्य ववन्दे चरणौ महान्}
{तत्तीर्थदर्शनौत्सुक्य विह्वलीकृतमानसः}% ३४

\twolineshloka
{आदिदेश निजामात्यं मन्त्रवित्तममुत्तमम्}
{तीर्थयात्रेच्छया सर्वान्सह नेतुं मनो दधत्}% ३५

\twolineshloka
{मन्त्रिन्पौरजनान्सर्वानादिश त्वं ममाज्ञया}
{पुरुषोत्तमपादाब्जदर्शनप्रीतिहेतवे}% ३६

\twolineshloka
{ये मदीये पुरे लोका ये च मद्वाक्यकारकाः}
{सर्वे निर्यान्तु मत्पुर्या मया सह नरोत्तमाः}% ३७

\twolineshloka
{ये तु मद्वाक्यमुल्लङ्घ्य स्थास्यन्ति पुरुषा गृहे}
{ते दण्ड्या यमदण्डेन पापिनोऽधर्महेतवः}% ३८

\twolineshloka
{किं तेन सुतवृन्देन बान्धवैः किं सुदुर्नयैः}
{यैर्नदृष्टः स्वचक्षुर्भ्यां पुण्यदः पुरुषोत्तमः}% ३९

\twolineshloka
{सूकरीयूथवत्तेषां प्रसूतिर्विट्प्रभक्षिका}
{येषां पुत्राश्च पौत्रा वा हरिं न शरणं गताः}% ४०

\twolineshloka
{यो देवो नाममात्रेण सर्वान्पावयितुं क्षमः}
{तं नमस्कुरुत क्षिप्रं मदीयाः प्रकृतिव्रजाः}% ४१

\twolineshloka
{इति वाक्यं मनोहारि भगवद्गुणगुम्फितम्}
{प्रजहर्ष महामात्य उत्तमः सत्यनामधृक्}% ४२

\twolineshloka
{हस्तिनं वरमारोप्य पटहेन व्यघोषयत्}
{यदादिष्टं नृपेणेह तीर्थयात्रां समिच्छता}% ४३

\twolineshloka
{गच्छन्तु त्वरिता लोका राज्ञा सह महागिरिम्}
{दृश्यतां पापसंहारी पुरुषोत्तमनामधृक्}% ४४

\twolineshloka
{क्रियतां सर्वसंसारसागरो गोष्पदं पुनः}
{भूष्यतां शङ्खचक्रादिचिह्नैः स्वस्व तनुर्नरैः}% ४५

\twolineshloka
{इत्यादिघोषयामास राज्ञादिष्टं यदद्भुतम्}
{सचिवो रघुनाथाङ्घ्रि ध्याननिर्वारितश्रमः}% ४६

\twolineshloka
{तच्छ्रुत्वा ताः प्रजाः सर्वा आनन्दरससम्प्लुताः}
{मनो दधुः स्वनिस्तारे पुरुषोत्तमदर्शनात्}% ४७

\twolineshloka
{निर्ययुर्ब्राह्मणास्तत्र शिष्यैः सह सुवेषिणः}
{आशिषं वरदानाढ्यां ददतो भूमिपं प्रति}% ४८

\twolineshloka
{क्षत्त्रिया धन्विनो वीरा वैश्या वस्तुक्रयाञ्चिताः}
{शूद्राः संसारनिस्तारहर्षित स्वीयविग्रहाः}% ४९

\twolineshloka
{रजकाश्चर्मकाः क्षौद्राः किराता भित्तिकारकाः}
{सूचीवृत्त्या च जीवन्तस्ताम्बूलक्रयकारकाः}% ५०

\twolineshloka
{तालवाद्यधरा ये च ये च रङ्गोपजीविनः}
{तैलविक्रयिणश्चैव वस्त्रविक्रयिणस्तथा}% ५१

\twolineshloka
{सूता वदन्तः पौराणीं वार्तां हर्षसमन्विताः}
{मागधा बन्दिनस्तत्र निर्गता भूमिपाज्ञया}% ५२

\twolineshloka
{भिषग्वृत्त्या च जीवन्तस्तथा पाशककोविदाः}
{पाकस्वादुरसाभिज्ञा हास्यवाक्यानुरञ्जकाः}% ५३

\twolineshloka
{ऐन्द्रजालिकविद्याध्रास्तथा वार्तासुकोविदाः}
{प्रशंसन्तो महाराजं निर्ययुः पुरमध्यतः}% ५४

\twolineshloka
{राजापि तत्र निर्वर्त्य प्रातःसन्ध्यादिकाः क्रियाः}
{ब्राह्मणं तापसश्रेष्ठमानिनाय सुनिर्मलम्}% ५५

\twolineshloka
{तदाज्ञया महाराजो निर्जगाम पुराद्बहिः}
{लोकैरनुगतो राजा बभौ चन्द्र इवोडुभिः}% ५६

\twolineshloka
{क्रोशमात्रं स गत्वाथ क्षौरं कृत्वा विधानतः}
{दण्डं कमण्डलुं बिभ्रन्मृगचर्म तथा शुभम्}% ५७

\twolineshloka
{शुभवेषेण संयुक्तो हरिध्यानपरायणः}
{कामक्रोधादिरहितं मनो बिभ्रन्महायशाः}% ५८

\twolineshloka
{तदा दुन्दुभयो भेर्य आनकाः पणवास्तथा}
{शङ्खवीणादिकाश्चैवाध्मातास्तद्वादकैर्मुहुः}% ५९

\twolineshloka
{जय देवेश दुःखघ्न पुरुषोत्तमसंज्ञित}
{दर्शयस्व तनुं मह्यं वदन्तो निर्ययुर्जनाः}% ६०

{॥इति श्रीपद्मपुराणे पातालखण्डे शेषवात्स्यायनसंवादे रामाश्वमेधे रत्नग्रीवस्य तीर्थप्रयाणन्नामैकोनविंशोऽध्यायः॥१९॥}

\dnsub{विंशोऽध्यायः}%\resetShloka

\uvacha{सुमतिरुवाच}

\twolineshloka
{अथ प्रयाते भूपाले सर्वलोकसमन्विते}
{महाभागैर्वैष्णवैश्च गायकैः कृष्णकीर्तनम्}% १

\twolineshloka
{शुश्रावासौ महाराजो मार्गे गोविन्दकीर्तनम्}
{जय माधव भक्तानां शरण्य पुरुषोत्तम}% २

\twolineshloka
{मार्गे तीर्थान्यनेकानि कुर्वन्पश्यन्महोदयम्}
{तापसब्राह्मणात्तेषां महिमानमथा शृणोत्}% ३

\twolineshloka
{विचित्रविष्णुवार्ताभिर्विनोदितमना नृपः}
{मार्गेमार्गे महाविष्णुं गापयामास गायकान्}% ४

\twolineshloka
{दीनान्धकृपणानां च पङ्गूनां वासनोचितम्}
{दानं ददौ महाराजो बुद्धिमान्विजितेन्द्रियः}% ५

\twolineshloka
{अनेकतीर्थविरजमात्मानं भव्यतां गतम्}
{कुर्वन्ययौ स्वीयलोकैर्हरिध्यानपरायणः}% ६

\twolineshloka
{नृपो गच्छन्ददर्शाग्रे नदीं पापप्रणाशिनीम्}
{चक्राङ्कितग्रावयुतां मुनिमानस निर्मलाम्}% ७

\twolineshloka
{अनेकमुनिवृन्दानां बहुश्रेणिविराजिताम्}
{सारसादिपतत्रीणां कूजितैरुपशोभिताम्}% ८

\twolineshloka
{दृष्ट्वा पप्रच्छ विप्राग्र्यं तापसं धर्मकोविदम्}
{अनेकतीर्थमाहात्म्य विशेषज्ञानजृम्भितम्}% ९

\twolineshloka
{स्वामिन्केयं नदी पुण्या मुनिवृन्दनिषेविता}
{करोति मम चित्तस्य प्रमोदभरनिर्भरम्}% १०

\twolineshloka
{इति श्रुत्वा वचस्तस्य राजराजस्य धीमतः}
{वक्तुं प्रचक्रमे विद्वांस्तीर्थमाहात्म्यमुत्तमम्}% ११

\uvacha{ब्राह्मण उवाच}

\twolineshloka
{गण्डकीयं नदी राजन्सुरासुरनिषेविता}
{पुण्योदकपरीवाह हतपातकसञ्चया}% १२

\twolineshloka
{दर्शनान्मानसं पापं स्पर्शनात्कर्मजं दहेत्}
{वाचिकं स्वीय तोयस्य पानतः पापसञ्चयम्}% १३

\twolineshloka
{पुरा दृष्ट्वा प्रजानाथः प्रजाः सर्वा विपावनीः}
{स्वगण्डविप्रुषोनेक पापघ्नीं सृष्टवानिमाम्}% १४

\twolineshloka
{एनां नदीं ये पुण्योदां स्पृशन्ति सुतरङ्गिणीम्}
{ते गर्भभाजो नैव स्युरपि पापकृतो नराः}% १५

\twolineshloka
{अस्यां भवा ये चाश्मानश्चक्रचिह्नैरलङ्कृताः}
{ते साक्षाद्भगवन्तो हि स्वस्वरूपधराः पराः}% १६

\twolineshloka
{शिलां सम्पूजयेद्यस्तु नित्यं चक्रयुतां नरः}
{न जातु स जनन्या वै जठरं समुपाविशेत्}% १७

\twolineshloka
{पूजयेद्यो नरो धीमाञ्छालग्रामशिलां वराम्}
{तेनाचारवता भाव्यं दम्भलोभवियोगिना}% १८

\twolineshloka
{परदार परद्रव्यविमुखेन नरेण हि}
{पूजनीयः प्रयत्नेन शालग्रामः सचक्रकः}% १९

\twolineshloka
{द्वारवत्यां भवं चक्रं शिला वै गण्डकीभवा}
{पुंसां क्षणाद्धरत्येव पापं जन्मशतार्जितम्}% २०

\twolineshloka
{अपि पापसहस्राणां कर्ता तावन्नरो भवेत्}
{शालग्रामशिलातोयं पीत्वा पूतो भवेत्क्षणात्}% २१

\twolineshloka
{ब्राह्मणः क्षत्रियो वैश्यः शूद्रो वेदपथि स्थितः}
{शालग्रामं पूजयित्वा गृहस्थो मोक्षमाप्नुयात्}% २२

\twolineshloka
{न जातु चित्स्त्रिया कार्यं शालग्रामस्य पूजनम्}
{भर्तृहीनाथ सुभगा स्वर्गलोकहितैषिणी}% २३

\twolineshloka
{मोहात्स्पृष्ट्वापि महिला जन्मशीलगुणान्विता}
{हित्वा पुण्यसमूहं सा सत्वरं नरकं व्रजेत्}% २४

\twolineshloka
{स्त्रीपाणिमुक्तपुष्पाणि शालग्रामशिलोपरि}
{पवेरधिकपातानि वदन्ति ब्राह्मणोत्तमाः}% २५

\twolineshloka
{चन्दनं विषसङ्काशं कुसुमं वज्रसन्निभम्}
{नैवेद्यं कालकूटाभं भवेद्भगवतः कृतम्}% २६

\twolineshloka
{तस्मात्सर्वात्मना त्याज्यं स्त्रिया स्पर्शः शिलोपरि}
{कुर्वती याति नरकं यावदिन्द्राश्चतुर्दश}% २७

\twolineshloka
{अपि पापसमाचारो ब्रह्महत्यायुतोऽपि वा}
{शालग्रामशिलातोयं पीत्वा याति परां गतिम्}% २८

\twolineshloka
{तुलसीचन्दनं वारि शङ्खो घण्टाथ चक्रकम्}
{शिला ताम्रस्य पात्रं तु विष्णोर्नामपदामृतम्}% २९

\twolineshloka
{पदामृतं तु नवभिः पापराशिप्रदाहकम्}
{वदन्ति मुनयः शान्ताः सर्वशास्त्रार्थकोविदाः}% ३०

\twolineshloka
{सर्वतीर्थपरिस्नानात्सर्वक्रतुसमर्चनात्}
{पुण्यं भवति यद्राजन्बिन्दौ बिन्दौ तदद्भुतम्}% ३१

\twolineshloka
{शालग्रामशिला यत्र पूज्यते पुरुषोत्तमैः}
{तत्र योजनमात्रं तु तीर्थकोटिसमन्वितम्}% ३२

\twolineshloka
{शालग्रामाः समाः पूज्याः समेषु द्वितयं नहि}
{विषमा एव सम्पूज्या विषमेषु त्रयं नहि}% ३३

\twolineshloka
{द्वारावती भवं चक्रं तथा वै गण्डकीभवम्}
{उभयोः सङ्गमो यत्र तत्र गङ्गा समुद्रगा}% ३४

\twolineshloka
{रूक्षाः कुर्वन्ति पुरुषा नायुः श्रीबलवर्जितान्}
{तस्मात्स्निग्धा मनोहारि रूपिण्यो ददति श्रियम्}% ३५

\twolineshloka
{आयुष्कामो नरो यस्तु धनकामो हि यः पुमान्}
{पूजयन्सर्वमाप्नोति पारलौकिकमैहिकम्}% ३६

\twolineshloka
{प्राणान्तकाले पुंसस्तु भवेद्भाग्यवतो नृप}
{वाचि नाम हरेः पुण्यं शिला हृदि तदन्तिके}% ३७

\twolineshloka
{गच्छत्सु प्राणमार्गेषु यस्य विश्रम्भतोऽपि चेत्}
{शालग्रामशिला स्फूर्तिस्तस्य मुक्तिर्न संशयः}% ३८

\twolineshloka
{पुरा भगवता प्रोक्तमम्बरीषाय धीमते}
{ब्राह्मणा न्यासिनः स्निग्धाः शालग्रामशिलास्तथा}% ३९

\twolineshloka
{स्वरूपत्रितयं मह्यमेतद्धि क्षितिमण्डले}
{पापिनां पापनिर्हारं कर्तुं धृतमुदं च ता}% ४०

\twolineshloka
{निन्दन्ति पापिनो ये वा शालग्रामशिलां सकृत्}
{कुम्भीपाके पचन्त्याशु यावदाभूतसम्प्लवम्}% ४१

\twolineshloka
{पूजां समुद्यतं कर्तुं यो वारयति मूढधीः}
{तस्य मातापिताबन्धुवर्गा नरकभागिनः}% ४२

\twolineshloka
{यो वा कथयति प्रेष्ठं शालग्रामार्चनं कुरु}
{सकृतार्थो नयत्याशु वैकुण्ठं स्वस्य पूर्वजान्}% ४३

\twolineshloka
{अत्रैवोदाहरन्तीममितिहासं पुरातनम्}
{मुनयो वीतरागाश्च कामक्रोधविवर्जिताः}% ४४

\twolineshloka
{पुरा कीकटसंज्ञे वै देशे धर्मविवर्जिते}
{आसीत्पुल्कसजातीयो नरः शबरसंज्ञितः}% ४५

\twolineshloka
{नित्यं जन्तुवधोद्युक्तः शरासनधरो मुहुः}
{तीर्थं प्रति यियासूनां बलाद्धरति जीवितम्}% ४६

\twolineshloka
{अनेकप्राणिहत्याकृत्परस्वे निरतः सदा}
{सदा रागादिसंयुक्तः कामक्रोधादिसंयुतः}% ४७

\twolineshloka
{विचरत्यनिशं भीमे वने प्राणिवधङ्करः}
{विषसंसक्तबाणाग्र रूढचापगुणोद्धुरः}% ४८

\twolineshloka
{सैकदा पर्यटन्व्याधः प्राणिमात्रभयङ्करः}
{कालं प्राप्तं न जानाति समीपेऽप्युग्रमानसः}% ४९

\twolineshloka
{यमदूतास्तु सम्प्राप्ताः पाशमुद्गरपाणयः}
{ताम्रकेशा दीर्घनखा लम्बदंष्ट्रा भयानकाः}% ५०

\twolineshloka
{श्यामा लोहस्यनिगडान्बिभ्रतो मोहकारकाः}
{बध्नन्तु पापिनं ह्येनं प्राणिमात्रभयङ्करम्}% ५१

\twolineshloka
{कदाचिन्मनसा नायं प्राणिमात्रोपकारकः}
{परदार परद्रव्य परद्रोहपरायणः}% ५२

\twolineshloka
{एतस्य जिह्वां महतीमहं निष्कासयाम्यतः}
{एको वदति चैतस्य चक्षुरुत्पाटयाम्यहम्}% ५३

\twolineshloka
{एको वदति चैतस्य करौ कृन्तामि पापिनः}
{अन्यो वदत्यहं कर्णौ कर्तयामि दुरात्मनः}% ५४

\twolineshloka
{एवं वदन्तः सुभृशं दन्तैर्दन्तनिपीडकाः}
{आगत्य तं दुरात्मानं सायुधास्तस्थुरुन्मदाः}% ५५

\twolineshloka
{एको दूतस्तदा सर्परूपं धृत्वादशत्पदे}
{स दष्टमात्रः सहसा गतासुः पर्यजायत}% ५६

\twolineshloka
{तदा तं लोहपाशेन बद्ध्वा ते यमकिङ्कराः}
{कशाभिस्ताडयामासुर्मुद्गरैः प्राहरन्क्रुधा}% ५७

\twolineshloka
{अहो दुष्ट दुरात्मंस्त्वं कदाचिन्नाचरः शुभम्}
{मनसापि यतस्त्वां वै क्षेप्स्यामो रौरवेषु च}% ५८

\twolineshloka
{त्वङ्मांसं वायसा रौद्रा भक्षयिष्यन्ति वै क्रुधा}
{आजन्मतस्तु भवता न कृतं हरिसेवनम्}% ५९

\twolineshloka
{त्वया कलत्रपुत्राद्या द्रोहं कृत्वा सुपोषिताः}
{न कदाचित्स्मृतो देवः पापहारी जनार्दनः}% ६०

\twolineshloka
{तस्मात्त्वां लोहशङ्कौ वा कुम्भीपाके च रौरवे}
{धर्मराजाज्ञया सर्वे नेष्यामो बहुताडनैः}% ६१

\twolineshloka
{एवमुक्त्वा यदानेतुं समैच्छन्यमकिङ्कराः}
{तावत्प्राप्तो महाविष्णुचरणाब्जपरायणः}% ६२

\twolineshloka
{यमदूतास्तदा दृष्टा वैष्णवेन महात्मना}
{पाशमुद्गरदण्डादिदुष्टायुधधरा गणाः}% ६३

\twolineshloka
{पुल्कसं लोहनिगडैर्बद्ध्वा यातुं समुद्यताः}
{बन्ध बन्ध ग्रसच्छिन्धि भिन्धि भिन्धीति वादिनः}% ६४

\twolineshloka
{तदा कृपालुस्तं प्रेक्ष्य पद्मनाभपरायणः}
{अत्यन्तकृपयायुक्तं चेतस्तत्र तदाकरोत्}% ६५

\twolineshloka
{असौ महादुष्ट पीडां मा यातु मम सन्निधौ}
{मोचयाम्यहमद्यैव यमदूतेभ्य एव च}% ६६

\twolineshloka
{इति कृत्वा मतिं तस्मिन्कृपायुक्तो मुनीश्वरः}
{शालग्रामशिलां हस्ते गृहीत्वास्य गतोऽन्तिके}% ६७

\twolineshloka
{तस्य पादोदकं पुण्यं तुलसीदलमिश्रितम्}
{मुखे विनिक्षिपन्कर्णे रामनाम जजाप ह}% ६८

\twolineshloka
{तुलसीं मस्तके तस्य धारयामास वैष्णवः}
{शिलां हृदि महाविष्णोर्धृत्वा प्राह स वैष्णवः}% ६९

\twolineshloka
{गच्छन्तु यमदूता वै यातनासु परायणाः}
{शालग्रामशिलास्पर्शो दहतात्पातकं महत्}% ७०

\twolineshloka
{इत्युक्तवति तस्मिन्वै गणा विष्णोर्महाद्भुताः}
{आययुस्तस्य सविधे शिलास्पर्शाद्गतांहसः}% ७१

\twolineshloka
{पीतवस्त्राः शङ्खचक्रगदापद्मविराजिताः}
{आगत्य मोचयामासुर्लोहपाशाद्दुरासदात्}% ७२

\twolineshloka
{मोचयित्वा महापापकारकं पुल्कसं नरम्}
{ऊचुः किमर्थं बद्धोऽयं वैष्णवः पूज्यदेहभृत्}% ७३

\twolineshloka
{कस्याज्ञाकारका यूयं यदधर्मप्रकारकाः}
{मुञ्चन्तु वैष्णवं त्वेनं किमर्थं विधृतो ह्ययम्}% ७४

\twolineshloka
{इति वाक्यं समाकर्ण्य जगदुर्यमकिङ्कराः}
{धर्मराजाज्ञया प्राप्ता नेतुं पापिनमुद्यताः}% ७५

\twolineshloka
{नासौ कदाचिन्मनसा प्राणिमात्रोपकारकः}
{प्राणिहत्या महापापकारी दुष्टशरीरभृत्}% ७६

\twolineshloka
{नॄन्बहूंस्तीर्थयात्रायां गच्छतोऽसौ व्यलुण्ठयत्}
{परदाररतो नित्यं सर्वपापाधिकारकः}% ७७

\twolineshloka
{तस्मान्नेतुं वयं प्राप्ताः पापिनं पुल्कसं नरम्}
{भवद्भिर्मोचितः कस्मादकस्मादागतैर्भटैः}% ७८

\uvacha{विष्णुदूता ऊचुः}

\twolineshloka
{ब्रह्महत्यादिकं पापं प्राणिकोटिवधोद्भवम्}
{शालग्रामशिलास्पर्शः सर्वं दहति तत्क्षणात्}% ७९

\twolineshloka
{रामेति नाम यच्छ्रोत्रे विश्रम्भादागतं यदि}
{करोति पापसन्दाहं तूलं वह्निकणो यथा}% ८०

\twolineshloka
{तुलसी मस्तके यस्य शिला हृदि मनोहरा}
{मुखे कर्णेऽथवा राम नाम मुक्तस्तदैव सः}% ८१

\twolineshloka
{तस्मादनेन तुलसी मस्तके विधृता पुरा}
{श्रावितं रामनामाशु शिला हृदि सुधारिता}% ८२

\twolineshloka
{तस्मात्पापसमूहोऽस्य दग्धः पुण्यकलेवरः}
{यास्यते परमं स्थानं पापिनां यत्सुदुर्ल्लभम्}% ८३

\twolineshloka
{वर्षायुतं तत्र भुक्त्वा भोगान्सर्वमनोहरान्}
{भारते जन्म सम्प्राप्य समाराध्य जगद्गुरुम्}% ८४

\twolineshloka
{प्राप्स्यते परमं स्थानं सुरासुरसुदुर्ल्लभम्}
{न ज्ञातो महिमा सम्यक्छिलायाः परमेष्ठिनः}% ८५

\twolineshloka
{दृष्टा स्पृष्टार्चिता वापि सर्वपापहरा क्षणात्}
{इत्युक्त्वा विरताः सर्वे महाविष्णोर्गणा मुदा}% ८६

\twolineshloka
{याम्यास्ते किङ्करा राज्ञे कथयामासुरद्भुतम्}
{वैष्णवो हर्षमापेदे रघुनाथपरायणः}% ८७

\twolineshloka
{मुक्तोऽसौ यमपाशाच्च गमिष्यति परं पदम्}
{तदाजगाम विमलं किङ्किणीजालमण्डितम्}% ८८

\twolineshloka
{विमानं देवलोकात्तु मनोहारि महाद्भुतम्}
{तत्रारुह्य गतः स्वर्गं महापुण्यैर्निषेवितम्}% ८९

\twolineshloka
{भोगान्भुक्त्वा स विपुलानाजगाम महीतलम्}
{काश्यां जन्म समासाद्य शुचिवाडवसत्कुले}% ९०

\twolineshloka
{आराध्य जगतामीशं गतवान्परमं पदम्}
{स पापी साधुसङ्गत्या शालग्रामशिलां स्पृशन्}% ९१

\twolineshloka
{महापीडाविनिर्मुक्तो गतवान्परमं पदम्}
{मया तेऽभिहितं राजन्गण्डकीचरितं महत्}% ९२

\onelineshloka
{श्रुत्वा विमुच्यते पापैर्भुक्तिं मुक्तिं च विन्दति}% ९३

{॥इति श्रीपद्मपुराणे पातालखण्डे रामाश्वमेधे शेषवात्स्यायनसंवादे गण्डकीमाहात्म्यं नाम विंशोऽध्यायः॥२०॥}

\dnsub{एकविंशोऽध्यायः}%\resetShloka

\uvacha{सुमतिरुवाच}

\twolineshloka
{एतन्माहात्म्यमतुलं गण्डक्याः कर्णगोचरम्}
{कृत्वा कृतार्थमात्मानममन्यत नृपोत्तमः}% १

\twolineshloka
{स्नात्वा तीर्थे पितॄन्सर्वान्सन्तर्प्य जहृषे महान्}
{शालग्रामशिलापूजां कुर्वन्वाडववाक्यतः}% २

\twolineshloka
{चतुर्विंशच्छिलास्तत्र गृहीत्वा स नृपोत्तमः}
{पूजयामास प्रेम्णा च चन्दनाद्युपचारकैः}% ३

\twolineshloka
{तत्र दानानि दत्त्वा च दीनान्धेभ्यो विशेषतः}
{गन्तुं प्रचक्रमे राजा पुरुषोत्तममन्दिरम्}% ४

\twolineshloka
{एवं क्रमेण सम्प्राप्तो गङ्गासागरसङ्गमम्}
{कृत्वाक्षिगोचरं तं च ब्राह्मणं पृष्टवान्मुदा}% ५

\twolineshloka
{स्वामिन्वद कियद्दूरे नीलाख्यः पर्वतो महान्}
{पुरुषोत्तमसंवासः सुरासुरनमस्कृतः}% ६

\twolineshloka
{तदा श्रुत्वा महद्वाक्यं रत्नग्रीवस्य भूपतेः}
{उवाच विस्मयाविष्टो राजानं प्रति सादरम्}% ७

\twolineshloka
{राजन्नेतत्स्थलं नीलपर्वतस्य नमस्कृतम्}
{किमर्थं दृश्यते नैव महापुण्यफलप्रदम्}% ८

\twolineshloka
{पुनःपुनरुवाचेदं स्थलं नीलस्य भूभृतः}
{कथं न दृश्यते राजन्पुरुषोत्तमवासभृत्}% ९

\twolineshloka
{अत्र स्नातं मया सम्यगत्र भिल्लाक्षिगोचराः}
{अनेनैव पथा राजन्नारूढं पर्वतोपरि}% १०

\twolineshloka
{इति तद्वाक्यमाकर्ण्य विव्यथे मानसे नृपः}
{नीलभूधरदर्शाय कुर्वन्नुत्कण्ठितं मनः}% ११

\twolineshloka
{उवाच तत्कथं विप्र दृश्येत पुरुषोत्तमः}
{कथं वा दृश्यते नीलस्तदुपायं वदस्व नः}% १२

\twolineshloka
{तदा वाक्यं समाकर्ण्य रत्नग्रीवस्य भूपतेः}
{तापसो ब्राह्मणो वाक्यमुवाच नृप विस्मितः}% १३

\twolineshloka
{गङ्गासागरसंयोगे स्नात्वास्माभिर्महीपते}
{स्थातव्यं तावदेवात्र यावन्नीलो न दृश्यते}% १४

\twolineshloka
{गीयते पापहा देवः पुरुषोत्तमसंज्ञितः}
{करिष्यते कृपामाशु भक्तवत्सलनामधृक्}% १५

\twolineshloka
{त्यजत्यसौ न वै भक्तान्देवदेवशिरोमणिः}
{अनेके रक्षिता भक्तास्तद्गायस्व महामते}% १६

\twolineshloka
{इति वाक्यं समाकर्ण्य राजा व्यथितचेतसा}
{स्नात्वा गङ्गाब्धिसंयोगे ततोनशनमादधात्}% १७

\twolineshloka
{करिष्यति कृपां यर्हि दर्शने पुरुषोत्तमः}
{पूजां कृत्वाशनं कुर्यामन्यथानशनं व्रतम्}% १८

\twolineshloka
{इति कृत्वा स नियमं गङ्गासागररोधसि}
{गायन्हरिगुणग्राममुपवासमथाचरत्}% १९

\uvacha{राजोवाच}

\twolineshloka
{जय दीनदयाकरप्रभो जय दुःखापह मङ्गलाह्वय}
{जय भक्तजनार्तिनाशन कृतवर्ष्मञ्जयदुष्टघातक}% २०

\twolineshloka
{अम्बरीषमथ वीक्ष्य दुःखितं विप्रशापहतसर्वमङ्गलम्}
{धारयन्निजकरे सुदर्शनं संररक्ष जठराधिवासतः}% २१

\twolineshloka
{दैत्यराज पितृकारितव्यथः शूलपाशजलवह्निपातनैः}
{श्रीनृसिंहतनुधारिणा त्वया रक्षितः सपदि पश्यतः पितुः}% २२

\twolineshloka
{ग्राहवक्त्रपतिताङ्घ्रिमुद्भटं वारणेन्द्रमतिदुःखपीडितम्}
{वीक्ष्य साधुकरुणार्द्रमानसस्त्वं गरुत्मति कृतारुहक्रियः}% २३

\twolineshloka
{त्यक्तपक्षिपतिरात्तचक्रको वेगकम्पयुतमालिकाम्बरः}
{गीयसे सुभिरमुष्य न क्रतो मोचकः सपदि तद्विनाशकः}% २४

\twolineshloka
{यत्रयत्र तव सेवकार्दनं तत्र तत्र बत देहधारिणा}
{पाल्यते च भवता निजः प्रभो पापहारिचरितैर्मनोहरैः}% २५

\twolineshloka
{दीननाथ सुरमौलिहीरकाघृष्टपादतल भक्तवल्लभ}
{पापकोटिपरिदाहक प्रभो दर्शयस्व निजदर्शनं मम}% २६

\twolineshloka
{पापकृद्यदि जनोयमागतो मानसे तव तथा हि दर्शय}
{तावका वयमघौघनाशनं विस्मृतं नहि सुरासुरार्चित}% २७

\twolineshloka
{ये वदन्ति तव नाम निर्मलं ते तरन्ति सकलाघसागरम्}
{संस्मृतिर्यदि कृता तदा मया प्राप्यतां सकलदुःखवारक}% २८

\uvacha{सुमतिरुवाच}

\twolineshloka
{एवं गायन्गुणान्रात्रौ दिवा वापि महीपतिः}
{क्षणमात्रं न विश्रान्तो निद्रामाप न वै सुखम्}% २९

\twolineshloka
{गायन्गच्छन्गृणंस्तिष्ठन्वदत्येतदहर्निशम्}
{दर्शयस्व कृपानाथ स्वतनुं पुरुषोत्तम}% ३०

\twolineshloka
{एवं राज्ञः पञ्चदिनं गतं गङ्गाब्धिसङ्गमे}
{तदा कृपाब्धिः कृपया चिन्तयामास गोपतिः}% ३१

\twolineshloka
{असौ राजा मदीयेन गानेन विगताघकः}
{पश्य तान्मामकीं प्रेष्ठां सुरासुरनमस्कृताम्}% ३२

\twolineshloka
{इति सञ्चिन्त्य भगवान्कृपापूरितमानसः}
{सन्न्यासिवेषमास्थाय ययौ राज्ञोऽन्तिकं विभुः}% ३३

\twolineshloka
{तत्र गत्वा महाराज त्रिदण्डियतिवेषधृक्}
{भक्तानुकम्पया प्राप्तो वीक्षितस्तापसेन हि}% ३४

\twolineshloka
{ॐनमो विष्णवेत्युक्त्वा नमश्चक्रे नृपोत्तमः}
{अर्घ्यपाद्यासनैः पूजां चकार हरिमानसः}% ३५

\twolineshloka
{उवाच भाग्यमतुलं यद्भवानक्षिगोचरः}
{अतः परं दास्यते मे गोविन्दो निजदर्शनम्}% ३६

\twolineshloka
{इति श्रुत्वा तु तद्वाक्यं सन्न्यासी निजगाद तम्}
{राजञ्छृणुष्व कथितं मम वाक्यविनिःसृतम्}% ३७

\twolineshloka
{अहं ज्ञानेन जानामि भूतं भव्यं भवच्च यत्}
{तस्मादहं ब्रुवे किञ्चिच्छृणुष्वैकाग्रमानसः}% ३८

\twolineshloka
{श्वो मध्याह्ने हरिर्दाता दर्शनं ब्रह्मदुर्ल्लभम्}
{पञ्चभिः स्वजनैः साकं यास्यसे परमं पदम्}% ३९

\twolineshloka
{त्वममात्यश्च महिला तव तापस वाडवः}
{पुरे तव करम्बाख्यः साधुश्च तं तु वायकः}% ४०

\twolineshloka
{एतैः पञ्चभिरेतस्मिन्नीले पर्वतसत्तमे}
{यास्यसे ब्रह्मदेवेन्द्र वन्दितं सुरपूजितम्}% ४१

\twolineshloka
{इत्युक्त्वाऽदृश्यतां प्राप्तो यतिः क्वापि न दृश्यते}
{तदाकर्ण्य नृपो हर्षं प्राप चाशु सविस्मयम्}% ४२

\uvacha{राजोवाच}

\twolineshloka
{स्वामिन्कोऽसौ समागत्य सन्न्यासी मां यदूचिवान्}
{न दृश्यते पुनः कुत्र गतोऽसौ चित्तहर्षदः}% ४३

\uvacha{तापस उवाच}

\twolineshloka
{राजंस्तव महाप्रेम्णा कृष्टचित्तः समभ्यगात्}
{पुरुषोत्तमनामायं सर्वपापप्रणाशनः}% ४४

\twolineshloka
{श्वोमध्याह्ने तव पुरो भविष्यति महान्गिरिः}
{तमारुह्य हरिं दृष्ट्वा कृतार्थस्त्वं भविष्यसि}% ४५

\twolineshloka
{इतिवाक्यसुधापूर नाशितस्वान्त सञ्ज्वरः}
{हर्षं यमाप स नृपो ब्रह्मापि न हि वेत्ति तम्}% ४६

\twolineshloka
{तदा दुन्दुभयो नेदुर्वीणापणवगोमुखाः}
{महानन्दस्तदा ह्यासीद्राजराजस्य चेतसि}% ४७

\twolineshloka
{गायन्हरिं क्षणं तिष्ठन्हसञ्जल्पन्ब्रुवन्नमन्}
{आनन्दं प्राप सुघनं सर्वसन्तापनाशनम्}% ४८

{॥इति श्रीपद्मपुराणे पातालखण्डे शेषवात्स्यायनसंवादे रामाश्वमेधे सन्न्यासिदर्शनं नाम एकविंशोऽध्यायः॥२१॥}

\dnsub{द्वाविंशोऽध्यायः}%\resetShloka

\uvacha{सुमतिरुवाच}

\twolineshloka
{अथ सर्वं दिनं नीत्वा हरिस्मरणकीर्तनैः}
{रात्रौ सुष्वाप गङ्गाया रोधस्युरुफलप्रदे}% १

\twolineshloka
{ददर्श स्वप्नमध्ये तु स स्वात्मानं चतुर्भुजम्}
{शङ्खचक्रगदापद्मशार्ङ्गकोदण्डधारिणम्}% २

\fourlineindentedshloka
{नृत्यन्तं पुरुषोत्तमस्य पुरतः शर्वादि देवैः सह}
{श्रीमद्भिः स्वतनूयुतैररिगदाम्बूत्थाब्जहेत्यादिभिः}
{विष्वक्सेनवरैर्गणैः सुतनुभिः श्रीशंसदोपासितं}
{दृष्ट्वा विस्मयमाप लोकविषयं हर्षं तथात्यद्भुतम्}% ३

\twolineshloka
{ददतं मनसोऽभीष्टं पुरुषोत्तमसंज्ञितम्}
{आत्मानं च कृपापात्रममन्यत महामतिः}% ४

\twolineshloka
{इत्येवं स्वप्नविषये ददर्श नृपसत्तमः}
{प्रातः प्रबुद्धो विप्राय जगाद स्वप्नमीक्षितम्}% ५

\twolineshloka
{तच्छ्रुत्वा वाडवो धीमान्कथयामास विस्मितः}
{राजंस्त्वयासौ दृष्टो यः पुरुषोत्तमसंज्ञितः}% ६

\twolineshloka
{दास्यते शङ्खचक्रादिचिह्नितां स्वतनुं हरिः}
{इति श्रुत्वा तु तद्वाक्यं रत्नग्रीवो महामनाः}% ७

\twolineshloka
{दापयामास दानानि दीनानां मानसोचितम्}
{स्नात्वा गङ्गाब्धिसंयोगे तर्पयित्वा पितॄन्सुरान्}% ८

\twolineshloka
{गायन्हरिगुणग्रामं प्रत्यैक्षत च दर्शनम्}
{ततो मध्याह्नसमये दिविदुन्दुभयो मुहुः}% ९

\twolineshloka
{जघ्नुः सुरकराघात बहुशब्दसुशब्दिताः}
{अकस्मात्पुष्पवृष्टिश्च बभूव नृपमस्तके}% १०

\onelineshloka
{धन्योसि नृपवर्यस्त्वं नीलं पश्याक्षिगोचरम्}% ११

\twolineshloka
{शृणोतीति यदा वाक्यं नृपो देवप्रणोदितम्}
{तदा स सूर्यकोटीनामधिकान्ति धरोद्भुतः}% १२

\twolineshloka
{राज्ञोऽक्षिगोचरो जातो नीलनामा महागिरिः}
{राजतैः कानकैः शृङ्गैः समन्तात्परिराजितः}% १३

\twolineshloka
{किमग्निः प्रज्वलत्येष द्वितीयः किमु भास्करः}
{किमयं वैद्युतः पुञ्जो ह्यकस्मात्स्थिरकान्तिधृक्}% १४

\twolineshloka
{तापस ब्राह्मणो दृष्ट्वा नीलप्रस्थं सुशोभितम्}
{राज्ञे निवेदयामास एष पुण्यो महागिरिः}% १५

\twolineshloka
{तच्छ्रुत्वा नृपतिश्रेष्ठः शिरसा प्रणनाम ह}
{धन्योऽस्मि कृतकृत्योऽस्मि नीलो मे दृष्टिगोचरः}% १६

\twolineshloka
{अमात्यो राजपत्नी च करम्बस्तन्तुवायकः}
{नीलदर्शनसंहृष्टा बभूवुः पुरुषर्षभ}% १७

\twolineshloka
{पञ्चैते विजये काले नीलपर्वतमारुहन्}
{महादुन्दुभिनिर्घोषाञ्च्छृण्वन्तो ह्यमरैः कृतान्}% १८

\twolineshloka
{तस्योपरितने शृङ्गे चित्रपादपराजिते}
{ददर्श हाटकाबद्धं देवालयमनुत्तमम्}% १९

\twolineshloka
{ब्रह्मागत्य सदा पूजां करोति परमेष्ठिनः}
{नैवेद्यं कुरुते यत्र हरिसन्तोषकारकम्}% २०

\twolineshloka
{दृष्ट्वाथ तत्र विमलं देवायतनमुत्तमम्}
{प्रविवेश परीवारैः पञ्चभिः सह संवृतः}% २१

\twolineshloka
{तत्र दृष्ट्वा जातरूपे महामणिविचित्रिते}
{सिंहासने विराजन्तं चतुर्भुजमनोहरम्}% २२

\twolineshloka
{चण्ड प्रचण्ड विजय जयादिभिरुपासितम्}
{प्रणनाम सपत्नीको राजा सेवकसंयुतः}% २३

\twolineshloka
{प्रणम्य परमात्मानं महाराजः सुरोत्तमम्}
{स्नापयामास विधिवद्वेदोक्तैः स्नानमन्त्रकैः}% २४

\twolineshloka
{अर्घ्यपाद्यादिकं चक्रे प्रीतेन मनसा नृपः}
{चन्दनेन विलिप्यैनं सुवस्त्रे विनिवेद्य च}% २५

\twolineshloka
{धूपमारार्तिकं कृत्वा सर्वस्वादुमनोहरम्}
{नैवेद्यं भगवन्मूर्त्यै न्यवेदयदथो नृपः}% २६

\twolineshloka
{प्रणम्य च स्तुतिं चक्रे तापसब्राह्मणेन च}
{यथामतिगुणग्रामगुम्फितस्तोत्रसञ्चयैः}% २७

\uvacha{राजोवाच}

\twolineshloka
{एकस्त्वं पुरुषः साक्षाद्भगवान्प्रकृतेः परः}
{कार्यकारणतो भिन्नो महत्तत्त्वादिपूजितः}% २८

\twolineshloka
{त्वन्नाभिकमलाज्जज्ञे ब्रह्मा सृष्टिविचक्षणः}
{तथा संहारकर्ता च रुद्रस्त्वन्नेत्रसम्भवः}% २९

\twolineshloka
{त्वयाज्ञप्तः करोत्यस्य विश्वस्य परिचेष्टितम्}
{त्वत्तो जातं पुराणाद्यज्जगत्स्थास्नु चरिष्णु च}% ३०


\threelineshloka
{चेतनाशक्तिमाविश्य त्वमेनं चेतयस्यहो}
{तव जन्म तु नास्त्येव नान्तस्तव जगत्पते}
{वृद्धिक्षयपरीणामास्त्वयि सन्त्येव नो विभो}% ३१

\twolineshloka
{तथापि भक्तरक्षार्थं धर्मस्थापनहेतवे}
{करोषि जन्मकर्माणि ह्यनुरूपगुणानि च}% ३२

\twolineshloka
{त्वया मात्स्यं वपुर्धृत्वा शङ्खस्तु निहतोसुरः}
{वेदाः सुरक्षिता ब्रह्मन्महापुरुषपूर्वज}% ३३

\twolineshloka
{शेषो न वेत्ति महि ते भारत्यपि महेश्वरी}
{किमुतान्ये महाविष्णो मादृशास्तु कुबुद्धयः}% ३४

\twolineshloka
{मनसा त्वां न चाप्नोति वागियं परमेश्वरी}
{तस्मादहं कथं त्वां वै स्तोतुं स्यामीश्वरः प्रभो}% ३५

\twolineshloka
{इति स्तुत्वा स शिरसा प्रणाममकरोन्मुहुः}
{गद्गदस्वरसंयुक्तो रोमहर्षाङ्किताङ्गकः}% ३६

\twolineshloka
{इति स्तुत्या प्रहृष्टात्मा भगवान्पुरुषोत्तमः}
{उवाच वचनं सत्यं राजानं प्रति सार्थकम्}% ३७

\uvacha{श्रीभगवानुवाच}

\twolineshloka
{तव स्तुत्या प्रहर्षोऽभून्मम राजन्महामते}
{जानीहि त्वं महाराज मां च प्रकृतितः परम्}% ३८

\twolineshloka
{नैवेद्यभक्षणं त्वं हि शीघ्रं कुरु मनोहरम्}
{चतुर्भुजत्वमाप्तः सन्गन्तासि परमं पदम्}% ३९

\twolineshloka
{त्वत्कृत्स्तुतिरत्नेन यो मां स्तोष्यति मानवः}
{तस्यापि दर्शनं दास्ये भुक्तिमुक्तिवरप्रदम्}% ४०

\twolineshloka
{इत्येवं वचनं राजा श्रुत्वा भगवतोदितम्}
{नैवेद्यभक्षणं चक्रे चतुर्भिः सह सेवकैः}% ४१

\twolineshloka
{ततो विमानं सम्प्राप्तं किङ्किणीजालमण्डितम्}
{अप्सरोवृन्दसंसेव्यं सर्वभोगसमन्वितम्}% ४२

\twolineshloka
{पुरुषोत्तमसंज्ञं च पश्यन्राजा स धार्मिकः}
{ववन्दे चरणौ तस्य कृपापात्रकृतात्मकः}% ४३

\twolineshloka
{तदाज्ञया विमाने स आरुह्य महिलायुतः}
{जगाम पश्यतस्तस्य दिवि वैकुण्ठमद्भुतम्}% ४४

\twolineshloka
{मन्त्री धर्मपरो राज्ञः सर्वधर्मविदुत्तमः}
{ययौ साकं विमानेन ललनावृन्दसेवितः}% ४५

\twolineshloka
{तापसब्राह्मणस्तत्र सर्वतीर्थावगाहकः}
{चतुर्भुजत्वं सम्प्राप्तो ययौ देवैर्विमानिभिः}% ४६

\twolineshloka
{करम्बोऽपि महाराज गानपुण्येन दर्शनम्}
{प्राप्तो ययौ सुरावासं सर्वदेवादिदुर्ल्लभम्}% ४७

\twolineshloka
{सर्वे प्रचलिता विष्णुलोकं परममद्भुतम्}
{चतुर्भुजाः शङ्खचक्रगदापाथोजधारिणः}% ४८

\twolineshloka
{सर्वे मेघश्रियः शुद्धा लसदम्भोजपाणयः}
{हारकेयूरकटकैर्भूषिताङ्गा ययुर्दिवम्}% ४९

\twolineshloka
{तद्विमानावलीर्दृष्ट्वा लोकैः प्रकृतिभिस्तदा}
{दुन्दुभीनां तु निर्घोषस्तैः कृतः कर्णगोचरः}% ५०

\twolineshloka
{तदैको ब्राह्मणो ह्यासीद्विष्णुपादाब्जवल्लभः}
{गतस्तद्विरहाकृष्टचेता जातश्चतुर्भुजः}% ५१

\twolineshloka
{तच्चित्रं वीक्ष्य लोकास्ते प्रशंसन्तो महोदयम्}
{गङ्गासागरसंयोगे स्नात्वाऽगुस्तं पुरं प्रति}% ५२

\twolineshloka
{अहो भाग्यं भूमिपते रत्नग्रीवस्य सन्मतेः}
{जगामानेन देहेन तद्विष्णोः परमं पदम्}% ५३

\twolineshloka
{राजन्नसौ नीलगिरिः पुरुषोत्तमसत्कृतः}
{यं वीक्ष्यैव व्रजन्त्यद्धा वैकुण्ठं परमायनम्}% ५४

\twolineshloka
{एतन्नीलस्य माहात्म्यं यः शृणोति स भाग्यवान्}
{यः श्रावयति लोकान्वै तौ गच्छेतां परं पदम्}% ५५

\twolineshloka
{एतच्छ्रुत्वा तु दुःस्वप्नो नश्यति स्मृतिमात्रतः}
{प्रान्ते संसारनिस्तारं ददाति पुरुषोत्तमः}% ५६

\twolineshloka
{योऽसौ नीलाधिवासी च स रामः पुरुषोत्तमः}
{सीतासाक्षान्महालक्ष्मीः सर्वकारणकारणम्}% ५७

\twolineshloka
{हयमेधं चरित्वा स लोकान्वै पावयिष्यति}
{यन्नामब्रह्महत्यायाः प्रायश्चित्ते प्रदिश्यते}% ५८

\twolineshloka
{इदानीं त्वद्धयः प्राप्तो नीलेपर्वतसत्तमे}
{पुरुषोत्तमदेवं त्वं नमस्कुरु महामते}% ५९

\twolineshloka
{तत्र निष्पापिनो भूत्वा यास्यामः परमं पदम्}
{यस्य प्रसादाद्बहवो निस्तीर्णा भवसागरात्}% ६०

\twolineshloka
{एवं प्रवदतस्तस्य प्राप्तोऽश्वो नीलपर्वतम्}
{वायुवेगेन पृथिवीं कुर्वन्सङ्क्षुण्णमण्डलाम्}% ६१

\twolineshloka
{तदा राजापि तत्पृष्ठचारी नीलाभिधं गिरिम्}
{प्राप्तो गङ्गाब्धिसंयोगे स्नात्वागात्पुरुषोत्तमम्}% ६२

\twolineshloka
{स्तुत्वा नत्वा च देवेशं सुरासुरनमस्कृतम्}
{जातं कृतार्थमात्मानममन्यत स शत्रुहा}% ६३

{॥इति श्रीपद्मपुराणे पातालखण्डे शेषवात्स्यायनसंवादे रामाश्वमेधे नीलगिरिमहिमवर्णनं नाम द्वाविंशोऽध्यायः॥२२॥}

\dnsub{त्रयोविंशोऽध्यायः}%\resetShloka

\uvacha{शेष उवाच}

\twolineshloka
{क्षणं स्थित्वा तृणान्यत्त्वा ययौ वाजी मनोजवः}
{वीरश्रेणीवृतः पत्रं भाले धृत्वा सचामरः}% १

\twolineshloka
{शत्रुघ्नेन सुवीरेण लक्ष्मीनिधि नृपेण च}
{पुष्कलेनोग्रवाहेन प्रतापाग्र्येण रक्षितः}% २

\twolineshloka
{ययौ पुरीं स चक्राङ्कां सुबाहुपरिरक्षिताम्}
{अनेकवीरकोटीभी रक्षितोऽनुगतः प्रभो}% ३

\twolineshloka
{तदा पुत्रोस्य दमनो मृगयामास्थितो महान्}
{ददर्शाश्वं भालपत्रं चन्दनादिकचर्चितम्}% ४

\twolineshloka
{विलोक्य सेवकं प्राह कस्याश्वो मेऽक्षिगोचरः}
{भाले पत्रं धृतं किं नु चामरं किन्तु शोभनम्}% ५

\twolineshloka
{इति राज्ञोवचः श्रुत्वा सेवकः प्रययौ ततः}
{यत्रासौ वर्तते वाजी भालपत्रः सुशोभनः}% ६

\twolineshloka
{गृहीत्वा तं केशसङ्घे रत्नमालाविभूषितम्}
{निनाय चाग्रे भूपस्य सुबाहुकुलधारिणः}% ७

\twolineshloka
{स पत्रं वाचयामास सुन्दराक्षरशोभितम्}
{अयोध्याधिपतिश्चासीद्राजा दशरथो बली}% ८

\twolineshloka
{तस्यात्मजो रामभद्रः सर्वशूरशिरोमणिः}
{नान्योस्ति तत्समः पृथ्व्यां धनुर्धरणविक्रमः}% ९

\twolineshloka
{तेनासौ मोचितो वाजी चन्दनादिकचर्चितः}
{तं पालयति धर्मात्मा शत्रुघ्नः परवीरहा}% १०

\twolineshloka
{ये च शूरा वयं वीरा धनुर्हस्ता इमे वयम्}
{ते गृह्णन्तु बलाद्वाहं रत्नमालाविभूषितम्}% ११

\twolineshloka
{तं च मोक्ष्यति शत्रुघ्नः सर्ववीरशिरोमणिः}
{अन्यथा पादयोस्तस्य प्रणतिं यान्तु धन्विनः}% १२

\twolineshloka
{इत्यभिप्रायमालोक्य जगाद नृपनन्दनः}
{राम एव धनुर्धारी न वयं क्षत्त्रियाः स्मृताः}% १३

\twolineshloka
{ताते मेऽवस्थिते पृथ्व्यां कोऽयं गर्वो महान्भुवि}
{प्राप्नोतु गर्वस्य फलं मम निर्मुक्तसायकैः}% १४

\twolineshloka
{अद्य मे निशिता बाणाः शत्रुघ्नं किंशुकं यथा}
{पुष्पितं विदधत्वद्धा क्षतावृतशरीरकम्}% १५

\twolineshloka
{दारयन्तु कपोलांश्च सायका मम दन्तिनाम्}
{अश्वान्पश्यन्तु शतशो रुधिरौघपरिप्लुतान्}% १६

\twolineshloka
{पिबन्तु योगिनीसङ्घा रुधिराणि नृमस्तकैः}
{शिवा भवन्तु सन्तुष्टा मद्वैरिक्रव्यभक्षणैः}% १७

\twolineshloka
{पश्यन्तु सुभटास्तस्य मम बाहुबलं महत्}
{कोदण्डदण्डनिर्मुक्ताः शरकोटीर्विमुञ्चतः}% १८

\twolineshloka
{इत्थमुक्त्वा महीपस्य तनुजो दमनाभिधः}
{स्वपुरं प्रेषयित्वा तं प्रहृष्टोऽभवदुद्भटः}% १९

\twolineshloka
{सेनापतिमुवाचेदं सज्जीकुरु महामते}
{सेनां परिमितां मह्यं वैरिवृन्दनिवारणे}% २०

\twolineshloka
{सज्जां सेनां विधायाशु सम्मुखो रणमण्डले}
{स्थितवान्या वदत्युग्रस्तावत्प्राप्ता हयानुगाः}% २१

\twolineshloka
{क्वासौ हयो महाराज्ञो भालपत्रेण चिह्नितः}
{पप्रच्छुस्ते तु चान्योन्यमतिव्याकुलिता मुहुः}% २२

\twolineshloka
{तावद्ददर्श पुरतः प्रतापाग्र्यः परन्तपः}
{सज्जीभूतं तु कटकं वीरशब्दनिनादितम्}% २३

\twolineshloka
{तत्रावदञ्जनाः केचिन्नीतोऽश्वोऽनेन भूपते}
{अन्यथा सम्मुखस्तिष्ठेत्कथं वीरो बलानुगः}% २४

\twolineshloka
{इत्याकर्ण्य प्रतापाग्र्यः प्रेषयामास सेवकम्}
{स गत्वा तत्र पप्रच्छ कुत्राश्वो रामभूपतेः}% २५

\twolineshloka
{केन नीतः कुतो नीतो रामं जानाति नो कुधीः}
{यं शक्रप्रमुखा देवा बलिमादाय सन्नताः}% २६

\twolineshloka
{तस्य वै धर्मराजस्य कुपितं तु बलं महत्}
{सर्वथा हि ग्रसिष्येत प्रणतिं चेन्न यास्यति}% २७

\twolineshloka
{इत्थमुक्तं समाकर्ण्य तदा राजसुतो बली}
{तं वै धिक्कारयामास वाग्जालेन सुदुर्मनाः}% २८

\twolineshloka
{मयानीतो यज्ञहयः पत्रचिह्नाद्यलङ्कृतः}
{ये शूरास्ते तु मां जित्वा मोचयन्तु बलादिह}% २९

\twolineshloka
{सेवकस्तद्वचः श्रुत्वा रोषपूर्णो हसन्ययौ}
{राज्ञे निवेदयामास यथावदुपवर्णितम्}% ३०

\twolineshloka
{तच्छ्रुत्वा रोषताम्राक्षः प्रतापाग्र्यो महाबलः}
{ययौ योद्धुं राजपुत्रं महावीरपुरस्कृतम्}% ३१

\twolineshloka
{रथेन कनकाङ्गेन चतुर्वाजिसुशोभिना}
{सुकूबरेण सर्वास्त्रपूरितेन ययौ बली}% ३२

\twolineshloka
{धनुष्टङ्कारयामास महाबलसमन्वितः}
{पुनःपुनर्जहासोच्चैः कोपादुद्गमिताश्रुकः}% ३३

\twolineshloka
{अश्ववाहा गजारूढाः खड्गोल्लसितपाणयः}
{अन्वयुस्ते प्रतापाग्र्यं रोषपूर्णाकुलेक्षणम्}% ३४

\twolineshloka
{हस्तिनः पत्तयश्चैव कोटिशः प्रधनोद्यताः}
{चिरकालमभीप्सन्तो रणं वीरेणकारितम्}% ३५

\twolineshloka
{तदोद्यतं समाज्ञाय रिपुसैन्यं नृपात्मजः}
{प्रत्युज्जगाम वीराग्र्यो महाबलपरीवृतः}% ३६

\twolineshloka
{सन्नद्धः कवची खड्गी शरासनधरो युवा}
{लीलयैव ययौ योद्धुं मृगराड्गजतामिव}% ३७

\twolineshloka
{तदा योधाः प्रकुपिताः परस्परवधैषिणः}
{छिन्धि भिन्धीति भाषन्तो रणकार्यविशारदाः}% ३८

\twolineshloka
{पत्तयः पत्तिसङ्घेन गजारूढाश्च सादिभिः}
{रथारूढा रथस्थैश्च वाहारूढाश्वसंस्थितैः}% ३९

\twolineshloka
{गजा भिन्ना द्विधा जाता हयाश्च द्विदलीकृताः}
{अनेकनरमस्तिष्कैर्मेदिनीपूरिता ह्यभूत्}% ४०

\twolineshloka
{तदा प्रकुपितो राजा प्रतापाग्र्यो महाबलः}
{स्वसैन्यकदनोद्युक्तं राजपुत्रं समीक्ष्य च}% ४१

\twolineshloka
{उवाच सारथिं तत्र प्रापयाश्वान्यतो मम}
{सैन्यस्य कदनासक्तो राजपुत्रो महारथः}% ४२

\twolineshloka
{अथ वीरशिरोरत्न नमिताङ्घ्रिर्नृपात्मजः}
{ययौ सम्मुखमेवास्य प्रतापाग्र्यस्य वीर्यवान्}% ४३

\twolineshloka
{सारथिः प्रापयामास प्रतापाग्र्यस्य वाजिनः}
{यत्रासौ दमनो वीरः सर्वशूरशिरोमणिः}% ४४

\twolineshloka
{गत्वा तमाह्वयामास राजपुत्रं रणोद्यतम्}
{रथे पुरटनिर्णिक्ते तिष्ठन्कोदण्डदण्डभृत्}% ४५

\twolineshloka
{रे राजपुत्र क शिशो त्वया बद्धोऽश्वसत्तमः}
{न ज्ञातोसि महाराजः सर्ववीरेन्द्र सेवितः}% ४६

\twolineshloka
{यस्य प्रतापं दैत्येन्द्रो न शक्तः सोढुमद्भुतम्}
{तस्य त्वं वाजिनं नीत्वा गतोऽसि पुटभेदनम्}% ४७

\twolineshloka
{मां जानीहि पुरः प्राप्तं कालरूपं तु वैरिणम्}
{मुञ्चाश्वमर्भ गच्छाशु बालक्रीडनकं कुरु}% ४८

\twolineshloka
{कस्यात्मजस्त्वं कुत्रत्यः कथं नोऽदीर्घदर्शिना}
{धृतोऽश्वस्त्वथ सञ्जाता घृणा मम शिशो त्वयि}% ४९

\twolineshloka
{इत्थमाकर्ण्य दमनः स्मितं चक्रे महामनाः}
{उवाच च प्रतापाग्र्यं तृणीकुर्वंश्च तद्बलम्}% ५०

\uvacha{दमन उवाच}

\twolineshloka
{मया बद्धो बलादश्वो नीतः स्वपुटभेदनम्}
{नार्पयिष्येऽद्य सप्राणः कुरु युद्धं महाबल}% ५१

\twolineshloka
{त्वया यदुक्तं बालस्त्वं गत्वा क्रीडनकं कुरु}
{तन्मे पश्य महाराज क्रीडनं रणमूर्धनि}% ५२

\uvacha{शेष उवाच}

\twolineshloka
{इत्युक्त्वा सगुणं चापं विधाय सुभुजां गजः}
{शराणां शतमाधत्त प्रतापाग्र्यस्य वक्षसि}% ५३

\twolineshloka
{सन्धाय बाणशतकं शङ्खं दध्मौ प्रतापवान्}
{तेन शङ्खनिनादेन कातराणां भयं ह्यभूत्}% ५४

\twolineshloka
{ताडयामास हृदये बाणानां शतकेन सः}
{प्रतापाग्र्यः प्रचिच्छेद लघुहस्तः सुपर्वणः}% ५५

\twolineshloka
{स बाणच्छेदनं दृष्ट्वा कुपितो व्यसृजच्छरान्}
{कङ्कपक्षान्वितांस्तीक्ष्णभल्लान्राजात्मजो बली}% ५६

\twolineshloka
{आकाशे भुवि मध्ये च बाणा ददृशिरेऽञ्चिताः}
{स्वनामचिह्नितास्तीक्ष्णधारापातसुशोभिताः}% ५७

\twolineshloka
{शरास्तद्बाहु हृदये लग्ना वह्निकणान्बहून्}
{सृजन्तः कुर्वते सैन्यदाहनं तदभून्महत्}% ५८

\twolineshloka
{प्रतापाग्र्यः प्रकुपितस्तिष्ठतिष्ठेति च ब्रुवन्}
{शरेण दशसङ्ख्येन ताडयामास मूर्धनि}% ५९

\twolineshloka
{ते बाणा राजपुत्रस्य ललाटे परिनिष्ठिताः}
{विराजन्ते स्म च मुने दशशाखास्तरोरिव}% ६०

\twolineshloka
{तेन बाणप्रहारेण विव्यथेन महामनाः}
{यष्टिकाप्रहतो यद्वत्कुञ्जरः सप्तवर्षकः}% ६१

\twolineshloka
{बाणान्धनुषि सन्धाय मुमोच त्रिशताञ्छुभान्}
{सुवर्णपुङ्खरचितान्महाकालानलोपमान्}% ६२

\twolineshloka
{ते बाणास्तु प्रतापाग्र्य वक्षो भित्त्वा गता ह्यधः}
{शोणिताक्ता यथा रामचन्द्र भक्ति पराङ्मुखाः}% ६३

\twolineshloka
{प्रतापाग्र्यः प्रकुपितः शरान्मुञ्चन्सहस्रशः}
{अकरोद्विरथं सूनुं सुबाहोस्तत्क्षणाद्द्रुतम्}% ६४

\twolineshloka
{चतुर्भिश्च तुरो वाहान्द्वाभ्यां ध्वजमशातयत्}
{एकेन सारथेः कायाच्छिरो मह्यामपातयत्}% ६५

\twolineshloka
{चतुर्भिस्ताडयामास तं सूनुं नृपतेः पुनः}
{तत्क्षणाच्चापमेकेन गुणयुक्तं तु चिच्छिदे}% ६६

\twolineshloka
{सोऽन्यरथं समारुह्य हयरत्नसुशोभितम्}
{धनुः करे समादाय सज्यं चक्रे महामनाः}% ६७

\twolineshloka
{प्रत्युवाच प्रतापाग्र्यं त्वया विक्रान्तमद्भुतम्}
{पश्येदानीं पराक्रान्तिं धनुषो मम सद्भट}% ६८

\twolineshloka
{एवमुक्त्वा तु दमनो बाणान्दश समाददे}
{चतुर्भिश्चतुरो वाहान्निनाय यमसादनम्}% ६९

\twolineshloka
{चतुर्भिस्तिलशः कृत्तो रथश्चक्रसमन्वितः}
{एकेन हृदि विव्याध बाणेनैकेन सारथिम्}% ७०

\twolineshloka
{जगर्ज शङ्खमापूर्य शङ्खशब्दसमन्वितः}
{तत्कर्म पूजयामास साधु वीर महाबल}% ७१

\twolineshloka
{इति विक्रान्तमालोक्य प्रतापाग्र्यो रुषान्वितः}
{अन्यं रथं समास्थाय ययौ योद्धुं नृपात्मजम्}% ७२

\twolineshloka
{उवाच वीर पश्य त्वं मम विक्रान्तमद्भुतम्}
{इत्युक्त्वाशु मुमोचौघाञ्छराणां शितपर्वणाम्}% ७३

\twolineshloka
{शराः सर्वत्र दृश्यन्ते कुञ्जरेषु हयेषु च}
{परब्रह्मेव सर्वत्र व्याप्ताश्चान्तरगोचराः}% ७४

\fourlineindentedshloka
{तं राजपुत्रं शितबाणकोटिभि-}
{र्व्याप्तं विधायाशु जगर्ज विक्रमी}
{संहर्षयन्स्वीयगणान्परान्महान्}
{कुर्वन्हृदा शून्यतमान्गतासुकान्}% ७५

\fourlineindentedshloka
{स राजपुत्रः शितसायकव्रजैः}
{सम्पूर्णमात्मानमवेक्ष्य रोषितः}
{जग्राह शस्त्राणि दुरन्तविक्रमो}
{धनुश्च धुन्वन्भुजदण्डयोर्महान्}% ७६

\twolineshloka
{चकर्त सर्वाण्यस्त्राणि शस्त्राणि च महाबलः}
{रोषताम्रेक्षणो मुञ्चञ्छरान्वैरिविदारिणः}% ७७

\twolineshloka
{तच्छस्त्रजालं निर्धूय राजपुत्रो जगाद तम्}
{क्षमस्वैकं प्रहारं मे यदि शूरोसि मारिष}% ७८

\twolineshloka
{यद्यनेन भवन्तं वै रथाच्चेत्पातयामि न}
{प्रतिज्ञां शृणु मे वीर मम गर्वेण निर्मिताम्}% ७९

\twolineshloka
{वेदं निन्दन्ति ये मत्ता हेतुवादविचक्षणाः}
{तेषां पापं ममैवास्तु नरकार्णवमज्जकम्}% ८०

\twolineshloka
{इत्युक्त्वा बाणमासाद्य कोदण्डे कालसन्निभम्}
{ज्वालामालाकुलं तीक्ष्णं निषङ्गादुद्धृतं वरम्}% ८१

\twolineshloka
{स मुक्तो नृपवर्येण हृदि लक्ष्यीकृतः शरः}
{जगाम तरसा तं वै कालानलसमप्रभः}% ८२

\twolineshloka
{प्रतापाग्र्यः शरं दृष्ट्वा स्वपातनसमुद्यतम्}
{बाणान्धनुष्यथाधत्त शरच्छेदायवै शितान्}% ८३

\twolineshloka
{स बाणः सर्वबाणांस्तांश्छिन्दन्मध्यत एव हि}
{जगाम वै प्रतापाग्र्यहृदयं धैर्यसंयुतम्}% ८४

\twolineshloka
{संलग्नो हृदि नालीकः प्रविवेश तदन्तरम्}
{राजाकृतप्रहारस्तु पपात धरणीतले}% ८५

\twolineshloka
{मूर्च्छितं चेतनाहीनं रथोपस्थाद्गतं भुवि}
{सारथिस्तं समादायापोवाह रणमण्डलात्}% ८६

\twolineshloka
{हाहाकारोमहानासीद्बलं भग्नं गतं ततः}
{यत्र शत्रुघ्ननामासौ वीरकोटिपरीवृतः}% ८७

\twolineshloka
{राजात्मजो जयं प्राप्य प्रतापाग्र्यं विजित्य सः}
{प्रतीक्षां तु चकारास्य शत्रुघ्नस्य च भूपतेः}% ८८

{॥इति श्रीपद्मपुराणे पातालखण्डे शेषवात्स्यायनसंवादे रामाश्वमेधे राजपुत्रयुद्धकथनं नाम त्रयोविंशोऽध्यायः॥२३॥}

\dnsub{चतुर्विंशतितमोऽध्यायः}%\resetShloka

\uvacha{शेष उवाच}

\twolineshloka
{शत्रुघ्नस्तु क्रुधाविष्टो दन्तान्दन्तैर्विनिष्पिषन्}
{हस्तौ धुन्वंल्लेलिहानमधरं जिह्वया सकृत्}% १

\twolineshloka
{पुनः पुनस्तान्पप्रच्छ केनाश्वो नीयते मम}
{प्रतापाग्र्यः केन जितः सर्वशूरशिरोमणिः}% २

\twolineshloka
{सेवकास्ते तदा प्रोचुर्दमनो नाम शत्रुहन्}
{सुबाहुजः प्रतापाग्र्यं जितवान्हयमाहरत्}% ३

\twolineshloka
{इति श्रुत्वा हयं नीतं दमनेन स्ववैरिणा}
{आजगाम स वेगेन यत्राभूद्रणमण्डलम्}% ४

\twolineshloka
{तत्रापश्यत्स शत्रुघ्नो गजान्दीर्णकपोलकान्}
{पर्वतानिव रक्तोदे मज्जमानान्मदोद्धतान्}% ५

\twolineshloka
{हयास्तत्र निजारोहकर्तृभिः सहिताः क्षताः}
{मृता वीरेण ददृशे शत्रुघ्नेन सुकोपिना}% ६

\twolineshloka
{नरान्रथान्गजान्भग्नान्वीक्षमाणः स शत्रुहा}
{अतीव चुक्रुधे यद्वत्प्रलये प्रलयार्णवः}% ७

\twolineshloka
{पुरतो दमनं वीक्ष्य हयनेतारमुद्भटम्}
{प्रतापाग्र्यस्य जेतारं तृणीकृत्य निजं बलम्}% ८

\twolineshloka
{तदा राजा प्रत्युवाच योधान्कोपाकुलेक्षणः}
{कोऽसौ दमन जेताऽत्र सर्वशस्त्रास्त्रधारकः}% ९

\twolineshloka
{यो वै राजसुतं वीरं रणकर्मविशारदम्}
{जेष्यत्यस्त्रेण निर्भीतः सज्जीभूतो भवत्वयम्}% १०

\twolineshloka
{इति वाक्यं समाकर्ण्य पुष्कलः परवीरहा}
{दमनं जेतुमुद्युक्तो जगाद वचनं त्विदम्}% ११

\twolineshloka
{स्वामिन्क्वायं दमनकः क्व तेऽपरिमितं बलम्}
{जेष्येऽहं त्वत्प्रतापेन गच्छाम्येष महामते}% १२

\twolineshloka
{सेवके मयि युद्धाय स्थिते कैर्नीयते हयः}
{रघुनाथप्रतापोऽयं सर्वं कृत्यं करिष्यति}% १३

\twolineshloka
{स्वामिञ्छृणु प्रतिज्ञां मे तव मोदप्रदायिनीम्}
{विजेष्ये दमनं युद्धे रणकर्मविचक्षणम्}% १४

\twolineshloka
{रामचन्द्रपदाम्भोजमध्वास्वादवियोगिनाम्}
{यदघं तु भवेत्तन्मे दमनं न जयेयदि}% १५

\twolineshloka
{पुत्रो यो मातृपादान्यत्तीर्थं मत्वा तया सह}
{विरुद्ध्येत्तत्तमो मह्यं न जयेदमनं यदि}% १६

\twolineshloka
{अद्य मद्बाणनिर्भिन्न महोरस्को नृपाङ्गजः}
{अलङ्करोतु प्रधने भूतलं शयनेन हि}% १७

\uvacha{शेष उवाच}

\twolineshloka
{इति प्रतिज्ञामाकर्ण्य पुष्कलस्य रघूद्वहः}
{जहर्ष चित्ते तेजस्वी निदिदेश रणं प्रति}% १८

\twolineshloka
{आज्ञप्तोऽसौ ययौ सैन्यैर्बहुभिः परिवारितः}
{यत्रास्ते दमनो राजपुत्रः शूरकुलोद्भवः}% १९

\twolineshloka
{दमनोऽपि तमाज्ञाय ह्यागतं रणमण्डले}
{प्रत्युज्जगाम वीराग्र्यः स्वसैन्यपरिवारितः}% २०

\twolineshloka
{अन्योन्यं तौ सम्मिलितौ रथस्थौ रथशोभिनौ}
{समरे शक्रदैत्यौ किं युद्धार्थं रणमागतौ}% २१

\twolineshloka
{उवाच पुष्कलस्तं वै राजपुत्रं महाबलम्}
{राजपुत्र दमनक मां जानीहि समागतम्}% २२

\twolineshloka
{स प्रतिज्ञं तु युद्धाय भरतात्मजमुद्भटम्}
{पुष्कलेन स्वनाम्ना च लक्षितं विद्धिसत्तम}% २३

\twolineshloka
{रघुनाथपदाम्भोज नित्यसेवामधुव्रतम्}
{त्वां जेष्ये शस्त्रसङ्घेनसज्जीभव महामते}% २४

\twolineshloka
{इति वाक्यं समाकर्ण्य दमनः परवीरहा}
{प्रत्युवाच हसन्वाग्मी निर्भयोद्दृष्टविक्रमः}% २५

\twolineshloka
{सुबाहुपुत्रं दमनं पितृभक्ति हृताघकम्}
{विद्धि मामश्वनेतारं शत्रुघ्नस्य महीपतेः}% २६

\twolineshloka
{जयो दैवविसृष्टोऽयं यस्य चालङ्करिष्यति}
{स प्राप्नोति निरीक्षस्व बलं मे रणमूर्धनि}% २७

\twolineshloka
{इत्युक्त्वा स शरं चापं विधायाकर्णपूरितम्}
{मुमोच बाणान्निशितान्वैरिप्राणापहारिणः}% २८

\twolineshloka
{ते बाणास्त्वाविलीभूताश्छादयामासुरम्बरम्}
{सूर्यभानुप्रभा यत्र बाणच्छायानिवारिता}% २९

\twolineshloka
{गजानां कटभित्त्योघे लग्ना सायकसन्ततिः}
{अलङ्करोति धातूनां रागा इव विचित्रिताः}% ३०

\twolineshloka
{पतितास्तत्र दृश्यन्ते नरा वाहा गजा रथाः}
{शरव्रातेन नृपतेः सुतेन परिताडिताः}% ३१

\twolineshloka
{तद्विक्रान्तं समालोक्य पुष्कलः परवीरहा}
{शराणां छायया व्याप्तं रणमण्डलमीक्ष्य च}% ३२

\twolineshloka
{शरासने समाधत्त बाणं वह्न्यभिमन्त्रितम्}
{आचम्य सम्यग्विधिवन्मोचयामास सायकम्}% ३३

\twolineshloka
{ततोऽग्निप्रादुरभवत्तत्र सङ्ग्राममूर्धनि}
{ज्वालाभिर्विलिहन्व्योम प्रलयाग्निरिवोत्थितः}% ३४

\twolineshloka
{ततोऽस्य सैन्यं निर्दग्धं त्रासं प्राप्तं रणाङ्गणे}
{पलायनपरं जातं वह्निज्वालाभिपीडितम्}% ३५

\twolineshloka
{छत्राणि तु प्रदग्धानि चन्द्राकाराणि धन्विनाम्}
{दृश्यन्ते जातरूपाभ कान्तिधारीणि तत्र ह}% ३६

\twolineshloka
{हया दग्धाः पलायन्ते केसरेषु च वैरिणाम्}
{रथा अपि गता दाहं सुकूबरसमन्विताः}% ३७

\twolineshloka
{मणिमाणिक्यरत्नानि वहन्तः करभास्ततः}
{पलायन्ते दहनभू ज्वालामालाभिपीडिताः}% ३८

\twolineshloka
{कुत्रचिद्दन्तिनो नष्टाः कुत्रचिद्धयसादिनः}
{कुत्रचित्पत्तयो नष्टा वह्निदग्धकलेवराः}% ३९

\twolineshloka
{शराः सर्वे नृपसुतप्रमुक्ताः प्रलयं गताः}
{आशुशुक्षणिकीलाभिर्भस्मीभूताः समन्ततः}% ४०

\twolineshloka
{तदा स्वसैन्ये दग्धे च दमनो रोषपूरितः}
{सर्वास्त्रवित्तच्छान्त्यर्थं वारुणास्त्रमथा ददे}% ४१

\twolineshloka
{वारुणं वह्निशान्त्यर्थं मुक्तं तेन महीभृता}
{आप्लावयद्बलं तस्य रथवाजिसमाकुलम्}% ४२

\twolineshloka
{रथा विप्लावितास्तोये दृश्यन्ते परिपन्थिनाम्}
{गजाश्चापि परिप्लुष्टाः स्वीयाः शान्तिमुपागताः}% ४३

\twolineshloka
{वह्निश्च शान्तिमगमदग्न्यस्त्र परिमोचितः}
{शान्तिमाप बलं स्वीयं वह्निज्वालाभिपीडितम्}% ४४

\twolineshloka
{कम्पिताः शीततोयेन सीत्कुर्वन्ति च वैरिणः}
{करकावृष्टिभिः क्षिप्ता वायुना च प्रपीडिताः}% ४५

\twolineshloka
{तदा स्वबलमालोक्य तोयपूरेण पीडितम्}
{कम्पितं क्षुभितं नष्टं वारुणेन विनिर्हृतम्}% ४६

\twolineshloka
{तदातिकोपताम्राक्षः पुष्कलो भरतात्मजः}
{वायव्यास्त्रं समाधत्त धनुष्येकं महाशरम्}% ४७

\twolineshloka
{ततो वायुर्महानासीद्वायव्यास्त्रप्रचोदितः}
{नाशयामास वेगेन घनानीकमुपस्थितम्}% ४८

\twolineshloka
{वायुना स्फालिता नागाः परस्परसमाहताः}
{अश्वाश्च संहतान्योन्यं स्वस्वारोहसमन्विताः}% ४९

\twolineshloka
{नराः प्रभञ्जनोद्धूता मुक्तकेशा निरोजसः}
{पतन्तोऽत्र समीक्ष्यन्ते वेताला इव भूगताः}% ५०

\twolineshloka
{वायुना स्वबलं सर्वं परिभूतं विलोक्य सः}
{राजपुत्रः पर्वतास्त्रं धनुष्येवं समादधे}% ५१

\twolineshloka
{तदा तु पर्वताः पेतुर्मस्तकोपरि युध्यताम्}
{वायुः सञ्च्छादितस्तैस्तु न प्रचक्राम कुत्रचित्}% ५२

\twolineshloka
{पुष्कलो वज्रसंज्ञं तु समाधत्त शरासने}
{वज्रेण कृत्तास्ते सर्वे जाताश्च तिलशः क्षणात्}% ५३

\twolineshloka
{वज्रं नगान्रजः शेषान्कृत्वा बाणाभिमन्त्रितम्}
{राजपुत्रोरसि प्रोच्चैः पपात स्वनवद्भृशम्}% ५४

\twolineshloka
{सत्वाकुलितचेतस्को हृदि विद्धः क्षतो भृशम्}
{विव्यथे बलवान्वीरः कश्मलं परमाप सः}% ५५

\twolineshloka
{तं वै कश्मलितं दृष्ट्वा सारथिर्नयकोविदः}
{अपोवाह रणात्तस्मात्क्रोशमात्रं नरेन्द्रजम्}% ५६

\twolineshloka
{ततो योधा राजसूनोः प्रणष्टाः प्रपलायिताः}
{गत्वा पुरीं समाचख्युः कश्मलस्थं नृपात्मजम्}% ५७

\twolineshloka
{पुष्कलो जयमाप्यैवं रणमूर्धनि धर्मवित्}
{न प्रहर्तुं पुनः शक्तो रघुनाथवचः स्मरन्}% ५८

\twolineshloka
{ततो दुन्दुभिनिर्घोषो जयशब्दो महानभूत्}
{साधुसाध्विति वाचश्च प्रावर्तन्त मनोहराः}% ५९

\twolineshloka
{हर्षं प्राप स शत्रुघ्नो जयिनं वीक्ष्य पुष्कलम्}
{प्रशशंस सुमत्यादि मन्त्रिभिः परिवारितः}% ६०

{॥इति श्रीपद्मपुराणे पातालखण्डे शेषवात्स्यायनसंवादे रामाश्वमेधे पुष्कलविजयो नाम चतुर्विंशतितमोऽध्यायः॥२४॥}

\dnsub{पञ्चविंशोऽध्यायः}%\resetShloka

\uvacha{शेष उवाच}

\twolineshloka
{अथ वीक्ष्य भटान्निजान्नृपो रुधिरौघेण परिप्लुताङ्गकान्}
{सुखमाप न वै शुशोच तान्परिपप्रच्छ सुतस्य चेष्टितम्}% १

\twolineshloka
{गदताखिलकर्म तस्य वै स कथं चाहरदश्ववर्यकम्}
{कथयन्तु पुनः कियद्बलं बत वीराः कति योद्धुमागताः}% २

\twolineshloka
{अथ शत्रुबलोन्मुखः कथं मम वीरो दमनो रणं व्यधात्}
{विजयं च विधाय दुर्जयं किल वीरं बत कोऽप्यशातयत्}% ३

\twolineshloka
{इत्याकर्ण्य वचो राज्ञः प्रत्यूचुस्तेऽस्य सेवकाः}
{क्षतजेन परिक्लिन्न गात्रवस्त्रादिधारिणः}% ४

\twolineshloka
{राजन्नश्वं समालोक्य पत्रचिह्नाद्यलङ्कृतम्}
{ग्राहयामास गर्वेण तृणीकृत्य रघूत्तमम्}% ५

\twolineshloka
{ततो हयानुगः प्राप्तः स्वल्पसैन्यसमावृतः}
{तेन साकमभूद्युद्धं तुमुलं रोमहर्षणम्}% ६

\twolineshloka
{तं मूर्च्छितं ततः कृत्वा तव पुत्रः स्वसायकैः}
{यावत्तिष्ठत्यथायातः शत्रुघ्नः स्वबलैर्वृतः}% ७

\twolineshloka
{ततो युद्धं महदभूच्छस्त्रास्त्रपरिबृंहितम्}
{बहुशो जयमापेदे तव पुत्रो महाबलः}% ८

\twolineshloka
{इदानीं तेन मुक्त्वास्त्रं शत्रुघ्नभ्रातृसूनुना}
{मूर्च्छितः प्रधने राजन्कृतो वीरः सुतस्तव}% ९

\twolineshloka
{इति वाक्यं समाकर्ण्य रोषशोकसमन्वितः}
{स्थगिताङ्ग इवासीत्स समुद्र इव पर्वणि}% १०

\twolineshloka
{उवाच सेनाधिपतिं रोषप्रस्फुरिताधरः}
{दन्तैर्दताँल्लिहन्नोष्ठं जिह्वया शोककर्शितः}% ११

\twolineshloka
{सेनापते कुरुष्वारान्मम सेनां तु सज्जिताम्}
{योत्स्ये रामस्य सुभटैर्ममपुत्रोपघातकैः}% १२

\twolineshloka
{अद्याहं मम पुत्रस्य दुःखदं निशितैः शरैः}
{पातयिष्ये यदि ह्येनं रक्षितापि महेश्वरः}% १३

\twolineshloka
{सेनापतिरिदं वाक्यं प्रोक्तं सुभुजभूपतेः}
{निशम्य च तथा कृत्वा सज्जीभूतो भवत्स्वयम्}% १४

\twolineshloka
{राज्ञे निवेदयामास ससज्जां चतुरङ्गिणीम्}
{सेनां कालबलप्रख्यां हतदुर्जनकोटिकाम्}% १५

\twolineshloka
{श्रुत्वा सेनापतेर्वाक्यं सुबाहुः परवीरहा}
{निर्जगाम ततो यत्र शत्रुघ्नः स्वसुतार्दनः}% १६

\twolineshloka
{कुञ्जरैश्च मदोन्मत्तैर्हयैश्चापि मनोजवैः}
{रथैश्च सर्वशस्त्रास्त्रपूरितै रिपुजेतृभिः}% १७

\twolineshloka
{भूश्चकम्पे तदा तत्र सैन्यभारेण भूरिणा}
{सम्मर्दः सुमहानासीत्तत्र सैन्ये विसर्पति}% १८

\twolineshloka
{राजानं निर्गतं दृष्ट्वा रथेन कनकाङ्गिना}
{शत्रुघ्नबलमुद्युक्तं सर्ववैरिप्रहारकम्}% १९

\twolineshloka
{सुकेतुस्तस्य वै भ्राता गदायुद्धविशारदः}
{रथेनाश्वा जगामायं सर्वशस्त्रास्त्रपूरितः}% २०

\twolineshloka
{चित्राङ्गस्तु सुतो राज्ञः सर्वयुद्धविचक्षणः}
{जगाम स्वरथेनाशु शत्रुघ्नबलमुन्मदम्}% २१

\twolineshloka
{तस्यानुजो विचित्राख्यो विचित्ररणकोविदः}
{ययौ रथेन हैमेन भ्रातृदुःखेन पीडितः}% २२

\twolineshloka
{अन्ये शूरा महेष्वासाः सर्वशस्त्रास्त्रकोविदाः}
{ययुर्नृपसमादिष्टाः प्रधनं वीरपूरितम्}% २३

\twolineshloka
{राजा सुबाहुः संरोषादागतः प्रधनाङ्गणे}
{विलोकयामास सुतं मूर्च्छितं शरपीडितम्}% २४

\twolineshloka
{रथोपस्थस्थितं मूढं स्वसुतं दमनाभिधम्}
{वीक्ष्य दुःखं मुहुः प्राप वीजयामास पल्लवैः}% २५

\twolineshloka
{जलेन सिक्तः संस्पृष्टो राज्ञा कोमलपाणिना}
{संज्ञामाप शनैर्वीरो दमनः परमास्त्रवित्}% २६

\twolineshloka
{उत्थितः क्व धनुर्मेऽस्ति क्व पुष्कल इतो गतः}
{संसज्य समरं त्यक्त्वा मद्बाणव्रणपीडितः}% २७

\twolineshloka
{इति वाक्यं समाकर्ण्य सुबाहुः पुत्रभाषितम्}
{परमं हर्षमापेदे परिरभ्य सुतं स्वकम्}% २८

\twolineshloka
{दमनो वीक्ष्य जनकं नृपं नम्रशिरोधरः}
{पपात पादयोर्भक्त्या क्षतदेहोऽस्त्रराजिभिः}% २९

\twolineshloka
{स्वसुतं रथसंस्थं तु विधाय नृपतिः पुनः}
{जगाद सेनाधिपतिं रणकर्मविशारदः}% ३०

\twolineshloka
{व्यूहं रचय सङ्ग्रामे क्रौञ्चाख्यं रिपुदुर्जयम्}
{यमाविश्य जये सैन्यं शत्रुघ्नस्य महीपतेः}% ३१

\fourlineindentedshloka
{तद्वाक्यमाकर्ण्य सुबाहुभूपतेः}
{क्रौञ्चाख्यसद्व्यूहविशेषमादधात्}
{यन्नो विशन्ते सहसा रिपोर्गणा}
{महाबलाः शस्त्रसमूहधारिणः}% ३२

\twolineshloka
{मुखे सुकेतुस्तस्यासीद्गले चित्राङ्गसंज्ञकः}
{पक्षयो राजपुत्रौ द्वौ पुच्छे राजा प्रतिष्ठितः}% ३३

\twolineshloka
{मध्ये सैन्यं महत्तस्य चतुरङ्गैस्तु शोभितम्}
{कृत्वा न्यवेदयद्राज्ञे क्रौञ्चव्यूहं विचित्रितम्}% ३४

\twolineshloka
{राजा दृष्ट्वा सुसन्नद्धं क्रौञ्चव्यूहं सुनिर्मितम्}
{रणाय स्वमतिं चक्रे शत्रुघ्नकटके स्थितैः}% ३५

{॥इति श्रीपद्मपुराणे पातालखण्डे शेषवात्स्यायनसंवादे रामाश्वमेधे सुबाहुसैन्यसमागमो नाम पञ्चविंशोऽध्यायः॥२५॥}

\dnsub{षड्विंशतितमोऽध्यायः}%\resetShloka

\uvacha{शेष उवाच}

\twolineshloka
{शत्रुघ्नस्तद्बलं दृष्ट्वा भीषणाकृतिमेघवत्}
{हस्त्यश्वरथपादातैर्बहुभिः परिवारितम्}% १

\twolineshloka
{सुमतिं प्रत्युवाचेदं वचोगम्भीरशब्दयुक्}
{नानावाक्यविचारज्ञैः पण्डितैः परिसेवितः}% २

\uvacha{शत्रुघ्न उवाच}

\twolineshloka
{सुमते कस्य नगरं प्राप्तो मे हयसत्तमः}
{बलमेतन्निरीक्षेहं पयोदधितरङ्गवत्}% ३

\twolineshloka
{कस्यैतद्बलमुद्धर्षं चतुरङ्गसमन्वितम्}
{पुरतो भाति युद्धाय समुपस्थितमादरात्}% ४

\twolineshloka
{एतत्सर्वं समाचक्ष्व यथावत्पृच्छतो मम}
{यज्ज्ञात्वा युद्धसंस्थायै निर्दिशामि स्वकान्भटान्}% ५

\twolineshloka
{इति वाक्यं समाकर्ण्य सुमतिः शुभबुद्धिमान्}
{उवाच वचनं प्रीतः शत्रुघ्नं वैरितापनम्}% ६

\uvacha{सुमतिरुवाच}

\twolineshloka
{चक्राङ्का नगरी राजन्वर्तते सविधे शुभा}
{यस्यां सन्ति नराः पापरहिता विष्णुभक्तितः}% ७

\twolineshloka
{तस्याः पुर्याः पतिरयं सुबाहुर्धर्मवित्तमः}
{तवायं पुरतो भाति पुत्रपौत्रसमावृतः}% ८

\twolineshloka
{स्वदारनिरतो नित्यं परदारपराङ्मुखः}
{विष्णोः कथास्य कर्णस्थाना परार्थप्रकाशिनी}% ९

\twolineshloka
{परस्वं न समादत्ते षष्ठांशादधिकं नृपः}
{ब्राह्मणा विष्णुभक्त्यैव पूज्यन्ते तेन धर्मिणा}% १०

\twolineshloka
{नित्यं सेवारतो विष्णुपादपद्ममधुव्रतः}
{एष स्वधर्मनिरतः परधर्मपराङ्मुखः}% ११

\twolineshloka
{एतस्य बलतुल्यं हि न वीराणां बलं क्वचित्}
{पुत्रस्य पतनं श्रुत्वा रोषशोकसमाकुलः}% १२

\twolineshloka
{चतुरङ्गसमेतोऽयं युद्धाय समुपस्थितः}
{तवापि वीरा बहवो लक्ष्मीनिधिमुखा अमून्}% १३

\twolineshloka
{जेष्यन्ति शस्त्रसङ्घेन निर्दिशाशु परं हि तान्}
{शत्रुघ्नस्तद्वचः श्रुत्वा प्रोवाच स्वभटान्वरान्}% १४

\twolineshloka
{रणप्राप्तिभवोद्धर्षपूरपूरितमानसान्}
{क्रौञ्चव्यूहोऽद्य रचितः सुबाहुपरिसैनिकैः}% १५

\twolineshloka
{मुखपक्षस्थिता योधास्तान्को भेत्स्यति शस्त्रवित्}
{यस्य भेदे निजा शक्तिर्यो वीर विजयोद्यतः}% १६

\twolineshloka
{स गृह्णातु मदीयाद्धि पाणिपद्माच्च वीटकम्}
{तदा लक्ष्मीनिधिर्वीरो जग्राह क्रौञ्चभेदने}% १७

\twolineshloka
{सर्वशस्त्रास्त्रविद्वीरैर्बहुभिः परिवारितः}
{उवाच वचनं राजन्यास्येऽहं क्रौञ्चभेदने}% १८

\twolineshloka
{भार्गवः पूर्वमेवासीत्क्रौञ्चभेत्ता तथा ह्यहम्}
{तथान्यं वीरमावोचत्कोऽस्य सार्धं गमिष्यति}% १९

\twolineshloka
{पुष्कलः पृष्ठतस्तस्य यातुं चक्रे मतिं ततः}
{रिपुतापो नीलरत्न उग्राश्वो वीरमर्दनः}% २०

\twolineshloka
{सर्वे शत्रुघ्नसन्देशाद्ययुस्तत्क्रौञ्चभेदने}
{शत्रुघ्नोऽपि रथस्थश्च सर्वायुधधरः परः}% २१

\twolineshloka
{पृष्ठतोऽस्य परीयाय बहुभिः सैनिकैर्वृतः}
{तदा प्रचलितौ दृष्टावन्योन्यबलवारिधी}% २२

\twolineshloka
{प्रलयं कर्तुमुद्युक्तौ जगतः सुतरङ्गिणौ}
{तदा भेर्यः समाजघ्नुरुभयोः सेनयोर्दृढाः}% २३

\twolineshloka
{रणभेर्यः शङ्खनादाः श्रूयन्ते तत्र तत्र ह}
{हेषन्ते वाजिनस्तत्र गर्जन्ति द्विरदा भृशम्}% २४

\twolineshloka
{हुं हुं कुर्वन्ति वीराग्र्या नदन्ति रथनेमयः}
{तत्र प्रकुपिताः शूराः सुबाहुबलदर्पिताः}% २५

\twolineshloka
{छिन्धि भिन्धीति भाषन्तो दृश्यन्ते बहवो रणे}
{एवम्भूते रणोद्युक्ते सैन्ये शत्रुघ्नवैरिणोः}% २६

\onelineshloka*
{मुखसंस्थं सुकेतुं तं लक्ष्मीनिधिरुवाच ह}

\uvacha{लक्ष्मीनिधिरुवाच}

\onelineshloka
{जनकस्य सुतं विद्धि लक्ष्मीनिधिरिति स्मृतम्}% २७

\twolineshloka
{सर्वशस्त्रास्त्रकुशलं सर्वयुद्धविशारदम्}
{मुञ्चाश्वं रामचन्द्रस्य सर्वदानवदंशितुः}% २८

\twolineshloka
{नोचेन्मद्बाणनिर्भिन्नो यास्यसे यमसादनम्}
{इति ब्रुवन्तं वीराग्र्यं सुकेतुः सहसा त्वरन्}% २९

\twolineshloka
{सज्यं चापं विधायाशु बाणान्मुञ्चन्स्थिरोऽभवत्}
{ते बाणाः शितपर्वाणः स्वर्णपुङ्खाः समन्ततः}% ३०

\onelineshloka*
{दृश्यन्ते व्यापिनस्तत्र रणमध्ये सुदुर्भराः}

\fourlineindentedshloka
{तद्बाणजालं तरसा निहत्य}
{लक्ष्मीनिधिश्चापमथा ततज्यम्}
{विधाय तस्योरसि बाणषट्कं}
{मुमोच तीक्ष्णं शितपर्वशोभितम्}% ३१

\twolineshloka
{तद्बाणाः सुभुजभ्रातुर्हृदयं संविदार्य च}
{गतास्ते भुवि दृश्यन्ते रुधिराक्ता मलीमसाः}% ३२

\twolineshloka
{तद्बाणभिन्नहृदयः सुकेतुः कोपपूरितः}
{जघानशरविंशत्या तीक्ष्णया नतपर्वया}% ३३

\twolineshloka
{उभौ बाणविभिन्नाङ्गावुभौ क्षतजविप्लुतौ}
{सैनिकैः परिदृश्यन्ते किंशुकाविव पुष्पितौ}% ३४

\twolineshloka
{मुञ्चन्तौ बाणकोटीश्च सन्दधन्तौ त्वरा शरान्}
{न केनापि विलक्ष्येते लघुहस्तौ महाबलौ}% ३५

\twolineshloka
{कुण्डलीकृत सच्चापौ वर्षन्तौ बाणधारया}
{नवाम्बुदाविव दिवि शक्रनिर्देशकारिणौ}% ३६

\twolineshloka
{तयोर्बाणा गजान्वाहान्नराञ्छूरान्विमस्तकान्}
{कुर्वन्तः केवलं दृष्टा न च सन्धानमोक्षयोः}% ३७

\twolineshloka
{पृथिवी सुभटैः पूर्णा सकिरीटैः सकुण्डलैः}
{धनुर्बाणकरै रोषसन्दष्टाधरयुग्मकैः}% ३८

\twolineshloka
{तयोः प्रयुद्ध्यतोर्दर्पात्सर्वशस्त्रास्त्रवेदिनोः}
{युद्धं समभवद्घोरं देवविस्मापनं महत्}% ३९

\twolineshloka
{सम्मर्दोऽभवदत्यन्तं वीरकोटिविदारणः}
{न केनचित्क्वचिद्दृष्टं शरजालान्तरेऽम्बरम्}% ४०

\twolineshloka
{तस्मिंस्तु समये लक्ष्मीनिधिर्वीरोऽरिमर्दनः}
{बाणांश्चापे समाधत्त वसुसङ्ख्यान्दृढाञ्छितान्}% ४१

\twolineshloka
{चतुर्भिस्तुरगान्वीरः सुकेतोरनयत्क्षयम्}
{एकेन ध्वजमत्युग्रं चिच्छेद तरसा हसन्}% ४२

\twolineshloka
{एकेन सारथेः कायाच्छिरोभूमावपातयत्}
{एकेन चापं सगुणमच्छिनद्रोषपूरितः}% ४३

\twolineshloka
{एकेन हृदि विव्याध सुकेतोर्वेगवान्नृपः}
{तत्कर्माद्भुतमुद्वीक्ष्य वीरा विस्मयमाययुः}% ४४

\twolineshloka
{सच्छिन्नधन्वा विरथो हताश्वो हतसारथिः}
{महतीं स गदां धृत्वा योद्घुकामोऽभ्युपेयिवान्}% ४५

\twolineshloka
{तमायान्तं समालक्ष्य गदायुद्धविशारदम्}
{महत्या गदया युक्तं रथादवततार सः}% ४६

\twolineshloka
{गदामादाय महतीं सर्वायसविनिर्मिताम्}
{जातरूपविचित्राङ्गीं सर्वशोभापुरस्कृताम्}% ४७

\twolineshloka
{लक्ष्मीनिधिर्भृशं क्रुद्धः सुकेतोर्वक्षसि त्वरन्}
{ताडयामास हृदये गदां वज्राग्निसन्निभाम्}% ४८

\twolineshloka
{गदया ताडितो वीरो नाकम्पत महामुने}
{मदोन्मत्तो यथा दन्ती बालेन स्रग्भिराहतः}% ४९

\twolineshloka
{उवाच तं स वीराग्र्यो नृपं लक्ष्मीनिधिं तदा}
{सहस्वैकं प्रहारं मे यदि शूरः परन्तप}% ५०

\twolineshloka
{इत्युक्त्वा ताडयामास ललाटे गदया भृशम्}
{गदया ताडितो भालेऽसृग्वमन्कुपितो भृशम्}% ५१

\twolineshloka
{मूर्ध्नि तं ताडयामास गदया कालरूपया}
{सुकेतुरपि तं स्कन्धे ताडयामास धर्मवित्}% ५२

\twolineshloka
{एवं भृशं प्रकुपितौ गदायुद्धविशारदौ}
{गदायुद्धं प्रकुर्वाणौ परस्परजयैषिणौ}% ५३

\twolineshloka
{अन्योन्याघातविमतौ परस्परवधोद्यतौ}
{न कोपि तत्र हीयेत न को जीयेत संयुगे}% ५४

\twolineshloka
{मूर्ध्नि भाले तथा स्कन्धे हृदि गात्रेषु सर्वतः}
{रुधिरौघ परिक्लिन्नौ महाबलपराक्रमौ}% ५५

\twolineshloka
{तदा लक्ष्मीनिधिः क्रुद्धो गदामुद्यम्य वेगवान्}
{जगाम प्रबलं हन्तुं हृदि राजानुजं बली}% ५६

\twolineshloka
{तमायान्तमथालोक्य स्वगदां महतीं दधत्}
{ययौ तं तरसा हन्तुं राजभ्राता बलाद्बलम्}% ५७

\twolineshloka
{गदां तेन विनिक्षिप्तां स्वकरे धृतवानयम्}
{तयैव गदया तस्य हृदि जघ्ने महाबलः}% ५८

\twolineshloka
{स्वगदां तेन वै नीतां दृष्ट्वा लक्ष्मीनिधिर्नृपः}
{बाहुयुद्धेन तं योद्धुमियेष बलवत्तमम्}% ५९

\twolineshloka
{तदा राजानुजः क्रुद्धो बाहुभ्यामुपगृह्य तम्}
{युयुधे सर्वयुद्धस्य ज्ञातावीरेषु सत्तमः}% ६०

\twolineshloka
{तदा लक्ष्मीनिधिस्तस्य हृदि जघ्ने स्वमुष्टिना}
{तदा सोपि शिरस्येनं मुष्टिमुद्यम्य चाहनत्}% ६१

\twolineshloka
{मुष्टिभिर्वज्रसङ्काशैस्तलस्फोटैश्च दारुणैः}
{अन्योन्यं जघ्नतुः क्रुद्धौ सन्दष्टाधरपल्लवौ}% ६२

\twolineshloka
{मुष्टी मुष्टि दन्ता दन्ति कचा कचि नखा नखि}
{उभयोरभवद्युद्धं तुमुलं रोमहर्षणम्}% ६३

\twolineshloka
{तदा प्रकुपितो भ्राता नृपतेश्च रणे नृपम्}
{गृहीत्वा भ्रामयित्वाथ पातयामास भूतले}% ६४

\twolineshloka
{लक्ष्मीनिधिः करे गृह्य तं नृपानुजमुच्चकैः}
{भ्रामयित्वा शतगुणं गजोपस्थे जघान तम्}% ६५

\twolineshloka
{स तदा पतितो भूमौ संज्ञां प्राप्य क्षणादनु}
{तथैव भ्रामयामास व्योम्नि वेगेन विक्रमी}% ६६

\twolineshloka
{एवं प्रयुध्यमानौ तौ बाहुयुद्धं गतौ पुनः}
{पादे पादं करे पाणिं हृदि हृद्वदने मुखम्}% ६७

\twolineshloka
{एवं परस्परं श्लिष्टौ परस्परवधैषिणौ}
{उभावपि पराक्रान्तावुभावपि मुमूर्च्छतुः}% ६८

\twolineshloka
{तद्दृष्ट्वा विस्मयं प्राप्ताः प्रशशंसुः सहस्रशः}
{धन्यो लक्ष्मीनिधिर्भूपो धन्यो राजानुजो बली}% ६९

{॥इति श्रीपद्मपुराणे पातालखण्डे शेषवात्स्यायनसंवादे रामाश्वमेधे गदायुद्धं नाम षड्विंशतितमोऽध्यायः॥२६॥}

\dnsub{सप्तविंशतितमोऽध्यायः}%\resetShloka

\uvacha{शेष उवाच}

\twolineshloka
{चित्राङ्गः क्रौञ्चकण्ठस्थो रथस्थो वीरशोभितः}
{गाहयामास तत्सैन्यं वाराह इव वारिधिम्}% १

\twolineshloka
{धनुर्विस्फार्य सुदृढं मेघनादनिनादितम्}
{मुमोच बाणान्निशितान्वैरिकोटिविदाहकान्}% २

\twolineshloka
{तद्बाणभिन्नसर्वाङ्गाः शेरते सुभटा भृशम्}
{सकिरीटतनुत्राणाः सन्दष्टदशनच्छदाः}% ३

\twolineshloka
{एवं प्रवृत्ते सङ्ग्रामे ययौ योद्धुं स पुष्कलः}
{मणिचित्रितमादाय चापं वैरिप्रतापनम्}% ४

\twolineshloka
{तयोः सङ्गतयोरूपं दृश्यतेऽतिमनोहरम्}
{पुरा तारकसङ्ग्रामे स्कन्दतारकयोर्यथा}% ५

\twolineshloka
{विस्फारयन्धनुः शीघ्रं सव्यसाची तु पुष्कलः}
{ताडयामास तं क्षिप्रं शरैः सन्नतपर्वभिः}% ६

\twolineshloka
{चित्राङ्गोऽपि रुषाक्रान्तः शरासन इषूञ्छितान्}
{दधद्व्यमुञ्चद्बहुशो रणमण्डलमूर्धनि}% ७

\twolineshloka
{नादानं न च सन्धानं न मोचनमथापि वा}
{दृष्टं तावेव सन्दृष्टौ कुण्डलीकृतचापिनौ}% ८

\twolineshloka
{तदासौ पुष्कलः क्रुद्धः शराणां शतकेन तम्}
{विव्याध वक्षःस्थलके महायोद्धारमुद्भटम्}% ९

\twolineshloka
{चित्राङ्गस्ताञ्शरान्सर्वांश्चिच्छेद तिलशः क्षणात्}
{ताडयामास चाङ्गेषु पुष्कलं शितसायकैः}% १०

\twolineshloka
{पुष्कलस्तद्रथं दिव्यं भ्रामकास्त्रेण शोभिना}
{नभसि भ्रामयामास तदद्भुतमिवाभवत्}% ११

\twolineshloka
{भ्रान्त्वा मुहूर्तमात्रं तु सरथो हयसंयुतः}
{स्थितिर्लेभेतिकष्टेन सन्धृतो रणमण्डले}% १२

\twolineshloka
{स चास्य विक्रमं दृष्ट्वा चित्राङ्गः कुपितो भृशम्}
{उवाच पुष्कलं धीमान्सर्वास्त्रेषु विशारदः}% १३

\uvacha{चित्राङ्ग उवाच}

\twolineshloka
{त्वया साधुकृतं कर्म सुभटैर्युधिसम्मतम्}
{मद्रथो वाजिसंयुक्तो भ्रामितो नभसि क्षणम्}% १४

\twolineshloka
{पराक्रमं समीक्षस्व ममापि सुभटेरितम्}
{आकाशचारी तु भवान्भवत्वमरपूजितः}% १५

\twolineshloka
{इत्युक्त्वा स मुमोचास्त्रं रणे परमदारुणम्}
{धनुषा परमास्त्रज्ञः सर्वधर्मविदुत्तमः}% १६

\twolineshloka
{तेन बाणेन संविद्धः खे बभ्राम पतङ्गवत्}
{सरथः सहयः सङ्ख्ये सध्वजश्च ससारथिः}% १७

\twolineshloka
{भ्रान्त्वा सरथवर्यस्तु नभसि त्वरयान्वितः}
{यावत्स्थितिं न लभते तावन्मुक्तोऽपरः शरः}% १८

\twolineshloka
{पुनश्च परिबभ्राम रथः सूतसमन्वितः}
{तत्कर्मवीक्ष्य पुत्रस्य राज्ञो विस्मयमाप सः}% १९

\twolineshloka
{कथञ्चित्स्थितिमप्याप पुष्कलः परवीरहा}
{रथं जघान बाणैश्च ससूतहयमस्य च}% २०

\twolineshloka
{सभग्नस्यन्दनो वीरः पुनरन्यं समाश्रितः}
{सोऽपि भग्नः शरैराशु पुष्कलेन रणाङ्गणे}% २१

\twolineshloka
{पुनरन्यं समास्थाय यावदायाति सम्मुखम्}
{तावद्बभञ्ज निशितैः सायकैस्तद्रथं पुनः}% २२

\twolineshloka
{एवं दश रथा भग्ना नृपतेरात्मजस्य हि}
{पुष्कलेन तु वीरेण महासंयुगशालिना}% २३

\twolineshloka
{तदा चित्राङ्गकः सङ्ख्ये रथे स्थित्वा विचित्रिते}
{आजगाम ह वेगेन पुष्कलं प्रति योधितुम्}% २४

\twolineshloka
{पुष्कलं पञ्चभिर्बाणैस्ताडयामास संयुगे}
{तैर्बाणैर्निहतोऽत्यतं विव्यथे भरतात्मजः}% २५

\twolineshloka
{सक्रुद्धश्चापमुद्यम्य बाणान्दश शितान्महान्}
{मुमोच हृदये तस्य स्वर्णपुङ्खसुशोभितान्}% २६

\twolineshloka
{ते बाणाः पपुरेतस्य रुधिरं बहुदारुणाः}
{पीत्वा पेतुः क्षितौ कूटसाक्षिणः पूर्वजा इव}% २७

\twolineshloka
{तदा चित्राङ्गकः क्रुद्धो भल्लान्पञ्च समाददे}
{मुमोच भाले पुत्रस्य भरतस्य महौजसः}% २८

\twolineshloka
{तैर्भल्लैराहतः क्रुद्धः शरासनवरे शरम्}
{दधत्प्रतिज्ञामकरोच्चित्राङ्गनिधनं प्रति}% २९

\twolineshloka
{शृणु वीर मम क्षिप्रं प्रतिज्ञां त्वद्वधाश्रिताम्}
{तज्ज्ञात्वा सावधानेन योद्धव्यं च त्वयात्र हि}% ३०

\twolineshloka
{बाणेनानेन चेत्त्वां वै न कुर्यां प्राणवर्जितम्}
{सतीं सन्दूष्य वनितां शीलाचारसुशोभिताम्}% ३१

\twolineshloka
{यो लोकः प्राप्यते लोकैर्यमस्य वशवर्तिभिः}
{स लोको मम वै भूयात्सत्यं मम प्रतिश्रुतम्}% ३२

\twolineshloka
{इति श्रेष्ठं वचः श्रुत्वा जहास परवीरहा}
{उवाच मतिमान्वीरः पुष्कलं वचनं शुभम्}% ३३

\twolineshloka
{मृत्युर्वै प्राणिनां भाव्यः सर्वत्रैव च सर्वदा}
{तस्मान्मे निधने दुःखं नास्ति शूरशिरोमणे}% ३४

\twolineshloka
{प्रतिज्ञा या कृता वीर त्वया वीरत्वशालिना}
{सा सत्यैव पुनर्मेऽद्य श्रूयतां व्याहृतं महत्}% ३५

\twolineshloka
{त्वद्बाणं मद्वधोद्युक्तं न च्छिन्द्यां यदि चेदहम्}
{तदा प्रतिज्ञां शृणु मे सर्ववीराभिमानिनः}% ३६

\twolineshloka
{तीर्थं जिगमिषोर्यो वै कुर्यात्स्वान्तविखण्डनम्}
{एकादशीव्रतादन्यज्जानाति व्रतमुच्चकैः}% ३७

\twolineshloka
{तस्य पापं ममैवास्तु प्रतिज्ञापरिघातिनः}
{इति वाक्यमुदीर्यैव तूष्णीम्भूतो धनुर्दधे}% ३८

\twolineshloka
{तदानेन निषङ्गात्स्वादुद्धृत्य सायकं वरम्}
{कथयामास विशदं वाक्यं शत्रुवधावहम्}% ३९

\uvacha{पुष्कल उवाच}

\twolineshloka
{यदि रामाङ्घ्रियुगुलं निष्कापट्येन चेतसा}
{उपासितं मया तर्हि मम वाक्यमृतं भवेत्}% ४०

\twolineshloka
{यदि स्वमहिलां भुक्त्वा नान्यां जानामिचेतसा}
{तेन सत्येन मे वाक्यं सत्यं भवतु सङ्गरे}% ४१

\twolineshloka
{इति वाक्यमुदीर्याशु बाणं धनुषि सन्धितम्}
{कालानलोपमं वीरशिरश्छेदनमाक्षिपत्}% ४२

\twolineshloka
{तं बाणं मुक्तमालोक्य स तु राजसुतो बली}
{बाणं शरासने धत्त तीक्ष्णं कालानलोपमम्}% ४३

\twolineshloka
{तेन बाणेन सञ्छिन्नो बाणः स्ववधउद्यतः}
{हाहाकारो महानासीच्छिन्ने तस्मिञ्छरे तदा}% ४४

\twolineshloka
{परार्धं पतितं भूमौ पूर्वार्धं फलसंयुतम्}
{शिरोधरां चकर्ताशु पद्मनालमिव क्षणात्}% ४५

\twolineshloka
{तदा भूमौ पतन्तं तु दृष्ट्वा तत्तस्यसैनिकाः}
{हाहाकृत्वा भृशं सर्वे पलायनपरागताः}% ४६

\twolineshloka
{पृथ्व्यां तन्मस्तकं श्रेष्ठं सकिरीटं सकुण्डलम्}
{शुशुभेऽतीव पतितं चन्द्रबिम्बं दिवो यथा}% ४७

\twolineshloka
{तं वीक्ष्य पतितं वीरः पुष्कलो भरतात्मजः}
{व्यगाहत व्यूहमिमं सर्ववीरैकशोभितम्}% ४८

{॥इति श्रीपद्मपुराणे पातालखण्डे शेषवात्स्यायनसंवादे रामाश्वमेधे चित्राङ्गवधो नाम सप्तविंशतितमोऽध्यायः॥२७॥}

\dnsub{अष्टाविंशतितमोऽध्यायः}%\resetShloka

\uvacha{शेष उवाच}

\twolineshloka
{अथ पुत्रं समालोक्य पतितं व्यसुमुद्धतम्}
{विललाप भृशं राजा सुतदुःखेन दुःखितः}% १

\twolineshloka
{मूर्ध्नि सन्ताडयामास पाणिभ्यामतिदुःखितः}
{कम्पमानो भृशं चाश्रूण्यमुञ्चन्नयनाब्जयोः}% २

\twolineshloka
{गृहीत्वा पतितं वक्त्रं चन्द्रबिम्बमनोरमम्}
{पुष्कलेषु क्षतासृग्भिः क्लिन्नं कुण्डलशोभितम्}% ३

\twolineshloka
{कुटिलभ्रूयुगं श्रेष्ठं सन्दष्टाधरपल्लवम्}
{स चुम्बन्मुखपद्मेन विलपन्निदमब्रवीत्}% ४

\fourlineindentedshloka
{हा पुत्र वीर कथमुत्सुकचेतसं मां}
{किं नेक्षसे विशदनेत्रयुगेन शूर}
{किं मद्विनोदकथयारहितस्त्वमेव}
{रोषोदधिप्लुतमनाः किल लक्ष्यसे च}% ५

\twolineshloka
{वद पुत्र कथं मां त्वं प्रब्रूषे न हसन्पुनः}
{अमृतैर्मधुरास्वादैर्विनोदयसि पुत्रक}% ६

\twolineshloka
{शत्रुघ्नाश्वं गृहाण त्वं सितचामरशोभितम्}
{स्वर्णपत्रेण शोभाढ्यं त्यज निद्रां महामते}% ७

\twolineshloka
{एष प्रतापविशदः प्रतापाग्र्यः परन्तपः}
{धनुर्बिभ्रत्पुरो भाति पुष्कलः परवीरहा}% ८

\twolineshloka
{एनं वारय सत्तीक्ष्णैर्बाणैः कोदण्डनिर्गतैः}
{कथं त्वं रणमध्ये वै शेते वीरविमोहितः}% ९

\twolineshloka
{हस्तिनः पत्तयश्चैव रथारूढा भयार्दिताः}
{शरणं त्वां समायान्ति तानीक्षस्व महामते}% १०

\twolineshloka
{पुत्र त्वया विना सोढुं कथं शक्तो रणाङ्गणे}
{शत्रुघ्नसायकांस्तीक्ष्णांश्चण्डकोदण्डनिर्गतान्}% ११

\twolineshloka
{अतो मां तु त्वया हीनं को वा पालयितुं क्षमः}
{यदि त्यक्ष्यसि निद्रा त्वं जयायाहं क्षमस्तदा}% १२

\twolineshloka
{इत्थं विलप्य सुभृशं तताड हृदयं स्वकम्}
{बहुशः पाणिना राजा पुत्रदुःखेन दुःखितः}% १३

\twolineshloka
{तदा विचित्र दमनौ स्व स्व स्यन्दनसंस्थितौ}
{पितुश्चरणयोर्नत्वा ऊचतुः समयोचितम्}% १४

\twolineshloka
{राजन्नस्मासु जीवत्सु किं दुःखं हृदि तद्वद}
{वीराणां प्रधने मृत्युर्वाञ्च्छितो जायते महान्}% १५

\twolineshloka
{धन्योऽयं बत चित्राङ्गो यो वीर भुवि शोभते}
{सकिरीटस्तु सन्दष्टदन्तच्छदयुगः प्रभुः}% १६

\twolineshloka
{कथयाशु किमद्यैव कुर्वस्ते कार्यमीप्सितम्}
{शत्रुघ्नवाहिनीं सर्वां हन्व आवामनाथिनीम्}% १७

\twolineshloka
{अद्यैव पुष्कलं भ्रातुर्वधकारिणमाहवे}
{पातयावो रथाच्छित्त्वा शिरोमुकुटमण्डितम्}% १८

\twolineshloka
{त्यज शोकं सुदुःखार्तः कथं भासि महामते}
{आज्ञापयावां मानार्ह कुरु युद्धे मतिं तथा}% १९

\twolineshloka
{इति वाक्यं समाकर्ण्य पुत्रयोर्वीरमानिनोः}
{शोकं त्यक्त्वा महाराजो युद्धाय मतिमादधात्}% २०

\twolineshloka
{तावपि प्रतियोद्धारं वाञ्च्छन्तौ रणदुर्मदौ}
{जग्मतुः कटके शत्रोरनन्तभटपूरिते}% २१

\twolineshloka
{रिपुतापेन दमनो नीलरत्नेन चेतरः}
{युयुधाते रणे वीरौ प्रावृषीव बलाहकौ}% २२

\twolineshloka
{राजा कनकसन्नद्धे रथे मणिविचित्रिते}
{रत्नकूबरशोभाढ्ये तिष्ठंश्चापधरो बली}% २३

\twolineshloka
{ययौ योद्धुं तु शत्रुघ्नं वीरकोटभिरावृतम्}
{तृणीकुर्वन्महावीरान्धनुर्विद्याविशारदान्}% २४

\twolineshloka
{तं योद्धुमागतं दृष्ट्वा सुबाहुं रोषपूरितम्}
{पुत्रनाशेन कुर्वन्तं सर्वसैन्यवधादिकम्}% २५

\twolineshloka
{शत्रुघ्नपार्श्वसञ्चारी हनूमांस्तमुपाद्रवत्}
{नखायुधो महानादं कुर्वन्मेघ इवाहवे}% २६

\twolineshloka
{सुबाहुस्तं हनूमन्तमागच्छन्तं महारवम्}
{उवाच प्रहसन्वाक्यं रोषपूरितलोचनः}% २७

\twolineshloka
{क्व गतः पुष्कलो हत्वा मत्पुत्रं रणमण्डले}
{पातयाम्यद्य तस्याशु शिरो ज्वलितकुण्डलम्}% २८

\twolineshloka
{क्व शत्रुघ्नो वाहपालः क्व च रामः कुतो भटाः}
{प्राणहन्तारमायान्तं पश्यन्तु प्रधने तु माम्}% २९

\twolineshloka
{इति तद्वाक्यमाकर्ण्य हनूमान्निजगाद तम्}
{शत्रुघ्नो लवणच्छेत्ता वर्तते सैन्यपालकः}% ३०

\twolineshloka
{स कथं प्रधने युध्येत्सेवकेऽग्रे स्थिते नृप}
{मां विजित्य रणे तं च त्वं गन्तासि नरर्षभ}% ३१

\twolineshloka
{इत्युक्तवन्तं तरसा विव्याध दशसायकैः}
{हृदि वानरमत्युग्रं पर्वताग्र्यमिवस्थितम्}% ३२

\twolineshloka
{सबाणानागतांस्तांश्च गृहीत्वा करसम्पुटे}
{चूर्णयामास तिलशः शितान्वैरिविदारणान्}% ३३

\twolineshloka
{चूर्णयित्वा शरांस्तांस्तु निनदन्घनगर्जितैः}
{पुच्छेनावेष्ट्य तस्योच्चै रथं निन्ये महाबलः}% ३४

\twolineshloka
{तदा तं नृपवर्योऽसावाकाशे स्थित एव सः}
{लाङ्गूलं ताडयामास शिताग्रैः सायकैर्मुहुः}% ३५

\twolineshloka
{स ताडितस्तु पुच्छाग्रे शरैः सन्नतपर्वभिः}
{मुमोच तद्रथं दिव्यं कनकेन विचित्रितम्}% ३६

\twolineshloka
{स मुक्तस्तेन तरसा शरैस्तीक्ष्णैर्जघान तम्}
{हनूमन्तं कपिवरं रोषसम्पूरितेक्षणः}% ३७

\twolineshloka
{हनूमान्बाणविच्छिन्नः सर्वत्ररुधिराप्लुतः}
{महारोषं समाधत्त नृपोपरि कपीश्वरः}% ३८

\twolineshloka
{गृहीत्वा तस्य दंष्ट्राभी रथं हयसमन्वितम्}
{चूर्णयामस वेगेन तदद्भुतमिवाभवत्}% ३९

\twolineshloka
{स्वरथं भज्यमानं तु दृष्ट्वा राजा त्वरन्बली}
{अन्यं रथं समास्थाय युयुधे तं महाबलम्}% ४०

\twolineshloka
{पुच्छे मुखेऽथोरसि च भुजे चरणयोर्नृपः}
{जघान शरसन्धानकोविदः परमास्त्रवित्}% ४१

\twolineshloka
{तदा क्रुद्धः कपिवरस्ताडयामास वक्षसि}
{पादेनोत्प्लुत्य वेगेन राज्ञः सुभटशोभिनः}% ४२

\twolineshloka
{स पदा प्रहतो भूमौ पपात किल मूर्च्छितः}
{मुखाद्वमन्नसृक्चोष्णं श्वासपूरप्रवेपितः}% ४३

\twolineshloka
{तदा प्रकुपितोऽत्यन्तं हनूमान्प्रधनाङ्गणे}
{अश्वान्वीरान्गजांश्चापि चूर्णयामास वेगतः}% ४४

\twolineshloka
{तदा सुकेतुस्तद्भ्राता तथा लक्ष्मीनिधिर्नृपः}
{उभावपि सुसन्नद्धौ युद्धाय समुपस्थितौ}% ४५

\twolineshloka
{राजानं मूर्च्छितं दृष्ट्वा प्रपलाय्य गता नराः}
{इतस्ततो बाणसङ्घैः क्षताः पुष्कलवर्षितैः}% ४६

\twolineshloka
{तद्भज्यमानं स्वबलं वीक्ष्य राजात्मजो बली}
{दमनः स्तम्भयामास सेतुर्वार्धिमिवोच्चलम्}% ४७

\twolineshloka
{तदा तु मूर्च्छितो राजा स्वप्नमेकं ददर्श ह}
{रणमध्ये कपिवरप्रपदाघातताडितः}% ४८

\twolineshloka
{रामचन्द्रस्त्वयोध्यायां सरयूतीरमण्डपे}
{ब्राह्मणैर्याज्ञिकश्रेष्ठैर्बहुभिः परिवारितः}% ४९

\twolineshloka
{तत्र ब्रह्मादयो देवास्तत्र ब्रह्माण्डकोटयः}
{कृतप्राञ्जलयस्तं वै स्तुवन्ति स्तुतिभिर्मुहुः}% ५०

\twolineshloka
{रामं श्यामं सुनयनं मृगशृङ्गपरिग्रहम्}
{गायन्ति नारदाद्याश्च वीणोल्लसितपाणयः}% ५१

\twolineshloka
{नृत्यन्त्यप्सरसस्तत्र घृताची मेनकादयः}
{वेदा मूर्तिधरा भूत्वा उपतिष्ठन्ति राघवम्}% ५२

\twolineshloka
{यच्च किञ्चिद्वस्तुजातं सर्वशोभासमन्वितम्}
{तस्य दातारमखिलं भक्तानां भोगदायकम्}% ५३

\twolineshloka
{इत्येवमादिसम्पश्यञ्जाग्रत्संज्ञामवाप सः}
{ब्रह्मशापहतज्ञानः किं दृष्टमिति वै वदन्}% ५४

\twolineshloka
{उत्थाय प्रययौ पद्भ्यां शत्रुघ्नचरणं प्रति}
{भृत्यकोटिपरीवारो रथकोटिपरीवृतः}% ५५

\twolineshloka
{सुकेतुं तु समाहूय विचित्रं दमनं तथा}
{युद्धं कर्तुं समुद्युक्तान्वारयामास धर्मवित्}% ५६

\twolineshloka
{उवाच तान्महाराजो धर्मात्मा धर्मसंयुतः}
{भ्रातःपुत्रौ शृणुत मे वाक्यं धर्मसमन्वितम्}% ५७

\twolineshloka
{मा युद्धं कुरुत क्षिप्रमनयस्तु महानभूत्}
{यद्रामचन्द्रवाहं त्वमगृह्णाद मनोर्ज्जितम्}% ५८

\twolineshloka
{एष रामः परम्ब्रह्म कार्यकारणतः परम्}
{चराचरजगत्स्वामी न मानुषवपुर्धरः}% ५९

\twolineshloka
{एतद्धि ब्रह्मविज्ञानमधुना ज्ञातवानहम्}
{पुरासिताङ्गशापेन हृतज्ञानधनोऽनघाः}% ६०

\twolineshloka
{अहं पुरा तीर्थयात्रां गतस्तत्त्वविवित्सया}
{तत्रानेके मया दृष्टा मुनयो धर्मवित्तमाः}% ६१

\twolineshloka
{असिताङ्गं मुनिमहं गतवांज्ञातुमिच्छया}
{तदा प्रोवाच मां विप्रः कृपां कृत्वा ममोपरि}% ६२

\twolineshloka
{योऽसावयोध्याधिपतिः स परब्रह्मशब्दितः}
{तस्य या जानकी देवी साक्षात्सा चिन्मयी स्मृता}% ६३

\twolineshloka
{एनं तु योगिनः साक्षादुपासते यमादिभिः}
{दुस्तरा पारसंसारवारिधिं सन्तितीर्षवः}% ६४

\twolineshloka
{स्मृतमात्रो महापापहारी स गरुडध्वजः}
{य एनं सेवते विद्वान्स संसारं तरिष्यति}% ६५

\twolineshloka
{तदाहमहसं विप्रं कोऽयं रामस्तु मानुषः}
{केयं सा जानकी देवी हर्षशोकसमाकुला}% ६६

\twolineshloka
{अजन्मनः कथं जन्म अकर्तुः कृत्यमत्र किम्}
{जन्मदुःखजरातीतं कथयस्व मुने मम}% ६७

\twolineshloka
{इत्युक्तवन्तं मां क्रुद्धः शशाप स मुनीश्वरः}
{अज्ञात्वा तत्स्वरूपं त्वं प्रतिब्रूषे ममाधम}% ६८

\twolineshloka
{एनं निन्दसि रामं त्वं मानुषोऽयमिदं हसन्}
{तस्मात्त्वं तत्त्वसम्मूढो भविष्यस्युदरम्भरिः}% ६९

\twolineshloka
{तदाहं तस्य चरणावगृह्णं सदया युतः}
{दृष्ट्वा मे विनयं मां तु प्रावोचत्करुणानिधिः}% ७०

\twolineshloka
{त्वं रामस्य मखे विघ्नं करिष्यसि यदा नृप}
{तदा हनूमानङ्घ्रिं त्वां ताडयिष्यति वेगतः}% ७१

\twolineshloka
{तदा त्वं ज्ञास्यसे राजन्नान्यथा स्वमनीषया}
{पुराहमुक्तस्तेनैवं तद्दृष्टमधुना मया}% ७२

\twolineshloka
{यदा मां हनुमान्क्रुद्धस्ताडयामास वक्षसि}
{तदाऽदर्शं रमानाथं पूर्णब्रह्मस्वरूपिणम्}% ७३

\twolineshloka
{तस्मादश्वं तु शोभाढ्यमानयन्तु महाबलाः}
{धनानि चैव वासांसि राज्यं चेदं समर्पये}% ७४

\twolineshloka
{रामं दृष्ट्वा कृतार्थः स्यामहं यज्ञेति पुण्यदे}
{आनयन्तु हयं मह्यं रोचते तु तदर्पणम्}% ७५

{॥इति श्रीपद्मपुराणे पातालखण्डे शेषवात्स्यायनसंवादे रामाश्वमेधे सुबाहुपराजयो नाम अष्टाविंशतितमोऽध्यायः॥२८॥}

\dnsub{एकोनत्रिंशत्तमोऽध्यायः}%\resetShloka

\uvacha{शेष उवाच}

\twolineshloka
{ते तु तातवचः श्रुत्वा हर्षिताः सम्प्रहारिणः}
{तथेत्यूचुर्महाराजं रामदर्शनलालसम्}% १

\uvacha{पुत्रा ऊचुः}

\twolineshloka
{राजन्भवत्पदाम्भोजान्नान्यं जानीमहे वयम्}
{यत्तव स्वान्ततो जातं तद्भवत्वद्य वेगतः}% २

\twolineshloka
{अश्वोऽयं नीयतां तत्र सितचामरभूषितः}
{रत्नमालातिशोभाढ्यश्चन्दनादिकचर्चितः}% ३

\twolineshloka
{राज्यमाज्ञाफलं स्वामिन्कोशा बहुसमृद्धयः}
{वासांसि सुमहार्हाणि सूक्ष्माणि सुगुणानि च}% ४

\twolineshloka
{चन्दनं चन्द्रकं चैव वाजिनः सुमनोहराः}
{हस्तिनस्तु मदोद्धूता रथाः काञ्चनकूबराः}% ५

\twolineshloka
{विचित्रतरवर्णादि नानाभूषणभूषिताः}
{दास्यः शतसहस्रं च दासाश्च सुमनोरमाः}% ६

\twolineshloka
{मणयः सूर्यसङ्काशा रत्नानि विविधानि च}
{मुक्ताफलानि शुभ्राणि गजकुम्भभवानि च}% ७

\twolineshloka
{विद्रुमाः शतसाहस्रा यद्यद्वस्तुमहोदयम्}
{तत्सर्वं रामचन्द्राय देहि राजन्महामते}% ८

\twolineshloka
{सुतानस्मान्किङ्करान्नः सर्वानर्पय भूपते}
{कथं न कुरुषेराजंस्तदधीनं नृपासनम्}% ९

\uvacha{शेष उवाच}

\twolineshloka
{इति पुत्रवचः श्रुत्वा हर्षितोऽभून्महीपतिः}
{उवाच च सुतान्वीरान्स्ववाक्यकरणोद्यतान्}% १०

\uvacha{राजोवाच}

\twolineshloka
{आनयन्तु हयं सर्वे सन्नद्धाः शस्त्रपाणयः}
{नानारथपरीवारास्ततो यास्ये नृपं प्रति}% ११

\uvacha{शेष उवाच}

\twolineshloka
{इति राज्ञोवचः श्रुत्वा विचित्रो दमनस्तथा}
{सुकेतुः समरे शूरा जग्मुस्तस्याज्ञयोद्यताः}% १२

\twolineshloka
{ते गत्वाथ पुरीं शूरा वाजिनं सुमनोरमम्}
{सितचामरसंयुक्तं स्वर्णपत्राद्यलङ्कृतम्}% १३

\twolineshloka
{रत्नमालाविभूषाढ्यं चित्रपत्रेणशोभितम्}
{विचित्रमणिभूषाढ्यं मुक्ताजालस्वलङ्कृतम्}% १४

\twolineshloka
{रज्ज्वा धृतं महावीरैः पूर्वतः पृष्ठतो भटैः}
{महाशस्त्रास्त्रसंयुक्तैः सर्वशोभासमन्वितैः}% १५

\twolineshloka
{सितातपत्रमस्योच्चैर्भाति मूर्धनि वाजिनः}
{सुचामरद्वयं यस्य ध्रियते पुरतो मुहुः}% १६

\twolineshloka
{कृष्णागर्वादिधूपैश्च धूपितं वायुवेगिनम्}
{राज्ञः पुरो निनायाश्वं हयमेधस्य सत्क्रतोः}% १७

\twolineshloka
{तमानीतं हयं दृष्ट्वा रत्नमालाविभूषितम्}
{मनोजवं कामरूपं जहर्ष मतिमान्नृपः}% १८

\twolineshloka
{जगाम पद्भ्यां शत्रुघ्नं राजचिह्नाद्यलङ्कृतः}
{स्वपुत्रपौत्रैः संयुक्तो राजा परमधार्मिकः}% १९

\twolineshloka
{ययौ कर्तुं धनानां स सद्व्ययं चलगामिनाम्}
{एतद्विनश्वरं मत्वा दुःखदं सक्तचेतसाम्}% २०

\twolineshloka
{शत्रुघ्नं स ददर्शाथ सितच्छत्रेण शोभितम्}
{चामरैर्वीज्यमानञ्च सेवकैः पुरतः स्थितैः}% २१

\twolineshloka
{सुमतिं परिपृच्छन्तं रामचन्द्रकथानकम्}
{भयवार्ताविनिर्मुक्तं वीरशोभास्वलङ्कृतम्}% २२

\twolineshloka
{वीरैः कोटिभिराकीर्णं वाजिपालनकाङ्क्षिभिः}
{वानराणां सहस्रैश्च समन्तात्परिवारितम्}% २३

\twolineshloka
{दृष्ट्वा शत्रुघ्नचरणौ प्रणनाम सपुत्रकः}
{धन्योऽहमिति संहृष्टो वदन्रामैकमानसः}% २४

\twolineshloka
{शत्रुघ्नस्तं प्रणयिनं दृष्ट्वा राजानमुद्भटम्}
{उत्थायासनतः सर्वैर्भटैर्दोर्भ्यां स सस्वजे}% २५

\twolineshloka
{दृढं सम्पूज्य राजा तं शत्रुघ्नं परवीरहा}
{उवाच हर्षमापन्नो गद्गदस्वरया गिरा}% २६

\uvacha{सुबाहुरुवाच}

\twolineshloka
{अद्य धन्योस्मि ससुतः सकुटुम्बः सवाहनः}
{यद्युष्मच्चरणौ द्रक्ष्ये नृपकोटिभिरीडितौ}% २७

\twolineshloka
{अज्ञानिना सुतेनायं गृहीतो वाजिनां वरः}
{दमनेनानयं त्वस्य क्षमस्व करुणानिधे}% २८

\twolineshloka
{न जानाति रघूत्तंसं सर्वदेवाधिदैवतम्}
{लीलया विश्वस्रष्टारं हन्तारमपि पालकम्}% २९

\twolineshloka
{इदं राज्यं समृद्धाङ्गं समृद्धबलवाहनम्}
{इमे कोशा धनैः पूर्णा इमे पुत्रा इमे वयम्}% ३०

\twolineshloka
{सर्वे वयं रामनाथास्त्वदाज्ञा प्रतिपालकाः}
{गृहाण सर्वं सफलं न मेऽस्ति क्वचिदुन्मतम्}% ३१

\twolineshloka
{क्वासौ हनूमान्रामस्य चरणाम्भोजषट्पदः}
{यत्प्रसादादहं प्राप्स्ये राजराजस्य दर्शनम्}% ३२

\twolineshloka
{साधूनां सङ्गमे किं किं प्राप्यते न महीतले}
{यत्प्रसादादहं मूढो ब्रह्मशापमतीतरम्}% ३३

\twolineshloka
{दृष्ट्वा त्वद्य महाराजं पद्मपत्रनिभेक्षणम्}
{प्राप्स्यामि जन्मनः सर्वं फलं दुर्लभमत्र च}% ३४

\twolineshloka
{मम तावद्गतं चायुर्बहुरामवियोगिनः}
{स्वल्पमुर्वरितं तत्र कथं द्रक्ष्ये रघूत्तमम्}% ३५

\twolineshloka
{मह्यं दर्शयतं रामं यज्ञकर्मविचक्षणम्}
{यदङ्घ्रिरजसापूता शिलाभूता मुनिप्रिया}% ३६

\twolineshloka
{काकः परं पदं प्राप्तो यद्बाणस्पर्शनात्खगः}
{अनेके यस्य वक्त्राब्जं वीक्ष्य सङ्ख्ये पदं गताः}% ३७

\twolineshloka
{ये त्वस्य रघुनाथस्य नाम गृह्णन्ति सादराः}
{ते यान्ति परमं स्थानं योगिभिर्यद्विचिन्त्यते}% ३८

\twolineshloka
{धन्यायोध्याभवा लोका ये राममुखपञ्जम्}
{स्वलोचनपुटैः पीत्वा सुखं यान्ति महोदयम्}% ३९

\twolineshloka
{इति सम्भाष्य नृपतिं वाहं राज्यं धनानि च}
{सर्वं समर्प्य चावोचत्किङ्करोस्मि महीपते}% ४०

\twolineshloka
{इति वाक्यं समाकर्ण्य राज्ञः परपुरञ्जयः}
{प्रत्युवाचेति तं भूपं वाग्मी वाक्यविशारदः}% ४१

\uvacha{शत्रुघ्न उवाच}

\twolineshloka
{कथं राजन्निदं ब्रूषे त्वं वृद्धो मम पूजितः}
{सर्वं त्वदीयं त्वद्राज्यं दमनो विदधात्वयम्}% ४२

\twolineshloka
{क्षत्त्रियाणामिदं कृत्यं यत्सङ्ग्रामविधायकम्}
{सर्वं राज्यं धनं चेदं प्रतियातु ममाज्ञया}% ४३

\twolineshloka
{यथा मे रघुनाथस्तु पूज्यो वाङ्मनसा सदा}
{तथा त्वमपि मत्पूज्यो भविष्यसि महीपते}% ४४

\twolineshloka
{भवान्सज्जो भवत्वद्य हयस्यानुगमं प्रति}
{सन्नद्धः कवची खड्गी गजाश्वरथसंयुतः}% ४५

\twolineshloka
{इति वाक्यं समाकर्ण्य शत्रुघ्नस्य महीपतिः}
{पुत्रं राज्येऽभिषेच्यैव शत्रुघ्नेन सुपूजितः}% ४६

\twolineshloka
{महारथैः परिवृतो निजं पुत्रं रणाङ्गणे}
{पुष्कलेन हतं भूपः संस्कृत्य विधिपूर्वकम्}% ४७

\twolineshloka
{क्षणं शुशोच तत्त्वज्ञो लोकदृष्ट्या महारथः}
{ज्ञानेनानाशयच्छोकं रघुनाथमनुस्मरन्}% ४८

\twolineshloka
{सज्जीभूतो रथे तिष्ठन्महासैन्यसमावृतः}
{आजगाम स शत्रुघ्नं महारथिपुरस्कृतः}% ४९

\twolineshloka
{राजा तमागतं दृष्ट्वा सर्वसैन्यसमन्वितम्}
{गन्तुं चकार धिषणां हयवर्यस्य पालने}% ५०

\twolineshloka
{सोऽश्वो विमोचितस्तेन भाले पत्रेण चिह्नितः}
{वामावर्तं भ्रमन्प्रायात्पौर्वाञ्जनपदान्बहून्}% ५१

\twolineshloka
{तत्रतत्रत्य भूपालैर्महाशूराभिपूजितैः}
{प्रणतिः क्रियते तस्य न कोपि तमगृह्णत}% ५२

\twolineshloka
{केचिद्वासांसि चित्राणि केचिद्राज्यं स्वकं महत्}
{केचिद्धनं जनं केचिदानीय प्रणमन्ति तम्}% ५३

{॥इति श्रीपद्मपुराणे पातालखण्डे शेषवात्स्यायनसंवादे रामाश्वमेधे शत्रुघ्नस्य सुबाहुना सह निर्याणं नाम एकोनत्रिंशत्तमोऽध्यायः॥२९॥}

\dnsub{त्रिंशोऽध्यायः}%\resetShloka

\uvacha{शेष उवाच}

\twolineshloka
{अथ तेजःपुरं प्राप्तस्तुरगः पत्रशोभितः}
{यस्यां पालयते राजा प्रजाः सत्येन सत्यवान्}% १

\twolineshloka
{अथ कोटिपरीवारो रघुनाथानुजस्ततः}
{हयानुगो ययौ तस्य पुरतः पुरधर्षणः}% २

\twolineshloka
{तद्दृष्ट्वा नगरं रम्यं चित्रप्राकारशोभितम्}
{काञ्चनैः कलशैस्तत्र परितः प्रतिभासितम्}% ३

\twolineshloka
{देवायतनसाहस्रैः सर्वतश्च विराजितम्}
{यतीनां तु मठास्तत्र शोभन्ते यतिपूरिताः}% ४

\twolineshloka
{वहत्यत्र महादेवी शिखिलोचनमूर्धगा}
{हंसकारण्डवाकीर्णामुनिवृन्दनिषेविता}% ५

\twolineshloka
{ब्राह्मणानां प्रत्यगारमग्निहोत्रभवः पुनः}
{धूमस्तत्र पुनात्यङ्ग पातकाप्लुतमानसान्}% ६

\twolineshloka
{उवाच सुमतिं राजा शत्रुघ्नः शत्रुतापनः}
{तत्पुरप्रेक्षणोद्भूतहर्षविस्मितमानसः}% ७

\uvacha{शत्रुघ्न उवाच}

\twolineshloka
{मन्त्रिन्कथय कस्येदं पुरं मे दृष्टिगोचरम्}
{करोति मानसाह्लादं धर्मेण प्रतिपालितम्}% ८

\uvacha{शेष उवाच}

\twolineshloka
{इति वाक्यं समाकर्ण्य शत्रुघ्नस्य महीपतेः}
{उवाच सुमतिः सर्वं यथातथमनुद्धतम्}% ९

\uvacha{सुमतिरुवाच}

\twolineshloka
{शृणुष्वावहितः स्वामिन्वैष्णवस्य कथाः शुभाः}
{याः श्रुत्वा मुच्यते पापाद्ब्रह्महत्यासमादपि}% १०

\twolineshloka
{जीवन्मुक्तो वरीवर्ति रामाङ्घ्र्यम्बुजषट्पदः}
{सत्यवान्यज्ञयज्ञाङ्ग ज्ञाता कर्ताऽविता महान्}% ११

\twolineshloka
{धेनुं प्रसाद्य बहुभिर्व्रतैर्यं प्राप तत्पिता}
{ऋतम्भराख्यो जगति ख्यातः परमधार्मिकः}% १२

\twolineshloka
{गौः प्रसन्ना ददौ पुत्रमनेकगुणसंस्कृतम्}
{सत्यवन्तं सुशोभाढ्यं तं जानीहि नृपोत्तमम्}% १३

\uvacha{शत्रुघ्न उवाच}

\twolineshloka
{को वा ऋतम्भरो राजा किमर्थं धेनुपूजनम्}
{कथं प्राप्तः सुतस्तस्य वैष्णवो विष्णुसेवकः}% १४

\twolineshloka
{सर्वमेतत्समाचक्ष्व वैष्णवस्य कथानकम्}
{श्रुतं हरति जन्तूनां महापातकपर्वतम्}% १५

\uvacha{शेष उवाच}

\twolineshloka
{इति वाक्यं समाकर्ण्य शत्रुघ्नस्य महार्थकम्}
{कथयामास विशदं तदुत्पत्तिकथानकम्}% १६

\twolineshloka
{ऋतम्भरो नरपतिरनपत्यः पुराऽभवत्}
{कलत्राणि बहून्यस्य न पुत्रं प्राप तेषु वै}% १७

\twolineshloka
{तदा जाबालिनामानं मुनिं दैवादुपागतम्}
{प्रपच्छ कुशलोद्युक्तः सपुत्रोत्पत्तिकारणम्}% १८

\uvacha{ऋतम्भर उवाच}

\twolineshloka
{स्वामिन्वन्ध्यस्य मे ब्रूहि पुत्रोत्पत्तिकरं वचः}
{यत्कृत्वा जायतेऽपत्यं मम वंशधरं वरम्}% १९

\twolineshloka
{तज्ज्ञात्वा भवतो भव्यं प्रकुर्यां निश्चितं वचः}
{दानं व्रतं वा तीर्थं वा मखं वा मुनिसत्तम}% २०

\twolineshloka
{इति राज्ञोवचः श्रुत्वा जगाद मुनिसत्तमः}
{सुतोत्पत्तिकरं वाक्यं प्रणतस्य सुतार्थिनः}% २१

\twolineshloka
{अपत्यप्राप्तिकामस्य सन्त्युपायास्त्रयः प्रभो}
{विष्णोः प्रसादो गोश्चापि शिवस्याप्यथवा पुनः}% २२

\twolineshloka
{तस्मात्त्वं कुरु वै पूजां धेनोर्देवतनोर्नृप}
{यस्याः पुच्छे मुखे शृङ्गे पृष्ठे देवाः प्रतिष्ठिताः}% २३

\twolineshloka
{सा तुष्टा दास्यति क्षिप्रं वाञ्छितं धर्मसंयुतम्}
{एवं विदित्वा गोपूजां विधेहि त्वमृतम्भर}% २४

\twolineshloka
{यो वै नित्यं पूजयति गां गेहे यवसादिभिः}
{तस्य देवाश्च पितरो नित्यं तृप्ता भवन्ति हि}% २५

\twolineshloka
{यो वै गवाह्निकं दद्यान्नियमेन शुभव्रतः}
{तेन सत्येन तस्य स्युः सर्वे पूर्णा मनोरथाः}% २६

\twolineshloka
{तृषिता गौर्गृहे बद्धा गेहे कन्या रजस्वला}
{देवता च सनिर्माल्या हन्ति पुण्यं पुराकृतम्}% २७

\twolineshloka
{यो वै गां प्रतिषिद्ध्येत चरन्तीं स्वं तृणं नरः}
{तस्य पूर्वे च पितरः कम्पन्ते पतनोन्मुखाः}% २८

\twolineshloka
{यो वै यष्ट्या ताडयति धेनुं मर्त्यो विमूढधीः}
{धर्मराजस्य नगरं स याति करवर्जितः}% २९

\twolineshloka
{यो वै दंशान्वारयति तस्य पूर्वे ह्यधोगताः}
{नृत्यन्ति मत्सुतो ह्यस्मांस्तारयिष्यति भाग्यवान्}% ३०

\twolineshloka
{अत्रैवोदाहरन्तीममितिहासं पुरातनम्}
{जनकस्य पुरावृत्तं धर्मराजपुरेऽद्भुतम्}% ३१

\twolineshloka
{एकदा जनको राजा योगेनासून्समत्यजत्}
{तदा विमानं सम्प्राप्तं किङ्किणीजालभूषितम्}% ३२

\twolineshloka
{तदारुह्य गतो राजा सेवकैरूढदेहवान्}
{मार्गे जगाम धर्मस्य संयमिन्याः पुरोऽन्तिके}% ३३

\twolineshloka
{तदा नरककोटीषु पीड्यन्ते पापकारिणः}
{जनकस्याङ्गपवनं प्राप्य सौख्यं प्रपेदिरे}% ३४

\twolineshloka
{निरये दाहजापीडा जातैषां सुखकारिणी}
{महादुःखं तदा नष्टं जनकस्याङ्गवायुना}% ३५

\twolineshloka
{तदा तं निर्गतं दृष्ट्वा जन्तवः पापपीडिताः}
{अत्यन्तं चुक्रुशुर्भीतास्तद्वियोगमनिच्छवः}% ३६

\twolineshloka
{ऊचुस्ते करुणां वाचं मा गच्छ सुकृतिन्नितः}
{त्वदङ्गवायुसंस्पर्शात्सुखिनः स्यामपीडिताः}% ३७

\twolineshloka
{इति वाक्यं समाकर्ण्य राजा परमधार्मिकः}
{मानसे चिन्तयामास करुणापूरपूरितः}% ३८

\twolineshloka
{चेन्मत्तः प्राणिनां सौख्यं भवेदिह तदा पुनः}
{अत्रैव च पुरे स्थास्ये स्वर्ग एष मनोरमः}% ३९

\twolineshloka
{एवं कृत्वा नृपस्तस्थौ तत्रैव निरयाग्रतः}
{विदधत्प्राणिनां सौख्यमनुकम्पितमानसः}% ४०

\twolineshloka
{तत्र धर्मस्तु सम्प्राप्तो निरयद्वारि दुःखदे}
{कारयन्यातनास्तीव्रा नानापातककारिणाम्}% ४१

\twolineshloka
{तदा ददर्श राजानं जनकं द्वारिसंस्थितम्}
{विमानेन महापुण्यकारिणं दययायुतम्}% ४२

\twolineshloka
{तमुवाच प्रेतपतिर्जनकं सहसन्गिरा}
{राजन्कुतस्त्वं सम्प्राप्तः सर्वधर्मशिरोमणिः}% ४३

\twolineshloka
{एतत्स्थानं पातकिनां दुष्टानां प्राणघातिनाम्}
{नायान्ति पुरुषा भूप त्वादृशाः पुण्यकारिणः}% ४४

\twolineshloka
{अत्रायान्ति नरास्ते वै ये परद्रोहतत्पराः}
{परापवादनिरताः परद्रव्यपरायणाः}% ४५

\twolineshloka
{यो वै कलत्रं धर्मिष्ठं निजसेवापरायणम्}
{अपराधादृते जह्यात्सनरोऽत्र समाव्रजेत्}% ४६

\twolineshloka
{मित्रं वञ्चयते यस्तु धनलोभेन लोभितः}
{आगत्यात्र नरः पीडां मत्तः प्राप्नोति दारुणाम्}% ४७

\twolineshloka
{यो रामं मनसा वाचा कर्मणा दम्भतोऽपि वा}
{द्वेषाद्वाचोपहासाद्वा न स्मरत्येव मूढधीः}% ४८

\twolineshloka
{तं बध्नामि पुनस्त्वेषु निक्षिप्य श्रपयामि च}
{यैः स्मृतो न रमानाथो नरकक्लेशवारकः}% ४९

\twolineshloka
{तावत्पापं मनुष्याणामङ्गेषु नृप तिष्ठति}
{यावद्रामं न रसना गृणाति कलि दुर्मतेः}% ५०

\twolineshloka
{महापापकरा राजन्ये भवन्ति महामते}
{तानानयन्ति मद्भृत्यास्त्वादृशान्द्रष्टुमक्षमाः}% ५१

\twolineshloka
{तस्माद्गच्छ महाराज भुङ्क्ष्व भोगाननेकशः}
{विमानवरमारुह्य भुङ्क्ष्व पुण्यमुपार्जितम्}% ५२

\twolineshloka
{इति वाक्यं समाकर्ण्यध र्मराजस्य तत्पतेः}
{उवाच धर्मराजानं करुणापूरपूरितः}% ५३

\uvacha{जनक उवाच}

\twolineshloka
{अहं गच्छामि नो नाथ जीवानामनुकम्पया}
{मदङ्गवायुना ह्येते सुखं प्राप्ताः स्म संस्थिताः}% ५४

\twolineshloka
{एतान्मुञ्चसि चेद्राजन्सर्वान्वै निरयस्थितान्}
{ततो गच्छामि सुखितः स्वर्गं पुण्यजनाश्रितम्}% ५५

\uvacha{जाबालिरुवाच}

\twolineshloka
{इति वाक्यमथाश्रुत्य जनकं प्रत्युवाच सः}
{प्रत्येकं निर्दिशञ्जीवान्निरयस्थाननेकशः}% ५६

\uvacha{धर्म उवाच}

\twolineshloka
{अयं मित्र कलत्रं वै विश्वस्तमनुजग्मिवान्}
{तस्मादेनं लोहशङ्कौ वर्षायुतमपीपचम्}% ५७

\twolineshloka
{पश्चादेनं सूकराणां योनौ निक्षिप्य दोषिणम्}
{मानुषेष्ववतार्योऽयं षण्ढचिह्नेन चिह्नितः}% ५८

\twolineshloka
{अनेन परदाराश्च बलादालिगिता मुहुः}
{तस्मादयं पच्यतेऽत्र रौरवे शतहायनम्}% ५९

\twolineshloka
{अयं तु परकीयं स्वं मुषित्वा बुभुजे कुधीः}
{तस्मादस्य करौ छित्त्वा पचेयं पूयशोणिते}% ६०

\twolineshloka
{अयं सायन्तने प्राप्तमतिथिं क्षुधयार्दितम्}
{वाण्यापि नाकरोत्तस्य पूजनं स्वागतं न च}% ६१

\twolineshloka
{तस्मादयं पातनीयस्तामिस्रेन्धनपूरिते}
{भ्रमरैः पीडितो यातु यातनां शतहायनम्}% ६२

\twolineshloka
{अयं तावत्परस्योच्चैर्निन्दां कुर्वन्नलज्जितः}
{अयमप्यशृणोत्कर्णौ प्रेरयन्बहुशस्तु ताम्}% ६३

\twolineshloka
{तस्मादिमावन्धकूपे पतितौ दुःखदुःखितौ}
{अयं मित्रध्रुगुद्विग्नः पच्यते रौरवे भृशम्}% ६४

\twolineshloka
{तस्मादेतान्पापभोगान्कारयित्वा विमोचये}
{त्वं गच्छ नरशार्दूल पुण्यराशिविधायकः}% ६५

\uvacha{जाबालिरुवाच}

\twolineshloka
{एवं स निर्दिशञ्जीवांस्तूष्णीमासाघकारिणः}
{प्रोवाच रामभक्तोऽसौ करुणापूरितेक्षणः}% ६६

\uvacha{जनक उवाच}

\twolineshloka
{कथं निरयनिर्मुक्तिर्जीवानां दुःखिनां भवेत्}
{तदाशु कथय त्वं वै यत्कृत्वा सुखमाप्नुयुः}% ६७

\uvacha{धर्म उवाच}

\twolineshloka
{नैभिराराधितो विष्णुर्नैभिस्तस्य कथाः श्रुताः}
{कथं निरयनिर्मुक्तिर्भवेद्वै पापकारिणाम्}% ६८

\twolineshloka
{यदि त्वं मोचयस्येतान्महापापकरानपि}
{तर्ह्यर्पय महाराज पुण्यं तत्कथयामि यत्}% ६९

\twolineshloka
{एकदा प्रातरुत्थाय शुद्धभावेन चेतसा}
{ध्यातः श्रीरघुनाथोऽसौ महापापहराभिधः}% ७०

\twolineshloka
{रामरामेति यच्चोक्तं त्वया शुद्धेन चेतसा}
{तत्पुण्यमर्पयैतेभ्यो येन स्यान्निरयाच्च्युतिः}% ७१

\uvacha{जाबालिरुवाच}

\twolineshloka
{एतच्छ्रुत्वा वचस्तस्य धर्मराजस्य धीमतः}
{पुण्यं ददौ महाराज आजन्मसमुपार्जितम्}% ७२

\twolineshloka
{यदा जन्मकृतैः पुण्यै रघुनाथार्चनोद्भवैः}
{एतेषां निरयान्मुक्तिर्भवत्वत्र मनोरमा}% ७३

\twolineshloka
{एवं कथयतस्तस्य जीवा निरयसंस्थिताः}
{तत्क्षणान्निरयान्मुक्ता जाता दिव्यवपुर्धराः}% ७४

\twolineshloka
{ऊचुस्ते जनकं राजंस्त्वत्प्रसादाद्वयं क्षणात्}
{दुःखदान्निरयान्मुक्ता यास्यामः परमं पदम्}% ७५

\twolineshloka
{तान्दृष्ट्वा सूर्यसङ्काशान्नरान्निरयनिःसृतान्}
{तुतोष चित्ते सुभृशं सर्वभूतदयापरः}% ७६

\twolineshloka
{ते सर्वे प्रययुर्लोकं दिवं देवैरलङ्कृतम्}
{जनकं तु प्रशंसन्तो महाराजं दयानिधिम्}% ७७

{॥इति श्रीपद्मपुराणे पातालखण्डे शेषवात्स्यायनसंवादे रामाश्वमेधे सत्यवदाख्याने जनकेन नरकस्थप्राणिमोचनं नाम त्रिंशोऽध्यायः॥३०॥}

\dnsub{एकत्रिंशोऽध्यायः}%\resetShloka

\uvacha{जाबालिरुवाच}

\twolineshloka
{अथ तेषु प्रयातेषु नरकस्थेषु वै नृषु}
{राजा पप्रच्छ कीनाशं सर्वधर्मविदांवरम्}% १

\uvacha{राजोवाच}

\twolineshloka
{धर्मराज त्वया प्रोक्तं यत्पातककरा नराः}
{आयान्ति तव संस्थानं न च धर्मकथारताः}% २

\twolineshloka
{मदागमनमत्राभूत्केनपापेन धार्मिक}
{तद्वै कथय सर्वं मे पापकारणमादितः}% ३

\twolineshloka
{इति श्रुत्वा तु तद्वाक्यं धर्मराजः परन्तप}
{कथयामास तस्यैवं यमपुर्यागमं तदा}% ४

\uvacha{धर्मराज उवाच}

\twolineshloka
{राजंस्तव महत्पुण्यं नैतादृक्कस्य भूतले}
{रघुनाथपदद्वन्द्वमकरन्द मधुव्रत}% ५

\twolineshloka
{त्वत्कीर्ति स्वर्धुनी सर्वान्पापिनो मलसंयुतान्}
{पुनाति परमाह्लादकारिणी दुष्टतारिणी}% ६

\twolineshloka
{तथापि पापलेशस्ते वर्तते नृपसत्तम}
{येन संयमिनीपार्श्वमागतः पुण्यपूरितः}% ७

\twolineshloka
{एकदा तु चरन्तीं गां वारयामास वै भवान्}
{तेन पापविपाकेन निरयद्वारदर्शनम्}% ८

\twolineshloka
{इदानीं पापनिर्मुक्तो बहुपुण्यसमन्वितः}
{भुङ्क्ष्व भोगान्सुविपुलान्निजपुण्यार्जितान्बहून्}% ९

\twolineshloka
{एतेषां करुणावार्धी रघुनाथो सुखं हरन्}
{संयमिन्या महामार्गे प्रेरयामास वैष्णवम्}% १०

\twolineshloka
{नागमिष्यो यदि त्वं वै मार्गेणानेन सुव्रत}
{अभविष्यत्कथं तेषां निरयात्परिमोचनम्}% ११

\twolineshloka
{त्वादृशाः परदुःखेन दुःखिताः करुणालयाः}
{प्राणिनां दुःखविच्छेदं कुर्वन्त्येव महामते}% १२

\uvacha{जाबालिरुवाच}

\twolineshloka
{एवं वदन्तं शमनं प्रणम्य स दिवङ्गतः}
{दिव्येन सुविमानेन अप्सरोगणशोभिना}% १३

\twolineshloka
{तस्माद्गावोऽनिशं पूज्या मनसापि न गर्हयेत्}
{गर्हयन्निरयं याति यावदिन्द्राश्चतुर्दश}% १४

\twolineshloka
{तस्मात्त्वं नृपतिश्रेष्ठ गोपूजां वै समाचर}
{सा तुष्टा दास्यति क्षिप्रं पुत्रं धर्मपरायणम्}% १५

\uvacha{सुमतिरुवाच}

\twolineshloka
{तच्छ्रुत्वा धेनुपूजां स पप्रच्छ कथमादरात्}
{पूजनीया प्रयत्नेन कीदृशं कुरुते नरम्}% १६

\twolineshloka
{जाबालि कथयामास धेनुपूजां यथाविधि}
{प्रत्यहं विपिनं गच्छेच्चारणार्थं व्रती तु गोः}% १७

\twolineshloka
{गवे यवांस्तु सम्भोज्य गोमयस्थान्समाहरेत्}
{भक्षणीया यवास्ते तु पुत्रकामेन भूपते}% १८

\twolineshloka
{सा यदा पिबते तोयं तदा पेयं जलं शुचि}
{सोच्चैः स्थाने यदा तिष्ठेत्तदानीं चासनस्थितः}% १९

\twolineshloka
{दंशान्निवारयेन्नित्यं यवसं स्वयमाहरेत्}
{एवं प्रकुर्वतः पुत्रं दास्यते धर्मतत्परम्}% २०

\uvacha{सुमतिरुवाच}

\twolineshloka
{इति वाक्यं समाकर्ण्य पुत्रकाम ऋतम्भरः}
{व्रतं चकार धर्मात्मा धेनुपूजां समाचरन्}% २१

\twolineshloka
{प्रत्यहं कुरुते गां वै यवसाद्येन तोषिताम्}
{दंशान्न्यवारयद्धीमान्यवभक्षकृतादरः}% २२

\twolineshloka
{एवं धेनुं पूजयतो गतास्तु दिवसा घनाः}
{वनमध्ये तृणादींश्च चरन्तीमकुतोभयाम्}% २३

\twolineshloka
{एकदा नृपतिस्तस्य वनस्य श्रीनिरीक्षणे}
{न्यस्तदृष्टिः सपरितो बभ्राम स कुतूहली}% २४

\twolineshloka
{तदागत्याहनद्गां वै पञ्चास्यः काननान्तरात्}
{क्रोशन्तीं बहुधा दीनां सिंहभारेणदुःखिताम्}% २५

\twolineshloka
{तदा नृपः समागत्य विलोक्य निजमातरम्}
{सिंहेन निहतां पश्यन्रुरोदातीव विह्वलः}% २६

\twolineshloka
{स दुःखितः समागत्य जाबालिमुनिसत्तमम्}
{निष्कृतिं तस्य पप्रच्छ गोवधस्य प्रमादतः}% २७

\uvacha{ऋतम्भर उवाच}

\twolineshloka
{स्वामिंस्त्वदाज्ञया धेनुं पालयन्वनमास्थितः}
{कुतोप्यागत्य तां सिंहो जघानादृष्टिगोचरः}% २८

\twolineshloka
{तस्य पापस्य निष्कृत्यै किं करोमि त्वदाज्ञया}
{कथं वा व्रतसम्पूर्तिर्मम पुत्रप्रदायिनी}% २९

\twolineshloka
{इत्युक्तवन्तं तं भूपं जगाद मुनिसत्तमः}
{सन्त्युपाया महीपाल पापस्यास्यापनुत्तये}% ३०

\twolineshloka
{ब्रह्मघ्नस्य कृतघ्नस्य सुरापस्य महामते}
{प्रायश्चित्तानि वर्तन्ते सर्वपापहराणि च}% ३१

\twolineshloka
{कृच्छ्रैश्चान्द्रायणैर्दानैर्व्रतैः सनियमैर्यमैः}
{पापानि प्रलयं यान्ति नियमादनुतिष्ठतः}% ३२

\twolineshloka
{द्वयोश्च निष्कृतिर्नास्ति पापपुञ्जकृतोस्तयोः}
{मत्या गोवधकर्तुश्च नारायणविनिन्दितुः}% ३३

\twolineshloka
{गवां यो मनसा दुःखं वाञ्च्छत्यधमसत्तमः}
{स याति निरयस्थानं यावदिन्द्राश्चतुर्दश}% ३४

\twolineshloka
{योऽपि देवं हरिं निन्देत्सकृद्दुर्भाग्यवान्नरः}
{स चापि नरकं पश्येत्पुत्रपौत्रपरीवृतः}% ३५

\twolineshloka
{तस्माज्ज्ञात्वा हरिं निन्दन्गोषु दुःखं समाचरन्}
{कदापि नरकान्मुक्तिं न प्राप्नोति नरेश्वर}% ३६

\twolineshloka
{अज्ञानप्राप्तगोहत्या प्रायश्चित्तं तु विद्यते}
{रामभक्तं तु धीमन्तं याहि त्वमृतुपर्णकम्}% ३७

\twolineshloka
{स वै समदृशः सर्वाञ्छत्रून्मित्राणि पश्यति}
{तुभ्यं कथिष्यति क्षिप्रं गोवधस्यास्य निष्कृतिम्}% ३८

\twolineshloka
{तस्य देशांस्त्वमाक्रामंस्तेन निर्वासितः पुरा}
{वैरिभावं परित्यज्य गच्छ त्वमृतुपर्णकम्}% ३९

\twolineshloka
{स यद्वदिष्यति क्षिप्रं तत्कुरुष्व समाहितः}
{यथा त्वत्कृतपापस्य निष्कृतिर्हि भविष्यति}% ४०

\twolineshloka
{स तु तद्वचनं श्रुत्वा जगाम ऋतुपर्णकम्}
{रामभक्तं रिपौ मित्रे समदृष्ट्या समञ्जसम्}% ४१

\twolineshloka
{स तस्मै कथयामास यज्जातं गोवधादिकम्}
{तस्य पापस्य निष्कृत्यै ह्युपायं सोऽप्यचिन्तयत्}% ४२

\twolineshloka
{क्षणं ध्यात्वाथ तं राजा ऋतुपर्ण ऋतम्भरम्}
{उवाच प्रहसन्वाक्यं बुद्धिमान्धर्मकोविदः}% ४३

\twolineshloka
{कोऽहं राजन्मुनीनां वै पुरतः शास्त्रवेदिनाम्}
{तान्हित्वा किं तु मां प्राप्तो मूर्खम्पण्डितमानिनम्}% ४४

\twolineshloka
{तव मय्यस्ति चेच्छ्रद्धा तदा किञ्चिद्ब्रवीम्यहम्}
{शृणुष्व नरशार्दूल गदितं मम सादरः}% ४५

\twolineshloka
{भज श्रीरघुनाथं त्वं कर्मणा मनसा गिरा}
{नैष्कापट्येन लोकेशं तोषयस्व महामते}% ४६

\twolineshloka
{स तुष्टो दास्यते सर्वं त्वद्धृदिस्थं मनोरथम्}
{अज्ञानकृत गोहत्यापापनाशं करिष्यति}% ४७

\twolineshloka
{रामं स्मरंस्त्वं धर्मात्मन्धेनुं पालय सत्तम}
{दत्त्वा द्विजाय कनकं पापनिष्कृतिमाप्स्यसि}% ४८

\uvacha{सुमतिरुवाच}

\twolineshloka
{एतच्छ्रुत्वा तु तद्वाक्यमृतम्भरनृपस्तदा}
{विधाय रामस्मरणं पूतात्मा व्रतमाचरत्}% ४९

\twolineshloka
{पूर्ववत्पालयन्धेनुं जगाम विपिनं महत्}
{रामनामस्मरन्नित्यं सर्वभूतहिते रतः}% ५०

\twolineshloka
{तस्मै तुष्टा तु सुरभिः प्रोवाच परितोषिता}
{राजन्वरय मत्तो वै वरं हृत्स्थं मनोरथम्}% ५१

\twolineshloka
{तदा प्रोवाच वै राजा पुत्रं देहि मनोरमम्}
{रामभक्तं पितृरतं स्वधर्मप्रतिपालकम्}% ५२

\twolineshloka
{तुष्टा दत्त्वा वरं सापि तस्मै राज्ञे सुतार्थिने}
{जगामादर्शनं देवी कामधेनुः कृपावती}% ५३

\twolineshloka
{स काले प्राप्तवान्पुत्रं वैष्णवं रामसेवकम्}
{सत्यवत्संज्ञयायुक्तमकरोत्तत्र तत्पिता}% ५४

\twolineshloka
{सत्यवन्तं सुतं लब्ध्वा पितृभक्तिपरं महान्}
{परमं हर्षमापेदे शक्रतुल्यपराक्रमम्}% ५५

\twolineshloka
{स राजा धार्मिकं पुत्रं प्राप्य हर्षेणनिर्भरः}
{राज्यं तस्मिन्महन्न्यस्य जगाम तपसे वनम्}% ५६

\twolineshloka
{तत्राराध्य हृषीकेशं भक्तियुक्तेन चेतसा}
{निर्धूतपापः सतनुरगाद्धरिपदं नृपः}% ५७

{॥इति श्रीपद्मपुराणे पातालखण्डे शेषवात्स्यायनसंवादे रामाश्वमेधे सत्यवदाख्याने धेनुव्रतवर्णनं नाम एकत्रिंशोऽध्यायः॥३१॥}

\dnsub{द्वात्रिंशोऽध्यायः}%\resetShloka

\uvacha{सुमतिरुवाच}

\twolineshloka
{असावपि नृपः सौम्य सत्यवान्नाम विश्रुतः}
{निजधर्मेण लोकेशं रघुनाथमतोषयत्}% १

\twolineshloka
{अस्मै तुष्टो रमानाथो ददौ भक्तिमचञ्चलाम्}
{निजाङ्घ्रिपद्मे यजतां दुर्लभां पुण्यकोटिभिः}% २

\twolineshloka
{नित्यं श्रीरघुनाथस्य कथानकमनातुरः}
{कुरुते सर्वलोकानां पावनं कृपयायुतः}% ३

\twolineshloka
{यो न पूजयते देवं रघुनाथं रमापतिम्}
{स तेन ताड्यते दण्डैर्यमस्यापि भयावहैः}% ४

\twolineshloka
{अष्टमाद्वत्सरादूर्ध्वमशीतिवत्सरो भवेत्}
{तावदेकादशी सर्वैर्मानुषैः कारिताऽमुना}% ५

\twolineshloka
{तुलसी वल्लभा यस्य कदाचिद्यच्छिरोधराम्}
{न मुञ्चति रमानाथ पादपद्मस्रगुत्तमा}% ६

\twolineshloka
{ऋषीणामपि पूज्योयमितरेषां कथं नहि}
{रघुनाथस्मृतिप्रीतिर्धूतपाप्मा हताशुभः}% ७

\twolineshloka
{ज्ञात्वायं रामचन्द्रस्य वाजिनं परमाद्भुतम्}
{आगत्य तुभ्यं सन्दास्यत्येतद्राज्यमकण्टकम्}% ८

\twolineshloka
{त्वया यद्गदितं राजंस्तत्ते कथितमुत्तमम्}
{पुनः किं पृच्छसे स्वामिन्नाज्ञापय करोमि तत्}% ९

\uvacha{शेष उवाच}

\twolineshloka
{गतोऽश्वस्तत्पुरान्तस्तु नानाश्चर्यसमन्वितः}
{तं दृष्ट्वा जनताः सर्वा राज्ञे गत्वा न्यवेदयन्}% १०

\uvacha{जनता ऊचुः}

\twolineshloka
{कोऽप्यश्वः सितवर्णेन गङ्गाजलसमेन वै}
{भाले सौवर्णपत्रेण राजमानः समागतः}% ११

\twolineshloka
{तच्छ्रुत्वा वचनं रम्यं जनानां हृद्यमीरितम्}
{ताः प्रत्याह हसन्भूपो ज्ञायतां कस्य वै हयः}% १२

\twolineshloka
{ताश्चैनं कथयामासुः शत्रुघ्नेन प्रपालितः}
{आयात्यश्वो महीभर्तू रामस्य पुरमध्यतः}% १३

\twolineshloka
{रामस्य नाम स श्रुत्वा द्व्यक्षरं सुमनोरमम्}
{जहर्ष चित्ते सुभृशं गद्गदस्वरचिन्हितः}% १४

\twolineshloka
{मयायोध्यापतिर्नित्यं यो रामश्चिन्त्यते हृदि}
{तस्याश्वः सहशत्रुघ्नः समायातः पुरं मम}% १५

\twolineshloka
{हनूमांस्तत्र रामाङ्घ्रिसेवाकर्ता भविष्यति}
{कदाचिदपि यो रामं न विस्मरति मानसे}% १६

\twolineshloka
{गच्छामि यत्र शत्रुघ्नो यत्र मारुतनन्दनः}
{अन्येऽपि यत्र पुरुषा रामपादाब्जसेवकाः}% १७

\twolineshloka
{अमात्यमादिदेशाथ सर्वं राजधनं महत्}
{गृहीत्वा तु मया सार्द्धमागच्छ त्वरया युतः}% १८

\twolineshloka
{यास्येऽहं रघुनाथस्य हयं पालयितुं वरम्}
{कर्तुं च रामपादाब्जपरिचर्यां सुदुर्लभाम्}% १९

\twolineshloka
{इत्युक्त्वा निर्जगामाथ शत्रुघ्नं प्रति सैनिकैः}
{तावत्पुरीमथ प्राप्तो रामभ्राता ससैनिकः}% २०

\twolineshloka
{वीरा गर्जन्ति प्रबला रथाः सुनिनदन्ति च}
{जयशङ्खस्वनास्तत्र वेणुनादाश्च सर्वतः}% २१

\twolineshloka
{आगत्य सत्यवान्राजा मन्त्रिभिः सुसमन्वितः}
{चरणे प्रणिपत्यास्मै राज्यं प्रादान्महाधनम्}% २२

\twolineshloka
{शत्रुघ्नस्तं तु राजानं ज्ञात्वा राममनुव्रतम्}
{तद्राज्यं तस्य पुत्राय रुक्मनाम्ने ददौ महत्}% २३

\twolineshloka
{हनूमन्तं परीरभ्य सुबाहुं रामसेवकम्}
{अन्यान्वै रामभक्तांश्च परिरभ्य महायशाः}% २४

\twolineshloka
{कृतार्थमिव चात्मानं मेने सत्यसमन्वितः}
{ननन्द चेतसि तदा शत्रुघ्नेन समन्वितः}% २५

\twolineshloka
{हयस्तावद्गतो दूरं वीरैः सुपरिरक्षितः}
{शत्रुघ्नस्तेन भूपेन ययौ वीरसमन्वितः}% २६

{॥इति श्रीपद्मपुराणे पातालखण्डे शेषवात्स्यायनसंवादे रामाश्वमेधे सत्यवत्समागमो नाम द्वात्रिंशोऽध्यायः॥३२॥}

\dnsub{त्रयस्त्रिंशत्तमोऽध्यायः}%\resetShloka

\uvacha{शेष उवाच}

\twolineshloka
{गच्छत्सु रथिवर्येषु शत्रुघ्नादिषु भूरिषु}
{महाराजेषु सर्वेषु रथकोटियुतेषु च}% १

\twolineshloka
{अकस्मादभवन्मार्गे तमः परमदारुणम्}
{यस्मिन्स्वीयो न पारक्यो लक्ष्यते ज्ञातिभिर्नरैः}% २

\twolineshloka
{रजसा व्यावृतं व्योम विद्युत्स्तनितसङ्कुलम्}
{एतादृशे तु सम्मर्दे महाभयकरे ततः}% ३

\twolineshloka
{मेघा वर्षन्ति रुधिरं पूयामेध्यादिकं बहु}
{अत्याकुला बभूवुस्ते वीराः परमवैरिणः}% ४

\twolineshloka
{आकुलीकृतलोके तु किमिदं किमिति स्थितिः}
{तमोव्याप्तानि लोकानां चक्षूंषि प्रथितौजसाम्}% ५

\twolineshloka
{जहाराश्वं रावणस्य सुहृत्पातालसंस्थितः}
{विद्युन्मालीति विख्यातो राक्षसश्रेणिसंवृतः}% ६

\twolineshloka
{कामगे सुविमाने तु सर्वायसनिषेविणि}
{आरूढोऽश्वं तु वीराणां भयं कुर्वञ्जहार ह}% ७

\twolineshloka
{मुहूर्तात्तत्तमो नष्टमाकाशं विमलं बभौ}
{वीराः शत्रुघ्नमुख्यास्ते प्रोचुः कुत्र हयोऽस्ति सः}% ८

\twolineshloka
{ते सर्वे हयराजं तु लोकयन्तः परस्परम्}
{ददृशुर्न यदा वाहं हाहाकारस्तदाभवत्}% ९

\twolineshloka
{कुत्राश्वो हयमेधस्य केन नीतः कुबुद्धिना}
{इति वाचमवोचंस्ते तावत्स दनुजेश्वरः}% १०

\twolineshloka
{ददृशे सुभटैः सर्वै रथस्थैः शौर्यशोभितैः}
{विमानवरमारूढै राक्षसाग्र्यैः समावृतः}% ११

\twolineshloka
{दुमुर्खा विकरालास्या लम्बदंष्ट्रा भयानकाः}
{राक्षसास्तत्र दृश्यन्ते सैन्यग्रासाय चोद्यताः}% १२

\twolineshloka
{तदा तं वेदयामासुः शत्रुघ्नं नृवरोत्तमम्}
{हयो नीतो न जानीमः खे विमानविलासिना}% १३

\twolineshloka
{तमसा व्याकुलान्कृत्वा वीरानस्मान्समाययौ}
{जग्राह नृपशार्दूल हयं कुरु यथोचितम्}% १४

\twolineshloka
{शत्रुघ्नस्तद्वचः श्रुत्वा महारोषसमावृतः}
{कोऽस्त्येष राक्षसो यो मे हयं जग्राह वीर्यवान्}% १५

\twolineshloka
{विमानं तत्पतत्वद्य मद्बाणव्रजनिर्हतम्}
{पतत्वद्य शिरस्तस्य क्षुरप्रैर्मम वैरिणः}% १६

\twolineshloka
{सज्जीयन्तां रथाः सर्वैर्महाशस्त्रास्त्रपूरिताः}
{यान्तु तं प्रतिसंहर्तुं योद्धारो वाजिहारिणम्}% १७

\twolineshloka
{इत्युक्त्वा रोषताम्राक्ष उवाच निजमन्त्रिणम्}
{नयानयविदं शूरं युद्धकार्यविशारदम्}% १८

\uvacha{शत्रुघ्न उवाच}

\twolineshloka
{मन्त्रिन्कथय के योज्या राक्षसस्य वधोद्यताः}
{महाशस्त्रा महाशूराः परमास्त्रविदुत्तमाः}% १९

\twolineshloka
{कथयाशु विचार्यैवं तत्करोमि भवद्वचः}
{वीरान्कथय तस्यैवं योग्यान्सर्वास्त्रकोविदान्}% २०

\twolineshloka
{एतच्छ्रुत्वा तु सचिवः प्राह वाक्यं यथोचितम्}
{वीरान्रणवरे योग्यान्दर्शयंस्तरसा नतान्}% २१

\uvacha{सुमतिरुवाच}

\twolineshloka
{जेतुं गच्छतु तद्रक्षः समरे विजयोद्यतः}
{महाशस्त्रास्त्रसंयुक्तः पुष्कलः परतापनः}% २२

\twolineshloka
{तथा लक्ष्मीनिधिर्यातु शस्त्रसङ्घसमन्वितः}
{करोतु तस्य यानस्य भङ्गं तीक्ष्णैः स्वसायकैः}% २३

\twolineshloka
{हनूमान्धृष्टकर्मात्र राक्षसैर्योधितुं क्षमः}
{करोतु मुखपुच्छाभ्यां ताडनं रक्षसां प्रभो}% २४

\twolineshloka
{वानरा अपि ये वीरा रणकर्मविशारदाः}
{गच्छन्तु तेऽखिला योद्धुं तववाक्यप्रणोदिताः}% २५

\twolineshloka
{सुमदश्च सुबाहुश्च प्रतापाग्र्यश्च सत्तमाः}
{गच्छन्तु सायकैस्तीक्ष्णैस्तान्योद्धुं राक्षसाधमान्}% २६

\twolineshloka
{भवानपि महाशस्त्रपरिवारो रथे स्थितः}
{करोतु युद्धे विजयं राक्षसं हन्तुमुद्यतः}% २७

\twolineshloka
{एतन्मम मतं राजन्ये योधास्तत्प्रमर्दनाः}
{ते गच्छन्तु रणे शूराः किमन्यैर्बहुभिर्भटैः}% २८

\twolineshloka
{इत्युक्तवति वीराग्र्येऽमात्ये सुमतिसंज्ञिके}
{शत्रुघ्नः कथयामास वीरान्सङ्ग्रामकोविदान्}% २९

\twolineshloka
{भो वीराः पुष्कलाद्या ये सर्वशस्त्रास्त्रकोविदाः}
{ते वदन्तु प्रतिज्ञां वै मत्पुरो राक्षसार्दने}% ३०

\twolineshloka
{कृत्वा प्रतिज्ञां विपुलां स्वपराक्रमशोभिनीम्}
{गच्छन्तु रणमध्ये हि भवन्तो बलसंयुताः}% ३१

\twolineshloka
{इति वाक्यं समाकर्ण्य शत्रुघ्नस्य महाबलाः}
{स्वां स्वां प्रतिज्ञां महतीं चक्रुस्ते तेजसान्विताः}% ३२

\twolineshloka
{तत्रादौ पुष्कलो वीरः श्रुत्वा वाक्यं महीपतेः}
{परमोत्साहसम्पन्नः प्रतिज्ञामूचिवानिमाम्}% ३३

\uvacha{पुष्कल उवाच}

\twolineshloka
{शृणुष्व नृपशार्दूल मत्प्रतिज्ञां पराक्रमात्}
{विहितां सर्वलोकानां शृण्वतां परमाद्भुताम्}% ३४

\twolineshloka
{चेन्न कुर्यां क्षुरप्राग्रैस्तीक्ष्णैः कोदण्डनिर्गतैः}
{दैत्यं मूर्च्छासमाक्रान्तं कीर्णकेशाकुलाननम्}% ३५

\twolineshloka
{कन्या स्वभोक्तुर्यत्पापं यत्पापं देवनिन्दने}
{तत्पापं मम वै भूयाच्चेत्कुर्यां स्ववचोऽनृतम्}% ३६

\twolineshloka
{यदिमद्बाणनिर्भिन्नाः सैनिकाः सुमहाबलाः}
{न पतन्ति महाराज प्रतिज्ञां तत्र मे शृणु}% ३७

\twolineshloka
{विष्ण्वीशयोर्विभेदं यः शिवशक्त्योः करोत्यपि}
{तत्पापं मम वै भूयाच्चेन्न कुर्यामृतं वचः}% ३८

\twolineshloka
{सर्वं मद्वाक्यमित्युक्तं रघुनाथपदाम्बुजे}
{भक्तिर्मे निश्चला यास्ति सैव सत्यं करिष्यति}% ३९

\twolineshloka
{पुष्कलस्य प्रतिज्ञां तां श्रुत्वा लक्ष्मीनिधिर्नृपः}
{प्रतिज्ञां व्यदधात्सत्यां स्वपराक्रमशोभिताम्}% ४०

\uvacha{लक्ष्मीनिधिरुवाच}

\twolineshloka
{वेदानां निन्दनं श्रुत्वा आस्ते यो मौनिवन्नरः}
{मानसे रोचयेद्यस्तु सर्वधर्मबहिष्कृतः}% ४१

\twolineshloka
{ब्राह्मणो यो दुराचारो रसलाक्षादिविक्रयी}
{विक्रीणाति च गां मूढो धनलोभेन मोहितः}% ४२

\twolineshloka
{म्लेच्छकूपोदकं पीत्वा प्रायश्चित्तं तु नाचरेत्}
{तत्पापं मम वै भूयाद्विमुखश्चेद्भवाम्यहम्}% ४३

\twolineshloka
{तत्प्रतिज्ञामथाश्रुत्य हनूमान्रणकोविदः}
{रामाङ्घ्रिस्मरणं कृत्वा प्रोवाच वचनं शुभम्}% ४४

\twolineshloka
{मत्स्वामीहृदये नित्यं ध्येयो वै योगिभिर्मुहुः}
{यं देवाः सासुराः सर्वे नमन्ति मणिमौलिभिः}% ४५

\twolineshloka
{रामः श्रीमानयोध्यायाः पतिर्लोकेशपूजितः}
{तं स्मृत्वा यद्ब्रुवे वाक्यं तद्वै सत्यं भवष्यिति}% ४६

\twolineshloka
{राजन्कोयं लघुर्दैत्यो दुर्बलः कामगे स्थितः}
{कथयाशु मया कार्यमेकेन विनिपातनम्}% ४७

\twolineshloka
{मेरुं देवेन्द्रसहितं लाङ्गूलाग्रेण तोलये}
{जलधिं शोषये सर्वं सांवर्तं वा पिबाम्यहम्}% ४८

\twolineshloka
{राज्ञः श्रीरघुनाथस्य जानक्याः कृपया मम}
{तन्नास्ति भूतले राजन्यदसाध्यं कदा भवेत्}% ४९

\twolineshloka
{एतद्वाक्यं मया प्रोक्तमनृतं स्याद्यदि प्रभो}
{तदैव रघुनाथस्य भक्तिदूरो भवाम्यहम्}% ५०

\twolineshloka
{यः शूद्रः कपिलां गां वै पयोबुद्ध्यानुपालयेत्}
{तस्य पापं ममैवास्तु चेत्कुर्यामनृतं वचः}% ५१

\twolineshloka
{ब्राह्मणीं गच्छते मोहाच्छूद्रः कामविमोहितः}
{तस्य पापं ममैवास्तु चेत्कुर्यामनृतं वचः}% ५२

\twolineshloka
{यद्घ्राणान्नरकं गच्छेत्स्पर्शनाच्चापि रौरवम्}
{तां पिबेन्मदिरां यो वा जिह्वास्वादेन लोलुपः}% ५३

\twolineshloka
{तस्य यज्जायते पापं तन्ममैवास्तु निश्चितम्}
{चेन्न कुर्यां प्रतिज्ञातं सत्यं रामकृपाबलात्}% ५४

\twolineshloka
{एवमुक्ते महावीरैर्योद्धारस्तरसा युताः}
{चक्रुः प्रतिज्ञां महतीं स्वपराक्रमशालिनीम्}% ५५

\twolineshloka
{शत्रुघ्नोऽपि व्यधात्तत्र प्रतिज्ञां पश्यतां नृणाम्}
{साधुसाधु प्रशंसन्वै तान्वीरान्युद्धकोविदान्}% ५६

\twolineshloka
{कथयामि पुरो वः स्वां प्रतिज्ञां सत्त्वशोभिताम्}
{तच्छृण्वन्तु महाभागा युद्धोत्साहसमन्विताः}% ५७

\twolineshloka
{चेत्तस्य शिर आहत्य पातयामि न सायकैः}
{विमानाच्च कबन्धाच्च भिन्नं छिन्नं च भूतले}% ५८

\twolineshloka
{यत्पापं कूटसाक्ष्येण यत्पापं स्वर्णचौर्यतः}
{यत्पापं ब्रह्मनिन्दायां तन्ममास्त्वद्य निश्चयात्}% ५९

\twolineshloka
{इति शत्रुघ्नसद्वाक्यं श्रुत्वा ते वीरपूजिताः}
{धन्योसि राघवभ्रातः कस्त्वदन्यो परो भवेत्}% ६०

\twolineshloka
{त्वया वै निहतो दैत्यो देवदानवदुःखदः}
{लवणो नाम लोकेश मधुपुत्रो महाबलः}% ६१

\twolineshloka
{कोयं वै राक्षसो दुष्टः क्व चास्य बलमल्पकम्}
{करिष्यसि क्षणादेव तस्य नाशं महामते}% ६२

\twolineshloka
{इत्युक्त्वा ते महावीराः सज्जीभूता रणाङ्गणे}
{प्रतिज्ञां स्वामृतां कर्तुं ययुस्ते राक्षसं मुदा}% ६३

{॥इति श्रीपद्मपुराणे पातालखण्डे शेषवात्स्यायनसंवादे रामाश्वमेधे वीरप्रतिज्ञाकथनं नाम त्रयस्त्रिंशत्तमोऽध्यायः॥३३॥}

\dnsub{चतुस्त्रिंशत्तमोऽध्यायः}%\resetShloka

\uvacha{शेष उवाच}

\twolineshloka
{रथैः सदश्वैः शोभाढ्यैः सर्वशस्त्रास्त्रपूरितैः}
{नानारत्नसमायुक्तैर्ययुस्ते राक्षसाधमम्}% १

\twolineshloka
{तान्दृष्ट्वा कामगे याने स्थितः प्रोवाच राक्षसः}
{मेघगम्भीरया वाचा तर्जयन्निव भूरिशः}% २

\twolineshloka
{मायां तु सुभटा योद्धुं गच्छन्तु निजमन्दिरम्}
{मा त्यजन्तु स्वकान्प्राणान्न मोक्ष्ये वाजिनं वरम्}% ३

\twolineshloka
{विद्युन्मालीति विख्यातो रावणस्य सुहृत्सखा}
{मत्सख्युः प्रेतभूतस्य निष्कृतिं कर्तुमेयिवान्}% ४

\twolineshloka
{क्वासौ रामो य आहत्य सखायं रावणं गतः}
{तस्य भ्रातापि कुत्रास्ते सर्वशूरशिरोमणिः}% ५

\twolineshloka
{तं हत्वा निष्कृतिं तस्य प्राप्स्ये रामस्य चानुजम्}
{पिबन्रुधिरमुद्भूतं कण्ठनालस्य बुद्बुदैः}% ६

\twolineshloka
{इति वाक्यं समाकर्ण्य योधानां प्रवरोत्तमः}
{पुष्कलो निजगादैनं वीर्यशौर्यसमन्वितम्}% ७

\uvacha{पुष्कल उवाच}

\twolineshloka
{विकत्थनं न कुर्वन्ति सङ्ग्रामे सुभटा नराः}
{पराक्रमं दर्शयन्ति निजशस्त्रास्त्रवर्षणैः}% ८

\twolineshloka
{रावणो निहतो येन ससुहृत्स्वजनैर्वृतः}
{तस्य वाजिनमाहृत्य कुत्र गन्तासि दुर्मद}% ९

\twolineshloka
{पतिष्यसि त्वं शत्रुघ्नबाणैः कोदण्डनिर्गतैः}
{त्वामत्स्यन्ति शिवा भूमौ पतितं प्राणवर्जितम्}% १०

\twolineshloka
{मा गर्ज दुष्ट रामस्य सेवके मयि संस्थिते}
{गर्जन्ति सुभटा युद्धे शत्रुं जित्वा महोदयात्}% ११

\uvacha{शेष उवाच}

\twolineshloka
{एवं ब्रुवन्तं तं वीरं पुष्कलं रणदुर्मदम्}
{जघान शक्त्या सुभृशं हृदि राक्षससत्तमः}% १२

\twolineshloka
{आयान्तीं तां महाशक्तिमायसीं काञ्चनाश्रिताम्}
{चिच्छेद त्रिभिरत्युग्रैः शितैर्बाणैः स पुष्कलः}% १३

\twolineshloka
{सा त्रिधा ह्यपतद्भूमौ विशिखैर्निष्प्रभीकृता}
{पतन्ती विरराजासौ विष्णोः शक्तित्रयीव किम्}% १४

\twolineshloka
{तां छिन्नां शक्तिकां दृष्ट्वा राक्षसः परतापनः}
{जग्राह शूलं तरसा त्रिशिखं लोहनिर्मितम्}% १५

\twolineshloka
{तीक्ष्णाग्रं ज्वलनप्रख्यं राक्षसेन्द्रो व्यमोचयत्}
{आयान्तं तिलशश्चक्रे बाणैः पुष्कलसंज्ञितः}% १६

\twolineshloka
{छित्त्वा त्रिशूलं तरसा राघवस्य हि सेवकः}
{पुष्कलश्चाप आधत्त बाणांस्तीक्ष्णान्मनोजवान्}% १७

\twolineshloka
{ते बाणा हृदि तस्याशु लग्ना रागं बतासृजन्}
{वैष्णवस्य यथा स्वान्ते गुणा विष्णोर्मनोहराः}% १८

\twolineshloka
{तद्बाणवेधदुःखार्तो विद्युन्माली सुदुर्मदः}
{जग्राह मुद्गरं घोरं पुष्कलं हन्तुमुद्यतः}% १९

\twolineshloka
{मुद्गरः प्रहितस्तेन विद्युन्माल्यभिधेन हि}
{हृदि लग्नोसृजच्छीघ्रं कश्मलं तदकारयत्}% २०

\twolineshloka
{मुद्गरप्रहतो वीरः कम्पमानः सवेपथुः}
{पपात स्यन्दनोपस्थे पुष्कलः शत्रुतापनः}% २१

\twolineshloka
{उग्रदंष्ट्रोऽथ तद्भ्राता लक्ष्मीनिधिमयोधयत्}
{शस्त्रास्त्रैर्बहुधा मुक्तैर्वीरप्राणहृतिङ्करैः}% २२

\twolineshloka
{पुष्कलस्तत्क्षणात्प्राप्य संज्ञां राक्षसमब्रवीत्}
{धन्योसि राक्षसश्रेष्ठ महीयांस्ते पराक्रमः}% २३

\twolineshloka
{पश्येदानीं ममाप्युच्चैः प्रतिज्ञां शूरमानिताम्}
{विमानात्पातयाम्यद्य भूमौ त्वां शितसायकैः}% २४

\twolineshloka
{इत्युक्त्वा निशितं बाणं समगृह्णाद्दुरासदम्}
{ज्वलन्तमग्नितेजस्कं महौदार्यसमन्वितम्}% २५

\twolineshloka
{स यावत्तत्प्रतीकर्तुं विधत्ते स्वपराक्रमम्}
{तावद्धृदिगतोऽत्युग्रस्तीक्ष्णधारः ससायकः}% २६

\twolineshloka
{तेन बाणेन विभ्रान्तो भ्रमच्चित्तः स राक्षसः}
{पपात कामगोपस्थाद्भूमौ विगतचेतनः}% २७

\twolineshloka
{उग्रदंष्ट्रेण वै दृष्टः पतमानो निजाग्रजः}
{गृहीत्वा तं विमानान्तर्निनाय रिपुशङ्कितः}% २८

\twolineshloka
{प्राह चारिं महारोषात्पुष्कलं बलिनां वरम्}
{मद्भ्रातरं पातयित्वा कुत्र यास्यसि दुर्मते}% २९

\twolineshloka
{मां वै युधि विनिर्जित्य गन्तासि जयमुत्तमम्}
{स्थिते मयि तव स्वान्ते जयाशा विनिवर्त्य ताम्}% ३०

\twolineshloka
{एवं ब्रुवन्तं तरसा जघान दशभिः शरैः}
{हृदये तस्य दुष्टस्य रोषपूरितलोचनः}% ३१

\twolineshloka
{स ताडितो दशशरैः पुष्कलेन महात्मना}
{चुक्रोध हृदि दुर्बुद्धिस्तं हन्तुमुपचक्रमे}% ३२

\twolineshloka
{दन्तान्निष्पिष्य सक्रोधो मुष्टिमुद्यम्य चाहनत्}
{व्यनदद्वज्रनिर्घातपातशङ्कां सृजन्हृदि}% ३३

\twolineshloka
{मुष्टिनाभिहतो वीरः पुष्कलः परमास्त्रवित्}
{नाकम्पत विनिष्पेषं वाञ्छंस्तस्य दुरात्मनः}% ३४

\twolineshloka
{वत्सदन्तान्महातीक्ष्णान्मुमोच हृदि सायकान्}
{तैर्बाणैर्व्यथितो दैत्यस्त्रिशूलं तु समाददे}% ३५

\twolineshloka
{जाज्वल्यमानं त्रिशिखं ज्वालामालातिभीषणम्}
{लग्नं हृदि महावीर पुष्कलस्य तु दारुणम्}% ३६

\twolineshloka
{मूर्च्छितस्तेन शूलेन निहतो धन्विसत्तमः}
{कश्मलं परमं प्राप्तः पपात स्यन्दनोपरि}% ३७

\twolineshloka
{मूर्च्छां प्राप्तं तमाज्ञाय हनूमान्पवनात्मजः}
{कोपव्याकुलितस्वान्तो बभाषे तं तु राक्षसम्}% ३८

\twolineshloka
{कुत्र गच्छसि दुर्बुद्धे मयि योद्धरि संस्थिते}
{त्वां हन्मि चरणाघातैर्वाजिहर्तारमागतम्}% ३९

\twolineshloka
{एवमुक्त्वा महादैत्याञ्जघान परसैनिकान्}
{विमानस्थान्नखाग्रेण दारयन्नभसि स्थितः}% ४०

\twolineshloka
{लाङ्गूलेनाहताः केचित्केचित्पादतला हताः}
{बाहुभ्यां दारिताः केचित्पवनस्य तनूभुवा}% ४१

\twolineshloka
{नश्यन्ति केचिन्निहताः केचिन्मूर्च्छन्ति संहताः}
{पलायन्ते पदाघातभयपीडाहतास्ततः}% ४२

\twolineshloka
{अनेके निहतास्तत्र राक्षसाश्चातिदारुणाः}
{छिन्ना भिन्ना द्विधा जाताः पवनस्य सुतेन वै}% ४३

\twolineshloka
{कामगन्तुविमानं तद्भिन्नप्राकारतोरणम्}
{हाहा कुर्वद्भिरसुरैः समन्तात्परिवारितम्}% ४४

\twolineshloka
{हनूमति महाशूरे क्षणं भूमौ क्षणं दिवि}
{इतस्ततः प्रदृश्येत कामयानं दुरासदम्}% ४५

\twolineshloka
{यत्रयत्र विमानं तत्तत्रतत्र समीरजः}
{प्रहरन्नेव दृश्येत कामरूपधरः कपिः}% ४६

\twolineshloka
{एवं तदाकुलीभूते विमानस्थे महाजने}
{उग्रदंष्ट्रस्तु दैत्येन्द्रो हनूमन्तमुपेयिवान्}% ४७

\twolineshloka
{कपे त्वया महत्कर्म कृतं यद्भटपातनम्}
{क्षणं तिष्ठसि चेत्कुर्वे तव प्राणवियोजनम्}% ४८

\twolineshloka
{एवमुक्त्वा हनूमन्तं प्रजघान स दुर्मतिः}
{त्रिशूलेन सुतीक्ष्णेन ज्वलत्पावककान्तिना}% ४९

\twolineshloka
{तदागतं त्रिशूलं च मुखे जग्राह वीर्यवान्}
{चूर्णयामास सकलं सर्वलोहविनिर्मितम्}% ५०

\twolineshloka
{चूर्णयित्वा त्रिशूलं तदायसं दैत्यमोचितम्}
{जघान तं चपेटाभिर्बह्वीभिर्हनुमान्बली}% ५१

\twolineshloka
{स आहतः कपीन्द्रेण चपेटाभिरितस्ततः}
{व्यथितो व्यसृजन्मायां सर्वलोकभयङ्करीम्}% ५२

\twolineshloka
{तदा तमोभवत्तीव्रं यत्र को वा न लक्ष्यते}
{यत्र स्वीयो न पारक्यो विदामास जनान्बहून्}% ५३

\twolineshloka
{शिलाः पर्वतशृङ्गाभाः पतन्ति सुभटोपरि}
{ताभिर्हतास्तु ते सर्वे व्याकुला अथ जज्ञिरे}% ५४

\twolineshloka
{विद्युतो विलसन्त्यत्र गर्जन्ति जलदा घनम्}
{वर्षन्ति पूयरुधिरं मुञ्चन्ति समलं जलम्}% ५५

\twolineshloka
{आकाशात्पतमानानि कबन्धानि बहूनि च}
{दृश्यन्ते छिन्नशीर्षाणि सकुण्डलयुतानि च}% ५६

\twolineshloka
{नग्ना विरूपाः सुभृशं कीर्णकेशाः सुदुर्मुखाः}
{दृश्यन्ते सर्वतो दैत्या दारुणा भयकारिणः}% ५७

\twolineshloka
{तदा व्याकुलिता लोकाः परस्परभयाकुलाः}
{पलायनपरा जाता महोत्पातममंसत}% ५८

\twolineshloka
{तदा शत्रुघ्न आयातो रथे स्थित्वा महायशाः}
{श्रीरामस्मरणं कृत्वा चापे सन्धाय सायकान्}% ५९

\twolineshloka
{तां मायां स विधूयाथ मोहनास्त्रेण वीर्यवान्}
{शरधाराः किरन्व्योम्नि ववर्ष समरेसुरम्}% ६०

\twolineshloka
{तदादिशः प्रसेदुस्ता रविस्त्वपरिवेषवान्}
{मेघा यथागतं याता विद्युतः शान्तिमागताः}% ६१

\twolineshloka
{तदा विमानं पुरतो दृश्यते राक्षसैर्युतम्}
{छिन्धि भिन्धीति भाषाभिर्व्याकुलं सुतरां महत्}% ६२

\twolineshloka
{बाणाश्च शतसाहस्राः स्वर्णपुङ्खैश्च शोभिताः}
{पेतुर्विमाने नभसि स्थिते कामगमे मुहुः}% ६३

\twolineshloka
{तदा भग्नं विमानं हि दृश्यते न तदुच्चकैः}
{स्वपुरी खण्डमेकत्र भग्नाङ्गमिव भूतले}% ६४

\twolineshloka
{तदा प्रकुपितो दैत्यो बाणान्धनुषि सन्दधे}
{तैर्बाणैर्विकिरन्रामभ्रातरं चाभिगर्जितः}% ६५

\twolineshloka
{ते बाणाः शतशस्तस्य लग्ना वपुषि भूरिशः}
{शोभामापुः शोणितौघान्वहन्तस्तीक्ष्णवक्त्रिणः}% ६६

\twolineshloka
{शत्रुघ्नः परया शक्त्या संयुक्तो वायुदैवतम्}
{अस्त्रं धनुषि चाधत्त राक्षसानां प्रकम्पनम्}% ६७

\twolineshloka
{तेनास्त्रेण विमानात्खात्पतन्तो मुक्तमूर्धजाः}
{दृश्यन्ते भूतवेतालसङ्घा इव नभश्चराः}% ६८

\twolineshloka
{तदस्त्रं रघुनाथस्य भ्रात्रा मुक्तं विलोक्य सः}
{अस्त्रं च पाशुपत्यं स चापे धाद्दनुजात्मजः}% ६९

\twolineshloka
{ततः प्रवृत्ता वेताला भूताः प्रेतनिशाचराः}
{कपालकर्तरीयुक्ताः पिबन्तः शोणितं बहु}% ७०

\twolineshloka
{ते वै शत्रुघ्नवीराणां रुधिराणि पपुर्मुदा}
{जीवतामपि दुर्वाराः कर्तरीपाणिशोभिताः}% ७१

\twolineshloka
{तदस्त्रं व्याप्नुवद्दृष्ट्वा सर्ववीरप्रभञ्जनम्}
{मुमोच तन्निरासाय चास्त्रं नारायणाभिधम्}% ७२

\twolineshloka
{नारायणास्त्रं तान्सर्वान्वारयामास तत्क्षणात्}
{ते सर्वे विलयं प्रापुर्निशाचरप्रणोदिताः}% ७३

\twolineshloka
{तदा क्रुद्धो निशाचारी विद्युन्माली समाददे}
{त्रिशूलं निशितं घोरं शत्रुघ्नं हन्तुमुल्बणम्}% ७४

\twolineshloka
{शूलहस्तं समायान्तं विद्युन्मालिनमाहवे}
{सायकैः प्राहरत्तस्य भुजे त्वर्धशशिप्रभैः}% ७५

\twolineshloka
{तैर्बाणैश्छिन्नहस्तः स शिरसा हन्तुमुद्यतः}
{हतोसि याहि शत्रुघ्न कस्त्वां त्राता भविष्यति}% ७६

\twolineshloka
{इति ब्रुवाणं तरसा चिच्छेद शितसायकैः}
{मस्तकं तस्य बलिनः शूरस्य सह कुण्डलम्}% ७७

\twolineshloka
{तं छिन्नशिरसं दृष्ट्वा उग्रदंष्ट्रः प्रतापवान्}
{मुष्टिना हन्तुमारेभे शत्रुघ्नं शूरसेवितम्}% ७८

\twolineshloka
{शत्रुघ्नस्तु क्षुरप्रेण सायकेनाच्छिनच्छिरः}
{प्रधावतो रणे वीरान्सर्वशस्त्रास्त्रकोविदान्}% ७९

\twolineshloka
{हतशेषा ययुः सर्वे राक्षसा नाथवर्जिताः}
{शत्रुघ्नं प्रणिपत्याथ ददुर्वाजिनमाहृतम्}% ८०

\twolineshloka
{ततो वीणानिनादाश्च शङ्खनादाः समन्ततः}
{श्रूयन्ते शूरवीराणां जयनादा मनोहराः}% ८१

{॥इति श्रीपद्मपुराणे पातालखण्डे शेषवात्स्यायनसंवादे रामाश्वमेधे विद्युन्मालिनामराक्षसपराजयो नाम चतुस्त्रिंशत्तमोऽध्यायः॥३४॥}

\dnsub{पञ्चत्रिंशत्तमोऽध्यायः}%\resetShloka

\uvacha{शेष उवाच}

\twolineshloka
{प्राप्य तं वाजिनं राजा शत्रुघ्नो राक्षसैर्हृतम्}
{अत्यन्तं हर्षमापेदे पुष्कलेन समन्वितः}% १

\twolineshloka
{रुधिरैः सिक्तगात्रास्ते योधा लक्ष्मीनिधिस्तथा}
{रणोत्साहेन संयुक्तं प्रशशंसुर्महानृपम्}% २

\twolineshloka
{हते तस्मिन्महादैत्ये विद्युन्मालिनि दुर्जये}
{सुराः सर्वे भयं त्यक्त्वा सुखमापुर्मुनेमहत्}% ३

\twolineshloka
{नद्यस्तु विमला जाता रविस्तु विमलोऽभवत्}
{वाता ववुः सुगन्धोद सिक्ता विमलशुष्मिणः}% ४

\twolineshloka
{सन्नद्धास्ते महावीरा रथस्था विमलाङ्गकाः}
{राजानमूचुस्ते सर्वे जयलक्ष्म्या समन्विताः}% ५

\uvacha{वीरा ऊचुः}

\twolineshloka
{दिष्ट्या हतस्त्वया दैत्यो विद्युन्माली महामते}
{यद्भयात्त्रासमापन्नाः सुराः स्वर्गान्निराकृताः}% ६

\twolineshloka
{दिष्ट्या प्राप्तो महावाजी रघुनाथस्य शोभनः}
{दिष्ट्या गन्तासि सर्वत्र जयं तु क्षितिमण्डले}% ७

\twolineshloka
{स्वामिन्मुञ्चत्विमं वाहं मनोवेगं मनोरमम्}
{समयस्य विलम्बो मा भवत्वत्र महामते}% ८

\uvacha{शेष उवाच}

\twolineshloka
{इति श्रुत्वा तु तद्वाक्यं वीराणां समयोचितम्}
{साधु साधु प्रशस्यैतान्मुमोच हयसत्तमम्}% ९

\twolineshloka
{स मुक्तश्चोत्तरामाशां बभ्रामाथ सुरक्षितः}
{रथपत्तिहयश्रेष्ठैः सर्वशस्त्रास्त्रकोविदैः}% १०

\twolineshloka
{तत्र यद्वृत्तमेतस्य शत्रुघ्नस्य मनोहरम्}
{वात्स्यायन शृणुष्वैतत्पापराशिप्रदाहकम्}% ११

\twolineshloka
{रेवातीरमथ प्राप्तो मुनिवृन्दनिषेवितम्}
{नीलरत्नसमूहस्य रसः किं तु पयो मिषात्}% १२

\twolineshloka
{तांस्तान्मुनिवरान्सर्वान्प्रणमञ्छूरसेवितः}
{जगाम हयरत्नस्य पृष्ठतः कामगामिनः}% १३

\twolineshloka
{गच्छंस्तत्राश्रमं जीर्णं पलाशपर्णनिर्मितम्}
{रेवायाजलकल्लोलैः सिक्तं पापहराश्रयम्}% १४

\twolineshloka
{तं दृष्ट्वा सुमतिं प्राह सर्वज्ञं नयकोविदम्}
{शत्रुघ्नः सर्वधर्मार्थकर्मकर्तव्यकोविदः}% १५

\uvacha{राजोवाच}

\twolineshloka
{मन्त्रिन्कथय कस्यायमाश्रमः पुण्यदर्शनः}
{विचारचतुरश्रेष्ठ वदैतन्मम पृच्छतः}% १६

\uvacha{शेष उवाच}

\twolineshloka
{इति वाक्यं समाकर्ण्य सुमतिः प्राह तं नृपम्}
{विशद स्मेरया वाचा दर्शयन्नात्मसौहृदम्}% १७

\uvacha{सुमतिरुवाच}

\twolineshloka
{एनं दृष्ट्वा महाराज धूतपापा वयं खलु}
{भविष्यामो मुनिश्रेष्ठं सर्वशास्त्रपरायणम्}% १८

\twolineshloka
{तस्मान्नत्वा तमापृच्छ सर्वं ते कथयिष्यति}
{रघुनाथपदाम्भोजमकरन्दाति लोलुपः}% १९

\twolineshloka
{नाम्ना त्वारण्यकं ख्यातं रघुनाथाङ्घ्रिसेवकम्}
{अत्युग्रतपसा पूर्णं सर्वशास्त्रार्थकोविदम्}% २०

\twolineshloka
{इति श्रुत्वा तु तद्वाक्यं धर्मार्थपरिबृंहितम्}
{जगाम तमथो द्रष्टुं स्वल्पसेवकसंयुतः}% २१

\twolineshloka
{हनूमान्पुष्कलो वीरः सुमतिर्मन्त्रिसत्तमः}
{लक्ष्मीनिधिः प्रतापाग्र्यः सुबाहुः सुमदस्तथा}% २२

\twolineshloka
{एतैः परिवृतो राजा शत्रुघ्नः प्रापदाश्रमम्}
{नमस्कर्तुं द्विजवरमारण्यकमुदारधीः}% २३

\twolineshloka
{गत्वा तं तापसश्रेष्ठं नमस्कारमथाकरोत्}
{सर्वैस्तैः सहितो वीरैर्विनयानतकन्धरैः}% २४

\twolineshloka
{तान्दृष्ट्वा सन्नतान्सर्वाञ्छत्रुघ्नप्रमुखान्नृपान्}
{अर्घ्यपाद्यादिकं चक्रे फलमूलादिभिस्तदा}% २५

\twolineshloka
{उवाच तान्नृपान्सर्वान्भवन्तः कुत्र सङ्गताः}
{कथमत्र समायातास्तत्सर्वं वदतानघाः}% २६

\twolineshloka
{तच्छ्रुत्वा वाक्यमेतस्य मुनिवर्यस्य वाडव}
{सुमतिः कथयामास वाक्यं वादविचक्षणः}% २७

\uvacha{सुमतिरुवाच}

\twolineshloka
{रघुवंशनृपस्यायमश्वो वै पाल्यतेऽखिलैः}
{यागं करिष्यते वीरः सर्वसम्भारसम्भृतम्}% २८

\twolineshloka
{तच्छ्रुत्वा वचनं तेषां जगाद मुनिसत्तमः}
{दन्तकान्त्याखिलं घोरं तमोनिर्वारयन्निव}% २९

\uvacha{आरण्यक उवाच}

\twolineshloka
{किं यागैर्विविधैरन्यैः सर्वसम्भारसम्भृतैः}
{स्वल्पपुण्यप्रदैर्नूनं क्षयिष्णुपददातृभिः}% ३०

\twolineshloka
{मूढो लोको हरिं त्यक्त्वा करोत्यन्यसमर्चनम्}
{रघुवीरं रमानाथं स्थिरैश्वर्यपदप्रदम्}% ३१

\twolineshloka
{यो नरैः स्मृतमात्रोपि हरते पापपर्वतम्}
{तं मुक्त्वा क्लिश्यते मूढो यागयोगव्रतादिभिः}% ३२

\twolineshloka
{अहो पश्यत मूढत्वं लोकानामतिवञ्चितम्}
{सुलभं रामभजनं मुक्त्वा दुर्ल्लभमाचरेत्}% ३३

\twolineshloka
{सकामैर्योगिभिर्वापि चिन्त्यते कामवर्जितैः}
{अपवर्गप्रदं नॄणां स्मृतमात्राखिलाघहम्}% ३४

\twolineshloka
{पुराहं तत्त्ववित्सायां ज्ञानिनं सुविचारयन्}
{अगमं बहुतीर्थानि तत्त्वं कोपि न मेऽदिशत्}% ३५

\twolineshloka
{तदैकं हि महद्भाग्यात्प्राप्तं वै लोमशं मुनिम्}
{स्वर्गलोकात्समायातं तीर्थयात्राचिकीर्षया}% ३६

\twolineshloka
{तमहं प्रणिपत्याथ पर्यपृच्छं महामुनिम्}
{महायुषं महायोगिसंसेवितपदद्वयम्}% ३७

\twolineshloka
{स्वामिन्मयाद्य मानुष्यं प्राप्तमद्भुतदुर्ल्लभम्}
{संसारघोरजलधिं किं कर्तव्यं तितीर्षुणा}% ३८

\twolineshloka
{विचार्य कथय त्वं तद्व्रतं दानं जपो मखः}
{देवो वा विद्यते यो वै संसृत्यम्भोधितारकः}% ३९

\twolineshloka
{यज्ज्ञात्वा संसृतिं घोरां तरामि त्वत्कृपाब्धितः}
{तन्मे कथय योगेश सर्वशास्त्रार्थपारग}% ४०

\twolineshloka
{इति मद्वाक्यमाकर्ण्य जगाद मुनिसत्तमः}
{शृणुष्वैकमना विप्र श्रद्धया परया युतः}% ४१

\twolineshloka
{सन्ति दानानि तीर्थानि व्रतानि नियमा यमाः}
{योगा यज्ञास्तथानेके वर्तन्ते स्वर्गदायकाः}% ४२

\twolineshloka
{परं गुह्यं प्रवक्ष्यामि सर्वपापप्रणाशनम्}
{तच्छृणुष्व महाभाग संसाराम्भोधितारकम्}% ४३

\twolineshloka
{नास्तिकाय न वक्तव्यं न चाऽश्रद्धालवे पुनः}
{निन्दकाय शठायापि न देयं भक्तिवैरिणे}% ४४

\twolineshloka
{रामभक्ताय शान्ताय कामक्रोधवियोगिने}
{वक्तव्यं सर्वदुःखस्य नाशकारकमुत्तमम्}% ४५

\twolineshloka
{रामान्नास्ति परो देवो रामान्नास्ति परं व्रतम्}
{न हि रामात्परो योगो न हि रामात्परो मखः}% ४६

\twolineshloka
{तं स्मृत्वा चैव जप्त्वा च पूजयित्वा नरः परम्}
{प्राप्नोति परमामृद्धिमैहिकामुष्मिकीं तथा}% ४७

\twolineshloka
{संस्मृतो मनसा ध्यातः सर्वकामफलप्रदः}
{ददाति परमां भक्तिं संसाराम्भोधितारिणीम्}% ४८

\twolineshloka
{श्वपाकोपि हि संस्मृत्य रामं याति परां गतिम्}
{ये वेदशास्त्रनिरतास्त्वादृशास्तत्र किं पुनः}% ४९

\twolineshloka
{सर्वेषां वेदशास्त्राणां रहस्यं ते प्रकाशितम्}
{समाचर तथा त्वं वै यथा स्यात्ते मनीषितम्}% ५०

\twolineshloka
{एको देवो रामचन्द्रो व्रतमेकं तदर्चनम्}
{मन्त्रोऽप्येकश्च तन्नाम शास्त्रं तद्ध्येव तत्स्तुतिः}% ५१

\twolineshloka
{तस्मात्सर्वात्मना रामचन्द्रं भजमनोहरम्}
{यथा गोष्पदवत्तुच्छो भवेत्संसारसागरः}% ५२

\twolineshloka
{श्रुत्वा मया तु तद्वाक्यं पुनः प्रश्नमकारिषम्}
{कथं वा ध्यायते देवः कथं वा पूज्यते नरैः}% ५३

\twolineshloka
{कथयस्व महाबुद्धे सर्वज्ञ मम विस्तरात्}
{यज्ज्ञात्वाहं कृतार्थः स्यां त्रिलोक्यां मुनिसत्तम}% ५४

\twolineshloka
{एतच्छ्रुत्वा तु मद्वाक्यं विचार्य स तु लोमशः}
{कथयामास मे सर्वं रामध्यानपुरःसरम्}% ५५

\twolineshloka
{शृणु विप्रेन्द्र वक्ष्यामि यत्पृष्टं तु त्वयानघ}
{यथा तुष्येद्रमानाथः संसारज्वरदाहकः}% ५६

\twolineshloka
{अयोध्यानगरे रम्ये चित्रमण्डपशोभिते}
{ध्यायेत्कल्पतरोर्मूले सर्वकामसमृद्धिदे}% ५७

\twolineshloka
{महामरकतस्वर्णनीलरत्नादिशोभितम्}
{सिंहासनं चित्तहरं कान्त्या तामिस्रनाशनम्}% ५८

\twolineshloka
{तस्योपरि समासीनं रघुराजं मनोरमम्}
{दूर्वादलश्यामतनुं देवदेवेन्द्रपूजितम्}% ५९

\twolineshloka
{राकायां पूर्णशीतांशुकान्तिधिक्कारिवक्त्रिणम्}
{अष्टमीचन्द्रशकलसमभालाधिधारिणम्}% ६०

\twolineshloka
{नीलकुन्तलशोभाढ्यं किरीटमणिरञ्जितम्}
{मकराकारसौन्दर्यकुण्डलाभ्यां विराजितम्}% ६१

\twolineshloka
{विद्रुमच्छवि सत्कान्तिरदच्छदविराजितम्}
{तारापतिकराकार द्विजराजि सुशोभितम्}% ६२

\twolineshloka
{जपापुष्पाभया माध्व्या जिह्वया शोभिताननम्}
{यस्यां वसन्ति निगमा ऋगाद्याः शास्त्रसंयुताः}% ६३

\twolineshloka
{कम्बुकान्तिधरग्रीवा शोभया समलङ्कृतम्}
{सिंहवदुच्चकौ स्कन्धौ मांसलौ बिभ्रतं वरम्}% ६४

\twolineshloka
{बाहू दधानं दीर्घाङ्गौ केयूरकटकाङ्कितौ}
{मुद्रिकाहीरशोभाभिर्भूषितौ जानुलम्बिनौ}% ६५

\twolineshloka
{वक्षो दधानं विपुलं लक्ष्मीवासेन शोभितम्}
{श्रीवत्सादिविचित्राङ्कैरङ्कितं सुमनोहरम्}% ६६

\twolineshloka
{महोदरं महानाभिं शुभकट्याविराजितम्}
{काञ्च्या वै मणिमत्या च विशेषेण श्रियान्वितम्}% ६७

\twolineshloka
{ऊरुभ्यां विमलाभ्यां वै जानुभ्यां शोभितं श्रिया}
{चरणाभ्यां वज्ररेखा यवाङ्कुशसुरेखया}% ६८

\twolineshloka
{युताभ्यां योगिध्येयाभ्यां कोमलाभ्यां विराजितम्}
{ध्यात्वा स्मृत्वा च संसारसागरं त्वं तरिष्यसि}% ६९

\twolineshloka
{तमेव पूजयन्नित्यं चन्दनादिभिरिच्छया}
{प्राप्नोति परमामृद्धिमैहिकामुष्मिकीं पराम्}% ७०

\twolineshloka
{त्वया पृष्टं महाराज रामस्य ध्यानमुत्तमम्}
{तत्ते कथितमेतद्वै संसारजलधिं तर}% ७१

{॥इति श्रीपद्मपुराणे पातालखण्डे शेषवात्स्यायनसंवादे रामाश्वमेधे आरण्यको पाख्याने लोमशारण्यकसंवादो नाम पञ्चत्रिंशत्तमोऽध्यायः॥३५॥}

\dnsub{षट्त्रिंशत्तमोऽध्यायः}%\resetShloka

\uvacha{शेषउवाच}

\twolineshloka
{एतच्छ्रुत्वा तु विप्रेन्द्रो लोमशात्परमं महत्}
{पुनः पप्रच्छ तमृषिं सर्वज्ञं योगिनां वरम्}% १

\uvacha{आरण्यक उवाच}

\twolineshloka
{मुनिश्रेष्ठ वदैतन्मे पृच्छामि त्वां महामते}
{गुरवः कृपया युक्ता भाषन्ते सेवकेऽखिलम्}% २

\twolineshloka
{कोऽसौ रामो महाभाग यो नित्यं ध्यायते त्वया}
{तस्य कानि चरित्राणि वदस्व त्वं द्विजर्षभ}% ३

\twolineshloka
{किमर्थमवतीर्णोऽसौ कस्मान्मानुषतां गतः}
{तत्सर्वं कथयाशु त्वं मम संशयनुत्तये}% ४

\uvacha{शेष उवाच}

\twolineshloka
{इति वाक्यं समाकर्ण्य मुनेः परमशोभनम्}
{लोमशः कथयामास रामचारित्रमद्भुतम्}% ५

\twolineshloka
{लोकान्निरयसम्मग्नांज्ञात्वा योगेश्वरेश्वरः}
{कीर्तिं प्रथयितुं लोके यया घोरं तरिष्यति}% ६

\twolineshloka
{एवं ज्ञात्वा दयावार्धिः परमेशो मनोहरः}
{अवतारं चकारात्र चतुर्धा सश्रियान्वितः}% ७

\twolineshloka
{पुरा त्रेतायुगे प्राप्ते पूर्णांशो रघुनन्दनः}
{सूर्यवंशे समुत्पन्नो रामो राजीवलोचनः}% ८

\twolineshloka
{स रामो लक्ष्मणसखः काकपक्षधरो युवा}
{तातस्य वचनात्तौ तु विश्वामित्रमनुव्रतौ}% ९

\twolineshloka
{यज्ञसंरक्षणार्थाय राज्ञा दत्तौ कुमारकौ}
{दान्तौ धनुर्धरौ वीरौ विश्वामित्रमनुव्रतौ}% १०

\twolineshloka
{पथि प्रव्रजतोस्तत्र ताटका नाम राक्षसी}
{सङ्गता च वने घोरे तयोर्वै विघ्नकारणात्}% ११

\twolineshloka
{ऋषेरनुज्ञया रामस्ताटकां यमयातनाम्}
{प्रावेशयद्धनुर्वेदविद्याभ्यासेन राघवः}% १२

\twolineshloka
{यस्य पादतलस्पर्शाच्छिला वासवयोगजा}
{अहल्या गौतमवधूः पुनर्जाता स्वरूपिणी}% १३

\twolineshloka
{विश्वामित्रस्य यज्ञे तु सुप्रवृत्ते रघूत्तमः}
{मारीचं च सुबाहुं च जघान परमेषुभिः}% १४

\twolineshloka
{ईश्वरस्य धनुर्भग्नं जनकस्य गृहे स्थितम्}
{रामः पञ्चदशे वर्षे षड्वर्षामथ मैथिलीम्}% १५

\twolineshloka
{उपयेमे विवाहेन रम्यां सीतामयोनिजाम्}
{कृतकृत्यस्तदा जातः सीतां सम्प्राप्य राघवः}% १६

\twolineshloka
{ततो द्वादश वर्षाणि रेमे रामस्तया सह}
{सप्तविंशतिमे वर्षे यौवराज्यमकल्पयत्}% १७

\twolineshloka
{राजानमथ कैकेयी वरद्वयमयाचत}
{तयोरेकेन रामस्तु ससीतः सह लक्ष्मणः}% १८

\twolineshloka
{जटाधरः प्रव्रजतुवर्षाणीह चतुर्दश}
{भरतस्तु द्वितीयेन यौवराज्याधिपोऽस्तु मे}% १९

\twolineshloka
{जानकी लक्ष्मणसखं रामं प्राव्राजयन्नृपः}
{त्रिरात्रमुदकाहारश्चतुर्थेऽह्नि फलाशनः}% २०

\twolineshloka
{पञ्चमे चित्रकूटे तु रामस्थानमकल्पयत्}
{अथ त्रयोदशे वर्षे पञ्चवट्यां महामुने}% २१

\twolineshloka
{रामो विरूपयामास शूर्पणखां निशाचरीम्}
{वने विचरतस्तस्य जानक्या सहितस्य च}% २२

\twolineshloka
{आगतो राक्षसस्तां तु हर्तुं पापविपाकतः}
{ततो माघासिताष्टम्यां मुहूर्ते वृन्दसंज्ञिते}% २३

\twolineshloka
{राघवाभ्यां विना सीतां जहार दशकन्धरः}
{तेनैवं ह्रियमाणा सा चक्रन्द कुररी यथा}% २४

\twolineshloka
{रामरामेति मां रक्ष रक्ष मां रक्षसा हृताम्}
{यथा श्येनः क्षुधाक्रान्तः क्रन्दन्तीं वर्तिकां नयेत्}% २५

\twolineshloka
{तथा कामवशं प्राप्तो रावणो जनकात्मजाम्}
{नयत्येवं जनकजां जटायुः पक्षिराट्तदा}% २६

\twolineshloka
{युयुधे राक्षसेन्द्रेण स रावणहतोऽपतत्}
{मार्गशुक्लनवम्यां तु वसन्तीं रावणालये}% २७

\twolineshloka
{सम्पातिर्दशमे मास आचख्यौ वानरेषु ताम्}
{एकादश्यां महेन्द्राद्रे पुःप्लुवे शतयोजनम्}% २८

\twolineshloka
{हनूमान्निशि तस्यां तु लङ्कायां पर्यकालयत्}
{तद्रात्रिशेषे सीताया दर्शनं हि हनूमतः}% २९

\twolineshloka
{द्वादश्यां शिंशपावृक्षे हनूमान्पर्यवस्थितः}
{तस्यां निशायां जानक्या विश्वासाय च सङ्कथा}% ३०

\twolineshloka
{अक्षादिभिस्त्रयोदश्यां ततो युद्धमवर्तत}
{ब्रह्मास्त्रेण चतुर्दश्यां बद्धः शक्रजिता कपिः}% ३१

\twolineshloka
{वह्निना पुच्छयुक्तेन लङ्काया दहनं कृतम्}
{पूर्णिमायां महेन्द्राद्रौ पुनरागमनं कपेः}% ३२

\twolineshloka
{मार्गासितप्रतिपदः पञ्चभिः पथिवासरैः}
{पुनरागत्य षष्ठेऽह्नि ध्वस्तं मधुवनं किल}% ३३

\twolineshloka
{सप्तम्यां प्रत्यभिज्ञानदानं सर्वनिवेदनम्}
{अष्टम्युत्तरफल्गुन्यां मुहूर्ते विजयाभिधे}% ३४

\twolineshloka
{मध्यं प्राप्ते सहस्रांशौ प्रस्थानं राघवस्य च}
{रामः कृत्वा प्रतिज्ञां तु प्रयातो दक्षिणां दिशम्}% ३५

\twolineshloka
{तीर्त्वाहं सागरमपि हनिष्ये राक्षसेश्वरम्}
{दक्षिणाशां प्रयातस्य सुग्रीवोऽप्यभवत्सखा}% ३६

\twolineshloka
{वासरैः सप्तभिः सिन्धोः स्कन्धावारनिवेशनम्}
{पौषशुक्लप्रतिपदस्तृतीयायावदम्बुधेः}% ३७

\twolineshloka
{उपस्थानं ससैन्यस्य राघवस्य बभूव ह}
{बिभीषणश्चतुर्थ्यां तु रामेण सह सङ्गतः}% ३८

\twolineshloka
{समुद्रतरणार्थाय पञ्चम्यां मन्त्र उद्यतः}
{प्रायोपवेशनं चक्रे रामो दिनचतुष्टयम्}% ३९

\twolineshloka
{समुद्रवरलाभश्च सहोपायप्रदर्शनम्}
{ततो दशम्यामारम्भस्त्रयोदश्यां समापनम्}% ४०

\twolineshloka
{चतुर्दश्यां सुवेलाद्रौ रामः सैन्यं न्यवेशयत्}
{पौर्णमास्यां द्वितीयां तं त्रिदिनैः सैन्यतारणम्}% ४१

\twolineshloka
{तीर्त्वा तोयनिधिं रामो वानरेश्वरसैन्यवान्}
{रुरोध च पुरीं लङ्कां सीतार्थं सह लक्ष्मणः}% ४२

\twolineshloka
{तृतीयादि दशम्यन्तं निवेशश्च दिनाष्टकम्}
{शुकसारणयोस्तत्र प्राप्तिरेकादशे दिने}% ४३

\twolineshloka
{पौषासिताख्यद्वादश्यां सैन्यसङ्ख्यानमेव च}
{शार्दूलेन कपीन्द्राणां सहसारोपवर्णनम्}% ४४

\twolineshloka
{त्रयोदश्या अमावास्यां लङ्कायां दिवसैस्त्रिभिः}
{रावणः सैन्यसङ्ख्यानं रणोत्साहं तदाकरोत्}% ४५

\twolineshloka
{प्रययावङ्गदो दौत्यं माघशुक्लाद्यवासरे}
{सीतायाश्च ततो भर्तुर्मायामूर्द्धादिदर्शनम्}% ४६

\twolineshloka
{माघद्वितीयादि दिनैः सप्तभिर्यावदष्टमी}
{रक्षसां वानराणां च युद्धमासीच्च सङ्कुलम्}% ४७

\twolineshloka
{माघशुक्लनवम्यां तु रात्राविन्द्रजिता रणे}
{रामलक्ष्मणयोर्नागपाशबन्धः कृतः किल}% ४८

\twolineshloka
{आकुलेषु कपीशेषु निरुत्साहेषु सर्वशः}
{नागपाशविमोक्षार्थं दशम्यां पवनोऽजपत्}% ४९

\twolineshloka
{कर्णे स्वरूपं रामस्य गरुडागमनं ततः}
{एकादश्यां च द्वादश्यां धूम्राक्षस्य वधः कृतः}% ५०

\twolineshloka
{त्रयोदश्यां तु तेनैव निहतः कम्पनो रणे}
{माघशुक्लचतुर्दश्या यावत्कृष्णादिवासरम्}% ५१

\twolineshloka
{त्रिदिनेन प्रहस्तस्य नीलेन विहितो वधः}
{माघकृष्णद्वितीयायाश्चतुर्थ्यं तं त्रिभिर्दिनैः}% ५२

\twolineshloka
{रामेण तुमुले युद्धे रावणो द्रावितो रणात्}
{पञ्चम्या अष्टमीयावद्रावणेन प्रबोधितः}% ५३

\twolineshloka
{कुम्भकर्णस्तदा चक्रेऽभ्यवहारं चतुर्दिनम्}
{कुम्भकर्णो दिनैः षड्भिर्नवम्यास्तु चतुर्दशीम्}% ५४

\twolineshloka
{रामेण निहतो युद्धे बहुवानरभक्षकः}
{अमावास्यादिने शोकादवहारो बभूव ह}% ५५

\twolineshloka
{फाल्गुनादिप्रतिपदश्चतुर्थ्यन्तं चतुर्दिनैः}
{बिसतन्तुप्रभृतयो निहताः पञ्चराक्षसाः}% ५६

\twolineshloka
{पञ्चम्याः सप्तमी यावदतिकायवधस्तथा}
{अष्टम्याद्वादशी यावन्निहतौ दिनपञ्चकात्}% ५७

\twolineshloka
{निकुम्भकुम्भावूर्ध्वं तु मकराक्षस्त्रिभिर्दिनैः}
{फाल्गुनासितद्वितीयायां दिने शक्रजिता जितम्}% ५८

\twolineshloka
{तृतीयादिसप्तम्यन्तं दिनपञ्चकमेव च}
{ओषध्यानयनव्यग्रादवहारो बभूव ह}% ५९

\twolineshloka
{ततस्त्रयोदशीयावद्दिनैः पञ्चभिरिन्द्रजित्}
{लक्ष्मणेन हतो युद्धे विख्यातबलपौरुषः}% ६०

\twolineshloka
{चतुर्दश्यां दशग्रीवो दीक्षां प्रापावहारतः}
{अमावास्यादिने प्रायाद्युद्धाय दशकन्धरः}% ६१

\twolineshloka
{चैत्रशुक्लप्रतिपदः पञ्चमीदिनपञ्चकैः}
{रावणे युद्ध्यमाने तु प्रचुरो रक्षसां वधः}% ६२

\twolineshloka
{चैत्रषष्ठ्याष्टमी यावन्महापार्श्वादि मारणम्}
{चैत्रशुक्लनवम्यां तु सौमित्रेः शक्तिभेदनम्}% ६३

\twolineshloka
{कोपाविष्टेन रामेण द्रावितो दशकन्धरः}
{द्रोणाद्रिराञ्जनेयेन लक्ष्मणार्थमुपाहृतः}% ६४

\twolineshloka
{दशम्यामवहारोभूद्रात्रौ युद्धे तु रक्षसाम्}
{एकादश्यां तु रामाय रथं मातलिसारथिः}% ६५

\twolineshloka
{प्रेरितो वासवेनाजावर्पयामास भक्तितः}
{कोपवानथ द्वादश्या यावत्कृष्णचतुर्दशी}% ६६

\twolineshloka
{अष्टादशदिनै रामो रावणं द्वैरथेऽवधीत्}
{सङ्ग्रामे तुमुले जाते रामो जयमवाप्तवान्}% ६७

\twolineshloka
{माघशुक्लद्वितीयायाश्चैत्रकृष्ण चतुर्दशीम्}
{सप्ताशीतिदिनेष्वेव मध्यं पञ्चदशाहकम्}% ६८

\twolineshloka
{युद्धावहारः सङ्ग्रामो द्वासप्तति दिनान्यभूत्}
{संस्कारो रावणादीनाममावस्या दिनेऽभवत्}% ६९

\twolineshloka
{वैशाखादि तिथौ राम उवास रणभूमिषु}
{अभिषिक्तो द्वितीयायां लङ्काराज्ये विभीषणः}% ७०

\twolineshloka
{सीताशुद्धिस्तृतीयायां देवेभ्यो वरलम्भनम्}
{हत्वा चिरेण लङ्केशं लक्ष्मणाग्रज एव सः}% ७१

\twolineshloka
{गृहीत्वा जानकीं पुण्यां दुःखितां राक्षसेन तु}
{आदाय परया प्रीत्या जानकीं स न्यवर्तत}% ७२

\twolineshloka
{वैशाखस्य चतुर्थ्यां तु रामः पुष्पकमाश्रितः}
{विहायसा निवृत्तस्तु भूयोऽयोध्यां पुरीं प्रति}% ७३

\twolineshloka
{पूर्णे चतुर्दशे वर्षे पञ्चम्यां माधवस्य तु}
{भरद्वाजाश्रमे रामः सगणः समुपाविशत्}% ७४

\twolineshloka
{नन्दिग्रामे तु षष्ठ्यां स भरतेन समागतः}
{सप्तम्यामभिषिक्तोऽसावयोध्यायां रघूद्वहः}% ७५

\twolineshloka
{दशैकाधिकमासांस्तुचतुर्दशाहानि मैथिली}
{उवास राम रहिता रावणस्य निवेशने}% ७६

\twolineshloka
{द्विचत्वारिंशक वर्षे रामो राज्यमकारयत्}
{सीतायाश्च त्रयस्त्रिंशद्वत्सराश्च तदाभवन्}% ७७

\twolineshloka
{स चतुर्दशवर्षान्ते प्रविश्य च पुरीं प्रभुः}
{अयोध्यां मुदितो रामो हत्वा रावणमाहवे}% ७८

\twolineshloka
{भ्रातृभिः सहितस्तत्र रामो राज्यमथाकरोत्}
{राज्यं प्रकुर्वतस्तस्य पुरोधा वदतां वरः}% ७९

\twolineshloka
{अगस्त्यः कुम्भसम्भूतिस्तमागन्ता रघोः पतिम्}
{तद्वाक्याद्रघुनाथोऽसौ करिष्यति हयक्रतुम्}% ८०

\twolineshloka
{तस्यागमिष्यति हयो ह्याश्रमे तव सुव्रत}
{तस्य योधाः प्रमुदिता आयास्यन्ति तवाश्रमम्}% ८१

\twolineshloka
{तेषामग्रे रामकथाः करिष्यसि मनोहराः}
{तैः साकं त्वमयोध्यायां गन्तासि वै द्विजर्षभ}% ८२

\twolineshloka
{दृष्ट्वा राममयोध्यायां पद्मपत्रनिभेक्षणम्}
{तत्क्षणादेव संसारवार्धिनिस्तारवान्भव}% ८३

\twolineshloka
{इत्युक्त्वा मां मुनिवरो लोमशः सर्वबुद्धिमान्}
{उवाच ते किं प्रष्टव्यं तदाहमवदं हि तम्}% ८४

\twolineshloka
{ज्ञातं त्वत्कृपया सर्वं रामचारित्रमद्भुतम्}
{त्वत्प्रसादादवाप्स्येऽहं रामस्य चरणाम्बुजम्}% ८५

\twolineshloka
{मया नमस्कृतः पश्चाज्जगाम स मुनीश्वरः}
{तत्प्रसादान्मयावाप्तं रामस्य चरणार्चनम्}% ८६

\twolineshloka
{सोऽहं स्मरामि रामस्य चरणावन्वहं मुहुः}
{गायामि तस्य चरितं मुहुर्मुहुरतन्द्रितः}% ८७

\twolineshloka
{पावयामि जनानन्यान्गानेन स्वान्तहारिणा}
{हृष्यामि तन्मुनेर्वाक्यं स्मारंस्मारं तदीक्षया}% ८८

\twolineshloka
{धन्योऽहं कृतकृत्योऽहं सभाग्योऽहं महीतले}
{रामचन्द्र पदाम्भोज दिदृक्षा मे भविष्यति}% ८९

\twolineshloka
{तस्मात्सर्वात्मना रामो भजनीयो मनोहरः}
{वन्दनीयो हि सर्वेषां संसाराब्धितितीर्षया}% ९०

\twolineshloka
{तस्माद्यूयं किमर्थं वै प्राप्ताः को वानराधिपः}
{यागं करोति धर्मात्मा हयमेधं महाक्रतुम्}% ९१

\twolineshloka
{तत्सर्वं कथयन्त्वत्र यां तु वाहस्य पालने}
{स्मरन्तु रघुनाथाङ्घ्रिं स्मृत्वा स्मृत्वा पुनः पुनः}% ९२

\twolineshloka
{इति वाक्यं समाकर्ण्य मुनेर्विस्मयमागताः}
{रघुनाथं स्मरन्तस्ते प्रोचुरारण्यकं मुनिम्}% ९३

{॥इति श्रीपद्मपुराणे पातालखण्डे शेषवात्स्यायनसंवादे रामाश्वमेधे लोमशारण्यकसंवादे रामचरित्रकथनं नाम षट्त्रिंशत्तमोऽध्यायः॥३६॥}

\dnsub{सप्तत्रिंशत्तमोऽध्यायः}%\resetShloka

\uvacha{शेष उवाच}

\twolineshloka
{ते पृष्टा मुनिवर्येण रामचारित्रमद्भुतम्}
{धन्यं सभाग्यं मन्वानाः प्रोचुरात्मानमादरात्}% १

\uvacha{जना ऊचुः}

\twolineshloka
{पवित्रिता वयं सर्वे दर्शनेन तवाधुना}
{यद्रामकथयास्मान्वै पावयस्यधुना जनान्}% २

\twolineshloka
{शृणुष्व वचनं तथ्यं भवान्ब्रह्मर्षिसत्तमः}
{त्वया पृष्टं यदस्मभ्यं सर्वं तत्कथयाम वै}% ३

\twolineshloka
{अगस्त्यवाक्याच्छ्रीरामो विप्रहत्यापनुत्तये}
{यागं करोति सुमहान्सर्वसम्भारसम्भृतम्}% ४

\twolineshloka
{तं पालयानाः सर्वे वै त्वदाश्रममुपागताः}
{अश्वेन सहिता विप्र तज्जानीहि महामते}% ५

\twolineshloka
{इति वाक्यं समाकर्ण्य मनोहारि रसायनम्}
{अत्यन्तं हर्षमापेदे ब्राह्मणो रामभक्तिमान्}% ६

\twolineshloka
{अद्य मे फलितो वृक्षो मनोरथश्रियान्वितः}
{अद्य मे जननी धन्या जातं मां सुषुवे तु या}% ७

\twolineshloka
{अद्य राज्यं मया प्राप्तं कण्टकेन विवर्जितम्}
{अद्य कोशाः सुसम्पन्ना अद्य देवाः सुतोषिताः}% ८

\twolineshloka
{अग्निहोत्रफलं त्वद्य प्राप्तं मे हविषा हुतम्}
{यद्द्रक्ष्ये रामचन्द्रस्य चरणाम्भोरुहोर्युगम्}% ९

\twolineshloka
{यो नित्यं ध्यायते स्वान्ते अयोध्यायाः पतिः प्रभुः}
{स मे दृग्गोचरो नूनं भविष्यति मनोहरः}% १०

\twolineshloka
{हनूमान्मां समालिङ्ग्य प्रक्ष्यते कुशलं मम}
{भक्तिं मे महतीं दृष्ट्वा तोषं प्राप्स्यति सत्तमः}% ११

\twolineshloka
{इति वाक्यं समाकर्ण्य हनूमान्कपिसत्तमः}
{जग्राह पादयुगलं मुनेरारण्यकस्य हि}% १२

\twolineshloka
{स्वामिन्हनूमान्विप्रर्षे सेवकोऽहं पुरःस्थितः}
{जानीहि रामदासस्य रेणुकल्पं मुनीश्वर}% १३

\twolineshloka
{इत्युक्तवति तस्मिन्वै मुनिः परमहर्षितः}
{आलिलिङ्ग हनूमन्तं रामभक्त्या सुशोभितम्}% १४

\twolineshloka
{उभौ प्रेमविनिर्भिन्नावुभावपि सुधाप्लुतौ}
{स्थगितौ चित्रलिखिताविव तत्र बभूवतुः}% १५

\twolineshloka
{उपविष्टौ कथास्तत्र चक्रतुः सुमनोहराः}
{रघुनाथपदाम्भोजप्रीतिनिर्भरमानसौ}% १६

\twolineshloka
{हनूमांस्तमुवाचेदं वचो विविधशोभनम्}
{आरण्यकं मुनिवरं रामाङ्घ्रिध्याननिर्भृतम्}% १७

\twolineshloka
{स्वामिन्नयं दशरथकुलहीराङ्कुरो महान्}
{रामभ्राता महाशूरः शत्रुघ्नः प्रणमत्यसौ}% १८

\twolineshloka
{लवणो येन निहतः सर्वलोकभयङ्करः}
{कृताश्च सुखिनः सर्वे मुनयः सुतपोधनाः}% १९

\twolineshloka
{एष पुष्कलनामा त्वां नमत्युद्भटसेवितः}
{येनाधुना महावीरा जिताः समरमण्डले}% २०

\twolineshloka
{जानीह्येनं बहुगुणं रामामात्यं महाबलम्}
{प्राणप्रियं रघुपतेः सर्वज्ञं धर्मकोविदम्}% २१

\twolineshloka
{सुबाहुरयमत्युग्रो वैरिवंशदवानलः}
{रामपादाब्जरोलम्बो नमति त्वां महायशाः}% २२

\twolineshloka
{सुमदोऽप्येष पार्वत्या दत्तरामाङ्घ्रिसेवया}
{प्राप्तोऽधुनासौ संसारवार्धिनिस्तरणं महत्}% २३

\twolineshloka
{सत्यवानयमश्वं यः प्राप्तमाश्रुत्य सेवकात्}
{राज्यं निवेदयामास स त्वां प्रणमति क्षितौ}% २४

\twolineshloka
{इति वाक्यं समाकर्ण्य समालिङ्ग्य समादरात्}
{चकारारण्यक ऋषिः स्वागतं फलकादिना}% २५

\twolineshloka
{ते हृष्टास्तत्र वसतिं चक्रुर्मुनिवराश्रमे}
{प्रातर्नित्यक्रियां कृत्वा रेवायां ते महोद्यमाः}% २६

\twolineshloka
{नरयानमथारोप्य सेवकैः सहितं मुनिम्}
{शत्रुघ्नः प्रापयामासायोध्यां रामकृतालयाम्}% २७

\twolineshloka
{स दूरान्नगरीं दृष्ट्वा सूर्यवंशनृपोषिताम्}
{पदातिरभवद्वेगाद्रघुनाथदिदृक्षया}% २८

\twolineshloka
{सम्प्राप्य नगरीं रम्यामयोध्यां जनशोभिताम्}
{मनोरथसहस्रेण संरूढो रामदर्शने}% २९

\twolineshloka
{ददर्श तत्र सरयूतीरे मण्डपशोभिते}
{रामं दूर्वादलश्यामं कञ्जकान्तिविलोचनम्}% ३०

\twolineshloka
{मृगशृङ्गं कटौ रम्यं धारयन्तं श्रियान्वितम्}
{ऋषिवृन्दैर्व्यासमुख्यैर्वृतं शूरैः सुसेवितम्}% ३१

\twolineshloka
{भरतेन सुमित्रायास्तनूजेन परीवृतम्}
{ददतं दीनसन्धेभ्यो दानानि प्रार्थितानि तम्}% ३२

\twolineshloka
{विलोक्यारण्यकाख्योऽसौ कृतार्थ इत्यमन्यत}
{मल्लोचने पद्मदलसमाने रामलोकके}% ३३

\twolineshloka
{अद्य मे सर्वशास्त्रस्य ज्ञातृत्वं बहुसार्थकम्}
{येन श्रीराममाज्ञाय प्राप्तोऽयोध्यापुरीमिमाम्}% ३४

\fourlineindentedshloka
{इत्येवमादिवचनानि बहूनि हृष्टो}
{रामाङ्घ्रिदर्शनसुहर्षित गात्रशोभी}
{प्रायाद्रमेश्वरसमीपमगम्यमन्यै-}
{र्योगेश्वरैरपि विचारपरैः सुदूरम्}% ३५

\twolineshloka
{धन्योऽहमद्य रामस्य चरणावक्षिगोचरौ}
{करिष्यामि वचो रम्यं वदन्राममवेक्षयन्}% ३६

\twolineshloka
{रामोऽपि वाडवश्रेष्ठं ज्वलन्तं स्वेन तेजसा}
{तपोमूर्तिधरं वीक्ष्य प्रत्युत्थानमथाकरोत्}% ३७

\twolineshloka
{रामचन्द्रस्तस्य पादौ सुचिरं नतवान्महान्}
{ब्रह्मण्यदेवपावित्र्यं कृतमद्यतनोर्मम}% ३८

\twolineshloka
{इति वाक्यं वदंस्तस्य पादयोः पतितः प्रभुः}
{सुरासुरनमन्मौलिमणिनीराजिताङ्घ्रिकः}% ३९

\twolineshloka
{प्रणतं तं नृपश्रेष्ठं वाडवेन्द्रो महातपाः}
{गृहीत्वा भुजयोर्मध्यमालिलिङ्ग प्रियं प्रभुम्}% ४०

\twolineshloka
{कौसल्यातनयस्तं वा उच्चैर्मणिमयासने}
{संस्थाप्य च पदोर्युग्मं जलेनाक्षालयत्प्रभुः}% ४१

\twolineshloka
{पादावनेजनोदं तु मस्तकेऽधाद्धरिः स्वयम्}
{पवित्रितोऽद्य सगणः सकुटुम्ब इति ब्रुवन्}% ४२

\twolineshloka
{चन्दनेन विलिप्याथ गां च प्रादात्पयस्विनीम्}
{उवाच च वचो रम्यं देवदेवेन्द्र सेवितः}% ४३

\twolineshloka
{स्वामिन्मखो मया वाजिमेधसंज्ञः क्रियेत ह}
{सोयं त्वच्चरणा यातादद्यपूर्णो भविष्यति}% ४४

\twolineshloka
{अद्य मे ब्रह्महत्योत्थ पापहानिं करिष्यति}
{अश्वमेधः क्रतुर्युष्मच्चरणेन पवित्रितः}% ४५

\twolineshloka
{इति वाक्यं ब्रुवाणं तं राजराजेन्द्रसेवितम्}
{आरण्यक उवाचेदं हसन्माध्व्या गिरा मुनिः}% ४६

\twolineshloka
{स्वामिंस्तव तु युक्तं हि वचो ब्रह्मण्यभूमिप}
{त्वन्मूर्तयो महाराज ब्राह्मणा वेदपारगाः}% ४७

\twolineshloka
{त्वं यदा ब्रह्मपूजादि शुभं कर्म करिष्यसि}
{ततोऽखिला नृपा विप्रं पूजयिष्यन्ति भूमिप}% ४८

\twolineshloka
{त्वयोक्तं यन्महाराज विप्रहत्यापनुत्तये}
{यागं करोमि विमलं तत्तु हास्यकरं वचः}% ४९

\twolineshloka
{त्वन्नामस्मरणान्मूढः सर्वशास्त्रविवर्जितः}
{सर्वपापाब्धिमुत्तीर्य स गच्छेत्परमं पदम्}% ५०

\twolineshloka
{सर्ववेदेतिहासानां सारार्थोऽयमिति स्फुटम्}
{यद्रामनामस्मरणं क्रियते पापतारकम्}% ५१

\twolineshloka
{तावद्गर्जन्ति पापानि ब्रह्महत्यासमानि च}
{न यावत्प्रोच्यते नाम रामचन्द्र तव स्फुटम्}% ५२

\twolineshloka
{त्वन्नामगर्जनं श्रुत्वा महापातककुञ्जराः}
{पलायन्ते महाराज कुत्रचित्स्थानलिप्सया}% ५३

\twolineshloka
{तस्मात्तव कथं हत्या महापुण्यददर्शन}
{राम त्वत्सुकथां श्रुत्वा पूतः सद्यो भविष्यति}% ५४

\twolineshloka
{मया पूर्वं कृतयुगे गङ्गायास्तीरवासिनाम्}
{ऋषीणां मुखतो वाक्यं श्रुतमेतत्पुराविदाम्}% ५५

\twolineshloka
{तावत्पापभियः पुंसां कातराणां सुपापिनाम्}
{यावन्न वदते वाचा रामनाममनोहरम्}% ५६

\twolineshloka
{तस्माद्धन्योऽहमधुना मम संसृतिनाशनम्}
{साम्प्रतं सुलभं रामचन्द्र त्वद्दर्शनादभूत्}% ५७

\twolineshloka
{इत्युक्तवन्तं स मुनिं पूजयामास तत्र वै}
{सर्वे मुनिजनाः साधु साधु वाक्यमिति ब्रुवन्}% ५८

\uvacha{शेष उवाच}

\twolineshloka
{अत्याश्चर्यमभूत्तत्र तन्मे निगदतः शृणु}
{वात्स्यायनमुनिश्रेष्ठ रामभक्तिपरायण}% ५९

\twolineshloka
{रामं दृष्ट्वा महाराजं यादृशं ध्यानगोचरम्}
{अत्यन्तं हर्षमापन्नो जगाद स मुनीश्वरान्}% ६०

\twolineshloka
{मुनीश्वराः संशृणुत मद्वाक्यं सुमनोहरम्}
{मादृशः को न भूलोके भविष्यति सुभाग्यवान्}% ६१

\twolineshloka
{नास्ति मत्सदृशः कोपि न जातो न भविष्यति}
{यद्रामभद्रो मां नत्वा स्वागतं परिपृष्टवान्}% ६२

\twolineshloka
{यत्पादपङ्कजरजः श्रुतिमृग्यं सदैव हि}
{सोऽद्य मत्पादयोः पाथः पीत्वा पूतममन्यत}% ६३

\twolineshloka
{एवं प्रवदतस्तस्य ब्रह्मस्फोटोऽभवत्तदा}
{निर्गतं तद्भवं तेजो विवेश रघुनायके}% ६४

\twolineshloka
{पश्यतां सर्वलोकानां सरयूतीरमण्डपे}
{सायुज्यमुक्तिं सम्प्राप दुर्ल्लभां योगिभिर्जनैः}% ६५

\twolineshloka
{दिवि तूर्यनिनादोऽभूद्वीणानादोऽभवत्तदा}
{पुष्पवृष्टिः पपाताग्रे पश्यतां चित्रमद्भुतम्}% ६६

\twolineshloka
{मुनयोऽप्येतदीक्षित्वा प्रशंसन्तो मुनीश्वरम्}
{कृतार्थोयं मुनिश्रेष्ठो यद्रामवपुषीक्षितः}% ६७

{॥इति श्रीपद्मपुराणे पातालखण्डे शेषवात्स्यायनसंवादे रामाश्वमेधे आरण्यकमुनेर्विष्णुलोकगमनं नाम सप्तत्रिंशत्तमोऽध्यायः॥३७॥}

\dnsub{अष्टत्रिंशत्तमोऽध्यायः}%\resetShloka

\uvacha{सूत उवाच}

\twolineshloka
{एतदाख्यानकं श्रुत्वा वात्स्यायन उदारधीः}
{परमं हर्षमापेदे जगाद च फणीश्वरम्}% १

\uvacha{वात्स्यायन उवाच}

\twolineshloka
{कथां संशृण्वते मह्यं तृप्तिर्नास्ति फणीश्वर}
{रघुनाथस्य भक्तार्तिहारिकीर्तिकरस्य वै}% २

\twolineshloka
{धन्य आरण्यको नाम मुनिर्वेदधरः परः}
{रघुनाथं समालोक्य देहं तत्याज नश्वरम्}% ३

\twolineshloka
{ततो राज्ञो हयः कुत्र गतः केन नियन्त्रितः}
{कथं तत्र रमानाथ कीर्तिर्जाता फणीश्वर}% ४

\twolineshloka
{सर्वं कथय मे तथ्यं सर्वज्ञोऽस्ति यतो भवान्}
{धराधरवपुर्धारी साक्षात्तस्य स्वरूपधृक्}% ५

\uvacha{व्यास उवाच}

\twolineshloka
{इति वाक्यं समाकर्ण्य प्रहृष्टेनान्तरात्मना}
{उवाच रामचारित्रं तत्तद्गुणकथोदयम्}% ६

\uvacha{शेष उवाच}

\twolineshloka
{साधु पृच्छसि विप्रर्षे रघुनाथगुणान्मुहुः}
{श्रुता न श्रुतवत्कृत्वा तेषु लोलुपतां दधत्}% ७

\twolineshloka
{ततो निरगमद्वाहः सैनिकैर्बहुभिवृतः}
{रेवातीरे मनोज्ञे तु मुनिवृन्दनिषेविते}% ८

\twolineshloka
{सेनाचरास्ततः सर्वे यत्र वाहस्ततस्ततः}
{प्रसर्पन्ति निरीक्षन्तस्तन्मार्गं रणकोविदाः}% ९

\twolineshloka
{वाजी गतोऽथ रेवाया ह्रदेऽगाधजलान्विते}
{भाले स्वर्णभवं पत्रं धारयन्पूजिताङ्गकः}% १०

\twolineshloka
{ततो जले ममज्जासौ रामचन्द्र हयो वरः}
{तदा सर्वे महाशूरास्तत्र विस्मयमागताः}% ११

\twolineshloka
{तैः परस्परमेवोचे कथं हयसमागमः}
{कोऽत्र गन्ता जले वाहमानेतुं तं महोदयम्}% १२

\twolineshloka
{इति यावत्समुद्विग्ना मन्त्रयन्ते परस्परम्}
{तावद्वीरशतैः सार्धमाजगाम रघोः पतिः}% १३

\twolineshloka
{तान्सर्वान्विमनस्कान्स दृष्ट्वा शत्रुघ्नसंज्ञितः}
{पप्रच्छ मेघगम्भीरवाचा वीरशिरोमणिः}% १४

\twolineshloka
{किं स्थितं निखिलैरद्य युष्माभिः सङ्घशो जले}
{कुत्राश्वो रघुनाथस्य स्वर्णपत्रेण शोभितः}% १५

\twolineshloka
{जले किं विनिमग्नोऽसौ हृतो वा केन मानिना}
{तन्मे कथयत क्षिप्रं कथं यूयं विमोहिताः}% १६

\uvacha{शेष उवाच}

\twolineshloka
{इति वाक्यं समाकर्ण्य राज्ञो रघुवरस्य हि}
{कथयामासुस्ते सर्वं वीराः शूरशिरोमणिम्}% १७

\uvacha{जना ऊचुः}

\twolineshloka
{स्वामिन्वयं न जानीमो मुहूर्तमभवज्जले}
{निममज्ज ततो नायाद्धयस्तव मनोहरः}% १८

\twolineshloka
{त्वमेव तत्र गत्वेमं वाहमानय वेगतः}
{अस्माभिस्तत्र गन्तव्यं त्वया सार्द्धं महामते}% १९

\twolineshloka
{इति श्रुत्वा वचस्तेषां सैनिकानां रघूद्वहः}
{खेदं प्राप जनान्पश्यञ्जलसन्तरणोद्यतान्}% २०

\twolineshloka
{उवाच मन्त्रिमुख्यं स किं कर्तव्यमतः परम्}
{कथं वाहस्य सम्प्राप्तिर्भविष्यति वदस्व तत्}% २१

\twolineshloka
{के तत्र शूराः संयोज्या जलेऽन्वेषयितुं हयम्}
{को वा नयिष्यते वाहं केनोपायेन तद्वद}% २२

\twolineshloka
{इति राज्ञोवचः श्रुत्वा सुमतिर्मन्त्रिसत्तमः}
{उवाच समये योग्यं शत्रुघ्नं हर्षयन्निव}% २३

\twolineshloka
{स्वामिन्नस्ति तव श्रीमञ्छक्तिरद्भुतकर्मणः}
{पातालगमने शक्तिर्जलमध्यादिह स्फुटम्}% २४

\twolineshloka
{अन्यच्च पुष्कलस्यापि शक्तिरस्ति महात्मनः}
{हनूमतोऽपि रामस्य पादसेवापरस्य च}% २५

\twolineshloka
{तस्माद्यूयं त्रयो गत्वा हयमानयत ध्रुवम्}
{यतो भवेद्वाहमेधो रघुनाथस्य धीमतः}% २६

\uvacha{शेष उवाच}

\twolineshloka
{इति वाक्यं समाश्रुत्य शत्रुघ्नः परवीरहा}
{स्वयं विवेश तोयान्तर्हनुमत्पुष्कलान्वितः}% २७

\twolineshloka
{यावज्जलं विवेशासौ तावत्पुरमदृश्यत}
{अनेकोद्यानशोभाढ्यममेयं पुटभेदनम्}% २८

\twolineshloka
{तत्र माणिक्यरचिते स्तम्भे स्वर्णमये हयम्}
{बद्धं ददर्श रामस्य स्वर्णपत्रसुशोभितम्}% २९

\twolineshloka
{स्त्रियस्तत्र मनोहारि रूपधारिण्य उत्तमाः}
{सेवन्ते सुन्दरीमेकां पर्यङ्के सुखमास्थिताम्}% ३०

\twolineshloka
{तान्दृष्ट्वा ताः स्त्रियः सर्वाः प्रावोचन्स्वामिनीं प्रति}
{एतेऽल्पवर्ष्मवयसो मांसपुष्टकलेवराः}% ३१

\twolineshloka
{भविष्यन्ति तव श्रेष्ठमाहारस्य फलं महत्}
{एतेषां शोणितं स्वादु पुरुषाणां गतायुषाम्}% ३२

\twolineshloka
{एतद्वचः समाकर्ण्य सेवकीनां वराङ्गना}
{जहास किञ्चिद्वदनं नर्तयन्ती भ्रुवानघा}% ३३

\twolineshloka
{तावत्त्रयस्ते सम्प्राप्ताः सन्नाहश्री विशोभिताः}
{शिरस्त्राणानि दधतः शौर्यवीर्यसमन्विताः}% ३४

\twolineshloka
{ता दृष्ट्वा महिलास्तत्र सौन्दर्यश्रीसमन्विताः}
{प्रोचुस्ते विस्मयं विप्र किमिदं दृश्यते महत्}% ३५

\twolineshloka
{नमश्चक्रुर्महात्मानः सर्वे देववराङ्गनाः}
{किरीटमणिविद्योतद्योतिताङ्घ्रियुगास्ततः}% ३६

\twolineshloka
{सा तान्पप्रच्छ पुरुषान्सर्वश्रेष्ठा तु भामिनी}
{के यूयमत्र सम्प्राप्ताः कथं चापधरा नराः}% ३७

\twolineshloka
{मत्स्थलं सर्वदेवानामगम्यं मोहनं महत्}
{अत्र प्राप्तस्य तु क्वापि निवृत्तिर्न भवत्युत}% ३८

\twolineshloka
{अश्वोऽयं कस्य राज्ञो वै कथं चामरवीजितः}
{स्वर्णपत्रेण शोभाढ्यः कथयन्तु ममाग्रतः}% ३९

\uvacha{शेष उवाच}

\twolineshloka
{इति तस्या वचः श्रुत्वा मोहनाचारसंयुतम्}
{हनूमांस्तां प्रत्युवाच गतभीः प्रहसन्निव}% ४०

\twolineshloka
{वयं वै किङ्करा राज्ञस्त्रैलोक्यस्य शिखामणेः}
{त्रिलोकीयं प्रणमते सर्वदेवशिरोमणिम्}% ४१

\twolineshloka
{रामभद्रस्य जानीहि हयमेधप्रवर्तितुः}
{प्रमुञ्च वाहमस्माकं कथं बद्धो वराङ्गने}% ४२

\twolineshloka
{वयं सर्वास्त्रकुशलाः सर्वशस्त्रास्त्रकोविदाः}
{नयिष्यामो बलाद्वाहं हत्वा तत्प्रतिरोधकान्}% ४३

\twolineshloka
{इति वाक्यं समाकर्ण्य प्लवङ्गस्य वराङ्गना}
{विवरस्था प्रत्युवाच हसन्ती वाक्यकोविदा}% ४४

\twolineshloka
{मयानीतमिमं वाहं न कोमोचयितुं क्षमः}
{वर्षायुतेन निशितैर्बाणकोटिभिरुच्छिखैः}% ४५

\twolineshloka
{परं रामस्य पादाब्जसैवकी कर्मकारिणी}
{न ग्रहीष्यामि तद्वाहं राजराजस्य धीमतः}% ४६

\twolineshloka
{महान विनयो जातो मम नेत्र्याः सुवाजिनः}
{क्षमताद्रामचन्द्रस्तच्छरण्यो भक्तवत्सलः}% ४७

\twolineshloka
{यूयं क्लिष्टास्तत्पुरुषा हयार्थं तस्य रक्षितुः}
{याचध्वं वरमप्राप्यं देवानामपि सत्तमाः}% ४८

\twolineshloka
{यथा मेमीवमत्युग्रं क्षमेत पुरुषोत्तमः}
{व्रीडां त्यक्त्वाखिलां यूयं वृणुध्वं वरमुत्तमम्}% ४९

\twolineshloka
{तस्या वचः परं श्रुत्वा हनूमान्नि जगाद ताम्}
{रघुनाथप्रसादेन सर्वमस्माकमूर्जितम्}% ५०

\fourlineindentedshloka
{तथापि याचे वरमेकमुत्तमं}
{विधेहि तन्मे मनसः समीहितम्}
{भवे भवे नो रघुनायकः पति-}
{र्वयं च तत्कर्मकराश्च किङ्कराः}% ५१

\twolineshloka
{एतद्वचनमाकर्ण्य प्लवगस्य तदाङ्गना}
{उवाच वाक्यं मधुरं प्रहस्य गुणपूजितम्}% ५२

\twolineshloka
{भवद्भिः प्रार्थितं यद्वै दुर्ल्लभं सर्वदैवतैः}
{तद्भविष्यत्यसन्देहः सेवकास्तद्रघोः पतेः}% ५३

\twolineshloka
{अथापि वरमेकं वै दास्यामि कृतहेलना}
{रघुनाथस्य तुष्ट्यर्थं तदृतं मे भवेद्वचः}% ५४

\twolineshloka
{अग्रे वीरमणिर्भूपो महावीरसमन्वितः}
{ग्रहीष्यति भवद्वाहं शिवेन परिरक्षितः}% ५५

\twolineshloka
{तज्जयार्थे महास्त्रं मे गृह्णीत सुमहाबलाः}
{द्वैरथे स तु योद्धव्यः शत्रुघ्नेन त्वया महान्}% ५६

\twolineshloka
{इदमस्त्रं यदा त्वं तु क्षेपयिष्यसि सङ्गरे}
{अनेन पूतो रामस्य स्वरूपं ज्ञास्यते पुनः}% ५७

\twolineshloka
{ज्ञात्वा तं वाजिनं दत्वा चरणे प्रपतिष्यति}
{तस्माद्गृह्णीध्वमस्त्रं तन्मम वैरिविदारणम्}% ५८

\twolineshloka
{तच्छ्रुत्वा रघुनाथस्य भ्राता जग्राह चास्त्रकम्}
{उदङ्मुखः पवित्राङ्गो योगिन्या दत्तमद्भुतम्}% ५९

\twolineshloka
{तत्प्राप्यास्त्रं महातेजा बभूव रिपुकर्शनः}
{दुष्प्रधर्ष्यो दुराराध्यो वैरिवारणसत्सृणिः}% ६०

\twolineshloka
{तां नत्वा राघवश्रेष्ठः शत्रुघ्नो हयसत्तमम्}
{गृहीत्वागाज्जलात्तस्माद्रेवातीरे सुखोचिते}% ६१

\twolineshloka
{तं दृष्ट्वा सैनिकाः सर्वे प्रहृष्टाङ्गा मुदान्विताः}
{साधुसाधु प्रशंसन्तः पप्रच्छुर्हयनिर्गमम्}% ६२

\twolineshloka
{हनूमान्कथयामास हयस्यागमनं महत्}
{वरप्राप्तिं च ताभ्यो वै तेऽपि श्रुत्वा मुदं गताः}% ६३

{॥इति श्रीपद्मपुराणे पातालखण्डे शेषवात्स्यायनसंवादे रामाश्वमेधे शत्रुघ्नस्य योगिनीदर्शनजलमध्याद्वाहप्राप्तिर्नाम अष्टत्रिंशत्तमोऽध्यायः॥३८॥}

\dnsub{एकोनचत्वारिंशत्तमोऽध्यायः}%\resetShloka

\uvacha{शेष उवाच}

\twolineshloka
{निनदत्सुमृदङ्गेषु वीणानादेषु सर्वतः}
{मुक्तो वाहस्ततो देव पुरं देवविनिर्मितम्}% १

\twolineshloka
{यत्र स्फाटिक कुड्यानां रचनाभिर्गृहा नृणाम्}
{हसन्ति विन्ध्यं विमलं पर्वतं नागसेवितम्}% २

\twolineshloka
{राजतानि गृहाण्यत्र दृश्यन्ते प्रकृतेरपि}
{विचित्रमणिसन्नद्धा नानामाणिक्यगोपुराः}% ३

\twolineshloka
{पद्मिन्यो यत्र लोकानां गेहे गेहे मनोहराः}
{हरन्ति चित्तानि नृणां मुखपद्मकलेक्षिताः}% ४

\twolineshloka
{पद्मरागमणिर्यत्र गेहे गेहे सुभूमिषु}
{बद्धः संलक्ष्यते विप्र तदोष्ठस्पर्धया नु किम्}% ५

\twolineshloka
{क्रीडाशैलाः प्रत्यगारं नीलरत्नविनिर्मिताः}
{कुर्वन्ति शङ्कां मेघस्य मयूराणां कलापिनाम्}% ६

\twolineshloka
{हंसा यत्र नृणां गेहे स्फाटिकेषु नियन्त्रिताः}
{कुर्वन्ति मेघान्नो भीतिं मानसं न स्मरन्ति च}% ७

\twolineshloka
{निरन्तरं शिवस्थाने ध्वस्तं चन्द्रिकया तमः}
{शुक्लकृष्णविभेदो न पक्षयोस्तत्र वै नृणाम्}% ८

\twolineshloka
{तत्र वीरमणी राजा धार्मिकेष्वग्रणीर्महान्}
{राज्यं करोति विपुलं सर्वभोगसमन्वितम्}% ९

\twolineshloka
{तस्य पुत्रो महाशूरो नाम्ना रुक्माङ्गदो बली}
{वनिताभिर्गतो रम्यदेहाभिः क्रीडितुं वनम्}% १०

\twolineshloka
{तासां मञ्जीरसंरावः कङ्कणानां रवस्तथा}
{मनो हरति कामस्य किमन्यस्य कथात्र भोः}% ११

\twolineshloka
{वनं जगाम सुमहत्सुपुष्पनगसंयुतम्}
{सदाशिवकृतस्थानमृतुषट्कैर्विराजितम्}% १२

\twolineshloka
{चम्पका यत्र बहुशः फुल्लकोरकशोभिताः}
{कुर्वन्ति कामिनां तत्र हृच्छयार्तिं विलोकिताः}% १३

\twolineshloka
{चूताः फलादिभिर्नम्रा मञ्जरीकोटिसंयुताः}
{नागाः पुन्नागवृक्षाश्च शालास्तालास्तमालकाः}% १४

\twolineshloka
{कोकिलानां समारावा यत्र च श्रुतिगोचराः}
{सदा मधुपझङ्कार गतनिद्राः सुमल्लिकाः}% १५

\twolineshloka
{दाडिमानां समूहाश्च कर्णिकारैः समन्विताः}
{केतकीकानकीवन्यवृक्षराजिविराजिताः}% १६

\twolineshloka
{तस्मिन्वने प्रमदसंयुतचित्तवृत्तिर्गायन्कलं मधुरवाग्विचिकीर्षयोच्चैः}
{उद्यत्कुचाभिरभितो वनिताभिरागाच्छोभानिधान वपुरुद्गतभीर्विवेश}% १७

\twolineshloka
{काश्चित्तं नृत्यविद्याभिस्तोषयन्ति स्म शोभनम्}
{काश्चिद्गानकलाभिश्च काश्चिद्वाक्चतुरोचितैः}% १८

\twolineshloka
{भ्रूसंज्ञया पराः काश्चित्तोषयामासुरुन्मदाः}
{परिरम्भणचातुर्यैस्तं हृष्टं विदधुः स्त्रियः}% १९

\twolineshloka
{ताभिः पुष्पोच्चयं कृत्वा भूषयामास ताः स्त्रियः}
{वाण्या कोमलया शंसन्रेमे कामवपुर्धरः}% २०

\twolineshloka
{एवं प्रवृत्ते समये राजराजस्य धीमतः}
{प्रायात्तद्वनदेशं स हयः परमशोभनः}% २१

\fourlineindentedshloka
{तं स्वर्णपत्ररचितैकललाटदेशं}
{गङ्गासमं घुसृणकुङ्कुम पिञ्जराङ्गम्}
{गत्यासमं पवनवेगतिरस्करिण्या}
{दृष्ट्वा स्त्रियः परमकौतुकधामदेहम्}% २२

\fourlineindentedshloka
{ऊचुः पतिं कमलमध्यपिशङ्गवर्णा-}
{स्ताम्राधरप्रतिभयाहतविद्रुमाभाः}
{दन्तव्रजप्रमितहास्यसुशोभिवक्त्राः}
{कामस्य बाणनयनादिविमोहनाभाः}% २३

\uvacha{स्त्रिय ऊचुः}

\twolineshloka
{कान्तकोयं महानर्वास्वर्णपत्रैकशोभितः}
{कस्य वा भाति शोभाढ्यो गृहाण स्वबलादिमम्}% २४

\uvacha{शेष उवाच}

\twolineshloka
{तदुक्तं वच आकर्ण्य लीलाललितलोचनः}
{जग्राह हयमेकेन करपद्मेन लीलया}% २५

\twolineshloka
{वाचयित्वा स्वर्णपत्रं स्पष्टवर्णसमन्वितम्}
{जहास महिलामध्ये जगाद वचनं पुनः}% २६

\uvacha{रुक्माङ्गद उवाच}

\twolineshloka
{पृथिव्यां नास्ति मे पित्रा समः शौर्येण च श्रिया}
{तस्मिन्कथं विधत्ते स उत्सेकं रामभूमिपः}% २७

\twolineshloka
{यस्य रक्षां प्रकुरुते सदा रुद्रः पिनाकधृक्}
{यं देवा दानवा यक्षा नमन्ति मणिमौलिभिः}% २८

\twolineshloka
{कुरुताद्वाजिमेधं वै जनको मे महाबलः}
{या त्वेष वाजिशालायां बध्नन्तु मम वै भटाः}% २९

\twolineshloka
{इति वाक्यं समाकर्ण्य महिलास्ता मनोहराः}
{प्रहर्षवदना जाताः कान्तं तु परिरेभिरे}% ३०

\twolineshloka
{गृहीत्वा च हयं पुत्रो राज्ञो वीरमणेर्महान्}
{पुरं पत्नीसमायुक्तो महोत्साहमवीविशत्}% ३१

\twolineshloka
{मृदङ्गध्वनिषु प्रोच्चैराहतेषु समन्ततः}
{बन्दिभिः संस्तुतः प्रागात्स्वपितुर्मन्दिरं महत्}% ३२

\twolineshloka
{तस्मै स कथयामास हयं नीतं रघोः पतेः}
{वाजिमेधाय निर्मुक्तं स्वच्छन्दगतिमद्भुतम्}% ३३

\twolineshloka
{रक्षितं शत्रुसूदेन महाबलसमेतिना}
{तच्छ्रुत्वा वचनं तस्य नृपो वीरमणिर्महान्}% ३४

\twolineshloka
{नातिप्रशंसयामास तत्कर्म सुमहामतिः}
{नीत्वा पुनः समायान्तं चौरस्येव विचेष्टितम्}% ३५

\twolineshloka
{कथयामास जामात्रे शिवायाद्भुतकर्मणे}
{अर्धाङ्गनाधरायाङ्गभूषाय चन्द्रधारिणे}% ३६

\twolineshloka
{तेन सम्मन्त्रयामास नृपो वीरमणिर्महान्}
{पुत्रसृष्टं महत्कर्म विनिन्द्यं महतां मतः}% ३७

\uvacha{शिव उवाच}

\twolineshloka
{राजन्पुत्रेण भवतः कृतं कर्म महाद्भुतम्}
{यो जहार महावाहंरामचन्द्रस्य धीमतः}% ३८

\twolineshloka
{अद्य युद्धं महद्भाति सुरासुरविमोहनम्}
{शत्रुघ्नेन महाराज्ञा वीरकोट्येकरक्षिणा}% ३९

\twolineshloka
{मया यो ध्रियते स्वान्ते जिह्वया प्रोच्यते हि यः}
{तस्य रामस्ययज्ञाङ्गं जहार तव पुत्रकः}% ४०

\twolineshloka
{परमत्र महाँल्लाभो भविष्यतितरां रणे}
{यद्रामचरणाम्भोजं द्रक्ष्यामः स्वीयसेवितम्}% ४१

\twolineshloka
{अत्र यत्नो महान्कार्यो हयस्य परिरक्षणे}
{नयिष्यन्ति बलाद्वाहं मया रक्षितमप्यमुम्}% ४२

\twolineshloka
{तस्मादिमं महाराज राज्येन सह सन्नतः}
{वाजिनं भोजनं दत्वा प्रेक्षस्वाङ्घ्रियुगं ततः}% ४३

\twolineshloka
{इति वाक्यं समाकर्ण्य शिवस्य स नृपोत्तमः}
{उवाच तं सुरेन्द्रादिवन्द्यपादाम्बुजद्वयम्}% ४४

\uvacha{वीरमणिरुवाच}

\twolineshloka
{क्षत्रियाणामयं धर्मो यत्प्रतापस्य रक्षणम्}
{तदसौ क्रान्तुमुद्युक्तः क्रतुना हयसंज्ञिना}% ४५

\twolineshloka
{तस्माद्रक्ष्यः स्वप्रतापो येनकेनापि मानिना}
{यावच्छक्यं कर्म कृत्वा शरीरव्ययकारिणा}% ४६

\twolineshloka
{सर्वं कृतं सुतेनेदं गृहीतोऽश्व पुनर्यतः}
{कोपितं रामभूपालं समयार्हं कुरु प्रभो}% ४७

\twolineshloka
{क्षत्त्रियाणामिदं कर्म कर्तव्यार्हं भवेन्नहि}
{यदकस्माद्रिपोः पादौ प्रणमेद्भयविह्वलः}% ४८

\twolineshloka
{रिपवो विहसन्त्येनं कातरोऽयं नृपाधमः}
{क्षुद्रः प्राकृतवन्नीचो नतवान्भयविह्वलः}% ४९

\twolineshloka
{तस्माद्भवान्यथायोग्यं योद्धव्ये समुपस्थिते}
{यद्विधेयं विचार्यैव कर्तव्यं भक्तरक्षणम्}% ५०

\uvacha{शेष उवाच}

\twolineshloka
{इति वाक्यं समाकर्ण्य चन्द्रचूडोवदद्वचः}
{प्रहसन्मेघगम्भीरवाण्या सम्मोहयन्मनः}% ५१

\twolineshloka
{यदि देवास्त्रयस्त्रिंशत्कोटयः समुपस्थिताः}
{तथापि त्वत्तः केनाश्वो गृह्यते मम रक्षितुः}% ५२

\twolineshloka
{यदि रामः समागत्य स्वात्मानं दर्शयिष्यति}
{तदाहं चरणौ तस्य प्रणमामि सुकोमलौ}% ५३

\twolineshloka
{स्वामिना न हि योद्धव्यं महान नय उच्यते}
{अन्ये वीरास्तृणप्रायाः किञ्चित्कर्तुं न वै क्षमाः}% ५४

\twolineshloka
{तस्माद्युद्ध्यस्व राजेन्द्र रक्षके मयि सुस्थिते}
{को गृह्णाति बलाद्वाहं त्रिलोकी यदि सङ्गता}% ५५

\uvacha{शेष उवाच}

\twolineshloka
{एतद्वचः परं श्रुत्वा चन्द्रचूडस्य भूमिपः}
{जहर्ष मानसेऽत्यन्तं युद्धकर्मणि कौतुकी}% ५६

{॥इति श्रीपद्मपुराणे पातालखण्डे शेषवात्स्यायनसंवादे रामाश्वमेधे वीरमणिपुत्रेण हयग्रहणं नाम एकोनचत्वारिंशत्तमोऽध्यायः॥३९॥}

\dnsub{चत्वारिंशत्तमोऽध्यायः}%\resetShloka

\uvacha{शेष उवाच}

\twolineshloka
{सेनाचरा महाराज्ञो महाबलसमन्विताः}
{समागतास्तं पश्यन्तो हयं रामस्य भूपतेः}% १

\twolineshloka
{क्वा सावश्वः केन नीतः कथं वा दृश्यते न सः}
{को गन्ता यमपुर्यां वै वाहं हृत्वा सुमन्दधीः}% २

\twolineshloka
{विलोकयन्तस्तन्मार्गं यावत्सेनाचरा रघोः}
{तावत्प्राप्तो महाराजो महासैन्यपरीवृतः}% ३

\twolineshloka
{पप्रच्छ सेवकान्सर्वान्कुत्राश्वो मम साम्प्रतम्}
{न दृश्यते कथं वाहः स्वर्णपत्रसुशोभितः}% ४

\twolineshloka
{इति तद्वचनं श्रुत्वा सेवकास्ते हयानुगाः}
{प्रोचुर्नाथ मनोवेगो वाहः केनापि कानने}% ५

\twolineshloka
{हृतो न लक्ष्यते तस्मादस्माभिर्मार्गकोविदैः}
{तदत्र यत्नः कर्तव्यो हयप्राप्तिं प्रति प्रभो}% ६

\twolineshloka
{तेषां वचनमाकर्ण्य पप्रच्छ सुमतिं नृपः}
{शत्रुघ्नः शत्रुसंहारकारीमोहनरूपधृक्}% ७

\uvacha{शत्रुघ्न उवाच}

\twolineshloka
{कोऽत्र राजा निवसति कथं वाहस्य सङ्गमः}
{कियद्बलं भूमिपतेर्येन मेऽद्य हृतो हयः}% ८

\uvacha{सुमतिरुवाच}

\twolineshloka
{राजन्देवपुरं ह्येतद्देवेनैव विनिर्मितम्}
{कैलासमिव दुर्गम्यं वैरिसङ्घैः सुसंहतैः}% ९

\twolineshloka
{अस्मिन्वीरमणी राजा महाशूरः प्रतापवान्}
{राज्यं करोति धर्मेण शिवेन परिरक्षितः}% १०

\twolineshloka
{योऽसौ प्रलयकारी स आस्ते भक्त्या वशीकृतः}
{चन्द्रचूडः स्वभक्तस्य पक्षपातं सृजन्सदा}% ११

\twolineshloka
{तस्मादत्र महद्युद्धं गृहीतश्चेद्भविष्यति}
{यत्ताः सन्तः प्रकुर्वन्तु रक्षणं कटकस्य हि}% १२

\twolineshloka
{एवं श्रुत्वा स शत्रुघ्नः सर्वभूपशिरोमणिः}
{सैन्यव्यूहं रचित्वासौ तिष्ठति स्म महायशाः}% १३

\twolineshloka
{अथ तं सुखमासीनं मन्त्रयन्तं सुमन्त्रिणा}
{आजगाम स देवर्षिर्युद्धकौतुकसंयुतः}% १४

\twolineshloka
{तमागतं मुनिं दृष्ट्वा शत्रुघ्नस्तपसां निधिम्}
{अभ्युत्थायासने स्थाप्य मधुपर्कमथार्पयत्}% १५

\twolineshloka
{स्वागतेन च सन्तुष्टं नारदं मुनिसत्तमम्}
{उवाच प्रीणयन्वाचा वाक्यवादविशारदः}% १६

\uvacha{शत्रुघ्न उवाच}

\twolineshloka
{मदीयोऽश्व कुत्र विप्र कथयस्व महामते}
{न लक्ष्यते गतिस्तस्य सेवकैर्मम कोविदैः}% १७

\twolineshloka
{शंस तं येन वा नीतं क्षत्त्रियेण च मानिना}
{कथमत्र हयप्राप्तिर्भविष्यति तपोधन}% १८

\twolineshloka
{इति वाक्यं समाकर्ण्य शत्रुघ्नस्य स नारदः}
{उवाच वीणां रणयन्गायन्रामकथां मुहुः}% १९

\uvacha{नारद उवाच}

\twolineshloka
{एतद्देवपुरं राजन्भूपो वीरमणिर्महान्}
{तत्पुत्रेण वनस्थेन गृहीतस्तव वाजिराट्}% २०

\twolineshloka
{तत्र युद्धं महत्तेऽद्य भविष्यति सुदारुणम्}
{अत्र वीराः पतिष्यन्ति बलशौर्यसमन्विताः}% २१

\twolineshloka
{तस्मादत्र महायत्नात्स्थातव्यं ते महाबल}
{रचय व्यूहरचनां दुर्गमां परसैनिकैः}% २२

\twolineshloka
{जयस्ते भविता राजन्कृच्छ्रेणास्मान्नृपोत्तमात्}
{रामं को नु पराजीयाद्भुवने सकले ह्यपि}% २३

\twolineshloka
{इत्युक्त्वान्तर्दधे विप्रो नभसि स्थितवांस्ततः}
{युद्धं सुदारुणं द्रक्ष्यन्देवदानवयोरिव}% २४

\uvacha{शेष उवाच}

\twolineshloka
{अथ राजा वीरमणिः सर्वशूरशिरोमणिः}
{पटहं घोषितुं स्वीये पुरमध्ये महारवम्}% २५

\twolineshloka
{आह्वयामास सेनान्यं रिपुवीरं महोन्नतम्}
{कथयामास च क्षिप्रं मेघगम्भीरया गिरा}% २६

\uvacha{वीरमणिरुवाच}

\twolineshloka
{सेनानीः पटहस्याज्ञां देहि मे शोभने पुरे}
{तच्छ्रुत्वा मे सुसन्नद्धाः शत्रुघ्नं प्रति यान्तु ते}% २७

\twolineshloka
{इति वाक्यं समाकर्ण्य राज्ञो वीरमणेस्तदा}
{कारयामास पटहं महारवनिनादितम्}% २८

\twolineshloka
{गेहे गेहे च रथ्यायां श्रूयते पटहध्वनिः}
{शत्रुघ्नं यान्तु ये सर्वे वीरा राजपुरे स्थिताः}% २९

\twolineshloka
{ये वै राज्ञः समुल्लङ्घ्य शासनं वीरमानिनः}
{पुत्रा वा भ्रातरो वापि ते वध्याः स्युर्नृपाज्ञया}% ३०

\twolineshloka
{शृण्वन्तु वीराः पुनरप्याह ते पटहे रवम्}
{श्रुत्वा विधीयतामाशु कर्तव्यं मा विलम्बितम्}% ३१

\uvacha{शेष उवाच}

\fourlineindentedshloka
{इति पटहरवं स्वकर्णगोचरं}
{नरवरवीरवरा ययुर्नृपोत्तमम्}
{कनककवचभूषितस्वदेहाः}
{समरमहोत्सव हृष्टचित्तकोशाः}% ३२

\twolineshloka
{केचिद्ययुः शिरस्त्राणं धृत्वा शिरसि शोभनम्}
{कवचेन सुशोभाढ्याः शतकोटिसुशोभिताः}% ३३

\twolineshloka
{रथेन हययुग्मेन मणिकाञ्चनशोभिना}
{ययुस्ते राजसन्देशान्नृवरालयमुन्मदाः}% ३४

\twolineshloka
{केचिन्मतङ्गजैर्मत्तैः केचिद्वाहैः सुशोभनैः}
{ययुर्नपगृहं सर्वे राजसन्देशकारकाः}% ३५

\twolineshloka
{विविक्तस्वर्णकवचशिरस्त्राणसुशोभितः}
{रुक्माङ्गदोऽपि च निजे रथे तिष्ठन्मनोजवे}% ३६

\twolineshloka
{शुभाङ्गदोऽनुजस्तस्य महारत्नमयं दधत्}
{कवचं वपुषि श्रेष्ठं निजं प्रायाद्रणोत्सवे}% ३७

\twolineshloka
{राजभ्राता वीरसिंहः सर्वशस्त्रास्त्रकोविदः}
{ययौ नृपाज्ञया तत्र शासनं भूमिपस्य हि}% ३८

\twolineshloka
{जामेयस्तस्य राज्ञोऽपि बलमित्र इति स्मृतः}
{सन्नद्धः कवची खड्गी जगाम नृपमन्दिरम्}% ३९

\twolineshloka
{सेनानी रिपुवारोऽपि सेनां तां चतुरङ्गिणीम्}
{सज्जां विधाय भूपाय न्यवेदयदथो महान्}% ४०

\twolineshloka
{अथ राजा वीरमणिः सर्वशस्त्रास्त्रपूरितम्}
{मणिसृष्टोच्चचक्रोच्चमारोहत्स्यन्दनोत्तमम्}% ४१

\twolineshloka
{ततो वीरार्णवे शङ्खनिनादश्च समन्ततः}
{श्रूयते कातरान्वीरान्प्रेरयन्निव सङ्गरे}% ४२

\twolineshloka
{भेर्यः समन्ततो जघ्नुः शुभवादकवादिताः}
{अनीकान्यत्र तस्यासन्सङ्ग्रामाय प्रतस्थुषः}% ४३

\twolineshloka
{सर्वे कृतस्वस्त्ययनाः सर्वाभरणभूषिताः}
{सर्वशस्त्रास्त्रसम्पूर्णा ययुः समरमण्डलम्}% ४४

\twolineshloka
{भेरीशङ्खनिनादेन पूरिताश्च नगा गुहाः}
{आकारितुं गतः किं नु तद्रवः स्वर्गसंस्थितान्}% ४५

\twolineshloka
{तस्मिन्कोलाहले वृत्ते राजा वीरमणिर्महान्}
{रणोत्साहेन संयुक्तो ययौ प्रधनमण्डलम्}% ४६

\twolineshloka
{आगत्य संस्थितस्तावद्रथपत्तिसमाकुलम्}
{समुद्र इव तत्स्थानात्प्लावितुं पुरुषानयात्}% ४७

\twolineshloka
{तदागतं बलं दृष्ट्वा रथिभिः शस्त्रकोविदैः}
{कोलाहलीकृतं सर्वमुवाच सुमतिं नृपः}% ४८

\uvacha{शत्रुघ्न उवाच}

\twolineshloka
{समागतो वीरमणिर्मम वाजिधरो बली}
{योद्धुं मां महता भूयः सैन्येन चतुरङ्गिणा}% ४९

\twolineshloka
{कथं युद्धं प्रकर्तव्यं के योत्स्यन्ति बलोत्कटाः}
{तान्सर्वान्दिश मे वीरान्यथा स्याज्जय ईप्सितः}% ५०

\uvacha{सुमतिरुवाच}

\twolineshloka
{स्वामिन्नसौ महाराजो महासैन्यपरीवृतः}
{समागतः स युद्धार्थं शिवभक्तिसमन्वितः}% ५१

\twolineshloka
{साम्प्रतं युद्ध्यतां वीरः पुष्कलः परमास्त्रवित्}
{अन्येपि नीलरत्नाद्या योद्धारो युद्धकोविदाः}% ५२

\twolineshloka
{शिवेन सह योद्धव्यं राज्ञा वा भवतानघ}
{द्वन्द्वयुद्धेन जेतव्यो महाबलपराक्रमः}% ५३

\twolineshloka
{अनेन विधिना राजञ्जयस्तेऽत्र भविष्यति}
{पश्चाद्यद्रोचते स्वामिंस्तत्कुरुष्व महामते}% ५४

\uvacha{शेष उवाच}

\twolineshloka
{इति वाक्यं समाकर्ण्य शत्रुघ्नः परवीरहा}
{सुभटानादिदेशाथ युद्धाय कृतनिश्चयः}% ५५

\twolineshloka
{सर्वैः ससैन्यैर्युद्धार्थं राजभिः शस्त्रकोविदैः}
{यथा स्यान्मे जयः क्षिप्रं यतितव्यं तथा पुनः}% ५६

\twolineshloka
{जयार्थं राघवस्यैव श्रुत्वा ते रणकोविदाः}
{महोत्साहेन संयुक्ता ययुर्योद्धुं तु सैनिकैः}% ५७

{॥इति श्रीपद्मपुराणे पातालखण्डे शेषवात्स्यायनसंवादे रामाश्वमेधे वीरमणिना सह युद्धनिश्चयो नाम चत्वारिंशत्तमोऽध्यायः॥४०॥}

\dnsub{एकचत्वारिंशत्तमोऽध्यायः}%\resetShloka

\uvacha{शेष उवाच}

\twolineshloka
{युद्धाय ते सुसन्नद्धाः शत्रुघ्नस्य महाबलाः}
{ययुर्वीरमणेः सैन्यमध्ये शौर्यसमन्विताः}% १

\twolineshloka
{शरान्विमुञ्चमानास्ते भिन्दन्तः सैनिकान्बहून्}
{व्यदृश्यन्त रणान्तःस्थाः शरासनधरा नराः}% २

\twolineshloka
{अनेके निहतास्तत्र गजा मणिमया रथाः}
{भग्ना वाहसमेताश्च दृश्यन्ते रणमण्डले}% ३

\twolineshloka
{विहितं कदनं तेषां श्रुत्वा रुक्माङ्गदो बली}
{रथे मणिमये तिष्ठन्ययौ योद्धुं ससैनिकान्}% ४

\twolineshloka
{शरासने शरान्धास्यन्निषुधी अक्षयौ दधत्}
{शोणनेत्रान्तरो भीमो महाकोपसमन्वितः}% ५

\twolineshloka
{अनेकबाणसंविग्नान्कुर्वञ्छूरान्सहस्रशः}
{हाहाकारं कारयंस्तद्ययौ रुक्माङ्गदो बली}% ६

\twolineshloka
{राजपुत्रः स्वसदृशं बलेन यशसाश्रिया}
{आह्वयामास शत्रुघ्नं भारतिं पुष्कलं बली}% ७

\uvacha{रुक्माङ्गद उवाच}

\twolineshloka
{आगच्छ वीरकर्मा त्वं महाबलपराक्रम}
{मया योद्धुं तु बलिना राजपुत्रेण भास्वता}% ८

\twolineshloka
{किमन्यैस्त्रासितैर्वीर निहतैः कोटिभिर्नरैः}
{मया समं महायुद्धं विधाय जयमाप्नुहि}% ९

\twolineshloka
{इत्युक्तवं तं तरसा प्रहसन्पुष्कलो बली}
{जघान विपुले मध्ये वक्षसस्तीक्ष्णपर्वभिः}% १०

\twolineshloka
{तदमृष्यन्राजपुत्रो महाचापे दधच्छरान्}
{जघान दशभिर्वीरं पुष्कलं वक्षसोऽन्तरे}% ११

\twolineshloka
{उभौ समरसंरब्धावुभावपि जयैषिणौ}
{रेजाते समरे तौ हि कुमारतारकौ यथा}% १२

\twolineshloka
{बाणान्धनुषि सन्धाय दशसङ्ख्यान्महाशितान्}
{अकरोत्पुष्कलो वीरो विरथं राजपुत्रकम्}% १३

\twolineshloka
{चतुर्भिश्चतुरोवाहान्द्वाभ्यां सूतमपातयत्}
{एकेन ध्वजमेतस्य द्वाभ्यां स्यन्दनरक्षकौ}% १४

\twolineshloka
{एकेन हृदि विव्याध राजपुत्रस्य वेगवान्}
{तदद्भुतं कर्म दृष्ट्वा सर्वे वीराः प्रतोषिताः}% १५

\twolineshloka
{सच्छिन्नधन्वा विरथो हताश्वो हतसारथिः}
{अत्यन्तं कोपमापन्नः स्यन्दनं परमाविशत्}% १६

\twolineshloka
{स स्थित्वा स्यन्दनवरे हयरत्नेन भूषिते}
{शरासनं महद्धृत्वा सुदृढं गुणपूरितम्}% १७

\twolineshloka
{उवाच पुष्कलं वीरं रुक्माङ्गद इदं वचः}
{महत्पराक्रमं कृत्वा क्व यास्यसि परन्तप}% १८

\twolineshloka
{पश्य मेऽद्यपराक्रान्तिं यद्बलेन विनिर्मिताम्}
{यत्नात्तिष्ठस्व भो वीर नयामि त्वद्रथं नभः}% १९

\twolineshloka
{इत्युक्त्वा शरमत्युग्रं दधार स्वशरासने}
{मन्त्रयित्वा ततश्चास्त्रं भ्रामकं पौष्कले रथे}% २०

\twolineshloka
{मुमोच निशितं बाणं स्वर्णपङ्खैकशोभितम्}
{तेन बाणेन नीतोऽस्य रथो योजनमात्रकम्}% २१

\twolineshloka
{धृतः कृच्छ्रेण सूतेन रथो बभ्राम भूतले}
{कृच्छ्रेण प्राप्य स्वस्थानं पुष्कलः परमास्त्रवित्}% २२

\twolineshloka
{जगाद वचनं तं वै बाणं बिभ्रच्छरासने}
{स्वर्गं प्राप्नुहि वीराग्र्य सर्वदेवैश्च सेवितम्}% २३

\twolineshloka
{त्वादृशाः पृथिवीयोग्या न भवन्ति नृपोत्तम}
{शतक्रतुसभायोग्यास्तद्गच्छ त्वं सुरालयम्}% २४

\twolineshloka
{इत्युक्त्वा स मुमोचास्त्रमाकाशप्रापकं महत्}
{तेन बाणेन सरथो ययौ खमनुलोमतः}% २५

\twolineshloka
{सर्वांल्लोकानतिक्रामन्ययौ सूर्यस्य मण्डलम्}
{तज्ज्वालया रथो दग्धो हयसूतसमन्वितः}% २६

\twolineshloka
{तत्करैर्दग्धभूयिष्ठ कलेवरः सुदुःखितः}
{पपात चन्द्रचूडं स धृत्वा हृद्यसुखार्दनम्}% २७

\twolineshloka
{भूमौ निपतितस्तत्र करदग्धकलेवरः}
{अत्यन्तं दुःखमापन्नो मुमूर्च्छ रणमण्डले}% २८

\twolineshloka
{तस्मिन्निपतिते भूमौ मूर्च्छिते राजपुत्रके}
{हाहाकारो महानासीत्तत्र सङ्ग्राममूर्धनि}% २९

\twolineshloka
{वैरिणो जयलक्ष्मीं ते प्रापुः पुष्कलमुख्यकाः}
{पलायनपरा जाता वैरिणो हयरक्षकाः}% ३०

\twolineshloka
{तदा पुत्रस्य वै मूर्च्छां दृष्ट्वा वीरमणिर्नृपः}
{प्रायात्समरमध्यस्थं पुष्कलं कोपपूरितः}% ३१

\twolineshloka
{तदा भूमिश्चचालेयं सपर्वतवनोत्तमा}
{शूरा वै हर्षमापन्नाः कातरा भयपीडिताः}% ३२

\twolineshloka
{चापं महद्दधानः स इषुधी अक्षयावपि}
{रोषान्निःश्वासमामुञ्चन्नाह्वयामास वैरिणम्}% ३३

{॥इति श्रीपद्मपुराणे पातालखण्डे शेषवात्स्यायनसंवादे रामाश्वमेधे रुक्माङ्गदपराजय पुष्कलविजयो नाम एकचत्वारिंशत्तमोऽध्यायः॥४१॥}

\dnsub{द्विचत्वारिंशत्तमोऽध्यायः}%\resetShloka

\uvacha{शेष उवाच}

\twolineshloka
{आह्वयन्तं महासैन्यवारिधौ पुष्कलं नृपम्}
{समालक्ष्य कपीन्द्रोऽपि हनूमांस्तमधावत}% १

\fourlineindentedshloka
{लाङ्गूलमुद्यम्य विशालदेहं}
{सरावमातत्य पयोदघोषम्}
{रणस्थितान्वीरवरान्कपीन्द्रो}
{जगाम तं वीरमणिं नृपेन्द्रम्}% २

\twolineshloka
{आयान्तं तं हनूमन्तं वीक्ष्य पुष्कल उद्भटः}
{विलोकयामासदृशा वैरिक्रोध सुशोणया}% ३

\twolineshloka
{जगाद तं हनूमन्तं पुष्कलः परमास्त्रवित्}
{मेघगम्भीरया वाचा नादयन्रणमण्डलम्}% ४

\uvacha{पुष्कल उवाच}

\twolineshloka
{कथं त्वं समरे योद्धुमागतोसि महाकपे}
{कियद्बलं स्वल्पमेतद्राज्ञो वीरमणेर्महत्}% ५

\twolineshloka
{यत्र त्रिजगती सर्वा सम्मुखे समुपागता}
{तत्र त्वं लीलया योद्धुं यातुमिच्छसि वा न वा}% ६

\twolineshloka
{कोयं राजा वीरमणिः कियद्बलमथाल्पकम्}
{अत्रागमनमत्युग्रं तव वीर न भाव्यते}% ७

\twolineshloka
{रघुनाथकृपापाङ्गादहं निस्तीर्य दुस्तरम्}
{क्षणान्निर्यामि कीशेन्द्र मा चित्तं कुरु सङ्गरे}% ८

\twolineshloka
{त्वया राक्षसपाथोधिस्तीर्णो रामकृपाव्रजात्}
{तथा रामं सुसंस्मृत्य निस्तरिष्यामि दुस्तरम्}% ९

\twolineshloka
{ये केचिद्दुस्तरं प्राप्य रघुनाथं स्मरन्ति च}
{तेषां दुःखोदधिः शुष्को भविष्यति न संशयः}% १०

\twolineshloka
{तस्माद्व्रज महावीर शत्रुघ्नसविधे बलिन्}
{एष आयामि निर्जित्य भूपं वीरमणिं क्षणात्}% ११

\uvacha{शेष उवाच}

\twolineshloka
{इति धीरां समाकर्ण्य वाणीं पुष्कलभाषिताम्}
{जगाद वचनं भूयः पुष्कलं परवीरहा}% १२

\uvacha{हनुमानुवाच}

\twolineshloka
{पुत्र मा साहसं कार्षीर्भूपं वीरमणिं प्रति}
{एष दाता शरण्यश्च बलशौर्यसमन्वितः}% १३

\twolineshloka
{त्वं बालः स्थविरो भूपोऽखिलशस्त्रास्त्रवित्तमः}
{अनेके विजिताः सङ्ख्ये वीराः शौर्यसुशोभिनः}% १४

\twolineshloka
{जानीहि पार्श्वे तस्य त्वं रक्षितारं सदाशिवम्}
{भक्त्या वशीकृतं स्थाणुं सोमं चैतत्पुरिस्थितम्}% १५

\twolineshloka
{तस्मादहमनेनैव योत्स्ये भूपेन पुष्कल}
{अन्यान्वीरान्विजित्वा त्वं कीर्तिमाप्नुहि पुष्कलाम्}% १६

\uvacha{पुष्कल उवाच}

\twolineshloka
{शिवो भक्त्या वशीकृत्य स्वपुरे स्थापितोऽमुना}
{परमस्याशु हृदयेन तिष्ठति महेश्वरः}% १७

\twolineshloka
{सदाशिवोयमाराध्य परमं स्थानमागतः}
{स रामो मन्मनस्त्यक्त्वा न क्वापि परिगच्छति}% १८

\twolineshloka
{यत्र रामस्तत्र विश्वं सर्वं स्थास्नु चरिष्णु च}
{तस्मादहं जयिष्यामि रणे वीरमणिं नृपम्}% १९

\twolineshloka
{व्रज त्वं समरे योद्धुमन्यान्मानिवरान्नृपान्}
{वीरसिंहमुखान्कीश मच्चिन्तां मा कुरु प्रभो}% २०

\twolineshloka
{वाचमित्थं समाकर्ण्य हनूमान्धीरसेविताम्}
{जगाम समरे योद्धुं वीरसिंहं नृपानुजम्}% २१

\twolineshloka
{लक्ष्मीनिधिः सुतेनास्य शुभाङ्गदसुसंज्ञिना}
{द्वैरथेन प्रयुयुधे महाशस्त्रास्त्रवेदिना}% २२

\twolineshloka
{बलमित्रेण सुमदः स्वप्रतापबलोर्जितः}
{योद्धुं सशस्त्रः सङ्ग्रामं विचचार नृपात्मजः}% २३

\twolineshloka
{आह्वयन्तं नृपं दृष्ट्वा द्वैरथे युद्धकोविदः}
{पुष्कलो रुक्मखचिते रथे तिष्ठन्ययौ हि तम्}% २४

\twolineshloka
{राजा तमागतं दृष्ट्वा पुष्कलं युद्धकोविदम्}
{उवाच निर्भिया वाण्या रणमध्ये सुभाषितः}% २५

\uvacha{वीरमणिरुवाच}

\twolineshloka
{बालमायाहि मां क्रुद्धं सङ्ग्रामे चण्डकोपनम्}
{गच्छ प्राणपरीप्सायै मा युद्धं कुरु मे सह}% २६

\twolineshloka
{त्वादृशान्बालकान्भूपा मादृशाः कृपयन्ति हि}
{प्रहरन्ति न चैतान्वै तस्माद्गच्छ रणाद्बहिः}% २७

\twolineshloka
{यावत्त्वं न मया दृष्टश्चक्षुर्भ्यां तावदुन्मनाः}
{साम्प्रतं त्वां प्रहर्तुं न मनः समभिकाङ्क्षति}% २८

\twolineshloka
{यत्त्वया मत्सुतो बाणैर्भिन्नो मूर्च्छीकृतः पुनः}
{सर्वं मया क्षान्तमद्य तवबालधियो महत्}% २९

\onelineshloka*
{इति वाक्यं समाकर्ण्य पुष्कलो निजगाद तम्}

\uvacha{पुष्कल उवाच}

\onelineshloka
{बालोऽहं त्वं महावृद्धः सर्वशस्त्रास्त्रकोविदः}% ३०

\twolineshloka
{क्षत्रियाणां मतं चैव ये बलाधिक्यसंयुताः}
{त एव वृद्धा भूपाग्र्य न वयोवृद्धतां गताः}% ३१

\twolineshloka
{मया ते मूर्च्छितः पुत्रः सशौर्यबलदर्पितः}
{इदानीं त्वामहं शस्त्रैः पातयिष्यामि सङ्गरे}% ३२

\twolineshloka
{तस्मात्त्वं यत्नतस्तिष्ठ राजन्सङ्ग्राममूर्धनि}
{रामभक्तं न मां कश्चिज्जयतीन्द्रपदे स्थितः}% ३३

\twolineshloka
{इत्थं भाषितमाश्रुत्य पुष्कलस्य नृपाग्रणीः}
{जहास बालं संवीक्ष्य कोपं च व्यदधात्पुनः}% ३४

\twolineshloka
{तं वै कुपितमालक्ष्य भरतात्मज उन्मदः}
{जघान शरविंशत्या राजानं हृदि तीक्ष्णया}% ३५

\twolineshloka
{राजा तानागतान्दृष्ट्वा बाणांस्तेन विमोचितान्}
{चिच्छेद परमक्रुद्धः शरैस्तीक्ष्णैरनेकधा}% ३६

\twolineshloka
{तद्बाणच्छेदनं दृष्ट्वा भारतिः परवीरहा}
{चुकोप हृदयेऽत्यन्तं राजानं च त्रिभिः शरैः}% ३७

\twolineshloka
{विव्याध भाले भूपाल पुत्रः पुष्कलसंज्ञकः}
{तत्र लग्ना विरेजुस्ते त्रिकूटशिखराणि किम्}% ३८


\threelineshloka
{तैर्बाणैर्व्यथितो राजा जघान नवभिः शरैः}
{हृदये पुष्कलं वीरं महाकोपसमन्वितः}
{तैर्वत्सदन्तैर्बह्वस्रं पीतं रामानुजाङ्गजम्}% ३९

\twolineshloka
{सर्पा आशीविषा यद्वत्क्रुद्धास्तद्वपुषि स्थिताः}
{परमं कोपमापन्नः पुष्कलो भूमिपं पुनः}% ४०

\twolineshloka
{बाणानां शतकेनाशु बिभेद शितपर्वणाम्}
{तैर्बाणैः कवचं भिन्नं किरीटः सशिरस्त्रकः}% ४१

\twolineshloka
{रथो धनुर्महत्सज्यं छिन्नं कोपपरिप्लवात्}
{क्षतजेन परिप्लुष्टो बाणभिन्नकलेवरः}% ४२

\twolineshloka
{अन्यं स्यन्दनमारुह्य जगाम भरतात्मजम्}
{धन्योसि वीर रामस्य चरणाब्जमधुव्रत}% ४३

\twolineshloka
{महत्कृतं कर्म तेऽद्य यदहं विरथीकृतः}
{प्राणान्रक्षस्व भो वीर साम्प्रतं मयि युद्ध्यति}% ४४

\twolineshloka
{सुलभा न तव प्राणाः कालरूपे मयि स्थिते}
{इत्युक्त्वा व्यहनद्बाणैरसङ्ख्यैः शस्त्रकोविदः}% ४५

\twolineshloka
{भूमौ दिशि च तद्बाणा नान्यद्दृश्येत तत्र ह}
{अनेके गजसाहस्रा भिन्ना अश्वाः समन्ततः}% ४६

\twolineshloka
{रथारथियुतास्तेन छिन्ना भिन्ना द्विधाकृताः}
{शोणितौघा सरित्तत्र प्रसुस्राव रणाङ्गणे}% ४७

\twolineshloka
{यत्रोन्मदा हि मातङ्गा दृश्यन्ते शैलशृङ्गवत्}
{केशाः शैवाललक्ष्यास्ते मुहुः प्राणिशिरः स्थिताः}% ४८

\twolineshloka
{अनेके पाणयश्छिन्ना वीराणां मुद्रिकाश्रियः}
{दृश्यन्ते अहिवत्तत्र चन्दनादिकरूषिताः}% ४९

\twolineshloka
{शिरांसि च भटाग्र्याणां कच्छपाभां वहन्ति वै}
{मांसानि पङ्का यत्रासन्वीराणां महतां ततः}% ५०

\twolineshloka
{एवं व्यतिकरे वृत्ते योगिन्यः शतशो रणे}
{पपुः पात्रेण रुधिरं प्राणिनां रणपातिनाम्}% ५१

\twolineshloka
{मांसानि बुभुजुस्ता वै हर्षकौतुकसंयुताः}
{पीत्वा तु शोणितं तत्र भक्षित्वा मांसमुन्मदाः}% ५२

\twolineshloka
{ननृतुर्जहसुः प्रोच्चैरुज्जगुः प्रधनाङ्गणे}
{पिशाचास्तत्र समरे प्राणिनां मस्तकानि वै}% ५३

\twolineshloka
{धृत्वा कराभ्यां मत्ताङ्गास्तालवद्वादनोद्यताः}
{शिवास्तत्र महामांसं पतितानां रणाङ्गणे}% ५४

\twolineshloka
{भक्षित्वा व्यनदन्मत्ताः कातराणां भयप्रदाः}
{कातरास्त्राः समापन्ना गताः कुञ्जरकोटरे}% ५५

\twolineshloka
{भक्षिता योगिनीभिस्ते पापिनां क्वापि न स्थितिः}
{एतत्कदनमालक्ष्य स्वसैन्यस्य रथाग्रणीः}% ५६

\twolineshloka
{पुष्कलोऽपि चकारात्र कदनं रणमण्डले}
{भिद्यन्ते गजशीर्षाणि पतन्ति मौक्तिकानि तु}% ५७


\threelineshloka
{दृश्यते लोमभिः पूर्णा ताम्रपर्णीव तन्नदी}
{पुष्कलप्रहिता बाणा नृणामङ्गेषु सङ्गताः}
{कुर्वन्ति प्राणविच्छेदं वीराणामपि सर्वतः}% ५८

\twolineshloka
{सर्वे रुधिरसिक्ताङ्गाः सर्वे च्छिन्ननिजाङ्गकाः}
{दृश्यन्ते किंशुका यद्वत्सुभटाः प्रधनाङ्गणे}% ५९

\twolineshloka
{एतस्मिन्समये क्रुद्धः समाभाष्य महीपतिम्}
{जघान बहुबाणैस्तं रोषपूरपरिप्लुतः}% ६०

\twolineshloka
{तद्बाणवेधभिन्नाङ्गो विशीर्णकवचो नृपः}
{महाबलं तं मन्वानः प्राहरच्छरकोटिभिः}% ६१

\twolineshloka
{तैर्बाणैः कवचान्मुक्तं सुस्राव बहुशोणितम्}
{वपुर्बभूव रुचिरं शरपञ्जरगोचरम्}% ६२

\twolineshloka
{शरपञ्जरमध्यस्थो विह्वलीकृतमानसः}
{शरान्नेतुं च सन्धातुं न क्षमः स च भारतिः}% ६३

\twolineshloka
{रामं स्मृत्वा धनुर्धृत्वा करे सज्जं महद्दृढम्}
{मुमोच बाणान्निशितान्वैरिवृन्दनिवारणान्}% ६४

\twolineshloka
{तैर्बाणैः शरजालं तद्विधूय मुनिपुङ्गव}
{शङ्खं प्रध्माय समरे जगाद गतभीर्नृपम्}% ६५

\uvacha{पुष्कल उवाच}

\twolineshloka
{त्वया कृतं महत्कर्म यन्मां बाणस्य पञ्जरे}
{गोचरं कृतवान्वीर वीरतापनमुद्भटम्}% ६६

\twolineshloka
{वृद्धत्वान्मम मान्योसि साम्प्रतं रणमण्डले}
{पश्यमेऽद्य पराक्रान्तं राजन्वीरमणे महत्}% ६७

\twolineshloka
{बाणत्रयेण भो वीर मूर्च्छितं करवै नहि}
{तर्हि प्रतिज्ञां शृणु वै सर्ववीरविमोहिनीम्}% ६८

\twolineshloka
{गङ्गां प्राप्यापि यो वै तां निन्दित्वा पापहारिणीम्}
{न मज्जति महापापो महामूढविचेष्टितः}% ६९

\twolineshloka
{तस्य पापं ममैवास्तु चेन्न त्वां रणमण्डले}
{पातये मूर्च्छया वीर सन्नद्धो भव भूपते}% ७०

\twolineshloka
{इति वाक्यं समाकर्ण्य पुष्कलस्य नृपोत्तमः}
{चुकोप भृशमुद्विग्नः सन्दधे निशिताञ्छरान्}% ७१

\twolineshloka
{ते शरा हृदयं भित्त्वा गतास्ते भारतेर्महत्}
{पेतुः क्षितावधो यद्वद्रामभक्तिपराङ्मुखाः}% ७२

\twolineshloka
{ततः शरं मुमोचास्मै निशितं वह्निसप्रभम्}
{लक्षीकृत्य महद्वक्षः कपाटतटविस्तृतम्}% ७३

\twolineshloka
{स बाणो भूमिपतिना द्विधा छिन्नः शरेण हि}
{पपात रथमध्ये स रविमण्डलवज्ज्वलन्}% ७४

\twolineshloka
{अपरं बाणमाधत्त मातृभक्तिभवं ततः}
{निधाय पुण्यं सोऽप्येष चिच्छेद महता पुनः}% ७५

\twolineshloka
{तदा खिन्नः स हृदये किङ्कर्तव्यमिति स्मरन्}
{रामं हृदि निजार्तिघ्नं मुमोच परमास्त्रवित्}% ७६

\twolineshloka
{स बाणस्तस्य हृदये लग्न आशीविषोपमः}
{मूर्च्छामप्रापयत्तं वै ज्वलन्सूर्यसमप्रभः}% ७७

\twolineshloka
{ततो हाहाकृतं सर्वं पलायनपरायणम्}
{राज्ञि सम्मूर्च्छिते जाते पुष्कलो जयमाप्तवान्}% ७८

{॥इति श्रीपद्मपुराणे पातालखण्डे शेषवात्स्यायनसंवादे रामाश्वमेधे वीरमणेः पराभवो नाम द्विचत्वारिंशत्तमोऽध्यायः॥४२॥}

\dnsub{त्रिचत्वारिंशत्तमोऽध्यायः}%\resetShloka

\uvacha{शेष उवाच}

\twolineshloka
{हनूमान्वीरसिंहं तु समागत्याब्रवीद्वचः}
{तिष्ठ यासि कुतो वीर जेष्यामि त्वां क्षणादिह}% १

\twolineshloka
{एवमुक्तं समाकर्ण्य प्लवगस्य वचो महत्}
{कोपपूरपरिप्लुष्टः कार्मुकं जलदस्वनम्}% २

\twolineshloka
{विनद्य घोरान्निशितान्बाणान्मुञ्चन्बभौ रणे}
{आषाढे जलदस्येव धारासारे मनोहरः}% ३

\twolineshloka
{तान्दृष्ट्वा निशितान्बाणान्स्वदेहे सुविलग्नकान्}
{चुकोप हृदयेऽत्यतं हनूमानञ्जनी सुतः}% ४

\twolineshloka
{मुष्टिना ताडयामास हृदये वज्रसारिणा}
{समुष्टिना हतो वीरः पपात धरणीतले}% ५

\twolineshloka
{मूर्च्छितं तं समालोक्य पितृव्यं स शुभाङ्गदः}
{रुक्माङ्गदोऽपि सम्मूर्च्छां त्यक्त्वागाद्रणमण्डलम्}% ६

\twolineshloka
{बाणान्समभिवर्षन्तौ मेघाविव महास्वनौ}
{कुर्वन्तौ कदनं घोरं प्लवङ्गं प्रति जग्मतुः}% ७

\twolineshloka
{तौ दृष्ट्वा समरे वीरौ समायातौ कपीश्वरः}
{लाङ्गूलेन च संवेष्ट्य सरथौ चापधारकौ}% ८

\twolineshloka
{स्फोटयामास भूदेशे तत्क्षणान्मूर्च्छितावुभौ}
{निश्चेष्टौ समभूतां तौ रुधिरारक्तदेहकौ}% ९

\twolineshloka
{बलमित्रश्चिरं युद्धं विधाय सुमदेन हि}
{मूर्च्छामप्रापयत्तं वै बाणैः सुशितपर्वभिः}% १०

\twolineshloka
{पुष्कलेन क्षणान्नीतो मूर्च्छां चैतन्यवर्जिताम्}
{जयमाप्तं तु कटकं शत्रुघ्नस्य भटार्दनम्}% ११

\twolineshloka
{एतस्मिन्समये साम्बः स्यन्दनं वरमास्थितः}
{विस्फारयन्धनुर्दिव्यमुपाधावद्भटानिमान्}% १२

\twolineshloka
{जटाजूटान्तरगतां चन्द्ररेखां वहन्महान्}
{सर्पाभूषां मनःस्पर्शां दधदाजगवं धनुः}% १३

\twolineshloka
{मूर्च्छितान्स्वजनान्दृष्ट्वा भक्तार्तिघ्नो महेश्वरः}
{योद्धुं प्रायान्महासैन्यैः शत्रुघ्नस्य भटानिमान्}% १४

\twolineshloka
{सगणः सपरीवारः कम्पयन्पृथिवीतलम्}
{भक्तरक्षार्थमागच्छंस्त्रिपुरं तु पुरा यथा}% १५

\twolineshloka
{कोपाच्छोणतरे नेत्रे वहन्प्रलयकारकः}
{पश्यन्वीरान्बहुमतीन्पिनाकी देववन्दितः}% १६

\twolineshloka
{तमागतं महेशानं वीक्ष्य रामानुजो बली}
{जगाम समरे योद्धुं सर्वदेवशिरोमणिम्}% १७

\twolineshloka
{अथागतं तु शत्रुघ्नं रुद्रो वीक्ष्य पिनाकधृक्}
{उवाच परमापन्नः कोपं सगुणचापधृक्}% १८

\twolineshloka
{पुष्कलेन महत्कर्म कृतं रामाङ्घ्रिसेविना}
{मद्भक्तं यो रणे हत्वा गतः समरमण्डलम्}% १९

\twolineshloka
{अद्य क्वास्ति परो वीरः पुष्कलः परमास्त्रवित्}
{तं हत्वा सुखमाप्स्यामि समरे भक्तपीडनम्}% २०

\uvacha{शेष उवाच}

\twolineshloka
{इत्युक्त्वा वीरभद्रं स प्रेषयामास पुष्कलम्}
{याहि त्वं समरे योद्धुं पुष्कलं सेवकार्दनम्}% २१

\twolineshloka
{नन्दिनं प्रेषयामास हनूमन्तं महाबलम्}
{कुशध्वजं प्रचण्डं तु भृङ्गिणं च सुबाहुकम्}% २२

\twolineshloka
{सुमदं चण्डनामानं गणं स्वीयं समादिशत्}
{पुष्कलस्तु समायान्तं वीरभद्रं महागणम्}% २३

\twolineshloka
{महारुद्रस्य संवीक्ष्य योद्धुं प्रायान्महामनाः}
{पुष्कलः पञ्चभिर्बाणैस्ताडयामास संयुगे}% २४

\twolineshloka
{तैर्बाणैः क्षतगात्रस्तु त्रिशूलं स समादिशत्}
{स त्रिशूलं क्षणाच्छित्त्वा व्यगर्जत महाबलः}% २५

\twolineshloka
{छिन्नं स्वीयं त्रिशूलं वै वीक्ष्य रुद्रानुगो बली}
{खट्वाङ्गेन जघानाशु मस्तके भारतिं द्विज}% २६

\twolineshloka
{खट्वाङ्गाभिहतः सोऽथ मुमूर्च्छ क्षणमुद्भटः}
{विहाय मूर्च्छां सद्वीरः पुष्कलः परमास्त्रवित्}% २७

\twolineshloka
{शरैश्चिच्छेद खट्वाङ्गं करस्थं तस्य तत्क्षणात्}
{वीरभद्रः स्वकेच्छिन्ने खट्वाङ्गे करसंस्थिते}% २८

\twolineshloka
{परमक्रोधमापन्नो बभञ्ज रथिनो रथम्}
{भङ्क्त्वा रथं तु वीरस्य पदातिं च विधाय सः}% २९

\twolineshloka
{बाहुयुद्धेन युयुधे पुष्कलेन महात्मना}
{स पुष्कलो रथं त्यक्त्वा चूर्णितं तेन वेगतः}% ३०

\twolineshloka
{मुष्टिना ताडयामास वीरभद्रं महाबलः}
{अन्योन्यं मुष्टिभिर्घ्नन्तावूरुभिर्जानुभिस्तथा}% ३१

\twolineshloka
{परस्परवधोद्युक्तौ परस्परजयैषिणौ}
{एवं चतुर्दिनमभूद्रात्रिं दिवमपीशयोः}% ३२

\twolineshloka
{न कोपि तत्र हीयेत न जीयेत महाबलः}
{पञ्चमे तु दिने वृत्ते वीरभद्रो महाबलः}% ३३

\twolineshloka
{गृहीत्वा नभ उड्डीनो महावीरं तु पुष्कलम्}
{तत्र युद्धं तयोरासीद्देवासुरविमोहनम्}% ३४

\twolineshloka
{मुष्टिना चरणाघातैर्बाहुभिः सुखुरैर्महत्}
{तदात्यन्तं प्रकुपितः पुष्कलो वीरभद्रकम्}% ३५

\twolineshloka
{गृहीत्वा कण्ठदेशे तु ताडयामास भूतले}
{तत्प्रहारेण व्यथितो वीरभद्रो महाबलः}% ३६

\twolineshloka
{गृहीत्वा पुष्कलं पादे जघानास्फालयन्मुहुः}
{ताडयित्वा महीदेशे पुष्कलं सुमहाबलः}% ३७

\twolineshloka
{त्रिशूलेन चकर्ताशु शिरो ज्वलितकुण्डलम्}
{जगर्ज पुष्कलं हत्वा वीरभद्रो महाबलः}% ३८

\twolineshloka
{गर्जता तेन शार्वेण प्रापितास्त्रा समुद्भटाः}
{हाहाकारो महानासीत्पुष्कले पतिते रणे}% ३९

\twolineshloka
{त्रासं प्रापुर्जनाः सर्वे रणमध्ये सुकोविदाः}
{ते शशंशुश्च शत्रुघ्नं पुष्कलं पतितं रणे}% ४०

\twolineshloka
{निहतं वीरभद्रेण महेश्वरगणेन वै}
{इत्याश्रुत्य महावीरः पुष्कलस्य वधं तदा}% ४१

\twolineshloka
{दुःखं प्राप्तो रणेऽत्यतं कम्पमानः शुचा महान्}
{तं दुःखितं च शत्रुघ्नं ज्ञात्वा रुद्रो ऽब्रवीद्वचः}% ४२

\twolineshloka
{शत्रुघ्नं समरे वीरं शोचन्तं पुष्कले हते}
{रे शत्रुघ्न रणे शोकं मा कृथाः सुमहाबल}% ४३

\twolineshloka
{वीराणां रणमध्ये तु पातनं कीर्तये स्मृतम्}
{धन्यो वीरः पुष्कलाख्यो यश्च वै दिनपञ्चकम्}% ४४

\twolineshloka
{युयुधे वीरभद्रेण महाप्रलयकारिणा}
{येन क्षणाद्विनिहतो दक्षो मदपमानकृत्}% ४५

\twolineshloka
{क्षणाद्विनिहता येन दैत्यास्त्रिपुरसैनिकाः}
{तस्माद्युद्ध्स्व राजेन्द्र शोकं त्यक्त्वा महाबल}% ४६

\twolineshloka
{यत्नात्तिष्ठाद्य वीराग्र्य मयि योद्धरि संस्थिते}
{शोकं सन्त्यज्य शत्रुघ्नो वीरश्चुक्रोध शङ्करम्}% ४७

\twolineshloka
{आत्तसज्जधनुर्बाणैः प्रचच्छाद महेश्वरम्}
{ते बाणाः सुरशीर्षण्य वपुषं क्षतविक्षतम्}% ४८

\twolineshloka
{अकुर्वत महच्चित्रं भक्तरक्षार्थमागतम्}
{ते बाणाः शङ्करस्यापि बाणा नभसि संस्थिताः}% ४९

\twolineshloka
{व्याप्यैतत्सकलं विश्वं चित्रकारि मुनेरपि}
{तद्बाणयोर्युद्धबलं वीक्ष्य सर्वत्र मेनिरे}% ५०

\twolineshloka
{प्रलयं लोकसंहारकारकं सर्वमोहकम्}
{आकाशे तु विमानानि संश्रित्य स्वपुरस्थिताः}% ५१

\twolineshloka
{विलोकयितुमागत्य प्रशंसन्ति तयोर्भृशम्}
{अयं लोकत्रयस्यास्य प्रलयोत्पत्तिकारकः}% ५२

\twolineshloka
{असावपि महाराज रामचन्द्रस्य चानुजः}
{किमिदं भविता को वा जेष्यति क्षितिमण्डले}% ५३

\twolineshloka
{पराजयं वा को वीरः प्राप्स्यते रणमूर्धनि}
{एवमेकादशाहानि वृत्तं युद्धं परस्परम्}% ५४

\twolineshloka
{द्वादशे दिवसे प्राप्ते मुमोचास्त्रं नराधिपः}
{ब्रह्मसंज्ञं महादेवं हन्तुं क्रोधसमन्वितः}% ५५

\twolineshloka
{सविज्ञाय महास्त्रं तन्मुक्तं शत्रुघ्नवैरिणा}
{हसन्नप्यपिबत्तेन मुक्तं ब्रह्मशिरो महत्}% ५६

\twolineshloka
{अत्यन्तं विस्मयं प्राप्य किं कर्तव्यमतः परम्}
{एवं विचारयुक्तस्य हृदये ज्वलनोपमम्}% ५७

\twolineshloka
{शरं वै निचखानाशु देवदेव शिरोमणिः}
{तेन बाणेन शत्रुघ्नो मूर्च्छितो रणमण्डले}% ५८

\twolineshloka
{हाहाभूतमभूत्सर्वं कटकं भटसेवितम्}
{वीराः सर्वे रुद्रगणैः पातिताः पृथिवीतले}% ५९

\twolineshloka
{सुबाहुसुमदोन्मुख्याः स्वबाहुबलदर्पिताः}
{पतितं मूर्च्छया वीक्ष्य शत्रुघ्नं शरपीडितम्}% ६०

\twolineshloka
{पुष्कलं तु रथे स्थाप्य सेवकैः परिरक्षितुम्}
{हनूमानागतो योद्धुं शिवं संहारकारकम्}% ६१

\twolineshloka
{श्रीरामस्मरणं योधान्स्वीयान्विप्र प्रहर्षितान्}
{प्रकुर्वन्रोषितस्तीव्रं लाङ्गूलं च प्रकम्पयन्}% ६२

{॥इति श्रीपद्मपुराणे पातालखण्डे शेषवात्स्यायनसंवादे रामाश्वमेधे पुष्कलशत्रुघ्नपराजयो नाम त्रिचत्वारिंशत्तमोऽध्यायः॥४३॥}

\dnsub{चतुश्चत्वारिंशत्तमोऽध्यायः}%\resetShloka

\uvacha{शेष उवाच}

\twolineshloka
{आगत्य सविधे रुद्रं समराङ्गणमूर्धनि}
{जगाद हनुमान्वीरः सञ्जिहीर्षुः सुराधिपम्}% १

\uvacha{हनूमानुवाच}

\twolineshloka
{त्वं यदाचरसे रुद्र धर्मस्य प्रतिकूलनम्}
{तस्मात्त्वां शास्तुमिच्छामि रामभक्तवधोद्यतम्}% २

\twolineshloka
{मया पुरा श्रुतं देव ऋषिभिर्बहुधोदितम्}
{रघुनाथपदस्मारी नित्यं रुद्रः पिनाकभृत्}% ३

\twolineshloka
{तत्सर्वं तु मृषा जातं शत्रुघ्नं प्रति युध्यतः}
{पुष्कलो मे हतः शूरः शत्रुघ्नोऽपि विमूर्च्छितः}% ४

\twolineshloka
{तस्मात्त्वां पातयाम्यद्य त्रैलोक्यप्रलयोद्यतम्}
{यत्नात्तिष्ठस्व भोः शर्व रामभक्तिपराङ्मुखः}% ५

\uvacha{शेष उवाच}

\twolineshloka
{इत्युक्तवन्तं प्लवगं प्रोवाच स महेश्वरः}
{धन्योऽसि वीरवर्यस्त्वं भवान्वदति नो मृषा}% ६

\twolineshloka
{मत्स्वामी रामचन्द्रोऽयं सुरासुरनमस्कृतः}
{तदश्वमानयामास शत्रुघ्नः परवीरहा}% ७

\twolineshloka
{तद्रक्षार्थं समायातस्तद्भक्त्या तु वशीकृतः}
{यथाकथञ्चिद्भक्तोऽसौ रक्ष्यः स्वात्मा इति स्थितिः}% ८

\twolineshloka
{रघुनाथः कृपां कृत्वा विलोकय तु निस्त्रपम्}
{मां स्वभक्त सुदुःखेन किञ्चित्कोपं दधन्महान्}% ९

\uvacha{शेष उवाच}

\twolineshloka
{एवं वदति चण्डीशे हनूमान्कुपितो भृशम्}
{शिलामादाय महतीं ताडयामास तद्रथम्}% १०

\twolineshloka
{शिलया ताडितस्तस्य रथः शकलतां गतः}
{ससूतः सहयः केतुपताकाभिः समन्वितः}% ११

\twolineshloka
{नभःस्था देवताः सर्वाः प्रशशंसुः कपीश्वरम्}
{धन्योसि प्लवगाधीश महत्कर्म त्वया कृतम्}% १२

\twolineshloka
{श्रीशिवं विरथं दृष्ट्वा नन्दी तं समुपाद्रवत्}
{उवाच श्रीमहादेवं मत्पृष्ठं गम्यतामिति}% १३

\twolineshloka
{वृषस्थितं तु भूतेशं हनूमान्कुपितो भृशम्}
{शिलामुत्पाट्य तरसा प्राहनद्धृदये तदा}% १४

\twolineshloka
{तदाहतो भूतपतिः शूलं तीक्ष्णं समाददे}
{जाज्वल्यमानं त्रिशिखं वह्निज्वालासमप्रभम्}% १५

\twolineshloka
{आयातं तन्महद्दृष्ट्वा शूलं प्रज्वलनप्रभम्}
{हस्ते गृहीत्वा तरसा बभञ्ज तिलशः क्षणात्}% १६

\twolineshloka
{भग्ने त्रिशूले तरसा कपीन्द्रेण क्षणाच्छिवः}
{शक्तिं करे समाधत्त सर्वलोहविनिर्मिताम्}% १७

\twolineshloka
{सा शक्तिः शिवनिर्मुक्ता हृदये तस्य धीमतः}
{लग्ना क्षणादभूत्तत्र विक्लवः प्लवगाधिपः}% १८

\twolineshloka
{क्षणाच्च तद्व्यथां तीर्त्वा गृहीत्वा वृक्षमुल्बणम्}
{ताडयामास हृदये महाव्यालविभूषिते}% १९

\twolineshloka
{ताडितास्तेन वीरेण फणीन्द्रास्त्रा समागताः}
{इतस्ततस्ते तं मुक्त्वा गताः पातालमुज्जवाः}% २०

\twolineshloka
{शिवस्तस्मिन्नगे मुक्ते वक्षसि स्वे निरीक्ष्य च}
{कुपितो व्यदधाद्घोरं मुसलं करयुग्मके}% २१

\twolineshloka
{हतोसि गच्छ सङ्ग्रामात्पलाय्य प्लवगाधम}
{एष ते प्राणहन्ताहं मुसलेन क्षणादिह}% २२

\twolineshloka
{मुसलं वीक्ष्य निर्मुक्तं शिवेन कुपितेन वै}
{कीशस्तद्वञ्चयामास महावेगाद्धरिं स्मरन्}% २३

\twolineshloka
{मुसलं तत्पपाताधः शिवमुक्तं महायसम्}
{विदार्य पृथिवीं सर्वां जगाम च रसातलम्}% २४

\twolineshloka
{तदा प्रकुपितोऽत्यतं हनूमान्रामसेवकः}
{गृहीत्वा पर्वतं हस्ते ताडयामास वक्षसि}% २५

\twolineshloka
{स यावत्पर्वतं छेत्तुं मतिं चक्रे सतीपतिः}
{तावद्धतः कपीन्द्रेण शालेन बहुशाखिना}% २६

\twolineshloka
{तमपिच्छेत्तुमुद्युक्तो यावत्तावच्छिलाहतः}
{शिलास्ता भेदितुं स्वान्तं चकार मृड उद्यतः}% २७

\twolineshloka
{तावद्वृष्टिं चकारायं शिलाभिर्नगपर्वतैः}
{लाङ्गूलेन च संवेष्ट्य ताडयत्येष भूतपम्}% २८

\twolineshloka
{शिलाभिः पर्वतैर्वृक्षैः पुच्छास्फोटेन भूरिशः}
{नन्दी प्राप्तो महात्रासं चन्द्रोऽपि शकलीकृतः}% २९

\twolineshloka
{अत्यन्तं विह्वलो जातो महेशानः प्रकोपनः}
{क्षणेक्षणे प्रहारेण विह्वलं कुर्वतं भृशम्}% ३०

\twolineshloka
{जगाद प्लवगाधीशं धन्योसि रघुपानुग}
{महत्कर्म कृतं तेऽद्य यत्तेहं सुप्रतोषितः}% ३१

\twolineshloka
{न दानेन न यज्ञेन नाल्पेन तपसा ह्यहम्}
{सुलभोऽस्मि महावेग तस्मात्प्रार्थय मे वरम्}% ३२

\uvacha{शेष उवाच}

\twolineshloka
{एवं ब्रुवन्तं तं दृष्ट्वा हनूमान्निजगाद तम्}
{प्रहसन्निर्भिया वाण्या महेशानं सुतोषितम्}% ३३

\uvacha{हनूमानुवाच}

\twolineshloka
{रघुनाथप्रसादेन सर्वं मेऽस्ति महेश्वर}
{तथापि याचे हि वरं त्वत्तः समरतोषितात्}% ३४

\twolineshloka
{एष पुष्कलसंज्ञो नः समरे पतितो हतः}
{तथैव रामावरजः शत्रुघ्नो मूर्च्छितो रणे}% ३५

\twolineshloka
{अन्ये च वीरा बहवः पतिताः शरविक्षताः}
{मूर्च्छिताः पतिताः केचित्तान्रक्षस्व गणैः सह}% ३६

\twolineshloka
{यथा चैतान्महाभूता वेतालाश्च पिशाचकाः}
{न हरन्ति न खादन्ति श्वशृगालादयस्तथा}% ३७

\twolineshloka
{एतेषां वपुषो भेदो न भवेत्त्वं तथाचर}
{यावदिन्द्रगणं जित्वा न यामि द्रोणपर्वतम्}% ३८

\twolineshloka
{तत्रस्था औषधीर्वापि नीत्वा संस्थापितान्भटान्}
{जीवयामि बलात्सर्वांस्तावत्त्वं रक्ष सर्वशः}% ३९

\twolineshloka
{एष गच्छामि तं नेतुं द्रोणं पर्वतसत्तमम्}
{यस्मिन्वसन्त्योषधयः प्राणिसञ्जीवनङ्कराः}% ४०

\twolineshloka
{एतद्वचः समाकर्ण्य तथेति निजगाद तम्}
{याहि शीघ्रं नगं तं तु रक्षामि त्वद्भटान्मृतान्}% ४१

\twolineshloka
{तच्छ्रुत्वा वाक्यमीशस्य जगाम द्रोणपर्वतम्}
{द्वीपान्सर्वानतिक्रम्य जगाम क्षीरसागरम्}% ४२

\twolineshloka
{अत्र तु स्वगणैश्चायं रक्षति स्म शिवो महान्}
{श्मशानं तद्गणैः स्वीयैर्महाबलपराक्रमैः}% ४३

\twolineshloka
{हनूमान्द्रोणमासाद्य द्रोणं नाम महागिरिम्}
{लाङ्गूले तं निधायाशु प्रतस्थे रणमण्डलम्}% ४४

\twolineshloka
{तं नेतुमुद्यते विप्र चकम्पे स च पर्वतः}
{कम्पमानं तु तं दृष्ट्वा तत्पाला देवतागणाः}% ४५

\twolineshloka
{हाहेति कृत्वा प्रोचुस्ते किमिदं भविता गिरौ}
{को ह्येनं नयते वीरो महाबलपराक्रमः}% ४६

\twolineshloka
{एवं कृत्वा सुराः सर्वे संहता ददृशुः कपिम्}
{मुञ्चैनमिति तं प्रोच्य जघ्नुः शस्त्रास्त्रकोटिभिः}% ४७

\twolineshloka
{तान्सर्वान्निघ्नतो दृष्ट्वा हनूमान्कुपितो भृशम्}
{जघान तान्क्षणाद्वीरः शक्रः सर्वासुरान्यथा}% ४८

\twolineshloka
{केचित्पदाहतास्तत्र केचित्करविमर्दिताः}
{लाङ्गूलेन हताः केचित्केचिच्छृङ्गेण चाहताः}% ४९

\twolineshloka
{सर्वे ते नाशमापन्नाः क्षणात्कीशेन ताडिताः}
{केचिन्निपतिता भूमौ रुधिरेण परिप्लुताः}% ५०

\twolineshloka
{केचित्कीशभयात्त्रस्ता जग्मुः शक्रं सुराधिपम्}
{क्षतेन च परिप्लुष्टा रुधिरक्षतदेहिनः}% ५१

\twolineshloka
{तान्दृष्ट्वा भयसंविग्नान्रुधिरेण परिप्लुतान्}
{सुराञ्जगाद विमनाः शक्रः सर्वसुरोत्तमः}% ५२

\twolineshloka
{कथं यूयं भयत्रस्ताः कथं रुधिरविप्लुताः}
{केन दैत्येन निहता राक्षसेनाधमेन वा}% ५३

\twolineshloka
{सर्वं शंसत मे तथ्यं यथा ज्ञात्वा व्रजामि तम्}
{निहत्य बद्ध्वा चायामि युष्मद्घातकमुन्मदम्}% ५४

\twolineshloka
{इति वाक्यं समाकर्ण्य तुरासाहं सुरोत्तमाः}
{जगदुर्दीनया वाचा सुरासुरनमस्कृतम्}% ५५

\uvacha{देवा ऊचुः}

\twolineshloka
{इहागत्य न जानीमः कश्चिद्वानररूपधृक्}
{नेतुं द्रोणं समुद्युक्तो लाङ्गूले वेष्ट्य तं गिरिम्}% ५६

\twolineshloka
{गन्तुं कृतमतिस्तावद्वयं सर्वे सुसंहताः}
{युद्धं चक्रुः सुसन्नद्धाः सर्वशस्त्रास्त्रवर्षिणः}% ५७

\twolineshloka
{तेन सर्वे वयं युद्धे निर्जिता बलशालिना}
{अनेके निहतास्तत्र भूमौ पेतुः सुरोत्तमाः}% ५८

\twolineshloka
{वयं तु बहुभिः पुण्यैर्जीविता इह चागताः}
{शोणितेन सुसिक्ताङ्गाः क्षतपीडासमन्विताः}% ५९

\twolineshloka
{एतद्वाक्यं समाकर्ण्य सुराणां स पुरन्दरः}
{आदिदेश सुरान्सर्वान्महाबलसमन्वितान्}% ६०

\twolineshloka
{यात महाद्रोणगिरिं कपिं बद्धुं महाबलम्}
{बद्ध्वा नयत यूयं वै सुराणां रणपातकम्}% ६१

\twolineshloka
{इत्याज्ञप्ता ययुस्ते वै द्रोणं पर्वतसत्तमम्}
{यत्रास्ते बलवान्वीरो हनूमान्कपिसत्तमः}% ६२

\twolineshloka
{गत्वा ते प्राहरन्सर्वे हनूमन्तं महाबलम्}
{हनूमता ते निहता मुष्टिभिः करताडनैः}% ६३

\twolineshloka
{पतितास्ते क्षणात्तत्र रुधिरक्षतविग्रहाः}
{अन्ये पलायनपरा जग्मुस्ते त्रिदिवेश्वरम्}% ६४

\twolineshloka
{तच्छ्रुत्वा कुपितः शक्रः सर्वानमरसत्तमान्}
{आदिदेश महावीरं वानरेन्द्रं सुरोत्तमः}% ६५

\twolineshloka
{तदाज्ञप्ता ययुस्ते वै यत्र कीशेश्वरो बली}
{तान्सर्वानागतान्दृष्ट्वा जगाद कपिसत्तमः}% ६६

\twolineshloka
{मायां तु वीराः समरे संहर्तारं हि मां बलात्}
{नेष्यामि युष्मानधुना संयमिन्याः पुरोऽन्तिके}% ६७

\twolineshloka
{इत्युक्ता अपि ते सर्वे सन्नद्धाः प्राहरन्कपिम्}
{शस्त्रास्त्रैर्बहुधा मुक्तैर्महाबलसमन्विताः}% ६८

\twolineshloka
{केचिच्छूलैः परशुभिः केचित्खड्गैश्च पट्टिशैः}
{मुसलैः शक्तिभिः केचित्क्रोधेन कलुषीकृताः}% ६९

\twolineshloka
{स आहतोऽमरवरैर्विविधैरायुधैर्बली}
{शिलाभिस्ताञ्जघानाशु सर्वानमरसत्तमान्}% ७०

\twolineshloka
{केचित्पलाय्य आहुस्ते गताः शक्रसमीपकम्}
{तदुक्तं वाक्यमाकर्ण्य भयं प्राप सुराधिपः}% ७१

\twolineshloka
{बृहस्पतिं सुराध्यक्षं मन्त्रिणं स्वर्गवासिनाम्}
{पप्रच्छ सविधे गत्वा नत्वा सुरगुरुं वरम्}% ७२

\uvacha{इन्द्र उवाच}

\twolineshloka
{कोऽसौ यो वानरो द्रोणं नेतुं स्वामिन्समागतः}
{येन मे निहता वीरा अमराः शस्त्रधारिणः}% ७३

\uvacha{शेष उवाच}

\twolineshloka
{एतच्छ्रुत्वा तु तद्वाक्यमुक्तमाङ्गिरसो महान्}
{जगाद भयसंविग्नं तुरासाहं सुराधिपम्}% ७४

\uvacha{बृहस्पतिरुवाच}

\twolineshloka
{यो रावणमहन्सङ्ख्ये कुम्भकर्णमदीदहत्}
{येन ते वैरिणः सर्वे हतास्तस्यैव सेवकः}% ७५

\twolineshloka
{येन लङ्का सत्रिकूटा निर्दग्धा पुच्छवह्निना}
{अक्षश्च निहतो येन हनूमन्तमवेहि तम्}% ७६

\twolineshloka
{तेन सर्वे विनिहता द्रोणार्थमयमुद्यतः}
{हयमेधं महाराजः करोति बलिसत्तमः}% ७७

\twolineshloka
{तस्याश्वं शिवभक्तस्तु नृपो वीरमणिर्महान्}
{जहार तत्र समभूद्रणं सुरविमोहनम्}% ७८

\twolineshloka
{शिवेन निहताः सङ्ख्ये वीरा रामस्य भूरिशः}
{तान्वै जीवयितुं द्रोणं नेष्यत्येव महाबलः}% ७९

\twolineshloka
{नायं वर्षशतैर्जेयो भवता बलसंयुतः}
{तस्मात्प्रसादय कपिं देहि तत्रत्यमौषधम्}% ८०

{॥इति श्रीपद्मपुराणे पातालखण्डे शेषवात्स्यायनसंवादे रामाश्वमेधे द्रोणगिरौ देवानां पराजयो नाम चतुश्चत्वारिंशत्तमोऽध्यायः॥४४॥}

\dnsub{पञ्चचत्वारिंशत्तमोऽध्यायः}%\resetShloka

\uvacha{शेष उवाच}

\twolineshloka
{गुरुभाषितमाकर्ण्य वृषपर्वरिपुः स्वराट्}
{ज्ञात्वा रामस्य कार्यार्थमागतं पवनात्मजम्}% १

\twolineshloka
{भयं तत्याज मनसि वानरात्समुपस्थितम्}
{जहर्ष चित्ते च भृशं वाचस्पतिमुवाच ह}% २

\uvacha{इन्द्र उवाच}

\twolineshloka
{कथं कार्यं सुराधीश द्रोणोऽयं नीयते यदि}
{देवानां जीवनं भूयः कथं स्यादिति मे वद}% ३

\twolineshloka
{इदानीं पवनोद्भूतं प्रसादय यथातथम्}
{रामः प्रीतिं परां याति देवानां च सुखं भवेत्}% ४

\twolineshloka
{देवाधिपस्य वचनं श्रुत्वा वाचस्पतिस्तदा}
{शक्रं तु पुरतः कृत्वा सर्वदेवैः परीवृतम्}% ५

\twolineshloka
{जगाम तत्र यत्रास्ते हनूमान्निर्भयः कपिः}
{गर्जति प्रसभं जित्वा सुरान्सर्वान्सुखासिनः}% ६

\twolineshloka
{ते गत्वा सन्निधौ तस्य बृहस्पतिपुरोगमाः}
{पेतुस्ते चरणौ नत्वा समीरतनुजस्य हि}% ७

\twolineshloka
{बृहस्पतिश्च तं वीरं जगाद प्रेरितोऽमुना}
{सुराधीशेन लोकस्य गुरुणा वदतां वरः}% ८

\twolineshloka
{अजानद्भिः कृतं कर्म देवैस्तव पराक्रमम्}
{श्रीरामचरणस्य त्वं सेवकोऽसि महामते}% ९

\twolineshloka
{किमर्थमयमारम्भः कथमत्र समागमः}
{तत्करिष्यामहे सर्वे सन्नतास्तव भाषितम्}% १०

\twolineshloka
{रोषं त्यक्त्वा कृपां कृत्वा देवाधीशं विलोकय}
{पवनात्मज दैत्यानां भयङ्करवपुर्दधत्}% ११

\uvacha{शेष उवाच}

\twolineshloka
{इत्थं भाषितमाकर्ण्य देवानां स गुरोर्वचः}
{उवाच देवान्सकलान्गुरुं चैव महयशाः}% १२

\twolineshloka
{राज्ञो वीरमणेः सङ्ख्ये हताः शर्वेण भूरिशः}
{भटास्तान्वै जीवयितुं द्रोणं नेष्यामि पर्वतम्}% १३

\twolineshloka
{तं ये निवारयिष्यन्ति स्ववीर्यबलदर्पिताः}
{तान्नेष्यामि क्षणादेव यमस्य सदनं प्रति}% १४

\twolineshloka
{तस्माद्वदत मे यूयं द्रोणं वाथ तदौषधम्}
{येन सञ्जीवयिष्यामि मृतान्वीरान्रणाङ्गणे}% १५

\uvacha{शेष उवाच}

\twolineshloka
{इति वाक्यं समाकर्ण्य वायुसूनोर्महात्मनः}
{ते सर्वे प्रणतिं गत्वा ददुः सञ्जीवनौषधम्}% १६

\twolineshloka
{ते प्रहृष्टा भयं त्यक्त्वा सुराः स्वर्गौकसः स्वयम्}
{ययुः सुरपतिं कृत्वा पुरः सौख्य समन्विताः}% १७

\twolineshloka
{हनुमान्भेषजं तत्तु समादायागतो रणम्}
{स्तुतः सर्वैः सुरगणैर्महाकर्मसमुत्सुकैः}% १८

\twolineshloka
{तमागतं हनूमन्तं वीक्ष्य सर्वेऽपि वैरिणः}
{साधुसाधुप्रशंसन्तमद्भुतं मेनिरे कपिम्}% १९

\fourlineindentedshloka
{कपिः समागत्य महामुदायुतः}
{पुरो भटं पुष्कलमागतं मृतम्}
{शिवेन संरक्षितमुग्रमण्डले}
{श्रीरामचित्तं सविधे जगाम ह}% २०

\twolineshloka
{सुमतिं च समाहूय मन्त्रिणं महतां मतम्}
{उवाच जीवयाम्यद्य सर्वान्वीरान्रणे मृतान्}% २१

\twolineshloka
{एवमुक्त्वा भेषजं तत्पुष्कलस्य महोरसि}
{शिरः कायेन सन्धाय जगाद वचनं शुभम्}% २२

\twolineshloka
{यद्यहं मनसा वाचा कर्मणा राघवं पतिम्}
{जानामि तर्हि एतेन भेषजेनाशु जीवतु}% २३

\twolineshloka
{इति वाक्यं यदा वक्ति तावत्पुष्कल उत्थितः}
{रणाङ्गणेऽदशद्रोषाद्दन्तान्वीरशिरोमणिः}% २४

\twolineshloka
{क्व गतो वीरभद्रोऽसौ मां सम्मूर्च्छ्य रणाङ्गणे}
{सद्योऽहं पातयाम्येनं क्वास्ति मे धनुरुत्तमम्}% २५

\twolineshloka
{इति तं भाषमाणं वै प्राह वीरं कपीन्द्रकः}
{धन्योऽसि वीर यद्भूयो वदस्येनं रणाङ्गणे}% २६

\twolineshloka
{त्वं हतो वीरभद्रेण रघुनाथप्रसादतः}
{पुनः सञ्जीवितोऽस्येहि शत्रुघ्नं याम मूर्च्छितम्}% २७

\twolineshloka
{इत्युक्त्वा प्रययौ तत्र सङ्ग्रामवरमूर्धनि}
{श्वसन्नास्ते स शत्रुघ्नः शिवबाणप्रपीडितः}% २८

\twolineshloka
{तत्र गत्वा समीपं तच्छत्रुघ्नस्य महात्मनः}
{निधाय भेषजं तस्य वक्षसि श्वासमागते}% २९

\twolineshloka
{उवाच हनुमांस्तं वै जीव शत्रुघ्नसत्तम}
{मूर्च्छितोऽसि रणे कस्मान्महाबलपराक्रम}% ३०

\twolineshloka
{यद्यहं ब्रह्मचर्यं च जन्मपर्यन्तमुद्यतः}
{पालयामि तदा वीरः शत्रुघ्नो जीवतु क्षणात्}% ३१

\twolineshloka
{उक्तमात्रेण तेनेदं जीवितः क्षणमात्रतः}
{क्व शिवः क्व शिवो यातो विहायरणमण्डलम्}% ३२

\twolineshloka
{अनेके निहताः सङ्ख्ये श्रीरुद्रेण पिनाकिना}
{ते सर्वे जीविता वीराः कपीन्द्रेण महात्मना}% ३३

\twolineshloka
{तदा सर्वे सुसन्नद्धा रोषपूरितमानसाः}
{स्वेस्वे रथे स्थिताः शत्रून्प्रययुः क्षतविग्रहाः}% ३४

\twolineshloka
{पुष्कलो वीरभद्रं तु चण्डं चैव कुशध्वजः}
{नन्दिनं हनुमान्वीरः शत्रुघ्नः सङ्गरे शिवम्}% ३५

\twolineshloka
{धनुर्विस्फारयन्तं तं शत्रुघ्नं बलिनां वरम्}
{सङ्ग्रामे शिवमाहूय तिष्ठन्तं प्रययौ नृपः}% ३६

\twolineshloka
{राजा वीरमणिर्वीरः शत्रुघ्नः समरे बली}
{अन्योन्यं चक्रतुर्युद्धं मुनिविस्मयकारकम्}% ३७

\twolineshloka
{राज्ञा च वीरमणिना रथा भग्नाः शताधिकाः}
{शत्रुघ्नस्य नरेन्द्रस्य तिलशः क्षणतो द्विज}% ३८

\twolineshloka
{तदा प्रकुपितोऽत्यन्तं शत्रुघ्नो रणमण्डले}
{आग्नेयास्त्रं मुमोचामुं दग्धुं सैन्यसमन्वितम्}% ३९

\twolineshloka
{दाहकं तन्महद्दृष्ट्वा महास्त्रं शत्रुमोचितम्}
{अत्यन्तं कुपितो राजा वारुणास्त्रं समाददे}% ४०

\twolineshloka
{वारुणास्त्रेण शीतार्तं वीक्ष्य रामानुजो बली}
{वायव्यास्त्रं मुमोचास्मै तेन वायुर्महानभूत्}% ४१

\twolineshloka
{वायुना संहता मेघा ययुस्ते सर्वतोदिशम्}
{इतस्ततो गताः सर्वे सैन्यं तत्सुखितं बभौ}% ४२

\twolineshloka
{सैन्ये पवनपीडार्ते नृपो वीरमतिर्महान्}
{पर्वतास्त्रं रिपूद्धारि जग्राह च शरासने}% ४३

\twolineshloka
{पर्वतैः स्तम्भितो वायुर्न चासर्पत सङ्गरे}
{तद्वीक्ष्य रामावरजो वज्रास्त्रं तु समाददे}% ४४

\twolineshloka
{वज्रास्त्रेण हताः सर्वे नगास्तु तिलशः कृताः}
{चूर्णतां प्रापुरेतस्मिन्रणे वीरवरार्चिते}% ४५

\twolineshloka
{वज्रास्त्रेण विदीर्णाङ्गा वीराः शोणितशोभिताः}
{बभूवुः समरप्रान्ते चित्रं समभवद्रणम्}% ४६

\twolineshloka
{तदा प्रकुपितोऽत्यन्तं राजा वीरमणिर्महान्}
{ब्रह्मास्त्रं चाप आधत्त वैरिदाहकमद्भुतम्}% ४७

\twolineshloka
{शत्रुघ्नः शरमादाय सस्मार सुमनोहरम्}
{अस्त्रं तद्योगिनीदत्तं सर्ववैरिविमोहनम्}% ४८

\twolineshloka
{ब्रह्मास्त्रं तत्करभ्रष्टमागतं वैरिणं प्रति}
{तावच्छत्रुघ्ननाम्ना तु तन्मुक्तं मोहनास्त्रकम्}% ४९

\twolineshloka
{मोहनास्त्रेण तद्ब्राह्मं द्विधाछिन्नं क्षणादिह}
{लग्नं राज्ञो हृदि क्षिप्रं मूर्च्छां सम्प्रापयन्नृपम्}% ५०

\twolineshloka
{ते बाणाः शतशो मुक्ताः शत्रुघ्नेन महीभृता}
{सर्वेपि मूर्च्छिता वीरा गणा रुद्रस्य ये पुनः}% ५१

\twolineshloka
{शिवस्य चरणोपस्थे मूढाः पेतुर्महीतले}
{तदा शिवः प्रकुपितो रथे तिष्ठन्ययौ नृपम्}% ५२

\twolineshloka
{शिवेन सहसा योद्धुं समायातो रणाङ्गणे}
{शत्रुघ्नः सज्जमात्तज्यं धनुः कृत्वा व्ययुद्ध्यत}% ५३

\twolineshloka
{तयोः समभवद्युद्धं घोरं वैरिविदारणम्}
{शस्त्रास्त्रैर्बहुधामुक्तैरादीपित दिगन्तरम्}% ५४

\twolineshloka
{अस्त्रप्रत्यस्त्रसङ्घातैस्ताडनप्रतिताडनैः}
{देवानामपि दैत्यानां नैतादृग्रणमण्डलम्}% ५५

\twolineshloka
{तदा व्याकुलितोऽत्यन्तं शत्रुघ्नः शिवसङ्गरे}
{सस्मार स्वामिनं तत्र पावनेरुपदेशतः}% ५६

\twolineshloka
{हा नाथ भ्रातरत्युग्रः शिवः प्राणापहारणम्}
{करोति धनुरुद्यम्य त्रायस्व रणमण्डले}% ५७

\twolineshloka
{अनेके दुःखपाथोधिं तीर्णा राम तवाख्यया}
{मामप्युद्धर दुःखस्थं रामराम कृपानिधे}% ५८

\twolineshloka
{इत्थं वक्ति यदा तावद्वीक्षितो रणमण्डले}
{नीलोत्पलदलश्यामो रामो राजीवलोचनः}% ५९

\twolineshloka
{मृगशृङ्गं कटौ धृत्वा दीक्षितं वपुरुद्वहन्}
{तं दृष्ट्वा विस्मयं प्राप शत्रुघ्नः समराङ्गणे}% ६०

{॥इति श्रीपद्मपुराणे पातालखण्डे शेषवात्स्यायनसंवादे रामाश्वमेधे श्रीरामसमागमो नाम पञ्चचत्वारिंशत्तमोऽध्यायः॥४५॥}

\dnsub{षट्चत्वारिंशत्तमोऽध्यायः}%\resetShloka

\uvacha{शेष उवाच}

\twolineshloka
{आगतं वीक्ष्य श्रीरामं शत्रुघ्नः प्रणतार्तिहम्}
{भ्रातरं सकलाद्दुःखान्मुक्तोऽभूद्द्विजसत्तम}% १

\twolineshloka
{हनूमान्वीक्ष्य विभ्रान्तो रामस्य चरणौ मुदा}
{ववन्दे भक्तरक्षार्थमागतं निजगाद च}% २

\twolineshloka
{स्वामिंस्तवैतद्युक्तं तु स्वभक्तपरिपालनम्}
{यत्सङ्ग्रामे जितं सर्वं पाशबद्धममोचयः}% ३

\twolineshloka
{वयं त्विदानीं धन्या वै यद्द्रक्ष्यामो भवत्पदे}
{जेष्यामोऽरीन्क्षणादेव त्वत्कृपातो रघूद्वह}% ४

\uvacha{शेष उवाच}

\twolineshloka
{स्थाणुस्तदागतं रामं योगिनां ध्यानगोचरम्}
{पतित्वा पादयोर्विप्र जगाद प्रणताभयम्}% ५

\twolineshloka
{एकस्त्वं पुरुषः साक्षात्प्रकृतेः पर ईर्यसे}
{यः स्वांशकलया विश्वं सृजस्यवसि हंसि च}% ६

\twolineshloka
{अरूपस्त्वमशेषस्य जगतः कारणं परम्}
{एक एव त्रिधारूपं गृह्णासि कुहकान्वितः}% ७

\twolineshloka
{सृष्टौ विधातृरूपेण पालने स्वयमास च}
{प्रलये जगतः साक्षादहं शर्वाख्यतां गतः}% ८

\twolineshloka
{तव यत्परमेशस्य हयमेधक्रतुक्रिया}
{ब्रह्महत्यापनोदाय तद्विडम्बनमद्भुतम्}% ९

\twolineshloka
{यत्पादशौचममलं गङ्गाख्यं शिरसोऽन्तरा}
{वहामि पापशान्त्यर्थं तस्य ते पातकं कुतः}% १०

\twolineshloka
{मया बह्वपकाराय कृतं कर्म तव स्फुटम्}
{क्षम्यतां तत्कृपालो हि भवतो व्यवधायकम्}% ११

\twolineshloka
{किं करोमि मया सत्यपालनार्थमिदं कृतम्}
{जानन्प्रभावं भवतो भक्तरक्षार्थमागतः}% १२

\twolineshloka
{असौ पुरा उज्जयिन्यां महाकालनिकेतने}
{स्नात्वा शिप्राख्य सरिति तपस्तेपे महाद्भुतम्}% १३

\twolineshloka
{ततः प्रसन्नो जातोऽहं जगाद भूमिपं प्रति}
{याचस्वेति महाराज स वव्रे राज्यमद्भुतम्}% १४

\twolineshloka
{मया प्रोक्तं देवपुरे तव राज्यं भविष्यति}
{यावद्रामहयः पुर्यामागमिष्यति याज्ञिकः}% १५

\twolineshloka
{तावत्प्रभृत्यहं स्थास्ये तव रक्षार्थमुद्यतः}
{एतद्दत्तवरो राम किं करोमि स्वसत्यतः}% १६

\twolineshloka
{घृणितोऽस्म्यधुना राज्ञा सपुत्रपशुबान्धवः}
{हयं समर्प्यते पादसेवां राजा विधास्यति}% १७

\uvacha{शेष उवाच}

\twolineshloka
{इति वाक्यं समाकर्ण्य महेशस्य रघूत्तमः}
{उवाच धीरया वाचा कृपया पूर्णलोचनः}% १८

\uvacha{राम उवाच}

\twolineshloka
{देवानामयमेवास्ति धर्मो भक्तस्य पालनम्}
{त्वया साधुकृतं कर्म यद्भक्तो रक्षितोऽधुना}% १९

\twolineshloka
{ममासि हृदये शर्व भवतो हृदये त्वहम्}
{आवयोरन्तरं नास्ति मूढाः पश्यन्ति दुर्धियः}% २०

\twolineshloka
{ये भेदं विदधत्यद्धा आवयोरेकरूपयोः}
{कुम्भीपाकेषु पच्यन्ते नराः कल्पसहस्रकम्}% २१

\twolineshloka
{ये त्वद्भक्तास्त एवासन्मद्भक्ता धर्मसंयुताः}
{मद्भक्ता अपि भूयस्या भक्त्या तव नतिङ्कराः}% २२

\uvacha{शेष उवाच}

\twolineshloka
{इत्थं भाषितमाकर्ण्य शर्वो वीरमणिं नृपम्}
{मूर्च्छितं जीवयामास करस्पर्शादिना प्रभुः}% २३

\twolineshloka
{अन्यानपि सुतानस्य मूर्च्छिताञ्छरपीडितान्}
{जीवयामास स मृडः समर्थः प्रभुरीश्वरः}% २४

\twolineshloka
{सज्जं विधाय तं भूपं श्रीरामपदयोर्नतिम्}
{कारयामास भूतेशः पुत्रपौत्रैः परीवृतम्}% २५

\twolineshloka
{धन्यो राजा वीरमणिर्यो ददर्श रघूत्तमम्}
{योगिभिर्योगनिष्ठाभिर्दुष्प्रापमयुतायुतैः}% २६

\twolineshloka
{ते नत्वा रघुनाथं तं कृतार्थी कृतविग्रहाः}
{ब्रह्मादिभिः पूज्यतमा अभूवन्द्विजसत्तम}% २७

\twolineshloka
{शत्रुघ्न हनुमद्भ्यां च पुष्कलादिभिरुद्भटैः}
{परिष्टुताय रामाय ददौ राजा हयोत्तमम्}% २८

\twolineshloka
{राज्येन सहितं सर्वं सपुत्रपशुबान्धवम्}
{शर्वेण प्रेरितः प्रादाद्भूपो वीरमणिस्तदा}% २९

\twolineshloka
{ततो रामो नुतः सर्वैर्वैरिभिर्निजसेवकैः}
{शत्रुघ्नादिभिरत्यन्तमुत्सुकैश्च विशेषतः}% ३०

\twolineshloka
{रथे मणिमये तिष्ठन्बभूव स तिरोहितः}
{अन्तर्हिते रामभद्रे सर्वे प्रापुः सुविस्मयम्}% ३१

\twolineshloka
{मा जानीहि मनुष्यं तं रामं लोकैकवन्दितम्}
{जले स्थले च सर्वत्र वर्तते संस्थितः सदा}% ३२

\twolineshloka
{ततो वीरा अलं हृष्टा अन्योन्यं परिरेभिरे}
{तूर्यमङ्गलवादित्रैः सुमहानुत्सवोऽभवत्}% ३३

\twolineshloka
{ततो मुक्तो हयः सर्वैर्वीरैः शस्त्रास्त्रकोविदैः}
{सर्वैरनुगतः प्रीतैर्विस्मयेन समन्वितैः}% ३४

\twolineshloka
{शर्वः सत्यप्रतिज्ञश्च तमनुज्ञाप्य सेवकम्}
{श्रीरामं शरणं प्रोच्य याहि लोकैकदुर्ल्लभम्}% ३५

\twolineshloka
{स्वयमन्तर्हितस्तत्र प्रलयोत्पत्तिकारकः}
{कैलासमगमच्छर्वः सेवकैः परिशोभितः}% ३६

\twolineshloka
{भूपो वीरमणिर्ध्यायञ्छ्रीरामचरणोदजम्}
{शत्रुघ्नेन ययौ साकं बलिना बलसंयुतः}% ३७

\twolineshloka
{एतद्रामस्य चरितं ये शृण्वन्ति नरोत्तमाः}
{तेषां संसारजं दुःखं न भविष्यति कर्हिचित्}% ३८

{॥इति श्रीपद्मपुराणे पातालखण्डे शेषवात्स्यायनसंवादे रामाश्वमेधे हयप्रस्थानं नाम षट्चत्वारिंशत्तमोऽध्यायः॥४६॥}

\dnsub{सप्तचत्वारिंशत्तमोऽध्यायः}%\resetShloka

\uvacha{शेष उवाच}

\twolineshloka
{हयो गतो हेमकूटं भारतान्ते ततो द्विज}
{अनेकभटसाहस्रै रक्षितो बद्धचामरः}% १

\twolineshloka
{यो वै विस्तरतो दैर्घ्याद्योजनानां समं ततः}
{अयुतेन सुशृङ्गैश्च राजतैः काञ्चनादिभिः}% २

\twolineshloka
{तत्रोद्यानं महच्छ्रेष्ठं पादपैः परिशोभितम्}
{शालैस्तालैस्तमालैश्च कर्णिकारैः समन्ततः}% ३

\twolineshloka
{हिन्तालैर्नागपुन्नागैः कोविदारैः सबिल्वकैः}
{चम्पकैर्बकुलैर्मेघैर्मदनैः कुटजादिभिः}% ४

\twolineshloka
{जातिकाभिर्यूथिकाभिर्नवमालिकया तथा}
{आम्रैर्माधवद्राक्षाभिर्दाडिमैः शोभितं वनम्}% ५

\twolineshloka
{अनेकपक्षिसङ्घुष्टं भ्रमरैर्निनदीकृतम्}
{मयूरकेकारवितं सर्वर्तुसुखदं हयः}% ६

\twolineshloka
{प्रविवेश स शत्रुघ्नो मनोवेगसमन्वितः}
{स्वर्णपत्रं विशाले स्वे भाले बिभ्रन्मनोहरम्}% ७

\twolineshloka
{गच्छतस्तस्य वाहस्य हयमेधक्रतोस्तदा}
{अकस्मादभवच्चित्रं तच्छृणुष्व द्विजोत्तम}% ८

\twolineshloka
{गात्रस्तम्भोऽभवत्तस्य न चचाल पथिस्थितः}
{हेमकूटइवाचाल्यो बभूव हयसत्तमः}% ९

\twolineshloka
{तदा तद्रक्षकाः सर्वे कशाघातान्वितेनिरे}
{तदाहतेऽपि न ययौ स्तब्धगात्रो हयोत्तमः}% १०

\twolineshloka
{शत्रुघ्नं सविधे गत्वा चुक्रुशुर्वाहरक्षकाः}
{स्वामिन्वयं न जानीमः किमभूद्धयसत्तमे}% ११

\twolineshloka
{गच्छतो वाहवर्यस्य मनोवेगस्य भूपते}
{आकस्मिकोऽभवत्तस्य गात्रस्तम्भो महामते}% १२

\twolineshloka
{कशाभिस्ताडितोऽस्माभिः परं तत्र चचाल न}
{एवं विचार्य यत्कर्म तत्कुरुष्व नृपोत्तम}% १३

\twolineshloka
{तदा विस्मयमापन्नो भूपतिः सह सैनिकैः}
{जगाम सहितः सर्वैर्हयस्य महतोऽन्तिके}% १४

\twolineshloka
{पुष्कलो बाहुना धृत्वा चरणौ तस्य भूतलात्}
{उत्पाटयामास तदा परं नो चेलतुस्ततः}% १५

\twolineshloka
{बलेन बलिनाक्रान्तो नाकम्पत हयस्तदा}
{हनूमांस्तं समुद्धर्तुं मतिं चक्रे महामनाः}% १६

\twolineshloka
{लाङ्गूलेन समावेष्ट्य बलेन बलिनां वरः}
{आचकर्ष बलाद्वाहं न चचाल तथापि सः}% १७

\twolineshloka
{तदोवाच कपिश्रेष्ठो हनूमान्विस्मयान्वितः}
{शत्रुघ्नं बलिनां श्रेष्ठं वीराणां परिशृण्वताम्}% १८

\twolineshloka
{मया द्रोणो लाङ्गुलेन लीलयोत्पाटितोऽधुना}
{परमत्र महाश्चर्यं कम्पते न हयोऽल्पकः}% १९

\twolineshloka
{दृष्टमत्र निदानं हि वीरैर्बलिभिरुद्धतैः}
{आकृष्टोऽपि न च स्थानाच्चचाल तिलमात्रतः}% २०

\twolineshloka
{कपिभाषितमाकर्ण्य शत्रुघ्नो विस्मयान्वितः}
{सुमतिं मन्त्रिणां श्रेष्ठमुवाच वदतां वरः}% २१

\uvacha{शत्रुघ्न उवाच}

\twolineshloka
{मन्त्रिन्किमभवद्वाहे स्तम्भनं वपुषोऽनघ}
{कोऽत्रोपायो विधेयः स्याद्येन वाहगतिर्भवेत्}% २२

\uvacha{सुमतिरुवाच}

\twolineshloka
{स्वामिन्कश्चिन्मुनिर्मृग्योऽखिलज्ञानविचक्षणः}
{देशोद्भवमहं जाने प्रत्यक्षं न परोक्षजम्}% २३

\uvacha{शेष उवाच}

\twolineshloka
{इति वाक्यं समाकर्ण्य सुमतेर्धर्मकोविदः}
{अन्वेषयामास मुनिं सेवकैः सह शोभनम्}% २४

\twolineshloka
{ते सर्वे सर्वतो गत्वा मुनिं धर्मविदं भटाः}
{व्यालोकयन्तः सर्वत्र न चापश्यन्मुनीश्वरम्}% २५

\twolineshloka
{एकस्त्वनुचरो विप्र गतो योजनमात्रतः}
{पूर्वस्यां दिशि चोद्युक्तः पश्यति स्म महाश्रमम्}% २६

\twolineshloka
{यत्र निर्वैरिणः सर्वे पशवो जनतास्तथा}
{गङ्गास्नानहताशेषकिल्बिषाः सुमनोहराः}% २७

\twolineshloka
{यत्र केचित्तपः श्रेष्ठं कुर्वन्ति स्म हुताशनैः}
{धूमैरधोमुखाः पत्रैर्वायुभिः स्वोदरम्भराः}% २८

\twolineshloka
{यत्राग्निहोत्रजो धूमः पवित्रयति सर्वदा}
{अनेकमुनिसंहृष्टो मुक्तपत्रलतोत्तमः}% २९

\twolineshloka
{तमाश्रमं मुनेर्ज्ञात्वा शौनकस्य मनोहरम्}
{न्यवेदयन्नृपायासौ विस्मयाविष्टचेतसे}% ३०

\twolineshloka
{तच्छ्रुत्वा हर्षितोऽत्यन्तं शत्रुघ्नः सह सेवकैः}
{हनूमत्पुष्कलाद्यैश्च सयुतोऽगात्तदाश्रमम्}% ३१

\twolineshloka
{तत्र वीक्ष्य मुनिश्रेष्ठं सम्यग्घुतहुताशनम्}
{प्रणम्य दण्डवत्तस्य चरणौ पापहारिणौ}% ३२

\twolineshloka
{तमागतं नृपं ज्ञात्वा शत्रुघ्नं बलिनां वरम्}
{अर्घ्यपाद्यादिकं चक्रे प्रीतस्तद्दर्शनादभूत्}% ३३

\twolineshloka
{सुखोपविष्टं विश्रान्तं नृपं प्राह मुनीश्वरः}
{किमर्थमटनं देव महत्पर्यटनं तव}% ३४

\twolineshloka
{त्वादृशाः पृथिवीं सर्वां नृपा वै न भ्रमन्ति चेत्}
{तदा दुष्टजनाः साधून्बाधन्ते विगतज्वरान्}% ३५

\twolineshloka
{कथयस्व महीपाल शत्रुघ्न बलिनां वर}
{सर्वं शुभायनो भूयात्तव पर्यटनादिकम्}% ३६

\uvacha{शेष उवाच}

\twolineshloka
{इत्युक्तवन्तं भूदेवं प्रत्युवाच महीश्वरः}
{गद्गद स्वरया वाण्या हर्षित स्वीयविग्रहः}% ३७

\uvacha{शत्रुघ्न उवाच}

\twolineshloka
{अकस्मादभवच्चित्रं रामाश्वस्य मनोहृतः}
{नातिदूरे त्वदावासात्तच्छृणुष्व विदांवर}% ३८

\twolineshloka
{उद्याने तव शोभाढ्ये यदृच्छातो हयो गतः}
{तत्प्रान्ते तस्य वाहस्य गात्रस्तम्भोऽभवत्क्षणात्}% ३९

\twolineshloka
{तदा मे बलिनो वीराः पुष्कलाद्या मदोत्कटाः}
{बलादाचकृषुर्वाहं न चचाल तथाप्यसौ}% ४०

\twolineshloka
{अस्मानपारदुःखाब्धौ मग्नान्प्रतितरिः स्मृतः}
{दैवाद्दृष्टः सुभाग्यैस्त्वं कथयस्व निदानकम्}% ४१

\uvacha{शेष उवाच}

\twolineshloka
{एवं पृष्टो मुनिवरः क्षणं दध्यौ महामतिः}
{ततः कारणसंयुक्तं विचारेण दधद्धयम्}% ४२

\twolineshloka
{क्षणात्तज्ज्ञानतां प्राप्य विस्मयोत्फुल्ललोचनः}
{जगाद स महीपालं दुःखितं संशयान्वितम्}% ४३

\uvacha{शौनक उवाच}

\twolineshloka
{शृणु राजन्प्रवक्ष्यामि हयस्तम्भस्य कारणम्}
{यच्छ्रुत्वा मुच्यते दुःखादतिचित्रकथानकम्}% ४४

\twolineshloka
{गौडदेशे महारण्ये कावेरीतीरभूषिते}
{वाडवः सात्वको नाम्ना चचार परमं तपः}% ४५

\twolineshloka
{एकाहं पयसः प्राशी दिनैकं वायुभक्षकः}
{दिनैकं तु निराहार एवं त्रिदिनमुन्नयेत्}% ४६

\twolineshloka
{एवं व्रते प्रवृत्तस्य कालः सर्वक्षयङ्करः}
{जग्राह स्वस्य दंष्ट्रायां मृतिं प्राप महाव्रती}% ४७

\twolineshloka
{विमाने सर्वशोभाढ्ये सर्वरत्नविभूषिते}
{अप्सरोभिः सह क्रीडन्ययौ मेरोः शिखास्थितौ}% ४८

\twolineshloka
{जम्बूनाममहावृक्षस्तत्र सेव्यरसोऽभवत्}
{नदी जाम्बवती संज्ञा स्वर्णद्रवसमन्विता}% ४९

\twolineshloka
{तस्यां मुनयइच्छाभिः क्रीडन्ते कुतुकान्विताः}
{अनेकतपसा पुण्याः सर्वसौख्यसमन्विताः}% ५०

\twolineshloka
{तत्रासौ स्वेच्छया क्रीडन्नप्सरोभिर्मुदान्वितः}
{प्रतीपमाचरत्तेषां स्वाभिमानमदोद्धतः}% ५१

\twolineshloka
{ततः शप्तः स मुनिभी राक्षसो भव दुर्मुखः}
{ततोऽतिदुःखितः प्राह मुनीन्विद्यातपोधनान्}% ५२

\twolineshloka
{अनुगृह्णन्तु मां सर्वे विप्रा यूयं कृपालवः}
{तदा तैरनुगृहीतो यदा रामहयं भवान्}% ५३

\twolineshloka
{स्तम्भयिष्यति वेगेन ततो रामकथाश्रुतिः}
{पश्चान्मुक्तिर्भवित्री ते शापादस्मात्सुदारुणात्}% ५४

\twolineshloka
{स प्रोक्तो मुनिभिर्देवो राक्षसत्वमितः प्रभो}
{स्तम्भयामास रामाश्वं मोचयानघकीर्तनैः}% ५५

{॥इति श्रीपद्मपुराणे पातालखण्डे शेषवात्स्यायनसंवादे रामाश्वमेधे शापकीर्तनं नाम सप्तचत्वारिंशत्तमोऽध्यायः॥४७॥}

\dnsub{अष्टचत्वारिंशत्तमोऽध्यायः}%\resetShloka

\uvacha{शेष उवाच}

\twolineshloka
{इति प्रोक्तं तु मुनिना संश्रुत्य परवीरहा}
{विस्मयं मानयामास हृदि शौनकमब्रवीत्}% १

\uvacha{शत्रुघ्न उवाच}

\twolineshloka
{कर्मणो गहना वार्ता यया सात्वकनामधृत्}
{दिवं प्राप्तोऽपि महता कर्मणा राक्षसीकृतः}% २

\twolineshloka
{स्वामिन्वद महर्षे त्वं कर्मणां स्वगतिर्यथा}
{येन कर्मविपाकेन यादृशं नरकं भवेत्}% ३

\uvacha{शौनक उवाच}

\twolineshloka
{धन्योसि राघवश्रेष्ठ यत्ते मतिरियं शुभा}
{जानन्नपि हितार्थाय लोकानां त्वं ब्रवीषि भोः}% ४

\twolineshloka
{कथयामि विचित्राणां कर्मणां विविधा गतीः}
{ताः शृणुष्व महाराज यच्छ्रुत्वा मोक्षमाप्नुयात्}% ५

\twolineshloka
{परवित्तं परापत्यं कलत्रं पारकं च यः}
{बलात्कारेण गृह्णाति भोगबुद्ध्या च दुर्मतिः}% ६

\twolineshloka
{कालपाशेन सम्बद्धो यमदूतैर्महाबलैः}
{तामिस्रे पात्यते तावद्यावद्वर्षसहस्रकम्}% ७

\twolineshloka
{तत्र ताडनमुद्धूताः कुर्वन्ति यमकिङ्कराः}
{पापभोगेन सन्तप्तस्ततो योनिं तु शौकरीम्}% ८

\twolineshloka
{तत्र भुक्त्वा महादुःखं मानुषत्वं गमिष्यति}
{रोगादिचिह्नितं तत्र दुर्यशो ज्ञापकं स्वकम्}% ९

\twolineshloka
{भूतद्रोहं विधायैव केवलं स्वकुटुम्बकम्}
{पुष्णाति पापनिरतः सोऽन्धतामिस्रके पतेत्}% १०

\twolineshloka
{ये नरा इह जन्तूनां वधं कुर्वन्ति वै मृषा}
{ते रौरवे निपात्यन्ते भिद्यन्ते रुरुभी रुषा}% ११

\twolineshloka
{यः स्वोदरार्थे भूतानां वधमाचरति स्फुटम्}
{महारौरवसंज्ञे तु पात्यते स यमाज्ञया}% १२

\twolineshloka
{यो वै निजं तु जनकं ब्राह्मणं द्वेष्टि पापकृत्}
{कालसूत्रे महादुष्टे योजनायुतविस्तृते}% १३

\twolineshloka
{यावन्ति पशुरोमाणि गवां द्वेषं करोति यः}
{तावद्वर्षसहस्राणि पच्यते यमकिङ्करैः}% १४

\twolineshloka
{यो भूमौ भूपतिर्भूत्वा दण्डायोग्यं तु दण्डयेत्}
{करोति ब्राह्मणस्यापि देहदण्डं च लोलुपः}% १५

\twolineshloka
{स सूकरमुखैर्दुष्टैः पीड्यते यमकिङ्करैः}
{पश्चाद्दुष्टासु योनीषु जायते पापमुक्तये}% १६

\twolineshloka
{ब्राह्मणानां गवां ये तु द्रव्यं वृत्तं तथाल्पकम्}
{वृत्तिं वा गृह्णते मोहाल्लुम्पन्ति स्वबलान्नराः}% १७

\twolineshloka
{ते परत्रान्धकूपे च पात्यन्ते च महार्दिताः}
{योऽन्नं स्वयमुपाहृत्य मधुरं चात्तिलोलुपः}% १८

\twolineshloka
{न देवाय न सुहृदे ददाति रसनापरः}
{स पतत्येव नरके कृमिभोजनसंज्ञिते}% १९

\twolineshloka
{अनापदि नरो यस्तु हिरण्यादीन्यपाहरेत्}
{ब्रह्मस्वं वा महादुष्टे सन्दंशे नरके पतेत्}% २०

\twolineshloka
{यः स्वदेहं प्रपुष्णाति नान्यं जानाति मूढधीः}
{स पात्यते तैलतप्ते कुम्भीपाकेऽतिदारुणे}% २१

\twolineshloka
{यो नागम्यां स्त्रियं मोहाद्योषिद्भावाच्च कामयेत्}
{तं तया किङ्कराः सूर्म्या परिरम्भं च कुर्वते}% २२

\twolineshloka
{ये बलाद्वेदमर्यादां लुम्पन्ति स्वबलोद्धताः}
{ते वैतरण्यां पतिता मांसशोणितभक्षकाः}% २३

\twolineshloka
{वृषलद्यं यः स्त्रियं कृत्वा तया गार्हस्थ्यमाचरेत्}
{पूयोदे निपतत्येव महादुःखसमन्वितः}% २४

\twolineshloka
{ये दम्भमाश्रयन्ते वै धूर्ता लोकस्य वञ्चने}
{वैशसे नरके मूढाः पतन्ति यमताडिताः}% २५

\twolineshloka
{ये सवर्णां स्त्रियं मूढा रेतः स्वं पाययन्ति च}
{रेतःकुल्यासु ते पापा रेतःपानेषु तत्पराः}% २६

\twolineshloka
{ये चौरा वह्निदा दुष्टा गरदा ग्रामलुण्ठकाः}
{सारमेयादने ते वै पात्यन्ते पातकान्विताः}% २७

\twolineshloka
{कूटसाक्ष्यं वदत्यद्धा पुरुषः पापसम्भृतः}
{परकीयं तु द्रव्यं यो हरति प्रसभं बली}% २८

\twolineshloka
{सोऽवीचिनरके पापी अवाग्वक्त्रः पतत्यधः}
{तत्र दुःखं महद्भुक्त्वा पापिष्ठां योनिमाव्रजेत्}% २९

\twolineshloka
{यो नरो रसनास्वादात्सुरां पिबति मूढधीः}
{तं पाययन्ति लोहस्य रसं धर्मस्य किङ्कराः}% ३०

\twolineshloka
{यो गुरूनवमन्येत स्वविद्याचारदर्पितः}
{स मृतः पात्यते क्षारनरकेऽधोमुखः पुमान्}% ३१

\twolineshloka
{विश्वासघातं कुर्वन्ति ये नरा धर्मनिष्कृताः}
{शूलप्रोते च नरके पात्यन्ते बहुयातने}% ३२

\twolineshloka
{पिशुनो यो नरान्सर्वानुद्वेजयति वाक्यतः}
{दन्दशूके च पतितो दन्दशूकैः स दश्यते}% ३३

\twolineshloka
{एवं राजन्ननेके वै नरकाः पापकारिणाम्}
{पापं कृत्वा प्रयान्त्येते पीडां यान्ति सुदारुणाम्}% ३४

\twolineshloka
{यैर्न श्रुता रामकथा न परोपकृतिः कृता}
{तेषां सर्वाणि दुःखानि भवन्ति नरकान्तरे}% ३५

\twolineshloka
{अत्र यस्य सुखं स्वर्गे भूयात्तस्य इतीर्यते}
{ये दुःखिनो रोगयुता नरकादागताश्च ते}% ३६

\uvacha{शेष उवाच}

\twolineshloka
{एतच्छ्रुत्वा महीपालः कम्पमानः क्षणे क्षणे}
{पप्रच्छ भूयस्तं विप्रं सर्वसंशयनुत्तये}% ३७

\twolineshloka
{तत्तत्पापस्य चिह्नानि कथयस्व महामुने}
{केन पापेन किं चिह्नं भूलोके उपजायते}% ३८

\twolineshloka
{इति श्रुत्वा तु तद्वाक्यं मुनिः प्रोवाच भूपतिम्}
{शृणु राजन्प्रवक्ष्यामि चिह्नानि पापकारिणाम्}% ३९

\uvacha{शौनक उवाच}

\twolineshloka
{सुरापः श्यामदन्तश्च नरकान्ते प्रजायते}
{अभक्ष्यभक्षकारी च जायते गुल्मकोदरः}% ४०

\twolineshloka
{उदक्यावीक्षितं भुक्त्वा जायते कृमिलोदरः}
{श्वमार्जारादिसंस्पृष्टं भुक्त्वा दुर्गन्धिमान्भवेत्}% ४१

\twolineshloka
{अनिवेद्य सुरादिभ्यो भुञ्जानो जायते नरः}
{उदरे रोगवान्दुःखी महारोगप्रपीडितः}% ४२

\twolineshloka
{परान्नविघ्नकरणादजीर्णमभिजायते}
{मन्दोदराग्निर्भवति सति द्रव्ये कदन्नदः}% ४३

\twolineshloka
{विषदश्छर्दिरोगी स्यान्मार्गहा पादरोगवान्}
{पिशुनो नरकस्यान्ते जायते श्वासकासवान्}% ४४

\twolineshloka
{धूर्तोऽपस्माररोगी स्याच्छूली च परतापनः}
{दावाग्निदायकश्चैव रक्तातीसारवान्भवेत्}% ४५

\twolineshloka
{सुरालये जले वापि शकृत्क्षेपं करोति यः}
{गुदरोगो भवेत्तस्य पापरूपः सुदारुणः}% ४६

\twolineshloka
{गर्भपातनजा रोगाः क्षयमेहजलोदराः}
{प्रतिमा भङ्गकारी च अप्रतिष्ठश्च जायते}% ४७

\twolineshloka
{दुष्टवादी खण्डितः स्यात्खल्वाटः परनिन्दकः}
{सभायां पक्षपाती च जायते पक्षघातवान्}% ४८

\twolineshloka
{परोक्तहास्यकृत्काणः कुनखी विप्रहेमहृत्}
{तुन्दीवरी ताम्रचौरः कांस्यहृत्पुण्डरीकिकः}% ४९

\twolineshloka
{त्रपुहारी च पुरुषो जायते पिङ्गमूर्द्धजः}
{शीसहारी च पुरुषो जायते शीर्षरोगवान्}% ५०

\twolineshloka
{घृतचौरस्तु पुरुषो जायते नेत्ररोगवान्}
{लोहहारी च पुरुषो बर्बराङ्गः प्रजायते}% ५१

\twolineshloka
{चर्महारी च पुरुषो जायते मेदसा वृतः}
{मधुचौरस्तु पुरुषो जायते बस्तिगन्धवान्}% ५२

\twolineshloka
{तैलचौर्येण भवति नरः कण्ड्वातिपीडितः}
{आमान्नहरणाच्चैव दन्तहीनः प्रजायते}% ५३

\twolineshloka
{पक्वान्नहरणाच्चैव जिह्वारोगयुतो भवेत्}
{मातृगामी च पुरुषो जायते लिङ्गवर्जितः}% ५४

\twolineshloka
{गुरुजायाभिगमनान्मूत्रकृच्छ्रः प्रजायते}
{भगिनीं चैव गमने पीतकुष्ठः प्रजायते}% ५५

\twolineshloka
{स्वसुतागमने चैव रक्तकुष्ठः प्रजायते}
{भ्रातृभार्याभिगमने गुल्मकुष्ठः प्रजायते}% ५६

\twolineshloka
{स्वामिगम्यादिगमने जायते दद्रुमण्डलम्}
{विश्वस्तभार्यागमने गजचर्मा प्रजायते}% ५७

\twolineshloka
{पितृष्वस्रभिगमने दक्षिणाङ्गे व्रणी भवेत्}
{मातुलान्यास्तु गमने वामाङ्गे व्रणवान्भवेत्}% ५८

\twolineshloka
{पितृव्यपत्नीगमने कटौ कुष्ठः प्रजायते}
{मित्रभार्याभिगमने मृतभार्यः प्रजायते}% ५९

\twolineshloka
{स्वगोत्रस्त्रीप्रसङ्गेन जायते च भगन्दरः}
{तपस्विनीप्रसङ्गेन प्रमेहो जायते नरे}% ६०

\twolineshloka
{श्रोत्रियस्त्रीप्रसङ्गेन जायते नासिकाव्रणी}
{दीक्षितस्त्रीप्रसङ्गेन जायते दुष्टरक्तसृक्}% ६१

\twolineshloka
{स्वजातिजायागमने जायते हृदयव्रणी}
{जात्युन्नतस्त्रीगमने जायते मस्तकव्रणी}% ६२

\twolineshloka
{पशुयोनौ च गमनान्मूत्रघातः प्रजायते}
{एते दोषा नराणां स्युर्नरकान्ते न संशयः}% ६३

\twolineshloka
{स्त्रीणामपि भवन्त्येते तत्तत्पुरुषसङ्गमात्}
{एवं राजन्हि चिह्नानि कीर्तितानि सुपापिनाम्}% ६४

\twolineshloka
{दानपुण्यप्रसङ्गेन तीर्थादिक्रियया तथा}
{रामस्य चरितं श्रुत्वा तपसा वाक्षयं व्रजेत्}% ६५

\twolineshloka
{सर्वेषामेव पापानां हरिकीर्तिधुनी नृणाम्}
{क्षालयेत्पापिनां पङ्कं नात्र कार्या विचारणा}% ६६

\twolineshloka
{यो नावमन्येत हरिं तस्य यागाविधि श्रुताः}
{तीर्थान्यपि सुपुण्यानि पावितुं न क्षमाणि तम्}% ६७

\twolineshloka
{हसते कीर्त्यमानं यश्चरित्रं ज्ञानदुर्बलः}
{न तस्य नरकान्मुक्तिः कल्पान्तेऽपि भविष्यति}% ६८

\twolineshloka
{या हि राजन्विमोक्षार्थं हयस्यानुचरैः सह}
{श्रावय श्रीशचरितं यतो वाहगतिर्भवेत्}% ६९

\uvacha{शेष उवाच}

\twolineshloka
{इति श्रुत्वा प्रहृष्टोऽभूच्छत्रुघ्नः परवीरहा}
{प्रणम्य तं परिक्रम्य ययौ सेवकसंयुतः}% ७०

\twolineshloka
{तत्र गत्वा स हनुमान्हयवर्यस्य पार्श्वतः}
{उवाच रामचरितं महादुर्गतिनाशकम्}% ७१

\twolineshloka
{याहि देव विमानं स्वं रामकीर्तनपुण्यतः}
{स्वैरं चर स्वलोके त्वं मुक्तो भव कुयोनितः}% ७२

\twolineshloka
{इति वाक्यं समाकर्ण्य शत्रुघ्नो यावदास्थितः}
{तावद्ददर्श विमलं देवं वैमानिकं वरम्}% ७३

\twolineshloka
{स उवाच विमुक्तोऽहं रामकीर्तनसंश्रुतेः}
{यामि स्वं भवनं राजन्नाज्ञापय महामते}% ७४

\twolineshloka
{इत्युक्त्वा प्रययौ देवो विमाने स्वे परिस्थितः}
{तदा विस्मयमापुस्ते शत्रुघ्नेन सहानुगाः}% ७५

\twolineshloka
{ततो वाहो विनिर्मुक्तो गात्रस्तम्भाच्च भूतलात्}
{ययौ तद्विपिनं सर्वं भ्रमन्पक्षिसमाकुलम्}% ७६

{॥इति श्रीपद्मपुराणे पातालखण्डे रामाश्वमेधे शेषवात्स्यायनसंवादे हयनिर्मुक्तिर्नामाष्टचत्वारिंशत्तमोऽध्यायः॥४८॥}

\dnsub{एकोनपञ्चाशत्तमोऽध्यायः}%\resetShloka

\uvacha{शेष उवाच}

\twolineshloka
{मासाः सप्ताभवंस्तस्य हयवर्यस्य हेलया}
{चरतो भारतं वर्षमनेकनृपपूरितम्}% १

\twolineshloka
{स पूजितो भूपवरैः परीत्य वरभारतम्}
{परीवृतो वीरवरैः शत्रुघ्नादिभिरुद्भटैः}% २

\twolineshloka
{स बभ्राम बहून्देशान्हिमालयसमीपतः}
{न कोपि तं निजग्राह हयं रामबलं स्मरन्}% ३

\twolineshloka
{अङ्गवङ्गकलिङ्गानां राजभिः संस्तुतो हयः}
{जगाम राज्ञो नगरे सुरथस्य मनोहरे}% ४

\twolineshloka
{कुण्डलं नाम नगरमदितेर्यत्र कुण्डलम्}
{कर्णयोः पतितं भूमौ हर्षभयसुकम्पयोः}% ५

\twolineshloka
{यत्र धर्मव्यतिक्रान्तिं न करोति कदापिना}
{श्रीरामस्मरणं प्रेम्णा करोति जनतान्वहम्}% ६

\twolineshloka
{अश्वत्थानां तु यत्रार्चा तुलस्याः प्रत्यहं नृभिः}
{क्रियते रघुनाथस्य सेवकैः पापवर्जितैः}% ७

\twolineshloka
{यत्र देवालया रम्या राघवप्रतिमायुताः}
{पूज्यन्ते प्रत्यहं शुद्धचित्तैः कपटवर्जितैः}% ८

\twolineshloka
{वाचि नाम हरेर्यत्र न वै कलहसङ्कथा}
{हृदि ध्यानं तु तस्यैव न च कामफलस्मृतिः}% ९

\twolineshloka
{देवनं यत्र रामस्य वार्त्ताभिः पूतदेहिनाम्}
{न जातुचिन्नृणामस्ति सत्यव्यसनमानिनाम्}% १०

\twolineshloka
{तस्मिन्वसति धर्मात्मा सुरथः सत्यवान्बली}
{रघुनाथपदस्मारहृष्टचित्तः परोन्मदः}% ११

\twolineshloka
{किं वर्णयामि रामस्य सेवकं सुरथं वरम्}
{यस्याशेषगुणा भूमौ विस्तृताः पावयन्त्यघम्}% १२

\twolineshloka
{सेवकास्तस्य भूपस्य पर्यटन्तः कदाचन}
{अपश्यन्हयमेधस्य हयं चन्दनचर्चितम्}% १३

\twolineshloka
{ते दृष्ट्वा विस्मयं प्राप्ता हयपत्रमलोकयन्}
{स्पष्टाक्षरसमायुक्तं चन्दनादिकचर्चितम्}% १४

\twolineshloka
{ज्ञात्वा रामेण सम्मुक्तं हयं नेत्रमनोहरम्}
{हृष्टा राज्ञे सभास्थाय कथयामासुरुत्सुकाः}% १५

\twolineshloka
{स्वामिन्नयोध्यानगरीपतिस्तस्यास्तु राघवः}
{हयमेधक्रतोर्योग्यो हयो मुक्तः परिभ्रमन्}% १६

\twolineshloka
{स ते पुरस्य निकटे प्राप्तः सेवकसंयुतः}
{गृहाण त्वं महाराज हयं तं सुमनोहरम्}% १७

\uvacha{शेष उवाच}

\twolineshloka
{इति श्रुत्वा निजप्रोक्तं वाक्यं हर्षपरिप्लुतः}
{उवाच वीरान्बलिनो मेघगम्भीरया गिरा}% १८

\uvacha{सुरथ उवाच}

\twolineshloka
{धन्या वयं राममुखं पश्यामः सह सेवकाः}
{ग्रहीष्यामि हयं तस्य भटकोटिपरीवृतम्}% १९

\twolineshloka
{तदा मोक्ष्यामि वाहं तं यदा रामः समाव्रजेत्}
{कृतार्थं मम भक्तस्य चिरं ध्यानरतस्य वै}% २०

\uvacha{शेष उवाच}

\twolineshloka
{इत्थमुक्त्वा महीपालः सेवकान्स्वयमादिशत्}
{गृह्णन्तु वाहं प्रसभं मोच्यो नाश्वोऽक्षिगोचरः}% २१

\twolineshloka
{अनेन सुमहाँल्लाभो भविष्यति तु मे मतम्}
{यद्रामचरणौ प्रेक्षे ब्रह्मशक्रादिदुर्ल्लभौ}% २२

\twolineshloka
{स एव धन्यः स्वजनः पुत्रो वा बान्धवोऽथवा}
{पशुर्वा वाहनं वापि रामाप्तिर्येन सम्भवेत्}% २३

\twolineshloka
{तस्माद्गृहीत्वा क्रत्वश्वं स्वर्णपत्रेण शोभितम्}
{बध्नन्तु वाजिशालायां कामवेगं मनोरमम्}% २४

\twolineshloka
{इत्युक्तास्ते ततो गत्वा वाहं रामस्य शोभितम्}
{गृहीत्वा तरसा राज्ञे ददुः सर्वं शुभाङ्गिनम्}% २५

\twolineshloka
{राजा प्राप्य मुदा चाश्वं रामस्य दनुजार्दनः}
{सेवकान्प्राह बलिनो धर्मकृत्यविचक्षणः}% २६

\twolineshloka
{वात्स्यायन महाबुद्धे शृणुष्वैकाग्रमानसः}
{न तस्य विषये कश्चित्परदाररतो नरः}% २७

\twolineshloka
{न परद्रव्यनिरतो न च कामेषु लम्पटः}
{न जिह्वाभिरतोन्मार्गे कीर्त्तयेद्रामकीर्तनात्}% २८

\twolineshloka
{यः सेवकान्नृपो वक्ति यूयं सेवार्थमागताः}
{कथयन्तु भवच्चेष्टां धर्मकर्मविशारदाः}% २९

\twolineshloka
{एकपत्नीव्रतधरा न परद्रव्यलोलुपाः}
{परापवादानिरता न च वेदोत्पथं गताः}% ३०

\twolineshloka
{श्रीरामस्मरणादीनि कुर्वन्ति प्रत्यहं भटाः}
{तानहं रामसेवार्थं रक्षाम्यन्तक कोपवान्}% ३१

\twolineshloka
{एतद्विरुद्धधर्माणो ये नराः पापसंयुताः}
{तानहं विषये मह्यं वासयामि न दुर्मतीन्}% ३२

\twolineshloka
{तस्य देशे न पापिष्ठाः पापं कुर्वन्ति मानसे}
{हरिध्यानहताशेष पातकामोदसंयुताः}% ३३

\twolineshloka
{यदैवमभवद्देशो राजा धर्मेण संयुतः}
{तदा तत्स्था नराः सर्वे मृता गच्छन्ति निर्वृतिम्}% ३४

\twolineshloka
{यमानुचरनिर्वेशो नाभवत्सौरथे पुरे}
{तदा यमो मुनेरूपं धृत्वा प्रागान्महीश्वरम्}% ३५

\twolineshloka
{वल्कलाम्बरधारी च जटाशोभितशीर्षकः}
{सुरथं तु सभामध्ये ददर्श हरिसेवकम्}% ३६

\twolineshloka
{तुलसीमस्तके यस्य वाचि नाम हरेः परम्}
{धर्मकर्मरतां वार्त्तां श्रावयन्तं निजाञ्जनान्}% ३७

\twolineshloka
{तदा मुनिं नृपो दृष्ट्वा तपोमूर्तिमिव स्थितम्}
{ववन्दे चरणौ तस्य पाद्यादिकमथाकरोत्}% ३८

\twolineshloka
{सुखोपविष्टं विश्रान्तं मुनिं प्राह नृपाग्रणीः}
{धन्यमद्य जनुर्मह्यं धन्यमद्य गृहं मम}% ३९

\twolineshloka
{कथाः कथयतान्मह्यं रामस्य विविधा वराः}
{याः शृण्वतां पापहानिर्भविष्यति पदे पदे}% ४०

\twolineshloka
{इत्थमुक्तं समाकर्ण्य जहास स मुनिर्भृशम्}
{दन्तान्प्रदर्शयन्सर्वांस्तालास्फालितपाणिकः}% ४१

\twolineshloka
{हसन्तं तं मुनिं प्राह हसने कारणं किमु}
{कथयस्व प्रसादेन यथा स्यान्मनसः सुखम्}% ४२

\twolineshloka
{ततो मुनिर्नृपं प्राह शृणु राजन्धियायुतः}
{यदहं तेऽभिधास्यामि स्मिते कारणमुत्तमम्}% ४३

\twolineshloka
{त्वया प्रोक्तं हरेः कीर्तिं कथयस्व ममाग्रतः}
{को हरिः कस्य वा कीर्तिः सर्वे कर्मवशा नराः}% ४४

\twolineshloka
{कर्मणा प्राप्यते स्वर्गः कर्मणा नरकं व्रजेत्}
{कर्मणैव भवेत्सर्वं पुत्रपौत्रादिकं बहु}% ४५

\twolineshloka
{शक्रः शतं क्रतूनां तु कृत्वागात्परमं पदम्}
{ब्रह्मापि कर्मणा लोकं प्राप्य सत्याख्यमद्भुतम्}% ४६

\twolineshloka
{अनेके कर्मणा सिद्धा मरुदादय ईशिनः}
{कुर्वन्ति भोगसौख्यं च अप्सरोगणसेविताः}% ४७

\twolineshloka
{तस्मात्कुरुष्व यज्ञादीन्यजस्व किल देवताः}
{यथा ते विमलाकीर्तिर्भविष्यति महीतले}% ४८

\twolineshloka
{इति श्रुत्वा तु तद्वाक्यं कोपक्षुभितमानसः}
{उवाच रामैकमना विप्रं कर्मविशारदम्}% ४९

\twolineshloka
{मा ब्रूहि कर्मणो वार्तां क्षयिष्णुफलदायिनीम्}
{गच्छ मन्नगरोपान्ताद्बहिर्लोकविगर्हितः}% ५०

\twolineshloka
{इन्द्रः पतिष्यति क्षिप्रं पतिष्यत्यपि पद्मजः}
{न पतिष्यन्ति मनुजा रामस्य भजनोत्सुकाः}% ५१

\twolineshloka
{पश्य ध्रुवं च प्रह्लादं बिभीषणमथाद्भुतम्}
{ये चान्ये रामभक्ता वै कदापि न पतन्ति ते}% ५२

\twolineshloka
{ये रामनिन्दका दुष्टास्तानि मे यमकिङ्कराः}
{ताडयिष्यन्ति लोहस्य मुद्गरैः पाशबन्धनैः}% ५३

\twolineshloka
{ब्राह्मणत्वाद्देहदण्डं न कुर्यां ते द्विजाधम}
{गच्छ गच्छ मदालोकात्ताडयिष्यामि चान्यथा}% ५४

\twolineshloka
{इत्थमुक्तवति श्रेष्ठे भूपे सुरथसंज्ञिते}
{सेवका बाहुना धृत्वा निष्कासयितुमुद्यताः}% ५५

\twolineshloka
{तदा यमो निजं रूपं धृत्वा लोकैकवन्दितम्}
{प्राह भूपं प्रतुष्टोऽस्मि याचस्व हरिसेवक}% ५६

\twolineshloka
{मया प्रलोभितो वाग्भिर्बह्वीभिरपि सुव्रत}
{चलितोसि न रामस्य सेवायाः साधुसेवितः}% ५७

\twolineshloka
{तदा प्रोवाच भूमीशो यमं दृष्ट्वा सुतोषितम्}
{उवाच यदि तुष्टोसि देहि मे वरमुत्तमम्}% ५८

\twolineshloka
{तावन्मम न वै मृत्युर्यावद्रामसमागमः}
{न भयं मे भवत्तो हि कदाचन हि धर्मराट्}% ५९

\twolineshloka
{तदोवाच यमो भूपमिदं तव भविष्यति}
{सर्वं त्वदीप्सितं तथ्यं करिष्यति रघोःपतिः}% ६०

\twolineshloka
{इत्युक्त्वान्तर्हितो धर्मो जगाम स्वपुरं प्रति}
{प्रशस्य तस्य चरितं हरिभक्तिपरात्मनः}% ६१

\twolineshloka
{स राजा धार्मिको रामसेवकः परया मुदा}
{गृहीत्वाश्वं प्रत्युवाच सेवकान्हरिसेवकान्}% ६२

\twolineshloka
{मया गृहीतो वाहोऽसौ राघवस्य महीपतेः}
{सज्जी भवन्तु सर्वत्र यूयं रणविशारदाः}% ६३

\twolineshloka
{इति प्रोक्तास्तु ते सर्वे भटा राज्ञो महाबलाः}
{सज्जीभूताः क्षणादेव सभायां जग्मुरुत्सुकाः}% ६४

\twolineshloka
{राज्ञो वीरा दशसुताश्चम्पको मोहकस्तथा}
{रिपुञ्जयोऽतिदुर्वारः प्रतापीबलमोदकः}% ६५

\twolineshloka
{हर्यक्षः सहदेवश्च भूरिदेवः सुतापनः}
{इति राज्ञो दश सुताः सज्जीभूता रणाङ्गणे}% ६६

\twolineshloka
{यातुमिच्छामकुर्वंस्ते महोत्साहसमन्विताः}
{राजापि स्वरथं चित्रं हेमशोभाविनिर्मितम्}% ६७

\twolineshloka
{आह्वयामास सुजवैर्वाजिभिः समलङ्कृतम्}
{रणोत्साहेन संयुक्तः सर्वसैन्यपरीवृतः}% ६८

\onelineshloka
{सभायां सेवकान्सर्वान्दिशन्नास्ते महीपतिः}% ६९

{॥इति श्रीपद्मपुराणे पातालखण्डे शेषवात्स्यायनसंवादे रामाश्वमेधे सुरथराज्ञा हयग्रहणं नाम एकोनपञ्चाशत्तमोऽध्यायः॥४९॥}

\dnsub{पञ्चाशत्तमोऽध्यायः}%\resetShloka

\uvacha{शेष उवाच}

\twolineshloka
{अथ रामानुजो वेगात्समागत्य स्वसेवकान्}
{पप्रच्छ कुत्र वाहोऽसौ याज्ञिकः सुमनोहरः}% १

\twolineshloka
{तदा ते वचनं प्रोचुः शत्रुघ्नं सुमहाबलाः}
{न जानीमो भटाः केचिद्धयं नीत्वा गताः पुरे}% २

\twolineshloka
{वयं च धिक्कृताः सर्वे बलिभी राजसेवकैः}
{अत्र प्रमाणं भगवानिति कर्तव्य तां प्रति}% ३

\twolineshloka
{तच्छ्रुत्वा वचनं तेषां शत्रुघ्नः कुपितो भृशम्}
{दशन्रोषात्स्वदशनाञ्जिह्वया लेलिहन्मुहुः}% ४

\twolineshloka
{उवाच वीरो मद्वाहं हृत्वा कुत्र गमिष्यसि}
{इदानीं पातये बाणैः पुरञ्जनसमन्वितम्}% ५

\twolineshloka
{इत्युक्त्वा सुमतिं प्राह कस्येदं पुटभेदनम्}
{को वर्ततेऽस्याधिपतिर्यो मे वाहमजीहरत्}% ६

\uvacha{शेष उवाच}

\twolineshloka
{इति वाक्यं समाकर्ण्य भूपतेः कोपसंयुतम्}
{जगाद मन्त्री सुगिरा स्फुटाक्षरसमन्वितम्}% ७

\twolineshloka
{विद्धीदं कुण्डलं नाम नगरं सुमनोहरम्}
{अस्मिन्वसति धर्मात्मा सुरथः क्षत्त्रियो बली}% ८

\twolineshloka
{नित्यं धर्मपरो रामचरणद्वन्द्वसेवकः}
{मनसा कर्मणा वाचा हनूमानिव सेवकः}% ९

\twolineshloka
{चरितान्यस्य शतशो वर्तन्ते धर्मकारिणः}
{महाबलपरीवारः सुरथः सर्वशोभनः}% १०

\twolineshloka
{महद्युद्धं भवेदत्र हृतश्चेद्वाहसत्तमः}
{अनेके प्रपतिष्यन्ति वीरा रणविशारदाः}% ११

\twolineshloka
{एवमुक्तं समाश्रुत्य शत्रुघ्नः सचिवं प्रति}
{उवाच पुनरप्येवं वचनं वदतां वरः}% १२

\uvacha{शत्रुघ्न उवाच}

\twolineshloka
{कथमत्र प्रकर्तव्यं रामाश्वोऽनेन चेद्धृतः}
{नायाति योद्धुं प्रबलं कटकं वीरसेवितम्}% १३

\uvacha{सुमतिरुवाच}

\twolineshloka
{दूतः प्रेष्यो महाराज राजानं प्रति वाग्मिकः}
{यद्वाक्येन समायाति बलेन बलिनां वरः}% १४

\twolineshloka
{नोचेदज्ञानतो वाहो धृतः केनापि मानिना}
{अर्पयिष्यति नः साधुमश्वं क्रतुवरं शुभम्}% १५

\twolineshloka
{इति श्रुत्वातु तद्वाक्यं शत्रुघ्नो बुद्धिमान्बली}
{अङ्गदं प्रत्युवाचेदं वचनं विनयान्वितम्}% १६

\uvacha{शत्रुघ्न उवाच}

\twolineshloka
{याहि त्वं निकटस्थे वै सुरथस्य महापुरे}
{दूतत्वेन ततो गत्वा प्रब्रूहि नृपतिं प्रति}% १७

\twolineshloka
{त्वया धृतो रामवाहो ज्ञानतोऽज्ञानतोपि वा}
{अर्पयतु न वा यातु प्रधनं वीरसंयुतम्}% १८

\twolineshloka
{रामस्य दौत्यं लङ्कायां रावणं प्रति यत्कृतम्}
{तथैव कुरु भूयिष्ठ बलसंयुतबुद्धिमन्}% १९

\uvacha{शेष उवाच}

\twolineshloka
{एतच्छ्रुत्वाङ्गदो वीर ओमिति प्रोच्य भूमिपम्}
{जगाम संसदो मध्ये वीरश्रेणिसमन्विते}% २०

\twolineshloka
{ददर्श सुरथं भूपं तुलसीमञ्जरीधरम्}
{रामभद्रं रसनया ब्रुवन्तं सेवकान्निजान्}% २१

\twolineshloka
{राजापि दृष्ट्वा प्लवगं मनोहरवपुर्धरम्}
{शत्रुघ्नदूतं मत्वापि वालिजं प्रत्यभाषत}% २२

\uvacha{सुरथ उवाच}

\twolineshloka
{प्लवगाधिप कस्मात्त्वमागतोऽत्र कथं भवान्}
{ब्रूहि मे कारणं सर्वं यथा ज्ञात्वा करोमि तत्}% २३

\uvacha{शेष उवाच}

\twolineshloka
{इति सम्भाषमाणं तं प्रत्युवाच कपीश्वरः}
{विस्मयंश्चेतसि भृशं रामसेवाकरं नृपम्}% २४

\twolineshloka
{जानीहि मां नृपश्रेष्ठ वालिपुत्रं हरीश्वरम्}
{शत्रुघ्नेन च दूतत्वे प्रेषितो भवतोऽन्तिकम्}% २५

\twolineshloka
{सेवकैः कैश्चिदागत्य धृतोऽश्वो मम साम्प्रतम्}
{अज्ञानतो महान्याय्यं कुर्वद्भिः सहसा नृप}% २६

\twolineshloka
{तमश्वं सह राज्येन सहपुत्रैर्मुदान्वितः}
{शत्रुघ्नं याहि चरणे पतित्वाशु प्रदेहि च}% २७

\twolineshloka
{नोचेच्छत्रुघ्ननिर्मुक्तनाराचैः क्षतविग्रहः}
{पृथ्वीतलमलं कुर्वञ्छयिष्यसि विशीर्षकः}% २८

\twolineshloka
{येन लङ्कापतिर्नाशं प्रापितो लीलया क्षणात्}
{तस्याश्वं यागयोग्यं तु हृत्वा कुत्र गमिष्यसि}% २९

\uvacha{शेष उवाच}

\twolineshloka
{इत्यादिभाषमाणं तं प्रत्युवाच महीश्वरः}
{सर्वं तथ्यं ब्रवीषि त्वं नानृतं तव भाषितम्}% ३०

\twolineshloka
{परं शृणुष्व मद्वाक्यं शत्रुघ्नपदसेवक}
{मया धृतो महानश्वो रामचन्द्रस्य धीमतः}% ३१

\twolineshloka
{न मोक्ष्ये सर्वथा वाहं शत्रुघ्नादिभयादहम्}
{चेद्रामः स्वयमागत्य दर्शनं दास्यते मम}% ३२

\twolineshloka
{तदाहं चरणौ नत्वा दास्यामि सुतसंयुतः}
{सर्वं राज्यं कुटुम्बं च धनं धान्यं बलं बहु}% ३३

\twolineshloka
{क्षत्त्रियाणामयं धर्मः स्वामिनापि विरुद्ध्यते}
{धर्मेण युद्धं तत्रापि रामदर्शनमिच्छता}% ३४

\twolineshloka
{शत्रुघ्नादीन्प्रवीरांस्तानधुनाहं क्षणादपि}
{जित्वा बध्नामि मद्गेहे नोचेद्रामः समाव्रजेत्}% ३५

\uvacha{शेष उवाच}

\twolineshloka
{इति श्रुत्वाङ्गदो धीमाञ्जहास नृपतिं तदा}
{उवाच च महद्वाक्यं महाधैर्यसमन्वितम्}% ३६

\uvacha{अङ्गद उवाच}

\twolineshloka
{बुद्धिहीनः प्रवदसि वृद्धत्वात्सागता तव}
{यत्त्वं शत्रुघ्ननृपतिं धिक्करोषि धिया बली}% ३७

\twolineshloka
{यो मान्धातृरिपुं दैत्यं लवणं लीलयावधीत्}
{येनानेके जिताः सङ्ख्ये वैरिणः प्रबलोद्धताः}% ३८

\twolineshloka
{विद्युन्माली हतो येन राक्षसः कामगे स्थितः}
{त्वं तं बध्नासि वीरेन्द्रं मतिहीनः प्रभासि मे}% ३९

\twolineshloka
{भ्रातृजो यस्य सुबली पुष्कलः परमास्त्रवित्}
{येन रुद्रगणः सङ्ख्ये वीरभद्रः सुतोषितः}% ४०

\twolineshloka
{वर्णयामि किमेतस्य पराक्रान्तिं बलोर्जिताम्}
{येन नास्ति समः पृथ्व्यां बलेन यशसा श्रिया}% ४१

\twolineshloka
{हनूमान्यस्य निकटे रघुनाथपदाब्जधीः}
{यस्यानेकानि कर्माणि भविष्यन्ति श्रुतानि ते}% ४२

\twolineshloka
{सत्रिकूटा राक्षसपूर्दग्धा येन क्षणाद्बलात्}
{अक्षो येन हतः पुत्रो राक्षसेन्द्रस्य दुर्मतेः}% ४३

\twolineshloka
{द्रोणो नाम गिरिर्येन पुच्छाग्रेण सदैवतः}
{आनीतो जीवनार्थं तु सैनिकानां मुहुर्मुहुः}% ४४

\twolineshloka
{जानाति रामश्चारित्रं नान्यो जानाति मूढधीः}
{यं कपीन्द्रं मनाक्स्वान्तान्न विस्मरति सेवकम्}% ४५

\twolineshloka
{सुग्रीवाद्याः कपीन्द्रा ये पृथ्वीं सर्वां ग्रसन्ति ये}
{ते शत्रुघ्नं नृपं सर्वे सेवन्ते प्रेक्षणोत्सुकाः}% ४६

\twolineshloka
{कुशध्वजो नीलरत्नो रिपुतापो महास्त्रवित्}
{प्रतापाग्र्यः सुबाहुश्च विमलः सुमदस्तथा}% ४७

\twolineshloka
{राजा वीरमणिः सत्ययुतो रामस्य सेवकः}
{एतेऽन्येपि नृपा भूमेः पतयः पर्युपासते}% ४८

\twolineshloka
{तत्र त्वं वीर जलधौ मशकः को भवानिति}
{तज्ज्ञात्वा गच्छ शत्रुघ्नं कृपालुं पुत्रकैर्युतः}% ४९

\twolineshloka
{वाहं समर्प्य गन्तासि रामं राजीवलोचनम्}
{दृष्ट्वा कृतार्थी कुरुषे स्वाङ्गानि जनुषा सह}% ५०

\uvacha{शेष उवाच}

\twolineshloka
{राजा प्रोवाच तं दूतं प्रब्रुवन्तमनेकधा}
{एतान्दर्शयसि क्षिप्रं सर्वे न ममगोचराः}% ५१

\twolineshloka
{यादृशं मद्बलं दूत तादृशं न हनूमतः}
{यो रामं पृष्ठतः कृत्वा प्रागाद्यागस्य पालने}% ५२

\twolineshloka
{यद्यहं मनसा वाचा कर्मणा कुतुकान्वितः}
{भजामि रामं तर्ह्याशु दर्शयिष्यति स्वां तनुम्}% ५३

\twolineshloka
{अन्यथा हनुमन्मुख्या वीरा बध्नन्तु मां बलात्}
{गृह्णन्तु वाहं तरसा रामभक्तिसमन्विताः}% ५४

\twolineshloka
{गच्छ त्वं नृप शत्रुघ्नं कथयस्व ममोदितम्}
{सज्जीभवन्तु सुभटा एष यामि रणे बली}% ५५

\twolineshloka
{स विचार्य यथायुक्तं करिष्यति रणाङ्गणे}
{मोचयन्तु महावाहं न वामा मा ददन्तु ते}% ५६

\uvacha{शेष उवाच}

\twolineshloka
{इति श्रुत्वास्मि तं कृत्वा ययौ वीरो यतो नृपः}
{गत्वा निवेदयामास यथोक्तं सुरथेन वै}% ५७

{॥इति श्रीपद्मपुराणे पातालखण्डे शेषवात्स्यायनसंवादे रामाश्वमेधे सुरथदूतयोः संवादो नाम पञ्चाशत्तमोऽध्यायः॥५०॥}

\dnsub{एकपञ्चाशत्तमोऽध्यायः}%\resetShloka

\uvacha{शेष उवाच}

\twolineshloka
{तच्छ्रुत्वा भाषितं तस्य सुरथस्याङ्गदाननात्}
{सज्जीभूता रणे सर्वे रथस्था रणकोविदाः}% १

\twolineshloka
{पटहानां निनादोऽभूद्भेरीनादस्तथैव च}
{वीराणां गर्जनानादाः प्रादुर्भूता रणाङ्गणे}% २

\twolineshloka
{रथचीत्कारशब्देन गजानां बृंहितेन च}
{व्याप्तं तत्सकलं विश्वं दिवं यातो महारवः}% ३

\twolineshloka
{रणोत्साहेन संयुक्ता वीरा रणविशारदाः}
{कुर्वन्ति विविधान्नादान्कातरस्य भयङ्करान्}% ४

\twolineshloka
{एवं कोलाहले वृत्ते सुरथो नाम भूमिपः}
{स्वसुतैः सैनिकैश्चाथ वृतः प्रायाद्रणाङ्गणे}% ५

\twolineshloka
{गजैरथैर्हयैः पत्तिव्रजैः पूर्णां तु मेदिनीम्}
{कुर्वन्समुद्रइव तां प्लावयन्ददृशे भटैः}% ६

\twolineshloka
{शङ्खनादेन सङ्घुष्टं जयनादैस्तथैव च}
{वीक्ष्य तं प्रधनोद्युक्तं सुमतिं प्राह भूमिपः}% ७

\uvacha{शत्रुघ्न उवाच}

\twolineshloka
{एष राजा समायातो महासैन्यपरीवृतः}
{अत्र यत्कृत्यमस्माकं तद्वदस्व महामते}% ८

\uvacha{सुमतिरुवाच}

\twolineshloka
{योद्धव्यमत्र बहुभिर्वीरै रणविशारदैः}
{पुष्कलादिभिरत्युग्रैः सर्वशस्त्रास्त्रकोविदैः}% ९

\twolineshloka
{राज्ञा सह समीरस्य पुत्रः परमशौर्यवान्}
{युद्धं करोतु सुबलः परयुद्धविशारदः}% १०

\uvacha{शेष उवाच}

\twolineshloka
{इति ब्रूते महामात्यो यावत्तावन्नृपात्मजाः}
{रणाङ्गणे धनूंष्यद्धा स्फारयामासुरुद्धताः}% ११

\twolineshloka
{तान्वीक्ष्य योधाः सुबलाः पुष्कलाद्या रणोत्कटाः}
{अभिजग्मुः स्यन्दनैः स्वैर्धनुर्बाणकरा मताः}% १२

\twolineshloka
{चम्पकेन महावीरः पुष्कलः परमास्त्रवित्}
{द्वैरथेनैव युयुधे महावीरेण शालिना}% १३

\twolineshloka
{मोहकं योधयामास जानकिः स कुशध्वजः}
{रिपुञ्जयेन विमलो दुर्वारेण सुबाहुकः}% १४

\twolineshloka
{प्रतापिना प्रतापाग्र्यो बलमोदेन चाङ्गदः}
{हर्यक्षेण नीलरत्नः सहदेवेन सत्यवान्}% १५

\twolineshloka
{राजा वीरमणिर्भूरि देवेन युयुधे बली}
{असुतापेन चोग्राश्वो युयुधे बलसंयुतः}% १६

\twolineshloka
{द्वैरथं तु महद्युद्धमकुर्वन्युद्धकोविदाः}
{सर्वे शस्त्रास्त्रकुशलाः सर्वे युद्धविशारदाः}% १७

\twolineshloka
{एवं प्रवृत्ते सङ्ग्रामे सुरथस्य सुतैस्तदा}
{अत्यन्तं कदनं तत्र बभूव मुनिसत्तम}% १८

\twolineshloka
{पुष्कलश्चम्पकं प्राह किं नामासि नृपात्मज}
{धन्योसि यो मया सार्धं रणमध्यमुपेयिवान्}% १९

\twolineshloka
{इदानीं तिष्ठ किं यासि कथं ते जीवितं भवेत्}
{एहि युद्धं मया सार्धं सर्वशस्त्रास्त्रकोविद}% २०

\twolineshloka
{इत्यभिव्याहृतं तस्य श्रुत्वा राजात्मजो बली}
{जगाद पुष्कलं वीरो मेघगम्भीरया गिरा}% २१

\uvacha{चम्पक उवाच}

\twolineshloka
{न नाम्ना न कुलेनेदं युद्धमत्र भविष्यति}
{तथापि तव वक्ष्येऽहं स्वनामबलपूर्वकम्}% २२

\twolineshloka
{मम माता राघवेशो मत्पिता राघवः स्मृतः}
{मम बन्धू रामचन्द्र स्वःजनो मम राघवः}% २३

\twolineshloka
{मन्नाम रामदासश्च सदा रामस्य सेवकः}
{तारयिष्यति मां युद्धे रामो भक्तकृपाकरः}% २४

\twolineshloka
{लोकानां मतमास्थाय प्रब्रवीमि तवाधुना}
{सुरथस्य सुतश्चाहं माता वीरवतीमम}% २५

\twolineshloka
{मन्नामयो मधौ सर्वाञ्छोभनान्विदधाति वै}
{मधुपायंरसावा सन्त्यजन्ति मधुमोहिताः}% २६

\twolineshloka
{वर्णेन स्वर्णसदृशो मध्ये लिङ्गवपुर्धरः}
{तदाख्ययाभिधां वीर जानीहि मम मोहिनीम्}% २७

\twolineshloka
{युध्यस्व बाणैः प्रधनेन को जेतुं हि मां क्षमः}
{इदानीं दर्शयिष्यामि स्वपराक्रममद्भुतम्}% २८

\uvacha{शेष उवाच}

\twolineshloka
{इति श्रुत्वा महद्वाक्यं पुष्कलो हृदि तोषितः}
{तं दुर्जयं मन्यमानः शरान्मुञ्चन्रणेऽभवत्}% २९

\twolineshloka
{शरसङ्घं प्रमुञ्चन्तं कोटिधा पुष्कलं ययौ}
{चम्पकः कोपसंयुक्तो धनुः सज्यमथाकरोत्}% ३०

\twolineshloka
{मुमोच निशितान्बाणान्वैरिवृन्दविदारणान्}
{स्वनामचिह्नितान्स्वर्णपुङ्खभागसमन्वितान्}% ३१

\twolineshloka
{तांश्चिच्छेद महावीरः पुष्कलः प्रधनाङ्गणे}
{शरान्धकारं सर्वत्र मुञ्चन्बाणाञ्छिलाशितान्}% ३२

\twolineshloka
{स्वबाणच्छेदनं दृष्ट्वा कृतं वीरेण चम्पकः}
{आह्वयामास बलिनं पुष्कलं कोपपूरितः}% ३३

\twolineshloka
{मा प्रयाहि रणं त्यक्त्वेति ब्रुवन्समरे पुनः}
{पुष्कलं हृदये बाणैर्विव्याध दशभिस्त्वरन्}% ३४

\twolineshloka
{ते बाणाः पुष्कलस्याहो हृदये तीव्रवेगिनः}
{आगत्य सुभृशं लग्नाः शोणितं पपुरूर्जितम्}% ३५

\twolineshloka
{तैर्बाणैर्व्यथितो वीरः शरान्पञ्च समाददे}
{सुतीक्ष्णाग्रान्महाकोपाद्वारयन्पर्वतानिव}% ३६

\twolineshloka
{ते बाणास्तस्य बाणाश्च परस्परमथोर्जिताः}
{आकाशे रचिताश्छिन्नाः शतधा राजसूनुना}% ३७

\twolineshloka
{छित्त्वा बाणान्सुतीक्ष्णाग्रान्सुरथाङ्गोद्भवो बली}
{बाणाञ्छतं समाधत्त पुष्कलं ताडितुं हृदि}% ३८

\twolineshloka
{ते बाणाः शतधाच्छिन्नाः पुष्कलेन महात्मना}
{अपतन्समरोपान्ते शरवेगप्रपीडिताः}% ३९

\twolineshloka
{तदा तत्सुमहत्कर्म दृष्ट्वा राज्ञः सुतो बली}
{सहस्रेण शराणां च ताडयन्वक्षसि स्फुटम्}% ४०

\twolineshloka
{तानप्याशु प्रचिच्छेद पुष्कलः परमास्त्रवित्}
{पुनरप्याशु स्वे चापे समाधत्तायुतं शरान्}% ४१

\twolineshloka
{तानप्याशु प्रचिच्छेद पुष्कलः परमास्त्रवित्}
{ततोऽत्यतं प्रकुपितः शरवृष्टिमथाकरोत्}% ४२

\twolineshloka
{शरवृष्टिं समायान्तीं मत्वा चम्पक वीरहा}
{साधुसाधुप्रशंसन्तं पुष्कलं समताडयत्}% ४३

\twolineshloka
{पुष्कलश्चम्पकं दृष्ट्वा महावीर्यसमन्वितम्}
{ब्रह्मणोऽस्त्रसमाधत्त स्वे चापे सर्वशस्त्रवित्}% ४४

\twolineshloka
{तेन मुक्तं महाशस्त्रं प्रजज्वाल दिशो दश}
{खं रोदसी व्याप्य विश्वं प्रलयं कर्तुमुद्यतम्}% ४५

\twolineshloka
{चम्पको मुक्तमस्त्रं तद्दृष्ट्वा सर्वास्त्रकोविदः}
{तत्संहर्तुं तदेवास्त्रं मुमोच रिपुमुद्यतम्}% ४६

\twolineshloka
{द्वयोरेकतमं तेजः प्रलयं मेनिरे जनाः}
{सञ्जहार तदास्त्रास्त्रमेकीभूतं परास्त्रकम्}% ४७

\twolineshloka
{तत्कर्मचाद्भुतं दृष्ट्वा पुष्कलस्तिष्ठतिष्ठ च}
{ब्रुवञ्छरानमोघांस्तु चम्पकं स क्रुधाहनत्}% ४८

\twolineshloka
{चम्पकस्ताञ्छरान्मुक्तानगणय्य महामनाः}
{रामास्त्रं प्रमुमोचाथ पुष्कलं प्रति दारुणम्}% ४९

\twolineshloka
{तन्मुक्तमस्त्रमालोक्य चम्पकेन महात्मना}
{छेत्तुं यावन्मनश्चक्रे तावद्ग्रस्तः शरेण सः}% ५०

\twolineshloka
{बद्धश्चम्पकवीरेण रथे स्वे स्थापितः पुनः}
{पुरं प्रेषयितुं तावन्मनश्चक्रे महामनाः}% ५१

\twolineshloka
{हाहाकारो महानासीद्बद्धे पुष्कलसंज्ञिके}
{शत्रुघ्नं प्रययुर्योधाः पलायनपरायणाः}% ५२

\twolineshloka
{भग्नांस्तान्वीक्ष्य शत्रुघ्नो हनूमन्तमुवाच ह}
{केन वीरेण मे भग्नं बलं वीरैरलङ्कृतम्}% ५३

\twolineshloka
{तदोवाच महीनाथ पुष्कलं परवीरहा}
{बद्ध्वा नयति वीरोऽसौ चम्पकः स्वपदोद्धुरः}% ५४

\twolineshloka
{तस्येदृग्वाक्यमाकर्ण्य शत्रुघ्नः कोपसंयुतः}
{उवाच पवनोद्भूतं मोचयाशु नृपात्मजात्}% ५५

\twolineshloka
{महाबलः सुतश्चास्य बद्ध्वा यः पुष्कलं भटम्}
{तस्मान्मोचय वीराग्र्य कथं तिष्ठसि चाहवे}% ५६

\twolineshloka
{एतद्वाक्यं समाकर्ण्य हनूमानोमिति ब्रुवन्}
{जगाम तं मोचयितुं पुष्कलं चम्पकाद्भटात्}% ५७

\twolineshloka
{हनूमन्तमथालोक्य तं मोचयितुमागतम्}
{बाणैः शतैश्च साहस्रैर्जघान परकोपनः}% ५८

\twolineshloka
{बाणांस्तान्स बभञ्जाशु मुक्तांस्तेन महात्मना}
{पुनरप्येनमेवाशु बाणान्मुञ्चन्महानभूत्}% ५९

\twolineshloka
{तान्सर्वांश्चूर्णयामास नाराचान्वैरिमोचितान्}
{शालं करे समाधृत्य जघान नृपनन्दनम्}% ६०

\twolineshloka
{शालं तेन विनिर्मुक्तं तिलशः कृतवान्बली}
{गजो हनूमता मुक्तो नृपनन्दन मस्तके}% ६१

\twolineshloka
{सोऽप्याहतश्चम्पकेन मृतो भूमौ पपातसः}
{शिलाः सम्मोचयामास हनूमान्परमास्त्रवित्}% ६२

\twolineshloka
{चम्पकस्ताः शिलाः सर्वाः क्षणाच्चूर्णितवान्भृशम्}
{बाणयन्त्रिकया ब्रह्मन्महच्चित्रमभूदिदम्}% ६३

\twolineshloka
{स्वमुक्तास्ताः शिलाः सर्वाश्चूर्णिता वीक्ष्य मारुतिः}
{चुकोप हृदयेऽत्यतं बहुवीर्यमिति स्मरन्}% ६४

\twolineshloka
{आगत्य च करे धृत्वा नभस्युत्पतितः कपिः}
{तावद्ययौ नेत्रपथादुपरि क्षिप्रवेगवान्}% ६५

\twolineshloka
{चम्पकस्तं हनूमन्तं युयुधे नभसि स्थितः}
{बाहुयुद्धेन महता ताडितः कपिपुङ्गवः}% ६६

\twolineshloka
{चुकोप मानसे वीरो गर्वपर्वतदारुणः}
{पदा धृत्वा चम्पकं तं ताडयामास भूतले}% ६७

\twolineshloka
{ताडितोऽसौ कपीन्द्रेण क्षणादुत्थाय वेगवान्}
{हनूमन्तं तु लाङ्गूले धृत्वा बभ्राम सर्वतः}% ६८

\twolineshloka
{कपीन्द्रस्तद्बलं वीक्ष्य हसन्पादेऽग्रहीत्पुनः}
{भ्रामयित्वा शतगुणं गजोपस्थे ह्यपातयत्}% ६९

\twolineshloka
{पपात भूमौ सुबलो राजसूनुः स चम्पकः}
{मूर्च्छितो वीरभूषाढ्यमलङ्कुर्वन्रणाङ्गणम्}% ७०

\twolineshloka
{तदा हाहेति वै लोकाश्चुक्रुशुश्चम्पकानुगाः}
{पुष्कलं मोचयामास बद्धं चम्पकपाशतः}% ७१

{॥इति श्रीपद्मपुराणे पातालखण्डे शेषवात्स्यायनसंवादे रामाश्वमेधे पुष्कलमोचनं नामैकपञ्चाशत्तमोऽध्यायः॥५१॥}

\dnsub{द्विपञ्चाशत्तमोऽध्यायः}%\resetShloka

\uvacha{शेष उवाच}

\twolineshloka
{चम्पकं पतितं दृष्ट्वा सुरथः क्षत्रियो बली}
{पुत्रदुःखपरीताङ्गो जगाम स्यन्दने स्थितः}% १

\twolineshloka
{कपीन्द्रमाजुहावाथ सुरथः कोपसंयुतः}
{निःश्वासवेगं सम्मुञ्चन्महाबलसमन्वितः}% २

\twolineshloka
{आह्वयानं नृपं दृष्ट्वा निजं वीरः कपीश्वरः}
{जगाम तं महावीरो महावेगसमन्वितः}% ३

\twolineshloka
{तमागतं हनूमन्तं तृणीकुर्वं तमुद्भटान्}
{उवाच सुरथो राजा मेघगम्भीरसुस्वरः}% ४

\uvacha{सुरथ उवाच}

\twolineshloka
{धन्योसि कपिवर्य त्वं महाबलपराक्रमः}
{येन राममहत्कृत्यं कृतं राक्षसके पुरे}% ५

\twolineshloka
{त्वं रामचरणस्यासि सेवको भक्तिसंयुतः}
{त्वया वीरेण मत्पुत्रः पातितश्चम्पको बली}% ६

\twolineshloka
{इदानीं त्वां तु सम्बध्य गन्तास्मि नगरेमम}
{यत्नात्तिष्ठ कपीशेशसत्यमुक्तं मया स्मृतम्}% ७

\twolineshloka
{इति भाषितमाकर्ण्य सुरथस्य कपीश्वरः}
{उवाच धीरया वाण्या रणे वीरैकभूषिते}% ८

\uvacha{हनूमानुवाच}

\twolineshloka
{त्वं रामचरणस्मारी वयं रामस्य सेवकाः}
{बध्नासि चेन्मां प्रसभं मोचयिष्यति मत्प्रभुः}% ९

\twolineshloka
{कुरु वीर भवत्स्वान्तस्थितं सत्यं प्रतिश्रुतम्}
{रामं स्मरन्वै दुःखान्तं याति वेदा वदन्त्यदः}% १०

\uvacha{शेष उवाच}

\twolineshloka
{इति ब्रुवन्तं सुरथः प्रशस्य पवनात्मजम्}
{विव्याध बाणैर्बहुभिः शितैः शाणेन दारुणैः}% ११

\twolineshloka
{तान्मुक्तानगणय्याथ बाणाञ्छोणितपातिनः}
{करे जग्राह कोदण्डं सज्यं शरसमन्वितम्}% १२

\twolineshloka
{गृहीत्वा करयोश्चापं बभञ्ज कुपितः कपिः}
{चीत्कुर्वंस्त्रासयन्वीरान्नखैर्दीर्णान्सृजन्भटान्}% १३

\twolineshloka
{तेन भग्नं धनुर्दृष्ट्वा स्वकीयं गुणसंयुतम्}
{अपरं धनुरादत्त महद्गुणविशोभितम्}% १४

\twolineshloka
{तच्चापि जगृहे रोषात्कपिश्चापं बभञ्ज तत्}
{अन्यच्चापं समादत्त तद्बभञ्ज महाबलः}% १५

\twolineshloka
{तस्मिंश्चापे प्रभग्नेऽपि सोऽन्यद्धनुरुपाददत्}
{सोपि चापं बभञ्जाशु महावेगसमन्वितः}% १६

\twolineshloka
{एवं राज्ञस्तु चापानामशीतिर्विदलीकृता}
{क्षणे क्षणे महारोषात्कुर्वन्नादाननेकधा}% १७

\twolineshloka
{तदात्यन्तं प्रकुपितः शक्तिमुग्रामथाददे}
{शक्त्या स ताडितो वीरः पपात क्षणमुत्सुकः}% १८

\twolineshloka
{उत्थाय स्यन्दनं राज्ञो जग्राह कुपितो भृशम्}
{उड्डीनस्तं गृहीत्वा तु समुद्रमतिवेगतः}% १९

\twolineshloka
{तमुड्डीनं समालक्ष्य सुरथः परवीरहा}
{ताडयामास परिघैर्हृदि मारुतिमुद्यतम्}% २०

\twolineshloka
{मुक्तस्तेन रथो दूराच्चूर्णीभूतोऽभवत्क्षणात्}
{सोऽन्यरथं समारुह्य ययौ वेगात्समीरजम्}% २१

\twolineshloka
{हनूमांस्तद्रथं पुच्छे संवेष्ट्य प्रधनाङ्गणे}
{हयसारथिसंयुक्तं बभञ्ज सपताकिनम्}% २२

\twolineshloka
{अन्यं रथं समास्थाय ययौ राजा महाबलः}
{बभञ्ज तं रथं वेगान्मारुतिः कुपिताङ्गकः}% २३

\twolineshloka
{भग्नं तं स्यन्दनं वीक्ष्य सुरथोऽन्यसमाश्रितः}
{भग्नः स तेन सहसा हयसारथिसंयुतः}% २४

\twolineshloka
{एवमेकोनपञ्चाशद्रथा भग्ना हनूमता}
{तत्कर्म वीक्ष्य राजापि विसिस्माय ससैनिकः}% २५

\twolineshloka
{कुपितः प्राह कीशेन्द्रं धन्योसि पवनात्मज}
{पराक्रमन्निदं कर्म न कर्ता न करिष्यति}% २६

\twolineshloka
{क्षणमेकं प्रतीक्षस्व यावत्सज्यं धनुस्त्वहम्}
{करोमि पवनोद्भूत रामपादाब्जषट्पद}% २७

\twolineshloka
{इत्युक्त्वा चापमात्तज्यं कृत्वा रोषपरिप्लुतः}
{अस्त्रं पाशुपतं नाम सन्दधे शर उल्बणे}% २८

\twolineshloka
{ततो भूताश्च वेतालाः पिशाचा योगिनीमुखाः}
{प्रादुर्बभूवुः सहसा भीषयन्तः समीरजम्}% २९

\twolineshloka
{कपिः पाशुपतैरस्त्रैर्बद्धो लोकैरभीक्षितः}
{हाहेति च वदन्त्येते यावत्तावत्समीरजः}% ३०

\twolineshloka
{स्मृत्वा रामं स्वमनसा त्रोटयामास तत्क्षणात्}
{स मुक्तगात्रः सहसा युयुधे सुरथं नृपम्}% ३१

\twolineshloka
{तं मुक्तगात्रं संवीक्ष्य सुरथः परमास्त्रवित्}
{महाबलं मन्यमानो ब्राह्ममस्त्रं समाददे}% ३२

\twolineshloka
{मारुतिर्ब्राह्ममस्त्रं तु निजगाल हसन्बली}
{तन्निगीर्णं नृपो दृष्ट्वा रामं सस्मार भूमिपः}% ३३

\twolineshloka
{स्मृत्वा दाशरथिं रामं रामास्त्रं स्वशरासने}
{सन्धाय तं जगादेदं बद्धोसि कपिपुङ्गव}% ३४

\twolineshloka
{श्रुत्वा तत्प्रक्रमेद्यावत्तावद्बद्धो रणाङ्गणे}
{राज्ञा रामास्त्रतो वीरो हनूमान्रामसेवकः}% ३५

\twolineshloka
{उवाच भूपं हनुमान्किङ्करोमि महीभुज}
{मत्स्वाम्यस्त्रेण सम्बद्धो नान्येन प्राकृतेन वै}% ३६

\twolineshloka
{तन्मानयामिभूपालनयस्वस्वपुरम्प्रति}
{मोचयिष्यति मत्स्वामी आगत्य स दयानिधिः}% ३७

\twolineshloka
{बद्धे समीरजे क्रुद्धः पुष्कलो भूमिपं ययौ}
{तं पुष्कलं समायान्तं विव्याध वसुभिः शरैः}% ३८

\twolineshloka
{अनेकबाणसाहस्रैर्निजघान नृपं बली}
{राज्ञानेके शरास्तस्य च्छिन्नाः प्रधनमण्डले}% ३९

\twolineshloka
{एवं समरसङ्क्रुद्धे सुरथे पुष्कले तथा}
{बाणैर्व्याप्तं जगत्सर्वं स्थास्नुभूयश्चरिष्णु च}% ४०

\twolineshloka
{तेषां रणोद्यमं वीक्ष्य मुमुहुः सुरसैनिकाः}
{मानवानां तु का वार्ता क्षणात्त्रासं समीयुषाम्}% ४१

\twolineshloka
{अस्त्रप्रत्यस्त्रविगमैर्महामन्त्रपरिस्तुतैः}
{बभूव तुमुलं युद्धं वीराणां रोमहर्षणम्}% ४२

\twolineshloka
{तदा प्रकुपितो राजा नाराचं तु समाददे}
{छिन्नः स तु क्रुधा मुक्तैर्वत्सदन्तैः सभारतैः}% ४३

\twolineshloka
{छिन्ने तस्मिञ्छरे राजा कोपादन्यं समाददे}
{छिनत्ति यावत्स शरं तावल्लग्नो हृदि क्षतः}% ४४

\twolineshloka
{मूर्च्छां प्राप महातेजाः पुष्कलो महदद्भुतम्}
{युद्धं विधाय सुमहद्राज्ञा सह महामतिः}% ४५

\twolineshloka
{पुष्कले पतिते राजा शत्रुघ्नः शत्रुतापनः}
{सुरथं प्रति सङ्क्रुद्धो जगाम स्यन्दनस्थितः}% ४६

\twolineshloka
{उवाच सुरथं भूपं रामभ्राता महाबलः}
{त्वया महत्कृतं कर्म यद्बद्धः पवनात्मजः}% ४७

\twolineshloka
{पुष्कलोऽपि महावीरस्तथान्ये मम सैनिकाः}
{पातिताः प्रधने घोरे महाबलपराक्रमाः}% ४८

\twolineshloka
{इदानीं तिष्ठ मद्वीरान्पातयित्वा रणाङ्गणे}
{कुत्र यास्यसि भूमीश सहस्व मम सायकान्}% ४९

\twolineshloka
{इत्थमाश्रुत्य वीरस्य भाषितं सुरथो बली}
{जगाद रामपादाब्जं दधच्चेतसि शोभनम्}% ५०

\twolineshloka
{मया ते पातिताः सङ्ख्ये वीरा मारुतजोन्मुखाः}
{इदानीं पातयिष्यामि त्वामपि प्रधनाङ्गणे}% ५१

\twolineshloka
{स्मरस्व रामं यो वीरः स्वमागत्य प्ररक्षति}
{अन्यथा जीवितं नास्ति मत्पुरः शत्रुतापन}% ५२

\twolineshloka
{इत्युक्त्वा बाणसाहस्रैस्तं जघान महीपतिः}
{शत्रुघ्नं शरसङ्घातपञ्जरे न्यदधात्परम्}% ५३

\twolineshloka
{शत्रुघ्नः शरसङ्घातं मुञ्चन्तं वह्निदैवतम्}
{अस्त्रं मुमोच दाहार्थं शराणां नतपर्वणाम्}% ५४

\twolineshloka
{तदस्त्रं मुक्तमालोक्य राजा वै सुरथो महान्}
{वारुणास्त्रेण शमयन्विव्याध शरकोटिभिः}% ५५

\twolineshloka
{तदा तद्योगिनीदत्तमस्त्रं धनुषि सन्दधे}
{मोहनं सर्ववीराणां निद्राप्रापकमद्भुतम्}% ५६

\twolineshloka
{तन्मोहनं महास्त्रं स वीक्ष्य राजा हरिंस्मरन्}
{जगाद शत्रुघ्नमयं सर्वशस्त्रास्त्रकोविदः}% ५७

\twolineshloka
{मोहितस्य मम श्रीमद्रामस्य स्मरणेन ह}
{नान्यन्मोहनमाभाति ममापि भयतापदम्}% ५८

\twolineshloka
{इत्युक्तवति वीरे तु मुमोच स महास्त्रकम्}
{तेन बाणेन सञ्छिन्नं पपात रणमण्डले}% ५९

\twolineshloka
{तन्मोहनं महास्त्रं तु निष्फलं वीक्ष्य भूमिपे}
{अत्यन्तं विस्मयं प्राप्य बाणं धनुषि सन्दधे}% ६०

\twolineshloka
{लवणो येन निहतो महासुरविमर्दनः}
{तं बाणं चाप आधत्त घोरं कान्त्यानलप्रभम्}% ६१

\twolineshloka
{तं वीक्ष्य राजा प्रोवाच बाणोऽयमसतां हृदि}
{लगते रामभक्तस्य सम्मुखेऽपि न भात्यसौ}% ६२

\twolineshloka
{इत्येवं भाषमाणं तु बाणेनानेन शत्रुहा}
{विव्याध हृदये क्षिप्रं वह्निज्वालासमप्रभम्}% ६३

\twolineshloka
{तेन बाणेन दुःखार्तो महापीडासमन्वितः}
{रथोपस्थे क्षणं मूर्च्छामवाप परतापनः}% ६४

\twolineshloka
{स क्षणात्तां व्यथां तीर्त्वा जगाद रिपुमग्रतः}
{सहस्वैकं प्रहारं मे कुत्र यासि ममाग्रतः}% ६५

\twolineshloka
{एवमुक्त्वा महासङ्ख्ये बाणमाधत्त सायके}
{ज्वालामालापरीताङ्गं स्वर्णपुङ्खसमन्वितम्}% ६६

\twolineshloka
{स बाणो धनुषो मुक्तः शत्रुघ्नेन पथिस्थितः}
{छिन्नोऽप्यग्रफलेनाशु हृदये समपद्यत}% ६७

\twolineshloka
{तेन बाणेन सम्मूर्छ्य पपात स्यन्दनोपरि}
{ततो हाहाकृतं सैन्यं भग्नं सर्वं पराद्रवत्}% ६८

\twolineshloka
{सुरथो जयमापेदे सङ्ग्रामे रामसेवकः}
{दशवीरा दशसुतैर्मूर्च्छिताः पतिताः क्वचित्}% ६९

{॥इति श्रीपद्मपुराणे पातालखण्डे शेषवात्स्यायनसंवादे रामाश्वमेधे सुरथविजयो नाम द्विपञ्चाशत्तमोऽध्यायः॥५२॥}

\dnsub{त्रिपञ्चाशत्तमोऽध्यायः}%\resetShloka

\uvacha{शेष उवाच}

\twolineshloka
{सुग्रीवस्तु तत्कटकं भग्नं वीक्ष्य रणाङ्गणे}
{स्वामिनं मूर्च्छितं वापि ययौ योद्धुं नृपं प्रति}% १

\twolineshloka
{आगच्छ भूप सर्वान्नो मूर्च्छयित्वा कुतो भवान्}
{गच्छति क्षिप्रं मां देहि युद्धं रणविशारद}% २

\twolineshloka
{एवमुक्त्वा नगं कञ्चिद्विशालं शाखया युतम्}
{उत्पाट्य प्राहरत्तस्य मस्तके बलसंयुतः}% ३

\fourlineindentedshloka
{तेन प्रहारेण महाबलो नृपः}
{संवीक्ष्यसु ग्रीवमथो स्वचापे}
{बाणान्समाधाय शितान्सरोषा}
{ज्जघान वक्षस्यतिपौरुषो बली}% ४

\twolineshloka
{तान्बाणान्व्यधमत्सर्वान्सुग्रीवः सहसा हसन्}
{ताडयामास हृदये सुरथं सुमहाबलः}% ५

\twolineshloka
{पर्वतैः शिखरैश्चैव नगैर्द्विरदवर्ष्मभिः}
{वेगात्सन्ताडयामास दारयन्सुरथं नखैः}% ६

\twolineshloka
{तमप्याशु बबन्धास्त्राद्रामसंज्ञात्सुदारुणात्}
{बद्धः कपिवरो मेने सुरथं रामसेवकम्}% ७

\twolineshloka
{गजो यथायसमयीं शृङ्खलां पादलम्बिताम्}
{प्राप्य किञ्चिन्न वै कर्तुं शक्नोति स तथा ह्यभूत्}% ८

\twolineshloka
{जितं तेन महाराज्ञा सुरथेन सुपुत्रिणा}
{सर्वान्वीरान्रथे स्थाप्य ययौ स्वनगरं प्रति}% ९

\twolineshloka
{गत्वा सभायां सुमहान्बद्धं मारुतिमब्रवीत्}
{स्मर श्रीरघुनाथं त्वं दयालुं भक्तपालकम्}% १०

\twolineshloka
{यथा त्वां बन्धनात्सद्यो मोचयिष्यति सुष्ठुधीः}
{नान्यथायुतवर्षेण मोचयिष्यामि बन्धनात्}% ११

\fourlineindentedshloka
{इत्युक्तमाकर्ण्य समीरजस्तदा}
{सुबद्धमात्मानमवेक्ष्य वीरान्}
{सम्मूर्च्छिताञ्छत्रुशराभिघात-}
{पीडायुतान्बन्धनमुक्तये स्मरत्}% १२

\twolineshloka
{श्रीरामचन्द्रं रघुवंशजातं सीतापतिं पङ्कजपत्रनेत्रम्}
{स्वमुक्तये बन्धनतः कृपालुं सस्मार सर्वैः करणैर्विशोकैः}% १३

\uvacha{हनूमानुवाच}

\fourlineindentedshloka
{हा नाथ हा नरवरोत्तम हा दयालो}
{सीतापते रुचिरकुन्तलशोभिवक्त्र}
{भक्तार्तिदाहक मनोहररूपधारिन्}
{मां बन्धनात्सपदि मोचय मा विलम्बम्}% १४

\fourlineindentedshloka
{सम्मोचितास्तु भवता गजपुङ्गवाद्याः}
{देवाश्च दानवकुलाग्नि सुदह्यमानाः}
{तत्सुन्दरीशिरसिसंस्थितकेशबन्धः}
{सम्मोचितस्तु करुणालय मां स्मरस्व}% १५

\fourlineindentedshloka
{त्वं यागकर्मनिरतोऽसि मुनीश्वरेन्द्रै}
{र्धर्मं विचारयसि भूमिपतीड्यपाद}
{अत्राहमद्य सुरथेन विगाढपाश-}
{बद्धोस्मि मोचय महापुरुषाशु देव}% १६

\fourlineindentedshloka
{नो मोचयस्यथ यदि स्मरणातिरेकात्}
{त्वं सर्वदेववरपूजितपादपद्म}
{लोको भवन्तमिदमुल्लसितोऽहसिष्य-}
{त्तस्माद्विलम्बमिह माचर मोचयाशु}% १७

\twolineshloka
{इति श्रुत्वा जगन्नाथो रघुवीरः कृपानिधिः}
{भक्तं मोचयितुं प्रागात्पुष्पकेणाशुवेगिना}% १८

\twolineshloka
{लक्ष्मणेनानुगेनाथ भरतेन सुशोभितम्}
{मुनिवृन्दैर्व्यासमुख्यैः समेतं ददृशे कपिः}% १९

\twolineshloka
{तमागतं निजं नाथं वीक्ष्य भूपं समब्रवीत्}
{पश्य राजन्निजं मोक्तुमायातं कृपया हरिम्}% २०

\twolineshloka
{अनेके मोचिताः पूर्वं स्मरणात्सेवका निजाः}
{तथा मां पाशतो बद्धं सम्मोचयितुमागतः}% २१

\twolineshloka
{श्रीरामभद्रमायान्तं वीक्ष्यासौ सुरथः क्षणात्}
{नतीश्च शतशश्चक्रे भक्तिपूरपरिप्लुतः}% २२

\twolineshloka
{श्रीरामस्तं निजैर्दोर्भिः परिरेभे चतुर्भुजः}
{मूर्ध्नि सिञ्चन्नश्रुजलैर्हर्षाद्भक्तं स्वकं मुहुः}% २३

\twolineshloka
{उवाच धन्यदेहोऽसि महत्कर्म कृतं त्वया}
{कपीश्वरस्त्वया बद्धो हनूमान्सर्वतो बलः}% २४

\twolineshloka
{श्रीरामः कपिवर्यं तं मोचयामास बन्धनात्}
{मूर्छितांस्तान्भटान्सर्वान्वीक्ष्य दृष्ट्या स्वजीवयत्}% २५

\twolineshloka
{ते मूर्च्छां तत्यजुर्दृष्टा रामेण सुरसेविना}
{उत्थिता ददृशुः श्रीमद्रामचन्द्रं मनोरमम्}% २६

\twolineshloka
{प्रणतास्ते रघुपतिं तेन पृष्टा अनामयम्}
{सुखीभूता नृपं प्रोचुः सर्वं स्वकुशलं नृपाः}% २७

\twolineshloka
{सुरथो वीक्ष्य रामं च कृपार्थं सेवकात्मनः}
{आगतं सकलं राज्यं सहयं सुमुदार्पयत्}% २८

\threelineshloka
{अनेकवरिवस्याभिः श्रीरामं समतोषयत्}
{कथयामास मेऽन्याय्यं कृतं ते क्षम राघव}
{श्रीरामस्तमुवाचाथ कृतं ते वाहरक्षणम्}% २९

\twolineshloka
{क्षत्त्रियाणामयं धर्मः स्वामिना सह युद्ध्यते}
{त्वया साधुकृतं कर्म रणे वीराः प्रतोषिताः}% ३०

\twolineshloka
{इत्युक्तवन्तं नृहरिं पूजयन्ससुतोऽभवत्}
{श्रीरामस्त्रिदिनं स्थित्वा ययौ तमनुमन्त्र्य च}% ३१

\twolineshloka
{कामगेन विमानेन मुनिभिः सहितो महान्}
{तं दृष्ट्वा विस्मितास्तस्य कथाश्चक्रुर्मनोहराः}% ३२

\twolineshloka
{चम्पकं स्वपुरे स्थाप्य सुरथः क्षत्रियो बली}
{शत्रुघ्नेन समं यातुं मनश्चक्रे महाबलः}% ३३

\twolineshloka
{शत्रुघ्नः स्वहयं प्राप्य भेरीनादानकारयत्}
{शङ्खनादान्बहुविधान्सर्वत्र समवादयत्}% ३४

\twolineshloka
{सुरथेन समं वीरो यज्ञवाहममूमुचत्}
{स बभ्रामापरान्देशान्न कोपि जगृहे बली}% ३५

\twolineshloka
{यत्रयत्र गतो वाहस्तत्रतत्र परिभ्रमन्}
{सैन्येन महता यातः शत्रुघ्नः सुरथेन च}% ३६

\twolineshloka
{कदाचिज्जाह्नवीतीरे वाल्मीकेराश्रमं वरम्}
{गतो मुनिवरैर्जुष्टं प्रातर्धूमेन चिह्नितम्}% ३७

{॥इति श्रीपद्मपुराणे पातालखण्डे शेषवात्स्यायनसंवादे रामाश्वमेधे रघुनाथसमागमो नाम त्रिपञ्चाशत्तमोऽध्यायः॥५३॥}

\dnsub{चतुःपञ्चाशत्तमोऽध्यायः}%\resetShloka

\uvacha{शेष उवाच}

\twolineshloka
{गतः प्रातःक्रियां कर्तुं समिधस्तत्क्रियार्हकाः}
{आनेतुं जानकीसूनुर्वृतो मुनिसुतैर्लवः}% १

\twolineshloka
{ददर्श तत्र यज्ञाश्वं स्वर्णपत्रेण चिह्नितम्}
{कुङ्कुमागरुकस्तूरी दिव्यगन्धेन वासितम्}% २

\twolineshloka
{विलोक्य जातकुतुको मुनिपुत्रानुवाच सः}
{अर्वा कस्य मनोवेगः प्राप्तो दैवान्मदाश्रमम्}% ३

\twolineshloka
{आगच्छन्तु मया सार्धं प्रेक्षन्तां मा भयं कृथाः}
{इत्युक्त्वा स लवस्तूर्णं वाहस्य निकटे गतः}% ४

\twolineshloka
{स रराज समीपस्थो वाहस्य रघुवंशजः}
{धनुर्बाणधरः स्कन्धे जयन्त इव दुर्जयः}% ५

\twolineshloka
{गत्वा मुनिसुतैः सार्धं वाचयामास पत्रकम्}
{भालस्थितं स्पष्टवर्णराजिराजितमुत्तमम्}% ६

\twolineshloka
{विवस्वतो महान्वंशः सर्वलोकेषु विश्रुतः}
{यत्र कोपि पराबाधी न परद्रव्यलम्पटः}% ७

\twolineshloka
{सूर्यवंशध्वजो धन्वी धनुर्दीक्षा गुरुर्गुरुः}
{यं देवाः सामराः सर्वे नमन्ति मणिमौलिभिः}% ८

\twolineshloka
{तस्यात्मजो वीर बलदर्पहारी रघूद्वहः}
{रामचन्द्रो महाभागः सर्वशूरशिरोमणिः}% ९

\twolineshloka
{तन्माता कोशलनृपपुत्रीरत्नसमुद्भवा}
{तस्याः कुक्षिभवं रत्नं रामः शत्रुक्षयङ्करः}% १०

\twolineshloka
{करोति हयमेधं स ब्राह्मणेन सुशिक्षितः}
{रावणाभिधविप्रेन्द्र वधपापापनुत्तये}% ११

\twolineshloka
{मोचितस्तेन वाहानां मुख्योऽसौ याज्ञिको हयः}
{महाबलपरीवारो परिखाभिः सुरक्षितः}% १२

\twolineshloka
{तद्रक्षकोऽस्ति मद्भ्राता शत्रुघ्नो लवणान्तकः}
{हस्त्यश्वरथपादातसङ्घसेनासमन्वितः}% १३

\twolineshloka
{यस्य राज्ञ इति श्रेष्ठो मानो जायेत्स्वकान्मदात्}
{शूरा वयं धनुर्धारि श्रेष्ठा वयमिहोत्कटाः}% १४

\twolineshloka
{ते गृह्णन्तु बलाद्वाहं रत्नमालाविभूषितम्}
{मनोवेगं कामजवं सर्वगत्याधिभास्वरम्}% १५

\twolineshloka
{ततो मोचयिता भ्राता शत्रुघ्नो लीलया हठात्}
{शरासनविनिर्मुक्त वत्सदन्तकृतव्यथात्}% १६

\fourlineindentedshloka
{ये क्षत्रियाः क्षत्रियकन्यकायां}
{जाताश्च सत्क्षेत्रकुलेषु सत्सु}
{गृह्णन्तु ते तद्विपरीतदेहा}
{नमन्तु राज्यं रघवे निवेद्य}% १७

\twolineshloka
{इति संवाच्य कुपितो लवः शस्त्रधनुर्धरः}
{उवाच मुनिपुत्रांस्तान्रोषगद्गदभाषितः}% १८

\twolineshloka
{पश्यत क्षिप्रमेतस्य धृष्टत्वं क्षत्रियस्य वै}
{लिलेख यो भालपत्रे स्वप्रतापबलं नृपः}% १९

\twolineshloka
{कोऽसौ रामः कः शत्रुघ्नः कीटाः स्वल्पबलाश्रिताः}
{क्षत्रियाणां कुले जाता एते न वयमुत्तमाः}% २०

\twolineshloka
{एतस्य वीरसूर्माता जानकी न कुशप्रसूः}
{या रत्नं कुशसंज्ञं तु दधाराग्निमिवारणिः}% २१

\twolineshloka
{इदानीं क्षत्रियत्वादि दर्शयिष्यामि सर्वतः}
{यदि क्षत्रियभूरेष भविष्यति च शत्रुहा}% २२

\twolineshloka
{गृहीष्यति मया बद्धं वाहं यज्ञक्रियोचितम्}
{नोचेत्क्षत्रत्वमुन्मुच्य कुशस्य चरणार्चकः}% २३

\twolineshloka
{अधुना मद्धनुर्मुक्तैः शरैः सुप्तो भविष्यति}
{अन्ये ये च महावीरा रणमण्डलभूषणाः}% २४

\twolineshloka
{इत्यादिवाक्यमुच्चार्य लवो जग्राह तं हयम्}
{तृणीकृत्य नृपान्सर्वांश्चापबाणधरो वरः}% २५

\twolineshloka
{तदा मुनिसुताः प्रोचुर्लवं हयजिहीर्षकम्}
{अयोध्यानृपती रामो महाबलपराक्रमः}% २६

\twolineshloka
{तस्य वाहं न गृह्णाति शक्रोऽपि स्वबलोद्धतः}
{मा गृहाण शृणुष्वेदं मद्वाक्यं हितसंयुतम्}% २७

\twolineshloka
{इत्युक्तं स श्रुतौ धृत्वा जगाद स द्विजात्मजान्}
{यूयं बलं न जानीथ क्षत्रियाणां द्विजोत्तमाः}% २८

\twolineshloka
{क्षत्रिया वीर्यशौण्डीर्या द्विजा भोजनशालिनः}
{तस्माद्यूयं गृहे गत्वा भुञ्जन्तु जननी हृतम्}% २९

\twolineshloka
{इत्युक्तास्तेऽभवंस्तूष्णीं प्रेक्षन्तस्तत्पराक्रमम्}
{लवस्य मुनिपुत्रास्ते सन्तस्थुर्दूरतो बहिः}% ३०

\twolineshloka
{एवं व्यतिकरे वृत्ते सेवकास्तस्य भूपतेः}
{आयातास्तं हयं बद्धं दृष्ट्वा प्रोचुस्तदा लवम्}% ३१

\twolineshloka
{बबन्ध को हयमहो रुष्टः कस्य च धर्मराट्}
{को बाणव्रजमध्यस्थः प्राप्स्यते परमां व्यथाम्}% ३२

\twolineshloka
{तदा लवो जगादाशु मया बद्धोऽश्व उत्तमः}
{यो मोचयति तस्याशु रुष्टो भ्राता कुशो महान्}% ३३

\twolineshloka
{यमः करिष्यति किमु ह्यागतोऽपि स्वयं प्रभुः}
{नत्वा गमिष्यति क्षिप्रं शरवृष्ट्या सुतोषितः}% ३४

\uvacha{शेष उवाच}

\twolineshloka
{इति वाक्यं समाकर्ण्य बालोयमिति तेब्रुवन्}
{समागता मोचयितुं हयं बद्धं तु ये हरेः}% ३५

\twolineshloka
{तान्वैमोचयितुं प्राप्ताञ्छत्रुघ्नस्य च सेवकान्}
{कोदण्डं करयोर्धृत्वा क्षुरप्रान्सममूमुचत्}% ३६

\twolineshloka
{ते छिन्नबाहवः शोकाच्छत्रुघ्नं प्रतिसङ्गताः}
{पृष्टास्ते जगदुः सर्वे लवात्स्वभुजकृन्तनम्}% ३७

{॥इति श्रीपद्मपुराणे पातालखण्डे शेषवात्स्यायनसंवादे रामाश्वमेधे लवेन हयबन्धनं नाम चतुःपञ्चाशत्तमोऽध्यायः॥५४॥}

\dnsub{पञ्चपञ्चाशत्तमोऽध्यायः}%\resetShloka

\uvacha{व्यास उवाच}

\twolineshloka
{एतां श्रुत्वा कथां रम्यां लवस्य बलिनो मुनिः}
{संशयानः पर्यपृच्छन्नागं दशशताननम्}% १

\uvacha{श्रीवात्स्यायन उवाच}

\twolineshloka
{त्वयोक्तं तु पुरा रामः सीतामेकाकिनीं वने}
{रजकस्य दुरुक्त्यासौ तत्याज महि लोलुपः}% २

\twolineshloka
{जानक्यां क्व सुतौ जातौ क्व धनुर्धरतां गतौ}
{कथं च शिक्षिता विद्या यो रामहयमाहरत्}% ३

\uvacha{व्यास उवाच}

\twolineshloka
{इति श्रुत्वा मुनेर्वाक्यं शेषो नागो महामतिः}
{प्रशस्य विप्रं जगदे रामचारित्रमद्भुतम्}% ४

\uvacha{शेष उवाच}

\twolineshloka
{रामो राज्यमयोध्यायां भ्रातृभिः सहितोऽकरोत्}
{धर्मेण पालयन्सर्वं क्षितिखण्डं स्वया स्त्रिया}% ५

\twolineshloka
{सीता दधार तद्वीर्यं मासाः पञ्चाभवंस्तदा}
{अत्यन्तं शुशुभे देवी त्रयीव पुरुषं धरा}% ६

\twolineshloka
{कदाचित्समये रामः पप्रच्छ च विदेहजाम्}
{कीदृशो दोहदः साध्वि मया ते साध्यते हि सः}% ७

\twolineshloka
{रहस्येव तु सा पृष्टा त्रपमाणा पतिं सती}
{लज्जा गद्गद वाग्रामं निजगाद वचोऽमृतम्}% ८

\uvacha{सीतोवाच}

\twolineshloka
{त्वत्कृपातो मया सर्वं भुक्तं भोक्ष्यामि शोभनम्}
{न कश्चिन्मानसे कान्त विषयो ह्यतिरिच्यते}% ९

\twolineshloka
{यस्याभवादृशः स्वामी देवसंस्तुतसत्पदः}
{तस्याः सर्वं वरीवर्ति न किञ्चिदवशिष्यते}% १०

\twolineshloka
{त्वमाग्रहात्पृच्छसि मां दोहदं मनसि स्थितम्}
{ब्रवीमि पुरतः सत्यं तव स्वामिन्मनोहर}% ११

\twolineshloka
{चिरं जातं मया सत्यो लोपामुद्रादिकाः स्त्रियः}
{दृष्ट्वा स्वामिन्मनो द्रष्टुं ता उत्सुकति सुन्दरीः}% १२

\twolineshloka
{राज्यं प्राप्ता त्वया सार्द्धमनेकसुखमास्थिता}
{कृतघ्नाहं कदापीह ता नमस्कर्तुमानसा}% १३

\twolineshloka
{तत्र गत्वा तपःकोशान्वस्त्राद्यैः परिपूजये}
{रत्नानि चैव भास्वन्ति भूषा अपि समर्पये}% १४

\twolineshloka
{यथा मे तोषिताः सत्यो ददत्याशीर्मनोहराः}
{एष मे दोहदः कान्त परिपूरय मानसः}% १५

\twolineshloka
{इत्थमाकर्ण्य वचनं सीतायाः सुमनोहरम्}
{जगाद परमप्रीतो रामचन्द्रः प्रियां प्रति}% १६

\twolineshloka
{धन्यासि जानकी प्रातर्गमिष्यसि तपोधनाः}
{प्रेक्ष्यतास्तु कृतार्था त्वमागमिष्यसि मेऽन्तिकम्}% १७

\twolineshloka
{इति रामवचः श्रुत्वा परमां प्रीतिमाप सा}
{प्रातर्मम भवत्यद्धा तापसीनां समीक्षणम्}% १८

\twolineshloka
{अथ तन्निशि रामेण चाराः कीर्तिं निजां श्रुताम्}
{प्रेक्षितुं प्रेषितास्ते तु निशीथे ह्यगमनञ्छनैः}% १९

\twolineshloka
{ते प्रत्यहं रामकथाः शृण्वन्तः सुमनोहराः}
{तद्दिने गतवन्तस्तु धनाढ्यस्य गृहं महत्}% २०

\twolineshloka
{दीपं वीक्ष्य प्रज्वलन्तं वचनं वीक्ष्य मानुषम्}
{स्थितास्तत्र क्षणं चाराः समशृण्वन्यशो भृशम्}% २१

\twolineshloka
{तत्र काचन वामाक्षी बालकं प्रति हर्षिता}
{स्तनं धयन्तं निजगौ वाक्यं तु सुमनोहरम्}% २२

\twolineshloka
{पिब पुत्र यथेष्टं त्वं स्तन्यं मम मनोहरम्}
{पश्चात्तव सुदुष्प्रापं भविष्यति ममात्मज}% २३

\twolineshloka
{एतत्पुर्याः पती रामो नीलोत्पलदलप्रभः}
{तत्पुरीस्थजनानां तु न भविष्यति वै जनुः}% २४

\twolineshloka
{जन्माभावात्कथं पानं स्तन्यस्य भुवि जायते}
{तस्मात्पिब मुहुः स्तन्यं दुर्ल्लभं हृदि मन्य च}% २५

\twolineshloka
{ये श्रीरामं स्मरिष्यन्ति ध्यायन्ति च वदन्ति ये}
{तेषामपि पयःपानं न भविष्यति जातुचित्}% २६

\twolineshloka
{इत्यादिवाक्यं संश्रुत्य श्रीरामयशसोऽमृतम्}
{हर्षिताः प्रययुर्गेहमन्यद्भाग्यवतो महत्}% २७

\twolineshloka
{तावदन्यश्चरस्तत्र मनोरममिदं गृहम्}
{मत्वा तिष्ठन्हि रामस्य क्षणं शुश्रूषया यशः}% २८

\twolineshloka
{तत्र काचिन्निजं कान्तं पर्यङ्कोपरि सुस्थितम्}
{ताम्बूलं चर्वती दत्तं भर्त्तास्नेहेन सुन्दरी}% २९

\twolineshloka
{कङ्कणस्वरशोभाढ्या कर्पूरागरुधूपिता}
{कान्तं वीक्ष्य चलन्नेत्रा कामरूपमवोचत}% ३०

\twolineshloka
{नाथ त्वं तादृशो मह्यं भासि यादृग्रघोः पतिः}
{अत्यन्तं सुन्दरतरं वपुर्बिभ्रत्सुकोमलम्}% ३१

\twolineshloka
{पद्मप्रान्तं नेत्रयुग्मं वक्षो मोहनविस्तृतम्}
{भुजौ च साङ्गदौ बिभ्रत्साक्षाद्राम इवासि मे}% ३२

\twolineshloka
{इति वाक्यं समाकर्ण्य कान्तायाः सुमनोहरम्}
{उवाच नेत्रयोः प्रान्तं नर्तयन्कामसुन्दरः}% ३३

\twolineshloka
{शृणु कान्ते त्वया प्रोक्तं साध्व्या तु सुमनोहरम्}
{पतिव्रतानां तद्योग्यं स्वकान्तो राम एव हि}% ३४

\twolineshloka
{परं क्वाहं मन्दभाग्यः क्व रामो भाग्यवान्महान्}
{क्व चाहं कीटवत्तुच्छः क्व ब्रह्मादिसुरार्चितः}% ३५

\twolineshloka
{खद्योतः क्व नभोरत्नं शलभः क्व नु पामरः}
{गजारिः क्व मृगेन्द्रोऽसौ शशकः क्व नु मन्दधीः}% ३६

\twolineshloka
{क्व च सा जाह्नवी देवी क्व रथ्या जलमुत्पथम्}
{क्व मेरुः सुरसंवासः क्व गुञ्जापुञ्जकोल्पकः}% ३७

\twolineshloka
{तथाहं क्व क्व रामोऽसौ यत्पादरजसाङ्गना}
{शिलीभूता क्षणाज्जाता ब्रह्ममोहनरूपधृक्}% ३८

\twolineshloka
{इति वाक्यं प्रब्रुवाणं परिरेभे निजं पतिम्}
{जातकामा हृतप्रेम्णा नर्तित भ्रू धनुर्धरा}% ३९

\twolineshloka
{इत्यादि वाक्यं संश्रुत्य गतश्चान्यनिवेशनम्}
{तावदन्यश्चरो वाक्यं शुश्राव यशसान्वितम्}% ४०

\twolineshloka
{काचित्पुष्पमयीं शय्यां चन्दनं सह चन्द्रकम्}
{सर्वं विधाय कामार्हं जगाद वचनं पतिम्}% ४१

\twolineshloka
{पते कुरुष्व भोगार्हे शयनं पुष्पमञ्चके}
{चन्दनादिकलेपं च तथा भोगमनेकधा}% ४२

\twolineshloka
{त्वादृशा एव भोगार्हा न च रामपराङ्मुखाः}
{सर्वं रामकृपाप्राप्तमुपभुङ्क्ष्व यथातथम्}% ४३

\twolineshloka
{मत्सदृशी कामिनी ते चन्दनं तापहारकम्}
{पर्यङ्कः पुष्परचितः सर्वं रामकृपाभवम्}% ४४

\twolineshloka
{ये रामं न भजिष्यन्ति ते नरा जठरं स्वयम्}
{न भर्तुं शक्नुवन्त्येव वस्त्रभोगादि वर्जिताः}% ४५

\twolineshloka
{इति ब्रुवन्तीं महिलां हर्षितः पतिरब्रवीत्}
{सर्वं तथ्यं ब्रवीषि त्वं मम रामकृपाभवम्}% ४६

\twolineshloka
{इत्येवं रामभद्रस्य यशः श्रुत्वा गतश्चरः}
{तावदन्यस्य वेश्मस्थश्चरोऽन्य शुश्रुवे वचः}% ४७

\twolineshloka
{काचित्कान्तेन पर्यङ्के वीणावादनतत्परा}
{कान्तेन रामसत्कीर्तिं गायमाना पतिं जगौ}% ४८

\twolineshloka
{स्वामिन्वयं धन्यतमा येषां पुर्याः पतिः प्रभुः}
{श्रीरामः स्वप्रजाः पुत्रानिव पाति च रक्षकः}% ४९

\twolineshloka
{यो महत्कर्मदुःसाध्यं कृतवान्सुलभं न तत्}
{समुद्रं यो निजग्राह सेतुं तत्र बबन्ध च}% ५०

\twolineshloka
{रावणं यो रिपुं हत्वा लङ्कां सम्भज्य वानरैः}
{जानकीमाजहारात्र महदाचारमाचरत्}% ५१

\twolineshloka
{इति प्रोक्तं समाकर्ण्य वचः सुमधुराक्षरम्}
{पतिः स्मितं चकारेमां वाक्यं पुनरथाब्रवीत्}% ५२

\twolineshloka
{मुग्धेनेदं महत्कर्म रामचन्द्रस्य भामिनी}
{दशाननवधादीनि समुद्र दमनानि च}% ५३

\twolineshloka
{लीलयायोऽवनिं प्राप्तो ब्रह्मादिप्रार्थितो महान्}
{करोति सच्चरित्राणि महापापहराणि च}% ५४

\twolineshloka
{मा जानीहि नरं रामं कौसल्यानन्ददायकम्}
{सृजत्यवति हन्त्येतद्विश्वं लीलात्तमानुषः}% ५५

\twolineshloka
{धन्या वयं ये रामस्य पश्यामो मुखपङ्कजम्}
{ब्रह्मादिसुरदुर्दर्शं महत्पुण्यकृतो वयम्}% ५६

\twolineshloka
{अशृणोद्रामचन्द्रस्य चरित्रं श्रुतिसौख्यदम्}
{इत्यादिवाक्यं शुश्राव चारो द्वारिस्थितो मुहुः}% ५७

\twolineshloka
{अन्यो ह्यन्यं गृहं गत्वा तस्थौ श्रोतुं हरेर्यशः}
{तत्रापि रामभद्रस्य यशः शुश्राव शोभनम्}% ५८

\twolineshloka
{खेलन्ती स्वामिना सार्धं द्यूतेन सुमनोहरा}
{उवाच वाक्यं मधुरं नर्तयन्तीव कङ्कणे}% ५९

\fourlineindentedshloka
{जितं मया कान्त जवेन सर्वं}
{करिष्यसि त्वं किमु हारिमानसः}
{इत्यादि वाक्यं परिहासपूर्वकं}
{कृत्वा स्वकान्तं परिषस्वजे मुदा}% ६०

\twolineshloka
{उवाच कान्तश्चार्वङ्गि जितमेव सुशोभने}
{रामं मे स्मरतो नित्यं न कुत्रापि पराजयः}% ६१

\twolineshloka
{इदानीं त्वां तु जेष्यामि रामं स्मृत्वा मनोहरम्}
{देवा यथा पुरा स्मृत्वा दितिजानजयन्क्षणात्}% ६२

\twolineshloka
{एवमुक्त्वा पाशकानां परिवर्तनमाकरोत्}
{तावज्जयं प्रपेदेऽसौ हर्षितो वाक्यमब्रवीत्}% ६३

\twolineshloka
{मम प्रोक्तमृतं जातं जिता त्वं नवयौवना}
{रामस्मारी कदाप्येव न भवेद्रिपुतो भयी}% ६४

\twolineshloka
{इत्येवं तौ वदन्तौ च परस्परमथोत्सुकौ}
{परिरभ्य दृढं प्रेम्णा ततश्चारो गतो गृहम्}% ६५

\twolineshloka
{एवं पञ्चमहाचारा राज्ञः संश्रुत्य वै यशः}
{परस्परं प्रशंसन्तो गेहं स्वं स्वं ययुर्मुदा}% ६६

\twolineshloka
{एकः षष्ठश्चरः कारुगेहानालोक्य तत्र ह}
{जगाम श्रोतुकामोऽसौ यशो राज्ञो महीपतेः}% ६७

\twolineshloka
{रजकः क्रोधसंस्पृष्टो भार्यामन्यगृहोषिताम्}
{पदा सन्ताडयामास धिक्कुर्वञ्छोणनेत्रवान्}% ६८

\twolineshloka
{गच्छ त्वं मद्गृहात्तस्य गेहं यत्रोषिता दिनम्}
{नाहं गृह्णामि भवतीं दुष्टां वचनलङ्घिनीम्}% ६९

\twolineshloka
{तदास्य माता प्रोवाच मा त्यजैनां गृहागताम्}
{अपराधेन रहितां दुष्टकर्मविवर्जिताम्}% ७०

\twolineshloka
{मातरं प्रत्युवाचाथ रजकः क्रोधसंयुतः}
{नाहं रामइव प्रेष्ठां गृह्णाम्यन्यगृहोषिताम्}% ७१

\twolineshloka
{स राजा यत्करोत्येव तत्सर्वं नीतिमद्भवेत्}
{अहं गृह्णामि नो भार्यां परवेश्मनि संस्थिताम्}% ७२

\twolineshloka
{पुनःपुनरुवाचेदं नाहं रामो महीश्वरः}
{यः परस्य गृहे संस्थां जानकीं वै ररक्ष सः}% ७३

\twolineshloka
{इति वाक्यं समाश्रुत्य चारः क्रोधपरिप्लुतः}
{खड्गं गृहीत्वा स्वकरे तं हन्तुं विदधे मनः}% ७४

\twolineshloka
{स रामोक्तं च सस्मार न वध्यः कोपि मे जनः}
{इति ज्ञात्वा सरोषं तु सञ्जहार महामनाः}% ७५

\twolineshloka
{तदा श्रुत्वा सुदुःखार्तः पञ्चचारा यतः स्थिताः}
{ततो गतः प्रकुपितो निःश्वसन्मुहुरुच्छ्वसन्}% ७६

\twolineshloka
{ते वै परस्परं तत्र मिलितास्तु समब्रुवन्}
{स्वश्रुतं रामचरितं सर्वलोकैकपूजितम्}% ७७

\twolineshloka
{ते तद्भाषितमाकर्ण्य परस्परममन्त्रयन्}
{न वाच्यं रघुनाथाया वाच्यं दुष्टजनोदितम्}% ७८

\twolineshloka
{इति सम्मन्त्र्य ते गेहं गत्वा सुषुपुरुत्सुकाः}
{प्राता राज्ञे प्रशंसाम इति बुद्ध्या व्यवस्थिताः}% ७९

{॥इति श्रीपद्मपुराणे पातालखण्डे शेषवात्स्यायनसंवादे रामाश्वमेधे चारनिरीक्षणं नाम पञ्चपञ्चाशत्तमोऽध्यायः॥५५॥}

\dnsub{षट्पञ्चाशत्तमोऽध्यायः}%\resetShloka

\uvacha{शेष उवाच}

\twolineshloka
{प्रातर्नित्यं विधायासौ ब्राह्मणान्वेदवित्तमान्}
{हिरण्यदानैः सन्तर्प्य विधिवत्संसदं ययौ}% १

\twolineshloka
{लोकाः सर्वे नमस्कर्तुं रघुनाथं महीपतिम्}
{पुत्रवत्स्वप्रजाः सर्वाः पालयन्तं ययुः सभाम्}% २

\twolineshloka
{लक्ष्मणेनातपत्रं तु धृतं मूर्धनि भूपतेः}
{तदा भरतशत्रुघ्नौ चामरद्वन्द्व धारिणौ}% ३

\twolineshloka
{वसिष्ठप्रमुखास्तत्र मुनयः पर्युपासत}
{सुमन्त्रप्रमुखास्तत्र मन्त्रिणो न्यायकर्तृकाः}% ४

\twolineshloka
{एवं प्रवृत्ते समये षट्चारास्ते स्वलङ्कृताः}
{समाजग्मुर्नरपतिं नमस्कर्तुं सभास्थितम्}% ५

\twolineshloka
{तान्वक्तुकामान्संवीक्ष्य चारान्नृपतिसत्तमः}
{सभायामन्तरावेश्म रहः प्राविशदुत्सुकः}% ६

\twolineshloka
{एकान्ते तांश्चरान्सर्वान्पप्रच्छ सुमतिर्नृपः}
{कथयन्तु चरा मह्यं यथातथ्यमरिन्दमाः}% ७

\twolineshloka
{लोका ब्रुवन्ति मां कीदृग्भार्याया मम कीदृशम्}
{मन्त्रिणां कीदृशं लोका वदन्ति चरितं कथम्}% ८

\twolineshloka
{इति वाक्यं समाकर्ण्य चारा राममथाब्रुवन्}
{मेघगम्भीरया वाचा पृच्छन्तं रघुनायकम्}% ९

\uvacha{चारा ऊचुः}

\twolineshloka
{नाथ कीर्तिर्जनान्सर्वान्पावयत्यधुना भुवि}
{गृहेगृहे श्रुतास्माभिः पुरुषैः स्त्रीभिरीडिता}% १०

\twolineshloka
{विवस्वतो महान्वंशो भवता परमेष्ठिना}
{अलङ्कर्तुं गतं भूमौ कीर्तिर्विस्तारिता बहुः}% ११

\twolineshloka
{अनेके सगराद्याश्च कीर्तिमन्तो महाबलाः}
{अभवंस्तादृशी कीर्तिस्तेषां नाभूद्यथा तव}% १२

\twolineshloka
{त्वया नाथेन सकलाः कृतार्थाश्च प्रजाः कृताः}
{यासां नाकालमरणं न च रागाद्युपद्रुतिः}% १३

\twolineshloka
{यादृशश्चन्द्रमालोके यादृशी जाह्नवी सरित्}
{तादृक्तव च सत्कीर्तिः प्रकाशयति भूतलम्}% १४

\twolineshloka
{ब्रह्मादिका भवत्कीर्तिमाकर्ण्य त्रपिता भृशम्}
{नाथ सर्वत्र ते कीर्तिः पावयत्यधुना जनान्}% १५

\twolineshloka
{वयं धन्यतमाः सर्वे ये चारास्तव भूपते}
{क्षणेक्षणे तव मुखं लोकयाम मनोहरम्}% १६

\twolineshloka
{इत्यादिवाक्यं चाराणां पञ्चानां वीक्ष्य राघवः}
{षष्ठं पप्रच्छ चारं तं विलक्षणमुखाङ्कितम्}% १७

\uvacha{राम उवाच}

\twolineshloka
{सत्यं वद महाबुद्धे यच्छ्रुतं लोकसङ्करे}
{तादृक्प्रशंस मे सर्वमन्यथा पातकादिकृत्}% १८

\twolineshloka
{पुनः पुनश्च तं रामः पप्रच्छाशु सविस्तरम्}
{तथापि न ब्रवीत्येव रामं लौकिकभाषितम्}% १९

\twolineshloka
{तदा रामः प्रत्यवोचच्चारं मुखविलक्षितम्}
{शपामि त्वां तु सत्येन शंस सर्वं यथातथम्}% २०

\twolineshloka
{तदा रामं प्रत्युवाच चारो वाक्यं शनैः शनैः}
{अकथ्यमपि ते वाच्यं वाक्यं कारुजनोदितम्}% २१

\uvacha{चार उवाच}

\twolineshloka
{स्वामिन्सर्वत्र ते कीर्तिर्दशाननवधादिका}
{अन्यत्र राक्षसगृहे स्थितायास्ते स्त्रिया अहो}% २२

\twolineshloka
{कारुरेकस्तु रजको निशीथे महिलां स्वकाम्}
{अन्यगेहोषितां दृष्ट्वा धिक्कुर्वन्समताडयत्}% २३

\twolineshloka
{तन्माता प्रत्युवाचेमां कथं ताडयसेऽनघाम्}
{गृहाण मा कृथा निन्दां स्त्रियं मद्वाक्यमाचर}% २४

\twolineshloka
{तदावोचत्स रजको नाहं रामो महीपतिः}
{यद्राक्षसगृहेध्युष्टां सीतामङ्गीचकार सः}% २५

\twolineshloka
{सर्वं राज्ञः कृतं कर्म नीतिमद्भवति प्रभो}
{अन्येषां पुण्यकर्तॄणामपि कृत्यमनीतिमत्}% २६

\twolineshloka
{पुनः पुनरुवाचासौ नाहं रामो महीपतिः}
{चुक्रुधे समये राजन्मया वाक्यं तव स्मृतम्}% २७

\twolineshloka
{तदानीं शिर आच्छिद्य पातयामि महीतले}
{कृतः पुनर्विचारोमे क्व रामो रजकः क्व नु}% २८

\twolineshloka
{अयं दुष्टोऽनृतं वक्ति न हीदं तथ्यमुच्यते}
{आज्ञापयसि चेद्राम साम्प्रतं मारयामि तम्}% २९

\twolineshloka
{अवाच्यमपि ते प्रोक्तं त्वदाग्रहत उन्नयम्}
{राजा प्रमाणमत्रेदं विचारयतु सङ्गतम्}% ३०

\uvacha{शेष उवाच}

\twolineshloka
{इति वाक्यं समाकर्ण्य महावज्रनिभाक्षरम्}
{निःश्वसन्मुहुरुच्छ्वासमाचरन्मूर्च्छितोऽपतत्}% ३१

\twolineshloka
{तं मूर्च्छितं नृपं दृष्ट्वा चारा दुःखसमन्विताः}
{वीजयामासुर्वासोग्रैर्दुःखापनय हेतवे}% ३२

\twolineshloka
{स लब्धसंज्ञो नृपतिर्मुहूर्तेन जगाद तान्}
{गच्छन्तु भरतं शीघ्रं प्रेषयन्तु च मां प्रति}% ३३

\twolineshloka
{ते दुःखिताश्चरास्तूर्णं भरतस्य गृहं गताः}
{कथयामासु रामस्य सन्देशं नयहारकाः}% ३४

\twolineshloka
{भरतो रामसन्देशं श्रुत्वा धीमान्ययौ सदः}
{रामं प्रति रहःसंस्थं श्रुत्वा तं त्वरया युतः}% ३५

\twolineshloka
{आगत्य तं प्रतीहारं प्रत्युवाच महामनाः}
{कुत्रास्ते रामभद्रोऽसौ मम भ्राता कृपानिधिः}% ३६

\twolineshloka
{तन्निर्दिष्टं गृहं वीरो ययौ रत्नमनोहरम्}
{रामं विलोक्य विक्लान्तं भयमाप स मानसे}% ३७

\twolineshloka
{किं वासौ कुपितो रामः किं वा दुःखमिदं विभोः}
{तदा प्रोवाच नृपतिं निःश्वसन्तं मुहुर्मुहुः}% ३८

\twolineshloka
{स्वामिन्सुखसमाराध्यं वक्त्रं ते कथमानतम्}
{अश्रुभिर्लक्ष्यते राहुग्रस्तदेहः शशीव ते}% ३९

\twolineshloka
{सर्वं मे कारणं तथ्यं ब्रूहि मां किं करोमि ते}
{त्यज दुःखं महाराज कथं दुःखस्य भाजनम्}% ४०

\twolineshloka
{एवं भ्रात्रा प्रोच्यमानो गद्गदस्वरया गिरा}
{प्रोवाच भ्रातरं वीरो रामचन्द्रश्च धार्मिकः}% ४१

\twolineshloka
{शृणु भ्रातर्वचो मह्यं मम दुःखस्य कारणम्}
{तन्मार्जनं कुरुष्वाद्य भ्रातः प्रातर्महामते}% ४२

\twolineshloka
{वंशे वैवस्वते राजा न कश्चिदयशः क्षतः}
{मत्कीर्तिरद्य कलुषा गङ्गायमुनया गता}% ४३

\twolineshloka
{येषां यशो नृणां भूमौ तेषामेव सुजीवितम्}
{अपकीर्तिक्षतानां तु जीवितं मृतकैः समम्}% ४४

\twolineshloka
{येषां यशो भवेद्भूमौ तेषां लोकाः सनातनाः}
{अपकीर्त्युरगी दष्टास्तेषां भूयादधोगतिः}% ४५

\twolineshloka
{अद्य मे कलुषा कीर्तिः स्वर्धुनी लोकविश्रुता}
{तच्छृणुष्व वचो मेऽद्य रजकेन यथोदितम्}% ४६

\twolineshloka
{अस्मिन्पुरेऽद्य रजक उक्तवाञ्जानकीभवम्}
{किञ्चिद्वाच्यं ततो भ्रातः किं करोमि महीतले}% ४७

\twolineshloka
{किमात्मानं जहाम्यद्य किमेनां जानकीं स्त्रियम्}
{उभयोः किं मया कार्यं तत्तथ्यं ब्रूहि मे भवान्}% ४८

\twolineshloka
{इत्युक्त्वा निर्गलद्बाष्पो वेपथु क्षुभिताङ्गकः}
{पपात भूमौ विरजो धार्मिकाणां शिरोमणिः}% ४९

\twolineshloka
{भ्रातरं पतितं दृष्ट्वा भरतो दुःखसंयुतः}
{संवीक्ष्य शनकै रामं प्राप्तसंज्ञं चकार सः}% ५०

\twolineshloka
{संज्ञां प्राप्तं तु संवीक्ष्य रामचन्द्रं सुदुःखितम्}
{उवाच दुःखनाशाय वाक्यं तु सुमनोहरम्}% ५१

\twolineshloka
{कोऽयं वै रजकः किन्तु वाच्यं वाक्यं यथाब्रवीत्}
{जिह्वाच्छेदं करिष्यामि जानकीवाच्यकारिणः}% ५२

\twolineshloka
{तदा रामोऽब्रवीद्वाक्यं रजकस्य मुखोद्गतम्}
{श्रुतं चारेण तत्सर्वं भरताय महात्मने}% ५३

\twolineshloka
{तच्छ्रुत्वा भरतः प्राह भ्रातरं दुःखशोकिनम्}
{जानकीवह्निशुद्धाभूल्लङ्कायां वीरपूजिता}% ५४

\twolineshloka
{ब्रह्माब्रवीदियं शुद्धा पिता दशरथस्तव}
{कथं सा रजकोक्तित्वाद्धातव्या लोकपूजिता}% ५५

\twolineshloka
{ब्रह्मादिसंस्तुता कीर्तिस्तवलोकान्पुनाति हि}
{सा कथं रजकोक्त्या वै कलुषाद्य भविष्यति}% ५६

\twolineshloka
{तस्मात्त्यज महादुःखं सीतावाच्यसमुद्भवम्}
{कुरु राज्यं तया सार्धमन्तर्वत्न्या सुभाग्यया}% ५७

\twolineshloka
{त्वं कथं स्वशरीरं तु हातुमिच्छसि शोभनम्}
{वयं हताः स्म सर्वेऽद्य त्वां विना दुःखनाशकम्}% ५८

\twolineshloka
{क्षणं सीता न जीवेत त्वां विना सुमहोदया}
{तस्मात्पतिव्रता साकं भुनक्तु विपुलां श्रियम्}% ५९

\twolineshloka
{इति वाक्यं समाकर्ण्य भरतस्य च धार्मिकः}
{पुनरेव जगादेमं वाक्यं वाक्यविदां वरः}% ६०

\twolineshloka
{यत्त्वं कथयसि भ्रातस्तत्सर्वं धर्मसंयुतम्}
{परं यद्वच्म्यहं वाक्यं तत्कुरुष्व ममाज्ञया}% ६१

\twolineshloka
{जानाम्येनां वह्निशुद्धां पवित्रां लोकपूजिताम्}
{लोकापवादाद्भीतोऽहं त्यजामि स्वां तु जानकीम्}% ६२

\twolineshloka
{तस्मात्करे शितं धृत्वा करवालं सुदारुणम्}
{शिरश्छिन्ध्यथवा जायां जानकीं मुञ्च वै वने}% ६३

\twolineshloka
{इति वाक्यं समाकर्ण्य रामस्य भरतोऽपतत्}
{मूर्च्छितः सन्क्षितौ देहे कम्पयुक्तः सबाष्पकः}% ६४

{॥इति श्रीपद्मपुराणे पातालखण्डे शेषवात्स्यायनसंवादे रामाश्वमेधे भरतवाक्यं नाम षट्पञ्चाशत्तमोऽध्यायः॥५६॥}

\dnsub{सप्तपञ्चाशत्तमोऽध्यायः}%\resetShloka

\uvacha{वात्स्यायन उवाच}

\twolineshloka
{जगत्पवित्रसत्कीर्ति जानक्या वाच्यवाचनम्}
{कथं समकरोत्स्वामी तन्मे कथय सुव्रत}% १


\threelineshloka
{यथा मे मनसः सौख्यं भविष्यति सुशोभनम्}
{तथा कुरुष्व शेषाद्य त्वन्मुखान्निःसृतामृतम्}
{पिबतस्तृप्तिरेव स्याद्यया संसृतिकृं तनम्}% २

\uvacha{शेष उवाच}

\twolineshloka
{मिथिलायां महापुर्यां जनको नाम भूपतिः}
{तस्यां करोति सद्राज्यं धर्मेणाराधयन्प्रजाः}% ३

\twolineshloka
{तस्य सङ्कर्षतो भूमिं सीतया दीर्घमुख्यया}
{सीरध्वजस्य निरगात्कुमारी ह्यतिसुन्दरी}% ४

\twolineshloka
{तदात्यन्तं मुदं प्राप्तः सीरकेतुर्महीपतिः}
{सीता नामाकरोत्तस्या मोहिन्या जगतः श्रियः}% ५

\twolineshloka
{सैकदोद्यानविपिने खेलन्ती सुमनोहरा}
{अपश्यत्स्वमनःकान्तं शुकशुक्योर्युगं वदत्}% ६

\twolineshloka
{अत्यन्तं हर्षमापन्नमत्यन्तं कामलोलुपम्}
{परस्परं भाषमाणं स्नेहेन शुभया गिरा}% ७

\twolineshloka
{रममाणं तदा युग्मं नभसि क्षिप्रवेगतः}
{समुत्पतन्नगोपस्थे स्थितं शब्दं चकार तत्}% ८

\twolineshloka
{रामो महीपतिर्भूमौ भविष्यति मनोहरः}
{तस्य सीतेति नाम्ना तु भविष्यति महेलिका}% ९

\twolineshloka
{स तया सह वर्षाणां सहस्राण्येकयुग्दश}
{राज्यं करिष्यते धीमान्कर्षन्भूमिपतीन्बली}% १०

\twolineshloka
{धन्या सा जानकी देवी धन्योऽसौ रामसंज्ञितः}
{यौ परस्परमापन्नौ पृथिव्यां रंस्यतो मुदा}% ११

\twolineshloka
{इति सम्भाष्यमाणं तु शुकयुग्मं तु मैथिली}
{ज्ञात्वेदं देवतायुग्मं वाणीं तस्य विलोक्य च}% १२

\twolineshloka
{मदीयास्तु कथा रम्याः कुरुते शुकयोर्युगम्}
{एतद्गृहीत्वा पृच्छामि सर्वं वाक्यं गतार्थकम्}% १३

\twolineshloka
{एवं विचार्य सा स्वीयाः सखीः प्रतिजगाद ह}
{गृह्णन्तु शनकैरेतत्पक्षियुग्मं मनोहरम्}% १४

\twolineshloka
{सख्यस्तास्तद्गिरिं गत्वा गृह्णन्पक्षियुगं वरम्}
{निवेदयामासुरिदं स्वसख्याः प्रियकाम्यया}% १५

\twolineshloka
{बहुधा विविधाञ्छब्दान्कुर्वद्वीक्ष्य मनोहरम्}
{आश्वासयामास तदा पप्रच्छ तदिदं वचः}% १६

\uvacha{सीतोवाच}

\twolineshloka
{मा भैषातां युवां रम्यौ कौ वां कुत्र समागतौ}
{को रामः का च सा सीता तज्ज्ञानं तु कुतः स्मृतम्}% १७

\twolineshloka
{तत्सर्वं शंसतं क्षिप्रं मत्तो वां व्येतु यद्भयम्}
{इति पृष्टं तया पक्षियुगं सर्वमशंसत}% १८

\uvacha{पक्षियुग्ममुवाच}

\twolineshloka
{वाल्मीकिरास्ते सुमहानृषिर्धर्मविदुत्तमः}
{आवां तदाश्रमस्थाने सर्वदा सुमनोहरे}% १९

\twolineshloka
{सशिष्यान्गापयामास भावि रामायणं मुनिः}
{प्रत्यहं तत्पदस्मारी सर्वभूतहिते रतः}% २०

\twolineshloka
{तदावाभ्यां श्रुतं सर्वं भावि रामायणं महत्}
{मुहुर्मुहुर्गीयमानमायातं परिपाठतः}% २१

\twolineshloka
{शृण्वावां तेऽभिधास्यावो यो रामो या च जानकी}
{यद्यद्भविष्यते तस्या रामेण क्रीडितात्मना}% २२

\twolineshloka
{ऋष्यशृङ्गकृतेष्ट्यां च चतुर्धा त्वङ्गतो हरिः}
{प्रादुर्भविष्यति श्रीमान्सुरस्त्रीगीतसत्कथः}% २३

\twolineshloka
{स कौशिकेन मिथिलां प्राप्स्यते भ्रातृसंयुतः}
{धनुष्पाणिस्तदा दृष्ट्वा दुर्ग्राह्यमन्यभूभुजाम्}% २४

\twolineshloka
{धनुर्भङ्क्त्वा जनकजां प्राप्स्यते सुमनोहराम्}
{तया सह महद्राज्यं करिष्यति श्रुतं वरे}% २५

\twolineshloka
{एतदन्यच्च तत्रस्थैः श्रुतमस्माभिरुद्गतैः}
{कथितं तव चार्वङ्गि मुञ्चावां गन्तुकामुकौ}% २६

\twolineshloka
{इति वाक्यं तयोर्धृत्वा श्रोत्रयोः सुमनोहरम्}
{पुनरेवजगादेदं वाक्यं पक्षियुगं प्रति}% २७

\twolineshloka
{स रामः कुत्र वर्तेत कस्य पुत्रः कथं तु ताम्}
{परिग्रहीष्यति वरः कीदृग्रूपधरो नरः}% २८

\twolineshloka
{मया पृष्टमिदं सर्वं कथयन्तु यथातथम्}
{पश्चात्सर्वं करिष्यामि प्रियं युष्मन्मनोहरम्}% २९

\twolineshloka
{तच्छ्रुत्वा तां तु कामेन पीडितां वीक्ष्य सा शुकी}
{जानकीं हृदये ज्ञात्वा पपाठ पुरतस्ततः}% ३०

\twolineshloka
{सूर्यवंशध्वजो धीमान्राजा पङ्क्तिरथो बली}
{यं देवाः श्रित्य सर्वारीन्विजेष्यन्ति च सर्वशः}% ३१

\twolineshloka
{तस्य भार्यात्रयं भावि शक्रमोहनरूपधृक्}
{तस्मिन्नपत्य चातुष्कं भविष्यति बलोन्नतम्}% ३२

\twolineshloka
{सर्वेषामग्रजो रामो भरतस्तदनुस्मृतः}
{लक्ष्मणस्तदनु श्रीमाञ्छत्रुघ्नः सर्वतो बली}% ३३

\twolineshloka
{रघुनाथ इति ख्यातिं गमिष्यति महामनाः}
{तेषामनन्तनामानि रामस्य बलिनः सखि}% ३४

\fourlineindentedshloka
{पद्मकोश इव शोभनं मुखं}
{पङ्कजाभनयने सुदीर्घके}
{उन्नता पृथुमनोहरा नसा}
{वल्गुसङ्गत मनोहरे भ्रुवौ}% ३५

\fourlineindentedshloka
{जानुलम्बित मनोहरौ भुजौ}
{कम्बुशोभिगलकोऽतिह्रस्वकः}
{सत्कपाटतलविस्तृतश्रिकं}
{वक्ष एतदमलं सलक्ष्मकम्}% ३६

\fourlineindentedshloka
{शोभनोरुकटिशोभया युतं}
{जानुयुग्मममलं स्वसेवितम्}
{पादपद्ममखिलैर्निजैः सदा}
{सेवितं रघुपतिं सुशोभनम्}% ३७

\twolineshloka
{एतद्रूपधरो रामो मया किं तु स वर्ण्यते}
{शताननोपि नो याति पक्षिणः किमु मादृशाः}% ३८

\twolineshloka
{यद्रूपं वीक्ष्य ललिता मनोहरवपुर्धरा}
{लक्ष्मीर्मुमोह भुविका वर्तते या न मोहति}% ३९

\twolineshloka
{महाबलो महावीर्यो महामोहनरूपधृक्}
{किं वर्णयामि श्रीरामं सर्वैश्वर्यगुणान्वितम्}% ४०

\twolineshloka
{धन्या सा जानकीदेवी महामोहनरूपधृक्}
{रंस्यते येन सहिता वर्षाणामयुतं मुदा}% ४१

\twolineshloka
{त्वं कासि किं तु नामासि बत सुन्दरि यत्तु माम्}
{परिपृच्छसि वैदग्ध्याद्रामकीर्तनमादरात्}% ४२

\twolineshloka
{एतद्वाक्यं समाकर्ण्य जानकी पक्षिणोर्युगम्}
{उवाच जन्मललितं शंसन्ती स्वस्य मोहनम्}% ४३

\twolineshloka
{या त्वया जानकी प्रोक्ता साहं जनकपुत्रिका}
{स रामो मां यदाभ्येत्य प्राप्स्यते सुमनोहरः}% ४४

\twolineshloka
{तदा वां मोचयाम्यद्धानान्यथा वाक्यलोभिता}
{लालयामि सुखेनास्तां मद्गेहे मधुराक्षरौ}% ४५

\twolineshloka
{इत्युक्तं तत्समाकर्ण्य वेपतुर्भयतां गतौ}
{परस्परं प्रक्षुभितौ जानकीमित्यवोचताम्}% ४६

\twolineshloka
{वयं वै पक्षिणः साध्वि वनस्था वृक्षगोचराः}
{परिभ्रमाम सर्वत्र नो सुखं नो भवेद्गृहे}% ४७

\twolineshloka
{अन्तर्वत्नी स्वके स्थाने गत्वा संसूय पुत्रकान्}
{त्वत्स्थानमागमिष्यामि सत्यं मे ह्युदितं वचः}% ४८

\twolineshloka
{एवं प्रोक्ता तया सा तु न मुमोच शुकीं स्वयम्}
{तदापतिस्तां प्रोवाच विनीतवदनुत्सुकः}% ४९

\twolineshloka
{सीते मुञ्च कथं भार्यां रक्षसे मे मनोहरीम्}
{आवां गच्छाव विपिने विचरावः सुखं वने}% ५०

\twolineshloka
{अन्तर्वत्नी तु वर्तेत भार्या मम मनोरमा}
{तस्याः प्रसूतिं कृत्वा त्वामागमिष्यामि शोभने}% ५१

\twolineshloka
{इत्युक्ता निजगादेमं सुखं गच्छ महामते}
{एतां रक्षामि सुखिनीं मत्पार्श्वे प्रियकारिणीम्}% ५२

\twolineshloka
{इत्युक्तो दुःखितः पक्षी तामूचे करुणान्वितः}
{योगिभिः प्रोच्यते यद्वै तद्वचस्तथ्यमेव हि}% ५३

\twolineshloka
{न वक्तव्यं न वक्तव्यं मौनमाश्रित्य तिष्ठताम्}
{नोचेत्स वाक्यदोषेण प्राप्नोत्यालानमुन्मदः}% ५४

\twolineshloka
{वयं चेदत्र वाक्यं नाकरिष्याम नगोपरि}
{बन्धनं कथमावां स्यात्तस्मान्मौनं समाचरेत्}% ५५

\twolineshloka
{इत्युक्त्वा तां प्रत्युवाच नाहं जीवामि सुन्दरि}
{एतया भार्यया सीते तस्मान्मुञ्च मनोहरे}% ५६

\twolineshloka
{अनेकविधवाक्यैः सा बोधिता नामुचत्तदा}
{कुपिता दुःखिता भार्या शशाप जनकात्मजाम्}% ५७

\twolineshloka
{यथा त्वं पतिना सार्धं वियोजयसि मामितः}
{तथा त्वमपि रामेण वियुक्ता भव गर्भिणी}% ५८

\twolineshloka
{इत्युक्तवत्यां तस्यां तु दुःखितायां पुनः पुनः}
{प्राणा निरगमन्दुःखात्पतिदुःखेन पूरितात्}% ५९

\twolineshloka
{रामं रामं स्मरन्त्याश्च वदन्त्यांश्च पुनः पुनः}
{विमानमागतं सुष्ठु पक्षिणी स्वर्गता बभौ}% ६०

\twolineshloka
{तस्यां मृतायां दुःखार्तो भर्ता तस्याः स पक्षिराट्}
{परमं क्रोधमापन्नो जाह्नव्यां दुःखितोऽपतत्}% ६१

\twolineshloka
{तथा भवामि रामस्य नगरे जनपूरिते}
{मद्वाक्यादियमुद्विग्ना वियोगेन सुदुःखिता}% ६२

\twolineshloka
{इत्युक्त्वा स पपातोदे जाह्नव्या भ्रमशोभिते}
{दुःखितः कुपितो भीतस्तद्वियोगेन कम्पितः}% ६३

\twolineshloka
{क्रुद्धत्वाद्दुःखितत्वाच्च सीताया अपमाननात्}
{अन्त्यजत्वं परं प्राप्तो रजकः क्रोधनाभिधः}% ६४

\twolineshloka
{यः क्रोधाच्च स्वकान्प्राणान्महतां दुष्टमाचरन्}
{सन्त्यजेत्स मृतो याति अन्त्यजत्वं द्विजोत्तमः}% ६५

\twolineshloka
{तज्जातं रजकोक्त्यासौ निन्दिता च वियोगिता}
{रजकस्य च शापेन वियुक्ता सा वनं गता}% ६६

\twolineshloka
{एतत्ते कथितं विप्र यत्ते पृष्टं विदेहजाम्}
{पुनरत्र परं वृत्तं शृणुष्व निगदामि तत्}% ६७


{॥इति श्रीपद्मपुराणे पातालखण्डे शेषवात्स्यायनसंवादे रामाश्वमेधे रजकप्राग्जन्मकथनन्ना सप्तपञ्चाशत्तमोऽध्यायः॥५७॥}

\dnsub{अष्टपञ्चाशत्तमोऽध्यायः}%\resetShloka

\uvacha{शेष उवाच}

\twolineshloka
{भरतं मूर्च्छितं दृष्ट्वा रघुनाथः सुदुःखितः}
{प्रतीहारमुवाचेदं शत्रुघ्नं प्रापयाशु माम्}% १

\twolineshloka
{तद्वाक्यं प्रोक्तमाकर्ण्य क्षणाच्छत्रुघ्नमानयत्}
{यत्र रामो निजभ्राता भरतेन सह स्थितः}% २

\twolineshloka
{भरतं मूर्च्छितं दृष्ट्वा रघुनाथं च दुःखितम्}
{प्रणम्य दुःखितोऽवोचत्किमिदं दारुणं महत्}% ३

\twolineshloka
{तदा रामोऽन्त्यजप्रोक्तं वाक्यं लोकविगर्हितम्}
{तं प्रत्युवाच रामोऽसौ शत्रुघ्नं पदसेवकम्}% ४

\twolineshloka
{अधोमुखो दीनरवो गद्गदाक्षरवेपथुः}
{शृणु भ्रातर्वचो मेऽद्य कुरु तत्क्षिप्रमादरात्}% ५

\twolineshloka
{यथा स्याद्विमलाकीर्तिर्गङ्गेव पृथिवीं गता}
{सीताया वाच्यमतुलं लोके श्रुत्वान्त्यजोदितम्}% ६

\twolineshloka
{हातुमिच्छामि देहं स्वमेनां वा जानकीं किल}
{इति वाक्यं समाकर्ण्य रामस्य किल शत्रुहा}% ७

\twolineshloka
{सवेपथुः पपातोर्व्यां दुःखितः परदारणः}
{संज्ञां प्राप्य मुहूर्तेन रघुनाथमवोचत}% ८

\uvacha{शत्रुघ्न उवाच}

\twolineshloka
{किमेतदुच्यते स्वामिञ्जानकीं प्रति दारुणम्}
{पाखण्डैर्दुष्टचित्तैश्च सर्वधर्मबहिष्कृतैः}% ९

\twolineshloka
{निन्दिता श्रुतिरग्राह्या भवति त्वग्र्यजन्मनाम्}
{जाह्नवी सर्वलोकानां पापघ्नी दुरितापहा}% १०

\twolineshloka
{निस्पृष्टा पापिभिः पुम्भिः सा स्पर्शेनार्हिता सताम्}
{सूर्यो जगत्प्रकाशाय समुदेति जगत्यहो}% ११

\twolineshloka
{उलूकानां रुचिकरो न भवेत्तत्र का क्षतिः}
{तस्मात्त्वमेनां गृह्णीष्व मा त्यजानिन्दितां स्त्रियम्}% १२

\twolineshloka
{श्रीरामभद्रकृपया कुरुष्व वचनं मम}
{एतच्छ्रुत्वा वचस्तस्य शत्रुघ्नस्य महात्मनः}% १३

\twolineshloka
{पुनःपुनर्जगादेदं यदुक्तं भरतं प्रति}
{तन्निशम्य वचो भ्रातुर्दुःखपूरपरिप्लुतः}% १४

\twolineshloka
{पपात मूर्च्छितो भूमौ छिन्नमूल इव द्रुमः}
{भ्रातरं पतितं वीक्ष्य शत्रुघ्नं दुःखितो भृशम्}% १५

\twolineshloka
{प्रतीहारमुवाचेदं लक्ष्मणं त्वानयान्तिकम्}
{स लक्ष्मणगृहं गत्वा न्यवेदयदिदं वचः}% १६

\uvacha{प्रतीहार उवाच}

\twolineshloka
{स्वामिन्रामो भवन्तं तु समाह्वयति वेगतः}
{स तच्छ्रुत्वा समाह्वानं रामचन्द्रेण वेगतः}% १७

\twolineshloka
{जगाम तरसा तत्र यत्र स भ्रातृकोऽनघः}
{भरतं मूर्च्छितं दृष्ट्वा शत्रुघ्नमपि मूर्च्छितम्}% १८

\twolineshloka
{श्रीरामचन्द्रं दुःखार्तं दुःखितो वाक्यमब्रवीत्}
{किमेतद्दारुणं राजन्दृश्यते मूर्च्छनादिकम्}% १९

\twolineshloka
{तदाशु शंस मां सर्वं कारणं मुख्यतोऽनघ}
{एवं वदन्तं नृपतिर्वृत्तान्तं सर्वमादितः}% २०

\twolineshloka
{शशंस लक्ष्मणं क्षिप्रं दुःखपूरपरिप्लुतम्}
{लक्ष्मणस्तद्वचः श्रुत्वा सीतायास्त्यागसम्भवम्}% २१

\twolineshloka
{निःश्वसन्मुहुरुच्छ्वासं स्तब्धगात्र इवाभवत्}
{भ्रातरं स्तब्धगात्रं च कम्पमानं मुहुर्मुहुः}% २२

\twolineshloka
{न किञ्चन वदन्तं तं वीक्ष्य शोकार्दितोऽब्रवीत्}
{किं करिष्याम्यहं भूमौ स्थित्वा दुर्यशसाङ्कितः}% २३

\twolineshloka
{त्यजामीदं वपुः श्रीमल्लोकभीत्या च शोकवान्}
{सर्वदा भ्रातरो मह्यं मद्वाक्यकरणोत्सुकाः}% २४

\twolineshloka
{इदानीं तेपि दैवेन प्रतिकूलवचः कराः}
{कुत्र गच्छामि कं यामि हसिष्यन्ति नृपा भुवि}% २५

\twolineshloka
{दुर्यशो लाञ्छितं मां वै कुष्ठिनं रूपवन्नराः}
{मनोर्वंशे पुरा भूपा जाता जाता गुणाधिकाः}% २६

\twolineshloka
{इदानीं मयि जाते तु विपरीतं बभूव तत्}
{इति सम्भाषमाणं तं रामभद्रं समीक्ष्य सः}% २७

\twolineshloka
{संस्तभ्याश्रूणि विपुलान्युवाच विकल स्वरः}
{स्वामिन्विषादं मा कार्षीः कथं तव मतिर्हृता}% २८

\twolineshloka
{सीतामनिन्दितां को नु त्यजति श्रुतवान्भवान्}
{आकारयामि रजकं परिपृच्छामि तं प्रति}% २९

\twolineshloka
{कथं त्वयानिन्दिता सा जानकी योषितां वरा}
{तव देशे बलात्कश्चिद्बाध्यते न जनोऽल्पकः}% ३०

\twolineshloka
{तस्मात्तस्य यथास्वान्ते प्रतीतिः स्यात्तथाचर}
{किमर्थं त्यज्यते भीरुः पतिव्रतपरायणा}% ३१

\twolineshloka
{मनसा वचसा नान्यं जानाति जनकात्मजा}
{तस्मादेनां गृहाण त्वमेतां मा त्यज जानकीम्}% ३२

\twolineshloka
{ममोपरि कृपां कृत्वा मदुक्तं संश्रयाशु तत्}
{एवं वदन्तं प्रत्यूचे रामः शोकेन कर्षितः}% ३३

\onelineshloka
{लक्ष्मणं धर्मवाक्येन बोधयंस्त्यजनोद्यमः}% ३४

\uvacha{राम उवाच}

\twolineshloka
{कथं तु मां ब्रवीषि त्वं मा त्यजैनामनिन्दिताम्}
{लोकापवादात्त्यक्ष्येऽहं जानन्नपि विपापिनीम्}% ३५

\twolineshloka
{स्वयशः कारणेऽहं स्वं देहं त्यक्ष्याम्यशोभनम्}
{त्वामपि भ्रातरं त्यक्ष्ये लोकवादाद्विगर्हितम्}% ३६

\twolineshloka
{किमुतान्ये गृहाः पुत्रा मित्राणि वसुशोभनम्}
{स्वयशःकारणे सर्वं त्यजामि किमु मैथिलीम्}% ३७

\twolineshloka
{न तथा मे प्रियो भ्राता न कलत्रं न बान्धवाः}
{यथा मे विमलाकीर्तिर्वल्लभा लोकविश्रुता}% ३८

\twolineshloka
{इदानीं रजको नैव प्रष्टव्यो भवति ध्रुवम्}
{कालेन सर्वं भविता लोकचित्तस्य रञ्जनम्}% ३९

\twolineshloka
{आमयो यद्वदामस्तु न चिकित्स्यो भवेत्क्षितौ}
{सकालेन परीपाकाद्भेषजादेव नश्यति}% ४०

\twolineshloka
{तथा कालेन सम्भावि साम्प्रतं मा विलम्बय}
{त्यजैनां विपिने साध्वीं मां वा खड्गेन घातय}% ४१

\twolineshloka
{इत्युक्तं वाक्यमाकर्ण्य दुःखितोऽभूत्तदा महान्}
{चिन्तयामास च स्वान्ते लक्ष्मणः शोककर्षितः}% ४२

\twolineshloka
{पित्राज्ञप्तो जामदग्न्यो मातरं चाप्यघातयत्}
{गुरोराज्ञा नैव लङ्घ्या युक्ताऽयुक्तापि सर्वथा}% ४३

\twolineshloka
{तस्मादेनां त्यजाम्येव रामस्य प्रियकाम्यया}
{इति सञ्चिन्त्य मनसि भ्रातरं प्रत्युवाच सः}% ४४

\uvacha{लक्ष्मणउवाच}

\twolineshloka
{अकृत्यमपि कार्यं वै गुर्वाज्ञां नैव लङ्घयेत्}
{तस्मात्कुर्वे भवद्वाक्यं यत्त्वं वदसि सुव्रत}% ४५

\twolineshloka
{इत्येवं भाषमाणं तं लक्ष्मणं प्रत्युवाच सः}
{साधुसाधु महाप्राज्ञ त्वया मे तोषितं मनः}% ४६

\twolineshloka
{अद्यैव रात्रौ जानक्या दोहदस्तापसी क्षणे}
{तन्मिषेण रथे स्थाप्य मोचयैनां महावने}% ४७

\twolineshloka
{इत्थं भाषितमाकर्ण्य विशुष्यद्वदनोऽभितः}
{रुदन्बाष्पकलां मुञ्चञ्जगाम स्वं निवेशनम्}% ४८

\twolineshloka
{सुमन्त्रं तु समाहूय वचनं तमथाब्रवीत्}
{रथं मे कुरु सज्जं वै सदश्ववरभूषितम्}% ४९

\twolineshloka
{स तद्वाक्यं समाकर्ण्य रथमानीतवांस्तदा}
{आनीतं तं रथं दृष्ट्वा लक्ष्मणः शोककर्षितः}% ५०

\twolineshloka
{परमं दुःखमापन्नः संरुह्य स्यन्दनं वरम्}
{निःश्वसञ्जानकीगेहं प्रतस्थे भ्रातृसेवकः}% ५१

\twolineshloka
{गत्वा चान्तःपुरे भ्राता रामस्य मिथिलात्मजाम्}
{प्रत्यूचे निःश्वसन्वाक्यं दुःखपूरपरिप्लुतः}% ५२

\twolineshloka
{मातर्जानकि रामेण प्रेषितो भवनं तव}
{तापसीः प्रति याहि त्वं दोहदप्राप्तिहेतवे}% ५३

\twolineshloka
{इति वाक्यं समाकर्ण्य लक्ष्मणस्य विदेहजा}
{परमं हर्षमापन्ना लक्ष्मणं प्रत्यभाषत}% ५४

\uvacha{जानक्युवाच}

\twolineshloka
{धन्याहं मैथिली राज्ञी रामस्य चरणस्मरा}
{यस्या दोहदपूर्त्यर्थं प्रेषयामास लक्ष्मणम्}% ५५

\twolineshloka
{अद्याहं ता वनचरीस्तापसीः पतिदेवताः}
{नमस्कुर्यां च वासोभिः पूजयामि मनोहराः}% ५६

\twolineshloka
{इत्युक्त्वा रम्यवस्त्राणि महार्हाभरणानि च}
{मणीन्विमलमुक्ताश्च कर्पूरादिसुगन्धिमत्}% ५७

\twolineshloka
{चन्दनादिकवस्तूनि विचित्राणि सहस्रधा}
{जग्राह रघुनाथस्य पत्नी स्वप्रियकाम्यया}% ५८

\twolineshloka
{सीता गृहीत्वा सर्वाणि दासीनां करयोर्मुहुः}
{लक्ष्मणं प्रतिगच्छन्ती देहल्यां चास्खलत्तदा}% ५९

\twolineshloka
{अविचार्य तदौत्सुक्याल्लक्ष्मणं प्रियकारिणम्}
{उवाच कुत्र सरथो येन मां प्रापयिष्यसि}% ६०

\twolineshloka
{स निःश्वसन्रथं हैमं जानक्या सह निर्विशत्}
{सुमन्त्रं प्रत्युवाचासौ चालयाश्वान्मनोजवान्}% ६१

\twolineshloka
{स सुयुक्तं रथं वाक्याल्लक्ष्मणस्य तु चाह्वयत्}
{अश्रुपूर्णमुखं पश्यँल्लक्ष्मणं स मुहुर्मुहुः}% ६२

\twolineshloka
{आहतास्तेन कशया वाहास्तस्यापतन्पथि}
{न चलन्ति यदा वाहास्तदा लक्ष्मणमब्रवीत्}% ६३

\uvacha{सुमन्त्र उवाच}

\twolineshloka
{स्वामिंश्चलन्ति नो वाहा यत्नेन परिचालिताः}
{किं करोमि न जानेऽत्र कारणं वाहपातने}% ६४

\twolineshloka
{एवं ब्रुवन्तं प्रत्यूचे लक्ष्मणो गद्गदस्वरः}
{सारथिं धैर्यमास्थाय ताडयैतान्कशादिभिः}% ६५

\twolineshloka
{एतच्छ्रुत्वोदितं यन्ता कथञ्चित्समचालयत्}
{तदा स्फुरद्दक्षनेत्रं जानक्या दुःखशंसकम्}% ६६

\twolineshloka
{तदैव हृदये शोकः समभूद्दुःखशंसकः}
{तदैव पक्षिणः पुण्याः कुर्वन्ति परिवर्तनम्}% ६७

\twolineshloka
{एवं वीक्ष्यैव वैदेही प्रत्युवाचाथ देवरम्}
{कथं मे तापसीक्षायै यातुमिच्छा रघूद्वह}% ६८

\twolineshloka
{रामे भूयाद्धि कल्याणं भरते वा तथानुजे}
{तत्प्रजासु च सर्वत्र मा भवन्तु विपर्ययाः}% ६९

\twolineshloka
{एवं ब्रुवन्तीं संवीक्ष्य जानकीं च स लक्ष्मणः}
{न किञ्चिदुक्तवान्रुद्ध कण्ठो बाष्पप्रपूरितः}% ७०

\twolineshloka
{सा गच्छन्ती मृगान्वामं परिवर्तनकारकान्}
{अपश्यद्दुःखसङ्घातकारकान्समभाषत}% ७१

\twolineshloka
{अद्य यन्मे मृगा वामं वर्तयन्ति तदिष्यते}
{श्रीरामचरणौ मुक्त्वा गच्छन्त्यायुक्तमेव तत्}% ७२

\twolineshloka
{महिलानां परोधर्मः स्वभर्तृचरणार्चनम्}
{तन्मुक्त्वान्यत्र यान्त्या मे यद्भवेद्युक्तमेव तत्}% ७३

\twolineshloka
{एवं पथि विचारं तु कुर्वन्त्या परमार्थतः}
{जाह्नवी ददृशे देव्या मुनिवृन्दैकसेविता}% ७४

\twolineshloka
{यस्यां जलस्य कल्लोला दृश्यन्ते दुग्धसन्निभाः}
{तरङ्गो दृश्यते यत्र स्वर्गसोपानमूर्तिभृत्}% ७५

\twolineshloka
{यस्या वारिकणस्पर्शान्महापातकसञ्चयः}
{पलायते न कुत्रापि स्थानमीक्षन्समन्ततः}% ७६

\twolineshloka
{गङ्गां प्राप्याथ सौमित्रिर्जानकीं स्यन्दने स्थिताम्}
{उवाच निर्गलद्बाष्प एहि सीते रथाद्भुवि}% ७७

\twolineshloka
{सीता तद्वाक्यमाकर्ण्य क्षणादवततार सा}
{लक्ष्मणेन धृता बाहौ स्खलन्ती पथि कण्टकैः}% ७८

{॥इति श्रीपद्मपुराणे पातालखण्डे शेषवात्स्यायनसंवादे रामाश्वमेधे जानक्या गङ्गादर्शनं नाम अष्टपञ्चाशत्तमोऽध्यायः॥५८॥}

\dnsub{एकोनषष्टितमोऽध्यायः}%\resetShloka

\uvacha{शेष उवाच}

\twolineshloka
{अथ नावा समुत्तीर्य जाह्नवीं लक्ष्मणस्तदा}
{जानकीं परतस्तीरे हस्ते धृत्वा वनं ययौ}% १

\twolineshloka
{सा चलन्ती पथि तदा शुष्यद्वदनलक्षिता}
{कण्टकक्षतसत्पादा स्खलन्ती च पदे पदे}% २

\twolineshloka
{लक्ष्मणस्तां महाघोरे विपिने दुःखदायिनि}
{प्रवेशयामास तदा राघवाज्ञाविधायकः}% ३

\twolineshloka
{यत्र वृक्षा महाघोरा बर्बूलाः खदिरा घनाः}
{श्लेष्मातकाश्चिञ्चिणीकाः शुष्का दावेन वह्निना}% ४

\twolineshloka
{कोटरस्था महासर्पाः फूत्कुर्वन्ति सुकोपिताः}
{घूका घूत्कुर्वते यत्र लोकचित्तभयङ्कराः}% ५

\twolineshloka
{व्याघ्राः सिंहाः सृगालाश्च द्वीपिनोऽतिभयङ्कराः}
{दृश्यन्ते यत्र सरला मनुष्यादाः सुकोपनाः}% ६

\twolineshloka
{महिषाः सूकरा दुष्टा दंष्ट्रा द्वयविलक्षिताः}
{कुर्वन्ति प्राणिनां तापं मानसस्य मदोद्धुराः}% ७

\twolineshloka
{ईदृग्वनं प्रपश्यन्ती भयेनोपगतज्वरा}
{कण्टकैर्दष्टचरणा लक्ष्मणं वाक्यमब्रवीत्}% ८

\uvacha{जानक्युवाच}

\twolineshloka
{वीरर्षिमुनिसंसेव्या नाश्रमान्नेत्रसौख्यदान्}
{नाहं पश्यामि नो तेषां पत्नीश्च सुतपोधनाः}% ९

\twolineshloka
{पश्यामि केवलं घोरान्पक्षिणः शुष्कवृक्षकान्}
{दावानलेन तत्सर्वं दह्यमानमिदं वनम्}% १०

\twolineshloka
{त्वां च पश्यामि दुःखार्तमश्रुपूर्णाकुलेक्षणम्}
{शकुनेतरसाहस्रं भवेन्मम पदे पदे}% ११

\twolineshloka
{तन्मे कथय वीराग्र्य कथं मुक्ता महात्मना}
{रामेण दुष्टहृदया क्षिप्रं कथय मे हि तत्}% १२

\twolineshloka
{इति वाक्यं समाकर्ण्य लक्ष्मणः शोककर्शितः}
{संरुद्धबाष्पवदनो न किञ्चित्प्रोक्तवांस्तदा}% १३

\twolineshloka
{तदेव विपिनं घोरं गच्छन्ती लक्ष्मणान्विता}
{पुनरप्याह तं वीरं दुःखार्ता पश्यती मुखम्}% १४

\twolineshloka
{तदापि स न तां वक्ति किमपि प्रेक्षुलोलुपः}
{तदा सात्यन्तनिर्बन्धं चकार परिपृच्छती}% १५

\twolineshloka
{आग्रहेण यदा पृष्टो लक्ष्मणः सीतया तदा}
{रुद्धकण्ठो मुहुः शोचन्नवदत्त्यागसम्भवम्}% १६

\twolineshloka
{तद्वाक्यं पविना तुल्यं निशम्य मुनिसत्तम}
{सुलताकृत्तमूलेव बभूवाकल्पवर्जिता}% १७

\twolineshloka
{तदैव पृथिवी तां न जग्राह तनयामिमाम्}
{रामो विपापिनीं सीतां न जह्यादिति शङ्किनी}% १८

\twolineshloka
{पतितां तां तु वैदेहीं दृष्ट्वा सौमित्रिरुत्सुकः}
{पल्लवाग्र्यसमीरेण संज्ञितां तु चकार सः}% १९

\twolineshloka
{संज्ञां प्राप्ता ह्युवाचेदं मा हास्यं कुरु देवर}
{कथं मां पापरहितां त्यजते स रघूद्वहः}% २०

\twolineshloka
{एवं बहुविलप्याथ लक्ष्मणं दुःखसंयुतम्}
{संवीक्ष्यमू र्च्छिता भूमौ पपात परिदुःखिता}% २१

\twolineshloka
{मुहूर्तेनापि संज्ञां सा प्राप्य दुःखपरिप्लुता}
{जगाद रामचरणौ स्मंरती शोकविक्षता}% २२

\twolineshloka
{रघुनाथो महाबुद्धिस्त्यजते मां कथं महान्}
{यो मदर्थे पयोराशिं बद्धवान्वानरैर्युतः}% २३

\twolineshloka
{स कथं मां महावीरो निष्पापां रजकोक्तितः}
{त्यजिष्यति ममैवात्र दैवं तु प्रतिकूलितम्}% २४

\twolineshloka
{एवं वदन्ती पुनरपि मूर्च्छां प्राप्ता विदेहजा}
{मूर्च्छितां तां समीक्ष्याथ रुरोद विकृतस्वरः}% २५

\twolineshloka
{पुनः संज्ञामवाप्यैवं सौमित्रिं निजगाद सा}
{दुःखातुरं वीक्षमाणा रुद्धकण्ठं सुदुःखिता}% २६

\twolineshloka
{सौमित्रे गच्छ रामं त्वं धर्ममूर्तिं यशोनिधिम्}
{मद्वाक्यमेवैतद्ब्रूयाः समक्षं तपसां निधेः}% २७

\twolineshloka
{मां तत्याज भवान्यद्वै जानन्नपि विपापिनीम्}
{कुलस्य सदृशं किं वा शास्त्रज्ञानस्य तत्फलम्}% २८

\twolineshloka
{नित्यं तव पदे रक्तां त्वदुच्छिष्टभुजं हि माम्}
{भवांस्तत्याज तत्सर्वं मम दैवं तु कारणम्}% २९

\twolineshloka
{कल्याणं तव सर्वत्र भूयाद्वीरवरोत्तम}
{अहं तावद्वने त्वां हि स्मरन्ती प्राणधारिका}% ३०

\twolineshloka
{मनसा कर्मणा वाचा भवानेव ममोत्तमः}
{अन्ये तुच्छीकृताः सर्वे मनसा रघुवंशज}% ३१

\twolineshloka
{भवेभवे भवानेव पतिर्भूयान्महीश्वर}
{त्वत्पदस्मरणानेक हतपापा सतीश्वरी}% ३२

\twolineshloka
{श्वश्रूजनं ब्रूहि सर्वं मत्सन्देशं रघूत्तम}
{त्यक्ता वने महाघोरे रामेण निरघा सती}% ३३

\twolineshloka
{स्मरामि चरणौ युष्मद्वने मृगगणैर्युते}
{अन्तर्वत्नी वने त्यक्ता रामेण सुमहात्मना}% ३४

\twolineshloka
{सौमित्रे शृणु मद्वाक्यं भद्रं भूयाद्रघूत्तमे}
{इदानीं सन्त्यजे प्राणान्रामवीर्यं सुरक्षती}% ३५

\twolineshloka
{त्वं रामवचनं तथ्यं यत्करोषि शुभं तव}
{परतन्त्त्रेण तत्कार्यं रामपादाब्जसेविना}% ३६

\twolineshloka
{गच्छ त्वं राम सविधे शिवाः पन्थान एव ते}
{ममोपरि कृपा कार्या स्मर्तव्याहं कदा कदा}% ३७

\twolineshloka
{इत्युक्त्वा मूर्च्छिता भूमौ पपात पुरतस्तदा}
{लक्ष्मणो दुःखमापेदे वीक्ष्य मूर्च्छितजानकीम्}% ३८

\twolineshloka
{वीजयामास वासोग्रैः संज्ञां प्राप्तां प्रकृत्य च}
{सौमित्रिः सान्त्वयामास वचनैर्मधुरैर्मुहुः}% ३९

\uvacha{लक्ष्मण उवाच}

\twolineshloka
{एष गच्छामि रामं वै गत्वा शंसामि सर्वशः}
{समीपे ते मुनेरस्ति वाल्मीकेराश्रमो महान्}% ४०

\twolineshloka
{इत्युक्त्वा तां परिक्रम्य दुःखितो बाष्पपूरितः}
{मुञ्चन्नश्रुकलां दुःखाद्ययौ रामं महीपतिम्}% ४१

\twolineshloka
{जानकी देवरं यान्तं वीक्ष्य विस्मितलोचना}
{हसत्ययं महाभागो लक्ष्मणो देवरो मम}% ४२

\twolineshloka
{कथं मां प्राणतः प्रेष्ठां विपापां राघवोऽत्यजत्}
{इति सञ्चिन्तयन्ती सा तमैक्षदनिमेषणा}% ४३

\twolineshloka
{जाह्नवीं सर्वथोत्तीर्णां ज्ञात्वा सत्यं स्वहापनम्}
{पतिता प्राणसन्देहं प्राप्ता मूर्च्छां गता तदा}% ४४

\twolineshloka
{तदा हंसाः स्वपक्षाभ्यां जलमानीय सर्वतः}
{सिषिचुर्मधुरो वायुर्ववौ पुष्पसुगन्धिमान्}% ४५

\twolineshloka
{करिणः पुष्करैः स्वीयैर्जलपूर्णैः समन्ततः}
{व्याप्तं शरीरं रजसा क्षालयन्त इवागताः}% ४६

\twolineshloka
{मृगास्तदन्तिकं प्राप्य सन्तस्थुर्विस्मितेक्षणाः}
{नगाः पुष्पयुता आसंस्तत्कालं मधुना विना}% ४७

\twolineshloka
{एतस्मिन्समये वृत्ते संज्ञां प्राप्य तदा सती}
{विललाप सुदुःखार्ता रामरामेति जल्पती}% ४८

\twolineshloka
{हा नाथ दीनबन्धो हे करुणामयसन्निधे}
{अपराधादृते मां त्वं कथं त्यजसि वै वने}% ४९

\twolineshloka
{इत्येवमादिभाषन्ती विलपन्ती मुहुर्मुहुः}
{इतस्ततः प्रपश्यन्ती सम्मूर्च्छन्ती पुनःपुनः}% ५०

\twolineshloka
{तदा स्वशिष्यैर्भगवान्वाल्मीकिः सङ्गतो वनम्}
{शुश्राव रुदितं तत्र करुणास्वरभाषितम्}% ५१

\twolineshloka
{शिष्यान्प्रति जगादाथ पश्यन्तु वनमध्यतः}
{को रोदिति महाघोरे विपिने दुःखितस्वरः}% ५२

\twolineshloka
{ते प्रयुक्ताश्च मुनिना सञ्जग्मुर्यत्र जानकी}
{रामरामेति भाषन्ती बाष्पपूरपरिप्लुता}% ५३

\twolineshloka
{तां दृष्ट्वा स्त्रियमौत्सुक्याद्वाल्मीकिं प्रत्यगुर्मुनिम्}
{श्रुत्वा तदीरितं वाक्यं जगामासौ ततो मुनिः}% ५४

\twolineshloka
{दृष्ट्वा तं तपसां राशिं जानकी पतिदेवता}
{नमोस्तु मुनये वेदमूर्तये व्रतवार्धये}% ५५

\twolineshloka
{इत्युक्तवन्तीं तां सीतामाशीर्भिरभ्यनन्दयत्}
{भर्त्रा सह चिरञ्जीव पुत्रौ प्राप्नुहि शोभनौ}% ५६

\twolineshloka
{कासि त्वं किं वने घोरे सङ्गतासि किमीदृशी}
{सर्वं मे शंस जानीयां तव दुःखस्य कारणम्}% ५७

\twolineshloka
{सा तदा प्रत्युवाचेमं रामस्य महिला मुनिम्}
{निःश्वसन्ती करुणया गिरासञ्जातवेपथुः}% ५८

\twolineshloka
{शृणु मे वाक्यमर्थोक्तं सर्वदुःखस्य कारणम्}
{जानीहि मां भूमिपते रघुनाथस्य सेवकीम्}% ५९

\twolineshloka
{अपराधं विना त्यक्ता न जाने तत्र कारणम्}
{लक्ष्मणो मां विमुच्यात्र गतवान्राघवाज्ञया}% ६०

\twolineshloka
{इत्युक्त्वाश्रुकलापूर्णं बिभ्रतीं मुखपङ्कजम्}
{वाल्मीकिः सान्त्वयन्प्राह जानकीं कमलेक्षणाम्}% ६१

\twolineshloka
{वाल्मीकिं मां विजानीहि पितुस्तव गुरुं मुनिम्}
{दुःखं मा कुरु वैदेहि ह्यागच्छ मम चाश्रमम्}% ६२

\twolineshloka
{भिन्नस्थाने पितुर्गेहं जानीहि पतिदेवते}
{ईदृशे कर्मणि मम रोषोस्त्वेव महीपतेः}% ६३

\twolineshloka
{एवं वचः समाकर्ण्य जानकी पतिदेवता}
{दुःखपूर्णाश्रुवदना किञ्चित्सुखमवाप सा}% ६४

\uvacha{शेष उवाच}

\twolineshloka
{वाल्मीकिः सान्त्वयित्वैनां दुःखपूराकुलेक्षणाम्}
{निनाय स्वाश्रमं पुण्यं तापसीवृन्दपूरितम्}% ६५

\twolineshloka
{सा गच्छन्ती पृष्ठतोऽस्य वाल्मीकेस्तपसां निधेः}
{रराजेन्दोः पृष्ठतो वै तारकेव मनोहरा}% ६६

\twolineshloka
{वाल्मीकिः प्राप्य च स्वीयमाश्रमं मुनिपूरितम्}
{तापसीः प्रतिसञ्चख्यौ जानकीं स्वाश्रमं गताम्}% ६७

\twolineshloka
{वैदेही तापसीः सर्वा नमश्चक्रे महामनाः}
{परस्परं प्रहृषिताः परिरम्भं समाचरन्}% ६८

\twolineshloka
{वाल्मीकिर्निजशिष्यान्स प्रत्युवाच तपोनिधिः}
{रच्यतां बत जानक्याः पर्णशाला मनोरमा}% ६९

\twolineshloka
{इत्युक्तं वाक्यमाकर्ण्य वाल्मीकेः सुमनोरमम्}
{व्यरचन्पत्रकैः शालां दारुभिः सुमनोहराम्}% ७०

\twolineshloka
{तत्रावसद्धि वैदेही पतिव्रतपरायणा}
{वाल्मीकेः परिचर्यां च कुर्वन्ती फलभक्षिका}% ७१

\twolineshloka
{जपन्ती रामरामेति मनसा वचसा स्वयम्}
{निनाय दिवसांस्तत्र जानकी पतिदेवता}% ७२

\twolineshloka
{काले सासूत पूत्रौ द्वौ मनोहरवपुर्धरौ}
{रामचन्द्र प्रतिनिधी ह्यश्विनाविव जानकी}% ७३

\twolineshloka
{तच्छ्रुत्वा तु मुनिर्हृष्टो जानक्याः पुत्रसम्भवम्}
{चकार जातकर्मादि संस्कारान्मन्त्रवित्तमः}% ७४

\twolineshloka
{कुशैर्लवैश्च वाल्मीकिर्मुनिः कर्माणि चाचरत्}
{तन्नाम्ना पुत्रयोराख्या कुशो लव इति स्फुटा}% ७५

\twolineshloka
{वाल्मीकिर्यत्र विरजा मङ्गलं तद्यथाचरत्}
{अत्यन्तं हृष्टचित्ता सा बभूव कमलेक्षणा}% ७६

\twolineshloka
{तद्दिने लवणं हत्वा शत्रुघ्नः स्वल्पसैनिकः}
{आगमच्चाश्रमे चास्य वाल्मीकेर्निशि शोभने}% ७७

\twolineshloka
{तदा वाल्मीकिना शिष्टः शत्रुघ्नो रघुनायकम्}
{मा शंस जानकीपुत्रौ कथयिष्याम्यहं पुरः}% ७८

\twolineshloka
{जानकीपुत्रकौ तत्र ववृधाते मनोरमौ}
{कन्दमूलफलैः पुष्टौ व्यदधादुन्मदौ वरौ}% ७९

\twolineshloka
{शुक्लप्रतिपदायाश्च शशीव सुमनोहरौ}
{कालेन संस्कृतौ जातावुपनीतौ मनोहरौ}% ८०

\twolineshloka
{उपनीयमुनिर्वेदं साङ्गमध्यापयत्सुतौ}
{सरहस्यं धनुर्वेदं रामायणमपाठयत्}% ८१

\twolineshloka
{वाल्मीकिना च धनुषी दत्ते स्वर्णसुभूषिते}
{अभेद्ये सगुणे श्रेष्ठे वैरिवृन्दविदारणे}% ८२

\twolineshloka
{इषुधी बाणसम्पूर्णौ अक्षये करवालके}
{चर्माण्यभेद्यानि ददौ जानक्यात्मजयोस्तदा}% ८३

\twolineshloka
{धनुर्धरौ धनुर्वेदपारगावाश्रमे मुदा}
{चरन्तौ तत्र रेजाते अश्विनाविव शोभनौ}% ८४

\twolineshloka
{जानकी वीक्ष्य पुत्रौ द्वौ खड्गचर्मधरौ वरौ}
{परमं हर्षमापन्ना विरहोद्भवमत्यजत्}% ८५

\twolineshloka
{एष ते कथितो विप्र जानक्याः पुत्रसम्भवः}
{अतः शृणुष्व यद्वृत्तं वीरबाहुविकृन्तनम्}% ८६

{॥इति श्रीपद्मपुराणे पातालखण्डे शेषवात्स्यायनसंवादे रामाश्वमेधे कुशलवोत्पत्तिकथानकन्नामैकोनषष्टितमोऽध्यायः॥५९॥}

\dnsub{षष्टितमोऽध्यायः}%\resetShloka

\uvacha{शेष उवाच}

\twolineshloka
{शत्रुघ्नो निजवीराणां छिन्नान्बाहून्निरीक्षयन्}
{उवाच तान्सुकुपितो रोषसन्दंशिताधरः}% १

\twolineshloka
{केन वीरेण वो बाहुकृन्तनं समकारि भोः}
{तस्याहं बाहू कृन्तामि देवगुप्तस्य वै भटाः}% २

\twolineshloka
{न जानाति महामूढो रामचन्द्र बलं महत्}
{इदानीं दर्शयिष्यामि पराक्रान्त्या बलं स्वकम्}% ३

\twolineshloka
{स कुत्र वर्तते वीरो हयः कुत्र मनोरमः}
{को वाऽगृह्णात्सुप्तसर्पान्मूढो ज्ञात्वा पराक्रमम्}% ४

\twolineshloka
{इति ते कथिता वीरा विस्मिता दुःखिता भृशम्}
{रामचन्द्र प्रतिनिधिं बालकं समशंसत}% ५

\twolineshloka
{स श्रुत्वा रोषताम्राक्षो बालकेन हयग्रहम्}
{सेनान्यं वै कालजितमाज्ञापयद्युयुत्सुकः}% ६

\twolineshloka
{सेनानीः सकलां सेनां व्यूहयस्व ममाज्ञया}
{रिपुः सम्प्रति गन्तव्यो महाबलपराक्रमः}% ७

\twolineshloka
{नायं बालो हरिर्नूनं भविष्यति हयन्धरः}
{अथवा त्रिपुरारिः स्यान्नान्यथा मद्धयापहृत्}% ८

\twolineshloka
{अवश्यं कदनं भाविसैन्यस्य बलिनो महत्}
{स्वच्छन्दचरितैः खेलन्नास्ते निर्भयधीः शिशुः}% ९

\twolineshloka
{तत्र गन्तव्यमस्माभिः सन्नद्धै रिपुदुर्जयैः}
{एतन्निशम्य वचनं शत्रुघ्नस्य ससैन्यपः}% १०

\twolineshloka
{सज्जीचकार सेनां तां दुर्व्यूढां चतुरङ्गिणीम्}
{सज्जां तां शत्रुजिद्दृष्ट्वा चतुरङ्गयुतां वराम्}% ११

\twolineshloka
{आज्ञापयत्ततो गन्तुं यत्र बालो हयन्धरः}
{सा चचाल तदा सेना चतुरङ्गसमन्विता}% १२

\twolineshloka
{कम्पयन्ती महीभागं त्रासयन्ती रिपून्बलात्}
{सेनानीस्तं ददर्शाथ बालकं रामरूपिणम्}% १३

\twolineshloka
{विचार्य रामप्रतिममब्रवीद्वचनं हितम्}
{बाल मुञ्च हयश्रेष्ठं रामस्य बलशालिनः}% १४

\twolineshloka
{सेनानीः कालजिन्नाम तस्य भूपस्य दुर्मदः}
{त्वां रामप्रतिमं दृष्ट्वा कृपा मे हृदि जायते}% १५

\twolineshloka
{अन्यथा तव मे दौस्थ्याज्जीवितं न भविष्यति}
{एतद्वाक्यं समाकर्ण्य शत्रुघ्नस्य भटस्य हि}% १६

\twolineshloka
{जहास किञ्चिदाकोपादुवाच च वचोद्भुतम्}
{गच्छ मुक्तोसि तं रामं कथयस्व हयग्रहम्}% १७

\twolineshloka
{त्वत्तो बिभेमि नो शूर वाक्येन नयशालिना}
{ममात्र गणना नास्ति त्वादृशाः कोटयो यदि}% १८

\twolineshloka
{मातृपादप्रसादेन तूलीभूता न संशयः}
{कालजित्तव यन्नाम मात्राकारि मनोज्ञया}% १९

\twolineshloka
{पक्वबिम्बफलस्येव वर्णतो न च वीर्यतः}
{दर्शयस्वाधुना वीर्यं स्वनामबलचिह्नितः}% २०

\onelineshloka*
{मां कालं तव सञ्जित्य सत्यनामा भविष्यसि}

\uvacha{शेष उवाच}

\onelineshloka
{स वाक्यैः पविनातुल्यैर्भिन्नः सुभटशेखरः}% २१


\threelineshloka
{चुकोप हृदयेऽत्यतं जगाद वचनं पुनः}
{कस्मिन्कुले समुत्पत्तिः किं नामासि च बालक}
{त्वन्नाम नाभिजानामि कुलं शीलं वयस्तथा}% २२

\twolineshloka
{पादचारं रथस्थोऽहमधर्मेण कथं जये}
{तदात्यन्तं प्रकुपितो जगाद वचनं पुनः}% २३

\twolineshloka
{कुलेन किं च शीलेन नाम्ना वा सुमनोहृदा}
{लवोऽहं लवतः सर्वाञ्जेष्यामि रिपुसैनिकान्}% २४

\twolineshloka
{इदानीं त्वामपि भटं करिष्ये पादचारिणम्}
{इत्थमुक्त्वा धनुः सज्यं चकार स लवो बली}% २५

\twolineshloka
{टङ्कारयामास तदा वीरानाकम्पयन्हृदि}
{वाल्मीकिं प्रथमं स्मृत्वा जानकीं मातरं लवः}% २६

\twolineshloka
{मुमोच बाणान्निशितान्सद्यः प्राणापहारिणः}
{कालजित्स्वधनुः कृत्वा सज्यं कोपसमन्वितः}% २७

\twolineshloka
{ताडयामास जवनो लवं रणविशारदः}
{तद्बाणाञ्छतधा छित्त्वा क्षणाद्वेगात्कुशानुजः}% २८

\twolineshloka
{सेनान्यं विरथं चक्रे वसुभिः स्वशरोत्तमैः}
{विरथो गजमानीतमारुरोह भटैर्निजैः}% २९

\twolineshloka
{मदोन्मत्तं महावेगं सप्तधा प्रस्रवान्वितम्}
{गजारूढं तु तं दृष्ट्वा दशभिर्धनुषोगतैः}% ३०

\twolineshloka
{बाणैर्विव्याध विहसन्सर्वान्रिपुगणाञ्जयी}
{कालजित्तस्य वीर्यं तु दृष्ट्वा विस्मितमानसः}% ३१

\twolineshloka
{गदां मुमोच महतीं महायस विनिर्मिताम्}
{आपतन्तीं गदां वेगाद्भारायुतविनिर्मिताम्}% ३२

\twolineshloka
{त्रिधा चिच्छेद तरसा क्षुरप्रैः सकुशानुजः}
{परिघं निशितं घोरं वैरिप्राणहरोदितम्}% ३३

\twolineshloka
{मुक्तं पुनस्तेन लवश्चिच्छेद तरसान्वितः}
{छित्त्वा तत्परिघं घोरं कोपादारक्तलोचनः}% ३४

\twolineshloka
{गजोपस्थे समारूढं मन्यमानश्चुकोप ह}
{तत्क्षणादच्छिनत्तस्य शुण्डां खड्गेन दन्तिनः}% ३५

\twolineshloka
{दन्तयोश्चरणौ धृत्वा रुरोह गजमस्तके}
{मुकुटं शतधा कृत्वा कवचं तु सहस्रधा}% ३६

\twolineshloka
{केशेष्वाकृष्य सेनान्यं पातयामास भूतले}
{पातितः स गजोपस्थात्सेनानीः कुपितः पुनः}% ३७

\twolineshloka
{हृदये ताडयामास मुष्टिना वज्रमुष्टिना}
{स आहतो मुष्टिभिस्तु क्षुरप्रान्निशिताञ्छरान्}% ३८

\twolineshloka
{मुमोच हृदये क्षिप्रं कुण्डलीकृतधन्ववान्}
{स रराज रणोपान्ते कुण्डलीकृत चापवान्}% ३९

\twolineshloka
{शिरस्त्रं कवचं बिभ्रदभेद्यं शरकोटिभिः}
{स विद्धः सायकैस्तीक्ष्णैस्तं हन्तुं खड्गमाददे}% ४०

\twolineshloka
{दशन्रोषात्स्वदशनान्निःश्वसन्नुच्छ्वसन्मुहुः}
{खड्गहस्तं समायान्तं शूरं सेनापतिं लवः}% ४१

\twolineshloka
{चिच्छेद भुजमध्यं तु स खड्गः पाणिरापतत्}
{छिन्नं खड्गधरं हस्तं वीक्ष्य कोपाच्चमूपतिः}% ४२

\twolineshloka
{वामेन गदया हन्तुं प्रचक्राम भुजेन तम्}
{सोऽपि च्छिन्नो भुजस्तस्य साङ्गदस्तीक्ष्णसायकैः}% ४३

\twolineshloka
{तदा प्रकुपितो वीरः पादाभ्यामहनल्लवम्}
{लवः पादाहतस्तस्य न चचाल रणाङ्गणे}% ४४

\twolineshloka
{स्रजाहतो द्विप इव चरणच्छेदनं व्यधात्}
{तदापि तं मौलिनासौ प्रहर्तुमुपचक्रमे}% ४५

\twolineshloka
{तदा लवश्चमूनाथं मन्यमानोऽधिपौरुषम्}
{करवालं समादाय करे कालानलोपमम्}% ४६

\twolineshloka
{अच्छिनच्छिर एतस्य महामुकुटशोभितम्}
{हाहाकारो महानासीच्चमूनाथे निपातिते}% ४७

\twolineshloka
{सैनिकाः परिसङ्क्रुद्धा लवं हन्तुं समागताः}
{लवस्तान्स्वशराघातैः पलायनपरान्व्यधात्}% ४८

\twolineshloka
{छिन्नाभिन्नाङ्गकाः केचिद्गता केचिद्रणाङ्गणात्}
{स निवार्याखिलान्योधान्विजगाह चमूं मुदा}% ४९

\twolineshloka
{वाराह इव निःश्वस्य प्रलये सुमहार्णवम्}
{गजा भिन्ना द्विधा जाता मौक्तिकैः पूरिता मही}% ५०

\twolineshloka
{दुर्गमाभूद्भटाग्र्याणां पर्वतैर्व्यापृता यथा}
{अश्वाः कनकपल्याणा रुचिरारत्नराजिताः}% ५१

\twolineshloka
{अपतन्रुधिराप्लुष्टे ह्रदे बल सुशोभिताः}
{रथिनः करमध्यस्थ धनुर्दण्डसुशोभिनः}% ५२

\twolineshloka
{रथोपस्थे निपतिताः स्वर्गगा इव वै सुराः}
{सन्दष्टौष्ठपुटा वक्त्र भ्रमल्लक्ष्मीविलक्षिताः}% ५३

\twolineshloka
{पतितास्तत्र दृश्यन्ते वीरा रणविशारदाः}
{सुस्राव शोणितसरिद्धयमस्तककच्छपा}% ५४

\twolineshloka
{महाप्रवाहललिता वैरिणां भयकारिका}
{केषाञ्चिद्बाहविश्छिन्नाः केषां पादा विकर्तिता}% ५५

\twolineshloka
{केषां कर्णाश्च नासाश्च केषां कवचकुण्डले}
{एवं तु कदनं जातं सेनान्यां पतिते रणे}% ५६

\twolineshloka
{सर्वे निपतिता वीरा न केचिज्जीवितास्ततः}
{लवो जयं रणे प्राप्य वैरिवृन्दं विजित्य च}% ५७

\twolineshloka
{अन्यागमनशङ्कायां मनः कुर्वन्नवैक्षत}
{केचिदुर्वरिता युद्धाद्भाग्येन न रणे मृताः}% ५८

\twolineshloka
{शत्रुघ्नं सविधे जग्मुः शंसितुं वृत्तमद्भुतम्}
{गत्वा ते कथयामासुर्यथावृत्तं रणाङ्गणे}% ५९

\twolineshloka
{कालजिन्निधनं बालाच्चित्रकारि रणोद्यमम्}
{तच्छ्रुत्वा विस्मयं प्राप्तः शत्रुघ्नस्तानुवाच ह}% ६०

\twolineshloka
{हसन्रोषाद्दशन्दन्तान्बालग्राह हयं स्मरन्}
{रे वीराः किं मदोन्मत्ता यूयं किं वा छलग्रहाः}% ६१

\twolineshloka
{किं वा वैकल्यमायातं कालजिन्मरणं कथम्}
{यः सङ्ख्ये वैरिवृन्दानां दारुणः समितिञ्जयः}% ६२

\twolineshloka
{तं कथं बालको जीयाद्यमस्यापि दुरासदम्}
{शत्रुघ्नवाक्यं संश्रुत्य वीराः प्रोचुरसृक्प्लुताः}% ६३

\twolineshloka
{नास्माकं मदमत्तादि न च्छलो न च देवनम्}
{कालजिन्मरणं सत्यं लवाज्जानीहि भूपते}% ६४

\twolineshloka
{बलं च कृत्स्नं मथितं बालेनातुलशौण्डिना}
{अतः परं तु यत्कार्यं ये प्रेष्या नृवरोत्तमाः}% ६५

\twolineshloka
{बालं ज्ञात्वा भवान्नात्र करोतु बलसाहसम्}
{इति श्रुत्वा वचस्तेषां वीराणां शत्रुहा तदा}% ६६

\onelineshloka
{सुमतिं च मतिश्रेष्ठमुवाच रणकारणे}% ६७

{॥इति श्रीपद्मपुराणे पातालखण्डे शेषवात्स्यायनसंवादे रामाश्वमेधे कुशलवयुद्धे सैन्यपराजय कालजित्सेनानीमरणं नाम षष्टितमोऽध्यायः॥६०॥}

\dnsub{एकषष्टितमोऽध्यायः}%\resetShloka

\uvacha{शत्रुघ्न उवाच}

\twolineshloka
{जानासि किं महामन्त्रिन्को बालो हयमाहरत्}
{येन मे क्षपितं सर्वं बलं वारिधिसन्निभम्}% १

\uvacha{सुमतिरुवाच}

\twolineshloka
{स्वामिन्नयं मुनिश्रेष्ठ वाल्मीकेराश्रमो महान्}
{क्षत्त्रियाणामत्र वासो नास्त्येव परतापन}% २

\twolineshloka
{इन्द्रो भविष्यति परममर्षी हयमाहरत्}
{पुरारिर्वान्यथा वाहं तव कः समुपाहरेत्}% ३

\twolineshloka
{कालजिद्येन नाशं वै प्राप्तः परमदारुणः}
{तं प्रति श्रीमहाराज गन्ता कः पुष्कलान्यतः}% ४

\twolineshloka
{त्वं च वीरैर्भटैः सर्वैराजभिः परिवारितः}
{तत्र गच्छस्व सैन्येन महता शत्रुकृन्तन}% ५

\twolineshloka
{गत्वा स जीवितं वीरं बद्ध्वा तु कुतुकार्थिने}
{दर्शयिष्यामि रामाय मतं मे त्विदमादृतम्}% ६

\twolineshloka
{इति वाक्यं समाकर्ण्य वीरान्सर्वान्समादिशत्}
{सैन्येन महता यात यूयमायामि पृष्ठतः}% ७

\twolineshloka
{निर्दिष्टास्ते क्षणाद्वीरा जग्मुर्यत्र लवो बली}
{धनुर्विस्फारयंस्तत्र सुदृढं गुणपूरितम्}% ८

\twolineshloka
{आयातं तन्महद्दृष्ट्वा बलं वीरप्रपूरितम्}
{न किञ्चिन्मनसा बिभ्येलवेन बलशालिना}% ९

\twolineshloka
{लवः सिंह इवोत्तस्थौ मृगान्मत्वाऽखिलान्भटान्}
{धनुर्विस्फारयन्रोषाच्छरान्मुञ्चन्सहस्रशः}% १०

\twolineshloka
{ते शरैः पीड्यमानास्तु महारोषेण पूरिताः}
{वीरं बालं मन्यमानाः सम्मुखं प्राद्रवंस्तदा}% ११

\twolineshloka
{वीरान्सहस्रशो दृष्ट्वा भ्रमिभिः पर्यवस्थितान्}
{लवो जवेन सन्धाय शरान्रोषप्रपूरितः}% १२

\twolineshloka
{भ्रमिराद्या सहस्रेण द्वितीयायुतसङ्ख्यया}
{तृतीयायुतयुग्मेन तुरीयायुतपञ्चभिः}% १३

\twolineshloka
{पञ्चमी लक्षयोधानां षष्ठी योधायुताधिकैः}
{सप्तमी लक्षयुग्मेन सप्तभिर्भ्रमिभिर्वृतः}% १४

\twolineshloka
{मध्ये लवो भ्रमिव्याप्तः स चरन्वह्निवत्तदा}
{दाहयामास सर्वान्वै सैनिकान्भ्रमिकारकान्}% १५

\twolineshloka
{काचित्खङ्गैः शरैः काचित्काचित्प्रासैश्च कुन्तलैः}
{पट्टिशैः परिघैः सर्वा भ्रमिर्भग्ना महात्मना}% १६

\twolineshloka
{सप्तभिर्भ्रमिभिर्मुक्तो रराज स कुशानुजः}
{मेघवृन्दविनिर्मुक्तः शशीव शरदागमे}% १७

\twolineshloka
{प्राहरत्सर्वथा योधान्भिन्दन्गजकरान्बहून्}
{छिन्दञ्छिरांसि वीराणां चक्रभ्रूणि महान्ति च}% १८

\twolineshloka
{अनेके पतिता वीरा लवबाणप्रपीडिताः}
{मुमुहुः समरेऽथान्ये नष्टा अन्ये सुकातराः}% १९

\twolineshloka
{पलायनपरं सैन्यं लवबाणप्रपीडितम्}
{वीक्ष्य वीरो रणे योद्धुं प्रायात्पुष्कलसंज्ञकः}% २०

\twolineshloka
{तिष्ठतिष्ठेति च वदन्रोषपूरितलोचनः}
{रथे सुहयशोभाढ्ये तिष्ठन्प्रायाल्लवं बली}% २१

\twolineshloka
{स लवं प्रत्युवाचाथ पुष्कलः परमास्त्रवित्}
{तिष्ठ दत्ते मया सङ्ख्ये रथे सुहयशोभिते}% २२


\threelineshloka
{पदातिना त्वया युद्धं करोमि किमथाहवे}
{तस्मात्तिष्ठ रथे पश्चाद्युद्ध्येऽहं भवता सह}
{एतद्वाक्यं निशम्यासौ लवः पुष्कलमब्रवीत्}% २३

\twolineshloka
{त्वया दत्ते रथे स्थित्वा युद्धं कुर्यामहं रणे}
{तदा मे पापमेव स्याज्जयः सन्दिग्ध एव हि}% २४

\twolineshloka
{न वयं ब्राह्मणा वीर प्रतिग्रहपरायणाः}
{वयं तु क्षत्रिया नित्यं दानकर्मक्रियारताः}% २५

\twolineshloka
{इदानीं त्वद्रथं कोपाद्भनज्मि प्रत्यहं भवान्}
{पादचारी भवत्येव पश्चाद्युद्धं करिष्यति}% २६

\twolineshloka
{पुष्कलो वाक्यमाकर्ण्य धर्मधैर्यसमन्वितम्}
{विसिस्माय चिरं चित्ते धनुः सज्यमथाकरोत्}% २७

\twolineshloka
{तमात्तधनुषं दृष्ट्वा लवः कोपसमन्वितः}
{चापं चिच्छेद पाणिस्थं शरसन्धानमाचरन्}% २८

\twolineshloka
{स यावत्स गुणं चापं कुरुते तावदुद्धतः}
{रथभङ्गं चकारास्य समरे प्रहसन्बली}% २९

\twolineshloka
{भग्नं रथं स्वकं वीक्ष्य धनुश्छिन्नं महात्मना}
{महावीरं मन्यमानः पदातिः प्राद्रवद्रणे}% ३०

\twolineshloka
{उभौ धनुर्धरौ वीरावुभावपि शरोद्धतौ}
{उभौ क्षतजविप्लुष्टौ छिन्नसन्नाहितावुभौ}% ३१

\twolineshloka
{परस्परं बाणघातविशीर्णावपुलक्षितौ}
{जयाकाङ्क्षां प्रकुर्वन्तौ परस्परवधैषिणौ}% ३२

\twolineshloka
{जयन्तकार्तिकेयौ वा पुरारिः पुरभिद्यथा}
{एवं परस्परं युद्धं प्रकुर्वाणौ रणाङ्गणे}% ३३

\twolineshloka
{पुष्कलः प्रत्युवाचाथ बालं शूरशिरोमणे}
{त्वादृशो न मया दृष्टः कश्चिद्वीरशिरोमणिः}% ३४

\twolineshloka
{शिरस्ते पातयाम्यद्य बाणैः शितसुपर्वभिः}
{मा पलायस्व समरे प्राणान्रक्षस्व संयतः}% ३५

\twolineshloka
{एवमुक्त्वा लवं वीरं चकार शरपञ्जरे}
{पुष्कलस्य शरा भूमौ नभसि व्याप्य संस्थिताः}% ३६

\twolineshloka
{शरपञ्जरमध्यस्थो लवः पुष्कलमब्रवीत्}
{महत्कर्म कृतं वीर यन्मां बाणैरपीडयत्}% ३७

\twolineshloka
{इत्युक्त्वा बाणसङ्घातं प्रच्छिद्य वचनं पुनः}
{जगाद पुष्कलं वीरः शरसन्धानकोविदः}% ३८

\twolineshloka
{पालयात्मानमाजिस्थं मच्छराघातपीडितः}
{पतिष्यसि महीपृष्ठे रुधिरेण परिप्लुतः}% ३९

\twolineshloka
{एवमुक्तं समाकर्ण्य पुष्कलः कोपसंयुतः}
{रणे संयोधयामास लवं वीरं महाबलम्}% ४०

\twolineshloka
{लवः प्रकुपितो बाणं तीक्ष्णं वैरिविदारणम्}
{जग्राह लवतः कोशादाशीविषमिव क्रुधा}% ४१

\twolineshloka
{जाज्वल्यमानः सशरश्चापमुक्तो लवस्य च}
{हृदयं भेत्तुमुद्युक्तश्छिन्नो भारतिनाशु सः}% ४२

\twolineshloka
{छिन्ने भारतिना सङ्ख्ये शरेण प्राणहारिणा}
{अत्यन्तं कुपितो घोरं शरमन्यं समाददे}% ४३

\twolineshloka
{आकर्णाकृष्टचापेन स मुक्तो निशितः शरः}
{बिभेद हृदयं तस्य पुष्कलस्य महारणे}% ४४

\twolineshloka
{भिन्नो वक्षसि वीरेण सायकेनाशुगामिना}
{पपात धरणीपृष्ठे महाशूरशिरोमणिः}% ४५

\twolineshloka
{पतितं तं समालोक्य पुष्कलं पवनात्मजः}
{गृहीत्वा राघवभ्रात्रे ददौ मूर्च्छासमन्वितम्}% ४६

\twolineshloka
{मूर्च्छितं तं समालोक्य शोकविह्वलमानसः}
{हनूमन्तं लवं हन्तुं निदिदेश क्रुधान्वितः}% ४७

\twolineshloka
{हनूमान्क्रोधसम्प्लुष्टो लवं सङ्ख्ये महाबलम्}
{विजेतुं तरसा प्रागाद्वृक्षमुद्यम्य शाल्मलिम्}% ४८

\twolineshloka
{वृक्षेण हतवान्मूर्ध्नि लवस्य हनुमान्बली}
{तमापतन्तं तरसा चिच्छेद शतधा लवः}% ४९

\twolineshloka
{छिन्ने नगे पुनः कोपाद्वृक्षानुत्पाट्य मूलतः}
{ताडयामास हृदये मस्तके च महाबलः}% ५०

\twolineshloka
{यान्यान्वृक्षान्समाहृत्याताडयत्पवनात्मजः}
{तांस्तांश्चिच्छेद तरसा बलवान्नतपर्वभिः}% ५१

\twolineshloka
{तदा शिलाः समुत्पाट्य गण्डशैलोपमाः कपिः}
{पातयामास शिरसि क्षिप्रवेगेन मारुतिः}% ५२

\twolineshloka
{स आहतः शिलासङ्घैः सङ्ख्ये कोदण्डमुन्नयन्}
{बाणैस्ताश्चूर्णयामास यन्त्रिकाभिर्यथा कणाः}% ५३

\twolineshloka
{तदात्यन्तं प्रकुपितो मारुतिः पुच्छवेष्टनम्}
{चकार समरोपान्ते लवस्य बलिनः कृती}% ५४

\twolineshloka
{स्वं पुच्छेन समाविद्धं वीक्ष्य स्वाम्बां हृदि स्मरन्}
{मुष्टिना ताडयामास लाङ्गूलं मारुतेर्बली}% ५५

\twolineshloka
{तन्मुष्टिघातव्यथितो मारुतिस्तममूमुचत्}
{स मुक्तः पुच्छतो युद्धे शरान्मुञ्चन्नभूद्बली}% ५६

\twolineshloka
{दुर्वारशरघातेन सम्पीडिततनुः कपिः}
{बाणवर्षं मन्यमानो दुःसहं समरे बहु}% ५७

\twolineshloka
{किङ्कर्तव्यमितोऽस्माभिः पलाय्य यदि गम्यते}
{तदा मे स्वामिनो लज्जा ताडयेद्बालकोऽत्र माम्}% ५८

\twolineshloka
{ब्रह्मदत्तवरत्वात्तु मूर्च्छा न मरणं नहि}
{दुःसहा बाणपीडात्र किं कर्तव्यं मयाधुना}% ५९

\twolineshloka
{शत्रुघ्नः समरे गत्वा जयं प्राप्नोतु बालकात्}
{अहं तावज्जयाकाङ्क्षी शये कपटमूर्च्छया}% ६०

\twolineshloka
{इत्येवं मानसे कृत्वा पपात रणमण्डले}
{पश्यतां सर्ववीराणां कपटेन विमूर्च्छितः}% ६१

\twolineshloka
{तं मूर्च्छितं समाज्ञाय हनूमन्तं महाबलम्}
{जघान सर्वान्नृपतीञ्छरमोक्षविचक्षणः}% ६२

{॥इति श्रीपद्मपुराणे पातालखण्डे शेषवात्स्यायनसंवादे रामाश्वमेधे हनुमत्पतनन्नामैकषष्टितमोऽध्यायः॥६१॥}

\dnsub{द्विषष्टितमोऽध्यायः}%\resetShloka

\uvacha{शेष उवाच}

\twolineshloka
{मूर्च्छितं मारुतिं श्रुत्वा शत्रुघ्नः शोकमाययौ}
{किङ्कर्तव्यं मया सङ्ख्ये बालकोऽयं महाबलः}% १

\twolineshloka
{स्वयं रथे हेममये तिष्ठन्वीरवरैः सह}
{योद्धुं प्रागाल्लवो यत्र विचित्ररणकोविदः}% २

\twolineshloka
{लवं ददर्श शिशुतां प्राप्तं राममिव क्षितौ}
{धनुर्बाणकरं वीरान्क्षिपन्तं रणमूर्धनि}% ३

\twolineshloka
{विचारयामास तदा कोऽयं रामस्वरूपधृक्}
{नीलोत्पलदलश्यामं वपुर्बिभ्रन्मनोहरम्}% ४

\twolineshloka
{एष वै देहतनुजा सुतो भवति नान्यथा}
{अस्मान्विजित्य समरे यास्यते मृगराडिव}% ५

\twolineshloka
{अस्माकं नो जयो भाव्यः शक्त्या विरहितात्मनाम्}
{अशक्ताः किं करिष्यामः समरे रणकोविदाः}% ६

\twolineshloka
{इत्येवं स विचार्याथ बालकं तु वचोऽब्रवीत्}
{रणे कुतुककर्तारं वीरकोटिनिपातकम्}% ७

\twolineshloka
{कस्त्वं बाल रणेऽस्माकं वीरान्पातयसि क्षितौ}
{न जानीषे बलं राज्ञो रामस्य दनुजार्दिनः}% ८

\twolineshloka
{का ते माता पिता कस्ते सुभाग्यो जयमाप्तवान्}
{नाम किं विश्रुतं लोके जानीयां ते महाबल}% ९

\twolineshloka
{मुञ्च वाहः कथं बद्धः शिशुत्वात्तत्क्षमामि ते}
{आयाहि रामं वीक्षस्व दास्यते बहुलं तव}% १०

\twolineshloka
{इत्युक्तो बालको वीरो वचः शत्रुघ्नमावदत्}
{किं ते नाम्नाथ पित्रा वा कुलेन वयसा तथा}% ११

\twolineshloka
{युध्यस्व समरे वीर चेत्त्वं बलयुतो भवेः}
{कुशं वीरं नमस्कृत्य पादयोर्याहि नान्यथा}% १२

\twolineshloka
{भ्राता रामस्य वीरो भूर्नावयोर्बलिनां वरः}
{वाहं विमोचय बलाच्छक्तिस्ते विद्यते यदि}% १३

\twolineshloka
{इत्युक्त्वा शरसन्धानं कृत्वा प्राहरदुद्भटः}
{हृदये मस्तके चैव भुजयो रणमण्डले}% १४

\twolineshloka
{तदा प्रकुपितो राजा धनुः सज्यमथाकरोत्}
{नादयन्मेघगम्भीरं त्रासयन्निव बालकम्}% १५

\twolineshloka
{बाणानपरिसङ्ख्यातान्मुमोच बलिनां वरः}
{बालो बलेन चिच्छेद सर्वांस्तान्सायकव्रजान्}% १६

\twolineshloka
{लवस्यानेकधा मुक्तैर्बाणैर्व्याप्तं महीतलम्}
{व्यतीपाते प्रदत्तस्य दानस्येवाक्षयं गताः}% १७

\twolineshloka
{ते बाणा व्योमसकलं व्याप्नुवँल्लवसन्धिताः}
{सूर्यमण्डलमासाद्य प्रवर्तन्ते समन्ततः}% १८

\twolineshloka
{मारुतो नाविशद्यत्र बाणपञ्जरगोचरे}
{मनुष्याणां तु का वार्ता क्षणजीवितशंसिनाम्}% १९

\twolineshloka
{तद्बाणान्विस्तृतान्दृष्ट्वा शत्रुघ्नो विस्मयं गतः}
{अच्छिनच्छतसाहस्रं बाणमोचनकोविदः}% २०

\twolineshloka
{ताञ्छिन्नान्सायकान्सर्वान्स्वीयान्दृष्ट्वा कुशानुजः}
{धनुश्चिच्छेद तरसा शत्रुघ्नस्य महीपतेः}% २१

\twolineshloka
{सोऽन्यद्धनुरुपादाय यावन्मुञ्चति सायकान्}
{तावद्बभञ्ज सरथं सायकैः शितपर्वभिः}% २२

\twolineshloka
{करस्थमच्छिनच्चापं सुदृढं गुणपूरितम्}
{तत्कर्मापूजयन्वीरा रणमण्डलवर्तिनः}% २३

\twolineshloka
{सच्छिन्नधन्वा विरथो हताश्वो हतसारथिः}
{अन्यं रथं समास्थाय ययौ योद्धुं लवं बलात्}% २४

\twolineshloka
{अनेकबाणनिर्भिन्नः स्रवद्रक्तकलेवरः}
{पुष्पितः किंशुक इव शुशुभे रणमध्यगः}% २५

\twolineshloka
{शत्रुघ्नबाणप्रहतः परं कोपमुपागमत्}
{बाणसन्धानचतुरः कुण्डलीकृत चापवान्}% २६

\twolineshloka
{विशीर्णकवचं देहं शिरोमुकुटवर्जितम्}
{स्रवद्रक्तपरिप्लुष्टं शत्रुघ्नस्य चकार सः}% २७

\twolineshloka
{तदा रामानुजः क्रुद्धो दशबाणाञ्छिताग्रकान्}
{मुमोच प्राणसंहारकारकान्कुपितो भृशम्}% २८

\twolineshloka
{स तांस्तांस्तिलशः कृत्वा बाणैर्निशितपर्वभिः}
{ताडयामास हृदये शत्रुघ्नस्याष्टभिः शरैः}% २९

\twolineshloka
{अत्यन्तं बाणपीडार्तो लवं बलिनमुत्स्मरन्}
{दुःसहं मन्यमानस्तं शरान्मुञ्चन्नभूत्तदा}% ३०

\twolineshloka
{तदा लवेन तीक्ष्णेन हृदि भिन्नो विशालके}
{अर्धचन्द्रसमानेन तीक्ष्णपर्वसुशोभिना}% ३१

\twolineshloka
{स विद्धो हृदि बाणेन पीडां प्राप्तः सुदारुणाम्}
{पपात स्यन्दनोपस्थे धनुःपाणिः सुशोभितः}% ३२

\twolineshloka
{शत्रुघ्नं मूर्छितं दृष्ट्वा नृपाश्च सुरथादयः}
{दुद्रुवुर्लवमुद्युक्ता जयप्राप्त्यै रणे तदा}% ३३

\twolineshloka
{सुरथो विमलो वीरो राजा वीरमणिस्तथा}
{सुमदो रिपुतापाद्याः परिवव्रुश्च संयुगे}% ३४

\twolineshloka
{केचित्क्षुरप्रैर्मुसलैः केचिद्बाणैः सुदारुणैः}
{प्रासैः परशुभिः केचित्सर्वतः प्राहरन्नृपाः}% ३५

\twolineshloka
{तानधर्मेण युद्धोत्कान्दृष्ट्वा वीरशिरोमणिः}
{दशभिर्दशभिर्बाणैस्ताडयामास संयुगे}% ३६

\twolineshloka
{ते बाणवर्षविहता रणमध्ये सुकोपनाः}
{केचित्पलायिताः केचिन्मुमुहुर्युद्धमण्डले}% ३७

\twolineshloka
{तावत्स राजा शत्रुघ्नो मूर्च्छां सन्त्यज्य सङ्गरे}
{लवं प्रायान्महावीरं योद्धुं बलसमन्वितः}% ३८

\twolineshloka
{आगत्य तं लवं प्राह धन्योसि शिशुसन्निभः}
{न बालस्त्वं सुरः कश्चिच्छलितुं मां समागतः}% ३९

\twolineshloka
{केनापि नहि वीरेण पातितो रणमण्डले}
{त्वयाहं प्रापितो मूर्च्छां समक्षं मम पश्यतः}% ४०

\twolineshloka
{इदानीं पश्य मे वीर्यं त्वां सङ्ख्ये पातयाम्यहम्}
{सहस्व बाणमेकं त्वं मापलायस्व बालक}% ४१

\twolineshloka
{इत्युक्त्वा समरे बालं शरमेकं समाददे}
{यमवक्त्रसमं घोरं लवणो येन घातितः}% ४२

\twolineshloka
{सन्धाय बाणं जाज्वल्यं हृदि भेत्तुं मनो दधत्}
{लवं वीरसहस्राणां वह्निवत्सर्वदाहकम्}% ४३

\twolineshloka
{तं बाणं प्रज्वलन्तं स द्योतयन्तं दिशो दश}
{दृष्ट्वा सस्मार बलिनं कुशं वैरिनिपातिनम्}% ४४

\twolineshloka
{यद्यस्मिन्समये वीरो भ्राता स्याद्बलवान्मम}
{तदा शत्रुघ्नवशता न मे स्याद्भयमुल्बणम्}% ४५

\twolineshloka
{एवं तर्कयतस्तस्य लवस्य च महात्मनः}
{हृदि लग्नो महाबाणो घोरः कालानलोपमः}% ४६

\twolineshloka
{मूर्च्छां प्राप तदा वीरो भूपसायकसंहतः}
{सङ्गरे सर्ववीराणां शिरोभिः समलङ्कृते}% ४७

{॥इति श्रीपद्मपुराणे पातालखण्डे शेषवात्स्यायनसंवादे रामाश्वमेधे लवमूर्च्छा नाम द्विषष्टितमोऽध्यायः॥६२॥}

\dnsub{त्रिषष्टितमोऽध्यायः}%\resetShloka

\uvacha{शेष उवाच}

\twolineshloka
{लवं विमूर्च्छितं दृष्ट्वा बलिवैरिविदारणम्}
{शत्रुघ्नो जयमापेदे रणमूर्ध्नि महाबलः}% १

\twolineshloka
{लवं बालं रथे स्थाप्य शिरस्त्राणाद्यलङ्कृतम्}
{रामप्रतिनिधिं मूर्त्या ततो गन्तुमियेष सः}% २

\twolineshloka
{स्वमित्रं शत्रुणा ग्रस्तमिति दुःखसमन्विताः}
{बालामात्रेऽस्य सीतायै त्वरिताः सन्न्यवेदयन्}% ३

\uvacha{बाला ऊचुः}

\twolineshloka
{मातर्जानकि ते पुत्रो बलाद्वाहमपाहरत्}
{कस्यचिद्भूपवर्यस्य बलयुक्तस्य मानिनः}% ४

\twolineshloka
{ततो युद्धमभूद्घोरं तस्य सैन्येन जानकि}
{तदा वीरेण पुत्रेण तव सर्वं निपातितम्}% ५

\twolineshloka
{पश्चादपि जयं प्राप्तः सुतस्तव मनोहरः}
{तं भूपं मूर्छितं कृत्वा जयमाप रणाङ्गणे}% ६

\twolineshloka
{ततो मूर्च्छां विहायैष राजा परमदारुणः}
{सङ्कुप्य पातयामास तव पुत्रं रणाङ्गणे}% ७

\twolineshloka
{अस्माभिर्वारितः पूर्वं मा गृहाण हयोत्तमम्}
{अस्मान्सर्वांश्च धिक्कृत्य ब्राह्मणान्वेदपारगान्}% ८

\twolineshloka
{इति वाक्यं शिशूनां सा समाकर्ण्य सुदारुणम्}
{पपात भूतलोपस्थे दुःखयुक्ता रुरोद ह}% ९

\uvacha{सीतोवाच}

\twolineshloka
{कथं नृपो दयाहीनो बालेन सह युध्यति}
{अधर्मकृतदुर्बुद्धिर्यो मद्बालं न्यपातयत्}% १०

\twolineshloka
{लव वीरभवान्कुत्र वर्ततेऽति बलान्वितः}
{कथं त्वं निष्कृपस्याहो राज्ञोऽहार्षीद्धयोत्तमम्}% ११

\twolineshloka
{त्वं बालस्ते दुराक्रान्ताः सर्वशस्त्रविशारदाः}
{रथस्था विरथस्त्वं वै कथं युद्धं समं भवेत्}% १२

\twolineshloka
{ताताहं तु त्वया सार्द्धं रामत्यागासुखं जहौ}
{इदानीं रहिता युष्मत्कथं जीवामि कानने}% १३

\twolineshloka
{एहि मां मुञ्च यज्ञाश्वं गच्छत्वेष महीपतिः}
{मद्दुःखं नाभिजानासि मम दुःखप्रमार्जकः}% १४

\twolineshloka
{कुशो यद्यभविष्यत्स रणे वीरशिरोमणिः}
{अमोचयिष्यदधुना भवन्तं भूपपार्श्वतः}% १५

\twolineshloka
{सोऽपि मद्दैवतो नास्ति समीपे किं करोम्यतः}
{दैवमेव ममाप्यत्र कारणं दुःखसम्भवे}% १६

\twolineshloka
{एवमादि बहुश्रीमत्येषा वै विललाप ह}
{पादाङ्गुष्ठेन लिखती भूमिं नेत्रद्वयाश्रुभिः}% १७

\twolineshloka
{बालान्प्रति जगादासौ पृथुकः स च भूपतिः}
{कथं मत्सुतमापात्य रणे कुत्र गमिष्यति}% १८

\twolineshloka
{इति वाक्यं वदत्येषा जानकी पतिदेवता}
{तावत्कुशस्तु सम्प्राप्त उज्जयिन्या महर्षिभिः}% १९

\twolineshloka
{माघासितचतुर्दश्यां महाकालं समर्च्य च}
{प्राप्य भूरिवरांस्तस्मादागमन्मातृसन्निधौ}% २०

\twolineshloka
{जानकीं विह्वलां दृष्ट्वा नेत्रोद्भूताश्रु विक्लवाम्}
{शोकविह्वलदीनाङ्गीं बभाषे यावदुत्सुकः}% २१

\twolineshloka
{तदा स्वबाहुरवदत्स्फुरद्युद्धाभिशंसनः}
{हृदये चरणोत्साहो बभूवातिरथस्य हि}% २२

\twolineshloka
{स प्रत्युवाच जननीं दीनगद्गदभाषिणीम्}
{मातस्तव गतं दुःखं मयि पुत्र उपस्थिते}% २३

\twolineshloka
{मयि जीवति ते नेत्रादश्रूणि भुवि नो पतन्}
{प्रसूमुवाचाश्रुखिन्नां दीनगद्गदभाषिणीम्}% २४

\twolineshloka
{कुशो दुःखमितः सद्यो दुःखितां धीरमानसः}
{मम भ्राता लवः कुत्र वर्तते वैरिमर्दनः}% २५

\twolineshloka
{सदा मामागतं ज्ञात्वा प्रहर्षन्सन्निधावियात्}
{न दृश्यते कथं वीरः कुत्र रन्तुं गतो बली}% २६

\twolineshloka
{केन वा सह बालत्वाद्गतो मां वै निरीक्षितुम्}
{किं त्वं रोदिषि मे मातर्लवः कुत्र स वर्तते}% २७

\twolineshloka
{तन्मे कथय सर्वं यत्तव दुःखस्य कारणम्}
{तच्छ्रुत्वा पुत्रवाक्यं सा दुःखिता कुशमब्रवीत्}% २८

\twolineshloka
{लवो धृतो नृपेणात्र केनचिद्धयरक्षिणा}
{बबन्ध बालको मेत्र हयं यागक्रियोचितम्}% २९

\twolineshloka
{तद्रक्षकान्बहूञ्जिग्ये एकोऽनेकान्रिपून्बली}
{राजा तं मूर्च्छितं कृत्वा बबन्ध रणमूर्धनि}% ३०

\twolineshloka
{बालका इति मामूचुः सहगन्तार एव हि}
{ततोऽहं दुःखिता जाता निशम्य लवमाधृतम्}% ३१


\threelineshloka
{त्वं मोचय बलात्तस्मात्काले प्राप्तो नृपोत्तमात्}
{निशम्य मातुर्वचनं कुशः कोपसमन्वितः}
{जगाद तां दशन्नोष्ठं दन्तैर्दन्तान्विनिष्पिषन्}% ३२

\uvacha{कुश उवाच}

\twolineshloka
{मातर्जानीहि तं मुक्तं लवं पाशस्य बन्धनात्}
{इदानीं हन्मि तं बाणैः समग्रबलवाहनम्}% ३३

\twolineshloka
{यदि देवोऽमरो वापि यदि शर्वः समागतः}
{तथापि मोचये तस्माद्बाणैर्निशितपर्वभिः}% ३४

\twolineshloka
{मा रोदिषि मातरिह वीराणां रणमूर्छितम्}
{कीर्तयेऽत्र भवत्येव पलायनमकीर्तये}% ३५

\twolineshloka
{देहि मे कवचं दिव्यं धनुर्गुणसमन्वितम्}
{शिरस्त्राणं च मे मातः करवालं तथाशितम्}% ३६

\twolineshloka
{इदानीं यामि समरे पातयामि बलं महत्}
{मोचयामि भ्रातरं स्वं रणमध्याद्विमूर्छितम्}% ३७

\twolineshloka
{न मोचयाम्यद्य पुत्रं तव मातर्महारणात्}
{तदा तौ मे भवत्पादौ संरुष्टौ भवतां क्षितौ}% ३८

\uvacha{शेष उवाच}

\twolineshloka
{इति वाक्येन सन्तुष्टा जानकी शुभलक्षणा}
{सर्वं प्रादादस्त्रवृन्दं जयाशीर्भिर्नियुज्यतम्}% ३९

\twolineshloka
{प्रयाहि पुत्र सङ्ग्रामं लवं मोचय मूर्च्छितम्}
{इत्याज्ञप्तः कुशः सङ्ख्ये कवची कुण्डली बली}% ४०

\twolineshloka
{मुकुटी करवाली च चर्मधारी धनुर्धरः}
{अक्षयाविषुधी कृत्वा स्कन्धयोः सिंहवीर्ययोः}% ४१

\twolineshloka
{जगाम तरसा नत्वा मातृपादौ रथाग्रणीः}
{वेगेन यावद्युद्धाय गच्छति क्षिप्रमाहवे}% ४२

\twolineshloka
{तावद्ददर्श स्वलवं वैरिवृन्दनिपातकम्}
{आयान्तं तं कुशं वीरा ददृशुः सुमहाभटाः}% ४३

\twolineshloka
{कृतान्तमिव संहर्तुं सर्वं विश्वमुपस्थितम्}
{लवो महाबलं दृष्ट्वा कुशं भ्रातरमागतम्}% ४४

\twolineshloka
{अत्यन्तं वह्निवद्युद्धे दिदीपे वायुना समम्}
{रथादुन्मुच्य चात्मानं युद्धाय स विनिर्गतः}% ४५

\twolineshloka
{कुशः सर्वान्रणस्थान्वै वीरान्पूर्वदिशि क्षिपत्}
{पश्चिमायां दिशि लवः कोपात्सर्वान्समैरयत्}% ४६

\twolineshloka
{कुशबाणव्यथाव्याप्ता लवसायकपीडिताः}
{सैन्ये जना मुने सर्वे उत्कल्लोलाम्बुधिभ्रमाः}% ४७

\twolineshloka
{कुशेन च लवेनाथ शरव्रातैः प्रपीडितम्}
{न शर्म लेभे सकलं सैन्यं वीरप्रपूरितम्}% ४८

\twolineshloka
{इतस्ततः प्रभग्नं तद्बलं त्रस्तं पुनः पुनः}
{न कुत्रचिद्रणे स्थित्वा युद्धमैच्छद्बलान्वितः}% ४९

\twolineshloka
{एतस्मिन्समये राजा शत्रुघ्नः परतापनः}
{कुशं वीरं ययौ योद्धुं तादृशं लवसन्निभम्}% ५०

\twolineshloka
{कुशं दृष्ट्वा बलाक्रान्तं राममूर्तिसमप्रभम्}
{रथे तिष्ठन्हेममये जगाद परवीरहा}% ५१

\uvacha{शत्रुघ्न उवाच}

\twolineshloka
{कोऽसि त्वं सन्निभो भ्रात्रा लवेन सुमहाबलः}
{किं नामासि महावीर कस्ते तातः क्व ते प्रसूः}% ५२

\twolineshloka
{कथं वने द्विजैर्जुष्टे तिष्ठसि त्वं नरर्षभ}
{सर्वं शंस यथायुध्ये त्वया सह महाबल}% ५३

\twolineshloka
{इति वाक्यं समाकर्ण्य कुशः प्रोवाच भूमिपम्}
{मेघगम्भीरया वाचा नादयन्रणमण्डलम्}% ५४

\twolineshloka
{केवलं सुषुवे सीता पतिव्रतपरायणा}
{वने वसावो वाल्मीकेश्चरणार्चनतत्परौ}% ५५

\twolineshloka
{मातृसेवासमुद्युक्तौ सर्वविद्याविशारदौ}
{कुशो लव इति प्रख्यामागतौ भूपतेऽनघ}% ५६

\twolineshloka
{कस्त्वं वीरो रणश्लाघी किमर्थं हयसत्तमः}
{मुक्तोऽस्ति समरे त्वद्य जेतासि बलसंयुतः}% ५७

\twolineshloka
{युध्यस्व त्वं मया सार्द्धं यदि वीरोऽसि भूमिप}
{इदानीं पातयिष्यामि भवन्तं रणमूर्धनि}% ५८

\twolineshloka
{शत्रुघ्नस्तं सुतं ज्ञात्वा सीताया रामसम्भवम्}
{विसिष्माय स्वयं चित्ते कोपाद्धनुरुपाददत्}% ५९

\twolineshloka
{तमात्तधनुषं दृष्ट्वा कुशः कोपसमन्वितः}
{विस्फारयामास धनुः स्वीयं सुदृढमुत्तमम्}% ६०

\twolineshloka
{मुमोच बाणान्निशिताञ्छत्रुघ्नः सर्वशस्त्रवित्}
{तांश्चिच्छेद कुशः सर्वांल्लीलया प्रहसन्रणे}% ६१

\twolineshloka
{बाणाश्च शतसाहस्राः कुशस्य च नृपस्य च}
{भुवनं व्याप्नुवन्सर्वं तच्चित्रमभवन्मुने}% ६२

\twolineshloka
{अग्न्यस्त्रेण कुशः सर्वान्ददाह तरसा बली}
{शमयामास तं भूपः पर्जन्यास्त्रेण वीर्यवान्}% ६३


\threelineshloka
{शमयामास तं भूपो वायव्येनातिविक्रमः}
{तदा वायुरभूत्तीव्रः सर्वतो रणमण्डले}
{पर्वतास्त्रेण तं वायुं क्षोभयन्तं समावृणोत्}% ६४

\threelineshloka
{वज्रास्त्रेण नृपः सङ्ख्ये चिच्छेद सनगोपलान्}
{तदा नारायणास्त्रं स मुमोच कुश उद्भटः}
{नारायणं तदा भूपं नाशकत्परिबाधितुम्}% ६५

\twolineshloka
{तदा प्रकुपितोऽत्यतं कुशः कोपपरायणः}
{उवाच भूपं शत्रुघ्नं महाबलपराक्रमम्}% ६६

\twolineshloka
{जानामि त्वां महावीरं सङ्ग्रामे जयकारिणम्}
{यत्त्वां नारायणास्त्रं मे न बबाधे भयानकम्}% ६७

\twolineshloka
{इदानीं पातयाम्यद्य भूमौ त्वां नृपते शरैः}
{त्रिभिश्चेन्नकरोम्येतत्प्रतिज्ञां तर्हि मे शृणु}% ६८

\twolineshloka
{यो मनुष्यवपुः प्राप्य दुर्लभं पुण्यकोटिभिः}
{तन्नाद्रियेत सम्मोहात्तस्य मेस्त्वत्र पातकम्}% ६९

\twolineshloka
{सावधानो भवानत्र भवतु प्रधनाङ्गणे}
{पातयामि क्षितौ सद्य इत्युक्त्वा स्वशरासने}% ७०

\twolineshloka
{शरं संरोपयामास घोरं कालानलप्रभम्}
{लक्षीकृत्य रिपोर्वक्षो विपुलं कठिनं महत्}% ७१

\twolineshloka
{तं सन्धितं शरं दृष्ट्वा शत्रुघ्नः कोपमूर्च्छितः}
{मुमोच बाणान्निशितान्कुशत्वग्भेदकारकान्}% ७२

\twolineshloka
{स बाणो हृदयं तस्य भेत्तुं तत्प्रचचाल वै}
{घोररूपो वह्निसमआशीविषवदुच्छ्वसन्}% ७३

\twolineshloka
{स बाणो नृपवर्येण रामं स्मृत्वाशुलक्षितः}
{चिच्छेद कुशमुक्तं स सायकं शितपर्वकम्}% ७४

\twolineshloka
{तदात्यन्तं प्रकुपितः कुशो बाणस्य कृन्तनात्}
{अपरं सायकं चापे दधार शितपर्वकम्}% ७५

\twolineshloka
{स यावत्तदुरो भेत्तुं करोति च बलोद्धुरः}
{तं तावदच्छिनत्तस्य शरं कालानलप्रभम्}% ७६

\twolineshloka
{तदा कुशो मातृपादौ स्मृत्वा रोषसमन्वितः}
{तृतीयं चापके स्वीये दधार शरमद्भुतम्}% ७७

\twolineshloka
{शत्रुघ्नस्तमपि क्षिप्रं च्छेत्तुं बाणं समाददे}
{तावद्विद्धः शरेणासौ पपात धरणीतले}% ७८

\twolineshloka
{हाहाकारो महानासीच्छत्रुघ्ने विनिपातिते}
{जयमापकुशस्तत्र स्वबाहुबलदर्पितः}% ७९

{॥इति श्रीपद्मपुराणे पातालखण्डे शेषवात्स्यायनसंवादे रामाश्वमेधे शत्रुघ्नमूर्च्छने कुशजयो नाम त्रिषष्टितमोऽध्यायः॥६३॥}

\dnsub{चतुःषष्टितमोऽध्यायः}%\resetShloka

\uvacha{शेष उवाच}

\twolineshloka
{शत्रुघ्नं पतितं वीक्ष्य सुरथः प्रवरो नृपः}
{प्रययौ मणिना सृष्टे रथे तिष्ठन्महाद्भुते}% १

\twolineshloka
{पुष्कलस्तु रणे पूर्वं पातितः स विचारयन्}
{लवं ययौ तदा योद्धुं महावीरबलोन्नतम्}% २

\twolineshloka
{सुरथः कुशमासाद्य बाणान्मुञ्चन्ननेकधा}
{व्यथयामास समरे महावीरशिरोमणिः}% ३

\twolineshloka
{सुरथं विरथं चक्रे बाणैर्दशभिरुच्छिखैः}
{धनुश्चिच्छेद तरसा सुदृढं गुणपूरितम्}% ४

\twolineshloka
{अस्त्रप्रत्यस्त्रसंहारैः क्षैपणैः प्रतिक्षेपणैः}
{अभवत्तुमुलं युद्धं वीराणां रोमहर्षणम्}% ५

\twolineshloka
{अत्यन्तं समरोद्युक्ते सुरथे दुर्जये नृपे}
{कुशः सञ्चिन्तयामास किङ्कर्तव्यं रणे मया}% ६

\twolineshloka
{विचार्य निशितं घोरं सायकं समुपाददे}
{हननाय नृपस्यास्य महाबलसमन्वितः}% ७

\twolineshloka
{तमागतं शरं दृष्ट्वा कालानलसमप्रभम्}
{छेत्तुं मतिं चकाराशु तावल्लग्नो महाशरः}% ८

\twolineshloka
{मुमूर्च्छ समरे वीरो महावीरबलस्ततः}
{पपात स्यन्दनोपस्थे सारथिस्तमुपाहरत्}% ९

\twolineshloka
{सुरथे पतिते दृष्ट्वा कुशं जयसमन्वितम्}
{त्रासयन्तं वीरगणानियाय पवनात्मजः}% १०

\twolineshloka
{समीरसूनुं प्रबलमायान्तं वीक्ष्य वानरम्}
{जहास दर्शयन्दन्तान्कोपयन्निव तं क्रुधा}% ११

\twolineshloka
{उवाच च हनूमन्तमेहि त्वं मम सम्मुखम्}
{भेत्स्ये बाणसहस्रेण मृतो यास्यसि यामिनीम्}% १२

\twolineshloka
{इत्युक्तो हनुमांज्ञात्वा रामसूनुं महाबलम्}
{स्वामिकार्यं प्रकर्तव्यमिति कृत्वा प्रधावितः}% १३

\twolineshloka
{शालमुत्पाट्य तरसा विशालं शतशाखिनम्}
{कुशं वक्षसि संलक्ष्य ययौ योद्धुं महाबलः}% १४

\twolineshloka
{शालहस्तं समायान्तं हनूमन्तं महाबलम्}
{त्रिभिः क्षुरप्रैर्विव्याध हृदि चन्द्रोपमैर्बली}% १५

\twolineshloka
{स बाणविद्धस्तरसा कुशेन बलशालिना}
{शालेन हृदि सञ्जघ्ने दन्तान्निष्पिष्य मारुतिः}% १६

\twolineshloka
{शालाहतस्तदा बालः किञ्चिन्नाकम्पत स्मयात्}
{तदा वीराः प्रशंसां तु प्रचक्रुस्तस्य बाल्यतः}% १७

\twolineshloka
{स शालेन हतो वीरः संहारास्त्रं समाददे}
{संहन्तुं वैरिणं कोपात्कुशः स परमास्त्रवित्}% १८

\twolineshloka
{संहारास्त्रं समालोक्य दुर्जयं कुशमोचितम्}
{दध्यौ रामं स्वमनसा भक्तविघ्नविनाशकम्}% १९

\twolineshloka
{तदा मुक्तं कुशेनाशु तदस्त्रं हृदि मारुतेः}
{लग्नं महाव्यथाकारि तेन मूर्च्छामितः पुनः}% २०

\twolineshloka
{मूर्च्छां प्राप्तं तु तं दृष्ट्वा प्लवङ्गं बलसंयुतः}
{विव्याध सायकैस्तीक्ष्णैः सैन्यं तत्सकलं महत्}% २१

\twolineshloka
{तस्य बाणायुतैर्भग्नं बलं सर्वं रणाङ्गणे}
{पलायनपरं जातं चतुरङ्गसमन्वितम्}% २२

\twolineshloka
{तदा कपिपतिः कोपात्सुग्रीवो रक्षको महान्}
{अभ्यधावन्नगान्नैकानुत्पाट्य कुशमुद्भटम्}% २३

\twolineshloka
{कुशः सर्वान्प्रचिच्छेद लीलया प्रहसन्नगान्}
{पुनरप्यागतान्वृक्षांश्चिच्छेद तरसा बली}% २४

\twolineshloka
{अनेकबाणव्यथितः सुग्रीवः समराङ्गणे}
{जग्राह पर्वतं घोरं कुशमस्तकमध्यतः}% २५

\twolineshloka
{कुशस्तं नगमायान्तं वीक्ष्य बाणैरनेकधा}
{निष्पिपेष चकाराशु महारुद्राङ्गयोग्यताम्}% २६

\twolineshloka
{सुग्रीवस्तन्महत्कर्म दृष्ट्वा बालेन निर्मितम्}
{जयाशाप्रतिनिर्वृत्तो बभूव समराङ्गणे}% २७

\twolineshloka
{रणमध्ये दुराक्रान्तं कुशं लाङ्गूलताडकम्}
{अत्यमर्षीरुषाक्रान्तस्तं हन्तुं नगमाददे}% २८

\twolineshloka
{आत्मानं हन्तुमुद्युक्तं वीक्ष्य सुग्रीवमादरात्}
{ताडयामास बहुभिः सायकैः शितपर्वभिः}% २९

\twolineshloka
{स ताडितो बहुविधैः शरैः पीडासमन्वितः}
{कुशं हन्तुं समारब्धो ययौ शालं समाददे}% ३०

\twolineshloka
{तदापि च कुशो वीरो वारुणास्त्रं समाददे}
{बबन्ध तं च पाशेन दृढेन स लवाग्रजः}% ३१

\twolineshloka
{स बद्धः पाशकैः स्निग्धैः कुशेन बलशालिना}
{पपात रणमध्ये वै महावीरैरलङ्कृते}% ३२

\twolineshloka
{सुग्रीवं पतितं दृष्ट्वा वीराः सर्वत्र दुद्रुवुः}
{जयमाप लवभ्राता महावीरशिरोमणिः}% ३३

\twolineshloka
{तावल्लवो भटाञ्जित्वा पुष्कलं चाङ्गदं तथा}
{प्रतापाग्र्यं वीरमणिं तथान्यानपि भूभुजः}% ३४

\twolineshloka
{जयं प्राप्य रणे वीरो लवो भ्रातरमागमत्}
{सङ्ग्रामे जयकर्तारं वैरिकोटिनिपातकम्}% ३५

\twolineshloka
{परस्परं प्रहृषितौ परिरम्भं प्रकुर्वतः}
{जयं प्राप्तौ तदा वार्तां मुने चक्रतुरुन्मदौ}% ३६

\uvacha{लव उवाच}

\twolineshloka
{भ्रातस्तव प्रसादेन निस्तीर्णो रणतोयधिः}
{इदानीं वीररणकं शोधयावः सुशोभितम्}% ३७

\twolineshloka
{इत्युक्त्वा त्वरितं वीरो जग्मतुस्तौ कुशीलवौ}
{राज्ञो मौलिमणिं चित्रं जग्राह कनकाचितम्}% ३८

\twolineshloka
{पुष्कलस्य लवो वीरो जग्राह मुकुटं शुभम्}
{अङ्गदे च महानर्घ्ये शत्रुघ्नस्यापरस्य च}% ३९

\twolineshloka
{गृहीत्वा शस्त्रसङ्घातं हनूमन्तं कपीश्वरम्}
{सुग्रीवं सविधे गत्वा उभावपि बबन्धतुः}% ४०

\twolineshloka
{पुच्छे वायुसुतस्यायं गृहीत्वा तु कुशानुजः}
{भ्रातरं प्रत्युवाचेदं नेष्यामि स्वकमन्दिरम्}% ४१

\twolineshloka
{आवयोर्जननी प्रीत्यै गृहीत्वा पुच्छके त्वहम्}
{क्रीडार्थमृषिपुत्राणां कौतुकार्थं ममैव च}% ४२

\twolineshloka
{एतच्छ्रुत्वा ततो वाक्यमुवाच च कुशो लवम्}
{अहमेनं ग्रहीष्यामि वानरं बलिनं दृढम्}% ४३

\twolineshloka
{इत्येवं भाषमाणौ तौ बद्ध्वा तौ बलिनां वरौ}
{पुच्छयोर्बलिनौ धृत्वा जग्मतुः स्वाश्रमं प्रति}% ४४

\twolineshloka
{स्वाश्रमाय प्रगच्छन्तौ वीक्ष्य तौ कपिसत्तमौ}
{कम्पमानौ जगदतुरन्योन्यं भीतया गिरा}% ४५

\twolineshloka
{हनूमान्कपिराजानं प्रत्युवाच भयार्द्रधीः}
{एतौ रामसुतावस्मान्नेष्यतः स्वाश्रमं प्रति}% ४६

\twolineshloka
{मया पूर्वं कृतं कर्म जानकीं प्रतिगच्छता}
{तत्र मे जानकी देवी सम्मुखाभून्मनोहरा}% ४७

\twolineshloka
{सा मां द्रक्ष्यति वैदेही बद्धं पाशेन वैरिणा}
{तदा हसिष्यति वरा त्रपा मेऽत्र भविष्यति}% ४८

\twolineshloka
{मया किमत्र कर्तव्यं प्राणत्यागो भविष्यति}
{महद्दुःखं चापतितं स रामः किं करिष्यति}% ४९

\twolineshloka
{सुग्रीवस्तद्वचः श्रुत्वा ममाप्येवं महाकपे}
{नेष्यते यदि मामेवं निधनं तु भविष्यति}% ५०

\twolineshloka
{एवं कथयतोरेव ह्यन्योन्यं भयभीतयोः}
{कुशो लवश्च भवनं मातुः प्रापतुरोजसा}% ५१

\twolineshloka
{तावायातौ समीक्ष्यैव जहर्ष जननी तयोः}
{अन्योन्यं परमप्रीत्या परिरेभे निजौ सुतौ}% ५२

\twolineshloka
{ताभ्यां पुच्छगृहीतौ तौ वानरौ वीक्ष्य जानकी}
{हनूमन्तं च सुग्रीवं सर्ववीरं कपीश्वरम्}% ५३

\twolineshloka
{जहास पाशबद्धौ तौ वीक्षमाणा वराङ्गना}
{उवाच च विमोक्षार्थं वदन्ती वचनं वरम्}% ५४

\twolineshloka
{पुत्रौ प्रमुञ्चतं कीशौ महावीरौ महाबलौ}
{ईक्षन्तौ मां यदि स्फीतौ प्राणत्यागं करिष्यतः}% ५५

\twolineshloka
{अयं वै हनुमान्वीरो यो ददाह दनोः पुरीम्}
{अयमप्यृक्षराजो हि सर्ववानरभूमिपः}% ५६

\twolineshloka
{किमर्थं विधृतौ कुत्र किं वा कृतमनादरात्}
{पुच्छे युवाभ्यां विधृतौ स महान्विस्मयोऽस्ति मे}% ५७

\twolineshloka
{इति मातुर्वचः श्लक्ष्णं वीक्ष्यतां पुत्रकौ तदा}
{ऊचतुर्विनयश्रेष्ठौ महाबलसमन्वितौ}% ५८

\twolineshloka
{मातः कश्चन भूपालो रामो दाशरथिर्बली}
{तेन मुक्तो हयः स्वर्णभालपत्रः सुशोभितः}% ५९

\twolineshloka
{तत्रैवं लिखितं मातरेकवीराप्रसूर्मम}
{ये क्षत्रियास्ते गृह्णन्तु नोचेत्पादतलार्चकाः}% ६०

\twolineshloka
{तदा मया विचारो वै कृतः स्वान्ते पतिव्रते}
{भवती क्षत्रिया किं न वीरसूः किं न वा भवेत्}% ६१

\twolineshloka
{धार्ष्ट्यं तद्वीक्ष्य भूपस्य गृहीतोऽश्वो मया बलात्}
{जितं कुशेन वीरेण सैन्यं तत्पातितं रणे}% ६२

\twolineshloka
{मुकुटोऽयं भूमिपतेर्जानीहि पतिदेवते}
{अयमप्यन्यवीरस्य पुष्कलस्य महात्मनः}% ६३

\twolineshloka
{जानीहि मुकुटं त्वन्यं मणिमुक्ताविराजितम्}
{अश्वोऽयं मे मनोहारी कामयानो हि भूपतेः}% ६४

\twolineshloka
{आरोहणाय मद्भ्रातुर्जानीहि बलिनो वरे}
{इमौ कीशौ मया रन्तुमानीतौ बलिनां वरौ}% ६५

\twolineshloka
{कौतुकार्थं तवैवैतौ सङ्ग्रामे युद्धकारकौ}
{इति वाक्यं समाकर्ण्य जानकी पतिदेवता}% ६६

\onelineshloka*
{जगाद पुत्रौ तौ वीरौ मोचयेथां पुनः पुनः}

\uvacha{सीतोवाच}

\onelineshloka
{युवाभ्यामनयः सृष्टो हृतो रामहयो महान्}% ६७

\twolineshloka
{अनेके पातिता वीरा इमौ बद्धौ कपीश्वरौ}
{पितुस्तव हयो वीरो यागार्थं मोचितोऽमुना}% ६८

\twolineshloka
{तस्यापि हृतवन्तौ किं वाजिनं मखसत्तमे}
{मुञ्चतं प्लवगावेतौ मुञ्चतं वाजिनां वरम्}% ६९

\twolineshloka
{क्षाम्यतां भूपतेर्भ्राता शत्रुघ्नः परकोपनः}
{जनन्यास्तद्वचः श्रुत्वा ऊचतुस्तां बलान्वितौ}% ७०

\twolineshloka
{क्षात्रधर्मेण तं भूपं जितवन्तौ बलान्वितम्}
{नास्माकमनयोर्भावि क्षात्रधर्मेण युध्यताम्}% ७१

\onelineshloka
{वाल्मीकिना पुरा प्रोक्तमस्माकं पठतां पुरः}% ७२

\twolineshloka
{कण्वस्याश्रमकेवाहं धृत्वा यागक्रियोचितम्}
{तस्मात्सुतः स्वपित्रापि युध्येद्भ्रात्रापि चानुजः}% ७३

\twolineshloka
{गुरुणा शिष्य एवापि तस्मान्नो पापसम्भवः}
{त्वदाज्ञातो ऽधुना चावां दास्यावो हयमुत्तमम्}% ७४

\twolineshloka
{मोक्ष्यावः कीशावेतौ हि करिष्यावो वचस्तव}
{इत्युक्त्वा मातरं वीरौ गतौ रणे कपीश्वरौ}% ७५

\twolineshloka
{अमुञ्चतां हयं चापि हयमेधक्रियोचितम्}
{सीतादेवी स्वपुत्राभ्यां श्रुत्वा सैन्यं निपातितम्}% ७६

\twolineshloka
{श्रीरामं मनसा ध्यात्वा भानुमैक्षत साक्षिणम्}
{यद्यहं मनसा वाचा कर्मणा रघुनायकम्}% ७७

\twolineshloka
{भजामि नान्यं मनसा तर्हि जीवेदयं नृपः}
{सैन्यं चापि महत्सर्वं यन्नाशितमिदं बलात्}% ७८

\twolineshloka
{पुत्राभ्यां तत्तु जीवेत मत्सत्याज्जगताम्पते}
{इति यावद्वचो ब्रूते जानकीपतिदेवता}% ७९

\onelineshloka
{तावद्बलं च तत्सर्वं जीवितं रणमूर्द्धनि}% ८०

{॥इति श्रीपद्मपुराणे पातालखण्डे शेषवात्स्यायनसंवादे रामाश्वमेधे सैन्यजीवनं नाम चतुःषष्टितमोऽध्यायः॥६४॥}

\dnsub{पञ्चषष्टितमोऽध्यायः}%\resetShloka

\uvacha{शेष उवाच}

\twolineshloka
{क्षणान्मूर्च्छां जहौ वीरः शत्रुघ्नः समराङ्गणे}
{अन्येऽपि वीराबलिनो मूर्च्छां प्राप्ताः सुजीविताः}% १

\twolineshloka
{शत्रुघ्नो वाजिनां श्रेष्ठं ददर्श पुरतः स्थितम्}
{आत्मानं च शिरस्त्राणरहितं सैन्यजीवितम्}% २

\twolineshloka
{वीक्ष्य चित्रमिदं स्वान्ते चकारच जगाद ह}
{सुमतिं मन्त्रिणां श्रेष्ठं मूर्च्छाविरहितं तदा}% ३

\twolineshloka
{कृपां कृत्वा हयं प्रादाद्बालो यज्ञस्य पूर्तये}
{गच्छाम रामं तरसा हयागमनकाङ्क्षिणम्}% ४

\twolineshloka
{इत्युक्त्वा स रथे स्थित्वा हयमादाय वेगतः}
{ययौ तदाश्रमाद्दूरं भेरीशङ्खविवर्जितः}% ५

\twolineshloka
{तत्पृष्ठतो महासैन्यं चतुरङ्गसमन्वितम्}
{चचाल कुर्वन्सम्भग्नं स्वभारेण फणीश्वरम्}% ६

\twolineshloka
{जवेन जाह्नवीं तीर्त्वा कल्लोलजलशालिनीम्}
{जगाम विषये स्वीये स्वकीयजनशोभिते}% ७

\twolineshloka
{पुष्कलेन युतो राजा सुरथेन समन्वितः}
{रथे मणिमये तिष्ठन्महत्कोदण्डधारकः}% ८

\twolineshloka
{हयं तं पुरतः कृत्वा रत्नमालाविभूषितम्}
{श्वेतातपत्रं तस्यैव मूर्ध्नि चामरभूषितम्}% ९

\twolineshloka
{अनेकरथसाहस्रैः परितो बलिभिर्नृपैः}
{उद्यत्कोदण्डललितैर्वीरनादविभूषितैः}% १०

\twolineshloka
{क्रमेण नगरीं प्राप सूर्यवंशविभूषिताम्}
{अनेकैः केतुभिः श्रेष्ठैर्भूषितां दुर्गराजिताम्}% ११

\twolineshloka
{रामः श्रुत्वा हयं प्राप्तं शत्रुघ्नेन सहामुना}
{पुष्कलेन च वीरेण ययौ हर्षमनेकधा}% १२

\twolineshloka
{कटकं निर्दिदेशासौ चतुरङ्गं महाबलम्}
{लक्ष्मणं प्रेषयामास भ्रातरं बलिनां वरम्}% १३

\twolineshloka
{लक्ष्मणः सैन्यसहितो गत्वा भ्रातरमागतम्}
{परिरेभे मुदाक्रान्तः क्षतशोभितगात्रकम्}% १४

\twolineshloka
{सर्वत्र कुशलं पृष्टो वार्तां चात्र चकार सः}
{परमं हर्षमापन्नः शत्रुघ्नः सङ्गतो मुदा}% १५

\twolineshloka
{सौमित्रिः स्वरथे स्थित्वा भ्रात्रा सह महामनाः}
{सैन्येन महता वीरो ययौ स्वनगरीं प्रति}% १६

\twolineshloka
{सरयूः पुण्यसलिला पवित्रित जगत्त्रया}
{रामपादरजः पूता शरच्चन्द्रसमप्रभा}% १७

\twolineshloka
{हंसकारण्डवाकीर्णा चक्रवाकोपशोभिता}
{विचित्रतरवर्णैश्च पक्षिभिर्नादिता भृशम्}% १८

\twolineshloka
{मण्डपास्तत्र बहुशो रामचन्द्रेणकारिताः}
{ब्राह्मणानां वेदविदां पृथक्पाठनिनादकाः}% १९

\twolineshloka
{क्षत्रियास्तत्र बहवो धनुःपाणि सुशोभिताः}
{ज्याटङ्कारेण बहुना नादयन्तो महीतलम्}% २०

\twolineshloka
{भुञ्जते ब्राह्मणा यत्र विचित्रान्नैर्मनोहरैः}
{परस्परं प्रपश्यन्तो वार्तां चक्रुर्मनोहराम्}% २१

\twolineshloka
{पायसान्नानि शुभ्राणि चन्द्रकान्तिसमानि च}
{क्षीराज्यबहुयुक्तानि शर्करामिश्रितानि च}% २२

\twolineshloka
{अपूपास्तत्र बहुलाश्चन्द्रबिम्बसमाः श्रिया}
{कर्पूरादिसुगन्धेन वासिताः सुमनोहराः}% २३

\twolineshloka
{फेनिकाघटकाः स्निग्धाः शतच्छिद्रा विरन्ध्रकाः}
{शष्कुल्यो मण्डकामृष्टा मधुरान्नसमन्विताः}% २४

\twolineshloka
{भक्तं कुमुदसङ्काशं मुद्गदालिविमिश्रितम्}
{सुगन्धेन समायुक्तमत्यन्तं प्रीतिदायकम्}% २५

\twolineshloka
{ओदनो दधिना युक्तो भीमसेनसमन्वितः}
{स्वादुपाककरैः सृष्टः पात्रे मुक्तः प्रवेषकैः}% २६

\twolineshloka
{तत्र केचिद्द्विजाः पात्रे निक्षिप्तं वीक्ष्य पायसम्}
{परस्परं ते प्रत्यूचुः किमिदं दृश्यतेऽद्भुतम्}% २७

\twolineshloka
{किं चन्द्रबिम्बं नभसः पतितं तमसो भयात्}
{अमृतं तु भवत्यत्र मृत्युनाशकमद्भुतम्}% २८

\twolineshloka
{तच्छ्रुत्वा रोषताम्राक्षः प्रोवाचान्यो द्विजोत्तमः}
{नभवत्येव चन्द्रस्य बिम्बं त्वमृतविप्लुतम्}% २९

\twolineshloka
{एकमिन्दोर्वपुस्त्वेतद्दृश्यते सदृशं कथम्}
{ब्राह्मणानां सहस्रस्य पात्रे पात्रे पृथक्पृथक्}% ३०

\twolineshloka
{ततो जानीहि कुमुदं कर्पूरं वा भविष्यति}
{मा जानीहि मृगाङ्कस्य बिम्बं शुभ्रश्रियान्वितम्}% ३१

\twolineshloka
{तावदन्यो रुषाक्रान्तो धुन्वन्स्वं मस्तकं तथा}
{न जानन्ति द्विजा मूढाः स्वादुज्ञाना विचक्षणाः}% ३२

\twolineshloka
{इदं तु क्षौद्रकन्दस्यरसेन परिपाचितम्}
{जानीहि शतपत्रस्य पुष्पाणि मधुराणि च}% ३३

\twolineshloka
{एवं परस्परं विप्राः कन्दमूलफलाशिनः}
{तर्कयन्ति मुने प्रीता रसज्ञानेऽतिलोलुपाः}% ३४

\twolineshloka
{तावदन्यो द्विजः प्राह क्षत्त्रियाणां वरं जनुः}
{भोक्ष्यन्ते तादृशं त्वन्नं महत्पुण्यैरुपस्कृतम्}% ३५

\twolineshloka
{तदा तं प्राब्रवीद्विप्रो दत्तस्य फलमीदृशम्}
{ये ददत्यग्रजन्मभ्यः प्राप्नुवन्ति त ईप्सितम्}% ३६

\twolineshloka
{यैरर्चितो नैव हरिर्नैवेद्यैर्विविधैर्मुहुः}
{तेषामेतादृशं भोज्यं न भवेदक्षिगोचरम्}% ३७

\twolineshloka
{यैर्नरैरग्रजन्मानो भोजिता विविधै रसैः}
{भुञ्जते ते स्वादुरसं पापिनां चक्षुरुज्झितम्}% ३८

\twolineshloka
{एवंविधैरसैर्मिष्टैर्भोजिता द्विजसत्तमाः}
{मण्डपे विपठन्तस्ते शब्दब्रह्मविचक्षणाः}% ३९

\twolineshloka
{नृत्यन्त्येके हसन्त्येके नदन्त्येके प्रहर्षिताः}
{उत्सवो बहुरुद्भाति तत्र शत्रुघ्न आगमत्}% ४०

\twolineshloka
{रामः शत्रुघ्नमायान्तं पुष्कलेन समन्वितम्}
{निरीक्ष्यमुदमुद्भूतां रक्षितुं नाशकत्तदा}% ४१

\twolineshloka
{यावदुत्तिष्ठते रामो भ्रातरं हयपालकम्}
{तावद्रामपदेलग्नः शत्रुघ्नो भ्रातृवत्सलः}% ४२

\twolineshloka
{पादयोः पतितं वीक्ष्य भ्रातरं विनयान्वितम्}
{परिरेभे दृढं प्रीतः क्षतसंशोभिताङ्गकम्}% ४३

\twolineshloka
{अश्रूणि बहुधा मुञ्चन्हर्षाच्छिरसि राघवः}
{अत्यन्तं परमां प्राप मुदं वचनदूरगाम्}% ४४

\twolineshloka
{पुष्कलं स्वीयपदयोर्नम्रं विनयविह्वलः}
{सुदृढं भुजयोर्मध्ये विनीयापीडयद्भृशम्}% ४५

\twolineshloka
{हनूमन्तं तथा वीरं सुग्रीवं चाङ्गदं तथा}
{लक्ष्मीनिधिं जनकजं प्रतापाग्र्यं रिपुञ्जयम्}% ४६

\twolineshloka
{सुबाहुं सुमदं वीरं विमलं नीलरत्नकम्}
{सत्यवन्तं वीरमणिं सुरथं रामसेवकम्}% ४७

\twolineshloka
{अन्यानपि महाभागान्रघुनाथः स्वयं तदा}
{परिरेभे दृढं स्निग्धान्पादयोः प्रणतान्नृपान्}% ४८

\twolineshloka
{सुमतिः श्रीरघुपतिं भक्तानुग्रहकारकम्}
{परिरभ्य दृढं प्रीतः सम्मुखे तिष्ठदुन्नतः}% ४९

\twolineshloka
{तदा रामो निजामात्यं वीक्ष्य सान्निध्यमागतम्}
{उवाच परमप्रीत्या मन्त्रिणं वदतां वरः}% ५०

\twolineshloka
{सुमते मन्त्रिणां श्रेष्ठ शंश मे वाग्मिनां वर}
{क एते भूमिपाः सर्वे कथमत्र समागताः}% ५१

\twolineshloka
{कुत्रकुत्र हयः प्राप्तः केनकेन नियन्त्रितः}
{कथं वै मोचितो भ्रात्रा महाबलसुशालिना}% ५२

\uvacha{शेष उवाच}

\twolineshloka
{इत्युक्तो मन्त्रिणां श्रेष्ठः सुमतिः प्राह राघवम्}
{प्रहसन्मेघगम्भीर नादेन च सुबुद्धिमान्}% ५३

\uvacha{सुमतिरुवाच}

\twolineshloka
{सर्वज्ञस्य पुरस्तेऽद्य मया कथमुदीर्यते}
{पृच्छसि त्वं लोकरीत्या सर्वं जानासि सर्वदृक्}% ५४

\twolineshloka
{तथापि तव निर्देशं शिरस्याधाय सर्वदा}
{ब्रवीमि तच्छृणुष्वाद्य सर्वराजशिरोमणे}% ५५

\twolineshloka
{त्वत्प्रसादादहो स्वामिन्सर्वत्र जगतीतले}
{परिबभ्राम ते वाहो भालपत्रसुशोभितः}% ५६

\twolineshloka
{न कश्चित्तं निजग्राह स्वनामबलदर्पितः}
{स्वं स्वं राज्यं समर्प्याथ प्रणेमुस्ते पदाम्बुजम्}% ५७

\twolineshloka
{को वा रावण दैत्येन्द्र निहन्तुर्वाजिसत्तमम्}
{गृह्णाति विजयाकाङ्क्षी जरामरणवर्जितः}% ५८

\twolineshloka
{अहिच्छत्रां गतस्तावत्तव वाजी मनोरमः}
{तद्राजा सुमदः श्रुत्वा हयं प्राप्तं तव प्रभो}% ५९

\twolineshloka
{सपुत्रः प्रबलः सर्वसैन्येन बलिना वृतः}
{सर्वं समर्पयामास राज्यं निहतकण्टकम्}% ६०

\twolineshloka
{यो राजा जगतां नेत्रीं मातरं जगदम्बिकाम्}
{प्रसाद्य चिरमायुष्यं लेभे राज्यमकण्टकम्}% ६१

\twolineshloka
{स एष त्वां प्रणमति सुमदः प्रभुसेवितम्}
{तं गृहाण कृपादृष्ट्या चिराद्दर्शनकाङ्क्षकम्}% ६२

\twolineshloka
{ततः सुबाहुभूपस्य नगरे बलपूरिते}
{दमनस्तस्य वै पुत्रः प्रजग्राह हयोत्तमम्}% ६३

\twolineshloka
{तेन साकं महद्युद्धं बभूव दमनेन च}
{पुष्कलो जयमापेदे सम्मूर्छ्य सुभुजात्मजम्}% ६४

\twolineshloka
{ततः सुबाहुः सङ्क्रुद्धो रणे पवनजं बलात्}
{युयुधे तव पादाब्जसेवकं बलिनां वरम्}% ६५

\twolineshloka
{तस्य पादाहतो ज्ञानं प्राप्य शापतिरस्कृतम्}
{तुभ्यं समर्प्य सकलं वाजिनः पालकोऽभवत्}% ६६

\twolineshloka
{एष त्वां सुभुजो राजा प्रणमत्युन्नताङ्गकः}
{कृपादृष्ट्याभिषिञ्च त्वं सुबाहुं नयकोविदम्}% ६७

\twolineshloka
{ततो मुक्तो हयो रेवाह्रदे स निममज्ज ह}
{तत्र प्राप्तं मोहनास्त्रं शत्रुघ्नेन बलीयसा}% ६८

\twolineshloka
{ततो देवपुरे प्रागाच्छिववासविभूषिते}
{तत्रत्यं तु विजानासि यतस्त्वं तत्र चागतः}% ६९

\twolineshloka
{विद्युन्माली हतो दैत्यः सत्यवान्सङ्गतस्ततः}
{सुरथेन समं युद्धं जानासि त्वं महामते}% ७०

\twolineshloka
{ततः कुण्डलकान्मुक्तो हयो बभ्राम सर्वतः}
{न कश्चित्तं निजग्राह स्ववीर्यबलदर्पितः}% ७१

\twolineshloka
{वाल्मीकेराश्रमे रम्ये हयः प्राप्तो मनोरमः}
{तत्र यत्कुतुकं जातं तच्छृणुष्व नरोत्तम}% ७२

\twolineshloka
{तत्रार्भस्तव सारूप्यं बिभ्रत्षोडशवार्षिकः}
{जग्राह वीक्ष्य पत्राङ्कं वाजिनं बलवत्तमः}% ७३

\twolineshloka
{तत्र कालजिता युद्धं महज्जातं नरोत्तम}
{निहतस्तेन वीरेण शितधारेण हेतिना}% ७४

\twolineshloka
{अनेके निहताः सङ्ख्ये पुष्कलाद्या महाबलाः}
{मूर्च्छितं चापि शत्रुघ्नं चक्रे वीरशिरोमणिः}% ७५

\twolineshloka
{तदा राजा महद्दुःखं विचार्य हृदिसंयुगे}
{कोपेन मूर्च्छितं चक्रे वीरो हि बलिनां वरः}% ७६

\twolineshloka
{स यावन्मूर्च्छितो राज्ञा तावदन्यः समागतः}
{तेनैतेन च सञ्जीव्य नाशितं कटकं तव}% ७७

\twolineshloka
{सर्वेषां मूर्च्छितानां तु शस्त्राण्याभरणानि च}
{गृहीत्वा वानरौ बद्धौ जग्मतुः स्वाश्रमं प्रति}% ७८

\twolineshloka
{कृपां कृत्वा पुनस्तेन दत्तोऽश्वो यज्ञियो महान्}
{जीवनं प्रापितं सर्वं कटकं नष्टजीवितम्}% ७९

\twolineshloka
{वयं गृहीत्वा तं वाहं प्राप्तास्तव समीपतः}
{एतदेव मया ज्ञातं तदुक्तं ते पुरोवचः}% ८०

{॥इति श्रीपद्मपुराणे पातालखण्डे शेषवात्स्यायनसंवादे रामाश्वमेधे सुमतिनिवेदनं नाम पञ्चषष्टितमोऽध्यायः॥६५॥}

\dnsub{षट्षष्टितमोऽध्यायः}%\resetShloka

\uvacha{शेष उवाच}

\twolineshloka
{कथितौ वै सुमतिना वाल्मीकेराश्रमे शिशू}
{पुत्रौ स्वीयाविति ज्ञात्वा वाल्मीकिं प्रति सञ्जगौ}% १

\uvacha{श्रीराम उवाच}

\twolineshloka
{कौ शिशू मम सारूप्यधारकौ बलिनां वरौ}
{किमर्थं तिष्ठतस्तत्र धनुर्विद्याविशारदौ}% २

\twolineshloka
{अमात्यकथितौ श्रुत्वा विस्मयो मम जायते}
{यौ शत्रुघ्नं हनूमन्तं लीलयाङ्ग बबन्धतुः}% ३

\twolineshloka
{तस्माच्छंस मुने सर्वं बालयोश्च विचेष्टितम्}
{यथा मे परमा प्रीतिर्भवत्येवमभीप्सिता}% ४

\twolineshloka
{इति तत्कथितं श्रुत्वा राजराजस्य धीमतः}
{उवाच परमं वाक्यं स्पष्टाक्षरसमन्वितम्}% ५

\uvacha{वाल्मीकिरुवाच}

\twolineshloka
{तवान्तर्यामिणो नॄणां कथं ज्ञानं च नो भवेत्}
{तथापि कथयाम्यत्र तव सन्तोषहेतवे}% ६

\twolineshloka
{राजन्यौ बालकौ मह्यमाश्रमे बलिनां वरौ}
{त्वत्सारूप्यधरौ स्वाङ्गमनोहरवपुर्धरौ}% ७

\twolineshloka
{त्वया यदा वने त्यक्ता जानकी वै निरागसी}
{अन्तर्वत्नी वने घोरे विलपन्ती मुहुर्मुहुः}% ८

\twolineshloka
{कुररीमिव दुःखार्तां वीक्ष्याहं तव वल्लभाम्}
{जनकस्य सुतां पुण्यामाश्रमे त्वानयं तदा}% ९

\twolineshloka
{तस्याः पर्णकुटीरम्या रचिता मुनिपुत्रकैः}
{तस्यामसूत पुत्रौ द्वौ भासयन्तौ दिशो दश}% १०

\twolineshloka
{तयोरकरवं नाम कुशो लव इति स्फुटम्}
{ववृधातेऽनिशं तत्र शुक्लपक्षे यथा शशी}% ११

\twolineshloka
{कालेनोपनयाद्यानि सर्वाणि कृतवानहम्}
{वेदान्साङ्गानहं सर्वान्ग्राहयामास भूपते}% १२

\twolineshloka
{सर्वाणि सरहस्यानि शृणुष्व मुखतो मम}
{आयुर्वेदं धनुर्विद्यां शस्त्रविद्यां तथैव च}% १३

\twolineshloka
{विद्यां जालन्धरीं चाथ सङ्गीतकुशलौ कृतौ}
{गङ्गाकूले गायमानौ लताकुञ्जवनेषु च}% १४

\twolineshloka
{चञ्चलौ चलचित्तौ तौ सर्वविद्याविशारदौ}
{तदाहमतिसन्तोषं प्राप्तश्चाहं रघूत्तम}% १५


\threelineshloka
{दत्त्वा सर्वाणि चास्त्राणि मस्तके निहितः करः}
{अतीवगानकुशलौ दृष्ट्वा लोका विसिष्मिरे}
{षड्जमध्यमगान्धारस्वरभेदविशारदौ}% १६

\twolineshloka
{तथाविधौ विलोक्याहं गापयामि मनोहरम्}
{भविष्यज्ञानयोगाच्च कृतं रामायणं शुभम्}% १७

\twolineshloka
{मृदङ्गपणवाद्यादि यन्त्रवीणाविशारदौ}
{वनेवने च गायन्तौ मृगपक्षिविमोहकौ}% १८

\twolineshloka
{अद्भुतं गीतमाधुर्यं तव रामकुमारयोः}
{श्रोतुं तौ वरुणो बाला वा निनाय विभावरीम्}% १९

\twolineshloka
{मनोहरवयोरूपौ गानविद्याब्धिपारगौ}
{कुमारौ जगदुस्तत्र लोकेशादेशतः कलम्}% २०

\twolineshloka
{परमं मधुरं रम्यं पवित्रं चरितं तव}
{शुश्राव वरुणः सार्द्धं कुटुम्बेन च गायकैः}% २१

\twolineshloka
{शृण्वन्नैव गतस्तृप्तिं मित्रेण वरुणः सह}
{सुधातोऽपि परं स्वादुचरितं रघुनन्दन}% २२

\twolineshloka
{गानानन्दमहालोभ हृतप्राणेन्द्रियक्रियः}
{प्रत्यागन्तुं दिदेशासौ कुमारौ न हि तावकौ}% २३

\twolineshloka
{रमणीय महाभोगैर्लोभितावपि बालकौ}
{चलितौ न गुरोश्चात्ममातुः पादाम्बुजस्मृतेः}% २४

\twolineshloka
{अहं चापि गतः पश्चाद्वरुणालयमुत्तमम्}
{वरुणः प्रेमसहितः पूजां चक्रे मम प्रभो}% २५

\twolineshloka
{पृच्छते जन्मकर्मादि सर्वज्ञायापि बालयोः}
{वरुणायाब्रुवं सर्वं जन्मविद्याद्युपागमम्}% २६

\twolineshloka
{श्रुत्वा सीतासुतौ देवः स चक्रेम्बरभूषणैः}
{देवदत्तमिति ग्राह्यमिति मद्वाक्यगौरवात्}% २७


\threelineshloka
{आहृतं राजपुत्राभ्यां यद्दत्तं वरुणेन तत्}
{प्रसन्नेन तयोर्वाद्यगानविद्यावयोगुणैः}
{ततो मामब्रवीत्सीतामुद्दिश्य वरुणः कृती}% २८

\twolineshloka
{सीतापति व्रताधुर्या रूपशीलवयोन्विता}
{वीरपुत्रा महाभागा त्यागं नार्हति कर्हिचित्}% २९

\twolineshloka
{महती हानिरेतस्यास्त्यागे हि रघुनन्दन}
{सिद्धीनां परमासिद्धिरेषा ते ह्यनपायिनी}% ३०

\twolineshloka
{पामरैर्महिमानास्या ज्ञायते यदि दूषितैः}
{का हानिस्तावता राम पुण्यश्रवणकीर्तन}% ३१

\twolineshloka
{अस्मत्साक्षिकमेतस्याः पावनं चरितं सदा}
{सद्यस्ते सिद्धिमायान्ति ये सीतापदचिन्तकाः}% ३२

\twolineshloka
{यस्याः सङ्कल्पमात्रेण जन्मस्थितिलयादिकाः}
{भवन्ति जगतां नित्यं व्यापारा ऐश्वरा अमी}% ३३

\twolineshloka
{सीता मृत्युःसुधा चेयं तपत्येषा च वर्षति}
{स्वर्गो मोक्षस्तपो योगो दानं च तव जानकी}% ३४

\twolineshloka
{ब्रह्माणं शिवमन्यांश्च लोकपालान्मदादिकान्}
{करोत्येषा करोत्येव नान्या सीता तव प्रिया}% ३५

\twolineshloka
{त्वं पिता सर्वलोकानां सीता च जननीत्यतः}
{कुदृष्टिरत्र तु क्षेमयोग्या न तव कर्हिचित्}% ३६

\twolineshloka
{वेत्ति सीतां सदा शुद्धां सर्वज्ञो भगवान्स्वयम्}
{भवानपि सुतां भूमेः प्राणादपि गरीयसीम्}% ३७

\twolineshloka
{आदर्तव्या त्वया तस्मात्प्रिया शुद्धेति जानकी}
{न च शापपराभूतिः सीतायां त्वयि वा विभो}% ३८

\twolineshloka
{इमानि मम वाक्यानि वाच्यानि जगतां पतिम्}
{रामं प्रति त्वया साक्षाद्वाल्मीके मुनिसत्तम}% ३९

\twolineshloka
{इत्युक्तो वरुणेनाहं सीतासङ्ग्रहकारणात्}
{एवमेव हि सर्वैश्च लोकपालैरपि प्रभो}% ४०

\twolineshloka
{श्रुतं रामायणोद्गानं पुत्राभ्यां ते सुरासुरैः}
{गन्धर्वैरपि सर्वैश्च कौतुकाविष्टमानसैः}% ४१

\twolineshloka
{प्रसन्ना एव सर्वेऽपि प्रशशंसुः सुतौ च ते}
{त्रैलोक्यं मोहितं ताभ्यां रूपगानवयोगुणैः}% ४२

\twolineshloka
{दत्तं यल्लोकपालैस्तु सुताभ्यां स्वीकृतं हि तत्}
{ऋषिभिश्च वरा आभ्यामन्येभ्यः कीर्तिरेव च}% ४३

\twolineshloka
{एकरामं जगत्सर्वं पूर्वं मुनिविलोकितम्}
{त्रिराममधुना जान्तं सुताभ्यां तेखिलेक्षितम्}% ४४

\twolineshloka
{एककामपरामूर्तिर्लोके पूर्वमवेक्षिता}
{कामैश्चतुर्भिरद्यायं जायते च यतस्ततः}% ४५

\twolineshloka
{सर्वत्रान्यत्र राजेन्द्र रामपुत्रौ कुशीलवौ}
{गीयते अत्र सङ्कोचः किं कृतो विदुषि त्वयि}% ४६

\twolineshloka
{कृतेषु तव सर्वेषु श्रूयते महती स्तुतिः}
{त्यागादन्यत्र सीतायाः पुण्यश्लोकशिरोमणे}% ४७

\twolineshloka
{त्वया त्रैलोक्यनाथेन गार्हस्थ्यमनुकुर्वता}
{अङ्गीकार्यौ सुतौ रामविद्याशीलगुणान्वितौ}% ४८

\twolineshloka
{न तौ स्वां मातरं हित्वा स्थास्यतोऽभवदन्तिके}
{जनन्या सहितौ तस्मादाकार्यौ भवता सुतौ}% ४९

\twolineshloka
{दत्त एव तयेदानीं सेनासञ्जीवनात्पुनः}
{प्रत्ययः सर्वलोकानां पावनः पततामपि}% ५०

\twolineshloka
{नाज्ञातं तेन चास्माकं नामराणां च मानद}
{शुद्धौ तस्यास्तु लोकानां यन्नष्टं तदिह ध्रुवम्}% ५१

\uvacha{शेष उवाच}

\twolineshloka
{इति वाल्मीकिना रामः सर्वज्ञोऽप्यवबोधितः}
{स्तुत्वा नत्वा च वाल्मीकिं प्रत्युवाच स लक्ष्मणम्}% ५२

\twolineshloka
{गच्छ ताताधुना सीतामानेतुं धर्मचारिणीम्}
{सपुत्रां रथमास्थाय सुमन्त्रसहितः सखे}% ५३

\twolineshloka
{श्रावयित्वा ममेमानि मुनेश्च वचनान्यपि}
{सम्बोध्य च पुरीमेतां सीतां प्रत्यानयाशु ताम्}% ५४

\uvacha{लक्ष्मण उवाच}

\twolineshloka
{यास्यामि तव सन्देशात्सर्वेषां नः प्रभोर्विभो}
{देव्या यास्यति चेद्देव यात्रा मे सफला ततः}% ५५

\twolineshloka
{मयि सामाभ्यसूयैव पूर्वदोषवशात्सती}
{अनागतायां तस्यां तु क्षमस्वागन्तुकं मम}% ५६

\twolineshloka
{इत्युक्त्वा लक्ष्मणो रामं रथे स्थित्वा नृपाज्ञया}
{सुमित्रमुनिशिष्याभ्यां युतोऽगाद्भूमिजाश्रमम्}% ५७

\twolineshloka
{कथं प्रसादनीया स्यात्सीता भगवती मया}
{पूर्वदोषं विजानन्ती रामाधीनस्य मे सदा}% ५८

\twolineshloka
{एवं सञ्चिन्तयन्नन्तर्हर्षसङ्कोच मध्यगः}
{लक्ष्मणः प्राप सीताया आश्रमं श्रमनाशनम्}% ५९

\twolineshloka
{रथात्सोथावरुह्यारादश्रुरुद्धविलोचनः}
{आर्ये पूज्ये भगवति शुभे इति वदन्मुहुः}% ६०

\twolineshloka
{पपात पादयोस्तस्या वेपमानाखिलाङ्गकः}
{उत्थापितस्तया देव्या प्रीतिविह्वलया स च}% ६१

\twolineshloka
{किमर्थमागतः सौम्य वनं मुनिजनप्रियम्}
{आस्ते स कुशली देवः कौसल्याशुक्तिमौक्तिकः}% ६२

\twolineshloka
{अरोषो मयि कश्चित्स कीर्त्या केवलयादृतः}
{कीर्त्यते सर्वलोकैश्च कल्याणगुणसागरः}% ६३

\twolineshloka
{अकीर्तिभीतिमापन्नस्त्यक्तुं मां त्वां नियुक्तवान्}
{यदि ततश्च लोकेषु कीर्तिस्तस्यामलाभवत्}% ६४

\twolineshloka
{मृत्वापि पतिसत्कीर्तिं कुर्वन्त्या मे हि सुस्थिरा}
{पतिसामीप्यमेवाशु भूयादेव हि देवर}% ६५

\twolineshloka
{त्यक्तयापि मया तेन नासौ त्यक्तो मनागपि}
{फलं हि साधनायत्तं हेतुः फलवशो न तु}% ६६

\twolineshloka
{कौसल्याशल्यशून्यासौ कृपापूर्णा सदा मयि}
{आस्ते कुशलिनी यस्याः पुत्रस्त्रैलोक्यपालकः}% ६७

\twolineshloka
{सर्वे कुशलिनः सन्ति भरताद्याश्च बान्धवाः}
{सुमित्रा च महाभागा यस्याः प्राणादहं प्रिया}% ६८

\twolineshloka
{मद्वत्किं त्वमपि त्यक्तः सर्वलोकेषु कीर्तये}
{राज्ञः किं दुस्त्यजं तस्य स्वात्मापि यस्य न प्रियः}% ६९

\twolineshloka
{इत्येवं बहुधा पृष्टस्तया रामानुजः सताम्}
{उवाच कुशली देवः कुशलं त्वयि पृच्छति}% ७०

\twolineshloka
{कौसल्या च सुमित्रा च याश्चान्या राजयोषितः}
{पप्रच्छुः कुशलं देवि प्रीत्या त्वामाशिषा सह}% ७१

\twolineshloka
{कुशलप्रश्नपूर्वं हि तव पादाभिवन्दनम्}
{निवेदयामि शत्रुघ्न भरताभ्यां कृतं शुभे}% ७२

\twolineshloka
{गुरुभिर्गुरुपत्नीभिः सर्वाभिरपि ते शुभे}
{दत्ताशीः कुशलप्रश्नः कृतश्च त्वयि जानकि}% ७३

\twolineshloka
{आकारयति देवस्त्वां निर्व्यलीकेन चात्मवान्}
{अलभ्यान्यरतिस्त्वत्तोऽन्यत्र सर्वत्र भामिनि}% ७४

\twolineshloka
{शून्या एव दिशः सर्वास्त्वां विना जनकात्मजे}
{पश्यन्रोदिति नाथो नो रोदयन्नितरानपि}% ७५

\twolineshloka
{यत्र देवि स्थितासि त्वं नित्यं स्मरति राघवः}
{अशून्यं तु तमेवासौ मन्यमानो विदेहजे}% ७६

\twolineshloka
{धन्योऽयमाश्रमो जातो वाल्मीकेर्यत्र जानकी}
{कालं क्षपति वार्ताभिर्मदीयाभिर्वदन्निति}% ७७

\twolineshloka
{उक्तवान्यद्रुदन्किञ्चित्स्वामी नस्त्वयि तच्छृणु}
{व्यक्तीभवति वक्तुर्यद्धृद्गतं तदसंशयम्}% ७८

\twolineshloka
{लोका वदन्ति मामेव सर्वेषामीश्वरेश्वरम्}
{अहं त्वदृष्टमेवैषां स्वतन्त्रं कारणं ब्रुवे}% ७९

\twolineshloka
{अदृष्टमेव कार्येषु सर्वेशोऽप्यनुगच्छति}
{ईशनीयाः कुतो नैतदन्वीयुः सुखदुःखयोः}% ८०

\twolineshloka
{धनुर्भङ्गे मतिभ्रंशे कैकय्या मरणे पितुः}
{अरण्यगमने तत्र हरणे तव वारिधेः}% ८१

\twolineshloka
{तरणे रक्षसां भर्तुर्मारणेऽपि रणेरणे}
{सहायीभवने मह्यमृक्षवानररक्षसाम्}% ८२

\twolineshloka
{लाभे तव प्रतिज्ञायाः सत्यत्वे च सतीमणे}
{पुनः स्वबन्धुसम्बन्धे राज्यप्राप्तौ च भामिनि}% ८३

\twolineshloka
{पुनः प्रियावियोगे च कारणं यदवारणम्}
{प्रसीदति तदेवाद्य संयोगे पुनरावयोः}% ८४

\twolineshloka
{वेदोऽन्यथा कृतो येन लोकोत्पत्ति लयौ यतः}
{लोकाननुगतस्तस्मात्कारणं प्रथमं त्वहम्}% ८५

\twolineshloka
{अदृष्टमनुवर्तन्ते लोकाः सम्प्रतिबोधकाः}
{भोगेन जीर्यतेऽदृष्टं तत्तद्भुक्तं त्वया वने}% ८६

\twolineshloka
{स्नेहोऽकारणकः सीते वर्धमानो मम त्वयि}
{लोकादृष्टे तिरस्कृत्य त्वामाह्वयत आदरात्}% ८७

\twolineshloka
{शङ्कितेनापि दोषेण स्नेहनैर्मल्यमज्जनम्}
{भवतीति स वै शुद्ध आस्वाद्यो विबुधैः सदा}% ८८

\twolineshloka
{स्नेहशुद्धिरियं भद्रे कृता मे त्वयि नान्यथा}
{मन्तव्यं रक्षितोऽप्येष लोकः शिष्टानुवर्तिना}% ८९

\twolineshloka
{आवयोर्निन्दया देवि सर्वावस्था सुशुद्धये}
{लोको नश्येद्धि सम्मूढश्चरितैर्महतामयम्}% ९०

\twolineshloka
{आवयोरुज्ज्वला कीर्तिरावयोरुज्ज्वलो रसः}
{आवयोरुज्ज्वलौ वंशावावयोरुज्ज्वलाः क्रियाः}% ९१

\twolineshloka
{भवेयुरावयोः कीर्तिर्गायका उज्ज्वला भुवि}
{आवयोर्भक्तिमन्तो ये ते यान्त्यन्ते भवाम्बुधेः}% ९२

\twolineshloka
{इत्युक्ता भवती तेन प्रीयमाणेन ते गुणैः}
{पत्युः पादाम्बुजे द्रष्टुं करोतु सदयं मनः}% ९३

\twolineshloka
{वासांसि रमणीयानि भूषणानि महान्ति च}
{अङ्गरागस्तथा गन्धा मनोज्ञास्त्वयि योजिताः}% ९४

\twolineshloka
{रथो दास्यश्च रामेण प्रेषिता उत्सवायते}
{छत्रं च चामरे शुभ्रे गजा अश्वाश्च शोभने}% ९५

\twolineshloka
{स्तूयमाना द्विजश्रेष्ठैः सूतमागधबन्दिभिः}
{वन्द्यमाना पुरस्त्रीभिः सेव्यमाना च योद्धृभिः}% ९६

\twolineshloka
{पुष्पैः सञ्छाद्यमाना च देवीदेवाङ्गनादिभिः}
{धनानि ददती तेभ्यो द्विजातिभ्यो यथेप्सितम्}% ९७

\twolineshloka
{गजारूढौ कुमारौ च पुरस्कृत्य जनेश्वरी}
{मयानुगम्यमाना च गच्छायोध्यां निजां पुरीम्}% ९८

\twolineshloka
{त्वयि तत्र गतायां तु सङ्गतायां प्रियेण ते}
{सर्वासां राजनारीणामागतानां च सर्वशः}% ९९

\twolineshloka
{सर्वासामृषिपत्नीनां कौसलानां तथैव च}
{मङ्गलैर्वाद्यगीताद्यैर्भवत्वद्य महोत्सवः}% १००

\uvacha{शेष उवाच}

\twolineshloka
{इतिविज्ञापनां देवी श्रुत्वा सीता तमाह सा}
{नाहं कीर्तिकरी राज्ञो ह्यपकीर्तिः स्वयं त्वहम्}% १०१

\twolineshloka
{किं मया तस्य साध्यं स्याद्धर्मकामार्थशून्यया}
{सत्येवं भवतां भूपे को विश्वासो निरङ्कुशे}% १०२

\twolineshloka
{प्रत्यक्षा वा परोक्षा वा भर्तुर्दोषा मनःस्थिताः}
{न वाच्या जातु मादृश्या कल्याणकुलजातया}% १०३

\twolineshloka
{पाणिग्रहणकाले मे यद्रूपो हृदये स्थितः}
{तद्रूपो हृदयान्नासौ कदाचिदपसर्पति}% १०४

\twolineshloka
{लक्ष्मणेमौ कुमारौ मे तत्तेजोंशसमुद्भवौ}
{वंशाङ्कुरौ महाशूरौ धनुर्विद्याविशारदौ}% १०५

\twolineshloka
{नीत्वा पितुः समीपं तु लालनीयौ प्रयत्नतः}
{तपसाराधयिष्यामि रामं काममिह स्थिता}% १०६

\twolineshloka
{वाच्यं त्वया महाभाग पूज्यपादाभिवन्दनम्}
{सर्वेभ्यः कुशलं चापि गत्वेतो मदपेक्षया}% १०७

\twolineshloka
{पुत्रौ समादिशत्सीता गच्छतं पितुरन्तिकम्}
{शुश्रूषणीय एवासौ भवद्भ्यां स्वपदप्रदः}% १०८

\twolineshloka
{आज्ञप्तावप्यनिच्छन्तौ तौ कुमारौ कुशीलवौ}
{वाल्मीकिवचनात्तत्र जग्मतुश्च सलक्ष्मणौ}% १०९

\twolineshloka
{वाल्मीकेरेव पादाब्जसमीपं तत्सुतौ गतौ}
{लक्ष्मणोऽपि ववन्दे तं गत्वा बालकसंयुतः}% ११०

\twolineshloka
{वाल्मीकिर्लक्ष्मणस्तौ तु कुमारौ मिलिता अमी}
{सभायां संस्थितं रामं ज्ञात्वा ते जग्मुरुत्सुकाः}% १११

\twolineshloka
{लक्ष्मणः प्रणिपत्याथ सीतावाक्यादिसर्वशः}
{कथयामास रामाय हर्षशोकयुतः सुधीः}% ११२

\twolineshloka
{सीतासन्देशवाक्येभ्यो रामो मूर्च्छां समन्वभूत्}
{संज्ञामवाप्य चोवाच लक्ष्मणं नयकोविदम्}% ११३

\twolineshloka
{गच्छ मित्र पुनस्तत्र यत्नेन महता च ताम्}
{शीघ्रमानय भद्रं ते मद्वाक्यानि निवेद्य च}% ११४

\twolineshloka
{अरण्ये किं तपस्यन्त्या गतिरन्या विचिन्तिता}
{श्रुता दृष्टाथ वा मत्तो यन्नागच्छसि जानकि}% ११५

\twolineshloka
{त्वदिच्छया त्वमेवेतो गतारण्यं मुनिप्रियम्}
{पूजिता मुनिपत्न्यस्ता दृष्टा मुनिगणास्त्वया}% ११६

\twolineshloka
{पूर्णो मनोरथस्तेऽद्य किं नागच्छसि भामिनि}
{न दोषं मयि पश्येस्त्वं स्वात्मेच्छाया विलोकनात्}% ११७

\twolineshloka
{गत्वा गत्वाथ वामोरु पतिरेव गतिः स्त्रियाः}
{निर्गुणोपि गुणाम्भोधिः किम्पुनर्मनसेप्सितः}% ११८

\twolineshloka
{याया क्रियाकुलस्त्रीणां सासा पत्युः प्रतुष्टये}
{पूर्वमेवप्रतुष्टोऽहमिदानीं सुतरां त्वयि}% ११९

\twolineshloka
{यागो जपस्तपोदानं व्रतं तीर्थं दयादिकम्}
{देवाश्च मयि सन्तुष्टे तुष्टमेतदसंशयम्}% १२०

\uvacha{शेष उवाच}

\twolineshloka
{इति सन्देशमादाय सीतां प्रति जगत्पतेः}
{आह लक्ष्मण आत्मेशमानतः प्रणयाद्धरौ}% १२१

\twolineshloka
{सीतानयनमुद्दिश्य प्रसन्नस्त्वं यदूचिवान्}
{कथयिष्यामि तद्वाक्यं विनयेन समन्वितम्}% १२२

\twolineshloka
{इत्युक्त्वा पादयोर्नत्वा रघुनाथस्य लक्ष्मणः}
{जगाम त्वरितः सीतां रथे तिष्ठन्महाजवे}% १२३

\twolineshloka
{वाल्मीकिः श्रीयुतौ वीक्ष्य रामपुत्रौ महौजसौ}
{उवाच स्मितमाधाय मुखं कृत्वा मनोहरम्}% १२४

\twolineshloka
{युवां प्रगायतां पुत्रौ रामचारित्रमद्भुतम्}
{वीणां प्रवादयन्तौ च कलगानेन शोभितम्}% १२५

\twolineshloka
{इत्यक्तौ तौ सुतौ रामचारित्रं बहुपुण्यदम्}
{अगायतां महाभागौ सुवाक्यपदचित्रितम्}% १२६

\twolineshloka
{यस्मिन्धर्मविधिः साक्षात्पातिव्रत्यं तु यत्स्थितम्}
{भ्रातृस्नेहो महान्यत्र गुरुभक्तिस्तथैव च}% १२७

\twolineshloka
{स्वामिसेवकयोर्यत्र नीतिर्मूर्तिमती किल}
{अधर्मकरशास्तिं वै यत्र साक्षाद्रघूद्वहात्}% १२८

\twolineshloka
{तद्गानेन जगद्व्याप्तं दिवि देवा अपि स्थिताः}
{किन्नरा अपि यद्गानं श्रुत्वा मूर्च्छामिताः क्षणात्}% १२९

\twolineshloka
{वीणायारणितं श्रुत्वा तालमानेन शोभितम्}
{निखिला परिषत्तत्र शालभं जीवचित्रिता}% १३०

\twolineshloka
{हर्षादश्रूणिमुञ्चन्तो रामाद्या भूमिपास्तथा}
{तद्गानपञ्चमालापमोहिताश्चित्रितोपमाः}% १३१

\twolineshloka
{तत्र रामः सुतौ दृष्ट्वा महागानविमोहकौ}
{अदात्ताभ्यां सुवर्णस्य लक्षं लक्षं पृथक्पृथक्}% १३२

\twolineshloka
{तदा दानपरं दृष्ट्वा वाल्मीकिं मुनिसत्तमम्}
{अब्रूतां प्रहसन्तौ तौ किञ्चिद्वक्रभ्रुवौ ततः}% १३३

\twolineshloka
{मुने महानयोनेन क्रियते भूमिपेन वै}
{यदावाभ्यां सुवर्णानि दातुमिच्छति लोभयन्}% १३४

\twolineshloka
{प्रतिग्रहो ब्राह्मणानां शस्यते नेतरेषु वै}
{प्रतिग्रहपरो राजा नरकायैव कल्पते}% १३५

\twolineshloka
{आवयोः कृपया मुक्तं राज्यं भुङ्क्ते महीपतिः}
{कथं दातुं सुवर्णानि वाञ्छति श्रेयसाञ्चितः}% १३६

\twolineshloka
{इत्युक्तवन्तौ तौ दृष्ट्वा वाल्मीकिः कृपयायुतः}
{अशंसद्युष्मत्पितरं जानीथां नीतिवित्तमौ}% १३७

\twolineshloka
{इति श्रुत्वा मुनेर्वाक्यं बालकौ नृपपादयोः}
{लग्नौ विनयसंयुक्तौ मातृभक्त्यातिनिर्मलौ}% १३८

\twolineshloka
{रामो बालौ दृढं स्वाङ्गे परिरभ्य मुदान्वितः}
{मेने स्त्रियास्तदा धर्मौ मूर्तिमन्तावुपस्थितौ}% १३९

\twolineshloka
{सभापि रामसुतयोर्वीक्ष्य वक्त्रे मनोरमे}
{जानकीपतिभक्तित्वं सत्यं मेने मुनीश्वर}% १४०

\twolineshloka
{इति शेषमुखप्रोक्तं श्रुत्वा वात्स्यायनोऽब्रवीत्}
{रामायणं श्रोतुमनाः सर्वधर्मसमन्वितम्}% १४१

\uvacha{वात्स्यायन उवाच}

\twolineshloka
{कस्मिन्काले कृतं स्वामिन्रामायणमिदं महत्}
{कस्माच्चकार किन्तत्र वर्णनं तद्वदस्व मे}% १४२

\uvacha{शेष उवाच}

\twolineshloka
{एकदा गतवान्विप्रो वाल्मीकिर्विपिनं महत्}
{यत्र तालास्तमालाश्च किंशुका यत्र पुष्पिताः}% १४३


\threelineshloka
{केतकी यत्र रजसा कुर्वती सौरभं वनम्}
{शशिप्रभेव महती दृश्यते शुभ्रकर्णभृत्}
{चम्पकोबकुलश्चापि कोविदारः कुरण्टकः}% १४४

\twolineshloka
{अनेके पुष्पिता यत्र पादपाः शोभने वने}
{कोकिलानां विरावेण षट्पदानां च शब्दितैः}% १४५

\twolineshloka
{सङ्घुष्टं सर्वतो रम्यं मनोहरवयोन्वितम्}
{तत्र क्रौञ्चयुगं रम्यं कामबाणप्रपीडितम्}% १४६

\twolineshloka
{परस्परं प्रहृषितं रेमे स्निग्धतया स्थितम्}
{तदा व्याधः समागत्य तयोरेकं मनोहरम्}% १४७

\twolineshloka
{अवधीन्निर्दयः कश्चिन्मांसास्वादनलोलुपः}
{तदा क्रौञ्ची व्याधहतं स्वपतिं वीक्ष्य दुःखिता}% १४८

\twolineshloka
{विललाप भृशं दुःखान्मुञ्चन्ती रावमुच्चकैः}
{तदा मुनिः प्रकुपितो निषादं क्रौञ्चघातकम्}% १४९

\twolineshloka
{शशाप वार्युपस्पृश्य सरितः पावनं शुभम्}
{मा निषाद प्रतिष्ठां त्वमगमः शाश्वतीः समाः}% १५०

\twolineshloka
{यत्क्रौञ्चपक्षिणोरेकमवधीः काममोहितम्}
{तदा प्रबन्धं श्लोकस्य जातं मत्वा ह्यनुद्विजाः}% १५१

\twolineshloka
{ऊचुर्मुनिं प्रहृष्टास्ते शंसन्तः साधुसाध्विति}
{स्वामिञ्छापोदिते वाक्ये भारतीश्लोकमातनोत्}% १५२


\threelineshloka
{अत्यन्तं मोहनो जातः श्लोकोऽयं मुनिसत्तम}
{तदा मुनिः प्रहृष्टात्मा बभूव वाडवर्षभ}
{तस्मिन्काले समागत्य ब्रह्मा पुत्रैः समन्वितः}% १५३

\twolineshloka
{वचो जगाद वाल्मीकिं धन्योसि त्वं मुनीश्वर}
{भारती त्वन्मुखे स्थित्वा श्लोकत्वं समपद्यत}% १५४

\twolineshloka
{तस्माद्रामायणं रम्यं कुरुष्व मधुराक्षरम्}
{येन ते विमला कीर्तिराकल्पान्तं भविष्यति}% १५५

\twolineshloka
{धन्या सैव मुखे वाणी रामनाम्ना समन्विता}
{अन्या कामकथा नॄणां जनयत्येव पातकम्}% १५६

\twolineshloka
{तस्मात्कुरुष्व रामस्य चरितं लोकविश्रुतम्}
{येन स्यात्पापिनां पापहानिरेव पदेपदे}% १५७

\twolineshloka
{इत्युक्त्वान्तर्दधे स्रष्टा सर्वदेवैः समन्वितः}
{ततः सचिन्तयामास कथं रामायणं भवेत्}% १५८

\twolineshloka
{तदा ध्यानपरो जातो नदीतीरे मनोरमे}
{तस्य चेतस्यथो रामः प्रादुर्भूतो मनोहरः}% १५९

\twolineshloka
{नीलोत्पलदलश्यामं रामं राजीवलोचनम्}
{निरीक्ष्य तस्य चरितं भूतम्भाविभवच्च यत्}% १६०

\twolineshloka
{तदात्यन्तं मुदं प्राप्तो रामायणमथासृजत्}
{मनोरमपदैर्युक्तं वृत्तैर्बहुविधैरपि}% १६१

\twolineshloka
{षट्काण्डानि सुरम्याणि यत्र रामायणेऽनघ}
{बालमारण्यकं चान्यत्किष्किन्धा सुन्दरं तथा}% १६२

\twolineshloka
{युद्धमुत्तरमन्यच्च षडेतानि महामते}
{शृणुयाद्यो नरः पुण्यात्सर्वपापैः प्रमुच्यते}% १६३

\twolineshloka
{तत्र बाले तु सन्तुष्टः पुत्रेष्ट्या चतुरस्सुतान्}
{प्राप पङ्क्तिरथः साक्षाद्धरिं ब्रह्मसनातनम्}% १६४

\twolineshloka
{स कौशिकमखं गत्वा सीतामुद्वाह्य भार्गवम्}
{आगत्य पुरमुत्कृष्टो यौवराज्यप्रकल्पनम्}% १६५

\twolineshloka
{मातृवाक्याद्वनं प्रागाद्गङ्गामुत्तीर्य पर्वतम्}
{चित्रकूटं महिलया लक्ष्मणेन समन्वितः}% १६६

\twolineshloka
{भरतस्तं वने श्रुत्वा जगाम भ्रातरं नयी}
{तमप्राप्य स्वयं नन्दिग्रामे वासमचीकरत्}% १६७

\twolineshloka
{बालमेतच्छृणुष्वान्यदारण्यकसमुद्भवम्}
{मुनीनामाश्रमे वासस्तत्र तत्रोपवर्णनम्}% १६८

\twolineshloka
{नासाच्छेदः शूर्पणख्याः खरदूषणनाशनम्}
{मायामारीचहननं दैत्याद्रामापहारणम्}% १६९

\twolineshloka
{वने विरहिणा भ्रान्तं मनुष्यचरितं भृतम्}
{कबन्धप्रेक्षणं तत्र पम्पायां गमनं तथा}% १७०

\twolineshloka
{हनूमता सङ्गमनमित्येतद्वनसंज्ञितम्}
{अपरं च शृणु मुने सङ्क्षिप्य कथयाम्यहम्}% १७१

\twolineshloka
{सप्ततालप्रभेदश्च वालेर्मारणमद्भुतम्}
{सुग्रीवे राज्यदानं च नगवर्णनमित्युत}% १७२

\twolineshloka
{लक्ष्मणात्कर्मसन्देशः सुग्रीवस्य विवासनम्}
{तथा सैन्यसमुद्देशः सीतान्वेषणमप्युत}% १७३

\twolineshloka
{सम्पातिप्रेक्षणं तत्र वारिधेर्लङ्घनं तथा}
{परतीरे कपिप्राप्तिः कैष्किन्धं काण्डमद्भुतम्}% १७४

\twolineshloka
{सुन्दरं शृणु काण्डं वै यत्र रामकथाद्भुता}
{प्रतिगेहे प्रति भ्रान्तिः कपेश्चित्रस्य दर्शनम्}% १७५

\twolineshloka
{सीतासन्दर्शनं तत्र जानक्याभाषणं तथा}
{वनभङ्गः प्रकुपितैर्बन्धनं वानरस्य च}% १७६

\twolineshloka
{ततो लङ्काप्रज्वलनं वानरैः सङ्गतिस्ततः}
{रामाभिज्ञानदानं च सैन्यप्रस्थानमेव च}% १७७

\twolineshloka
{समुद्रे सेतुकरणं शुकसारणसङ्गतिः}
{इति सुन्दरमाख्यातं युद्धे सीतासमागमः}% १७८

\twolineshloka
{उत्तरे ऋषिसंवादो यज्ञप्रारम्भ एव च}
{तत्रानेका रामकथाः शृण्वतां पापनाशकाः}% १७९

\twolineshloka
{इति षट्काण्डमाख्यातं ब्रह्महत्यापनोदनम्}
{सङ्क्षेपतो मया तुभ्यमाख्यातं सुमनोहरम्}% १८०

\twolineshloka
{चतुर्विंशतिसाहस्रं षट्काण्डपरिचिह्नितम्}
{तद्वै रामायणं प्रोक्तं महापातकनाशनम्}% १८१

\twolineshloka
{तच्छ्रुत्वा राघवः प्रीतः पुत्रावाधाय चासने}
{दृढं तौ परिरभ्याथ सीतां सस्मार वल्लभाम्}% १८२

{॥इति श्रीपद्मपुराणे पातालखण्डे शेषवात्स्यायनसंवादे रामाश्वमेधे रामायणगानं नाम षट्षष्टितमोऽध्यायः॥६६॥}

\dnsub{सप्तषष्टितमोऽध्यायः}%\resetShloka

\uvacha{शेष उवाच}

\twolineshloka
{अथ सौमित्रिरागत्य जानकीं नतवान्मुहुः}
{प्रेमगद्गदया शंसन्वाचं रामप्रणोदिताम्}% १

\twolineshloka
{सीता समागतं दृष्ट्वा लक्ष्मणं विनयान्वितम्}
{तन्मुखाद्रामसन्देशं श्रुत्वोवाच विलज्जिता}% २

\twolineshloka
{सौमित्रे कथमागच्छे रामत्यक्ता महावने}
{तिष्ठामि रामं स्मरन्ती वाल्मीकेराश्रमे त्वहम्}% ३

\twolineshloka
{तस्या मुखोदितं वाक्यं श्रुत्वा सौमित्रिरब्रवीत्}
{मातः पतिव्रते रामस्त्वामाकारयते मुहुः}% ४

\twolineshloka
{पतिव्रता पतिकृतं दोषं नानयते हृदि}
{तस्मादागच्छ हि मया स्थित्वा स्यन्दन उत्तमे}% ५

\twolineshloka
{इत्यादि वचनं श्रुत्वा जानकी पतिदेवता}
{मनोरोषं परित्यज्य तस्थौ सौमित्रिणा रथे}% ६

\twolineshloka
{तापसीः सकला नत्वा मुनींश्च निगमोज्ज्वलान्}
{रामं स्मरन्ती मनसा रथे स्थित्वागमत्पुरीम्}% ७

\twolineshloka
{क्रमेण नगरीं प्राप्ता महार्हाभरणान्विता}
{सरयूं सरितं प्राप यत्र रामः स्वयं स्थितः}% ८

\twolineshloka
{रथादुत्तीर्य ललिता लक्ष्मणेन समन्विता}
{रामस्य पादयोर्लग्ना पतिव्रतपरायणा}% ९

\twolineshloka
{रामस्तामागतां दृष्ट्वा जानकीं प्रेमविह्वलाम्}
{साध्वि त्वया सहेदानीं कुर्वे यज्ञसमापनम्}% १०

\twolineshloka
{वाल्मीकिं सा नमस्कृत्य तथान्यान्विप्रसत्तमान्}
{जगाम मातृपदयोः सन्नतिं कर्तुमुत्सुका}% ११

\twolineshloka
{कौशल्या तामथायान्तीं वीरसूं जानकीं प्रियाम्}
{आशीर्भिरभिसंयुज्य ययौ हर्षमनेकधा}% १२

\twolineshloka
{कैकेयीपदयोर्नम्रां वीक्ष्य वैदेहपुत्रिकाम्}
{भर्त्रा सह चिरं जीव सपुत्रेत्याशिषं व्यधात्}% १३

\twolineshloka
{सुमित्रा स्वपदेनम्रां वीक्ष्य वैदेहपुत्रिकाम्}
{आशिषं व्यदधात्तस्याः पुत्रपौत्रप्रदायिनीम्}% १४

\twolineshloka
{सीता ताः सर्वतो नत्वा रामचन्द्र प्रिया सती}
{परमं हर्षमापन्ना बभूव किल वाडव}% १५

\twolineshloka
{समागतां वीक्ष्य पत्नीं रामचन्द्रस्य कुम्भजः}
{सुवर्णपत्नीं धिक्कृत्य तामधाद्धर्मचारिणीम्}% १६

\twolineshloka
{रामस्तदा यज्ञमध्ये शुशुभे सीतया सह}
{तारयानुगतो यद्वच्छशीव शरदुत्प्रभः}% १७

\twolineshloka
{प्रयोगमकरोत्तत्र काले प्राप्ते मनोरमे}
{वैदेह्या धर्मचारिण्या सर्वपापापनोदनम्}% १८

\twolineshloka
{सीतया सहितं रामं प्रसक्तं यज्ञकर्मणि}
{निरीक्ष्य जहृषुस्तत्र कौतुकेन समन्विताः}% १९

\twolineshloka
{वसिष्ठं प्राह सुमतिं रामस्तत्र क्रतौ वरे}
{किं कर्तव्यं मया स्वामिन्नतः परमवश्यकम्}% २०

\twolineshloka
{रामस्य वचनं श्रुत्वा गुरुः प्राह महामतिः}
{ब्राह्मणानां प्रकर्तव्या पूजा सन्तोषकारिका}% २१

\twolineshloka
{मरुत्तेन क्रतुः सृष्टः पूर्वं सम्भारसम्भृतः}
{ब्राह्मणास्तत्र वित्ताद्यैस्तोषिता अभवंस्तदा}% २२

\twolineshloka
{अत्यन्तं वित्तसम्भारं नेतुं विप्राशकन्नहि}
{प्राक्षिपन्हिमवद्देशे वित्तभारासहा द्विजाः}% २३

\twolineshloka
{तस्मात्त्वमपि राजाग्र्य लक्ष्मीवान्नृपसत्तम}
{देहि दानादि विप्रेभ्यो यथा स्यात्प्रीतिरुत्तमा}% २४

\twolineshloka
{एतच्छ्रुत्वा स राजाग्र्यः पूज्यं मत्वा घटोद्भवम्}
{प्रथमं पूजयामास ब्रह्मपुत्रं तपोधनम्}% २५

\twolineshloka
{अनेकरत्नसम्भारैः स्वर्णभारैरनेकधा}
{देशैर्जनैः परिवृतैरत्यन्तप्रीतिदायकैः}% २६

\twolineshloka
{अगस्त्यं पूजयामास सपत्नीकं मनोरमम्}
{तथैव रत्नैः स्वर्णैश्च देशैश्च विविधैरपि}% २७

\twolineshloka
{व्यासं सत्यवतीपुत्रं तथैव समपूजयत्}
{च्यवनं भार्यया साकं सुरत्नैः समपूजयत्}% २८

\twolineshloka
{अन्यानपि मुनीन्सर्वानृत्विजस्तपसां निधीन्}
{पूजयामास रत्नौघैः स्वर्णभारैरनेकधा}% २९

\twolineshloka
{अदात्तदा क्रतौ रामो विप्रेभ्यो भूरिदक्षिणाम्}
{लक्षंलक्षं सुवर्णस्य प्रत्येकं त्वग्रजन्मने}% ३०

\twolineshloka
{दीनान्धकृपणेभ्यश्च ददौ दानमनेकधा}
{यथासन्तोषविहितैर्वित्तै रत्नैर्मनोहरैः}% ३१

\twolineshloka
{वासांसि च विचित्राणि भोजनानि मृदूनि च}
{तत्र प्रादाद्यथाशास्त्रं सर्वेषां प्रीतिदायकम्}% ३२

\twolineshloka
{हृष्टपुष्टजनाकीर्णं सर्वसत्त्वोपबृंहितम्}
{अत्यन्तमभवद्धृष्टं पुरं पुंस्त्रीसमावृतम्}% ३३

\twolineshloka
{दानं ददन्तं सर्वेषां वीक्ष्य कुम्भोद्भवो मुनिः}
{अत्यन्तपरमप्रीतिं ययौ क्रतुवरे द्विजः}% ३४

\twolineshloka
{तदाभिषेकस्नानार्थं पानीयममृतोपमम्}
{आनेतुं च चतुःषष्टि नृपान्सस्त्रीन्समाह्वयत्}% ३५

\twolineshloka
{रामस्तु सीतया सार्द्धमानेतुमुदकं ययौ}
{घटेन स्वर्णवर्णेन सर्वालङ्कारशोभया}% ३६

\twolineshloka
{सौमित्रिरप्यूर्मिलया माण्डव्या भरतो नृपः}
{शत्रुघ्नः श्रुतकीर्त्या च कान्तिमत्या च पुष्कलः}% ३७

\twolineshloka
{सुबाहुः सत्यवत्या च सत्यवान्वीरभूषया}
{सुमदस्तत्र सत्कीर्त्या राज्ञ्या च विमलो नृपः}% ३८

\twolineshloka
{राजावीरमणिस्तत्र श्रुतवत्या मनोज्ञया}
{लक्ष्मीनिधिः कोमलया रिपुतापोङ्गसेनया}% ३९

\twolineshloka
{विभीषणो महामूर्त्या प्रतापाग्र्यः प्रतीतया}
{उग्राश्वः कामगमया नीलरत्नोधिरम्यया}% ४०

\twolineshloka
{सुरथः सुमनोहार्या तथा मोहनया कपिः}
{इत्यादीन्नृपतीन्विप्रो वसिष्ठः प्राहिणोन्मुनिः}% ४१

\twolineshloka
{वसिष्ठः सरयूं गत्वा शिवपुण्यजलाप्लुताम्}
{उदकं मन्त्रयामास वेदमन्त्रेण मन्त्रवित्}% ४२

\twolineshloka
{पयः पुनीह्यमुं वाहमुदकेन मनोहृता}
{यज्ञार्थं रामचन्द्रस्य सर्वलोकैकरक्षितुः}% ४३

\twolineshloka
{उदकं तन्मुनिस्पृष्टं सर्वे रामादयो नृपाः}
{आजह्रुर्मण्डपतले विप्रवर्यैरुपस्तुते}% ४४

\twolineshloka
{पयोभिर्निर्मलैः स्नाप्य वाजिनं क्षीरसन्निभम्}
{मन्त्रेण मन्त्रयामास राम हस्तेन कुम्भजः}% ४५

\twolineshloka
{पुनीहि मां महावाह अस्मिन्ब्रह्मसमाकुले}
{त्वन्मेधेनाखिला देवाः प्रीणन्तु परितोषिताः}% ४६

\twolineshloka
{इत्युक्त्वा स नृपो रामः सीतया सममस्पृशत्}
{तदा सर्वे द्विजाश्चित्रममन्यन्त कुतूहलात्}% ४७

\twolineshloka
{परस्परमवोचंस्ते यन्नामस्मरणान्नराः}
{महापापात्प्रमुच्यन्ते स रामः किं वदत्यहो}% ४८

\twolineshloka
{इत्युक्तवति भूमीशे रामे कुम्भोद्भवो मुनिः}
{करवालं चाभिमन्त्र्य ददौ रामकरे मुनिः}% ४९

\twolineshloka
{करवाले धृते स्पृष्टे रामेण स हयः क्रतौ}
{पशुत्वं तु विहायाशु दिव्यरूपमपद्यत}% ५०

\twolineshloka
{विमानवरमारूढश्चाप्सरोभिः समन्ततः}
{चामरैर्वीज्यमानश्च वैजयन्त्या विभूषितः}% ५१

\twolineshloka
{तदा तं वाजितां त्यक्त्वा दिव्यरूपधरं वरम्}
{वीक्ष्य लोकाः क्रतौ सर्वे विस्मयं प्राप्नुवंस्तदा}% ५२

\twolineshloka
{तदा रामः स्वयं जानंज्ञापयन्सर्वतो नरान्}
{पप्रच्छ दिव्यरूपं तं सुरं परमधार्मिकः}% ५३

\twolineshloka
{कस्त्वं दिव्यवपुः प्राप्तः कस्मात्त्वं वाजितां गतः}
{कथं सुरस्त्रीसहितः किं चिकीर्षसि तद्वद}% ५४

\twolineshloka
{रामस्य वचनं श्रुत्वा देवः प्रोवाच भूमिपम्}
{हसन्मेघरवां वाणीमवदत्सुमनोहराम्}% ५५

\twolineshloka
{तवाज्ञातं न सर्वत्र बाह्याभ्यन्तरचारिणः}
{तथापि पृच्छते तुभ्यं कथयामि यथातथम्}% ५६

\twolineshloka
{अहं पुराभवे राम द्विजः परमधार्मिकः}
{अचरं प्रतिकूलं वै वेदस्य रिपुतापन}% ५७

\twolineshloka
{कदाचिद्धुतपापायास्तीरेऽहं गतवान्पुरा}
{अनेकवृक्षललिते सर्वत्रसुमनोरमे}% ५८

\twolineshloka
{तत्र स्नात्वा पितॄंस्तृप्त्वा दानं दत्त्वा यथाविधि}
{ध्यानं तव महाबाहो कृतवान्वेदसम्मितम्}% ५९

\twolineshloka
{तदा जनाः समायाता बहवस्तत्र भूपते}
{तेषां प्रवञ्चनार्थाय दम्भमेनमकारिषम्}% ६०

\twolineshloka
{अनेकक्रतुसम्भारैः पूर्णमजिरमुत्तमम्}
{वासोभिश्छादितं रम्यं चषालादियुतं महत्}% ६१

\twolineshloka
{अग्निहोत्रोद्भवोधूमः सर्वतो नभसोङ्गणम्}
{चकार रम्यमतुलं चित्रकारिवपुर्धरः}% ६२

\twolineshloka
{अनेकतिलकश्रीभिः शोभिताङ्गो महत्तपाः}
{दर्भशोभः समित्पाणिर्दम्भो मूर्तिधरः किमु}% ६३

\twolineshloka
{दुर्वासास्तत्र स्वच्छन्दं पर्यटञ्जगतीतलम्}
{प्राप तत्र महातेजा धूतपापसरित्तटे}% ६४

\twolineshloka
{ददर्श मां दम्भकरं मौनधारिणमग्रतः}
{अनर्घ्यकरमुन्मत्तमस्वागतवचः करम्}% ६५

\twolineshloka
{दृष्ट्वातीव क्रुधाक्रान्तः समुद्र इव पर्वणि}
{शशापासौ मुनिस्तीव्रो दम्भिनं मां महामतिः}% ६६

\twolineshloka
{दम्भं करोषि चेत्तीरे सरितस्त्वं सुदुर्मते}
{तस्मात्प्राप्नुहि निर्वाच्यं पशुत्वं तापसाधम}% ६७

\twolineshloka
{शापं प्रदत्तं संश्रुत्य दुःखितोऽहं तदाभवम्}
{अग्राहिषं पदे तस्य मुनेर्दुर्वाससः किल}% ६८

\twolineshloka
{तदा मे कृतवान्राम द्विजोऽनुग्रहमुत्तमम्}
{वाजितां प्राप्नुहि मखे राजराजस्य तापस}% ६९

\twolineshloka
{पश्चात्तद्धस्तसम्पर्काद्याहि तत्परमं पदम्}
{दिव्यं वपुर्मनोहारि धृत्वा दम्भविवर्जितम्}% ७०

\twolineshloka
{तेन शापोपिसन्दिष्टो ममानुग्रहतां गतः}
{यदहं तव हस्तस्य स्पर्शं प्राप्तो मनोरमम्}% ७१

\twolineshloka
{यदेव राम देवादिदुर्लभं बहुजन्मभिः}
{तत्तेऽहं करजस्पर्शं प्राप्तवानिह दुर्लभम्}% ७२

\twolineshloka
{आज्ञापय महाराज त्वत्प्रसादादहं महत्}
{गच्छामि शाश्वतं स्थानं तव दुःखादिवर्जितम्}% ७३

\twolineshloka
{न यत्र शोको न जरा न मृत्युः कालविभ्रमः}
{तत्स्थानं देव गच्छामि त्वत्प्रसादान्नराधिप}% ७४

\twolineshloka
{इत्युक्त्वा तं परिक्रम्य विमानवरमारुहत्}
{अनेकरत्नखचितं सर्वदेवाधिवन्दितम्}% ७५

\twolineshloka
{गतोऽसौ शाश्वतस्थानं रामपादप्रसादतः}
{पुनरावृत्तिरहितं शोकमोहविवर्जितम्}% ७६

\twolineshloka
{तेन तत्कथितं श्रुत्वा रामं ज्ञात्वेतरे जनाः}
{विस्मयं प्रापिरे सर्वे परस्परमुदुन्मदाः}% ७७

\twolineshloka
{शृणु द्विजमहाबुद्धे दम्भेनापि स्मृतो हरिः}
{ददाति मोक्षं सुतरां किं पुनर्दम्भवर्जनात्}% ७८

\twolineshloka
{यथाकथञ्चिद्रामस्य कर्तव्यं स्मरणं परम्}
{येन प्राप्नोति परमं पदं देवादिदुर्लभम्}% ७९

\twolineshloka
{तच्चित्रं वीक्ष्य मुनयः कृतार्थं मेनिरे निजम्}
{यद्रामचरणप्रेक्षा करस्पर्शपवित्रितम्}% ८०

\twolineshloka
{गते तस्मिन्सुरे स्वर्गं हयरूपधरे पुरा}
{उवाच रामस्तपसां निधीन्वेदविदुत्तमान्}% ८१

\twolineshloka
{किं कर्तव्यं मयाब्रह्मन्हयो नष्टो गतः सुखम्}
{होमः कथं पुरोभावी सर्वदैवततर्पकः}% ८२

\twolineshloka
{यथा स्यात्सुरसन्तृप्तिर्यथा मे मख उत्तमः}
{तथा कुर्वन्तु मुनयो यथा मे स्याद्विधिश्रुतम्}% ८३

\twolineshloka
{इति वाक्यं समाश्रुत्य जगाद मुनिसत्तमः}
{वसिष्ठः सर्वदेवानां चित्ताभिज्ञानकोविदः}% ८४

\twolineshloka
{कर्पूरमाहर क्षिप्रं येन देवाः स्वयं पुरा}
{प्राप्य हव्यं ग्रहीष्यन्ति मद्वाक्यप्रेरिताधुना}% ८५

\twolineshloka
{इति वाक्यं समाकर्ण्य रामः क्षिप्रमुपाहरत्}
{कर्पूरं बहुदेवानां प्रीत्यर्थं बहुशोभनम्}% ८६

\twolineshloka
{तदा मुनिः प्रहृष्टात्मा देवानाह्वयदद्भुतान्}
{ते सर्वे तत्क्षणात्प्राप्ताः स्वपरीवारसंवृताः}% ८७

{॥इति श्रीपद्मपुराणे पातालखण्डे शेषवात्स्यायनसंवादे रामाश्वमेधे यज्ञप्रारम्भो नाम सप्तषष्टितमोऽध्यायः॥६७॥}

\dnsub{अष्टषष्टितमोऽध्यायः}%\resetShloka

\uvacha{शेष उवाच}

\onelineshloka
{परिस्वादन्क्रतौ तृप्तिं न प्राप सुरसंयुतः}% १

\twolineshloka
{नारायणो महादेवो ब्रह्मा तत्र चतुर्मुखः}
{वरुणश्च कुबेरश्च तथान्ये लोकपालकाः}% २

\twolineshloka
{तत्रास्वाद्य हविः स्निग्धं वसिष्ठेन परिष्कृतम्}
{तत्र पुनर्हि विप्रेन्द्राः क्षुधार्ताइव भोजनात्}% ३

\twolineshloka
{सर्वान्देवांश्च सन्तर्प्य हविषा करुणानिधिः}
{वसिष्ठप्रेरितः सर्वमिति कर्तव्यमाचरत्}% ४

\twolineshloka
{ब्राह्मणादानसन्तुष्टा हविस्तुष्टाः सुरावराः}
{तृप्ताः सर्वे स्वकं भागं गृहीत्वा स्वालयं ययुः}% ५

\twolineshloka
{ऋषिभ्यो होतृमुख्येभ्यः प्रादाद्राज्यं चतुर्दिशम्}
{सन्तुष्टास्ते द्विजाराममाशीर्भिरददुः शुभम्}% ६

\twolineshloka
{पूर्णाहुतिं ततः कृत्वा वसिष्ठः प्राह सुस्त्रियः}
{वर्धापयन्तु भूमीशं यागपूर्तिकरं परम्}% ७

\twolineshloka
{तद्वाक्यं ताः स्त्रियः श्रुत्वा लाजैरवाकिरन्मुदा}
{लावण्यजितकन्दर्पं महामणिविभूषितम्}% ८

\twolineshloka
{ततोऽवभृथस्नानार्थं प्रेरयामास भूमिपम्}
{ययौ रामः सहस्वीयैः सरयूतीरमुत्तमम्}% ९

\twolineshloka
{अनेकराजकोटीभिः परीतः पादचारिभिः}
{जगाम स सरिच्छ्रेष्ठां पक्षिवृन्दसमाकुलाम्}% १०

\twolineshloka
{तारापतिरिव स्वाभिर्भार्याभिर्वृत उत्प्रभः}
{विरोचते तथा तद्वद्रामो राजगणैर्वृतः}% ११

\twolineshloka
{तदुत्सवं समाज्ञाय ययुर्लोकास्त्वरायुताः}
{सीतापतिमुखालोकनिश्चलीभूतलोचनाः}% १२

\twolineshloka
{राजेन्द्रं सीतया साकं गच्छन्तं सरितं प्रति}
{विलोक्य मुदिता लोकाश्चिरं दर्शनलालसाः}% १३

\twolineshloka
{अनेक नटगन्धर्वा गायन्तो यश उज्ज्वलम्}
{अनुजग्मुर्महीशानं सर्वलोकनमस्कृतम्}% १४

\twolineshloka
{नर्तक्यस्तत्र नृत्यन्त्यः क्षोभयन्त्यः पतेर्मनः}
{जलयन्त्रैश्च सिञ्चन्त्यो ययुः श्रीरामसेवनम्}% १५

\twolineshloka
{महाराजं विलिपन्त्यो हरिद्रा कुङ्कुमादिभिः}
{परस्परं प्रलिपन्त्यो मुदं प्रापुर्महत्तराम्}% १६

\twolineshloka
{कुचयुग्मोपरिन्यस्तमुक्ताहारसुशोभिताः}
{श्रवणद्वन्द्वसम्मृष्टस्वर्णकुण्डललक्षिताः}% १७

\twolineshloka
{अनेकनरनारीभिः सङ्कीर्णं मार्गमाचरन्}
{यथावत्सरितं प्राप शिवपुण्यजलाप्लुताम्}% १८

\twolineshloka
{तत्र गत्वा स वैदेह्या रामः कमललोचनः}
{प्रविवेश जलं पुण्यं वसिष्ठादिभिरन्वितः}% १९

\twolineshloka
{अनुप्रविविशुः सर्वे राजानो जनतास्तथा}
{तत्पादरजसा पूतजलं लोकैकवन्दितम्}% २०

\twolineshloka
{परस्परं प्रसिञ्चन्तो जलयन्त्रैर्मनोरमैः}
{सुशोणनयनाः सर्वे हर्षं प्रापुर्मनोधिकम्}% २१

\twolineshloka
{स रामः सीतया सार्धं चिरं पुण्यजलप्लवे}
{क्रीडित्वा जलकल्लोलैर्निरगाद्धर्मसंयुतः}% २२

\twolineshloka
{दुकूलवासाः सकिरीटकुण्डलः केयूरशोभावरकङ्कणान्वितः}
{कन्दर्पकोटिश्रियमुद्वहन्नृपो राजाग्र्यवर्यैरुपसंस्तुतो बभौ}% २३

\twolineshloka
{सयागयूपं वरवर्णशोभितं कृत्वा सरित्तीरवरे महामनाः}
{त्रैलोक्यलोकश्रियमाप ह्यद्भुतामन्यैर्दुरापां नृपतिर्भुजैर्निजैः}% २४

\twolineshloka
{एवं जनकपुत्र्यासौ हयमेधत्रयं चरन्}
{त्रैलोक्ये कीर्तिमतुलां प्राप देवैः सुदुर्लभाम्}% २५

\twolineshloka
{एवं ते वर्णितं तात यत्पृष्टो रामसत्कथाम्}
{विस्तृतः कथितो मेधो भूयः किं पृच्छसे द्विज}% २६

\twolineshloka
{यः शृणोति हरेर्भक्त्या रामचन्द्रस्य सन्मखम्}
{ब्रह्महत्यां क्षणात्तीर्त्वा ब्रह्मशाश्वतमाप्नुयात्}% २७

\twolineshloka
{अपुत्रो लभते पुत्रान्निर्धनो धनमाप्नुयात्}
{रोगार्तो मुच्यते रोगाद्बद्धो मुच्येत बन्धनात्}% २८

\twolineshloka
{यत्कथाश्रवणाद्दुष्टः श्वपचोऽपि परं पदम्}
{प्राप्नोति किमु विप्राग्र्यो रामभक्तिपरायणः}% २९

\twolineshloka
{रामं स्मृत्वा महाभागं पापिनः परमं पदम्}
{प्राप्नुयुः परमं स्वर्गं शक्रदेवादिदुर्लभम्}% ३०

\twolineshloka
{ते धन्या मानवा लोके ये स्मरन्ति रघूत्तमम्}
{ते क्षणात्संसृतिं तीर्त्वा गच्छन्ति सुखमव्ययम्}% ३१

\twolineshloka
{प्रत्येकमक्षरं ब्रह्महत्यावंशदवानलः}
{तं यः श्रावयते धीमांस्तं गुरुं सम्प्रपूजयेत्}% ३२

\twolineshloka
{श्रुत्वा कथां वाचकाय गवां द्वन्द्वं प्रदापयेत्}
{सपत्नीकाय सम्पूज्य वस्त्रालङ्कारभोजनैः}% ३३

\twolineshloka
{कुण्डलाभ्यां विराजन्त्यौ मुद्रिकाभिरलङ्कृते}
{रामसीते स्वर्णमय्यौ प्रतिमे शोभने वरे}% ३४

\twolineshloka
{कृत्वा तु वाचकायैव दीयते भो द्विजोत्तम}
{तस्य देवाश्च पितरो वैकुण्ठं प्राप्नुयुस्तदा}% ३५

\twolineshloka
{त्वया पृष्टा रामकथा मया ते कथिता पुरा}
{किमन्यत्कथ्यतां ब्रह्मन्पुरतस्तव धीमतः}% ३६

\twolineshloka
{शृण्वन्ति ये कथामेतां ब्रह्महत्यौघनाशिनीम्}
{ते यान्ति परमं स्थानं यच्च देवैः सुदुर्लभम्}% ३७

\twolineshloka
{गोघ्नश्चापि सुतघ्नश्च सुरापो गुरुतल्पगः}
{क्षणात्पूतो भवत्येव नात्र संशयितुं क्षमम्}% ३८

{॥इति श्रीपद्मपुराणे पातालखण्डे शेषवात्स्यायनसंवादे रामाश्वमेधे श्रवणपठनपुण्यवर्णनं नामाष्टषष्टितमोऽध्यायः॥६८॥}

{॥इति रामाश्वमेधप्रकरणं समाप्तम्॥}

    
    \part{अनुबन्धाः}
    \begingroup
    \let\chapt\sect
    % !TeX program = XeLaTeX
% !TeX root = ../../shloka.tex

\chapt{नामरामायणम्}
\newcommand{\jaya}{
\smallskip
\twolineshloka*
{राम राम जय राजा राम}
{राम राम जय सीता राम}
\vspace{0.5cm}}
\begin{large}
\begin{multicols}{2}
\dnsub{बालकाण्डः}
शुद्धब्रह्मपरात्पर\hfill राम\\
कालात्मकपरमेश्वर\hfill राम\\
शेषतल्पसुखनिद्रित\hfill राम\\
ब्रह्माद्यमरप्रार्थित\hfill राम\\
चण्डकिरणकुलमण्डन\hfill राम\\
श्रीमद्दशरथनन्दन\hfill राम\\
कौसल्यासुखवर्धन\hfill राम\\
विश्वामित्रप्रियधन\hfill राम\\
घोरताटकाघातक\hfill राम\\
मारीचादिनिपातक\hfill राम\\
कौशिकमखसंरक्षक\hfill राम\\
श्रीमदहल्योद्धारक\hfill राम\\
गौतममुनिसम्पूजित\hfill राम\\
सुरमुनिवरगणसंस्तुत\hfill राम\\
नाविकधावितमृदुपद\hfill राम\\
मिथिलापुरजनमोहक\hfill राम\\
विदेहमानसरञ्जक\hfill राम\\
त्र्यम्बककार्मुखभञ्जक\hfill राम\\
सीतार्पितवरमालिक\hfill राम\\
कृतवैवाहिककौतुक\hfill राम\\
भार्गवदर्पविनाशक\hfill राम\\
श्रीमदयोध्यापालक\hfill राम\\
\jaya
\dnsub{अयोध्याकाण्डः}
अगणितगुणगणभूषित\hfill राम\\
अवनीतनयाकामित\hfill राम\\
राकाचन्द्रसमानन\hfill राम\\
पितृवाक्याश्रितकानन\hfill राम\\
प्रियगुहविनिवेदितपद\hfill राम\\
तत्क्षालितनिजमृदुपद\hfill राम\\
भरद्वाजमुखानन्दक\hfill राम\\
चित्रकूटाद्रिनिकेतन\hfill राम\\
दशरथसन्ततचिन्तित\hfill राम\\
कैकेयीतनयार्थित\hfill राम\\
विरचितनिजपितृकर्मक\hfill राम\\
भरतार्पितनिजपादुक\hfill राम\\
\jaya
\dnsub{अरण्यकाण्डः}
दण्डकावनजनपावन\hfill राम\\
दुष्टविराधविनाशन\hfill राम\\
शरभङ्गसुतीक्ष्णार्चित\hfill राम\\
अगस्त्यानुग्रहवर्धित\hfill राम\\
गृध्राधिपसंसेवित\hfill राम\\
पञ्चवटीतटसुस्थित\hfill राम\\
शूर्पणखार्त्तिविधायक\hfill राम\\
खरदूषणमुखसूदक\hfill राम\\
सीताप्रियहरिणानुग\hfill राम\\
मारीचार्तिकृताशुग\hfill राम\\
विनष्टसीतान्वेषक\hfill राम\\
गृध्राधिपगतिदायक\hfill राम\\
कबन्धबाहुच्छेदन\hfill राम\\
शबरीदत्तफलाशन\hfill राम\\
\jaya

\dnsub{किष्किन्धाकाण्डः}
हनुमत्सेवितनिजपद\hfill राम\\
नतसुग्रीवाभीष्टद\hfill राम\\
गर्वितवालिसंहारक\hfill राम\\
वानरदूतप्रेषक\hfill राम\\
हितकरलक्ष्मणसंयुत\hfill राम\\
\jaya

\dnsub{सुन्दरकाण्डः}
कपिवरसन्ततसंस्मृत\hfill राम\\
तद्गतिविघ्नध्वंसक\hfill राम\\
सीताप्राणाधारक\hfill राम\\
दुष्टदशाननदूषित\hfill राम\\
शिष्टहनूमद्भूषित\hfill राम\\
सीतावेदितकाकावन\hfill राम\\
कृतचूडामणिदर्शन\hfill राम\\
कपिवरवचनाश्वासित\hfill राम\\
\jaya

\dnsub{युद्धकाण्डः}
रावणनिधनप्रस्थित\hfill राम\\
वानरसैन्यसमावृत\hfill राम\\
शोषितसरिदीशार्थित\hfill राम\\
विभीषणाभयदायक\hfill राम\\
पर्वतसेतुनिबन्धक\hfill राम\\
कुम्भकर्णशिरश्छेदक\hfill राम\\
राक्षससङ्घविमर्दक\hfill राम\\
अहिमहिरावणचारण\hfill राम\\
संहृतदशमुखरावण\hfill राम\\
विधिभवमुखसुरसंस्तुत\hfill राम\\
खःस्थितदशरथवीक्षित\hfill राम\\
सीतादर्शनमोदित\hfill राम\\
अभिषिक्तविभीषणनत\hfill राम\\
पुष्पकयानारोहण\hfill राम\\
भरद्वाजाभिनिषेवण\hfill राम\\
भरतप्राणप्रियकर\hfill राम\\
साकेतपुरीभूषण\hfill राम\\
सकलस्वीयसमानत\hfill राम\\
रत्नलसत्पीठास्थित\hfill राम\\
पट्टाभिषेकालङ्कृत\hfill राम\\
पार्थिवकुलसम्मानित\hfill राम\\
विभीषणार्पितरङ्गक\hfill राम\\
कीशकुलानुग्रहकर\hfill राम\\
सकलजीवसंरक्षक\hfill राम\\
समस्तलोकाधारक\hfill राम\\
\jaya

\dnsub{उत्तरकाण्डः}
आगतमुनिगणसंस्तुत\hfill राम\\
विश्रुतदशकण्ठोद्भव\hfill राम\\
सितालिङ्गननिर्वृत\hfill राम\\
नीतिसुरक्षितजनपद\hfill राम\\
विपिनत्याजितजनकज\hfill राम\\
कारितलवणासुरवध\hfill राम\\
स्वर्गतशम्बुकसंस्तुत\hfill राम\\
स्वतनयकुशलवनन्दित\hfill राम\\
अश्वमेधक्रतुदीक्षित\hfill राम\\
कालावेदितसुरपद\hfill राम\\
आयोध्यकजनमुक्तिद\hfill राम\\
विधिमुखविबुधानन्दक\hfill राम\\
तेजोमयनिजरूपक\hfill राम\\
संसृतिबन्धविमोचक\hfill राम\\
धर्मस्थापनतत्पर\hfill राम\\
भक्तिपरायणमुक्तिद\hfill राम\\
सर्वचराचरपालक\hfill राम\\
सर्वभवामयवारक\hfill राम\\
वैकुण्ठालयसंस्थित\hfill राम\\
नित्यानन्दपदस्थित\hfill राम\\
\jaya
\end{multicols}
\vspace{-0.5cm}
॥इति श्रीलक्ष्मणाचार्यविरचितं नामरामायणं सम्पूर्णम्॥
\end{large}

\closesection
    \chapt{वाल्मीकि-रामायण-ध्यान-श्लोकाः}

\twolineshloka*
{शुक्लाम्बरधरं विष्णुं शशिवर्णं चतुर्भुजम्}
{प्रसन्नवदनं ध्यायेत् सर्वविघ्नोपशान्तये}

\twolineshloka*
{वागीशाद्याः सुमनसः सर्वार्थानामुपक्रमे}
{यं नत्वा कृतकृत्याः स्युस्तं नमामि गजाननम्}

\dnsub{श्री-गुरु-प्रार्थना}

\twolineshloka*
{गुरुर्ब्रह्मा गुरुर्विष्णुर्गुरुर्देवो महेश्वरः}
{गुरुः साक्षात् परं ब्रह्म तस्मै श्री-गुरवे नमः}

\twolineshloka*
{सदाशिवसमारम्भां शङ्कराचार्यमध्यमाम्}
{अस्मदाचार्यपर्यन्तां वन्दे गुरुपरम्पराम्}

\twolineshloka*
{अखण्डमण्डलाकारं व्याप्तं येन चराचरम्}
{तत्पदं दर्शितं येन तस्मै श्री-गुरवे नमः}

\dnsub{श्री-सरस्वती-प्रार्थना}
\fourlineindentedshloka*
{दोर्भिर्युक्ता चतुर्भिः स्फटिकमणिनिभैरक्षमालां दधाना}
{हस्तेनैकेन पद्मं सितमपि च शुकं पुस्तकं चापरेण}
{भासा कुन्देन्दुशङ्खस्फटिकमणिनिभा भासमानाऽसमाना}
{सा मे वाग्देवतेयं निवसतु वदने सर्वदा सुप्रसन्ना}


\dnsub{श्री-वाल्मीकि-नमस्क्रिया}
\twolineshloka
{कूजन्तं राम रामेति मधुरं मधुराक्षरम्}
{आरुह्य कविताशाखां वन्दे वाल्मीकिकोकिलम्}

\twolineshloka
{वाल्मीकेर्मुनिसिंहस्य कवितावनचारिणः}
{शृण्वन् रामकथानादं को न याति परां गतिम्}

\twolineshloka
{यः पिबन् सततं रामचरितामृतसागरम्}
{अतृप्तस्तं मुनिं वन्दे प्राचेतसमकल्मषम्}

\resetShloka
\dnsub{श्री-हनुमन्नमस्क्रिया}

\twolineshloka
{गोष्पदीकृत-वाराशिं मशकीकृत-राक्षसम्}
{रामायण-महामाला-रत्नं वन्देऽनिलात्मजम्}

\twolineshloka
{अञ्जनानन्दनं वीरं जानकीशोकनाशनम्}
{कपीशमक्षहन्तारं वन्दे लङ्काभयङ्करम्}

\twolineshloka
{उल्लङ्घ्य सिन्धोः सलिलं सलीलं यः शोकवह्निं जनकात्मजायाः}
{आदाय तेनैव ददाह लङ्कां नमामि तं प्राञ्जलिराञ्जनेयम्}

\twolineshloka
{आञ्जनेयमतिपाटलाननं काञ्चनाद्रि-कमनीय-विग्रहम्}
{पारिजात-तरुमूल-वासिनं भावयामि पवमान-नन्दनम्}

\twolineshloka
{यत्र यत्र रघुनाथकीर्तनं तत्र तत्र कृतमस्तकाञ्जलिम्}
{बाष्पवारिपरिपूर्णलोचनं मारुतिं नमत राक्षसान्तकम्}

\twolineshloka
{मनोजवं मारुततुल्यवेगं जितेन्द्रियं बुद्धिमतां वरिष्ठम्}
{वातात्मजं वानरयूथमुख्यं श्रीरामदूतं शिरसा नमामि}

\resetShloka

\dnsub{श्री-रामायण-प्रार्थना}
\fourlineindentedshloka
{यः कर्णाञ्जलिसम्पुटैरहरहः सम्यक् पिबत्यादरात्}
{वाल्मीकेर्वदनारविन्दगलितं रामायणाख्यं मधु}
{जन्म-व्याधि-जरा-विपत्ति-मरणैरत्यन्त-सोपद्रवं}
{संसारं स विहाय गच्छति पुमान् विष्णोः पदं शाश्वतम्}

\twolineshloka
{तदुपगत-समास-सन्धियोगं सममधुरोपनतार्थ-वाक्यबद्धम्}
{रघुवरचरितं मुनिप्रणीतं दशशिरसश्च वधं निशामयध्वम्}

\twolineshloka
{वाल्मीकि-गिरिसम्भूता रामसागरगामिनी}
{पुनातु भुवनं पुण्या रामायणमहानदी}

\twolineshloka
{श्लोकसारजलाकीर्णं सर्गकल्लोलसङ्कुलम्}
{काण्डग्राहमहामीनं वन्दे रामायणार्णवम्}

\twolineshloka
{वेदवेद्ये परे पुंसि जाते दशरथात्मजे}
{वेदः प्राचेतसादासीत् साक्षाद्रामायणात्मना}

\resetShloka
\dnsub{श्री-राम-ध्यानम्}

\fourlineindentedshloka
{वैदेहीसहितं सुरद्रुमतले हैमे महामण्डपे}
{मध्ये पुष्पकमासने मणिमये वीरासने सुस्थितम्}
{अग्रे वाचयति प्रभञ्जनसुते तत्त्वं मुनिभ्यः परं}
{व्याख्यान्तं भरतादिभिः परिवृतं रामं भजे श्यामलम्}

\fourlineindentedshloka
{वामे भूमिसुता पुरश्च हनुमान् पश्चात् सुमित्रासुतः}
{शत्रुघ्नो भरतश्च पार्श्वदलयोर्वाय्वादिकोणेषु च}
{सुग्रीवश्च विभीषणश्च युवराट् तारासुतो जाम्बवान्}
{मध्ये नीलसरोजकोमलरुचिं रामं भजे श्यामलम्}

\twolineshloka
{रामं रामानुजं सीतां भरतं भरतानुजम्}
{सुग्रीवं वायुसूनुं च प्रणमामि पुनः पुनः}

\twolineshloka
{नमोऽस्तु रामाय सलक्ष्मणाय देव्यै च तस्यै जनकात्मजायै}
{नमोऽस्तु रुद्रेन्द्रयमानिलेभ्यो नमोऽस्तु चन्द्रार्कमरुद्गणेभ्यः}

\centerline{\textbf{ॐ श्री-गुरुभ्यो नमः।}}

\resetShloka

    \chapt{मङ्गलश्लोकाः}

\fourlineindentedshloka
{स्वस्ति प्रजाभ्यः परिपालयन्तां}
{न्यायेन मार्गेण महीं महीशाः}
{गोब्राह्मणेभ्यः शुभमस्तु नित्यं}
{लोकाः समस्ताः सुखिनो भवन्तु}

\twolineshloka
{काले वर्षतु पर्जन्यः पृथिवी सस्यशालिनी}
{देशोऽयं क्षोभरहितो ब्राह्मणाः सन्तु निर्भयाः}

\twolineshloka
{अपुत्राः पुत्रिणः सन्तु पुत्रिणः सन्तु पौत्रिणः}
{अधनाः सधनाः सन्तु जीवन्तु शरदां शतम्}

\twolineshloka
{चरितं रघुनाथस्य शतकोटि-प्रविस्तरम्}
{एकैकमक्षरं पुंसां महापातकनाशनम्}

\twolineshloka
{शृण्वन् रामायणं भक्त्या यः पादं पदमेव वा}
{स याति ब्रह्मणः स्थानं ब्रह्मणा पूज्यते सदा}

\twolineshloka
{रामाय रामभद्राय रामचन्द्राय वेधसे}
{रघुनाथाय नाथाय सीतायाः पतये नमः}

\twolineshloka
{यन्मङ्गलं सहस्राक्षे सर्वदेवनमस्कृते}
{वृत्रनाशे समभवत् तत्ते भवतु मङ्गलम्}

\twolineshloka
{यन्मङ्गलं सुपर्णस्य विनताऽकल्पयत् पुरा}
{अमृतं प्रार्थयानस्य तत्ते भवतु मङ्गलम्}

\twolineshloka
{अमृतोत्पादने दैत्यान् घ्नतो वज्रधरस्य यत्}
{अदितिर्मङ्गलं प्रादात् तत्ते भवतु मङ्गलम्}

\twolineshloka
{त्रीन् विक्रमान् प्रक्रमतो विष्णोरमिततेजसः}
{यदासीन्मङ्गलं राम तत्ते भवतु मङ्गलम्}

\twolineshloka
{ऋषयः सागरा द्वीपा वेदा लोका दिशश्च ते}
{मङ्गलानि महाबाहो दिशन्तु तव सर्वदा}

\twolineshloka
{मङ्गलं कोसलेन्द्राय महनीयगुणाब्धये}
{चक्रवर्तितनूजाय सार्वभौमाय मङ्गलम्}

\fourlineindentedshloka*
{कायेन वाचा मनसेन्द्रियैर्वा}
{बुद्‌ध्याऽऽत्मना वा प्रकृतेः स्वभावात्}
{करोमि यद्यत् सकलं परस्मै}
{नारायणायेति समर्पयामि}

    \input{../../namavali-manjari/100/Rama_108}
    \input{../../namavali-manjari/100/Sita_108}
    \input{../../namavali-manjari/100/Anjaneya_108}
    \input{../../namavali-manjari/100/Rama_Ramarahasya_108}
    \input{../../namavali-manjari/100/Sita_Ramarahasya_108}
    \input{../../namavali-manjari/100/Anjaneya_Ramarahasya_108}
    \endgroup
    
\end{center}


\newpage\mbox{}
\clearemptydoublepage

\end{document}
