\sect{अष्टमोऽध्यायः --- श्रीरामावतारकथावर्णनम्}

\src{स्कन्दपुराणम्}{खण्डः १ (माहेश्वरखण्डः)}{केदारखण्डः}{अध्यायः ०८}
\vakta{}
\shrota{}
\tags{}
\notes{Describes Rāma's avatara very briefly, in the context of Lingārchana mahatmyam. Also includes Nandikeshvara's curse to Rāvaṇa, and the story of Rāmāyana in brief.}
\textlink{https://sa.wikisource.org/wiki/स्कन्दपुराणम्/खण्डः_१_(माहेश्वरखण्डः)/केदारखण्डः/अध्यायः_०८}
\translink{https://www.wisdomlib.org/hinduism/book/the-skanda-purana/d/doc365955.html}

\storymeta




\uvacha{लोमश उवाच}

\twolineshloka
{तस्करोऽपि पुरा ब्रह्मन्सर्वधर्मबाहिष्कृतः}
{ब्रह्मघ्नोऽसौ सुरापश्च सुवर्णस्य च तस्करः}%॥ १ ॥

\twolineshloka
{लम्पटोहि महापाप उत्तमस्त्रीषु सर्वदा}
{द्यूतकारी सदा मन्दः कितवैः सह सङ्गतः}%॥ २ ॥

\twolineshloka
{एकदा क्रीडता तेन हारितं द्यूतमद्भुतम्}
{कितवैर्मर्द्यमानो हि तदा नोवाच किञ्चन}%॥ ३ ॥

\twolineshloka
{पीडितोऽप्यभवत्तूष्णीं तैरुक्तः पापकृत्तमः}
{द्यूते त्वया च तद्द्रव्यं हारितं किं प्रयच्छसि}%॥ ४ ॥

\twolineshloka
{नो वा तत्कथ्यतां शीघ्रं याथातथ्येन दुर्मते}
{यद्धारितं प्रयच्छामि रात्रावित्यब्रवीच्च सः}%॥ ५ ॥

\twolineshloka
{तैर्मुक्तस्तेन वाक्येन गतास्ते कितवादयः}
{तदा निशीथसमये गतोऽसौ शिवमन्दिरम्}%॥ ६ ॥

\twolineshloka
{शिरोधिरुह्य शम्भोश्च घण्टामादातुमुद्यतः}
{तावत्कैलासशिखरे शम्भुः प्रोवाच किङ्करान्}%॥ ७ ॥

\twolineshloka
{अनेन यत्कृतं चाद्य सर्वेषामधिकं भुवि}
{सर्वेषामेव भक्तानां वरिष्ठोऽयं च मत्प्रियः}%॥ ८ ॥

\twolineshloka
{इति प्रोक्त्वान यामास वीरभद्रादिभिर्गणैः}
{ते सर्वे त्वरिता जग्मुः कैलासाच्छिववल्लभात्}%॥ ९ ॥

\threelineshloka
{सर्वैर्डमरुनादेन नादितं भुवनत्रयम्}
{तान्दृष्ट्वा सहसोत्तीर्य तस्करोसौ दुरात्मवान्}
{लिङ्गस्य मस्तकात्सद्यः पलायनपरोऽभवत्}%॥ १० ॥

\onelineshloka
{पलायमानं तं दृष्ट्वा वीरभद्रः समाह्वयत्}%॥ ११ ॥

\twolineshloka
{कस्माद्विभेपि रे मन्द देवदेवो महेस्वरः}
{प्रसन्नस्तव जातोद्य उदारचरितो ह्यसौ}%॥ १२ ॥

\twolineshloka
{इत्युक्त्वा तं विमाने च कृत्वा कैलासमाययौ}
{पार्षदो हि कृतस्तेन तस्करो हि महात्मना}%॥ १३ ॥

\twolineshloka
{तस्माद्भाव्या शिवे भक्तिः सर्वेषामपि देहिनाम्}
{पशवोऽपि हि पूज्याः स्युः किं पुनर्मानवाभुवि}%॥ १४ ॥

\twolineshloka
{ये तार्किकास्तर्कपरास्तथ मीमांसकाश्च ये}
{अन्योन्यवादिनश्चान्ये चान्ये वात्मवितर्ककाः}%॥ १५ ॥

\twolineshloka
{एकवाक्यं न कुर्वन्ति शिवार्चनबहिष्कृताः}
{तर्को हि क्रियते यैश्च तेसर्वे किं शिवं विना}%॥ १६ ॥

\twolineshloka
{तथा किं बहुनोक्तेन सर्वेऽपि स्थिरजङ्गमाः}
{प्राणिनोऽपि हि जायन्ते केवलं लिङ्गधारिणः}%॥ १७ ॥

\twolineshloka
{पिण्डीयुक्तं यथा लिङ्गं स्थापितं च यथाऽभवत्}
{तथा नरा लिङ्गयुक्ताः पिण्डीभूतास्तथा स्त्रियः}%॥ १८ ॥

\twolineshloka
{शिवशक्तियुतं सर्वं जगदेतच्चराचरम्}
{तं शिवं मौढ्यतस्त्यक्त्वा मूढाश्चान्यं भजन्ति ये}%॥ १९ ॥

\twolineshloka
{धर्ममात्यन्तिकं तुच्छं नश्वरं क्षणभङ्गुरम्}
{यो विष्णुः स शिवो ज्ञेयो यः शिवो विष्णुरेव सः}%॥ २० ॥

\twolineshloka
{पीठिका विष्णुरूपं स्याल्लिङ्गरूपी महेश्वरः}
{तस्माल्लिङ्गार्चनं श्रेष्ठं सर्वेषामपि वै द्विजाः}%॥ २१ ॥

\twolineshloka
{ब्रह्मा मणिमयं लिङ्गं पूजयत्यनिशं शुभम्}
{इन्द्रो रत्नमयं लिङ्गं चन्द्रो मुक्तामयं तथा}%॥ २२ ॥

\twolineshloka
{भानुस्ताम्रमयं लिङ्गं पूजयत्यनिशं शुभम्}
{रौक्मं लिङ्गं कुबेरश्च पाशी चारक्तमेव च}%॥ २३ ॥

\twolineshloka
{यमो नीलमयं लिङ्गं राजतं नैर्ऋतस्तथा}
{काश्मीरं पवनो लिङ्गमर्चयत्यनिशं विभोः}%॥ २४ ॥

\twolineshloka
{एवं ते लिङ्गिताः सर्वे लोकपालाः सवासवाः}
{तथा सर्वेऽपि पाताले गन्धर्वाः किन्नरैः सह}%॥ २५ ॥

\twolineshloka
{दैत्यानां वैष्णवाः केचित्प्रह्लादप्रमुखा द्विजाः}
{तथाहि राक्षसानां च विभीषणपुरोगमाः}%॥ २६ ॥

\twolineshloka
{बलिश्च नमुचिश्चैव हिरण्यकशिपुस्तथा}
{वृषपर्वा वृषश्चैव संह्रादो बाण एव च}%॥ २७ ॥

\twolineshloka
{एते चान्ये च बहवः शिष्याः शुक्रस्य धीमतः}
{एवं शिवार्चनरताः सर्वे ते दैत्यदानवाः}%॥ २८ ॥

\twolineshloka
{राक्षसा एव ते सर्वे शिवपूजान्विताः सदा}
{हेतिः प्रहेतिः संयातिर्विघसः प्रघसस्तथा}%॥ २९ ॥

\twolineshloka
{विद्युज्जिह्वस्तीक्ष्णदंष्ट्रो धूम्राक्षो भीमविक्रमः}
{माली चैव सुमाली च माल्यवानतिभीषमः}%॥ ३० ॥

\twolineshloka
{विद्युत्कैशस्तडिज्जिह्वो रावणश्च महाबलः}
{कुम्भकर्णो दुराधर्षो वेगदर्शी प्रतापवान्}%॥ ३१ ॥

\twolineshloka
{एते हि राक्षसाः श्रेष्ठा शिवार्चनरताः सदा}
{लिङ्गमभ्यर्च्य च सदा सिद्धिं प्राप्ताः पुरा तु ते}%॥ ३२ ॥

\twolineshloka
{रावणेन तपस्तप्तं सर्वेषामपि दुःखहम्}
{तपोधिपो महादेवस्तुतोष च तदा भृशम्}%॥ ३३ ॥

\twolineshloka
{वरान्प्रायच्छत तदा सर्वेषामपि दुर्लभान्}
{ज्ञानं विज्ञानसहितं लब्धं तेन सदाशिवात्}%॥ ३४ ॥

\twolineshloka
{अजेयत्वं च सङ्ग्रामे द्वैगुण्यं शिरसामपि}
{पञ्चवक्त्रो महा देवो दशवक्त्रोऽथ रावणः}%॥ ३५ ॥

\twolineshloka
{देवानृषीन्पितॄंश्चैव निर्जित्य तपसा विभुः}
{महेशस्य प्रसादाच्च सर्वेषामधिकोऽभवत्}%॥ ३६ ॥

\twolineshloka
{राजा त्रिकूटाधिपतिर्महेशेन कृतो महान्}
{सर्वेषां राक्षसानां च परमासनमास्तितः}%॥ ३७ ॥

\twolineshloka
{तपस्विनां परीक्षायै यदृषीणां विहिंसनम्}
{कृतं तेन तदा विप्रा रावणेन तपस्विना}%॥ ३८ ॥

\twolineshloka
{अजेयो हि महाञ्जातो रावणो लोकरावणः}
{सृष्ट्यन्तरं कृतं येन प्रसादाच्छङ्करस्य च}%॥ ३९ ॥

\twolineshloka
{लोकपाला जितास्तेन प्रतापेन तपस्विना}
{ब्रह्मापि विजितो येन तपसा परमेण हि}%॥ ४० ॥

\twolineshloka
{अमृतांशुकरो भूत्वा जितो येन शशी द्विजाः}
{दाहकत्वाज्जितो वह्निरीशः कैलासतोलनात्}%॥ ४१ ॥

\twolineshloka
{ऐश्वर्येण जितश्चेन्द्रो विष्णुः सर्वगतस्तथा}
{लिङ्गार्चनप्रसादेन त्रैलोक्यं च वशीकृतम्}%॥ ४२ ॥

\twolineshloka
{तदा सर्वे सुरगणा ब्रह्मविष्णुपुरोगमाः}
{मेरुपृष्ठं समासाद्य सुमन्त्रं चक्रिरे तदा}%॥ ४३ ॥

\twolineshloka
{पीडिताः स्मो रावणेन तपसा दुष्करेण वै}
{गोकर्णाख्ये गिरौ देवाः श्रूयतां परमाद्भुतम्}%॥ ४४ ॥

\threelineshloka
{साक्षाल्लिङ्गार्चनं येन कृतमस्ति महात्मना}
{ज्ञानज्ञेयं ज्ञानगम्यं यद्यत्परममद्भुतम्}
{तत्कृतं रावणेनैव सर्वेषां दुरतिक्रमम्}%॥ ४५ ॥

\twolineshloka
{वैराग्यं परमास्थाय औदार्यं च ततोऽधिकम्}
{तेनैव ममता त्यक्ता रावणेन महात्मना}%॥ ४६ ॥

\twolineshloka
{संवत्सरसहस्राच्च स्वशिरो हि महाभुजः}
{कृत्त्वा करेण लिङ्गस्य पूजनार्थं समर्पयत्}%॥ ४७ ॥

\twolineshloka
{रावणस्य कबन्धं च तदग्रे च समीपतः}
{योगधारणया युक्तं परमेण समाधिना}%॥ ४८ ॥

\threelineshloka
{लिङ्गे लयं समाधाय कयापि कलया स्थितम्॥}
{अन्यच्छिरोविवृश्च्यैवं तेनापि शिवपूजनम्}
{कृतं नैवान्यमुनिना तथा चैवापरेणहि}%॥ ४९ ॥

\twolineshloka
{एवं शिरांस्येव बहूनि तेन समर्पितान्येव शिवार्चनार्थे}
{भूत्वा कबन्धो हि पुनः पुनश्च शिवोऽसौ वरदो बभूव}%॥ ५० ॥

\twolineshloka
{मया विनासुरस्तत्र पिण्डीभूतेन वै पुरा}
{वरान्वरय पौलस्त्य यथेष्टं तान्ददाम्यहम्}%॥ ५१ ॥

\twolineshloka
{रावणेन तदा चोक्तः शिवः परममङ्गलः}
{यदि प्रसन्नो भगवन्देयो मे वर उत्तमः}%॥ ५२ ॥

\twolineshloka
{न कामयेऽन्यं च वरमाश्रये त्वत्पदाम्बुजम्}
{यथा तथा प्रदातव्यं यद्यस्ति च कृपा मयि}%॥ ५३ ॥

\twolineshloka
{तदा सदाशिवेनोक्तो रावणो लोकरावणः}
{मत्प्रसादाच्च सर्वं त्वं प्राप्स्यसे मनसेप्सितम्}%॥ ५४ ॥

\twolineshloka
{एवं प्राप्तं शिवात्सर्वं रावणेन सुरेश्वराः}
{तस्मात्सर्वैर्भवद्भिश्च तपसा परमेण हि}%॥ ५५ ॥

\twolineshloka
{विजेतव्यो रावणोयमिति मे मनसि स्थितम्}
{अच्युतस्य वचः श्रुत्वा ब्रह्माद्या देवतागणाः}%॥ ५६ ॥

\twolineshloka
{चिन्तामापेदिरे सर्वे चिरं ते विषयान्विताः}
{ब्रह्मापि चेन्द्रियग्रस्तः सुता रमितुमुद्यतः}%॥ ५७ ॥

\twolineshloka
{इन्द्रो हि जारभावाच्च चन्द्रो हि गुरुतल्पगः}
{यमः कदर्यभावाच्च चञ्चलत्वात्सदागतिः}%॥ ५८ ॥

\twolineshloka
{पावकः सर्वभक्षित्वात्तथान्ये देवतागणाः}
{अशक्ता रावणं जेतुं तपसा च विजृम्भितम्}%॥ ५९ ॥

\twolineshloka
{शैलादो हि महातेजा गणश्रेष्ठः पुरातनः}
{बुद्धि मान्नीतिनिपुणो महाबलपराक्रमी}%॥ ६० ॥

\twolineshloka
{शिवप्रियो रुद्ररूपी महात्मा ह्युवाच सर्वानथ चेन्द्रमुख्यान्}
{कस्माद्यूयं सम्भ्रमादागताश्च एतत्सर्वं कथ्यतां विस्तरेण}%॥ ६१

\onelineshloka
{नन्दिना च तदा सर्वे पृष्टाः प्रोचुस्त्वरान्विताः}%॥ ६२ ॥

\uvacha{देवा ऊचुः}

\twolineshloka
{रावणेन वयं सर्वे निर्जिता मुनिभिः सह}
{प्रसादयितुमायाताः शिवं लोकेश्वरेश्वरम्}%॥ ६३ ॥

\threelineshloka
{प्रहस्य भगवान्नन्दी ब्रह्माणं वै ह्युवाच ह}
{क्व यूयं क्व शिवः शम्भुस्तपसा परमेण हि}
{द्रष्टव्यो हृदि मध्यस्थः सोऽद्य द्रष्टुं न पार्यते}%॥ ६४ ॥

\twolineshloka
{यावद्भावा ह्यनेकाश्च इन्द्रियार्थास्तथैव च}
{यावच्च ममताभावस्तावदीशो हि दुर्लभः}%॥ ६५ ॥

\twolineshloka
{जितेन्द्रियाणां शान्तानां तन्निष्ठानां महात्मनाम्}
{सुलभो लिङ्गरूपी स्याद्भवतां हि सुदुर्लभः}%॥ ६६ ॥

\threelineshloka
{तदा ब्रह्मादयो देवा ऋषयश्च विपश्चितः}
{प्रणम्य नन्दिनं प्राहुः कस्मात्त्वं वानराननः}
{तत्सर्वं कथयान्यं च रावणस्य तपोबलम्}%॥ ६७ ॥

\uvacha{नन्दीश्वर उवाच}

\twolineshloka
{कुबेरोऽधिकृतस्तेन शङ्करेण महात्मना}
{धनानामादिपत्ये च तं द्रष्टुं रावणोऽत्र वै}%॥ ६८ ॥

\twolineshloka
{आगच्छत्त्वरया युक्तः समारुह्य स्ववाहनम्}
{मां दृष्ट्वा चाब्रवीत्क्रुद्धः कुबेरो ह्यत्र आगतः}%॥ ६९ ॥

\twolineshloka
{त्वया दृष्टोऽथ वात्रासौ कथ्यतामविलम्बितम्}
{किं कार्यं धनदेनाद्य इति पृष्टो मया हि सः}%॥ ७० ॥

\twolineshloka
{तदोवाच महातेजा रावणो लोकरावणः}
{मय्यश्रद्धान्वितो भूत्वा विषयात्मा सुदुर्मदः}%॥ ७१ ॥

\threelineshloka
{शिक्षापयितुमारब्धो मैवं कार्यमिति प्रभो॥}
{यथाहं च श्रिया युक्त आढ्योऽहं बलवानहम्}
{तथा त्वं भव रे मूढ मा मूढत्वमुपार्जय}%॥ ७२ ॥

\twolineshloka
{अहं मूढः कृतस्तेन कुबेरेण महात्मना}
{मया निराकृतो रोषात्तपस्तेपे स गुह्यकः}%॥ ७३ ॥

\twolineshloka
{कुबेरः स हि नन्दिन्किमागतस्तव मन्दिरम्}
{दीयतां च कुबेरोद्य नात्र कार्या विचारणा}%॥ ७४ ॥

\twolineshloka
{रावणस्य वचः श्रुत्वा ह्यवोचं त्वरितोऽप्यहम्}
{लिङ्गकोसि महाभाग त्वमहं च तथाविधः}%॥ ७५ ॥

\twolineshloka
{उभयोः समनां ज्ञात्वा वृथा जल्पसि दुर्मते}
{यथोक्तः स त्ववादीन्मां वदनार्थे बलोद्धतः}%॥ ७६ ॥

\threelineshloka
{यथा भवद्भिः पृष्टोऽहं वदनार्थे महात्मभिः}
{पुरावृत्तं मया प्रोक्तं शिवार्चनविधेः फलम्}
{शिवेन दत्तं सालूप्यं न गृहीतं मया तदा}%॥ ७७ ॥

\twolineshloka
{याचितं च मया शम्भोर्वदनं वानरस्य च}
{शिवेन कृपया दत्तं मम कारुण्यशालिना}%॥ ७८ ॥

\twolineshloka
{निराभिमानिनो ये च निर्दभा निष्परिग्रहाः}
{शम्भोः प्रियास्ते विज्ञेया ह्यन्ये शिववबहिष्कृताः}%॥ ७९ ॥

\twolineshloka
{तथावदन्मया सार्द्धं रावणस्तपसो बलात्}
{मया च याचितान्येव दश वक्त्राणि धीमता}%॥ ८० ॥

\twolineshloka
{उपहासकरं वाक्यं पौलस्त्यस्य तदा सुराः}
{मया तदा हि शप्तोऽसौ रावणो लोकरावणः}%॥ ८१ ॥

\threelineshloka
{ईदृशान्येव वक्त्राणि येषां वै सम्भवन्ति हि}
{तैः समेतो यदा कोऽपि नरवर्यो महातपाः}
{मां पुरस्कृत्य सहसा हनिष्यति न संशयः}%॥ ८२ ॥

\twolineshloka
{एवं शप्तो मया ब्रह्मन्रावणो लोकरावणः}
{अर्चितं केवलं लिङ्गं विना तेन महात्मना}%॥ ८३ ॥

\twolineshloka
{पीठिकारूपसंस्थेन विना तेन सुरोत्तमाः}
{विष्णुना हि महाभागास्तस्मात्सर्वं विधास्यति}%॥ ८४ ॥

\twolineshloka
{देवदेवो महादेवो विष्णुरूपी महेश्वरः}
{सर्वे यूयं प्रार्थयन्तु विष्णुं सर्वगुहाशयम्}%॥ ८५ ॥

\threelineshloka
{अहं हि सर्वदेवानां पुरोवर्ती भवाम्यतः}
{ते सर्वे नन्दिनो वाक्यं श्रुत्वा मुदितमानसाः}
{वैकुण्ठमागता गीर्भिर्विष्णुं स्तोतुं प्रचक्रिरे}%॥ ८६ ॥

\uvacha{देवा ऊचुः}

\twolineshloka
{नमो भगवते तुभ्यं देवदेव जगत्पते}
{त्वदाधारमिदं सर्वं जगदेतच्चराचरम्}%॥ ८७ ॥

\twolineshloka
{एतल्लिङ्गं त्वया विष्णो धृतं वै पिण्डिरूपिणा}
{महाविष्णुस्वरूपेण घातितौ मधुकैटभौ}%॥ ८८ ॥

\twolineshloka
{तथा कमठरूपेण धृतो वै मन्दराचलः}
{वराहरूपमास्थाय हिरण्याक्षो हतस्त्वया}%॥ ८९ ॥

\twolineshloka
{हिरण्यकशिपुर्दैत्यो हतो नृहरिरूपिणा}
{त्वया चैव बलिर्बद्धो दैत्यो वामनरूपिणा}%॥ ९० ॥

\twolineshloka
{भृगूणामन्वये भूत्वा कृतवीर्यात्मजो हतः}
{इतोप्यस्मान्महाविष्णो तथैव परिपालय}%॥ ९१

\onelineshloka
{रावमस्य भयादस्मात्त्रातुं भूयोर्हसि त्वरम्}%॥ ९२ ॥

\twolineshloka
{एवं सम्प्रार्थितो देवैर्भगवान्भूतभावनः}
{उवाच च सुरान्सर्वान्वासुदेवो जगन्मयः}%॥ ९३ ॥

\threelineshloka
{हे देवाः श्रूयतां वाक्यं प्रस्तावसदृशं महत्}
{शैलादिं च पुरस्कृत्य सर्वे यूयं त्वरान्विताः}
{अवतारान्प्रकुर्वन्तु वानरीं तनुमाश्रिताः}%॥ ९४ ॥

\threelineshloka
{अहं हि मानुषो भूत्वा ह्यज्ञानेन समावृतः}
{सम्भविष्याम्ययोध्यायं गृहे दशरथस्य च}
{ब्रह्मविद्यासहायोस्मि भवतां कार्यसिद्धये}%॥ ९५ ॥

\twolineshloka
{जनकस्य गृहे साक्षाद्ब्रह्मविद्या जनिष्यति}
{भक्तो हि रावणः साक्षाच्छिवध्यानपरायणः}%॥ ९६ ॥

\twolineshloka
{तपसा महता युक्तो ब्रह्मविद्यां यदेच्छति}
{तदा सुसाध्यो भवति पुरुषो धर्मनिर्जितः}%॥ ९७ ॥

\twolineshloka
{एवं सम्भाष्य भगवान्विष्णुः परममङ्गलः}
{वाली चेन्द्रांशसम्भूतः सुग्रीवों शुमतः सुतः}%॥ ९८ ॥

\twolineshloka
{तथा ब्रह्मांशसम्भूतो जाम्बवान्नृक्षकुञ्जरः}
{शिलादतनयो नन्दी शिवस्यानुचरः प्रियः}%॥ ९९ ॥

\twolineshloka
{यो वै चैकादशो रुद्रो हनूमान्स महाकपिः}
{अवतीर्णः सहायार्थं विष्णोरमिततेजसः}%॥ १०० ॥

\twolineshloka
{मैन्दादयोऽथ कपयस्ते सर्वे सुरसत्तमाः}
{एवं सर्वे सुरगणा अवतेरुर्यथा तथम्}%॥ १०१ ॥

\twolineshloka
{तथैव विष्णुरुत्पन्नः कौशल्यानन्दवर्द्धनः}
{विश्वस्य रमणाच्चैव राम इत्युच्यते बुधैः}%॥ १०२

\onelineshloka
{शेषोपि भक्त्या विष्णोश्च तपसाऽवातरद्भुवि}%॥ १०३ ॥

\twolineshloka
{दोर्दण्डावपि विष्णोश्च अवतीर्णौ प्रतापिनौ}
{शत्रुघ्नभरताख्यौ च विख्यातौ भुवनत्रये}%॥ १०४ ॥

\threelineshloka
{मिथिलाधिपतेः कन्या या उक्ता ब्रह्मवादिभिः}
{सा ब्रह्मविद्यावतरत्सुराणां कार्य्यसिद्धये}
{सीता जाता लाङ्गलस्य इयं भूमिविकर्षणात्}%॥ १०५ ॥

\twolineshloka
{तस्मात्सीतेति विख्याता विद्या सान्वीक्षिकी तदा}
{मिथिलायां समुत्पन्ना मैथितीत्यभिधीयते}%॥ १०६ ॥

\twolineshloka
{जनकस्य कुले जाता विश्रुता जनकात्मजा}
{ख्याता वेदवती पूर्वं ब्रह्मविद्याघनाशिनी}%॥ १०७

\onelineshloka
{सा दत्ता जनकेनैव विष्णवे परमात्मने}%॥ १०८ ॥

\twolineshloka
{तयाथ विद्यया सार्द्धं देवदेवो जगत्पतिः}
{उग्रे तपसि लीनोऽसौ विष्णुः परमदुष्करम्}%॥ १०९ ॥

\twolineshloka
{रावणं जेतुकामो वै रामो राजीवलोचनः}
{अरण्यवासमकरोद्देवानां कार्यसिद्धये}%॥ ११० ॥

\twolineshloka
{शेषावतारोऽपि महांस्तपः परमदुष्करम्}
{तताप परया शक्त्या देवानां कार्यसिद्धये}%॥ १११

\onelineshloka
{शत्रुघ्नो भरतश्चैव तेपतुः परमं तपः}%॥ ११२ ॥

\threelineshloka
{ततोऽसौ तपसा युक्तः सार्द्धं तैर्देवतागणैः}
{सगणं रावणं रामः षड्भिर्मासैरजीहनत्}
{विष्णुना घातितः शस्त्रैः शिवसारूप्यमाप्तवान्}%॥ ११३

\onelineshloka
{सगमः स पुनः सद्यो बन्धुभिः सह सुव्रताः}%॥ ११४ ॥

\threelineshloka
{शिवप्रसादात्सकलं द्वैताद्वैतमवाप ह}
{द्वैताद्वैतविवेकार्थमृपयोप्यत्र मोहिताः}
{तत्सर्वं प्राप्नुवन्तीह शिवार्चनरता नराः}%॥ ११५ ॥

\threelineshloka
{येऽर्चयन्ति शिवं नित्यं लिङ्गरूपिणमेव च}
{स्त्रियो वाप्यथ वा शूद्राः श्वपचा ह्यन्त्यवासिनः}
{तं शिवं प्राप्नुवन्त्येव सर्वदुःखोपनाशनम्}%॥ ११६

\onelineshloka
{पशवोऽपि परं याताः किं पुनर्मानुषादयः}%॥ ११७ ॥

\twolineshloka
{ये द्विजा ब्रह्मचर्येण तपः परममास्थिताः}
{वर्षैरनेकैर्यज्ञानां तेऽपि स्वर्गपरा भवन्}%॥ ११८ ॥

\twolineshloka
{ज्योतिष्टोमो वाजपेयो ह्यतिरात्रादयो ह्यमी}
{यज्ञाः स्वर्गं प्रयच्छन्ति सत्त्रीणां नात्र संशयः}%॥ ११९ ॥

\twolineshloka
{तत्र स्वर्गसुखं भुक्त्वा पुण्यक्षयकरं महत्}
{पुण्यक्षयेऽपि यज्वानो मर्त्यलोकं पतन्ति वै}%॥ १२० ॥

\twolineshloka
{पतितानां च संसारे दैवाद्बुद्धिः प्रजायते}
{गुणत्रयमयी विप्रास्तासुतास्त्विह योनिषु}%॥ १२१ ॥

\twolineshloka
{यथा सत्त्वं सम्भवति सत्त्वयुक्तभवं नराः}
{राजसाश्च तथा ज्ञेयास्ता मसाश्चैव ते द्विजाः}%॥ १२२ ॥

\twolineshloka
{एवं संसारचक्रेऽस्मिन्भ्रमिता बहवो जनाः}
{यदृच्छया दैवगत्या शिवं संसेवते नरः}%॥ १२३ ॥

\twolineshloka
{शिवध्यानपराणां च नराणां यतचेतसाम्}
{मायानिरसनं सद्यो भविष्यति न चान्यथा}%॥ १२४ ॥

\twolineshloka
{मायानिरसनात्सद्यो नश्यत्येव गुणत्रयम्}
{यदा गुणत्रयातीतो भवतीति स मुक्तिभाक्}%॥ १२५ ॥

\twolineshloka
{तस्माल्लिङ्गार्चनं भाव्यं सर्वेषामपि देहिनाम्}
{लिङ्गरूपी शिवो भूत्वा त्रायते सञ्चराचरम्}%॥ १२६ ॥

\twolineshloka
{पुरा भवद्भिः पृष्टोऽहं लिङ्गरूपी कथं शिवः}
{तत्सर्वं कथितं विप्रा याथातथ्येन सम्प्रति}%॥ १२७ ॥

\twolineshloka
{कथं गरं भक्षितवाञ्छिवो लोकमहेश्वरः}
{तत्सर्वं श्रूयतां विप्रा यतावत्कथयामि वः}%॥ १२८ ॥

॥इति श्रीस्कान्दे महापुराण एकाशीतिसाहस्र्यां संहितायां प्रथमे माहेश्वरखण्डे केदारखण्डे शिवशास्त्रे शिवलिङ्गार्चनमाहात्म्यकथने श्रीरामावतारकथावर्णनं नामाष्टमोऽध्यायः॥८॥