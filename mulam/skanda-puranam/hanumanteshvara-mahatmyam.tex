\sect{त्र्यशीतितमोऽध्यायः --- हनूमन्तेश्वरतीर्थमाहात्म्यवर्णनम्}

\src{स्कन्दपुराणम्}{खण्डः ५ (अवन्तीखण्डः)}{रेवा खण्डम्}{अध्यायः ०८३}
\vakta{}
\shrota{}
\tags{}
\notes{}
\textlink{https://sa.wikisource.org/wiki/स्कन्दपुराणम्/खण्डः_५_(अवन्तीखण्डः)/रेवा_खण्डम्/अध्यायः_०८३}
\translink{https://www.wisdomlib.org/hinduism/book/the-skanda-purana/d/doc425812.html}

\storymeta




\uvacha{श्रीमार्कण्डेय उवाच}


\threelineshloka
{ततो गच्छेन्महाराज तीर्थं परमशोभनम्}
{ब्रह्महत्याहरं प्रोक्तं रेवातटसमाश्रयम्}
{हनूमताभिधं ह्यत्र विद्यते लिङ्गमुत्तमम्}%॥ १ ॥

\uvacha{युधिष्ठिर उवाच}

\twolineshloka
{हनूमन्तेश्वरं नाम कथं जातं वदस्व मे}
{ब्रह्महत्याहरं तीर्थं रेवादक्षिणसंस्थितम्}%॥ २ ॥

\uvacha{श्रीमार्कण्डेय उवाच}

\twolineshloka
{साधु साधु महाबाहो सोमवंशविभूषण}
{गुह्याद्गुह्यतरं तीर्थं नाख्यातं कस्यचिन्मया}%॥ ३ ॥

\twolineshloka
{तव स्नेहात्प्रवक्ष्यामि पीडितो वार्द्धकेन तु}
{पूर्वं जातं महद्युद्धं रामरावणयोरपि}%॥ ४ ॥

\twolineshloka
{पुलस्त्यो ब्रह्मणः पुत्रो विश्रवास्तस्य वै सुतः}
{रावणस्तेन सञ्जातो दशास्यो ब्रह्मराक्षसः}%॥ ५ ॥

\twolineshloka
{त्रैलोक्यविजयी भूतः प्रसादाच्छूलिनः स च}
{गीर्वाणा विजिताः सर्वे रामस्य गृहिणी हृता}%॥ ६ ॥

\twolineshloka
{वारितः कुम्भकर्णेन सीतां मोचय मोचय}
{विभीषणेन वै पापो मन्दोदर्या पुनःपुनः}%॥ ७ ॥

\twolineshloka
{त्वं जितः कार्तवीर्येण रैणुकेयेन सोऽपि च}
{स रामो रामभद्रेण तस्य सङ्ख्ये कथं जयः}%॥ ८ ॥

\uvacha{रावण उवाच}

\twolineshloka
{वानरैश्च नरैरृक्षैर्वराहैश्च निरायुधैः}
{देवासुरसमूहैश्च न जितोऽहं कदाचन}%॥ ९ ॥

\uvacha{श्रीमार्कण्डेय उवाच}

\twolineshloka
{सुग्रीवहनुमद्भ्यां च कुमुदेनाङ्गदेन च}
{एतैरन्यैः सहायैश्च रामचन्द्रेण वै जितः}%॥ १० ॥

\twolineshloka
{रामचन्द्रेण पौलस्त्यो हतः सङ्ख्ये महाबलः}
{वनं भग्नं हताः शूराः प्रभञ्जनसुतेन च}%॥ ११ ॥

\twolineshloka
{रावणस्य सुतो जन्ये हतश्चाक्षकुमारकः}
{आयामो रक्षसां भीमः सम्पिष्टो वानरेण तु}%॥ १२ ॥

\twolineshloka
{एवं रामायणे वृत्ते सीतामोक्षे कृते सति}
{अयोध्यां तु गते रामे हनुमान्स महाकपिः}%॥ १३ ॥

\twolineshloka
{कैलासाख्यं गतः शैलं प्रणामाय महेशितुः}
{तिष्ठ तिष्ठेत्यसौ प्रोक्तो नन्दिना वानरोत्तमः}%॥ १४ ॥

\twolineshloka
{ब्रह्महत्यायुतस्त्वं हि राक्षसानां वधेन हि}
{भैरवस्य सभा नूनं न द्रष्टव्या त्वया कपे}%॥ १५ ॥

\uvacha{हनुमानुवाच}

\twolineshloka
{नन्दिनाथ हरं पृच्छ पातकस्योपशान्तिदम्}
{पापोऽहं प्लवगो यस्मात्सञ्जातः कारणान्तरात्}%॥ १६ ॥

\uvacha{नन्द्युवाच}

\twolineshloka
{रुद्रदेहोद्भवा किं ते न श्रुता भूतले स्थिता}
{श्रवणाज्जन्मजनितं द्विगुणं कीर्तनाद्व्रजेत्}%॥ १७ ॥

\twolineshloka
{त्रिंशज्जन्मार्जितं पापं नश्येद्रेवावगाहनात्}
{तस्मात्त्वं नर्मदातीरं गत्वा चर तपो महत्}%॥ १८ ॥

\twolineshloka
{गन्धर्वाहसुतोऽप्येवं नन्दिनोक्तं निशम्य च}
{प्रयातो नर्मदातीरमौर्व्यादक्षिणसङ्गमम्}%॥ १९ ॥

\twolineshloka
{दध्यौ सुदक्षिणे देवं विरूपाक्षं त्रिशूलिनम्}
{जटामुकुटसंयुक्तं व्यालयज्ञोपवीतिनम्}%॥ २० ॥

\twolineshloka
{भस्मोपचितसर्वाङ्गं डमरुस्वरनादितम्}
{उमार्द्धाङ्गहरं शान्तं गोनाथासनसंस्थितम्}%॥ २१ ॥

\twolineshloka
{वत्सरान् सुबहून् यावदुपासाञ्चक्र ईश्वरम्}
{तावत्तुष्टो महादेव आजगाम सहोमया}%॥ २२ ॥

\twolineshloka
{उवाच मधुरां वाणीं मेघगम्भीरनिस्वनाम्}
{साधु साध्वित्युवाचेशः कष्टं वत्स त्वया कृतम्}%॥ २३ ॥

\twolineshloka
{न च पूर्वं त्वया पापं कृतं रावणसङ्क्षये}
{स्वामिकार्यरतस्त्वं हि सिद्धोऽसि मम दर्शनात्}%॥ २४ ॥

\threelineshloka
{हनुमांश्च हरं दृष्ट्वा उमार्द्धाङ्गहरं स्थिरम्}
{साष्टाङ्गं प्रणतोऽवोचज्जय शम्भो नमोऽस्तु ते}
{जयान्धकविनाशाय जय गङ्गाशिरोधर}%॥ २५ ॥

\twolineshloka
{एवं स्तुतो महादेवो वरदो वाक्यमब्रवीत्}
{वरं प्रार्थय मे वत्स प्राणसम्भवसम्भव}%॥ २६ ॥

\uvacha{श्रीहनुमानुवाच}

\twolineshloka
{ब्रह्मरक्षोवधाज्जाता मम हत्या महेश्वर}
{न पापोऽहं भवेदेव युष्मत्सम्भाषणे क्षणात्}%॥ २७ ॥

\uvacha{ईश्वर उवाच}

\twolineshloka
{नर्मदातीर्थमाहात्म्याद्धर्मयोगप्रभावतः}
{मन्मूर्तिदर्शनात्पुत्र निष्पापोऽसि न संशयः}%॥ २८ ॥

\twolineshloka
{अन्यं च ते प्रयच्छामि वरं वानरपुङ्गव}
{उपकाराय लोकानां नामानि तव मारुते}%॥ २९ ॥

\twolineshloka
{हनूमानं जनिसुतो वायुपुत्रो महाबलः}
{रामेष्टः फाल्गुनो गोत्रः पिङ्गाक्षोऽमितविक्रमः}%॥ ३० ॥

\twolineshloka
{उदधिक्रमणश्रेष्ठो दशग्रीवस्य दर्पहा}
{लक्ष्मणप्राणदाता च सीताशोकनिवर्तनः}%॥ ३१ ॥

\twolineshloka
{इत्युक्त्वान्तर्दधे देव उमया सह शङ्करः}
{हनूमानीश्वरं तत्र स्थापयामास भक्तितः}%॥ ३२ ॥

\threelineshloka
{आत्मयोगबलेनैव ब्रह्मचर्यप्रभावतः}
{ईश्वरस्य प्रसादेन लिङ्गं कामप्रदं हि तत्}
{अच्छेद्यमप्रतर्क्यं च विनाशोत्पत्तिवर्जितम्}%॥ ३३ ॥

\ldots

॥इति श्रीस्कान्दे महापुराण एकाशीतिसाहस्र्यां संहितायां पञ्चम आवन्त्यखण्डे रेवाखण्डे हनूमन्तेश्वरतीर्थमाहात्म्यवर्णनं नाम त्र्यशीतितमोऽध्यायः॥८३॥