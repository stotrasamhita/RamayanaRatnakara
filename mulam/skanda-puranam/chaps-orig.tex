https://sa.wikisource.org/wiki/स्कन्दपुराणम्/खण्डः_१_(माहेश्वरखण्डः)/केदारखण्डः/अध्यायः_०८
https://www.wisdomlib.org/hinduism/book/the-skanda-purana/d/doc365955.html

॥लोमश उवाच॥
तस्करोऽपि पुरा ब्रह्मन्सर्वधर्मबाहिष्कृतः॥
ब्रह्मघ्नोऽसौ सुरापश्च सुवर्णस्य च तस्करः॥ ८.१ ॥
लंपटोहि महापाप उत्तमस्त्रीषु सर्वदा॥
द्यूतकारी सदा मन्दः कितवैः सह सङ्गतः॥ ८.२ ॥
एकदा क्रीडता तेन हारितं द्यूतमद्भुतम्॥
कितवैर्मर्द्यमानो हि तदा नोवाच किञ्चन॥ ८.३ ॥
पीडितोऽप्यभवत्तूष्णीं तैरुक्तः पापकृत्तमः॥
द्यूते त्वया च तद्द्रव्यं हारितं किं प्रयच्छसि॥ ८.४ ॥
नो वा तत्कथ्यतां शीघ्रं याथातथ्येन दुर्मते॥
यद्धारितं प्रयच्छामि रात्रावित्यब्रवीच्च सः॥ ८.५ ॥
तैर्मुक्तस्तेन वाक्येन गतास्ते कितवादयः॥
तदा निशीथसमये गतोऽसौ शिवमन्दिरम्॥ ८.६ ॥
शिरोधिरुह्य शम्भोश्च घण्टामादातुमुद्यतः॥
तावत्कैलासशिखरे शंभुः प्रोवाच किङ्करान्॥ ८.७ ॥
अनेन यत्कृतं चाद्य सर्वेषामधिकं भुवि॥
सर्वेषामेव भक्तानां वरिष्ठोऽयं च मत्प्रियः॥ ८.८ ॥
इति प्रोक्त्वान यामास वीरभद्रादिभिर्गणैः॥
ते सर्वे त्वरिता जग्मुः कैलासाच्छिववल्लभात्॥ ८.९ ॥
सर्वैर्डमरुनादेन नादितं भुवनत्रयम्॥
तान्दृष्ट्वा सहसोत्तीर्य तस्करोसौ दुरात्मवान्॥
लिङ्गस्य मस्तकात्सद्यः पलायनपरोऽभवत्॥ ८.१० ॥
पलायमानं तं दृष्ट्वा वीरभद्रः समाह्वयत्॥ ८.११ ॥
कस्माद्विभेपि रे मन्द देवदेवो महेस्वरः॥
प्रसन्नस्तव जातोद्य उदारचरितो ह्यसौ॥ ८.१२ ॥
इत्युक्त्वा तं विमाने च कृत्वा कैलासमाययौ॥
पार्षदो हि कृतस्तेन तस्करो हि महात्मना॥ ८.१३ ॥
तस्माद्भाव्या शिवे भक्तिः सर्वेषामपि देहिनाम्॥
पशवोऽपि हि पूज्याः स्युः किं पुनर्मानवाभुवि॥ ८.१४ ॥
ये तार्किकास्तर्कपरास्तथ मीमांसकाश्च ये॥
अन्योन्यवादिनश्चान्ये चान्ये वात्मवितर्ककाः॥ ८.१५ ॥
एकवाक्यं न कुर्वन्ति शिवार्चनबहिष्कृताः॥
तर्को हि क्रियते यैश्च तेसर्वे किं शिवं विना॥ ८.१६ ॥
तथा किं बहुनोक्तेन सर्वेऽपि स्थिरजङ्गमाः॥
प्राणिनोऽपि हि जायन्ते केवलं लिङ्गधारिणः॥ ८.१७ ॥
पिण्डीयुक्तं यथा लिङ्गं स्थापितं च यथाऽभवत्॥
तथा नरा लिङ्गयुक्ताः पिण्डीभूतास्तथा स्त्रियः॥ ८.१८ ॥
शिवशक्तियुतं सर्वं जगदेतच्चराचरम्॥
तं शिवं मौढ्यतस्त्यक्त्वा मूढाश्चान्यं भजन्ति ये॥ ८.१९ ॥
धर्ममात्यन्तिकं तुच्छं नश्वरं क्षणभङ्गुरम्॥
यो विष्णुः स शिवो ज्ञेयो यः शिवो विष्णुरेव सः॥ ८.२० ॥
पीठिका विष्णुरूपं स्याल्लिङ्गरूपी महेश्वरः॥
तस्माल्लिङ्गार्चनं श्रेष्ठं सर्वेषामपि वै द्विजाः॥ ८.२१ ॥
ब्रह्मा मणिमयं लिङ्गं पूजयत्यनिशं शुभम्॥
इन्द्रो रत्नमयं लिङ्गं चन्द्रो मुक्तामयं तथा॥ ८.२२ ॥
भानुस्ताम्रमयं लिङ्गं पूजयत्यनिशं शुभम्॥
रौक्मं लिङ्गं कुबेरश्च पाशी चारक्तमेव च॥ ८.२३ ॥
यमो नीलमयं लिङ्गं राजतं नैर्ऋतस्तथा॥
काश्मीरं पवनो लिङ्गमर्चयत्यनिशं विभोः॥ ८.२४ ॥
एवं ते लिङ्गिताः सर्वे लोकपालाः सवासवाः॥
तथा सर्वेऽपि पाताले गन्धर्वाः किन्नरैः सह॥ ८.२५ ॥
दैत्यानां वैष्णवाः केचित्प्रह्लादप्रमुखा द्विजाः॥
तथाहि राक्षसानां च विभीषणपुरोगमाः॥ ८.२६ ॥
बलिश्च नमुचिश्चैव हिरण्यकशिपुस्तथा॥
वृषपर्वा वृषश्चैव संह्रादो बाण एव च॥ ८.२७ ॥
एते चान्ये च बहवः शिष्याः शुक्रस्य धीमतः॥
एवं शिवार्चनरताः सर्वे ते दैत्यदानवाः॥ ८.२८ ॥
राक्षसा एव ते सर्वे शिवपूजान्विताः सदा॥
हेतिः प्रहेतिः संयातिर्विघसः प्रघसस्तथा॥ ८.२९ ॥
विद्युज्जिह्वस्तीक्ष्णदंष्ट्रो धूम्राक्षो भीमविक्रमः॥
माली चैव सुमाली च माल्यवानतिभीषमः॥ ८.३० ॥
विद्युत्कैशस्तडिज्जिह्वो रावणश्च महाबलः॥
कुंभकर्णो दुराधर्षो वेगदर्शी प्रतापवान्॥ ८.३१ ॥
एते हि राक्षसाः श्रेष्ठा शिवार्चनरताः सदा॥
लिङ्गमभ्यर्च्य च सदा सिद्धिं प्राप्ताः पुरा तु ते॥ ८.३२ ॥
रावणेन तपस्तप्तं सर्वेषामपि दुःखहम्॥
तपोधिपो महादेवस्तुतोष च तदा भृशम्॥ ८.३३ ॥
वरान्प्रायच्छत तदा सर्वेषामपि दुर्लभान्॥
ज्ञानं विज्ञानसहितं लब्धं तेन सदाशिवात्॥ ८.३४ ॥
अजेयत्वं च सङ्ग्रामे द्वैगुण्यं शिरसामपि॥
पञ्चवक्त्रो महा देवो दशवक्त्रोऽथ रावणः॥ ८.३५ ॥
देवानृषीन्पितॄंश्चैव निर्जित्य तपसा विभुः॥
महेशस्य प्रसादाच्च सर्वेषामधिकोऽभवत्॥ ८.३६ ॥
राजा त्रिकूटाधिपतिर्महेशेन कृतो महान्॥
सर्वेषां राक्षसानां च परमासनमास्तितः॥ ८.३७ ॥
तपस्विनां परीक्षायै यदृषीणां विहिंसनम्॥
कृतं तेन तदा विप्रा रावणेन तपस्विना॥ ८.३८ ॥
अजेयो हि महाञ्जातो रावणो लोकरावणः॥
सृष्ट्यन्तरं कृतं येन प्रसादाच्छङ्करस्य च॥ ८.३९ ॥
लोकपाला जितास्तेन प्रतापेन तपस्विना॥
ब्रह्मापि विजितो येन तपसा परमेण हि॥ ८.४० ॥
अमृतांशुकरो भूत्वा जितो येन शशी द्विजाः॥
दाहकत्वाज्जितो वह्निरीशः कैलासतोलनात्॥ ८.४१ ॥
ऐश्वर्येण जितश्चेन्द्रो विष्णुः सर्वगतस्तथा॥
लिङ्गार्चनप्रसादेन त्रैलोक्यं च वशीकृतम्॥ ८.४२ ॥
तदा सर्वे सुरगणा ब्रह्मविष्णुपुरोगमाः॥
मेरुपृष्ठं समासाद्य सुमन्त्रं चक्रिरे तदा॥ ८.४३ ॥
पीडिताः स्मो रावणेन तपसा दुष्करेण वै॥
गोकर्णाख्ये गिरौ देवाः श्रूयतां परमाद्भुतम्॥ ८.४४ ॥
साक्षाल्लिङ्गार्चनं येन कृतमस्ति महात्मना॥
ज्ञानज्ञेयं ज्ञानगम्यं यद्यत्परममद्भुतम्॥
तत्कृतं रावणेनैव सर्वेषां दुरतिक्रमम्॥ ८.४५ ॥
वैराग्यं परमास्थाय औदार्यं च ततोऽधिकम्॥
तेनैव ममता त्यक्ता रावणेन महात्मना॥ ८.४६ ॥
संवत्सरसहस्राच्च स्वशिरो हि महाभुजः॥
कृत्त्वा करेण लिङ्गस्य पूजनार्थं समर्पयत्॥ ८.४७ ॥
रावणस्य कबन्धं च तदग्रे च समीपतः॥
योगधारणया युक्तं परमेण समाधिना॥ ८.४८ ॥
लिङ्गे लयं समाधाय कयापि कलया स्थितम्॥
अन्यच्छिरोविवृश्च्यैवं तेनापि शिवपूजनम्॥
कृतं नैवान्यमुनिना तथा चैवापरेणहि॥ ८.४९ ॥
एवं शिरांस्येव बहूनि तेन समर्पितान्येव शिवार्चनार्थे॥
भूत्वा कबन्धो हि पुनः पुनश्च शिवोऽसौ वरदो बभूव॥ ८.५० ॥
मया विनासुरस्तत्र पिण्डीभूतेन वै पुरा॥
वरान्वरय पौलस्त्य यथेष्टं तान्ददाम्यहम्॥ ८.५१ ॥
रावणेन तदा चोक्तः शिवः परममङ्गलः॥
यदि प्रसन्नो भगवन्देयो मे वर उत्तमः॥ ८.५२ ॥
न कामयेऽन्यं च वरमाश्रये त्वत्पदांबुजम्॥
यथा तथा प्रदातव्यं यद्यस्ति च कृपा मयि॥ ८.५३ ॥
तदा सदाशिवेनोक्तो रावणो लोकरावणः॥
मत्प्रसादाच्च सर्वं त्वं प्राप्स्यसे मनसेप्सितम्॥ ८.५४ ॥
एवं प्राप्तं शिवात्सर्वं रावणेन सुरेश्वराः॥
तस्मात्सर्वैर्भवद्भिश्च तपसा परमेण हि॥ ८.५५ ॥
विजेतव्यो रावणोयमिति मे मनसि स्थितम्॥
्च्युतस्य वचः श्रुत्वा ब्रह्माद्या देवतागणाः॥ ८.५६ ॥
चिन्तामापेदिरे सर्वे चिरं ते विषयान्विताः॥
ब्रह्मापि चेन्द्रियग्रस्तः सुता रमितुमुद्यतः॥ ८.५७ ॥
इन्द्रो हि जारभावाच्च चन्द्रो हि गुरुतल्पगः॥
यमः कदर्यभावाच्च चञ्चलत्वात्सदागतिः॥ ८.५८ ॥
पावकः सर्वभक्षित्वात्तथान्ये देवतागणाः॥
अशक्ता रावणं जेतुं तपसा च विजृंभितम्॥ ८.५९ ॥
शैलादो हि महातेजा गणश्रेष्ठः पुरातनः॥
बुद्धि मान्नीतिनिपुणो महाबलपराक्रमी॥ ८.६० ॥
शिवप्रियो रुद्ररूपी महात्मा ह्युवाच सर्वानथ चेन्द्रमुख्यान्॥
कस्माद्यूयं संभ्रमादागताश्च एतत्सर्वं कथ्यतां विस्तरेण॥ ८.६१ ॥
नन्दिना च तदा सर्वे पृष्टाः प्रोचुस्त्वरान्विताः॥ ८.६२ ॥
॥देवा ऊचुः॥
रावणेन वयं सर्वे निर्जिता मुनिभिः सह॥
प्रसादयितुमायाताः शिवं लोकेश्वरेश्वरम्॥ ८.६३ ॥
प्रहस्य भगवान्नन्दी ब्रह्माणं वै ह्युवाच ह॥
क्व यूयं क्व शिवः शंभुस्तपसा परमेण हि॥
द्रष्टव्यो हृदि मध्यस्थः सोऽद्य द्रष्टुं न पार्यते॥ ८.६४ ॥
यावद्भावा ह्यनेकाश्च इन्द्रियार्थास्तथैव च॥
यावच्च ममताभावस्तावदीशो हि दुर्लभः॥ ८.६५ ॥
जितेन्द्रियाणां शान्तानां तन्निष्ठानां महात्मनाम्॥
सुलभो लिङ्गरूपी स्याद्भवतां हि सुदुर्लभः॥ ८.६६ ॥
तदा ब्रह्मादयो देवा ऋषयश्च विपश्चितः॥
प्रणम्य नन्दिनं प्राहुः कस्मात्त्वं वानराननः॥
तत्सर्वं कथयान्यं च रावणस्य तपोबलम्॥ ८.६७ ॥
॥नन्दीश्वर उवाच॥
कुबेरोऽधिकृतस्तेन शङ्करेण महात्मना॥
धनानामादिपत्ये च तं द्रष्टुं रावणोऽत्र वै॥ ८.६८ ॥
आगच्छत्त्वरया युक्तः समारुह्य स्ववाहनम्॥
मां दृष्ट्वा चाब्रवीत्क्रुद्धः कुबेरो ह्यत्र आगतः॥ ८.६९ ॥
त्वया दृष्टोऽथ वात्रासौ कथ्यतामविलंबितम्॥
किं कार्यं धनदेनाद्य इति पृष्टो मया हि सः॥ ८.७० ॥
तदोवाच महातेजा रावणो लोकरावणः॥
मय्यश्रद्धान्वितो भूत्वा विषयात्मा सुदुर्मदः॥ ८.७१ ॥
शिक्षापयितुमारब्धो मैवं कार्यमिति प्रभो॥
यथाहं च श्रिया युक्त आढ्योऽहं बलवानहम्॥
तथा त्वं भव रे मूढ मा मूढत्वमुपार्जय॥ ८.७२ ॥
अहं मूढः कृतस्तेन कुबेरेण महात्मना॥
मया निराकृतो रोषात्तपस्तेपे स गुह्यकः॥ ८.७३ ॥
कुबेरः स हि नन्दिन्किमागतस्तव मन्दिरम्॥
दीयतां च कुबेरोद्य नात्र कार्या विचारणा॥ ८.७४ ॥
रावणस्य वचः श्रुत्वा ह्यवोचं त्वरितोऽप्यहम्॥
लिङ्गकोसि महाभाग त्वमहं च तथाविधः॥ ८.७५ ॥
उभयोः समनां ज्ञात्वा वृथा जल्पसि दुर्मते॥
यथोक्तः स त्ववादीन्मां वदनार्थे बलोद्धतः॥ ८.७६ ॥
यथा भवद्भिः पृष्टोऽहं वदनार्थे महात्मभिः॥
पुरावृत्तं मया प्रोक्तं शिवार्चनविधेः फलम्॥
शिवेन दत्तं सालूप्यं न गृहीतं मया तदा॥ ८.७७ ॥
याचितं च मया शंभोर्वदनं वानरस्य च॥
शिवेन कृपया दत्तं मम कारुण्यशालिना॥ ८.७८ ॥
निराभिमानिनो ये च निर्दभा निष्परिग्रहाः॥
शंभोः प्रियास्ते विज्ञेया ह्यन्ये शिववबहिष्कृताः॥ ८.७९ ॥
तथावदन्मया सार्द्धं रावणस्तपसो बलात्॥
मया च याचितान्येव दश वक्त्राणि धीमता॥ ८.८० ॥
उपहासकरं वाक्यं पौलस्त्यस्य तदा सुराः॥
मया तदा हि शप्तोऽसौ रावणो लोकरावणः॥ ८.८१ ॥
ईदृशान्येव वक्त्राणि येषां वै संभवन्ति हि॥
तैः समेतो यदा कोऽपि नरवर्यो महातपाः॥
मां पुरस्कृत्य सहसा हनिष्यति न संशयः॥ ८.८२ ॥
एवं शप्तो मया ब्रह्मन्रावणो लोकरावणः॥
अर्चितं केवलं लिङ्गं विना तेन महात्मना॥ ८.८३ ॥
पीठिकारूपसंस्थेन विना तेन सुरोत्तमाः॥
विष्णुना हि महाभागास्तस्मात्सर्वं विधास्यति॥ ८.८४ ॥
देवदेवो महादेवो विष्णुरूपी महेश्वरः॥
सर्वे यूयं प्रार्थयन्तु विष्णुं सर्वगुहाशयम्॥ ८.८५ ॥
अहं हि सर्वदेवानां पुरोवर्ती भवाम्यतः॥
ते सर्वे नन्दिनो वाक्यं श्रुत्वा मुदितमानसाः॥
वैकुण्ठमागता गीर्भिर्विष्णुं स्तोतुं प्रचक्रिरे॥ ८.८६ ॥
॥देवा ऊचुः॥
नमो भगवते तुभ्यं देवदेव जगत्पते॥
त्वदाधारमिदं सर्वं जगदेतच्चराचरम्॥ ८.८७ ॥
एतल्लिङ्गं त्वया विष्णो धृतं वै पिण्डिरूपिणा॥
महाविष्णुस्वरूपेण घातितौ मधुकैटभौ॥ ८.८८ ॥
तथा कमठरूपेण धृतो वै मन्दराचलः॥
वराहरूपमास्थाय हिरण्याक्षो हतस्त्वया॥ ८.८९ ॥
हिरण्यकशिपुर्दैत्यो हतो नृहरिरूपिणा॥
त्वया चैव बलिर्बद्धो दैत्यो वामनरूपिणा॥ ८.९० ॥
भृगूणामन्वये भूत्वा कृतवीर्यात्मजो हतः॥
इतोप्यस्मान्महाविष्णो तथैव परिपालय॥ ८.९१ ॥
रावमस्य भयादस्मात्त्रातुं भूयोर्हसि त्वरम्॥ ८.९२ ॥
एवं संप्रार्थितो देवैर्भगवान्भूतभावनः॥
उवाच च सुरान्सर्वान्वासुदेवो जगन्मयः॥ ८.९३ ॥
हे देवाः श्रूयतां वाक्यं प्रस्तावसदृशं महत्॥
शैलादिं च पुरस्कृत्य सर्वे यूयं त्वरान्विताः॥
अवतारान्प्रकुर्वन्तु वानरीं तनुमाश्रिताः॥ ८.९४ ॥
अहं हि मानुषो भूत्वा ह्यज्ञानेन समावृतः॥
संभविष्याम्ययोध्यायं गृहे दशरथस्य च॥
ब्रह्मविद्यासहायोस्मि भवतां कार्यसिद्धये॥ ८.९५ ॥
जनकस्य गृहे साक्षाद्ब्रह्मविद्या जनिष्यति॥
भक्तो हि रावणः साक्षाच्छिवध्यानपरायणः॥ ८.९६ ॥
तपसा महता युक्तो ब्रह्मविद्यां यदेच्छति॥
तदा सुसाध्यो भवति पुरुषो धर्मनिर्जितः॥ ८.९७ ॥
एवं संभाष्य भगवान्विष्णुः परममङ्गलः॥
वाली चेन्द्रांशसंभूतः सुग्रीवों शुमतः सुतः॥ ८.९८ ॥
तथा ब्रह्मांशसंभूतो जाम्बवान्नृक्षकुञ्जरः॥
शिलादतनयो नन्दी शिवस्यानुचरः प्रियः॥ ८.९९ ॥
यो वै चैकादशो रुद्रो हनूमान्स महाकपिः॥
अवतीर्णः सहायार्थं विष्णोरमिततेजसः॥ ८.१०० ॥
मैन्दादयोऽथ कपयस्ते सर्वे सुरसत्तमाः॥
एवं सर्वे सुरगणा अवतेरुर्यथा तथम्॥ ८.१०१ ॥
तथैव विष्णुरुत्पन्नः कौशल्यानन्दवर्द्धनः॥
विश्वस्य रमणाच्चैव राम इत्युच्यते बुधैः॥ ८.१०२ ॥
शेषोपि भक्त्या विष्णोश्च तपसाऽवातरद्भुवि॥ ८.१०३ ॥
दोर्दण्डावपि विष्णोश्च अवतीर्णौ प्रतापिनौ॥
शत्रुघ्नभरताख्यौ च विख्यातौ भुवनत्रये॥ ८.१०४ ॥
मिथिलाधिपतेः कन्या या उक्ता ब्रह्मवादिभिः॥
सा ब्रह्मविद्यावतरत्सुराणां कार्य्यसिद्धये॥
सीता जाता लाङ्गलस्य इयं भूमिविकर्षणात्॥ ८.१०५ ॥
तस्मात्सीतेति विख्याता विद्या सान्वीक्षिकी तदा॥
मिथिलायां समुत्पन्ना मैथितीत्यभिधीयते॥ ८.१०६ ॥
जनकस्य कुले जाता विश्रुता जनकात्मजा॥
ख्याता वेदवती पूर्वं ब्रह्मविद्याघनाशिनी॥ ८.१०७ ॥
सा दत्ता जनकेनैव विष्णवे परमात्मने॥ ८.१०८ ॥
तयाथ विद्यया सार्द्धं देवदेवो जगत्पतिः॥
उग्रे तपसि लीनोऽसौ विष्णुः परमदुष्करम्॥ ८.१०९ ॥
रावणं जेतुकामो वै रामो राजीवलोचनः॥
अरण्यवासमकरोद्देवानां कार्यसिद्धये॥ ८.११० ॥
शेषावतारोऽपि महांस्तपः परमदुष्करम्॥
तताप परया शक्त्या देवानां कार्यसिद्धये॥ ८.१११ ॥
शत्रुघ्नो भरतश्चैव तेपतुः परमं तपः॥ ८.११२ ॥
ततोऽसौ तपसा युक्तः सार्द्धं तैर्देवतागणैः॥
सगणं रावणं रामः षड्भिर्मासैरजीहनत्॥
विष्णुना घातितः शस्त्रैः शिवसारूप्यमाप्तवान्॥ ८.११३ ॥
सगमः स पुनः सद्यो बन्धुभिः सह सुव्रताः॥ ८.११४ ॥
शिवप्रसादात्सकलं द्वैताद्वैतमवाप ह॥
द्वैताद्वैतविवेकार्थमृपयोप्यत्र मोहिताः॥
तत्सर्वं प्राप्नुवन्तीह शिवार्चनरता नराः॥ ८.११५ ॥
येऽर्चयन्ति शिवं नित्यं लिङ्गरूपिणमेव च॥
स्त्रियो वाप्यथ वा शूद्राः श्वपचा ह्यन्त्यवासिनः॥
तं शिवं प्राप्नुवन्त्येव सर्वदुःखोपनाशनम्॥ ८.११६ ॥
पशवोऽपि परं याताः किं पुनर्मानुषादयः॥ ८.११७ ॥
ये द्विजा ब्रह्मचर्येण तपः परममास्थिताः॥
वर्षैरनेकैर्यज्ञानां तेऽपि स्वर्गपरा भवन्॥ ८.११८ ॥
ज्योतिष्टोमो वाजपेयो ह्यतिरात्रादयो ह्यमी॥
यज्ञाः स्वर्गं प्रयच्छन्ति सत्त्रीणां नात्र संशयः॥ ८.११९ ॥
तत्र स्वर्गसुखं भुक्त्वा पुण्यक्षयकरं महत्॥
पुण्यक्षयेऽपि यज्वानो मर्त्यलोकं पतन्ति वै॥ ८.१२० ॥
पतितानां च संसारे दैवाद्बुद्धिः प्रजायते॥
गुणत्रयमयी विप्रास्तासुतास्त्विह योनिषु॥ ८.१२१ ॥
यथा सत्त्वं संभवति सत्त्वयुक्तभवं नराः॥
राजसाश्च तथा ज्ञेयास्ता मसाश्चैव ते द्विजाः॥ ८.१२२ ॥
एवं संसारचक्रेऽस्मिन्भ्रमिता बहवो जनाः॥
यदृच्छया दैवगत्या शिवं संसेवते नरः॥ ८.१२३ ॥
शिवध्यानपराणां च नराणां यतचेतसाम्॥
मायानिरसनं सद्यो भविष्यति न चान्यथा॥ ८.१२४ ॥
मायानिरसनात्सद्यो नश्यत्येव गुणत्रयम्॥
यदा गुणत्रयातीतो भवतीति स मुक्तिभाक्॥ ८.१२५ ॥
तस्माल्लिङ्गार्चनं भाव्यं सर्वेषामपि देहिनाम्॥
लिङ्गरूपी शिवो भूत्वा त्रायते सञ्चराचरम्॥ ८.१२६ ॥
पुरा भवद्भिः पृष्टोऽहं लिङ्गरूपी कथं शिवः॥
तत्सर्वं कथितं विप्रा याथातथ्येन संप्रति॥ ८.१२७ ॥
कथं गरं भक्षितवाञ्छिवो लोकमहेश्वरः॥
तत्सर्वं श्रूयतां विप्रा यतावत्कथयामि वः॥ ८.१२८ ॥
इति श्रीस्कान्दे महापुराण एकाशीतिसाहास्र्यां संहितायां प्रथमे माहेश्वरखण्डे केदारखण्डे शिवशास्त्रे शिवलिङ्गार्चनमाहात्म्यकथने श्रीरामावतारकथावर्णनन्नामाष्टमोऽध्यायः॥ ८ ॥ छ ॥


===

https://sa.wikisource.org/wiki/स्कन्दपुराणम्/खण्डः_३_(ब्रह्मखण्डः)/सेतुखण्डः/अध्यायः_०२
https://www.wisdomlib.org/hinduism/book/the-skanda-purana/d/doc423570.html

॥ऋषय ऊचुः॥
कथं सूत महाभाग रामेणाक्लिष्टकर्मणा ।।
सेतुर्बद्धो नदीनाथे ह्यगाधे वरुणालये ।।१।।
सेतौ च कति तीर्थानि गन्धमादनपर्वते ।।
एतन्नः श्रद्दधानानां ब्रूहि पौराणिकोत्तम ।। २ ।।
॥श्रीसूत उवाच॥
रामेण हि यथासेतुर्निबद्धो वरुणालये ।।
तदहं संप्रवक्ष्यामि युष्माकं मुनिपुङ्गवाः ।। ३ ।।
आज्ञया हि पितू रामो न्यवसद्दण्डकानने।।
सीतालक्ष्मणसंयुक्तः पञ्चवट्यां समाहितः।। ४ ।।
तस्मिन्निव सतस्तस्य राघवस्य महात्मनः ।।
रावणेन हृता भार्या मारीचच्छद्मना द्विजाः ।।५ ।।
मार्गमाणो वने भार्यां रामो दशरथात्मजः।।
पंपातीरे जगा मासौ शोकमोहसमन्वितः ।।६।।
दृष्टवान्वानरं तत्र कञ्चिद्दशरथात्मजः ।।
वानरेणाथ पृष्टोऽयं को भवानिति राघवः ।। ७ ।।
आदितः स्वस्य वृत्तान्त्तं तस्मै प्रोवाच तत्त्वतः ।।
अथ राघवसंपृष्टो वानरः को भवानिति ।। ८ ।।
सोपि विज्ञापयामास राघवाय महात्मने ।।
अहं सुग्रीवसचिवो हनूमा न्नाम वानरः ।। ९ ।।
तेन च प्रेरितोऽभ्यागां युवाभ्यां सख्यमिच्छता ।।
आगच्छतं तद्भद्रं वां सुग्रीवान्तिकमाशु वै ।। १० ।।
तथास्त्विति स रामो पि तेन साकं मुनीश्वराः ।।
सुग्रीवान्तिकमागप्य सख्यं चक्रेऽग्निसाक्षिकम् ।। ११ ।।
प्रतिजज्ञेऽथ रामोऽपि तस्मै वालिवधं प्रति ।।
सुग्रीवश्चापि वै देह्याः पुनरानयनं द्विजाः ।। १२ ।।
इत्येवं समयं कृत्वा विश्वास्य च परस्परम् ।।
मुदा परमया युक्तौ नरेश्वरकपीश्वरौ ।।१३।।
आसाते ब्राह्मणश्रेष्ठा ऋष्यमूकगिरौ तथा ।।
सुग्रीवप्रत्ययार्थं च दुन्दुभेः कायमाशु वै।। १४ ।।
पादाङ्गुष्ठेन चिक्षेप राघवो बहुयोजनम् ।।
सप्तताला विनिर्भिन्ना राघवेण महात्मना ।। १५ ।।
ततः प्रीतमना वीरः सुग्रीवो राममब्रवीत् ।।
इन्द्रादिदेवताभ्योऽपि नास्ति राघव मे भयम्।। १६ ।।
भवान्मित्रं मया लब्धो यस्मादति पराक्रमः ।।
अहं लङ्केश्वरं हत्वा भार्यामानयितास्मि ते ।। १७ ।।
ततः सुग्रीवसहितो रामचन्द्रो महाबलः ।।
सलक्ष्मणो ययौ तूर्णं किष्किन्धां वालिपालिताम् ।। १८ ।।
ततो जगर्ज सुग्रीवो वाल्यागमनकाङ्क्षया ।।
अमृष्यमाणो वाली च गर्जितं स्वानुजस्य वै।। ।। १९ ।।
अन्तःपुराद्विनिष्क्रम्य युयुधेऽवरजेन सः ।।
वालिमुष्टिप्रहारेण ताडितो भृशविह्वलः ।। २० ।।
सुग्रीवो निर्गतस्तूर्णं यत्र रामो महाबलः ।।
ततो रामो महाबाहुस्सुग्रीवस्य शिरोधरे ।। २१ ।।
लतामाबध्य चिह्नं तु युद्धायाचोदयत्तदा ।।
गर्जितेन समाहूय सुग्रीवो वालिनं पुनः ।। २२ ।।
रामप्रेरणया तेन बाहुयुद्धमथाकरोत् ।।
ततो वालिनमाजघ्ने शरेणैकेन राघवः ।। २३ ।।
हते वालिनि सुग्रीवः किष्किन्धां प्रत्यपद्यत ।।
ततो वर्षास्वतीतासु सुग्रीवो वानराधिपः ।। २४ ।।
सीतामानयितुं तूर्णं वानराणां महाचमूम् ।।
समादाय समागच्छदन्तिकं नृपपुत्रयोः ।। २५ ।।
प्रस्थापयामास कपीन्सीतान्वेषणकाङ्क्षया ।।
विदितायां तु वैदेह्या लङ्कायां वायुसूनुना ।। २६ ।।
दत्ते चूडामणौ चापि राघवो हर्षशोकवान्।।
सुग्रीवेणानुजेनापि वायुपुत्रेण धीमता ।। २७ ।।
तथान्यैः कपिभिश्चैव जांबवन्नलमुख्यकैः ।।
अन्वीयमानो रामोऽसौ मुहूर्तेऽभिजिति द्विजाः ।।२८।।
विलङ्घ्य विविधा न्देशान्महेन्द्रं पर्वतं ययौ ।।
चक्रतीर्थं ततो गत्वा निवासमकरोत्तदा ।। २९ ।।
तत्रैव तु स धर्मात्मा समागच्छद्विभीषणः ।।
भ्राता वै राक्षसेन्द्रस्य चतुर्भिः सचिवैः सह ।। ३० ।।
प्रतिजग्राह रामस्तं स्वागतेन महात्मना ।।
सुग्रीवस्य तु शङ्काऽभूत्प्रणिधिः स्यादयं त्विति ।। ३१ ।।
राघवस्तस्य चेष्टाभिः सम्यक्स्वचरितैर्हितैः ।।
अदुष्टमेनं दृष्ट्वैव तत एनमपूजयत् ।। ३२ ।।
सर्वराक्षसराज्ये तमभ्यषिञ्चद्विभीषणम् ।।
चक्रे च मन्त्रिप्रवरं सदृशं रविसूनुना ।। ३३ ।।
चक्रतीर्थं समासाद्य निवसद्रघुनन्दनः ।।
चिन्तयन्राघवः श्रीमान्सुग्रीवादीनभाषत ।। ३४ ।।
मध्ये वानरमु ख्यानां प्राप्तकालमिदं वचः ।।
उपायः को नु भवतामेतत्सागरलङ्घने ।। ३५ ।।
इयं च महती सेना सागरश्चापि दुस्तरः ।।
अंभोराशिरयं नीलश्चञ्चलोर्म्मिसमाकुलः ।। ३६ ।।
उद्यन्मत्स्यो महानक्रशङ्खशुक्तिसमाकुलः ।।
क्वचिदौर्वानलाक्रान्तः फेनवानतिभीषणः ।। ।। ३७ ।।
प्रकृष्टपवनाकृष्टनीलमेघसमन्वितः ।।
प्रलयांभोधरारावः सारवाननिलोद्धतः ।। ३८ ।।
कथं सागरमक्षोभ्यं तरामो वरुणा लयम् ।।
सैन्यैः परिवृताः सर्वे वानराणां महौजसाम् ।। ३९ ।।
उपायैरधिगच्छामो यथा नदनदीपतिम् ।।
कथं तरामः सहसा ससैन्या वरुणालयम् ।। ४० ।।
शतयोजनमायातं मनसापि दुरासदम् ।।
अतो नु विघ्ना बहवः कथं प्राप्या च मैथिली ।। ४१ ।।
कष्टात्कष्टतरं प्राप्ता वयमद्य निराश्रयाः ।।
महाजले महावाते समुद्रे हि निराश्रये ।। ४२ ।।
उपायं कं विधास्यामस्तरणार्थं वनौकसाम् ।।
राज्याद्भ्रष्टो वनं प्राप्तो हृता सीता मृतः पिता ।। ४३ ।।
इतोऽपि दुःसहं दुःखं यत्सागरविलङ्घनम् ।।
धिग्धिग्गर्जितमंभोधे धिगेतां वारिराशिताम् ।। ४४ ।।
कथं तद्वचनं मिथ्या महर्षेः कुम्भजन्मनः ।।
हत्वा त्वं रावणं पापं पवित्रे गन्धमादने ।।
पापोपशमनायाशु गच्छस्वेति यदीरितम् ।।४५।।
॥श्रीसूत उवाच॥
इति रामवचः श्रुत्वा सुग्रीवप्रमुखास्तदा ।। ४६ ।।
ऊचुः प्राञ्जलयः संर्मे राघवं तं महाबलम् ।।
नौभिरेनं तरिष्यामः प्लवैश्च विविधैरिति ।। ४७ ।।
मध्ये वानरकोटीनां तदोवाच विभीषणः ।।
समुद्रं राघवो राजा शरणं गन्तुमर्हति ।। ४८ ।।
खनितः सागरैरेष समुद्रो वरुणालयः ।।
कर्तुमर्हति रामस्य तज्ज्ञातेः कार्यमंबुधिः ।। ४९ ।।
विभीषणेनैवमुक्तो राक्षसेन विपश्चिता ।।
सान्त्वयन्राघवः सर्वान्वानरानिदमब्रवीत् ।। ५० ।।
शतयोजन विस्तारमशक्ताः सर्ववानराः ।।
तर्तुं प्लवोडुपैरेनं समुद्रमतिभीषणम् ।। ५१ ।।
नावो न सन्ति सेनाया बह्व्या वानरपुङ्गवाः ।।
वणिजामुपघातं च कथमस्मद्विधश्चरेत् ।। ५२ ।।
विस्तीर्णं चैव नः सैन्यं हन्याच्छिद्रेषु वा परः ।।
प्लवोडुपप्रतारोऽतो नैवात्र मम रोचते ।। ५३ ।।
विभीषेणोक्तमे वेदं मोदते मम वानराः ।।
अहं त्विमं जलनिधिमुपास्ये मार्गसिद्धये ।। ५४ ।।
नो चेद्दर्शयिता मार्गं धक्ष्याम्येनमहं तदा ।।
महास्त्रैरप्रतिहतैरत्यग्निपवनोज्ज्वलैः ।। ५५ ।।
इत्युक्त्वा सहसौमित्रिरुपस्पृश्याथ राघवः ।।
प्रतिशिश्ये जलनिधिं विधिवत्कुशसंस्तरे ।। ५६ ।।
तदा रामः कुशा स्तीर्णे तीरे नदनदीपतेः ।।
संविवेश महाबाहुर्वेद्यामिव हुताशनः ।। ५७ ।।
शेषभोगनिभं बाहुमुपधाय रघूद्वहः ।।
दक्षिणो दक्षिणं बाहुमुपास्ते मकरालयम् ।। ५८ ।।
तस्य रामस्य सुप्तस्य कुशास्तीर्णे महीतले ।।
नियमादप्रमत्तस्य निशास्तिस्रोऽतिचक्रमुः ।। ५९ ।।
स त्रिरात्रोषितस्तत्र नयज्ञो धर्मतत्परः ।।
उपास्तेस्म तदा रामः सागरं मार्गसिद्धये ।। ६० ।।
न च दर्शयते मन्दस्तदा रामस्य सागरः ।।
प्रयतेनापि रामेण यथार्हमपि पूजितः ।। ६१ ।।
तथापि सागरो रामं न दर्शयति चात्मनः ।।
समुद्राय ततः क्रुद्धो रामो रक्तान्तलोचनः ।। ६२ ।।
समीपवर्तिनं चेदं लक्ष्मणं प्रत्यभाषत ।।
अद्य मद्बाणनिर्भिन्नैर्मकरैर्वरुणालयम् ।। ६३ ।।
निरुद्धतोयं सौमित्रे करिष्यामि क्षणादहम् ।।
सशङ्खशुक्ताजालं हि समीनमकरं शनैः ।। ६४ ।।
अद्य बाणैरमोघास्त्रैर्वारिधिं परिशोषये ।।
क्षमया हि समायुक्तं मामयं मकरालयः ।। ६९ ।।
असमर्थं विजानाति धिक्क्षमामीदृशे जने ।।
न दर्शयति साम्ना मे सागरो रूपमात्मनः ।। ६६ ।।
चापमानय सौमित्रे शरांश्चाशीविषोपमान्।।
सागरं शोषयिष्यामि पद्भ्यां यान्तु प्लवङ्गमाः ।। ६७ ।।
एनं लङ्घितमर्यादं सहस्रोर्मिसमाकुलम् ।।
निर्मर्यादं करिष्यामि सायकैर्वरुणालयम् ।। ६८ ।।
महार्णवं शोषयिष्ये महादानवसङ्कुलम् ।।
महामकरनक्राढ्यं महावीचिसमाकुलम् ।। ६९ ।।
एवमुक्त्वा धनुष्पाणिः क्रोधपर्याकुलेक्षणः ।।
रामो बभूव दुर्धर्षस्त्रिपुरघ्नो यथा शिवः ।। ७० ।।
आकृष्य चापं कोपेन कम्पयित्वा शरैर्जगत् ।।
मुमोच विशिखानुग्रांस्त्रिपुरेषु यथा भवः ।। ७१ ।।
दीप्ता बाणाश्च ये घोरा भासयन्तो दिशो दश ।।
प्राविशन्वारिधेस्तोयं दृप्तदानवसङ्कुलम् ।।७२।।
समुद्रस्तु ततो भीतो वेपमानः कृताञ्जलिः ।।
अनन्यशरणो विप्राः पाता लात्स्वयमुत्थितः।।७३।।
शरणं राघवं भेजे कैवल्यपदकारणम् ।।
तुष्टाव राघवं विप्रा भूत्वा शब्दैर्मनोरमैः ।। ७४ ।।
॥समुद्र उवाच॥
नमामि ते राघव पादपङ्कजं सीतापते सौख्यद पादसेवनात् ।।
नमामि ते गौतमदारमोक्षजं श्रीपादरेणुं सुरवृन्दसेव्यम्।।७६।।
सुन्दप्रियादेहविदारिणे नमो नमोस्तु ते कौशिकयागरक्षिणे ।।
नमो महादेवशरासभेदिने नमो नमो राक्षससङ्घनाशिने ।। ७६ ।।
रामराम नमस्यामि भक्तानामिष्टदायिनम् ।।
अवतीर्णो रघुकुले देवकार्यचिकीर्षया ।। ७७ ।।
नारायणमनाद्यन्तं मोक्षदं शिवमच्युतम् ।।
रामराम महाबाहो रक्ष मां शरणागतम् ।। ७८ ।।
कोपं संहर राजेन्द्र क्षमस्व करुणालय ।।
भूमिर्वातो वियच्चापो ज्योतींषि च रघूद्वह ।। ७९ ।।
यत्स्वभावानि सृष्टानि ब्रह्मणा परमेष्ठिना ।।
वर्तन्ते तत्स्वभा वानि स्वभावो मे ह्यगाधता ।। ८० ।।
विकारस्तु भवेद्गाध एतत्सत्यं वदाम्यहम् ।।
लोभात्कामाद्भयाद्वापि रागाद्वापि रघूद्वह ।। ८१ ।।
न वंशजं गुणं हातुमुत्सहेयं कथञ्चन ।।
तत्करिष्ये च साहाय्यं सेनायास्तरणे तव ।। ८२ ।।
इत्युक्तवन्तं जलधिं रामोऽवादीन्नदीपतिम् ।।
ससैन्योऽहं गमि ष्यामि लङ्कां रावणपालिताम् ।। ८३ ।।
तच्छोषमुपयाहि त्वं तरणार्थं ममाधुना ।।
इत्युक्तस्तं पुनः प्राह राघवं वरुणालयः ।। ८४ ।।
शृणुष्वाव हितो राम श्रुत्वा कर्तव्यमाचर ।।
यद्याज्ञया ते शुष्यामि ससैन्यस्य यियासतः ।।८५।।
अन्येऽप्याज्ञापयिष्यन्ति मामेवं धनुषो बलात् ।।
उपायमन्यं वक्ष्यामि तरणार्थं बलस्य ते ।। ८१६ ।।
अस्ति ह्यत्र नलोनाम वानरः शिल्पिसंमतः ।।
त्वष्टुः काकुत्स्थ तनयो बलवान्विश्वकर्मणः ।। ८७ ।।
स यत्काष्ठं तृणं वापि शिलां वा क्षेप्स्यते मयि ।।
सर्वं तद्धारयिष्यामि स ते सेतुर्भविष्यति ।। ८८ ।।
सेतुना तेन गच्छ त्वं लङ्कां रावणपालि ताम् ।।
उक्त्वेत्यन्तर्हिते तस्मिन्रामो नलमुवाच ह ।। ८९ ।।
कुरु सेतुं समुद्रे त्वं शक्तो ह्यसि महामते ।।
तदाऽब्रवीन्नलो वाक्यं रामं धर्मभृतां वरम् ।। ९० ।।
अहं सेतुं विधास्यामि ह्यगाधे वरुणालये ।।
पित्रा दत्तवरश्चाहं सामर्थ्ये चापि तत्समः ।। ९१ ।।
मातुर्मम वरो दत्तो मन्दरे विश्वक र्मणा ।।
शिल्पकर्मणि मत्तुल्यो भविता ते सुतस्त्विति ।। ९२ ।।
पुत्रोऽहमौरसस्तस्य तुल्यो वै विश्वकर्मणा ।।
अद्यैव कामं बध्नन्तु सेतुं वानरपुं गवाः ।। ९३ ।।
ततो रामनिसृष्टास्ते वानरा बलवत्तराः ।।
पर्वतान्गिरिशृङ्गाणि लतातृणमहीरुहान् ।। ९४ ।।
समाजह्रुर्महाकाया गरुडानिलरंहसः ।।
नलश्चक्रे महासेर्तुमध्ये नदनदीपतेः ।। ९५ ।।
दशयोजनविस्तीर्णं शतयोजनमायतम् ।।
जानकीरमणो रामः सेतुमेवमकारयत् ।। ९६ ।।।
नलेन वानरेन्द्रेण विश्वकर्मसुतेन वै ।।
तमेवं सेतुमासाद्य रामचन्द्रेण कारितम् ।। ९७ ।।
सर्वे पातकिनो मर्त्या मुच्यन्ते सर्वपातकैः ।।
व्रतदान तपोहोमैर्न तथा तुष्यते शिवः ।। ९८ ।।
सेतुमज्जनमात्रेण यथा तुष्यति शङ्करः ।।
न तुल्यं विद्यते तेजोयथा सौरेण तेजसा ।। ९९ ।।
सेतुस्नानेन च तथा न तुल्यं विद्यते क्वचित् ।।
तत्सेतुमूलं लङ्कायां यत्ररामो यियासया ।। १०० ।।
वानरैः सेतुमारेभे पुण्यं पाप प्रणाशनम् ।।
तद्दर्भशयनं नाम्ना पश्चाल्लोकेषु विश्रुतम् ।। १०१।।
एवमुक्तं मया विप्राः समुद्रे सेतुबन्धनम् ।।
अत्र तीर्थान्यनेकानि सन्ति पुण्यान्यनेकशः ।। १०२ ।।
न सङ्ख्यां नामधेयं वा शेषो गणयितुं क्षमः ।।
किं त्वहं प्रब्रवीम्यद्य तत्र तीर्थानि कानिचित् ।। १०३ ।।
चतुर्विंशति तीर्थानि सन्ति सेतौ प्रधानतः।।
प्रथमं चकतीर्थं स्याद्वेतालवरदं ततः ।। १०४ ।।
ततः पापविनाशार्थं तीर्थं लोकेषु विश्रुतम् ।।
ततः सीतासरः पुण्यं ततो मङ्गलतीर्थकम्।। ।। १०५ ।।
ततः सकलपापघ्नी नाम्ना चामृतवापिका ।।
ब्रह्मकुण्डं ततस्तीर्थं ततः कुण्डं हनूमतः ।। १०६ ।।
आगस्त्यं हि ततस्तीर्थं रामतीर्थ मतः परम् ।।
ततो लक्ष्मणतीर्थं स्याज्जटातीर्थमतः परम् ।। १०७ ।।
ततो लक्ष्म्याः परं तीर्थमग्नितीर्थमतः परम् ।।
चक्रतीर्थं ततः पुण्यं शिवतीर्थमतः परम् ।। १०८ ।।
ततः शङ्खाभिधं तीर्थं ततो यामुनतीर्थकम् ।।
गङ्गातीर्थं ततः पश्चाद्गयातीर्थमनन्तरम् ।। १०९ ।।
ततः स्यात्कोटितीर्थाख्यं साध्यानाममृतं ततः ।।
मानसाख्यं ततस्तीर्थं धनुष्कोटिस्ततः परम् ।। ११० ।।
प्रधानतीर्थान्येतानि महापापहराणि च ।।
कथितानि द्विजश्रेष्ठास्सेतुमध्यगतानि वै ।। १११ ।।
यथा सेतुश्च बद्धोऽभूद्रामेण जलधौ महान् ।।
कथितं तच्च विप्रेन्द्राः पुण्यं पापहारं तथा ।। ११२ ।।
यच्छ्रुत्वा च पठित्वा च मुच्यते मानवो भुवि ।। ११३ ।।
अध्यायमेनं पठते मनुष्यः शृणोति वा भक्तियुतो द्विजेन्द्राः ।।
सो नन्तमाप्नोति जयं परत्र पुनर्भवक्लेशमसौ न गच्छेत् ।। ११४ ।।
इति श्रीस्कान्दे महापुराण एकाशीतिसाहस्र्यां संहितायां तृतीये ब्रह्मखण्डे सेतुमाहात्म्ये सेतुनिर्माणादिवर्णनन्नाम द्वितीयोऽध्यायः ।। २ ।। ।।


===

https://sa.wikisource.org/wiki/स्कन्दपुराणम्/खण्डः_३_(ब्रह्मखण्डः)/सेतुखण्डः/अध्यायः_४४
https://www.wisdomlib.org/hinduism/book/the-skanda-purana/d/doc423612.html

।। ऋषय ऊचुः ।। ।।
सर्ववेदार्थतत्त्वज्ञ पुराणार्णवपारग ।।
व्यासपादांबुजद्वन्द्वनमस्कारहृताशुभ ।। १ ।।
पुराणार्थोपदेशेन सर्वप्राण्युपका रक ।।
त्वया ह्यनुगृहीताः स्म पुराणकथनाद्वयम् ।। २ ।।
अधुना सेतुमाहात्म्यकथनात्सुतरां मुने ।।
वयं कृतार्थाः सञ्जाता व्यासशिष्य महामते ।। ।। ३ ।।
यथा प्रातिष्ठिपल्लिङ्गं रामो दशरथात्मजः ।।
तच्छ्रोतुं वयमिच्छामस्त्वमिदानीं वदस्व नः ।। ४ ।।
।। श्रीसूत उवाच ।। ।।
यदर्थं स्थापितं लिङ्गं गन्धमादनपर्वते ।।
रामचन्द्रेण विप्रेन्द्र तदिदानीं ब्रवीमि वः ।। ५ ।।
हृतभार्यो वनाद्रामो रावणेन बलीयसा ।।
कपिसेनायुतो धीरः ससौमि त्रिर्महाबलः ।।६।।
महेन्द्रं गिरिमासाद्य व्यलोकयत वारिधिम् ।।
तस्मिन्नपारे जलधौ कृत्वा सेतुं रघूद्वहः ।।७।।
तेन गत्वा पुरीं लङ्कां रावणेनाभिरक्षि ताम् ।।
अस्तङ्गते सहस्रांशौ पौर्णमास्यां निशामुखे।।८।।
रामः ससैनिको विप्राः सुवेलगिरिमारुहत् ।।
ततः सौधस्थितं रात्रौ दृष्ट्वा लङ्केश्वरं बली। ।। ।। ९ ।।
सूर्यपुत्रोऽस्य मुकुटं पातयास भूतले ।।
राक्षसो भग्नमुकुटः प्रविवेश गृहोदरम् ।। 3.1.44.१० ।।
गृहं प्रविष्टे लङ्केशे रामः सुग्रीवसंयुतः ।।
सानुजः सेनया सार्द्धमवरुह्य गिरेस्तटात् ।।११।।
सेनां न्यवेशयद्वीरो रामो लङ्कासमीपतः ।।
ततो निवेशमानांस्तान्वानरान्रावणानुगाः ।। १२।।
अभिजग्मुर्महाकायाः सायुधाः सहसैनिकाः ।।
पर्वणः पूतनो जृंभः खरः क्रोधवशो हरिः ।।१३।।
प्रारुजश्चारुजश्चैव प्रहस्तश्चेतरे तथा ।।
ततोऽभिपततां तेषामदृश्यानां दुरात्मनाम् ।। १४ ।।
अन्तर्धानवधं तत्र चकार स्म विभीषणः ।।
ते दृश्यमाना बलिभिर्हरिभिर्दूरपातिभिः ।। १५ ।।
निहताः सर्वतश्चैते न्यपतन्वै गतासवः ।।
अमृष्यमाणः सबलो रावणो निर्ययावथ ।। १६ ।।
व्यूह्य तान्वानरान्सर्वान्न्यवारयत सायकैः ।।
राघवस्त्वथ निर्याय व्यूढानीको दशाननम् ।। १७ ।।
प्रत्ययुध्यत वेगेन द्वन्द्वयुद्धमभूत्तदा ।।
युयुधे लक्ष्मणेनाथ इन्द्रजिद्रावणात्मजः ।। १८ ।।
विरूपाक्षेण सुग्रीवस्तारेयेणापि खर्वटः ।।
पौण्ड्रेण च नलस्तत्र पुटेशः पनसेन च ।। १९ ।।
अन्येपि कपयो वीरा राक्षसैर्द्वन्द्वमेत्य तु ।।
चक्रुर्युद्धं सुतुमुलं भीरूणां भयवर्द्धनम् ।। 3.1.44.२० ।।
अथ रक्षांसि भिन्नानि वानरैर्भीमविक्रमैः ।।
प्रदुद्रुवू रणादाशु लङ्कां रावणपालिताम् ।। २१ ।।
भग्नेषु सर्वसैन्येषु रावणप्रेरितेन वै ।।
पुत्रेणेन्द्रजिता युद्धे नागास्त्रैरतिदारुणैः ।। २२ ।।
विद्धौ दाशरथी विप्रा उभौ तौ रामलक्ष्मणौ ।।
मोचितौ वैनतेयेन गरुडेन महात्मना ।। २३ ।।
तत्र प्रहस्तस्तरसा समभ्येत्य विभीषणम् ।।
गदया ताडयामास विनद्य रणकर्कशः ।। २४ ।।
स तयाभिहतो धीमान्गदया भामिवेगया ।।
नाकंपत महाबाहुर्हिमवानिव सुस्थितः ।। २५ ।।
ततः प्रगृह्य विपुलामष्टघण्टां विभीषणः ।।
अभिमन्त्र्य महाशक्तिं चिक्षे पास्य शिरः प्रति।।२६।।
पतन्त्या स तया वेगाद्राक्षसोऽशनिना यथा।
हृतोत्तमाङ्गो ददृशे वातरुग्ण इव द्रुमः।।२७।।
तं दृष्ट्वा निहतं सङ्ख्ये प्रहस्तं क्षणदाचरम्।।
अभिदुद्राव धूम्राक्षो वेगेन महता कपीन्।।२८।।
कपिसैन्यं समालोक्य विद्रुतं पवनात्मजः।।
धूम्राक्षमाजघानाशु शरेण रणमूर्धनि।।२९।।
धूम्राक्षं निहतं दृष्ट्वा हतशेषा निशाचराः ।।
सर्वं राज्ञे यथावृत्तं रावणाय न्यवेदयन् ।।3.1.44.३०।।
ततः शयानं लङ्केशः कुम्भकर्णमबोधयत् ।।
प्रबुद्धं प्रेषयामास युद्धाय स च रावणः ।। ३१ ।।
आगतं कुम्भकर्णं तं ब्रह्मास्त्रेण तु लक्ष्मणः ।।
जघान समरे क्रुद्धो गतासुर्न्यपतच्च सः ।। ३२ ।।
दूषणस्यानुजौ तत्र वत्रवेगप्रमाथिनौ ।।
हनुमन्नीलनिहतौ रावणप्रतिमौ रणे ।। ३३ ।।
वज्रदंष्ट्रं समवधीद्विश्वकर्मसुतो नलः ।।
अकंपनं च न्यहनत्कुमुदो वानरर्षभः ।। ३४ ।।
षष्ठ्यां पराजितो राजा प्राविशच्च पुरीं ततः ।।
अतिकायो लक्ष्मणेन हतश्च त्रिशिरास्तथा ।। ३५ ।।
सुग्रीवेण हतौ युद्धे देवान्त कनरान्तकौ ।।
हनूमता हतौ युद्धे कुम्भकर्णसुतावुभौ ।। ३६ ।।
विभीषणेन निहतो मकराक्षः खरात्मजः ।।
तत इन्द्रजितं पुत्रं चोदयामास रावणः।।३७।।
इन्द्रजिन्मोहयित्वा तौ भ्रातरौ रामलक्षमणौ।।
घोरैः शरैरङ्गदेन हतवाहो दिवि स्थितः।।३८।।
कुमुदाङ्गदसुग्रीवनलजांबवदादिभिः।।
सहिता वानराः सर्वे न्यपतंस्तेन घातिताः।।३९।।
एवं निहत्य समरे ससैन्यौ रामलक्ष्मणौ।।
अन्तर्दधे तदा व्योम्नि मेघनादो महाबलः ।। 3.1.44.४० ।।
ततो विभीषणो राममिक्ष्वाकुकुलभूषणम् ।।
उवाच प्राञ्जलिर्वाक्यं प्रणम्य च पुनःपुनः ।। ४१ ।।
अयमंभो गृहीत्वा तु राजराजस्य शासनात् ।।
गुह्यकोऽभ्यागतो राम त्वत्सकाशमरिन्दम ।। ४२ ।।
इदमंभः कुबेरस्ते महाराज प्रयच्छति ।।
अन्तर्हितानां भूतानां दर्शनार्थं परं तप ।। ४३ ।।
अनेन स्पृष्टनयनो भूतान्यन्तर्हितान्यपि ।।
भवान्द्रक्ष्यति यस्मै वा भवानेतत्प्रदास्यति ।। ४४ ।।
सोऽपि द्रक्ष्यति भूतानि वियत्त्यन्तर्हितानि वै ।।
तथेति रामस्तद्वारि प्रतिगृह्याथ सत्कृतम् ।। ४५ ।।
चकार नेत्रयोः शौचं लक्ष्मणश्च महाबलः ।
सुग्रीवजांबवन्तौ च हनुमानङ्गदस्तथा ।। ४६ ।।
मैन्दद्विविदनीलाश्च ये चान्ये वानरास्तथा ।।
ते सर्वे रामदत्तेन वारिणा शुद्धचक्षुषः ।। ४७ ।।
आकाशेन्तर्हितं वीरमपश्यन्रावणा त्मजम् ।।
ततस्तमभिदुद्राव सौमित्रिर्दृष्टिगोचरम् ।। ४८ ।।
ततो जघान सङ्कुद्धो लक्ष्मणः कृतलक्षणः ।।
कुवेरप्रेषितजलैः पवित्रीकृतलोचनः ।। ।। ४९ ।।
ततः समभवद्युद्धं लक्ष्मणेन्द्रजितोर्महत् ।।
अतीव चित्रमाश्चर्यं शक्रप्रह्लादयोरिव ।। 3.1.44.५० ।।
ततस्तृतीयदिवसे यत्नेन महता द्विजाः ।।
इन्द्रजिन्निहतो युद्धे लक्ष्मणेन बलीयसा ।। ५१ ।।
ततो मूलबलं सर्वं हतं रामेण धीमता ।।
अथ क्रुद्धो दशग्रीवः प्रियपुत्रे निपातिते ।। ५२ ।।
निर्ययौ रथमास्थाय नगराद्बहुसैनिकः ।।
रावणो जानकीं हन्तुमुद्युक्तो विन्ध्यवारितः ।। ५३ ।।
ततो हर्यश्वयुक्तेन रथेनादित्यवर्चसा ।।
उपतस्थे रणे रामं मातलिः शक्रसारथिः ।। ५४ ।।
ऐन्द्रं रथं समारुह्य रामो धर्मभृतां वरः ।।
शिरांसि राक्षसेन्द्रस्य ब्रह्मास्त्रेणावधीद्रणे ।। ५५ ।।
ततो हतदशग्रीवं रामं दशरथात्मजम् ।।
आशीर्भिर्जययुक्ताभिर्देवाः सर्षिपुरोगमाः ।। ५६ ।।
तुष्टुवुः परिसन्तुष्टाः सिद्धविद्याधरास्तथा ।।
रामं कमलपत्राक्षं पुष्प वर्षेरवाकिरन् ।। ५७ ।।
रामस्तैः सुरसङ्घातैः सहितः सैनिकैर्वृतः ।।
सीतासौमित्रिसहितः समारुह्य च पुष्पकम् ।। ५८ ।।
तथाभिषिच्य राजानं लङ्कायां च विभीषणम् ।।
कपिसेनावृतो रामो गन्धमादनमन्वगात् ।।५९।।
परिशोध्य च वैदेहीं गन्धमादनपर्वते ।।
रामं कमलपत्राक्षं स्थितवानर संवृतम् ।। 3.1.44.६० ।।
हतलङ्केश्वरं वीरं सानुजं सविभीषणम् ।।
सभार्यं देववृन्दैश्च सेवितं मुनिपुङ्गवैः ।। ६१ ।।
मुनयोऽभ्यागता द्रष्टुं दण्डकारण्य वासिनः ।।
अगस्त्यं ते पुरस्कृत्य तुष्टुवुर्मैथिलीपतिम् ।। ६२ ।।
।। मुनय ऊचुः ।। ।।
नमस्ते रामचन्द्राय लोकानुग्रहकारिणे ।।
अरावणं जगत्कर्तुमवतीर्णाय भूतले ।। ६३ ।।
ताटिकादेहसंहर्त्रे गाधिजाध्वररक्षिणे ।।
नमस्ते जितमारीच सुवाहुप्राणहारिणे ।। ६४ ।।
अहल्यामुक्तिसन्दायिपादपङ्कजरेणवे ।।
नमस्ते हरकोदण्डलीलाभञ्जनकारिणे ।। ६५ ।।
नमस्ते मैथिलीपाणिग्रहणोत्सवशालिने ।।
नमस्ते रेणुकापुत्रपराजयविधायिने ।। ६६ ।।
सहलक्ष्मणसीताभ्यां कैकेय्यास्तु वरद्वयात् ।।
सत्यं पितृवचः कर्तुं नमो वनमुपे युषे ।। ६७ ।।
भरतप्रार्थनादत्तपादुकायुगुलाय ते ।।
नमस्ते शरभङ्गस्य स्वर्गप्राप्त्यैकहेतवे ।। ६८ ।।
नमो विराधसंहर्त्रे गृधराजस खाय ते ।।
मायामृगमहाक्रूरमारीचाङ्गविदारिणे ।। ६९ ।।
सीतापहारिलोकेशयुद्धत्यक्तकलेवरम् ।।
जटायुषं तु सन्दह्य तत्कैवल्यप्रदायिने ।। ।। 3.1.44.७० ।।
नमः कबन्धसंहर्त्रे शवरीपूजिताङ्घ्रये ।।
प्राप्तसुग्रीवसख्याय कृतवालिवधाय ते ।। ७१ ।।
नमः कृतवते सेतुं समुद्रे वरुणालये ।।
सर्वराक्षससंहर्त्रे रावणप्राणहारिणे ।। ७२ ।।
संसारांबुधिसन्तारपोतपादांबुजाय ते ।।
नमो भक्तार्तिसंहर्त्रे सच्चिदानन्दरूपिणे ।। ७३ ।।
नमस्ते राम भद्राय जगतामृद्धिहेतवे ।।
रामादिपुण्यनामानि जपतां पापहारिणे ।। ७४ ।।
नमस्ते सर्वलोकानां सृष्टिस्थित्यन्तकारिणे ।।
नमस्ते करुणामूर्ते भक्तरक्षणदीक्षित ।। ७८५ ।।
ससीताय नमस्तुभ्यं विभीषणसुखप्रद ।।
लङ्केश्वरवधाद्राम पालितं हि जगत्त्वया ।। ७६ ।।
रक्षरक्ष जगन्नाथ पाह्य स्माञ्जानकीपते ।।
स्तुत्वैवं मुनयः सर्वे तूष्णीं तस्थुर्द्विजोत्तमाः ।। ७७ ।।
।। श्रीसूत उवाच ।। ।।
य इदं रामचन्द्रस्य स्तोत्रं मुनिभिरीरितम् ।।
त्रिसन्ध्यं पठते भक्त्या भुक्तिं मुक्तिं च विन्दति ।। ७८ ।।
प्रयाणकाले पठतो न् भीतिरुपजायते ।।
एतत्स्तोत्रस्य पठनाद्भूतवेतालकादयः ।। ।। ७९ ।।
नश्यन्ति रोगा नश्यन्ति नश्यते पापसञ्चयः ।।
पुत्रकामो लभेत्पुत्रं कन्या विन्दति सत्पतिम् ।। 3.1.44.८० ।।
मोक्षकामो लभेन्मोक्षं धनकामो धनं लभेत्।।
सर्वान्कामानवाप्नोति पठन्भक्त्या त्विमं स्तवम् ।। ८१ ।।
ततो रामो मुनीन्प्राह प्रणम्य च कृताञ्जलिः ।।
अहं विशुद्धये प्राप्यः सकलैरपि मानवैः ।। ८२ ।।
मद्दृष्टिगोचरो जन्तुर्नित्यमोक्षस्य भाजनम् ।।
तथापि मुनयो नित्यं भक्तियुक्तेन चेतसा ।। ८३ ।।
स्वात्मलाभेन सन्तुष्टान्साधून्भूतसुहृत्तमान् ।।
निरहङ्कारिणः शान्तान्नमस्याम्यूर्ध्वरेतसः ।। ८४ ।।
यस्माद्ब्रह्मण्यदेवोऽहमतो विप्रान्भजे सदा ।।
युष्मान्पृच्छाम्यहं किञ्चित्तद्वदध्वं विचार्य तु ।। ८५ ।।
रावणस्य वधाद्विप्रा यत्पापं मम वर्तते ।।
तस्य मे निष्कृतिं ब्रूत पौलस्त्यवधजस्य हि ।।
यत्कृत्वा तेन पापे न मुच्येऽहं मुनिपुङ्गवाः ।। ८६ ।।
।। मुनय ऊचुः ।। ।।
सत्यव्रत जगन्नाथ जगद्रक्षाधुरन्धर ।। ८७ ।।
सर्वलोकोपकारार्थं कुरु राम शिवार्चनम् ।।
गन्धमादनशृङ्गेऽस्मिन्महापुण्ये विमुक्तिदे ।। ८८ ।।
शिवलिङ्गप्रतिष्ठां त्वं लोकसङ्ग्रहकाम्यया ।।
कुरु राम दशग्रीववधदोषापनुत्तये ।। ।।८९।।
लिङ्गस्थापनजं पुण्यं चतुर्वक्त्रोऽपि भाषितुम्।।
न शक्नोति ततो वक्तुं किं पुनर्मनुजेश्वर।।3.1.44.९।।।
यत्त्वया स्थाप्यते लिगं गन्धमादनपर्वते ।।
अस्य सन्दर्शनं पुंसां काशीलिङ्गावलोकनात् ।। ९१ ।।
अधिकं कोटिगुणितं फलवत्स्यान्न संशयः ।।
तव नाम्ना त्विदं लिङ्गं लोके ख्यातिं समश्नुताम् ।। ९२ ।।
नाशकं पुण्यपापाख्यकाष्ठानां दहनोपमम् ।।
इदं रामेश्वरं लिङ्गं ख्यातं लोके भविष्यति ।। ९३ ।।
मा विलंबं कुरुष्वातो लिङ्गस्थापनकर्मणि ।।
रामचन्द्र महाभाग करुणापूर्णविग्रह ।। ९४ ।।
।। श्रीसूत उवाच ।। ।।
इति श्रुत्वा वचो रामो मुनीनां तं मुनीश्वराः ।।
पुण्यकालं विचार्याथ द्विमुहूर्तं जगत्पतिः ।। ९५ ।।
कैलासं प्रेषयामास हनुमन्तं शिवालयम् ।।
शिवलिङ्गं समानेतुं स्थापनार्थं रघूद्वहः ।। ९६ ।।
।। राम उवाच ।। ।।
हनूमन्नञ्जनीसूनो वायुपुत्र महाबल ।।
कैलासं त्वरितो गत्वा लिङ्गमानय मा चिरम् ।। ९७ ।।
इत्याज्ञप्तस्स रामेण भुजावास्फाल्य वीर्यवान् ।।
मुहूर्तद्वितयं ज्ञात्वा पुण्यकालं कपीश्वरः ।। ९८ ।।
पश्यतां सर्वदेवानामृषीणां च महात्मनाम् ।।
उत्पपात महावेगश्चालयन्गन्धमादनम् ।।९९।।
लङ्घयन्स वियन्मार्गं कैलासं पर्वतं ययौ ।।
न ददर्श महादेवं लिङ्गरूपधरं कपिः।।3.1.44.१००।।
कैलासे पर्वते तस्मिन्पुण्ये शङ्करपालिते ।।
आञ्जनेयस्तपस्तेपे लिङ्गप्राप्त्यर्थमादरात् ।। १ ।।
प्रागग्रेषु समासीनः कुशेषु मुनिपुङ्गवाः ।।
ऊर्ध्वबाहुर्निरालम्बो निरुच्छ्वासो जितेन्द्रियः ।। २ ।।
प्रसादयन्महादेवं लिङ्गं लेभे स मारुतिः ।।
एतस्मिन्नन्तरे विप्रा मुनिभिस्तत्त्वदर्शिभिः ।। ३ ।।
अनागतं हनूमन्तं कालं स्वल्पावशेषितम् ।।
ज्ञात्वा प्रकथितं तत्र रामं प्रति महामतिम् ।। ४ ।।
रामराम महाबाहो कालो ह्यत्येति सांप्रतम् ।।
जानक्या यत्कृतं लिङ्गं सैकतं लीलया विभो ।। ५ ।।
तल्लिङ्गं स्थापयस्वाद्य महालिङ्गमनुत्तमम् ।।
श्रुत्वैतद्वचनं रामो जानक्या सह सत्वरम् ।। ६ ।।
मुनिभिः सहितः प्रीत्या कृतकौतुकमङ्गलः ।।
ज्येष्ठे मासे सिते पक्षे दशम्यां बुधहस्तयोः ।। ७ ।।
गरानन्दे व्यतीपाते कन्या चन्द्रे वृषे रवौ ।।
दशयोगे महापुण्ये गन्धमादनपर्वते ।। ८ ।।
सेतुमध्ये महादेवं लिङ्गरूपधरं हरम् ।।
ईशानं कृत्तिवसनं गङ्गाचन्द्रकलाधरम् ।। ९ ।।
रामो वै स्थापयामास शिवलिङ्गमनुत्तमम् ।।
लिङ्गस्थं पूजयामास राघवः सांबमीश्वरम् ।। 3.1.44.११० ।।
लिङ्गस्थः स महादेवः पार्वत्या सह शङ्करः ।।
प्रत्यक्षमेव भगवान्दत्तवान्वरमुत्तमम् ।। ११ ।।
सर्वलोकशरण्याय राघवाय महात्मने ।।
त्वयात्र स्थापितं लिङ्गं ये पश्यन्ति रघूद्वह ।। १२ ।।
महापातकयुक्ताश्च तेषां पापं प्रणश्यति ।।
सर्वाण्यपि हि पापानि धनुष्कोटौ निमज्जनात् ।। १३ ।।
दर्शनाद्रामलिङ्गस्य पातकानि महान्त्यपि ।।
विलयं यान्ति राजेन्द्र रामचन्द्र न संशयः ।। १४ ।।
प्रादादेवं हि रामाय वरं देवोंऽबिकापतिः ।।
तदग्रे नन्दिकेशं च स्थापयामास राघवः ।। ।। १५ ।।
ईश्वरस्याभिषेकार्थं धनुष्कोट्याथ राघवः ।।
एकं कूपं धरां भित्त्वा जनयामास वै द्विजाः ।। १६ ।।
तस्माज्जलमुपादाय स्नापयामास शङ्करम् ।।
कोटितीर्थमिति प्रोक्तं तत्तीर्थं पुण्यमुत्तमम् ।। १७ ।।
उक्तं तद्वैभवं पूर्वमस्माभिर्मुनिपुङ्गवाः ।।
देवाश्च मुनयो नागा गन्धर्वाप्स रसां गणाः ।।
सर्वेपि वानरा लिङ्गमेकैकं चक्रुरादरात् ।। १८ ।।
।। श्रीसूत उवाच ।। ।।
एवं वः कथितं विप्रा यथा रामेण धीमत ।।५९ ।।।।
स्थापितं शिवलिङ्गं वै भुक्तिमुक्तिप्रदायकम् ।।
इमां लिङ्गप्रतिष्ठां यः शृणोति पठतेऽथवा ।। 3.1.44.१२० ।।
स रामेश्वरलिङ्गस्य सेवाफलमवाप्नुयात् ।।।।
सायुज्यं च समाप्नोति रामनाथस्य वैभवात् ।। १२१ ।।
इति श्रीस्कान्दे महापुराण एकाशीतिसाहस्र्यां संहितायां तृतीये ब्रह्मखण्डे सेतुमाहात्म्ये रामनाथलिङ्गप्रतिष्ठाविधिवर्णनन्नाम चतुश्चत्वारिंशोऽध्यायः ।। ४४ ।।


===

https://sa.wikisource.org/wiki/स्कन्दपुराणम्/खण्डः_३_(ब्रह्मखण्डः)/सेतुखण्डः/अध्यायः_४५
https://www.wisdomlib.org/hinduism/book/the-skanda-purana/d/doc423613.html

।। श्रीसूत उवाच ।। ।।
एवं प्रतिष्ठिते लिङ्गे रामेणाक्लिष्टकारिणा ।।
लिङ्गं वरं समादाय मारुतिः सहसाऽऽययौ ।। १ ।।
रामं दाशरथिं वीरमभिवाद्य स मारुतिः ।।
वैदेहीलक्ष्मणौ पश्चात्सुग्रीवं प्रणनाम च ।। २ ।।
सीता सैकतलिङ्गं तत्पूजयन्तं रघूद्वहम् ।।
दृष्ट्वाथ मुनिभिः सार्द्धं चुकोप पवनात्मजः ।। ३ ।।
अत्यन्तं खेदखिन्नः सन्वृथाकृतपरिश्रमः ।।
उवाच रामं धर्मज्ञं हनूमानञ्जनात्मजः ।। ४ ।।
।। हनूमानुवाच ।। ।।
दुर्जातोऽहं वृथा राम लोके क्लेशाय केवलम् ।।
खिन्नोऽस्मि बहुशो देव राक्षसैः क्रूरकर्मभिः ।। ५ ।।
मा स्म सीमन्तिनी काचिज्जनयेन्मादृशं सुतम् ।।
यतोऽनुभूयते दुःखमनन्तं भवसागरे ।। ६ ।।
खिन्नोऽस्मि सेवया पूर्वं युद्धेनापि ततोधिकम्।।
अनन्तं दुःखमधुना यतो मामवमन्यसे ।। ७ ।।
सुग्रीवेण च भार्यार्थं राज्यार्थं राक्षसेन च ।।
रावणावरजेन त्वं सेवितो ऽसि रघूद्वह ।। ८ ।।
मया निर्हेतुकं राम सेवितोऽसि महामते ।।
वानराणामनेकेषु त्वयाज्ञप्तोऽहमद्य वै ।। ९ ।।
शिवलिङ्गं समानेतुं कैलासात्पर्वतो त्तमात् ।।
कैलासं त्वरितो गत्वा न चापश्यं पिनाकिनम् ।। 3.1.45.१० ।।
तपसा प्रीणयित्वा तं सांबं वृषभवाहनम् ।।
प्राप्तलिङ्गो रघुपते त्वरितः समु पागतः ।। ११ ।।
अन्यलिङ्गं त्वमधुना प्रतिष्ठाप्य तु सैकतम् ।।
मुनिभिर्देवगन्धर्वैः साकं पूजयसे विभो ।। १२ ।।
मयानीतमिदं लिङ्गं कैलासा त्पर्वताद्वृथा ।।
अहो भाराय मे देहो मन्दभाग्यस्यजायते ।। १३ ।।
भूतलस्य महाराज जानकीरमण प्रभो ।।
इदं दुःखमहं सोढुं न शक्नोमि रघूद्वह ।। १४
किं करिष्यामि कुत्राहं गमिष्यामि न मे गतिः ।।
अतः शरीरं त्यक्ष्यामि त्वयाहमवमानितः ।। १५ ।।
।। श्रीसूत उवाच ।।
एवं स बहुशो विप्राः क्रुशित्वा पवनात्मजः ।।
दण्डवत्प्रणतो भूमौ क्रोधशोकाकुलोऽभवत् ।। १६ ।।
तं दृष्ट्वा रघुनाथोऽपि प्रहसन्निदमब्रवीत् ।।
पश्यतां सवदेवानां मुनीनां कपिरक्षसाम् ।।
सान्त्वयन्मारुतिं तत्र दुःखं चास्य प्रमार्जयन् ।। १७ ।।
।। श्रीराम उवाच ।।
सर्वं जानाम्यहं कार्यमात्मनोऽपि परस्य च ।। १८ ।।
जातस्य जायमानस्य मृतस्यापि सदा कपे ।।
जायते म्रियते जन्तुरेक एव स्वकर्मणा १९ ।।
प्रयाति नरकं चापि परमात्मा तु निर्गुणः ।।
एवं तत्त्वं विनिश्चित्य शोकं मा कुरु वानर ।। 3.1.45.२० ।।
लिङ्गत्रयविनिर्मुक्तं ज्योतिरेकं निरञ्जनम् ।।
निराश्रयं निर्विकारमात्मानं पश्य नित्यशः ।। २१ ।।
किमर्थं कुरुषे शोकं तत्त्वज्ञानस्य बाधकम् ।।
तत्त्वज्ञाने सदा निष्ठां कुरु वानरसत्तम ।। २२ ।।
स्वयंप्रकाशमात्मानं ध्यायस्व सततं कपे ।।
देहादौ ममतां मुञ्च तत्त्वज्ञानविरोधिनीम् ।। २३ ।।
धर्मं भजस्व सततं प्राणिहिंसां परित्यज ।।
सेवस्व साधुपुरुषाञ्जहि सर्वेन्द्रियाणि च ।। २४ ।।
परित्यजस्व सततमन्येषां दोषकीर्तनम् ।।
शिवविष्ण्वादिदेवानामर्चां कुरु सदा कपे ।। २५ ।।
सत्यं वदस्व सततं परित्यज शुचं कपे ।।
प्रत्यग्ब्रह्मैकताज्ञानं मोहवस्तुसमुद्गतम् ।। २६ ।।
शोभनाशोभना भ्रान्तिः कल्पि तास्मिन्यथार्थवत् ।।
अध्यास्ते शोभनत्वेन पदार्थे मोहवैभवात् ।। २७ ।।
रोगो विजायते नृणां भ्रान्तानां कपिसत्तम ।।
रागद्वेषबलाद्बद्धा धर्मा धर्मवशङ्गताः ।। २८ ।।
देवतिर्यङ्मनुष्याद्या निरयं यान्ति मानवाः ।।
चन्दनागरुकर्पूरप्रमुखा अतिशोभनाः ।। २९ ।।
मलं भवन्ति यत्स्पर्शात्तच्छरीरं कथं सुखम्।।
भक्ष्यभोज्यादयः सर्वे पदार्था अतिशोभनाः ।।3.1.45.३०।।
विष्ठा भवन्ति यत्सङ्गात्तच्छरीरं कथं सुखम्।।
सुगन्धि शीतलं तोयं मूत्रं यत्सङ्गमाद्भवेत् ।।३१।।
तत्कथं शोभनं पिण्डं भवेद्ब्रूहि कपेऽधुना ।।
अतीव धवलाः शुद्धाः पटा यत्सङ्गमेनहि ।।३२।।
भवन्ति मलिनाः स्वेदात्तत्कथं शोभनं भवेत ।।
श्रूयतां परमार्थो मे हनूमन्वायुनन्दन ।। ३३ ।।
अस्मिन्संसारगर्ते तु किञ्चित्सौख्यं न विद्यते ।।
प्रथमं जन्तुराप्नोति जन्म बाल्यं ततः परम् ।। ।।३४।।
पश्चाद्यौवनमाप्नोति ततो वार्धक्यमश्नुते ।।
पश्चान्मृत्युमवाप्नोति पुनर्जन्म तदश्नुते ।।३५।।
अज्ञानवैभवादेव दुःखमाप्नोति मानवः ।।
तदज्ञान निवृत्तौ तु प्राप्नोति सुखमुत्तमम् ।।३६।।
अज्ञानस्य निवृत्तिस्तु ज्ञानादेव न कर्मणा ।।
ज्ञानं नाम परं ब्रह्म ज्ञानं वेदान्तवाक्यजम् ।।३७।।
तज्ज्ञानं च विरक्तस्य जायते नेतरस्य हि ।।
मुख्याधिकारिणः सत्यमाचार्यस्य प्रसादतः ।।३९।।
यदा सर्वे प्रमुच्यन्ते कामा येऽस्य हृदि स्थिताः ।।
तदा मर्त्योऽमृतोऽत्रैव परं ब्रह्म समश्नुते ।। ३९ ।।
जाग्रतं च स्वपन्तं च भुञ्जन्तं च स्थितं तथा ।।
इमं जनं सदा क्रूरः कृतान्तः परिकर्षति ।। 3.1.45.४० ।।
सर्वे क्षयान्ता निचयाः पतनान्ताः समुच्छ्रयाः ।।
संयोगा विप्रयोगान्ता मरणान्तं च जीवितम् ।। ४१ ।।
यथा फलानां पक्वानां नान्यत्र पतनाद्भयम् ।।
यथा नराणां जातानां नान्यत्र पतनाद्भयम् ।। ४२ ।।
यथा गृहं दृढस्तंभं जीर्णं काले विनश्यति ।।
एवं विनश्यन्ति नरा जरामृत्युवशङ्गताः ।। ४३ ।।
अहोरात्रस्य गमनान्नृणामायुर्विनश्यति ।।
आत्मानमनुशोच त्वं किमन्यमनुशोचसि ।। ४४ ।।
नश्यत्यायुः स्थितस्यापि धावतोऽपि कपीश्वर ।।
सहैव मृत्युर्व्रजति सह मृत्युर्निषीदति ।। ४५ ।।
चरित्वा दूरदेशं च सह मृत्युर्निवर्तते ।।
शरीरे वलयः प्राप्ताः श्वेता जाताः शिरोरुहाः ।। ४६ ।।
जीर्यते जरया देहः श्वासकासादिना तथा ।।
यथा काष्ठं च काष्ठं च समेयातां महोदधौ ।। ४७ ।।
समेत्य च व्यपेयातां कालयोगेन वानर ।।
एवं भार्या च पुत्रश्च वधुक्षेत्रधनानि च ।। ४८ ।।
क्वचित्संभूय गच्छन्ति पुनरन्यत्र वानर ।।
यथा हि पान्थं गच्छन्तं पथि कश्चित्पथि स्थितः।। ४९ ।।
अहमप्या गमिष्यामि भवद्भिः साकमित्यथ ।।
कञ्चित्कालं समेतौ तौ पुनरन्यत्र गच्छतः ।। 3.1.45.५० ।।
एवं भार्यासुतादीनां सङ्गमो नश्वरः कपे ।।
शरीरजन्मना साकं मृत्युः सञ्जायते ध्रुवम् ।। ५१ ।।
अवश्यंभाविमरणे न हि जातु प्रतिक्रिया ।।
एतच्छरीरपाते तु देही कर्मगतिं गतः ।। ५२ ।।
प्राप्य पिण्डान्तरं वत्स पूर्वपिण्डं त्यजत्यसौ ।।
प्राणिनां न सदैकत्र वासो भवति वानर ।। ५३ ।।
स्वस्वकर्मवशात्सर्वे वियुज्यन्ते पृथक्पृथक् ।।
यथा प्राणिशरीराणि नश्यन्ति च भवन्ति च ।। ५४ ।।
आत्मनो जन्ममरणे नैव स्तः कपिसत्तम ।।
अतस्त्वमञ्जनासूनो विशोकं ज्ञानमद्वयं ।।। ।। ५५ ।।
सद्रूपममलं ब्रह्म चिन्तयस्व दिवानिशम् ।।
त्वत्कृतं मत्कृतं कर्म मत्कृतं त्वाकृतं तथा ।। ५६ ।।
मल्लिङ्गस्थापनं तस्मात्त्वल्लिङ्ग स्थापनं कपे ।।
मुहूर्तातिक्रमाल्लिङ्गं सैकतं सीतया कृतम् ।। ५७ ।।
मयात्र स्थापितं तस्मात्कोपं दुःखं च मा कुरु ।।
कैलासादागतं लिङ्गं स्थापयास्मिच्छुभे दिने ।। ५८ ।।
तव नाम्ना त्विदं लिङ्गं यातु लोकत्रये प्रथाम् ।।
हनूमदीश्वरं दृष्ट्वा द्रष्टव्यो राघवेश्वरः ।। ५९ ।।
ब्रह्मराक्षसयूथानि हतानि भवता कपे ।।
अतः स्वनाम्ना लिङ्गस्य स्थापनात्त्वं प्रमोक्ष्यसे ।। 3.1.45.६० ।।
स्वयं हरेण दत्तं तु हनूमन्नामकं शिवम् ।।
संपश्यन्रामनाथं च कृतकृत्यो भवेन्नरः ।। ६१ ।।
योजनानां सहस्रेऽपि स्मृत्वा लिङ्गं हनूमतः ।।
रामनाथेश्वरं चापि स्मृत्वा सायुज्यमाप्नुयात् ।। ६२ ।।
तेनेष्टं सर्वयज्ञैश्च तपश्चाकारि कृत्स्नशः ।।
येन दृष्टौ महादेवौ हनूमद्राघवेश्वरौ ।। ६३ ।।
हनूमता कृतं लिङ्गं यच्च लिङ्गं मया कृतम् ।।
जानकीयं च यल्लिङ्गं यल्लिङ्गं लक्ष्मणेश्वरम् ।। ६४ ।।
सुग्रीवेण कृतं यच्च सेतुकर्त्रा नलेन च ।।
अङ्गदेन च नीलेन तथा जांबवता कृतम् ।। ६५ ।।
विभीषणेन यच्चापि रत्नलिङ्गं प्रतिष्ठितम् ।।
इन्द्राद्यैश्च कृतं लिङ्गं यच्छेषाद्यैः प्रतिष्ठितम् ।। ६६ ।।
इत्येकादशरूपोऽयं शिवः साक्षाद्विभासते ।।
सदा ह्येतेषु लिङ्गेषु सन्निधत्ते महेश्वरः ।। ६७ ।।
तत्स्वपापौघशुद्ध्यर्थं स्थापयस्व महेश्वरम् ।।
अथ चेत्त्वं महाभाग लिङ्गमुत्सादयिष्यसि ।। ६८ ।।
मयात्र स्थापितं वत्स सीतया सैकतं कृतम् ।।
स्थापयिष्यामि च ततो लिङ्गमेतत्त्वया कृतम् ।। ६९ ।।
पातालं सुतलं प्राप्य वितलं च रसातलम् ।।
तलातलं च तदिदं भेदयित्वा तु तिष्ठति ।। 3.1.45.७० ।।
प्रतिष्ठितं मया लिङ्गं भेत्तुं कस्य बलं भवेत् ।।
उत्तिष्ठ लिङ्गमुद्वास्य मयैतत्स्थापितं कपे ।। ।। ७१ ।।
त्वया समाहृतं लिङ्गं स्थापयस्वाशु मा शुचः ।।
इत्युक्तस्तं प्रणम्याथाज्ञातसत्त्वोऽथ वानरः ।। ७२ ।।
उद्वासयामि वेगेन सैकतं लिङ्गमुत्त मम्।।
संस्थापयामि कैलासादानीतं लिङ्गमादरात् ।। ७३ ।।
उद्वासने सैकतस्य कियान्भारो भवेन्मम ।।
चेतसैवं विचार्यायं हनूमान्मारुता त्मजः ।। ७४ ।।
पश्यतां सर्वदेवानां मुनीनां कपिरक्षसाम् ।।
पश्यतो रामचन्द्रस्य लक्ष्मणस्यापि पश्यतः ।। ७५ ।।
पश्यन्त्या अपि वैदेह्या लिङ्गं तत्सैकतं बलात् ।।
पाणिना सर्वयत्नेन जग्राह तरसा बली ।। ७६ ।।
यत्नेन महता चायं चालयन्नपि मारुतिः ।।
नालं चालयितुं ह्यासीत्सैकतं लिङ्गमोजसा ।। ७७ ।।
ततः किलकिलाशब्दं कुर्वन्वानरपुङ्गवः ।।
पुच्छमुद्यम्य पाणिभ्यां निरास्थत्तन्निजौजसा ।। ७८ ।।
इत्यनेकप्रकारेण चाल यन्नपि वानरः ।।
नैव चालयितुं शक्तो बभूव पवनात्मजः ।। ७९ ।।
तद्वेष्टयित्वा पुच्छेन पाणिभ्यां धरणीं स्पृशन् ।।
उत्पपाताथ तरसा व्योम्नि वायुसुतः कपिः ।। 3.1.45.८० ।।
कंपयन्स धरां सर्वां सप्तद्वीपां सपर्वतम् ।।
लिङ्गस्य क्रोशमात्रे तु मूर्च्छितो रुधिरं वमन् ।। ८१ ।।
पपात हनुमान्विप्राः कंपिताङ्गो धरातले ।।
पततो वायुपुत्रस्य वक्त्राच्च नयनद्वयात् ।। ८२ ।।
नासापुटाच्छ्रोत्ररन्ध्रादपानाच्च द्विजोत्तमाः ।।
रुधिरौघः प्रसुस्राव रक्तकुण्ड मभूच्च तत्।। ८३ ।।
ततो हाहाकृतं सर्वं सदेवासुरमानुषम् ।।
धावन्तौ कपिभिः सार्द्धमुभौ तौ रामलक्ष्मणौ ।। ८४ ।।
जानकीसहितौ विप्रा ह्यास्तां शोकाकुलौ तदा ।।
सीतया सहितौ वीरौ वानरैश्च महाबलौ ।। ८५ ।।
रुरुचाते तदा विप्रा गन्धमादनपर्वते ।।
यथा तारागणयुतौ रजन्यां शशि भास्करौ ।। ८६ ।।
ददर्शतुर्हनूमन्तं चूर्णीकृतकलेवरम्।।
मूर्च्छितं पतितं भूमौ वमन्तं रुधिरं मुखात्।। ८७ ।।
विलोक्य कपयः सर्वे हाहाकृत्वाऽपतन्भुवि ।।
कराभ्यां सदयं सीता हनूमन्तं मरुत्सुतम्।। ८८ ।।
ताततातेति पस्पर्श पतितं धरणीतले ।।
रामोऽपि दृष्ट्वा पतितं हनूमन्तं कपीश्वरम् ।। । ८९ ।।
आरोप्याङ्कं स्वपाणिभ्यामाममर्श कलेवरम् ।।
विमुञ्चन्नेत्रजं वारि वायुजं चाव्रवीद्द्विजाः ।। 3.1.45.९० ।।
इति श्रीस्कान्दे महापुराण एकाशीति साहस्र्यां संहितायां तृतीये ब्रह्मखण्डे सेतुमाहात्म्ये रामचन्द्रतत्त्वज्ञानोपदेशवर्णनन्नाम पञ्चचत्वारिंशोऽध्यायः ।। ४५ ।।


===

https://sa.wikisource.org/wiki/स्कन्दपुराणम्/खण्डः_३_(ब्रह्मखण्डः)/सेतुखण्डः/अध्यायः_४६
https://www.wisdomlib.org/hinduism/book/the-skanda-purana/d/doc423614.html

।। श्रीराम उवाच।।
पंपारण्ये वयं दीनास्त्वया वानरपुङ्गव।।
आश्वासिताः कारयित्वा सख्यमादित्यसूनुना।।१।।
त्वां दृष्ट्वा पितरं बन्धून्कौसल्यां जननीमपि ।।
न स्मरामो वयं सर्वान्मे त्वयोपकृतं बहु।। २ ।।
मदर्थं सागरस्तीर्णो भवता बहु योजनः ।।
तलप्रहाराभिहतो मैनाकोऽपि नगोत्तमः ।। ३ ।।
नागमाता च सुरसा मदर्थं भवता जिता ।।
छायाग्रहां महाक्रूराम वधीद्राक्षसीं भवान् ।। ४ ।।
सायं सुवेलमासाद्य लङ्कामाहत्य पाणिना ।।
अयासी रावणगृहं मदर्थं त्वं महाकपे ।। ५ ।।
सीतामन्विष्य लङ्कायां रात्रौ गतभयो भवान्।।
अदृष्ट्वा जानकीं पश्चादशोकवनिकां ययौ ।। ६ ।।
नमस्कृत्य च वैदेहीमभिज्ञानं प्रदाय च ।।
चूडामणिं समादाय मदर्थं जानकीकरात् ।। ७ ।।
अशोकवनिकावृक्षानभाङ्क्षीस्त्वं महाकपे ।।
ततस्त्वशीतिसाहस्रान्किङ्करान्नाम राक्षसान् ।। ८ ।।
रावणप्रतिमान्युद्धे पत्यश्वेभरथाकुलान् ।।
अवधीस्त्वं मदर्थे वै महाबलपराक्रमान् ।। ९ ।।
ततः प्रहस्ततनयं जंबुमालिनमागतम् ।।
अवधीन्मन्त्रितनयान्सप्त सप्तार्चिवर्चसः ।। 3.1.46.१० ।।
पञ्च सेनापतीन्पश्चादनयस्त्वं यमालयम् ।।
कुमारमक्षमवधीस्ततस्त्वं रणमूर्धनि ।। ११ ।।
तत इन्द्रजिता नीतो राक्षसेन्द्र सभां शुभाम् ।।
तत्र लङ्केश्वरं वाचा तृणीकृत्यावमन्य च ।। १२ ।।
अभाङ्क्षीस्त्वं पुरीं लङ्कां मदर्थं वायुनन्दन ।।
पुनः प्रतिनिवृत्तस्त्वमृष्यमूकं महागिरिम् ।। १३ ।।
एवमादि महादुःखं मदर्थं प्राप्तवानसि ।।
त्वमत्र भूतले शेषे मम शोकमुदीरयन् ।। १४ ।।
अहं प्राणान्परित्यक्ष्ये मृतोऽसि यदि वायुज ।।
सीतया मम किं कार्यं लक्ष्मणेनानुजेन वा ।। १५ ।।
भरतेनापि किं कार्यं शत्रुघ्नेन श्रियापि वा ।।
राज्येनापि न मे कार्यं परेतस्त्वं कपे यदि ।। १६ ।।
उत्तिष्ठ हनुमन्वत्स किं शेषेऽद्य महीतले ।।
शय्यां कुरु महाबाहो निद्रार्थं मम वानर ।। १७ ।।
कन्दमूलफलानि त्वमाहारार्थं ममाहर ।।
स्नातुमद्य गमिष्यामि शीघ्रं कलशमानय ।। १८ ।।
अजिनानि च वासांसि दर्भांश्च समुपाहर ।।
ब्रह्मास्त्रेणावबद्धोऽहं मोचितश्च त्वया हरे ।। १९ ।।
लक्ष्मणेन सह भ्रात्रा ह्यौषधानयनेन वै ।।
लक्ष्मणप्राणदाता त्वं पौलस्त्यमदनाशनः ।। 3.1.46.२० ।।
सहायेन त्वया युद्धे राक्षसा न्रावणादिकान् ।।
निहत्यातिबलान्वीरानवापं मैथिलीमहम् ।। २१ ।।
हनूमन्नञ्जनासूनो सीताशोकविनाशन ।।
कथमेवं परित्यज्य लक्ष्मणं मां च जानकीम् ।। २२ ।।
अप्रापयित्वायोध्यां त्वं किमर्थं गतवानसि ।।
क्व गतोसि महावीर महाराक्षसकण्टक ।। २३ ।।
इति पश्यन्मुखं तस्य निर्वाक्यं रघुनन्दनः ।।
प्ररुदन्नश्रुजालेन सेचयामास वायुजम् ।। २४ ।।
वायुपुत्रस्ततो मूर्च्छामपहाय शनैर्द्विजाः ।।
पौलस्त्यभयसन्त्रस्तलोकरक्षार्थमागतम्।।२५।।
आश्रित्य मानुषं भावं नारायणमजं विभुम्।।
जानकीलक्ष्मणयुतं कपिभिः परिवारितम्।।२६।।
कालांभोधरसङ्काशं रणधूलिसमुक्षितम् ।।
जटामण्डलशोभाढ्यं पुण्डरीकायतेक्षणम् ।। २७ ।।
खिन्नं च बहुशो युद्धे ददर्श रघुनन्दनम् ।।
स्तूयमानममित्रघ्नं देवर्षिपितृकिन्नरैः ।। २८ ।।
दृष्ट्वा दाशरथिं रामं कृपाबहुलचेतसम् ।।
रघुनाथकरस्पर्शपूर्णगात्रः स वानरः ।। २९ ।।
पतित्वा दण्डवद्भूमौ कृताञ्जलिपुटो द्विजाः ।।
अस्तौषीज्जानकीनाथं स्तोत्रैः श्रुतिमनोहरैः ।। 3.1.46.३० ।।
।। हनूमानुवाच ।। ।।
नमो रामाय हरये विष्णवे प्रभविष्णवे ।।
आदिदेवाय देवाय पुराणाय गदाभृते ।। ३१ ।।
विष्टरे पुष्पकं नित्यं निविष्टाय महात्मने ।।
प्रहृष्टवानरानीकजुष्टपादांबुजाय ते ।। ३२ ।।
निष्पिष्ट राक्षसेन्द्राय जगदिष्टविधायिने ।।
नमः सहस्रशिरसे सहस्रचरणाय च ।। ३३ ।।
सहस्राक्षाय शुद्धाय राघवाय च विष्णवे ।।
भक्तार्तिहारिणे तुभ्यं सीतायाः पतये नमः ।। ३४ ।।
हरये नारसिंहाय दैत्यराजविदारिणे ।।
नमस्तुभ्यं वराहाय दंष्ट्रोद्धृतवसुन्धर ।। ३५ ।।
त्रिविक्रमाय भवते बलियज्ञ विभेदिने ।।
नमो वामनरूपाय नमो मन्दरधारिणे ।। ३६ ।।
नमस्ते मत्स्यरूपाय त्रयीपालनकारिणे ।।
नमः परशुरामाय क्षत्रियान्तकराय ते ।।३७।।
नमस्ते राक्षसघ्नाय नमो राघवरूपिणे ।।
महादेव महाभीम महाकोदण्डभेदिने ।। ३८ ।।
क्षत्रियान्तकरक्रूरभार्गवत्रासकारिणे ।।
नमोऽस्त्वहिल्या सन्तापहारिणे चापहारिणे ।। ३९ ।।
नागायुतवलोपेतताटकादेहहारिणे ।।
शिलाकठिनविस्तारवालिवक्षोविभेदिने ।। 3.1.46.४० ।।
नमो माया मृगोन्माथकारिणेऽज्ञानहारिणे ।।
दशस्यन्दनदुःखाब्धिशोषणागस्त्यरूपिणे ।। ४१ ।।
अनेकोर्मिसमाधूतसमुद्रमदहारिणे ।।
मैथिलीमानसां भोजभानवे लोकसाक्षिणे ।। ४२ ।।
राजेन्द्राय नमस्तुभ्यं जानकीपतये हरे ।।
तारकब्रह्मणे तुभ्यं नमो राजीवलोचन ।। ४३ ।।
रामाय रामचन्द्राय वरेण्याय सुखात्मने ।।
विश्वामित्रप्रियायेदं नमः खरविदारिणे ।। ४४ ।।
प्रसीद देवदेवेश भक्तानामभयप्रद ।।
रक्ष मां करु णासिन्धो रामचन्द्र नमोऽस्तु ते ।। ४५ ।।
रक्ष मां वेदवचसामप्यगोचर राघव ।।
पाहि मां कृपया राम शरणं त्वामुपैम्यहम् ।। ४६ ।।
रघुवीर महामोहमपाकुरु ममाधुना ।।
स्नाने चाचमने भुक्तो जाग्रत्स्वप्नसुषुप्तिषु ।। ४७ ।।
सर्वावस्थासु सर्वत्र पाहि मां रघुनन्दन ।।
महिमानं तव स्तोतुं कः समर्थो जगत्त्रये ।। ४८ ।।
त्वमेव त्वन्महत्त्वं वै जानासि रघुनन्दन ।।
इति स्तुत्वा वायुपुत्रो रामचन्द्रं घृणानिधिम् ।। ४९ ।।
सीतामप्यभितुष्टाव भक्तियुक्तेन चेतसा ।।
जानकि त्वां नमस्यामि सर्वपापप्रणाशिनीम् ।। 3.1.46.५० ।।
दारिद्र्यरणसंहर्त्रीं भक्तानामिष्टदायिनीम् ।।
विदेहराजतनयां राघवानन्दकारिणीम् ।। ५१ ।।
भूमेर्दुहितरं विद्यां नमामि प्रकृतिं शिवाम् ।।
पौलस्त्यैश्वर्यसंहर्त्रीं भक्ताभीष्टां सरस्वतीम् ।। ५२ ।।
पतिव्रताधुरीणां त्वां नमामि जनकात्मजाम् ।।
अनुग्रहपरामृद्धिमनघां हरिवल्लभाम् ।। ५३ ।।
आत्मविद्यां त्रयीरूपामुमारूपां नमाम्य हम् ।।
प्रसादाभिमुखीं लक्ष्मीं क्षीराब्धितनयां शुभाम् ।। ५४ ।।
नमामि चन्द्रभगिनीं सीतां सर्वाङ्गसुन्दरीम् ।।
नमामि धर्मनिलयां करुणां वेदमातरम् ।। ५५ ।।
पद्मालयां पद्महस्तां विष्णुवक्षस्थलालयाम् ।।
नमामि चन्द्रनिलयां सीतां चन्द्रनिभाननाम् ।। ५६ ।।
आह्लादरूपिणीं सिद्धिं शिवां शिवकरीं सतीम् ।।
नमामि विश्वजननीं रामचन्द्रेष्टवल्लभाम् ।।
सीतां सर्वानवद्याङ्गीं भजामि सततं हृदा ।। ५७ ।।
।। श्रीसूत उवाच ।। ।।
स्तुत्वैवं हनुमान्सीतारामचन्द्रौ सभक्तिकम् ।। ५८ ।।
आनन्दाश्रुपरिक्लिन्नस्तूष्णीमास्ते द्विजोत्तमाः ।।
य इदं वायुपुत्रेण कथितं पापनाशनम् ।। ५९ ।।
स्तोत्रं श्रीरामचन्द्रस्य सीतायाः पठतेऽन्वहम् ।।
स नरो महदैश्वर्यमश्नुते वाञ्छितं स दा ।। 3.1.46.६० ।।
अनेकक्षेत्रधान्यानि गाश्च दोग्ध्रीः पयस्विनीः ।।
आयुर्विद्याश्च पुत्रांश्च भार्यामपि मनोरमाम् ।। ६१ ।।
एतत्स्तोत्रं सकृ द्विप्राः पठन्नाप्नोत्यसंशयः ।।
एतत्स्तोत्रस्य पाठेन नरकं नैव यास्यति ।। ६२ ।।
ब्रह्महत्यादिपापानि नश्यन्ति सुमहान्त्यपि ।।
सर्वपापविनिर्मुक्तो देहान्ते मुक्तिमाप्नुयात् ।। ६३ ।।
इति स्तुतो जगन्नाथो वायुपुत्रेण राघवः ।।
सीतया सहितो विप्रा हनूमन्तमथाब्रवीत् ।। ६४ ।।
।। श्रीराम उवाच ।। ।।
अज्ञानाद्वा नरश्रेष्ठ त्वयेदं साहसं कृतम् ।।
ब्रह्मणा विष्णुना वापि शक्रादित्रिदशैरपि ।। ६५ ।।
नेदं लिङ्गं समुद्धर्तुं शक्यते स्थापितं मया ।।
महादेवापराधेन पतितोऽस्यद्य मूर्च्छितः ।। ६६ ।।
इतः परं मा क्रियतां द्रोहः सांबस्य शूलिनः ।।
अद्यारभ्य त्विदं कुण्डं तव नाम्ना जगत्त्रये ।। ६७ ।।
ख्यातिं प्रयातु यत्र त्वं पतितो वानरोत्तम ।।
महापातकसङ्घानां नाशः स्यादत्र मज्जनात् ।। ६८ ।।
महादेवजटाजाता गौतमी सरितां वरा ।।
अश्वमेधसहस्रस्य फलदा स्नायिनां नृणाम् ।। ६९ ।।
ततः शतगुणा गङ्गा यमुना च सरस्वती ।।
एतन्नदीत्रयं यत्र स्थले प्रवहते कपे ।। 3.1.46.७० ।।
मिलित्वा तत्र तु स्नानं सहस्रगुणितं स्मृतम् ।।
नदीष्वेतासु यत्स्नानात्फलं पुंसां भवेत्कपे ।। ७१ ।।
तत्फलं तव कुण्डेऽस्मिन्स्नानात्प्राप्नोत्यसंशयम् ।।
दुर्लभं प्राप्य मानुष्यं हनूमत्कुण्डतीरतः ।। ७२ ।।
श्राद्धं न कुरुते यस्तु भक्तियुक्तेन चेतसा ।।
निराशास्तस्य पितरः प्रयान्ति कुपिताः कपे ।। ७३ ।।
कुप्यन्ति मुनयोऽप्यस्मै देवाः सेन्द्राः सचारणाः ।।
न दत्तं न हुतं येन हनूमत्कुण्डतीरतः ।। ७४ ।।
वृथाजीवित एवासाविहामुत्र च दुःखभाक् ।।
हनूमत्कुण्डसविधे येन दत्तं तिलोदकम् ।।
मोदन्ते पितरस्तस्य घृतकुल्याः पिबन्ति च ।। ७५ ।।
।। श्रीसूत उवाच ।। ।।
श्रुत्वैतद्वचनं विप्रा रामेणोक्तं स वायुजः ।। ७६ ।।
उत्तरे रामनाथस्य लिङ्गं स्वेनाहृतं मुदा ।।
आज्ञया रामचन्द्रस्य स्थापयामास वायुजः ।। ७७ ।।
प्रत्यक्षमेव सर्वेषां कपिलाङ्गूलवेष्टितम् ।।
हरोपि तत्पुच्छजा तं बिभर्ति च वलित्रयम् ।।
तदुत्तरायां ककुभि गौरीं संस्थापयन्मुदा ।। ७८ ।।
।। श्रीसूत उवाच ।।
एवं वः कथितं विप्रा यदर्थं राघवेण तु ।।।
लिङ्गं प्रतिष्ठितं सेतौ भुक्तिमुक्तिप्रदं नृणाम् ।। ७९ ।।
यः पठेदिममध्यायं शृणुयाद्वा समाहितः ।।
स विधूयेह पापानि शिवलोके महीयते ।। 3.1.46.८० ।।
इति श्रीस्कान्दे महापुराण एकाशीतिसाहस्र्यां संहितायां तृतीये ब्रह्मखण्डे सेतुमाहात्म्ये रामनाथलिङ्गप्रतिष्ठाकारणवर्णनन्नाम षट्चत्वारिंशोऽध्यायः ।। ४६ ।।

===

https://sa.wikisource.org/wiki/स्कन्दपुराणम्/खण्डः_३_(ब्रह्मखण्डः)/सेतुखण्डः/अध्यायः_४७
https://www.wisdomlib.org/hinduism/book/the-skanda-purana/d/doc423615.html

।। ऋषय ऊचुः ।। ।।
राक्षसस्य वधात्सूत रावणस्य महामुने ।।
ब्रह्महत्या कथमभूद्राघवस्य महात्मनः ।। ।। १ ।।
ब्राह्मणस्य वधात्सूत ब्रह्महत्याभिजायते ।।
न ब्राह्मणो दशग्रीवः कथं तद्वद नो मुने ।। २ ।।
ब्रह्महत्या भवेत्क्रूरा रामचन्द्रस्य धीमतः ।।
एतन्नः श्रद्दधानानां वद कारुण्यतोऽधुना ।। ३ ।।
इति पृष्टस्ततः सूतो नैमिषारण्यवासिभिः ।।
वक्तुं प्रचक्रमे तेषां प्रश्नस्योत्तरमुत्तमम्।। ४ ।। ।। ।
। श्रीसूत उवाच ।। ।।
ब्रह्मपुत्रो महातेजाः पुलस्त्योनाम वै द्विजाः ।।
बभूव तस्य पुत्रोऽभूद्विश्रवा इति विश्रुतः ।। ५ ।।
तस्य पुत्रः पुलस्त्य स्य विश्रवा मुनिपुङ्गवाः ।।
चिरकालं तपस्तेपे देवैरपि सुदुष्करम् ।। ६ ।।
तपः कुर्वति तस्मिंस्तु सुमाली नाम राक्षसः ।।
पाताललोकाद्भूलोकं सर्वं वै विचचार ह ।। ७ ।।
हेमनिष्काङ्गदधरः कालमेघनिभच्छविः ।।
समादाय सुतां कन्यां पद्महीनामिव श्रियम् ।। ८ ।।
विचरन्स महीपृष्ठे कदाचित्पुष्पकस्थितम् ।।
दृष्ट्वा विश्रवसः पुत्रं कुबेरं वै धनेश्वरम् ।। ९ ।।
चिन्तयामास विप्रेन्द्राः सुमाली स तु राक्षसः ।।
कुबेरसदृशः पुत्रो यद्यस्माकं भविष्यति ।। 3.1.47.१० ।।
वयं वर्द्धामहे सर्वे राक्षसा ह्यकुतोभयाः ।।
विचार्यैवं निजसुतामब्रवीद्राक्षसेश्वरः ।। ११ ।।
सुते प्रदानकालोऽद्य तव कैकसि शोभने ।।
अद्य ते यौवनं प्राप्तं तद्देया त्वं वराय हि ।। १२ ।।
अप्रदानेन पुत्रीणां पितरो दुःखमाप्नुयुः ।।
किं च सर्वगुणोत्कृष्टा लक्ष्मीरिव सुते शुभे ।। १३ ।।
प्रत्याख्यानभयात्पुंभिर्न च त्वं प्रार्थ्यसे शुभे ।।
कन्यापितृत्वं दुःखाय सर्वेषां मानकाङ्क्षिणाम् ।।१४ ।।
न जानेऽहं वरः को वा वरयेदिति कन्यके ।।
सा त्वं पुलस्त्यतनयं मुनिं विश्रवसं द्विजम् ।। १५ ।।
पितामहकुलोद्भूतं वरयस्व स्वयङ्गता ।।
कुबेरतुल्यास्तनया भवेयुस्ते न संशयः ।। १६ ।।
कैकसी तद्वचः श्रुत्वा सा कन्या पितृगौरवात् ।।
अङ्गीचकार तद्वाक्यं तथास्त्विति शुचिस्मिता ।। १७ ।।
पर्णशालां मुनिश्रेष्ठा गत्वा विश्रवसो मुनेः ।।
अतिष्ठदन्तिके तस्य लज्जमाना ह्यधोमुखी ।। १८ ।।
तस्मिन्नवसरे विप्राः पुलस्त्यतनयः सुधीः ।।
अग्निहोत्रमुपास्ते स्म ज्वलत्पावकसन्निभः ।। १९ ।।
सन्ध्याकालमतिक्रूरमविचिन्त्य तु कैकसी ।।
अभ्येत्य तं मुनिं सुभ्रूः पितुर्वचनगौरवात् ।। 3.1.47.२० ।।
तस्थावधोमुखी भूमिं लिखत्यङ्गुष्ठकोटिना ।।
विश्रवास्तां विलोक्याथ कैकसीं तनुमध्यमाम् ।।
उवाच सस्मितो विप्राः पूर्णचन्द्रनिभाननाम् ।। २१ ।। ।।
।। विश्रवा उवाच ।। ।।
शोभने कस्य पुत्री त्वं कुतो वा त्वमिहागता ।।२२।।
कार्यं किं वा त्वमुद्दिश्य वर्तसेऽत्र शुचिस्मिते ।।
यथार्थतो वदस्वाद्य मम सर्वमनिन्दिते ।। २३ ।।
इतीरिता कैकसी सा कन्या बद्धाञ्जलिर्द्विजाः ।।
उवाच तं मुनिं प्रह्वा विनयेन समन्विता ।। २४ ।।
तपः प्रभावेन मुने मदभिप्रायमद्य तु ।।
वेत्तुमर्हसि सम्यक्त्वं पुलस्त्यकुलदीपन ।।२५।।
अहं तु कैकसी नाम सुमालिदुहिता मुने ।।
मत्तातस्याज्ञया ब्रह्मंस्तवान्तिकमुपागता ।।२६।।
शेष त्वं ज्ञानदृष्ट्याद्य ज्ञातुमर्हस्यसंशयः।।
क्षणं ध्यात्वा मुनिः प्राह विश्रवाः स तु कैकसीम् ।।२७।।
मया ते विदितं सुभ्रूर्मनोगतमभीप्सितम्।।
पुत्राभिलाषिणी सा त्वं मामगाः सांप्रतं शुभे ।।२८।।
सायङ्कालेऽधुना क्रूरे यस्मान्मां त्वमुपागता ।।
पुत्राभिलाषिणी भूत्वा तस्मात्त्वां प्रब्रवीम्यहम्।।२९।।
शृणुष्वावहिता रामे कैकसी त्वमनिन्दिते ।।
दारुणान्दारुणाकारान्दारुणाभिजनप्रियान्।।3.1.47.३० ।।
जनयिष्यसि पुत्रांस्त्वं राक्षसान्क्रूरकर्मणः ।।
श्रुत्वा तद्वचनं सा तु कैकसी प्रणिपत्य तम् ।। ३१ ।।
पुलस्त्यतनयं प्राह कृताञ्जलिपुटा द्विजाः ।।
भगवदीदृशाः पुत्रास्त्वत्तः प्राप्तुं न युज्यते ।। ३२ ।।
इत्युक्तः स मुनिः प्राह कैकसीं तां सुमध्यमाम् ।।
मद्वंशानुगुणः पुत्रः पश्चिमस्ते भविष्यति ।। ३३ ।।
धार्मिकः शास्त्रविच्छान्तो न तु राक्षसचेष्टितः ।।
इत्युक्ता कैकसी विप्राः काले कतिपये गते ।। ३४ ।।
सुषुवे तनयं क्रूरं रक्षोरूपं भयङ्करम् ।।
द्विपञ्चशीर्षं कुमतिं विंशद्बाहुं भयानकम्।। ३५ ।।
ताम्रोष्ठं कृष्णवदनं रक्तश्मश्रु शिरोरुहम् ।।
महादंष्ट्रं महाकायं लोकत्रासकरं सदा ।। ३६ ।।
दशग्रीवाभिधः सोऽभूत्तथा रावण नामवान्।।
रावणानन्तरं जातः कुम्भकर्णाभिधः सुतः ।। ३७ ।।
ततः शूर्पणखा नाम्ना क्रूरा जज्ञे च राक्षसी ।।
ततो बभूव कैकस्या विभीषण इति श्रुतः ।। ३८ ।।
पश्चिमस्तनयो धीमान्धार्मिको वेदशास्त्रवित् ।।
एते विश्रवसः पुत्रा दशग्रीवादयो द्विजाः ।। ३९ ।।
अतो दशग्रीववधात्कुम्भकर्णवधादपि ।।
ब्रह्महत्या समभवद्रामस्याक्लिष्टकर्मणः ।। 3.1.47.४० ।।
अतस्तच्छान्तये रामो लिङ्गं रामेश्वराभिधम् ।।
स्थापयामास विधिना वैदिकेन द्विजोत्तमाः ।। ४१ ।।
एवं रावणघातेन ब्रह्महत्यासमुद्भवः ।।
समभूद्रामचन्द्रस्य लोककान्तस्य धीमतः ।। ४२ ।।
तत्सहैतुकमाख्यातं भवतां ब्रह्मघातजम्।।
पापं यच्छान्तये रामो लिङ्गं प्रातिष्ठिपत्स्वयम् ।। ४३ ।।
एवं लिङ्गं प्रतिष्ठाप्य रामचन्द्रोऽतिधार्मिकः ।।
मेने कृतार्थमात्मानं ससीता वरजो द्विजाः ।। ४४ ।।
ब्रह्महत्या गता यत्र रामचन्द्रस्य भूपतेः ।।
तत्र तीर्थमभूत्किञ्चिद्ब्रह्महत्याविमोचनम् ।। ४५ ।।
तत्र स्नानं महापुण्यं ब्रह्महत्याविनाशनम् ।।
दृश्यते रावणोऽद्यापि छायारूपेण तत्र वै ।। ४६ ।।
तदग्रे नागलोकस्य बिलमस्ति महत्तरम् ।।
दशग्रीववधोत्पन्नां ब्रह्महत्यां बलीयसीम् ।। ४७ ।।
तद्बिलं प्रापयामास जानकीरमणो द्विजाः ।।
तस्योपरि बिलस्याथ कृत्वा मण्डपमुत्तमम् ।। ४८ ।।
भैरवं स्थापयामास रक्षार्थं तत्र राघवः ।।
भैरवाज्ञापरित्रस्ता ब्रह्महत्या भयङ्करी ।। ४९ ।।
नाशक्नोत्तद्बिलादूर्ध्वं निर्गन्तुं द्विजसत्तमाः ।।
तस्मिन्नेव बिले तस्थौ ब्रह्महत्या निरुद्यमा ।। 3.1.47.५० ।।
रामनाथमहालिङ्गं दक्षिणे गिरिजा मुदा ।।
वर्तते परमानन्दशिवस्यार्धशरीरिणी ।। ५१ ।।
आदित्यसोमौ वर्तेते पार्श्वयोस्तत्र शूलिनः ।।
देवस्य पुरतो वह्नी रामनाथस्य वर्तते ।। ५२ ।।
आस्ते शतक्रतुः प्राच्यामाग्नेयां च तथाऽनलः ।।
आस्ते यमो दक्षिणस्यां रामनाथस्य सेवकः ।। ५३ ।।
नैर्ऋते निर्ऋतिर्विप्रा वर्तते शङ्करस्य तु ।।
वारुण्यां वरुणो भक्त्या सेवते राघवेश्वरम् ।। ५४ ।।
वायव्ये तु दिशो भागे वायुरास्ते शिवस्य तु ।।
उत्तरस्यां च धनदो रामनाथस्य वर्तते ।। ५५ ।।
ईशान्येऽस्य च दिग्भागे महेशो वर्तते द्विजाः ।।
विनायक कुमारौ च महादेवसुतावुभौ ।। ५६ ।।
यथाप्रदेशं वर्तेते रामनाथालयेऽधुना ।।
वीरभद्रादयः सर्वे महेश्वरगणेश्वराः ।। ५७ ।।
यथाप्रदेशं वर्तन्ते रामनाथालये सदा ।।
मुनयः पन्नगाः सिद्धा गन्धर्वाप्सरसां गणाः ।। ५८ ।।
सन्तुष्यमाणहृदया यथेष्टं शिवसन्निधौ ।।
वर्तन्ते रामनाथस्य सेवार्थं भक्तिपूर्वकम् ।। ५९ ।।
रामनाथस्य पूजार्थं श्रोत्रियान्ब्राह्मणान्बहून् ।।
रामेश्वरे रघुपतिः स्थापयामास पूजकान् ।। 3.1.47.६० ।।
रामप्रतिष्ठितान्विप्रान्हव्यकव्यादिनार्चयेत् ।।
तुष्टास्ते तोषिताः सर्वा पितृभिः सहदेवताः ।। ६१ ।।
तेभ्यो बहुधनान्ग्रामान्प्रददौ जानकीपतिः ।।
रामनाथमहादेव नैवेद्यार्थमपि द्विजाः ।। ६२ ।।
बहून्ग्रामान्बहुधनं प्रददौ लक्ष्मणाग्रजः ।।
हारकेयूरकटकनिष्काद्याभरणानि च ।। ६३ ।।
अनेकपट्ट वस्त्राणि क्षौमाणि विविधानि च ।।
रामनाथाय देवाय ददौ दशरथात्मजः ।। ६४ ।।
गङ्गा च यमुना पुण्या सरयूश्च सरस्वती ।।
सेतौ रामेश्वरं देवं भजन्ते स्वाघशान्तये ।। ६५ ।।
एतदध्यायपठनाच्छ्रवणादपि मानवः ।।
विमुक्तः सर्वपापेभ्यः सायुज्यं लभते हरेः ।। ६६ ।।
इति श्रीस्कान्दे महापुराण एकाशीतिसाहस्र्यां संहितायां तृतीये ब्रह्मखण्डे सेतुमाहात्म्ये रामस्य ब्रह्महत्योत्पत्तिहेतुनिरूपणं नाम सप्तचत्वारिंशोऽध्यायः ।। ४७ ।। ।। ।।

===

https://sa.wikisource.org/wiki/स्कन्दपुराणम्/खण्डः_३_(ब्रह्मखण्डः)/धर्मारण्य_खण्डः/अध्यायः_३०
https://www.wisdomlib.org/hinduism/book/the-skanda-purana/d/doc423651.html

।। व्यास उवाच ।। ।।
पुरा त्रेतायुगे प्राप्ते वैष्णवांशो रघूद्वहः ।।
सूर्यवंशे समुत्पन्नो रामो राजीवलोचनः ।। १ ।।
स रामो लक्ष्मणश्चैव काकपक्षधरावुभौ ।।
तातस्य वचनात्तौ तु विश्वामित्रमनुव्रतौ ।। २ ।।
यज्ञसंरक्षणार्थाय राज्ञा दत्तौ कुमारकौ ।।
धनुःशरधरौ वीरौ पितुर्वचनपालकौ ।। ३ ।।
पथि प्रव्रजतो यावत्ताडकानाम राक्षसी ।।
तावदागम्य पुरतस्तस्थौ वै विघ्नकारणात् ।। ।। ४ ।।
ऋषेरनुज्ञया रामस्ताडकां समघातयत्।।
प्रादिशच्च धनुर्वेदविद्यां रामाय गाधिजः ।।५।।
तस्य पादतलस्पर्शाच्छिला वासवयोगतः ।।
अहल्या गौतमवधूः पुनर्जाता स्वरूपिणी ।। ६ ।।
विश्वामित्रस्य यज्ञे तु संप्रवृत्ते रघूत्तमः ।।
मारीचं च सुबाहुं च जघान परमेषुभिः ।।७।।
ईश्वरस्य धनुर्भग्नं जनकस्य गृहे स्थितम् ।।
रामः पञ्चदशे वर्षे षड्वर्षां चैव मैथिलीम् ।। ८ ।।
उपयेमे तदा राजन्रम्यां सीतामयोनिजाम् ।।
कृतकृत्यस्तदा जातः सीतां संप्राप्य राघवः ।। ९ ।।
अयोध्यामगमन्मार्गे जामदग्न्यमवेक्ष्य च ।।
सङ्ग्रामोऽभूत्तदा राजन्देवानामपि दुःसहः ।। 3.2.30.१० ।।
ततो रामं पराजित्य सीतया गृहमागतः ।।
ततो द्वादशवर्षाणि रेमे रामस्तया सह ।। ११ ।।
एकविंशतिमे वर्षे यौवराज्यप्रदायकम् ।।
राजानमथ कैकेयी वरद्वयमयाच त ।। १२ ।।
तयोरेकेन रामस्तु ससीतः सहलक्ष्मणः ।।
जटाधरः प्रव्रजतां वर्षाणीह चतुर्दश ।। १३ ।।
भरतस्तु द्वितीयेन यौवराज्याधिपोस्तु मे ।।
मन्थरावचनान्मूढा वरमेतमयाचत ।। १४ ।।
जानकीलक्ष्मणसखं रामं प्राव्राजयन्नृपः ।।
त्रिरात्रमुदकाहारश्चतुर्थेह्नि फलाशनः ।। १५ ।।
पञ्चमे चित्रकूटे तु रामो वासमकल्पयत् ।।
तदा दशरथः स्वर्गं गतो राम इति ब्रुवन् ।। १६ ।।
ब्रह्मशापं तु सफलं कृत्वा स्वर्गं जगाम किम् ।।
ततो भरत शत्रुघ्नौ चित्रकूटे समागतौ ।। १७ ।।
स्वर्गतं पितरं राजन्रामाय विनिवेद्य च ।।
सान्त्वनं भरतस्यास्य कृत्वा निवर्तनं प्रति ।। १८ ।।
ततो भरत शत्रुघ्नौ नन्दिग्रामं समागतौ ।।
पादुकापूजनरतौ तत्र राज्यधरावुभौ ।। १९ ।।
अत्रिं दृष्ट्वा महात्मानं दण्डकारण्यमागमत ।।
रक्षोगणवधारम्भे विराधे विनिपातिते ।। 3.2.30.२० ।।
अर्द्धत्रयोदशे वर्षे पञ्चवट्यामुवास ह ।।
ततो विरूपयामास शूर्पणखां निशाचरीम् ।।
वने विचरतरतस्य जानकीसहितस्य च ।। २१ ।।
आगतो राक्षसो घोरः सीतापहरणाय सः ।।
ततो माघासिताष्टम्यां मुहूर्ते वृन्दसंज्ञके ।। २२ ।।
राघवाभ्यां विना सीतां जहार दश कन्धरः ।।
मारीचस्याश्रमं गत्वा मृगरूपेण तेन च ।। २३ ।।
नीत्वा दूरं राघवं च लक्ष्मणेन समन्वितम् ।।
ततो रामो जघानाशु मारीचं मृगरू पिणम् ।। २४ ।।
पुनः प्राप्याश्रमं रामो विना सीतां ददर्श ह ।।
तत्रैव ह्रियमाणा सा चक्रन्द कुररी यथा ।। २५ ।।
रामरामेति मां रक्ष रक्ष मां रक्षसा हृताम्।।
यथा श्येनः क्षुधायु्क्तः क्रन्दन्तीं वर्तिकां नयेत् ।। २६ ।।
तथा कामवशं प्राप्तो राक्षसो जनकात्मजाम् ।।
नयत्येष जनकजां तच्छ्रुत्वा पक्षिराट् तदा ।। २७ ।।
युयुधे राक्षसेन्द्रेण रावणेन हतोऽपतत् ।।
माघासितनवम्यां तु वसन्तीं रावणालये ।। २८ ।।
मार्गमाणौ तदा तौ तु भ्रातरौ रामलक्ष्मणौ ।। २९ ।।
जटायुषं तु दृष्ट्वैव ज्ञात्वा राक्षससंहृताम् ।।
सीतां ज्ञात्वा ततः पक्षी संस्कृतस्तेन भक्तितः ।। 3.2.30.३० ।।
अग्रतः प्रययौ रामो लक्ष्मणस्तत्पदानुगः ।।
पंपाभ्याशमनुप्राप्य शबरीमनुगृह्य च ।।३१।।
तज्जलं समुपस्पृश्य हनुमद्दर्शनं कृतम् ।।
ततो रामो हनुमता सह सख्यं चकार ह ।। ३२ ।।
ततः सुग्रीवमभ्येत्य अहनद्वालिवानरम् ।।
प्रेषिता रामदेवेन हनुमत्प्रमुखाः प्रियाम् ।। ३३ ।।
अङ्गुलीयकमादाय वायुसूनुस्तदागतः ।।
संपातिर्दशमे मासि आचख्यौ वानराय ताम् ।।३४।।
ततस्तद्वचनादब्धिं पुप्लुवे शतयोजनम् ।।
हनुमान्निशि तस्यां तु लङ्कायां परितोऽचिनोत्।।३५।।
तद्रात्रिशेषे सीताया दर्शनं तु हनूमतः ।।
द्वादश्यां शिंशपावृक्षे हनुमान्पर्यवस्थितः ।। ३६ ।।
तस्यां निशायां जानक्या विश्वासायाह सङ्कथाम् ।।
अक्षादिभिस्त्रयोदश्यां ततो युद्धमवर्त्तत ।। ३७ ।।
ब्रह्मास्त्रेण त्रयोदश्यां बद्धः शक्रजिता कपिः ।।
दारुणानि च रूक्षाणि वाक्यानि राक्षसाधिपम् ।। ।। ३८ ।।
अब्रवीद्वायुसूनुस्तं बद्धो ब्रह्मास्त्रसंयुतः ।।
वह्निना पुच्छयुक्तेन लङ्काया दहनं कृतम् ।। ३९ ।।
पूर्णिमायां महेन्द्राद्रौ पुनरागमनं कपेः ।।
मार्गशीर्षप्रतिपदः पञ्चभिः पथि वासरैः ।। 3.2.30.४० ।।
पुनरागत्य वर्षेह्नि ध्वस्तं मधुवनं किल ।।
सप्तम्यां प्रत्यभिज्ञानदानं सर्वनिवेदनम् ।। ४१ ।।
मणिप्रदानं सीतायाः सर्वं रामाय शंसयत् ।।
अष्टम्युत्तरफाल्गुन्यां मुहूर्ते विजयाभिधे ।। ४२ ।।
मध्यं प्राप्ते सहस्रांशौ प्रस्थानं राघवस्य च ।।
रामः कृत्वा प्रतिज्ञां हि प्रयातुं दक्षिणां दिशम्।। ४३ ।।
तीर्त्वाहं सागरमपि हनिष्ये राक्षसेश्वरम् ।।
दक्षिणाशां प्रयातस्य सुग्रीवोऽथाभव त्सखा ।। ४४ ।।
वासरैः सप्तभिः सिन्धोस्तीरे सैन्यनिवेशनम् ।।
पौषशुक्लप्रतिपदस्तृतीयां यावदंबुधौ ।।
उपस्थानं ससैन्यस्य राघवस्य बभूव ह।।४५।।
विभीषणश्चतुर्थ्यां तु रामेण सह सङ्गतः ।।
समुद्रतरणार्थाय पञ्चम्यां मन्त्र उद्यतेः ।। ४६ ।।
प्रायोपवेशनं चक्रे रामो दिनचतुष्टयम् ।।
समुद्राद्वरलाभश्च सहोपायप्रदर्शनः ।। ४७ ।।
सेतोर्दशम्यामारंभस्त्रयोदश्यां समापनम् ।।
चतुर्दश्यां सुवेलाद्रौ रामः सेनां न्यवे शयत् ।। ४८ ।।
पूर्णिमास्या द्वितीयायां त्रिदिनैः सैन्यतारणम् ।।
तीर्त्वा तोयनिधिं रामः शूरवानरसैन्यवान् ।। ४९ ।।
रुरोध च पुरीं लङ्कां सीतार्थं शुभलक्षणः ।।
तृतीयादिदशम्यन्तं निवेशश्च दिनाष्टकः ।। 3.2.30.५० ।।
शुकसारणयोस्तत्र प्राप्तिरेकादशीदिने ।।
पौषासिते च द्वादश्यां सैन्यसङ्ख्यानमेव च ।। ५१ ।।
शार्दूलेन कपीन्द्राणां सारासारोपवर्णनम् ।।
त्रयोदश्याद्यमान्ते च लङ्कायां दिवसैस्त्रिभिः ।। ५२ ।।
रावणः सैन्यसं ख्यानं रणोत्साहं तदाऽकरोत् ।।
प्रययावङ्गदो दौत्ये माघशुक्लाद्यवासरे ।। ५३ ।।
सीतायाश्च तदा भर्तुर्मायामूर्धादिदर्शनम् ।।
माघशुक्लद्वितीया यां दिनैः सप्तभिरष्टमीम् ।। ५४ ।।
रक्षसां वानराणां च युद्धमासीच्च सङ्कुलम् ।।
माघशुक्लनवम्यां तु रात्राविन्द्रजिता रणे ।। ५५ ।।
रामलक्ष्मणयोर्ना गपाशबन्धः कृतः किल ।।
आकुलेषु कपीशेषु हताशेषु च सर्वशः ।। ५६ ।।
वायूपदेशाद्गरुडं सस्मार राघवस्तदा ।।
नागपाशविमोक्षार्थं दशम्यां गरु डोऽभ्यगात् ।। ५७ ।।
अवहारो माघशुक्लैस्यैकादश्यां दिनद्वयम् ।।
द्वादश्यामाञ्जनेयेन धूम्राक्षस्य वधः कृतः ।। ५८ ।।
त्रयोदश्यां तु तेनैव निहतोऽकंपनो रणे ।।
मायासीतां दर्शयित्वा रामाय दशकन्धरः ।। ५९ ।।
त्रासयामास च तदा सर्वान्सैन्यगतानपि ।।
माघशुक्लचतुर्द्दश्यां यावत्कृष्णादिवासरम् ।। 3.2.30.६० ।।
त्रिदिनेन प्रहस्तस्य नीलेन विहितो वधः ।।
माघकृष्णद्वितीयायाश्चतुर्थ्यन्तं त्रिभिर्दिनैः ।। ६१ ।।
रामेण तुमुले युद्धे रावणो द्रावितो रणात् ।।
पञ्चम्या अष्टमी यावद्रावणेन प्रबोधितः ।। ६२ ।।
कुंभकर्णस्तदा चक्रेऽभ्यवहारं चतुर्दिनम् ।।
कुम्भकर्णोकरोद्युद्धं नवम्यादिचतुर्दिनैः ।। ६३ ।।
रामेण निहतो युद्धे बहुवानरभक्षकः ।।
अमावास्यादिने शोकाऽभ्यवहारो बभूव ह ।। ६४ ।।
फाल्गुनप्रतिपदादौ चतुर्थ्यन्तैश्चतुर्दिनैः ।।
नरान्तकप्रभृतयो निहताः पञ्च राक्षसाः ।। ६५ ।।
पञ्चम्याः सप्तमीं यावदतिकायवधस्त्र्यहात् ।।
अष्टम्या द्वादशीं यावन्निहतो दिनपञ्चकात् ।। ६६ ।।
निकुम्भकुम्भौ द्वावेतौ मकराक्षश्चतुर्दिनैः ।।
फाल्गुनासितद्वितीयाया दिने वै शक्रजिज्जितः ।। ६७ ।।
तृतीयादौ सप्तम्यन्तदिनपञ्चकमेव च ।।
ओषध्यानयवैयग्र्यादवहारो बभूव ह ।। ६८ ।।
अष्टम्यां रावणो मायामैथिलीं हतवान्कुधीः।।
शोकावेगात्तदा रामश्चक्रे सैन्यावधारणम् ।। ६९ ।।
ततस्त्रयोदशीं यावद्दिनैः पञ्चभिरिन्द्रजित् ।।
लक्ष्मणेन हतो युद्धे विख्यातबलपौरुषः ।। 3.2.30.७० ।।
चतुर्द्दश्यां दशग्रीवो दीक्षामापावहारतः ।।
अमावास्यादिने प्रागाद्युद्धाय दशकन्धरः ।। ७१ ।।
चैत्रशुक्लप्रतिपदः पञ्चमीदिनपञ्चके ।।
रावणो युध्यमानो ऽभूत्प्रचुरो रक्षसां वधः ।। ७२ ।।
चैत्रशुक्लाष्टमीं यावत्स्यन्दनाश्वादिसूदनम् ।।
चैत्रशुक्लनवम्यां तु सौमित्रेः शक्तिभेदने ।। ७३ ।।
कोपाविष्टेन रामेण द्रावितो दशकन्धरः ।।
विभीषणोपदेशेन हनुमद्युद्धमेव च ।। ७४ ।।
द्रोणाद्रेरोषधीं नेतुं लक्ष्मणार्थमुपागतः ।।
विशल्यां तु समादाय लक्ष्मणं तामपाययत् ।। ७५ ।।
दशम्यामवहारोऽभूद्रात्रौ युद्धं तु रक्षसाम् ।।
एकादश्यां तु रामाय रथो मातलिसारथिः ।। ७६ ।।
प्राप्तो युद्धाय द्वादश्यां यावत्कृष्णां चतुर्दशीम् ।।
अष्टादशदिने रामो रावणं द्वैरथेऽवधीत् ।। ७७ ।।
संस्कारा रावणादीनाममावा स्यादिनेऽभवन् ।।
सङ्ग्रामे तुमुले जाते रामो जयमवाप्तवान् ।। ७८ ।।
माघशुक्लद्वितीयादिचैत्रकृष्णचतुर्द्दशीम् ।।
सप्ताशीतिदिनान्येवं मध्ये पंवदशा हकम् ।। ७९ ।।
युद्धावहारः सङ्ग्रामो द्वासप्ततिदिनान्यभूत् ।।
वैशाखादि तिथौ राम उवास रणभूमिषु ।।
अभिषिक्तो द्वितीयायां लङ्काराज्ये विभी षणः ।। 3.2.30.८० ।।
सीताशुद्धिस्तृतीयायां देवेभ्यो वरलंभनम् ।।
दशरथस्यागमनं तत्र चैवानुमोदनम् ।। ८१ ।।
हत्वा त्वरेण लङ्केशं लक्ष्मणस्याग्रजो विभुः।।।
गृहीत्वा जानकीं पुण्यां दुःखितां राक्षसेन तु ।। ८२ ।।
आदाय परया प्रीत्या जानकीं स न्यवर्तत ।।
वैशाखस्य चतुर्थ्यां तु रामः पुष्पकमा श्रितः ।। ८३ ।।
विहायसा निवृत्तस्तु भूयोऽयोध्यां पुरीं प्रति ।।
पूर्णे चतुर्दशे वर्षे पञ्चम्यां माधवस्य च ।। ८४ ।।
भारद्वाजाश्रमे रामः सगणः समु पाविशत् ।।
नन्दिग्रामे तु षष्ठ्यां स पुष्पकेण समागतः ।। ८५ ।।
सप्तम्यामभिषिक्तोऽसौ भूयोऽयोध्यायां रघूद्वहः ।।
दशाहाधिकमासांश्च चतुर्दश हि मैथिली ।। ८५ ।।
उवास रामरहिता रावणस्य निवेशने ।।
द्वाचत्वारिंशके वर्षे रामो राज्यमकारयत् ।। ८७ ।।
सीतायास्तु त्रयस्त्रिंशद्वर्षाणि तु तदा भवन् ।।
स चतुर्दशवर्षान्ते प्रविष्टः स्वां पुरीं प्रभुः ।। ८८ ।।
अयोध्यां नाम मुदितो रामो रावणदर्पहा ।।
भ्रातृभिः सहितस्तत्र रामो राज्यमकार यत् ।। ८९ ।।
दशवर्षसहस्राणि दशवर्षशतानि च ।।
रामो राज्यं पालयित्वा जगाम त्रिदिवालयम् ।। 3.2.30.९० ।।
रामराज्ये तदा लोका हर्षनिर्भरमा नसाः ।।
बभूवुर्धनधान्याढ्याः पुत्रपौत्रयुता नराः ।। ९१ ।।
कामवर्षी च पर्जन्यः सस्यानि गुणवन्ति च ।।
गावस्तु घटदोहिन्यः पादपाश्च सदा फलाः ।। ९२ ।।
नाधयो व्याधयश्चैव रामराज्ये नराधिप ।।
नार्यः पतिव्रताश्चासन्पितृभक्तिपरा नराः ।। ९३ ।।
द्विजा वेदपरा नित्यं क्षत्रिया द्विज सेविनः ।।
कुर्वते वैश्यवर्णाश्च भक्तिं द्विजगवां सदा ।। ९४ ।।
न योनिसङ्करश्चासीत्तत्र नाचारसङ्करः ।।
न वन्ध्या दुर्भगा नारी काकवन्ध्या मृत प्रजा ।। ९५ ।।
विधवा नैव काप्यासीत्सभर्तृका न लप्यते ।।
नावज्ञां कुर्वते केपि मातापित्रोर्गुरोस्तथा ।। ९६ ।।
न च वाक्यं हि वृद्धानामुल्लं घयति पुण्यकृत् ।।
न भूमिहरणं तत्र परनारीपराङ्मुखाः ।। ९७ ।।
नापवादपरो लोको न दरिद्रो न रोगभाक् ।।
न स्तेयो द्यूतकारी च मैरेयी पापिनो नहि ।। ९८ ।।
न हेमहारी ब्रह्मघ्नो न चैव गुरुतल्पगः ।।
न स्त्रीघ्नो न च बालघ्नो न चैवानृतभाषणः ।। ९९ ।।
न वृत्तिलोपकश्चासीत्कूट साक्षी न चैव हि ।।
न शठो न कृतघ्नश्च मलिनो नैव दृश्यते ।। 3.2.30.१०० ।।
सदा सर्वत्र पूज्यन्ते ब्राह्मणा वेदपारगाः ।।
नावैष्णवोऽव्रती राजन्राम राज्येऽतिविश्रुते ।। १०१ ।।
राज्यं प्रकुर्वतस्तस्य पुरोधा वदतां वरः ।।
वसिष्ठो मुनिभिः सार्द्धं कृत्वा तीर्थान्यनेकशः ।। २ ।।
आजगाम ब्रह्मपुत्रो महाभागस्तपोनिधिः ।।
रामस्तं पूजयामास मुनिभिः सहितं गुरुम् ।। ३ ।।
अभ्युत्थानार्घपाद्यैश्च मधुपर्कादिपूजया ।।
प्रपच्छ कुशलं रामं वसिष्ठो मुनिपुङ्गवः ।। ४ ।।
राज्ये चाश्वे गजे कोशे देशे सद्भ्रातृभृत्ययोः ।।
कुशलं वर्त्तते राम इति पृष्टे मुनेस्तदा ।। ५ ।।
।। राम उवाच ।। ।।
सर्वत्र कुशलं मेऽद्य प्रसादाद्भवतः सदा ।।
पप्रच्छ कुशलं रामो वसिष्ठं मुनिपुङ्गवम् ।। ६ ।।
सर्वतः कुशली त्वं हि भार्यापुत्रसमन्वितः ।।
स सर्वं कथयामास यथा तीर्थान्यशेषतः ।। ७ ।।
सेवितानि धरापृष्ठे क्षेत्राण्यायतनानि च ।।
रामाय कथयामास सर्वत्र कुशलं तदा ।। ८ ।।
ततः स विस्मयाविष्टो रामो राजीवलोचनः ।।
पप्रच्छ तीर्थमाहात्म्यं यत्तीर्थेषूत्तमोत्तमम् ।। १०९ ।।
इति श्रीस्कान्दे महापुराण एकाशीतिसाहस्र्यां संहितायां तृतीये ब्रह्मखण्डे पूर्वभागे धर्मारण्यमाहात्म्ये रामचरित्रवर्णनन्नाम त्रिंशोऽध्यायः ।। ३० ।। ।। छ ।। ।।


श्रीराम उवाच ।। ।।
भगवन्यानि तीर्थानि सेवितानि त्वया विभो ।।
एतेषां परमं तीर्थं तन्ममाचक्ष्व मानद ।। १ ।।
मया तु सीताहरणे निहता ब्रह्मराक्षसाः ।।
तत्पापस्य विशुदयर्थं वद तीर्थोत्तमोत्तमम् ।। २ ।।
वसिष्ठ उवाच ।। ।।
गङ्गा च नर्मदा तापी यमुना च सरस्वती ।।
गण्डकी गोमती पूर्णा एता नद्यः सुपावनाः ।। ३ ।।
एतासां नर्मदा श्रेष्ठा गङ्गा त्रिपथगामिनी ।।
दहते किल्बिषं सर्वं दर्शनादेव राघव ।। ४ ।।
दृष्ट्वा जन्मशतं पापं गत्वा जन्मशतत्रयम् ।।
स्नात्वा जन्मसहस्रं च हन्ति रेवा कलौ युगे ।। ५ ।।
नर्मदातीरमाश्रित्य शाकमूलफलैरपि ।।
एकस्मिन्भोजिते विप्रे कोटि भोजफलं लभेत ।। ६ ।।
गङ्गा गङ्गेति यो ब्रूयाद्योजनानां शतैरपि ।।
मुच्यते सर्वपापेभ्यो विष्णुलोकं स गच्छति ।। ७ ।।
फाल्गुनान्ते कुहूं प्राप्य तथा प्रौष्ठपदेऽसिते ।।
पक्षे गङ्गामधि प्राप्य स्नानं च पितृतर्पणम् ।। ८ ।।
कुरुते पिण्डदानानि सोऽक्षयं फलमश्नुते ।।
शुचौ मासे च संप्राप्ते स्नानं वाप्यां करोति यः ।। ९ ।।
चतुरशीतिनरकान्न पश्यति नरो नृप ।।
तपत्याः स्मरणे राम महापातकिनामपि ।। 3.2.31.१० ।।
उद्धरेत्सप्तगोत्राणि कुलमेकोत्तरं शतम् ।।
यमुनायां नरः स्नात्वा सर्वपापैः प्रमुच्यते ।। ११ ।।
महापातकयुक्तोऽपि स गच्छेत्परमां गतिम् ।।
कार्त्तिक्यां कृत्तिकायोगे सरस्वत्यां निमज्जयेत् ।। १२ ।।
गच्छेत्स गरुडारूढः स्तूयमानः सुरोत्तमैः ।।
स्नात्वा यः कार्तिके मासि यत्र प्राची सरस्वती ।। १३ ।।
प्राचीं माधवमास्तूय स गच्छेत्परमां गतिम् ।।
गण्डकीपुण्यतीर्थे हि स्नानं यः कुरुते नरः ।। १४ ।।
शालग्रामशिलामर्च्य न भूयः स्तनपो भवेत् ।।
गोमतीजलकल्लोलैर्मज्जयेत्कृष्णसन्निधौ ।। १५ ।।
चतुर्भुजो नरो भूत्वा वैकुण्ठे मोदते चिरम् ।।
चर्मण्वतीं नमस्कृत्य अपः स्पृशति यो नरः ।। १६ ।।
स तारयति पूर्वजान्दश पूर्वान्दशापरान् ।।
द्वयोश्च सङ्गमं दृष्ट्वा श्रुत्वा वा सागरध्वनिम् ।। १७ ।।
ब्रह्महत्यायुतो वापि पूतो गच्छेत्परां गतिम् ।।
माघमासे प्रयागे तु मज्जनं कुरुते नरः ।। १८ ।।
इह लोके सुखं भुक्त्वा अन्ते विष्णुपदं व्रजेत् ।।
प्रभासे ये नरा राम त्रिरात्रं ब्रह्मचारिणः ।। १९ ।।
यमलोकं न पश्येयुः कुंभीपाकादिकं तथा ।।
नैमिषारण्यवासी यो नरो देवत्वमाप्नुयात् ।। 3.2.31.२० ।।
देवानामालयं यस्मात्तदेव भुवि दुर्लभम् ।।
कुरुक्षेत्रे नरो राम ग्रहणे चन्द्रसूर्ययोः ।। २१ ।।
हेमदानाच्च राजेन्द्र न भूयः स्तनपो भवेत् ।।
श्रीस्थले दर्शनं कृत्वा नरः पापात्प्रमुच्यते ।। २२ ।।
सर्वदुःखविनाशे च विष्णुलोके महीयते ।।
काश्यपीं स्पर्शयेद्यो गां मानवो भुवि राघव ।। ।। २३ ।।
सर्वकामदुघावासमृषिलोकं स गच्छति ।।
उज्जयिन्यां तु वैशाखे शिप्रायां स्नानमाचरेत् ।। २४ ।।
मोचयेद्रौरवाद्घोरात्पूर्वजांश्च सहस्रशः ।।
सिन्धुस्नानं नरो राम प्रकरोति दिनत्रयम् ।। २५ ।।
सर्वपापविशुद्धात्मा कैलासे मोदते नरः ।।
कोटितीर्थे नरः स्नात्वा दृष्ट्वा कोटीश्वरं शिवम् ।। २६ ।।
ब्रह्महत्यादिभिः पापैर्लिप्यते न च स क्वचित् ।।
अज्ञानामपि जन्तूनां महाऽमेध्ये तु गच्छताम् ।। ।। २७ ।।
पादोद्भूतं पयः पीत्वा सर्वपापं प्रणश्यति ।।
वेदवत्यां नरो यस्तु स्नाति सूर्योदये शुभे ।। २८ ।।
सर्वरोगात्प्रमुच्येत परं सुखमवाप्नुयात् ।।
तीर्थानि राम सर्वत्र स्नानपानावगाहनैः ।। २९ ।।
नाशयन्ति मनुष्याणां सर्वपापानि लीलया ।।
तीर्थानां परमं तीर्थं धर्मारण्यं प्रचक्षते ।। 3.2.31.३० ।।
ब्रह्मविष्णुशिवाद्यैर्यदादौ संस्थापितं पुरा ।।
अरण्यानां च सर्वेषां तीर्थानां च विशेषतः ।। ३१ ।।
धर्मारण्यात्परं नास्ति भुक्तिमुक्तिप्रदायकम् ।।
स्वर्गे देवाः प्रशंसन्ति धर्मारण्यनिवासिनः ।। ३२ ।।
ते पुण्यास्ते पुण्यकृतो ये वसन्ति कलौ नराः ।।
धर्मारण्ये रामदेव सर्वकिल्बिषनाशने ।। ३३ ।।
ब्रह्महत्यादिपापानि सर्वस्तेयकृतानि च ।।
परदारप्रसङ्गादि अभक्ष्यभक्षणादि वै ।। ३४ ।।
अगम्यागमना यानि अस्पर्शस्पर्शनादि च ।।
भस्मीभवन्ति लोकानां धर्मारण्यावगाहनात् ।। ३५ ।।
ब्रह्मघ्नश्च कृतघ्नश्च बालघ्नोऽनृतभाषणः ।।
स्त्रीगोघ्नश्चैव ग्रामघ्रो धर्मारण्ये विमुच्यते ।। ३६ ।।
नातः परं पावनं हि पापिनां प्राणिनां भुवि ।।
स्वर्ग्यं यशस्यमायुष्यं वाञ्छितार्थप्रदं शुभम् ।। ३७ ।।
कामिनां कामदं क्षेत्रं यतीनां मुक्तिदायकम् ।।
सिद्धानां सिद्धिदं प्रोक्तं धर्मारण्यं युगेयुगे ।। ३८ ।।
।। ब्रह्मोवाच ।। ।।
वसिष्ठवचनं श्रुत्वा रामो धर्मभृतां वरः ।।
परं हर्षमनुप्राप्य हृदयानन्दकारकम् ।। ३९ ।।
प्रोत्फुल्लहृदयो रामो रोमाचिन्ततनूरुहः ।।
गमनाय मतिं चक्रे धर्मारण्ये शुभव्रतः ।। ।। 3.2.31.४० ।।
यस्मिन्कीटपतङ्गादिमानुषाः पशवस्तथा ।।
त्रिरात्रसेवनेनैव मुच्यन्ते सर्वपातकैः ।। ४१ ।।
कुशस्थली यथा काशी शूलपाणिश्च भैरवः ।।
यथा वै मुक्तिदो राम धर्मारण्यं तथोत्तमम् ।। ४२ ।।
ततो रामो महेष्वासो मुदा परमया युतः ।।
प्रस्थितस्तीर्थयात्रायां सीतया भ्रातृभिः सह ।। ४३ ।।
अनुजग्मुस्तदा रामं हनुमांश्च कपीश्वरः ।।
कौशल्या च सुमित्रा च कैकेयी च मुदान्विता ।। ४४ ।।
लक्ष्मणो लक्षणोपेतो भरतश्च महामतिः ।।
शत्रुघ्नः सैन्यसहितोप्ययोध्यावासिनस्तथा ।। ४५ ।।
प्रकृतयो नरव्याघ्र धर्मारण्ये विनिर्ययुः ।।
अनुजग्मुस्तदा रामं मुदा परमया युताः ।। ४६ ।।
तीर्थयात्राविधिं कर्तुं गृहात्प्रचलितो नृपः ।।
वसिष्ठं स्वकुलाचार्यमिदमाह महीपते ।। ४७ ।।
।। श्रीराम उवाच ।। ।।
एतदाश्चर्यमतुलं किमादि द्वारकाभवत् ।।
कियत्कालसमुत्पन्ना वसिष्ठेदं वदस्व मे ।। ४८ ।।
।। वसिष्ठ उवाच ।। ।।
न जानामि महाराज कियत्कालादभूदिदम् ।।
लोमशो जांबवांश्चैव जानातीति च कारणम् ।। ४९ ।।
शरीरे यत्कृतं पापं नानाजन्मान्तरेष्वपि ।।
प्रायश्चितं हि सर्वेषामेतत्क्षेत्र परं स्मृतम् ।। 3.2.31.५० ।।
श्रुत्वेति वचनं तस्य रामं ज्ञानवतां वरः ।।
गन्तुं कृतमतिस्तीर्थं यात्राविधिमथाचरत् ।। ५१ ।।
वसिष्ठं चाग्रतः कृत्वा महामाण्डलिकैर्नृपैः ।।
पुनश्चरविधिं कृत्वा प्रस्थितश्चोत्तरां दिशम् ।। ५२ ।।
वसिष्ठं चाग्रतः कृत्वा प्रतस्थे पश्चिमां दिशम् ।।
ग्रामाद्ग्राममतिक्रम्य देशाद्देशं वनाद्वनम् ।।५३।।
विमुच्य निर्ययौ रामः ससैन्यः सपरिच्छदः ।।
गजवाजिसहस्रौघै रथैर्यानैश्च कोटिभिः।। ।। ५४ ।।
शिबिकाभिश्चासङ्ख्याभिः प्रययौ राघवस्तदा ।।
गजारूढः प्रपश्यंश्च देशान्विविधसौहृदान् ।। ५५ ।।
श्वेतातपत्रं विधृत्य चामरेण शुभेन च ।।
वीजितश्च जनौघेन रामस्तत्र समभ्यगात् ।। ५६ ।।
वादित्राणां स्वनैघोरैर्नृत्यगीतपुरःसरैः ।।
स्तूयमानोपि सूतैश्च ययौ रामो मुदान्वितः ।। ५७ ।।
दशमेऽहनि संप्राप्तं धर्मारण्यमनुत्तमम् ।।
अदूरे हि ततो रामो दृष्ट्वा माण्डलिकं पुरम् ।। ५८ ।।
तत्र स्थित्वा ससैन्यस्तु उवास निशि तां पुरीम् ।।
श्रुत्वा तु निर्जनं क्षेत्रमुद्वसं च भयानकम् ।। ५९ ।।
व्याघ्रसिंहाकुलं तत्र यक्षराक्षससेवितम् ।।
श्रुत्वा जनमुखाद्रामो धर्मारण्यमरण्यकम् ।।
तच्छ्रुत्वा रामदेवस्तु न चिन्ता क्रियतामिति ।। 3.2.31.६० ।।
तत्रस्थान्वणिजः शूरान्दक्षान्स्वव्यवसायके ।। ६१ ।।
समर्थान्हि महाकायान्महाबलपराक्रमान् ।।
समाहूय तदा काले वाक्यमेतदथाब्रवीत् ।। ६२ ।।
शिबिकां सुसुवणां मे शीघ्रं वाहयताचिरम् ।।
यथा क्षणेन चैकेन धर्मरण्यं व्रजाम्यहम् ।। ६३ ।।
तत्र स्नात्वा च पीत्वा च सर्वपापात्प्रमुच्यते ।।
एवं ते वणिजः सर्वै रामेण प्रेरितास्तदा ।। ६४ ।।
तथेत्युक्त्वा च ते सर्वे ऊहुस्तच्छिबिकां तदा ।।
क्षेत्रमध्ये यदा रामः प्रविष्टः सहसैनिकः ।। ।। ६५ ।।
तद्यानस्य गतिर्मन्दा सञ्जाता किल भारत ।।
मन्दशब्दानि वाद्यानि मातङ्गा मन्दगामिनः ।। ६६ ।।
हयाश्च तादृशा जाता रामो विस्मय मागतः ।।
गुरुं पप्रच्छ विनयाद्वशिष्ठं मुनिपुङ्गवम् ।। ६७ ।।
किमेतन्मन्दगतयश्चित्रं हृदि मुनीश्वर ।।
त्रिकालज्ञो मुनिः प्राह धर्मक्षेत्रमुपागतम् ।। ।। ६८ ।।
तीर्थे पुरातने राम पादचारेण गम्यते ।।
एवं कृते ततः पश्चात्सैन्यसौख्यं भविष्यति ।। ६९ ।।
पादचारी ततौ रामः सैन्येन सह संयुतः ।।
मधुवासनके ग्रामे प्राप्तः परमभावनः ।। 3.2.31.७० ।।
गुरुणा चोक्तमार्गेण मातॄणां पूजनं कृतम् ।।
नानोपहारैर्विविधैः प्रतिष्ठाविधिपूर्वकम् ।। ७३ ।।
ततो रामो हरिक्षेत्रं सुवर्णादक्षिणे तटे ।।
निरीक्ष्य यज्ञयोग्याश्च भूमीर्वै बहुशस्तथा ।। ७२ ।।
कृतकृत्यं तदात्मानं मेने रामो रघूद्वहः ।।
धर्मस्थानं निरीक्ष्याथ सुवर्णाक्षोत्तरे तटे ।। ७३ ।।
सैन्यसङ्घं समुत्तीर्य्य बभ्राम क्षेत्रमध्यतः ।।
तत्र तीर्थेषु सर्वेषु देवतायतनेषु च ।।७४।।
यथोक्तानि च कर्माणि रामश्चक्रे विधानतः ।।
श्राद्धानि विधिवच्चक्रे श्रद्धया परया युतः ।। ७५ ।।
स्थापयामास रामेशं तथा कामेश्वरं पुनः ।।
स्थानाद्वायुप्रदेशे तु सुवर्णो भयतस्तटे ।। ७६ ।।
कृत्वैवं कृतकृत्योऽभूद्रामो दशरथात्मजः ।।
कृत्वा सर्वविधिं चैव सभायां समुपाविशत् ।। ७७ ।।
तां निशां स नदीतीरे सुष्वाप रघुनन्दनः ।।
ततोऽर्द्धरात्रे सञ्जाते रामो राजीवलोचनः ।। ७८ ।।
जागृतस्तु तदा काल एकाकी धर्मवत्सलः ।।
अश्रौषीच्च क्षणे तस्मिन्रामो नारीविरोदनम् ।। ७९ ।।
निशायां करुणैर्वाक्यै रुदन्तीं कुररीमिव ।।
चारैर्विलोकयामास रामस्तामतिसंभ्रमात् ।। 3.2.31.८० ।।
दृष्ट्वातिविह्वलां नारीं क्रन्दन्तीं करुणैः स्वरैः ।।
पृष्टा सा दुःखिता नारी रामदूतैस्तदानघ ।। ८१ ।।
।। दूता ऊचुः ।। ।।
कासि त्वं सुभगे नारि देवी वा दानवी नु किम् ।।
केन वा त्रासितासि त्वं मुष्टं केन धनं तव ।।८२।।
विकला दारुणाञ्छब्दानुद्गिरन्ती मुहुर्मुहुः ।।
कथयस्व यथातथ्यं रामो राजाभिपृच्छति।। ।। ८३ ।।
तयोक्तं स्वामिनं दूताः प्रेषयध्वं ममान्तिकम् ।।
यथाहं मानसं दुःखं शान्त्यै तस्मै निवेदये ।। ८४ ।।
तथेत्युक्त्वा ततो दूता राममागत्य चाब्रुवन् ।। ८५ ।।
इति श्रीस्कान्दे महापुराण एकाशीतिसाहस्र्यां संहितायां तृतीये ब्रह्मखण्डे पूर्वभागे धर्मारण्यमाहात्म्ये दूतागमनन्नामैकत्रिंशो ऽध्यायः ।। ३१ ।। छ ।।

।। व्यास उवाच ।। ।।
ततश्च रामदूतास्ते नत्वा राममथाब्रुवन् ।।
रामराम महाबाहो वरनारी शुभानना ।। १ ।।
सुवस्त्रभूषाभरणां मृदुवाक्यपरायणाम् ।।
एकाकिनीं क्रदमानाम दृष्ट्वा तां विस्मिता वयम् ।। २ ।।
समीपवर्तिनो भूत्वा पृष्टा सा सुरसुन्दरी ।।
का त्वं देवि वरारोहे देवी वा दानवी नु किम् ।। ३ ।।
रामः पृच्छति देवि त्वां ब्रूहि सर्वं यथातथम् ।।
तच्छ्रुत्वा वचनं रामा सोवाच मधुरं वचः ।। ४ ।।
रामं प्रेषयत भद्रं वो मम दुःखापहं परम् ।। ५ ।।
तदाकर्ण्य ततो रामः संभ्रमात्त्वरितो ययौ ।।
दृष्ट्वा तां दुःखसन्तप्तां स्वयं दुःखमवाप सः ।।
उवाच वचनं रामः कृताञ्जलिपुटस्तदा ।। ६ ।।
।। श्रीराम उवाच ।। ।।
का त्वं शुभे कस्य परिग्रहो वा केनावधूता विजने निरस्ता ।।
मुष्टं धनं केन च तावकीनमाचक्ष्व मातः सकलं ममाग्रे ।। ७ ।।
इत्युक्त्वा चातिदुःखार्तो रामो मतिमतां वरः ।।
प्रणामं दण्डवच्चक्रे चक्रपाणिरिवापरः ।।८।।
तयाभिवन्दितो रामः प्रगम्य च पुनःपुनः ।।
तुष्टया परया प्रीत्या स्तुतो मधुरया गिरा ।। ९ ।।
परमात्मन्परेशान दुःखहारिन्सनातन ।।
यदर्थमवतारस्ते तच्च कार्यं त्वया कृतम् ।। 3.2.32.१० ।।
रावणः कुम्भकर्णश्च शक्रजित्प्रमुखास्तथा ।।
खरदूषणत्रिशिरोमारीचाक्षकुमारकाः ।। ११ ।।
असङ्ख्या निर्जिता रौद्रा राक्षसाः समराङ्गणे ।। १२ ।।
किं वच्मि लोकेश सुकीर्त्तिमद्य ते वेधास्त्वदीयाङ्गजपद्मसंभवः ।।
विश्वं निविष्टं च ततो ददर्श वटस्य पत्रे हि यथो वटो मतः ।। १३ ।।
धन्यो दशरथो लोके कौशल्या जननी तव ।।
ययोर्जातोसि गोविन्द जगदीश परः पुमान् ।। १४ ।।
धन्यं च तत्कुलं राम यत्र त्वमागतः स्वयम् ।।
धन्याऽयोध्यापुरी राम धन्यो लोकस्त्वदाश्रयः ।। १५ ।।
धन्यः सोऽपि हि वाल्मीकिर्येन रामायणं कृतम् ।।
कविना विप्रमुख्येभ्य आत्मबुद्ध्या ह्यनागतम् ।। १६ ।।
त्वत्तोऽभवत्कुलं चेदं त्वया देव सुपावितम् ।।१७।।
नरपतिरिति लोकैः स्मर्यते वैष्णवांशः स्वयमसि रमणीयैस्त्वं गुणैर्विष्णुरेव ।।
किमपि भुवनकार्यं यद्विचिन्त्यावतीर्य तदिह घटयतस्ते वत्स निर्विघ्नमस्तु ।।१८।।
स्तुत्वा वाचाथ रामं हि त्वयि नाथे नु सांप्रतम् ।।
शून्या वर्ते चिरं कालं यथा दोषस्तथैव हि ।। १९ ।।
धर्मारण्यस्य क्षेत्रस्य विद्धि मामधिदेवताम् ।।
वर्षाणि द्वादशेहैव जातानि दुःखि तास्म्यहम् ।। 3.2.32.२० ।।
निर्जनत्वं ममाद्य त्वमुद्धरस्व महामते ।।
लोहासुरभयाद्राम विप्राः सर्वे दिशो दश ।। २१ ।।
गताश्च वणिजः सर्वे यथास्थानं सुदुःखिताः ।।
स दैत्यो घातितो राम देवैः सुरभयङ्करः ।। २२ ।।
आक्रम्यात्र महामायो दुराधर्षो दुरत्ययः ।।
न ते जनाः समायान्ति तद्भयादति शङ्किताः ।। २३ ।।
अद्य वै द्वादश समाः शून्यागारमनाथवत् ।।
यस्माच्च दीर्घिकायां मे स्नानदानोद्यतो जनः ।। २४ ।।
राम तस्यां दीर्घिकायां निपतन्ति च शूकराः ।।
यत्राङ्गना भर्तृयुता जलक्रीडापरायणाः ।। २५ ।।
चिक्रीडुस्तत्र महिषा निपतन्ति जलाशये ।।
यत्र स्थाने सुपुष्पाणां प्रकरः प्रचुरोऽभवत् ।। २६ ।।
तद्रुद्धं कण्टकैर्वृक्षैः सिंहव्याघ्रसमाकुलैः ।।
सञ्चिक्रीडुः कुमाराश्च यस्यां भूमौ निरन्तरम् ।।२७।।
कुमार्यश्चित्रकाणां च तत्र क्रीडं ति हर्षिताः ।।
अकुर्वन्वाडवा यत्र वेदगानं तिरन्तरम् ।।२८।।
शिवानां तत्र फेत्काराः श्रूयन्तेऽतिभयङ्कराः ।।
यत्र धूमोऽग्निहोत्राणां दृश्यते वै गृहेगृहे ।। २९ ।।
तत्र दावाः सधूमाश्च दृश्यन्तेऽत्युल्बणा भृशम् ।।
नृत्यन्ते नर्त्तका यत्र हर्षिता हि द्विजाग्रतः ।। 3.2.32.३० ।।
तत्रैव भूतवेताला प्रेताः नृत्यन्ति मोहिताः ।।
नृपा यत्र सभायां तु न्यषीदन्मन्त्रतत्पराः ।। ३१ ।।
तस्मिन्स्थाने निषीदन्ति गवया ऋक्षशल्लकाः ।।
आवासा यत्र दृश्यन्ते द्विजानां वणिजां तथा ।।३२।।
कुट्टिमप्रतिमा राम दृश्यन्तेत्र बिलानि वै ।।
कोटराणीह वृक्षाणां गवाक्षाणीह सर्वतः ।। ३३ ।।
चतुष्का यज्ञवेदिर्हि सोच्छ्राया ह्यभवत्पुरा ।।
तेऽत्र वल्मीकनिचयैर्दृश्यन्ते परिवेष्टिताः ।। ३४ ।।
एवंविधं निवासं मे विद्धि राम नृपोत्तम ।।
शून्यं तु सर्वतो यस्मान्निवासाय द्विजा गताः ।। ३५ ।।
तेन मे सुमहद्दुःखं तस्मात्त्राहि नरेश्वर ।।
एतच्छ्रुत्वा वचो राम उवाच वदतां वरः ।। ३६ ।।
।। श्रीराम उवाच ।। ।।
न जाने तावकान्विप्रांश्चतुर्दिक्षु समाश्रितान् ।।
न तेषां वेद्म्यहं सङ्ख्यां नामगोत्रे द्विजन्मनाम् ।। ३७ ।।
यथा ज्ञातिर्यथा गोत्रं याथातथ्यं निवेदय ।।
तत आनीय तान्सर्वान्स्वस्थाने वासयाम्यहम् ।। ३८ ।।
।। श्रीमातोवाच ।। ।।
ब्रह्मविष्णुमहेशैश्च स्थापिता ये नरेश्वर ।।
अष्टादश सहस्राणि ब्राह्मणा वेदपारगाः ।। ३९ ।।
त्रयीविद्यासु विख्याता लोकेऽस्मिन्नमितद्युते ।।
चतुष्षष्टिकगोत्राणां वाडवा ये प्रतिष्ठिताः ।। ।। 3.2.32.४० ।।
श्रीमातादात्त्रयीविद्यां लोके सर्वे द्विजोत्तमाः ।।
षट्त्रिंशच्च सहस्राणि वैश्या धर्मपरायणाः ।। ४१ ।।
आर्यवृत्तास्तु विज्ञेया द्विजशुश्रूषणे रताः ।।
बहुलार्को नृपो यत्र संज्ञया सह राजते ।। ४२ ।।
कुमारावश्विनौ देवौ धनदो व्ययपूरकः ।।
अधिष्ठात्री त्वहं राम नाम्ना भट्टारिका स्मृता ।। ४३ ।।
।। श्रीसूत उवाच ।। ।।
स्थानाचाराश्च ये केचित्कुलाचारास्तथैव च ।।
श्रीमात्रा कथितं सर्वं रामस्याग्रे पुरातनम् ।। ४४ ।।
तस्यास्तु वचनं श्रुत्वा रामो मुदमवाप ह ।।
सत्यंसत्यं पुनः सत्यं सत्यं हि भाषितं त्वया ।। ४५ ।।
यस्मात्सत्यं त्वया प्रोक्तं तन्नाम्ना नगरं शुभम् ।।
वासयामि जगन्मातः सत्यमन्दिरमेव च ।। ४६ ।।
त्रैलोक्ये ख्यातिमाप्नोतु सत्यमन्दिरमु त्तमम् ।। ४७ ।।
एतदुक्त्वा ततो रामः सहस्रशतसङ्ख्यया ।।
स्वभृत्यान्प्रेषयामास विप्रानयनहेतवे ।। ४८ ।।
यस्मिन्देशे प्रदेशे वा वने वा सरि तस्तटे ।।
पर्यन्ते वा यथास्थाने ग्रामे वा तत्रतत्र च ।। ४९ ।।
धर्मारण्यनिवासाश्च याता यत्र द्विजोत्तमाः ।।
अर्घपाद्यैः पूजयित्वा शीघ्रमानयतात्र तान् ।। 3.2.32.५० ।।
अहमत्र तदा भोक्ष्ये यदा द्रक्ष्ये द्विजोत्तमान् ।। ५१ ।।
विमान्य च द्विजानेतानागमिष्यति यो नरः ।।
स मे वध्यश्च दण्ड्यश्च निर्वास्यो विषयाद्बहिः ।। ५२ ।।
तच्छ्रुत्वा दारुणं वाक्यं दुःसहं दुःप्रधर्षणम् ।।
रामाज्ञाकारिणो दूता गताः सर्वे दिशो दश ।। ५३ ।।
शोधिता वाडवाः सर्वे लब्धाः सर्वे सुहर्षिताः ।।
यथोक्तेन विधानेन अर्घपाद्यैरपूजयन् ।। ५४ ।।
स्तुतिं चक्रुश्च विधिवद्विनयाचारपूर्वकम् ।।
आमन्त्र्य च द्विजान्सर्वान्रामवाक्यं प्रकाशयन् ।। ५५ ।।
ततस्ते वाडवाः सर्वे द्विजाः सेवकसंयुताः ।।
गमनायोद्यताः सर्वे वेदशास्त्रपरायणाः ।। ५६ ।।
आगता रामपार्श्वं च बहुमानपुरःसराः ।।
समागतान्द्विजान्दृष्ट्वा रोमाञ्चिततनूरुहः ।। ५७ ।।
कृतकृत्यमिवात्मानं मेने दाशरथिर्नृपः ।।
स संभ्रमात्समुत्थाय पदातिः प्रययौ पुरः ।। ५८ ।।
करसंपुटकं कृत्वा हर्षाश्रु प्रतिमुञ्चयन् ।।
जानुभ्यामवनिं गत्वा इदं वचनमब्रवीत् ।। ५९ ।।
विप्रप्रसादात्कमलावरोऽहं विप्रप्रसादाद्धरणीधरोऽहम् ।।
विप्रप्रसादाज्जगतीपतिश्च विप्रप्रसादान्मम रामनाम ।। 3.2.32.६० ।।
इत्येवमुक्ता रामेण वाड वास्ते प्रहर्षिताः ।।
जयाशीर्भिः प्रपूज्याथ दीर्घायुरिति चाब्रुवन् ।। ६१ ।।
आवर्जितास्ते रामेण पाद्यार्घ्यविष्टरादिभिः ।।
स्तुतिं चकार विप्राणां दण्डवत्प्रणिपत्य च ।। ६२ ।।
कृताञ्जलिपुटः स्थित्वा चक्रे पादाभिवन्दनम् ।।
आसनानि विचित्राणि हैमान्याभरणानि च ।। ६३ ।।
समर्पयामास ततो रामो दशरथात्मजः ।।
अङ्गुलीयकवासांसि उपवीतानि कर्णकान् ।। ६४ ।।
प्रददौ विप्रमुख्येभ्यो नानावर्णाश्च धेनवः ।।
एकैकशत सङ्ख्याका घटोध्नीश्च सवत्सकाः ।। ६५ ।।
सवस्त्रा बद्धघण्टाश्च हेमशृङ्गविभूषिताः ।।
रूप्यखुरास्ताम्रपृष्ठीः कांस्यपात्रसमन्विताः ।। ६६ ।।
इति श्रीस्कान्दे महापुराण एकाशीतिसाहस्र्यां। संहितायां तृतीये ब्रह्मखण्डे पूर्वभागे धर्मारण्यमाहात्म्ये ब्रह्मनारदसंवादे सत्यमन्दिरस्थापन वर्णनोनाम द्वात्रिंशोऽध्यायः ।। ३२ ।। ।। छ ।।

।। राम उवाच ।। ।।
जीर्णोद्धारं करिष्यामि श्रीमातुर्वचनादहम् ।।
आज्ञा प्रदीयतां मह्यं यथादानं ददामि वः ।। १ ।।
पात्रे दानं प्रदातव्यं कृत्वा यज्ञवरं द्विजाः ।।
नापात्रे दीयते किञ्चिद्दत्तं न तु सुखावहम् ।। २ ।।
सुपात्रं नौरिव सदा तारयेदुभयोरपि ।।
लोहपिण्डोपमं ज्ञेयं कुपात्रं भञ्जनात्मकम् ।। ३ ।।
जातिमात्रेण विप्रत्वं जायते न हि भो द्विजाः ।।
क्रिया बलवती लोके क्रियाहीने कुतः फलम् ।। ४ ।।
पूज्यास्तस्मात्पूज्यतमा ब्राह्मणाः सत्यवादिनः ।।
यज्ञकार्ये समुत्पन्ने कृपां कुर्वन्तु सर्वदा ।। ५ ।।
।। ब्रह्मोवाच ।। ।।
ततस्तु मिलिताः सर्वे विमृश्य च परस्परम् ।।
केचिदूचुस्तदा रामं वयं शिलोञ्छजीविकाः ।। ६ ।।
सन्तोषं परमास्थाय स्थिता धर्मपरायणाः ।।
प्रतिग्रहप्रयोगेण न चास्माकं प्रयोजनम् ।। ७ ।।
दशसूनासमश्चक्री दशचक्रिसमो ध्वजः ।।
दशध्वजसमा वेश्या दशवेश्यासमो नृपः ।। ८ ।।
राजप्रतिग्रहो घोरो राम सत्यं न संशयः ।।
तस्माद्वयं न चेच्छामः प्रतिग्रहं भया वहम् ।। ९ ।।
एकाहिका द्विजाः केचित्केचित्स्वामृतवृत्तयः ।।
कुम्भीधान्या द्विजाः केचित्केचित्षट्कर्मतत्पराः ।। 3.2.33.१० ।।
त्रिमूर्तिस्थापिताः सर्वे पृथग्भावाः पृथग्गुणाः ।।
केचिदेवं वदन्ति स्म त्रिमूर्त्याज्ञां विना वयम् ।। ११ ।।
प्रतिग्रहस्य स्वीकारं कथं कुर्याम ह द्विजाः ।।
न तांबूलं स्त्रीकृतं नो ह्यद्मो दानेन भषितम् ।। १२ ।।
विमृश्य स तदा रामो वसिष्ठेन महात्मना ।।
ब्रह्मविष्णुशिवादीनां सस्मार गुरुणा सह ।।
स्मृतमात्रास्ततो देवास्तं देशं समुपागमन्।।
सूर्यकोटिप्रतीकाशीवमानावलिसंवृताः ।। १४ ।।
रामेण ते यथान्यायं पूजिताः परया मुदा ।। १३ ।।
निवेदितं तु तत्सर्वं रामेणातिसुबुद्धिना ।। १५ ।।
अधिदेव्या वचनतो जीर्णोद्धारं करोम्यहम् ।।
धर्मारण्ये हरिक्षेत्रे धर्मकूपसमीपतः ।। १६ ।।
ततस्ते वाडवाः सर्वे त्रिमूर्त्तीः प्रणिपत्य च ।।
महता हर्षवृन्देन पूर्णाः प्राप्तमनोरथाः ।। १७ ।।
अर्घ्यपाद्यादिविधिना श्रद्धया तानपूजयन् ।।
क्षणं विश्रम्य ते देवा ब्रह्मविष्णुशिवादयः ।। १८ ।।
ऊचू रामं महाशक्तिं विनयात्कृतसंपुटम् ।। १९ ।।
।। देवा ऊचुः ।। ।।
देवद्रुहस्त्वया राम ये हता रावणादयः ।।
तेन तुष्टा वयं सर्वे भानुवंशविभूषण ।। 3.2.33.२० ।।
उद्धरस्व महास्थानं महतीं कीर्तिमाप्नुहि ।। २१ ।।
लब्ध्वा स तेषामाज्ञां तु प्रीतो दशरथात्मजः ।।
जीर्णोद्धारेऽनन्तगुणं फलमिच्छन्निलापतिः ।। २२ ।।
देवानां सन्निधौ तेषां कार्यारंभमथाकरोत् ।।
स्थण्डिलं पूर्वतः कृत्वा महागिरि समं शुभम्।।२३।।
तस्योपरि बहिःशाला गृहशाला ह्यनेकशः ।।
ब्रह्मशालाश्च बहुशो निर्ममे शोभनाकृतीः ।।२४।।
निधानैश्च समायुक्ता गृहोपकरणै र्वृताः ।।
सुवर्णकोटिसंपूर्णा रसवस्त्रादिपूरिताः ।। २५ ।।
धनधान्यसमृद्धाश्च सर्वधातुयुतास्तथा ।।
एतत्सर्वं कारयित्वा ब्राह्मणेभ्यस्तदा ददौ ।।२६।।
एकैकशो दशदश ददौ धेनूः पयस्विनीः ।।
चत्वारिंशच्छतं प्रादाद्ग्रामाणां चतुराधिकम् ।। २७ ।।
त्रैविद्यद्विजविप्रेभ्यो रामो दशरथात्मजः ।।
काजेशेन त्रयेणैव स्थापिता द्विजसत्तमाः ।। २८ ।।
तस्मात्त्रयीविद्य इति ख्यातिर्लोके बभूव ह ।।
एवंविधं द्विजेभ्यः स दत्त्वा दानं महाद्भुतम् ।। ।। २९ ।।
आत्मानं चापि मेने स कृतकृत्यं नरेश्वरः ।।
ब्रह्मणा स्थापिताः पूर्वं विष्णुना शङ्करेण ये ।। 3.2.33.३० ।।
ते पूजिता राघवेण जीर्णोद्धारे कृते सति ।।
षट्त्रिंशच्च सहस्राणि गोभुजा ये वणिग्वराः ।। ३१ ।।
शुश्रूषार्थं प्रदत्ता वै देवैर्हरिहरादिभिः।।
सन्तुष्टेन तु शर्वेण तेभ्यो दत्तं तु चेत नम्।। ३२ ।।
श्वेताश्वचामरौ दत्तौ खङ्गं दत्तं सुनिर्मलम् ।।
तदा प्रबोधितास्ते च द्विजशुश्रूषणाय वै ।। ३३ ।।
विवाहादौ सदा भाव्यं चामरै मङ्गलं वरम् ।।
खङ्गं शुभं तदा धार्य्यं मम चिह्नं करे स्थितम् ।। ३४ ।।
गुरुपूजा सदा कार्या कुलदेव्याः पुनःपुनः ।।
वृद्ध्यागमेषु प्राप्तेषु वृद्धि दायकदक्षिणा ।। ३५ ।।
एकादश्यां शनेर्वारे दानं देयं द्विजन्मने ।।
प्रदेयं बालवृद्धेभ्यो मम रामस्य शासनात् ।। ३६ ।।
मण्डलेषु च ये शुद्धा वणिग्वृत्तिरताः पराः ।।
सपादलक्षास्ते दत्ता रामशासनपालकाः ।। ३७ ।।
माण्डलीकास्तु ते ज्ञेया राजानो मण्डलेश्वराः ।।
द्विज शुश्रूषणे दत्ता रामेण वणिजां वराः ।। ३८ ।।
चामरद्वितयं रामो दत्तवान्खड्गमेव च ।।
कुलस्य स्वामिनं सूर्यं प्रतिष्ठाविधिपूर्वकम् ।। ।। ३९ ।।
ब्रह्माणं स्थापयामास चतुर्वेदसमन्वितम् ।।
श्रीमातरं महाशक्तिं शून्यस्वामिहरिं तथा ।। 3.2.33.४० ।।
विघ्नापध्वंसनार्थाय दक्षिणद्वारसंस्थितम् ।।
गणं संस्थापयामास तथान्याश्चैव देवताः ।। ४१ ।।
कारितास्तेन वीरेण प्रासादाः सप्तभूमिकाः ।।
यत्किं चित्कुरुते कार्यं शुभं माङ्गल्यरूपकम् ।। ४२ ।।
पुत्रे जाते जातके वान्नाशने मुण्डनेऽपि वा ।।
लक्षहोमे कोटिहोमे तथा यज्ञक्रियासु च ।। ४३ ।।
वास्तुपूजाग्रहशान्त्योः प्राप्ते चैव महोत्सवे ।।
यत्किञ्चित्कुरुते दानं द्रव्यं वा धान्यमुत्तमम् ।। ४४ ।।
वस्त्रं वा धेनवो नाथ हेम रूप्यं तथैव च ।।
विप्राणामथ शूद्राणां दीनानाथान्धकेषु च ।। ४५ ।।
प्रथमं बकुलार्कस्य श्रीमातुश्चैव मानवः ।।
भागं दद्याच्च निर्विघ्नकार्यसिद्ध्यै निरन्तरम् ।। ४६ ।।
वचनं मे समुल्लङ्घ्य कुरुते योऽन्यथा नरः ।।
तस्य तत्कर्मणो विघ्नं भविष्यति न संशयः ।। ४७ ।।
एवमुक्त्वा ततो रामः प्रहृष्टेनान्तरात्मना ।।
देवानामथ वापीश्च प्राकारांस्तु सुशोभनान् ।। ४८ ।।
दुर्गोपकरणैर्युक्तान्प्रतोलीश्च सुविस्तृताः ।।
निर्ममे चैव कुण्डानि सरांसि सरसीस्तथा ।। ४९ ।।
धर्मवापीश्च कूपांश्च तथान्यान्देवनिर्मितान् ।।
एतत्सर्वं च विस्तार्य धर्मारण्ये मनोरमे ।। 3.2.33.५० ।।
ददौ त्रैविद्यमुख्येभ्यः श्रद्धया परया पुनः ।।
ताम्रपट्टस्थितं रामशासनं लोपयेत्तु यः ।।५१।।
पूर्वजास्तस्य नरके पतन्त्यग्रे न सन्ततिः ।।
वायुपुत्रं समाहूय ततो रामोऽब्रवीद्वचः ।। ।। ५२ ।।
वायुपुत्र महावीर तव पूजा भविष्यति ।।
अस्य क्षेत्रस्य रक्षायै त्वमत्र स्थितिमाचर ।। ५३ ।।
आञ्जनेयस्तु तद्वाक्यं प्रणम्य शिरसादधौ ।।
जीर्णोद्धारं तदा कृत्वा कृतकृत्यो बभूव ह ।। ५४ ।।
श्रीमातरं तदाभ्यर्च्य प्रसन्नेनान्तरात्मना ।।
श्रीमातरं नमस्कृत्य तीर्थान्यन्यानि राघवः ।।५५।।
तेऽपि देवाः स्वकं स्थानं ययुर्बह्मपुरोगमाः ।।५६।।
दत्त्वाशिषं तु रामाय वाञ्छितं ते भविष्यति ।।
रम्यं कृतं त्वया राम विप्राणां स्थापनादिकम् ।। ।। ५७ ।।
अस्माकमपि वात्सल्यं कृतं पुण्यवता त्वया ।।
इति स्तुवन्तस्ते देवाः स्वानि स्थानानि भेजिरे ।। ५८ ।।
इति श्रीस्कान्दे महापुराण एका शीतिसाहस्र्यां संहितायां तृतीये ब्रह्मखण्डे पूर्वार्धे धर्मारण्यमाहात्म्ये श्रीरामचन्द्रस्य पुरप्रत्यागमनवर्णनन्नाम त्रयस्त्रिंशोऽध्यायः ।। ३३ ।। छ ।।


। ।। व्यास उवाच ।। ।।
एवं रामेण धर्मज्ञ जीर्णोद्धारः पुरा कृतः ।।
द्विजानां च हितार्थाय श्रीमातुर्वचनेन च ।। १ ।।
।। युधिष्ठिर उवाच ।। ।।
कीदृशं शासनं ब्रह्मन्रामेण लिखितं पुरा ।।
कथयस्व प्रसादेन त्रेतायां सत्यमन्दिरे ।। २ ।।
।। व्यास उवाच ।। ।।
धर्मारण्ये वरे दिव्ये बकुलार्के स्वधिष्ठिते ।।
शून्यस्वामिनि विप्रेन्द्र स्थिते नारायणे प्रभौ ।। ३ ।।
रक्षणाधिपतौ देवे सर्वज्ञे गुणनायके ।।
भवसागर मग्नानां तारिणी यत्र योगिनी ।। ४ ।।
शासनं तत्र रामस्य राघवस्य च नामतः ।।
शृणु ताम्राश्रयं तत्र लिखितं धर्मशास्त्रतः ।। ५ ।।
महाश्चर्यकरं तच्च ह्यनेकयुगसंस्थितम्।।
सर्वो धातुः क्षयं याति सुवर्णं क्षयमेति च ।। ६ ।।
प्रत्यक्षं दृश्यते पुत्र द्विजशासनमक्षयम् ।।
अविनाशो हि ताम्रस्य कारणं तत्र विद्यते ।। ७ ।।
वेदोक्तं सकलं यस्माद्विष्णुरेव हि कथ्यते ।।
पुराणेषु च वेदेषु धर्मशास्त्रेषु भारत ।। ८ ।।
सर्वत्र गीयते विष्णुर्नाना भावसमाश्रयः ।।
नानादेशेषु धर्मेषु नानाधर्मनिषेविभिः ।। ९ ।।
नानाभेदैस्तु सर्वत्र विष्णुरेवेति चिन्त्यते ।।
अवतीर्णः स वै साक्षात्पुराणपुरुषो त्तमः ।। 3.2.34.१० ।।
देववैरिविनाशाय धर्मसंरक्षणाय च ।।
तेनेदं शासनं दत्तमविनाशात्मकं सुत ।। ११ ।।
यस्य प्रतापादृषद(य)स्तारिता जलमध्यतः ।।
वानरैर्वेष्टिता लङ्का हेलया राक्षसा हताः ।। १२ ।।
मुनिपुत्रं मृतं रामो यमलोकादुपानयत् ।।
दुन्दुभिर्निहतो येन कबन्धोऽभिहतस्तथा ।। १३ ।।
निहता ताडका चैव सप्तताला विभेदिताः ।।
खरश्च दूषणश्चैव त्रिशिराश्च महासुरः ।। १४ ।।
चतुर्दशसहस्राणि जवेन निहता रणे ।।
तेनेदं शासनं दत्तमक्षयं न कथं भवेत् ।। १५ ।।
स्ववंशवर्णनं तत्र लिखित्वा स्वयमेव तु ।।
देशकालादिकं सर्वं लिलेख विधिपूर्वकम् ।। १६ ।।
स्वमुद्राचिह्नितं तत्र त्रैविद्येभ्यस्तथा ददौ।।
चतुश्चत्वारिंशवर्षो रामो दशरथात्मजः ।। १७ ।।
तस्मिन्काले महाश्चर्यं सन्दत्तं किल भारत।।
तत्र स्वर्णोपमं चापि रौप्योपमम थापि च ।। १८ ।।
उवाह सलिलं तीर्थे देवर्षिपितृतृप्तिदम् ।।
स्ववंशनायकस्याग्रे सूर्येण कृतमेव तत् ।। १९ ।।
तद्दृष्ट्वा महदाश्चर्यं रामो विष्णुं प्रपूज्य च ।।
रामलेखविचित्रैस्तु लिखितं धर्मशासनम् ।। 3.2.34.२० ।।
यद्दृष्ट्वाथ द्विजाः सर्वे संसारभयबन्धनम् ।।
कुर्वते नैव यस्माच्च तस्मान्निखिलरक्षकम् ।। ।। २१ ।।
ये पापिष्ठा दुराचारा मित्रद्रोहरताश्च ये ।।
तेषां प्रबोधनार्थाय प्रसिद्धिमकरोत्पुरा ।। २२ ।।
रामलेखविचित्रैस्तु विचित्रे ताम्रपट्टके ।।
वाक्यानीमानि श्रूयन्ते शासने किल नारद ।। २३ ।।
आस्फोटयन्ति पितरः कथयन्ति पितामहाः ।।
भूमिदोऽस्मत्कुले जातः सोऽस्मान्सन्तारयिष्यति ।। २४ ।।
बहुभिर्बहुधा भुक्ता राजभिः पृथिवी त्वियम् ।।
यस्ययस्य यदा भूमिस्तस्यतस्य तदा फलम् ।। २५ ।।
षष्टिवर्षसहस्राणि स्वर्गे वसति भूमिदः ।।
आच्छेत्ता चानुमन्ता च तान्येव नरकं व्रजेत् ।। २६ ।।
सन्दंशैस्तुद्यमानस्तु मुद्गरैर्विनिहत्य च ।।
पाशैः सुबध्यमानस्तु रोरवीति महास्वरम् ।। २७ ।।
ताड्यमानः शिरे दण्डैः समालिङ्ग्य विभावसुम् ।।
क्षुरिकया छिद्यमानो रोरवीति महास्वनम् ।। २८ ।।
यमदूतैर्महाघोरैर्ब्रह्मवृत्तिविलोपकः ।।
एवंविधैर्महादुष्टैः पीड्यन्ते ते महागणैः ।। २९ ।।
ततस्तिर्यक्त्वमाप्नोति योनिं वा राक्षसीं शुनीम् ।।
व्यालीं शृगालीं पैशाचीं महाभूतभयङ्करीम् ।। 3.2.34.३० ।।
भूमेरङ्गुलहर्ता हि स कथं पापमाचरेत् ।।
भूमेरङ्गुलदाता च स कथं पुण्यमाचरेत् ।। ३१ ।।
अश्वमेधसहस्राणां राजसूयशतस्य च ।।
कन्याशतप्रदानस्य फलं प्राप्नोति भूमिदः ।। ३२ ।।
आयुर्यशः सुखं प्रज्ञा धर्मो धान्यं धनं जयः ।।
सन्तानं वर्द्धते नित्यं भूमिदः सुखमश्मुते ।। ३३ ।।
भूमेरङ्गुलमेकं तु ये हरन्ति खला नराः ।।
वन्ध्याटवीष्वतोयासु शुष्ककोटरवासिनः ।।
कृष्णसर्पाः प्रजायन्ते दत्तदायापहारकाः ।। ३४ ।।
तडागानां सहस्रेण अश्वमेधशतेन वा ।।
गवां कोटिप्रदानेन भूमिहर्त्ता विशुध्यति ।। ३५ ।।
यानीह दत्तानि पुनर्धनानि दानानि धर्मार्थयशस्कराणि ।।
औदार्यतो विप्रनिवेदितानि को नाम साधुः पुनराददीत ।। ३६ ।।
चलदलदललीलाचञ्चले जीवलोके तृणलवलघुसारे सर्वसंसारसौख्ये ।।
अपहरति दुराशः शासनं ब्राह्मणानां नरकगहनगर्त्तावर्तपातोत्सुको यः ।।३७ ।।
ये पास्यन्ति महीभुजः क्षितिमिमां यास्यन्ति भुक्त्वाखिलां नो याता न तु याति यास्यति न वा केनापि सार्द्धं धरा।।
यत्किञ्चिद्भुवि तद्विनाशि सकलं कीर्तिः परं स्थायिनी त्वेवं वै वसुधापि यैरुपकृता लोप्या न सत्कीर्तयः ।। ३८ ।।
एकैव भगिनी लोके सर्वेषामेव भूभुजाम् ।।
न भोज्या न करग्राह्या विप्रदत्ता वसुन्धरा ।। ३९ ।।
दत्त्वा भूमिं भाविनः पार्थिवेशान्भूयोभूयो याचते रामचन्द्रः ।।
सामान्योऽयं धर्मसेतुर्नृपाणां स्वे स्वे काले पालनीयो भवद्भिः ।।3.2.34.४०।।
अस्मिन्वंशे क्षितौ कोपि राजा यदि भविष्यति ।।
तस्याहं करलग्नोस्मि मद्दत्तं यदि पाल्यते।।४१।।
लिखित्वा शासनं रामश्चातुर्वेद्यद्विजोत्तमान्।।
संपूज्य प्रददौ धीमान्वसिष्ठस्य च सन्निधौ।।४२।।
ते वाडवा गृहीत्वा तं पट्टं रामाज्ञया शुभम्।।
ताम्रं हैमाक्षरयुतं धर्म्यं धर्मविभूषणम्।।४३।।
पूजार्थं भक्तिकामार्थास्तद्रक्षणमकुर्वत।।
चन्दनेन च दिव्येन पुष्पेण च सुगन्धिना।।४४।।
तथा सुवर्णपुष्पेण रूप्यपुष्पेण वा पुनः।।
अहन्यहनि पूजां ते कुर्वते वाडवाः शुभाम्।।४५।।
तदग्रे दीपकं चैव घृतेन विमलेन हि।।
सप्तवर्तियुतं राजन्नर्घ्यं प्रकुर्वते द्विजाः।।४६।।
नैवेद्यं कुर्वते नित्यं भक्तिपूर्वं द्विजोत्तमाः।।
रामरामेति रामेति मन्त्रमप्युच्चरन्ति हि।।४७।।
अशने शयने पाने गमने चोपवेशने ।।
सुखे वाप्यथवा दुःखे राममन्त्रं समुच्चरेत् ।। ४८ ।।
न तस्य दुःखदौर्भाग्यं नाधिव्याधिभयं भवेत् ।।
आयुः श्रियं बलं तस्य वर्द्धयन्ति दिने दिने ।। ४९ ।।
रामेति नाम्ना मुच्येत पापाद्वै दारुणादपि ।।
नरकं नहि गच्छेत गतिं प्राप्नोति शाश्वतीम् ।। 3.2.34.५० ।। ।।
।। व्यास उवाच ।। ।।
इति कृत्वा ततो रामः कृतकृत्यममन्यत ।।
प्रदक्षिणीकृत्य तदा प्रणम्य च द्विजान्बहून् ।। ५१ ।।
दत्त्वा दानं भूरितरं गवाश्वमहिषीरथम् ।।
ततः सर्वान्निजांस्तांश्च वाक्यमेतदुवाच ह ।। ५२ ।।
अत्रैव स्थीयतां सर्वैर्यावच्चन्द्रदिवाकरौ ।।
यावन्मेरुर्महीपृष्ठे सागराः सप्त एव च ।। ५३ ।।
तावदत्रैव स्थातव्यं भवद्भिर्हि न संशयः ।।
यदा हि शासनं विप्रा न मन्यन्ते नृपा भुवि ।। ५४ ।।
अथवा वणिजः शूरा मदमायाविमोहिताः ।।
मदाज्ञां न प्रकुर्वन्ति मन्यन्ते वा न ते जनाः ।। ५५ ।।
तदा वै वायुपुत्रस्य स्मरणं क्रियतां द्विजाः ।।
स्मृतमात्रो हनूमान्वै समागत्य करिष्यति ।। ५६ ।।
सहसा भस्म तान्सत्यं वचनान्मे न संशयः ।।
य इदं शासनं रम्यं पालयिष्यति भूपतिः ।। ।। ५७ ।।
वायुपुत्रः सदा तस्य सौख्यमृद्धिं प्रदास्यति ।।
ददाति पुत्रान्पौत्रांश्च साध्वीं पत्नीं यशो जयम् ।। ५८ ।।
इत्येवं कथयित्वा च हनुमन्तं प्रबोध्य च ।।
निवर्तितो रामदेवः ससैन्यः सपरिच्छदः ।। ५९ ।।
वादित्राणां स्वनैर्विष्वक्सूच्यमानशुभागमः ।।
श्वेतातपत्रयुक्तोऽसौ चामरैर्वी जितो नरैः ।।
अयोध्यां नगरीं प्राप्य चिरं राज्यं चकार ह ।। 3.2.34.६० ।।
इति श्रीस्कान्दे महापुराण एकाशीतिसाहस्र्यां संहितायां तृतीये ब्रह्मखण्डे पूर्वभागे धर्मारण्यमाहात्म्ये ब्रह्मनारदसंवादे श्रीरामेण ब्राह्मणेभ्यः शासनपट्टप्रदानवर्णनन्नाम चतुस्त्रिंशोऽध्यायः ।। ३४ ।। ।। छ । ।। ।।

।। नारद उवाच ।। ।।
भगवन्देवदेवेश सृष्टिसंहारकारक ।।
गुणातीतो गुणैर्युक्तो मुक्तीनां साधनं परम् ।। १ ।।
संस्थाप्य वेदभवनं विधिवद्द्विज सत्तमान् ।।
किं चक्रे रघुनाथस्तु भूयोऽयोध्यां गतस्तदा ।। २ ।।
स्वस्थाने ब्राह्मणास्तत्र कानि कर्माणि चक्रिरे ।।
।। ब्रह्मोवाच ।। ।।
इष्टापूर्तरताः शान्ताः प्रतिग्रहपराङ्मुखाः ।। ३ ।।
राज्यं चक्रुर्वनस्यास्य पुरोधा द्विजसत्तमः ।।
उवाच रामपुरतस्तीर्थमाहात्म्यमुत्तमम् ।। ४ ।।
प्रयागस्य च माहात्म्यं त्रिवेणीफलमुत्तमम् ।।
प्रयागतीर्थमहिमा शुक्लतीर्थस्य चैव हि ।। ५ ।।
सिद्धक्षेत्रस्य काश्याश्च गङ्गाया महिमा तथा ।।
वसिष्ठः कथया मास तीर्थान्यन्यानि नारद ।। ६ ।।
धर्मारण्यसुवर्णाया हरिक्षेत्रस्य तस्य च ।।
स्नानदानादिकं सर्वं वाराणस्या यवाधिकम् ।। ७ ।।
एतच्छ्रुत्वा रामदेवः स चमत्कृतमानसः ।।
धर्मारण्ये पुनर्यात्रां कर्त्तुकामः समभ्यगात् ।। ८ ।।
सीतया सह धर्मज्ञो गुरुसैन्यपुरःसरः ।।
लक्ष्मणेन सह भ्रात्रा भरतेन सहायवान् ।। ९ ।।
शत्रुघ्नेन परिवृतो गतो मोहेरके पुरे ।।
तत्र गत्वा वसिष्ठं तु पृच्छतेऽसौ महामनाः ।। 3.2.35.१० ।।
।। राम उवाच ।। ।।
धर्मारण्ये महाक्षेत्रे किं कर्त्तव्यं द्विजोत्तम ।।
दानं वा नियमो वाथ स्नानं वा तप उत्तमम् ।। ११ ।।
 ध्यानं वाथ क्रतुं वाथ होमं वा जपमुत्तमम् ।।
दानं वा नियमं वाथ स्नानं वा तप उत्तमम्।। १२ ।।
येन वै क्रियमाणेन तीर्थेऽस्मिन्द्विजसत्तम ।।
ब्रह्महत्यादिपापेभ्यो मुच्यते तद्ब्रवीहि मे ।। १३ ।।
।। वसिष्ठ उवाच ।। ।।
यज्ञं कुरु महाभाग धर्मारण्ये त्वमुत्तमम् ।।
दिनेदिने कोटिगुणं यावद्वर्षशतं भवेत् ।। १४ ।।
तच्छ्रुत्वा चैव गुरुतो यज्ञारंभं चकार सः ।।
तस्मिन्नवसरे सीता रामं व्यज्ञापयन्मुदा ।। १५ ।।
स्वामिन्पूर्वं त्वया विप्रा वृता ये वेदपारगाः ।।
ब्रह्मविष्णुमहेशेन निर्मिता ये पुरा द्विजाः ।। १६ ।।
कृते त्रेतायुगे चैव धर्मारण्यनिवासिनः ।।
विप्रांस्तान्वै वृणुष्व त्वं तैरेव साधकोऽध्वरः ।। १७ ।।
तच्छ्रुत्वा रामदेवेन आहूता ब्राह्मणास्तदा ।।
स्थापिताश्च यथापूर्वमस्मिन्मोहे रके पुरे ।। १८ ।।
तैस्त्वष्टादशसङ्ख्याकैस्त्रैविद्यैर्मेहिवाडवैः ।।
यज्ञं चकार विधिवत्तैरेवायतबुद्धिभिः ।। १९ ।।
कुशिकः कौशिको वत्स उपमन्युश्च काश्यपः।।
कृष्णात्रेयो भरद्वाजो धारिणः शौनको वरः।।3.2.35.२०।।
माण्डव्यो भार्गवः पैङ्ग्यो वात्स्यो लौगाक्ष एव च।।
गाङ्गायनोथ गाङ्गेयः शुनकः शौनकस्तथा।। २१।।
।।ब्रह्मोवाच।।
एभिर्विप्रैः क्रतुं रामः समाप्य विधिवन्नृपः।।
चकारावभृथं रामो विप्रान्संपूज्य भक्तितः।।२२।।
यज्ञान्ते सीतया रामो विज्ञप्तः सुविनीतया।।
अस्याध्वरस्य संपत्ती दक्षिणां देहि सुव्रत।।२३।।
मन्नाम्ना च पुरं तत्र स्थाप्यतां शीघ्रमेव च ।।
सीताया वचनं श्रुत्वा तथा चक्रे नृपोत्तमः ।। २४ ।।
तेषां च ब्राह्मणानां च स्थानमेकं सुनिर्भयम् ।।
दत्तं रामेण सीतायाः सन्तोषाय महीभृता ।। २५ ।।
सीतापुरमिति ख्यातं नाम चक्रे तदा किल ।।
तस्याधिदेव्यौ वर्त्तेते शान्ता चैव सुमङ्गला ।। २६ ।।
मोहेरकस्य पुरतो ग्रामद्वादशकं पुरः ।।
ददौ विप्राय विदुषे समुत्थाय प्रहर्षितः ।। २७ ।।
तीर्थान्तरं जगामाशु काश्यपीसरितस्तटे ।।
वाडवाः केऽपि नीतास्ते रामेण सह धर्मवित् ।। २८ ।।
धर्मालये गतः सद्यो यत्र माला कमण्डलुः ।।
पुरा धर्मेण सुमहत्कृतं यत्र तपो मुने ।। २९ ।।
तदारभ्य सुविख्यातं धर्मालयमिति
श्रुतम् ।। ददौ दाशरथिस्तत्र महादानानि षोडश । 3.2.35.३० ।।
पञ्चाशत्तदा ग्रामाः सीतापुरसमन्विताः ।।
सत्यमन्दिरपर्यन्ता रघुना थेन वै तदा ।। ३१ ।।
सीताया वचनात्तत्र गुरुवाक्येन चैव हि ।।
आत्मनो वंशवृद्ध्यर्थं द्विजेभ्योऽदाद्रघूत्तमः ।। ३२ ।।
अष्टादशसहस्राणां द्विजानामभवत्कुलम् ।।
वात्स्यायन उपमन्युर्जातूकर्ण्योऽथ पिङ्गलः ।। ३३ ।।
भारद्वाजस्तथा वत्सः कौशिकः कुश एव च ।।
शाण्डिल्यः कश्यपश्चैव गौतमश्छान्धनस्तथा ।। ३४ ।।
कृष्णात्रेयस्तथा वत्सो वसिष्ठो धारणस्तथा ।।
भाण्डिलश्चैव विज्ञेयो यौवनाश्वस्ततः परम् ।। ३५ ।।
कृष्णायनोपमन्यू च गार्ग्यमुद्गलमौखकाः ।।
पुशिः पराशरश्चैव कौण्डिन्यश्च ततः परम् ।। ३६ ।।
पञ्चपञ्चाशद्ग्रामाणां नामान्येवं यथाक्रमम् ।।
सीतापुरं श्रीक्षेत्रं च मुशली मुद्गली तथा ।। ३७ ।।
ज्येष्ठला श्रेयस्थानं च दन्ताली वटपत्रका ।।
राज्ञः पुरं कृष्णवाटं देहं लोहं चनस्थनम् ।। ३८ ।।
कोहेचं चन्दनक्षेत्रं थलं च हस्तिनापुरम् ।।
कर्पटं कन्नजह्नवी वनोडफनफावली ।। ३९ ।।
मोहोधं शमोहोरली गोविन्दणं थलत्यजम् ।।
चारणसिद्धं सोद्गीत्राभाज्यजं वटमालिका ।।3.2.35.४०।।
गोधरं मारणजं चैव मात्रमध्यं च मातरम् ।।
बलवती गन्धवती ईआम्ली च राज्यजम् ।।४१।।
रूपावली बहुधनं छत्रीटं वंशञ्जं तथा ।।
जायासंरणं गोतिकी च चित्रलेखं तथैव च ।। ४२ ।।
दुग्धावली हंसावली च वैहोलं चैल्लजं तथा ।।
नालावली आसावली सुहाली कामतः परम् ।।४३।।
रामेण पञ्चपञ्चाशद्ग्रामाणि वसनाय च ।।
स्वयं निर्माय दत्तानि द्विजेभ्यस्तेभ्य एव च ।। ४४ ।।
तेषां शुश्रूषणार्थाय वैश्यान्रामो न्यवे दयत् ।।
षट्त्रिंशच्च सहस्राणि शूद्रास्तेभ्यश्चतुर्गुणान् ।।४५।।
तेभ्यो दत्तानि दानानि गवाश्ववसनानि च ।।
हिरण्यं रजतं ताम्रं श्रद्धया परया मुदा ।। ४६ ।।
नारद उवाच ।।
अष्टादशसहस्रास्ते ब्राह्मणा वेदपारगाः ।।
कथं ते व्यभजन्ग्रामान्द्रामो()त्पन्नं तथा वसु ।।
वस्त्राद्यं भूषणाद्यं च तन्मे कथय सुव्र तम् ।।४७। ।।
ब्रह्मोवाच ।। ।।
यज्ञान्ते दक्षिणा यावत्सर्त्विग्भिः स्वीकृता सुत ।।
महादानादिकं सर्वं तेभ्य एव समर्पितम् ।।४८।।
ग्रामाः साधारणा दत्ता महास्थानानि वै तदा ।।
ये वसन्ति च यत्रैव तानि तेषां भवन्त्विति ।। ४९ ।।
वशिष्ठवचनात्तत्र ग्रामास्ते विप्रसात्कृताः ।।
रघूद्वहेन धीरेण नोद्व सन्ति यथा द्विजाः ।।3.2.35.५०।।
धान्यं तेषां प्रदत्तं हि विप्राणां चामितं वसु ।।
कृताञ्जलिस्ततो रामो ब्राह्मणानिदमब्रवीत् ।।५१।।
यथा कृतयुगे विप्रास्त्रेतायां च यथा पुरा ।।
तथा चाद्यैव वर्त्तव्यं मम राज्ये न संशयः ।। ५२ ।।
यत्किञ्चिद्धनधान्यं वा यानं वा वसनानि वा ।।
मणयः काञ्चनादींश्च हेमादींश्च तथा वसु ।। ५३ ।।
ताम्राद्यं रजतादींश्च प्रार्थयध्वं ममाधुना ।।
अधुना वा भविष्ये वाभ्यर्थनीयं यथोचितम् ।। ५४ ।।
प्रेषणीयं वाचिकं मे सर्वदा द्विजसत्तमाः ।।
यंयं कामं प्रार्थयध्वं तं तं दास्याम्यहं विभो ।। ५५।।
ततो रामः सेवकादीनादरात्प्रत्यभाषत ।।
विप्राज्ञा नोल्लङ्घनीया सेव नीया प्रयत्नतः ।। ५६ ।।
यंयं कामं प्रार्थयन्ते कारयध्वं ततस्ततः ।।
एवं नत्वा च विप्राणां सेवनं कुरुते तु यः ।। ५७ ।।
स शूद्रः स्वर्गमाप्नोति धनवान्पुत्रवान्भवेत् ।।
अन्यथा निर्धनत्वं हि लभते नात्र संशयः ।। ५८ ।।
यवनो म्लेच्छजातीयो दैत्यो वा राक्षसोपि वा।।
योत्र विघ्नं करोत्येव भस्मीभवति तत्क्षणात् ।। ५९ ।।
।। ब्रह्मोवाच ।। ।।
ततः प्रदक्षिणीकृत्य द्विजान्रामोऽतिहर्षितः ।।
प्रस्थानाभिमुखो विप्रैराशीर्भिरभिनन्दितः ।। 3.2.35.६० ।।
आसीमान्तमनुव्रज्य स्नेहव्याकुललोचनाः ।।
द्विजाः सर्वे विनिर्वृत्ता धर्मारण्ये विमोहिताः ।। ६१ ।।
एवं कृत्वा ततो रामः प्रतस्थे स्वां पुरीं प्रति ।।
काश्यपाश्चैव गर्गाश्च कृतकृत्या दृढव्रताः ।। ६२ ।।
गुर्वासनसमाविष्टाः सभार्या ससुहृत्सुताः ।।
राजधानीं तदा प्राप रामोऽयोध्यां गुणान्विताम् ।। ६३ ।।
दृष्ट्वा प्रमुदिताः सर्वे लोकाः श्रीरघुनन्दनम् ।।
ततो रामः स धर्मात्मा प्रजापालनतत्परः ।। ६४ ।।
सीतया सह धर्मात्मा राज्यं कुर्वंस्तदा सुधीः ।।
जानक्या गर्भमाधत्त रविवंशोद्भवाय च ।। ६५ ।।
इति श्रीस्कान्दे महापुराण एकाशीतिसाहस्र्यां संहितायां तृतीये ब्रह्मखण्डे पूर्वभागे धर्मारण्यमाहात्म्ये श्रीरामरुद्रकृतधर्मारण्यतीर्थक्षेत्रजीर्णोद्धारवर्णनं नाम पञ्चत्रिंशोऽध्यायः ।। ३५ ।।

===

https://sa.wikisource.org/wiki/स्कन्दपुराणम्/खण्डः_५_(अवन्तीखण्डः)/अवन्तीस्थचतुरशीतिलिङ्गमाहात्म्यम्/अध्यायः_७९
https://www.wisdomlib.org/hinduism/book/the-skanda-purana/d/doc425723.html


।। श्रीमहादेव उवाच ।। ।।
एकोनाशीतिकं विद्धि हनुमत्केश्वरं प्रिये ।।
यस्य दर्शनमात्रेण समीहितफलं लभेत् ।। १ ।।
प्राप्तराज्यस्य रामस्य राक्षसानां वधे कृते ।।
आगता मुनयो देवि राघवं प्रतिनन्दितुम् ।। २ ।।
रामेण पूजिताः सर्वे ह्यगस्तिप्रमुखा द्विजाः ।।
प्रहृष्टमनसो विप्रा रामं वचनमब्रुवन् ।। ३ ।।
दिष्ट्या तु निहतो राम रावणः पुत्रपौत्रवान् ।।
दिष्ट्या विजयिनं त्वाऽद्य पश्यामः सह भार्यया ।। ४ ।।
हनूमता च सहितं वानरेण महात्मना ।।
दिष्ट्या पवनपुत्रेण राक्षसान्तकरेण च ।। ५ ।।
चिरं जीवतु दीर्घायुर्वानरो हनुमान्सदा ।।
अञ्जनीगर्भसंभूतो रुद्रांशो हि धरातले ।। ६ ।।
आखण्डलोऽग्निर्भगवान्यमो वै निऋतिस्तथा ।।
वरुणः पवनश्चैव धनाध्यक्षस्तथा शिवः ।।
ब्रह्मणा सहिताश्चैव दिक्पालाः पातु सर्वदा ।। ७ ।।
श्रुत्वा तेषां तु वचनं मुनीनां भावितात्मनाम् ।।
विस्मयं परमं गत्वा रामः प्राञ्जलिरब्रवीत् ।। ८ ।।
किमर्थं लक्ष्मणं त्यक्त्वा वानरोऽयं प्रशंसितः ।।
कीदृशः किंप्रभावो वा किंवीर्यः किंपराक्रमः ।। ९ ।।
अथोचुः सत्यमेवैतत्कारणं वानरोत्तमे ।।
न त्वस्य सदृशो वीर्ये विद्यते भुवनत्रये ।। 5.2.79.१० ।।
एष देव महाप्राज्ञो योजनानां शतं प्लुतः ।।
धर्षयित्वा पुरीं लङ्कां रावणान्तःपुरं गतः ।। ११ ।।
प्रादेशमात्रप्रतिमं कृतं रूपमनेन वै ।।
दृष्टा संभाषिता सीता पृष्टा विश्वासिता तथा ।। १२ ।।
सेनाग्रगा मन्त्रिपुत्राः किङ्करा रावणात्मजाः ।।
हता हनुमता तत्र ताडिता रावणालये ।। १३ ।।
भूयो बन्धविमुक्तेन संभाष्य तु दशाननम् ।।
लङ्का भस्मीकृता तेन पातकेनेव मेदिनी ।। १४ ।।
न कालस्य न शक्रस्य न विष्णोर्वेधसोऽपि वा ।।
श्रूयन्ते तानि कर्माणि यादृशानि हनूमतः ।। १५ ।। ।।।
।। राम उवाच ।। ।।
एतस्य बाहुवीर्येण लङ्का सीता च लक्ष्मणः ।।
प्राप्तो मम जयश्चैव राज्यं मित्राणि बान्धवाः ।। १६ ।।
सखायं वानरपतिर्मुक्त्वैनं हरिपुङ्गवम् ।।
प्रवृत्तिमपि को वेत्तुं जानक्याः शक्तिमान्भवेत् ।। १७ ।।
वाली किमर्थमेतेन सुग्रीवप्रियकाम्यया ।।
तदा वैरे समुत्पन्ने न दग्धस्तृणवत्कथम् ।। १८ ।।
नायं विदितवान्मन्ये हनुमानात्मनो बलम् ।।
उपेक्षितः क्लिश्यिमाने किमर्थं वानराधिपे ।। १९ ।।
एवं ब्रुवाणं रामं तु मुनयो वाक्यमब्रुवन् ।।
सत्यमेतद्रघुश्रेष्ठ यद्ब्रवीषि हनूमतः ।। 5.2.79.२० ।।
न बले विद्यते तुल्यो न गतौ न मतावपि ।।
अमोघवाक्यैः शापस्तु दत्तोऽस्य मुनिभिः पुरा ।। २१ ।।
न ज्ञातं हि बलं येन बलिना वालिमर्दने ।।
बाल्येऽप्यनेन यत्कर्म कृतं नाम महात्मना ।। २२ ।।
तन्न वर्णयितुं शक्यमेतस्य तु बलं महत् ।।
यदि श्रोतुं तवेच्छास्ति निशामय वदामहे ।। २३ ।।
असौ हि जातमात्रोऽपि बालार्क इव मूर्त्तिमान् ।।
ग्रहीतुकामो बालार्कं पुप्लावांबरमध्यतः ।। २४ ।।
तूर्णमाधावतो राम शक्रेण विदितात्मना ।।
हनुस्तेनास्य सहसा कुलिशेनैव ताडितः ।। २५ ।।
ततो गिरौ पपातैष शक्रवज्राभिताडितः ।।
पततोस्य महावेगाद्वामो हनुरभज्यत ।।
अस्मिंस्तु पतिते बाले मृतकल्पेऽशनिक्षतात् ।। २६ ।।
ततो वायुः समादाय महा कालवनं गतः ।।
लिङ्गमाराधयामास पुत्रार्थं पवनस्तदा ।। २७ ।।
स्पृष्टमात्रस्तु लिङ्गेन समुत्तस्थौ प्लवङ्गमः ।।
जलसिक्तं यथा सस्यं पुनर्जीवि तमाप्तवान् ।। २८ ।।
प्राणवन्तमिमं दृष्ट्वा पवनो हर्षितस्तदा ।।
प्रत्युवाच प्रसन्नात्मा पुत्रमादाय सत्वरम् ।। २९ ।।
स्पर्शनादस्य लिङ्गस्य मम पुत्रः समुत्थितः ।।
हनुमत्केश्वरो देवो विख्यातोऽयं भविष्यति ।। 5.2.79.३० ।।
एतस्मिन्नन्तरे शक्रः समायातः सुरैर्वृतः ।।
नीलोत्पलमयीं मालां संप्रगृह्येदमब्रवीत् ।। ३१ ।।
मत्करोत्सृष्टवज्रेण यस्मादस्यहनुर्हतः ।।
तदेष कपिशार्दूलो हनुमांस्तु भविष्यति ।। ३२ ।।
वरुणोऽस्व वरं प्रादान्नास्य मृत्युर्भविष्यति ।।
यमो दण्डादवध्यत्वमारोग्यं धनदो ददौ ।। ३३ ।।
सूर्येण च प्रभा दत्ता पवनेन गतिर्द्रुता ।।
लिङ्गेन च वरो दत्तो देवानां सन्निधौ तदा ।। ३४ ।।
आयुधानां हि सर्वेषामवध्योऽयं भविष्यति ।।
अजरश्चामरश्चैव भविष्यति न संशयः ।। ३५ ।।
अमित्रभयदो ह्येष मित्राणामभय प्रदः ।।
अजेयो भविता युद्धे लिङ्गेनोक्तं पुनःपुनः ।। ३६ ।।
शत्रोर्बलोत्सादनाय राघवप्रीतये सदा ।।
कियत्कालं बलं स्वीयं न स्मरिष्यति शापतः ।। ३७ ।।
हते तु रावणे भूयो रामस्यानुमते स्थितः ।।
विभीषणं प्रार्थयित्वा मामत्र स्थापयिष्यति ।। ३८ ।।
ततो मां त्रिदशाः सर्वे पूजयिष्यन्ति भाविताः ।।
तेनैव नाम्ना विख्यातिं पुनर्यास्यामि भूतले ।। ३९ ।।
अथ गन्धवहः पुत्रं प्रगृह्य गृहमानयत् ।।
अञ्जनायै तदाचख्यौ वरलब्धिं च लिङ्गतः ।। 5.2.79.४० ।।
एवं लिङ्गप्रभावाच्च बलवान्मारुतात्मजः ।।
स जातस्त्रिषुलोकेषु राम तस्मात्प्रशस्यते ।। ४१ ।।
पराक्रमोत्साहमति प्रतापैः सौशील्यमाधुर्यनयादिकैश्च ।।
गांभीर्यचातुर्यसुवीर्यधैर्यैर्हनूमतः कोऽभ्यधिकोऽस्ति लोके ।। ४२ ।।
ममेव विक्षोभितसागरस्य लोकान्दि धक्षोरिव पावकस्य ।।
प्रजा जिहीर्षोरिव चातकस्य हनूमतः स्थास्यति कः पुरस्तात् ।। ४३ ।।
एतद्वै कथित तुभ्यं यन्मां त्वं परि पृच्छसि ।।
हनूमतोऽस्य बालस्य कर्माण्यद्भुतविक्रम ।। ४४ ।।
दृष्टः सभाजितश्चापि राम गच्छामहे वयम् ।।
एवमुक्त्वा गताः सर्वे मुनयोऽवन्तिमण्डलम् ।।
पूजयामासुरीशानं हनुमत्केश्वरं शिवम् ।। ।।
समर्चयन्ति ये भक्त्या लिङ्गं त्रिदशपूजितम् ।।
हनुमत्केश्वरं देवं ते कृतार्थाः कलौ युगे ।। ४६ ।।
व्रजन्त्येव सुदुष्प्राप्यं ब्रह्मसायुज्यमव्ययम्।
संप्राप्य तु पुनर्जन्म लभन्ते मोक्षमव्ययम् ।। ४७ ।।
यः पश्यति नरो लिङ्गं हनुमत्केश्वरॆ प्रिय ।।
सोऽधिकं फलमाप्नोति सर्वदुःखविवर्जितः ।। ४८ ।।
सर्वलोकेषु तस्यैव गतिर्न प्रतिहन्यते ।।
दिव्येनैश्वर्ययोगेन युज्यते नात्र संशयः ।। ४९ ।।
बालसूर्यप्रतीकाशविमानेन सुवर्चसा ।।
वृतः स्त्रीणां सहस्रैस्तु स्वच्छदगमनागमः ।। 5.2.79.५० ।।
विचरत्यविचारेण सर्वलोकान्दिवौकसाम् ।।
स्पृहणीयतमः पुंसां सर्ववर्णोत्तमोधुना ।। ५१ ।।
स्वर्गाच्च्युतः प्रजायेत कुले महति रूपवान् ।।
धर्मज्ञो रुद्रभक्तश्च सर्वविद्यार्थपारगः ।।५२।।
राजा वा राजतुल्यो वा दर्शनादस्य जायते ।।
स्पर्शनात्परमं पुण्यं यजनात्परमं पदम् ।।५३।।
एष ते कथितो देवि प्रभावः पापनाशनः ।।
हनुमत्केश्वरेशस्य स्वप्नेश्वरमथो शृणु ।। ५४ ।।
इति श्रीस्कान्दे महापुराण एकाशीतिसाहस्र्यां संहितायां पञ्चम आवन्त्यखण्डे चतुरशीतिलिङ्गमाहात्म्य उमामहेश्वरसंवादे हनुमत्केश्वरमाहात्म्यवर्णनन्नामैकोनाशीतितमोऽध्यायः ।। ७९ ।। ।। छ ।।

===

https://sa.wikisource.org/wiki/स्कन्दपुराणम्/खण्डः_५_(अवन्तीखण्डः)/रेवा_खण्डम्/अध्यायः_०८३
https://www.wisdomlib.org/hinduism/book/the-skanda-purana/d/doc425812.html

अध्याय ८३

श्रीमार्कण्डेय उवाच -
ततो गच्छेन्महाराज तीर्थं परमशोभनम् ।
ब्रह्महत्याहरं प्रोक्तं रेवातटसमाश्रयम् ।
हनूमताभिधं ह्यत्र विद्यते लिङ्गमुत्तमम् ॥ ८३.१ ॥

युधिष्ठिर उवाच -
हनूमन्तेश्वरं नाम कथं जातं वदस्व मे ।
ब्रह्महत्याहरं तीर्थं रेवादक्षिणसंस्थितम् ॥ ८३.२ ॥

श्रीमार्कण्डेय उवाच -
साधु साधु महाबाहो सोमवंशविभूषण ।
गुह्याद्गुह्यतरं तीर्थं नाख्यातं कस्यचिन्मया ॥ ८३.३ ॥
तव स्नेहात्प्रवक्ष्यामि पीडितो वार्द्धकेन तु ।
पूर्वं जातं महद्युद्धं रामरावणयोरपि ॥ ८३.४ ॥
पुलस्त्यो ब्रह्मणः पुत्रो विश्रवास्तस्य वै सुतः ।
रावणस्तेन सञ्जातो दशास्यो ब्रह्मराक्षसः ॥ ८३.५ ॥
त्रैलोक्यविजयी भूतः प्रसादाच्छूलिनः स च ।
गीर्वाणा विजिताः सर्वे रामस्य गृहिणी हृता ॥ ८३.६ ॥
वारितः कुम्भकर्णेन सीतां मोचय मोचय ।
विभीषणेन वै पापो मन्दोदर्या पुनःपुनः ॥ ८३.७ ॥
त्वं जितः कार्तवीर्येण रैणुकेयेन सोऽपि च ।
स रामो रामभद्रेण तस्य सङ्ख्ये कथं जयः ॥ ८३.८ ॥

रावण उवाच -
वानरैश्च नरैरृक्षैर्वराहैश्च निरायुधैः ।
देवासुरसमूहैश्च न जितोऽहं कदाचन ॥ ८३.९ ॥

श्रीमार्कण्डेय उवाच -
सुग्रीवहनुमद्भ्यां च कुमुदेनाङ्गदेन च ।
एतैरन्यैः सहायैश्च रामचन्द्रेण वै जितः ॥ ८३.१० ॥
रामचन्द्रेण पौलस्त्यो हतः सङ्ख्ये महाबलः ।
वनं भग्नं हताः शूराः प्रभञ्जनसुतेन च ॥ ८३.११ ॥
रावणस्य सुतो जन्ये हतश्चाक्षकुमारकः ।
आयामो रक्षसां भीमः सम्पिष्टो वानरेण तु ॥ ८३.१२ ॥
एवं रामायणे वृत्ते सीतामोक्षे कृते सति ।
अयोध्यां तु गते रामे हनुमान्स महाकपिः ॥ ८३.१३ ॥
कैलासाख्यं गतः शैलं प्रणामाय महेशितुः ।
तिष्ठ तिष्ठेत्यसौ प्रोक्तो नन्दिना वानरोत्तमः ॥ ८३.१४ ॥
ब्रह्महत्यायुतस्त्वं हि राक्षसानां वधेन हि ।
भैरवस्य सभा नूनं न द्रष्टव्या त्वया कपे ॥ ८३.१५ ॥

हनुमानुवाच -
नन्दिनाथ हरं पृच्छ पातकस्योपशान्तिदम् ।
पापोऽहं प्लवगो यस्मात्सञ्जातः कारणान्तरात् ॥ ८३.१६ ॥

नन्द्युवाच -
रुद्रदेहोद्भवा किं ते न श्रुता भूतले स्थिता ।
श्रवणाज्जन्मजनितं द्विगुणं कीर्तनाद्व्रजेत् ॥ ८३.१७ ॥
त्रिंशज्जन्मार्जितं पापं नश्येद्रेवावगाहनात् ।
तस्मात्त्वं नर्मदातीरं गत्वा चर तपो महत् ॥ ८३.१८ ॥
गन्धर्वाहसुतोऽप्येवं नन्दिनोक्तं निशम्य च ।
प्रयातो नर्मदातीरमौर्व्यादक्षिणसङ्गमम् ॥ ८३.१९ ॥
दध्यौ सुदक्षिणे देवं विरूपाक्षं त्रिशूलिनम् ।
जटामुकुटसंयुक्तं व्यालयज्ञोपवीतिनम् ॥ ८३.२० ॥
भस्मोपचितसर्वाङ्गं डमरुस्वरनादितम् ।
उमार्द्धाङ्गहरं शान्तं गोनाथासनसंस्थितम् ॥ ८३.२१ ॥
वत्सरान् सुबहून् यावदुपासाञ्चक्र ईश्वरम् ।
तावत्तुष्टो महादेव आजगाम सहोमया ॥ ८३.२२ ॥
उवाच मधुरां वाणीं मेघगम्भीरनिस्वनाम् ।
साधु साध्वित्युवाचेशः कष्टं वत्स त्वया कृतम् ॥ ८३.२३ ॥
न च पूर्वं त्वया पापं कृतं रावणसङ्क्षये ।
स्वामिकार्यरतस्त्वं हि सिद्धोऽसि मम दर्शनात् ॥ ८३.२४ ॥
हनुमांश्च हरं दृष्ट्वा उमार्द्धाङ्गहरं स्थिरम् ।
साष्टाङ्गं प्रणतोऽवोचज्जय शम्भो नमोऽस्तु ते ।
जयान्धकविनाशाय जय गङ्गाशिरोधर ॥ ८३.२५ ॥
एवं स्तुतो महादेवो वरदो वाक्यमब्रवीत् ।
वरं प्रार्थय मे वत्स प्राणसम्भवसम्भव ॥ ८३.२६ ॥

श्रीहनुमानुवाच -
ब्रह्मरक्षोवधाज्जाता मम हत्या महेश्वर ।
न पापोऽहं भवेदेव युष्मत्सम्भाषणे क्षणात् ॥ ८३.२७ ॥

ईश्वर उवाच -
नर्मदातीर्थमाहात्म्याद्धर्मयोगप्रभावतः ।
मन्मूर्तिदर्शनात्पुत्र निष्पापोऽसि न संशयः ॥ ८३.२८ ॥
अन्यं च ते प्रयच्छामि वरं वानरपुङ्गव ।
उपकाराय लोकानां नामानि तव मारुते ॥ ८३.२९ ॥
हनूमानं जनिसुतो वायुपुत्रो महाबलः ।
रामेष्टः फाल्गुनो गोत्रः पिङ्गाक्षोऽमितविक्रमः ॥ ८३.३० ॥
उदधिक्रमणश्रेष्ठो दशग्रीवस्य दर्पहा ।
लक्ष्मणप्राणदाता च सीताशोकनिवर्तनः ॥ ८३.३१ ॥
इत्युक्त्वान्तर्दधे देव उमया सह शङ्करः ।
हनूमानीश्वरं तत्र स्थापयामास भक्तितः ॥ ८३.३२ ॥
आत्मयोगबलेनैव ब्रह्मचर्यप्रभावतः ।
ईश्वरस्य प्रसादेन लिङ्गं कामप्रदं हि तत् ।
अच्छेद्यमप्रतर्क्यं च विनाशोत्पत्तिवर्जितम् ॥ ८३.३३ ॥

% श्रीमार्कण्डेय उवाच -
% हनूमन्तेश्वरे पुत्र प्रत्यक्षप्रत्ययं शृणु ।
% यद्वृत्तं द्वापरस्यादौ त्रेतान्ते पाण्डुनन्दन ॥ ८३.३४ ॥
% सुपर्वा नाम भूपालो बभूव वसुधातले ।
% तस्य राज्ञः सदा सौख्यं नरा दीर्घायुषः सदा ॥ ८३.३५ ॥
% स पुत्रधनसंयुक्तश्चौरोपद्रववर्जितः ।
% शतबाहुर्बभूवास्य पुत्रो भीमपराक्रमः ॥ ८३.३६ ॥
% आसक्तोऽसौ सदा कालं पापधर्मैर्नरेश्वर ।
% अटाट्यत धरां सर्वां पर्वतांश्च वनानि च ॥ ८३.३७ ॥
% वधार्थं मृगयूथानामागतो विन्ध्यपर्वतम् ।
% तरुजातिसमाकीर्णे हस्तियूथसमाचिते ॥ ८३.३८ ॥
% सिंहचित्रकशोभाढ्ये मृगवाराहसङ्कुले ।
% क्रीडित्वा स वने राजा नर्मदामानतः क्वचित् ॥ ८३.३९ ॥
% हनूमन्तवने प्राप्तः शतक्रोशप्रमाणके ।
% चिञ्चिणीवनशोभाढ्ये कदम्बतरुसङ्कुले ॥ ८३.४० ॥
% नित्यं पालाशजम्बीरैः करञ्जखदिरैस्तथा ।
% पाटलैर्बदरैर्युक्तैः शमीतिन्दुकशोभितम् ॥ ८३.४१ ॥
% मृगयूथैः समाछन्नशिखण्डिस्वरनादितम् ।
% पारावतकसङ्घानां समन्तात्स्वरशोभितम् ॥ ८३.४२ ॥
% शरत्कालेऽरमद्राजा बहुले चाश्विनस्य सः ।
% वनमध्यं गतोऽद्राक्षीद्भ्रमन्तं पिङ्गलद्विजम् ॥ ८३.४३ ॥
% पुस्तिकाकरसंस्थं च पप्रच्छ चपलं द्विजम् ॥ ८३.४४ ॥

% शतबाहुरुवाच -
% एकाकी त्वं वने कस्माद्भ्रमसे पुस्तिकाकरः ।
% इतस्ततोऽपि सम्पश्यन् कथयस्व द्विजोत्तम ॥ ८३.४५ ॥

% ब्राह्मण उवाच -
% कान्यकुब्जात्समायातः प्रेषितो राजकन्यया ।
% अस्थिक्षेपाय वै राजन्हनूमन्तेश्वरे जले ॥ ८३.४६ ॥

% राजोवाच -
% अस्थिक्षेपो जले कस्माद्धनूमन्तेश्वरे द्विज ।
% क्रियते केन कार्येण साश्चर्यं कथ्यतां मम ॥ ८३.४७ ॥
% सुपर्वणः सुतो यानं त्यक्त्वा भूमौ प्रणम्य च ।
% कृताञ्जलिपुटो भूत्वा ब्राह्मणाय नरेश्वर ।
% समस्तं कथयामास वृत्तान्तं स्वं पुरातनम् ॥ ८३.४८ ॥

% ब्राह्मण उवाच -
% शिखण्डी नाम राजास्ति कन्यकुब्जे प्रतापवान् ।
% अपुत्रोऽसौ महीपालः कन्या जाता मनोरथैः ॥ ८३.४९ ॥
% जातिस्मरा सुचार्वङ्गी नर्मदायाः प्रभावतः ।
% पित्रा च सैकदा कन्या विवाहाय प्रजल्पिता ॥ ८३.५० ॥
% अनित्ये पुत्रि संसारे कन्यादानं ददाम्यहम् ।
% श्वःकृत्यमद्य कुर्वीत पूर्वाह्णे चापराह्णिकम् ।
% न हि प्रतीक्षते मृत्युः कृतं चास्य न चाकृतम् ॥ ८३.५१ ॥

% कन्योवाच -
% इच्छेयं यत्र काले हि तत्र देया त्वया पितुः ।
% पुत्रीवाक्यादसौ राजा विस्मितो वाक्यमब्रवीत् ॥ ८३.५२ ॥

% शिखण्ड्युवाच -
% कथ्यतां मे महाभागे साश्चर्यं भाषितं त्वया ।
% पितुर्वाक्येन सा बालोत्तमा ह्यागतान्तिकम् ॥ ८३.५३ ॥
% कथयामास यद्वृत्तं हनूमन्तेश्वरे नृप ।
% कलापिनी ह्यहं तात युता भर्त्रावसं तदा ॥ ८३.५४ ॥
% रेवौर्व्यासङ्गमन्तिस्था रेवाया दक्षिणे तटे ।
% हनूमन्तवने पुण्ये चिक्रीडाहं यदृच्छया ॥ ८३.५५ ॥
% भर्तृयुक्ता च संसुप्ता रजन्यां सरले नगे ।
% आगता लुब्धकास्तत्र क्षुधार्ता वनमुत्तमम् ॥ ८३.५६ ॥
% भर्तृयोगयुता पापैर्दृष्टाहं वधचिन्तकैः ।
% पाशबन्धं समादाय बद्धाहं स्वामिना सह ॥ ८३.५७ ॥
% ग्रीवां ते मोटयामासुः पिच्छाछोटनकं कृतम् ।
% हुताशनमुखे तैस्तु सह कान्तेन लुब्धकैः ॥ ८३.५८ ॥
% परिभर्ज्यावयोर्मांसं भक्षयित्वा यथेष्टतः ।
% सुप्ताः स्वस्थेन्द्रिया रात्रौ सा गता शर्वरी क्षयम् ॥ ८३.५९ ॥
% प्रभाते मांसशेषं च जम्बुकैर्गृध्रघातिभिः ।
% मच्छरीरोद्भवं चास्थि स्नायुमांसेन चावृतम् ॥ ८३.६० ॥
% गृहीतं घातिनैकेन चाकाशात्पतितं तदा ।
% तं मांसभक्षणं दृष्ट्वा परे पक्षिण आगताः ॥ ८३.६१ ॥
% दृष्ट्वा पक्षिसमूहं तु अस्थिखण्डं व्यसर्जयत् ।
% विहगानां समस्तानां धावतां चैव पश्यताम् ॥ ८३.६२ ॥
% पतितं नर्मदातोये हनूमन्तेश्वरे नृप ।
% मदीयमस्थिखण्डं च पतितं नर्मदाजले ॥ ८३.६३ ॥
% तस्य तीर्थस्य पुण्येन जाताहं पुत्रिका तव ।
% भूपकन्या त्वहं जाता पूर्णचन्द्रनिभानना ॥ ८३.६४ ॥
% जातिस्मरा नरेन्द्रस्य सञ्जाता भवतः कुले ।
% तस्माद्विवाहं नेच्छामि मम भर्ता नृपोत्तम ॥ ८३.६५ ॥
% विषमे वर्ततेऽद्यापि शकुन्तमृगजातिषु ।
% तस्यास्थिशेषं राजेन्द्र तस्मिंस्तीर्थे भविष्यति ॥ ८३.६६ ॥
% तत्क्षेपणार्थं वै तात प्रेषयाद्य द्विजोत्तमम् ।
% एतत्ते सर्वमाख्यातं कारणं नृपसत्तम ॥ ८३.६७ ॥
% मद्भर्ता विषमे स्थाने शकुन्तमृगजातिषु ।
% यदि प्रेषयसे तात कञ्चित्त्वं नर्मदातटे ॥ ८३.६८ ॥
% तस्याहं कथयिष्यामि स्थानैश्चिह्नैश्च लक्षितम् ।
% शिखण्डिनाप्यहं तत्र ह्याहूतो ह्यवनीपते ॥ ८३.६९ ॥
% दास्यामि विंशतिग्रामान्गच्छ त्वं नर्मदातटे ।
% प्रेषणं मे प्रतिज्ञातमलक्ष्म्या पीडितेन तु ॥ ८३.७० ॥

% कन्योवाच -
% गच्छ त्वं नर्मदां पुण्यां सर्वपापक्षयङ्करीम् ।
% आग्नेय्यां सोमनाथस्य हनूमन्तेश्वरः परः ॥ ८३.७१ ॥
% अर्धक्रोशेन रेवाया विस्तीर्णो वटपादपः ।
% करञ्जः कटहश्चैव सन्निधाने वटस्य च ॥ ८३.७२ ॥
% न्यग्रोधमूलसान्निध्ये सूक्ष्मान्यस्थीनि द्रक्ष्यसि ।
% समूह्य तानि सङ्गृह्य गच्छ रेवां द्विजोत्तम ॥ ८३.७३ ॥
% आश्विनस्यासिते पक्षे त्रिपुरारिस्तु वै तिथौ ।
% स्नाप्य त्रिशूलिनं भक्त्या रात्रौ त्वं कुरु जागरम् ॥ ८३.७४ ॥
% क्षिपेः प्रभाते तानि त्वं नाभिमात्रजलस्थितः ।
% इत्युच्चार्य द्विजश्रेष्ठ विमुक्तिस्तस्य जायताम् ॥ ८३.७५ ॥
% क्षिप्त्वास्थीनि पुनः स्नानं कर्तव्यं त्वघनाशनम् ।
% एवं कृते तु राजेन्द्र गतिस्तस्य भविष्यति ॥ ८३.७६ ॥
% कथितं कन्यया यच्च तत्सर्वं पुस्तिकाकृतम् ।
% आगतोऽहं नृपश्रेष्ठ तीर्थेऽत्र दुरितापहे ॥ ८३.७७ ॥
% सोऽभिज्ञानं ततो दृष्ट्वा नीत्वास्थीनि नरेश्वर ।
% पूर्वोक्तेन विधानेन प्राक्षिपं नार्मदा मसिपुष्पवृष्टिःऽशु साधु साध्विति पाण्डव ।
% विमानं च ततो दिव्यमागतं बर्हिणस्तदा ॥ ८३.७८ ॥
% दिव्यरूपधरो भूत्वा गतो नाके कलापवान् ।
% एवं तु प्रत्ययं दृष्ट्वा हनूमन्तेश्वरे नृप ॥ ८३.७९ ॥
% चकारानशनं विप्रः शतबाहुश्च भूपतिः ।
% शोषयामासतुस्तौ स्वमीश्वराराधने रतौ ॥ ८३.८० ॥
% ध्यायन्तौ तस्थतुर्देवं शतबाहुद्विजोत्तमौ ।
% मासार्धेन मृतो राजा शतबाहुर्महामनाः ॥ ८३.८१ ॥
% किङ्कणीजालशोभाढ्यं विमानं तत्र चागतम् ।
% साधु साधु नृपश्रेष्ठ विमानारोहणं कुरु ॥ ८३.८२ ॥

% शतबाहुरुवाच -
% नायामि स्वर्गमार्गाग्रं विप्रो यावन्न संस्थितः ।
% उपदेशप्रदो मह्यं गुरुरूपी द्विजोत्तमः ॥ ८३.८३ ॥
% अप्सरस ऊचुः ।
% लोभावृतो ह्ययं विप्रो लोभात्पापस्य सङ्ग्रहः ।
% हनूमन्तेश्वरे राजन्ये मृताः सत्त्वमास्थिताः ॥ ८३.८४ ॥
% ते यान्ति शाङ्करे लोके सर्वपापक्षयङ्करे ।
% नैव पापक्षयश्चास्य ब्राह्मणस्य नरेश्वर ॥ ८३.८५ ॥
% गृहं च गृहिणी चित्ते ब्राह्मणस्य प्रवर्तते ।
% शतबाहुस्ततो विप्रमुवाच विनयान्वितः ॥ ८३.८६ ॥
% त्यज मूलमनर्थस्य लोभमेनं द्विजोत्तम ।
% इत्युक्त्वा स्वर्ययौ राजा स्वर्गकन्यासमावृतः ॥ ८३.८७ ॥
% दिनैः कैश्चिद्गतो विप्रः स्वर्गं वैतालिकैर्वृतः ।
% बर्ही च काशीराजस्य पुत्रस्तीर्थप्रभावतः ॥ ८३.८८ ॥
% आत्मानं कन्यया दत्तं पूर्वजन्म व्यचिन्तयन् ।
% सा च तं प्रौढमालोक्य पितुराज्ञामवाप्य च ।
% स्वयंवरे स्वभर्तारं लेभे साध्वी नृपात्मजम् ॥ ८३.८९ ॥

% श्रीमार्कण्डेय उवाच -
% एतद्वृत्तान्तमभवत्तस्मिंस्तीर्थे नृपोत्तम ।
% एतस्मात्कारणान्मेध्यं तीर्थमेतत्सदा नृप ॥ ८३.९० ॥
% अष्टम्यां वा चतुर्दश्यां सर्वकालं नरेश्वर ।
% विशेषाच्चाश्विने मासि कृष्णपक्षे चतुर्दशीम् ॥ ८३.९१ ॥
% स्नापयेदीश्वरं भक्त्या क्षौद्रक्षीरेण सर्पिषा ।
% दध्ना च खण्डयुक्तेन कुशतोयेन वै पुनः ॥ ८३.९२ ॥
% श्रीखण्डेन सुगन्धेन गुण्ठयेच्च महेश्वरम् ।
% ततः सुगन्धपुष्पैश्च बिल्वपत्रैश्च पूजयेत् ॥ ८३.९३ ॥
% मुचकुन्देन कदेन जातीकाशकुशोद्भवैः ।
% उन्मत्तमुनिपुष्पौघैः पुष्पैस्तत्कालसम्भवैः ॥ ८३.९४ ॥
% अर्चयेत्परया भक्त्या हनूमन्तेश्वरं शिवम् ।
% घृतेन दापयेद्दीपं तैलेन तदभावतः ॥ ८३.९५ ॥
% श्राद्धं च कारयेत्तत्र ब्राह्मणैर्वेदपारगैः ।
% सर्वलक्षणसम्पूर्णैः कुलीनैर्गृहपालकैः ॥ ८३.९६ ॥
% तर्पयेद्ब्राह्मणान् भक्त्या वसनान्नहिरण्यतः ।
% नरकस्था दिवं यान्तु प्रोच्येति प्रणमेद्द्विजान् ॥ ८३.९७ ॥
% पतितान् वर्जयेद्विप्रान् वृषली यस्य गेहिनी ।
% स्ववृषं चापरित्यज्य वृषैरन्यैर्वृषायते ॥ ८३.९८ ॥
% वृषलीं तां विदुर्देवा न शूद्री वृषली भवेत् ।
% ब्रह्महत्या सुरापानं गुरुदारनिषेवणम् ॥ ८३.९९ ॥
% सुवर्णहरणन्यासमित्रद्रोहोद्भवं तथा ।
% नश्यते पातकं सर्वमित्येवं शङ्करोऽब्रवीत् ॥ ८३.१०० ॥

% श्रीमार्कण्डेय उवाच -
% वाक्प्रलापेन भो वत्स बहुनोक्तेन किं मया ।
% सर्वपातकसंयुक्तो दद्याद्दानं द्विजन्मने ॥ ८३.१०१ ॥
% गोदानं च प्रकर्तव्यमस्मिंस्तीर्थे विशेषतः ।
% गोदानं हि यतः पार्थ सर्वदानाधिकं स्मृतम् ॥ ८३.१०२ ॥
% सर्वदेवमया गावः सर्वे देवास्तदात्मकाः ।
% शृङ्गाग्रेषु महीपाल शक्रो वसति नित्यशः ॥ ८३.१०३ ॥
% उरः स्कन्दः शिरो ब्रह्मा ललाटे वृषभध्वजः ।
% चन्द्रार्कौ लोचने देवौ जिह्वायां च सरस्वती ॥ ८३.१०४ ॥
% मरुद्गणाः सदा साध्या यस्या दन्ता नरेश्वर ।
% हुङ्कारे चतुरो वेदान् विद्यात्साङ्गपदक्रमान् ॥ ८३.१०५ ॥
% ऋषयो रोमकूपेषु ह्यसङ्ख्यातास्तपस्विनः ।
% दण्डहस्तो महाकायः कृष्णो महिषवाहनः ॥ ८३.१०६ ॥
% यमः पृष्ठस्थितो नित्यं शुभाशुभपरीक्षकः ।
% चत्वारः सागराः पुण्याः क्षीरधाराः स्तनेषु च ॥ ८३.१०७ ॥
% विष्णुपादोद्भवा गङ्गा दर्शनात्पापनाशनी ।
% प्रस्रावे संस्थिता यस्मात्तस्माद्वन्द्या सदा बुधैः ॥ ८३.१०८ ॥
% लक्ष्मीश्च गोमये नित्यं पवित्रा सर्वमङ्गला ।
% गोमयालेपनं तस्मात्कर्तव्यं पाण्डुनन्दन ॥ ८३.१०९ ॥
% गन्धर्वाप्सरसो नागाः खुराग्रेषु व्यवस्थिताः ।
% पृथिव्यां सागरान्तायां यानि तीर्थानि भारत ।
% तानि सर्वाणि जानीयाद्गौर्गव्यं तेन पावनम् ॥ ८३.११० ॥

% युधिष्ठिर उवाच -
% सर्वदेवमयी धेनुर्गीर्वाणाद्यैरलङ्कृता ।
% एतत्कथय मे तात कस्माद्गोषु समाश्रिताः ॥ ८३.१११ ॥

% श्रीमार्कण्डेय उवाच -
% सर्वदेवमयो विष्णुर्गावो विष्णुशरीरजाः ।
% देवास्तदुभयात्तस्मात्कल्पिता विविधा जनैः ॥ ८३.११२ ॥
% श्वेता वा कपिला वापि क्षीरिणी पाण्डुनन्दन ।
% सवत्सा च सुशीला च सितवस्त्रावगुण्ठिता ॥ ८३.११३ ॥
% कांस्यदोहनिका देया स्वर्णशृङ्गी सुभूषिता ।
% हनूमन्तेश्वरस्याग्रे भक्त्या विप्राय दापयेत् ॥ ८३.११४ ॥
% नियमस्थेन सा देया स्वर्गमानन्त्यमिच्छता ।
% असमर्थाय ये दद्युर्विष्णुलोके प्रयान्ति ते ॥ ८३.११५ ॥
% असौ लोके च्युतो राजन्भूतले द्विजमन्दिरे ।
% कुशलो जायते पुत्रो गुणविद्याधनर्द्धिमान् ॥ ८३.११६ ॥
% सर्वपापहरं तीर्थं हनूमन्तेश्वरं नृप ।
% शृण्वन्विमुच्यते पापाद्वर्णसङ्करसंभवात् ॥ ८३.११७ ॥
% दूरस्थश्चिन्तयन् पश्यन्मुच्यते नात्र संशयः ॥ ८३.११८ ॥

॥ इति श्रीस्कान्दे महापुराण एकाशीतिसाहस्र्यां संहितायां पञ्चम आवन्त्यखण्डे रेवाखण्डे हनूमन्तेश्वरतीर्थमाहात्म्यवर्णनं नाम त्र्यशीतितमोऽध्यायः ॥


===

https://sa.wikisource.org/wiki/स्कन्दपुराणम्/खण्डः_६_(नागरखण्डः)/अध्यायः_०९९
https://www.wisdomlib.org/hinduism/book/the-skanda-purana/d/doc493463.html

॥ ऋषय ऊचुः ॥ ॥
यदेतद्भवता प्रोक्तं तत्र रामेण निर्मितः ॥
रामेश्वरस्तथा सीता तेन तत्र विनिर्मिता ॥ १ ॥
तथा च लक्ष्मणार्थाय निर्मितस्तेन संश्रयः ॥
एतन्महद्विरुद्धं ते प्रतिभाति वचोऽखिलम् ॥ २ ॥
त्वया सूत पुरा प्रोक्तं रामो लक्ष्मणसंयुतः ॥
सीतया सहितः प्राप्तः क्षेत्रेऽत्र प्रस्थितो वने ॥ ३ ॥
श्राद्धं कृत्वा गयाशीर्षे लक्ष्मणेन विरुद्ध्य च ॥
पुनः संप्रस्थितोऽरण्यं क्रोधाविष्टश्च तं प्रति । ४ ॥
यत्त्वयोक्तं तदा तेन निर्मितोऽत्र महेश्वरः ॥
एतच्च सर्वमाचक्ष्व सन्देहं सूतनन्दन ॥ ५ ॥
॥ सूत उवाच ॥ ॥
अत्र मे नास्ति सन्देहो युष्माकं च पुनः स्थितः ॥
ततो वक्ष्याम्यशेषेण श्रूयतां द्विजसत्तमाः ॥
एतत्क्षेत्रं पुनश्चाद्यं न क्षयं याति कुत्रचित् ॥ ६ ॥
अन्यस्मिन्दिवसे प्राप्ते स तदा रघुनन्दनः ॥
यदा विरोधमापन्नः सार्धं सौमित्रिणा सह ॥ ७ ॥
एतत्पुनर्दिनं चान्यद्यत्र तेन प्रतिष्ठितः ॥
रामेश्वरः स्वयं भक्त्या दुःखितेन महात्मना ॥ ८ ॥
॥ ऋषय ऊचुः ॥ ॥
अन्यस्मिन्दिवसे तत्र कस्मिन्काले रघूत्तमः ॥
संप्राप्तस्तस्य किं दुःखं सञ्जातं तत्प्रकीर्तय ॥ ९ ॥ ।
॥ सूत उवाच ॥ ॥
कृत्वा सीतापरित्यागं रामो राजीवलोचनः ॥
लोकापवादसन्त्रस्तस्ततो राज्यं चकार सः ॥ ६.९९.१० ॥
कृत्वा स्वर्णमयीं सीतां पत्नीं यज्ञप्रसिद्धये ॥
न स चक्रे महाभागो भार्यामन्यां कथञ्चन ॥ १॥।
दशवर्षसहस्राणि दशवर्षशतानि च ॥
ब्रह्मचर्येण चक्रे स राज्यं निहतकण्टकम् ॥ १२ ॥
ततो वर्षसहस्रान्ते प्राप्ते चैकादशे द्विजाः ॥
देवदूतः समायातो रामस्य सदनं प्रति ॥।३॥
तेनोक्तं देवराजेन प्रेषितोऽहं तवान्तिकम् ॥
तस्मात्कुरु समालोकं विजने त्वं मया सह ॥ १४ ॥
एवमुक्तस्तदा तेन दूतेन रघुनन्दनः ॥
परं रहः समासाद्य मन्त्रं चक्रे ततः परम् ॥ १-५ ॥
तस्यैवमुपविष्टस्य मन्त्रस्थाने महात्मनः ॥
बहुत्वादिष्टलोकस्य न रहस्यं प्रजायते ॥ १६ ॥
ततः कोपपरीतात्मा दूतः प्रोवाच सादरम् ॥
विहस्य जनसंसर्गं दृष्ट्वैकान्तेऽपि संस्थिते ॥ १७ ॥
यथा दंष्ट्राच्युतः सर्पो नागो वा मदवर्जितः ॥
आज्ञाहीनस्तथा राजा मानवैः परिभूयते ॥ १८ ॥
सेयं तव रघुश्रेष्ठ नाज्ञास्ति प्रतिवेद्म्यहम् ॥
शक्रालापमपि त्वं च नैकान्ते श्रोतुमर्हसि ॥ १९ ॥
तस्य तद्वचनं श्रुत्वा कोपसंरक्तलोचनः ॥
त्रिशाखां भृकुटीं कृत्वा ततः स प्राह लक्ष्मणम्॥ ६.९९.२० ॥
ममात्र सन्निविष्टस्य सहानेन प्रजल्पतः ॥
यदि कश्चिन्नरो मोहादागमिष्यति लक्ष्मण ॥
स्वहस्तेन न सन्देहः सूदयिष्यामि तं द्रुतम् ॥ २१ ॥
न हन्मि यदि तं प्राप्तमत्र मे दृष्टिगोचरम्॥
तन्मा भून्मे गतिः श्रेष्ठा धर्मिणां या प्रपद्यते ॥ ॥ २२ ॥
एवं ज्ञात्वा प्रयत्नेन त्वया भाव्यमसंशयम् ॥
राजद्वारि यथा कश्चिन्न मया वध्यतेऽधुना ॥ २३ ॥
तमोमित्येव संप्रोच्य लक्ष्मणः शुभलक्षणः ॥
राजद्वारं समासाद्य चकार विजनं ततः॥ २४ ॥
देवदूतोऽपि रामेण समं चक्रे ततः परम्॥
मन्त्रं शक्रसमादिष्टं तथान्यैः स्वर्गवासिभिः ॥ २५ ॥
॥ देवदूत उवाच ॥ ॥
त्वं रावणविनाशार्थमवतीर्णो धरातले ॥
स च व्यापादितो दुष्टः पापस्त्रैलोक्यकण्टकः ॥ २६ ॥
कृतं सर्वं महाभाग देव कृत्यं त्वयाऽधुना ॥
तस्मात्सन्तु सनाथास्ते देवाः शक्रपुरोगमाः॥२७॥
यदि ते रोचते चित्ते नोपरोधेन सांप्रतम् ॥
प्रसादं कुरु देवानां तस्मादागच्छ सत्वरम् ॥
स्वर्गलोकं परित्यज्य मर्त्यलोकं सुनिन्दितम् ॥२८॥
॥ सूत उवाच ॥ ॥
एतस्मिन्नन्तरे प्राप्तो दुर्वासा मुनिसत्तमः ॥
प्रोवाचाथ क्षुधाविष्टः क्वासौ क्वासौ रघूत्तमः ॥ २९ ॥
॥ लक्ष्मण उवाच ॥ ॥
व्यग्रः स पार्थिवश्रेष्ठो देवकार्येण केनचित् ॥
तस्मादत्रैव विप्रेन्द्र मुहूर्तं परिपालय ॥६.९९.३०॥।
यावत्सान्त्वयते रामो दूतं शक्रसमुद्भवम् ॥
ममोपरि दयां कृत्वा विनयावनतस्य हि ॥ ३१ ॥
॥ दुर्वासा उवाच ॥ ॥
यदि यास्यति नो दृष्टिं मम द्राक्स रघूत्तमः ॥
शापं दत्त्वा कुलं सर्वं तद्धक्ष्यामि न संशयः ॥ ३२ ॥
ममापि दर्शनादन्यन्न किञ्चिद्विद्यते गुरु ॥
कृत्यं लक्ष्मण यावत्त्वमन्यन्मूढ़ प्रकत्थसे ॥ ३३ ॥
तच्छ्रुत्वा लक्ष्मणश्चित्ते चिन्तयामास दुःखितः ॥
वरं मे मृत्युरेकस्य मा भूयात्कुलसङ्क्षयः ॥ ३४ ॥
एवं स निश्चयं कृत्वा ततो राममुपाद्रवत् ॥
उवाच दण्डवद्भूमौ प्रणिपत्य कृताञ्जलिः ॥ ३५ ॥
दुर्वासा मुनिशार्दूलो देव ते द्वारि तिष्ठति ॥
दर्शनार्थी क्षुधाविष्टः किं करोमि प्रशाधि माम् ॥ ३६ ॥
तस्य तद्वचनं श्रुत्वा ततो दूतमुवाच तम् ॥
गत्वेमं ब्रूहि देवेशं मम वाक्यादसंशयम् ॥
अहं संवत्सरस्यान्त आगमिष्यामि तेंऽतिके ॥ ३७ ॥
एवमुक्त्वा विसृज्याथ तं दूतं प्राह लक्ष्मणम् ॥
प्रवेशय द्रुतं वत्स तं त्वं दुर्वाससं मुनिम् ॥ ३८ ॥
ततश्चार्घ्यं च पाद्यं च गृहीत्वा सम्मुखो ययौ ॥
रामदेवः प्रहृष्टात्मा सचिवैः परिवारितः ॥ ३९ ॥
दत्त्वार्घ्यं विधिवत्तस्य प्रणिपत्य मुहुर्मुहुः ॥
प्रोवाच रामदेवोऽथ हर्षगद्गदया गिरा ॥ ६.९९.४० ॥
स्वागतं ते मुनिश्रेष्ठ भूयः सुस्वागतं च ते ॥
एतद्राज्यममी पुत्रा विभवश्च तव प्रभो ॥ ४१ ॥
कृत्वा मम प्रसादं च गृहाण मुनिसत्तम ॥
धन्योऽस्म्यनुगृहीतोऽस्मि यत्त्वं मे गृहमागतः ॥
पूज्यो लोकत्रयस्यापि निःशेषतपसान्निधिः ॥ ४२ ॥
॥ मुनिरुवाच ॥ ॥
चातुर्मास्यव्रतं कृत्वा निराहारो रघूत्तम ॥
अद्य ते भवनं प्राप्य आहारार्थं बुभुक्षितः ॥ ४३ ॥
तस्मात्त्वं यच्छ मे शीघ्रं भोजनं रघुनन्दन ॥
नान्येन कारणं किञ्चित्सन्न्यस्तस्य धनादिना ॥ ४४ ॥
ततस्तं भोजयामास श्रद्धापूतेन चेतसा ॥
स्वयमेवाग्रतः स्थित्वा मृष्टान्नैर्विविधैः शुभैः ॥ ४५ ॥
लेह्यैश्चोष्यैस्तथा चर्व्यैः खाद्यैरेव पृथग्विधैः ॥
यावदिच्छा मुनेस्तस्य तथान्नैर्विविधैरपि ॥ ४६ ॥
इति श्रीस्कान्दे महापुराण एकाशीतिसाहस्र्यां संहितायां षष्ठे नागरखण्डे हाटकेश्वरक्षेत्रमाहात्म्ये रामेश्वरस्थापनप्रस्तावे श्रीरामंप्रति दुर्वासः समागमनवृत्तान्तवर्णनन्नामैकोनशततमोऽ ध्यायः ॥ ९९ ॥ ॥ छ ॥


॥ सूत उवाच ॥ ॥
एवं भुक्त्वा स विप्रर्षिर्वाञ्छया राममन्दिरे ॥
दत्ताशीर्निर्गतः पश्चादामन्त्र्य रघुनन्दनम् ॥ १ ॥
अथ याते मुनौ तस्मिन्दुर्वाससि तदन्तिकात् ॥
लक्ष्मणः खङ्गमादाय रामदेवमुवाच ह ॥ २ ॥
एतत्खङ्गं गृहीत्वाशु मां प्रभो विनिपातय ॥
येन ते स्यादृतं वाक्यं प्रतिज्ञातं च यत्पुरा ॥ ३ ॥
ततो रामश्चिरात्स्मृत्वा तां प्रतिज्ञां स्वयं कृताम् ॥
वधार्थं संप्रविष्टस्य समीपे पुरुषस्य च ॥ ४ ॥
ततोऽतिचिन्तयामास व्याकुलेनान्तरात्मना ॥
बाष्पव्याकुलनेत्रश्च निःष्वसन्पन्नगो यथा ॥ ५ ॥
तं दीनवदनं दृष्ट्वा निःष्वसन्तं मुहुर्मुहुः ॥
भूयः प्रोवाच सौमित्रिर्विनयावनतः स्थितः ॥ ६ ॥
एष एव परो धर्मो भूपतीनां विशेषतः ॥
यथात्मीयं वचस्तथ्यं क्रियते निर्विकल्पितम् ॥ ७ ॥
तस्मात्त्वया प्रभो प्रोक्तं स्वयमेव ममाग्रतः ॥
तस्यैव देवदूतस्य तारनादेन कोपतः ॥ ९ ॥
योऽत्रागच्छति सौमित्रे मम दूतस्य सन्निधौ ॥
तं चेद्धन्मि स्वहस्तेन नाहं तस्मात्सुपापकृत् ॥ ९ ॥
तदहं चागतस्तात भयाद्दुर्वाससो मुनेः ॥
निषिद्धोऽपि त्वयातीव तस्माच्छीघ्रं तु घातय ॥ ६.१००.१० ॥
ततः संमन्त्र्य सुचिरं मन्त्रिभिः सहितो नृपः ॥
ब्राह्मणैर्धर्मशास्त्रज्ञैस्तथान्यैर्वेदपारगैः ॥ ११ ॥
प्रोवाच लक्ष्मणं पश्चाद्विनयावनतं स्थितम् ॥
वाष्पक्लिन्नमुखो रामो गद्गदं निःश्वसन्मुहुः ॥ १२ ॥
व्रज लक्ष्मण मुक्तस्त्वं मया देशातरं द्रुतम् ॥
त्यागो वाथ वधो वाथ साधूनामुभयं समम् ॥ १३ ॥
न मया दर्शनं भूयस्तव कार्यं कथञ्चन ॥
न स्थातव्यं च देशेऽपि यदि मे वाञ्छसि प्रियम् ॥ १४ ॥
तस्य तद्वचनं श्रुत्वा प्रणिपत्य ततः परम् ॥
निर्ययौ नगरात्तस्मात्तत्क्षणादेव लक्ष्मणः ॥ १५ ॥
अकृत्वापि समालापं केनचिन्निजमन्दिरे ॥
मात्रा वा भार्यया वाथ सुतेन सुहृदाथवा ॥ १६ ॥
ततोऽसौ सरयूं गत्वाऽवगाह्याथ च तज्जलम् ॥
शुचिर्भूत्वा निविष्टोथ तत्तीरे विजने शुभे ॥ १७ ॥
पद्मासनं विधायाथ न्यस्यात्मानं तथात्मनि ॥
ब्रह्मद्वारेण तं पश्चात्तेजोरूपं व्यसर्जयत् ॥ १८ ॥
अथ तद्राघवो दृष्ट्वा महत्तेजो वियद्गतम् ॥
विस्मयेन समायुक्तोऽचिन्तयत्किमिदं ततः ॥ १९ ॥
अथ मर्त्ये परित्यक्ते तेजसा तेन तत्क्षणात् ॥
वैष्णवेन तुरीयेण भागेन द्विजसत्तमाः ॥ ६.१००.२० ॥
पपात भूतले कायं काष्ठलोष्टोपमं द्रुतम्॥
लक्ष्मणस्य गतश्रीकं सरय्वाः पुलिने शुभे ॥ २१ ॥
ततस्तु राघवः श्रुत्वा लक्ष्मणं गतजीवितम् ॥
पतितं सरितस्तीरे विललाप सुदुःखितः ॥ २२ ॥
स्वयं गत्वा तमुद्देशं सामात्यः ससुहृज्जनः ॥
लक्ष्मणं पतितं दृष्ट्वा करुणं पर्यदेवयत् ॥ २३ ॥
हा वत्स मां परित्यज्य किं त्वं संप्रस्थितो दिवम् ॥
प्राणेष्टं भ्रातरं श्रेष्ठं सदा तव मते स्थितम् ॥ २४ ॥
तस्मिन्नपि महारण्ये गच्छमानः पुरादहम् ॥।?
अपि सन्धार्यमाणेन अनुयातस्त्वया तदा ॥ २५ ॥
संप्राप्तेऽपि कबन्धाख्ये राक्षसे बलवत्तरे ॥
त्वया रात्रिमुखे घोरे सभार्योऽहं प्ररक्षितः ॥ २६ ॥
येनेन्द्रजिद्धतो युद्धे तादृग्रूपो निशाचरः ॥
स एष पतितः शेते गतासुर्धरणीतले ॥ ९७ ॥
येन शूर्पणखा ध्वस्ता राक्षसी सा च दारुणा ॥
लीलयापि ममादेशात्सोयमेवंविधः स्थितः ॥ २८ ॥
यद्बाहुबलमाश्रित्य मया ध्वस्ता निशाचराः ॥
सोऽयं निपतितः शेते मम भ्राता ह्यनाथवत् ॥।? ॥ २९ ॥
हा वत्स क्व गतो मां त्वं विमुच्य भ्रातरं निजम् ॥
ज्येष्ठं प्राणसमं किं ते स्नेहोऽद्य विगतः क्वचित् ॥ ६.१००.३० ॥
॥ सूत उवाच ॥ ॥
एवं बहुविधान्कृत्वा प्रलापान्रघुनन्दनः ॥
मातृभिः सहितो दीनः शोकेन महतान्वितः ॥ ३१ ॥
ततस्ते मन्त्रिणस्तस्य प्रोचुस्तं वीक्ष्य दुःखितम् ॥।
विलपन्तं रघुश्रेष्ठं स्त्रीजनेन समन्वितम् ॥ ३२ ॥
॥ मन्त्रिण ऊचुः ॥ ॥
मा शोकं कुरु राजेन्द्र यथान्यः प्राकृतः स्थितः ॥
कुरुष्व च यथेदं स्यात्सांप्रतं चौर्ध्वदैहिकम्॥ ३३ ॥
नष्टं मृतमतीतं च ये शोचन्ति कुबुद्धयः ॥
धीराणां तु पुरा राजन्नष्टं नष्टं मृतं मृतम् ॥ ३४ ॥
एवं ते मन्त्रिणः प्रोच्य ततस्तस्य कलेवरम् ॥
लक्ष्मणस्य विलप्यौच्चैश्चन्दनोशीरकुङ्कुमैः ॥ ३५ ॥
कर्पूरागुरुमिश्रैश्च तथान्यैः सुसुगन्धिभिः ॥
परिवेष्ट्य शुभैर्वस्त्रैः पुष्पैः संभूष्य शोभनैः ॥ ३६ ॥
चन्दनागुरुकाष्ठैश्च चितिं कृत्वा सुविस्तराम् ॥
न्यदधुस्तस्य तद्गात्रं तत्र दक्षिणदिङ्मुखम् ॥ ३७ ॥
एतस्मिन्नन्तरे जातं तत्राश्चर्यं द्विजोत्तमाः ॥
तन्मे निगदतः सर्वं शृण्वन्तु सकलं द्विजाः ॥ ३८ ॥
यावत्तेंऽतः समारोप्य चितां तस्य कलेवरम् ॥
प्रयच्छन्ति हविर्वाहं तावन्नष्टं कलेवरम् ॥ ३९ ॥
एतस्मिन्नन्तरे वाणी निर्गता गगनाङ्गणात् ॥
नादयन्ती दिशः सर्वाः पुष्पवर्षादनन्तरम् ॥ ६.१००.४० ॥
रामराम महाबाहो मा त्वं शोकपरो भव ॥
न चास्य युज्यते वह्निर्दातुं चैव कथञ्चन ॥ ४१ ॥
ब्रह्मज्ञानप्रयुक्तस्य सन्न्यस्तस्य विशेषतः ॥
अग्निदानं न युक्तं स्यात्सर्वेषामपि योगिनाम् ॥ ४२ ॥
तवायं बान्धवो राम ब्रह्मणः सदनं गतः।
ब्रह्मद्वारेण चात्मानं निष्क्रम्य सुमहायशाः ॥ ४३ ॥
अथ ते मन्त्रिणः प्रोचुस्तच्छ्रुत्वाऽऽकाशगं वचः ॥
अशोच्यो यं महाराज संसिद्धिं परमां गतः ॥
लक्ष्मणो गम्यतां शीघ्रं तस्मात्स्वभवने विभो ॥ ४४ ॥
चिन्त्यन्तां राजकार्याणि तथा यच्चौर्ध्वदैहिकम् ॥
कुरु स्नेहोचितं तस्य पृष्ट्वा ब्राह्मणसत्तमान् ॥ ४५ ॥
॥ राम उवाच ॥
नाहं गृहं गमिष्यामि लक्ष्मणेन विनाऽधुना ॥
प्राणानत्र विहास्यामि यथा तेन महात्मना ॥ ४६ ॥
एष पुत्रो मया दत्तः कुशाख्यो मम संमतः ॥
युष्मभ्यं क्रियतां राज्ये मदीये यदि रोचते ॥ ४७ ॥
एवमुक्त्वा ततो रामो गन्तुकामो दिवालयम् ॥
चिन्तयामास भूयोऽपि स्मृत्वा मित्रं विभीषणम् ॥ ४८ ॥
मया तस्य तदा दत्तं लङ्कायां राज्यमक्षयम् ॥
बहुभक्तिप्रतुष्टेन यावच्चन्द्रार्कतारकाः ॥ ४९ ॥
अतिक्रूरतरा जाती राक्षसानां यतः स्मृता ॥
विशेषाद्वरपुष्टानां जायतेऽत्र धरातले ॥ ६.१००.५० ॥
तच्चेद्राक्षसभावेन स महात्मा विभीषणः ॥
करिष्यति सुरैः सार्धं विरोधं रावणो यथा ॥ ५१ ॥
तं देवाः सूदयिष्यन्ति उपायैः सामपूर्वकैः ॥
त्रैलोक्यकण्टको यद्वत्तस्य भ्राता दशाननः ॥ ५२ ॥
ततो मे स्यान्मृषा वाणी तस्माद्गत्वा तदन्तिकम् ॥
शिक्षां ददामि तस्याहं यथा देवान्न दूषयेत् ॥ ५३ ॥
तथा मे परमं मित्रं द्वितीयं वानरः स्थितः ॥
सुग्रीवाख्यो महाभागो जांबवांश्च तथाऽपरः ॥ ५४ ॥
सभृत्यो वायुपुत्रश्च वालिपुत्रसमन्वितः ॥
कुमुदाख्यश्च तारश्च तथान्येऽपि च वानराः ॥५५॥
तस्मात्तानपि संभाष्य सर्वान्संमन्त्र्य सादरम् ॥
ततो गच्छामि देवानां कृतकृत्यो गृहं प्रति ॥ ५६ ॥
एवं सञ्चिन्त्य सुचिरं समाहूय च पुष्पकम् ॥
तत्रारुह्य ययौ तूर्णं किष्किन्धाख्यां पुरीं प्रति ॥५७॥
अथ ते वानरा दृष्ट्वा प्रोद्द्योतं पुष्पकोद्भवम् ॥
विज्ञाय राघवं प्राप्तं सत्वरं सम्मुखा ययुः ॥ ५८ ॥
ततः प्रणम्य ते दूराज्जानुभ्यामवनिं गताः ॥
जयेति शब्दमादाय मुहुर्मुहुरितस्ततः ॥५९॥
ततस्तेनैव संयुक्ताः किष्किन्धां तां महापुरीम् ॥
विविशुः सत्पताकाभिः समन्तात्समलङ्कृताम् ॥ ६.१००.६०
अथोत्तीर्य विमानाग्र्यात्सुग्रीवभवने शुभे ॥
प्रविवेश द्रुतं रामः सर्वतः सुविभूषिते ॥ ६१ ॥
तत्र रामं निविष्टं ते विश्रान्तं वीक्ष्य वानराः॥
अर्घ्यादिभिश्च संपूज्य पप्रच्छुस्तदनन्तरम् ॥६२॥
॥ वानरा ऊचुः ॥ ॥
तेजसा त्वं विनिर्मुक्तो दृश्यसे रघुनन्दन॥
कृशोऽस्यतीव चोद्विग्नः कच्चित्क्षेमं गृहे तव ॥६३॥
काये वाऽनुगतो नित्यं तथा ते लक्ष्मणोऽनुजः॥
न दृश्यते समीपस्थः किमद्य तव राघव ॥६४॥
तथा प्राणसमाऽभीष्टा सीता तव प्रभो ॥
दृश्यते किं न पार्श्वस्था एतन्नः कौतुकं परम् ॥६५॥
॥सूत उवाच॥
तेषां तद्वचनं श्रुत्वा चिरं निःश्वस्य राघवः ॥
वाष्पपूर्णेक्षणो भूत्वा सर्वं तेषां न्यवेदयत् ॥६६॥
अथ सीता परित्यक्ता तथा भ्राता स लक्ष्मणः॥
यदर्थं तत्र संप्राप्तः स्वयमेव द्विजोत्तमाः ॥६७॥
तच्छ्रुत्वा वानराः सर्वे सुग्रीवप्रमुखास्ततः ॥
रुरुदुस्ते सुदुःखार्ताः समालिङ्ग्य ततः परम्॥६८॥
एवं चिरं प्रलप्योच्चैस्ततः प्रोचू रघूत्तमम्॥
आदेशो दीयतां राजन्योऽस्माभिरिह सिध्यति ॥६९॥
धन्या वयं धरापृष्ठे येषां त्वं रघुसत्तम ॥
ईदृक्स्नेहसमायुक्तः समागच्छसि मन्दिरे ॥ ६.१००.७० ॥
॥ राम उवाच ॥ ॥
उषित्वा रजनीमेकां सुग्रीव तव मन्दिरे ॥
प्रातर्लङ्कां गमिष्यामि यत्रास्ते स विभीषणः ७१ ॥
प्रधानामात्ययुक्तेन त्वयापि कपिसत्तम ॥
आगन्तव्यं मया सार्धं विभीषणगृहं प्रति ॥ ७२ ॥
इति श्रीस्कान्दे महापुराण एकाशीतिसाहस्र्यां संहितायां षष्ठे नागरखण्डे हाटकेश्वरक्षेत्रमाहात्म्ये श्रीरामेश्वरस्थापन प्रस्तावे लक्ष्मणनिर्वाणोत्तरं श्रीरामस्य सुग्रीवनगरींप्रति गमनवर्णनन्नाम शततमोऽध्यायः

॥ सूत उवाच ॥ ॥
एवं तां रजनीं तत्र स उषित्वा रघूत्तमः ॥
उपास्यमानः सर्वैस्तैः सद्भक्त्या वानरोत्तमैः ॥ १ ॥
ततः प्रभाते विमले प्रोद्गते रविमण्डले ॥
कृत्वा प्राभातिकं कर्म समाहूयाथ पुष्पकम् ॥ २ ॥
सुग्रीवेण सुषेणेन तारेण कुमुदेन च ॥
अङ्गदेनाथ कुण्डेन वायुपुत्रेण धीमता ॥ ३ ॥
गवाक्षेण नलेनेव तथा जांबवतापि च ॥
दशभिर्वानरैः सार्धं समारूढः स पुष्पके ॥ ४ ॥
ततः संप्रस्थितः काले लङ्कामुद्दिश्य राघवः ॥
मनोजवेन तेनैव विमानेन सुवर्चसा ॥ ५ ॥
संप्राप्तस्तत्क्षणादेव लङ्काख्यां च महापुरीम् ॥
वीक्षयंस्तान्प्रदेशांश्च यत्र युद्धं पुराऽभवत् ॥ ६ ॥
ततो विभीषणो दृष्ट्वा प्रोद्द्योतं पुष्पकोद्भवम् ॥
रामं विज्ञाय संप्राप्तं प्रहृष्टः सम्मुखो ययौ ॥
मन्त्रिभिः सकलैः सार्धं तथा भृत्यैः सुतैरपि ॥ ७ ॥
अथ दृष्ट्वा सुदूरात्तं रामदेवं विभीषणः ॥
पपात दण्डवद्भूमौ जयशब्दमुदीरयन् ॥ ८ ॥
तथागतं परिष्वज्य सादरं स विभीषणम् ॥
तेनैव सहितः पश्चाल्लङ्कां तां प्रविवेश ह ॥ ९ ॥
विभीषणगृहं प्राप्य तत्र सिंहासने शुभे ॥
निविष्टो वानरैस्तैश्च समन्तात्परिवारितः ॥ ६.१०१.१० ॥
ततो निवेदयामास तस्मै सर्वं विभीषणः ॥
राज्यं पुत्रकलत्रादि यच्चान्यदपि किञ्चन ॥ ११ ॥
ततः प्रोवाच विनयात्कृताञ्जलिपुटः स्थितः ॥
आदेशो दीयतां देव ब्रूहि कृत्यं करोमि किम् ॥ १२ ॥
अकस्मादेव संप्राप्तः किमर्थं वद मे प्रभो ॥
किं नायातः स सौमित्रिस्त्वया सार्ध च जानकी ॥।३॥
॥ सूत उवाच ॥ ॥
निवेद्य राघवस्तस्मै सर्वं गद्गदया गिरा ॥
वाष्पपूरप्रतिच्छन्नवक्त्रो भूयो विनिःश्वसन् ॥ १४ ॥
ततः प्रोवाच सत्यार्थं विभीषणकृते हितम् ॥
तं चापि शोकसन्तप्तं संबोध्य रघुनन्दनः ॥ १५ ॥
अहं राज्यं परित्यज्य सांप्रतं राक्षसोत्तम ॥
यास्यामि त्रिदिवं तूर्णं लक्ष्मणो यत्र संस्थितः ॥ ॥ १६ ॥
न तेन रहितो मर्त्ये मुहूर्तमपि चोत्सहे ॥
स्थातुं राक्षसशार्दूल बान्धवेन महात्मना ॥ १७ ॥
अहं शिक्षापणार्थाय तव प्राप्तो विभीषण ॥
तस्मादव्यग्रचित्तेन संशृणुष्व कुरुष्व च ॥ १८ ॥
एषा राज्योद्भवा लक्ष्मीर्मदं सञ्जनयेन्नृणाम् ॥
मद्यवत्स्वल्पबुद्धीनां तस्मात्कार्यो न स त्वया ॥ १९ ॥
शक्राद्या अमराः सर्वे त्वया पूज्याः सदैव हि ॥
मान्याश्च येन ते राज्यं जायते शाश्वतं सदा । ६.१०१.२० ॥
मम सत्यं भवेद्वाक्य मेतस्मादहमागतः ॥
प्राप्तराज्यप्रतिष्ठोऽपि तव भ्राता महाबलः ॥ २१ ॥
विनाशं सहसा प्राप्तस्तस्मान्मान्याः सुराः सदा ॥
यदि कश्चित्समायाति मानुषोऽत्र कथञ्चन ॥
मत्काय एव द्रष्टव्यः सर्वैरेव निशाचरैः ॥ २२ ॥
तथा निशाचराः सर्वे त्वया वार्या विभीषण ॥
मम सेतुं समुल्लङ्घ्य न गन्तव्यं धरातले ॥ २३ ॥
॥ विभीषण उवाच ॥ ॥
एवं विभो करिष्यामि तवादेशमसंशयम् ॥
परं त्वया परित्यक्ते मर्त्ये मे जीवितं व्रजेत् ॥ २४ ॥
तस्मान्मामपि तत्रैव त्वं विभो नेतुमर्हसि ॥
आत्मना सह यत्रास्ते प्राग्गतो लक्ष्मणस्तव ॥ २५ ॥
॥ श्रीराम उवाच ॥ ॥
मया तेऽक्षयमादिष्टं राज्यं राक्षससत्तम ॥
तस्मान्नार्हसि मां कर्तुं मिथ्याचारं कथञ्चन ॥ २६ ॥
अहमस्मिन्स्वके सेतौ शङ्करत्रितयं शुभम् ॥
स्थापयिष्यामि कीर्त्यर्थं तत्पूज्यं भवता सदा ॥
भक्तिमान्प्रतिसन्धाय यावच्चन्द्रार्कतारकम् ॥ २७ ॥
एवमुक्त्वा रघुश्रेष्ठो राक्षसेन्द्रं विभीषणम् ॥
दशरात्रं तत्र तस्थौ लङ्कायां वानरैः सह ॥ २८ ॥
कुर्वन्युद्धकथाश्चित्रा याः कृताः पूर्वमेव हि ॥
पश्यन्युद्धस्य सर्वाणि स्थानानि विविधानि च ॥ २९ ॥
शंसमानः प्रवीरांस्तान्राक्षसान्बलवत्तरान् ॥
कुम्भकर्णेन्द्रजित्पूर्वान्सङ्ख्ये चाभिमुखागतान् ॥ ६.१०१.३० ॥
ततश्चैकादशे प्राप्ते दिवसे रघुनन्दनः ॥
पुष्पकं तत्समारुह्य प्रस्थितः स्वपुरीं प्रति ॥ ३१ ॥
वानरैस्तैः समोपेतो विभीषणपुरःसरः ॥
ततः संस्थापयामास सेतुप्रान्ते महेश्वरम् ॥ ३२
मध्ये चैव तथादौ च श्रद्धापूतेन चेतसा ॥
रामेश्वरत्रयं राम एवं तत्र विधाय सः ॥ ३३ ॥
सेतुबन्धं तथासाद्य प्रस्थितः स्वगृहं प्रति ॥
तावद्विभीषणेनोक्तः प्रणिपत्य मुहुर्मुहुः ॥ ३४ ॥
॥ विभीषण उवाच ॥ ॥
अनेन सेतुमार्गेण रामेश्वरदिदृक्षया ॥
मानवा आगमिष्यन्ति कौतुकाच्छ्रद्धयाविताः ॥ ३५ ॥
राक्षसानां महाराज जातिः क्रूरतमा मता ॥
दृष्ट्वा मानुषमायान्तं मांसस्येच्छा प्रजायते ॥ ३६ ॥
यदा कश्चिज्जनं कश्चिद्राक्षसो भक्षयिष्यति ।
आज्ञाभङ्गो ध्रुवं भावी मम भक्तिरतस्य च ॥ ३७ ॥
भविष्यन्ति कलौ काले दरिद्रा नृपमानवाः ॥
तेऽत्र स्वर्णस्य लोभेन देवतादर्शनाय च ॥ ३८ ॥
नित्यं चैवागमिष्यन्ति त्यक्त्वा रक्षःकृतं भयम् ।
तेषां यदि वधं कश्चिद्राक्षसात्प्रापयिष्यति ॥ ३९ ॥
भविष्यति च मे दोषः प्रभुद्रोहोद्भवः प्रभो ॥
तस्मात्कञ्चिदुपायं त्वं चिन्तयस्व यथा मम ।
आज्ञाभङ्गकृतं पापं जायते न गुरो क्वचित् ॥ ६.१०१.४० ॥
तस्य तद्वचनं श्रुत्वा ततः स रघुसत्तमः ॥
बाढमित्येव चोक्त्वाथ चापं सज्जीचकार सः ॥। ॥ ४१ ॥
ततस्तं कीर्तिरूपं च मध्यदेशे रघूत्तमः॥
अच्छिनन्निशितैर्बाणैर्दशयोजनविस्तृतम्॥ ४२ ॥
तेन संस्थापितो यत्र शिखरे शङ्करः स्वयम्॥।
शिखरं तत्सलिङ्गं च पतितं वारिधेर्जले ॥ ४३ ॥
एवं मार्गमगम्यं तं कृत्वा सेतुसमुद्भवम् ॥
वानरै राक्षसैः सार्धं ततः संप्रस्थितो गृहम् ॥ ४४ ॥।
इति श्रीस्कान्दे महापुराणएकाशीतिसाहस्र्यां संहितायां षष्ठे नागरखण्डे हाटकेश्वरक्षेत्रमाहात्म्ये सेतुमध्ये श्रीरामकृतरामेश्वरप्रतिष्ठावर्णनन्नामैको त्तरशततमोऽध्यायः ॥ १०१ ॥

॥ सूत उवाच ॥ ॥
संप्रस्थितस्य रामस्य स्वकीयं सदनं प्रति ॥
यदाश्चर्यमभून्मार्गे श्रूयतां द्विजसत्तमाः ॥ १ ॥
नभोमार्गेण गच्छत्तद्विमानं पुष्पकं द्विजाः ॥
अकस्मादेव सञ्जातं निश्चलं चित्रकृन्नृणाम् ॥ २ ॥
अथ तन्निश्चलं दृष्ट्वा पुष्पकं गगनाङ्गणे ॥
रामो वायुसुतस्येदं वचनं प्राह विस्मयात् ॥ ३
त्वं गत्वा मारुते शीघ्रं भूमिं जानीहि कारणम् ॥
किमेतत्पुष्पकं व्योम्नि निश्चलत्वमुपागतम् ॥ ४ ॥
कदाचिद्धार्यते नास्य गतिः कुत्रापि केनचित् ॥
ब्रह्मदृष्टिप्रसूतस्य पुष्पकस्य महात्मनः ॥ ५ ॥
बाढमित्येव स प्रोच्य हनूमान्धरणीतलम् ॥
गत्वा शीघ्रं पुनः प्राह प्रणिपत्य रघूत्तमम् ॥ ६ ॥
अत्रास्याधः शुभं क्षेत्रं हाटकेश्वर संज्ञितम् ॥
यत्र साक्षाज्जगत्कर्ता स्वयं ब्रह्मा व्यवस्थितः ॥ ७ ॥
आदित्या वसवो रुद्रा देववैद्यौ तथाश्विनौ ॥
तत्र तिष्ठन्ति ते सर्वे तथान्ये सिद्धकिन्नराः ॥ ८ ॥
एतस्मात्कारणान्नैतदतिक्रामति पुष्पकम् ॥
तत्क्षेत्रं निश्चलीभूतं सत्यमेतन्मयोदितम् ॥ ९ ॥
॥ सूत उवाच ॥ ॥
तस्य तद्वचनं श्रुत्वा कौतूहलसमवितः ॥
पुष्पकं प्रेरयामास तत्क्षेत्रं प्रति राघवः ॥६.१०२.॥॥
सर्वैस्तैर्वानरैः सार्धं राक्षसैश्च पृथग्विधैः ॥
अवतीर्य ततो हृष्टस्तस्मिन्क्षेत्रे समन्ततः ॥ ११ ॥
तीर्थमालोकयामास पुण्यान्यायतनानि च ॥
ततो विलोकयामास पितामहविनिर्मिताम् ॥
चामुण्डां तत्र च स्नात्वा कुण्डे कामप्रदायिनि ॥ १२ ॥
ततो विलोकयामास पित्रा तस्य विनिर्मितम्॥
रामः स्वमिव देवेशं दृष्ट्वा देवं चतुर्भुजम्॥ ॥ १३ ॥
राजवाप्यां शुचिर्भूत्वा स्नात्वा तर्प्य निजान्पितॄन् ॥
ततश्च चिन्तयामास क्षेत्रे त्र बहुपुण्यदे ॥ १४ ॥
लिङ्गं संस्थापयाम्येव यद्वत्तातेन केशवः ॥
तथा मे दयितो भ्राता लक्ष्मणो दिवमाश्रितः ॥ १५ ॥
यस्तस्य नामनिर्दिष्टं लिङ्गं संस्थापयाम्यहम् ॥
तं चापि मूर्तिमन्तं च सीतया सहितं शुभम् ॥
क्षेत्रे मेध्यतमे चात्र तथात्मानं दृषन्मयम् ॥ १६ ॥
एवं स निश्चयं कृत्वा प्रासादानां च पञ्चकम् ॥
स्थापयामास सद्भक्त्या रामः शस्त्रभृतां वरः ॥ १७ ॥
ततस्ते वानराः सर्वे राक्षसाश्च विशेषतः ॥
लिङ्गानि स्थापयामासुः स्वानिस्वानि पृथक्पृथक् ॥ १८ ॥
तत्रैव सुचिरं कालं स्थितास्ते श्रद्धयाऽन्विताः ॥
ततो जग्मुरयोध्यायां विमानवरमाश्रिताः ॥ १९ ॥
एतद्वः सर्वमाख्यातं यथा रामेश्वरो महान् ॥
लक्ष्मणेश्वरसंयुक्तस्तस्मिंस्तीर्थे सुशोभने ॥ ६.१०२.२० ॥
यस्तौ प्रातः समुत्थाय सदा पश्यति मानवः ॥
स कृत्स्नं फलमाप्नोति श्रुते रामायणेऽत्र यत् ॥ २१ ॥
अथाष्टम्यां चतुर्दश्यां यो रामचरितं पठेत् ॥
तदग्रे वाजिमेधस्य स कृत्स्नं लभते फलम् ॥ २२ ॥
इति श्रीस्कान्दे महापुराण एकाशीतिसाहस्र्यां संहितायां षष्ठे नागरखण्डे हाटकेश्वरक्षेत्रमाहात्म्ये श्रीरामचन्द्रेण हाटकेश्वरक्षेत्रे लक्ष्मणादिप्रासादपञ्चकनिर्माणप्रतिष्ठापनवर्णनन्नाम द्व्युत्तरशततमोऽध्यायः ॥ १०२ ॥

===

https://sa.wikisource.org/wiki/स्कन्दपुराणम्/खण्डः_७_(प्रभासखण्डः)/प्रभासक्षेत्र_माहात्म्यम्/अध्यायः_१११
https://www.wisdomlib.org/hinduism/book/the-skanda-purana/d/doc626899.html

॥ ईश्वर उवाच ॥ ॥
ततो गच्छेन्महादेवि पुष्करारण्यमुत्तमम् ॥
तस्मादीशानकोणस्थं धनुषां षष्टिभिः स्थितम् ॥ १ ॥
तत्र कुण्डं महादेवि ह्यष्टपुष्करसंज्ञितम् ॥
सर्व पापहरं देवि दुष्प्राप्यमकृतात्मभिः ॥ २ ॥
तत्र कुण्डसमीपे तु पुरा रामेशधीमता ।।
स्थापितं तन्महालिङ्गं रामेश्वर इति स्मृतम् ॥ ३ ॥
तस्य पूजनमात्रेण मुच्यते ब्रह्महत्यया ॥ ४ ॥
॥ श्रीदेव्युवाच ॥ ॥
भगवन्विस्तराद्ब्रूहि रामेश्वरसमुद्भवम् ॥
कथं तत्रागमद्रामः ससीतश्च सलक्ष्मणः ॥ ५ ॥
कथं प्रतिष्ठितं लिङ्गं पुष्करे पापतस्करे ॥
एतद्विस्तरतो ब्रूहि फलं माहात्म्यसंयुतम् ॥ ६ ॥
॥ ईश्वर उवाच ॥ ॥
चतुर्विंशयुगे रामो वसिष्ठेन पुरोधसा ॥
पुरा रावणनाशार्थं जज्ञे दशरथात्मजः ॥ ७ ॥
ततः कालान्तरे देवि ऋषिशापान्महातपाः ॥
ययौ दाशरथी रामः ससीतः सहलक्ष्मणः ॥ ८ ॥
वनवासाय निष्क्रान्तो दिव्यैर्ब्रह्मर्षिभिर्वृतः ॥
ततो यात्राप्रसङ्गेन प्रभासं क्षेत्रमागतः ॥ ९ ॥
तं देशं तु समासाद्य सुश्रान्तो निषसाद ह ॥
अस्तं गते ततः सूर्ये पर्णान्यास्तीर्य भूतले ॥ 7.1.111.१० ॥
सुष्वापाथ निशाशेषे ददृशे पितरं स्वकम् ॥
स्वप्ने दशरथं देवि सौम्यरूपं महाप्रभम् ॥ ११ ॥
प्रातरुत्थाय तत्सर्वं ब्राह्मणेभ्यो न्यवेदयत् ॥
यथा दशरथः स्वप्ने दृष्टस्तेन महात्मना ॥ १२ ॥
॥ ब्राह्मणा ऊचुः ॥ ॥
वृद्धिकामाश्च पितरो वरदास्तव राघव ॥
दर्शनं हि प्रयच्छन्ति स्वप्नान्ते हि स्ववंशजे ॥ १३ ॥
एतत्तीर्थं महापुण्यं सुगुप्तं शार्ङ्गधन्वनः ॥
पुष्करेति समाख्यातं श्राद्धमत्र प्रदीयताम् ॥ १४ ॥
नूनं दशरथो राजा तीर्थे चास्मिन्समीहते ॥
त्वया दत्तं शुभं पिण्डं ततः स दर्शनं गतः ॥ १५ ॥
॥ ईश्वर उवाच ॥ ॥
तेषां तद्वचनं श्रुत्वा रामो राजीवलोचनः ॥
निमन्त्रयामास तदा श्राद्धार्हान्ब्राह्मणाञ्छुभान् ॥ १६ ॥
अब्रवील्लक्ष्मणं पार्श्वे स्थितं विनतकन्धरम् ॥
फलार्थं व्रज सौमित्रे श्राद्धार्थं त्वरयाऽन्वितः ॥ १७ ॥
स तथेति प्रतिज्ञाय जगाम रघुनन्दनः ॥
आनयामास शीघ्रं स फलानि विविधानि च ॥ १८ ॥
बिल्वानि च कपित्थानि तिन्दुकानि च भूरिशः ॥
बदराणि करीराणि करमर्दानि च प्रिये ॥ १९ ॥
चिर्भटानि परूषाणि मातुलिङ्गानि वै तथा ॥
नालिकेराणि शुभ्राणि इङ्गुदीसंभवानि च ॥ 7.1.111.२० ॥
अथैतानि पपाचाशु सीता जनकनन्दिनी ॥
ततस्तु कुतपे काले स्नात्वा वल्कलभृच्छुचिः ॥ २१ ॥
ब्राह्मणानानयामास श्राद्धार्हान्द्विजसत्तमान् ॥
गालवो देवलो रैभ्यो यवक्रीतोऽथ पर्वतः ॥ २२ ॥
भरद्वाजो वसिष्ठश्च जावालिर्गौतमो भृगुः ॥
एते चान्ये च बहवो ब्राह्मणा वेदपारगाः ॥ २३ ॥
श्राद्धार्थं तस्य संप्राप्ता रामस्याक्लिष्टकर्मणः ॥
एतस्मिन्नेव काले तु रामः सीतामभाषत ॥ २४ ॥
एहि वैदेहि विप्राणां देहि पादावनेजनम् ॥
एतच्छ्रुत्वाऽथ सा सीता प्रविष्टा वृक्षमध्यतः ॥ २५ ॥
गुल्मैराच्छाद्य चात्मानं रामस्यादर्शने स्थिता ॥
मुहुर्मुहुर्यदा रामः सीतासीतामभाषत ॥ २६ ॥
ज्ञात्वा तां लक्ष्मणो नष्टां कोपाविष्टं च राघवम् ॥
स्वयमेव तदा चक्रे ब्राह्मणार्ह प्रतिक्रियाम् ॥ ॥ २७ ॥
अथ भुक्तेषु विप्रेषु कृत पिण्डप्रदानके ॥
आगता जानकी सीता यत्र रामो व्यवस्थितः ॥ २८ ॥
तां दृष्ट्वा परुषैर्वाक्यैर्भर्त्सयामास राघवः ॥
धिग्धिक्पापे द्विजांस्त्यक्त्वा पितृकृत्यमहोदयम् ॥
क्व गताऽसि च मां हित्वा श्राद्धकाले ह्युपस्थिते ॥ २९ ॥
॥ ईश्वर उवाच ॥ ॥
तस्य तद्वचनं श्रुत्वा भयभीता च जानकी ॥ 7.1.111.३० ॥
कृताञ्जलिपुटा भूत्वा वेपमाना ह्यभाषत ॥
मा कोपं कुरु कल्याण मा मां निर्भर्त्सय प्रभो ॥ ३१ ॥
शृणु यस्माद्विभोऽन्यत्र गता त्यक्त्वा तवान्तिकम् ॥
दृष्टस्तत्र पिता मेऽद्य तथा चैव पितामहः ॥ ३२ ॥
तस्य पूर्वतरश्चापि तथा मातामहादयः ॥
अङ्गेषु ब्राह्मणेन्द्राणामाक्रान्तास्ते पृथक्पृथक् ॥ ३३ ॥
ततो लज्जा समभवत्तत्र मे रघुनन्दन ॥
पित्रा तत्र महाबाहो मनोज्ञानि शुभानि च ॥ ३४ ॥
भक्ष्याणि भक्षितान्येव यानि वै गुणवन्ति च ॥
स कथं सुकषायाणि क्षाराणि कटुकानि च ॥
भक्षयिष्यति राजेन्द्र ततो मे दुःखमाविशत्॥ ३६ ॥
एतस्मात्कारणान्नष्टा लज्जयाऽहं रघूद्वह ॥
दृष्ट्वा श्वशुरवर्गं स्वं तस्मात्कोपं परित्यज ॥ ३६ ॥
तस्यास्तद्वचनं श्रुत्वा विस्मितो राघवोऽभवत् ॥
विशेषेण ददौ तस्मिञ्छ्राद्धं तीर्थे तु पुष्करे ॥ ३७ ॥
तत्र पुष्करसान्निध्ये दक्षिणे धनुषां त्रये ॥
लिङ्गं प्रतिष्ठयामास रामेश्वरमिति श्रुतम् ॥ ३५ ॥
यस्तं पूजयते भक्त्या गन्धपुष्पादिभिः क्रमात् ॥
स प्राप्नोति परं स्थानं य्रत्र देवो जनार्दनः ॥ ३९ ॥
किमत्र बहुनोक्तेन द्वादश्यां यत्प्रदापयेत् ॥
न तत्र परिसङ्ख्यानं त्रिषु लोकेषु विद्यते ॥ 7.1.111.४० ॥
शुक्राङ्गारकसंयुक्ता चतुर्थी या भवेत्क्वचित् ॥
षष्ठी वात्र वरारोहे तत्र श्राद्धे महत्फलम् ॥ ४१ ॥
यावद्द्वादशवर्षाणि पितरश्च पितामहाः ॥
तर्पिता नान्यमिच्छन्ति पुष्करे स्वकुलोद्भवे ॥ ४२ ॥
तत्र यो वाजिनं दद्यात्सम्यग्भक्तिसमन्वितः ॥
अश्वमेधस्य यज्ञस्य फलं प्राप्नोति मानवः ॥ ४३ ॥
इति ते कथितं सम्यङ्माहात्म्यं पापनाशनम् ॥
रामेश्वरस्य देवस्य पुष्करस्य च भामिनि ॥ ४४ ॥
इति श्रीस्कान्दे महापुराण एकाशीतिसाहस्र्यां संहितायां सप्तमे प्रभासखण्डे प्रथमे प्रभासक्षेत्रमाहात्म्ये पुष्करमाहात्म्ये रामेश्वरक्षेत्रमाहात्म्यवर्णनन्नामैकादशोत्तरशततमोऽध्यायः ॥ १११ ॥ ॥

===

https://sa.wikisource.org/wiki/स्कन्दपुराणम्/खण्डः_७_(प्रभासखण्डः)/प्रभासक्षेत्र_माहात्म्यम्/अध्यायः_११२
https://www.wisdomlib.org/hinduism/book/the-skanda-purana/d/doc626900.html


॥ ईश्वर उवाच ॥ ॥
ततो गच्छेन्महादेवि लक्ष्मणेश्वरमुत्तमम् ॥
रामेशात्पूर्वदिग्भागे धनुस्त्रिंशकसंस्थितम् ॥ १ ॥
स्थापितं लक्ष्मणेनैव तत्र यात्रागतेन वै ॥
महापापहरं देवि तल्लिङ्गं सुरपूजितम् ॥ २ ॥
यस्तं पूजयते भक्त्या नृत्यगीतादिवादनैः ॥
होमजाप्यैः समाधिस्थः स याति परमां गतिम् ॥ ३ ॥
अन्नोदकं हिरण्यं च तत्र देयं द्विजातये ॥
संपूज्य देवदेवेशं गन्धपुष्पादिभिः क्रमात् ॥ ॥ ४ ॥
माघे कृष्णचतुर्दश्यां विशेषस्तत्र पूजने ॥
स्नानं दानं जपस्तत्र भवेदक्षयकारकम् ॥ ५ ॥
इति श्रीस्कान्दे महापुराण एकाशीति साहस्र्यां संहितायां सप्तमे प्रभासखण्डे प्रथमे प्रभासक्षेत्रमाहात्म्ये रामेश्वरक्षेत्रमाहात्म्ये लक्ष्मणेश्वरमाहात्म्यवर्णनन्नाम द्वादशोत्तरशततमो ऽध्यायः ॥ ११२ ॥


===

https://sa.wikisource.org/wiki/स्कन्दपुराणम्/खण्डः_७_(प्रभासखण्डः)/प्रभासक्षेत्र_माहात्म्यम्/अध्यायः_११३
https://www.wisdomlib.org/hinduism/book/the-skanda-purana/d/doc626901.html

॥ ईश्वर उवाच ॥ ॥
ततो गच्छेन्महादेवि जानकीश्वरमुत्तमम् ॥
रामेशान्नैऋते भागे धनुस्त्रिंशकसंस्थितम् ॥ १ ॥
पापघ्नं सर्वजन्तूनां जानक्याऽऽराधितं पुरा ॥
प्रतिष्ठितं विशेषेण सम्यगाराध्यशङ्करम् ॥ २ ॥
पूर्वं तस्यैव लिङ्गस्य वसिष्ठेशेति नाम वै ॥
तत्पश्चाज्जानकीशेति त्रेतायां प्रथितं क्षितौ ॥ ३ ॥
ततः षष्टिसहस्राणि वालखिल्या महर्षयः ॥
तत्र सिद्धिमनुप्राप्तास्तेन सिद्धेश्वरेति च ॥ ॥ ४ ॥
ख्यातं कलौ महादेवि युगलिङ्गं महाप्रभम्॥
तद्दृष्ट्वा मुच्यते पापैर्दुःखदौर्भाग्यसंभवैः ॥ ५ ॥
यस्तं पूजयते भक्त्या नारी वा पुरुषोऽपि वा ॥
संस्नाप्य विधिवद्भक्त्या स मुक्तः पातकैर्भवेत् ॥ ६ ॥
स्नात्वा च पुष्करे तीर्थे यस्तल्लिगं प्रपूजयेत् ॥
नियतो नियताहारो मासमेकं निरन्तरम्॥ ७ ॥
दिनेदिने भवेत्तस्य वाजिमेधाधिकं फलम् ॥
माघे मासि तृतीयायां या नारी तं प्रपूजयेत् ॥
तदन्वयेऽपि दौर्भाग्यं दुःखं शोकश्च नो भवेत् ॥ ८ ॥
इति ते कथितं देवि माहात्म्यं पापनाशनम् ॥
श्रुतं हरति पापानि सौभाग्यं संप्रयच्छति ॥ ९ ॥
इति श्रीस्कान्दे महापुराण एकाशीतिसाहस्र्यां संहितायां सप्तमे प्रभासखण्डे प्रथमे प्रभासक्षेत्रमाहात्म्ये जानकीश्वरमाहात्म्यवर्णनन्नाम त्रयोदशोत्तरशततमोऽध्यायः ॥ ॥ ११३ ॥

===

https://sa.wikisource.org/wiki/स्कन्दपुराणम्/खण्डः_७_(प्रभासखण्डः)/प्रभासक्षेत्र_माहात्म्यम्/अध्यायः_१७१
https://www.wisdomlib.org/hinduism/book/the-skanda-purana/d/doc626959.html

॥ ईश्वर उवाच ॥ ॥
ततो गच्छेन्महादेवि देवीमेकल्लवीरिकाम् ॥
एकल्लवीरायाम्ये तु नातिदूरे व्यवस्थिताम् ॥ १ ॥
पूर्वं दशरथो योऽसौ सूर्यवंशविभूषणः ॥
प्रभासं क्षेत्रमासाद्य तपश्चक्रे सुदुश्चरम् ॥ २ ॥
लिङ्गं तत्र प्रतिष्ठाप्य तोषयामास शाङ्करम् ॥
स देवं प्रार्थयामास पुत्रं चैवामितौजसम् ॥ ॥ ३ ॥
ददौ तस्य तदा पुत्रं देवं त्रैलोक्यपूजितम् ॥
रामेति नाम यस्यासीत्त्रैलोक्ये प्रथितं यशः ॥ ४ ॥
यस्याद्यापीह गायन्ति भूर्भुवःस्वर्नि वासिनः ॥
देवदैत्यासुराः सर्वे वाल्मीक्याद्या महर्षयः ॥ ५ ॥
तल्लिङ्गस्य प्रभावेन प्राप्तं राज्ञा महद्यशः ॥
कार्तिक्यां कार्तिके मासि विधिना यस्तमर्चयेत् ॥
दीपपूजोपहारेण यशस्वी सोऽपि जायते ॥ ६ ॥
इति श्रीस्कान्दे महापुराण एकाशीतिसाहस्र्यां संहितायां सप्तमे प्रभासखण्डे प्रथमे प्रभासक्षेत्रमाहात्म्ये दशरथेश्वरमाहात्म्यवर्णनन्नामैकसप्तत्युत्तरशततमोऽध्यायः ॥ १७१ ॥

===

https://sa.wikisource.org/wiki/स्कन्दपुराणम्/खण्डः_७_(प्रभासखण्डः)/वस्त्रापथक्षेत्रमाहात्म्यम्/अध्यायः_१८
https://www.wisdomlib.org/hinduism/book/the-skanda-purana/d/doc627173.html

Vamana → Narada!

लङ्कायां रावणो राज्यं करिष्यति महाबलः ॥
त्रैलोक्यकण्टकं नाम यदासौ धारयिष्यति ॥ १८१ ॥
तदा दाशरथी रामः कौसल्यानन्दवर्द्धनः ॥
भविष्ये भ्रातृभिः सार्द्धं गमिष्ये यज्ञमण्डपे ॥ १८२ ॥
ताडकां ताडयित्वाहं सुबाहुं यज्ञमन्दिरे ॥
नीत्वा यज्ञाद्गमिष्यामि सीतायास्तु स्वयंवरे ॥ १८३ ॥
परिणेष्याभि तां सीतां भङ्क्त्वा माहेश्वरं धनुः ॥
त्यक्त्वा राज्यं गमिष्यामि वने वर्षांश्चतुर्दश ॥ १८४ ॥
सीताहरणजं दुःखं प्रथमं मे भविष्यति ॥
नासाकर्णविहीनां तां करिष्ये राक्षसीं वने ॥ १८५ ॥
चतुर्द्दशसहस्राणि त्रिशिरःखरदूषणान् ॥
द्हत्वा हनिष्ये मारीचं राक्षसं मृगरूपिणम् ॥१८६॥
हृतदारो गमिष्यामि दग्ध्वा गृध्रं जटायुषम् ॥
सुग्रीवेण समं मैत्रीं कृत्वा हत्वाऽथ वालिनम् ॥ १८७ ॥
समुद्रं बन्धयिष्यामि नलप्रमुखवानरैः ॥
लङ्कां संवेष्टयिष्यामि मारयिष्यामि राक्षसान् ॥ १८८ ॥
कुम्भकर्णं निहत्याजौ मेघनादं ततो रणे ॥
निहत्य रावणं रक्षः पश्यतां सर्वरक्षसाम् ॥ १८९ ॥
विभीषणाय दास्यामि लङ्कां देवविनिर्मिताम्॥
अयोध्यां पुनरागत्य कृत्वा राज्यमकण्टकम् ॥7.2.18.१९०॥
कालदुर्वाससोश्चित्रचरित्रेणामरावतीम् ॥
यास्येऽहं भ्रातृभिः सार्धं राज्यं पुत्रे निवेद्य च ॥१९१॥