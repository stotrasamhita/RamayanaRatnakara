\sect{त्रिंशोऽध्यायः --- रामचरित्रवर्णनम्}

\src{स्कन्दपुराणम्}{खण्डः ३ (ब्रह्मखण्डः)}{धर्मारण्य खण्डः}{अध्यायः ३०}
\vakta{}
\shrota{}
\tags{}
\notes{}
\textlink{https://sa.wikisource.org/wiki/स्कन्दपुराणम्/खण्डः_३_(ब्रह्मखण्डः)/धर्मारण्य_खण्डः/अध्यायः_३०}
\translink{https://www.wisdomlib.org/hinduism/book/the-skanda-purana/d/doc423651.html}

\storymeta




\uvacha{व्यास उवाच}

\twolineshloka
{पुरा त्रेतायुगे प्राप्ते वैष्णवांशो रघूद्वहः}
{सूर्यवंशे समुत्पन्नो रामो राजीवलोचनः}%॥ १ ॥

\twolineshloka
{स रामो लक्ष्मणश्चैव काकपक्षधरावुभौ}
{तातस्य वचनात्तौ तु विश्वामित्रमनुव्रतौ}%॥ २ ॥

\twolineshloka
{यज्ञसंरक्षणार्थाय राज्ञा दत्तौ कुमारकौ}
{धनुःशरधरौ वीरौ पितुर्वचनपालकौ}%॥ ३ ॥

\twolineshloka
{पथि प्रव्रजतो यावत्ताडकानाम राक्षसी}
{तावदागम्य पुरतस्तस्थौ वै विघ्नकारणात्}%॥ ४ ॥


\twolineshloka
{ऋषेरनुज्ञया रामस्ताडकां समघातयत्॥}
{प्रादिशच्च धनुर्वेदविद्यां रामाय गाधिजः}%॥५॥

\twolineshloka
{तस्य पादतलस्पर्शाच्छिला वासवयोगतः}
{अहल्या गौतमवधूः पुनर्जाता स्वरूपिणी}%॥ ६ ॥


\twolineshloka
{विश्वामित्रस्य यज्ञे तु सम्प्रवृत्ते रघूत्तमः}
{मारीचं च सुबाहुं च जघान परमेषुभिः}%॥७॥

\twolineshloka
{ईश्वरस्य धनुर्भग्नं जनकस्य गृहे स्थितम्}
{रामः पञ्चदशे वर्षे षड्वर्षां चैव मैथिलीम्}%॥ ८ ॥

\twolineshloka
{उपयेमे तदा राजन्रम्यां सीतामयोनिजाम्}
{कृतकृत्यस्तदा जातः सीतां सम्प्राप्य राघवः}%॥ ९ ॥

\twolineshloka
{अयोध्यामगमन्मार्गे जामदग्न्यमवेक्ष्य च}
{सङ्ग्रामोऽभूत्तदा राजन्देवानामपि दुःसहः}%॥ १० ॥

\twolineshloka
{ततो रामं पराजित्य सीतया गृहमागतः}
{ततो द्वादशवर्षाणि रेमे रामस्तया सह}%॥ ११ ॥

\twolineshloka
{एकविंशतिमे वर्षे यौवराज्यप्रदायकम्}
{राजानमथ कैकेयी वरद्वयमयाच त}%॥ १२ ॥

\twolineshloka
{तयोरेकेन रामस्तु ससीतः सहलक्ष्मणः}
{जटाधरः प्रव्रजतां वर्षाणीह चतुर्दश}%॥ १३ ॥

\twolineshloka
{भरतस्तु द्वितीयेन यौवराज्याधिपोस्तु मे}
{मन्थरावचनान्मूढा वरमेतमयाचत}%॥ १४ ॥

\twolineshloka
{जानकीलक्ष्मणसखं रामं प्राव्राजयन्नृपः}
{त्रिरात्रमुदकाहारश्चतुर्थेह्नि फलाशनः}%॥ १५ ॥

\twolineshloka
{पञ्चमे चित्रकूटे तु रामो वासमकल्पयत्}
{तदा दशरथः स्वर्गं गतो राम इति ब्रुवन्}%॥ १६ ॥

\twolineshloka
{ब्रह्मशापं तु सफलं कृत्वा स्वर्गं जगाम किम्}
{ततो भरत शत्रुघ्नौ चित्रकूटे समागतौ}%॥ १७ ॥

\twolineshloka
{स्वर्गतं पितरं राजन्रामाय विनिवेद्य च}
{सान्त्वनं भरतस्यास्य कृत्वा निवर्तनं प्रति}%॥ १८ ॥

\twolineshloka
{ततो भरत शत्रुघ्नौ नन्दिग्रामं समागतौ}
{पादुकापूजनरतौ तत्र राज्यधरावुभौ}%॥ १९ ॥

\twolineshloka
{अत्रिं दृष्ट्वा महात्मानं दण्डकारण्यमागमत}
{रक्षोगणवधारम्भे विराधे विनिपातिते}%॥ २० ॥

\threelineshloka
{अर्द्धत्रयोदशे वर्षे पञ्चवट्यामुवास ह}
{ततो विरूपयामास शूर्पणखां निशाचरीम्}
{वने विचरतरतस्य जानकीसहितस्य च}%॥ २१ ॥

\twolineshloka
{आगतो राक्षसो घोरः सीतापहरणाय सः}
{ततो माघासिताष्टम्यां मुहूर्ते वृन्दसंज्ञके}%॥ २२ ॥

\twolineshloka
{राघवाभ्यां विना सीतां जहार दश कन्धरः}
{मारीचस्याश्रमं गत्वा मृगरूपेण तेन च}%॥ २३ ॥

\twolineshloka
{नीत्वा दूरं राघवं च लक्ष्मणेन समन्वितम्}
{ततो रामो जघानाशु मारीचं मृगरू पिणम्}%॥ २४ ॥

\twolineshloka
{पुनः प्राप्याश्रमं रामो विना सीतां ददर्श ह}
{तत्रैव ह्रियमाणा सा चक्रन्द कुररी यथा}%॥ २५ ॥

\twolineshloka
{रामरामेति मां रक्ष रक्ष मां रक्षसा हृताम्}
{यथा श्येनः क्षुधायु्क्तः क्रन्दन्तीं वर्तिकां नयेत्}%॥ २६ ॥

\twolineshloka
{तथा कामवशं प्राप्तो राक्षसो जनकात्मजाम्}
{नयत्येष जनकजां तच्छ्रुत्वा पक्षिराट् तदा}%॥ २७ ॥

\twolineshloka
{युयुधे राक्षसेन्द्रेण रावणेन हतोऽपतत्}
{माघासितनवम्यां तु वसन्तीं रावणालये}%॥ २८

\onelineshloka
{मार्गमाणौ तदा तौ तु भ्रातरौ रामलक्ष्मणौ}%॥ २९ ॥

\twolineshloka
{जटायुषं तु दृष्ट्वैव ज्ञात्वा राक्षससंहृताम्}
{सीतां ज्ञात्वा ततः पक्षी संस्कृतस्तेन भक्तितः}%॥ ३० ॥

\twolineshloka
{अग्रतः प्रययौ रामो लक्ष्मणस्तत्पदानुगः}
{पम्पाभ्याशमनुप्राप्य शबरीमनुगृह्य च}%॥ ३१ ॥

\twolineshloka
{तज्जलं समुपस्पृश्य हनुमद्दर्शनं कृतम्}
{ततो रामो हनुमता सह सख्यं चकार ह}%॥ ३२ ॥

\twolineshloka
{ततः सुग्रीवमभ्येत्य अहनद्वालिवानरम्}
{प्रेषिता रामदेवेन हनुमत्प्रमुखाः प्रियाम्}%॥ ३३ ॥

\twolineshloka
{अङ्गुलीयकमादाय वायुसूनुस्तदागतः}
{सम्पातिर्दशमे मासि आचख्यौ वानराय ताम्}%॥ ३४ ॥

\twolineshloka
{ततस्तद्वचनादब्धिं पुप्लुवे शतयोजनम्}
{हनुमान्निशि तस्यां तु लङ्कायां परितोऽचिनोत्}%॥ ३५ ॥

\twolineshloka
{तद्रात्रिशेषे सीताया दर्शनं तु हनूमतः}
{द्वादश्यां शिंशपावृक्षे हनुमान्पर्यवस्थितः}%॥ ३६ ॥

\twolineshloka
{तस्यां निशायां जानक्या विश्वासायाह सङ्कथाम्}
{अक्षादिभिस्त्रयोदश्यां ततो युद्धमवर्त्तत}%॥ ३७ ॥

\twolineshloka
{ब्रह्मास्त्रेण त्रयोदश्यां बद्धः शक्रजिता कपिः}
{दारुणानि च रूक्षाणि वाक्यानि राक्षसाधिपम्}%॥ ३८ ॥

\twolineshloka
{अब्रवीद्वायुसूनुस्तं बद्धो ब्रह्मास्त्रसंयुतः}
{वह्निना पुच्छयुक्तेन लङ्काया दहनं कृतम्}%॥ ३९ ॥

\twolineshloka
{पूर्णिमायां महेन्द्राद्रौ पुनरागमनं कपेः}
{मार्गशीर्षप्रतिपदः पञ्चभिः पथि वासरैः}%॥ ४० ॥

\twolineshloka
{पुनरागत्य वर्षेह्नि ध्वस्तं मधुवनं किल}
{सप्तम्यां प्रत्यभिज्ञानदानं सर्वनिवेदनम्}%॥ ४१ ॥

\twolineshloka
{मणिप्रदानं सीतायाः सर्वं रामाय शंसयत्}
{अष्टम्युत्तरफाल्गुन्यां मुहूर्ते विजयाभिधे}%॥ ४२ ॥

\twolineshloka
{मध्यं प्राप्ते सहस्रांशौ प्रस्थानं राघवस्य च}
{रामः कृत्वा प्रतिज्ञां हि प्रयातुं दक्षिणां दिशम्}%॥ ४३ ॥

\twolineshloka
{तीर्त्वाहं सागरमपि हनिष्ये राक्षसेश्वरम्}
{दक्षिणाशां प्रयातस्य सुग्रीवोऽथाभव त्सखा}%॥ ४४ ॥

\threelineshloka
{वासरैः सप्तभिः सिन्धोस्तीरे सैन्यनिवेशनम्}
{पौषशुक्लप्रतिपदस्तृतीयां यावदम्बुधौ}
{उपस्थानं ससैन्यस्य राघवस्य बभूव ह}%॥ ४५ ॥

\twolineshloka
{विभीषणश्चतुर्थ्यां तु रामेण सह सङ्गतः}
{समुद्रतरणार्थाय पञ्चम्यां मन्त्र उद्यतेः}%॥ ४६ ॥

\twolineshloka
{प्रायोपवेशनं चक्रे रामो दिनचतुष्टयम्}
{समुद्राद्वरलाभश्च सहोपायप्रदर्शनः}%॥ ४७ ॥

\twolineshloka
{सेतोर्दशम्यामारम्भस्त्रयोदश्यां समापनम्}
{चतुर्दश्यां सुवेलाद्रौ रामः सेनां न्यवे शयत्}%॥ ४८ ॥

\twolineshloka
{पूर्णिमास्या द्वितीयायां त्रिदिनैः सैन्यतारणम्}
{तीर्त्वा तोयनिधिं रामः शूरवानरसैन्यवान्}%॥ ४९ ॥

\twolineshloka
{रुरोध च पुरीं लङ्कां सीतार्थं शुभलक्षणः}
{तृतीयादिदशम्यन्तं निवेशश्च दिनाष्टकः}%॥ ५० ॥

\twolineshloka
{शुकसारणयोस्तत्र प्राप्तिरेकादशीदिने}
{पौषासिते च द्वादश्यां सैन्यसङ्ख्यानमेव च}%॥ ५१ ॥

\twolineshloka
{शार्दूलेन कपीन्द्राणां सारासारोपवर्णनम्}
{त्रयोदश्याद्यमान्ते च लङ्कायां दिवसैस्त्रिभिः}%॥ ५२ ॥

\twolineshloka
{रावणः सैन्यसं ख्यानं रणोत्साहं तदाऽकरोत्}
{प्रययावङ्गदो दौत्ये माघशुक्लाद्यवासरे}%॥ ५३ ॥

\twolineshloka
{सीतायाश्च तदा भर्तुर्मायामूर्धादिदर्शनम्}
{माघशुक्लद्वितीया यां दिनैः सप्तभिरष्टमीम्}%॥ ५४ ॥

\twolineshloka
{रक्षसां वानराणां च युद्धमासीच्च सङ्कुलम्}
{माघशुक्लनवम्यां तु रात्राविन्द्रजिता रणे}%॥ ५५ ॥

\twolineshloka
{रामलक्ष्मणयोर्ना गपाशबन्धः कृतः किल}
{आकुलेषु कपीशेषु हताशेषु च सर्वशः}%॥ ५६ ॥

\twolineshloka
{वायूपदेशाद्गरुडं सस्मार राघवस्तदा}
{नागपाशविमोक्षार्थं दशम्यां गरु डोऽभ्यगात्}%॥ ५७ ॥

\twolineshloka
{अवहारो माघशुक्लैस्यैकादश्यां दिनद्वयम्}
{द्वादश्यामाञ्जनेयेन धूम्राक्षस्य वधः कृतः}%॥ ५८ ॥

\twolineshloka
{त्रयोदश्यां तु तेनैव निहतोऽकम्पनो रणे}
{मायासीतां दर्शयित्वा रामाय दशकन्धरः}%॥ ५९ ॥

\twolineshloka
{त्रासयामास च तदा सर्वान्सैन्यगतानपि}
{माघशुक्लचतुर्द्दश्यां यावत्कृष्णादिवासरम्}%॥ ६० ॥

\twolineshloka
{त्रिदिनेन प्रहस्तस्य नीलेन विहितो वधः}
{माघकृष्णद्वितीयायाश्चतुर्थ्यन्तं त्रिभिर्दिनैः}%॥ ६१ ॥

\twolineshloka
{रामेण तुमुले युद्धे रावणो द्रावितो रणात्}
{पञ्चम्या अष्टमी यावद्रावणेन प्रबोधितः}%॥ ६२ ॥

\twolineshloka
{कुम्भकर्णस्तदा चक्रेऽभ्यवहारं चतुर्दिनम्}
{कुम्भकर्णोकरोद्युद्धं नवम्यादिचतुर्दिनैः}%॥ ६३ ॥

\twolineshloka
{रामेण निहतो युद्धे बहुवानरभक्षकः}
{अमावास्यादिने शोकाऽभ्यवहारो बभूव ह}%॥ ६४ ॥

\twolineshloka
{फाल्गुनप्रतिपदादौ चतुर्थ्यन्तैश्चतुर्दिनैः}
{नरान्तकप्रभृतयो निहताः पञ्च राक्षसाः}%॥ ६५ ॥

\twolineshloka
{पञ्चम्याः सप्तमीं यावदतिकायवधस्त्र्यहात्}
{अष्टम्या द्वादशीं यावन्निहतो दिनपञ्चकात्}%॥ ६६ ॥

\twolineshloka
{निकुम्भकुम्भौ द्वावेतौ मकराक्षश्चतुर्दिनैः}
{फाल्गुनासितद्वितीयाया दिने वै शक्रजिज्जितः}%॥ ६७ ॥

\twolineshloka
{तृतीयादौ सप्तम्यन्तदिनपञ्चकमेव च}
{ओषध्यानयवैयग्र्यादवहारो बभूव ह}%॥ ६८ ॥

\twolineshloka
{अष्टम्यां रावणो मायामैथिलीं हतवान्कुधीः}
{शोकावेगात्तदा रामश्चक्रे सैन्यावधारणम्}%॥ ६९ ॥

\twolineshloka
{ततस्त्रयोदशीं यावद्दिनैः पञ्चभिरिन्द्रजित्}
{लक्ष्मणेन हतो युद्धे विख्यातबलपौरुषः}%॥ ७० ॥

\twolineshloka
{चतुर्द्दश्यां दशग्रीवो दीक्षामापावहारतः}
{अमावास्यादिने प्रागाद्युद्धाय दशकन्धरः}%॥ ७१ ॥

\twolineshloka
{चैत्रशुक्लप्रतिपदः पञ्चमीदिनपञ्चके}
{रावणो युध्यमानो ऽभूत्प्रचुरो रक्षसां वधः}%॥ ७२ ॥

\twolineshloka
{चैत्रशुक्लाष्टमीं यावत्स्यन्दनाश्वादिसूदनम्}
{चैत्रशुक्लनवम्यां तु सौमित्रेः शक्तिभेदने}%॥ ७३ ॥

\twolineshloka
{कोपाविष्टेन रामेण द्रावितो दशकन्धरः}
{विभीषणोपदेशेन हनुमद्युद्धमेव च}%॥ ७४ ॥

\twolineshloka
{द्रोणाद्रेरोषधीं नेतुं लक्ष्मणार्थमुपागतः}
{विशल्यां तु समादाय लक्ष्मणं तामपाययत्}%॥ ७५ ॥

\twolineshloka
{दशम्यामवहारोऽभूद्रात्रौ युद्धं तु रक्षसाम्}
{एकादश्यां तु रामाय रथो मातलिसारथिः}%॥ ७६ ॥

\twolineshloka
{प्राप्तो युद्धाय द्वादश्यां यावत्कृष्णां चतुर्दशीम्}
{अष्टादशदिने रामो रावणं द्वैरथेऽवधीत्}%॥ ७७ ॥

\twolineshloka
{संस्कारा रावणादीनाममावा स्यादिनेऽभवन्}
{सङ्ग्रामे तुमुले जाते रामो जयमवाप्तवान्}%॥ ७८ ॥

\twolineshloka
{माघशुक्लद्वितीयादिचैत्रकृष्णचतुर्द्दशीम्}
{सप्ताशीतिदिनान्येवं मध्ये पंवदशा हकम्}%॥ ७९ ॥

\threelineshloka
{युद्धावहारः सङ्ग्रामो द्वासप्ततिदिनान्यभूत्}
{वैशाखादि तिथौ राम उवास रणभूमिषु}
{अभिषिक्तो द्वितीयायां लङ्काराज्ये विभी षणः}%॥ ८० ॥

\twolineshloka
{सीताशुद्धिस्तृतीयायां देवेभ्यो वरलम्भनम्}
{दशरथस्यागमनं तत्र चैवानुमोदनम्}%॥ ८१ ॥

\twolineshloka
{हत्वा त्वरेण लङ्केशं लक्ष्मणस्याग्रजो विभुः}
{गृहीत्वा जानकीं पुण्यां दुःखितां राक्षसेन तु}%॥ ८२ ॥

\twolineshloka
{आदाय परया प्रीत्या जानकीं स न्यवर्तत}
{वैशाखस्य चतुर्थ्यां तु रामः पुष्पकमा श्रितः}%॥ ८३ ॥

\twolineshloka
{विहायसा निवृत्तस्तु भूयोऽयोध्यां पुरीं प्रति}
{पूर्णे चतुर्दशे वर्षे पञ्चम्यां माधवस्य च}%॥ ८४ ॥

\twolineshloka
{भारद्वाजाश्रमे रामः सगणः समु पाविशत्}
{नन्दिग्रामे तु षष्ठ्यां स पुष्पकेण समागतः}%॥ ८५ ॥

\twolineshloka
{सप्तम्यामभिषिक्तोऽसौ भूयोऽयोध्यायां रघूद्वहः}
{दशाहाधिकमासांश्च चतुर्दश हि मैथिली}%॥ ८५ ॥

\twolineshloka
{उवास रामरहिता रावणस्य निवेशने}
{द्वाचत्वारिंशके वर्षे रामो राज्यमकारयत्}%॥ ८७ ॥

\twolineshloka
{सीतायास्तु त्रयस्त्रिंशद्वर्षाणि तु तदा भवन्}
{स चतुर्दशवर्षान्ते प्रविष्टः स्वां पुरीं प्रभुः}%॥ ८८ ॥

\twolineshloka
{अयोध्यां नाम मुदितो रामो रावणदर्पहा}
{भ्रातृभिः सहितस्तत्र रामो राज्यमकार यत्}%॥ ८९ ॥

\twolineshloka
{दशवर्षसहस्राणि दशवर्षशतानि च}
{रामो राज्यं पालयित्वा जगाम त्रिदिवालयम्}%॥ ९० ॥

\twolineshloka
{रामराज्ये तदा लोका हर्षनिर्भरमा नसाः}
{बभूवुर्धनधान्याढ्याः पुत्रपौत्रयुता नराः}%॥ ९१ ॥

\twolineshloka
{कामवर्षी च पर्जन्यः सस्यानि गुणवन्ति च}
{गावस्तु घटदोहिन्यः पादपाश्च सदा फलाः}%॥ ९२ ॥

\twolineshloka
{नाधयो व्याधयश्चैव रामराज्ये नराधिप}
{नार्यः पतिव्रताश्चासन्पितृभक्तिपरा नराः}%॥ ९३ ॥

\twolineshloka
{द्विजा वेदपरा नित्यं क्षत्रिया द्विज सेविनः}
{कुर्वते वैश्यवर्णाश्च भक्तिं द्विजगवां सदा}%॥ ९४ ॥

\twolineshloka
{न योनिसङ्करश्चासीत्तत्र नाचारसङ्करः}
{न वन्ध्या दुर्भगा नारी काकवन्ध्या मृत प्रजा}%॥ ९५ ॥

\twolineshloka
{विधवा नैव काप्यासीत्सभर्तृका न लप्यते}
{नावज्ञां कुर्वते केपि मातापित्रोर्गुरोस्तथा}%॥ ९६ ॥

\twolineshloka
{न च वाक्यं हि वृद्धानामुल्लं घयति पुण्यकृत्}
{न भूमिहरणं तत्र परनारीपराङ्मुखाः}%॥ ९७ ॥

\twolineshloka
{नापवादपरो लोको न दरिद्रो न रोगभाक्}
{न स्तेयो द्यूतकारी च मैरेयी पापिनो नहि}%॥ ९८ ॥

\twolineshloka
{न हेमहारी ब्रह्मघ्नो न चैव गुरुतल्पगः}
{न स्त्रीघ्नो न च बालघ्नो न चैवानृतभाषणः}%॥ ९९ ॥

\twolineshloka
{न वृत्तिलोपकश्चासीत्कूट साक्षी न चैव हि}
{न शठो न कृतघ्नश्च मलिनो नैव दृश्यते}%॥ १०० ॥

\twolineshloka
{सदा सर्वत्र पूज्यन्ते ब्राह्मणा वेदपारगाः}
{नावैष्णवोऽव्रती राजन्राम राज्येऽतिविश्रुते}%॥ १०१ ॥

\twolineshloka
{राज्यं प्रकुर्वतस्तस्य पुरोधा वदतां वरः}
{वसिष्ठो मुनिभिः सार्द्धं कृत्वा तीर्थान्यनेकशः}%॥ २ ॥

\twolineshloka
{आजगाम ब्रह्मपुत्रो महाभागस्तपोनिधिः}
{रामस्तं पूजयामास मुनिभिः सहितं गुरुम्}%॥ ३ ॥

\twolineshloka
{अभ्युत्थानार्घपाद्यैश्च मधुपर्कादिपूजया}
{प्रपच्छ कुशलं रामं वसिष्ठो मुनिपुङ्गवः}%॥ ४ ॥

\twolineshloka
{राज्ये चाश्वे गजे कोशे देशे सद्भ्रातृभृत्ययोः}
{कुशलं वर्त्तते राम इति पृष्टे मुनेस्तदा}%॥ ५ ॥

\uvacha{राम उवाच}

\twolineshloka
{सर्वत्र कुशलं मेऽद्य प्रसादाद्भवतः सदा}
{पप्रच्छ कुशलं रामो वसिष्ठं मुनिपुङ्गवम्}%॥ ६ ॥

\twolineshloka
{सर्वतः कुशली त्वं हि भार्यापुत्रसमन्वितः}
{स सर्वं कथयामास यथा तीर्थान्यशेषतः}%॥ ७ ॥

\twolineshloka
{सेवितानि धरापृष्ठे क्षेत्राण्यायतनानि च}
{रामाय कथयामास सर्वत्र कुशलं तदा}%॥ ८ ॥

\twolineshloka
{ततः स विस्मयाविष्टो रामो राजीवलोचनः}
{पप्रच्छ तीर्थमाहात्म्यं यत्तीर्थेषूत्तमोत्तमम्}%॥ १०९ ॥

॥इति श्रीस्कान्दे महापुराण एकाशीतिसाहस्र्यां संहितायां तृतीये ब्रह्मखण्डे पूर्वभागे धर्मारण्यमाहात्म्ये रामचरित्रवर्णनं नाम त्रिंशोऽध्यायः॥३०॥
