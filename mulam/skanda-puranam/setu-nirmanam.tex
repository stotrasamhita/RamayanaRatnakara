\sect{द्वितीयोऽध्यायः --- सेतुनिर्माणादिवर्णनम्}

\src{स्कन्दपुराणम्}{खण्डः ३ (ब्रह्मखण्डः)}{सेतुखण्डः}{अध्यायः ०२}
\vakta{}
\shrota{}
\tags{}
\notes{}
\textlink{https://sa.wikisource.org/wiki/स्कन्दपुराणम्/खण्डः_३_(ब्रह्मखण्डः)/सेतुखण्डः/अध्यायः_०२}
\translink{https://www.wisdomlib.org/hinduism/book/the-skanda-purana/d/doc423570.html}

\storymeta




\uvacha{ऋषय ऊचुः}

\twolineshloka
{कथं सूत महाभाग रामेणाक्लिष्टकर्मणा}
{सेतुर्बद्धो नदीनाथे ह्यगाधे वरुणालये}% ॥ १ ॥

\twolineshloka
{सेतौ च कति तीर्थानि गन्धमादनपर्वते}
{एतन्नः श्रद्दधानानां ब्रूहि पौराणिकोत्तम}%॥ २ ॥

\uvacha{श्रीसूत उवाच}

\twolineshloka
{रामेण हि यथासेतुर्निबद्धो वरुणालये}
{तदहं सम्प्रवक्ष्यामि युष्माकं मुनिपुङ्गवाः}%॥ ३ ॥

\twolineshloka
{आज्ञया हि पितू रामो न्यवसद्दण्डकानने}
{सीतालक्ष्मणसंयुक्तः पञ्चवट्यां समाहितः}%॥ ४ ॥

\twolineshloka
{तस्मिन्निव सतस्तस्य राघवस्य महात्मनः}
{रावणेन हृता भार्या मारीचच्छद्मना द्विजाः}% ॥ ५ ॥

\twolineshloka
{मार्गमाणो वने भार्यां रामो दशरथात्मजः}
{पम्पातीरे जगा मासौ शोकमोहसमन्वितः}% ॥ ६ ॥

\twolineshloka
{दृष्टवान्वानरं तत्र कञ्चिद्दशरथात्मजः}
{वानरेणाथ पृष्टोऽयं को भवानिति राघवः}%॥ ७ ॥

\twolineshloka
{आदितः स्वस्य वृत्तान्त्तं तस्मै प्रोवाच तत्त्वतः}
{अथ राघवसम्पृष्टो वानरः को भवानिति}%॥ ८ ॥

\twolineshloka
{सोपि विज्ञापयामास राघवाय महात्मने}
{अहं सुग्रीवसचिवो हनूमा न्नाम वानरः}%॥ ९ ॥

\twolineshloka
{तेन च प्रेरितोऽभ्यागां युवाभ्यां सख्यमिच्छता}
{आगच्छतं तद्भद्रं वां सुग्रीवान्तिकमाशु वै}%॥ १० ॥

\twolineshloka
{तथास्त्विति स रामो पि तेन साकं मुनीश्वराः}
{सुग्रीवान्तिकमागप्य सख्यं चक्रेऽग्निसाक्षिकम्}%॥ ११ ॥

\twolineshloka
{प्रतिजज्ञेऽथ रामोऽपि तस्मै वालिवधं प्रति}
{सुग्रीवश्चापि वै देह्याः पुनरानयनं द्विजाः}%॥ १२ ॥

\twolineshloka
{इत्येवं समयं कृत्वा विश्वास्य च परस्परम्}
{मुदा परमया युक्तौ नरेश्वरकपीश्वरौ}%॥ १३ ॥

\twolineshloka
{आसाते ब्राह्मणश्रेष्ठा ऋष्यमूकगिरौ तथा}
{सुग्रीवप्रत्ययार्थं च दुन्दुभेः कायमाशु वै}%॥ १४ ॥

\twolineshloka
{पादाङ्गुष्ठेन चिक्षेप राघवो बहुयोजनम्}
{सप्तताला विनिर्भिन्ना राघवेण महात्मना}%॥ १५ ॥

\twolineshloka
{ततः प्रीतमना वीरः सुग्रीवो राममब्रवीत्}
{इन्द्रादिदेवताभ्योऽपि नास्ति राघव मे भयम्}%॥ १६ ॥

\twolineshloka
{भवान्मित्रं मया लब्धो यस्मादति पराक्रमः}
{अहं लङ्केश्वरं हत्वा भार्यामानयितास्मि ते}%॥ १७ ॥

\twolineshloka
{ततः सुग्रीवसहितो रामचन्द्रो महाबलः}
{सलक्ष्मणो ययौ तूर्णं किष्किन्धां वालिपालिताम्}%॥ १८ ॥

\twolineshloka
{ततो जगर्ज सुग्रीवो वाल्यागमनकाङ्क्षया}
{अमृष्यमाणो वाली च गर्जितं स्वानुजस्य वै}%॥ १९ ॥

\twolineshloka
{अन्तःपुराद्विनिष्क्रम्य युयुधेऽवरजेन सः}
{वालिमुष्टिप्रहारेण ताडितो भृशविह्वलः}%॥ २० ॥

\twolineshloka
{सुग्रीवो निर्गतस्तूर्णं यत्र रामो महाबलः}
{ततो रामो महाबाहुस्सुग्रीवस्य शिरोधरे}%॥ २१ ॥

\twolineshloka
{लतामाबध्य चिह्नं तु युद्धायाचोदयत्तदा}
{गर्जितेन समाहूय सुग्रीवो वालिनं पुनः}%॥ २२ ॥

\twolineshloka
{रामप्रेरणया तेन बाहुयुद्धमथाकरोत्}
{ततो वालिनमाजघ्ने शरेणैकेन राघवः}%॥ २३ ॥

\twolineshloka
{हते वालिनि सुग्रीवः किष्किन्धां प्रत्यपद्यत}
{ततो वर्षास्वतीतासु सुग्रीवो वानराधिपः}%॥ २४ ॥

\twolineshloka
{सीतामानयितुं तूर्णं वानराणां महाचमूम्}
{समादाय समागच्छदन्तिकं नृपपुत्रयोः}%॥ २५ ॥

\twolineshloka
{प्रस्थापयामास कपीन्सीतान्वेषणकाङ्क्षया}
{विदितायां तु वैदेह्या लङ्कायां वायुसूनुना}%॥ २६ ॥

\twolineshloka
{दत्ते चूडामणौ चापि राघवो हर्षशोकवान्}
{सुग्रीवेणानुजेनापि वायुपुत्रेण धीमता}%॥ २७ ॥

\twolineshloka
{तथान्यैः कपिभिश्चैव जाम्बवन्नलमुख्यकैः}
{अन्वीयमानो रामोऽसौ मुहूर्तेऽभिजिति द्विजाः}%॥ २८ ॥

\twolineshloka
{विलङ्घ्य विविधा न्देशान्महेन्द्रं पर्वतं ययौ}
{चक्रतीर्थं ततो गत्वा निवासमकरोत्तदा}%॥ २९ ॥

\twolineshloka
{तत्रैव तु स धर्मात्मा समागच्छद्विभीषणः}
{भ्राता वै राक्षसेन्द्रस्य चतुर्भिः सचिवैः सह}%॥ ३० ॥

\twolineshloka
{प्रतिजग्राह रामस्तं स्वागतेन महात्मना}
{सुग्रीवस्य तु शङ्काऽभूत्प्रणिधिः स्यादयं त्विति}%॥ ३१ ॥

\twolineshloka
{राघवस्तस्य चेष्टाभिः सम्यक्स्वचरितैर्हितैः}
{अदुष्टमेनं दृष्ट्वैव तत एनमपूजयत्}%॥ ३२ ॥

\twolineshloka
{सर्वराक्षसराज्ये तमभ्यषिञ्चद्विभीषणम्}
{चक्रे च मन्त्रिप्रवरं सदृशं रविसूनुना}%॥ ३३ ॥

\twolineshloka
{चक्रतीर्थं समासाद्य निवसद्रघुनन्दनः}
{चिन्तयन्राघवः श्रीमान्सुग्रीवादीनभाषत}%॥ ३४ ॥

\twolineshloka
{मध्ये वानरमु ख्यानां प्राप्तकालमिदं वचः}
{उपायः को नु भवतामेतत्सागरलङ्घने}%॥ ३५ ॥

\twolineshloka
{इयं च महती सेना सागरश्चापि दुस्तरः}
{अम्भोराशिरयं नीलश्चञ्चलोर्म्मिसमाकुलः}%॥ ३६ ॥

\twolineshloka
{उद्यन्मत्स्यो महानक्रशङ्खशुक्तिसमाकुलः}
{क्वचिदौर्वानलाक्रान्तः फेनवानतिभीषणः}%॥ ३७ ॥

\twolineshloka
{प्रकृष्टपवनाकृष्टनीलमेघसमन्वितः}
{प्रलयाम्भोधरारावः सारवाननिलोद्धतः}%॥ ३८ ॥

\twolineshloka
{कथं सागरमक्षोभ्यं तरामो वरुणा लयम्}
{सैन्यैः परिवृताः सर्वे वानराणां महौजसाम्}%॥ ३९ ॥

\twolineshloka
{उपायैरधिगच्छामो यथा नदनदीपतिम्}
{कथं तरामः सहसा ससैन्या वरुणालयम्}%॥ ४० ॥

\twolineshloka
{शतयोजनमायातं मनसापि दुरासदम्}
{अतो नु विघ्ना बहवः कथं प्राप्या च मैथिली}%॥ ४१ ॥

\twolineshloka
{कष्टात्कष्टतरं प्राप्ता वयमद्य निराश्रयाः}
{महाजले महावाते समुद्रे हि निराश्रये}%॥ ४२ ॥

\twolineshloka
{उपायं कं विधास्यामस्तरणार्थं वनौकसाम्}
{राज्याद्भ्रष्टो वनं प्राप्तो हृता सीता मृतः पिता}%॥ ४३ ॥

\twolineshloka
{इतोऽपि दुःसहं दुःखं यत्सागरविलङ्घनम्}
{धिग्धिग्गर्जितमम्भोधे धिगेतां वारिराशिताम्}%॥ ४४ ॥

\threelineshloka
{कथं तद्वचनं मिथ्या महर्षेः कुम्भजन्मनः}
{हत्वा त्वं रावणं पापं पवित्रे गन्धमादने}
{पापोपशमनायाशु गच्छस्वेति यदीरितम्}%॥ ४५ ॥

\uvacha{श्रीसूत उवाच}

\onelineshloka
{इति रामवचः श्रुत्वा सुग्रीवप्रमुखास्तदा}%॥ ४६ ॥

\twolineshloka
{ऊचुः प्राञ्जलयः संर्मे राघवं तं महाबलम्}
{नौभिरेनं तरिष्यामः प्लवैश्च विविधैरिति}%॥ ४७ ॥

\twolineshloka
{मध्ये वानरकोटीनां तदोवाच विभीषणः}
{समुद्रं राघवो राजा शरणं गन्तुमर्हति}%॥ ४८ ॥

\twolineshloka
{खनितः सागरैरेष समुद्रो वरुणालयः}
{कर्तुमर्हति रामस्य तज्ज्ञातेः कार्यमम्बुधिः}%॥ ४९ ॥

\twolineshloka
{विभीषणेनैवमुक्तो राक्षसेन विपश्चिता}
{सान्त्वयन्राघवः सर्वान्वानरानिदमब्रवीत्}%॥ ५० ॥

\twolineshloka
{शतयोजन विस्तारमशक्ताः सर्ववानराः}
{तर्तुं प्लवोडुपैरेनं समुद्रमतिभीषणम्}%॥ ५१ ॥

\twolineshloka
{नावो न सन्ति सेनाया बह्व्या वानरपुङ्गवाः}
{वणिजामुपघातं च कथमस्मद्विधश्चरेत्}%॥ ५२ ॥

\twolineshloka
{विस्तीर्णं चैव नः सैन्यं हन्याच्छिद्रेषु वा परः}
{प्लवोडुपप्रतारोऽतो नैवात्र मम रोचते}%॥ ५३ ॥

\twolineshloka
{विभीषेणोक्तमे वेदं मोदते मम वानराः}
{अहं त्विमं जलनिधिमुपास्ये मार्गसिद्धये}%॥ ५४ ॥

\twolineshloka
{नो चेद्दर्शयिता मार्गं धक्ष्याम्येनमहं तदा}
{महास्त्रैरप्रतिहतैरत्यग्निपवनोज्ज्वलैः}%॥ ५५ ॥

\twolineshloka
{इत्युक्त्वा सहसौमित्रिरुपस्पृश्याथ राघवः}
{प्रतिशिश्ये जलनिधिं विधिवत्कुशसंस्तरे}%॥ ५६ ॥

\twolineshloka
{तदा रामः कुशा स्तीर्णे तीरे नदनदीपतेः}
{संविवेश महाबाहुर्वेद्यामिव हुताशनः}%॥ ५७ ॥

\twolineshloka
{शेषभोगनिभं बाहुमुपधाय रघूद्वहः}
{दक्षिणो दक्षिणं बाहुमुपास्ते मकरालयम्}%॥ ५८ ॥

\twolineshloka
{तस्य रामस्य सुप्तस्य कुशास्तीर्णे महीतले}
{नियमादप्रमत्तस्य निशास्तिस्रोऽतिचक्रमुः}%॥ ५९ ॥

\twolineshloka
{स त्रिरात्रोषितस्तत्र नयज्ञो धर्मतत्परः}
{उपास्तेस्म तदा रामः सागरं मार्गसिद्धये}%॥ ६० ॥

\twolineshloka
{न च दर्शयते मन्दस्तदा रामस्य सागरः}
{प्रयतेनापि रामेण यथार्हमपि पूजितः}%॥ ६१ ॥

\twolineshloka
{तथापि सागरो रामं न दर्शयति चात्मनः}
{समुद्राय ततः क्रुद्धो रामो रक्तान्तलोचनः}%॥ ६२ ॥

\twolineshloka
{समीपवर्तिनं चेदं लक्ष्मणं प्रत्यभाषत}
{अद्य मद्बाणनिर्भिन्नैर्मकरैर्वरुणालयम्}%॥ ६३ ॥

\twolineshloka
{निरुद्धतोयं सौमित्रे करिष्यामि क्षणादहम्}
{सशङ्खशुक्ताजालं हि समीनमकरं शनैः}%॥ ६४ ॥

\twolineshloka
{अद्य बाणैरमोघास्त्रैर्वारिधिं परिशोषये}
{क्षमया हि समायुक्तं मामयं मकरालयः}%॥ ६९ ॥

\twolineshloka
{असमर्थं विजानाति धिक्क्षमामीदृशे जने}
{न दर्शयति साम्ना मे सागरो रूपमात्मनः}%॥ ६६ ॥

\twolineshloka
{चापमानय सौमित्रे शरांश्चाशीविषोपमान्}
{सागरं शोषयिष्यामि पद्भ्यां यान्तु प्लवङ्गमाः}%॥ ६७ ॥

\twolineshloka
{एनं लङ्घितमर्यादं सहस्रोर्मिसमाकुलम्}
{निर्मर्यादं करिष्यामि सायकैर्वरुणालयम्}%॥ ६८ ॥

\twolineshloka
{महार्णवं शोषयिष्ये महादानवसङ्कुलम्}
{महामकरनक्राढ्यं महावीचिसमाकुलम्}%॥ ६९ ॥

\twolineshloka
{एवमुक्त्वा धनुष्पाणिः क्रोधपर्याकुलेक्षणः}
{रामो बभूव दुर्धर्षस्त्रिपुरघ्नो यथा शिवः}%॥ ७० ॥

\twolineshloka
{आकृष्य चापं कोपेन कम्पयित्वा शरैर्जगत्}
{मुमोच विशिखानुग्रांस्त्रिपुरेषु यथा भवः}%॥ ७१ ॥

\twolineshloka
{दीप्ता बाणाश्च ये घोरा भासयन्तो दिशो दश}
{प्राविशन्वारिधेस्तोयं दृप्तदानवसङ्कुलम्}%॥ ७२ ॥

\twolineshloka
{समुद्रस्तु ततो भीतो वेपमानः कृताञ्जलिः}
{अनन्यशरणो विप्राः पाता लात्स्वयमुत्थितः}%॥ ७३ ॥

\twolineshloka
{शरणं राघवं भेजे कैवल्यपदकारणम्}
{तुष्टाव राघवं विप्रा भूत्वा शब्दैर्मनोरमैः}%॥ ७४ ॥

\uvacha{समुद्र उवाच}

\twolineshloka
{नमामि ते राघव पादपङ्कजं सीतापते सौख्यद पादसेवनात्}
{नमामि ते गौतमदारमोक्षजं श्रीपादरेणुं सुरवृन्दसेव्यम्}%॥ ७६ ॥

\twolineshloka
{सुन्दप्रियादेहविदारिणे नमो नमोस्तु ते कौशिकयागरक्षिणे}
{नमो महादेवशरासभेदिने नमो नमो राक्षससङ्घनाशिने}%॥ ७६ ॥

\twolineshloka
{रामराम नमस्यामि भक्तानामिष्टदायिनम्}
{अवतीर्णो रघुकुले देवकार्यचिकीर्षया}%॥ ७७ ॥

\twolineshloka
{नारायणमनाद्यन्तं मोक्षदं शिवमच्युतम्}
{रामराम महाबाहो रक्ष मां शरणागतम्}%॥ ७८ ॥

\twolineshloka
{कोपं संहर राजेन्द्र क्षमस्व करुणालय}
{भूमिर्वातो वियच्चापो ज्योतींषि च रघूद्वह}%॥ ७९ ॥

\twolineshloka
{यत्स्वभावानि सृष्टानि ब्रह्मणा परमेष्ठिना}
{वर्तन्ते तत्स्वभा वानि स्वभावो मे ह्यगाधता}%॥ ८० ॥

\twolineshloka
{विकारस्तु भवेद्गाध एतत्सत्यं वदाम्यहम्}
{लोभात्कामाद्भयाद्वापि रागाद्वापि रघूद्वह}%॥ ८१ ॥

\twolineshloka
{न वंशजं गुणं हातुमुत्सहेयं कथञ्चन}
{तत्करिष्ये च साहाय्यं सेनायास्तरणे तव}%॥ ८२ ॥

\twolineshloka
{इत्युक्तवन्तं जलधिं रामोऽवादीन्नदीपतिम्}
{ससैन्योऽहं गमि ष्यामि लङ्कां रावणपालिताम्}%॥ ८३ ॥

\twolineshloka
{तच्छोषमुपयाहि त्वं तरणार्थं ममाधुना}
{इत्युक्तस्तं पुनः प्राह राघवं वरुणालयः}%॥ ८४ ॥

\twolineshloka
{शृणुष्वाव हितो राम श्रुत्वा कर्तव्यमाचर}
{यद्याज्ञया ते शुष्यामि ससैन्यस्य यियासतः}%॥ ८५ ॥

\twolineshloka
{अन्येऽप्याज्ञापयिष्यन्ति मामेवं धनुषो बलात्}
{उपायमन्यं वक्ष्यामि तरणार्थं बलस्य ते}%॥ ८६ ॥

\twolineshloka
{अस्ति ह्यत्र नलोनाम वानरः शिल्पिसम्मतः}
{त्वष्टुः काकुत्स्थ तनयो बलवान्विश्वकर्मणः}%॥ ८७ ॥

\twolineshloka
{स यत्काष्ठं तृणं वापि शिलां वा क्षेप्स्यते मयि}
{सर्वं तद्धारयिष्यामि स ते सेतुर्भविष्यति}%॥ ८८ ॥

\twolineshloka
{सेतुना तेन गच्छ त्वं लङ्कां रावणपालि ताम्}
{उक्त्वेत्यन्तर्हिते तस्मिन्रामो नलमुवाच ह}%॥ ८९ ॥

\twolineshloka
{कुरु सेतुं समुद्रे त्वं शक्तो ह्यसि महामते}
{तदाऽब्रवीन्नलो वाक्यं रामं धर्मभृतां वरम्}%॥ ९० ॥

\twolineshloka
{अहं सेतुं विधास्यामि ह्यगाधे वरुणालये}
{पित्रा दत्तवरश्चाहं सामर्थ्ये चापि तत्समः}%॥ ९१ ॥

\twolineshloka
{मातुर्मम वरो दत्तो मन्दरे विश्वक र्मणा}
{शिल्पकर्मणि मत्तुल्यो भविता ते सुतस्त्विति}%॥ ९२ ॥

\twolineshloka
{पुत्रोऽहमौरसस्तस्य तुल्यो वै विश्वकर्मणा}
{अद्यैव कामं बध्नन्तु सेतुं वानरपुं गवाः}%॥ ९३ ॥

\twolineshloka
{ततो रामनिसृष्टास्ते वानरा बलवत्तराः}
{पर्वतान्गिरिशृङ्गाणि लतातृणमहीरुहान्}%॥ ९४ ॥

\twolineshloka
{समाजह्रुर्महाकाया गरुडानिलरंहसः}
{नलश्चक्रे महासेर्तुमध्ये नदनदीपतेः}%॥ ९५ ॥

\twolineshloka
{दशयोजनविस्तीर्णं शतयोजनमायतम्}
{जानकीरमणो रामः सेतुमेवमकारयत्}%॥ ९६ ॥

\twolineshloka
{नलेन वानरेन्द्रेण विश्वकर्मसुतेन वै}
{तमेवं सेतुमासाद्य रामचन्द्रेण कारितम्}%॥ ९७ ॥

\twolineshloka
{सर्वे पातकिनो मर्त्या मुच्यन्ते सर्वपातकैः}
{व्रतदान तपोहोमैर्न तथा तुष्यते शिवः}%॥ ९८ ॥

\twolineshloka
{सेतुमज्जनमात्रेण यथा तुष्यति शङ्करः}
{न तुल्यं विद्यते तेजोयथा सौरेण तेजसा}%॥ ९९ ॥

\twolineshloka
{सेतुस्नानेन च तथा न तुल्यं विद्यते क्वचित्}
{तत्सेतुमूलं लङ्कायां यत्ररामो यियासया}%॥ १०० ॥

\twolineshloka
{वानरैः सेतुमारेभे पुण्यं पाप प्रणाशनम्}
{तद्दर्भशयनं नाम्ना पश्चाल्लोकेषु विश्रुतम्}%॥ १०१ ॥

\twolineshloka
{एवमुक्तं मया विप्राः समुद्रे सेतुबन्धनम्}
{अत्र तीर्थान्यनेकानि सन्ति पुण्यान्यनेकशः}%॥ १०२ ॥

\twolineshloka
{न सङ्ख्यां नामधेयं वा शेषो गणयितुं क्षमः}
{किं त्वहं प्रब्रवीम्यद्य तत्र तीर्थानि कानिचित्}%॥ १०३ ॥

\twolineshloka
{चतुर्विंशति तीर्थानि सन्ति सेतौ प्रधानतः}
{प्रथमं चकतीर्थं स्याद्वेतालवरदं ततः}%॥ १०४ ॥

\twolineshloka
{ततः पापविनाशार्थं तीर्थं लोकेषु विश्रुतम्}
{ततः सीतासरः पुण्यं ततो मङ्गलतीर्थकम्}%॥ १०५ ॥

\twolineshloka
{ततः सकलपापघ्नी नाम्ना चामृतवापिका}
{ब्रह्मकुण्डं ततस्तीर्थं ततः कुण्डं हनूमतः}%॥ १०६ ॥

\twolineshloka
{आगस्त्यं हि ततस्तीर्थं रामतीर्थ मतः परम्}
{ततो लक्ष्मणतीर्थं स्याज्जटातीर्थमतः परम्}%॥ १०७ ॥

\twolineshloka
{ततो लक्ष्म्याः परं तीर्थमग्नितीर्थमतः परम्}
{चक्रतीर्थं ततः पुण्यं शिवतीर्थमतः परम्}%॥ १०८ ॥

\twolineshloka
{ततः शङ्खाभिधं तीर्थं ततो यामुनतीर्थकम्}
{गङ्गातीर्थं ततः पश्चाद्गयातीर्थमनन्तरम्}%॥ १०९ ॥

\twolineshloka
{ततः स्यात्कोटितीर्थाख्यं साध्यानाममृतं ततः}
{मानसाख्यं ततस्तीर्थं धनुष्कोटिस्ततः परम्}%॥ ११० ॥

\twolineshloka
{प्रधानतीर्थान्येतानि महापापहराणि च}
{कथितानि द्विजश्रेष्ठास्सेतुमध्यगतानि वै}%॥ १११ ॥

\twolineshloka
{यथा सेतुश्च बद्धोऽभूद्रामेण जलधौ महान्}
{कथितं तच्च विप्रेन्द्राः पुण्यं पापहारं तथा}%॥ ११२

\onelineshloka
{यच्छ्रुत्वा च पठित्वा च मुच्यते मानवो भुवि}%॥ ११३ ॥

\twolineshloka
{अध्यायमेनं पठते मनुष्यः शृणोति वा भक्तियुतो द्विजेन्द्राः}
{सो नन्तमाप्नोति जयं परत्र पुनर्भवक्लेशमसौ न गच्छेत्}%॥ ११४ ॥

॥इति श्रीस्कान्दे महापुराण एकाशीतिसाहस्र्यां संहितायां तृतीये ब्रह्मखण्डे सेतुमाहात्म्ये सेतुनिर्माणादिवर्णनं नाम द्वितीयोऽध्यायः॥२॥
