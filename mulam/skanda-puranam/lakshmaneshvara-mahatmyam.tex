\sect{द्वादशोत्तरशततमोऽध्यायः --- लक्ष्मणेश्वरमाहात्म्यवर्णनम्}

\src{स्कन्दपुराणम्}{खण्डः ७ (प्रभासखण्डः)}{प्रभासक्षेत्र माहात्म्यम्}{अध्यायः ११२}
\vakta{}
\shrota{}
\tags{}
\notes{}
\textlink{https://sa.wikisource.org/wiki/स्कन्दपुराणम्/खण्डः_७_(प्रभासखण्डः)/प्रभासक्षेत्र_माहात्म्यम्/अध्यायः_११२}
\translink{https://www.wisdomlib.org/hinduism/book/the-skanda-purana/d/doc626900.html}

\storymeta




\uvacha{ईश्वर उवाच}

\twolineshloka
{ततो गच्छेन्महादेवि लक्ष्मणेश्वरमुत्तमम्}
{रामेशात्पूर्वदिग्भागे धनुस्त्रिंशकसंस्थितम्}%॥ १ ॥

\twolineshloka
{स्थापितं लक्ष्मणेनैव तत्र यात्रागतेन वै}
{महापापहरं देवि तल्लिङ्गं सुरपूजितम्}%॥ २ ॥

\twolineshloka
{यस्तं पूजयते भक्त्या नृत्यगीतादिवादनैः}
{होमजाप्यैः समाधिस्थः स याति परमां गतिम्}%॥ ३ ॥

\twolineshloka
{अन्नोदकं हिरण्यं च तत्र देयं द्विजातये}
{सम्पूज्य देवदेवेशं गन्धपुष्पादिभिः क्रमात्}%॥ ४ ॥

\twolineshloka
{माघे कृष्णचतुर्दश्यां विशेषस्तत्र पूजने}
{स्नानं दानं जपस्तत्र भवेदक्षयकारकम्}%॥ ५ ॥
॥इति श्रीस्कान्दे महापुराण एकाशीतिसाहस्र्यां संहितायां सप्तमे प्रभासखण्डे प्रथमे प्रभासक्षेत्रमाहात्म्ये रामेश्वरक्षेत्रमाहात्म्ये लक्ष्मणेश्वरमाहात्म्यवर्णनं नाम द्वादशोत्तरशततमोऽध्यायः॥११२॥