% !TeX program = XeLaTeX
% !TeX root = ./ramayana-sangraha-mulam.tex
{\font \x="Sanskrit 2003:script=deva" at 12pt\x
\centerline{॥ॐ श्री-गणेशाय नमः॥}
\centerline{॥ॐ श्री-गुरुभ्यो नमः॥}
\centerline{॥हरिः ॐ॥}
}
\thispagestyle{empty}

\begin{center}
\chapter*{\texorpdfstring{\scshape{Preface}}{Preface}}
\end{center}

\twolineshloka*{सदाशिवसमारम्भां शङ्कराचार्यमध्यमाम्}
{अस्मदाचार्यपर्यन्तां वन्दे गुरुपरम्पराम्}

\twolineshloka*
{एष सेतुर्विधरणो लोकासम्भेदहेतवे}
{कोदण्डेन च दण्डेन रामेण गुरुणा कृतः}

रामायण-श्रोतॄणां कदापि तृप्तिर्न जायते! यथा भगवान् वाल्मीकिः वदति--- ``रामो रामो राम इति प्रजानामभवन् कथाः'', तद्वत् इतिहासपुराणानामपि श्रीरामचन्द्रस्य बहवः कथाः यत्र कुत्रापि लभ्यन्ते। तेषाम् एकत्र प्रस्तुतिं कर्तुम् एतत् परिश्रमः। सीतादेवी अपि अध्यात्मरामायणे रामस्य अरण्यगमनप्रसङ्गे वदति---

``रामायणानि बहुशः श्रुतानि बहुभिर्द्विजैः॥२-४-७७॥'' 

अत्र अनेकरामकथाः प्रस्तुताः सन्ति। तासां वक्तॄन् व्यासं वाल्मीकिं च नमस्कृत्य एतस्य ग्रन्थस्य पारायणम् आरभामहे। रामे अनन्यभक्तिः सदा भवतु नः। 

\twolineshloka*
{नारायणं नमस्कृत्य नरं चैव नरोत्तमम्}
{देवीं सरस्वतीं चैव ततो जयमुदीरयेत्}

\twolineshloka*
{कूजन्तं राम रामेति मधुरं मधुराक्षरम्}
{आरुह्य कविताशाखां वन्दे वाल्मीकिकोकिलम्}


\twolineshloka*
{यत्र यत्र रघुनाथकीर्तनं तत्र तत्र कृतमस्तकाञ्जलिम्}
{बाष्पवारिपरिपूर्णलोचनं मारुतिं नमत राक्षसान्तकम्}

\twolineshloka*
{रामं रामानुजं सीतां भरतं भरतानुजम्}
{सुग्रीवं वायुसूनुं च प्रणमामि पुनः पुनः}

\twolineshloka*
{नमोऽस्तु रामाय सलक्ष्मणाय देव्यै च तस्यै जनकात्मजायै}
{नमोऽस्तु रुद्रेन्द्रयमानिलेभ्यो नमोऽस्तु चन्द्रार्कमरुद्गणेभ्यः}


बलं विष्णोः प्रवर्धताम्!



\centerline{सर्वम् श्री-सीतारामचन्द्रार्पणमस्तु॥}

\medskip
\noindent\today \hfill \textsc{Karthik Raman}
