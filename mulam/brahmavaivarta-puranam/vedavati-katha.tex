\chapt{ब्रह्मवैवर्त-पुराणम्}

\sect{वेदवती-प्रस्तावः}

\src{ब्रह्मवैवर्त-पुराणम्}{खण्डः २ (प्रकृतिखण्डः)}{अध्यायः १४}{}
% \tags{concise, complete}
\notes{This chapter briefly describes the stories of Vedavati, Sita and Draupadi, across the three yugas.}
\textlink{https://sa.wikisource.org/wiki/ब्रह्मवैवर्तपुराणम्/खण्डः_२_(प्रकृतिखण्डः)/अध्यायः_१४}
\translink{https://archive.org/details/brahma-vaivarta-purana-all-four-kandas-english-translation/page/n255/mode/2up}

\storymeta


\uvacha{नारायण उवाच}

\twolineshloka
{लक्ष्मीं तौ च समाराध्य चोग्रेण तपसा मुने}
{प्रत्येकं वरमिष्टं च सम्प्रापतुरभीप्सितम्}% ।। १ ।।

\twolineshloka
{महालक्ष्म्या वरेणैव तौ पृध्वीशौ बभूवतुः}
{धनवन्तौ पुत्रवन्तौ धर्मध्वजकुशध्वजौ}% ।। २ ।।

\twolineshloka
{कुशध्वजस्य पत्नी च देवी मालावती सती}
{सा सुषाव च कालेन कमलांशां सुतां सतीम्}% ।। ३ ।।

\twolineshloka
{सा च भूतलसम्बन्धाज्ज्ञानयुक्ता बभूव ह}
{कृत्वा वेदध्वनिं स्पष्टमुत्तस्थौ सूतिकागृहे।}% ।। ४ ।।

\twolineshloka
{वेदध्वनिं सा चकार जातमात्रेण कन्यका}
{तस्मात्तां ते वेदवतीं प्रवदन्ति मनीषिणः}% ।। ५ ।।

\twolineshloka
{जातमात्रेण सुस्नाता जगाम तपसे वनम्}
{सर्वैर्निषिद्धा यत्नेन नारायणपरायणा}% ।। ६ ।।

\twolineshloka
{एकमन्वन्तरं चैव पुष्करे च तपस्विनी}
{अत्युग्रां वै तपस्यां तु लीलया च चकार सा}% ।। ७ ।।

\twolineshloka
{तथाऽपि पुष्टा न कृशा नवयौवनसंयुता}
{शुश्राव खे च सहसा सा वाचमशरीरिणीम्}% ।। ।। ८ ।।

\twolineshloka
{जन्मान्तरे ते भर्त्ता च भविष्यति हरिः स्वयम्}
{ब्रह्मादिभिर्दुराराध्यं पतिं लप्स्यसि सुन्दरि}% ।। ९ ।।

\twolineshloka
{इति श्रुत्वा तु सा रुष्टा चकार च पुनस्तपः}
{अतीव निर्जनस्थाने पर्वते गन्धमादने}% ।। 2.14.१० ।।

\twolineshloka
{तत्रैवं सुचिरं तप्त्वा विश्वस्य समुवास सा}
{ददर्श पुरतस्तत्र रावणं दुर्निवारणम्}% ।। ११ ।।

\twolineshloka
{दृष्ट्वा साऽतिथिभक्त्या च पाद्यं तस्मै ददौ किल}
{सुस्वादु फलमूलं च जलं चापि सुशीतलम्}% ।। १२ ।।

\twolineshloka
{तच्च भुक्त्वा स पापिष्ठश्चावात्सीत्तत्समीपतः}
{चकार प्रश्नमिति तां का त्वं कल्याणि चेति च}% ।। १३ ।।

\twolineshloka
{तां च दृष्ट्वा वरारोहां पीनोन्नतपयोधराम्}
{शरत्पद्मनिभास्यां च सस्मितां सुदतीं सतीम्}% ।। १४ ।।

\twolineshloka
{मूर्च्छामवाप कृपणः कामबाणप्रपीडितः}
{तां करेण समाकृष्य सम्भोगं कर्तुमुद्यतः}% ।। १५ ।।

\twolineshloka
{सा सती कोपदृष्ट्या च स्तम्भितं तं चकार ह}
{स जडो हस्तपादैश्च किञ्चिद्वक्तुं न च क्षमः}% ।। १६ ।।

\twolineshloka
{तुष्टाव मनसा देवीं पद्मांशां पद्मलोचनाम्}
{सा तत्स्तवेन सन्तुष्टा प्राकृतं तं मुमोच ह}% ।। १७ ।।

\twolineshloka
{शशाप च मदर्थे त्वं विनश्यसि सबान्धवः}
{स्पृष्टाऽहं च त्वया कामाद्विसृजाम्यवलोकय}% ।। १८ ।।

\twolineshloka
{इत्युक्त्वा सा च योगेन देहत्यागं चकार ह}
{गङ्गायां तां च सन्न्यस्य स्व गृहं रावणो ययौ}% ।। १९ ।।

\twolineshloka
{अहो किमद्भुतं दृष्टं किं कृतं वा मयाऽधुना}
{इति सञ्चिन्त्य संस्मृत्य विललाप पुनः पुनः}% ।। ।। 2.14.२० ।।

\twolineshloka
{सा च कालान्तरे साध्वी बभूव जनकात्मजा}
{सीतादेवीति विख्याता यदर्थे रावणो हतः}% ।। २१ ।।

\twolineshloka
{महातपस्विनी सा च तपसा पूर्वजन्मनः}
{लेभे रामं च भर्त्तारं परिपूर्णतमं हरिम्}% ।। २२ ।।

\twolineshloka
{सम्प्राप्य तपसाऽऽराध्य स्वामिनं च जगत्पतिम्}
{सा रमा सुचिरं रेमे रामेण सह सुन्दरी}% ।। २३ ।।

\twolineshloka
{जातिस्मरा स्म स्मरति तपसश्च क्रमं पुरा}
{सुखेन तज्जहौ सर्वं दुःखं चापि सुखं लभेत्}% ।। ।। २४ ।।

\twolineshloka
{नानाप्रकारविभवं चकार सुचिरं सती}
{सम्प्राप्य सुकुमारं तमतीव नवयौवनम्}% ।। २५ ।।

\twolineshloka
{गुणिनं रसिकं शान्तं कान्तवेषमनुत्तमम्}
{स्त्रीणां मनोज्ञं रुचिरं तथा लेभे यथेप्सितम्}% ।। २६ ।।

\twolineshloka
{पितुर्वचःपालनार्थं सत्यसन्धो रघूत्तमः}
{जगाम काननं पश्चात्कालेन च बलीयसा}% ।। २७ ।।

\twolineshloka
{तस्थौ समुद्रनिकटे सीतया लक्ष्मणेन च}
{ददर्श तत्र वह्निं च विप्ररूपधरं हरिः}% ।। २८ ।।

\twolineshloka
{तं रामं दुःखितं दृष्ट्वा स च दुःखी बभूव ह}
{उवाच किञ्चित्सत्येष्टं सत्यं सत्यपरायणः}% ।। २९ ।।
वह्निरुवाच ।।

\twolineshloka
{भगवञ्छ्रूयतां वाक्यं कालेन यदुपस्थितम्}
{सीताहरणकालोऽयं तवैव समुपस्थितः}% ।। 2.14.३० ।।

\twolineshloka
{दैवं च दुर्निवार्य्यं वै न च दैवात्परं बलम्}
{मत्प्रसू मयि सन्न्यस्य च्छायां रक्षान्तिकेऽधुना}% ।। ३१ ।।

\twolineshloka
{दास्यामि सीतां तुभ्यं च परीक्षासमये पुनः}
{देवैः प्रस्थापितोऽहं च न च विप्रो हुताशनः}% ।। ३२ ।।

\twolineshloka
{रामस्तद्वचनं श्रुत्वा न प्रकाश्य च लक्ष्मणम्}
{स्वच्छन्दं स्वीचकारासौ हृदयेन विदूयता}% ।। ३३ ।।

\twolineshloka
{वह्निर्योगेन सीतावन्मायासीतां चकार ह}
{तत्तुल्यगुणरूपां तां ददौ रामाय नारद}% ।। ३४ ।।

\twolineshloka
{सीतां गृहीत्वा स ययौ गोप्यं वक्तुं निषेध्य च}
{लक्ष्मणो नैव बुबुधे गोप्यमन्यस्य का कथा}% ।। ३९ ।।

\twolineshloka
{एतस्मिन्नन्तरे रामो ददर्श कनकं मृगम्}
{सीता तं प्रेरयामास तदर्थे यत्नपूर्वकम्}% ।। ३६ ।।

\twolineshloka
{सन्न्यस्य लक्ष्मणं रामो जानक्या रक्षणे वने}
{स्वयं जगाम हन्तुं तं विव्यधे सायकेन च}% ।। ३७ ।।

\twolineshloka
{लक्ष्मणेति च शब्दं वै कृत्वा मायामृगस्तदा}
{प्राणांस्तत्याज सहसा पुरो दृष्ट्वा हरिं स्मरन्}% ।। ३८ ।।

\twolineshloka
{मृगरूपं परित्यज्य दिव्यरूपं विधाय च}
{रत्ननिर्मितयानेन वैकुण्ठं स जगाम ह}% ।। ३९ ।।

\twolineshloka
{वैकुण्ठस्य महाद्वारं किङ्करो द्वारपालयोः}
{जय विजययोश्चैव बलवांश्च जयाभिधः}% ।। 2.14.४० ।।

\twolineshloka
{शापेन सनकादीनां सम्प्राप्तो राक्षसीं तनुम्}
{पुनर्जगाम तद्द्वारमादौ स द्वारपालयोः}% ।। ४१ ।।

\twolineshloka
{अथ शब्दं च सा श्रुत्वा लक्ष्मणेति च विक्लवम्}
{सीता तं प्रेरयामास लक्ष्मणं रामसन्निधौ}% ।। ४२ ।।

\twolineshloka
{गते च लक्ष्मणे रामं रावणो दुर्निवारणः}
{सीतां गृहीत्वा प्रययौ लङ्कामेव स्वलीलया}% ।। ४३ ।।

\twolineshloka
{विषसाद च रामश्च वने दृष्ट्वा च लक्ष्मणम्}
{तूर्णं च स्वाश्रमं गत्वा सीतां नैव ददर्श सः}% ।। ४४ ।।

\twolineshloka
{मूर्च्छां सम्प्राप्य सुचिरं विललाप भृशं पुनः}
{पुनर्बभ्राम गहने तदन्वेषणपूर्वकम्}% ।। ४५ ।।

\twolineshloka
{काले सम्प्राप्य तद्वार्तां गृधद्वारा नदीतटे}
{सहायं वानरं कृत्वा चाबध्नात्सागरं हरिः}% ।। ।। ४६ ।।

\twolineshloka
{लङ्कां गत्वा रघुश्रेष्ठश्चावधीत्सायकेन च}
{सबान्धवं रावणं च सीतां सम्प्राप दुःखिताम्}% ।। ४७ ।।

\twolineshloka
{तां च वह्निपरीक्षां व कारयामास सत्वरम्}
{हुताशनस्तत्र काले वास्तवीं जानकीं ददौ}% ।। ४८ ।।

\twolineshloka
{छाया चोवाच वह्निं च रामं च विनयान्विता}
{करिष्यामीति किमहं तदुपायं वदस्व मे}% ।। ४९ ।।
वह्निरुवाच ।।

\twolineshloka
{त्वं गच्छ तपसे देवि पुष्करं च सुपुण्यदम्}
{कृत्वा तपस्यां तत्रैव स्वर्गलक्ष्मीर्भविष्यसि}% ।। 2.14.५० ।।

\twolineshloka
{सा च तद्वचनं श्रुत्वा प्रतेपे पुष्करे तपः}
{दिव्यं त्रिलक्षवर्ष च स्वर्गे लक्ष्मीर्बभूव ह}% ।। ५१ ।।

\twolineshloka
{सा च कालेन तपसा यज्ञकुण्डसमुद्भवा}
{कामिनी पाण्डवानां च द्रौपदी द्रुपदात्मजा}% ।। ५२ ।।

\twolineshloka
{कृते युगे वेदवती कुशध्वजसुता शुभा}
{त्रेतायां रामपत्नी च सीतेति जनकात्मजा}% ।। ५३ ।।

\twolineshloka
{तच्छाया द्रौपदी देवी द्वापरे द्रुपदात्मजा}
{त्रिहायणीति सा प्रोक्ता विद्यमाना युगत्रये}% ।। ५४ ।।

\uvacha{नारद उवाच}

\twolineshloka
{प्रियाः पञ्च कथं तस्या बभूवुर्मुनिपुङ्गव}
{इति वै चित्तसन्देहं दूरीकुरु महाप्रभो}% ।। ५५ ।।

\uvacha{नारायण उवाच}

\twolineshloka
{लङ्कायां वस्तुतः सीता रामं सम्प्राप नारद}
{रूपयौवनसम्पन्ना छाया सा बहुविह्वला}% ।। ५६ ।।

\twolineshloka
{रामाग्न्योराज्ञया तप्त्वा ययाचे शङ्करं वरम्}
{कामातुरा पतिव्यग्रा प्रार्थयन्ती पुनः पुनः}% ।। ५७ ।।

\twolineshloka
{पतिं देहि पतिं देहि पतिं देहि त्रिलोचन}
{पतिं देहि पतिं देहि पञ्चवारं पतिव्रता}% ।। ५८ ।।

\twolineshloka
{शिवस्तत्प्रार्थनां श्रुत्वा सस्मितो रसिकेश्वरः}
{प्रिये तव प्रियाः पञ्च भवन्तीति वरं ददौ}% ।। ५९ ।।

\twolineshloka
{तेनासीत्पाण्डवानां च पञ्चानां कामिनी प्रिया}
{इत्येवं कथितं सर्वं प्रस्तुतं वस्तुतः शृणु}% ।। 2.14.६० ।।

\twolineshloka
{अथ सम्प्राप्य लङ्कायां सीतां रामो मनोहराम्}
{विभीषणाय तां लङ्कां दत्त्वाऽयोध्यां ययौ पुनः}% ।। ६१ ।।

\twolineshloka
{एकादशसहस्राब्दं कृत्वा राज्यं च भारते}
{जगाम सर्वैर्लोकैश्च सार्द्धं वैकुण्ठमेव च}% ।। ६२ ।।

\twolineshloka
{कमलांशा वेदवती कमलायां विवेश सा}
{कथितं पुण्यमाख्यानं पुण्यदं पापनाशनम्}% ।। ६३ ।।

\twolineshloka
{सततं मूर्तिमन्तश्च वेदाश्चत्वार एव च}
{सन्ति यस्याश्च जिह्वाग्रे सा च वेदवती स्मृता}% ।। ६४ ।।

\twolineshloka
{कुशध्वजसुताख्यानमुक्तं सङ्क्षेपतस्तव}
{धर्मध्वजसुताख्यानं निबोध कथयामि ते}% ।। ६५ ।।

॥इति श्रीब्रह्मवैवर्त्ते महापुराणे द्वितीये प्रकृतिखण्डे नारदनारायणसंवादे तुलस्युपाख्याने वेदवतीप्रस्तावे चतुर्दशोऽध्यायः॥१४॥
