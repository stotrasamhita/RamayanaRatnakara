\chapt{श्रीरामावतारवर्णनम्}

\src{अग्निपुराणम्}{}{अध्यायः ५}{श्लोकाः १--१४}
\vakta{}
\shrota{}
\notes{}
\textlink{https://sa.wikisource.org/wiki/अग्निपुराणम्/अध्यायः_५}
\translink{}

\storymeta

sect{पञ्चमोऽध्यायः - बाल-काण्ड-वर्णनम्}

\uvacha{अग्निरुवाच}
\twolineshloka
{रामायणमहं वक्ष्ये नारदेनोदितं पुरा}
{वाल्मीकये यथा तद्वत् पठितं भुक्तिमुक्तिदम्} %।। १ ।।

\uvacha{नारद उवाच}
\twolineshloka
{विष्णुनाभ्यव्जजो ब्रह्मा मरीचिर्ब्रह्मणः सुतः}
{मरीचेः कश्यपस्तस्मात् सूर्यो वैवस्वतो मनुः} %।। २ ।।

\twolineshloka
{ततस्तस्मात्तथेक्ष्वाकुस्तस्य वंशे ककुत्स्थकः}
{ककुत्स्थस्य रघुस्तस्मादजो दशरथस्ततः} %।। ३ ।।

\twolineshloka
{रावणादेर्वधार्थाय चतुर्द्धाभूत् स्वयं हरिः}
{राज्ञो दशरथाद्रामः कौशल्यायां बभूव ह} %।। ४ ।।

\twolineshloka
{कैकेय्यां भरतः पुत्रः सुमित्रायाञ्च लक्ष्मणः}
{शत्रुघ्नः ऋष्यश्रृङ्गेण तासु सन्दत्तपायसात्} %।। ५ ।।

\twolineshloka
{प्राशिताद्यज्ञसंसिद्धाद्रामाद्याश्च समाः पितुः}
{यज्ञविध्नविनाशाय विश्वामित्रार्थितो नृपः} %।। ६ ।।

\twolineshloka
{रामं सम्प्रेषयामास लक्ष्मणं मुनिना सह}
{रामो गतोऽस्त्रशस्त्राणि शिक्षितस्ताडकान्तकृत्} %।। ७ ।।

\twolineshloka
{मारीचं मानवास्त्रेण मोहितं दूरतोऽनयत्}
{सुबाहुं यज्ञहन्तारं सबलञ्चावधीद् बली} %।। ८ ।।

\twolineshloka
{सिद्धाश्रमनिवासी च विश्वामित्रादिभिः सह}
{गतः क्रतुं मैथिलस्य द्रष्टुञ्चापंसहानुजः} %।। ९ ।।

\twolineshloka
{शतानन्दनिमित्तेन विश्वामित्रप्रभावतः}
{रामाय कथितो राज्ञा समुनिः पूजितः क्रतौ} %।। १० ।।

\twolineshloka
{धनुरापूरयामास लीलया स बभञ्ज तत् }
{वीर्यशुल्कञ्च जनकः सीतां कन्यान्त्वयोनिजाम्} %।। ११ ।।

\twolineshloka
{ददौ रामाय रामोऽपि पित्रादौ हि समागते}
{उपयेमे जानकीन्तामुर्मिलां लक्ष्मणस्तथा} %।। १२ ।।

\twolineshloka
{श्रुतकीर्त्तिं माण्डवीञ्च कुशध्वजसुते तथा}
{जनकस्यानुजस्यैते शत्रुघ्नभरतावुभौ} %।। १३ ।।

\threelineshloka
{कन्ये द्वे उपयेमाते जनकेन सुपूजितः}
{रामोऽगात्सवशिष्ठाद्यैर्जामदग्न्यं विजित्य च}
{अयोध्यां भरतोभ्यागात् सशत्रुघ्नो युधाजितः} %।। १४ ।।

॥इत्यादिमहापुराणे आग्नेये रामायणे बालकाण्डवर्णनं नाम पञ्चमोऽध्यायः॥


\sect{षष्ठोऽध्यायः --- अयोध्या-काण्ड-वर्णनम्}

\uvacha{नारद उवाच}

\twolineshloka
{भरतेऽथ गते रामः पित्रादीनभ्यपूजयत्}
{राजा दशरथो राममुवाच शृणु राघव} % ०१

\twolineshloka
{गुणानुरागाद्राज्ये त्वं प्रजाभिरभिषेचितः}
{मनसाहं प्रभाते ते यौवराज्यं ददामि ह} % ०२

\twolineshloka
{रात्रौ त्वं सीतया सार्धं संयतः सुव्रतो भव}
{राज्ञश्च मन्त्रिणश्चाष्टौ सवसिष्ठास्तथाब्रुवन्} % ०३

\twolineshloka
{सृष्टिर्जयन्तो विजयः सिद्धार्थो राष्ट्रवर्धनः}
{अशोको धर्मपालश्च सुमन्त्रः सवसिष्ठकः} % ०४

\twolineshloka
{पित्रादिवचनं श्रुत्वा तथेत्युक्त्वा स राघवः}
{स्थितो देवार्चनं कृत्वा कौशल्यायै निवेद्य तत्} % ०५

\twolineshloka
{राजोवाच वसिष्ठादीन् रामराज्याभिषेचने}
{सम्भारान् सम्भवन्तु स्म इत्युक्त्वा कैकेयीङ्गतः} % ०६

\twolineshloka
{अयोध्यालङ्कृतिं दृष्ट्वा ज्ञात्वा रामाभिषेचनं}
{भविष्यतीत्याचचक्षे कैकेयीं मन्थरा सखी} % ०७

\twolineshloka
{पादौ गृहीत्वा रामेण कर्षिता सापराधतः}
{तेन वैरेण सा राम वनवासञ्च काङ्क्षति} % ०८

\twolineshloka
{कैकेयि त्वं समुत्तिष्ठ रामराज्याभिषेचनं}
{मरणं तव पुत्रस्य मम ते नात्र संशयः} % ०९

\twolineshloka
{कुब्जयोक्तञ्च तच्छ्रुत्वा एकमाभरणं ददौ}
{उवाच मे यथा रामस्तथा मे भरतः सुतः} % १०

\twolineshloka
{उपायन्तु न पश्यामि भरतो येन राज्यभाक्}
{कैकेयीमब्रवीत्क्रुद्धा हारं त्यक्त्वाथ मन्थरा} % ११

\twolineshloka
{बालिशे रक्ष भरतमात्मानं माञ्च राघवात्}
{भविता राघवो राजा राघवस्य ततः सुतः} % १२

\twolineshloka
{राजवंशस्तु कैकेयि भरतात्परिहास्यते}
{देवासुरे पुरा युद्धे शम्बरेण हताः सुराः} % १३

\twolineshloka
{रात्रौ भर्ता गतस्तत्र रक्षितो विद्यया त्वया}
{वरद्वयं तदा प्रादाद्याचेदानीं नृपं च तत्} % १४

\twolineshloka
{रामस्य च वने वासं नव वर्षाणि पञ्च च}
{यौवराज्यं च भरते तदिदानीं प्रदास्यति} % १५

\twolineshloka
{प्रोत्साहिता कुब्जया सा अनर्थे चार्थदर्शिनी}
{उवाच सदुपायं मे कच्चित्तं कारयिष्यति} % १६

\twolineshloka
{क्रोधागारं प्रविष्टाथ पतिता भुवि मूर्छिता}
{द्विजादीनर्चयित्वाऽथ राजा दशरथस्तदा} % १७

\twolineshloka
{ददर्श केकयीं रुष्टामुवाच कथमीदृशी}
{रोगार्ता किं भयोद्विग्ना किमिच्छसि करोमि तत्} % १८

\twolineshloka
{येन रामेण हि विना न जीवामि मुहूर्तकम्}
{शपामि तेन कुर्यां वै वाञ्छितं तव सुन्दरि} % १९

\twolineshloka
{सत्यं ब्रूहीति सोवाच नृपं मह्यं ददासि चेत्}
{वरद्वयं पूर्वदत्तं सत्यात्त्वं देहि मे नृप} % २०

\twolineshloka
{चतुर्दशसमा रामो वने वसतु संयतः}
{सम्भारैरेभिरद्यैव भरतोऽत्राभिषेच्यताम्} % २१

\twolineshloka
{विषं पीत्वा मरिष्यामि दास्यसि त्वं न चेन्नृप}
{तच्छ्रुत्वा मूर्छितो भूमौ वज्राहत इवापतत्} % २२

\twolineshloka
{मुहूर्ताच्चेतनां प्राप्य कैकेयीमिदमब्रवीत्}
{किं कृतं तव रामेण मया वा पापनिश्चये} % २३

\twolineshloka
{यन्मामेवं ब्रवीषि त्वं सर्वलोकाप्रियङ्करि}
{केवलं त्वत्प्रियं कृत्वा भविष्यामि सुनिन्दितः} % २४

\twolineshloka
{या त्वं भार्या कालरात्री भरतो नेदृशः सुतः}
{प्रशाधि विधवा राज्यं मृते मयि गते सुते} % २५

\twolineshloka
{सत्यपाशनिबद्धस्तु राममाहूय चाब्रवीत्}
{कैकेय्या वञ्चितो राम राज्यं कुरु निगृह्य माम्} % २६

\twolineshloka
{त्वया वने तु वस्तव्यं कैकेयीभरतो नृपः}
{पितरञ्चैव कैकेयीं नमस्कृत्य प्रदक्षिणं} % २७

\twolineshloka
{कृत्वा नत्वा च कौशल्यां समाश्वस्य सलक्ष्मणः}
{सीतया भार्यया सार्धं सरथः ससुमन्त्रकः} % २८

\twolineshloka
{दत्वा दानानि विप्रेभ्यो दीनानाथेभ्य एव सः}
{मातृभिश्चैव विप्राद्यैः शोकार्तैर्निर्गतः पुरात्} % २९

\twolineshloka
{उषित्वा तमसातीरे रात्रौ पौरान् विहाय च}
{प्रभाते तमपश्यन्तोऽयोध्यां ते पुनरागताः} % ३०

\twolineshloka
{रुदन् राजाऽपि कौशल्या गृहमागात्सुदुःखितः}
{पौरा जना स्त्रियः सर्वा रुरुदू राजयोषितः} % ३१

\twolineshloka
{रामो रथस्थश्चीराढ्यः शृङ्गवेरपुरं ययौ}
{गुहेन पूजितस्तत्र इङ्गुदीमूलमाश्रितः} % ३२

\twolineshloka
{लक्ष्मणः स गुहो रात्रौ चक्रतुर्जागरं हि तौ}
{सुमन्त्रं सरथं त्यक्त्वा प्रातर्नावाथ जाह्नवीम्} % ३३

\twolineshloka
{रामलक्ष्मणसीताश्च तीर्णा आपुः प्रयागकम्}
{भरद्वाजं नमस्कृत्य चित्रकूटं गिरिं ययुः} % ३४

\twolineshloka
{वास्तुपूजां ततः कृत्वा स्थिता मन्दाकिनीतटे}
{सीतायै दर्शयामास चित्रकूटं च राघवः} % ३५

\twolineshloka
{नखैर्विदारयन्तं तां काकं तच्चक्षुराक्षिपत्}
{ऐषिकास्त्रेण शरणं प्राप्तो देवान् विहायसः} % ३६

\twolineshloka
{रामे वनं गते राजा षष्ठेऽह्नि निशि चाब्रवीत्}
{कौशल्यां स कथां पौर्वां यदज्ञानाद्धतः पुरा} % ३७

\twolineshloka
{कौमारे सरयूतीरे यज्ञदत्तकुमारकः}
{शब्दभेदाच्च कुम्भेन शब्दं कुर्वंश्च तत्पिता} % ३८

\twolineshloka
{शशाप विलपन्मात्रा शोकं कृत्वा रुदन्मुहुः}
{पुत्रं विना मरिष्यावस्त्वं च शोकान्मरिष्यसि} % ३९

\twolineshloka
{पुत्रं विना स्मरन् शोकात्कौशल्ये मरणं मम}
{कथामुक्त्वाऽथ हा राममुक्त्वा राजा दिवङ्गतः} % ४०

\twolineshloka
{सुप्तं मत्त्वाऽथ कौशल्या सुप्ता शोकार्तमेव सा}
{सुप्रभाते गायनाश्च सूतमागधवन्दिनः} % ४१

\twolineshloka
{प्रबोधका बोधयन्ति न च बुध्यत्यसौ मृतः}
{कौशल्या तं मृतं ज्ञात्वा हा हताऽस्मीति चाब्रवीत्} % ४२

\twolineshloka
{नरा नार्योऽथ रुरुदुरानीतो भरतस्तदा}
{वशिष्ठाद्यैः सशत्रुघ्नः शीघ्रं राजगृहात्पुरीम्} % ४३

\twolineshloka
{दृष्ट्वा सशोकां कैकेयीं निन्दयामास दुःखितः}
{अकीर्तिः पातिता मूर्ध्नि कौशल्यां स प्रशस्य च} % ४४

\twolineshloka
{पितरं तैलद्रोणिस्थं संस्कृत्य सरयूतटे}
{वशिष्ठाद्यैर्जनैरुक्तो राज्यं कुर्विति सोऽब्रवीत्} % ४५

\twolineshloka
{व्रजामि राममानेतुं रामो राजा मतो बली}
{शृङ्गवेरं प्रयागं च भरद्वाजेन भोजितः} % ४६

\twolineshloka
{नमस्कृत्य भरद्वाजं रामं लक्ष्मणमागतः}
{पिता स्वर्गं गतो राम अयोध्यायां नृपो भव} % ४७

\twolineshloka
{अहं वनं प्रयास्यामि त्वदादेशप्रतीक्षकः}
{रामः श्रुत्वा जलं दत्वा गृहीत्वा पादुके व्रज} % ४८

\threelineshloka
{राज्यायाहन्नयास्यामि सत्याच्चीरजटाधरः}
{रामोक्तो भरतश्चायान्नन्दिग्रामे स्थितो बली}
{त्यक्त्वायोध्यां पादुके ते पूज्य राज्यमपालयत्} % ४९

॥इत्यादिमहापुराणे आग्नेये रामायणेऽयोध्याकाण्डवर्णनं नाम षष्ठोऽध्यायः॥


\sect{सप्तमोऽध्यायः --- अरण्य-काण्ड-वर्णनम्}

\uvacha{नारद उवाच}
\twolineshloka
{रामो वशिष्ठं मातॄश्च नत्वाऽत्रिञ्च प्रणम्य सः}
{अनसूयाञ्च तत्पत्नीं शरभङ्गं सुतीक्ष्णकम्}% ।। १ ।।

\twolineshloka
{अगस्त्य भ्रातरं नत्वा अगस्त्यन्तत्प्रसादतः}
{धनुः खङ्गञ्च सम्प्राप्य दण्डकारण्यमागतः}% ।। २ ।।

\twolineshloka
{जनस्थाने पञ्चवट्यां स्थितो गोदावरीं तटे}
{तत्र सूर्पणखायाता भक्षितुं तान् भयङ्करी}% ।। ३ ।।

\twolineshloka
{रामं सुरूपं दृष्ट्वा सा कामिनी वाक्यमब्रवीत्}
{कस्त्वं कस्मात्समायातो भर्त्ता मे भव चार्थितः}% ।। ४ ।।

\twolineshloka
{एतौ च भक्षयिष्यामि इत्युक्त्वा तं समुद्यता }
{तस्या नासाञ्च कर्णौ च रामोक्तो लक्ष्मणोऽच्छिनत्}% ।। ५।।

\twolineshloka
{रक्तं क्षरन्ती प्रययौ खरं भ्रातरमब्रवीत्}
{मरीष्यामि विनासाऽहं खर जीवामि वै तदा}% ।। ६ ।।

\twolineshloka
{रामस्य भार्य्या सीताऽसौ तस्यासील्लक्ष्मणोऽनुजः}
{तेषां यद्रुधिरं सोष्णं पाययिष्यसि मां यदि}% ।। ७ ।।

\twolineshloka
{खरस्तथेति तामुक्त्वा यतुर्दृशसहस्त्रकैः}
{रक्षसां दूषणेनागाद्योद्धु त्रिशिरसा सह}% ।। ८ ।।

\twolineshloka
{रामं रामोऽपि युयुधे शरैर्विव्याध राक्षसान्}
{हस्त्यश्वरथपादातं बलं निन्ये यमक्षयम्}% ।। ९ ।।

\twolineshloka
{त्रिशीर्षाणं खरं रौद्रं युध्यन्तञ्चौव दूषणम्}
{ययौ सूर्पणखा लङ्कां रावणाग्रेपतद् भुवि}% ।। १० ।।

\twolineshloka
{अब्रवीद्रावणं क्रुद्धा न त्वं राजा न रक्षकः}
{खरादिहन्तू रामस्य सीतां भार्यां हरस्व च}% ।। ११ ।।

\twolineshloka
{रामलक्ष्मणरक्तस्य पानाज्जीवामि नान्यथा}
{तथेत्याह च तच्छ्रुत्वा मारीचं प्राह वै व्रज}% ।। १२ ।।

\twolineshloka
{स्वर्णचित्रमृगो भूत्वा रामलक्ष्मणकर्षकः}
{सीताग्रे तां हरिष्यामि अन्यथा मरणं तव}% ।। १३ ।।

\twolineshloka
{मारीचो रावणं प्राह रामो मृत्युर्धनुर्धरः}
{रावणादपि मर्त्तव्यं मर्त्तव्यं राघवादपि}% ।। १४ ।।

\twolineshloka
{अवश्यं यदि मर्त्तव्यं वरं रामो न रावणः}
{इति मत्वा मृगो भूत्वा सीताग्रे व्यचरन्मुहुः}% ।। १५ ।।

\twolineshloka
{सीतया प्रेरितो रामः शरेणाथावधीच्च तम्}
{म्रियमाणो मृगः प्राह हा सीते लक्ष्मणेति च}% ।। १६ ।।

\twolineshloka
{सौमित्रिः सीतयोक्तोऽथ विरुद्धं राममागतः}
{रावणोऽप्यहरत् सीतां हत्वा गृध्रं जटायुषम्}% ।। १७ ।।

\twolineshloka
{जटायुषा स भिन्नाङ्गः अङ्केनादाय जानकीम्}
{गतो लङ्कामशोकाख्ये धारयामास चाब्रवीत्}% ।। १८ ।।

\twolineshloka
{भव भार्य्या ममाग्र्या त्वं राक्षस्यो रक्ष्यतामियम् }
{रामो हत्वा तु मारीचं दृष्ट्वा लक्ष्मणमब्रवीत्}% ।। १९ ।।

\twolineshloka
{मायामृगोऽसौ सौमित्रे यथा त्वमिह चागतः }
{तथा सीता हृता नूनं नापश्यत् स गतोऽथ ताम्}% ।। २० ।।

\twolineshloka
{शुशोच विललापार्त्तो मां त्यक्त्वा क्क गतासि वै}
{लक्ष्मणाश्वासितो रामो मार्गयामास जानकीम्}% ।। २१ ।।

\threelineshloka
{दृष्ट्वा जटायुस्तं प्राह रावणो हृतवांश्च ताम्}
{मृतोऽथ संस्कृतस्तेन कबन्धञ्चावधीत्ततः}
{शापमुक्तोऽब्रवीद्रामं स त्वं सुग्रीवमाव्रज} %।। २२ ।।

॥इत्यादिमहापुराणे अग्नेये रामायणे अरण्यकाण्डवर्णनं नाम सप्तमोऽध्यायः॥


\sect{अष्टमोऽध्यायः --- किष्किन्धा-काण्ड-वर्णनम्}

\uvacha{नारद उवाच}


\twolineshloka
{रामः पम्पासरो गत्वा शोचन् स शर्वरीं ततः}
{हनूमताऽथ सुग्रीवं नीतो मित्रं चकार ह}% ।। १ ।।

\twolineshloka
{सप्त तालन् विनिर्भिद्य शरेणैकेन पश्यतः}
{पादेन दुन्दुभेः कायञ्चिक्षेप दशयोजनम्}% ।। २ ।।

\twolineshloka
{तद्रिपुं बालिनं हत्वा भ्रातरं वैरसारिणम्}
{किष्किन्धां कपिरज्यञ्च रुमान्तारां समर्पयत्}% ।। ३ ।।

\twolineshloka
{ऋष्यमूकेहरीशायकिष्किन्धेशोऽब्रवीत्सच }
{सीतां त्वं प्राश्यसेयद्वत् तथा राम करोमिते}% ।। ४ ।।

\twolineshloka
{तछ्रुत्वा माल्यवत्पृष्ठे चातुर्मास्यं चकारसः}
{किष्किन्धायाञ्च सुग्रीवो यदा नायाति दर्शनम्}% ।। ५ ।।

\twolineshloka
{तदाऽब्रवीत्तं रामोक्तं लक्ष्मणो व्रज राघवम्}
{न स सङ्कुचितः पन्था येन बाली हतो गतः}% ।। ६ ।।

\twolineshloka
{समये तिष्ठ सुग्रीव मा बालिपथमन्वगः}
{सुग्रीव आह संसक्तो गतं कालं न बुद्धवान्}% ।। ७ ।।

\twolineshloka
{इत्युक्त्वा स गतो रामं नत्वोवाच हरीश्वरः}
{आनीता वानराः सर्वे सीतायाश्च गवेषणे}% ।। ८ ।।

\twolineshloka
{त्वन्मतात् प्रेषयिष्यामि विचिन्वन्तु च जानकीम् }
{पूर्वादौ मासमायान्तु मासादूर्ध्वं निहन्मि तान्}% ।। ९ ।।

\twolineshloka
{इत्युक्ता वानराः पूर्वपश्चमोत्तरमार्गगाः}
{जग्मू रामं ससुग्रीवमपश्यन्तस्तु जानकीम्}% ।। १० ।।

\twolineshloka
{रामाङ्गुलीयं संगृह्य हनूमान् वानरैः सह}
{दक्षिणे मागयामास सुप्रभाया गुहान्तिके}% ।। ११ ।।

\twolineshloka
{मासादूर्ध्वञ्च विन्यस्ता अपश्यन्तस्तु जानकीम्}
{ऊचुर्वृथामरिष्यामो जटायुर्द्धन्य एव सः}% ।। १२ ।।

\twolineshloka
{सीतार्थे योऽत्यजत् प्राणान्रावणेन हतो रणे}
{तच्छ्रु त्वा प्राह सम्पातिर्विहाय कपिभक्षणम्}% ।। १३ ।।

\twolineshloka
{भ्राताऽसौ मे जटायुर्वै मयोड्डीनोऽर्कमण्डलम्}
{अर्क तापाद्रक्षितोऽगाद् दग्धपक्षोऽहमभ्रगः}% ।। १४ ।।

\twolineshloka
{रामवार्त्ताश्रवात् पक्षौ जातौ भूयोऽथ जानकीम्}
{पश्याम्यशोकवनिकागतां लङ्कागतां किल}% ।। १५ ।।

\twolineshloka
{शतयोजनविस्तीर्णे लवणाब्धौ त्रिकूटके}
{ज्ञात्वा रामं ससुग्रीवं वानराः कथयन्तु वै}% ।। १६ ।।

॥इत्यादिमहापुराणे आग्नेये रामायणे किष्किन्धाकाण्डर्णनं नाम अष्टमोऽध्यायः॥

\sect{नवमोऽध्यायः --- सुन्दरकाण्ड-वर्णनम्}

\uvacha{नारद उवाच}
\twolineshloka
{सम्पातिवचनं श्रुत्वा हनुमानङ्गदादयः}
{अब्धिं दृष्ट्वाऽब्रुवंस्तेऽब्धिं लङ्घयेत् को नु जीवयेत्}% ।। १ ।।

\twolineshloka
{कपीनां जीवनार्थाय रामकार्य्यप्रसिद्धये}
{शतयोजनविस्तीर्णं पुप्लुवेऽब्धिं स मारुतिः}% ।। २ ।।

\twolineshloka
{दृष्ट्वोत्थितञ्च मैनाकं सिंहिकां विनिपात्य च }
{लङ्कां दृष्ट्वा राक्षसानां गृहाणि वनितागृहे}% ।। ३ ।।

\twolineshloka
{दशग्रीवस्य कुम्भस्य कुम्भकर्णस्य रक्षसः}
{विभीषणस्येन्द्रजितो गृहेऽन्येषां च रक्षसाम्}% ।। ४ ।।

\twolineshloka
{नापश्यत् पानभूम्यादौ सीतां चिन्तापरायणः}
{अशोकवनिकां गत्वा दृष्टवाञ्छिंशपातले}% ।। ५ ।।

\twolineshloka
{राक्षसीरक्षितां सीतां भव भार्येति वादिनम्}
{रावणं शिशपास्थोऽथ नेति सीतान्तु वादिनीम्}% ।। ६ ।।

\twolineshloka
{भव भार्या रावणस्य राक्षसीर्वादिनीः कपिः}
{गते तु रावणे प्राह राजा दशरथोऽभवत्}% ।। ७ ।।

\twolineshloka
{रामोऽस्य लक्ष्ममः पुत्रौ वनवासङ्गतौ वरौ}
{रामपत्नी जानकी त्वं रावणेन हृता बलात्}% ।। ८ ।।

\twolineshloka
{रामः सुग्रीवमित्रस्त्वा मार्गयन् प्रैषयच्च माम् }
{साभिज्ञानञ्चांगुलीयं रामदत्तं गृहाण वै}% ।। ९ ।।

\twolineshloka
{सीताऽङ्गुलीयं जग्रह साऽपश्यन्मारुतिन्तरौ}
{भूयोऽग्रे चोपविष्टं तमुवाच यदि जीवति}% ।। १० ।।

\twolineshloka
{रामः कथं न नयति शङ्कितामब्रवीत् कपिः}
{रामः सीते न जानीते ज्ञात्वा त्वां स नयिष्यति}% ।। ११ ।।

\twolineshloka
{रावणं राक्षसं हत्वा सबलं देविमाशुच}
{साभिज्ञानं देहि मे त्वं मणिं सीताऽददत्कपौ}% ।। १२ ।।

\twolineshloka
{उवाच मां यथा रामो नयेच्छीघ्रं तथा कुरु}
{काकाक्षिपातनकथाम्प्रतियाहि हि शोकह}% ।। १३।।

\twolineshloka
{मणिं कथां गृहीत्वाह हनूमान्नेष्यते पतिः }
{अथवा ते त्वारा काचित् पृष्ठमारुह मे शुभे}% ।। १४ ।।

\twolineshloka
{अद्य त्वां दर्शयिष्यामि ससुग्रीवञ्च राघवम् }
{सीताऽब्रवीद्धनूमन्तं नयतां मां हि राघवः}% ।। १५ ।।

\twolineshloka
{हनूमान् स दशग्रीवदर्शनोपायमाकरोत्}
{वनं बभञ्च तत्पालान् हत्वा दन्तनखादिभिः}% ।। १६ ।।

\twolineshloka
{हत्वा तु किङ्करान् सर्वान् सप्त मन्त्रिसुतानपि}
{पुत्रमक्षं कुमारञ्च शक्रजिच्चबबन्ध तम्}% ।। १७ ।।

\twolineshloka
{नागपाशेन पिङ्गाक्षं दर्शयामास रावणम्}
{उवाच रावणः कस्त्वं मारुतिः प्राह रावणम्}% ।। १८ ।।

\twolineshloka
{रामदूतो राघवाय सीतां देहि मरिष्यसि}
{रामबाणैर्हतः सार्धं लङ्कास्थै राक्षसैर्ध्रुवम्}% ।। १९ ।।

\twolineshloka
{रावणो हन्तुमुद्युक्तो विभीषणनिवारितः}
{दीपयामास लाङ्गूलं दीप्तपुच्छः स मारुतिः}% ।। २० ।।

\twolineshloka
{दग्ध्वा लङ्कां राक्षसाश्च दृष्ट्वा सीतां प्रणम्य ताम्}
{समुद्रपारमागम्य दृष्ट्वा सीतेति चाब्रवीत्}% ।। २१ ।।

\twolineshloka
{अङ्गदादीनङ्गदाद्यैः पीत्वा मधुवने मधु}
{जित्वा दधिमुखादींश्च दृष्ट्वा रामं च तेऽब्रुवन्}% ।। २२ ।।

\twolineshloka
{दृष्टा सीतेति रामोऽपि हृष्टः पप्रच्छ मारुतिम्}
{कथं दृष्टा त्वया सीता किमुवाच च मां प्रति}% ।। २३ ।।

\twolineshloka
{सीताकथामृतेनैव सिञ्च मां कामवह्निगम्}
{हनूमानब्रवीद्रामं लङ्घयित्वाऽब्धिमागतः}% ।। २४ ।।

\twolineshloka
{सीतां दृष्ठ्वा पुरीं दग्ध्वा सीतामणिं गृहाण वै}
{हत्वा त्वं रावणं सीतां प्रास्यसे राम मा शुचः}% ।। २५ ।।

\twolineshloka
{गृहीत्वा तं मणिं रामो रुरोद विरहातुरः }
{मणिं दृष्ट्वा जानकी मे दृष्टा सीता नयस्व माम्}% ।। २६ ।।

\twolineshloka
{तथा विना न जीवामि सुग्रीवाद्यैः प्रबोधितः}
{समुद्रतीरं गतवान् तत्र रामं विभीषणः}% ।। २७ ।।

\twolineshloka
{गतस्तिरस्कृतो भ्रात्रा रावणेन दुरात्मना}
{रामाय देहि सीतां त्वमित्युक्तेनासहायवान्}% ।। २८ ।।

\twolineshloka
{रामो विभीषणं मित्रं लङ्कैवर्येऽभ्यषेचयत्}
{समुद्रं प्रार्थयन्मार्गं यदा नायात्तदा शरैः}% ।। २९ ।।

\twolineshloka
{भेदयामास रामञ्च उवाचाब्धि समागतः}
{नलेन सेतुं बद्‌ध्वाब्धौ लङ्कां व्रज गभीरकः}% ।। ३० ।।

\threelineshloka
{अहं त्वया कृतः पूर्वं रामोऽपि नलसेतुना}
{कृतेन तरुशैलाद्यैर्गतः पारं महोदधेः}
{वानरैः स सुवेलस्थः सह लङ्कां ददर्श वै} %।। ३१ ।।

॥इत्यादिमहापुराणे आग्नये रामायणे सुन्दरकाण्डवर्णनं नाम नवमोऽध्यायः॥

\sect{अष्टमोऽध्यायः --- युद्ध-काण्ड-वर्णनम्}

\uvacha{नारद उवाच}
\twolineshloka
{रामोक्तञ्चाङ्गदौ गत्वा रावणं प्राह जानकी}
{दीयतां राघवायाशु अन्यथा त्वं मरिप्यसि}% ।। १ ।।

\twolineshloka
{रावणो हन्तुमुद्युक्तः सङ्ग्रामोद्धतराक्षसः}
{रामयाह दशग्रीवो युद्धमेकं तु मन्यते}% ।। २ ।।

\twolineshloka
{रामो युद्धाय तच्छ्रुत्वा लङ्कां सकपिराययौ}
{वानरो हनुमान् मैन्दो द्विविदौ जाम्बवान्नलः}% ।। ३ ।।

\twolineshloka
{नीलस्तारोङ्गदो धूभ्रः सुषेणः केशरी गयः}
{पनसो विनतो रम्भः शरभः कथनो बली}% ।। ४ ।।

\twolineshloka
{गवाक्षो दधिवक्त्रश्च गवयो गन्धमादनः}
{एते चान्ये च सुग्रीव एतैर्युक्तो ह्यसङ्ख्यकैः}% ।। ५ ।।

\twolineshloka
{रक्षसां वानराणाञ्च युद्धं सङ्कुलमाबभौ}
{राक्षसा वानराञ्जघ्नुः शरशक्तिगदादिभिः}% ।। ६ ।।

\twolineshloka
{वानरा राक्षसाञ् जघ्नुर्नखदन्तशिलादिभिः}
{हस्त्थश्वरथपादातं राक्षसानां बलं हतम्} %।।७ ।।

\twolineshloka
{हनूमान् गिरिऋङ्गेण धूम्राक्षमवधीद्रिपुम्}
{अकम्पनं प्रहस्तञ्च युध्यन्तं नील आवधीत्}% ।। ८ ।।

\twolineshloka
{इन्द्रजिच्छरबन्धाच्च विमुक्तौ रामलक्ष्मणौ}
{तार्क्ष्यसन्दर्शनाद् बाणैर्जघ्ननू राक्षसं बलम्}% ।। ९ ।।

\twolineshloka
{रामः शरैर्जर्जरितं रावणञ्चाकरोद्रणे}
{रावणः कुम्बकर्णञ्च बौधयामास दुः खितः}% ।। १० ।।

\twolineshloka
{कुम्भकर्णः प्रबुद्धोऽथ पीत्वा घटसहस्त्रकम्}
{मद्यस्य महिषादीनां भक्षयित्वाह रावणम्}% ।। ११ ।।

\twolineshloka
{सीताया हरणं पापं कृतन्त्वं हि गुरुर्यतः}
{अतो गच्छामि युद्धाय रामं हन्मि सवानरम्}% ।। १२ ।।

\twolineshloka
{इत्युक्त्वा वानरान् सर्वान् कुम्भकर्णो ममर्द्द ह}
{गृहीतस्तेन सुग्रीवः कर्णनासं चकर्त्त सः}% ।। १३ ।।

\twolineshloka
{कर्णनासाविहीनोऽसौ भक्षयामास वानरान्}
{अथ कुम्भो निकुम्भश्च मकराक्षश्च राक्षसः}% ।। १४ ।।

\twolineshloka
{ततः पादौ ततश्छित्त्वा शिरो भूमौ व्यपातयत् }
{अथ कुम्भो निकुम्भश्च मकराक्षश्च राक्षसः}% ।। १५ ।।

\twolineshloka
{महोदरो महापार्श्वो मत्त उन्तत्तराक्षसः}
{प्रघसो भासकर्णश्च विरूपाक्षस्छ संयुगे}% ।। १६ ।।

\twolineshloka
{देवान्तको नरान्तश्च त्रिशिराश्चातिकायकः}
{रामेण लक्ष्मणेनैते वानरैः सविभीषणैः}% ।। १७ ।।

\twolineshloka
{युध्यमानास्तथाह्यन्ये राक्षसाभुवि पातिताः}
{इन्द्रजिन्मायया युध्यन् रामादीन् सम्बबन्ध ह}% ।। १८ ।।

\twolineshloka
{वरदत्तैर्नागबाणै रोषध्या तौ विशल्यकौ}
{विशल्ययाव्रणौ कृत्वा मारुत्यानीतपर्वने}% ।। १९ ।।

\twolineshloka
{हनूमान् धारयामास तत्रागं यत्र संश्थितः}
{निकुम्भिलायां होमादि कुर्वन्तं तं हि लक्ष्मणः}% ।। २० ।।

\twolineshloka
{शरैरिन्द्रजितं वीरं युद्धे तं तु व्यशातयत्}
{रावणः शोकसन्तप्तः सीतां हन्तुं समुद्यतः}% ।। २१ ।।

\twolineshloka
{अविन्ध्यवारितो राजरथस्यः सबलौययौ}
{इन्द्रोक्तो मातलीरामं रथस्थं प्रचकार तम्}% ।। २२ ।।

\twolineshloka
{रामरावणयोर्युद्धं रामरावणयोरिव}
{रावणो वानरान् हन्ति मारुत्याद्याश्च रावणम्}% ।। २३ ।।

\twolineshloka
{रामः शस्त्रैस्तमस्त्रैश्च ववर्ष जलदो यथा}
{तस्य ध्वजं स चिच्छेद रथमश्वांश्च सारथिम्}% ।। २४ ।।

\twolineshloka
{धनुर्बाहूञ्छिरांस्येव उत्तिष्ठन्ति शिरांसि हि}
{पैतामहेन हृदयं भित्त्वा रामेण रावणः}% ।। २५ ।।

\twolineshloka
{भूतले पातितः सर्वै राक्षसै रुरुदुः स्त्रियः}
{आश्वास्य तञ्च संस्कृत्य रामज्ञप्तो विभीषणः}% ।। २६ ।।

\twolineshloka
{हनृमतानयद्रामः सीतां शुद्धां गृहीतवान्}
{रामो वह्नौ प्रविष्टान्तां शुद्धामिन्द्रादिभिः स्तुतः}% ।। २७ ।।

\twolineshloka
{ब्रह्मणा दशरथेन त्वं विष्ण् राक्षसमर्द्दनः}
{इन्द्रौर्च्चितोऽमृतवृष्ट्या जीवयामास वानरान्}% ।। २८ ।।

\twolineshloka
{रामेण पूजिता जग्मुर्युद्धं दृष्ट्वा दिवञ्च ते }
{रामो विभीषणायादाल्लङ्कामभ्यर्च्य वानरान्}% ।। २९ ।।

\twolineshloka
{ससीतः पुष्पके स्थित्वा गतमार्गेण वै गतः}
{दर्शयन् वनदुर्गाणि सीतायै हृष्टमानसः}% ।। ३० ।।

\twolineshloka
{भरद्वाजं नमस्कृत्य नन्दिग्रामं समागतः}
{भरतेन नतश्चागादयोध्यान्तत्र संश्थितः}% ।। ३१ ।।

\twolineshloka
{वसिष्ठादीन्नमस्कृत्य कौशल्याञ्चैव केकयीम् }
{सुमित्रां प्राप्तराज्योऽथ द्विजादीन् सोऽभ्यपूजयत्}% ।। ३२ ।।

\twolineshloka
{वासुदेवं स्वमात्मानमश्वमेधैरथायजत्}
{सर्वदानानि स ददौ पालयामास स प्रजाः}% ।। ३३।।

\threelineshloka
{पुत्रवद्धर्म्मकामादीन् दुष्टनिग्रहणे रतः}
{सर्वधर्म्मपरो लोकः सर्वशस्या च मेदिनी}
{नाकालमरणञ्चासीद्रामे राज्यं प्रशासति} %।। ३४ ।।

इत्यादिमहापुराणे आग्नेये रामायणे युद्धकाण्डवर्णनं नाम दशमोऽध्यायः ॥

\sect{एकादशोऽध्यायः --- उत्तर-काण्ड-वर्णनम्}

\uvacha{नारद उवाच}

\twolineshloka
{राज्यस्थं राघवं जग्मुरगस्त्याद्याः सुपूजिताः}
{धन्यस्त्वं विजयी यस्मादिन्द्रजिद्विनिपातितः}% ।। १ ।।

\twolineshloka
{ब्रह्मात्मजः पुलस्त्योभूद् विश्रवास्तस्यनैकषी}
{पुष्पोत्कटाभूत् प्रथमा तत्पुत्रोभूद्धनेश्वरः}% ।। २ ।।

\twolineshloka
{नैकष्यां रावणो जज्ञे विंशद्बाहुर्द्दशाननः}
{तपसा ब्रह्मदत्तेन वरेण जितदैवतः}% ।। ३ ।।

\twolineshloka
{कुम्भकर्णः सनिद्रोऽभूद्धर्म्मिष्ठोऽभूद्धिभीषणः}
{स्वसा शूर्पणखा तेषां रावणान्मेघनादकः}% ।। ४ ।।

\twolineshloka
{इन्द्रं जित्वेन्द्रजिच्चाभूद्रावणादधिको बली}
{हतस्त्वया लक्ष्मणेन देवादेः क्षेममिच्छता}% ।। ५ ।।

\twolineshloka
{इत्युक्त्वा ते गता विप्रा अगस्त्याद्या नमस्कृताः}
{देवप्रार्थितरामोक्तः शत्रुघ्नो लवणार्द्दनः}% ।। ६ ।।

\twolineshloka
{अभूत् पूर्म्मथुरा काचिद् रामोक्तो भरतोऽवधीत्}
{कोटित्रयञ्च शैलूषपुत्राणां निशितैः शरैः}% ।। ७ ।।

\twolineshloka
{शैलूषं दुप्टगन्धर्वं सिन्धुतीरनिवासिनम्}
{तक्षञ्च पुष्करं पुत्रं स्थापयित्वाथ देशयोः}% ।। ८ ।।

\twolineshloka
{भरतोगात्सशत्रुघ्नो राघवं पूजयन् स्थितः}
{रामो दुष्टान्निहत्याजौ शिष्टान् सम्पाल्य मानवः}% ।। ९ ।।

\twolineshloka
{पुत्रौ कुशलवौ जातौ वाल्मीकेराश्रमे वरौ}
{लोकापवादात्त्यक्तायां ज्ञातौ सुचरितश्रवात्}% ।। १० ।।

\twolineshloka
{राज्येभिषिच्य ब्रह्माहमस्मीति ध्यानतत्परः}
{दशवर्षसहस्त्राणि दशवर्षसतानि च}% ।। ११ ।।

\twolineshloka
{राज्यं कृत्वा क्रतून् कृत्वा स्वर्गं देवार्च्चितो ययौ}
{सपौरः सानुजः सीतापुत्रो जनपदान्वितः}% ।। १२ ।।
                     
\uvacha{अग्निरुवाच}
\twolineshloka
{वाल्मीकिर्नारदाच्छ्रु त्वा रामायणमकारयत्}
{सविस्तरं यदेतच्च श्रृणुयात्स दिवं व्रजेत्}% ।। १३ ।।

॥इत्यादिमहापुराणे आग्नेये रामायणे उत्तरकाण्डवर्णनं नाम एकादशोऽध्यायः॥

\closesection