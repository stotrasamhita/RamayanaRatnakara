\chapt{श्रीमद्-भागवतम्}

\sect{दशमोऽध्यायः --- श्रीरामचरितम्}

\src{श्रीमद्-भागवतम्}{नवमः स्कन्धः}{अध्यायः १०}{श्लोकाः १---५६}
\vakta{शुकः}
\shrota{परीक्षितः}
\tags{concise, complete}
\notes{This chapter recounts the appearance of Lord Rāmacandra in the lineage of Mahārāja Khaṭvāṅga and details His divine exploits, including the slaying of Rāvaṇa and His triumphant return to Ayodhyā.}
\textlink{http://stotrasamhita.net/wiki/Bhagavatam/Skandha_09/Adhyaya_10}
\translink{}

\storymeta


\uvacha{श्रीशुक उवाच}

\twolineshloka
{खट्वाङ्गाद्दीर्घबाहुश्च रघुस्तस्मात्पृथुश्रवाः}
{अजस्ततो महाराजस्तस्माद्दशरथोऽभवत्} %1

\threelineshloka
{तस्यापि भगवानेष साक्षाद्ब्रह्ममयो हरि}
{अंशांशेन चतुर्धागात्पुत्रत्वं प्रार्थितः सुरै}
{रामलक्ष्मणभरत शत्रुघ्ना इति संज्ञया} %2

\twolineshloka
{तस्यानुचरितं राजन्नृषिभिस्तत्त्वदर्शिभिः}
{श्रुतं हि वर्णितं भूरि त्वया सीतापतेर्मुहुः} %3

\fourlineindentedshloka
{गुर्वर्थे त्यक्तराज्यो व्यचरदनुवनं पद्मपद्भ्यां प्रियायाः}
{पाणिस्पर्शाक्षमाभ्यां मृजितपथरुजो यो हरीन्द्रानुजाभ्याम्}
{वैरूप्याच्छूर्पणख्याः प्रियविरहरुषारोपितभ्रूविजृम्भ}
{त्रस्ताब्धिर्बद्धसेतुः खलदवदहनः कोसलेन्द्रोऽवतान्नः} %4

\twolineshloka
{विश्वामित्राध्वरे येन मारीचाद्या निशाचराः}
{पश्यतो लक्ष्मणस्यैव हता नैरृतपुङ्गवाः} %5

\fourlineindentedshloka
{यो लोकवीरसमितौ धनुरैशमुग्रं}
{सीतास्वयंवरगृहे त्रिशतोपनीतम्}
{आदाय बालगजलील इवेक्षुयष्टिं}
{सज्ज्यीकृतं नृप विकृष्य बभञ्ज मध्ये} %6

\fourlineindentedshloka
{जित्वानुरूपगुणशीलवयोऽङ्गरूपां}
{सीताभिधां श्रियमुरस्यभिलब्धमानाम्}
{मार्गे व्रजन्भृगुपतेर्व्यनयत्प्ररूढं}
{दर्पं महीमकृत यस्त्रिरराजबीजाम्} %7

\fourlineindentedshloka
{यः सत्यपाशपरिवीतपितुर्निदेशं}
{स्त्रैणस्य चापि शिरसा जगृहे सभार्यः}
{राज्यं श्रियं प्रणयिनः सुहृदो निवासं}
{त्यक्त्वा ययौ वनमसूनिव मुक्तसङ्गः} %8

\fourlineindentedshloka
{रक्षःस्वसुर्व्यकृत रूपमशुद्धबुद्धेस्}
{तस्याः खरत्रिशिरदूषणमुख्यबन्धून्}
{जघ्ने चतुर्दशसहस्रमपारणीय}
{कोदण्डपाणिरटमान उवास कृच्छ्रम्} %9

\fourlineindentedshloka
{सीताकथाश्रवणदीपितहृच्छयेन}
{सृष्टं विलोक्य नृपते दशकन्धरेण}
{जघ्नेऽद्भुतैणवपुषाश्रमतोऽपकृष्टो}
{मारीचमाशु विशिखेन यथा कमुग्रः} %10

\fourlineindentedshloka
{रक्षोऽधमेन वृकवद्विपिनेऽसमक्षं}
{वैदेहराजदुहितर्यपयापितायाम्}
{भ्रात्रा वने कृपणवत्प्रियया वियुक्तः}
{स्त्रीसङ्गिनां गतिमिति प्रथयंश्चचार} %11

\fourlineindentedshloka
{दग्ध्वात्मकृत्यहतकृत्यमहन्कबन्धं}
{सख्यं विधाय कपिभिर्दयितागतिं तैः}
{बुद्ध्वाथ वालिनि हते प्लवगेन्द्रसैन्यैर्}
{वेलामगात्स मनुजोऽजभवार्चिताङ्घ्रिः} %12

\fourlineindentedshloka
{यद्रोषविभ्रमविवृत्तकटाक्षपात}
{सम्भ्रान्तनक्रमकरो भयगीर्णघोषः}
{सिन्धुः शिरस्यर्हणं परिगृह्य रूपी}
{पादारविन्दमुपगम्य बभाष एतत्} %13

\fourlineindentedshloka
{न त्वां वयं जडधियो नु विदाम भूमन्}
{कूटस्थमादिपुरुषं जगतामधीशम्}
{यत्सत्त्वतः सुरगणा रजसः प्रजेशा}
{मन्योश्च भूतपतयः स भवान्गुणेशः} %14

\fourlineindentedshloka
{कामं प्रयाहि जहि विश्रवसोऽवमेहं}
{त्रैलोक्यरावणमवाप्नुहि वीर पत्नीम्}
{बध्नीहि सेतुमिह ते यशसो वितत्यै}
{गायन्ति दिग्विजयिनो यमुपेत्य भूपाः} %15

\fourlineindentedshloka
{बद्ध्वोदधौ रघुपतिर्विविधाद्रिकूटैः}
{सेतुं कपीन्द्रकरकम्पितभूरुहाङ्गैः}
{सुग्रीवनीलहनुमत्प्रमुखैरनीकैर्}
{लङ्कां विभीषणदृशाविशदग्रदग्धाम्} %16

\fourlineindentedshloka
{सा वानरेन्द्रबलरुद्धविहारकोष्ठ}
{श्रीद्वारगोपुरसदोवलभीविटङ्का}
{निर्भज्यमानधिषणध्वजहेमकुम्भ}
{शृङ्गाटका गजकुलैर्ह्रदिनीव घूर्णा} %17

\fourlineindentedshloka
{रक्षःपतिस्तदवलोक्य निकुम्भकुम्भ}
{धूम्राक्षदुर्मुखसुरान्तकनरान्तकादीन्}
{पुत्रं प्रहस्तमतिकायविकम्पनादीन्}
{सर्वानुगान्समहिनोदथ कुम्भकर्णम्} %18

\fourlineindentedshloka
{तां यातुधानपृतनामसिशूलचाप}
{प्रासर्ष्टिशक्तिशरतोमरखड्गदुर्गाम्}
{सुग्रीवलक्ष्मणमरुत्सुतगन्धमाद}
{नीलाङ्गदर्क्षपनसादिभिरन्वितोऽगात्} %19

\fourlineindentedshloka
{तेऽनीकपा रघुपतेरभिपत्य सर्वे}
{द्वन्द्वं वरूथमिभपत्तिरथाश्वयोधैः}
{जघ्नुर्द्रुमैर्गिरिगदेषुभिरङ्गदाद्याः}
{सीताभिमर्षहतमङ्गलरावणेशान्} %20

\fourlineindentedshloka
{रक्षःपतिः स्वबलनष्टिमवेक्ष्य रुष्ट}
{आरुह्य यानकमथाभिससार रामम्}
{स्वःस्यन्दने द्युमति मातलिनोपनीते}
{विभ्राजमानमहनन्निशितैः क्षुरप्रैः} %21

\fourlineindentedshloka
{रामस्तमाह पुरुषादपुरीष यन्नः}
{कान्तासमक्षमसतापहृता श्ववत्ते}
{त्यक्तत्रपस्य फलमद्य जुगुप्सितस्य}
{यच्छामि काल इव कर्तुरलङ्घ्यवीर्यः} %22

\fourlineindentedshloka
{एवं क्षिपन्धनुषि सन्धितमुत्ससर्ज}
{बाणं स वज्रमिव तद्धृदयं बिभेद}
{सोऽसृग्वमन्दशमुखैर्न्यपतद्विमानाद्}
{धाहेति जल्पति जने सुकृतीव रिक्तः} %23

\twolineshloka
{ततो निष्क्रम्य लङ्काया यातुधान्यः सहस्रशः}
{मन्दोदर्या समं तत्र प्ररुदन्त्य उपाद्रवन्} %24

\twolineshloka
{स्वान्स्वान्बन्धून्परिष्वज्य लक्ष्मणेषुभिरर्दितान्}
{रुरुदुः सुस्वरं दीना घ्नन्त्य आत्मानमात्मना} %25

\twolineshloka
{हा हताः स्म वयं नाथ लोकरावण रावण}
{कं यायाच्छरणं लङ्का त्वद्विहीना परार्दिता} %26

\twolineshloka
{न वै वेद महाभाग भवान्कामवशं गतः}
{तेजोऽनुभावं सीताया येन नीतो दशामिमाम्} %27

\twolineshloka
{कृतैषा विधवा लङ्का वयं च कुलनन्दन}
{देहः कृतोऽन्नं गृध्राणामात्मा नरकहेतवे} %28

\dnsub{श्रीशुक उवाच}


\twolineshloka
{स्वानां विभीषणश्चक्रे कोसलेन्द्रानुमोदितः}
{पितृमेधविधानेन यदुक्तं साम्परायिकम्} %29

\twolineshloka
{ततो ददर्श भगवानशोकवनिकाश्रमे}
{क्षामां स्वविरहव्याधिं शिंशपामूलमाश्रिताम्} %30

\twolineshloka
{रामः प्रियतमां भार्यां दीनां वीक्ष्यान्वकम्पत}
{आत्मसन्दर्शनाह्लाद विकसन्मुखपङ्कजाम्} %31

\twolineshloka
{आरोप्यारुरुहे यानं भ्रातृभ्यां हनुमद्युतः}
{विभीषणाय भगवान्दत्त्वा रक्षोगणेशताम्} %32

\twolineshloka
{लङ्कामायुश्च कल्पान्तं ययौ चीर्णव्रतः पुरीम्}
{अवकीर्यमाणः सुकुसुमैर्लोकपालार्पितैः पथि} %33

\twolineshloka
{उपगीयमानचरितः शतधृत्यादिभिर्मुदा}
{गोमूत्रयावकं श्रुत्वा भ्रातरं वल्कलाम्बरम्} %34

\twolineshloka
{महाकारुणिकोऽतप्यज्जटिलं स्थण्डिलेशयम्}
{भरतः प्राप्तमाकर्ण्य पौरामात्यपुरोहितैः} %35

\twolineshloka
{पादुके शिरसि न्यस्य रामं प्रत्युद्यतोऽग्रजम्}
{नन्दिग्रामात्स्वशिबिराद्गीतवादित्रनिःस्वनैः} %36

\twolineshloka
{ब्रह्मघोषेण च मुहुः पठद्भिर्ब्रह्मवादिभिः}
{स्वर्णकक्षपताकाभिर्हैमैश्चित्रध्वजै रथैः} %37

\twolineshloka
{सदश्वै रुक्मसन्नाहैर्भटैः पुरटवर्मभिः}
{श्रेणीभिर्वारमुख्याभिर्भृत्यैश्चैव पदानुगैः} %38

\twolineshloka
{पारमेष्ठ्यान्युपादाय पण्यान्युच्चावचानि च}
{पादयोर्न्यपतत्प्रेम्णा प्रक्लिन्नहृदयेक्षणः} %39

\twolineshloka
{पादुके न्यस्य पुरतः प्राञ्जलिर्बाष्पलोचनः}
{तमाश्लिष्य चिरं दोर्भ्यां स्नापयन्नेत्रजैर्जलैः} %40

\twolineshloka
{रामो लक्ष्मणसीताभ्यां विप्रेभ्यो येऽर्हसत्तमाः}
{तेभ्यः स्वयं नमश्चक्रे प्रजाभिश्च नमस्कृतः} %41

\twolineshloka
{धुन्वन्त उत्तरासङ्गान्पतिं वीक्ष्य चिरागतम्}
{उत्तराः कोसला माल्यैः किरन्तो ननृतुर्मुदा} %42

\twolineshloka
{पादुके भरतोऽगृह्णाच्चामरव्यजनोत्तमे}
{विभीषणः ससुग्रीवः श्वेतच्छत्रं मरुत्सुतः} %43

\twolineshloka
{धनुर्निषङ्गान्छत्रुघ्नः सीता तीर्थकमण्डलुम्}
{अबिभ्रदङ्गदः खड्गं हैमं चर्मर्क्षराण्नृप} %44

\twolineshloka
{पुष्पकस्थो नुतः स्त्रीभिः स्तूयमानश्च वन्दिभिः}
{विरेजे भगवान्राजन्ग्रहैश्चन्द्र इवोदितः} %45

\twolineshloka
{भ्रात्राभिनन्दितः सोऽथ सोत्सवां प्राविशत्पुरीम्}
{प्रविश्य राजभवनं गुरुपत्नीः स्वमातरम्} %46

\twolineshloka
{गुरून्वयस्यावरजान्पूजितः प्रत्यपूजयत्}
{वैदेही लक्ष्मणश्चैव यथावत्समुपेयतुः} %47

\twolineshloka
{पुत्रान्स्वमातरस्तास्तु प्राणांस्तन्व इवोत्थिताः}
{आरोप्याङ्केऽभिषिञ्चन्त्यो बाष्पौघैर्विजहुः शुचः} %48

\twolineshloka
{जटा निर्मुच्य विधिवत्कुलवृद्धैः समं गुरुः}
{अभ्यषिञ्चद्यथैवेन्द्रं चतुःसिन्धुजलादिभिः} %49

\twolineshloka
{एवं कृतशिरःस्नानः सुवासाः स्रग्व्यलङ्कृतः}
{स्वलङ्कृतैः सुवासोभिर्भ्रातृभिर्भार्यया बभौ} %50

\threelineshloka
{अग्रहीदासनं भ्रात्रा प्रणिपत्य प्रसादित}
{प्रजाः स्वधर्मनिरता वर्णाश्रमगुणान्विता}
{जुगोप पितृवद्रामो मेनिरे पितरं च तम्} %51

\twolineshloka
{त्रेतायां वर्तमानायां कालः कृतसमोऽभवत्}
{रामे राजनि धर्मज्ञे सर्वभूतसुखावहे} %52

\twolineshloka
{वनानि नद्यो गिरयो वर्षाणि द्वीपसिन्धवः}
{सर्वे कामदुघा आसन्प्रजानां भरतर्षभ} %53

\twolineshloka
{नाधिव्याधिजराग्लानि दुःखशोकभयक्लमाः}
{मृत्युश्चानिच्छतां नासीद्रामे राजन्यधोक्षजे} %54

\twolineshloka
{एकपत्नीव्रतधरो राजर्षिचरितः शुचिः}
{स्वधर्मं गृहमेधीयं शिक्षयन्स्वयमाचरत्} %55

\twolineshloka
{प्रेम्णाऽनुवृत्त्या शीलेन प्रश्रयावनता सती}
{भिया ह्रिया च भावज्ञा भर्तुः सीताऽहरन्मनः} %56

॥इति श्रीमद्भागवते महापुराणे पारमहंस्यां संहितायां नवमस्कन्धे दशमोऽध्यायः॥

\closesection