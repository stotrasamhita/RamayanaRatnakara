\sect{षोडशराजकीये रामचरितम्}

\src{श्रीमन्महाभारतम्}{द्रोण-पर्व}{अभिमन्युवधपर्व}{अध्यायाः ५९}
\vakta{नारदः}
\shrota{सृञ्जयः}
\tags{concise, complete}
\notes{Narada narrates the story of 16 great kings who no longer existed, to emphasise the impermanence of life. Narada extols Rama in this brief outline of Rama's life and kingship.}
% \textlink{http://stotrasamhita.net/wiki/Narayaniyam/Dashaka_34}
\translink{}

\storymeta

\dnsub{अध्यायः ५९}

\uvacha{नारद उवाच}

\threelineshloka
{रामं दाशरथिं चैव मृतं सृञ्जय शुश्रुम}
{यं प्रजा अन्वमोदन्त पिता पुत्रमिवौरसम्}
{असङ्ख्येया गुणा यस्मिन्नासन्नमिततेजसि}


\twolineshloka
{यश्चतुर्दशवर्षाणि निदेशात्पितुरच्युतः}
{वने वनितया सार्धमवसल्लक्ष्मणाग्रजः}


\twolineshloka
{जघान च जनस्थाने राक्षसान्मनुजर्षभः}
{तपस्विनां रक्षणार्थं सहस्राणि चतुर्दश}


\twolineshloka
{तत्रैव वसतस्तस्य रावणो नाम राक्षसः}
{जहार भार्यां वैदेहीं सम्मोह्यैनं सहानुजम्}


\twolineshloka
{`रामो हृतां राक्षसेन भार्यां श्रुत्वा जटायुषः}
{आतुरः शोकसन्तप्तो रामोऽगच्छद्धरीश्वरम्}


\twolineshloka
{तेन रामः सुसङ्गम्य वानरैश्च महाबलैः}
{आजगामोदधेः पारं सेतुं कृत्वा महार्णवे}


\twolineshloka
{तत्र हत्वा तु पौलस्त्यान्ससुहृद्गणबान्धवान्}
{मायाविनं महाघोरं रावणं लोककण्टकम्'}


\twolineshloka
{सुरासुरैरवध्यं तं देवब्राह्मणकण्टकम्}
{जघान स महाबाहुः पौलस्त्यं सगणं रणे}


\twolineshloka
{`हत्वा तत्र रिपुं सङ्ख्ये भार्यया सह सङ्गतः}
{स च लङ्केश्वरं चक्रे धर्मात्मानं विभीषणम्}


\twolineshloka
{भार्यया सह संयुक्तस्ततो वानरसेनया}
{अयोध्यामागतो वीरः पुष्पकेण विराजता}


\twolineshloka
{तत्र राजन्प्रविष्टः सन्नयोध्यायां महायशाः}
{मातॄर्वयस्यान्सचिवानृत्विजः सपुरोहितान्}


\twolineshloka
{शुश्रूषमाणः सततं मन्त्रिभिश्चाभिषेचितः}
{विसृज्य हरिराजानं हनुमन्तं सहाङ्गदम्}


\twolineshloka
{भ्रातरं भरतं वीरं शत्रुघ्नं चैव लक्ष्मणम्}
{पूजयन्परया प्रीत्या वैदेह्या चाभिपूजितः}


\twolineshloka
{दशवर्षसहस्राणि दशवर्षशतानि च}
{चतुःसगारपर्यन्तां पृथिवीमन्वशासत}


\twolineshloka
{अश्वमेधशतैरीजं क्रतुभिर्भूरिदक्षिणैः}
{यश्च विप्रप्रसादेन सर्वकामानवाप्य च'}


\twolineshloka
{सम्प्राप्य विधिवद्राज्यं सर्वभूतानुकम्पनः}
{`सर्वद्वीपानवष्टभ्य प्रजा धर्मेण पालयन्'}


\threelineshloka
{स निर्गलं मुख्यतममश्वमेधशतं प्रभुः}
{आजहार सुरेशस्य हविषा मुदमाहरन्}
{अन्यैश्च विविधैर्यज्ञैरीजे बहुगुणैर्नृपः}


\twolineshloka
{क्षुत्पिपासेऽजयद्रामः सर्वरोगांश्च देहिनाम्}
{सततं गुणसम्पन्नो दीप्यमानः स्वतेजसा}


\threelineshloka
{अतिसर्वाणि भूतानि रामो दाशरथिर्बभौ}
{ऋषीणां देवतानां च मानुषाणां च सर्वशः}
{पृथिव्यां सह वासोऽभूद्रामे राज्यं प्रशासति}


\twolineshloka
{नाहीयत तदा प्रामः प्राणिनां न तदा व्यथा}
{प्राणापानौ समावास्तां रामे राज्यं प्रशासति}


\twolineshloka
{पर्यदीप्यन्त तेजांसि तदाऽनर्थाश्च नाभवन्}
{दीर्घायुषः प्रजाः सर्वा युवा न म्रियते तदा}


\twolineshloka
{वेदैश्चतुर्भिः सुप्रीताः प्राप्नुवन्ति दिवौकसः}
{हव्यं कव्यं च विविधं निष्पूर्तं हुतमेव च}


\twolineshloka
{अदंशमशका देशा नष्टव्यालसरीसृपाः}
{नाप्यु प्राणभृतां मृत्युर्नाकाले ज्वलनोऽदहत्}


\twolineshloka
{अधर्मरुचयो लुब्धा मूर्खा वा नाभवंस्तदा}
{शिष्टेष्टप्राज्ञकर्माणः सर्वे वर्णास्तदाऽभवन्}


\twolineshloka
{स्वधां पूजां च रक्षोभिर्जनस्थाने प्रणाशिताम्}
{प्रादान्निहत्य रक्षांसि पितृदेवेभ्य ईश्वरः}


\twolineshloka
{सहस्रपुत्राः पुरुषा दशवर्षशतायुषः}
{न च ज्येष्ठाः कनिष्ठेभ्यस्तदा श्राद्धानि कुर्वते}


\twolineshloka
{श्यामो युवा लोहिताक्षो मत्तमातङ्गविक्रमः}
{आजानुबाहुः सुभुजः सिंहस्कन्धो महाबलः}


\twolineshloka
{दशवर्षसहस्राणि दशवर्षशतानि च}
{सर्वभूतमनःकान्तो रामो राज्यमकारयत्}


\twolineshloka
{रामो रामो राम इति प्रजानामभवत्कथा}
{रामभूतं जगदभूद्रामे राज्यं प्रशासति}


\twolineshloka
{चतुर्विधाः प्रजा रामः स्वर्गं नीत्वा दिवं गतः}
{आत्मेच्छया प्रतिष्ठाप्य राजवंशमिहाष्टधा}


\threelineshloka
{स चेन्ममार सृञ्जय चतुर्भद्रतरस्त्वया}
{पुत्रात्पुण्यतरस्तुभ्यं मा पुत्रमनुतप्यथाः}
{अयज्वानमदक्षिण्यमभि श्वैत्येत्युदाहरत्}


॥इति श्रीमन्महाभारते द्रोणपर्वणि अभिमन्युवध-पर्वणि एकोनषष्टितमोऽध्यायः॥६०॥

\closesection