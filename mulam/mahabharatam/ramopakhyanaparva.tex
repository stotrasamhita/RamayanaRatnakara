\sect{रामोपाख्यान-पर्व}

\src{श्रीमन्महाभारतम्}{वन-पर्व}{श्रीरामोपाख्यानपर्व}{अध्यायाः २७५--२९३}
\vakta{मार्कण्डेयः}
\shrota{युधिष्ठिरः}
\tags{concise, complete}
\notes{From this chapter begins the detailed account of Rama, as narrated by by Rishi Markandeya to Yudhishthira in the ``Ramopakhyana parva'', spanning 19 chapters and 750+ shlokas in the Vanaparva of the Mahabharata. It narrates the birth of Rama, his early life, exile, encounters with various beings, and the ultimate victory over Ravana.}
% \textlink{http://stotrasamhita.net/wiki/Narayaniyam/Dashaka_34}
\translink{}

\storymeta

\dnsub{अध्यायः २७५}\resetShloka

\uvacha{मार्कण्डेय उवाच}

\twolineshloka
{प्राप्तमप्रतिमं दुःखं रामेण भरतर्षभ}
{रक्षसा जानकी तस्य हृता भार्या बलीयसा}


\twolineshloka
{आश्रमाद्राक्षसेन्द्रेण रावणेन दुरात्मना}
{मायामास्थाय तरसा हत्वा गृध्रं जटायुषम्}


\twolineshloka
{प्रत्याजहार तां रामः सुग्रीवबलमाश्रितः}
{बद्ध्वा सेतुं समुद्रस्य दग्ध्वा लङ्कां शितैः शरैः}

\uvacha{युधिष्ठिर उवाच}


\twolineshloka
{कस्मिन् रामः कुले जातः किंवीर्यः किम्पराक्रमः}
{रावणः कस्य पुत्रो वा किं वैरं तस्य तेन ह}


\threelineshloka
{एतन्मे भगवन्सर्वं सम्यगाख्यातुमर्हसि}
{त्वया प्रत्यक्षतो दृष्टं यथासर्वमशेषतः}
{श्रोतुमिच्छामि चरितं रामस्याक्लिष्टकर्मणः}

\uvacha{मार्कण्डेय उवाच}


\twolineshloka
{अजो नामाभवद्राजा महानिक्ष्वाकुवंशजः}
{तस्य पुत्रो दशरथः शश्वत्स्वाध्यायवाञ्छुचिः}


\twolineshloka
{अभवंस्तस्य चत्वारः पुत्रा धर्मार्थकोविदाः}
{रामलक्ष्मणशत्रुघ्ना भरतश्च महाबलः}


\twolineshloka
{रामस्य माता कौसल्या कैकेयी भरतस्य तु}
{सुतौ लक्ष्मणशत्रुघ्नौ सुमित्रायाः परन्तपौ}


\twolineshloka
{विदेहराजो जनकः सीता तस्यात्मजा विमो}
{यां चकार स्वयं त्वष्टा रामस्य महिषीं प्रियाम्}


\twolineshloka
{एतद्रामस्य ते जन्म सीतायाश्च प्रकीर्तितम्}
{रावणस्यापि ते जन्म व्याख्यास्यामि जनेश्वर}


\twolineshloka
{पितामहो रावणस्य साक्षाद्देवः प्रजापतिः}
{स्वयभूः सर्वलोकानां प्रभुः स्रष्टा महातपाः}


\twolineshloka
{पुलस्त्यो नाम तस्यासीन्मानसो दयितः सुतः}
{तस्य वैश्रवणो नाम गवि पुत्रोऽभवत्प्रभुः}


\twolineshloka
{पितरं स समुत्सृज्य पितामहमुपस्थितः}
{तस्य कोपात्पिता राजन्ससर्जात्मानमात्मना}


\twolineshloka
{स जज्ञे विश्रवा नाम तस्यात्मार्धेन वै द्विजः}
{प्रतीकाराय सक्रोधस्ततो वैश्रवणस्य वै}


\twolineshloka
{पितामहस्तु प्रीतात्मा ददौ वैश्रवणस्य ह}
{अमरत्वं धनेशत्वं लोकपालत्वमेव च}


\twolineshloka
{ईशानन तथा सख्यं पुत्रं च नलकूवरम्}
{राजधानीनिवेसं च लङ्कां रक्षोगणान्विताम्}


\twolineshloka
{विमानं पुष्पकं नाम कामगं च ददौ प्रभुः}
{यक्षाणामाधिपत्यञ्च राजराजत्वमेव च}


॥इति श्रीमन्महाभारते अरण्यपर्वणि रामोपाख्यान-पर्वणि त्रिशततमोऽध्यायः॥२७५॥

\storymeta

\dnsub{अध्यायः २७६}\resetShloka
\uvacha{मार्कण्डेय उवाच}


\twolineshloka
{पुलस्त्यस्य तु यः क्रोधादर्धदेहोऽभवन्मुनिः}
{विश्रवानाम सक्रोधं पितरं राक्षसेश्वरः}


\twolineshloka
{बुबुधे तं तु सक्रोधं पितरं राक्षसेश्वरः}
{कुबेरस्तत्प्रसादार्थं यतते स्म सदा नृप}


\twolineshloka
{स राजराजो लङ्कायां न्यवसन्नरवाहनः}
{राक्षसीः प्रददौ तिस्रः पितुर्वै परिचारिकाः}


\twolineshloka
{ताः सदा तं महात्मानं सन्तोषयितुमुद्यताः}
{ऋषिं भरतशार्दूल नृत्यगीतविशारदाः}


\twolineshloka
{पुष्पोत्कटा च राका च मालिनी च विशापते}
{अन्योन्यस्पर्धयाराजञ्श्रेयस्कामाः सुमध्यमाः}


\twolineshloka
{स तासां भगवांस्तुष्टो महात्मा प्रददौ वरान्}
{लोकपालोपमान्पुत्रानकैकस्या यथेप्सितान्}


\twolineshloka
{पुष्पोत्कटायां जज्ञाते द्वौ पुत्रौ राक्षसेश्वरौ}
{कुम्भकर्णदशग्रीवौ बलेनाप्रतिमौ भुवि}


\twolineshloka
{मालिन जनयामास पुत्रमेकं विभीषणम्}
{राकार्या मिथुनं जज्ञे खरः शूर्पणखा तथा}


\twolineshloka
{विभीषणस्तु रूपेण सर्वेभ्योऽभ्यधिकोऽभवत्}
{स बभूव महाभागो धर्मगोप्ता क्रियारतिः}


\twolineshloka
{दशग्रीवस्तु सर्वेषां श्रेष्ठो राक्षसपुङ्गवः}
{महोत्साहो महावीर्यो महासत्वपराक्रमः}


\twolineshloka
{कुम्भकर्णो बलेनासीत्सर्वेभ्योऽभ्यधिको युधि}
{मायावी रणशौण्डश्च रौद्रश्च रजनीचरः}


\twolineshloka
{खरो धनुषि विक्रान्तो ब्रह्मद्विट् पिशिताशनः}
{सिद्धविघ्नकरी चापि रौद्री शूर्पणखा तदा}


\twolineshloka
{सर्वे वेदविदः शूराः सर्वेसुचरितव्रताः}
{ऊषुः पित्रा सह रता गन्धमादनपर्वते}


\twolineshloka
{ततो वैश्रवणं तत्र ददृशुर्नरवाहनम्}
{पित्रा सार्धं समासीनमृद्ध्या परमया युतम्}


\twolineshloka
{जातामर्षास्ततस्ते तु तपसे धृतनिश्चयाः}
{ब्रह्माणं तोषयामासुर्घोरेण तपसा तदा}


\twolineshloka
{अतिष्ठदेकपादेन सहस्रं परिवत्सरान्}
{वायुभक्षो दशग्रीवः पञ्चाग्निः सुसमाहितः}


\twolineshloka
{अधःशायी कुम्भकर्णो यताहारो यतव्रतः}
{विभीषणः शीर्णपर्णमेकमभ्यवहारयन्}


\twolineshloka
{उपवासरतिर्धीमान्सदा जप्यपरायणः}
{तमेव कालमातिष्ठत्तीव्रं तप उदारधीः}


\twolineshloka
{स्वरः शूर्पणखा चैव तेषां वै तप्यतां तपः}
{परिचर्यां च रक्षां च चक्रतुर्हष्टमानसौ}


\twolineshloka
{पूर्णे वर्षसहस्रेतु शिरश्छित्त्वा दशाननः}
{जुहोत्यग्नौ दुराधर्षस्तेनातुष्यज्जगत्प्रभुः}


\twolineshloka
{ततो ब्रह्मा स्वयं गत्वा तपसस्तान्न्यवारयत्}
{प्रलोभ्यवरदानेन सर्वानेवपृथक्पृथक्}

\uvacha{ब्राह्मोवाच}


\twolineshloka
{प्रीतोऽस्मि वो निवर्तध्वं वरान्वृणुत पुत्रकाः}
{यद्यदिष्टमृते त्वेकममरत्वं तथाऽस्तु तत्}


\twolineshloka
{यद्यदग्नौ हुतं सर्वं शिरस्ते महदीप्सया}
{तथैव तानि ते देहे भविष्यन्ति यथेप्सया}


\twolineshloka
{वैरूप्यं च न ते देहे कामरूपधरस्तथा}
{भविष्यसि रणेऽरीणां विजेता न च संशयः}

\uvacha{रावण उवाच}


\twolineshloka
{गन्धर्वदेवासुरतो यक्षराक्षसतस्तथा}
{सर्पकिन्नरभूतेभ्यो न मे भूयात्पराभवः}

\uvacha{ब्रह्मोवाच}


\twolineshloka
{य एते कीर्तिताः सर्वे न तेभ्योऽस्ति भयं तव}
{ऋते मनुष्याद्भद्रं ते तथा तद्विहितं मया}

\uvacha{मार्कण्डेय उवाच}


\twolineshloka
{एवमुक्तो दशग्रीवस्तुष्टः समभवत्तदा}
{अवमेने हि दुर्बुद्धिर्मनुष्यान्पुरुषादकः}


\threelineshloka
{कुम्भकर्णमथोवाच तथैव प्रपितामहः}
{वरं वृणीष्व भद्रं ते प्रीतोस्मीति पुनःपुनः}
{स वव्रे महतीं निद्रां तमसा ग्रस्तचेतनः}


\twolineshloka
{तथाभविष्यतीत्युक्त्वा विभीषणमुवाच ह}
{वरं वृणीष्व पुत्र त्वं प्रीतोऽस्मीति पुनःपुनः}

\uvacha{विभीषण उवाच}


\twolineshloka
{परमापद्गतस्यापि नाधर्मे मे मतिर्भवेत्}
{अशिक्षितं च भगवन्ब्रह्मास्त्रं प्रतिभातु मे}

\uvacha{ब्रह्मोवाच}


\twolineshloka
{यस्माद्राक्षसयोनौ ते जातस्यामित्रकर्शन}
{नाधर्मे धीयते बुद्धिरमरत्वं ददानि ते}

\uvacha{मार्कण्डेय उवाच}


\twolineshloka
{राक्षसस्तु वरंलब्ध्वा दशग्रीवो विशापते}
{लङ्कायाश्च्यावयामास युधि जित्वा धनेश्वरम्}


\twolineshloka
{हित्वास भगवाँल्लङ्कामाविशद्गन्धमादनम्}
{गन्धर्वयक्षानुगतो रक्षःकिपुरुषैः सह}


\twolineshloka
{विमानं पुष्पकं तस्य जहाराक्रम्य रावणः}
{शशाप तं वैश्रवणो न त्वामेतद्वहिष्यति}


\twolineshloka
{यस्तु त्वां समरे हन्ता तमेवैतद्वहिष्यति}
{अवमत्य गुरुं मां च क्षिप्रं त्वन्न भविष्यसि}


\twolineshloka
{विभीषणस्तु धर्मात्मा सतां मार्गमनुस्मरन्}
{अन्वगच्छन्महाराज श्रिया परमया युतः}


\twolineshloka
{तस्मै स भगवांस्तुष्टो भ्राता भ्रात्रे धनेश्वरः}
{सैनापत्यं ददौ धीमान्यक्षराक्षससेनयोः}


\twolineshloka
{राक्षसाः पुरुषादाश्च पिशाचाश्च महाबलाः}
{सर्वे समेत्य राजानमभ्यषिञ्चन्दशाननम्}


\twolineshloka
{दशग्रीवश्चदैत्यानां दानवानां बलोत्कटः}
{आक्रम्य रत्नान्यहरत्कामरूपी विहङ्गम}


\twolineshloka
{रावयामास लोकान्यत्तस्माद्रावण उच्यते}
{दशग्रीवः कामबलो देवानां भयमादधत्}


॥इति श्रीमन्महाभारते अरण्यपर्वणि रामोपाख्यान-पर्वणि त्रिशततमोऽध्यायः॥२७६॥

\storymeta

\dnsub{अध्यायः २७७}\resetShloka

\uvacha{मार्कण्डेय उवाच}


\twolineshloka
{ततो ब्रह्मर्षयः सर्वे सिद्धा देवर्षयस्तथा}
{हव्यवाहं पुरस्कृत्य ब्रह्माणं शरणं गताः}

\uvacha{अग्निरुवाच}


\twolineshloka
{योसौ विश्रवसः पुत्रो दशग्रीवो महाबलः}
{अवध्यो वरदानेन कृतो भगवता पुरा}


\twolineshloka
{स बाधते प्रजाः सर्वा विप्रकारैर्महाबलः}
{ततो नस्त्रातु भगवान्नान्यस्त्राता हि विद्यते}

\uvacha{ब्रह्मोवाच}


\twolineshloka
{न स देवासुरैः शक्यो युद्धे जेतुं विभावसो}
{विहितं तत्रयत्कार्यमभितस्तस्य निग्रहः}


\twolineshloka
{तदर्थमवतीर्णोऽसौ मन्नियोगाच्चतुर्भुजः}
{विष्णुः प्रहरतां श्रेष्ठः स तत्कर्म करिष्यति}

\uvacha{मार्कण्डेय उवाच}


\twolineshloka
{पितामहस्ततस्तेषां सन्निधौ शक्रमब्रवीत्}
{सर्वैर्देवगणैः सार्धं सभव त्वं महीतले}


\twolineshloka
{विष्णोः सहायानृक्षीषु वानरीषु च सर्वशः}
{जनयध्वं सुतान्वीरान्कामरूपबलान्वितान्}


\twolineshloka
{ते यथोक्ता भगवता तत्प्रतिश्रुत्य शासनम्}
{ससृजुर्देवगन्धर्वाः पुत्रान्वानररूपिणः}


\twolineshloka
{ततो भागानुभागेन देवगन्धर्वदानवाः}
{अवतर्तुं महीं सर्वे मन्त्रयामासुरञ्जसा}


\threelineshloka
{अवतेरुर्महीं स्वर्गादंशैश्च सहिताः सुराः}
{ऋषयश्च महात्मानः सिद्धाश्च सह किन्नरैः}
{चारणाश्चासृजन्घोरान्वानरान्वनचारिणः}


\twolineshloka
{यस्य देवस्य यद्रूपं वेषस्तेजश्च यद्विधम्}
{अजायन्त समास्तेन तस्य तस्य सुतास्तदा}


\twolineshloka
{तेषां समक्षं गन्धर्वी दुन्दुभीं नाम नामतः}
{शशास वरदो देवो गच्छ कार्यार्थसिद्धये}


\twolineshloka
{पितामहवचः श्रुत्वा गन्धर्वी दुन्दुभी ततः}
{मन्थरा मानुषे लोके कुब्जा समभवत्तदा}


\twolineshloka
{शक्रप्रभृतयश्चैव सर्वे ते सुरसत्तमाः}
{वानरर्क्षवरस्त्रीषु जनयामासुरात्मजान्}


\twolineshloka
{तेऽन्ववर्तन्पितॄन्सर्वे यशसा च बलेन च}
{भेत्तारो गिरिशृङ्गाणां सालतालशिलायुधाः}


\twolineshloka
{वज्चसंहननाः सर्वेसर्वेऽमोघवलास्तथा}
{कामवीर्यबलाश्चैवसर्वे बुद्धिविशारदाः}


\twolineshloka
{नागायुतसमप्राणा वायुवेगसमा जवे}
{यथेच्छविनिपाताश्च केचिदत्र वनौकसः}


\twolineshloka
{एवं विधाय तत्सर्वं भगवाँल्लोकभावनः}
{मन्थरां बोधयामास यद्यत्कार्यं त्वया तथा}


\twolineshloka
{सा तद्वच समाज्ञाय तथा चक्रे मनोजवा}
{इतश्चेतश्च गच्छन्ती वैरसन्धुक्षणे रता}


॥इति श्रीमन्महाभारते अरण्यपर्वणि रामोपाख्यान-पर्वणि त्रिशततमोऽध्यायः॥२७७॥

\storymeta

\dnsub{अध्यायः २७८}\resetShloka

\uvacha{युधिष्ठिर उवाच}

\twolineshloka
{उक्तं भगवता जन्म रामादीनां पृथक्पृथक्}
{प्रस्थानकारणं ब्रह्मञ्श्रोतुमिच्छामि कथ्यताम्}


\twolineshloka
{कथं दाशरथी वीरौ भ्रातरौ रामलक्ष्मणौ}
{प्रस्तापितौ वने ब्रह्मन्मैथिली च यशस्विनी}

\uvacha{मार्कण्डेय उवाच}


\twolineshloka
{जातपुत्रो दशरथः प्रीतिमानभवन्नृप}
{क्रियारतिर्धर्मरतः सततं वृद्धसेविता}


\twolineshloka
{क्रमेण चास्य ते पुत्रा व्यवर्धन्त महौजसः}
{वेदेषु सरहस्येषु धनुर्वेदेषु पारगाः}


\twolineshloka
{चरितब्रह्मचर्यास्ते कृतदाराश्च पार्थिव}
{दृष्ट्वा रामं दशरथः प्रीतिमानभवत्सुखी}


\twolineshloka
{ज्येष्ठो रामोऽभवत्तेषां रमयामास हि प्रजाः}
{मनोहरतया धीमान्पितुर्हृदयनन्दनः}


\twolineshloka
{ततः स राजा मतिमान्मत्वाऽऽत्मानं वयोधिकम्}
{मन्त्रयामास सचिवैर्मन्त्रज्ञैश्च पुरोहितैः}


\twolineshloka
{अभिषेकाय रामस्य यावैराज्येन भारत}
{प्राप्तकालं च ते सर्वे मेनिरे मन्त्रिसत्तमाः}


\twolineshloka
{लोहिताक्षं महाबाहुं मत्तमातङ्गगामिनम्}
{कम्बुग्रीवं महोरस्कं नीलकुञ्चितमूर्धजम्}


\twolineshloka
{दीप्यमानं श्रिया वीरं शक्रादनवरं बले}
{पारगं सर्वधर्माणां बृहस्पतिसमं मतौ}


\twolineshloka
{सर्वानुरक्तप्रकृतिं सर्वविद्याविशारदम्}
{जितेन्द्रियममित्राणामपि दृष्टिमनोहरम्}


\twolineshloka
{नियन्तारमसाधूनां गोप्तारं धर्मचारिणाम्}
{धृतिमन्तमनाधृष्यं जेतारमपराजितम्}


\twolineshloka
{पुत्रं राजा दशरथः कौसल्यानन्दवर्धनम्}
{सन्दृश्यपरमां प्रीतिमगच्छत्कुलनन्दनम्}


\twolineshloka
{चिन्तयंश्च महातेजा गुणान्रामस्य वीर्यवान्}
{अभ्यभाषत भद्रं ते प्रीयमाणः पुरोहितम्}


\twolineshloka
{अद्य पुष्यो निशि ब्रह्मन्पुण्यं योगमुपैष्यति}
{सभाराः सभ्रियन्तां मे रामश्चोपनिमन्त्र्यताम्}


\twolineshloka
{श्व एवपुष्यो भविता यत्ररामः सुतो मया}
{यौवराज्येऽभिषेक्तव्यः पौरेषु सहमन्त्रिभिः}


\twolineshloka
{इति तद्राजवचनं प्रतिश्रुत्याथ मन्थरा}
{कैकेयीमभिगम्येदं काले वचनमब्रवीत्}


\twolineshloka
{अद्य कैकेयि दौर्भाग्यं राज्ञा ते ख्यापितं महत्}
{आशीविषस्त्वां सङ्क्रुद्धश्छन्नो दशति दुर्भगे}


\twolineshloka
{सुभगा खलु कौसल्या यस्याः पुत्रोऽभिषेक्ष्यते}
{कुतो हि तव सौभाग्यं यस्याः पुत्रो न राज्यभाक्}


\twolineshloka
{सा तद्वचनमाज्ञाय सर्वाभरणभूषिता}
{वेदी विलग्नमध्येन बिभ्रती रूपमुत्तमम्}


\twolineshloka
{वविक्ते पतिमासाद्य हसन्तीव शुचिस्मिता}
{राजानं तर्जयन्तीव मधुरं वाक्यमब्रवीत्}


\threelineshloka
{सत्यप्रतिज्ञ यन्मे त्वं काममेकं विसृष्टवान्}
{उपाकुरुष्व तद्राजंस्तस्मान्मुञ्चस्व सङ्कटात्}
{तदद्य कुरु सत्यं मे वरं वरद भूपते}

\uvacha{राजोवाच}


\twolineshloka
{वरं ददानि ते हन्त तद्गृहाण यदिच्छसि}
{अवध्यो वध्यतां कोद्य वध्यः कोऽद्य विमुच्यताम्}


\twolineshloka
{धनं ददानि कस्याद् ह्रियतां कस्यरवापुनः}
{ब्राह्मणस्वादिहान्यत्रयत्किञ्चिद्वित्तमस्ति मे}


\twolineshloka
{पृथिव्यां राजराजोऽस्मि चातुर्वर्ण्यस्य रक्षिता}
{यस्तेऽभिलषितः कामो ब्रूहि कल्याणि माचिरम्}


\twolineshloka
{सातद्वचनमाज्ञाय परिगृह्य नराधिपम्}
{आत्मनो बलमाज्ञाय तत एनमुवाच ह}


\twolineshloka
{आभिषेचनिकं यत्ते रामार्थमुपकल्पितम्}
{भरतस्तदवाप्नोतु वनं गच्छतु राघवः}


\twolineshloka
{नव पञ्च च वर्षाणि दण्डकारण्यमाश्रितः}
{चीराजिनजटाधारी रामो भवतु तापसः}


\twolineshloka
{स तं राजा वरं श्रुत्वा विप्रियं दारुणोदयम्}
{दुःखार्तो भरतश्रेष्ठ न किञ्चिद्व्याजहार ह}


\twolineshloka
{ततस्तथोक्तं पितरं रामो विज्ञाय वीर्यवान्}
{वनं प्रतस्थे धर्मात्मा राजा सत्यो भवत्विति}


\twolineshloka
{तमन्वगच्छल्लक्ष्मीवान्धनुष्माँल्लक्ष्मणस्तदा}
{सीता च भार्या भद्रं ते वैदेही जनकात्मजा}


\twolineshloka
{ततो वनं गतेरामे राजा दशरथस्तदा}
{समयुज्यत देहस्य कालपर्यायधर्मणा}


\twolineshloka
{रामं तु गतमाज्ञाय राजानं च तथागतम्}
{अनार्या भरतं देवी कैकेयी वाक्यमब्रवीत्}


\twolineshloka
{गतोदशरथः स्वर्गं वनस्थौ रामलक्ष्मणौ}
{गृहाण राज्यंविपुलं क्षेमं निहतकण्टकम्}


\twolineshloka
{तामुवाच स धर्मात्मा नृशंसं बत ते कृतम्}
{पतिं हत्वाकुलं चेदमुत्साद्य धनलुब्धया}


\twolineshloka
{अयशः पातयित्वा मे मूर्ध्नि त्वं कुलपांसने}
{सकामा भव मे मातरित्युक्त्वा प्ररुरोद ह}


\twolineshloka
{स चारित्रं विशोध्याथ सर्वप्रकृतिसन्निधौ}
{अन्वयाद्धातरं रामं विनिवर्तनलालसः}


\twolineshloka
{कौसल्यां च सुमित्रां च कैकेयीं च सुदुःखितः}
{अग्रे प्रस्थाप्य यानैः स शत्रुघ्नसहितो ययौ}


\twolineshloka
{वसिष्ठवामदेवाभ्यां विप्रैश्चान्यैः सहस्रशः}
{पौरजानपदैः सार्धं रामानयनकाङ्क्षया}


\twolineshloka
{ददर्श चित्रकूटस्थं स रामं सहलक्ष्मणम्}
{तापसानामलङ्कारं धारयन्तं धनुर्धरम्}


\twolineshloka
{उवाच प्राञ्जलिर्भूत्वाप्रणिपत्य रघूत्तमम्}
{शशंस मरणं राज्ञः सोऽनाथांश्चापि कोसलान्}


% Check verse!
नाथ त्वं प्रतिपद्यस्व स्वराज्यमिति चोक्तवान्
\twolineshloka
{स तस्य वचनं श्रुत्वा रामः परमदुःखितः}
{चकार देवकल्पस्य पितुः स्नात्वोदकक्रियाम्}


\threelineshloka
{अब्रवीत्स तदारामो भ्रातरं भ्रातृवत्सलम्}
{पादुके मे भविष्येते राज्यगोप्त्र्यौ परन्तप}
{एवमस्त्विति तं प्राह भरतः प्रणतस्तदा}


\twolineshloka
{विसर्जितः स रामेण पितुर्वचनकारिणा}
{नन्दिग्रामेऽकरोद्राज्यं पुरस्कृत्यास्य पादुके}


\twolineshloka
{रामस्तु पुनराशङ्क्य पौरजानपदागमम्}
{प्रविवेश महारण्यं शरभङ्गाश्रमं प्रति}


\twolineshloka
{सत्कृत्य शरभङ्गं स दण्डकारण्यमाश्रितः}
{नदीं गोदावरीं रम्यामाश्रित्य न्यवसत्तदा}


\twolineshloka
{वसतस्तस्य रामस्य ततः शूर्पणखाकृतम्}
{खरेणासीन्महद्वैरं जनस्थाननिवासिना}


\twolineshloka
{रक्षार्थं तापसानां तु राघवो धर्मवत्सलः}
{चतुर्दशसहस्राणि जघान भुवि राक्षसान्}


\twolineshloka
{दूषणं च स्वरं चैवनिहत्य सुमहाबलौ}
{चक्रे क्षेमं पुनर्धीमान्धर्मारण्यं स राघवः}


\twolineshloka
{हतेषु तेषु रक्षःसु ततः शूर्पणखा पुनः}
{ययौ निकृत्तनासोष्ठी लङ्कां भ्रातुर्निवेशनम्}


\twolineshloka
{ततो रावणमभ्येत्य राक्षसी दुःखमूर्च्छिता}
{पपात पादयोर्भ्रातुः संशुष्करुधिरानना}


\twolineshloka
{तां तथा विकृतां दृष्ट्वा रावणः क्रोधमूर्च्छितः}
{उत्पपातासनात्क्रुद्धो दन्तैर्दन्तानुपस्पृशन्}


\twolineshloka
{स्वानमात्यान्विसृज्याथ विविक्ते तामुवाच सः}
{केनास्येवं कृता भद्रे मामचिन्त्यावमत्य च}


\twolineshloka
{कः शूलं तीक्ष्णमासाद्य सर्वगात्रेषु सेवते}
{कः शिरस्यग्निमाधाय विश्वस्तः स्वपते सुखम्}


\twolineshloka
{आशीविषं घोरतरं पादेन स्पृशतीह कः}
{सिंहं केसरिणं मत्तः स्पृष्ट्वा दंष्ट्रासु तिष्ठति}


\twolineshloka
{इत्येवं ब्रुवतस्तस्य नेत्रेभ्यस्तेजसोऽर्चिषः}
{निश्चेरुर्दह्यतो रात्रौ वृक्षस्येव स्वरन्ध्रतः}


\twolineshloka
{तस्य तत्सर्वमाचख्यौ भगिनी रामविक्रमम्}
{खरदूषणसंयुक्तं राक्षसानां पराभवम्}


\twolineshloka
{ततो ज्ञातिवधं श्रुत्वा रावणः कालचोदितः}
{रामस्य वधमाकाङ्क्षन्मारीचं मनसागमत्}


\twolineshloka
{स निश्चित्यततः कृत्यं सागरं लवणाकरम्}
{ऊर्ध्वमाचक्रमे राजा विधाय नगरे विधिम्}


\twolineshloka
{त्रिकूटं समतिक्रम्य कालपर्वतमेव च}
{ददर्श मकरावासं गम्भीरोदं महोदधिम्}


\twolineshloka
{तमतीत्याथ गोकर्णमभ्यगच्छद्दशाननः}
{दयितं स्तानमव्यग्रं शूलपाणेर्महात्मनः}


\twolineshloka
{तत्राभ्यगच्छन्मारीचं पूर्वामात्यं दशाननः}
{पुरा रामभयादेव तापसं प्रियजीवितम्}


॥इति श्रीमन्महाभारते अरण्यपर्वणि रामोपाख्यान-पर्वणि त्रिशततमोऽध्यायः॥२७८॥

\storymeta

\dnsub{अध्यायः २७९}\resetShloka

\uvacha{मार्कण्डेय उवाच}


\twolineshloka
{मारीचस्त्वथ सभ्रान्तो दृष्ट्वा रावणमागतम्}
{पूजयामास सत्कारैः फलमूलादिभिस्ततः}


\twolineshloka
{विश्रान्तं चैनमासीनमन्वासीनः स राक्षसः}
{उवाच प्रश्रितं वाक्यं वाक्यज्ञो वाक्यकोविदम्}


\twolineshloka
{न ते प्रकृतिमान्वर्णः कच्चित्क्षेमं पुरे तव}
{कच्चित्प्रकृतयः सर्वा भजन्ते त्वां यथा पुरा}


\twolineshloka
{किमिहागमने चापि कार्यं ते राक्षसेश्वर}
{कृतमित्येव तद्विद्धि यद्यपि स्यात्सुदुष्करम्}


\twolineshloka
{शशंस रावणस्तस्मै तत्सर्वं रामचेष्टितम्}
{समासेनैव कार्याणि क्रोधामर्षसमन्वितः}


\twolineshloka
{मारीचस्त्वब्रवीच्छ्रत्वा समासेनैव रावणम्}
{अलं ते राममासाद्य वीर्यज्ञो ह्यस्मि तस्य वै}


\threelineshloka
{बाणवेगं हि कस्तस्य शक्तः सोढुं महात्मनः}
{प्रव्रज्यायां हि मे हेतुः स एव पुरुषर्षभः}
{विनाशमुखमेतत्ते केनाख्यातं दुरात्मना}


\twolineshloka
{तमुवाचाथ सक्रोधो रावणः परिभर्त्सयन्}
{अकुर्वतोऽस्मद्वचनं स्यान्मृत्युरपि ते ध्रुवम्}


\twolineshloka
{मरीचश्चिन्तयामास विशिष्टान्मरणं वरम्}
{अवश्यं मरणे प्राप्ते करिष्याम्यस्य यन्मतम्}


\twolineshloka
{ततस्तं प्रत्युवाचाथ मारीचो रक्षसांवरम्}
{किं ते साह्यां मया कार्यं करिष्याम्यवशोपि तत्}


\twolineshloka
{तमब्रवीद्दशग्रीवो गच्छ सीतां प्रलोभय}
{रत्नशृङ्गो मृगो भूत्वा रत्नचित्रतनूरुहः}


\twolineshloka
{ध्रुवं सीता समालक्ष्यत्वां रामं चोदयिष्यति}
{अपक्रान्ते च काकुत्स्थे सीता वश्या भविष्यति}


\twolineshloka
{तामादायापनेष्यामि ततः स नभविष्यति}
{भार्यावियोगाद्दुर्बुद्धिरेतत्साह्यं कुरुष्व मे}


\twolineshloka
{इत्येवमुक्तोमारीचः कृत्वोदकमथात्मनः}
{रावणं पुरतो यान्तमन्वगच्छत्सुदुःखितः}


\twolineshloka
{ततस्तस्याश्रमं गत्वारामस्याक्लिष्टकर्मणः}
{चक्रतुस्तद्यथा सर्वमुभौ यत्पूर्वमन्त्रितम्}


\twolineshloka
{रावणस्तु यतिर्भूत्वा मुण्डः कुण्डीत्रिदण्डधृत्}
{मृगश्चभूत्वामारीचस्तं देशमुपजग्मतुः}


\twolineshloka
{दर्शयामास मारीचो वैदेहीं मृगरूपधृत्}
{चोदयामास तस्यार्थे सा रामं विधिचोदिता}


\twolineshloka
{रामस्तस्याः प्रियं कुर्वन्धनुरादाय सत्वरः}
{रक्षार्थे लक्ष्मणं न्यस्य प्रययौ मृगलिप्सया}


\twolineshloka
{स धन्वी बद्धतूणीरः खङ्गगोधाङ्गुलित्रवान्}
{अन्वधावन्मृगं रामो रुद्रस्तारामृगं यथा}


\twolineshloka
{सोऽन्तर्हितः पुनस्तस्य दर्शनं राक्षसो व्रजन्}
{चकर्ष महदध्वानं रामस्तं वुबुधे ततः}


\twolineshloka
{निशाचरं विदित्वा तं राघवः प्रतिभानवान्}
{अमोघं शरमादाय जघान मृगरूपिणम्}


\twolineshloka
{स रामवाणाभिहतः कृत्वा रामस्वरं तदा}
{हा सीते लक्ष्मणेत्येवं चुक्रोशार्तस्वरेण ह}


\twolineshloka
{शुश्राव तस्य वैदेही ततस्तां करुणां गिरम्}
{साप्रापतत्ततः सीता तामुवाचाथ लक्ष्मणः}


\twolineshloka
{अलं ते शङ्कया भीरु को रामं प्रहरिष्यति}
{मुहूर्ताद्द्रक्ष्यसे रामं भर्तारं त्वं शुचिस्मितम्}


\twolineshloka
{इत्युक्ता सा प्ररुदती पर्यशङ्कत लक्ष्मणम्}
{हता वै स्त्रीस्वभावेन शुद्धचारित्रभूषणा}


\twolineshloka
{सा तं परुषमारब्धा वक्तुं साध्वी पतिव्रता}
{नैष कामो भवेन्मूढ यं त्वं प्रार्थयसे हृदा}


\twolineshloka
{अप्यहंशस्त्रमादाय हन्यामात्मानमात्मना}
{पतेयं गिरिशृङ्गाद्वा विशेयं वा हुताशनम्}


\twolineshloka
{रामं भर्तारमुत्सुज्यन त्वहं त्वां कथञ्चन}
{निहीनमुपतिष्ठेयं शार्दूली क्रोष्टुकं यथा}


\twolineshloka
{एतादृशं वचः श्रुत्वा लक्ष्मणः प्रियराघ्नव}
{पिधायकर्णौ सद्वृत्तः प्रस्थितो येन राघवः}


\twolineshloka
{स रामस्य पदं गृह्य प्रससार धनुर्धरः}
{अवीक्षमाणो विम्बोष्ठीं प्रययौ लक्ष्मणस्तदा}


\twolineshloka
{एतस्मिन्नन्तरे रक्षो रावणः प्रत्यदृश्यत}
{अभव्यो भव्यरूपेण भस्मच्छन्न इवानलः}


\twolineshloka
{यतिवेपप्रतिच्छन्नो जिहीर्षुस्तामनिन्दिताम्}
{उपागच्छत्स वैदेहीं रावणः पापनिश्चयः}


\twolineshloka
{सा तमालक्ष्यसप्राप्तं धर्मज्ञा जनकात्मजा}
{निमन्त्रयामास तदा फलमूलाशनादिभिः}


\twolineshloka
{अवमत्यततः सर्वं स्वं रूपं प्रत्यपद्यत}
{सान्त्वयामास वैदेहीं कामी राक्षसपुङ्गवः}


\twolineshloka
{सीते राक्षसराजोऽहंरावणो नाम विश्रुतः}
{मम लङ्कापुरी नाम्ना रम्या पारे महोदधेः}


\twolineshloka
{तत्र त्वं नरनारीषु शोभिष्यसि मया सह}
{भार्या मे भव सुश्रोणि तापसं त्यज राघवम्}


\twolineshloka
{एवमादीनि वाक्यानि श्रुत्वा तस्याथ जानकी}
{पिधाय कर्णौ सुश्रोणी मैवमित्यब्रवीद्वचः}


\twolineshloka
{प्रपतेद्द्यौः सनक्षत्रा पृथिवी शकलीभवेत्}
{शुष्येत्तोयनिधौ तोयं चन्द्रः शीतांशुतां त्यजेत्}


\twolineshloka
{उष्णांशुत्वमथो जह्यादादित्यो वह्निरुष्णताम्}
{त्यक्त्वाशैत्यं भजेन्नाहं त्यजेयंरघुनन्दनम्}


\twolineshloka
{कथं हि भिन्नकरटं पद्मिनं वनगोचरम्}
{उपस्थाय महानागं करेणुः सूकरं स्पृशेत्}


\twolineshloka
{कथं हि पीत्वा माध्वीकं पीत्वा च मधुमाधवीम्}
{लोभं सौवीरके कुर्यान्नारी काचिदिति स्मरेः}


\twolineshloka
{इति सा तं समाभाष्य प्रविवेशाश्रमं ततः}
{क्रोधात्प्रस्फुरमाणौष्ठी विधुन्वाना करौ मुहुः}


% Check verse!
तामधिद्रुत्य सुश्रोणीं रावणः प्रत्यषेधयत्
\twolineshloka
{भर्त्सयित्वातु रूक्षेण स्वरेण गतचेतनाम्}
{मूर्धजेषु निजग्राह ऊर्ध्वमाचक्रमे ततः}


\twolineshloka
{तां ददर्श ततो गृध्रो जटायुर्गिरिगोचरः}
{रुदतीं रामरामेति हियमाणां तपस्विनीम्}


॥इति श्रीमन्महाभारते अरण्यपर्वणि रामोपाख्यान-पर्वणि त्रिशततमोऽध्यायः॥२७९॥

\storymeta

\dnsub{अध्यायः २८०}\resetShloka

\uvacha{गार्कण्डेय उवाच}


\twolineshloka
{सखा दशरथस्यासीज्जटायुररुणात्मजः}
{गृध्रराजो महावीरः सम्पातिर्यस्य सोदरः}


\twolineshloka
{स ददर्श तदा सीतां रावणाङ्कगतां स्नुषाम्}
{सक्रोधोऽभ्यद्रवत्पक्षी रावणं राक्षसेश्वरम्}


\twolineshloka
{अथैनमब्रवीद्गृध्रो मुञ्चमुञ्चेति मैथिलीम्}
{ध्रियमाणे मयि कथं हरिष्यसि निशाचर}


\twolineshloka
{न हिमे मोक्ष्यसे जीवन्यदि नोत्सृजसे वधूम्}
{उक्त्वैवं राक्षसेन्द्रं तं चकर्त नखरैर्भृशम्}


\twolineshloka
{पक्षतुण्डप्रहारैश्च शतशो जर्जरीकृतम्}
{चक्षार रुधिरं भूरि गिरिः प्रस्रवणैरिव}


\twolineshloka
{स वध्यमानो गृध्रेण रामप्रियहितैषिणा}
{खङ्गमादाय चिच्छेद भुजौ तस्य पतत्रिणः}


\twolineshloka
{निहत्य गृध्रराजं सभिन्नाभ्रशिखरोपमम्}
{ऊर्ध्वमाचक्रमे सीतां गृहीत्वाऽङ्केन राक्षसः}


\twolineshloka
{यत्रयत्रतु वैदेही पश्यत्याश्रममण्डलम्}
{सरोवा सरितो वाऽपि तत्र मुञ्चति भूषणम्}


\twolineshloka
{सा ददर्श गिरिप्रस्थे पञ्च वानरपुङ्गवान्}
{तत्र वासो महद्दिव्यमुत्ससर्ज मनस्विनी}


\twolineshloka
{तत्तेषां वानरेन्द्राणां पपात पवनोद्धतम्}
{मध्ये सुपीतं पञ्चानां विद्युन्मेघान्तरे यथा}


\twolineshloka
{अचिरेणातिचक्राम खेचरः खे चरन्निव}
{ददर्शाथ पुरीं रम्यां बहुद्वारां मनोरमाम्}


\twolineshloka
{प्राकारवप्रसबाधां निर्मितां विश्वकर्मणा}
{प्रविवेशपुरीं लङ्कां ससीतो राक्षसेश्वरः}


\twolineshloka
{एवं हृतायां वैदेह्यां रामो हत्वा महामृगम्}
{निवृत्तो ददृशे दूराद्भ्रातरं लक्ष्मणं तदा}


\twolineshloka
{कथमुत्सृज्य वैदेहीं वने राक्षससेविते}
{इति तं भ्रातरं दृष्ट्वा प्राप्तोऽसीति व्यगर्हयत्}


\twolineshloka
{मृगरूपधरेणाथ रक्षसासोपकर्षणम्}
{भ्रातुरागमनं चैवचिन्तयन्पर्यतप्यत}


\twolineshloka
{गर्हयन्नेव रामस्तु त्वरितस्तं समासदत्}
{अपि जीवति वैदेहीमिति पश्यामि लक्ष्मण}


\twolineshloka
{तस्य तत्सर्वमाचख्यौ सीताया लक्ष्मणो वचः}
{यदुक्तवत्यसदृशं वैदेही पश्चिमं वचः}


\twolineshloka
{दह्यमानेन तु हृदा रामोऽभ्यपतदाश्रमम्}
{स ददर्श तदा गृध्रं निहतं पर्वतोपमम्}


\twolineshloka
{राक्षसं शङ्कमानस्तं विकृष्य बलवद्धनुः}
{अभ्यधावत काकुत्स्थस्ततस्तं सहलक्ष्मणः}


\twolineshloka
{स तावुवाच तेजस्वी सहितौ रामलक्ष्मणौ}
{गृध्रराजेस्मि भद्रंवां सखा दशरथस्य वै}


\twolineshloka
{तस्य तद्वचनं श्रुत्वा सङ्गृह्य धनुषी शुभे}
{कोयं पितरमस्माकं नाम्नाऽऽहेत्यूचतुश्च तौ}


\twolineshloka
{ततो ददृशतुस्तौ तं छिन्नपक्षद्वयं खगम्}
{तयोः शशंस गृध्रस्तु सीतार्थे रावणाद्वधम्}


\twolineshloka
{अपृच्छद्राघवो गृध्रं रावणः कां दिशं गतः}
{तस्य गृध्रः शिरःकम्पैराचचक्षे ममार च}


\twolineshloka
{दक्षिणामिति काकुत्स्थो विदित्वाऽस्य तदिङ्गितम्}
{संस्कारं लम्भयामास सखायं पूजयन्पितुः}


\twolineshloka
{ततो दृष्ट्वाऽऽश्रमपदं व्यपविद्धबृसीकटम्}
{विध्वस्तकलशं शून्यं गोमायुशतसङ्कुलम्}


\twolineshloka
{दुःखशोकसमाविष्टौ वैदेहीहरणार्दितौ}
{जग्मतुर्दण्डकारण्यं दक्षिणेन परन्तपौ}


\twolineshloka
{वने महति तस्मिंस्तु रामः सौमित्रिणा सह}
{ददर्श मृगयूथनि द्रवमाणानि सर्वशः}


\twolineshloka
{शब्दं च घोरं सत्वानां दावाग्नरिववर्धतः}
{अपश्येतां मुहूर्ताच्च कबन्धं घोरदर्शनम्}


\twolineshloka
{मेघपर्वतसङ्काशं सालस्कन्धं महाभूजम्}
{उरोगतविशालाक्षं महोदरमहामुखम्}


\twolineshloka
{यदृच्छयाथ तद्रक्षः करे जग्राह लक्ष्मणम्}
{विषादमगमत्सद्यः सौमित्रिरथ भारत}


\twolineshloka
{स राममभिसप्रेक्ष्य कृष्यते येन तन्मुखम्}
{विषण्णश्चाब्रवीद्रामं पश्यावस्थामिमां मम}


\twolineshloka
{हरणं चैववैदेह्या मम चायमुपप्लवः}
{राज्यभ्रंशश्च भवतस्तातस्य मरणं तथा}


\twolineshloka
{नाहं त्वां मह वैदेह्या समेतं कोसलागतम्}
{द्रक्ष्यामि प्रथिते राज्येपितृपैतामहे स्थितम्}


\twolineshloka
{द्रक्ष्यन्त्यार्यस्य धन्या ये कुशलाजशमीदलैः}
{अभिषिक्तस्य वदनं सोमं शान्तघनं यथा}


\twolineshloka
{एवं बहुविधं धीमान्विललाप स लक्ष्मणः}
{तमुवाचाथकाकुत्स्थः सभ्रमेष्वप्यसभ्रमः}


\threelineshloka
{मा विषीद नरव्याघ्र नैष कश्चिन्मयि स्थिते}
{शक्तो धर्षयितुं वीर सुमित्रानन्दवर्धन}
{छिन्ध्यस्य दक्षिणं बाहुं छिन्नः सव्यो मया भुजः}


\twolineshloka
{इत्येवं वदता तस्य भुजो रामेण पातितः}
{खङ्गेन भृशतीक्ष्णेन निकृत्तस्तिलकाण्डवत्}


\twolineshloka
{ततोऽस्य दक्षिणं बाहुं स्वङ्गेनाजघ्निवान्बली}
{सौमित्रिरपि सप्रेक्ष्यभ्रातरं राघवं स्थितम्}


\twolineshloka
{पुनर्जघान पार्श्वे वै तद्रक्षो लक्ष्मणो भृशम्}
{गतासुरपतद्भूमौ कबन्धः सुमहांस्ततः}


\twolineshloka
{तस्य देहाद्विनिःसृत्य पुरुषो दिव्यदर्शनः}
{ददृशे दिवमास्थाय दिवि सूर्य इव ज्वलन्}


\twolineshloka
{पप्रच्छ रामस्तं वाग्मी कस्त्वं प्रब्रूहि पृच्छतः}
{कामया किमिदं चित्रमाश्चर्यं प्रतिभाति मे}


\twolineshloka
{तस्याचचक्षेगन्धर्वोविश्वावसुरहं नृप}
{प्राप्तो ब्राह्मणशपेन योनिं राक्षससेविताम्}


\twolineshloka
{रावणेन हृतासीता लङ्कायां सन्निवेशिता}
{सुग्रीवमभिगच्छस्वस ते साह्यं करिष्यति}


\twolineshloka
{एषा पम्पा शिवजला हंसकारण्डवायुता}
{ऋश्यमूकस्य शैलस्य सन्निकर्षे तटाकिनी}


\twolineshloka
{वसते तत्रसुग्रीवश्चतुर्भिः सचिवैः सह}
{भ्राता वानरराजस्य वालिनो हेममालिनः}


\twolineshloka
{तेन त्वं सहसङ्गम्य दुःखमूलं निवेदय}
{समानशीलो भवतः साहाय्यं स करिष्यति}


\twolineshloka
{एतावच्छक्यमस्माभिर्वक्तुं द्रष्टासि जानकीम्}
{ध्रुवं वानरराजस्य विदितो रावणालयः}


\twolineshloka
{इत्युक्त्वाऽन्तर्हितो दिव्यः पुरुषः स महाप्रभः}
{विस्मयं जग्मतुश्चोभौ प्रवीरौ रामलक्ष्मणौ}


॥इति श्रीमन्महाभारते अरण्यपर्वणि रामोपाख्यान-पर्वणि त्रिशततमोऽध्यायः॥२८०॥

\storymeta

\dnsub{अध्यायः २८१}\resetShloka

\uvacha{मार्कण्डेय उवाच}


\twolineshloka
{ततोऽविदूरे नलिनीं प्रभूतकमलोत्पलाम्}
{सीताहरणदुःखार्तः पम्पां रामः समासदत्}


\twolineshloka
{मारुतेन सुशीतेन सुखेनामृतगन्धिना}
{सेव्यमानो वने तस्मिञ्जगाम मनसा प्रियाम्}


\twolineshloka
{विललाप स राजेन्द्रस्तत्र कान्तामनुस्मरन्}
{कामबाणाभिसन्तप्तं सौमित्रिस्तमथाब्रवीत्}


\twolineshloka
{न त्वामेवंविधो भावः स्प्रष्टुमर्हति मानद}
{आत्मवन्तमिव व्याधिः पुरुषंवृद्धसेविनम्}


\twolineshloka
{प्रवृत्तिरुपलब्धा ते वैदेह्या रावणस्य च}
{तां त्वं पुरुषकारेण बुद्ध्या चैवोपपादय}


\twolineshloka
{अभिगच्छाव सुग्रीवं शैलस्थं हरिपुङ्गवम्}
{मयि शिष्ये च भृत्ये च सहाये च समाश्वस}


\twolineshloka
{एवं बहुविधैर्वाक्यैर्लक्ष्मणेन स राघवः}
{उक्तः प्रकृतिमापेदे कार्ये चानन्तरोऽभवत्}


\twolineshloka
{निषेव्य वारि पम्पायास्तर्पयित्वा पितृनपि}
{प्रतस्थतुरभौ वीरौ भ्रातरौ रामलक्ष्मणौ}


\twolineshloka
{तावृश्यमूकमभ्येत्य बहुमूलफलद्रुमम्}
{गिर्यग्रे वानरान्पञ्च वीरौ ददृशतुस्तदा}


\twolineshloka
{सुग्रीवः प्रेषयामास सचिवं वानरं तयोः}
{बुद्धिमन्तं हनूमन्तं हिमवन्तमिव स्थितम्}


\twolineshloka
{तेन सम्भाष्य पूर्वं तौ सुग्रीवमभिजग्मतुः}
{सख्यं वानरराजेन चक्रे रामस्तदा नृप}


\twolineshloka
{ततः सीतां हृतां श्रुत्वा सुग्रीवो वालिना कृतम्}
{दुःखमाख्यातवान्सर्वं रामायामिततेजसे}


\twolineshloka
{तद्वासो दर्शयामास तस्य कार्ये निवेदिते}
{वानराणां तु यत्सीता ह्रियमाणा व्यपासृजत्}


\twolineshloka
{तत्प्रत्ययकरं लब्ध्वा सुग्रीवं प्लवगाधिपम्}
{पृथिव्यां वानरैश्वर्ये स्वयंरामोऽभ्यषेचयत्}


\twolineshloka
{प्रतिजज्ञे चकाकुत्स्थः समरे वालिनो वधम्}
{सुग्रीवश्चापि वैदेह्याः पुनरानयनं नृप}


\twolineshloka
{इत्येवं समयं कृत्वाविश्वास्य च परस्परम्}
{अभ्येत्य सर्वकिष्किन्धां तस्थुर्युद्धाभिकाङ्क्षिणः}


\twolineshloka
{सुग्रीवः प्राप्यकिष्किन्धां ननादौघनिभस्वनः}
{नसाय् तन्ममृषे वाली तारा तं प्रत्यषेधयत्}


\twolineshloka
{यथानदतिसुग्रीवो बलवानेष वानरः}
{मन्ये चाश्रयवान्प्राप्तो न त्वं निष्क्रान्तुमर्हसि}


\twolineshloka
{हेममाली ततो वाली तारां ताराधिपाननाम्}
{प्रोवाच वचनं वाग्मी तां वानरपतिः पतिः}


% Check verse!
\twolineshloka
{सर्वभूतरुतज्ञा त्वं शृणु सर्वं कपीश्वर}
{केन चाश्रयवान्प्राप्तो ममैष भ्रातृगन्धिकः}

\twolineshloka
{चिन्तयित्वा मुहूर्तं तु तारा ताराधिपप्रभा}
{पतिमित्यब्रवीत्प्राज्ञा शृणु सर्वं कपीश्वर}


\twolineshloka
{हृतदारो महासत्वोरामो दशरथात्मजः}
{तुल्यारिमित्रतां प्राप्तः सुग्रीवेण धनुर्धरः}


\twolineshloka
{भ्राता चास्य महाबाहुः सौमित्रिरपराजितः}
{लक्ष्मणो नाम मेधावी स्थितः कार्यार्थसिद्धये}


\twolineshloka
{मैन्दश्च द्विविदश्चापि हनूमांश्चानिलात्मजः}
{जाम्बवानृक्षराजश्च सुग्रीवसचिवाः स्थिताः}


\twolineshloka
{सर्व एते महात्मानो बुद्धिमन्तो महाबलाः}
{अलं तव विनाशाय रामवीर्यव्यपाश्रयाः}


\twolineshloka
{तस्यास्तदाक्षिप्य वचो हितमुक्तं कपीश्वरः}
{पर्यशङ्कत तामीर्षुः सुग्रीवगतमानसाम्}


\twolineshloka
{तारां परुषमुक्त्वा तु निर्जगाम गुहामुखात्}
{स्थितं माल्यवतोऽभ्याशे सुग्रीवं सोभ्यभाषत}


\twolineshloka
{असकृत्त्वं मया क्लीव निर्जितो जीवितप्रियः}
{मुक्तो गच्छसि दुर्बुद्धे कथङ्कारं रणे पुनः}


\twolineshloka
{इत्युक्तः प्राहसुग्रीवो भ्रातरं हेतुमद्वचः}
{प्राप्तकालममित्रघ्नं रामं सम्बोधयन्निव}


\twolineshloka
{हृतराज्यस्य मे राजन्हृतदारस्य च त्वया}
{किं मे जीवितसामर्थ्यमिति विद्धि समागतम्}


\twolineshloka
{एवमुक्त्वाबहुविधं ततस्तौ सन्निपेततुः}
{समरे वालिसुग्रीवौ सालतालशिलायुधौ}


\twolineshloka
{उभौ जघ्नतुरन्योन्यमुभौ भूमौ निपेततुः}
{उभौ ववल्गतुश्चित्रं मुष्टिभिश्च निजघ्नतुः}


\twolineshloka
{उभौ रुधिरसंसिक्तौ नखदन्तपरिक्षतौ}
{शुशुभाते तदा वीरौ पुष्पिताविव किंशुकौ}


\twolineshloka
{न विशेषस्तयोर्युद्धे यदा कश्चन दृश्यते}
{सुग्रीवस्य तदा मालां हनुमान्कण्ठ आसजत्}


\twolineshloka
{स मालया तदा वीरः शुशुभे कण्ठसक्तया}
{श्रीमानिव महाशैलो मलयो मेघमालया}


\twolineshloka
{कृतचिह्नं तु सुग्रीवं रामो दृष्ट्वा महाधनुः}
{विचकर्ष धनुःश्रेष्ठं वालिमुद्दिश्य लक्षयन्}


\twolineshloka
{विष्फारस्तस्य धनुषो यन्त्रस्येव तदा बभौ}
{वितत्रास तदा वाली शरेणाभिहतो हृदि}


\twolineshloka
{स भिन्नहृदयो वाली वक्राच्छोणितमुद्वमन्}
{ददर्शावस्थितं रामं ततः सौमित्रिणा सह}


\twolineshloka
{गर्हयित्वास काकुत्स्थं पपात भुवि मूर्च्छितः}
{तारा ददर्श तं भूमौ तारापतिमिव च्युतम्}


\twolineshloka
{हते वालिनि सुग्रीवः किष्किन्धां प्रत्यपद्यत}
{तां च तारापतिमुखीं तारां निपतितेश्वराम्}


\twolineshloka
{रामस्तु चतुरो मासान्पृष्ठे माल्यवतः शुभे}
{निवासमकरोद्धीमान्सुग्रीवेणाभ्युपस्थितः}


\twolineshloka
{रावणोऽपिपुरीं गत्वालङ्कां कामबलात्कृतः}
{सीतां निवेशयामास भवने नन्दनोपमे}


\twolineshloka
{अशोकवनिकाभ्यासे तापसास्रमसन्निभे}
{भर्तृस्मरणतन्वङ्गी तापसीवेषधारिणी}


\twolineshloka
{उपवासतपःशीला ततः सा पृथुलेक्षणा}
{उवास दुःखवसतिं फलमूलकृताशना}


\twolineshloka
{दिदेश राक्षसीस्तत्ररक्षणे राक्षसाधिपः}
{प्रासासिशूलपरशुमुद्गरालातधारिणीः}


\twolineshloka
{द्व्यक्षीं त्र्यक्षीं ललाटक्षीं दीर्घजिह्वामजिह्विकाम्}
{त्रिस्तनीमेकपादां च त्रिजटामेकलोचनाम्}


\twolineshloka
{एताश्चान्याश्च दीप्ताक्ष्यः करभोत्कटमूर्धजाः}
{परिवार्यासते सीतां दिवारात्रमतन्द्रिताः}


\twolineshloka
{तास्तु तामायतापाङ्गीं पिशाच्यो दारुणस्वराः}
{तर्जयन्ति सदा रौद्राः परुषव्यञ्जनस्वराः}


\twolineshloka
{खादाम पाटयामैनां तिलशः प्रविभज्यताम्}
{येयं भर्तारमस्माकमवमत्येह जीवति}


\twolineshloka
{इत्येवं परिभर्त्सन्तीस्त्रासयानाः पुनः पुनः}
{भर्तृशोकसमाविष्टा निःश्वस्येदमुवाच ताः}


\twolineshloka
{आर्याः खादत मां शीघ्रं न मे लोभोस्ति जीविते}
{विना तं पुण्डरीकाक्षं नीलकुञ्चितमूर्धजम्}


\twolineshloka
{अद्यैवाहं निराहारा जीवितप्रियवर्जिता}
{शोषयिष्यामि गात्राणि बल्ली तलगता यथा}


\twolineshloka
{न त्वन्यमभिगच्छेयं पुमांसं राघवादृते}
{इति जानीत सत्यं मेक्रियतां यदनन्तरम्}


\twolineshloka
{तस्यास्तद्वचनं श्रुत्वा राक्षस्यस्ताः खरस्वनाः}
{आख्यातुं राक्षसेन्द्राय जन्मुस्तत्सर्वमादितः}


\twolineshloka
{गतासु तासु सर्वासु त्रिजटा नाम राक्षसी}
{सान्त्वयामास वैदेहीं धर्मज्ञा प्रियवादिनी}


\twolineshloka
{सीते वक्ष्यामि ते किञ्चिद्विश्वासं करु मे सखि}
{भयं त्वं त्यज वामोरु शृणु चेदं वचो मम}


\twolineshloka
{अविन्ध्यो नाम मेधावी वृद्धो राक्षसपुङ्गवः}
{स रामस्य हितान्वेषी त्वदर्थे मामचूचुदत्}


\twolineshloka
{सीता मद्वचनाद्वाच्या समाश्वास्य प्रसाद्य च}
{भर्ता तेकुशली रामोलक्ष्मणानुगतो बली}


\twolineshloka
{सख्यं वानरराजेन शक्रप्रतिमतेजसा}
{कृतवान्राघवः श्रीमांस्त्वदर्थे च समुद्यतः}


\twolineshloka
{मा च ते भूद्भयं भीरु रावणाल्लोकगर्हितात्}
{नलकूबरशापेन रक्षिता ह्यसि नन्दिनि}


\twolineshloka
{शप्तो ह्येष पुरा पापो वधूं रम्भां परामृशन्}
{न शक्रोत्यवशां नारीमुपैतुमजितेन्द्रियः}


\twolineshloka
{क्षिप्रमेष्यति ते भर्ता सुग्रीवेणाभिरक्षितः}
{सौमित्रिसहितो धीमांस्त्वां चेतो मोक्षयिष्यति}


\twolineshloka
{स्वप्ना हि सुमहाघोरा दृष्टा मेऽनिष्टदर्शनाः}
{विनाशायास्य दुर्बुद्धेः पौलस्त्यस्य कुलस्य च}


\twolineshloka
{दारुणो ह्येष दुष्टात्मा क्षुद्रकर्मा निशाचरः}
{स्वभावाच्छीलदोषेण सर्वेषां भयवर्धनः}


\twolineshloka
{स्पर्धते सर्वदेवैर्यः कालोपहतचेतनः}
{मया विनासलिङ्गानि स्वप्ने दृष्टानि तस्य वै}


\twolineshloka
{तैलाभिषिक्तो विकचो मज्जनप्के दशाननः}
{असकृत्स्वरयुक्ते तु रथे नृत्यन्निव स्थितः}


\twolineshloka
{कुम्भकर्णादयश्चेमे नग्नाः पतितमूर्धजाः}
{गच्छन्ति दक्षिणामाशां रक्तमाल्यानुलेपनाः}


\twolineshloka
{श्वेतातपत्रः सोष्णीषः शुक्लमाल्यानुलेपनः}
{श्वेतपर्वतमारूढ एक एव विभीषणः}


\twolineshloka
{सचिवाश्चास्य चत्वारः शुक्लमाल्यानुलेपनाः}
{श्वेतपर्वतमारूढा मोक्ष्यन्तेऽस्मान्महाभयात्}


\twolineshloka
{रामस्यास्त्रेण पृथिवी परिक्षिप्ता ससागरा}
{यशसा पृथिवीं कृत्स्नां पूरयिष्यति ते पतिः}


\twolineshloka
{हस्तिसक्थिसमारूढो भुञ्जानो मधुपायसम्}
{लक्ष्मणश्च मया दृष्टो दिधक्षुः सर्वतो दिशम्}


\twolineshloka
{रुदती रुधिरार्द्राङ्गी व्याघ्रेण परिरक्षिता}
{असकृत्त्वं मया दृष्टा गच्छन्ती दिशमुत्तराम्}


\twolineshloka
{हर्षमेष्यसि वैदेहि क्षिप्रं भर्त्रा समन्विता}
{राघवेण सहभ्रात्रा सीते त्वमचिरादिव}


\twolineshloka
{इत्येतन्मृगशावाक्षी तच्छ्रुत्वा त्रिजटावचः}
{बभूवाशावती बाला पुनर्भर्तृसमागमे}


\twolineshloka
{तावदभ्यागता रौद्र्यः पिशाच्यस्ताःसुदारुणाः}
{ददृशुस्तां त्रिजटया सहासीनां यथापुरम्}


॥इति श्रीमन्महाभारते अरण्यपर्वणि रामोपाख्यान-पर्वणि त्रिशततमोऽध्यायः॥२८१॥

\storymeta

\dnsub{अध्यायः २८२}\resetShloka

\uvacha{मार्कण्डेय उवाच}


\twolineshloka
{ततस्तां भर्तृशोकार्तां दीनां मलिनवाससम्}
{मणिशेषाभ्यलङ्कारां रुदतीं च पतिव्रताम्}


\twolineshloka
{राक्षसीभिरुपास्यन्तीं समासीनां शिलातले}
{रावणःकामबाणार्तो ददर्शोपससर्प च}


\twolineshloka
{देवदानवगन्धर्वयक्षकिपुरुषैर्युधि}
{अजितोशोकवनिकां ययौ कन्दर्पपीडितः}


\twolineshloka
{दिव्याम्बरधरः श्रीमन्सुमृष्टमणिकुण्डलः}
{विचित्रमाल्यमुकुटो वसन्त इव मूर्तिमान्}


\twolineshloka
{न कल्पवृक्षसदृशोयत्नादपि विभूषितः}
{श्मशानचैत्यद्रुमवद्भूषितोऽपि भयङ्करः}


\twolineshloka
{स तस्यास्तनुमध्यायाः समीपे रजनीचरः}
{ददृशे रोहिणीमेत्य शनैश्चर इव ग्रैहः}


\twolineshloka
{स तामामन्त्र्य सुश्रोणीं पुष्पकेतुशराहतः}
{इदमित्यब्रवीद्वाक्यं त्रस्तां रौहीमिवाबलाम्}


\twolineshloka
{सीते पर्याप्तमेतावत्कृतोभर्तुरनुग्रहः}
{प्रसादं कुरु तन्वङ्गि क्रियतां परिकर्म ते}


\twolineshloka
{भजस्वमां वरारोहे महार्हाभरणाम्बरा}
{भवमे सर्वनारीणामुत्तमा वरवर्णिनी}


\threelineshloka
{सन्ति मे देवकन्याश्च गन्धर्वाणां च योषितः}
{सन्ति दानवकन्याश्च दैत्यानां चापि योषितः}
{तासामद्य विशालाक्षि सर्वासां मे भवोत्तमा}


\twolineshloka
{चतुर्दश पिशाचीनां कोट्यो मे वचने स्थिताः}
{द्विस्तावत्पुरुषादानां रक्षसां भीमकर्मणाम्}


\twolineshloka
{ततो मे त्रिगुणा यक्षा ये मद्वचनकारिणः}
{केचिदेव धनाध्यक्षं भ्रातरं मे समाश्रिताः}


\twolineshloka
{गन्दर्वाप्सरसो भद्रे मामापानगतं सदा}
{उपतिष्ठन्ति वामोरु यथैव भ्रातरं मम}


\twolineshloka
{पुत्रोऽहमपि विप्रर्षेः साक्षाद्विश्रवसो मुनेः}
{पञ्चमो लोकपालानामिति मे प्रथितं यशः}


\twolineshloka
{दिव्यानि भक्ष्यभोज्यानि पानानि विविधानि च}
{यथैव त्रिदशेशस्यतथैव मम भामिनि}


\twolineshloka
{क्षीयतां दुष्कृतं कर्म वनवासकृतं तव}
{भार्या मे भवसुश्रोणि यथा मण्डोदरीतथा}


\twolineshloka
{इत्युक्ता तेन वैदेही परिवृत्य सुभानना}
{तृणमन्तरतः कृत्वा तमुवाच निशाचरम्}


\twolineshloka
{अशिवेनातिवामोरूरजस्रं नेत्रवारिणा}
{स्तनावपतितौ बाला संहतावभिवर्षती}


\twolineshloka
{व्यवस्थाप्यकथञ्चित्सा विषादादतिमोहिता}
{उवाच वाक्यं तं क्षुद्रं वैदेही पतिदेवता}


\threelineshloka
{असकृद्वदतो वाक्यमीदृशं राक्षसेश्वर}
{विषादयुक्तमेतत्ते मया श्रुतमभाग्यया}
{तद्भद्रमुख भद्रं ते मानसं विनिवर्त्यताम्}


\twolineshloka
{परदाराऽस्म्यलभ्या च सततं च पतिव्रता}
{न चैवौपयिकी भार्य मानुषी तव राक्षस}


\twolineshloka
{विवशां धर्षयित्वच कां त्वं प्रीतिमवाप्स्यसि}
{न च पालयसे धर्मं लोकपालसमः कथम्}


\twolineshloka
{भ्रातरं राजराजं तं महेश्वरसस्वं प्रभुम्}
{धनेश्वरं व्यपदिशन्कथं त्विह न लज्जसे}


\twolineshloka
{इत्युक्त्वा प्रारुदत्सीता कम्पयन्ती पयोधरौ}
{शिरोधरां च तन्वङ्गी मुस्वं प्रच्छाद्यवाससा}


\twolineshloka
{तस्य रुदत्या भामित्या दीर्घा वेणी सुसयता}
{ददृशे स्वसिता स्निग्धा काली व्यालीव मूर्धनि}


\twolineshloka
{श्रुत्वा तद्रावणो वाक्यं सीतयोक्तं सुनिषुरम्}
{प्रत्याख्यातोऽपिदुर्मेधाः पुनरेवाब्रवीद्वचः}


\twolineshloka
{काममङ्गनि मे सीते दुनोतु मकरध्वजः}
{नत्वामकामां सुश्रोणीं समेप्ये चारुहासिनीम्}


\twolineshloka
{किन्नु शक्यं मया कर्तुं यत्त्वमद्यापिमानुषम्}
{आहारभूतमस्माकं राममेवानुरुध्यसे}


\twolineshloka
{इत्युक्त्वा तामनिन्द्याङ्गीं स राक्षसमहेश्वरः}
{तत्रैवान्तर्हितो भूत्वा जगामाभिमतां दिशम्}


\twolineshloka
{राक्षसीभिः परिवृतावैदेही शोककशिन्ता}
{सेव्यमाना त्रिजटया तत्रैव न्यवसत्तदा}


॥इति श्रीमन्महाभारते अरण्यपर्वणि रामोपाख्यान-पर्वणि त्रिशततमोऽध्यायः॥२८२॥

\storymeta

\dnsub{अध्यायः २८३}\resetShloka

\uvacha{मार्कण्डेय उवाच}


\twolineshloka
{राघवः सहसौमित्रिः सुग्रीवेणाभिपालितः}
{वसन्माल्यवतः पृष्ठे ददर्श विमलं नभः}


\twolineshloka
{सदृष्ट्वाविमले व्योम्नि निर्मलं शसलक्षणम्}
{ग्रहनक्षत्रताराभिरनुयान्तममित्रहा}


\twolineshloka
{कुमुदोत्पलपद्मानां गन्धमादाय वायुना}
{महीधरस्थः शीतेन सहसाप्रतिबोधितः}


\twolineshloka
{प्रभाते लक्ष्मणं वीरमभ्यभाषत दुर्मनाः}
{सीतां संस्मृत् यधर्मात्मा रुद्धां राक्षसवेश्मनि}


\twolineshloka
{गच्छ लक्ष्मण जानीहि किष्किन्दायां कपीश्वरम्}
{प्रमत्तं ग्राम्यधर्मेषु कृतघ्नं स्वार्थपण्डितम्}


\twolineshloka
{योसौ कुलाधमो मूढो मया राज्येऽभिषेचितः}
{सर्ववानरगोपुच्छा यमृक्षाश्च भजन्ति वै}


\twolineshloka
{यदर्थं निहतो बाली मया रघुकुलोद्वह}
{त्वया सहमहाबाहो किष्किन्धोपवने तदा}


\twolineshloka
{कृतघ्नं तमहं मन्ये वानरापशदं भुवि}
{यो मामेवङ्गतो मूढो न जानीतेऽद्य लक्ष्मण}


\twolineshloka
{असौ मन्ये न जानीते समयप्रतिपालनम्}
{कृतोपकारं मां नूनमवमत्याल्पया धिया}


\twolineshloka
{यदितावदनुद्युक्तः शेते कामसुखात्मकः}
{नेतव्यो वालिमार्गेण सर्वभूतगतिं त्वया}


\twolineshloka
{अथापि घटतेऽस्माकमर्ते वानरपुङ्गवः}
{तमादायैव काकुत्स्थ त्वरावान्भव माचिरम्}


\twolineshloka
{इत्युक्तो लक्ष्मणो भ्रात्रा गुरुवाक्यहिते रतः}
{प्रतस्थे रुचिरं गृह्य समार्गणगुणं धनुः}


\twolineshloka
{किष्किन्धाद्वारमासाद्यप्रविवेशानिवारितः}
{सक्रोध इतितं मत्वाराजा प्रत्युद्ययौ हरिः}


\twolineshloka
{तं सदारोविनीतात्मा सुग्रीवः प्लवगाधिपः}
{पूजया प्रतिजग्राह प्रीयमाणस्तदर्हया}


\twolineshloka
{तमब्रवीद्रामवचः सौमित्रिरकुतोभयः}
{स तत्सर्वमशेषेण श्रुत्वा प्रह्वः कृताञ्जलिः}


\twolineshloka
{सभृत्यदारो राजेन्द्रसुग्रीवो वानराधिपः}
{इदमाह वचः प्रीतो लक्ष्मणं नरकुञ्जरम्}


\twolineshloka
{नास्मि लक्ष्मण दुर्मेधा नाकृतज्ञो न निर्घृणः}
{श्रूयतां यः प्रयत्नो मे सीतापर्येषणे कृतः}


\twolineshloka
{दिशः प्रस्थापिताः सर्वेविनीता हरयो मया}
{सर्वेषां च कृतः कालो मासेऽभ्यागमने पुनः}


\twolineshloka
{यैरियं सवना साद्रिः सपुरा सागराम्बरा}
{विचेतव्या मही वीर सग्रामनगराकरा}


\twolineshloka
{स मासः पञ्चरात्रेण पूर्णो भवितुमर्हति}
{ततः श्रोष्यसि रामेण सहितः सुमहत्प्रियम्}


\twolineshloka
{इत्युक्तो लक्ष्मणस्तेन वानरेन्द्रेण धीमता}
{त्यक्त्वा रोषमदीनात्मा सुग्रीवं प्रत्यपूजयत्}


\twolineshloka
{सरामं सहसुग्रीवो माल्यवत्पुष्ठमास्थितम्}
{अभिगम्योदयं तस्य कार्यस्य प्रत्यवेदयत्}


\twolineshloka
{इत्येवं वानरेन्द्रास्ते समाजग्मुः सहस्रशः}
{दिशस्तिस्रो विचित्याथ न तु ये दक्षिणां गताः}


\twolineshloka
{आचख्युस्तत्र रामाय महीं सागरमेखलाम्}
{विचितां न तु वैदेह्या दर्शनं रावणस्य वा}


\twolineshloka
{गतास्तु दक्षिणामाशां ये वै वानरपुङ्गवाः}
{आशावांस्तेषु काकुत्स्थः प्राणानार्तोऽभ्यधारयत्}


\twolineshloka
{द्विमासोपरमे काले व्यतीते प्लवगास्ततः}
{सुग्रीवमभिगम्येदं त्वरिता वाक्यमब्रुवन्}


\twolineshloka
{रक्षितंवालिना यत्तत्स्फीतं मधुवनं महत्}
{त्वया च प्लवगश्रेष्ठ तद्भुङ्क्ते पवनात्मजः}


\twolineshloka
{वालिपुत्रोऽङ्गदश्चैव ये चान्ये प्लवगर्षभाः}
{विचेतुं दक्षिणामाशां राजन्प्रस्थापितास्त्वया}


\twolineshloka
{तेषामपनयं श्रुत्वा मेने सकृतकृत्यताम्}
{कृतार्थानां हि भृत्यानामेतद्भवति चेष्टितम्}


\twolineshloka
{स तद्रामाय मेधावी शशंस प्लवगर्षभः}
{रामश्चाप्यनुमानेन मेने दृष्टां तु मैथिलीम्}


\twolineshloka
{हनुमत्प्रमुखाश्चापि विश्रान्तास्ते प्लवङ्गमाः}
{अभिजग्मुर्हरीन्द्रं तं रामलक्ष्मणसन्निधौ}


\twolineshloka
{गतिं च मुखवर्णं च दृष्ट्वारामो हनूमतः}
{अगमत्प्रत्ययं भूयो दृष्टा सीतेति भारत}


\twolineshloka
{हनूमत्प्रमुखास्ते तु वानराः पूर्णमानसाः}
{प्रणेमुर्विधिवद्रामं सुग्रीवं लक्ष्मणं तथा}


\twolineshloka
{तानुवाचानतान्रामः प्रगृह्य सशरं धनुः}
{अपि मां जीवयिष्यध्वमपि वः कृतकृत्यता}


\twolineshloka
{अपि राज्यमयोध्यायां कारयिष्याम्यहं पुनः}
{निहत्यसमरे शत्रूनाहृत्यजनकात्मजाम्}


\twolineshloka
{अमोक्षयित्वावैदेहीमहत्वा च रणे रिपून्}
{हृतदारोऽवधूतश्चनाहं जीवितुमुत्सहे}


\twolineshloka
{इत्युक्तवचनं रामं प्रत्युवाचानिलात्मजः}
{प्रियमाख्यामि ते राम दृष्टा सा जानकी मया}


\twolineshloka
{विचित्य दक्षिणामाशां सपर्वतवनाकराम्}
{श्रान्ताः काले व्यतीते स्म दृष्टवन्तो महागुहाम्}


\twolineshloka
{प्रविशामो वयं तां तु बहुयोजनमायताम्}
{अन्धकारां सुविपिनां गहनां कीटसेविताम्}


\twolineshloka
{गत्वा सुमहदध्वानमादित्यस्य प्रभां ततः}
{दृष्टवन्तः स्म तत्रैवभवनं दिव्यमन्तरा}


\twolineshloka
{मयस्य किल दैत्यस्य तदा सद्वेश्म राघव}
{तत्र प्रभावती नाम तपोऽतप्यत तापसी}


\twolineshloka
{तया दत्तानि भोज्यानि पानानि विविधानि च}
{भुक्त्वा लब्धबलाः सन्तस्तयोक्तेन पथा ततः}


\twolineshloka
{निर्याय तस्मादुद्देशात्पश्यामो लवणाम्भसः}
{समीपे सह्यमलयौ दर्दुरं च महागिरिम्}


\twolineshloka
{ततो मलयमारुह्य पश्यन्तो वरुणालयम्}
{विषण्णा व्यथिताः खिन्ना निराशा जीविते भृशम्}


\twolineshloka
{अनेकशतविस्तीर्णं योजनानां महोदधिम्}
{तिमिनक्रझषावासं चिन्तयन्तः सुदुःखिताः}


\twolineshloka
{तत्रानशनसङ्कल्पं कृत्वाऽऽसीना वयं तदा}
{ततः कथान्ते गृध्रस्य जटायोरभवत्कथा}


\twolineshloka
{ततः पर्वतशृङ्गाभं घोररूपं भयावहम्}
{पक्षिणं दृष्टवन्तः स्म वैनतियेमिवापरम्}


\twolineshloka
{सोऽस्मानतर्कयद्भोक्तुमथाभ्येत्य वचोऽब्रवीत्}
{भोः क एष मम भ्रातुर्जटायोः कुरुते कथाम्}


\twolineshloka
{सपातिर्नाम तस्याहं ज्येष्ठो भ्राता खगाधिपः}
{अन्योन्यस्पर्धया रूढावावामदित्यसत्पदम्}


\twolineshloka
{ततो दग्धाविमौ पक्षौ न दग्धौ तु जटायुषः}
{तस्मान्मे चिरदृष्टः स भ्राता गृध्रपतः प्रियः}


\twolineshloka
{निर्दग्धपक्षः पतितो ह्यहमस्मिन्महागिरौ}
{द्रष्टुं वीरं न शक्नोमि भ्रातरं वै जटायुषम्}


\twolineshloka
{तस्यैवं वदतोऽस्माभिर्हतो भ्राता निवेदितः}
{व्यसनं भवतश्चेदं सङ्क्षेपाद्वै निवेदितम्}


\twolineshloka
{स सम्पातिस्तदा राजञ्श्रुत्वासुमहदप्रियम्}
{विषण्णचेताः पप्रच्छ पुनरस्मानरिन्दम}


\twolineshloka
{कः सरामः कथं सीता जटायुश्च कथं हतः}
{इच्छामि सर्वमेवैतच्छ्रोतुं प्लवगसत्तमाः}


\twolineshloka
{तस्याहं सर्वमेवैतद्भवतो व्यसनागमम्}
{प्रायोपवेशने चैवहेतुं विस्तरशोऽब्रुवम्}


\twolineshloka
{सोऽस्मानाश्वासयामास वाक्येनानेन पक्षिराट्}
{रावणो विदितो मह्यं लङ्का चास्य महापुरी}


\twolineshloka
{दृष्टापारे समुद्रस्य त्रिकूटगिरिकन्दरे}
{भवित्री तत्र वैदेही न मेऽस्त्यत्र विचारणा}


\twolineshloka
{इतितस्य वचः श्रुत्वा वयमुत्थाय सत्वराः}
{सागरक्रमणे मन्त्रं मन्त्रयामः परन्तप}


\threelineshloka
{नाध्यवास्यद्यदा कश्चित्सागरस्य विलङ्घनम्}
{ततः पितरमाविश्य पुप्लुवेऽहमहार्णवम्}
{शतयोजनविस्तीर्णं निहत्य जलराक्षसीम्}


\twolineshloka
{उपवासतपःशीला भर्तृदर्शनलालसा}
{जटिला मलदिग्धाङ्गीकृश दीना तपस्विन}


\twolineshloka
{निमित्तैस्तामहं सीतामुपलभ्य पृथग्विधैः}
{उपसृत्याब्रवं चार्यामभिगम्य रहोगताम्}


\twolineshloka
{सीते रामस्य दूतोऽहंवानरोमारुतात्मजः}
{त्वद्दर्शनमभिप्रप्सुरिह प्राप्तो विहायसा}


\twolineshloka
{राजपुत्रौ कुशलिनौ भ्रातरौ रामलक्ष्मणौ}
{सर्वशाखामृगेन्द्रेण सुग्रीवेणाभिपालितौ}


\twolineshloka
{कुशलन्त्वाब्रवीद्रामःसीते सौमित्रिणा सह}
{सखिभावाच्च सुग्रीवः कुशलं त्वाऽनुपृच्छति}


\twolineshloka
{क्षिप्रमेष्यति ते भर्ता सर्वशाखामृगैः सह}
{प्रत्ययं कुरु मे देवि वानरोऽस्मि न राक्षसः}


\twolineshloka
{मुहूर्तमिवच ध्यात्वा सीता मां प्रत्युवाच ह}
{अवैमि त्वांहनूमन्तमविन्ध्यवचनादहम्}


\twolineshloka
{अविन्ध्यो हि महाबाहो राक्षसो वृद्धसमतः}
{कथितस्तेन सुग्रीवस्त्वद्विधैः सचिवैर्वृतः}


\twolineshloka
{गम्यतामिति चोक्त्वा मां सीता पादादिमं मणिम्}
{घारिता येन वैदेही कालमेतमनिन्दिता}


\twolineshloka
{प्रत्ययार्थं कथां चेमां कथयामास जानकी}
{क्षिप्तामिषीकां काकाय चित्रकूटे महागिरौ}


\twolineshloka
{भवता पुरुषव्याघ्र प्रत्यभिज्ञानकारणात्}
{एकाक्षिविकलः काकः सुदुष्टात्मा कृतश्चवै}


\twolineshloka
{ग्राहयित्वाऽहमात्मानं ततो दग्ध्वाच तां पुरीम्}
{सप्राप्त इतितं रामः प्रियवादिनमार्चयत्}


॥इति श्रीमन्महाभारते अरण्यपर्वणि रामोपाख्यान-पर्वणि त्रिशततमोऽध्यायः॥२८३॥

\storymeta

\dnsub{अध्यायः २८४}\resetShloka

\uvacha{मार्कण्डेय उवाच}


\twolineshloka
{ततस्तत्रैवरामस्य समासीनस्य तैः सह}
{समाजग्मुः कपिश्रेष्ठाः सुग्रीववचनात्तदा}


\twolineshloka
{वृतः कोटिसहस्रेण वानराणां तरस्विनाम्}
{श्वशुरो वालिनः श्रीमान्सुषेणो राममभ्ययात्}


\twolineshloka
{कोटीशतवृतोवाऽपिगजो गवय एव च}
{वानरेन्द्रौ महावीर्यौ पृथक्पृथगदृश्यताम्}


\twolineshloka
{षष्टिकोटिसहस्राणि प्रकर्षन्प्रत्यदृश्यत}
{गोलाङ्गूलो महाराज गवाक्षो भीमदर्शनः}


\twolineshloka
{गन्धमादनवासी तु प्रथितो गन्धमादनः}
{कोटीशतसहस्राणि हरीणां समकर्षत}


\twolineshloka
{पनसो नाम मेधावी वानरःसुमहाबलः}
{कोटीर्दश द्वादश च त्रिंशत्पञ्च प्रकर्षति}


\twolineshloka
{श्रीमान्दधिमुखो नाम हरिवृद्धोऽतिवीर्यवान्}
{प्रचकर्ष महासैन्यं हरीणां भीमतेजसाम्}


\twolineshloka
{कृष्णानां मुखपुण्ड्राणामृक्षाणां भीमकर्मणाम्}
{कोटीर्दश द्वादश च त्रिंशत्पञ्च प्रकर्षति}


\twolineshloka
{एते चान्ये च बहवो हरियूथपयूथपाः}
{असङ्ख्येया महाराज समीयू रामकारणात्}


\twolineshloka
{गिरिकूटनिभाङ्गानां सिंहानामिव गर्जताम्}
{श्रूयते तुमुलः शब्दस्तत्रतत्रप्रधावताम्}


\twolineshloka
{गिरिकूटनिभाः क्नचित्केचिन्महिषसन्निभाः}
{शरदभ्रप्रतीकाशाः केचिद्धिङ्गुलकाननाः}


\twolineshloka
{उत्पतन्तः पतन्तश्च प्लवमानाश्च वानराः}
{उद्धुन्वन्तोऽपरे रेणून्समाजग्मुः समन्ततः}


\twolineshloka
{सवानरमहासैन्यः पूर्णसागरसन्निभः}
{निवेशमकरोत्तत्रसुग्रीवानुमते तदा}


\twolineshloka
{ततस्तेषु हरीन्द्रेषु समावृत्तेषु सर्वशः}
{तिथौ प्रशस्ते नक्षत्रे मुहूर्ते चाभिपूजिते}


\twolineshloka
{तेन व्यूढेन सैन्येन लोकानुद्वर्तयन्निव}
{प्रययौ राघवः श्रीमान्सुग्रीवसहितस्तदा}


\twolineshloka
{मुखमासीत्तु सैन्यस्य हनूमान्मारुतात्मजः}
{जघनं पालयामास सौमित्रिरकुतोभयः}


\twolineshloka
{बद्धगोधाङ्गुलित्रणौ राघवौ तत्रजग्मतुः}
{वृतौ हरिमहामात्रैश्चन्द्रसूर्यौ ग्रहैरिव}


\twolineshloka
{प्रबभौ हरिसैन्यं तत्सालतालशिलायुधम्}
{सुमहच्छालिभवनं यथा सूर्योदयं प्रति}


\twolineshloka
{नलनीलाङ्गदक्राथमैन्दद्विविदपालिता}
{ययौ सुमहती सेना राघवस्यार्थसिद्धये}


\twolineshloka
{विविधेषु प्रशस्तेषु बहुमूलफलेषु च}
{प्रभूतमधुमांसेषु वारिमत्सु विवेषु च}


\twolineshloka
{निवसन्ती निराबाधा तथैवगिरिसानुषु}
{उपायाद्धिरिसेना सा क्षारोदमथ मागरम्}


\twolineshloka
{द्वितीयसागरनिमं तद्बलबहुलध्वजम्}
{वेलावनं समासाद्य निवासमकरोत्तदा}


\twolineshloka
{ततो दाशरथिः श्रीमान्सुग्रीवं प्रत्यभाषत}
{मध्ये वानरमुख्यानां प्राप्तकालमिदं वचः}


\twolineshloka
{उपायः कोनु भवतां मतः सागरलङ्घने}
{इयं हि महती सेना सागरश्चातिदुस्तरः}


\twolineshloka
{तत्रान्ये व्याहरन्ति स्म वानराः पटुमानिनः}
{समर्था लङ्घने सिन्दोर्न तत्कृत्स्नस्य वानराः}


\twolineshloka
{केचिन्नौभिर्व्यवस्यन्ति केचिच्च विविधैः प्लवैः}
{नेति रामस्तु तान्सर्वान्सान्त्वयन्प्रत्यभाषत}


\twolineshloka
{शतयोजनविस्तारं न शक्ताः सर्ववानराः}
{क्रान्तुं तोयनिधिं वीरानैषा वो नैष्ठिकी मतिः}


\twolineshloka
{नावो न सन्ति सेनाया बह्व्यस्तारयितुं तथा}
{वणिजामुपघातं च कथमस्मद्विधश्चरेत्}


\twolineshloka
{विस्तीर्णं चैव नः सैन्यं हन्याच्छिद्रेण वै परः}
{प्लवोडुपप्रतारश्च नैवात्र मम रोचते}


\twolineshloka
{अहं त्विमं जलनिधिं समारप्स्याम्युपायतः}
{प्रतिशेष्याम्युपवसन्दर्शयिष्यति मां ततः}


\twolineshloka
{न चेद्दर्शयिता मार्गं धक्ष्याम्यनमहं ततः}
{महास्त्रैरप्रतिहतैरत्यग्निपवनोज्ज्वलैः}


\twolineshloka
{इत्युक्त्वा सहसौमित्रिरुपस्पृश्याथ राघवः}
{प्रतिशिस्ये जलनिधं विधिवत्कुशसंस्तरे}


\twolineshloka
{सागरस्तु ततः स्वप्ने दर्शयामास राघवम्}
{देवो नदनदीमर्ता श्रीमान्यादोगणैर्वृतः}


\twolineshloka
{कौसल्यामातरित्येवमाभाष्य मधुरं वचः}
{इदमित्याह रत्नानामाकरैः शतशो वृतः}


\threelineshloka
{ब्रूहि किं ते करोम्यत्र साहाय्यं पुरुषर्षभ}
{ऐक्ष्वाको ह्यस्मि ते ज्ञाती राम सत्यपराक्रम}
{एवमुक्तः समुद्रेण रामो वाक्यमथाब्रवीत्}


\threelineshloka
{मार्गमिच्छामि सैन्यस्य दत्तं नदनदीपते}
{येन गत्वा दशग्रीवं हन्यामि कुलपांसनम्}
{राक्षसं सानुबन्धं तं मम भार्यापहारिणम्}


\twolineshloka
{यद्येवं याचतो मार्गं न प्रदास्यति मे भवान्}
{शरैस्त्वां शोषयिष्यामि दिव्यास्त्रयतिमन्त्रितैः}


\twolineshloka
{इत्येवं ब्रुवतः श्रुत्वा रामस्य वरुणालयः}
{उवाचव्यथितोवाक्यमितिबद्धाञ्जलिःस्थितः}


\twolineshloka
{नेच्छामि प्रतिघातं ते नास्मि विघ्नकरस्तव}
{शृणु चेदं वचोराम श्रुत्वा कर्तव्यमाचर}


\twolineshloka
{यदि दास्यामि ते मार्गं सैन्यस्य व्रजतोऽऽज्ञया}
{अन्येऽप्याज्ञापयिष्यन्ति मामेवं धनुषोबलात्}


\twolineshloka
{अस्तित्वत्रनलो नाम वानरः शिल्पिसमतः}
{त्वष्टुः काकुत्स्थ तनयो बलवान्विश्वकर्मणः}


\twolineshloka
{स यत्काष्ठं तृणं वाऽपिशिलां वा क्षेप्स्यते मयि}
{सर्वं तद्धारयिष्यामि स ते सेतुर्भविष्यति}


\twolineshloka
{इत्युक्त्वाऽन्तर्हिते तस्मिन्रामो नलमुवाच ह}
{कुरु सेतुं समुद्रे त्वंशक्तो ह्यसि मतो मम}


\twolineshloka
{तेनोपायेन काकुत्स्थः सतुबन्धमकारयत्}
{दशयोजनविस्तारमायतं शतयोजनम्}


\twolineshloka
{नलसेतुरिति ख्यातो योऽद्यापि प्रथितो भुवि}
{रामस्याज्ञां पुरस्कृत्य धार्यते गिरिसन्निभः}


\twolineshloka
{तत्रस्थं स तु धर्मात्मा समागच्चद्विभीषणः}
{भ्राता वै राक्षसेन्द्रस्य चतुर्भिः सचिवैः सह}


\twolineshloka
{प्रतिजग्राह रामस्तं स्वागतेन महामनाः}
{सुग्रीवस्य तु शङ्काऽभूत्प्रणिधिः स्यादिति स्मह}


\twolineshloka
{राघवः सत्यचेष्टाभिः सम्यक्व चरितेङ्गितैः}
{यदा तत्त्वेन तुष्टोऽभूत्तत एनमपूजयत्}


\twolineshloka
{सर्वराक्षसराज्येचाप्यभ्यपिञ्चद्विभीषणम्}
{चक्रे च मन्त्रसचिवं सहृदं लक्ष्मणस्य च}


\twolineshloka
{विभीषणमते चैव सोऽत्यक्रामन्महार्णवम्}
{ससैन्यः सेतुना तेन मार्गेणैव नराधिपः}


\twolineshloka
{ततो गत्वासमासाद्य लङ्कोद्यानान्यनेकशः}
{भेदयामास कपिभिर्महान्ति च बहूनि च}


\twolineshloka
{तत्रास्तां रावणामात्यौ राक्षसौ शुकसारणौ}
{चरौ वानररूपेण तौ जग्राह विभीषणः}


\twolineshloka
{प्रतिपन्नौ यदा रूपं राक्षसं तौ निशाचरौ}
{दर्शयित्वा ततः सैन्यं रामः पश्चादवासृजत्}


\twolineshloka
{निवेश्योपवने सैन्यं स शूरः प्राज्यवानरम्}
{प्रेषयामास दुत्येन रावणस्य ततोऽङ्गदम्}


॥इति श्रीमन्महाभारते अरण्यपर्वणि रामोपाख्यान-पर्वणि त्रिशततमोऽध्यायः॥२८४॥

\storymeta

\dnsub{अध्यायः २८५}\resetShloka

\uvacha{मार्कण्डेय उवाच}


\twolineshloka
{प्रभूतान्नोदकेतस्मिन्बहुमूलफले वने}
{सेनां निवेश्य काकुत्स्थो विधिवत्पर्यरक्षत}


\twolineshloka
{रावणः संविधं चक्रे लङ्कायां शास्त्रनिर्मिताम्}
{प्रकृत्यैवदुराधर्षा दृढप्राकारतोरणा}


\twolineshloka
{अगाधतोयाः परिखा मीननक्रसमाकुलाः}
{बभूवुः सप्त दुर्धर्षाः स्वादिरैः शङ्कुभिश्चिताः}


\twolineshloka
{कर्णाटयन्त्रा दुर्धर्षा बभूवुः सहुडोपलाः}
{साशीविषघटायोधाः ससर्जरसपांसवः}


\twolineshloka
{मुसलालातनाराचतोमरासिपरश्वथैः}
{अन्विताश्चशतघ्नीभिः समधूच्छिष्टमुद्गराः}


\twolineshloka
{पुरद्वारेषु सर्वेषु गुल्माः स्थावरजङ्गमाः}
{बभूवुः पत्तिबहुलाः प्रभूतगजवाजिनः}


\twolineshloka
{अङ्गदस्त्वथ लङ्कायां द्वारदेशमुपागतः}
{विदितो राक्षसेन्द्रस्य प्रविवेशगतव्यथः}


\twolineshloka
{मध्ये राक्षसकोटीनां बह्वीनां सुमहाबलः}
{शुशुभे मेघमालाभिरादित्य इव संवृतः}


\twolineshloka
{ससमासाद्य पौलस्त्यममात्यैरभिसंवृतम्}
{रामसन्देशमामन्त्र्य वाग्मी वक्तुं प्रचक्रमे}


\twolineshloka
{आह त्वां राघवो राजन्कोसलेन्द्रो महायशाः}
{प्राप्तकालमिदं वाक्यं तदादत्स्व सुदुर्मते}


\twolineshloka
{अकृतात्मानमासाद्य राजानमनये रतम्}
{विनश्यन्त्यनयाविष्टा देशाश्च नगराणि च}


\twolineshloka
{त्वयैकेनापराद्धं मे सीतामाहरता बलात्}
{वधायानपराद्धानामन्येषां तद्भविष्यति}


\twolineshloka
{ये त्वया बलदर्पाभ्यामाविष्टेन वनेचराः}
{ऋषयोहिंसिताः पूर्वन्देवाश्चाप्यवमानिताः}


\twolineshloka
{राजर्षयश्च निहता रुदत्यश्चाहृताः स्त्रियः}
{तदिदं समनुप्राप्तं फलन्तस्यानयस्य ते}


\twolineshloka
{हन्तास्मि त्वां सहामात्यैर्युध्यस्व पुरुषो भव}
{पश्य मे धनुषो वीर्यं मानुषस्य निशाचर}


\twolineshloka
{मुच्यतां जानकी सीता न मे मोक्ष्यसि कर्हिचित्}
{अराक्षसमिमं लोकङ्कर्ताऽस्मि निशितैः शरैः}


\twolineshloka
{इति तस्य ब्रुवाणस्य दूतस्य परुषं वचः}
{श्रुत्वा न ममृषे राजा रावणः क्रोधमूर्च्छितः}


\twolineshloka
{इङ्गितज्ञास्ततो भर्तुश्चत्वारो रजनीचराः}
{चतुर्ष्वङ्गेषु जगृहुः शार्दूलमिव पक्षिणः}


\twolineshloka
{तांस्तथाङ्गेषु संसक्तानङ्गदो रजनीचरान्}
{आदायैव खमुत्पत्य प्रासादतलमाविशत्}


\twolineshloka
{वेगेनोत्पततस्तस्य पेतुस्ते रजनीचराः}
{भुवि सभिन्नहृदयाः प्रहारवरपीडिताः}


\twolineshloka
{संसक्तोहर्म्यशिखरात्तस्मात्पुनरवापतत्}
{लङ्घयित्वा पुरं लङ्कां सुवेलस्य समीपतः}


\twolineshloka
{कोसलेन्द्रमथागम्य सर्वमावेद्य वानरः}
{विशश्राम स तेजस्वी राघवेणाभिनन्दितः}


\twolineshloka
{ततः सर्वाभिसारेण हरीणां वातरंहसाम्}
{भेदयामास लङ्कायाः ग्राकारं रघुनन्दनः}


\twolineshloka
{विभीषणर्क्षाधिपती पुरस्कृत्याथ लक्ष्मणः}
{दक्षिणं नगरद्वारमवामृद्गाद्दुरासदम्}


\twolineshloka
{करभारुणगात्राणां हरीणां युद्धशालिनाम्}
{कोटीशतसहस्रेण लङ्कामभ्यपतत्तदा}


\twolineshloka
{प्रलम्बबाहूरुकरजङ्घान्तरविलम्बिनाम्}
{ऋक्षाणां धूम्रवर्णानां तिस्रः कोठ्यो व्यवस्थिताः}


\twolineshloka
{उत्पतद्भिः पतद्भिश्च निपतद्भिश्च वानरैः}
{नादृश्यत तदा सूर्यो रजसा नाशितप्रभः}


\twolineshloka
{शालिप्रसूनसदृशैः शिरीपकुसुमप्रभैः}
{तरुणादित्यसदृशैः शणगौरैश्च वैनरैः}


\twolineshloka
{प्राकारं ददृशुस्ते तु समन्तात्कपिलीकृतम्}
{राक्षसा विस्मिता राजन्सस्त्रीवृद्धाः समन्ततः}


\twolineshloka
{बिभिदुस्ते मणिस्तम्भान्कर्णाट्टशिखराणि च}
{भग्नोन्मथितशृङ्गाणि यन्त्राणि च विचिक्षिपुः}


\twolineshloka
{परिगृह्य शतघ्नीश्च सचक्राः सगुडोपलाः}
{चिक्षिपुर्भुजवेगेन लङ्कामध्येमहास्वनाः}


\twolineshloka
{प्राकारस्थाश्चये केचिन्निशाचरगणास्तथा}
{प्रदुद्रुवुस्ते शतशः कपिभिः समभिद्रुताः}


\twolineshloka
{ततस्तु राजवचनाद्राक्षसाः कामरूपिणः}
{निर्ययुर्विकृताकाराः सहस्रशतसङ्घशः}


\twolineshloka
{शखवर्षाणि वर्षन्तो द्रावयित्वा वनौकसः}
{प्राकारं शोभयन्तस्ते परं विस्मयमास्थिताः}


\twolineshloka
{स मापराशिसदृशैर्बभूव क्षणादाचरैः}
{कृतो निर्वानरो भूयः प्राकारो भीमदर्शनैः}


\twolineshloka
{पेतुः शलविभिन्नाङ्गा बहवो वानरर्पभाः}
{स्तम्भतोरणभग्नाश्चपेतुस्तत्रनिशाचराः}


\twolineshloka
{केशाकेश्यभवद्युद्धं रक्षसां वानरैः सह}
{नखैर्दन्तैश्च वीराणां खादतां वै परस्परम्}


\twolineshloka
{निष्टनन्तो ह्युभयतस्तत्र वानरराक्षसाः}
{हतानिपतिता भूमौ न मुञ्चन्ति परस्परम्}


\twolineshloka
{रामस्तु शरजालानिववर्ष जलदो यथा}
{तानिलङ्कां समासाद्य जघ्रुस्तान्रजनीचरान्}


\twolineshloka
{सौमित्रिरपि नाराचैर्दृढधन्वा जितक्लमः}
{आदिश्यादिश्य दुर्गस्थान्पातयामास राक्षसान्}


\twolineshloka
{ततः प्रत्यवहारोऽभूत्सैन्यानां राधवाज्ञया}
{कृते विमर्दे लङ्कायां लब्धलक्ष्योजयोत्तरः}


॥इति श्रीमन्महाभारते अरण्यपर्वणि रामोपाख्यान-पर्वणि त्रिशततमोऽध्यायः॥२८५॥

\storymeta

\dnsub{अध्यायः २८६}\resetShloka

\uvacha{मार्कण्डेय उवाच}


\twolineshloka
{ततो निविशमानांस्तान्सैनिकान्रावणानुगाः}
{अभिजग्मुर्गणाऽनके पिशाचक्षुद्ररक्षसाम्}


\twolineshloka
{पर्वणः पतनो जम्भः खरः क्रोधवशो हरिः}
{प्ररुजश्चारुजश्चैव प्रघसश्चैवमादयः}


\twolineshloka
{ततोऽभिपततां तेषामदृश्यानां दुरात्मनाम्}
{अन्तर्धानवधं तज्ज्ञश्चकार स विभीषणः}


\twolineshloka
{ते दृश्यमाना हरिभिर्बलिभिर्दूरपातिभिः}
{निहताः सर्वशो राजन्महीं जग्मुर्गतासवः}


\twolineshloka
{अमृष्यमाणः सबलो रावणो निर्ययावथ}
{राक्षसानां बलैर्घोरैः पिशाचानाञ्च संवृतः}


\twolineshloka
{युद्धशास्त्रविधानज्ञ उशना इव चापरः}
{व्यूह्यचौशनसं व्यूहं हरीनभ्यवहारयत्}


\twolineshloka
{राघवस्तु विनिर्यान्तं व्यूढानीकं दशाननम्}
{बार्हस्पत्यं विधं कृत्वा प्रतिव्यूह्य ह्यदृश्यत}


\twolineshloka
{समेत्य युयुधे तत्र ततो रामेण रावणः}
{युयुधे लक्ष्मणश्चापि तथैवेन्द्रजिता सह}


\twolineshloka
{विरूपाक्षेण सुग्रीवस्तारेण च निस्वर्वटः}
{पौण्ड्रेण च नलस्तत्र पदुशः पनसेन च}


\twolineshloka
{विषह्यं यं हि यो मेने स स तेन समेयिवान्}
{युयुधे युद्धवेलायां स्वबाहुबलमाश्रितः}


\twolineshloka
{स सप्रहारो ववृधे भीरूणां भयवर्धनः}
{रोमसंहर्षणो घोरः पुरा देवासुरे यथा}


\twolineshloka
{रावणो राममानर्च्छच्छक्तिशूलासिवृष्टिभिः}
{निशितैरायसैस्तीक्ष्णै रावणं चापि राघवः}


\twolineshloka
{तथैवेन्द्रजितं यत्तं लक्ष्मणो मर्मभेदिभिः}
{इन्द्रजिच्चापि सौमित्रिं बिभेद बहुभिः शरैः}


\twolineshloka
{विभीषणः प्रहस्तं च प्रहस्तश्च विभीषणम्}
{खगपत्रैः शरैस्तीक्ष्णैरभ्यवर्षद्गतव्यथः}


\twolineshloka
{तेषां बलवतामासीन्महास्त्राणां समागमः}
{विव्यथुः सकला येन त्रयो लोकाश्चराचराः}


॥इति श्रीमन्महाभारते अरण्यपर्वणि रामोपाख्यान-पर्वणि त्रिशततमोऽध्यायः॥२८६॥

\storymeta

\dnsub{अध्यायः २८७}\resetShloka

\uvacha{मार्कण्डेय उवाच}


\twolineshloka
{ततः प्रहस्तः सहसा समभ्येत्य विभीषणम्}
{गदया ताडयामास विनद्य रणकर्कशम्}


\twolineshloka
{स तयाऽभिहतो धीमान्गदया भीमवेगया}
{नाकम्पत महाबाहुर्हिमवानिव सुस्थिरः}


\twolineshloka
{ततः प्रगृह्यविपुलां शतघण्टां विभीषणः}
{अनुमन्त्र्य महाशक्तिं चिक्षेपास्य शिरः प्रति}


\twolineshloka
{पतन्त्या स तया वेगाद्राक्षसोऽशनिवेगया}
{हृतोत्तामङ्गो ददृशे वातरुग्ण इव द्रुमः}


\twolineshloka
{तं दृष्ट्वा निहतं सङ्ख्ये प्रहस्तं क्षणदाचरम्}
{अभिदुद्राव धूम्राक्षो वेगेन महता कपीन्}


\twolineshloka
{तस्य मेघोपमं सैन्यमापतद्भीमदर्शनम्}
{दृष्ट्वैव सहसा दीर्णा रणे वानरपुङ्गवाः}


\twolineshloka
{ततस्तान्सहसा दीर्णान्दृष्ट्वा वानरपुङ्गवान्}
{निर्ययौ कपिशार्दूलो हनूमान्मारुतात्मजः}


\twolineshloka
{तं दृष्ट्वाऽवस्थितं सङ्ख्ये हरयः पवनात्मजम्}
{महत्या त्वरया राजत्सन्न्यवर्तन्त सर्वशः}


\twolineshloka
{ततः शब्दो महानासीत्तुमुलो रोमहर्षणः}
{रामरावणसैन्यानामन्योन्यमभिधावताम्}


\twolineshloka
{तस्मिन्प्रवृत्ते सङ्ग्रामे घोरे रुधिरकर्दमे}
{क्षूम्राक्षः कपिसैन्यं तद्द्रावयामास पत्रिभिः}


\twolineshloka
{तं स रक्षोमहामात्रमापतन्तं सपत्नजित्}
{प्रतिजग्राह हनुमांस्तरसा पवनात्मजः}


\twolineshloka
{तयोर्युद्धमभूद्घोरं हरिराक्षसवीरयोः}
{जिगीषतोर्युधाऽन्योन्यमिन्द्रप्रह्लादयोरिव}


\twolineshloka
{गदाभिः परिघैश्चैव राक्षसो जघ्निवान्कपिम्}
{कपिश्च जघ्निवान्रक्षः सस्कन्धविटपैर्द्रुमैः}


\twolineshloka
{ततस्तमतिकोपेन साश्वं सरथसारथिम्}
{धूम्राक्षमवधीत्क्रुद्धो हनूमान्मारुतात्मजः}


\twolineshloka
{ततस्तं निहतं दृष्ट्वा धूम्राक्षं राक्षसोत्तमम्}
{हरयो जातविश्रम्भा जघ्नुरन्ये च सैनिकान्}


\twolineshloka
{ते वध्यमाना हरिभिर्बलिभिर्जितकाशिभिः}
{राक्षसा भग्नसङ्कल्पा लङ्कामभ्यपतन्भयात्}


\twolineshloka
{तेऽभिपत्य पुरं भग्ना हतशेषा निशाचराः}
{सर्वं राज्ञे यथावृत्तं रावणाय न्यवेदयन्}


\twolineshloka
{श्रुत्वा तु रावणस्तेभ्यः प्रहस्तं निहतं युधि}
{धूम्राक्षं च महेष्वासं ससैन्यं सहराक्षसैः}


\twolineshloka
{सुदीर्घमिव निःश्वस्य समुत्पत्य वरासनात्}
{उवाच कुम्भकर्णस्य कर्मकालोऽयमागतः}


\twolineshloka
{इत्येवमुक्त्वा विविधैर्वादित्रैः सुमहास्वनैः}
{शयानमतिनिद्रालुं कुम्भकर्णमबोधयत्}


\threelineshloka
{प्रबोध्य महता चैनं यत्नेनाऽऽगतसाध्वसः}
{स्वस्थमासीनमव्यग्रं विनिद्रं राक्षसाधिपः}
{ततोऽब्रवीद्दशग्रीवः कुम्भकर्णं महाबलम्}


\twolineshloka
{धन्योसि यस्य ते निद्रा कुम्भकर्णेयमीदृशी}
{य इदं दारुणं कालं न जानीषे महाभयम्}


\twolineshloka
{एष तीर्त्वाऽर्णवं रामः सेतुना हरिभिः सह}
{अवमत्येह नः सर्वान्करोति कदनं महत्}


\twolineshloka
{मया त्वपहृता भार्या सीता नामास्य जानकी}
{तां नेतुं स इहायातो बद्ध्वा सेतुं महार्णवे}


\twolineshloka
{तेन चैव प्रहस्तादिर्महान्नः स्वजनो हतः}
{तस्य नान्यो निहन्ताऽस्ति त्वामृतेशत्रुकर्शन}


\twolineshloka
{सदंशितोऽभिनिर्याहि त्वमद्य बलिनांवर}
{रामादीन्समरे सर्वाञ्जहि शत्रूनरिन्दम}


\twolineshloka
{दूषणावरजौ चैव वज्रवेगप्रमाथिनौ}
{तौ त्वां बलेन महता सहितावनुयास्यतः}


\twolineshloka
{इत्युक्त्वा राक्षुसपतिः कुम्भकर्णं तरस्विनम्}
{सन्दिदेशेतिकर्तव्ये वज्रवेगप्रमाथिनौ}


\twolineshloka
{तथेत्युक्त्वा युतौ वीरौ रावणं दूषाणानुजौ}
{कुम्भकर्णं पुरस्कृत्य तूर्णं निर्ययतुः पुरात्}


॥इति श्रीमन्महाभारते अरण्यपर्वणि रामोपाख्यान-पर्वणि त्रिशततमोऽध्यायः॥२८७॥

\storymeta

\dnsub{अध्यायः २८८}\resetShloka

\uvacha{मार्कण्डेय उवाच}


\twolineshloka
{ततो निर्याय स्वपुरात्कुम्भकर्णः सहानुगः}
{अपश्यत्कपिसैन्यं तज्जितकाश्यग्रतः स्थितम्}


\twolineshloka
{स वीक्षमाणस्तत्सैन्यं रामदर्शनकाङ्क्षया}
{अपश्यच्चापि सौमित्रिं धनुष्पाणिं व्यवस्थितम्}


\twolineshloka
{तमभ्येत्याशु हरयः परिवब्रुः समन्ततः}
{शैलवृक्षायुधा नादानमुञ्चन्भीषणास्ततः}


\twolineshloka
{अभ्यघ्नंश्च महाकायैर्बहुभिर्जगतीरुहैः}
{करजैरतुदंश्चान्ये विहाय भयमुत्तमम्}


\twolineshloka
{बहुधा युध्यमानास्ते युद्धमार्गैः प्लवङ्गमाः}
{नानाप्रहरणैर्भीमै राक्षसेन्द्रमताडयन्}


\twolineshloka
{स ताड्यमानः प्रहसन्भक्षयामास वानरान्}
{बलं चण्डबलाख्यं च वज्रबाहुं च वानरम्}


\twolineshloka
{तद्दृष्ट्वा व्यथनं कर्म कुम्भकर्णस्य रक्षसः}
{उदक्रोशन्परित्रस्तास्तारप्रभृतयस्तदा}


\twolineshloka
{तानुच्चैः क्रोशतः सैन्याञ्श्रुत्वा स हरियूथपान्}
{अभिदुद्राव सुग्रीवः कुम्भकर्णमपेतभीः}


\twolineshloka
{ततो निपत्य वेगेन कुम्भकर्णं महामना}
{सालेन जघ्निवान्मूर्ध्निं बलेन कपिकुञ्जरः}


\twolineshloka
{स महात्मा महावेगः कुम्भकर्णस्य मूर्धनि}
{बिभेद सालं सुग्रीवो न चैवाव्यथयत्कपिः}


\twolineshloka
{ततो विनद्यसहसा सालस्पर्शविबोधितः}
{दोर्भ्यामादाय सुग्रीवं कुम्भकर्णोऽहरद्बलात्}


\twolineshloka
{ह्रियमाणं तु सुग्रीवं कुम्भकर्णेन रक्षसा}
{अवेक्ष्याभ्यद्रवद्वीरः सौमित्रिर्मित्रनन्दनः}


\twolineshloka
{सोऽभिपत्य महर्वेगं रुक्मपुङ्खं महाशरम्}
{प्राहिणोत्कुम्भकर्णाय लक्ष्मणः परवीरहा}


\twolineshloka
{स तस्य देहावरणं भित्त्वा देहं च सायकः}
{जगाम दारयन्भूमिं रुधिरेण समुक्षितः}


\twolineshloka
{तथा स भिन्नहृदयः समुत्सृज्य कपीश्वरम्}
{वेगेन महताऽऽविष्टस्तिष्ठतिष्ठेति चाब्रवीत्}


\twolineshloka
{कुम्भकर्णो महेष्वासः प्रगृहीतशिलायुधः}
{अभिदुद्राव सौमित्रिमुद्यम्य महतीं शिलाम्}


\twolineshloka
{तस्याभिपततस्तूर्णं क्षुराभ्यामुच्छितौ करौ}
{चिच्छेद निशिताग्राभ्यां स बभूव चतुर्भुजः}


\twolineshloka
{तानप्यस्य भुजान्सर्वान्प्रगृहीतशिलायुधान्}
{क्षुरैश्चिच्छेद लघ्वस्त्रं सौमित्रिः प्रतिदर्शयन्}


\twolineshloka
{स बभूवातिकायश्च बहुपादशिरोभुजः}
{तं ब्रह्मास्त्रेण सौमित्रिर्ददाराद्रिचयोपमम्}


\twolineshloka
{स पपात महावीर्यो दिव्यास्त्राभिहतो रणे}
{महाशनिविनिर्दग्धः पादपोऽङ्कुरवानिव}


\twolineshloka
{तं दृष्ट्वा वृत्रसङ्काशं कुम्भकर्णं तरस्विनम्}
{गतासुं पतितं भूमौ राक्षसाः प्राद्रवन्भयात्}


\twolineshloka
{तथातान्द्रवतो योधान्दृष्ट्वा तौ दूषणानुजौ}
{अवस्थाप्याथ सौमित्रिं सङ्क्रुद्धावभ्यधावताम्}


\twolineshloka
{तावाद्रवन्तौ सङ्क्रुद्धौ वज्रवेगप्रमाथिनौ}
{अभिजग्राह सौमित्रिर्विनद्योभौ पतत्रिभिः}


\twolineshloka
{ततः सुतुमुलं युद्धमभवद्रोमहर्षणम्}
{दूषणानुजयोः पार्थ लक्ष्मणस्य च धीमतः}


\twolineshloka
{महता शरवर्षेण राक्षसौ सोऽभ्यवर्पत}
{तं चापिवीरौ सङ्क्रुद्धावुभौ तौ समवर्षताम्}


\twolineshloka
{मुहूर्तमेवमभवद्वज्रवेगप्रमाथिनोः}
{सौमित्रेश्च महाबाहोः सप्रहारः सुदारुणः}


\twolineshloka
{अथाद्रिशृङ्गमादाय हनुमान्मारुतात्मजः}
{अभिद्रुत्याददे प्राणान्वज्रवेगस्य रक्षसः}


\twolineshloka
{नीलश्च महता ग्राव्णा दूपणावरजं हरिः}
{प्रमाथिनमभिद्रुत्य प्रममाथ महाबलः}


\twolineshloka
{ततः प्रावर्तत पुनः सङ्ग्रामः कटुकोदयः}
{रामरावणसैन्यानामन्योन्यमभिधावताम्}


\twolineshloka
{शतसो नैर्ऋतान्वन्या जघ्नुर्वन्यांश्च नैर्ऋताः}
{नैर्ऋतास्तत्रवध्यन्ते प्रायेण न तु वानराः}


॥इति श्रीमन्महाभारते अरण्यपर्वणि रामोपाख्यान-पर्वणि त्रिशततमोऽध्यायः॥२८८॥

\storymeta

\dnsub{अध्यायः २८९}\resetShloka

\uvacha{मार्कण्डेय उवाच}


\twolineshloka
{ततः श्रुत्वाहतं सङ्ख्ये कुम्भकर्णं सहानुगम्}
{प्रहस्तं च महेष्वासं धूम्राक्षं चातितेजसम्}


\twolineshloka
{पुत्रमिन्द्रजितं वीरं रावणः प्रत्यभाषत}
{जहिरामममित्रघ्न सुग्रीवं च सलक्ष्मणम्}


\twolineshloka
{त्वया हि मम सत्पुत्र यशो दीप्तमुपार्जितम्}
{जित्वावज्रधरं सङ्ख्ये सहस्राक्षं शचीपतिम्}


\twolineshloka
{अन्तर्हितः प्रकाशो वा दिव्यैर्दत्तवरैः शरैः}
{जहि शत्रूनमित्रघ्न मम शस्त्रभृतांवर}


\twolineshloka
{रामलक्ष्मणसुग्रीवाः शरस्पर्शं न तेऽनघ}
{समर्थाः प्रतिसोढुं च कुतस्तदनुयायिनः}


\twolineshloka
{अगता या प्रहस्तेन कुम्भकर्णेन चानघ}
{खरस्यापचितिः सङ्ख्ये तां गच्छ त्वे महाभुज}


\twolineshloka
{त्वमद्य निशितैर्बाणैर्हत्वा शत्रून्ससैनिकान्}
{प्रतिनन्दय मां पुत्र पुरा जित्वेव वासवम्}


\twolineshloka
{इत्युक्तः स तथेत्युक्त्वा रथमास्थाय दंशिथः}
{प्रययाविन्द्रजिद्राजंस्तूर्णमायोधनं प्रति}


\twolineshloka
{ततो विश्राव्य विस्पष्टं नाम राक्षसपुङ्गवः}
{आह्वयामास समरे लक्ष्मणं शुभलक्षणम्}


\twolineshloka
{तं लक्ष्मणोऽभ्यधावच्च प्रगृह्य सशरं धनुः}
{त्रासयंस्तलघोषेण सिंहः क्षुद्रमृगं यथा}


\twolineshloka
{तयोः समभवद्युद्धं सुमहज्जयगृद्धिनोः}
{दिव्यास्त्रविदुपोस्तीव्रमन्योन्यस्पर्धिनोस्तदा}


\twolineshloka
{रावणिस्तु यदा नैनं विशेषयति सायकैः}
{ततो गुरुतरं यत्नमातिष्ठद्बलिनां वरः}


\twolineshloka
{तत एवं महावेगैरर्दयामास तोमरैः}
{तानागतान्स चिच्छेद सौमित्रिर्निशितैः शरैः}


\twolineshloka
{ते निकृत्ताः शरैस्तीक्ष्णैर्न्यपतन्धरणीतले}
{साधका रावणेराजौ शतशः शकलीकृताः}


\twolineshloka
{तमङ्गदो वालिसुतः श्रीमानुद्यम्य पादपम्}
{अभिद्रुत्य महावेगस्ताडयामास मूर्धनि}


\twolineshloka
{तस्येन्द्रजिदसभ्रान्तः प्रासेनोरसि वीर्यवान्}
{प्रहर्तुमैच्छत्तं चास्य प्रासं चिच्छेद लक्ष्मणः}


\twolineshloka
{तमभ्याशगतं वीरमङ्गदं रावणात्मजः}
{गदयाऽताडयत्सव्ये पार्श्वेवानरपुङ्गवम्}


\twolineshloka
{तमचिन्त्य प्रहारं स बलवान्वालिनः सुतः}
{ससर्जेन्द्रजितः क्रोधात्सालस्कन्धं तथाङ्गदः}


\twolineshloka
{सोऽङ्गदेन रुपोत्सृष्टो वधायेन्द्रजितस्तरुः}
{जघानेन्द्रजितः पार्थ रथं साश्वं ससारथिम्}


\twolineshloka
{ततो हताश्वात्प्रस्कन्द्य रथात्स हतसारथिः}
{तत्रैवान्तर्दधे राजन्मायया रावणात्मजः}


\twolineshloka
{अन्तर्हितं विदित्वा तं बहुमायं च राक्षसम्}
{रामस्तं देशमागम्य तत्सैन्यं पर्यरक्षत}


\twolineshloka
{स राममुद्दिश्य शरैस्ततो दत्तवरैस्तदा}
{विव्याध सर्वगात्रेषु लक्ष्मणं च महाबलम्}


\twolineshloka
{तमदृश्यंशरैः शूरौ माययाऽन्तर्हितं तदा}
{योधयामासतुरुभौ रावणिं रामलक्ष्मणौ}


\twolineshloka
{स रुषा सर्वगात्रेषु तयोः पुरुषसिंहयोः}
{व्यसृजत्सायकान्भूयः शतशोऽथ सहस्रशः}


\twolineshloka
{तमदृश्यं विचिन्वन्तः सृजन्तमनिशं शरान्}
{हरयो विविशुर्व्योम प्रगृह्य महतीः शिलाः}


\twolineshloka
{तांश्च तौ चाप्यदृश्यः सशरैर्विव्याध राक्षसः}
{स भृशं ताडयामास रावणिर्मायया वृतः}


\twolineshloka
{तौ शरैरर्दितौ वीरौ भ्रातरौ रामलक्ष्मणौ}
{पेततुर्गगनाद्भूमिं सूर्याचन्द्रमसाविव}


॥इति श्रीमन्महाभारते अरण्यपर्वणि रामोपाख्यान-पर्वणि त्रिशततमोऽध्यायः॥२८९॥

\storymeta

\dnsub{अध्यायः २९०}\resetShloka

\uvacha{मार्कण्डेय उवाच}


\twolineshloka
{तावुभौ पतितौ दृष्ट्वा भ्रातरौ रामलक्ष्मणौ}
{बबन्ध रावणिर्भूयः शरैर्दत्तवरैस्तदा}


\twolineshloka
{तौ वीरौ शरजालेन बद्धाविन्द्रजिता रणे}
{रेजतुः पुरुषव्याघ्रौ शकुन्ताविव पञ्जरे}


\twolineshloka
{दृष्ट्वा निपतितौ भूमौ सर्वाङ्गेषु शराचितौ}
{सुग्रीवः कपिभिः सार्धं परिवार्योपतस्तिवान्}


\twolineshloka
{सुषेणमैन्दद्विविदैः कुमुदेनाङ्गदेन च}
{हनुमम्नीलतारैश्च नलेन च कपीश्वरः}


\twolineshloka
{ततस्तं देशमागम्य कृतकर्मा विभीषणः}
{बोधयामास तौ वीरौ प्रज्ञास्त्रेण प्रमोहितौ}


\twolineshloka
{विशल्यौ चापि सुग्रीवः क्षणेनैतौ चकार ह}
{विशल्यया महौषध्या दिव्यमन्त्रप्रयुक्तया}


\twolineshloka
{तौ लब्धसंज्ञौ नृवरौ विशल्यावुदतिष्ठताम्}
{उभौ गतक्लमौ चाऽऽस्तां क्षणेनैतौ महारथौ}


\twolineshloka
{ततो विभीषणः पार्थ राममिक्ष्वाकुनन्दनम्}
{उवाच विज्वरं दृष्ट्वा कृताञ्जलिरिदं वचः}


\twolineshloka
{अयमम्भो गृहीत्वातु राजराजस्य शासनात्}
{गुह्यकोऽभ्यागतः श्वेतात्त्वत्सकाशमरिन्दम}


\twolineshloka
{इदमम्भः कुबेरस्ते महाराज प्रयच्छति}
{अन्तर्हितानां भूतानां दर्शनार्थं परन्तप}


\twolineshloka
{अनेन मृष्टनयनो भूतान्यन्तर्हितान्युत}
{भवान्द्रक्ष्यति यस्मै च भवानेतत्प्रदास्यति}


\twolineshloka
{तथेति रामस्तद्वारि प्रतिगृह्याभिसंस्कृतम्}
{चकार नेत्रयोः शौचं लक्ष्मणश्च महामनाः}


\twolineshloka
{सुग्रीवजाम्बवन्तौ चहनुमानङ्गदस्तथा}
{मैन्दद्विविदनीलाश्च प्रायः प्लवगसत्तमाः}


\twolineshloka
{तथासमभवच्चापि यदुवाच विभीषणः}
{क्षणेनातीन्द्रियाण्येषां चक्षुंष्यासन्युधिष्ठिर}


\twolineshloka
{इन्द्रजित्कृतकर्मा तु पित्रे कर्म तदाऽऽत्मनः}
{निवेद्य पुनरागच्छत्त्वरयाऽऽजिशिरःप्रति}


\twolineshloka
{तमागतं तु सङ्क्रुद्धं पुनरेव युयुत्सया}
{अभिदुद्राव सौमित्रिर्विभीषणमते स्थितः}


\twolineshloka
{अकृताह्निकमेवैनं जिघांसुर्जितकाशिनम्}
{शरैर्जघान सङ्क्रुद्धः कृतसंज्ञोऽथ लक्ष्मणः}


\twolineshloka
{तयोः समभवद्युद्धं तदाऽन्योन्यं जीगीषतोः}
{अतीव चित्रमाश्चर्यं शक्रप्रह्लादयोरिव}


\twolineshloka
{अविध्यदिन्द्रजित्तीक्ष्णैः सौमित्रिं मर्मभेदिभिः}
{सौमित्रिश्चानलस्पर्शैरविध्यद्रावणिं शरैः}


\twolineshloka
{सौमित्रिशरसंस्पर्शाद्रावणिः क्रोधमूर्च्छितः}
{असृजल्लक्ष्मणायाष्टौ शरानाशीविषोपमान्}


\twolineshloka
{तस्येषून्पावकस्पर्शैः सौमित्रिः पत्रिभिस्त्रिभिः}
{वारयामास नाराचैः सौमित्रिर्मित्रनन्दनः}


\twolineshloka
{असृजल्लक्ष्मणश्चाष्टौ राक्षसाय शरान्पुनः}
{तथा तं न्यहनद्वीरस्तन्मे निगदतः शृणु}


\twolineshloka
{एकेनास्य धनुष्मन्तं बाहुं देहादपातयत्}
{द्वितीयेन तु बाणेन भुजमन्यमपातयत्}


\twolineshloka
{तृतीयेन तु बाणेन शितधारेण भास्वता}
{जहार सुनसं चापि शिरो ज्वलितकुण्डलम्}


\twolineshloka
{विनिकृत्तभुजस्कन्धः कबन्धाकृतिदर्शनः}
{पपात वसुधायां तु छिन्नमूल इवद्रुमः}


\twolineshloka
{तं हत्वा सूतमप्यस्त्रैर्जघान बलिनां वरः}
{लङ्कां प्रवेशयामासुस्तं रथं वाजिनस्तदा}


\threelineshloka
{ददर्श रावणस्तं च रथं पुत्रविनाकृतम्}
{स पुत्रं निहतं श्रुत्वा त्रासात्सभ्रान्तमानसः}
{रावणः शोकमोहार्तो वैदेहीं हन्तुमुद्यतः}


\twolineshloka
{अशोकवनिकास्थां तां रामदर्शनलालसाम्}
{खड्गमादाय दुष्टात्मा जवेनाभिपपात ह}


\twolineshloka
{तं दृष्ट्वा तस्य दुर्बुद्धेरविन्ध्यः पापनिश्चयम्}
{शमयामास सङ्क्रुद्धं श्रूयतां येन हेतुना}


\twolineshloka
{महाराज्ये स्थितो दीप्ते न स्त्रियं हन्तुमर्हसि}
{हतैवैषा यदा स्त्री च बन्धनस्था च ते वशे}


\twolineshloka
{न चैषा दहभेदेन हतास्यादिति मे मतिः}
{जहि भर्तारमेवास्या हते तस्मिन्हता भवेत्}


\twolineshloka
{न हि ते विक्रमे तुल्यः साक्षादपि शतक्रतुः}
{असकृद्धि त्वया सन्द्रास्त्रासितास्त्रिदसा युधि}


\twolineshloka
{एवं बहुविधैर्वाक्यैरविन्ध्यो रावणं तदा}
{क्रुद्धं संशमयामास जगृहे च स तद्वचः}


\twolineshloka
{निर्याणे स मतिं कृत्वा नियन्तारं क्षपाचरः}
{आज्ञापयामास तदारथो मे कल्प्यतामिति}


॥इति श्रीमन्महाभारते अरण्यपर्वणि रामोपाख्यान-पर्वणि त्रिशततमोऽध्यायः॥२९०॥

\storymeta

\dnsub{अध्यायः २९१}\resetShloka

\uvacha{मार्कण्डेय उवाच}


\twolineshloka
{ततः क्रुद्धो दशग्रीवः प्रिये पुत्रे निपातिते}
{निर्ययौ रथमास्थाय हेमरत्नविभूषितम्}


\twolineshloka
{संवृतोराक्षसैर्घेरैर्विविधायुधपाणिभिः}
{अभिदुद्राव रामं स पोथयन्हरियूथपान्}


\twolineshloka
{तमाद्रवन्तं सङ्क्रुद्ध मैन्दनीलनलाङ्गदाः}
{हनुमाञ्जाम्बवांश्चैव ससैन्याः पर्यवारयन्}


\twolineshloka
{ते दशग्रीवसैन्यं तदृक्षवानरपुङ्गवाः}
{द्रुमैर्विध्वंसयाञ्चक्रुर्दशग्रीवस्य पश्यतः}


\twolineshloka
{ततः स्वसैन्यमालोक्य वध्यमानमरातिभिः}
{मायावी चासृजन्मायां रावणो राक्षसाधिपः}


\twolineshloka
{तस्य देहविनिष्क्रान्ताः शतशोऽथ सहस्रशः}
{राक्षसाः प्रत्यदृश्यन्त शरशक्त्यृष्टिपाणयः}


\twolineshloka
{तान्रामो जघ्निवान्सर्वान्दिव्येनास्त्रेण राक्षसान्}
{अथ भूयोपि मायां स व्यदधाद्राक्षसाधिपः}


\twolineshloka
{कृत्वा रामस्य रूपाणि लक्ष्मणस्य च भारत}
{अभिदुद्राव रामं च लक्ष्मणं च दशाननः}


\twolineshloka
{ततस्ते राममर्च्छन्तो लक्ष्मणं च क्षपाचराः}
{अभिपेतुस्तदा रामं प्रगृहीतशरासनाः}


\twolineshloka
{तां दृष्ट्वा राक्षसेन्द्रस्य मायामिक्ष्वाकुनन्दनः}
{उवाच रामः सौमित्रिमसभ्रान्तो बृहद्वचः}


\twolineshloka
{जहीमान्राक्षसान्पापानात्मनः प्रतिरूपकान्}
{इत्युक्त्वाऽभ्यहनद्रामो लक्ष्मणश्चात्मरूपकान्}


\twolineshloka
{ततो हर्यश्वयुक्तेन रथेनादित्यवर्चसा}
{उपतस्थे रणे रामं मातलिः शक्रसारथिः}

\uvacha{मातलिरुवाच}


\twolineshloka
{अयं हर्यश्वयुग्जैत्रो मघोनः स्यन्दनोत्तमः}
{त्वदर्थमिह सप्राप्तः सन्देशाद्वै शतक्रतोः}


\twolineshloka
{अनेन शक्रः काकुत्स्थ समरे दैत्यदानवान्}
{शतशः पुरुषव्याघ्र रथोदारेण जघ्निवान्}


\twolineshloka
{तदनन नरव्याघ्र मया यत्तेन संयुगे}
{स्यन्दनेन जहिक्षिप्रं रावणं मा चिरं कृथाः}


\twolineshloka
{इत्युक्तो राघवस्तथ्यं वचोऽशङ्कत मातलेः}
{मायैषाराक्षसस्येति तमुवाच विबीषणः}


\twolineshloka
{नेयं माया नरव्याघ्र रावणस्य दुरात्मनः}
{तदातिष्ठ रथंशीघ्रमिमसैन्द्रं महाद्युते}


\twolineshloka
{ततः प्रहृष्टः काकुत्स्थस्तथेत्युक्त्वा विभीषणम्}
{रथेनाभिपपाताथ दशग्रीवं रुषाऽन्वितः}


\twolineshloka
{हाहाकुतानि भूतानि रावणे समभिद्रुते}
{सिंहनादाः सपटहादिति दिव्यास्तथाऽनदन्}


\twolineshloka
{दशकन्धरराजसून्वोस्तथा युद्धमभून्महत्}
{अलब्धोपममन्यत्रतयोरेव तथाऽभवत्}


\twolineshloka
{सरामाय महाघोरं विससर्ज निशाचरः}
{शूलमिन्द्राशनिप्रख्यं ब्रह्मदण्डभिवोद्यतम्}


\twolineshloka
{तच्छूलं सत्वरं रामश्चच्छेद निशितैः शरैः}
{तद्दृष्ट्वा दुष्करं कर्म रावणं भयमाविशत्}


\twolineshloka
{ततः क्रुद्धः ससर्जाशु दशग्रीवः शिताञ्छरान्}
{सहस्रायुतशो रामे शस्त्राणि विविधानि च}


\twolineshloka
{ततो भुशुण्डीः शूलानि मुसलानि परश्वथान्}
{शक्तीश्च विविधाकाराः शतघ्नीश्च शितान्क्षुरान्}


\twolineshloka
{तां मायांविविधां दृष्ट्वा दशग्रीवस्य रक्षसः}
{भयात्प्रदुद्रुवुः सर्वे वानराः सर्वतोदिशम्}


\twolineshloka
{ततः सुपत्रं सुमुखंहेमपुङ्गं शरोत्तमम्}
{तूणादादाय काकुत्स्थो ब्रह्मास्त्रेण युयोज ह}


\twolineshloka
{तं प्रेक्ष्यबाणं रामेण ब्रह्मास्त्रेणानुमन्त्रितम्}
{जहृषुर्देवगन्धर्वा दृष्ट्वा शक्रपुरोगमाः}


\twolineshloka
{अल्पावशेषमायुश्च ततोऽमन्यन्त रक्षसः}
{ब्रह्मास्त्रोदीरणाच्छत्रोर्देवदानवकिन्नराः}


\twolineshloka
{ततः ससर्ज तं रामः शरमप्रतिमौजसम्}
{रावणान्तकरं घोरं ब्रह्मदण्डमिवोद्यतम्}


\threelineshloka
{मुक्तमात्रेण रामेण दूराकृष्टेन भारत}
{स तेन राक्षसश्रेष्ठः सरथः साश्वसारथिः}
{प्रजज्वाल महाज्वालेनाग्निनाभिपरिप्लुतः}


\twolineshloka
{ततः प्रहृष्टास्त्रिदशाः सहगन्धर्वचारणाः}
{निहतं रावणं दृष्ट्वा रामेणाक्लिष्टकर्मणा}


\twolineshloka
{तत्यजुस्तं महाभागं पञ्चभूतानि रावणम्}
{भ्रंशितः सर्वलोकेषु स हि ब्रह्मास्त्रतेजसा}


\twolineshloka
{शरीरधातवो ह्यस्य मासं रुधिरमेव च}
{नेशुर्ब्रह्मास्त्रनिर्दग्दा न च भस्माप्यदृश्यत}


॥इति श्रीमन्महाभारते अरण्यपर्वणि रामोपाख्यान-पर्वणि त्रिशततमोऽध्यायः॥२९१॥

\storymeta

\dnsub{अध्यायः २९२}\resetShloka

\uvacha{मार्कण्डेय उवाच}


\twolineshloka
{स हत्वा रावणं क्षुद्रं राक्षसेन्द्रं सुरद्विषम्}
{बभूव हृष्टः ससुहृद्रामः सौमित्रिणा सह}


\twolineshloka
{ततो हते दशग्रीवे देवाः सर्षिपुरोगमाः}
{आशीर्भिर्जययुक्ताभिरानर्चुस्तं महाभुजम्}


\twolineshloka
{रामं कमलपत्राक्षं तुष्टुवुः सर्वदेवताः}
{गन्धर्वाः पुष्पवर्षैश्च वाग्भिश्च त्रिदशालयाः}


\twolineshloka
{पूजयित्वा रणे रामं प्रतिजग्मुर्यथागतम्}
{तन्महोत्सवसङ्काशमासीदाकाशमच्युत}


\twolineshloka
{ततो हत्वा दशग्रीवं लङ्कां रामो महायशाः}
{विभीषणाय प्रददौ प्रभुः परपुरञ्जयः}


\twolineshloka
{ततः सीतां पुरस्कृत्य विभीषणपुरस्कृताम्}
{अविन्ध्यो नाम सुप्रज्ञो वृद्धामात्यो विनिर्ययौ}


\twolineshloka
{उवाच च महात्मानं काकुत्स्थं दैन्यमास्थितम्}
{प्रतीच्छ देवीं सद्वृत्तां महात्मञ्जानकीमिति}


\twolineshloka
{एतच्छ्रुत्वा वचस्तस्मादवतीर्य रथोत्तमात्}
{बाष्पेणापिहितां सीतां ददर्शेक्ष्वाकुनन्दनः}


\twolineshloka
{तां दृष्ट्वा चारुसर्वाङ्गीं यानस्थां शोककर्शिताम्}
{मलोपचितसर्वाङ्गीं जटिलां कृष्णवाससम्}


\twolineshloka
{उवाच रामो वैदेहीं परामर्शविशङ्कितः}
{लक्षयित्वेङ्गितं सर्वं प्रियं तस्यै निवेद्य सः}


\threelineshloka
{गच्छ वैदेहि मुक्ता त्वं यत्कार्यं तन्मया कृतम्}
{मामासाद्यपतिं भद्रे न त्वं राक्षसवेश्मनि}
{जरां व्रजेथा इति मे निहतोऽसौ निशाचरः}


\twolineshloka
{कथं ह्यस्मद्विधो जातु जानन् धर्मविनिश्चयम्}
{परहस्तगतां नारीं मुहूर्तमपि धारयेत्}


\twolineshloka
{सुवृत्तामसुवृत्तां वाऽप्यहं त्वामद्य मैथिलि}
{नोत्सहे परिभोगाय श्वावलीढं हविर्यथा}


\twolineshloka
{ततः सा सहसा बाला तच्छ्रुत्वा दारुणं वचः}
{पपात देवी व्यथिता निकृत्ता कदली यथा}


\twolineshloka
{योऽप्यस्या हर्षसम्भूतो मुखरागः पुराऽभवत्}
{क्षणेन सपुनर्नष्टो निःश्वासादिव दर्पणे}


\twolineshloka
{ततस्ते हरयः सर्वे तच्छ्रुत्वा रामभाषितम्}
{गतासुकल्पा निश्चेष्टा बभूवुः सहलक्ष्मणाः}


\twolineshloka
{ततो देवो विशुद्धात्मा विमानेन चतुर्मुखः}
{पद्मयोनिर्जगत्स्रष्टा दर्शयामास राघवम्}


\twolineshloka
{शक्रश्चाग्निश्च वायुश्च यमो वरुण एव च}
{यक्षाधिपश्च भगवांस्तथा सप्तर्षयोऽमलाः}


\twolineshloka
{राजा दशरथश्चैव दिव्यभास्वरमूर्तिमान्}
{विमानेन महार्हेण हंसयुक्तेन भास्वता}


\twolineshloka
{ततोऽन्तरिक्षं तत्सर्वं देवगन्धर्वसङ्कुलम्}
{शुशुभे तारकाचित्रं शरदीव नभस्तलम्}


\twolineshloka
{तत उत्थाय वैदेही तेषां मध्ये यशस्विनी}
{उवाच वाक्यं कल्याणी रामं पृथुलवक्षसम्}


\twolineshloka
{राजपुत्र न ते कोपं करोमि विदिताहि मे}
{गतिः स्त्रीणां नराणां च शृणु चेदं वचो मम}


\twolineshloka
{अन्तश्चरति भूतानां मातरिश्वा सदागतिः}
{स मे विमुञ्चतु प्राणान्यदि पापं चराम्यहम्}


\twolineshloka
{अग्निरापस्तथाऽऽकाशं पृथिवी वायुरेव च}
{विमुञ्चन्तु मम प्राणान्यदि पापं चराम्यहम्}


\twolineshloka
{यथाऽहं त्वदृते वीर नान्यं स्वप्नेऽप्यचिन्तयम्}
{तथा मे देव निर्दिष्टस्त्वमेव हि पतिर्भव}


\twolineshloka
{ततोऽन्तरिक्षे वागारीत्सुभगा लोकसाक्षिणी}
{पुण्यासंहर्षणी तेषां वानराणां महात्मनाम्}

\uvacha{वायुरुवाच}

\twolineshloka
{भो भो राघव सत्यं वै वायुरस्मि सदागतिः}
{अपापा मैथिली राजन् सङ्गच्छ सह भार्यया}

\uvacha{अग्निरुवाच}

\twolineshloka
{अहमन्तःशरीरस्थो भूतानां रघुनन्दन}
{सुसूक्ष्ममपि काकुत्स्थ मैथिली नापराध्यति}

\uvacha{वरुण उवाच}

\twolineshloka
{रसा वै मत्प्रसूता हि भूतदेहेषु राघव}
{अहंवै त्वां प्रब्रवीमि मैथिली प्रतिगृह्यताम्}

\uvacha{यम उवाच}

\twolineshloka
{धर्मोऽहमस्मि काकुत्स्थ साक्षी लोकस्य कर्मणाम्}
{शुभाशुभानां सीतेयमपापा प्रतिगृह्यताम्}

\uvacha{ब्रह्मोवाच}

\twolineshloka
{पुत्र नैतदिहाश्चर्यं त्वयि राजर्षिधर्मणि}
{साधो सद्वृत्त काकुत्स्थ शृणु चेदं वचो मम}

\twolineshloka
{शत्रुरेष त्वया वीर देवगन्धर्वभोगिनाम्}
{यक्षाणां दानवानां च महर्षीणां च पातितः}

\twolineshloka
{अवध्यः सर्वभूतानां मत्प्रसादात्पुराऽभवत्}
{कस्माच्चित्कारणात्पापः कञ्चित्कालमुपेक्षितः}


\twolineshloka
{वधार्थमात्मनस्तेन हृता सीता दुरात्मना}
{नलकूबरशापेन रक्षा चास्याः कृता मया}


\twolineshloka
{यदि ह्यकामामासेवेत्स्त्रियमन्यामपि ध्रुवम्}
{शतधाऽस्य फलेन्मूर्धा इत्युक्तः सोभवत्पुरा}


\twolineshloka
{नात्रशङ्का त्वया कार्या प्रतीच्छेमां महामते}
{कृतं त्वया महत्कार्यं देवानाममितप्रभ}

\uvacha{दशरथ उवाच}


\twolineshloka
{प्रीतोस्मि वत्स भद्रं ते पिता दशरथोस्मि ते}
{अनुजानामि राज्यं च प्रशाधि पुरुषोत्तम}

\uvacha{राम उवाच}


\twolineshloka
{अभिवादये त्वां राजेन्द्र यदि त्वं जनको मम}
{गमिष्यामि पुरीं रम्यामयोध्यां शासनात्तव}

\uvacha{मार्कण्डेय उवाच}


\threelineshloka
{तमुवाच पिता भूयः प्रहृष्टो भरतर्षभ}
{गच्छायोध्यां प्रशाधि त्वंराम रक्तान्तलोचन}
{सपूर्णानीहवर्षाणि चतुर्दश महाद्युते}


\twolineshloka
{ततो देवान्नमस्कृत्य मुहृद्भिरभिनन्दितः}
{महेन्द्रइव पौलोम्या भार्यया स समेयिवान्}


\twolineshloka
{ततो वरं ददौ तस्मै ह्यविन्ध्याय परन्तपः}
{त्रिजटां चार्थमानाभ्यां योजयामास राक्षसीम्}


\twolineshloka
{तमुवाच ततो ब्रह्मा देवैः शक्रषुरोगमैः}
{कौसल्यामातरिष्टांस्ते वरानद्य ददानि कान्}


\twolineshloka
{वव्रेरामः स्थितिं धर्मे शत्रुभिश्चापराजयम्}
{राक्षसैर्निहतानां च वानराणां समुद्भवम्}


\twolineshloka
{ततस्ते ब्रह्मणा प्रोक्ते तथेतिवचने तदा}
{समुत्तस्थुर्महाराज वानरा लब्धचेतसः}


\twolineshloka
{सीता चापि महाभागा वरं हनुमते ददौ}
{रामकीर्त्या समं पुत्र जीवितं ते भविष्यति}


\twolineshloka
{दिव्यास्त्वामुपभोगाश्च मत्प्रसादकृताः सदा}
{उपस्थास्यन्ति हनुमन्निति स्म हरिलोचन}


\twolineshloka
{ततस्ते प्रेक्षमाणानां तेपामक्लिष्टकर्मणाम्}
{अन्तर्धानं ययुर्देवाः सर्वे शक्रपुरोगमाः}


\twolineshloka
{दृष्ट्वा रामं तु जानक्या सङ्गतं शक्रसारथिः}
{उवाच परमप्रीतसुहृन्मध्य इदं वचः}


\twolineshloka
{देवगन्धर्वयक्षाणां मानुषासुरभोगिनाम्}
{अपनीतं त्वया दुःखमिदं सत्यपराक्रम}


\twolineshloka
{सदेवासुरगन्धर्वा यक्षराक्षसपन्नगाः}
{कथयिष्यन्ति लोकास्त्वां यावद्भूमिर्धरिष्यति}


\twolineshloka
{इत्येवमुक्त्वाऽनुज्ञाप्यरामं शस्त्रभृतांवरम्}
{सपूज्यापाक्रमत्तेन रथेनादित्यवर्चसा}


\twolineshloka
{ततःसीतां पुरस्कृत्य रामः सौमित्रिणा सह}
{सुग्रीवप्रमुखैश्चैव सहितः सर्ववानरैः}


\twolineshloka
{विधाय रक्षां लङ्कायां विभीषणपुरस्कृतः}
{सन्ततार पुनस्तेन सेतुना मकरालयम्}


\twolineshloka
{पुष्पकेण विमानेन खेचरेण विराजता}
{कामगेन यथामुख्यैरमात्यैः संवृतो वसी}


\twolineshloka
{ततस्तीरे समुद्रस्य यत्र शिश्ये स पार्थिवः}
{तत्रैवोवास धर्मात्मा सहितः सर्ववानरैः}


\twolineshloka
{अथैनान्राघवः काले समानीयाभिपूज्य च}
{विसर्जयामास तदा रत्नैः सन्तोष्य सर्वशः}


\twolineshloka
{गतेषु वानरेन्द्रेषु गोपुच्छर्क्षेषु तेषु च}
{सुग्रीवसहितो रामः किष्किन्दां पुनरागमत्}


\twolineshloka
{विभीषणेनानुगतः सुग्रीवसहितस्तदा}
{पुष्पकेण विमानेन वैदेह्या दर्शयन्वनम्}


\twolineshloka
{किष्किन्धां तु समासाद्यरामः प्रहरतांवरः}
{अङ्गदं कृतकर्माणं यौवराज्येऽभ्यषेचयत्}


\twolineshloka
{ततस्तैरेव सहितो रामः सौमित्रिणा सह}
{यथागतेन मार्गेण प्रययौ स्वपुरं प्रति}


\twolineshloka
{अयोध्यां स समासाद्यपुरीं राष्ट्रपतिस्ततः}
{भरताय हनूमन्तं दूतं प्रास्थापयद्द्रुतम्}


\twolineshloka
{लक्षयित्वेङ्गितं सर्वप्रियं तस्मै निवेद्य वै}
{वायुपुत्रे पुनः प्राप्ते नन्दिग्राममुपाविशत्}


\threelineshloka
{सतत्रमलदिग्धाङ्गं भरतं चीरवाससम्}
{नन्दिग्रामगतंरामः सशत्रुघ्नं सराघवः}
{अग्रतःपादुके कृत्वा ददर्शासीनमासने}


\twolineshloka
{समेत्यभरतेनाथ शत्रुघ्नेन च वीर्यवान्}
{राघवः सहसौमित्रिर्मुमुदे भरतर्षभ}


\twolineshloka
{ततो भरतशत्रुघ्नौ समेतौ गुरुणा तदा}
{वैदेह्या दर्शनेनोभौ प्रहर्षं समवापतुः}


\twolineshloka
{तस्मै तद्भरतो राज्यमागतायातिसत्कृतम्}
{न्यासं निर्यातयामास युक्तः परमया मुदा}


\twolineshloka
{ततस्तं वैष्णवे शूरं नक्षत्रेऽभिजितेऽहनि}
{वसिष्ठो वामदेवश्च सहितावभ्यषिञ्चताम्}


\twolineshloka
{सोभिषिक्तः कपिश्रेष्ठं सुग्रीवं ससुहृज्जनम्}
{विभीषणं च पौलस्त्यमन्वजानाद्गृहान्प्रति}


\twolineshloka
{अभ्यर्च्य विविधै रत्नैः प्रीतियुक्तौ मुदा युतौ}
{समाधायेतिकर्तव्यं दुःखेन विससर्ज ह}


\twolineshloka
{पुष्पकं च विमानं तत्पूजयित्वा स राघवः}
{प्रादाद्वैश्रवणायैव प्रीत्या स रघुनन्दनः}


\twolineshloka
{ततो देवर्षिसहितः सरितं गोमतीमनु}
{शताश्वमेधानाजह्रे जारूथ्यान्स निरर्गलान्}


॥इति श्रीमन्महाभारते अरण्यपर्वणि रामोपाख्यान-पर्वणि त्रिशततमोऽध्यायः॥२९२॥

\storymeta

\dnsub{अध्यायः २९३}\resetShloka

\uvacha{मार्कण्डेय उवाच}


\twolineshloka
{एवमेतन्महाबाहो रामेणामिततेजसा}
{प्राप्तं व्यसनमत्युग्रं वनवासकृतं पुरा}


\twolineshloka
{मा शुचः परुषव्याघ्र क्षत्रियोसि परन्तप}
{बाहुवीर्याश्रयेमार्गे वर्तसे दीप्तनिर्णये}


\twolineshloka
{न हि ते वृजिनं किञ्चिद्दृश्यते परमण्वपि}
{अस्मिन्मार्गे निपीदेयुः सेन्द्रा अपि सुरासुराः}


\twolineshloka
{संहत्य निहतोवृत्रो मरुद्भिर्वज्रपाणिना}
{नमुचिश्चैवदुर्धर्षो दीर्गजिह्वा चराक्षसी}


\twolineshloka
{सहायवति सर्वार्थाः सतिष्ठन्तीह सर्वशः}
{किन्नु तस्याजितं सङ्ख्ये यस्य भ्राता धनञ्जयः}


\twolineshloka
{अयं च बलिनांश्रेष्ठो भीमो भीमपराक्रमाः}
{युवानौ च महेष्वासौ वीरौ माद्रवतीसुतौ}


\twolineshloka
{एभिः सहायैः कस्मात्त्वं विषीदसि परन्तप}
{य इमे वज्रिणः सेनां जयेयुः समरुद्गणाम्}


\twolineshloka
{त्वमप्येभिर्महेष्वासैः सहायैर्देवरूपिभिः}
{विजेष्यसि रणे सर्वानमित्रान्भरतर्षभ}


\twolineshloka
{इतश्च त्वमिमां पश्यसैन्धवेन दुरात्मना}
{बलिना वीर्यमत्तेन हृतामेभिर्महात्मभिः}


\twolineshloka
{आनीतां द्रौपदीं कृष्णां कृत्वा कर्म सुदुष्करम्}
{जयद्रथं च राजानं विजितं वशमागतम्}


\twolineshloka
{असहायेन रामेण वैदेही पुनराहृता}
{हत्वासङ्ख्ये दशग्रीवं राक्षसं भीमविक्रमम्}


\twolineshloka
{यस्य शाखामृगामित्राण्यृक्षाः कालमुखास्तथा}
{जात्यन्तरगता राजन्नेतद्बुद्ध्याऽनुचिन्तय}


\twolineshloka
{तस्मात्सर्वं कुरुश्रेष्ठ मा शुचो भरतर्षभ}
{त्वद्विधा हि महात्मानो न शोचन्ति परन्तप}

\uvacha{वैशम्पायन उवाच}


\twolineshloka
{एवमाश्वासितो राजामार्कण्डेयेन धीमता}
{त्यक्त्वा दुःखमदीनात्मा पुनरप्येनमब्रवीत्}


॥इति श्रीमन्महाभारते अरण्यपर्वणि रामोपाख्यान-पर्वणि त्रिशततमोऽध्यायः॥२९३॥

रामोपाख्यान-पर्व समाप्तम्॥१८॥ 

\closesection