\sect{हनूमता रामकथाकथनम्}

\src{श्रीमन्महाभारतम्}{वन-पर्व}{तीर्थयात्रापर्व}{अध्यायाः १४९--१५०}
\vakta{हनुमान्}
\shrota{भीमः}
\tags{concise, complete}
\notes{After an interesting encounter between Hanuman and Bhima, at Bhima's request, Hanuman narrates Ramayana to Bhima.}
% \textlink{http://stotrasamhita.net/wiki/Narayaniyam/Dashaka_34}
\translink{}

\storymeta

\dnsub{अध्यायः १४९}

\uvacha{हनूमानुवाच}

\addtocounter{shlokacount}{25}

\twolineshloka
{यत्ते मम परिज्ञाने कौतूहलमरिन्दम}
{तत्सर्वमखिलेन त्वं शृणु पाण्डवनन्दन}


\twolineshloka
{अहं केसरिणः क्षेत्रे वायुना जगदायुषा}
{जातः कमलपत्राक्ष हनूमान्नाम वानरः}


\twolineshloka
{सूर्यपुत्रं च सुग्रीवं शक्रपुत्रं च वालिनम्}
{सर्ववानरराजानौ सर्ववानरयूथपाः}


\twolineshloka
{उपतस्थुर्महावीर्या मम चामित्रकर्शन}
{सुग्रीवेणाभवत्प्रीतिरनिलस्याग्निना यथा}


\twolineshloka
{निकृतः स ततो भ्रात्रा कस्मिंश्चित्कारणान्तरे}
{ऋश्यमूके मया सार्धं सुग्रीवो न्यवसच्चिरम्}


\twolineshloka
{अथ दाशरथिर्वीरो रामो नाम महाबलः}
{विष्णुर्मानुषरूपेण चचार वसुधातलम्}


\twolineshloka
{स पितुः प्रियमन्विच्छन्सहभार्यः सहानुजः}
{सधनुर्धन्विनां श्रेष्ठो दण्डकारण्यमाश्रितः}


\twolineshloka
{तस्य भार्या जनस्थानाच्छलेनापहृता बलात्}
{राक्षसेन्द्रेण बलिना रावणेन दुरात्मना}


\twolineshloka
{सुवर्णरत्नचित्रेण मृगरूपेण रक्षसा}
{वञ्चयित्वा नरव्याघ्रं मारीचेन तदाऽनघ}

॥इति श्रीमन्महाभारते अरण्यपर्वणि तीर्थयात्रा-पर्वणि एकोनपञ्चाशदधिकशततमोऽध्यायः॥१४९॥


\dnsub{अध्यायः १५०}

\uvacha{हनूमानुवाच}

\twolineshloka
{हृतदारः सह भ्रात्रा पत्नीं मार्गन्स राघवः}
{दृष्टवाञ्शैलशिखरे सुग्रीवं वानरर्षभम्}


\twolineshloka
{तेन तस्याभवत्सख्यं राघवस्य महात्मनः}
{स हत्वा वालिनं राज्ये सुग्रीवं प्रत्यपादयत्}


\twolineshloka
{स राज्यं प्राप्य सुग्रीवः सीतायाः परिमार्गणे}
{वानरान्प्रेषयामास शतशोऽथ सहस्रशः}


\twolineshloka
{ततो वानरकोटीभिः सहितोऽहं नरर्षभ}
{सीतां मार्गन्महाबाहो प्रस्थितो दक्षिणां दिशम्}


\twolineshloka
{ततः प्रवृत्तिः सीताया गृध्रेण सुमहात्मना}
{सम्पातिना समाख्याता रावणस्य निवेशने}


\twolineshloka
{ततोऽहं कार्यसिद्ध्यर्थं रामस्याक्लिष्टकर्मणः}
{शतयोजनविस्तारमर्णवं सहसा प्लुतः}


\twolineshloka
{अहं स्ववीर्यादुत्तीर्य सागरं मकरालयम्}
{सुतां जनकराजस्य सीतां सुररसुतोपमाम्}


\twolineshloka
{दृष्टवान्भरतश्रेष्ठ रावणस्य निवेशने}
{समेत्य तामहं देवीं वैदेहीं राघवप्रियाम्}


\twolineshloka
{दग्ध्वा लङ्कामशेषेण साट्टप्राकारतोरणाम्}
{प्रत्यागतश्चास्य पुनर्नाम तत्र प्रकाश्य वै}


\threelineshloka
{मद्वाक्यं चावधार्याशु रामो राजीवलोचनः}
{अबद्धपूर्वमन्यैश्च बद्ध्वा सेतुं महोदधौ}
{वृतो वानरकोटीभिः समुत्तीर्णो महार्णवम्}


\twolineshloka
{ततो रामेण वीर्येण हत्वा तान्सर्वराक्षसान्}
{रणे तु राक्षसगणं रावणं लोकरावणम्}


\twolineshloka
{निशाचरेनद्रं हत्वा तु सभ्रातृसुतबान्धवम्}
{राज्येऽभिषिच्य लङ्कायां राक्षसेन्द्रं विभीषणम्}


\twolineshloka
{धार्मिकं भक्तिमन्तं च भक्तानुगतवत्सलः}
{प्रत्याहृत्य ततः सीतां नष्टां वेदश्रुतिं यथा}


\threelineshloka
{तयैव सहितः साध्व्या पत्न्या रामो महायशाः}
{गत्वा ततोऽतित्वरितः स्वां पुरीं रघुनन्दनः}
{अध्यावसत्ततोऽयोध्यामयोध्यां द्विषतां प्रभुः}


\twolineshloka
{ततः प्रतिष्ठितो राज्ये रामो नृपतिसत्तमः}
{वरं मया याचितोऽसौ रामो राजीवलोचनः}


\twolineshloka
{यावद्रामकथेयं ते भवेल्लोकेषु शत्रुहन्}
{तावज्जीवेयमित्येवं तथाऽस्त्विति च सोब्ऽरवीत्}


\twolineshloka
{सीताप्रसादाच्च सदा मामिहस्थमरिन्दम}
{उपतिष्ठन्ति दिव्या हि भोगा भीम यथेप्सिताः}


\twolineshloka
{दशवर्षसहस्राणि दशवर्षशतानि च}
{राज्यं कारितवान्रामस्ततः स्वभवनं गतः}


\twolineshloka
{तदिहाप्सरसस्तात गन्धर्वाश्च सदाऽनघ}
{तस्य वीरस्य चरितं गायन्त्यो रमयन्ति माम्}


\twolineshloka
{अयं च मार्गो मर्त्यानामगम्यः कुरुनन्दन}
{ततोऽहं रुद्धवान्मार्गं तवेमं देवसेवितम्}


\twolineshloka
{त्वामनेन पथा यान्तं यक्षो वा राक्षसोऽपि वा}
{धर्षयेद्वा शपेद्वाऽपि मा कश्चिदिति भारत}


\twolineshloka
{दिव्यो देवपथो ह्येष नात्र गच्छन्ति मानुषाः}
{यदर्थमागतश्चासि अत एव सरश्च तत्}

॥इति श्रीमन्महाभारते अरण्यपर्वणि तीर्थयात्रा-पर्वणि पञ्चाशदधिकशततमोऽध्यायः॥१५०॥

\closesection