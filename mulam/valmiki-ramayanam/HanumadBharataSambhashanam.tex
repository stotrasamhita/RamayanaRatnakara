\sect{हनूमद्भरतसम्भाषणम्}

\src{श्रीमद्-वाल्मीकि-रामायणम्}{युद्धकाण्डः}{अध्यायः १२९}{श्लोकाः १---५४}
\vakta{हनुमान्}
\shrota{भरतः}
\tags{concise, complete}
\notes{Narration of Rama's story, by Hanuman, from Bharata's departing Chitrakuta to the retrieval of Sita by conquering Lanka.}
\textlink{}
\translink{}

\storymeta


\twolineshloka
{बहूनि नाम वर्षाणि गतस्य सुमहद्धनम्}
{शृणोम्यहं प्रीतिकरं मम नाथस्य कीर्तनम्}

\twolineshloka
{कल्याणी बत गाथेयं लौकिकी प्रतिभाति मे}
{एति जीवन्तमानन्दो नरं वर्षशतादपि}

\twolineshloka
{राघवस्य हरीणां च कथमासीत् समागमः}
{कस्मिन् देशे किमाश्रित्य तत्त्वमाख्याहि पृच्छतः}

\twolineshloka
{स पृष्टो राजपुत्रेण बृस्यां समुपवेशितः}
{आचचक्षे ततः सर्वं रामस्य चरितं वने}

\twolineshloka
{यथा प्रव्राजितो रामो मातुर्दत्तो वरस्तव}
{यथा च पुत्रशोकेन राजा दशस्थो मृतः}

\twolineshloka
{यथा दूतैस्त्वमानीतस्तूर्णं राजगृहात् प्रभो}
{त्वयायोध्यां प्रविष्टेन यथा राज्यं न चेप्सितम्}

\twolineshloka
{चित्रकूटं गिरिं गत्वा राज्येनामित्रकर्शन}
{निमन्त्रितस्त्वया भ्राता धर्ममाचरता सताम्}

\twolineshloka
{स्थितेन राज्ञो वचने यथा राज्यं विसर्जितम्}
{आर्यस्य पादुके गृह्य यथासि पुनरागतः}

\twolineshloka
{सर्वमेतन्महाबाहो यथावद्विदितं तव}
{त्वयि प्रतिप्रयाते तु यद्वृत्तं तन्निबोध मे}


\twolineshloka
{अपयाते त्वयि तदा समुद्भ्रान्तमृगद्विजम्}
{परिद्यूनमिवात्यर्थं तद्वनं समपद्यत}

\twolineshloka
{तद्धस्तिमृदितं घोरं सिंहव्याघ्रमृगायुतम्}
{प्रविवेशाथ विजनं सुमहद्दण्डकावनम्}

\twolineshloka
{तेषां पुरस्ताद्बलवान् गच्छतां गहने वने}
{निनदन् सुमहानादं विराधः प्रत्यदृश्यत}

\twolineshloka
{तमुत्क्षिप्य महानादमूर्ध्वबाहुमधोमुखम्}
{निखाते प्रक्षिपन्ति स्म नदन्तमिव कुञ्जरम्}

\twolineshloka
{तत् कृत्वा दुष्करं कर्म भ्रातरौ रामलक्ष्मणौ}
{सायाह्ने शरभङ्गस्य रम्यमाश्रममीयतुः}

\twolineshloka
{शरभङ्गे दिवं प्राप्ते रामः सत्यपराक्रमः}
{अभिवाद्य मुनीन् सर्वाञ्जनस्थानमुपागमत्}

\twolineshloka
{ततः पश्चाच्छूर्पणखा रामपार्श्वमुपागता}
{ततो रामेण सन्दिष्टो लक्ष्मणः सहसोत्थितः}

\twolineshloka
{प्रगृह्य खड्गं चिच्छेद कर्णनासं महाबलः}
{चतुर्दश सहस्राणि रक्षसां भीमकर्मणाम्}

\twolineshloka
{हतानि वसता तत्र राघवेण महात्मना}
{एकेन सह सङ्गम्य रणे रामेण सङ्गताः}

\twolineshloka
{अतुर्थभागेन निःशेषा राक्षसाः कृताः}
{महाबला महावीर्यास्तपसो विघ्नकारिणः}

\twolineshloka
{निहता राघवेणाजौ दण्डकारण्यवासिनः}
{राक्षसाश्च विनिष्पिष्टाः खरश्च निहतो रणे}

\twolineshloka
{ततस्तेनार्दिता बाला रावणं समुपागता}
{रावणानुचरो घोरो मारीचो नाम राक्षसः}

\twolineshloka
{लोभयामास वैदेहीं भूत्वा रत्नमयो मृगः}
{अथैनमब्रवीद्रामं वैदेही गृह्यतामिति}

\twolineshloka
{अहो मनोहरः कान्त आश्रमो नो भविष्यति}
{ततो रामो धनुष्पाणिर्धावन्तमनुधावति}

\twolineshloka
{स तं जघान धावन्तं शरेणानतपर्वणा}
{अथ सौम्य दशग्रीवो मृगं याते तु राघवे}

\twolineshloka
{लक्ष्मणे चापि निष्क्रान्ते प्रविवेशाश्रमं तदा}
{जग्राह तरसा सीतां ग्रहः खे रोहिणीमिव}

\twolineshloka
{त्रातुकामं ततो युद्धे हत्वा गृध्रं जटायुषम्}
{प्रगृह्य सीतां सहसा जगामाशु स रावणः}

\twolineshloka
{ततस्त्वद्भुतसङ्काशाः स्थिताः पर्वतमूर्धनि}
{सीतां गृहीत्वा गच्छन्तं वानराः पर्वतोपमाः}

\twolineshloka
{दशुर्विस्मितास्तत्र रावणं राक्षसाधिपम्}
{प्रविवेश ततो लङ्कां रावणो लोकरावणः}

\twolineshloka
{तां सुवर्णपरिक्रान्ते शुभे महति वेश्मनि}
{प्रवेश्य मैथिलीं वाक्यैः सान्त्वयामास रावणः}

\twolineshloka
{तृणकद्भाषितं तस्य तं च नैर्ऋतपुङ्गवम्}
{अचिन्तयन्ती वैदेही अशोकवनिकां गता}

\twolineshloka
{न्यवर्तत ततो रामो मृगं हत्वा महावने}
{निवर्तमानः काकुत्स्थो दृष्ट्वा गृधं प्रविव्यथे}

\twolineshloka
{गृद्धं हतं ततो दग्ध्वा रामः प्रियसखं पितुः}
{मार्गमाणस्तु वैदेहीं राघवः सहलक्ष्मणः}

\twolineshloka
{गोदावरीमन्वचरद्वनोद्देशांश्च पुष्पितान्}
{आसेदतुर्महारण्ये कबन्धं नाम राक्षसम्}

\twolineshloka
{ततः कबन्धवचनाद्रामः सत्यपराक्रमः}
{ऋश्यमूकं गिरिं गत्वा सुग्रीवेण समागतः}

\twolineshloka
{तयोः समागमः पूर्वं प्रीत्या हार्दो व्यजायत}
{भ्रात्रा निरस्तः क्रुद्धेन सुग्रीवो वालिना पुरा}

\twolineshloka
{इतरेतरसंवादात् प्रगाढः प्रणयस्तयोः}
{रामस्य बाहुवीर्येण स्वराज्यं प्रत्यपादयत्}


\twolineshloka
{वालिनं समरे हत्वा महाकायं महाबलम्}
{सुग्रीवः स्थापितो राज्ये सहितः सर्ववानरैः}

\twolineshloka
{रामाय प्रतिजानीते राजपुत्र्याश्च मार्गणम्}
{आदिष्टा वानरेन्द्रेण सुग्रीवेण महात्मना}

\twolineshloka
{दश कोट्यः प्लवङ्गानां सर्वाः प्रस्थापिता दिशः}
{तेषां नो विप्रकृष्टानां विन्ध्ये पर्वतसत्तमे}

\twolineshloka
{भृशं शोकाभितप्तानां महान् कालोऽत्यवर्तत}
{भ्राता तु गृधराजस्य सम्पातिर्नाम वीर्यवान्}

\twolineshloka
{समाख्याति स्म वसतिं सीताया रावणालये}
{सोऽहं शोकपरीतानां दुःखं तज्ज्ञातिनां नुदन्}

\twolineshloka
{आत्मवीर्यं समास्थाय योजनानां शतं प्लुतः}
{तत्राहमेकामद्राक्षमशोकवनिकां गताम्}

\twolineshloka
{कौशेयवस्त्रां मलिनां निरानन्दां दृढव्रताम्}
{तया समेत्य विधिवत् पृष्ट्वा सर्वमनिन्दिताम्}

\twolineshloka
{अभिज्ञानं च मे दत्तमर्चिष्मान् स महामणिः}
{अभिज्ञानं मणि लब्ध्वा चरितार्थोऽहमागतः}

\twolineshloka
{मया च पुनरागम्य रामस्याकिष्टकर्मणः}
{अभिज्ञानं मया दत्तमर्चिष्मान् स महामणिः}

\twolineshloka
{श्रुत्वा तु मैथिली हृष्टस्त्वाशश॑से॒ च जीवितम्}
{जीवितान्तमनुप्राप्तः पीत्वामृतमिवातुरः}

\twolineshloka
{उद्योजयिष्यन्नुद्योगं दधे कामं वधे मनः}
{जिघांसुरिव लोकान्ते सर्वाल्लोकान् विभावसुः}

\twolineshloka
{ततः समुद्रमासाद्य नलं सेतुमकारयत्}
{अतरत् कपिवीराणां वाहिनी तेन सेतुना}

\twolineshloka
{प्रहस्तमवधीन्नीलः कुम्भकर्णं तु राघवः}
{लक्ष्मणो रावणसुतं स्वयं रामस्तु रावणम्}

\twolineshloka
{स शक्रेण समागम्य यमेन वरुणेन च}
{महेश्वरः स्वयं भूम्यां तथा दशरथेन च}

\twolineshloka
{तैश्च दत्तवरः श्रीमानृषिभिश्च समागतः}
{सुरर्षिभिश्च काकुत्स्थो वरौलेमे परन्तपः}

\twolineshloka
{स तु दत्तवरः प्रीत्या वानरैश्च समागतः}
{पुष्पकेण विमानेन किष्किन्धामभ्युपागमत्}

\twolineshloka
{तं गङ्गां पुनरासाद्य वसन्तं मुनिसन्निधौ}
{अविघ्नं पुष्ययोगेन श्वो रामं द्रष्टुमर्हसि}

\fourlineindentedshloka
{ततस्तु सत्यं हनुमद्वचो महन्-}
{निशम्य हृष्टो भरतः कृताञ्जलिः}
{उवाच वाणीं मनसः प्रहर्षिणीं}
{चिरस्य पूर्णः खलु मे मनोरथः} 


इत्यार्षे श्रीमद्रामायणे वाल्मीकीये आदिकाव्ये चतुर्विंशतिसहत्रिकायां संहितायाम् युद्धकाण्डे हनूमद्भरतसम्भाषणं नाम एकोनत्रिंशदुत्तरशततमः सर्गः॥

\closesection