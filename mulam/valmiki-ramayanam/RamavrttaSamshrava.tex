\chapt{रामवृत्तसंश्रवः}

\src{श्रीमद्-वाल्मीकि-रामायणम्}{सुन्दरकाण्डः}{अध्यायः ३१}{श्लोकाः १---१९}
\vakta{हनुमान्}
\shrota{सीता}
\tags{concise, complete}
\notes{Narration of Rama's story to Sita by Hanuman.}
\textlink{}
\translink{}

\storymeta


\twolineshloka
{एवं बहुविधां चिन्तां चिन्तयित्वा महाकपिः}
{संश्रवे मधुरं वाक्यं वैदेह्या व्याजहार ह}

\twolineshloka
{राजा दशरथो नाम स्थकुञ्जरवाजिमान्}
{पुण्यशीलो महाकीर्तिर्ऋजुरासीन्महायशाः}

\twolineshloka
{राजर्षीणां गुणश्रेष्ठस्तपसा चर्षिभिः समः}
{चक्रवर्तिकुले जातः पुरन्दरसमो बले}

\twolineshloka
{अहिंसारतिरक्षुद्रो घृणी सत्यपराक्रमः}
{मुख्यश्चेक्ष्वाकुवंशस्य लक्ष्मीवाँल्लक्ष्मिवर्धनः}

\twolineshloka
{पार्थिवव्यञ्जनैर्युक्तः पृथुश्रीः पार्थिवर्षभः}
{पृथिव्यां चतुरान्तायां विश्रुतः सुखदः सुखी}

\twolineshloka
{तस्य पुत्रः प्रियो ज्येष्ठस्ताराधिपनिभाननः}
{रामो नाम विशेषज्ञः श्रेष्ठः सर्वधनुष्मताम्}

\twolineshloka
{रक्षिता स्वस्य वृत्तस्य स्वजनस्यापि रक्षिता}
{रक्षिता जीवलोकस्य धर्मस्य च परन्तपः}

\twolineshloka
{तस्य सत्याभिसन्धस्य वृद्धस्य वचनात् पितुः}
{सभार्यः सह च भ्रात्रा वीरः प्रव्राजितो वनम्}

\twolineshloka
{तेन तत्र महारण्ये मृगयां परिधावता}
{राक्षसा निहताः शूरा बहवः कामरूपिणः}

\twolineshloka
{जनस्थानवधं श्रुत्वा हतौ च खरदूषणौ}
{ततस्त्वमर्षापहृता जानकी रावणेन तु}

\twolineshloka
{वञ्चयित्वा वने रामं मृगरूपेण मायया}
{स मार्गमाणस्तां देवीं रामः सीतामनिन्दिताम्}

\twolineshloka
{आससाद वने मित्रं सुग्रीवं नाम वानरम्}
{ततः स वालिनं हत्वा रामः परपुरञ्जयः}

\twolineshloka
{प्रायच्छत् कपिराज्यं तत् सुग्रीवाय महाबलः}
{सुग्रीवेणापि सन्दिष्टा हरयः कामरूपिणः}

\twolineshloka
{दिक्षु सर्वासु तां देवीं विचिन्वन्ति सहस्रशः}
{अहं सम्पातिवचनाच्छतयोजनमायतम्}

\twolineshloka
{अस्या हेतोर्विशालाक्ष्याः सागरं वेगवान् प्लुतः}
{यथारूपां यथावर्णां यथालक्ष्मीं च निश्चिताम्}

\twolineshloka
{अश्रौषं राघवस्याहं सेयमासादिता मया}
{विररामैवमुक्त्वासौ वाचं वानरपुङ्गवः}

\threelineshloka
{जानकी चापि तच्छ्रुत्वा विस्मयं परमं गता}
{ततः सा वक्रकेशान्ता सुकेशी केशसंवृतम्}
{उन्नम्य वदनं भीरुः शिंशपामन्ववैक्षत}

\fourlineindentedshloka
{निशम्य सीता वचनं कपेश्च}
{दिशश्च सर्वाः प्रदिशश्च वीक्ष्य}
{स्वयं प्रहर्षं परमं जगाम}
{सर्वात्मना राममनुस्मरन्ती}

\fourlineindentedshloka
{सा तिर्यगूर्ध्वं च तथाप्यधस्तान्-}
{निरीक्षमाणा तमचिन्त्यबुद्धिम्}
{ददर्श पिङ्गाधिपतेरमात्यं}
{वातात्मजं सूर्यमिवोदयस्थम्}

इत्यार्षे श्रीमद्रामायणे वाल्मीकीये आदिकाव्ये चतुर्विंशतिसहस्रिकायां संहितायां सुन्दरकाण्डे रामवृत्तसंश्रवो नाम एकत्रिंशः सर्गः॥


\closesection