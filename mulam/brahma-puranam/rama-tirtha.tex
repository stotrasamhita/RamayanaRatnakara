\chapt{ब्रह्म-पुराणम्}

\sect{रामतीर्थवर्णनम्}

\src{ब्रह्म-पुराणम्}{}{अध्यायः १२३}{}
% \tags{concise, complete}
\notes{This chapter describes the significance of Rama Tirtha, a sacred place associated with that helped Dasharatha expiate his sins.}
\textlink{https://sa.wikisource.org/wiki/ब्रह्मपुराणम्/अध्यायः_१२३}
\translink{}

\storymeta


\twolineshloka
{रामतीर्थमिति ख्यातं भ्रूणहत्याविनाशनम्}
{तस्य श्रवणमात्रेण सर्वपापैः प्रमुच्यते} %॥१॥

\twolineshloka
{इक्ष्वाकुवंशप्रभवः क्षत्रियो लोकविश्रुतः}
{बलवान्मतिमाञ्शूरो यथा शक्रः पुरन्दरः} %॥२॥

\twolineshloka
{पितृपैतामहं राज्यं कुर्वन्नास्ते यथा बलिः}
{तस्य तिस्रो महिष्यः स्यू राज्ञो दशरथस्य हि} %॥३॥

\twolineshloka
{कौशल्या च सुमित्रा च कैकेयी च महामते}
{एताः कुलीनाः सुभगा रूपलक्षणसंयुताः} %॥४॥

\twolineshloka
{तस्मिन् राजनि राज्ये तु स्थितेऽयोध्यापतौ मुने}
{वसिष्ठे ब्रह्मविच्छ्रेष्ठे पुरोधसि विशेषतः} %॥५॥

\twolineshloka
{न च व्याधिर्न दुर्भिक्षं न चावृष्टिर्न चाधयः}
{ब्रह्मक्षत्रविशां नित्यं शूद्राणां च विशेषतः} %॥६॥

\twolineshloka
{आश्रमाणां तु सर्वेषामानन्दोऽभूत्पृथक्पृथक्}
{तस्मिञ्शासति राजेन्द्र इक्ष्वाकूणां कुलोद्वहे} %॥७॥

\twolineshloka
{देवानां दानवानां तु राज्यार्थे विग्रहोऽभवत्}
{क्वापि तत्र जयं प्रापुर्देवाः क्वापि तथेतरे} %॥८॥

\twolineshloka
{एवं प्रवर्तमाने तु त्रैलोक्यमतिपीडितम्}
{अभून्नारद तत्राहमवदं दैत्यदानवान्} %॥९॥

\twolineshloka
{देवांश्चापि विशेषेण न कृतं तैर्मदीरितम्}
{पुनश्च सङ्गरस्तेषां बभूव सुमहान्मिथः} %॥१०॥

\twolineshloka
{विष्णुं गत्वा सुराः प्रोचुस्तथेशानं जगन्मयम्}
{तावूचतुरुभौ देवानसुरान् दैत्यदानवान्} %॥११॥

\twolineshloka
{तपसा बलिनो यान्तु पुनः कुर्वन्तु सङ्गरम्}
{तथेत्याहुर्ययुः सर्वे तपसे नियतव्रताः} %॥१२॥

\twolineshloka
{ययुस्तु राक्षसान् देवाः पुनस्ते मत्सरान्विताः}
{देवानां दानवानां च सङ्गरोऽभूत्सुदारुणः} %॥१३॥

\twolineshloka
{न तत्र देवा जेतारो नैव दैत्याश्च दानवाः}
{संयुगे वर्तमाने तु वागुवाचाशरीरिणी} %॥१४॥

\uvacha{आकाशवागुवाच}

\onelineshloka
{येषां दशरथो राजा ते जेतारो न चेतरे} %}%॥* १५॥

\uvacha{ब्रह्मोवाच}

\twolineshloka
{इति श्रुत्वा जयायोभौ जग्मतुर्देवदानवौ}
{तत्र वायुस्त्वरन् प्राप्तो राजानमवदत्तदा} %॥१६॥

\uvacha{वायुरुवाच}


\twolineshloka
{आगन्तव्यं त्वया राजन् देवदानवसङ्गरे}
{यत्र राजा दशरथो जयस्तत्रेति विश्रुतम्} %॥१७॥

\onelineshloka
{तस्मात्त्वं देवपक्षे स्या भवेयुर्जयिनः सुराः}%॥* १८॥

\uvacha{ब्रह्मोवाच}


\twolineshloka
{तद्वायुवचनं श्रुत्वा राजा दशरथो नृपः}
{आगम्यते मया सत्यं गच्छ वायो यथासुखम्} %॥१९॥

\twolineshloka
{गते वायौ तदा दैत्या आजग्मुर्भूपतिं प्रति}
{तेऽप्यूचुर्भगवन्नस्मत्साहाय्यं कर्तुमर्हसि} %॥२०॥

\twolineshloka
{राजन् दशरथ श्रीमन् विजयस्त्वयि संस्थितः}
{तस्मात्त्वं वै दैत्यपतेः साहाय्यं कर्तुमर्हसि} %॥२१॥

\twolineshloka
{ततः प्रोवाच नृपतिर्वायुना प्रार्थितः पुरा}
{प्रतिज्ञातं मया तच्च यान्तु दैत्याश्च दानवाः} %॥२२॥

\twolineshloka
{स तु राजा तथा चक्रे गत्वा चैव त्रिविष्टपम्}
{युद्धं चक्रे तथा दैत्यैर्दानवैः सह राक्षसैः} %॥२३॥

\twolineshloka
{पश्यत्सु देवसङ्घेषु नमुचेर्भ्रातरस्तदा}
{विविधुर्निशितैर्बाणैरथाक्षं नृपतेस्तथा} %॥२४॥

\twolineshloka
{भिन्नाक्षं तं रथं राजा न जानाति स सम्भ्रमात्}
{राजान्तिके स्थिता सुभ्रूः कैकेय्याज्ञायि नारद} %॥२५॥

\twolineshloka
{न ज्ञापितं तया राज्ञे स्वयमालोक्य सुव्रता}
{भग्नमक्षं समालक्ष्य चक्रे हस्तं तदा स्वकम्} %॥२६॥

\twolineshloka
{अक्षवन्मुनिशार्दूल तदेतन्महदद्भुतम्}
{रथेन रथिनां श्रेष्ठस्तया दत्तकरेण च} %॥२७॥

\twolineshloka
{जितवान् दैत्यदनुजान् देवैः प्राप्य वरान् बहून्}
{ततो देवैरनुज्ञातस्त्वयोध्यां पुनरभ्यगात्} %॥२८॥

\twolineshloka
{स तु मध्ये महाराजो मार्गे वीक्ष्य तदा प्रियाम्}
{कैकेय्याः कर्म तद्दृष्ट्वा विस्मयं परमं गतः} %॥२९॥

\twolineshloka
{ततस्तस्यै वरान् प्रादात्त्रींस्तु नारद सा अपि}
{अनुमान्य नृपप्रोक्तं कैकेयी वाक्यमब्रवीत्} %॥३०॥

\uvacha{कैकेय्युवाच}


\onelineshloka
{त्वयि तिष्ठन्तु राजेन्द्र त्वया दत्ता वरा अमी}%॥* ३१॥

\uvacha{ब्रह्मोवाच}


\twolineshloka
{विभूषणानि राजेन्द्रो दत्त्वा स प्रियया सह}
{रथेन विजयी राजा ययौ स्वनगरं सुखी} %॥३२॥

\twolineshloka
{योषितां किमदेयं हि प्रियाणामुचितागमे}
{स कदाचिद्दशरथो मृगयाशीलिभिर्वृतः} %॥३३॥

\twolineshloka
{अटन्नरण्ये शर्वर्यां वारिबन्धमथाकरोत्}
{सप्तव्यसनहीनेन भवितव्यं तु भूभुजा} %॥३४॥

\twolineshloka
{इति जानन्नपि च तच्चकार तु विधेर्वशात्}
{गर्तं प्रविश्य पानार्थमागतान्निशितैः शरैः} %॥३५॥

\twolineshloka
{मृगान् हन्ति महाबाहुः शृणु कालविपर्ययम्}
{गर्तं प्रविष्टे नृपतौ तस्मिन्नेव नगोत्तमे} %॥३६॥

\twolineshloka
{वृद्धो वैश्रवणो नाम न शृणोति न पश्यति}
{तस्य भार्या तथाभूता तावब्रूतां तदा सुतम्} %॥३७॥

\uvacha{मातापितरावूचतुः॒}


\twolineshloka
{आवां तृषार्तौ रात्रिश्च कृष्णा चापि प्रवर्तते}
{वृद्धानां जीवितं कृत्स्नं बालस्त्वमसि पुत्रक} %॥३८॥

\twolineshloka
{अन्धानां बधिराणां च वृद्धानां धिक्च जीवितम्}
{जराजर्जरदेहानां धिग्धिक्पुत्रक जीवितम्} %॥३९॥

\twolineshloka
{तावत्पुम्भिर्जीवितव्यं यावल्लक्ष्मीर्दृढं वपुः}
{यावदाज्ञाप्रतिहता तीर्थादावन्यथा मृतिः} %॥४०॥

\uvacha{ब्रह्मोवाच}

\twolineshloka
{इत्येतद्वचनं श्रुत्वा वृद्धयोर्गुरुवत्सलः}
{पुत्रः प्रोवाच तद्दुःखं गिरा मधुरया हरन्} %॥४१॥

\uvacha{पुत्र उवाच}

\twolineshloka
{मयि जीवति किं नाम युवयोर्दुःखमीदृशम्}
{न हरत्यात्मजः पित्रोर्यश्चरित्रैर्मनोरुजम्} %॥४२॥

\onelineshloka
{तेन किं तनुजेनेह कुलोद्वेगविधायिना}%॥* ४३॥

\uvacha{ब्रह्मोवाच}


\twolineshloka
{इत्युक्त्वा पितरौ नत्वा तावाश्वास्य महामनाः}
{तरुस्कन्धे समारोप्य वृद्धौ च पितरौ तदा} %॥४४॥

\twolineshloka
{हस्ते गृहीत्वा कलशं जगाम ऋषिपुत्रकः}
{स ऋषिर्न तु राजानं जानाति नृपतिर्द्विजम्} %॥४५॥

\twolineshloka
{उभौ सरभसौ तत्र द्विजो वारि समाविशत्}
{सत्वरं कलशे न्युब्जे वारि गृह्णन्तमाशुगैः} %॥४६॥

\twolineshloka
{द्विजं राजा द्विपं मत्वा विव्याध निशितैः शरैः}
{वनद्विपोऽपि भूपानामवध्यस्तद्विदन्नपि} %॥४७॥

\twolineshloka
{विव्याध तं नृपः कुर्यान्न किं किं विधिवञ्चितः}
{स विद्धो मर्मदेशे तु दुःखितो वाक्यमब्रवीत्} %॥४८॥

\uvacha{द्विज उवाच}


\twolineshloka
{केनेदं दुःखदं कर्म कृतं सद्ब्राह्मणस्य मे}
{मैत्रो ब्राह्मण इत्युक्तो नापराधोऽस्ति कश्चन} %॥४९॥

\uvacha{ब्रह्मोवाच}


\twolineshloka
{तदेतद्वचनं श्रुत्वा मुनेरार्तस्य भूपतिः}
{निश्चेष्टश्च निरुत्साहो शनैस्तं देशमभ्यगात्} %॥५०॥

\twolineshloka
{तं तु दृष्ट्वा द्विजवरं ज्वलन्तमिव तेजसा}
{असावप्यभवत्तत्र सशल्य इव मूर्च्छितः} %॥५१॥

\onelineshloka
{आत्मानमात्मना कृत्वा स्थिरं राजाब्रवीदिदम्}%॥* ५२॥

\uvacha{राजोवाच}


\twolineshloka
{को भवान् द्विजशार्दूल किमर्थमिह चागतः}
{वद पापकृते मह्यं वद मे निष्कृतिं पराम्} %॥५३॥

\twolineshloka
{ब्रह्महा वर्णिभिः किन्तु श्वपचैरपि जातुचित्}
{न स्प्रष्टव्यो महाबुद्धे द्रष्टव्यो न कदाचन} %॥५४॥

\uvacha{ब्रह्मोवाच}


\onelineshloka
{तद्राजवचनं श्रुत्वा मुनिपुत्रोऽब्रवीद्वचः}%॥* ५५॥

\uvacha{मुनिपुत्र उवाच}


\twolineshloka
{उत्क्रमिष्यन्ति मे प्राणा अतो वक्ष्यामि किञ्चन}
{स्वच्छन्दवृत्तिताज्ञाने विद्धि पाकं च कर्मणाम्} %॥५६॥

\twolineshloka
{आत्मार्थं तु न शोचामि वृद्धौ तु पितरौ मम}
{तयोः शुश्रूषकः कः स्यादन्धयोरेकपुत्रयोः} %॥५७॥

\twolineshloka
{विना मया महारण्ये कथं तौ जीवयिष्यतः}
{ममाभाग्यमहो कीदृक्पितृशुश्रूषणे क्षतिः} %॥५८॥

\twolineshloka
{जाता मेऽद्य विना प्राणैर्हा विधे किं कृतं त्वया}
{तथापि गच्छ तत्र त्वं गृहीतकलशस्त्वरन्} %॥५९॥

\onelineshloka
{ताभ्यां देह्युदपानं त्वं यथा तौ न मरिष्यतः}%॥* ६०॥

\uvacha{ब्रह्मोवाच}


\twolineshloka
{इत्येवं ब्रुवतस्तस्य गताः प्राणा महावने}
{विसृज्य सशरं चापमादाय कलशं नृपः} %॥६१॥

\twolineshloka
{तत्रागात्स तु वेगेन यत्र वृद्धौ महावने}
{वृद्धौ चापि तदा रात्रौ तावन्योन्यं समूचतुः} %॥६२॥

\uvacha{वृद्धावूचतुः॒}


\twolineshloka
{उद्विग्नः कुपितो वा स्यादथवा भक्षितः कथम्}
{न प्राप्तश्चावयोर्यष्टिः किं कुर्मः का गतिर्भवेत्} %॥६३॥

\twolineshloka
{न कोपि तादृशः पुत्रो विद्यते सचराचरे}
{यः पित्रोरन्यथा वाक्यं न करोत्यपि निन्दितः} %॥६४॥

\twolineshloka
{वज्रादपि कठोरं वा जीवितं तमपश्यतोः}
{शीघ्रं न यान्ति यत्प्राणास्तदेकायत्तजीवयोः} %॥६५॥

\uvacha{ब्रह्मोवाच}


\twolineshloka
{एवं बहुविधा वाचो वृद्धयोर्वदतोर्वने}
{तदा दशरथो राजा शनैस्तं देशमभ्यगात्} %॥६६॥

\onelineshloka
{पादसञ्चारशब्देन मेनाते सुतमागतम्}%॥* ६७॥

\uvacha{वृद्धावूचतुः॒}


\twolineshloka
{कुतो वत्स चिरात्प्राप्तस्त्वं दृष्टिस्त्वं परायणम्}
{न ब्रूषे किन्तु रुष्टोऽसि वृद्धयोरन्धयोः सुतः} %॥६८॥

\uvacha{ब्रह्मोवाच}


\twolineshloka
{सशल्य इव दुःखार्तः शोचन् दुष्कृतमात्मनः}
{स भीत इव राजेन्द्रस्तावुवाचाथ नारद} %॥६९॥

\twolineshloka
{उदपानं च कुरुतां तच्छ्रुत्वा नृपभाषितम्}
{नायं वक्ता सुतोऽस्माकं को भवांस्तत्पुरा वद} %॥७०॥

\onelineshloka
{पश्चात्पिबावः पानीयं ततो राजाब्रवीच्च तौ}%॥* ७१॥

\uvacha{राजोवाच}


\onelineshloka
{तत्र तिष्ठति वां पुत्रो यत्र वारिसमाश्रयः}%॥* ७२॥

\uvacha{ब्रह्मोवाच}


\twolineshloka
{तच्छ्रुत्वोचतुरार्तौ तौ सत्यं ब्रूहि न चान्यथा}
{आचचक्षे ततो राजा सर्वमेव यथातथम्} %॥७३॥

\twolineshloka
{ततस्तु पतितौ वृद्धौ तत्रावां नय मा स्पृश}
{ब्रह्मघ्नस्पर्शनं पापं न कदाचिद्विनश्यति} %॥७४॥

\twolineshloka
{निन्ये वै श्रवणं वृद्धं सभार्यं नृपसत्तमः}
{यत्रासौ पतितः पुत्रस्तं स्पृष्ट्वा तौ विलेपतुः} %॥७५॥

\uvacha{वृद्धावूचतुः॒}


\twolineshloka
{यथा पुत्रवियोगेन मृत्युर्नौ विहितस्तथा}
{त्वं चापि पाप पुत्रस्य वियोगान्मृत्युमाप्स्यसि} %॥७६॥

\uvacha{ब्रह्मोवाच}


\twolineshloka
{एवं तु जल्पतोर्ब्रह्मन् गताः प्राणास्ततो नृपः}
{अग्निना योजयामास वृद्धौ च ऋषिपुत्रकम्} %॥७७॥

\twolineshloka
{ततो जगाम नगरं दुःखितो नृपतिर्मुने}
{वसिष्ठाय च तत्सर्वं न्यवेदयदशेषतः} %॥७८॥

\twolineshloka
{नृपाणां सूर्यवंश्यानां वसिष्ठो हि परा गतिः}
{वसिष्ठोऽपि द्विजश्रेष्ठैः सम्मन्त्र्याह च निष्कृतिम्} %॥७९॥

\uvacha{वसिष्ठ उवाच}


\twolineshloka
{गालवं वामदेवं च जाबालिमथ कश्यपम्}
{एतानन्यान् समाहूय हयमेधाय यत्नतः} %॥८०॥

\onelineshloka
{यजस्व हयमेधैश्च बहुभिर्बहुदक्षिणैः}%॥* ८१॥

\uvacha{ब्रह्मोवाच}


\twolineshloka
{अकरोद्धयमेधांश्च राजा दशरथो द्विजैः}
{एतस्मिन्नन्तरे तत्र वागुवाचाशरीरिणी} %॥८२॥

\uvacha{आकाशवाण्युवाच}

पूतं शरीरमभवद्राज्ञो दशरथस्य हि।

\twolineshloka
{व्यवहार्यश्च भविता भविष्यन्ति तथा सुताः}
{ज्येष्ठपुत्रप्रसादेन राजापापो भविष्यति} %॥८३॥

\uvacha{ब्रह्मोवाच}


\twolineshloka
{ततो बहुतिथे काले ऋष्यशृङ्गान्मुनीश्वरात्}
{देवानां कार्यसिद्ध्यर्थं सुता आसन् सुरोपमाः} %॥८४॥

\twolineshloka
{कौशल्यायां तथा रामः सुमित्रायां च लक्ष्मणः}
{शत्रुघ्नश्चापि कैकेय्यां भरतो मतिमत्तरः} %॥८५॥

\twolineshloka
{ते सर्वे मतिमन्तश्च प्रिया राज्ञो वशे स्थिताः}
{तं राजानमृषिः प्राप्य विश्वामित्रः प्रजापतिः} %॥८६॥

\twolineshloka
{रामं च लक्ष्मणं चापि अयाचत महामते}
{यज्ञसंरक्षणार्थाय ज्ञाततन्महिमा मुनिः} %॥८७॥

\onelineshloka
{चिरप्राप्तसुतो वृद्धो राजा नैवेत्यभाषत}%॥* ८८॥

\uvacha{राजोवाच}


\twolineshloka
{महता दैवयोगेन कथञ्चिद्वार्धके मुने}
{जातावानन्दसन्दोह दायकौ मम बालकौ} %॥८९॥

\onelineshloka
{सशरीरमिदं राज्यं दास्ये नैव सुताविमौ}%॥* ९०॥

\uvacha{ब्रह्मोवाच}


\onelineshloka
{वसिष्ठेन तदा प्रोक्तो राजा दशरथस्त्विति}%॥* ९१॥

\uvacha{वसिष्ठ उवाच}


\onelineshloka
{रघवः प्रार्थनाभङ्गं न राजन् क्वापि शिक्षिताः}%॥* ९२॥

\uvacha{ब्रह्मोवाच}


\onelineshloka
{रामं च लक्ष्मणं चैव कथञ्चिदवदन्नृपः}%॥* ९३॥

\uvacha{राजोवाच}


\onelineshloka
{विश्वामित्रस्य ब्रह्मर्षेः कुरुतां यज्ञरक्षणम्}%॥* ९४॥

\uvacha{ब्रह्मोवाच}


\twolineshloka
{वदन्निति सुतौ सोष्णं निश्वसन् ग्लपिताधरः}
{पुत्रौ समर्पयामास विश्वामित्रस्य शास्त्रकृत्} %॥९५॥

\twolineshloka
{तथेत्युक्त्वा दशरथं नमस्य च पुनः पुनः}
{जग्मतू रक्षणार्थाय विश्वामित्रेण तौ मुदा} %॥९६॥

\twolineshloka
{ततः प्रहृष्टः स मुनिर्मुदा प्रादात्तदोभयोः}
{माहेश्वरीं महाविद्यां धनुर्विद्यापुरःसराम्} %॥९७॥

\twolineshloka
{शास्त्रीमास्त्रीं लौकिकीं च रथविद्यां गजोद्भवाम्}
{अश्वविद्यां गदाविद्यां मन्त्राह्वानविसर्जने} %॥९८॥

\twolineshloka
{सर्वविद्यामथावाप्य उभौ तौ रामलक्ष्मणौ}
{वनौकसां हितार्थाय जघ्नतुस्ताटकां वने} %॥९९॥

\twolineshloka
{अहल्यां शापनिर्मुक्तां पादस्पर्शाच्च चक्रतुः}
{यज्ञविध्वंसनायाताञ्जघ्नतुस्तत्र राक्षसान्} %॥१००॥

\twolineshloka
{कृतविद्यौ धनुष्पाणी चक्रतुर्यज्ञरक्षणम्}
{ततो महामखे वृत्ते विश्वामित्रो मुनीश्वरः} %॥१०१॥

\twolineshloka
{पुत्राभ्यां सहितो राज्ञो जनकं द्रष्टुमभ्यगात्}
{चित्रामदर्शयत्तत्र राजमध्ये नृपात्मजः} %॥१०२॥

\twolineshloka
{रामः सौमित्रिसहितो धनुर्विद्यां गुरोर्मताम्}
{तत्प्रीतो जनकः प्रादात्सीतां लक्ष्मीमयोनिजाम्} %॥१०३॥

\twolineshloka
{तथैव लक्ष्मणस्यापि भरतस्यानुजस्य च}
{शत्रुघ्नभरतादीनां वसिष्ठादिमते स्थितः} %॥१०४॥

\twolineshloka
{राजा दशरथः श्रीमान् विवाहमकरोन्मुने}
{ततो बहुतिथे काले राज्यं तस्य प्रयच्छति} %॥१०५॥

\twolineshloka
{नृपतौ सर्वलोकानामनुमत्या गुरोरपि}
{मन्थरात्मकदुर्दैव प्रेरिता मत्सराकुला} %॥१०६॥

\twolineshloka
{कैकेयी विघ्नमातस्थे वनप्रव्राजनं तथा}
{भरतस्य च तद्राज्यं राजा नैव च दत्तवान्} %॥१०७॥

\twolineshloka
{पितरं सत्यवाक्यं तं कुर्वन् रामो महावनम्}
{विवेश सीतया सार्धं तथा सौमित्रिणा सह} %॥१०८॥

\twolineshloka
{सतां च मानसं शुद्धं स विवेश स्वकैर्गुणैः}
{तस्मिन् विनिर्गते रामे वनवासाय दीक्षिते} %॥१०९॥

\twolineshloka
{समं लक्ष्मणसीताभ्यां राज्यतृष्णाविवर्जिते}
{तं रामं चापि सौमित्रिं सीतां च गुणशालिनीम्} %॥११०॥

\twolineshloka
{दुःखेन महताविष्टो ब्रह्मशापं च संस्मरन्}
{तदा दशरथो राजा प्राणांस्तत्याज दुःखितः} %॥१११॥

\twolineshloka
{कृतकर्मविपाकेन राजा नीतो यमानुगैः}
{तस्मै राज्ञे महाप्राज्ञ यावत्स्थावरजङ्गमे} %॥११२॥

\twolineshloka
{यमसद्मन्यनेकानि तामिस्रादीनि नारद}
{नरकाण्यथ घोराणि भीषणानि बहूनि च} %॥११३॥

\twolineshloka
{तत्र क्षिप्तस्तदा राजा नरकेषु पृथक्पृथक्}
{पच्यते छिद्यते राजा पिष्यते चूर्ण्यते तथा} %॥११४॥

\twolineshloka
{शोष्यते दश्यते भूयो दह्यते च निमज्ज्यते}
{एवमादिषु घोरेषु नरकेषु स पच्यते} %॥११५॥

\twolineshloka
{रामोऽपि गच्छन्नध्वानं चित्रकूटमथागमत्}
{तत्रैव त्रीणि वर्षाणि व्यतीतानि महामते} %॥११६॥

\twolineshloka
{पुनः स दक्षिणामाशामाक्रामद्दण्डकं वनम्}
{विख्यातं त्रिषु लोकेषु देशानां तद्धि पुण्यदम्} %॥११७॥

\twolineshloka
{प्राविशत्तन्महारण्यं भीषणं दैत्यसेवितम्}
{तद्भयादृषिभिस्त्यक्तं हत्वा दैत्यांस्तु राक्षसान्} %॥११८॥

\twolineshloka
{विचरन् दण्डकारण्ये ऋषिसेव्यमथाकरोत्}
{तत्रेदं वृत्तमाख्यास्ये शृणु नारद यत्नतः} %॥११९॥

\twolineshloka
{तावच्छनैस्त्वगाद्रामो यावद्योजनपञ्चकम्}
{गौतमीं समनुप्राप्तो राजापि नरके स्थितः} %॥१२०॥

\twolineshloka
{यमः स्वकिङ्करानाह रामो दशरथात्मजः}
{गौतमीमभितो याति पितरं तस्य धीमतः} %॥१२१॥

\twolineshloka
{आकर्षन्त्वथ राजानं नरकान्नात्र संशयः}
{उत्तीर्य गौतमीं याति यावद्योजनपञ्चकम्} %॥१२२॥

\twolineshloka
{रामस्तावत्तस्य पिता नरके नैव पच्यताम्}
{यदेतन्मद्वचः पुण्यं न कुर्युर्यदि दूतकाः} %॥१२३॥

\twolineshloka
{ततश्च नरके घोरे यूयं सर्वे निमज्जथ}
{या काप्युक्ता परा शक्तिः शिवस्य समवायिनी} %॥१२४॥

\twolineshloka
{तामेव गौतमीं सन्तो वदन्त्यम्भःस्वरूपिणीम्}
{हरिब्रह्ममहेशानां मान्या वन्द्या च सैव यत्} %॥१२५॥

\twolineshloka
{निस्तीर्यते न केनापि तदतिक्रमजं त्वघम्}
{पापिनोऽप्यात्मजः कश्चिद्यश्च गङ्गामनुस्मरेत्} %॥१२६॥

\twolineshloka
{सोऽनेकदुर्गनिरयान्निर्गतो मुक्ततां व्रजेत्}
{किं पुनस्तादृशः पुत्रो गौतमीनिकटे स्थितः} %॥१२७॥

\twolineshloka
{यस्यासौ नरके पक्तुं न कैरपि हि शक्यते}
{दक्षिणाशापतेर्वाक्यं निशम्य यमकिङ्कराः} %॥१२८॥

\twolineshloka
{नरके पच्यमानं तमयोध्याधिपतिं नृपम्}
{उत्तार्य घोरनरकाद्वचनं चेदमब्रुवन्} %॥१२९॥

\uvacha{यमकिङ्करा ऊचुः॒}


\twolineshloka
{धन्योऽसि नृपशार्दूल यस्य पुत्रः स तादृशः}
{इह चामुत्र विश्रान्तिः सुपुत्रः केन लभ्यते} %॥१३०॥

\uvacha{ब्रह्मोवाच}


\onelineshloka
{स विश्रान्तः शनै राजा किङ्करान् वाक्यमब्रवीत्}%॥* १३१॥

\uvacha{राजोवाच}


\twolineshloka
{नरकेष्वथ घोरेषु पच्यमानः पुनः पुनः}
{कथं त्वाकर्षितः शीघ्रं तन्मे वक्तुमिहार्हथ} %॥१३२॥

\uvacha{ब्रह्मोवाच}


\onelineshloka
{तत्र कश्चिच्छान्तमना राजानमिदमब्रवीत्}%॥* १३३॥

\uvacha{यमदूत उवाच}


\twolineshloka
{वेदशास्त्रपुराणादावेतद्गोप्यं प्रयत्नतः}
{प्रकाश्यते तदपि ते सामर्थ्यं पुत्रतीर्थयोः} %॥१३४॥

\twolineshloka
{रामस्तव सुतः श्रीमान् गौतमीतीरमागतः}
{तस्मात्त्वं नरकाद्घोरादाकृष्टोऽसि नरोत्तम} %॥१३५॥
यदि त्वां तत्र गौतम्यां स्मरेद्रामः सलक्ष्मणः।

\twolineshloka
{स्नानं कृत्वाथ पिण्डादि ते दद्यात्स नृपोत्तम}
{ततस्त्वं सर्वपापेभ्यो मुक्तो यासि त्रिविष्टपम्} %॥१३६॥

\uvacha{राजोवाच}


\twolineshloka
{तत्र गत्वा भवद्वाक्यमाख्यास्ये स्वसुतौ प्रति}
{भवन्त एव शरणमनुज्ञां दातुमर्हथ} %॥१३७॥

\uvacha{ब्रह्मोवाच}


\twolineshloka
{तद्राजवचनं श्रुत्वा कृपया यमकिङ्कराः}
{आज्ञां च प्रददुस्तस्मै राजा प्रागात्सुतौ प्रति} %॥१३८॥

\twolineshloka
{भीषणं यातनादेहमापन्नो निःश्वसन्मुहुः}
{निरीक्ष्य स्वं लज्जमानः कृतं कर्म च संस्मरन्} %॥१३९॥

\twolineshloka
{स्वेच्छया विहरन् गङ्गामाससाद च राघवः}
{गौतम्यास्तटमाश्रित्य रामो लक्ष्मण एव च} %॥१४०॥

\twolineshloka
{सीतया सह वैदेह्या सस्नौ चैव यथाविधि}
{नैव तत्राभवद्भोज्यं भक्ष्यं वा गौतमीतटे} %॥१४१॥

\twolineshloka
{तद्दिने तत्र वसतां गौतमीतीरवासिनाम्}
{तद्दृष्ट्वा दुःखितो भ्राता लक्ष्मणो राममब्रवीत्} %॥१४२॥

\uvacha{लक्ष्मण उवाच}


\twolineshloka
{पुत्रौ दशरथस्यावां तवापि बलमीदृशम्}
{नास्ति भोज्यमथास्माकं गङ्गातीरनिवासिनाम्} %॥१४३॥

\uvacha{राम उवाच}


\twolineshloka
{भ्रातर्यद्विहितं कर्म नैव तच्चान्यथा भवेत्}
{पृथिव्यामन्नपूर्णायां वयमन्नाभिलाषिणः} %॥१४४॥

\twolineshloka
{सौमित्रे नूनमस्माभिर्न ब्राह्मणमुखे हुतम्}
{अवज्ञया महीदेवांस्तर्पयन्त्यर्चयन्ति न} %॥१४५॥
ते ये लक्ष्मण जायन्ते सर्वदैव बुभुक्षिताः।

\twolineshloka
{स्नात्वा देवानथाभ्यर्च्य होतव्यश्च हुताशनः}
{ततः स्वसमये देवो विधास्यत्यशनं तु नौ} %॥१४६॥

\uvacha{ब्रह्मोवाच}


\twolineshloka
{भ्रात्रोः सञ्जल्पतोरेवं पश्यतोः कर्मणो गतिम्}
{शनैर्दशरथो राजा तं देशमुपजग्मिवान्} %॥१४७॥

\twolineshloka
{तं दृष्ट्वा लक्ष्मणः शीघ्रं तिष्ठ तिष्ठेति चाब्रवीत्}
{धनुराकृष्य कोपेन रक्षस्त्वं दानवोऽथवा} %॥१४८॥

\twolineshloka
{आसन्नं च पुनर्दृष्ट्वा याहि याह्यत्र पुण्यभाक्}
{रामो दाशरथी राजा धर्मभाक्पश्य वर्तते} %॥१४९॥

\twolineshloka
{गुरुभक्तः सत्यसन्धो देवब्राह्मणसेवकः}
{त्रैलोक्यरक्षादक्षोऽसौ वर्तते यत्र राघवः} %॥१५०॥

\twolineshloka
{न तत्र त्वादृशामस्ति प्रवेशः पापकर्मणाम्}
{यदि प्रविशसे पाप ततो वधमवाप्स्यसि} %॥१५१॥
तत्पुत्रवचनं श्रुत्वा शनैराहूय वाचया।

\twolineshloka
{उवाचाधोमुखो भूत्वा स्नुषां पुत्रौ कृताञ्जलिः}
{मुहुरन्तर्विनिध्यायन् गतिं दुष्कृतकर्मणः} %॥१५२॥

\uvacha{राजोवाच}

अहं दशरथो राजा पुत्रौ मे शृणुतं वचः।

\twolineshloka
{तिसृभिर्ब्रह्महत्याभिर्वृतोऽहं दुःखमागतः}
{छिन्नं पश्यत मे देहं नरकेषु च पातितम्} %॥१५३॥

\uvacha{ब्रह्मोवाच}


\twolineshloka
{ततः कृताञ्जली रामः सीतया लक्ष्मणेन च}
{भूमौ प्रणेमुस्ते सर्वे वचनं चैतदब्रुवन्} %॥१५४॥

\uvacha{सीतारामलक्ष्मणा ऊचुः॒}


\onelineshloka
{कस्येदं कर्मणस्तात फलं नृपतिसत्तम}%॥* १५५॥

\uvacha{ब्रह्मोवाच}


\onelineshloka
{स च प्राह यथावृत्तं ब्रह्महत्यात्रयं तथा}%॥* १५६॥

\uvacha{राजोवाच}


\onelineshloka
{निष्कृतिर्ब्रह्महन्तॄणां पुत्रौ क्वापि न विद्यते}%॥* १५७॥

\uvacha{ब्रह्मोवाच}


\twolineshloka
{ततो दुःखेन महता वृताः सर्वे भुवं गताः}
{राजानं वनवासं च मातरं पितरं तथा} %॥१५८॥
दुःखागमं कर्मगतिं नरके पातनं तथा।

\twolineshloka
{एवमाद्यथ संस्मृत्य मुमोह नृपतेः सुतः}
{विसंज्ञं नृपतिं दृष्ट्वा सीता वाक्यमथाब्रवीत्} %॥१५९॥

\uvacha{सीतोवाच}


\twolineshloka
{न शोचन्ति महात्मानस्त्वादृशा व्यसनागमे}
{चिन्तयन्ति प्रतीकारं दैव्यमप्यथ मानुषम्} %॥१६०॥

\twolineshloka
{शोचद्भिर्युगसाहस्रं विपत्तिर्नैव तीर्यते}
{व्यामोहमाप्नुवन्तीह न कदाचिद्विचक्षणाः} %॥१६१॥

\twolineshloka
{किमनेनात्र दुःखेन निष्फलेन जनेश्वर}
{देहि हत्यां प्रथमतो या जाता ह्यतिभीषणा} %॥१६२॥

\twolineshloka
{पितृभक्तः पुण्यशीलो वेदवेदाङ्गपारगः}
{अनागा यो हतो विप्रस्तत्पापस्यात्र निष्कृतिम्} %॥१६३॥

\twolineshloka
{आचरामि यथाशास्त्रं मा शोकं कुरुतं युवाम्}
{द्वितीयां लक्ष्मणो हत्यां गृह्णातु त्वपरां भवान्} %॥१६४॥

\uvacha{ब्रह्मोवाच}


\twolineshloka
{एतद्धर्मयुतं वाक्यं सीतया भाषितं दृढम्}
{तथेति चाहतुरुभौ ततो दशरथोऽब्रवीत्} %॥१६५॥

\uvacha{दशरथ उवाच}


\twolineshloka
{त्वं हि ब्रह्मविदः कन्या जनकस्य त्वयोनिजा}
{भार्या रामस्य किं चित्रं यद्युक्तमनुभाषसे} %॥१६६॥

\twolineshloka
{न कोपि भवतां किन्तु श्रमः स्वल्पोऽपि विद्यते}
{गौतम्यां स्नानदानेन पिण्डनिर्वपणेन च} %॥१६७॥

\twolineshloka
{तिसृभिर्ब्रह्महत्याभिर्मुक्तो यामि त्रिविष्टपम्}
{त्वया जनकसम्भूते स्वकुलोचितमीरितम्} %॥१६८॥

\twolineshloka
{प्रापयन्ति परं पारं भवाब्धेः कुलयोषितः}
{गोदावर्याः प्रसादेन किं नामास्त्यत्र दुर्लभम्} %॥१६९॥

\uvacha{ब्रह्मोवाच}


\twolineshloka
{तथेति क्रियमाणे तु पिण्डदानाय शत्रुहा}
{नैवापश्यद्भक्ष्यभोज्यं ततो लक्ष्मणमब्रवीत्} %॥१७०॥

\twolineshloka
{लक्ष्मणः प्राह विनयादिङ्गुद्याश्च फलानि च}
{सन्ति तेषां च पिण्याकमानीतं तत्क्षणादिव} %॥१७१॥

\twolineshloka
{पिण्याकेनाथ गङ्गायां पिण्डं दातुं तथा पितुः}
{मनः कुर्वंस्ततो रामो मन्दोऽभूद्दुःखितस्तदा} %॥१७२॥

\twolineshloka
{दैवी वागभवत्तत्र दुःखं त्यज नृपात्मज}
{राज्यभ्रष्टो वनं प्राप्तः किं वै निष्किञ्चनो भवान्} %॥१७३॥

\twolineshloka
{अशठो धर्मनिरतो न शोचितुमिहार्हसि}
{वित्तशाठ्येन यो धर्मं करोति स तु पातकी} %॥१७४॥

\twolineshloka
{श्रूयते सर्वशास्त्रेषु यद्राम शृणु यत्नतः}
{यदन्नः पुरुषो राजंस्तदन्नास्तस्य देवताः} %॥१७५॥

\twolineshloka
{पिण्डे निपतिते भूमौ नापश्यत्पितरं तदा}
{शवं च पतितं यत्र शवतीर्थमनुत्तमम्} %॥१७६॥

\twolineshloka
{महापातकसङ्घात विघातकृदनुस्मृतिः}
{तत्रागच्छंल्लोकपाला रुद्रादित्यास्तथाश्विनौ} %॥१७७॥

\twolineshloka
{स्वं स्वं विमानमारूढास्तेषां मध्येऽतिदीप्तिमान्}
{विमानवरमारूढः स्तूयमानश्च किन्नरैः} %॥१७८॥

\twolineshloka
{आदित्यसदृशाकारस्तेषां मध्ये बभौ पिता}
{तमदृष्ट्वा स्वपितरं देवान् दृष्ट्वा विमानिनः} %॥१७९॥

\twolineshloka
{कृताञ्जलिपुटो रामः पिता मे क्वेत्यभाषत}
{इति दिव्याभवद्वाणी रामं सम्बोध्य सीतया} %॥१८०॥

\twolineshloka
{तिसृभिर्ब्रह्महत्याभिर्मुक्तो दशरथो नृपः}
{वृतं पश्य सुरैस्तात देवा अप्यूचिरे च तम्} %॥१८१॥

\uvacha{देवा ऊचुः॒}


\twolineshloka
{धन्योऽसि कृतकृत्योऽसि राम स्वर्गं गतः पिता}
{नानानिरयसङ्घातात्पूर्वजानुद्धरेत्तु यः} %॥१८२॥

\twolineshloka
{स धन्योऽलङ्कृतं तेन कृतिना भुवनत्रयम्}
{एनं पश्य महाबाहो मुक्तपापं रविप्रभम्} %॥१८३॥

\twolineshloka
{सर्वसम्पत्तियुक्तोऽपि पापी दग्धद्रुमोपमः}
{निष्किञ्चनोऽपि सुकृती दृश्यते चन्द्रमौलिवत्} %॥१८४॥

\uvacha{ब्रह्मोवाच}


\onelineshloka
{दृष्ट्वाब्रवीत्सुतं राजा आशीर्भिरभिनन्द्य च}%॥* १८५॥

\uvacha{राजोवाच}


\twolineshloka
{कृतकृत्योऽसि भद्रं ते तारितोऽहं त्वयानघ}
{धन्यः स पुत्रो लोकेऽस्मिन् पितॄणां यस्तु तारकः} %॥१८६॥

\uvacha{ब्रह्मोवाच}

\onelineshloka*
{ततः सुरगणाः प्रोचुर्देवानां कार्यसिद्धये}

\twolineshloka
{रामं च पुरुषश्रेष्ठं गच्छ तात यथासुखम्}
{ततस्तद्वचनं श्रुत्वा रामस्तानब्रवीत्सुरान्} %॥१८७॥

\uvacha{राम उवाच}


\onelineshloka
{गुरौ पितरि मे देवाः किं कृत्यमवशिष्यते}%॥* १८८॥

\uvacha{देवा ऊचुः॒}


\twolineshloka
{नदी न गङ्गया तुल्या न त्वया सदृशः सुतः}
{न शिवेन समो देवो न तारेण समो मनुः} %॥१८९॥
त्वया राम गुरूणां च कार्यं सर्वमनुष्ठितम्।

\twolineshloka
{तारिताः पितरो राम त्वया पुत्रेण मानद}
{गच्छन्तु सर्वे स्वस्थानं त्वं च गच्छ यथासुखम्} %॥१९०॥

\uvacha{ब्रह्मोवाच}


\twolineshloka
{तद्देववचनाद्धृष्टः सीतया लक्ष्मणाग्रजः}
{तद्दृष्ट्वा गङ्गामाहात्म्यं विस्मितो वाक्यमब्रवीत्} %॥१९१॥

\uvacha{राम उवाच}


\twolineshloka
{अहो गङ्गाप्रभावोऽयं त्रैलोक्ये नोपमीयते}
{वयं धन्या यतो गङ्गा दृष्टास्माभिस्त्रिपावनी} %॥१९२॥

\uvacha{ब्रह्मोवाच}


\twolineshloka
{हर्षेण महता युक्तो देवं स्थाप्य महेश्वरम्}
{तं षोडशभिरीशानमुपचारैः प्रयत्नतः} %॥१९३॥

\twolineshloka
{सम्पूज्यावरणैर्युक्तं षट्त्रिंशत्कलमीश्वरम्}
{कृताञ्जलिपुटो भूत्वा रामस्तुष्टाव शङ्करम्} %॥१९४॥

\uvacha{राम उवाच}


\fourlineindentedshloka
{नमामि शम्भुं पुरुषं पुराणं}
{नमामि सर्वज्ञमपारभावम्} 
{नमामि रुद्रं प्रभुमक्षयं तं} 
{नमामि शर्वं शिरसा नमामि}% १९५

\fourlineindentedshloka
{नमामि देवं परमव्ययं तम्} 
{उमापतिं लोकगुरुं नमामि}
{नमामि दारिद्र्यविदारणं तं}
{नमामि रोगापहरं नमामि}% १९६

\fourlineindentedshloka
{नमामि कल्याणमचिन्त्यरूपं}
{नमामि विश्वोद्भवबीजरूपम्}
{नमामि विश्वस्थितिकारणं तं}
{नमामि संहारकरं नमामि}% १९७

\fourlineindentedshloka
{नमामि गौरीप्रियमव्ययं तं}
{नमामि नित्यं क्षरमक्षरं तम्}
{नमामि चिद्रूपममेयभावं}
{त्रिलोचनं तं शिरसा नमामि}% १९८

\fourlineindentedshloka
{नमामि कारुण्यकरं भवस्य}
{भयङ्करं वापि सदा नमामि}
{नमामि दातारमभीप्सितानां}
{नमामि सोमेशमुमेशमादौ}% १९९

\fourlineindentedshloka
{नमामि वेदत्रयलोचनं तं}
{नमामि मूर्तित्रयवर्जितं तम्}
{नमामि पुण्यं सदसद्व्यतीतं}
{नमामि तं पापहरं नमामि}% २००

\fourlineindentedshloka
{नमामि विश्वस्य हिते रतं तं}
{नमामि रूपाणि बहूनि धत्ते}
{यो विश्वगोप्ता सदसत्प्रणेता}
{नमामि तं विश्वपतिं नमामि}% २०१

\fourlineindentedshloka
{यज्ञेश्वरं सम्प्रति हव्यकव्यं}
{तथा गतिं लोकसदाशिवो यः}
{आराधितो यश्च ददाति सर्वं}
{नमामि दानप्रियमिष्टदेवम्}% २०२

\fourlineindentedshloka
{नमामि सोमेश्वरमस्वतन्त्रम्}
{उमापतिं तं विजयं नमामि}
{नमामि विघ्नेश्वरनन्दिनाथं}
{पुत्रप्रियं तं शिरसा नमामि}% २०३

\fourlineindentedshloka
{नमामि देवं भवदुःखशोक}
{विनाशनं चन्द्रधरं नमामि}
{नमामि गङ्गाधरमीशमीड्यम्}
{उमाधवं देववरं नमामि}% २०४

\fourlineindentedshloka
{नमाम्यजादीशपुरन्दरादि}
{सुरासुरैरर्चितपादपद्मम्}
{नमामि देवीमुखवादनानाम्}
{ईक्षार्थमक्षित्रितयं य ऐच्छत्}% २०५

\fourlineindentedshloka
{पञ्चामृतैर्गन्धसुधूपदीपैर्}
{विचित्रपुष्पैर्विविधैश्च मन्त्रैः}
{अन्नप्रकारैः सकलोपचारैः}
{सम्पूजितं सोममहं नमामि}% २०६

\uvacha{ब्रह्मोवाच}


\twolineshloka
{ततः स भगवानाह रामं शम्भुः सलक्ष्मणम्}
{वरान् वृणीष्व भद्रं ते रामः प्राह वृषध्वजम्} %॥२०७॥

\uvacha{राम उवाच}


\twolineshloka
{स्तोत्रेणानेन ये भक्त्या तोष्यन्ति त्वां सुरोत्तम}
{तेषां सर्वाणि कार्याणि सिद्धिं यान्तु महेश्वर} %॥२०८॥

\twolineshloka
{येषां च पितरः शम्भो पतिता नरकार्णवे}
{तेषां पिण्डादिदानेन पूता यान्तु त्रिविष्टपम्} %॥२०९॥

\twolineshloka
{जन्मप्रभृति पापानि मनोवाक्कायिकं त्वघम्}
{अत्र तु स्नानमात्रेण तत्सद्यो नाशमाप्नुयात्} %॥२१०॥

\twolineshloka
{अत्र ये भक्तितः शम्भो ददत्यर्थिभ्य अण्वपि}
{सर्वं तदक्षयं शम्भो दातॄणां फलकृद्भवेत्} %॥२११॥

\uvacha{ब्रह्मोवाच}


\twolineshloka
{एवमस्त्विति तं रामं शङ्करो हृषितोऽब्रवीत्}
{गते तस्मिन् सुरश्रेष्ठे रामोऽप्यनुचरैः सह} %॥२१२॥

\twolineshloka
{गौतमी यत्र चोत्पन्ना शनैस्तं देशमभ्यगात्}
{ततः प्रभृति तत्तीर्थं रामतीर्थमुदाहृतम्} %॥२१३॥

\twolineshloka
{दयालोरपतत्तत्र लक्ष्मणस्य कराच्छरः}
{तद्बाणतीर्थमभवत्सर्वापद्विनिवारणम्} %॥२१४॥

\twolineshloka
{यत्र सौमित्रिणा स्नानं शङ्करस्यार्चनं कृतम्}
{तत्तीर्थं लक्ष्मणं जातं तथा सीतासमुद्भवम्} %॥२१५॥

\twolineshloka
{नानाविधाशेषपाप सङ्घनिर्मूलनक्षमम्}
{यदङ्घ्रिसङ्गादभवद्गङ्गा त्रैलोक्यपावनी} %॥२१६॥

\twolineshloka
{स यत्र स्नानमकरोत्तद्वैशिष्ट्यं किमुच्यते}
{तद्रामतीर्थसदृशं तीर्थं क्वापि न विद्यते} %॥२१७॥

॥इति श्रीमहापुराणे आदिब्राह्मे तीर्थमाहात्म्ये रामतीर्थवर्णनं नाम त्रयोविंशत्यधिकशततमोऽध्यायः॥१२३॥
