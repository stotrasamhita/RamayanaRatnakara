\sect{सहस्रकुण्डाख्यतीर्थवर्णनम्}

\src{ब्रह्म-पुराणम्}{गौतमीमाहात्म्यम्}{अध्यायः १५४}{}
% \tags{concise, complete}
\notes{This chapter describes the significance of the Sahasrakunda Tirtha, where Lord Rama performed rituals and established a sacred site after defeating Ravana. It also narrates the events leading to the establishment of this Tirtha, including Rama's return to Ayodhya with Sita.}
\textlink{https://sa.wikisource.org/wiki/ब्रह्मपुराणम्/अध्यायः_१५४}
\translink{}

\storymeta

\uvacha{ब्रह्मोवाच}

\twolineshloka
{सहस्रकुण्डमाख्यातं तीर्थं वेदविदो विदुः}
{यस्य स्मरणमात्रेण सुखी सम्पद्यते नरः} %॥१॥

\twolineshloka
{पुरा दाशरथी रामः सेतुं बद्‌ध्वा महार्णवे}
{लङ्कां दग्ध्वा रिपून्हत्वा रावणादीन्रणे शरैः} %॥२॥

\twolineshloka
{वैदेहीं च समासाद्य रामो वचनमब्रवीत्}
{पश्यत्सु लोकपालेषु तस्याऽऽचार्ये पुरः स्थिते} %॥३॥

\twolineshloka
{अग्नौ शुद्धिगतां सीतां रामो लक्ष्मणसन्निधौ}
{एहि वैदेहि शुद्धऽसि अङ्कमारोढुमर्हसि} %॥४॥

\twolineshloka
{नेत्युवाच तदा श्रीमानङ्गदो हनुमांस्तथा}
{अयोध्यायां तु वैदेहि सार्धं यामः सुहृज्जनैः} %॥५॥

\twolineshloka
{तत्र शुद्धिमवाप्याथ पुनर्भातृषु मातृषु}
{लौकिकेष्वपि पश्यत्सु ततः शुद्धा नृपात्मजा} %॥६॥

\twolineshloka
{अयोध्यायां सुपुण्येऽह्नि अङ्कमारोढुमर्हंसि}
{अस्याश्चरित्रविषये सन्देहः कस्य जायते} %॥७॥

\twolineshloka
{लोकापवादस्तदपि निरस्यः स्वजनेषु हि}
{तयोर्वाक्यमनादृत्य लक्ष्मणः सविभीषणः} %॥८॥

\twolineshloka
{रामश्च जाम्बवांश्चैव तामाह्वयन्नृपात्मजाम्}
{स्वस्तीत्युक्ता देवताभी राज्ञोङ्कं चाऽऽरुरोह सा} %॥९॥

\twolineshloka
{मुदतिस्ते ययुः शीघ्रं पुष्पकेण विराजता}
{अयोध्यां नगरीं प्राप्य तथा राज्यं स्वकं तु यत्} %॥१०॥

\twolineshloka
{मुदितास्तेऽभवन्सर्वे सदा रामानुवर्तिनः}
{ततः कतिपयाहेषु अनार्येभ्यो विरूपिकाम्} %॥११॥

\twolineshloka
{वाचं श्रुत्वा स तत्याज गुर्विणीं तामयोनिजाम्}
{मिथ्यापवादमपि हि न सहन्ते कुलोन्नताः} %॥१२॥

\twolineshloka
{वाल्मीकेर्मुनिमुख्यस्य आश्रमस्य समीपतः}
{तत्याज लक्ष्मणः सीतामदुष्टां रुदतीं रुदन्} %॥१३॥

\twolineshloka
{नोल्लङ्घ्याऽऽज्ञा गुरूणामित्यसौ तदकरोद्भिया}
{ततः कतिपयाहेतु व्यतीतेषु नृपात्मजः} %॥१४॥

\twolineshloka
{रामः सौमित्रिणा सार्धं हयमेधाय दीक्षितः}
{तत्रैवाऽऽजग्मतुरुभौ रामपुत्रौ यशस्विनौ} %॥१५॥

\twolineshloka
{लवः कुशश्च विख्यातौ नारदाविव गायकौ}
{रामायणं समग्रं तद्‌गन्धर्वाविव सुस्वरौ} %॥१६॥

\twolineshloka
{रामाय चरितं सर्वं गायमानौ समीयतुः}
{यज्ञवाटं राजसुतौ हेतुभिर्लक्षितौ तदा} %॥१७॥

\twolineshloka
{रामपुत्रावुभौ शूरौ वैदेह्यास्तनयाविति}
{तावानीय ततः पुत्रावभिषच्य यथाक्रमम्} %॥१८॥

\twolineshloka
{अङ्कारूढौ ततः कृत्वा सस्वजे तौ पुनः पुनः}
{संसारदुःखिन्नानामगतीनां शरीरिणाम्} %॥१९॥

\twolineshloka
{पुत्रालिङ्गनमेवात्र परं विश्रान्तिकारणम्}
{मुहुरालिङ्ग्य तौ पुत्रौ मुहुः स्वजति चुम्बति} %॥२०॥

\twolineshloka
{किमप्यन्तर्ध्याति च निःश्वसत्यपि वै मुहुः}
{एतस्मिन्नन्तरे प्राप्ता राक्षसा लङ्कवासिनः} %॥२१॥

\twolineshloka
{सुग्रीवो हनुमांश्चैव अङ्गदो जाम्बवांस्तथा}
{अन्ये च वानराः सर्वे विभीषणपुरः सराः} %॥२२॥

\twolineshloka
{ते चाऽऽगत्य नृपं प्राप्ताः सिंहासनमुपस्थितम्}
{सीतामदृष्ट्वा हनुमानङ्गदः कनकाङ्गदः} %॥२३॥

\twolineshloka
{क्व गताऽयोनिजा माता एको रामोऽत्र दृश्यते}
{रामेण सा परित्यक्ता इत्यूचुर्द्वारपालकाः} %॥२४॥

\twolineshloka
{पश्यत्सु लोकपालेषु आर्ये तत्र प्रवादिनि}
{अग्नौ शुद्धिगतां(ता)सीतां(ता)किन्तु राजा निरङ्कुशः} %॥२५॥

\twolineshloka
{उत्पन्नैर्लौकिकैर्वाक्यै रामस्त्यजति तां प्रियाम्}
{मरिष्याव इति ह्युक्त्वा गौतमीं पुनरीयतुः} %॥२६॥

\twolineshloka
{रामस्तौ पृष्ठतोऽभ्येत्य(?)अयोध्यावासिभिः सह}
{आगत्य गौतमीं तत्राकुर्वंस्त परमं तपः} %॥२७॥

\twolineshloka
{स्मारं स्मारं निश्वसन्तस्तां सीतां लोकमातरम्}
{संसारास्थाविरहिता गौतमीसेवनोत्सुकाः} %॥२८॥

\twolineshloka
{लोकत्रयपतिः साक्षाद्रामोऽनुजसमन्वितः}
{प्राप्तं स्नात्वा च गौतम्यां शिवाराधनतत्परः} %॥२९॥

\twolineshloka
{परितापं हजौ सर्वं सहस्रपरिवारितः}
{यत्र चाऽऽसीत्स वृत्तान्तः सहस्रकुण्डमुच्यते} %॥३०॥

\twolineshloka
{दशापराणि तीर्थानि तत्र सर्वार्थदानि च}
{तत्र स्नानं च दानं च सहस्रफलदायकम्} %॥३१॥

\twolineshloka
{यत्र श्रीगौतमीतीरे वसिष्ठादिमुनीश्वरैः}
{सर्वापत्तारकं होममकारयदघान्तकम्} %॥३२॥

\twolineshloka
{सहस्रसङ्ख्यायुक्तेषु कुण्डेषु वसुधारया}
{सर्वानपेक्षितान्कामानवापासौ महातपाः} %॥३३॥

\twolineshloka
{गौतम्याः सरिदम्बायाः प्रसादाद्राक्षसान्तकः}
{सहस्रकुण्डाभिधं तदभूत्तीर्थं महाफलम्} %॥३४॥

॥इति श्रीमहापुराणे आदिब्राह्मे तीर्थमाहात्म्ये सहस्रकुण्डादिदशतीर्थवर्णनं नाम चतुष्पञ्चाशदधिकशततमोऽध्यायः॥१५४॥

॥गौतमीमाहात्म्ये पञ्चाशीतितमोऽध्यायः॥८५॥