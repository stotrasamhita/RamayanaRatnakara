\sect{रामाश्वमेधप्रकरणम्}
\dnsub{प्रथमोऽध्यायः}\resetShloka

\src{पद्म-पुराणम्}{पातालखण्डः}{अध्यायाः १--६८}{}
% \tags{concise, complete}
\notes{These 68 Chapters present a very detailed account of Rama's life after His return to Ayodhya. In over 4300 shlokas, a variety of events of discussed, as Rāma returns to Ayodhyā, reunites with Bharata, and is consecrated king. Sage Agastya visits and narrates Rāvaṇa's story, advising Rāma to perform a horse sacrifice. Śatrughna is appointed to guard the sacrificial horse, leading to a campaign across various regions, marked by battles, pilgrimages, and moral episodes. Numerous kings are encountered and defeated, including Subāhu, Damana, and Vīramaṇi. Kuśa and Lava bind the horse, leading to confrontations with Rāma's army. Eventually, the sacrifice is completed with the singing of the Rāmāyaṇa by the twins.}
\textlink{https://sa.wikisource.org/wiki/पद्मपुराणम्/खण्डः_५_(पातालखण्डः)/अध्यायः_००१}
\translink{https://www.wisdomlib.org/hinduism/book/the-padma-purana/d/doc365311.html}

\storymeta



\twolineshloka
{नारायणं नमस्कृत्य नरं चैव नरोत्तमम्}
{देवींसरस्वतीं व्यासं ततो जयमुदीरयेत्}% १

\uvacha{ऋषय ऊचुः}

\twolineshloka
{श्रुतं सर्वं महाभाग स्वर्गखण्डं मनोहरम्}
{त्वत्तोऽधुना वदायुष्मञ्छ्रीरामचरितं हि नः}% २

\uvacha{सूत उवाच}

\twolineshloka
{अथैकदा धराधारं पृष्टवान्भुजगेश्वरम्}
{वात्स्यायनो मुनिवरः कथामेतां सुनिर्मलाम्}% ३

\uvacha{श्रीवात्स्यायन उवाच}

\twolineshloka
{शेषाशेष कथास्त्वत्तो जगत्सर्गलयादिकाः}
{भूगोलश्च खगोलश्च ज्योतिश्चक्रविनिर्णयः}% ४

\twolineshloka
{महत्तत्त्वादिसृष्टीनां पृथक्तत्त्वविनिर्णयः}
{नानाराजचरित्राणि कथितानि त्वयानघ}% ५

\twolineshloka
{सूर्यवंशभवानां च राज्ञां चारित्रमद्भुतम्}
{तत्रानेकमहापापहरा रामकृता कथा}% ६

\twolineshloka
{तस्य वीरस्य रामस्य हयमेधकथा श्रुता}
{सङ्क्षेपतो मया त्वत्तस्तामिच्छामि सविस्तराम्}% ७

\twolineshloka
{या श्रुता संस्मृता चोक्ता महापातकहारिणी}
{चिन्तितार्थप्रदात्री च भक्तचित्तप्रतोषदा}% ८

\uvacha{शेष उवाच}

\twolineshloka
{धन्योसि द्विजवर्य त्वं यस्य ते मतिरीदृशी}
{रघुवीरपदद्वन्द्व मकरन्द स्पृहावती}% ९

\twolineshloka
{वदन्ति मुनयः सर्वे साधूनां सङ्गमं वरम्}
{यस्मात्पापक्षयकरी रघुनाथकथा भवेत्}% १०

\twolineshloka
{त्वया मेऽनुग्रहः सृष्टो यद्रामः स्मारितः पुनः}
{सुरासुरकिरीटौघ मणिनीराजिताङ्घ्रिकः}% ११

\twolineshloka
{रावणारिकथा वार्द्धौ मशको मादृशः कियान्}
{यत्र ब्रह्मादयो देवा मोहिता न विदन्त्यपि}% १२

\twolineshloka
{तथापि भो मया तुभ्यं वक्तव्यं स्वीयशक्तितः}
{पक्षिणः स्वगतिं श्रित्वा खे गच्छन्ति सुविस्तरे}% १३

\twolineshloka
{चरितं रघुनाथस्य शतकोटिप्रविस्तरम्}
{येषां वै यादृशी बुद्धिस्ते वदन्त्येव तादृशम्}% १४

\twolineshloka
{रघुनाथसतीकीर्तिर्मद्बुद्धिं निर्मलीमसाम्}
{करिष्यति स्वसम्पर्कात्कनकं त्वनलो यथा}% १५

\uvacha{सूत उवाच}

\twolineshloka
{इत्युक्त्वा तं मुनिवरं ध्यानस्तिमितलोचनः}
{ज्ञानेनालोकयाञ्चक्रे कथां लोकोत्तरां शुभाम्}% १६

\twolineshloka
{गद्गदस्वरसंयुक्तो महाहर्षाङ्किताङ्गकः}
{कथयामास विशदां कथां दाशरथेः पुनः}% १७

\uvacha{शेष उवाच}

\twolineshloka
{लङ्केश्वरे विनिहते देवदानवदुःखदे}
{अप्सरोगणवक्त्राब्जचन्द्रमः कान्तिहर्तरि}% १८

\twolineshloka
{सुराः सर्वे सुखं प्रापुरिन्द्र प्रभृतयस्तदा}
{सुखं प्राप्ताः स्तुतिं चक्रुर्दासवत्प्रणतिं गताः}% १९

\twolineshloka
{लङ्कायां च प्रतिष्ठाप्य धर्मयुक्तं बिभीषणम्}
{सीतयासहितो रामः पुष्पकं समुपाश्रितः}% २०

\twolineshloka
{सुग्रीवहनुमत्सीतालक्ष्मणैः संयुतस्तदा}
{बिभीषणोऽपि सचिवैरन्वगाद्विरहोत्सुकः}% २१

\twolineshloka
{लङ्कां स पश्यन्बहुधा भग्नप्राकारतोरणाम्}
{दृष्ट्वाऽशोकवनं तत्र सीतास्थानं मुमूर्च्छ ह}% २२

\twolineshloka
{शिंशपांस्तत्र वृक्षांश्च पुष्पितान्कोरकैर्युतान्}
{राक्षसीभिः समाकीर्णान्मृताभिर्हनुमद्भयात्}% २३

\twolineshloka
{इत्थं सर्वं विलोक्याशु रामः प्रायात्पुरीं प्रति}
{ब्रह्मादिदेवैः सहितः स्वीयस्वीयविमानकैः}% २४

\twolineshloka
{देवदुन्दुभिनिर्घोषाञ्छृण्वञ्छ्रोत्रसुखावहान्}
{तथैवाप्सरसां नृत्यैः पूज्यमानो रघूत्तमः}% २५

\twolineshloka
{सीतायै दर्शयन्मार्गे तीर्थान्याश्रमवन्ति च}
{मुनींश्च मुनिपुत्रांश्च मुनिपत्नीः पतिव्रताः}% २६

\twolineshloka
{यत्रयत्र कृतावासाः पूर्वं रामेण धीमता}
{तान्सर्वान्दर्शयामास लक्ष्मणेन समन्वितः}% २७

\twolineshloka
{इत्येवं दर्शयंस्तस्यै रामोऽद्राक्षीत्स्वकां पुरीम्}
{तस्याः पुनः समीपे तु नन्दिग्रामं ददर्श ह}% २८

\twolineshloka
{यत्र वै भरतो राजा पालयन्धर्ममास्थितः}
{भ्रातुर्वियोगजनितं दुःखचिह्नं वहन्बहु}% २९

\twolineshloka
{गर्तशायी ब्रह्मचारी जटावल्कलसंयुतः}
{कृशाङ्गयष्टिर्दुःखार्तः कुर्वन्रामकथां मुहुः}% ३०

\twolineshloka
{यवान्नमपि नो भुङ्क्ते जलं पिबति नो मुहुः}
{उद्यन्तं सवितारं यो नमस्कृत्य ब्रवीति च}% ३१

\twolineshloka
{जगन्नेत्रसुरस्वामिन्हर मे दुष्कृतं महत्}
{मदर्थे रामचन्द्रोऽपि जगत्पूज्यो वनं ययौ}% ३२

\twolineshloka
{सीतया सुकुमाराङ्ग्या सेव्यमानोऽटवीं गतः}
{या सीता पुष्पपर्यङ्के वृन्तमासाद्य दुःखिता}% ३३

\twolineshloka
{या सीता रविसन्तापं कदापि प्राप नो सती}
{मदर्थे जानकी सा च प्रत्यरण्यं भ्रमत्यहो}% ३४

\twolineshloka
{या सीता राजवृन्दैश्च न दृष्टा नयनैः कदा}
{सा सीता दृश्यते नूनं किरातैः कालरूपिभिः}% ३५

\twolineshloka
{या सीता मधुरं त्वन्नं भोजिता न बुभुक्षति}
{सा सीताद्य वनस्थानि फलानि प्रार्थयत्यहो}% ३६

\twolineshloka
{इत्येवमन्वहं सूर्यमुपस्थाय वदत्यदः}
{प्रातःप्रातर्महाराजो भरतो रामवल्लभः}% ३७

\twolineshloka
{यश्चोच्यमानः सचिवैः समदुःखसुखैर्बुधैः}
{नीतिज्ञैः शास्त्रनिपुणैरिति प्रोवाच तान्नृपः}% ३८

\twolineshloka
{अमात्या दुर्भगं मां किं प्रब्रूत पुरुषाधमम्}
{मदर्थे मेऽग्रजो रामो वनं प्राप्यावसीदति}% ३९

\twolineshloka
{दुर्भगस्य मम प्रस्वाः पापमार्जनमादरात्}
{करोमि रामचन्द्राङ्घ्रिं स्मारं स्मारं सुमन्त्रिणः}% ४०

\twolineshloka
{धन्या सुमित्रा सुतरां वीरसूः स्वपतिप्रिया}
{यस्यास्तनूजो रामस्य चरणौ सेवतेऽन्वहम्}% ४१

\twolineshloka
{यत्र ग्रामे स्थितो नूनं भरतो भ्रातृवत्सलः}
{विलापं प्रकरोत्युच्चैस्तं ग्रामं स ददर्श ह}% ४२

{॥इति श्रीपद्मपुराणे पातालखण्डे शेषवात्स्यायनसंवादे रामाश्वमेधे रघुनाथस्य भरतावासनन्दिग्रामदर्शनो नाम प्रथमोऽध्यायः॥१॥}

\dnsub{द्वितीयोऽध्यायः}\resetShloka

\uvacha{शेष उवाच}

\twolineshloka
{अथ तद्दर्शनोत्कण्ठा विह्वलीकृतचेतसा}
{पुनः पुनः स्मृतो भ्राता भरतो धार्मिकाग्रणीः}% १

\twolineshloka
{उवाच च हनूमन्तं बलवन्तं समीरजम्}
{प्रस्फुरद्दशनव्याज चन्द्रकान्तिहतान्धकः}% २

\twolineshloka
{शृणु वीर हनूमंस्त्वं मद्गिरं भ्रातृनोदिताम्}
{चिरन्तनवियोगेन गद्गदीकृतविह्वलाम्}% ३

\twolineshloka
{गच्छ तं भ्रातरं वीर समीरणतनूद्भव}
{मद्वियोगकृशां यष्टिं वपुषो बिभ्रतं हठात्}% ४

\twolineshloka
{यो वल्कलं परीधत्ते जटां धत्ते शिरोरुहे}
{फलानां भक्षणमपि न कुर्याद्विरहातुरः}% ५

\twolineshloka
{परस्त्री यस्य मातेव लोष्टवत्काञ्चनं पुनः}
{प्रजाः पुत्रानिवोद्रक्षेद्बान्धवो मम धर्मवित्}% ६

\twolineshloka
{मद्वियोगजदुःखाग्निज्वालादग्धकलेवरम्}
{मदागमनसन्देश पयोवृष्ट्याशु सिञ्चतम्}% ७

\twolineshloka
{सीतया सहितं रामं लक्ष्मणेन समन्वितम्}
{सुग्रीवादिकपीन्द्रैश्च रक्षोभिः सबिभीषणैः}% ८

\twolineshloka
{प्राप्तं निवेदय सुखात्पुष्पकासनसंस्थितम्}
{येन मे सोऽनुजः शीघ्रं सुखमेति मदागमात्}% ९

\twolineshloka
{इति श्रुत्वा ततो वाक्यं रघुवीरस्य धीमतः}
{जगाम भरतावासं नन्दिग्रामं निदेशकृत्}% १०

\twolineshloka
{गत्वा स नन्दिग्रामं तु मन्त्रिवृद्धैः सुसंयतम्}
{भरतं भ्रातृविरहक्लिन्नं धीमान्ददर्श ह}% ११

\twolineshloka
{कथयन्तं मन्त्रिवृद्धान्रामचन्द्रकथानकम्}
{तदीय पदापाथोज मकरन्दसुनिर्भरः}% १२

\twolineshloka
{नमश्चकार भरतं धर्मं मूर्तियुतं किल}
{विधात्रा सकलांशेन सत्त्वेनैव विनिर्मितम्}% १३

\twolineshloka
{तं दृष्ट्वा भरतः शीघ्रं प्रत्युत्थाय कृताञ्जलि}
{स्वागतं चेति होवाच रामस्य कुशलं वद}% १४

\twolineshloka
{इत्येवं वदतस्तस्य भुजो दक्षिणतोऽस्फुरत्}
{हृदयाच्च गतः शोको हर्षास्रैः पूरिताननः}% १५

\twolineshloka
{विलोक्य तादृशं भूपं प्रत्युवाच कपीश्वरः}
{निकटे हि पुरः प्राप्तं विद्धि रामं सलक्ष्मणम्}% १६

\twolineshloka
{रामागमनसन्देशामृतसिक्तकलेवरः}
{प्रापयद्धर्षपूरं हि सहस्रास्यो न वेद्म्यहम्}% १७

\twolineshloka
{जगाद मम तन्नास्ति यत्तुभ्यं दीयते मया}
{दासोऽस्मि जन्मपर्यन्तं रामसन्देशहारकः}% १८

\twolineshloka
{वसिष्ठोऽपि गृहीत्वार्घ्यं मन्त्रिवृद्धाः सुहर्षिताः}
{जग्मुस्ते रामचन्द्रं च हनुमद्दर्शिताध्वना}% १९

\twolineshloka
{दृष्ट्वा दूरात्समायान्तं रामचन्द्रं मनोरमम्}
{पुष्पकासनमध्यस्थं ससीतं सहलक्ष्मणम्}% २०

\twolineshloka
{रामोऽपि दृष्ट्वा भरतं पादचारेण सङ्गतम्}
{जटावल्कलकौपीन परिधानसमन्वितम्}% २१

\twolineshloka
{अमात्यान्भ्रातृवेषेण समवेषाञ्जटाधरान्}
{नित्यं तपः क्लिष्टतया कृशरूपान्ददर्श ह}% २२

\twolineshloka
{रामोऽपि चिन्तयामास दृष्ट्वा वै तादृशं नृपम्}
{अहो दशरथस्यायं राजराजस्य धीमतः}% २३

\twolineshloka
{पुत्रः पदातिरायाति जटावल्कलवेषभृत्}
{न दुःखं तादृशं मेऽन्यद्वनमध्यगतस्य हि}% २४

\twolineshloka
{यादृशं मद्वियोगेन चैतस्य परिवर्त्तते}
{अहो पश्यत मे भ्राता प्राणात्प्रियतमः सखा}% २५

\twolineshloka
{श्रुत्वा मां निकटे प्राप्तं मन्त्रिवृद्धैः सुहर्षितैः}
{द्रष्टुं मां भरतोऽभ्येति वसिष्ठेन समन्वितः}% २६

\twolineshloka
{इति ब्रुवन्नरपतिः पुष्पकान्नभसोऽङ्गणात्}
{बिभीषणहनूमद्भ्यां लक्ष्मणेन कृतादरः}% २७

\twolineshloka
{यानादवतताराशु विरहात्क्लिन्नमानसः}
{भ्रातर्भ्रातः पुनर्भ्रातर्भ्रातर्भ्रातर्वदन्मुहुः}% २८

\threelineshloka
{दृष्ट्वा समुत्तीर्णमिमं रामचन्द्रं सुरैर्युतम्}
{भरतो भ्रातृविरहक्लिन्नं धीमान्ददर्श ह}
{हर्षाश्रूणि प्रमुञ्चंश्च दण्डवत्प्रणनाम ह}% २९

\twolineshloka
{रघुनाथोऽपि तं दृष्ट्वा दण्डवत्पतितं भुवि}
{उत्थाप्य जगृहे दोर्भ्यां हर्षालोकसमन्वितः}% ३०

\twolineshloka
{उत्थापितोऽपि हि भृशं नोदतिष्ठद्रुदन्मुहुः}
{रामचन्द्रपदाम्भोजग्रहणासक्तबाहुभृत्}% ३१

\uvacha{भरत उवाच}

\twolineshloka
{दुराचारस्य दुष्टस्य पापिनो मे कृपां कुरु}
{रामचन्द्र महाबाहो कारुण्यात्करुणानिधे}% ३२

\twolineshloka
{यस्ते विदेहजा पाणिस्पर्शं क्रूरममन्यत}
{स एव चरणो राम वने बभ्राम मत्कृते}% ३३

\twolineshloka
{इत्युक्त्वाश्रुमुखो दीनः परिरभ्य पुनः पुनः}
{प्राञ्जलिः पुरतस्तस्थौ हर्षविह्वलिताननः}% ३४

\twolineshloka
{रघुनाथस्तमनुजं परिष्वज्य कृपानिधिः}
{प्रणम्य च महामन्त्रिमुख्यानापृच्छ्य सादरम्}% ३५

\twolineshloka
{भरतेन समं भ्रात्रा पुष्पकासनमास्थितः}
{सीतां ददर्श भरतो भ्रातृपत्नीमनिन्दिताम्}% ३६

\twolineshloka
{अनसूयामिवात्रेः किं लोपामुद्रां घटोद्भुवः}
{पतिव्रतां जनकजाममन्यतननाम च}% ३७

\twolineshloka
{मातः क्षमस्व यदघं मया कृतमबुद्धिना}
{त्वत्सदृश्यः पतिपराः सर्वेषां साधुकारिकाः}% ३८

\twolineshloka
{जानक्यापि महाभागा देवरं वीक्ष्य सादरम्}
{आशीर्भिरभियुज्याथ समपृच्छदनामयम्}% ३९

\twolineshloka
{विमानवरमारूढास्ते सर्वे नभसोऽङ्गणे}
{क्षणादालोकयाञ्चक्रे निकटे स्वपितुः पुरीम्}% ४०

{॥इति श्रीपद्मपुराणे पातालखण्डे रामाश्वमेधे शेषवात्स्यायनसंवादे रामराजधानीदर्शनो नाम द्वितीयोऽध्यायः॥२॥}

\dnsub{तृतीयोऽध्यायः}\resetShloka

\uvacha{शेष उवाच}

\twolineshloka
{दृष्ट्वा रामो राजधानीं निजलोकनिवासिनीम्}
{जहर्ष मतिमान्वीरश्चिरदर्शनलालसः}% १

\twolineshloka
{भरतोऽपि स्वकं मित्रं सुमुखं नगरं प्रति}
{प्रेषयामास सचिवं नगरोत्सवसिद्धये}% २

\uvacha{भरत उवाच}

\twolineshloka
{कुर्वन्तु लोकास्त्वरितं रघुनाथागमोत्सवम्}
{मन्दिरे मन्दिरे रम्यं कृतकौतुकचित्रकम्}% ३

\twolineshloka
{विपांसुका राजमार्गाश्चन्दनद्रवसिञ्चिताः}
{प्रसूनभरसङ्कॢप्ता हृष्टपुष्टजनावृताः}% ४

\twolineshloka
{विचित्रवर्णध्वजभा चित्रिताखिलस्वाङ्गणाः}
{मेघागमे धनुरिव पश्यन्त्वेव वलीमुखाः}% ५

\twolineshloka
{प्रतिगेहं तु लोकानां कारयन्त्वगरूक्षणम्}
{यद्धूमं वीक्ष्य शिखिनो नृत्यं कुर्वन्तु लीलया}% ६

\twolineshloka
{हस्तिनो मम शैलाभानाधोरणसुयन्त्रितान्}
{विचित्रयन्तु बहुशो गैरिकाद्युपधातुभिः}% ७

\twolineshloka
{वाजिनश्चित्रिता भूयः सुशोभन्तु मनोजवाः}
{यद्वेगवीक्षणादेव गर्वं त्यजति स्वर्हयः}% ८

\twolineshloka
{कन्याः सहस्रशो रम्याः सर्वाभरणभूषिताः}
{गजोपरि समारूढा मुक्ताभिर्विकिरन्तु च}% ९

\twolineshloka
{ब्राह्मण्यः पात्रहस्ताश्च दूर्वाहारिद्रसंयुताः}
{सुवासिन्यो महाराजं रामं नीराजयन्तु ताः}% १०

\twolineshloka
{कौसल्यापुत्रसंयोगसन्देश विधुरा सती}
{हर्षं प्राप्नोतु सुकृशा तदीक्षणसुलालसा}% ११

\twolineshloka
{इत्येवमादिरचनाः पुरशोभाविधायिकाः}
{करोतु जनता हृष्टा रामस्यागमनं प्रति}% १२

\uvacha{शेष उवाच}

\twolineshloka
{इति श्रुत्वा ततो वाक्यं सुमुखो मन्त्रवित्तमः}
{प्रययौ नगरीं कर्तुं कृतकौतुकतोरणाम्}% १३

\twolineshloka
{गत्वाथ नगरीं तां वै मन्त्री तु सुमुखाभिधः}
{ख्यापयामास लोकानां रामागममहोत्सवम्}% १४

\twolineshloka
{लोकाः श्रुत्वा पुरीं प्राप्तं रघुनाथं सुहर्षिताः}
{पूर्वं तदीय विरहत्यक्तभोगसुखादयः}% १५

\twolineshloka
{ब्राह्मणा वेदसम्पन्नाः पवित्रा दर्भपाणयः}
{धौतोत्तरीयवलिता जग्मुः श्रीरघुनायकम्}% १६

\twolineshloka
{क्षत्रिया ये शूरतमा धनुर्बाणधरा वराः}
{सङ्ग्रामे बहुशो वीरा जेतारो ययुरप्यमुम्}% १७

\twolineshloka
{वैश्या धनसमृद्धाश्च मुद्राशोभितपाणयः}
{शुभ्रवस्त्रपरीधाना अभिजग्मुर्नरेश्वरम्}% १८

\twolineshloka
{शूद्रा द्विजेषु ये भक्ताः स्वीयाचारसुनिष्ठिताः}
{वेदाचाररता ये वै तेऽपिजग्मुः पुरीपतिम्}% १९

\twolineshloka
{ये ये वृत्तिकरा लोकाः स्वे स्वे कर्मण्यधिष्ठिताः}
{स्वकं वस्तु समादाय ययुः श्रीरामभूपतिम्}% २०

\twolineshloka
{इत्थं भूपतिसन्देशात्प्रमोदाप्लवसंयुताः}
{नाना कौतुकसंयुक्ता आजग्मुर्मनुजेश्वरम्}% २१

\uvacha{शेष उवाच}

\twolineshloka
{रघुनाथोऽपि सकलैर्दैवतैः स्वस्वयानगैः}
{परीतः प्रविवेशोच्चैः पुरीं रचितमोहनाम्}% २२

\twolineshloka
{प्लवङ्गाः प्लवनैर्युक्ता आकाशपथचारिणः}
{स्वस्वशोभापरीताङ्गाश्चानुजग्मुः पुरोत्तमम्}% २३

\twolineshloka
{पुष्पकादवरुह्याशु नरयानमथारुहत्}
{सीतया सहितो रामः परिवारसमावृतः}% २४

\twolineshloka
{अयोध्यां प्रविवेशाथ कृतकौतुकतोरणाम्}
{हृष्टपुष्टजनाकीर्णामुत्सवैः परीभूषिताम्}% २५

\twolineshloka
{वीणापणवभेर्यादिवादित्रैराहतैर्भृशम्}
{शोभमानः स्तूयमानः सूतमागधबन्दिभिः}% २६

\twolineshloka
{जय राघवरामेति जय सूर्यकुलाङ्गद}
{जय दाशरथे देव जयताल्लोकनायकः}% २७

\twolineshloka
{इति शृण्वञ्छुभां वाचं पौराणां हर्षिताङ्गिनाम्}
{रामदर्शनसञ्जात पुलकोद्भेद शोभिनाम्}% २८

\twolineshloka
{प्रविवेश वरं मार्गं रथ्याचत्वरभूषितम्}
{चन्दनोदकसंसिक्तं पुष्पपल्लवसंयुतम्}% २९

\twolineshloka
{तदा पौराङ्गनाः काश्चिद्गवाक्षबिलमाश्रिताः}
{रघुनाथस्वरूपेक्षा जातकामा अथाब्रुवन्}% ३०

\uvacha{पौराङ्गना ऊचुः}

\twolineshloka
{धन्या अभूवन्बत भिल्लकन्या वनेषु या राममुखारविन्दम्}
{स्वलोचनेन्दीवरकैरथापिबन्स्वभाग्यसञ्जातमहोदया इमाः}% ३१

\twolineshloka
{धन्यं मुखं पश्यत वीरधाम्नः श्रीरामदेवस्य सरोजनेत्रम्}
{यद्दर्शनं धातृमुखाः सुरा अपि प्रापुर्महद्भाग्ययुता वयन्त्वहो}% ३२

\twolineshloka
{एतन्मुखं पश्यत चारुहासं किरीटसंशोभिनिजोत्तमाङ्गम्}
{बन्धूकधिक्कारलसच्छविप्रदं दन्तच्छदं बिभ्रतमुच्चनासम्}% ३३

\twolineshloka
{इति गदितवतीस्ताः स्नेहभारेण रामा नलिनदलसदृक्षैर्नेत्रपातैर्निरीक्ष्य}
{निखिलगुरुरनूनप्रेमभारं नृलोकं जननिगृहमियेष प्रोषिताङ्गेन हृष्टः}% ३४

{॥इति श्रीपद्मपुराणे पातालखण्डे शेषवात्स्यायनसंवादे रामाश्वमेधे रघुनाथस्य पुरप्रवेशनं नाम तृतीयोऽध्यायः॥३॥}

\dnsub{चतुर्थोऽध्यायः}\resetShloka

\uvacha{वात्स्यायन उवाच}

\twolineshloka
{भुजगाधीश्वरेशान धराभारधरक्षम}
{शृण्वेकं संशयं मह्यं कृपया कथयस्व तम्}% १

\twolineshloka
{रघुनाथस्य गमनं वनं प्रति यदा ह्यभूत्}
{तदा प्रभृति देहेन स्थिता शून्येन चेतसा}% २

\twolineshloka
{तद्विप्रयोगविधुरा कृशदेहातिदुःखिता}
{सुमुखान्मन्त्रिणः श्रुत्वा रघुनाथं समागतम्}% ३

\twolineshloka
{कथं जहर्ष किमभूत्कीदृशं तत्र चिह्नितम्}
{रामचन्द्रस्य सन्देशहर्तारं किमुवाच सा}% ४

\twolineshloka
{एतन्मे संशयं छिन्धि रघुनाथगुणोदयम्}
{यथावच्छृण्वते मह्यं कथयस्व प्रसादतः}% ५

\uvacha{शेष उवाच}

\twolineshloka
{साधुपृष्टं महाभाग द्विजवर्यपुरस्कृत}
{तन्मे निगदतः साक्षाच्छृणुष्वैकमनाः किल}% ६

\twolineshloka
{सा वै तद्वदनाम्भोज च्युतं रामागमामृतम्}
{पीत्वा पीत्वा बभूवाहो स्थगिताङ्गेन विह्वला}% ७

\twolineshloka
{किं मे स्वप्नो विमूढायाः किं वा भ्रमकरं वचः}
{ममवै मन्दभाग्यायाः कथं रामेक्षणं पुनः}% ८

\twolineshloka
{बहुना तपसा कृत्वा प्राप्तोऽयं वै सुतः शिशुः}
{केनचिन्मम पापेन विप्रयोगं गतः पुनः}% ९

\twolineshloka
{सुमन्त्रिन्कुशली रामः सीतालक्ष्मणसंयुतः}
{कथं मां स्मरते वीरो वनचारी सुदुःखिताम्}% १०


\threelineshloka
{इति सा विललापोच्चै रघुनाथस्मृतिं गता}
{न निवेद निजं किञ्चित्परकीयं विमोहिता}
{सुमुखोऽपि तथा दृष्ट्वा दुःखितां मातरं भृशम्}% ११

\twolineshloka
{वीजयामास वासोग्रैः संज्ञामाप च सा पुनः}
{उवाच जननीं सौम्यं वचोहर्षकरं मुहुः}% १२

\twolineshloka
{रघुनाथागमस्मार हृष्टां तां व्यदधात्पुनः}
{मातर्विद्धि गृहं प्राप्तं रघुनाथं सलक्ष्मणम्}% १३

\twolineshloka
{सीतया सहितं पश्य चाशीर्भिरभियुङ्क्ष्व च}
{इति तथ्यं वचः श्रुत्वा सुमुखेन प्रभाषितम्}% १४

\twolineshloka
{यादृशं हर्षमापेदे तादृशं वेद्म्यहं नहि}
{उत्थाय चाजिरे प्राप्ता रोमाञ्चिततनूरुहा}% १५

\twolineshloka
{हर्षविह्वलिताङ्ग्यश्रु मुञ्चन्ती राममैक्षत}
{तावत्स रामो राजेन्द्रो नरयानमधिश्रितः}% १६

\twolineshloka
{प्राप्तः स्वमातुर्भवनं कैकेय्याः सुनयः पुरः}
{कैकेय्यपि त्रपाभारनम्रा रामं पुरःस्थितम्}% १७

\twolineshloka
{नोवाच किञ्चिन्महतीं चिन्तां प्राप्तवती मुहुः}
{सूर्यवंशध्वजो रामो मातरं वीक्ष्य लज्जिताम्}% १८

\onelineshloka*
{उवाच सान्त्वयंस्तां च वाक्यैर्विनयमिश्रितैः}

\uvacha{श्रीराम उवाच}

\onelineshloka
{मातर्मया वनं गत्वा सर्वमाचरितं तथा}% १९

\twolineshloka
{अधुना करवै किं वा त्वदाज्ञातो जनन्यहो}
{मया न्यूनं कृतं नास्ति कथं मां नेक्ष्यसे पुनः}% २०

\twolineshloka
{आशीर्भिरभिनन्द्यैनं भरतं मां च वीक्षय}
{इति श्रुत्वापि तद्वाक्यं सा नम्रवदनानघ}% २१

\twolineshloka
{शनैः शनैः प्रत्युवाच राम गच्छ स्वमालयम्}
{रामोऽपि श्रुत्वा वचनं जनन्याः पुरुषोत्तमः}% २२

\twolineshloka
{नमस्कृत्य ययौ गेहं सुमित्रायाः कृपानिधिः}
{सुमित्रा पुत्रसहितं रामं दृष्ट्वा महामनाः}% २३

\twolineshloka
{चिरञ्जीव चिरञ्जीव ह्याशीर्भिरिति चाभ्यधात्}
{मातुश्च रामभद्रोऽपि चरणौ प्रणिपत्य च}% २४

\twolineshloka
{परिष्वज्य मुदायुक्तो जगाद वचनं पुनः}
{रत्नगर्भे मम भ्रात्रा केनापि न कृतं तथा}% २५

\onelineshloka
{यथायमकरोद्धीमान्ममदुःखापनोदनम्}% २६

\twolineshloka
{रावणेन हृता सीता मया यत्प्राप्यते पुनः}
{मातस्तत्सर्वमाविद्धि लक्ष्मणस्य विचेष्टितम्}% २७

\twolineshloka
{दत्तामाशिषमागृह्य शिरसायं सुमित्रया}
{निजमातुश्च भवनं प्रययौ विबुधैर्वृतः}% २८

\twolineshloka
{मातरं वीक्ष्य हृषितां निजदर्शनलालसाम्}
{स्वयानादवरुह्याशु चरणावग्रहीद्धरिः}% २९

\twolineshloka
{माता तद्दर्शनोत्कण्ठा विह्वलीकृतमानसा}
{परिष्वज्य परिष्वज्य रामं मुदमवाप सा}% ३०

\twolineshloka
{शरीरे रोमहर्षोऽभूद्गद्गदा वागभूत्तदा}
{हर्षाश्रूणि तु सोष्णानि प्रवाहं प्रापुरापदात्}% ३१

\twolineshloka
{जननीं वीक्ष्य विनयी ताटङ्कद्वयवर्जिताम्}
{कराकल्प पदाकल्परहितां बिभ्रतीं तनुम्}% ३२

\twolineshloka
{किञ्चित्स्वदर्शनाद्धृष्टां कृशाङ्गीं तां स शोकभाक्}
{दुःखस्य समयो नायमिति मत्वा जगाद ताम्}% ३३

\uvacha{श्रीराम उवाच}

\twolineshloka
{मातर्मया त्वच्चरणौ चिरकालं न सेवितौ}
{ततः क्षमस्वापराधं भाग्यहीनस्य वै मम}% ३४

\twolineshloka
{ये पुत्रा मातापित्रोर्न शुश्रूषायां समुत्सुकाः}
{ते मन्तव्याः परा मातः कीटका रेतसो भवाः}% ३५

\twolineshloka
{किं कुर्वे जनकाज्ञातो गतो वै दण्डकं वनम्}
{तत्रापि त्वत्कृपापाङ्गात्तीर्णोऽस्मि दुःखसागरम्}% ३६

\twolineshloka
{रावणेन हृता सीता लङ्कायां गमिता पुनः}
{त्वत्कृपातो मया लब्धा तं हत्वा राक्षसेश्वरम्}% ३७

\twolineshloka
{सीतेयं त्वच्चरणयोः पतिता वै पतिव्रता}
{सम्भावयाशु चकितां त्वत्पादार्पितमानसाम्}% ३८

\twolineshloka
{इति श्रुत्वा तु तद्वाक्यं पादयोः पतितां स्नुषाम्}
{आशीर्भिरभियुज्यैनां बभाषे तां पतिव्रताम्}% ३९

\twolineshloka
{सीते स्वपतिना सार्द्धं चिरं विलस भामिनि}
{पुत्रौ प्रसूय च कुलं स्वकं पावय पावने}% ४०

\twolineshloka
{त्वत्सदृश्यः पतिपराः पतिदुःखसुखानुगाः}
{भवन्ति दुःखभागिन्यो न हि सत्यं जगत्त्रये}% ४१

\twolineshloka
{विदेहपुत्रि स्वकुलं त्वया पावितमात्मना}
{रामपादाब्जयुगलमनुयान्त्या महावनम्}% ४२

\twolineshloka
{किं चित्रं यत्पुमांसस्तु वैरिकोटिप्रभञ्जनाः}
{येषां गेहे सती भार्या स्वपतिप्रियवाञ्छिका}% ४३

\twolineshloka
{इत्युक्त्वा रघुनाथस्य भार्यामञ्चितलोचनाम्}
{तूष्णीं बभूव हृषिता प्रहृष्टस्वतनूरुहा}% ४४

\twolineshloka
{अथ भ्रातास्य भरतः पित्रा दत्तं निजं महत्}
{राज्यं निवेदयामास रामचन्द्राय धीमते}% ४५

\twolineshloka
{मन्त्रिणस्ते प्रहृष्टाङ्गा दैवज्ञान्मन्त्रकोविदान्}
{आहूय सुमुहूर्तन्ते पप्रच्छुः परमादरात्}% ४६

\twolineshloka
{शुभे मुहूर्ते सुदिने शुभनक्षत्रसंयुते}
{अभिषेकं महाराज्ये कारयामासुरुद्यताः}% ४७

\twolineshloka
{सप्तद्वीपवतीं पृथ्वीं व्याघ्रचर्मणि सुन्दरे}
{लिखित्वोपरि राजेन्द्रो महाराजोधितस्थिवान्}% ४८

\twolineshloka
{तद्दिनादेव साधूनां मनांसि प्रमुदं ययुः}
{दुष्टानां चेतसो ग्लानिरभवत्परतापिनाम्}% ४९

\twolineshloka
{स्त्रियस्तु पतिभक्त्या च पतिव्रतपरायणाः}
{मनसापि कदा पापं नाचरन्ति जना मुने}% ५०

\twolineshloka
{दैत्यादेवास्तथा नागा यक्षासुरमहोरगाः}
{सर्वे न्यायपथे स्थित्वा रामाज्ञां शिरसा दधुः}% ५१

\twolineshloka
{परोपकरणेयुक्ताः स्वधर्मसुखनिर्वृताः}
{विद्याविनोदगमिता दिनरात्रिक्षणाः शुभाः}% ५२

\twolineshloka
{वातोऽपि मार्गसंस्थानां बलान्नाहरते महान्}
{वासांस्यपि तु सूक्ष्माणि तत्र चौरकथा नहि}% ५३

\twolineshloka
{धनदो ह्यर्थिनां रामः कारुण्यश्च कृपानिधिः}
{भ्रातृभिः सहितो नित्यं गुरुदेवस्तुतिं व्यधात्}% ५४

{॥इति श्रीपद्मपुराणे पातालखण्डे शेषवात्स्यायनसंवादे रामाश्वमेधे रघुवरस्य राज्याभिषेको नाम चतुर्थोऽध्यायः॥४॥}

\dnsub{पञ्चमोऽध्यायः}\resetShloka

\uvacha{शेष उवाच}

\twolineshloka
{अथाभिषिक्तं रामं तु तुष्टुवुः प्रणताः सुराः}
{रावणाभिधदैत्येन्द्र वधहर्षितमानसाः}% १

\uvacha{देवा ऊचुः}

\twolineshloka
{जय दाशरथे सुरार्तिहञ्जयजय दानववंशदाहक}
{जय देववराङ्गनागणग्रहणव्यग्रकरारिदारक}% २

\twolineshloka
{तवयद्दनुजेन्द्र नाशनं कवयो वर्णयितुं समुत्सुकाः}
{प्रलये जगतान्ततीः पुनर्ग्रससे त्वं भुवनेशलीलया}% ३

\twolineshloka
{जय जन्मजरादिदुःखकैः परिमुक्तप्रबलोद्धरोद्धर}
{जय धर्मकरान्वयाम्बुधौ कृतजन्मन्नजरामराच्युत}% ४

\twolineshloka
{तव देववरस्य नामभिर्बहुपापा अपि ते पवित्रिताः}
{किमु साधुद्विजवर्यपूर्वकाः सुतनुं मानुषतामुपागताः}% ५

\twolineshloka
{हरविरिञ्चिनुतं तव पादयोर्युगलमीप्सितकामसमृद्धिदम्}
{हृदि पवित्रयवादिकचिह्नितैः सुरचितं मनसा स्पृहयामहे}% ६

\twolineshloka
{यदि भवान्न दधात्यभयं भुवो मदनमूर्ति तिरस्करकान्तिभृत्}
{सुरगणा हि कथं सुखिनः पुनर्ननुभवन्ति घृणामय पावन}% ७

\twolineshloka
{यदा यदास्मान्दनुजाहि दुःखदास्तदा तदा त्वं भुवि जन्मभाग्भवेः}
{अजोऽव्ययोऽपीशवरोऽपि सन्विभो स्वभावमास्थाय निजं निजार्चितः}% ८

\twolineshloka
{मृतसुधासदृशैरघनाशनैः सुचरितैरवकीर्य महीतलम्}
{अमनुजैर्गुणशंसिभिरीडितः प्रविश चाशु पुनर्हि स्वकं पदम्}% ९

\twolineshloka
{अनादिराद्योजररूपधारी हारी किरीटी मकरध्वजाभः}
{जयं करोतु प्रसभं हतारिः स्मरारि संसेवितपादपद्मः}% १०

\twolineshloka
{इत्युक्त्वा ते सुराः सर्वे ब्रह्मेन्द्रप्रमुखा मुहुः}
{प्रणेमुररिनाशेन प्रीणिता रघुनायकम्}% ११

\twolineshloka
{इति स्तुत्यातिसंहृष्टो रघुनाथो महायशाः}
{प्रोवाच तान्सुरान्वीक्ष्य प्रणतान्नतकन्धरान्}% १२

\uvacha{श्रीराम उवाच}

\twolineshloka
{सुरा वृणुत मे यूयं वरं किञ्चित्सुदुर्ल्लभम्}
{यं कोऽपि देवो दनुजो न यक्षः प्राप सादरः}% १३

\uvacha{सुरा ऊचुः}

\twolineshloka
{स्वामिन्भगवतः सर्वं प्राप्तमस्माभिरुत्तमम्}
{यदयं निहतः शत्रुरस्माकं तु दशाननः}% १४

\twolineshloka
{यदायदाऽसुरोऽस्माकं बाधां परिदधाति भोः}
{तदा तदेति कर्तव्यमेतावद्वैरिनाशनम्}% १५

\onelineshloka*
{तथेत्युक्त्वा पुनर्वीरः प्रोवाच रघुनन्दनः}

\uvacha{श्रीराम उवाच}

\onelineshloka
{सुराः शृणुत मद्वाक्यमादरेण समन्विताः}% १६

\twolineshloka
{भवत्कृतं मदीयैर्वैगुणैर्ग्रथितमद्भुतम्}
{स्तोत्रं पठिष्यति मुहुः प्रातर्निशि सकृन्नरः}% १७

\twolineshloka
{तस्य वैरि पराभूतिर्न भविष्यति दारुणा}
{न च दारिद्र्यसंयोगो न च व्याधिपराभवौ}% १८

\twolineshloka
{मदीयचरणद्वन्द्वे भक्तिस्तेषां तु भूयसी}
{भविष्यति मुदायुक्ते स्वान्ते पुंसां तु पाठतः}% १९

\twolineshloka
{इत्युक्त्वा सोऽभवत्तूष्णीं नरदेवशिरोमणिः}
{सुराः सर्वे प्रहृष्टास्ते ययुर्लोकं स्वकं स्वकम्}% २०

\twolineshloka
{रघुनाथोऽपि भ्रातॄंस्तान्पालयंस्तातवद्बुधान्}
{प्रजाः पुत्रानिव स्वीयाल्लाँलयँल्लोकनायकः}% २१

\twolineshloka
{यस्मिञ्छासति लोकानां नाकालमरणं नृणाम्}
{न रोगादि पराभूतिर्गृहेषु च महीयसी}% २२

\twolineshloka
{नेतिः कदापि द्दश्येत वैरिजं भयमेव च}
{वृक्षाः सदैव फलिनो मही भूयिष्ठधान्यका}% २३

\twolineshloka
{पुत्रपौत्रपरीवार सनाथी कृतजीवनाः}
{कान्ता संयोगजसुखैर्निरस्तविरहक्लमाः}% २४

\twolineshloka
{नित्यं श्रीरघुनाथस्य पादपद्मकथोत्सुकाः}
{कदापि परनिन्दासु वाचस्तेषां भवन्ति न}% २५

\twolineshloka
{कारवोऽपि कदा पापं नाचरन्ति मनस्यहो}
{रघुनाथकराघातदुःखशङ्काभिशंसिनः}% २६

\twolineshloka
{सीतापतिमुखालोक निश्चलीभूतलोचनाः}
{लोका बभूवुः सततं कारुण्यपरिपूरिताः}% २७

\twolineshloka
{राज्यं प्राप्तमसापत्नं समृद्धबलवाहनम्}
{ऋषिभिर्हृष्टपुष्टैश्च रम्यं हाटकभूषणैः}% २८

\twolineshloka
{सम्पुष्टमिष्टापूर्तानां धर्माणां नित्यकर्तृभिः}
{सदा सम्पन्नसस्यं च सुवसुक्षेत्रसंयुतम्}% २९

\twolineshloka
{सुदेशं सुप्रजं स्वस्थं सुतृणं बहुगोधनम्}
{देवतायतनानां च राजिभिः परिराजितम्}% ३०

\twolineshloka
{सुपूर्णा यत्र वै ग्रामाः सुवित्तर्द्धिविराजिताः}
{सुपुष्पकृत्रिमोद्यानाः सुस्वादुफलपादपाः}% ३१

\twolineshloka
{सपद्मिनीककासारा यत्र राजन्ति भूमयः}
{सदम्भा निम्नगा यत्र न यत्र जनता क्वचित्}% ३२

\twolineshloka
{कुलान्येव कुलीनानां वर्णानां नाधनानि च}
{विभ्रमो यत्र नारीषु न विद्वत्सु च कर्हिचित्}% ३३

\twolineshloka
{नद्यः कुटिलगामिन्यो न यत्र विषये प्रजाः}
{तमोयुक्ताः क्षपा यत्र बहुलेषु न मानवाः}% ३४

\twolineshloka
{रजोयुजः स्त्रियो यत्र नाधर्मबहुला नराः}
{धनैरनन्धो यत्रास्ति जनो नैव च भोजने}% ३५

\twolineshloka
{अनयः स्यन्दनो यत्र न च वैराजपूरुषः}
{दण्डः परशुकुद्दालवालव्यजनराजिषु}% ३६

\twolineshloka
{आतपत्रेषु नान्यत्र क्वचित्क्रोधोपरोधजः}
{अन्यत्राक्षिकवृन्देभ्यः क्वचिन्न परिदेवनम्}% ३७

\twolineshloka
{आक्षिका एव दृश्यन्ते यत्र पाशकपाणयः}
{जाड्यवार्ता जलेष्वेव स्त्रीमध्या एव दुर्बलाः}% ३८

\twolineshloka
{कठोरहृदया यत्र सीमन्तिन्यो न मानवाः}
{औषधेष्वेव यत्रास्ति कुष्ठयोगो न मानवे}% ३९

\twolineshloka
{वेधो यत्र सुरत्नेषु शूलं मूर्तिकरेषु वै}
{कम्पः सात्विकभावोत्थो न भयात्क्वापि कस्यचित्}% ४०

\twolineshloka
{सञ्ज्वरः कामजो यत्र दारिद्र्यकलुषस्य च}
{दुर्ल्लभत्वं सदैवस्य सुकृतेन च वस्तुनः}% ४१

\twolineshloka
{इभा एव प्रमत्ता वै युद्धे वीच्यो जलाशये}
{दानहानिर्गजेष्वेव तीक्ष्णा एव हि कण्टकाः}% ४२

\twolineshloka
{बाणेषु गुणविश्लेषो बन्धोक्तिः पुस्तके दृढा}
{स्नेहत्यागः खलेष्वेव न च वै स्वजने जने}% ४३

\twolineshloka
{तं देशं पालयामास लालयँल्लालिताः प्रजाः}
{धर्मं संस्थापयन्देशे दुष्टे दण्डधरोपमः}% ४४

\twolineshloka
{एवं पालयतो देशं धर्मेण धरणीतलम्}
{सहस्रं च व्यतीयुर्वै वर्षाण्येकादश प्रभोः}% ४५

\twolineshloka
{तत्र नीचजनाच्छ्रुत्वा सीताया अपमानताम्}
{स्वां च निन्दां रजकतस्तां तत्याज रघूद्वहः}% ४६

\twolineshloka
{पृथ्वीं पालयमानस्य धर्मेण नृपतेस्तदा}
{सीतां विरहितामेकां निदेशेन सुरक्षिताम्}% ४७

\twolineshloka
{कदाचित्संसदो मध्ये ह्यासीनस्य महामतेः}
{आजगाम मुनिश्रेष्ठः कुम्भोत्पत्तिर्मुनिर्महान्}% ४८

\twolineshloka
{गृहीत्वार्घ्यं समुत्तस्थौ वसिष्ठेन समन्वितः}
{जनताभिर्महाराजो वार्धिशोषकमागतम्}% ४९

\twolineshloka
{स्वागतेन सुसम्भाव्य पप्रच्छ तमनामयम्}
{सुखोपविष्टं विश्रान्तं बभाषे रघुनन्दनः}% ५०

{॥इति श्रीपद्मपुराणे पातालखण्डे शेषवात्स्यायनसंवादे रामाश्वमेधे अगस्त्यसमागमो नाम पञ्चमोऽध्यायः॥५॥}

\dnsub{षष्ठोऽध्यायः}\resetShloka

\uvacha{शेष उवाच}

\twolineshloka
{इत्थं स्वागतसन्तुष्टं ब्रह्मचर्यतपोनिधिम्}
{उवाच मतिमान्वीरः सर्वलोकगुरुर्मुनिम्}% १

\twolineshloka
{स्वागतं ते महाभाग कुम्भयोने तपोनिधे}
{त्वद्दर्शनेन सर्वे वै पाविताः सकुटुम्बकाः}% २

\twolineshloka
{कच्चिन्मतिस्ते वेदेषु शास्त्रेषु परिवर्तते}
{त्वत्तपोविघ्नकर्ता वै नास्ति भूमण्डले क्वचित्}% ३

\twolineshloka
{लोपामुद्रा महाभाग या च ते धर्मचारिणी}
{यस्याः पतिव्रता धर्मात्सर्वं भवति शोभनम्}% ४

\twolineshloka
{अपि शंस महाभाग धर्ममूर्ते कृपानिधे}
{अलोलुपस्य किं कार्यं करवाणि मुनीश्वर}% ५

\twolineshloka
{त्वत्तपोयोगतः सर्वं भवति स्वेच्छया बहु}
{तथापि मयि कृत्वैव कृपां शंश मुनीश्वरः}% ६

\uvacha{शेष उवाच}

\twolineshloka
{इत्युक्तो लोकगुरुणा राजराजेन धीमता}
{उवाच रामं लोकेशं विनीततरभाषया}% ७

\uvacha{अगस्त्य उवाच}

\twolineshloka
{स्वामिंस्तव सुदुर्दर्शं दर्शनं दैवतैरपि}
{मत्वा समागतं विद्धि राजराज कृपानिधे}% ८

\twolineshloka
{हतस्त्वया रावणाख्यस्त्वसुरो लोककण्टकः}
{दिष्ट्याद्य देवाः सुखिनो दिष्ट्या राजा बिभीषणः}% ९

\twolineshloka
{राम त्वद्दर्शनान्मेऽद्य गतं वै दुष्कृतं किल}
{सम्पूर्णो मे मनःकोश आनन्देन सुरोत्तम}% १०

\twolineshloka
{इत्युक्त्वा स बभूवाशु तूष्णीं कुम्भसमुद्भवः}
{रामसन्दर्शनाह्लादविह्वलीकृतमानसः}% ११

\twolineshloka
{रामः पप्रच्छ तं भूयो मुनिं ज्ञानविशारदम्}
{लोकातीतं भवद्भावि सर्वं जानासि सर्वतः}% १२

\twolineshloka
{मुने कथय मे सर्वं पृच्छतो हि सुविस्तरम्}
{कोऽसौ मया हतो यो हि रावणो विबुधार्दनः}% १३

\twolineshloka
{कुम्भकर्णोऽपि कस्त्वेष का जातिर्वै दुरात्मनः}
{देवो दैत्यः पिशाचो वा राक्षसो वा महामुने}% १४

\twolineshloka
{सर्वमाख्याहि सर्वज्ञ सर्वं जानासि विस्तरात्}
{अतः कथय मे सर्वं कृपां कृत्वा ममोपरि}% १५

\twolineshloka
{इति श्रुत्वा ततो वाक्यं कुम्भजन्मा तपोनिधिः}
{यत्पृष्टं रघुराजेन प्रवक्तुं तत्प्रचक्रमे}% १६

\twolineshloka
{राजन्सृष्टिकरो ब्रह्मा पुलस्त्यस्तत्सुतोऽभवत्}
{ततस्तु विश्रवा जज्ञे वेदविद्याविशारदः}% १७

\twolineshloka
{तस्य पत्नीद्वयं जातं पातिव्रत्यचरित्रभृत्}
{एका मन्दाकिनी नाम्नी द्वितीया कैकसी स्मृता}% १८

\twolineshloka
{पूर्वस्यां धनदो जज्ञे लोकपालविलासभृत्}
{योऽसौ शिवप्रसादेन लङ्कावासमचीकरत्}% १९

\twolineshloka
{विद्युन्मालिसुतायां तु पुत्रत्रयमभून्महत्}
{रावणः कुम्भकर्णश्च तथा पुण्यो बिभीषणः}% २०

\twolineshloka
{राक्षस्युदरजन्मत्वात्सन्ध्यासमयसम्भवात्}
{द्वयोरधर्मनिपुणा मतिरासीन्महामते}% २१

\twolineshloka
{एकदा तु विमानेन पुष्पकेण सुशोभिना}
{काञ्चनीयोपकल्पेन किङ्किणीजालमालिना}% २२

\twolineshloka
{आरुह्य पितरौ द्रष्टुं प्रायाच्छोभासमन्वितः}
{स्वगणैः संस्तुतो भूत्वा नानारत्नविभूषणैः}% २३

\twolineshloka
{आगत्य पित्रोश्चरणे पतित्वा चिरमात्मजः}
{हर्षविह्वलितात्मा च रोमाञ्चिततनूरुहः}% २४

\twolineshloka
{उवाच मेऽद्य सुदिनं महाभाग्यफलोदयः}
{यन्मे युष्मत्पदौ दृष्टौ महापुण्यददर्शनौ}% २५

\twolineshloka
{इत्यादिभिः स्तुतिपदैः स्तुत्वागान्मन्दिरं स्वकम्}
{पितरावपि संहृष्टौ पुत्रस्नेहाद्बभूवतुः}% २६

\twolineshloka
{तं दृष्ट्वा रावणो धीमाञ्जगाद निजमातरम्}
{कोऽयं पुमान्सुरो वाथ यक्षो वाथ नरोत्तमः}% २७

\twolineshloka
{योऽसौ मम पितुःपादौ सन्निषेव्य गतः पुनः}
{महाभाग्यनिधिः स्वीयैर्गणैः सुपरिवारितः}% २८

\twolineshloka
{केनेदं तपसा लब्धं विमानं वायुवेगधृक्}
{उद्यानारामलीलादि विलासस्थानमुत्तमम्}% २९

\uvacha{शेष उवाच}

\twolineshloka
{इति वाक्यं समाकर्ण्य जननी रोषविक्लवा}
{उवाच पुत्रं विमनाः किञ्चिन्नेत्रविकारिणी}% ३०

\twolineshloka
{रे पुत्र शृणु मद्वाक्यं बहुशिक्षासमन्वितम्}
{एतस्य जन्मकर्मादि विचारचतुराधिकम्}% ३१

\twolineshloka
{सपत्न्या मम कुक्षिस्थं विधानं समुपस्थितम्}
{येन स्वमातुर्विमलं कुलमुज्ज्वलितं महत्}% ३२

\twolineshloka
{त्वं तु मत्कुक्षिजः कीटः पापः स्वोदरपूरकः}
{यथा खरः स्वकं भारं जानाति न च तद्गुणम्}% ३३

\twolineshloka
{तथा त्वं लक्ष्यसेऽज्ञानी शयनासनभोगवान्}
{सुप्तो गतः क्वचिद्भ्रष्ट इत्येव तव सम्भवः}% ३४

\twolineshloka
{अनेन तपसा लब्धं शिवसन्तोषकारिणा}
{लङ्कावासो मनोवेगं विमानं राज्यसम्पदः}% ३५

\twolineshloka
{सुधन्या जननी त्वस्य सुभाग्या सुमहोदया}
{यस्याः पुत्रो निजगुणैर्लब्धवान्महतां पदम्}% ३६

\twolineshloka
{इति क्रुधा भाषितमार्तया तया मात्रा स्वयाऽकर्ण्य दुरात्मसत्तमः}
{रोषं विधायात्मगतं पुनर्वचो जगाद तां निश्चयभृत्तपः प्रति}% ३७

\uvacha{रावण उवाच}

\twolineshloka
{जनन्याकर्णय वचो मम गर्वसमन्वितम्}
{रत्नगर्भा त्वमेवासि यस्याः पुत्रास्त्रयो वयम्}% ३८

\twolineshloka
{कोऽसौ कीटः स धनदः क्व तपः स्वल्पकं पुनः}
{कालं का किन्तु तद्राज्यं स्वल्पसेवकसंयुतम्}% ३९

\twolineshloka
{मातः शृणु ममोत्साहात्प्रतिज्ञां करुणान्विते}
{न केनापि कृतां कर्त्रा महाभाग्ये हि कैकसि}% ४०

\twolineshloka
{यद्यहं भुवनं सर्वं वशेन स्थापयामि वै}
{तपोभिर्दुष्कृतैः कृत्वा ब्रह्मसन्तोषकारकैः}% ४१

\twolineshloka
{अन्नोदके सदा त्यक्त्वा निद्रां क्रीडां तथा पुनः}
{चेत्तदा पितृलोकस्य घातात्पापं भवेन्मम}% ४२

\twolineshloka
{कुम्भकर्णोऽपि कृतवान्विभीषणसमन्वितः}
{रावणेन सहभ्रात्रेत्युक्त्वागाद्गिरिकाननम्}% ४३

{॥इति श्रीपद्मपुराणे पातालखण्डे शेषवात्स्यायनसंवादे रामाश्वमेधे रावणोत्पत्तिर्नाम षष्ठोऽध्यायः॥६॥}

\dnsub{सप्तमोऽध्यायः}\resetShloka

\uvacha{अगस्त्य उवाच}

\twolineshloka
{अथोग्रं स तपो दैत्यो दशवर्षसहस्रकम्}
{चकार भानुमक्ष्णा च पश्यन्नूर्ध्वपदे स्थितः}% १

\twolineshloka
{कुम्भकर्णोऽपि कृतवांस्तपः परमदुश्चरम्}
{विभीषणस्तु धर्मात्मा चचार परमं तपः}% २

\twolineshloka
{तदा प्रसन्नो भगवान्देवदेवः प्रजापतिः}
{देवदानवयक्षादि मुकुटैः परिसेवितः}% ३

\twolineshloka
{ददौ राज्यं च सुमहद्भुवनत्रयभास्वरम्}
{वपुश्च कृतवान्रम्यं देवदानवसेवितम्}% ४

\twolineshloka
{तदा सन्तापितो भ्राता धनदो धर्मबुद्धिमान्}
{विमानं तु ततो नीतं लङ्का च नगरी हठात्}% ५

\twolineshloka
{भुवनं तापितं सर्वं देवाश्चैव दिवो गताः}
{हतवान्ब्राह्मणकुलं मुनीनां मूलकृन्तनः}% ६

\twolineshloka
{तदातिदुःखिता देवाः सेन्द्रा ब्रह्माणमाययुः}
{स्तुतिं चक्रुर्महात्मानो दण्डवत्प्रणतिं गताः}% ७

\twolineshloka
{ते तुष्टुवुः सुराः सर्वे वाग्भिरर्थ्याभिरादृताः}
{ततः प्रसन्नो भगवान्किङ्करोमीति चाब्रवीत्}% ८

\twolineshloka
{ततो निवेदयाञ्चक्रुर्ब्रह्मणे विबुधाः पुरः}
{दशग्रीवाच्च सङ्कष्टं तथा निजपराभवम्}% ९

\twolineshloka
{क्षणं ध्यात्वा ययौ ब्रह्मा कैलासं त्रिदशैः सह}
{तस्य शैलस्य पार्श्वे तु वैचित्र्येण समाकुलाः}% १०

\twolineshloka
{स्थिताः सन्तुष्टुवुर्देवाः शम्भुं शक्रपुरोगमाः}
{नमो भवाय शर्वाय नीलग्रीवाय ते नमः}% ११

\twolineshloka
{नमः स्थूलाय सूक्ष्माय बहुरूपाय ते नमः}
{इति सर्वमुखेनोक्तां वाणीमाकर्ण्य शङ्करः}% १२

\twolineshloka
{प्रोवाच नन्दिनं देवा नानयेति ममान्तिकम्}
{एतस्मिन्नन्तरे देवा आहूता नन्दिना च ते}% १३

\twolineshloka
{प्रविश्यान्तःपुरे देवा ददृशुर्विस्मितेक्षणाः}
{ब्रह्मागत्य ददर्शाथ शङ्करं लोकशङ्करम्}% १४

\twolineshloka
{गणकोटिसहस्रैस्तु सेवितं मोदशालिभिः}
{नग्नैर्विरूपैः कुटिलैर्धूसरैर्विकटैस्तथा}% १५

\twolineshloka
{प्रणिपत्याग्रतः स्थित्वा सह देवैः पितामहः}
{उवाच देवदेवेशं पश्यावस्थां दिवौकसाम्}% १६

\twolineshloka
{कृपां कुरु महादेव शरणागतवत्सल}
{दुष्टदैत्यवधार्थं त्वं समुद्योगं विधेहि भोः}% १७

\twolineshloka
{सोऽपि तद्वचनं श्रुत्वा दैन्यशोकसमन्वितम्}
{त्रिदशैः सहितैः सर्वैराजगाम हरेः पदम्}% १८

\twolineshloka
{तुष्टुवुर्मुनयः सर्वे ससुरोरगकिन्नराः}
{जय माधव देवेश जय भक्तजनार्तिहन्}% १९

\twolineshloka
{विलोकय महादेव लोकयस्व स्वसेवकान्}
{इत्युच्चैर्जगदुः सर्वे देवाः शर्वपुरोगमाः}% २०

\twolineshloka
{इत्युक्तमाकर्ण्य सुराधिनाथो दृष्ट्वा सुरार्तिं परिचिन्त्य विष्णुः}
{जगाद देवाञ्जलदोच्चया गिरा दुःखं तु तेषां प्रशमं नयन्निव}% २१

\twolineshloka
{भो ब्रह्मशर्वेन्द्र पुरोगमामराः शृण्वन्तु वाचं भवतां हितेरताम्}
{जाने दशग्रीवकृतं भयं वस्तन्नाशयाम्यद्य कृतावतारः}% २२

\twolineshloka
{पुरी त्वयोध्या रविवंशजातैर्नृपैर्महादानमखादिसत्क्रियैः}
{प्रपालिता भूतलमण्डनीया विराजते राजतभूमिभागैः}% २३

\twolineshloka
{तस्यां दशरथो राजा निरपत्यः श्रियान्वितः}
{पालयत्यधुना राज्यं दिक्चक्रजयवान्विभुः}% २४

\twolineshloka
{स तु वन्द्यादृष्यशृङ्गात्प्रार्थितात्पुत्रकाम्यया}
{पुत्रेष्ट्यां विधिना यज्वा महाबलसमन्वितः}% २५

\twolineshloka
{ततोऽहं प्रार्थितः पूर्वं तपसा तेन भोः सुराः}
{पत्नीषु तिसृषु प्रीत्या चतुर्धापि भवत्कृते}% २६

\twolineshloka
{राम लक्ष्मण शत्रुघ्न भरताख्या समन्वितः}
{कर्तास्मि रावणोद्धारं समूल बलवाहनम्}% २७

\twolineshloka
{भवन्तोऽपि स्वकैरंशैरवतीर्य चरन्त्विह}
{ऋक्षवानररूपेण सर्वत्र पृथिवीतले}% २८

\twolineshloka
{इत्युक्त्वा विररामाशु नभसीरितवाङ्मुने}
{देवाः श्रुत्वा महद्वाक्यं सर्वे संहृष्टमानसाः}% २९

\twolineshloka
{ते चक्रुर्गदितं यादृग्देवदेवेन धीमता}
{स्वैःस्वैरंशैर्मही पूर्णा ऋक्षवानररूपिभिः}% ३०

\twolineshloka
{योऽसौ विष्णुर्महादेवो देवानां दुःखनाशकः}
{सत्वमेव महाराज भगवान्कृतविग्रहः}% ३१

\twolineshloka
{भरतोऽयं लक्ष्मणश्च शत्रुघ्नश्च महामते}
{तावकांशाद्दशग्रीवो जनितश्च सुरार्द्दनः}% ३२

\twolineshloka
{पूर्ववैरानुबन्धेन जानकीं हृतवान्पुनः}
{स त्वया निहतो दैत्यो ब्रह्मराक्षसजातिमान्}% ३३

\twolineshloka
{पुलस्त्यपुत्रो दैत्येन्द्र सर्वलोकैककण्टकः}
{पातितः पृथिवी सर्वा सुखमापमहेश्वर}% ३४

\twolineshloka
{ब्राह्मणानां सुखं त्वद्य मुनीनां तापसं बलम्}
{शिवानि सर्वतीर्थानि सर्वे यज्ञाः सुसंहिताः}% ३५

\twolineshloka
{त्वयि राज्ञि जगत्सर्वं सदेवासुरमानुषम्}
{सुखं प्रपेदे विश्वात्मञ्जगद्योने नरोत्तम}% ३६

\twolineshloka
{एतत्ते सर्वमाख्यातं यत्पृष्टोऽहं त्वयानघ}
{उत्पत्तिश्च विपत्तिश्च मया मत्यनुसारतः}% ३७

\twolineshloka
{इत्थं निशम्य दितिजेन्द्रकुलानुकारिवार्तां महापुरुष ईश्वरईशिता च}
{संरुद्धबाष्पगलदश्रुमुखारविन्दो भूमौ पपात सदसि प्रथितप्रभावः}% ३८

{॥इति श्रीपद्मपुराणे पातालखण्डे शेषवात्स्यायनसंवादे रामाश्वमेधे रावणोत्पत्तिविपत्तिकथनन्नामसप्तमोऽध्यायः॥७॥}

\dnsub{अष्टमोऽध्यायः}\resetShloka

\uvacha{शेष उवाच}

\twolineshloka
{वात्स्यायनमुनिश्रेष्ठ कथा पापप्रणाशिनी}
{ब्रह्मण्यदेवदेवस्य सर्वधर्मैकरक्षितुः}% १

\twolineshloka
{राजानं मूर्च्छितं दृष्ट्वा कुम्भजन्मा तपोनिधिः}
{शनैःशनैः करेणाशु पस्पर्शाश्रु जगाद च}% २

\twolineshloka
{भो रामाश्वसिहि क्षिप्रं किमर्थमवसीदसि}
{भवान्दैत्यकुलच्छेत्ता महाविष्णुः सनातनः}% ३

\twolineshloka
{भूतं भव्यं भवच्चैव जगत्स्थास्नु चरिष्णु च}
{त्वदृते नास्ति सञ्चारी किमर्थमिह मूर्च्छितः}% ४

\twolineshloka
{श्रुत्वा वाक्यं महाराजः कुम्भजन्मसमीरितम्}
{उत्तस्थौ विगलन्नेत्र बाष्पपूरितसन्मुखः}% ५

\twolineshloka
{उवाच दीनदीनं च विस्पष्टाक्षरविस्तरम्}
{त्रपाभर नमन्मूर्तिर्ब्रह्मद्रोहपराङ्मुखः}% ६

\uvacha{श्रीराम उवाच}

\twolineshloka
{अहो मे पश्यता ज्ञानं विमूढस्य दुरात्मनः}
{यद्ब्राह्मणकुले रूढं हतवान्कामलोलुपः}% ७

\twolineshloka
{महिलार्थे त्वहं विप्रं वेदशास्त्रविवेकवान्}
{हतवान्वाडवकुलं बुद्धिहीनोति दुर्मतिः}% ८

\twolineshloka
{इक्ष्वाकूणां कुले जातु ब्राह्मणो न दुरुक्तिभाक्}
{ईदृशं कुर्वता कर्म मयैतत्सुकलङ्कितम्}% ९

\twolineshloka
{ये ब्राह्मणास्तु पूजार्हा दानसम्मानभोजनैः}
{ते मया निहता विप्राः शरसङ्घातसंहितैः}% १०

\twolineshloka
{काँल्लोकान्नु गमिष्यामि कुम्भीपाकोऽपि दुःसहः}
{न तादृशं तीर्थमस्ति यन्मां पावयितुं क्षमम्}% ११

\twolineshloka
{न यज्ञो न तपो दानं न वा चैव व्रतादिकम्}
{यत्तु वै ब्राह्मणद्रोग्धुर्ममपावनतारकम्}% १२

\twolineshloka
{यैः कोपितं ब्रह्मकुलं नरैर्निरयगामिभिः}
{ते नरा बहुशो दुःखं भोक्ष्यन्ति निरयं गताः}% १३

\twolineshloka
{वेदा मूलं तु धर्माणां वर्णाश्रमविवेकिनाम्}
{तन्मूलं ब्राह्मणकुलं सर्ववेदैकशाखिनः}% १४

\twolineshloka
{मूलच्छेत्तुर्ममौद्धत्यात्को लोकोनु भविष्यति}
{किमद्यकरणीयं वै येन मे हि शिवं भवेत्}% १५

\uvacha{शेष उवाच}

\twolineshloka
{विलपन्तं भृशं रामं राजेन्द्रं रघुपुङ्गवम्}
{मायामनुष्यवपुषं कुम्भजन्माब्रवीद्वचः}% १६

\uvacha{अगस्त्य उवाच}

\twolineshloka
{मा विषादं महाधीर कुरु राजन्महामते}
{न ते ब्राह्मणहत्या स्याद्दुष्टानां नाशमिच्छतः}% १७

\twolineshloka
{त्वं पुराणः पुमान्साक्षादीश्वरः प्रकृतेः परः}
{कर्ता हर्ताऽविता साक्षी निर्गुणः स्वेच्छया गुणी}% १८

\twolineshloka
{सुरापो ब्रह्महत्याकृत्स्वर्णस्तेयी महाघकृत्}
{सर्वे त्वन्नामवादेन पूताः शीघ्रं भवन्ति हि}% १९

\twolineshloka
{इयं देवी जनकजा महाविद्या महामते}
{यस्याः स्मरणमात्रेण मुक्ता यास्यन्ति सद्गतिम्}% २०

\twolineshloka
{रावणोऽपि न वै दैत्यो वैकुण्ठे तव सेवकः}
{ऋषीणां शापतोऽवाप्तो दैत्यत्वं दनुजान्तक}% २१

\twolineshloka
{तस्यानुग्रहकर्ता त्वं न तु हन्ता द्विजन्मनः}
{एवं सञ्चिन्त्य मा भूयो निजं शोचितुमर्हसि}% २२

\twolineshloka
{इति श्रुत्वा ततो वाक्यं रामः परपुरञ्जयः}
{उवाच मधुरं वाक्यं गद्गदस्वरभाषितम्}% २३

\uvacha{श्रीराम उवाच}

\twolineshloka
{पातकं द्विविधं प्रोक्तं ज्ञाताज्ञातविभेदतः}
{ज्ञातं यद्बुद्धिपूर्वं हि अज्ञातं तद्विवर्जितम्}% २४

\twolineshloka
{बुद्धिपूर्वं कृतं कर्म भोगेनैव विनश्यति}
{नश्येदनुशयादन्यदिदं शास्त्रविनिश्चितम्}% २५

\twolineshloka
{कुर्वतो बुद्धिपूर्वं मे ब्रह्महत्यां सुनिन्दिताम्}
{न मे दुःखापनोदाय साधुवादः सुसम्मतः}% २६

\twolineshloka
{प्रब्रूहि तादृशं मह्यं यादृशं पापदाहकम्}
{व्रतं दानं मखं किञ्चित्तीर्थमाराधनं महत्}% २७

\twolineshloka
{येन मे विमला कीर्तिर्लोकान्वै पावयिष्यति}
{पापाचाराप्तकालुष्यान्ब्रह्महत्याहतप्रभान्}% २८

\uvacha{शेष उवाच}

\twolineshloka
{इत्युक्तवन्तं तं रामं जगाद स तपोनिधिः}
{सुरासुरनमन्मौलि मणिनीराजिताङ्घ्रिकम्}% २९

\twolineshloka
{शृणु राम महावीर लोकानुग्रहकारक}
{विप्रहत्यापनोदाय तव यद्वचनं ब्रुवे}% ३०

\twolineshloka
{सर्वं स पापं तरति योऽश्वमेधं यजेत वै}
{तस्मात्त्वं यज विश्वात्मन्वाजिमेधेन शोभिना}% ३१

\twolineshloka
{सप्ततन्तुर्महीभर्त्रा त्वया साध्यो मनीषिणा}
{महासमृद्धियुक्तेन महाबलसुशालिना}% ३२

\twolineshloka
{स वाजिमेधो विप्राणां हत्यायाः पापनोदनः}
{कृतवान्यं महाराजो दिलीपस्तव पूर्वजः}% ३३

\twolineshloka
{शतक्रतुः शतं कृत्वा क्रतूनां पुरुषर्षभः}
{पदमापामरावत्यां देवदैत्यसुसेवितम्}% ३४

\twolineshloka
{मनुश्च सगरो राजा मरुत्तो नहुषात्मजः}
{एते ते पूर्वजाः सर्वे यज्ञं कृत्वा पदं गताः}% ३५

\twolineshloka
{तस्मात्त्वं कुरु राजेन्द्र समर्थोऽसि समन्ततः}
{भ्रातरो लोकपालाभा वर्तन्ते तव भावुकाः}% ३६

\twolineshloka
{इत्युक्तमाकर्ण्य मुनेः स भाग्यवान् रघूत्तमो ब्राह्मणघातभीतः}
{पप्रच्छ यागे सुमतिं चिकीर्षन्विधिं पुरावित्परिगीयमानः}% ३७

{॥इति श्रीपद्मपुराणे पातालखण्डे शेषवात्स्यायनसंवादे रामाश्वमेधे रघुनाथस्यागस्त्योपदेशोनामाष्टमोऽध्यायः॥८॥}

\dnsub{नवमोऽध्यायः}\resetShloka

\uvacha{श्रीराम उवाच}

\twolineshloka
{कीदृशोऽश्वस्तत्र भाव्यः को विधिस्तत्र पूजने}
{कथं वा शक्यते कर्तुं के जेयास्तत्र वैरिणः}% १

\uvacha{अगस्त्य उवाच}

\twolineshloka
{गङ्गाजलसमानेन वर्णेन वपुषा शुभः}
{कर्णे श्यामो मुखे रक्तः पीतः पुच्छे सुलक्षितः}% २

\twolineshloka
{मनोवेगः सर्वगतिरुच्चैःश्रवस्समप्रभः}
{वाजिमेधे हयः प्रोक्तः शुभलक्षणलक्षितः}% ३

\twolineshloka
{वैशाखपूर्णमास्यां तु पूजयित्वा यथाविधि}
{पत्रं लिखित्वा भाले तु स्वनामबलचिह्नितम्}% ४

\twolineshloka
{मोचनीयः प्रयत्नेन रक्षकैः परिरक्षितः}
{यत्र गच्छति यज्ञाश्वस्तत्र गच्छेत्सुरक्षकः}% ५

\twolineshloka
{यस्तम्बलान्निबध्नाति स्ववीर्यबलदर्पितः}
{तस्मात्प्रसभमानेयः परिरक्षाकरैर्हयः}% ६

\twolineshloka
{कर्त्रा तावत्सुविधिना स्थातव्यं नियमादिह}
{मृगशृङ्गधरो भूत्वा ब्रह्मचर्यसमन्वितः}% ७

\twolineshloka
{व्रतं पालयमानस्य यावद्वर्षमतिक्रमेत्}
{तावद्दीनान्धकृपणाः परितोष्या धनादिभिः}% ८

\twolineshloka
{अन्नं तु बहुशो देयं धनं वा भूरि मारिष}
{यद्यत्प्रार्थयते धीमांस्तत्तदेव ददाति हि}% ९

\twolineshloka
{एवं प्रकुर्वतः कर्म यज्ञः सम्पूर्णतां गतः}
{करोति सर्वपापानां नाशनं रिपुनाशन}% १०

\twolineshloka
{तस्माद्भवान्समर्थोऽस्ति करणे पालनेऽर्चने}
{कृत्वा कीर्तिं सुविमलां पावयान्याञ्जनान्नृप}% ११

\uvacha{श्रीराम उवाच}

\twolineshloka
{विलोकय द्विजश्रेष्ठ वाजिशालां ममाधुना}
{तादृशाः सन्ति नो वाश्वाः शुभलक्षणलक्षिताः}% १२

\twolineshloka
{इति श्रुत्वा तु तद्वाक्यमगस्त्यः करुणाकरः}
{उत्तस्थौ वीक्षमाणोऽयं यागार्हान्वाजिनः शुभान्}% १३

\twolineshloka
{गत्वाथ तत्र शालायां रामचन्द्रसमन्वितः}
{ददर्शाश्वान्विचित्राङ्गान्मनोवेगान्महाबलान्}% १४

\twolineshloka
{अवनितलगताः किं वाजिराजस्य वंश्याः किमथ रघुपतीनामेकतः कीर्तिपिण्डाः}
{किमिदममृतराशिर्वाहरूपेण सिन्धोर्मुनिरिति मनसोन्तर्विस्मयं प्राप पश्यन्}% १५

\twolineshloka
{एकतः शोणदेहानां वाजिनां पङ्क्तिरुत्तमा}
{एकतः श्यामकर्णाश्च कस्तूरीकान्तिसप्रभाः}% १६

\twolineshloka
{एकतः कनकाभाश्च त्वन्यतो नीलवर्णिनः}
{एकतः शबलैर्वर्णैर्विशिष्टैर्वाजिभिर्वृताः}% १७

\twolineshloka
{एवं पश्यन्मुनिः सर्वान्कौतुकाविष्टमानसः}
{ययावन्यत्र तान्द्रष्टुं यागयोग्यान्हयान्मुनिः}% १८

\twolineshloka
{ददर्श तत्र शतशो बद्धांस्तादृशवर्णकान्}
{दृष्ट्वा विस्मयमापेदे स मुनिर्हर्षिताङ्गकः}% १९

\twolineshloka
{एकतः श्यामकर्णांश्च सर्वाङ्गैः क्षीरसन्निभान्}
{पीतपुच्छान्मुखे रक्ताञ्छुभलक्षणलक्षितान्}% २०

\twolineshloka
{निरीक्ष्य परितोऽनघान्विमलनीरधारानिभान्मनोजवनशोभितान्विमलकीर्तिपुञ्जप्रभान्}
{पयोनिधिविशोषको मुनिरुवाचसीतापतिं विचित्रहयदर्शनाद्धृषितनेत्रवक्त्रप्रभः}% २१

\uvacha{अगस्त्य उवाच}

\twolineshloka
{हयमेधक्रतौ योग्यान्वाहांस्ते बहुशः शुभान्}
{पश्यतो नेत्रयोर्मेऽद्य तृप्तिर्नास्ति रघूत्तम}% २२

\twolineshloka
{रामचन्द्र महाभाग सुरासुरनमस्कृत}
{यज्ञं कुरु महाराज हयमेधं सुविस्तरम्}% २३

\twolineshloka
{सुरपतिरिव सर्वान्यज्ञसङ्घान्करिष्यंस्तपन इव सुपर्वारातितोयं विशोष्यन्}
{हतरिपुगणमुख्यं साम्परायं विजित्य क्षितितलसुखभोगं कुर्विदं भूरिभाग}% २४

\twolineshloka
{इत्येवं वाक्यवादेन परितुष्टाखिलेन्द्रियः}
{सर्वान्वै यज्ञसम्भारानाजहार मनोहरान्}% २५

\twolineshloka
{मुन्यन्वितो महाराजः सरयूतीरमागतः}
{सुवर्णलाङ्गलैर्भूमिं विचकर्ष महीयसीम्}% २६

\twolineshloka
{विलिख्य भूमिं बहुशश्चतुर्योजनसम्मिताम्}
{मण्डपान्रचयामास यज्ञार्थं स नरोत्तमः}% २७

\twolineshloka
{कुण्डं तु विधिवत्कृत्वा योनिमेखलयान्वितम्}
{अनेकरत्नरचितं सर्वशोभासमन्वितम्}% २८

\twolineshloka
{मुनीश्वरो महाभागो वसिष्ठः सुमहातपाः}
{सर्वं तत्कारयामास वेदशास्त्रविधिश्रितम्}% २९

\twolineshloka
{प्रेषितास्तेन मुनिना शिष्या मुनिवराश्रमान्}
{कथयामासुरुद्युक्तं हयमेधे रघूत्तमम्}% ३०

\twolineshloka
{आकारितास्तदा सर्वे ऋषयस्तपतां वराः}
{आजग्मुः परमेशस्य दर्शने त्वतिलालसाः}% ३१

\twolineshloka
{नारदोसितनामा च पर्वतः कपिलो मुनिः}
{जातूकर्ण्योऽङ्गिरा व्यास आर्ष्टिषेणोऽत्रिरासुरिः}% ३२

\twolineshloka
{हारीतो याज्ञवल्क्यश्च संवर्तः शुकसंज्ञितः}
{इत्येवमादयो राम हयमेधवरं ययुः}% ३३

\twolineshloka
{तान्सर्वान्पूजयामास रघुराजो महामनाः}
{प्रत्युत्थानाभिवादाभ्यामर्घ्यविष्टरकादिभिः}% ३४

\twolineshloka
{गां हिरण्यं ददौ तेभ्यः प्रायशो दृष्टविक्रमः}
{महद्भाग्यं त्वद्यमेऽस्ति यद्यूयं दर्शनं गताः}% ३५

\uvacha{शेष उवाच}

\twolineshloka
{एवं समाकुले ब्रह्मन्नृषिवर्य समागमे}
{धर्मवार्ता बभूवाहो वर्णाश्रमसुसम्मता}% ३६

\uvacha{वात्स्यायन उवाच}

\twolineshloka
{का धर्मवार्ता तत्रासीत्किं वा कथितमद्भुतम्}
{साधवः सर्वलोकानां कारुण्यात्किमुताब्रुवन्}% ३७

\uvacha{शेष उवाच}

\twolineshloka
{तान्समेतान्मुनीन्दृष्ट्वा रामो दाशरथिर्महान्}
{पप्रच्छ सर्वधर्मांश्च सर्ववर्णाश्रमोचितान्}% ३८

\twolineshloka
{ते तु पृष्टा हि रामेण धर्मान्प्रोचुर्महागुणान्}
{तान्प्रवक्ष्यामि ते सर्वान्यथाविधि शृणुष्व तान्}% ३९

\uvacha{ऋषय ऊचुः}

\twolineshloka
{ब्राह्मणेन सदा कार्यं यजनाध्ययनादिकम्}
{वेदान्पठित्वा विरजो नैव गार्हस्थ्यमाविशेत्}% ४०

\twolineshloka
{ब्राह्मणेन सदा त्याज्यं नीचसेवानुजीवनम्}
{आपद्गतोऽपि जीवेत न श्ववृत्त्या कदाचन}% ४१

\twolineshloka
{ऋतुकालाभिगमनं धर्मोऽयं गृहिणः परः}
{स्त्रीणां वरमनुस्मृत्याऽपत्यकामोथवा भवेत्}% ४२

\twolineshloka
{दिवाभिगमनं पुंसामनायुष्यकरं मतम्}
{श्राद्धाहः सर्वपर्वाणि यतस्त्याज्यानि धीमता}% ४३

\twolineshloka
{तत्र गच्छेत्स्त्रियं मोहाद्धर्मात्प्रच्यवते परात्}
{ऋतुकालाभिगामी यः स्वदारनिरतश्च यः}% ४४

\twolineshloka
{सर्वदा ब्रह्मचारी ह विज्ञेयः स गृहाश्रमी}
{ऋतुः षोडशयामिन्यश्चतस्रस्ता सुगर्हिताः}% ४५

\twolineshloka
{पुत्रदास्तासु या युग्मा अयुग्माः कन्यकाप्रदाः}
{त्यक्त्वा चन्द्रमसं दुष्टं मघां मूलं विहाय च}% ४६

\twolineshloka
{शुचिः सन्निर्विशेत्पत्नीं पुन्नामर्क्षे विशेषतः}
{शुचिं पुत्रं प्रसूयेत पुरुषार्थप्रसाधनम्}% ४७

\twolineshloka
{आर्षे विवाहे गोद्वन्द्वं यदुक्तं तत्प्रशस्यते}
{शुल्कमण्वपि कन्यायाः कन्याक्रेतुस्तु पापकृत्}% ४८

\twolineshloka
{वाणिज्यं नृपतेः सेवा वेदानध्ययनं तथा}
{कुविवाहः क्रियालोपः कुलपातनहेतवः}% ४९

\twolineshloka
{अन्नोदक पयो मूलफलैर्वापि गृहाश्रमी}
{गोदानेन तु यत्पुण्यं पात्राय विधिपूर्वकम्}% ५०

\twolineshloka
{अनर्चितोऽतिथिर्गेहाद्भग्नाशो यस्य गच्छति}
{आजन्मसञ्चितात्पुण्यात्क्षणात्स हि बहिर्भवेत्}% ५१

\twolineshloka
{पितृदेवमनुष्येभ्यो दत्त्वाश्नीतामृतं गृही}
{स्वार्थं पचत्यघं भुङ्क्ते केवलं स्वोदरम्भरिः}% ५२

\twolineshloka
{षष्ठ्यष्टम्योर्विशेत्पापं तैले मांसे सदैव हि}
{चतुर्दश्यां तथामायां त्यजेत क्षुरमङ्गनाम्}% ५३

\twolineshloka
{रजस्वलां न सेवेत नाश्नीयात्सह भार्यया}
{एकवासा न भुञ्जीत न भुञ्जीतोत्कटासने}% ५४

\twolineshloka
{नाश्नन्तीं स्त्रियमीक्षेत तेजःकामो नरोत्तमः}
{मुखेनोपधमेन्नाग्निं नग्नां नेक्षेत योषितम्}% ५५

\twolineshloka
{नाङ्घ्री प्रतापयेदग्नौ न वस्त्वशुचि निक्षिपेत्}
{प्राणिहिंसां न कुर्वीत नाश्नीयात्सन्ध्ययोर्द्वयोः}% ५६

\twolineshloka
{नाचक्षीत धयन्तीं गां नेन्द्रचापं प्रदर्शयेत्}
{न दिवोद्गतसारं च भक्षयेद्दधिनो निशि}% ५७

\twolineshloka
{स्त्रीं धर्मिणीं नाभिवादेन्नाद्यादातृप्ति रात्रिषु}
{तौर्यत्रिकप्रियो न स्यात्कांस्ये पादौ न धावयेत्}% ५८

\twolineshloka
{न धारयेदन्यभुक्तं वासश्चोपानहावपि}
{न भिन्नभाजनेऽश्नीयान्नाश्नीतान्नं विदूषितम्}% ५९

\twolineshloka
{संविशेन्नार्द्रचरणो नोच्छिष्टः क्वचिदाव्रजेत्}
{शयानो वा न चाश्नीयान्नोच्छिष्टः संस्पृशेच्छिरः}% ६०

\twolineshloka
{न मनुष्यस्तुतिं कुर्यान्नात्मानमवमानयेत्}
{अभ्युद्यतं न प्रणमेत्परमर्माणि नो वदेत्}% ६१

\twolineshloka
{एवं गार्हस्थ्यमाश्रित्य वानप्रस्थाश्रमं व्रजेत्}
{सस्त्रीको वा गतस्त्रीको विरज्येत ततः परम्}% ६२

\twolineshloka
{इत्येवमादयो धर्मा गदिता ऋषिभिस्तदा}
{श्रुता रामेण महता सर्वलोकहितैषिणा}% ६३

{॥इति श्रीपद्मपुराणे पातालखण्डे शेषवात्स्यायनसंवादे रामाश्वमेधे सर्वधर्मोपदेशो नाम नवमोऽध्यायः॥९॥}

\dnsub{दशमोऽध्यायः}\resetShloka

\uvacha{शेष उवाच}

\twolineshloka
{इत्थं संशृण्वतो धर्मान्वसन्तः समुपस्थितः}
{यत्र यज्ञ क्रियादीनां प्रारम्भः सुमहात्मनाम्}% १

\twolineshloka
{दृष्ट्वा तं समयं धीमान्वसिष्ठः कलशोद्भवः}
{रामचन्द्रं महाराजं प्रत्युवाच यथोचितम्}% २

\uvacha{वसिष्ठ उवाच}

\twolineshloka
{रामचन्द्र महाबाहो समयः पर्यभूत्तव}
{हयो यत्र प्रमुच्येत यज्ञार्थं परिपूजितः}% ३

\twolineshloka
{सामग्री क्रियतां तत्र आहूयन्तां द्विजोत्तमाः}
{करोतु पूजां भगवान्ब्राह्मणानां यथोचिताम्}% ४

\twolineshloka
{दीनान्धकृपणानां च दानं स्वान्ते समुत्थितम्}
{ददातु विधिवत्तेषां प्रतिपूज्याधिमान्य च}% ५

\twolineshloka
{भवान्कनकसत्पत्न्या दीक्षितोऽत्र व्रतं चर}
{भूमिशायी ब्रह्मचारी वसुभोगविवर्जितः}% ६

\twolineshloka
{मृगशृङ्गधरः कट्यां मेखलाजिनदण्डभृत्}
{करोतु यज्ञसम्भारं सर्वद्रव्यसमन्वितम्}% ७

\twolineshloka
{इति श्रुत्वा महद्वाक्यं वसिष्ठस्य यथार्थकम्}
{उवाच लक्ष्मणं धीमान्नानार्थपरिबृंहितम्}% ८

\uvacha{श्रीराम उवाच}

\twolineshloka
{शृणु लक्ष्मण मद्वाक्यं श्रुत्वा तत्कुरु सत्वरम्}
{हयमानय यत्नेन वाजिमेधक्रियोचितम्}% ९

\uvacha{शेष उवाच}

\twolineshloka
{श्रुत्वा वाक्यं रघुपतेः शत्रुजिल्लक्ष्मणस्तदा}
{सेनापतिमुवाचेदं वचो विविधवर्णनम्}% १०

\uvacha{लक्ष्मण उवाच}

\fourlineindentedshloka
{वीराकर्णय मे वचः सुमधुरं श्रुत्वा त्वरातः पुनः}
{कार्यं तत्क्षितिपालमौलिमुकुटैर्घृष्टाङ्घ्रि रामाज्ञया}
{सेनां कालबलप्रभञ्जनबलप्रोद्यत्समर्थाङ्गिनीं}
{सज्जां सद्रथहस्तिपत्तिसुहयारोहैर्विधे ह्यन्विताम्}% ११

\twolineshloka
{सज्जीयतां वायुजवास्तुरङ्गास्तरङ्गमाला ललिताङ्घ्रिपाताः}
{सदश्वचारैर्बहुशस्त्रधारिभिः संरोहिता वैरिबलप्रहारिभिः}% १२

\twolineshloka
{संलक्ष्यतां हस्तिनः पर्वताभा आधोरणैः प्रासकुन्ताग्रहस्तैः}
{शूरैः सास्त्रैर्भूरिदानोपहाराः क्षीबाणस्ते सर्वशस्त्रास्त्रपूर्णाः}% १३

\twolineshloka
{विततबहुसमृद्धिभ्राजमाना रथा मे पवनजवनवेगैर्वाजिभिर्युज्यमानाः}
{विविधरिपुविनाशस्मारकैरायुधास्त्रैर्भृतवलभिविभागानीयतां सूतवृन्दैः}% १४

\twolineshloka
{पत्तयः शतशो मह्यमायान्त्वस्त्राग्न्यपाणयः}
{हयमेधार्हवाहस्य रक्षणे विततोद्यमाः}% १५

\twolineshloka
{इत्याकर्ण्य वचस्तस्य लक्ष्मणस्य महात्मनः}
{सेनानी कालजिन्नामा कारयामास सज्जताम्}% १६

\twolineshloka
{दशध्रुवकमण्डितो लघुसुरोमशोभान्वितो विविक्तगलशुक्तिभृद्विततकण्ठको शेमणिः मुखे}
{विशदकान्तिधृत्त्वसितकान्तिभृत्कर्णयोर्व्यराजत तदाह यो धृतकराग्ररश्मिच्छटः}% १७

\twolineshloka
{कलासंशोभितमुखः स्फुरद्रत्नविशोभितः}
{मुक्ताफलानां मालाभिः शोभितो निर्ययौ हयः}% १८

\twolineshloka
{श्वेतातपत्ररचितः सितचामरशोभितः}
{बहुशोभापरीताङ्गो निर्ययौ हयराट्ततः}% १९

\twolineshloka
{अग्रतो मध्यतश्चैके पृष्ठतः सैनिकास्तथा}
{देवा हरिं यथापूर्वं सेवन्ते सेवनोचितम्}% २०

\twolineshloka
{अथ सैन्यं समाहूय सर्वमाज्ञापयत्तदा}
{हस्त्यश्वरथपादातवृन्दैः सुबहुसङ्कुलम्}% २१

\twolineshloka
{ततस्ततः समेतानां सैन्यानां श्रूयते ध्वनिः}
{ततो दुन्दुभिनादोऽभूत्तस्मिन्पुरवरे तदा}% २२

\twolineshloka
{तन्निनादेन शूराणां प्रियेण महता तदा}
{कम्पन्ति गिरिशृङ्गाणि प्रासादा विचलन्ति च}% २३

\twolineshloka
{हेषारवो महानासीद्वाजिनां मुह्यतां नृप}
{रथाङ्गघातसङ्घुष्टा धरा सञ्चलतीव सा}% २४

\twolineshloka
{चलितैर्गजयूथैश्च पृथ्वी रुद्धा समन्ततः}
{रजस्तु प्रचलत्तत्र जनान्तर्द्धानमादधात्}% २५

\twolineshloka
{निर्जगाम महासैन्यं छत्रैः सञ्छाद्य भास्करम्}
{सेनान्याकालजिन्नाम्ना प्रेरितं जनसङ्कुलम्}% २६

\twolineshloka
{गर्जन्तस्तलवीराग्र्याः कुर्वन्तो रणसम्भ्रमम्}
{रघुनाथस्य यागाय सज्जास्ते प्रययुर्मुदा}% २७

\twolineshloka
{मृगमदमयमङ्गेष्वङ्गरागं दधानाः कुसुमविमलमालाशोभितस्वोत्तमाङ्गाः}
{मुकुटकटकभूषाभूषिताङ्गाः समस्ताः प्रययुरवनिनाथप्रेरितास्तेऽपि सर्वे}% २८

\twolineshloka
{इत्येवं ते महाराजं ययुः सेनाचरा वराः}
{धनुर्धराः पाशधराः खड्गधाराः स्फुटक्रमाः}% २९

\twolineshloka
{एवं शनैःशनैः प्राप्तो मण्डपं यागचिह्नितम्}
{हयः खुरक्षततलां भूमिं कुर्वन्नभः प्लवन्}% ३०

\twolineshloka
{रामो दृष्ट्वा हरिं प्राप्तं बहुसन्तुष्टमानसः}
{वसिष्ठं प्रेरयामास क्रियाकर्तव्यतां प्रति}% ३१

\twolineshloka
{वसिष्ठो राममाहूय स्वर्णपत्नीसमन्वितम्}
{प्रयोगं कारयामास ब्रह्महत्यापनोदनम्}% ३२

\twolineshloka
{ब्रह्मचर्यव्रतधरो मृगशृङ्गपरिग्रहः}
{तत्कर्म कारयामास रामः परपुरञ्जयः}% ३३

\twolineshloka
{प्रारेभे यागकर्मार्थं कुण्डं मण्डपसम्मितम्}
{तत्राचार्योभवद्धीमान्वेदशास्त्रविचारवित्}% ३४

\twolineshloka
{वसिष्ठो रघुनाथस्य कुलपूर्वगुरुर्मुनिः}
{ब्रह्मंस्तत्राचरद्ब्रह्मकर्मागस्त्यस्तपोनिधिः}% ३५

\twolineshloka
{वाल्मीकिर्मुनिरध्वर्युर्मुनिः कण्वस्तु द्वारपः}
{अष्टौ द्वाराणि तत्रासन्सतोरण शुभानि वै}% ३६

\twolineshloka
{द्वारि द्वारि द्वयं विप्र ब्राह्मणस्याधिमन्त्रवित्}
{पूर्वद्वारि मुनिश्रेष्ठौ देवलासित संज्ञितौ}% ३७

\twolineshloka
{दक्षिणद्वारि भूमानौ कश्यपात्री तपोनिधी}
{पश्चिमद्वारि ऋषभौ जातूकर्ण्योऽथ जाजलिः}% ३८

\twolineshloka
{उत्तरद्वारि तु मुनी द्वौ द्वितैकत तापसौ}
{एवं द्वारविधिं कृत्वा वसिष्ठः कलशोद्भवः}% ३९

\twolineshloka
{हयवर्यस्य सत्पूजां कर्तुमारभत द्विज}
{सुवासिन्यः स्त्रियस्तत्र वासोलङ्कारभूषिताः}% ४०

\twolineshloka
{हरिद्राक्षतगन्धाद्यैः पूजयामासुरर्चितम्}
{नीराजनं ततः कृत्वा धूपयित्वागुरूक्षणैः}% ४१

\twolineshloka
{वर्धापनं ततो वेश्याश्चक्रुस्ता वाडवाज्ञया}
{एवं सम्पूज्य विमले भाले चन्दनचर्चिते}% ४२

\twolineshloka
{कुङ्कुमादिकगन्धाढ्ये सर्वशोभासमन्विते}
{बबन्ध भास्वरं पत्रं तप्तहाटकनिर्मितम्}% ४३

\twolineshloka
{तत्रालिखद्दाशरथेः प्रतापबलमूर्जितम्}
{सूर्यवंशध्वजो धन्वी धनुर्दीक्षा गुरुर्गुरुः}% ४४

\twolineshloka
{यं देवाः सासुराः सर्वे नमन्ति मणिमौलिभिः}
{तस्यात्मजो वीरबलदर्पहारी रघूद्वहः}% ४५

\twolineshloka
{रामचन्द्रो महाभागः सर्वशूरशिरोमणिः}
{तन्माता कोसलनृपपत्नीगर्भसमुद्भवा}% ४६

\twolineshloka
{तस्याः कुक्षिभवं रत्नं रामः शत्रुक्षयङ्करः}
{करोति हयमेधं वै ब्राह्मणेन सुशिक्षितः}% ४७

\twolineshloka
{रावणाभिधविप्रेन्द्र वधपापापनुत्तये}
{मोचितस्तेन वाहानां मुख्योऽसौ वाजिनां वरः}% ४८

\twolineshloka
{महाबलपरीवार परिखाभिः सुरक्षितः}
{तद्रक्षकोऽस्ति तद्भ्राता शत्रुघ्नो लवणान्तकः}% ४९

\twolineshloka
{हस्त्यश्वरथपादात सेनासङ्घसमन्वितः}
{यस्य राज्ञ इति श्रेष्ठो मानः स्यात्स्वबलोन्मदात्}% ५०

\twolineshloka
{वयं धनुर्धराः शूराः श्रेष्ठा वयमिहोत्कटाः}
{ते गृह्णन्तु बलाद्वाहं रत्नमालाविभूषितम्}% ५१

\twolineshloka
{मनोवेगं कामजवं सर्वगत्यधिभास्वरम्}
{ततो मोचयिता भ्राता शत्रुघ्नो लीलया हयम्}% ५२

\onelineshloka
{शरासनविनिर्मुक्त वत्सदन्तैः शिखाशितैः}% ५३

\fourlineindentedshloka
{इत्येवमादि विलिलेख महामुनीन्द्रः}
{श्रीरामचन्द्र भुजवीर्यलसत्प्रतापम्}
{शोभानिधानमतिचञ्चलवायुवेगं}
{पातालभूतलविशेषगतिं मुमोच}% ५४

\twolineshloka
{शत्रुघ्नमादिदेशाथ रामः शस्त्रभृतां वरः}
{याहि वाहस्य रक्षार्थं पृष्ठतः स्वैरगामिनः}% ५५

\twolineshloka
{शत्रुघ्न गच्छ वाहस्य मार्गं भद्रं भवेत्तव}
{भवेतां शत्रुविजयौ रिपुकर्षण ते भुजौ}% ५६

\fourlineindentedshloka
{ये योद्धारः प्रतिरणगतास्ते त्वया वारणीया-}
{वाहं रक्ष स्वकगुणगणैः संयुतः सन्महोर्व्याम्}
{सुप्तान्भ्रष्टान्विगतवसनान्भीतभीतांस्तु नम्रां-}
{स्तान्मा हन्याः सुकृतकृतिनो येन शंसन्ति कर्म}% ५७

\twolineshloka
{विरथा भयसन्त्रस्ता ये वदन्ति वयं तव}
{ते त्वया न हि हन्तव्याः शत्रुघ्न सुकृतैषिणा}% ५८

\twolineshloka
{यो हन्याद्विमदं मत्तं सुप्तं मग्नं भयातुरम्}
{तावकोऽहं ब्रुवाणं च स व्रजत्यधमां गतिम्}% ५९

\twolineshloka
{परस्वे चित्तवृत्तिं त्वं मा कृथाः पारदारिके}
{नीचे रतिं न कुर्वीथाः सर्वसद्गुणपूरितः}% ६०

\twolineshloka
{वृद्धानां प्रेरणं पूर्वं मा कुर्वीथा रणं जय}
{पूज्यपूजातिक्रमं त्वं मा विधेहि दयान्वितः}% ६१

\twolineshloka
{गां विप्रं च नमस्कुर्या वैष्णवं धर्मसंयुतम्}
{अभिवाद्य यतो गच्छेस्तत्र सिद्धिमवाप्नुयाः}% ६२

\twolineshloka
{विष्णुः सर्वेश्वरः साक्षी सर्वव्यापकदेहभृत्}
{ये तदीया महाबाहो तद्रूपा विचरन्ति हि}% ६३

\twolineshloka
{ये स्मरन्ति महाविष्णुं सर्वभूतहृदि स्थितम्}
{ते मन्तव्या महाविष्णु समरूपा रघूद्वह}% ६४

\twolineshloka
{यस्य स्वीयो न पारक्यो यस्य मित्रसमो रिपुः}
{ते वैष्णवाः क्षणादेव पापिनं पावयन्ति हि}% ६५

\twolineshloka
{येषां प्रियं भागवतं येषां वै ब्राह्मणाः प्रियाः}
{वैकुण्ठात्प्रेषितास्तेऽत्र लोकपावनहेतवे}% ६६

\twolineshloka
{येषां वक्त्रे हरेर्नाम हृदि विष्णुः सनातनः}
{उदरे विष्णुनैवेद्यः स श्वपाकोऽपि वैष्णवः}% ६७

\twolineshloka
{येषां वेदाः प्रियतमा न च संसारजं सुखम्}
{स्वधर्मनिरता ये च तान्नमस्कुर्विहान्वितान्}% ६८

\twolineshloka
{शिवे विष्णौ न वा भेदो न च ब्रह्ममहेशयोः}
{तेषां पादरजः पूतं वहाम्यघविनाशनम्}% ६९

\twolineshloka
{गौरी गङ्गा महालक्ष्मीर्यस्य नास्ति पृथक्तया}
{ते मन्तव्या नराः सर्वे स्वर्गलोकादिहागताः}% ७०

\twolineshloka
{शरणागतरक्षी च मानदानपरायणः}
{यथाशक्ति हरेः प्रीत्यै स ज्ञेयो वैष्णवोत्तमः}% ७१

\twolineshloka
{यस्य नाम महापापराशिं दहति सत्वरम्}
{तदीय चरणद्वन्द्वे भक्तिर्यस्य स वैष्णवः}% ७२

\twolineshloka
{इन्द्रियाणि वशे येषां मनोऽपि हरिचिन्तकम्}
{तान्नमस्कृत्य पूयात्सह्या जन्ममरणान्तिकात्}% ७३

\twolineshloka
{परस्त्रियं त्वं करवालवत्त्यजन्भवेर्यशोभूषणभूतिभूमिः}
{एवं ममादेशमथाचरंश्च लभेः परं धाम सुयोगमीड्यम्}% ७४

{॥इति श्रीपद्मपुराणे पातालखण्डे शेषवात्स्यायनसंवादे रामाश्वमेधे शत्रुघ्नशिक्षाकथनं नाम दशमोऽध्यायः॥१०॥}

\dnsub{एकादशोऽध्यायः}\resetShloka

\uvacha{शेष उवाच}

\twolineshloka
{एवमाज्ञाप्य भगवान्रामश्चामित्रकर्षणः}
{वीरानालोकयन्भूयो जगाद शुभया गिरा}% १

\twolineshloka
{शत्रुघ्नस्य मम भ्रातुर्वाजिरक्षाकरस्य वै}
{को गन्ता पृष्ठतो रक्षंस्तन्निदेशप्रपालकः}% २

\twolineshloka
{यः सर्ववीरान्प्रतिमुख्यमागतान्विनिर्जयेन्मर्मभिदस्त्रसङ्घैः}
{गृह्णात्वसौ मे करवीटकं तद्भूमौ यशः स्वं प्रथयन्सुविस्तरम्}% ३

\twolineshloka
{इत्युक्तवति रामे तु पुष्कलो भरतात्मजः}
{जग्राह वीटकं तस्माद्रघुराजकराम्बुजात्}% ४

\twolineshloka
{स्वामिन्गच्छामि शत्रुघ्न पृष्ठरक्षाकरोऽन्वहम्}
{सन्नद्धः सर्वशस्त्रास्त्र चापबाणधरः प्रभो}% ५

\twolineshloka
{सर्वमद्य क्षितितलं त्वत्प्रतापो विजेष्यते}
{एते निमित्तभूता वै रामचन्द्र महामते}% ६

\twolineshloka
{भवत्कृपातः सकलं ससुरासुरमानुषम्}
{उपस्थितं प्रयुद्धाय तन्निषेधे क्षमो ह्यहम्}% ७

\twolineshloka
{सर्वं स्वामी ज्ञास्यति यन्ममविक्रम दर्शनात्}
{एष गन्तास्मि शत्रुघ्न पृष्ठरक्षाप्रकारकः}% ८

\twolineshloka
{एवं ब्रुवन्तं भरतात्मजं स प्रस्तूय साध्वित्यनुमोदमानः}
{शशंस सर्वान्कपिवीरमुख्यान्प्रभञ्जनोद्भूतमुखान्हरिः प्रभुः}% ९

\twolineshloka
{भो हनूमन्महावीर शृणु मद्वाक्यमादृतः}
{त्वत्प्रसादान्मया प्राप्तमिदं राज्यमकण्टकम्}% १०

\twolineshloka
{सीतया मम संयोगे यो भवाञ्जलधिं तरेः}
{चरितं तद्धरे वेद्मि सर्वं तव कपीश्वर}% ११

\twolineshloka
{त्वं गच्छ मम सैन्यस्य पालकः सन्ममाज्ञया}
{शत्रुघ्नः सोदरो मह्यं पालनीयस्त्वहं यथा}% १२

\twolineshloka
{यत्र यत्र मतिभ्रंशः शत्रुघ्नस्य प्रजायते}
{तत्र तत्र प्रबोद्धव्यो भ्राता मम महामते}% १३

\twolineshloka
{इति श्रुत्वा महद्वाक्यं रामचन्द्रस्य धीमतः}
{शिरसा तत्समाधाय प्रणाममकरोत्तदा}% १४

\twolineshloka
{अथादिशन्महाराजो जाम्बवन्तं कपीश्वरम्}
{रघुनाथस्य सेवायै कपिषूत्तमतेजसम्}% १५

\twolineshloka
{अङ्गदो गवयो मैन्दस्तथा दधिमुखः कपिः}
{सुग्रीवः प्लवगाधीशः शतवल्यक्षिकौ कपी}% १६

\twolineshloka
{नीलो नलो मनोवेगोऽधिगन्ता वानराङ्गजः}
{इत्येवमादयो यूयं सज्जीभूता भवन्तु भोः}% १७

\twolineshloka
{सर्वैर्गजैः सदश्वैश्च तप्तहाटकभूषणैः}
{कवचैः सशिरस्त्राणैर्भूषितायां तु सत्वराः}% १८

\uvacha{शेष उवाच}

\twolineshloka
{सुमन्त्रमाहूय सुमन्त्रिणं तदा जगाद रामो बलवीर्यशोभनः}
{अमात्यमौले वद केऽत्र योज्या नरा हयं पालयितुं समर्थाः}% १९

\twolineshloka
{तदुक्तमेवमाकर्ण्य जगाद परवीरहा}
{हयस्य रक्षणे योग्यान्बलिनोऽत्र नराधिपान्}% २०

\twolineshloka
{रघुनाथ शृणुष्वैतान्नववीरान्सुसंहितान्}
{धनुर्धरान्महाविद्यान्सर्वशस्त्रास्त्रकोविदान्}% २१

\twolineshloka
{प्रतापाग्र्यं नीलरत्नं तथा लक्ष्मीनिधिं नृपम्}
{रिपुतापं चोग्रहयं तथा शस्त्रविदं नृपम्}% २२

\twolineshloka
{राजन्योऽसौ नीलरत्नो महावीरो रथाग्रणीः}
{स एव लक्षं रक्षेत लक्षं युध्येत निर्भयः}% २३

\twolineshloka
{अक्षौहिणीभिर्दशभिर्यातु वाहस्य रक्षणे}
{दंशितैस्स शिरस्त्राणैर्मम बाहुभिरुद्धतैः}% २४

\twolineshloka
{प्रतापाग्र्यो यो ह्ययं च रिपुगर्वमशातयत्}
{सव्यापसव्यबाणानां मोक्ता सर्वास्त्रवित्तमः}% २५

\twolineshloka
{एषोऽक्षौहिणिविंशत्या यातु यज्ञहयावने}
{सन्नद्धो रिपुनाशाय युवाको दण्डदण्डभृत्}% २६

\twolineshloka
{तथा लक्ष्मीनिधिस्त्वेष यातु राजन्यसत्तमः}
{यस्तपोभिः शतधृतिं प्रसाद्यास्त्राणि चाभ्यसत्}% २७

\twolineshloka
{ब्रह्मास्त्रं पाशुपत्यास्त्रं गारुडं नागसंज्ञितम्}
{मायूरं नाकुलं रौद्रं वैष्णवं मेघसंज्ञितम्}% २८

\twolineshloka
{वज्रं पार्वतसंज्ञं च तथा वायव्यसंज्ञितम्}
{इत्यादिकानामस्त्राणां सम्प्रयोगविसर्गवित्}% २९

\twolineshloka
{स एष निजसैन्यानामक्षौहिण्यैकया युतः}
{प्रयातु शूरमुकुटः सर्ववैरिप्रभञ्जनः}% ३०

\twolineshloka
{रिपुतापोऽयमेवाद्य गच्छत्वग्र्यो धनुर्भृताम्}
{सर्वशस्त्रास्त्रकुशलो रिपुवंशदवानलः}% ३१

\twolineshloka
{गच्छतात्सेनया बह्व्या चतुरङ्गसमेतया}
{शत्रुघ्नाज्ञां शिरस्येते दधत्वद्य बलोत्कटाः}% ३२

\twolineshloka
{उग्राश्वोऽपि महाराजा तथा शस्त्रविदेष च}
{सर्वे यान्तु सुसन्नद्धास्तव वाहस्य पालकाः}% ३३

\twolineshloka
{इति भाषितमाकर्ण्य मन्त्रिणः प्रजहर्ष च}
{आज्ञापयामास च तान्सुमन्त्रकथितान्भटान्}% ३४

\twolineshloka
{तेऽनुज्ञां रघुनाथस्य प्राप्य मोदं प्रपेदिरे}
{चिरकालं साम्परायं वाञ्च्छन्तो युद्धदुर्मदाः}% ३५

\twolineshloka
{सन्नद्धाः कवचाद्यैश्च तथा शस्त्रास्त्रवर्तनैः}
{ययुः शत्रुघ्नसंवासं सीतापति प्रणोदिताः}% ३६

\uvacha{शेष उवाच}

\twolineshloka
{अथोक्त ऋषिणा रामो विधिना पूजयत्क्रमात्}
{आचार्यादीनृषीन्सर्वान्यथोक्तवरदक्षिणैः}% ३७

\twolineshloka
{आचार्याय ददौ रामो हस्तिनं षष्टिहायनम्}
{हयमेकं मनोवेगं रत्नमालाविभूषितम्}% ३८

\twolineshloka
{पौरटं रथमेकं च मणिरत्नविभूषितम्}
{चतुर्भिर्वाजिभिर्युक्तं सर्वोपस्करसंयुतम्}% ३९

\twolineshloka
{मणिलक्षं तु प्रत्यक्षं मुक्ताफलतुलाशतम्}
{विद्रुमस्य तुलानां तु सहस्रं स्फुटतेजसाम्}% ४०

\twolineshloka
{ग्राममेकं सुसम्पन्नं नानाजनसमाकुलम्}
{विचित्रसस्यनिष्पन्नं विविधैर्मन्दिरैर्वृतम्}% ४१

\twolineshloka
{ब्रह्मणेऽपि तथैवादाद्धोत्रेऽप्यध्वर्यवे ददौ}
{ऋत्विग्भ्यो भूरिशो दत्त्वा प्रणनाम रघूत्तमः}% ४२

\twolineshloka
{सर्वे ते विविधा वाग्भिराशीर्भिरभिपूजिताः}
{चिरञ्जीव महाराज रामचन्द्र रघूद्वह}% ४३

\twolineshloka
{कन्यादानं भूमिदानं गजदानं तथैव च}
{अश्वदानं स्वर्णदानं तिलदानं समौक्तिकम्}% ४४

\twolineshloka
{अन्नदानं पयोदानमभयं दानमुत्तमम्}
{रत्नदानानि सर्वाणि विप्रेभ्यश्चादिशन्महान्}% ४५

\twolineshloka
{देहि देहि धनं देहि मानेति ब्रूहि कस्यचित्}
{ददात्वन्नं ददात्वन्नं सर्वभोगसमन्वितम्}% ४६

\twolineshloka
{इत्थं प्रावर्तत मखो रघुनाथस्य धीमतः}
{सदक्षिणो द्विजवरैः पूर्णः सर्वशुभक्रियः}% ४७

\twolineshloka
{अथ रामानुजो गत्वा मातरं प्रणनाम ह}
{आज्ञापयाश्वरक्षार्थमेष गच्छामि शोभने}% ४८

\twolineshloka
{त्वत्कृपातो रिपुकुलं जित्वा शोभासमन्वितः}
{आयास्यामि महाराजैर्हयवर्यसमन्वितः}% ४९

\uvacha{मातोवाच}

\twolineshloka
{पुत्र गच्छ महावीर शिवाः पन्थान एव ते}
{सर्वान्रिपुगणाञ्जित्वा पुनरागच्छ सन्मते}% ५०

\twolineshloka
{पुष्कलं पालय निजभ्रातृजं धर्मवित्तमम्}
{महाबलिनमद्यापि बालकं लीलयायुतम्}% ५१

\twolineshloka
{पुत्रागच्छसि चेद्युक्तः पुष्कलेन शुभान्वितः}
{तदा मम प्रमोदः स्यादन्यथा शोकभागहम्}% ५२

\twolineshloka
{इति सम्भाष्यमाणां स्वां मातरं प्रत्युवाच सः}
{त्वदीयचरणद्वन्द्वं स्मरन्प्राप्स्यामि शोभनम्}% ५३

\twolineshloka
{पुष्कलं पालयित्वाहं निजाङ्गमिव शोभने}
{स्वनामसदृशं कृत्वा पुनरेष्यामि मोदवान्}% ५४

\twolineshloka
{इत्युक्त्वा प्रययौ वीरो रामं स मखमण्डपे}
{आसीनं मुनिवर्याग्र्यैर्यज्ञवेषधरं वरम्}% ५५

\twolineshloka
{उवाच मतिमान्वीरः सर्वशोभासमन्वितः}
{रामाज्ञापय रक्षार्थं हयस्यानुज्ञया तव}% ५६

\twolineshloka
{रघुनाथोऽपि तच्छ्रुत्वा भद्रमस्त्विति चाब्रवीत्}
{बालं स्त्रियं प्रमत्तं त्वं मा हन्याः शस्त्रवर्जितम्}% ५७

\twolineshloka
{तदा लक्ष्मीनिधिर्भ्राता जानक्या जनकात्मजः}
{प्रहस्य किञ्चिन्नयने नर्तयन्राममब्रवीत्}% ५८

\uvacha{लक्ष्मीनिधिरुवाच}

\twolineshloka
{रामचन्द्र महाबाहो सर्वधर्मपरायण}
{शत्रुघ्नं शिक्षय तथा यथा लोकोत्तरो भवेत्}% ५९

\twolineshloka
{कुलोचितं कर्म कुर्वन्नग्रजाचरितं तथा}
{गच्छेत्स परमं धाम तेजोबलसमन्वितम्}% ६०

\twolineshloka
{त्वया प्रोक्तं महाराज ब्राह्मणं नावमानयेत्}
{पित्रा तव हतो विप्रः पितृभक्तिपरायणः}% ६१

\twolineshloka
{त्वयापि सुमहत्कर्म कृतं लोकविगर्हितम्}
{अवध्यां महिलां यस्त्वं हतवान्नियतं ततः}% ६२

\twolineshloka
{अग्रजोऽस्य महाराज कृतवान्यं पराक्रमम्}
{सनकेन कृतः पूर्वं राक्षस्याः कर्णकर्तनम्}% ६३

\twolineshloka
{एवं करिष्यति नृपः शत्रुघ्नः शिक्षया तव}
{यदि नायं तथा कुर्यात्कुलस्यासदृशं भवेत्}% ६४

\twolineshloka
{इत्युक्तवन्तं तं रामः प्रत्युवाच हसन्निव}
{मेघगम्भीरया वाचा सर्ववाक्यविशारदः}% ६५

\twolineshloka
{शृण्वन्तु योगिनः शान्ताः समदुःखसुखाः पुनः}
{जानन्त्यपारसंसारनिस्तारतरणादिकम्}% ६६

\twolineshloka
{ये शूराः समहेष्वासाः सर्वशस्त्रास्त्रकोविदाः}
{ते च जानन्ति युद्धस्य वार्त्तां न तु भवादृशाः}% ६७

\twolineshloka
{परोपतापिनो ये वै ये चोत्पथविसारिणः}
{ते हन्तव्या नृपैः सर्वैः सर्वलोकहितैषिभिः}% ६८

\twolineshloka
{इत्युक्तमाकर्ण्य सभासदस्ते सर्वे स्मितं चक्रुररिन्दमस्य}
{कुम्भोद्भवः पूजितमेनमश्वं विमोचयामास सुशोभितं हि}% ६९

\twolineshloka
{इमं मन्त्रं समुच्चार्य वसिष्ठः कलशोद्भवः}
{कराग्रेण स्पृशन्नश्वं मुमोच जयकाङ्क्षया}% ७०

\twolineshloka
{वाजिन्गच्छ यथालीलं सर्वत्र धरणीतले}
{यागार्थे मोचितो येन पुनरागच्छ सत्वरः}% ७१

\twolineshloka
{अश्वस्तु मोचितः सर्वैर्भटैः शस्त्रास्त्रकोविदैः}
{परीतः प्रययौ प्राचीं दिशं वायुजवान्वितः}% ७२

\twolineshloka
{प्रचचार बलं सर्वं कम्पयद्धरणीतलम्}
{शेषोऽपि किञ्चिन्न तया फणया धृतवान्भुवम्}% ७३

\twolineshloka
{दिशः प्रसेदुः परितः क्ष्मातलं शोभयान्वितम्}
{वायवस्तं तु शत्रुघ्नं पृष्ठतो मन्दगामिनः}% ७४

\twolineshloka
{शत्रुघ्नस्य प्रयाणायाभ्युद्य तस्य भुजोऽस्फुरत्}
{दक्षिणः शुभमाशंसी जयाय च बभूव ह}% ७५

\twolineshloka
{पुष्कलः स्वगृहं रम्यं प्रविवेश समृद्धिमत्}
{वितर्दिभिर्वलक्षाभिः शोभितं रत्नवेदिकम्}% ७६

\twolineshloka
{तत्रापश्यन्निजां भार्यां पतिव्रतपरायणाम्}
{किञ्चित्स्वदर्शनाद्धृष्टां भर्तृदर्शनलालसाम्}% ७७

\twolineshloka
{मुखारविन्देन च नागवल्लीदलं सुकर्पूरयुतं च चर्वती}
{नासाफलं तोयभवं महाधनं बाह्वोर्मृणालीसदृशोः सुकङ्कणे}% ७८

\twolineshloka
{कुचौ तु मालूरफलोपमौ वरौ नितम्बबिम्बं वरनीवि शोभितम्}
{पादौ तुलाकोटिधरौ सुकोमलौ दधत्यहो एक्षत सत्पतिं स्वकम्}% ७९

\twolineshloka
{परिरभ्य प्रियां धीरो गद्गदस्वरभाषिणीम्}
{तदुरोजपरीरम्भनिर्भरीकृतदेहकाम्}% ८०

\twolineshloka
{उवाच भद्रे गच्छामि शत्रुघ्नपृष्ठरक्षकः}
{रामाज्ञया याज्ञमश्वं पालयन्रथसंयुतः}% ८१

\twolineshloka
{त्वया मे मातरः पूज्याः पादसंवाहनादिमिः}
{तदुच्छिष्टं हि भुञ्जाना तत्कर्मकरणादरा}% ८२

\twolineshloka
{सर्वाः पतिव्रता नार्यो लोपामुद्रादिकाः शुभाः}
{नावमान्यास्त्वया भीरु स्वतपोबलशोभिताः}% ८३

{॥इति श्रीपद्मपुराणे पातालखण्डे शेषवात्स्यायनसंवादे रामाश्वमेधे हयमोचनपुष्कलभार्यासमागमो नाम एकादशोऽध्यायः॥११॥}

\dnsub{द्वादशोऽध्यायः}\resetShloka

\uvacha{शेष उवाच}

\twolineshloka
{इत्युक्तवन्तं स्वपतिं वीक्ष्य प्रेम्णा सुनिर्भरम्}
{प्रत्युवाच हसन्तीव किञ्चिद्गद्गदभाषिणी}% १

\twolineshloka
{नाथ ते विजयोभूयात्सर्वत्र रणमण्डले}
{शत्रुघ्नाज्ञा प्रकर्तव्या हयरक्षा यथा भवेत्}% २

\twolineshloka
{स्मरणीया हि सर्वत्र सेविका त्वत्पदानुगा}
{कदापि मानसं नाथ त्वत्तो नान्यत्र गच्छति}% ३

\twolineshloka
{परमायोधने कान्त स्मर्तव्याहं न जातुचित्}
{सत्यां मयि तव स्वान्ते युद्धे विजयसंशयः}% ४

\twolineshloka
{पद्मनेत्र तथा कार्यमूर्मिलाद्या यथा मम}
{हास्यं नैव प्रकुर्वन्ति मां वीक्ष्य करताडनैः}% ५

\twolineshloka
{इयं पत्नी महाभीरोः सङ्ग्रामे प्रपलायितुः}
{कातरा यर्हि युद्ध्यन्ति शूराणां समयः कुतः}% ६

\twolineshloka
{इत्येवं न हसन्त्युच्चैर्यथा मे देवराङ्गनाः}
{तथा कार्यं महाबाहो रामस्य हयरक्षणे}% ७

\twolineshloka
{योद्धा त्वमादौ सर्वत्र परे ये तव पृष्ठतः}
{धनुष्टङ्कारबधिराः क्रियन्तां बलिनः परे}% ८

\twolineshloka
{तवप्रोद्यत्कराम्भोज करवालभिया बलम्}
{परेषां भवतात्क्षिप्रमन्योन्य भयव्याकुलम्}% ९

\twolineshloka
{कुलं महदलं कार्यं परान्विजयता त्वया}
{गच्छ स्वामिन्महाबाहो तव श्रेयो भवत्विह}% १०

\twolineshloka
{इदं धनुर्गृहाणाशु महद्गुणविभूषितम्}
{यस्य गर्जितमाकर्ण्य वैरिवृन्दं भयातुरम्}% ११

\twolineshloka
{इमौ ते त्विषुधी वीर बध्येतां शं यथा भवेत्}
{वैरिकोटिविनिष्पेष बाणकोटि सुपूरितौ}% १२

\twolineshloka
{कवचं त्विदमाधेहि शरीरे कामसुन्दरे}
{वज्रप्रभा महादीप्ति हतसन्तमसन्दृढम्}% १३

\twolineshloka
{शिरस्त्राणं निजोत्तंसे कुरु कान्त मनोरमम्}
{इमेव तंसे विशदे मणिरत्नविभूषिते}% १४

\fourlineindentedshloka
{इति सुविमलवाचं वीरपुत्रीं प्रपश्यन्}
{नयनकमलदृष्ट्या वीक्षमाणस्तन्दङ्गम्}
{अधिगतपरिमोदो भारतिः शत्रुजेता}
{रणकरणसमर्थस्तां जगादातिधीरः}% १५

\uvacha{पुष्कल उवाच}

\twolineshloka
{कान्ते यत्त्वं वदसि मां तथा सर्वं चराम्यहम्}
{वीरपत्नी भवेत्कीर्तिस्तव कान्तिमतीप्सिता}% १६

\twolineshloka
{इति कान्तिमतीदत्तं कवचं मुकुटं वरम्}
{धनुर्महेषुधीखड्गं सर्वं जग्राह वीर्यवान्}% १७

\twolineshloka
{परिधाय च तत्सर्वं बहुशो भासमन्वितः}
{शुशुभेऽतीव सुभटः सर्वशस्त्रास्त्रकोविदः}% १८

\twolineshloka
{तमस्त्रशस्त्रशोभाढ्यं वीरमालाविभूषितम्}
{कुङ्कुमागुरुकस्तूरी चन्दनादिकचर्चितम्}% १९

\twolineshloka
{नानाकुसुममालाभिराजानुपरिशोभितम्}
{नीराजयामास मुहुस्तत्र कान्तिमती सती}% २०

\twolineshloka
{नीराजयित्वा बहुशः किरन्ती मौक्तिकैर्मुहुः}
{गलदश्रुचलन्नेत्रा परिरेभे पतिं निजम्}% २१

\twolineshloka
{दृढं सपरिरभ्यैतां चिरमाश्वासयत्तदा}
{वीरपत्नि कान्तिमति विरहं मा कृथा मम}% २२

\twolineshloka
{एष गच्छामि सविधे तव भामे पतिव्रते}
{इत्युक्त्वा तां निजां पत्नीं रथमारुरुहे वरम्}% २३

\twolineshloka
{तं प्रयान्तं पतिं श्रेष्ठं नयनैर्निमिषोज्झितैः}
{विलोकयामास तदा पतिव्रतपरायणा}% २४

\twolineshloka
{स ययौ जनकं द्रष्टुं जननीं प्रेमविह्वलाम्}
{गत्वा पितरमम्बां च ववन्दे शिरसा मुदा}% २५

\twolineshloka
{माता पुत्रं परिष्वज्य स्वाङ्कमारोपयत्तदा}
{मुञ्चन्ती बाष्पनिचयं स्वस्त्यस्त्विति जगाद सा}% २६

\twolineshloka
{पितरं प्राह भरतं रामो यज्ञकरः परः}
{पालनीयो लक्ष्मणेन भवद्भिश्च महात्मभिः}% २७

\twolineshloka
{आज्ञप्तोऽसौ जनन्या च पित्रा हृषितया गिरा}
{ययौ शत्रुघ्नकटकं महावीरविभूषितम्}% २८

\twolineshloka
{रथिभिः पत्तिभिर्वीरैः सदश्वैः सादिभिर्वृतः}
{ययौ मुदा रघूत्तंस महायज्ञहयाग्रणीः}% २९

\twolineshloka
{गच्छन्पाञ्चालदेशांश्च कुरूंश्चैवोत्तरान्कुरून्}
{दशार्णाञ्छ्रीविशालांश्च सर्वशोभासमन्वितः}% ३०

\twolineshloka
{तत्र तत्रोपशृण्वानो रघुवीरयशोऽखिलम्}
{रावणासुरघातेन भक्तरक्षाविधायकम्}% ३१

\twolineshloka
{पुनश्च हयमेधादि कार्यमारभ्य पावनम्}
{यशो वितन्वन्भुवने लोकान्रामोऽविता भयात्}% ३२

\twolineshloka
{तेभ्यस्तुष्टो ददौ हारान्रत्नानि विविधानि च}
{महाधनानि वासांसि शत्रुघ्नः प्रवरो महान्}% ३३

\twolineshloka
{सुमतिर्नाम तेजस्वी सर्वविद्याविशारदः}
{रघुनाथस्य सचिवः शत्रुघ्नानुचरो वरः}% ३४

\twolineshloka
{ययौ तेन महावीरो ग्रामाञ्जनपदान्बहून्}
{रघुनाथप्रतापेन न कोपि हृतवान्हयम्}% ३५

\twolineshloka
{देशाधिपाये बहवो महाबलपराक्रमाः}
{हस्त्यश्वरथपादात चतुरङ्गसमन्विताः}% ३६

\twolineshloka
{सम्पदो बहुशो नीत्वा मुक्तामाणिक्यसंयुताः}
{शत्रुघ्नं हयरक्षार्थमागतं प्रणता मुहुः}% ३७

\twolineshloka
{इदं राज्यं धनं सर्वं सपुत्रपशुबान्धवम्}
{रामचन्द्रस्य सर्वं हि न मदीयं रघूद्वह}% ३८

\twolineshloka
{एवं तदुक्तमाकर्ण्य शत्रुघ्नः परवीरहा}
{आज्ञां स्वां तत्र संज्ञाप्य ययौ तैः सहितः पथि}% ३९

\twolineshloka
{एवं क्रमेण सम्प्राप्तः शत्रुघ्नो हयसंयुतः}
{अहिच्छत्रां पुरीं ब्रह्मन्नानाजनसमाकुलाम्}% ४०

\twolineshloka
{ब्रह्मद्विजसमाकीर्णां नानारत्नविभूषिताम्}
{सौवर्णैः स्फाटिकैर्हर्म्यैर्गोपुरैः समलङ्कृताम्}% ४१

\twolineshloka
{यत्र रम्भा तिरस्कारकारिण्यः कमलाननाः}
{दृश्यन्ते सर्वहर्म्येषु ललना लीलयान्विताः}% ४२

\twolineshloka
{यत्र स्वाचारललिताः सर्वभोगैकभोगिनः}
{धनदानुचरायद्वत्तथा लीलासमन्विताः}% ४३

\twolineshloka
{यत्र वीरा धनुर्हस्ताःशरसन्धानकोविदाः}
{कुर्वन्ति तत्र राजानं सुहृष्टं सुमदाभिधम्}% ४४

\twolineshloka
{एवंविधं ददर्शासौ नगरं दूरतः प्रभुः}
{पार्श्वे तस्य पुरश्रेष्ठमुद्यानं शोभयान्वितम्}% ४५

\twolineshloka
{पुन्नागैर्नागचम्पैश्च तिलकैर्देवदारुभिः}
{अशोकैः पाटलैश्चूतैर्मन्दारैःकोविदारकैः}% ४६

\twolineshloka
{आम्रजम्बुकदम्बैश्च प्रियालपनसैस्तथा}
{शालैस्तालैस्तमालैश्च मल्लिकाजातियूथिभिः}% ४७

\twolineshloka
{नीपैः कदम्बैर्बकुलैश्चम्पकैर्मदनादिभिः}
{शोभितं सददर्शाथशत्रुघ्नःपरवीरहा}% ४८

\fourlineindentedshloka
{हयोगतस्तद्वनमध्यदेशे}
{तमालतालादि सुशोभिते वै}
{ययौ ततः पृष्ठत एव वीरो}
{धनुर्धरैः सेवितपादपद्मः}% ४९

\twolineshloka
{ददर्श त रचितं देवायतनमद्भुतम्}
{इन्द्रनीलैश्च वैडूर्यैस्तथा मारकतैरपि}% ५०

\twolineshloka
{शोभितं सुरसेवार्हं कैलासप्रस्थसन्निभम्}
{जातरूपमयैः स्तम्भैःशोभितं सद्मनां वरम्}% ५१

\twolineshloka
{दृष्ट्वातद्रघुनाथस्य भ्राता देवालयं वरम्}
{पप्रच्छ सुमतिं स्वीयं मन्त्रिणं वदतांवरम्}% ५२

\uvacha{शत्रुघ्न उवाच}

\twolineshloka
{वदामात्य वरेदं किं कस्यदेवस्य केतनम्}
{का देवता पूज्यतेऽत्र कस्य हेतोः स्थितानघ}% ५३

\onelineshloka
{एवमाकर्ण्य यथावदिहसर्वशः}% ५४

\twolineshloka
{कामाक्षायाः परं स्थानं विद्धि विश्वैकशर्मदम्}
{यस्या दर्शनमात्रेण सर्वसिद्धिः प्रजापते}% ५५

\twolineshloka
{देवासुरास्तु यां स्तुत्वा नत्वा प्राप्ताखिलां श्रियम्}
{धर्मार्थकाममोक्षाणां दात्री भक्तानुकम्पिनी}% ५६

\twolineshloka
{याचिता सुमदेनात्राहिच्छत्रा पतिना पुरा}
{स्थिता करोति सकलं भक्तानां दुःखहारिणी}% ५७

\twolineshloka
{तां नमस्कुरु शत्रुघ्न सर्ववीर शिरोमणे}
{नत्वाशु सिद्धिं प्राप्नोषि ससुरासुरदुर्ल्लभाम्}% ५८

\twolineshloka
{इति श्रुत्वाथ तद्वाक्यं शत्रुघ्नः शत्रुतापनः}
{पप्रच्छ सकलां वार्तां भवान्याः पुरुषर्षभः}% ५९

\uvacha{शत्रुघ्न उवाच}

\twolineshloka
{कोऽहिच्छत्रापती राजा सुमदः किं तपः कृतम्}
{येनेयं सर्वलोकानां माता तुष्टात्र संस्थिता}% ६०

\twolineshloka
{वद सर्वं महामात्य नानार्थपरिबृंहितम्}
{यथावत्त्वं हि जानासि तस्माद्वद महामते}% ६१

\uvacha{सुमतिरुवाच}

\twolineshloka
{हेमकूटो गिरिः पूतः सर्वदेवोपशोभितः}
{तत्रास्ति तीर्थं विमलमृषिवृन्दसुसेवितम्}% ६२

\twolineshloka
{सुमदो हि तपस्तेपे हतमातृपितृप्रजः}
{अरिभिः सर्वसामन्तैर्जगाम तपसे हि तम्}% ६३

\twolineshloka
{वर्षाणि त्रीणि सपदा त्वेकेन मनसा स्मरन्}
{जगतां मातरं दध्यौ नासाग्रस्तिमितेक्षणः}% ६४

\twolineshloka
{वर्षाणि त्रीणि शुष्काणां पर्णानां भक्षणं चरन्}
{चकार परमुग्रं स तपः परमदुश्चरम्}% ६५

\twolineshloka
{अब्दानि त्रीणि सलिले शीतकाले ममज्ज सः}
{ग्रीष्मे चचार पञ्चाग्नीन्प्रावृट्सु जलदोन्मुखः}% ६६

\twolineshloka
{त्रीणि वर्षाणि पवनं संरुध्य स्वान्तगोचरम्}
{भवानीं संस्मरन्धीरो न च किञ्चन पश्यति}% ६७

\twolineshloka
{वर्षे तु द्वादशेऽतीते दृष्ट्वैतत्परमं तपः}
{विभाव्य मनसातीव शक्रः पस्पर्ध तं भयात्}% ६८

\twolineshloka
{आदिदेश सकामं तु परिवारपरीवृतम्}
{अप्सरोभिः सुसंयुक्तं ब्रह्मेन्द्रादिजयोद्यतम्}% ६९

\twolineshloka
{गच्छ कामसखे मह्यं प्रियमाचर मोहन}
{सुमदस्य तपोविघ्नं समाचर यथा भवेत्}% ७०

\twolineshloka
{इति श्रुत्वा महद्वाक्यं तुरासाहः स्वयम्प्रभुः}
{उवाच विश्वविजये प्रौढगर्वो रघूद्वह}% ७१

\uvacha{काम उवाच}

\twolineshloka
{स्वामिन्कोऽसौ हि सुमदः किं तपः स्वल्पकं पुनः}
{ब्रह्मादीनां तपोभङ्गं करोम्यस्य तु का कथा}% ७२

\twolineshloka
{मद्बाणबलनिर्भिन्नश्चन्द्रस्तारां गतः पुरा}
{त्वमप्यहल्यां गतवान्विश्वामित्रस्तु मेनिकाम्}% ७३

\twolineshloka
{चिन्तां मा कुरु देवेन्द्र सेवके मयि संस्थिते}
{एष गच्छामि सुमदं देवान्पालय मारिष}% ७४

\twolineshloka
{एवमुक्त्वा कामदेवो हेमकूटं गिरिं ययौ}
{वसन्तेन युतः सख्या तथैवाप्सरसाङ्गणैः}% ७५

\twolineshloka
{वसन्तस्तत्र सकलान्वृक्षान्पुष्पफलैर्युतान्}
{कोकिलान्षट्पदश्रेण्या घुष्टानाशु चकार ह}% ७६

\twolineshloka
{वायुः सुशीतलो वाति दक्षिणां दिशमाश्रितः}
{कृतमालासरित्तीर लवङ्गकुसुमान्वितः}% ७७

\twolineshloka
{एवंविधे वने वृत्ते रम्भानामाप्सरोवरा}
{सखीभिः संवृता तत्र जगाम सुमदान्तिकम्}% ७८

\twolineshloka
{तत्रारभत गानं सा किन्नरस्वरशोभना}
{मृदङ्गपणवानेकवाद्यभेदविशारदा}% ७९

\fourlineindentedshloka
{तद्गानमाकर्ण्य नराधिपोऽसौ}
{वसन्तमालोक्य मनोहरं च}
{तथान्यपुष्टारटितं मनोरमं}
{चकार चक्षुः परिवर्तनं बुधः}% ८०

\twolineshloka
{तं प्रबुद्धं नृपं वीक्ष्य कामः पुष्पायुधस्त्वरन्}
{चकार सत्वरं सज्यं धनुस्तत्पृष्ठतोऽनघ}% ८१

\fourlineindentedshloka
{एकाप्सरास्तत्र नृपस्य पादयोः}
{संवाहनं नर्तितनेत्रपल्लवा}
{चकार चान्या तु कटाक्षमोक्षणं}
{चकार काचिद्भृशमङ्गचेष्टितम्}% ८२

\twolineshloka
{अप्सरोभिस्तथाकीर्णः कामविह्वलमानसः}
{चिन्तयामास मतिमाञ्जितेन्द्रियशिरोमणिः}% ८३

\twolineshloka
{एता मे तपसो विघ्नकारिण्योऽप्सरसां वराः}
{शक्रेण प्रेषिताः सर्वाः करिष्यन्ति यथातथम्}% ८४

\twolineshloka
{इति सञ्चिन्त्य सुतपास्ता उवाच वराङ्गनाः}
{का यूयं कुत्र संस्थाः किं भवतीनां चिकीर्षितम्}% ८५

\twolineshloka
{अत्यद्भुतं जातमहो यद्भवत्योऽक्षिगोचराः}
{यास्तपोभिः सुदुष्प्राप्यास्ता मे तपस आगताः}% ८६

{॥इति श्रीपद्मपुराणे पातालखण्डे शेषवात्स्यायनसंवादे रामाश्वमेधे कामाक्षोपाख्यानं नाम द्वादशोऽध्यायः॥१२॥}

\dnsub{त्रयोदशोऽध्यायः}\resetShloka

\uvacha{शेष उवाच}

\twolineshloka
{इति वाक्यं समाकर्ण्य सुमदस्य तपोनिधेः}
{जगदुः कामसेनास्तं रम्भाद्यप्सरसो मुदा}% १

\twolineshloka
{त्वत्तपोभिर्वयं कान्त प्राप्ताः सर्ववराङ्गनाः}
{तासां यौवनसर्वस्वं भुङ्क्ष्व त्यज तपःफलम्}% २

\twolineshloka
{इयं घृताची सुभगा चम्पकाभशरीरभृत्}
{कर्पूरगन्धललितं भुनक्तु त्वन्मुखामृतम्}% ३

\fourlineindentedshloka
{एतां महाभाग सुशोभिविभ्रमां}
{मनोहराङ्गीं घनपीनसत्कुचाम्}
{कान्तोपभुङ्क्ष्वाशु निजोग्रपुण्यतः}
{प्राप्तां पुनस्त्वं त्यज दुःखजातम्}% ४

\fourlineindentedshloka
{मामप्यनर्घ्याभरणोपशोभितां}
{मन्दारमालापरिशोभिवक्षसम्}
{नानारताख्यानविचारचञ्चुरां}
{दृढं यथा स्यात्परिरम्भणं कुरु}% ५

\fourlineindentedshloka
{पिबामृतं मामकवक्त्रनिर्गतं}
{विमानमारुह्य वरं मया सह}
{सुमेरुशृङ्गं बहुपुण्यसेवितं}
{सम्प्राप्य भोगं कुरु सत्तपः फलम्}% ६

\fourlineindentedshloka
{तिलोत्तमा यौवनरूपशोभिता}
{गृह्णातु ते मूर्धनि तापवारणम्}
{सुचामरौ सन्ततधारयाङ्कितौ}
{गङ्गाप्रवाहाविव सुन्दरोत्तम}% ७

\fourlineindentedshloka
{शृणुष्व भोः कामकथां मनोहरां}
{पिबामृतं देवगणादिवाञ्छितम्}
{उद्यानमासाद्य च नन्दनाभिधं}
{वराङ्गनाभिर्विहरं कुरु प्रभो}% ८

\fourlineindentedshloka
{इत्युक्तमाकर्ण्य महामतिर्नृपो}
{विचारयामास कुतो ह्युपस्थिताः}
{मया सुसृष्टास्तपसा सुराङ्गनाः}
{प्रत्यूह एवात्र विधेयमेष किम्}% ९

\twolineshloka
{इति चिन्तातुरो राजा स्वान्ते सञ्चिन्तयन्सुधीः}
{जगाद मतिमान्वीरः सुमदो देवताङ्गनाः}% १०

\twolineshloka
{यूयं तु ममचित्तस्था जगन्मातृस्वरूपकाः}
{मया सञ्चिन्त्यते या हि सापि त्वद्रूपिणी मता}% ११

\twolineshloka
{इदं तुच्छं स्वर्गसुखं त्वयोक्तं सविकल्पकम्}
{मत्स्वामिनी मया भक्त्या सेविता दास्यते वरम्}% १२

\twolineshloka
{यत्कृपातो विधिः सत्यलोकं प्राप्तो महानभूत्}
{सा मे दास्यति सर्वं हि भक्तदुःखान्तकारिणी}% १३

\twolineshloka
{किं नन्दनं किं तु गिरिः कनकेन सुमण्डितः}
{किं सुधा स्वल्पपुण्येन प्राप्या दानवदुःखदा}% १४

\twolineshloka
{इति वाक्यं समाकर्ण्य कामस्तु विविधैः शरैः}
{प्राहरन्नरदेवस्य कर्तुं किञ्चिन्न वै प्रभुः}% १५

\twolineshloka
{कटाक्षैर्नूपुरारावैः परिरम्भैर्विलोकनैः}
{न तस्य चित्तं विभ्रान्तं कर्तुं शक्ता वराङ्गनाः}% १६

\twolineshloka
{गत्वा यथागतं शक्रं जगदुर्धीरधीर्नृपः}
{तच्छ्रुत्वा मघवा भीतः सेवामारभतात्मनः}% १७

\twolineshloka
{अथ निश्चितमालोक्य पादपद्मे स्वकेऽम्बिका}
{जितेन्द्रियं महाराजं प्रत्यक्षाभूत्सुतोषिता}% १८

\twolineshloka
{पञ्चास्यपृष्ठललिता पाशाङ्कुशधरावरा}
{धनुर्बाणधरा माता जगत्पावनपावनी}% १९

\twolineshloka
{तां वीक्ष्य मातरं धीमान्सूर्यकोटिसमप्रभाम्}
{धनुर्बाणसृणीपाशान्दधानां हर्षमाप्तवान्}% २०

\twolineshloka
{शिरसा बहुशो नत्वा मातरं भक्तिभाविताम्}
{हसन्तीं निजदेहेषु स्पृशन्तीं पाणिना मुहुः}% २१

\twolineshloka
{तुष्टाव भक्त्युत्कलितचित्तवृत्तिर्महामतिः}
{गद्गदस्वरसंयुक्तः कण्टकाङ्गोपशोभितः}% २२

\twolineshloka
{जय देवि महादेवि भक्तवृन्दैकसेविते}
{ब्रह्मरुद्रादिदेवेन्द्र सेविताङ्घ्रियुगेऽनघे}% २३

\twolineshloka
{मातस्तव कलाविद्धमेतद्भाति चराचरम्}
{त्वदृते नास्ति सर्वं तन्मातर्भद्रे नमोस्तु ते}% २४

\twolineshloka
{मही त्वयाऽधारशक्त्या स्थापिता चलतीह न}
{सपर्वतवनोद्यान दिग्गजैरुपशोभिता}% २५

\twolineshloka
{सूर्यस्तपति खे तीक्ष्णैरंशुभिः प्रतपन्महीम्}
{त्वच्छक्त्या वसुधासंस्थं रसं गृह्णन्विमुञ्चति}% २६

\twolineshloka
{अन्तर्बहिः स्थितो वह्निर्लोकानां प्रकरोतु शम्}
{त्वत्प्रतापान्महादेवि सुरासुरनमस्कृते}% २७

\twolineshloka
{त्वं विद्या त्वं महामाया विष्णोर्लोकैकपालिनः}
{स्वशक्त्या सृजसीदं त्वं पालयस्यपि मोहिनि}% २८

\twolineshloka
{त्वत्तः सर्वे सुराः प्राप्य सिद्धिं सुखमयन्ति वै}
{मां पालय कृपानाथे वन्दिते भक्तवल्लभे}% २९

\twolineshloka
{रक्ष मां सेवकं मातस्त्वदीयचरणारणम्}
{कुरु मे वाञ्छितां सिद्धिं महापुरुषपूर्वजे}% ३०

\uvacha{सुमतिरुवाच}

\twolineshloka
{एवं तुष्टा जगन्माता वृणीष्व वरमुत्तमम्}
{उवाच भक्तं सुमदं तपसा कृशदेहिनम्}% ३१

\twolineshloka
{इत्येतद्वाक्यमाकर्ण्य प्रहृष्टः सुमदो नृपः}
{वव्रे निजं हृतं राज्यं हतदुर्जनकण्टकम्}% ३२

\twolineshloka
{महेशीचरणद्वन्द्वे भक्तिमव्यभिचारिणीम्}
{प्रान्ते मुक्तिं तु संसारवारिधेस्तारिणीं पुनः}% ३३

\uvacha{कामाक्षोवाच}

\twolineshloka
{राज्यं प्राप्नुहि सुमद सर्वत्रहतकण्टकम्}
{महिलारत्नसञ्जुष्टपादपद्मद्वयो भव}% ३४

\twolineshloka
{ततवैरिपराभूतिर्माभूयात्सुमदाभिध}
{यदा तु रावणं हत्वा रघुनाथो महायशाः}% ३५

\twolineshloka
{करिष्यत्यश्वमेधं हि सर्वसम्भारशोभितम्}
{तस्य भ्राता महावीरः शत्रुघ्नः परवीरहा}% ३६

\twolineshloka
{पालयन्हयमायास्यत्यत्र वीरादिभिर्वृतः}
{तस्मै सर्वं समर्प्य त्वं राज्यमृद्धं धनादिकम्}% ३७

\twolineshloka
{पालयिष्यसि योधैः स्वैर्धनुर्धारिभिरुद्भटैः}
{ततः पृथिव्यां सर्वत्र भ्रमिष्यसि महामते}% ३८

\twolineshloka
{ततो रामं नमस्कृत्य ब्रह्मेन्द्रेशादिसेवितम्}
{मुक्तिं प्राप्स्यसि दुष्प्रापां योगिभिर्यमसाधनैः}% ३९

\twolineshloka
{तावत्कालमिहस्थास्ये यावद्रामहयागमः}
{पश्चात्त्वां तु समुद्धृत्य गन्तास्मि परमं पदम्}% ४०

\twolineshloka
{इत्युक्त्वान्तर्दधे देवी सुरासुरनमस्कृता}
{सुमदोऽप्यहिच्छत्रायां शत्रून्हत्वा नृपोऽभवत्}% ४१

\twolineshloka
{एष राजा समर्थोऽपि बलवाहनसंयुतः}
{न ग्रहीष्यति ते वाहं महामायासुशिक्षितः}% ४२

\twolineshloka
{श्रुत्वा प्राप्तं पुरी पार्श्वे हयमेधहयोत्तमम्}
{त्वां च सर्वैर्महाराजैः सेविताङ्घ्रिं महामतिम्}% ४३

\twolineshloka
{सर्वं दास्यति सर्वज्ञ राजा सुमदनामधृक्}
{अधुनातन्महाराज रामचन्द्र प्रतापतः}% ४४

\uvacha{शेष उवाच}

\twolineshloka
{इति वृत्तं समाकर्ण्य सुमदस्य महायशाः}
{साधुसाध्विति चोवाच जहर्ष मतिमान्बली}% ४५

\twolineshloka
{अहिच्छत्रापतिः सर्वैः स्वगणैः परिवारितः}
{सभायां सुखमास्ते यो बहुराजन्यसेवितः}% ४६

\twolineshloka
{ब्राह्मणा वेदविदुषो वैश्या धनसमृद्धयः}
{राजानं पर्युपासन्ते सुमदंशो भयान्वितम्}% ४७

\twolineshloka
{वेदविद्याविनोदेन न्यायिनो ब्राह्मणा वराः}
{आशीर्वदन्ति तं भूपं सर्वलोकैकरक्षकम्}% ४८

\twolineshloka
{एतस्मिन्समये कश्चिदागत्य नृपतिं जगौ}
{स्वामिन्न जाने कस्यास्ति हयः पत्रधरोऽन्तिके}% ४९

\twolineshloka
{तच्छ्रुत्वा सेवकं श्रेष्ठं प्रेषयामास सत्वरः}
{जानीहि कस्य राज्ञोऽयमश्वो मम पुरान्तिके}% ५०

\twolineshloka
{गत्वाथ सेवकस्तत्र ज्ञात्वा वृत्तान्तमादितः}
{निवेदयामास नृपं महाराजन्यसेवितम्}% ५१

\twolineshloka
{स श्रुत्वा रघुनाथस्य हयं नित्यमनुस्मरन्}
{आज्ञापयामास जनं सर्वं राजाविशारदः}% ५२

\twolineshloka
{लोका मदीयाः सर्वे ये धनधान्यसमाकुलाः}
{तोरणादीनि गेहेषु मङ्गलानि सृजन्त्विह}% ५३

\twolineshloka
{कन्याः सहस्रशो रम्याः सर्वाभरणभूषिताः}
{गजोपरिसमारूढा यान्तु शत्रुघ्नसम्मुखम्}% ५४

\twolineshloka
{इत्यादिसर्वमाज्ञाप्य ययौ राजा स्वयं ततः}
{पुत्रपौत्रमहिष्यादिपरिवारसमावृतः}% ५५

\twolineshloka
{शत्रुघ्नः सुमहामात्यैः सुभटैः पुष्कलादिभिः}
{संयुतो भूपतिं वीरं ददर्श सुमदाभिधम्}% ५६

\twolineshloka
{हस्तिभिः सादिसंयुक्तैः पत्तिभिः परतापनैः}
{वाजिभिर्भूषितैर्वीरैः संयुतं वीरशोभितम्}% ५७

\twolineshloka
{अथागत्य महाराजः शत्रुघ्नं नतवान्मुदा}
{धन्योऽस्मि कृतकृत्योऽस्मि सत्कृतं च कृतं वपुः}% ५८

\twolineshloka
{इदं राज्यं गृहाणाशु महाराजोपशोभितम्}
{महामाणिक्यमुक्तादि महाधनसुपूरितम्}% ५९

\twolineshloka
{स्वामिंश्चिरं प्रतीक्षेऽहं हयस्यागमनं प्रति}
{कामाक्षाकथितं पूर्वं जातं सम्प्रति तत्तथा}% ६०

\twolineshloka
{विलोकय पुरं मह्यं कृतार्थान्कुरु मानवान्}
{पावयास्मत्कुलं सर्वं रामानुज महीपते}% ६१

\twolineshloka
{इत्युक्त्वारोहयामास कुञ्जरं चन्द्रसुप्रभम्}
{पुष्कलं च महावीरं तथा स्वयमथारुहत्}% ६२

\twolineshloka
{भेरीपणवतूर्याणां वीणादीनां स्वनस्तदा}
{व्याप्नोति स्म महाराज सुमदेन प्रणोदितः}% ६३

\fourlineindentedshloka
{कन्याः समागत्य महानरेन्द्रं-}
{शत्रुघ्नमिन्द्रादिकसेविताङ्घ्रिम्}
{करिस्थिता मौक्तिकवृन्दसङ्घै-}
{र्वर्धापयामासुरिनप्रयुक्ताः}% ६४

\twolineshloka
{शनैःशनैः समागत्य पुरीमध्ये जनैर्मुदा}
{वर्धापितो गृहं प्राप तोरणादिकभूषितम्}% ६५

\twolineshloka
{हयरत्नेन संयुक्तस्तथा वीरैः सुशोभितः}
{राज्ञा पुरस्कृतो राजा शत्रुघ्नः प्राप मन्दिरम्}% ६६

\twolineshloka
{अर्घादिभिः पूजयित्वा रघुनाथानुजं तदा}
{सर्वं समर्पयामास रामचन्द्राय धीमते}% ६७

{॥इति श्रीपद्मपुराणे पातालखण्डे शेषवात्स्यायनसंवादे रामाश्वमेधे शत्रुघ्नाहिच्छत्रापुरीप्रवेशो नाम त्रयोदशोऽध्यायः॥१३॥}

\dnsub{चतुर्दशोऽध्यायः}\resetShloka

\uvacha{शेष उवाच}

\twolineshloka
{अथ स्वागतसन्तुष्टं शत्रुघ्नं प्राह भूमिपः}
{रघुनाथकथां श्रेष्ठां शुश्रूषुः पुरुषर्षभः}% १

\uvacha{सुमद उवाच}

\twolineshloka
{कच्चिदास्ते सुखं रामः सर्वलोकशिरोमणिः}
{भक्तरक्षावतारोऽयं ममानुग्रहकारकः}% २

\twolineshloka
{धन्या लोका इमे पुर्यां रघुनाथमुखाम्बुजम्}
{ये पिबन्त्यनिशं चाक्षिपुटकैः परिमोदिताः}% ३

\twolineshloka
{अर्थजातं मदीयं च नितरां पुरुषर्षभ}
{कृतार्थं कुलभूम्यादि वस्तुजातं महामते}% ४

\twolineshloka
{कामाक्षया प्रसादो मे कृतः पूर्वं दयार्द्रया}
{रघुनाथमुखाम्भोजं द्रक्ष्येद्य सकुटुम्बकः}% ५

\twolineshloka
{इत्युक्तवति वीरे तु सुमदे पार्थिवोत्तमे}
{सर्वं तत्कथयामास रघुनाथगुणोदयम्}% ६

\twolineshloka
{त्रिरात्रं तत्र संस्थित्य रघुनाथानुजः परम्}
{गन्तुं चकार धिषणां राज्ञा सह महामतिः}% ७

\twolineshloka
{तज्ज्ञात्वा सुमदः शीघ्रं पुत्रं राज्येऽभ्यषेचयत्}
{शत्रुघ्नेन महाराज्ञा पुष्कलेनानुमोदितः}% ८

\twolineshloka
{वासांसि बहुरत्नानि धनानि विविधानि च}
{शत्रुघ्नसेवकेभ्योऽसौ प्रादात्तत्र महामतिः}% ९

\twolineshloka
{ततो गमनमारेभे मन्त्रिभिर्बहुवित्तमैः}
{पत्तिभिर्वाजिभिर्नागैः सदश्वैरथ कोटिभिः}% १०

\twolineshloka
{शत्रुघ्नः सहितस्तेन सुमदेन धनुर्भृता}
{जगाम मार्गे विहसन्रघुनाथप्रतापभृत्}% ११

\twolineshloka
{पयोष्णीतीरमासाद्य जगाम स हयोत्तमः}
{पृष्ठतोऽनुययुः सर्वे योधा वै हयरक्षिणः}% १२

\twolineshloka
{आश्रमान्विविधान्पश्यन्नृषीणां सुतपोभृताम्}
{तत्रतत्र विशृण्वानो रघुनाथगुणोदयम्}% १३

\twolineshloka
{एष धीमान्हरिर्याति हरिणा परिरक्षितः}
{हरिभिर्हरिभक्तैश्च हरिवर्यानुगैर्मुहुः}% १४

\twolineshloka
{इति शृण्वञ्छुभा वाचो मुनीनां परितः प्रभुः}
{तुतोष भक्त्युत्कलितचित्तवृत्तिभृतां महान्}% १५

\twolineshloka
{ददर्श चाश्रमं शुद्धं जनजन्तुसमाकुलम्}
{वेदध्वनिहताशेषा मङ्गलं शृण्वतां नृणाम्}% १६

\twolineshloka
{अग्निहोत्रहविर्धूम पवित्रितनभोखिलम्}
{मुनिवर्यकृतानेक यागयूपसुशोभितम्}% १७

\twolineshloka
{यत्र गावस्तु हरिणा पाल्यन्ते पालनोचिताः}
{मूषका न खनन्त्यस्मिन्बिडालस्य भयाद्बिलम्}% १८

\twolineshloka
{मयूरैर्नकुलैः सार्द्धं क्रीडन्ति फणिनोनिशम्}
{गजैः सिंहैर्नित्यमत्र स्थीयते मित्रतां गतैः}% १९

\twolineshloka
{एणास्तत्रत्य नीवारभक्षणेषु कृतादराः}
{न भयं कुर्वते कालाद्रक्षिता मुनिवृन्दकैः}% २०

\twolineshloka
{गावः कुम्भसमोधस्का नन्दिनी समविग्रहाः}
{कुर्वन्ति चरणोत्थेन रजसेलां पवित्रिताम्}% २१

\twolineshloka
{मुनिवर्याः समित्पाणि पद्मैर्धर्मक्रियोचिताम्}
{दृष्ट्वा पप्रच्छसुमतिं सर्वज्ञं राम मन्त्रिणम्}% २२

\uvacha{शत्रुघ्न उवाच}

\twolineshloka
{सुमते कस्य संस्थानं मुनेर्भाति पुरोगतम्}
{निर्वैरिजन्तु संसेव्यं मुनिवृन्दसमाकुलम्}% २३

\twolineshloka
{श्रोष्यामि मुनिवार्तां च विदधामि पवित्रताम्}
{निजं वपुस्तदीयाभिर्वार्ताभिर्वर्णनादिभिः}% २४

\twolineshloka
{इति श्रुत्वा महद्वाक्यं शत्रुघ्नस्य महात्मनः}
{कथयामास सचिवो रघुनाथस्य धीमतः}% २५

\uvacha{सुमतिरुवाच}

\twolineshloka
{च्यवनस्याश्रमं विद्धि महातापसशोभितम्}
{निर्वैरिजन्तुसङ्कीर्णं मुनिपत्नीभिरावृतम्}% २६

\twolineshloka
{योऽसौ महामुनिः स्वर्गवैद्ययोर्भागमादधात्}
{स्वायम्भुवमहायज्ञे शक्रमानविभेदनः}% २७

\twolineshloka
{महामुनेः प्रभावोऽयं न केनापि समाप्यते}
{तपोबलसमृद्धस्य वेदमूर्तिधरस्य ह}% २८

\twolineshloka
{श्रुत्वा रामानुजो वार्तां च्यवनस्य महामुनेः}
{सर्वं पप्रच्छ सुमतिं शक्रमानादिभञ्जनम्}% २९

\uvacha{शत्रुघ्न उवाच}

\twolineshloka
{कदासौ दस्रयोर्भागं चकार सुरपङ्क्तिषु}
{किं कृतं देवराजेन स्वायम्भुव महामखे}% ३०

\uvacha{सुमतिरुवाच}

\twolineshloka
{ब्रह्मवंशेऽतिविख्यातो मुनिर्भृगुरिति श्रुतः}
{कदाचिद्गतवान्सायं समिदाहरणं प्रति}% ३१

\twolineshloka
{तदा मखविनाशाय दमनो राक्षसो बली}
{आगत्योच्चैर्जगादेदं महाभयकरं वचः}% ३२

\twolineshloka
{कुत्रास्ति मुनिबन्धुः स कुत्र तन्महिलानघा}
{पुनः पुनरुवाचेदं वचो रोषसमाकुलः}% ३३

\twolineshloka
{तदाहुतवहो ज्ञात्वा राक्षसाद्भयमागतम्}
{दर्शयामास तज्जायामन्तर्वत्नीमनिन्दिताम्}% ३४

\twolineshloka
{जहार राक्षसस्तां तु रुदन्तीं कुररीमिव}
{भृगो रक्षपते रक्ष रक्ष नाथ तपोनिधे}% ३५

\twolineshloka
{एवं वदन्तीमार्तां तां गृहीत्वा निरगाद्बहिः}
{दुष्टो वाक्यप्रहारेण बोधयन्स भृगोः सतीम्}% ३६

\twolineshloka
{ततो महाभयत्रस्तो गर्भश्चोदरमध्यतः}
{पपात प्रज्वलन्नेत्रो वैश्वानर इवाङ्गजः}% ३७

\twolineshloka
{तेनोक्तं मा व्रजाशु त्वं भस्मी भव सुदुर्मते}
{न हि साध्वी परामर्शं कृत्वा श्रेयोऽधियास्यसि}% ३८

\twolineshloka
{इत्युक्तः स पपाताशु भस्मीभूतकलेवरः}
{माता तदार्भकं नीत्वा जगामाश्रममुन्मनाः}% ३९

\twolineshloka
{भृगुर्वह्निकृतं सर्वं ज्ञात्वा कोपसमाकुलः}
{शशाप सर्वभक्षस्त्वं भव दुष्टारिसूचक}% ४०

\twolineshloka
{तदा शप्तोऽतिदुःखार्तो जग्राहाङ्घ्र्याशुशुक्षणिः}
{कुरु मेऽनुग्रहं स्वामिन्कृपार्णव महामते}% ४१

\twolineshloka
{मयानृतं वचोभीत्या कथितं न गुरुद्रुहा}
{तस्मान्ममोपरि कृपां कुरु धर्मशिरोमणे}% ४२

\twolineshloka
{तदानुग्रहमाधाच्च सर्वभक्षो भवाञ्छुचिः}
{इत्युक्तवान्हुतभुजं दयार्द्रो मुनितापसः}% ४३

\twolineshloka
{गर्भाच्च्युतस्य पुत्रस्य जातकर्मादिकं शुचिः}
{चकार विधिवद्विप्रो दर्भपाणिः सुमङ्गलः}% ४४

\twolineshloka
{च्यवनाच्च्यवनं प्राहुः पुत्रं सर्वे तपस्विनः}
{शनैःशनैः स ववृधे शुक्ले प्रतिपदिन्दुवत्}% ४५

\twolineshloka
{स जगाम तपः कर्तुं रेवां लोकैकपावनीम्}
{शिष्यैः परिवृतः सर्वैस्तपोबलसमन्वितैः}% ४६

\twolineshloka
{गत्वा तत्र तपस्तेपे वर्षाणामयुतं महान्}
{अंसयोः किंशुकौ जातौ वल्मीकोपरिशोभितौ}% ४७

\twolineshloka
{मृगा आगत्य तस्याङ्गे कण्डूं विदधुरुत्सुकाः}
{न किञ्चित्स हि जानाति दुर्वारतपसावृतः}% ४८

\twolineshloka
{कदाचिन्मनुरुद्युक्तस्तीर्थयात्रां प्रति प्रभुः}
{सकुटुम्बो ययौ रेवां महाबलसमावृतः}% ४९

\twolineshloka
{तत्र स्नात्वा महानद्यां सन्तर्प्य पितृदेवताः}
{दानानि ब्राह्मणेभ्यश्च प्रादाद्विष्णुप्रतुष्टये}% ५०

\twolineshloka
{तत्कन्या विचरन्ती सा वनमध्ये इतस्ततः}
{सखीभिः सहिता रम्या तप्तहाटकभूषणा}% ५१

\twolineshloka
{तत्र दृष्ट्वाथ वल्मीकं महातरुसुशोभितम्}
{निमेषोन्मेषरहितं तेजः किञ्चिद्ददर्श सा}% ५२

\twolineshloka
{गत्वा तत्र शलाकाभिरतुदद्रुधिरं स्रवत्}
{दृष्ट्वा राज्ञाङ्गजा खेदं प्राप्तवत्यतिदुःखिता}% ५३

\twolineshloka
{न जनन्यै तथा पित्रे शशंसाघेन विप्लुता}
{स्वयमेवात्मनात्मानं सा शुशोच भयातुरा}% ५४

\twolineshloka
{तदा भूश्चलिता राजन्दिवश्चोल्का पपात ह}
{धूम्रा दिशो भवन्सर्वाः सूर्यश्च परिवेषितः}% ५५

\twolineshloka
{तदा राज्ञो हया नष्टा हस्तिनो बहवो मृताः}
{धनं नष्टं रत्नयुतं कलहोभून्मिथस्तदा}% ५६

\twolineshloka
{तदालोक्य नृपो भीतः किञ्चिदुद्विग्नमानसः}
{जनानपृच्छत्केनापि मुनये त्वपराधितम्}% ५७

\twolineshloka
{पारम्पर्येण तज्ज्ञात्वा स्वपुत्र्याः परिचेष्टितम्}
{ययौ सुदुःखितस्तत्र समृद्धबलवाहनः}% ५८

\twolineshloka
{तं वै तपोनिधिं वीक्ष्य महता तपसायुतम्}
{स्तुत्वा प्रसादयामास मुनिवर्य दयां कुरु}% ५९

\twolineshloka
{तस्मै तुष्टो जगादायं मुनिवर्यो महातपाः}
{तवात्मजाकृतं सर्वमुत्पाताद्यमवेहि तत्}% ६०

\twolineshloka
{तव पुत्र्या महाराज चक्षुर्विस्फोटनं कृतम्}
{बहुसुस्राव रुधिरं जानती त्वामुवाच न}% ६१

\twolineshloka
{तस्मादियं महाभूप मह्यं देया यथाविधि}
{ततश्चोत्पातशमनं भविष्यति न संशयः}% ६२

\twolineshloka
{तच्छ्रुत्वा दुःखितो राजा प्रज्ञाचाक्षुष आत्मजाम्}
{ददौ कुलवयोरूप शीललक्षणसंयुताम्}% ६३

\twolineshloka
{दत्ता यदा नृपेणेयं कन्या कमललोचना}
{तदोत्पाताः शमं याताः सर्वे मुनिरुषोद्गताः}% ६४

\twolineshloka
{राजा दत्त्वात्मजां तस्मै मुनये तपसान्निधे}
{प्राप स्वां नगरीं भूयो दुःखितोऽयं दयायुतः}% ६५

{॥इति श्रीपद्मपुराणे पातालखण्डे शेषवात्स्यायनसंवादे रामाश्वमेधे च्यवनोपाख्यानं नाम चतुर्दशोऽध्यायः॥१४॥}

\dnsub{पञ्चदशोऽध्यायः}\resetShloka

\uvacha{सुमतिरुवाच}

\twolineshloka
{अथर्षिः स्वाश्रमं गत्वा मानव्या सह भार्यया}
{मुदं प्राप हताशेष पातको योगयुक्तया}% १

\fourlineindentedshloka
{सा मानवी तं वरमात्मनः पतिं}
{नेत्रेणहीनं जरसा गतौजसम्}
{सिषेव एनं हरिमेधसोत्तमं}
{निजेष्टदात्रीं कुलदेवतां यथा}% २

\fourlineindentedshloka
{शूश्रूषती स्वं पतिमिङ्गितज्ञा}
{महानुभावं तपसां निधिं प्रियम्}
{परां मुदं प्राप सती मनोहरा}
{शची यथा शक्रनिषेवणोद्यता}% ३

\twolineshloka
{चरणौ सेवते तन्वी सर्वलक्षणलक्षिता}
{राजपुत्री सुन्दराङ्गी फलमूलोदकाशना}% ४

\twolineshloka
{नित्यं तद्वाक्यकरणे तत्परा पूजने रता}
{कालक्षेपं प्रकुरुते सर्वभूतहिते रता}% ५

\twolineshloka
{विसृज्य कामं दम्भं च द्वेषं लोभमघं मदम्}
{अप्रमत्तोद्यता नित्यं च्यवनं समतोषयत्}% ६

\twolineshloka
{एवं तस्य प्रकुर्वाणा सेवां वाक्कायकर्मभिः}
{सहस्राब्दं महाराज सा च कामं मनस्यधात्}% ७

\twolineshloka
{कदाचिद्देवभिषजावागतावाश्रमे मुनेः}
{स्वागतेन सुसम्भाव्य तयोः पूजां चकार सा}% ८

\fourlineindentedshloka
{शर्यातिकन्याकृतपूजनार्घ-}
{पाद्यादिना तोषितचित्तवृत्ती}
{तावूचतुः स्नेहवशेन सुन्दरौ}
{वरं वृणुष्वेति मनोहराङ्गीम्}% ९

\twolineshloka
{तुष्टौ तौ वीक्ष्य भिषजौ देवानां वरयाचने}
{मतिं चकार नृपतेः पुत्री मतिमतां वरा}% १०

\twolineshloka
{पत्यभिप्रायमालक्ष्य वाचमूचे नृपात्मजा}
{दत्तं मे चक्षुषी पत्युर्यदि तुष्टौ युवां सुरौ}% ११

\twolineshloka
{इत्येतद्वचनं श्रुत्वा सुकन्या या मनोहरम्}
{सतीत्वं च विलोक्येदमूचतुर्भिषजां वरौ}% १२

\twolineshloka
{त्वत्पतिर्यदि देवानां भागं यज्ञे दधात्यसौ}
{आवयोरधुना कुर्वश्चक्षुषोः स्फुटदर्शनम्}% १३

\twolineshloka
{च्यवनोऽप्योमिति प्राह भागदाने वरौजसोः}
{तदा हृष्टावश्विनौ तमूचतुस्तपतां वरम्}% १४

\twolineshloka
{निमज्जतां भवानस्मिन्ह्रदे सिद्धविनिर्मिते}
{इत्युक्तो जरयाग्रस्त देहो धमनिसन्ततः}% १५

\twolineshloka
{ह्रदं प्रवेशितोऽश्विभ्यां स्वयं चामज्जतां ह्रदे}
{पुरुषास्त्रय उत्तस्थुरपीच्या वनिताप्रियाः}% १६

\twolineshloka
{रुक्मस्रजः कुण्डलिनस्तुल्यरूपाः सुवाससः}
{तान्निरीक्ष्य वरारोहा सुरूपान्सूर्यवर्चसः}% १७

\twolineshloka
{अजानती पतिं साध्वी ह्यश्विनौ शरणं ययौ}
{दर्शयित्वा पतिं तस्यै पातिव्रत्येन तोषितौ}% १८

\twolineshloka
{ऋषिमामन्त्र्य ययतुर्विमानेन त्रिविष्टपम्}
{यक्ष्यमाणे क्रतौ स्वीयभागकार्याशयायुतौ}% १९

\twolineshloka
{कालेन भूयसा क्षामां कर्शितां व्रतचर्यया}
{प्रेमगद्गदया वाचा पीडितः कृपयाब्रवीत्}% २०

\fourlineindentedshloka
{तुष्टोऽहमद्य तव भामिनि मानदायाः}
{शुश्रूषया परमया हृदि चैकभक्त्या}
{यो देहिनामयमतीव सुहृत्स्वदेहो}
{नावेक्षितः समुचितः क्षपितुं मदर्थे}% २१

\fourlineindentedshloka
{ये मे स्वधर्मनिरतस्य तपः समाधि}
{विद्यात्मयोगविजिता भगवत्प्रसादाः}
{तानेव ते मदनुसेवनयाऽविरुद्धान्}
{दृष्टिं प्रपश्य वितराम्यभयानशोकान्}% २२

\fourlineindentedshloka
{अन्ये पुनर्भगवतो भ्रुव उद्विजृम्भ-}
{विस्रंसितार्थरचनाः किमुरुक्रमस्य}
{सिद्धासि भुङ्क्ष्व विभवान्निजधर्मदोहान्}
{दिव्यान्नरैर्दुरधिगान्नृपविक्रियाभिः}% २३

\fourlineindentedshloka
{एवं ब्रुवाणमबलाखिलयोगमाया}
{विद्याविचक्षणमवेक्ष्य गताधिरासीत्}
{सम्प्रश्रयप्रणयविह्वलया गिरेषद्}
{व्रीडाविलोकविलसद्धसिताननाह}% २४

\uvacha{सुकन्योवाच}

\fourlineindentedshloka
{राद्धं बत द्विजवृषैतदमोघयोग-}
{मायाधिपे त्वयि विभो तदवैमि भर्तः}
{यस्तेऽभ्यधायि समयः सकृदङ्गसङ्गो}
{भूयाद्गरीयसि गुणः प्रसवः सतीनाम्}% २५

\fourlineindentedshloka
{तत्रेति कृत्यमुपशिक्ष्य यथोपदेशं}
{येनैष कर्शिततमोति रिरंसयात्मा}
{सिध्येत ते कृतमनोभव धर्षिताया}
{दीनस्तदीशभवनं सदृशं विचक्ष्व}% २६

\uvacha{सुमतिरुवाच}

\twolineshloka
{प्रियायाः प्रियमन्विच्छंश्च्यवनो योगमास्थितः}
{विमानं कामगं राजंस्तर्ह्येवाविरचीकरत्}% २७

\twolineshloka
{सर्वकामदुघं रम्यं सर्वरत्नसमन्वितम्}
{सर्वार्थोपचयोदर्कं मणिस्तम्भैरुपस्कृतम्}% २८

\twolineshloka
{दिव्योपस्तरणोपेतं सर्वकालसुखावहम्}
{पट्टिकाभिः पताकाभिर्विचित्राभिरलङ्कृतम्}% २९

\twolineshloka
{स्रग्भिर्विचित्रमालाभिर्मञ्जुसिञ्जत्षडङ्घ्रिभिः}
{दुकूलक्षौमकौशेयैर्नानावस्त्रैर्विराजितम्}% ३०

\twolineshloka
{उपर्युपरि विन्यस्तनिलयेषु पृथक्पृथक्}
{कॢप्तैः कशिपुभिः कान्तं पर्यङ्कव्यजनादिभिः}% ३१

\twolineshloka
{तत्रतत्र विनिक्षिप्त नानाशिल्पोपशोभितम्}
{महामरकतस्थल्या जुष्टं विद्रुमवेदिभिः}% ३२

\twolineshloka
{द्वाःसु विद्रुमदेहल्या भातं वज्रकपाटकम्}
{शिखरेष्विन्द्रनीलेषु हेमकुम्भैरधिश्रितम्}% ३३

\twolineshloka
{चक्षुष्मत्पद्मरागाग्र्यैर्वज्रभित्तिषु निर्मितैः}
{जुष्टं विचित्रवैतानैर्मुक्ताहारावलम्बितैः}% ३४

\twolineshloka
{हंसपारावतव्रातैस्तत्र तत्र निकूजितम्}
{कृत्रिमान्मन्यमानैस्तानधिरुह्याधिरुह्य च}% ३५

\twolineshloka
{विहारस्थानविश्राम संवेश प्राङ्गणाजिरैः}
{यथोपजोषं रचितैर्विस्मापनमिवात्मनः}% ३६

\twolineshloka
{एवं गृहं प्रपश्यन्तीं नातिप्रीतेन चेतसा}
{सर्वभूताशयाभिज्ञः स्वयं प्रोवाच तां प्रति}% ३७

\twolineshloka
{निमज्ज्यास्मिन्ह्रदे भीरु विमानमिदमारुह}
{सुभ्रूर्भर्तुः समादाय वचः कुवलयेक्षणा}% ३८

\twolineshloka
{सरजो बिभ्रती वासो वेणीभूतांश्च मूर्द्धजान्}
{अङ्गं च मलपङ्केन सञ्छन्नं शबलस्तनम्}% ३९

\twolineshloka
{आविवेश सरस्तत्र मुदा शिवजलाशयम्}
{सान्तःसरसि वेश्मस्थाः शतानि दशकन्यकाः}% ४०

\twolineshloka
{सर्वाः किशोरवयसो ददर्शोत्पलगन्धयः}
{तां दृष्ट्वा शीघ्रमुत्थाय प्रोचुः प्राञ्जलयः स्त्रियः}% ४१

\twolineshloka
{वयं कर्मकरीस्तुभ्यं शाधि नः करवाम किम्}
{स्नानेन ता महार्हेण स्नापयित्वा मनस्विनीम्}% ४२

\twolineshloka
{दुकूले निर्मले नूत्ने ददुरस्यै च मानद}
{भूषणानि परार्घ्यानि वरीयांसि द्युमन्ति च}% ४३

\twolineshloka
{अन्नं सर्वगुणोपेतं पानं चैवामृतासवम्}
{अथादर्शे स्वमात्मानं स्रग्विणं विरजोम्बरम्}% ४४

\twolineshloka
{ताभिः कृतस्वस्त्ययनं कन्याभिर्बहुमानितम्}
{हारेण च महार्हेण रुचकेन च भूषितम्}% ४५

\twolineshloka
{निष्कग्रीवं वलयिनं क्वणत्काञ्चननूपुरम्}
{श्रोण्योरध्यस्तया काञ्च्या काञ्चन्या बहुरत्नया}% ४६

\twolineshloka
{सुभ्रुवा सुदता शुक्लस्निग्धापाङ्गेन चक्षुषा}
{पद्मकोशस्पृधा नीलैरलकैश्च लसन्मुखम्}% ४७

\twolineshloka
{यदा सस्मार दयितमृषीणां वल्लभं पतिम्}
{तत्र चास्ते सहस्त्रीभिर्यत्रास्ते स मुनीश्वरः}% ४८

\twolineshloka
{भर्तुः पुरस्तादात्मानं स्त्रीसहस्रवृतं तदा}
{निशाम्य तद्योगगतिं संशयं प्रत्यपद्यत}% ४९

\twolineshloka
{सतां कृत मलस्नानां विभ्राजन्तीमपूर्ववत्}
{आत्मनो बिभ्रतीं रूपं संवीतरुचिरस्तनीम्}% ५०

\twolineshloka
{विद्याधरी सहस्रेण सेव्यमानां सुवाससम्}
{जातभावो विमानं तदारोहयदमित्रहन्}% ५१

\fourlineindentedshloka
{तस्मिन्नलुप्तमहिमा प्रिययानुषक्तो}
{विद्याधरीभिरुपचीर्णवपुर्विमाने}
{बभ्राज उत्कचकुमुद्गणवानपीच्य}
{स्ताराभिरावृत इवोडुपतिर्नभःस्थः}% ५२

\fourlineindentedshloka
{तेनाष्टलोकपविहारकुलाचलेन्द्र-}
{द्रोणीष्वनङ्गसखमारुतसौभगासु}
{सिद्धैर्नुतोद्युधुनिपातशिवस्वनासु}
{रेमे चिरं धनदवल्ललनावरूथी}% ५३

\twolineshloka
{वैश्रम्भके सुरवने नन्दने पुष्पभद्रके}
{मानसे चैत्ररथ्ये च सरे मे रामया रतः}% ५४

{॥इति श्रीपद्मपुराणे पातालखण्डे शेषवात्स्यायनसंवादे रामाश्वमेधे च्यवनस्य तपोभोगवर्णनं नाम पञ्चदशोऽध्यायः॥१५॥}

\dnsub{षोडशोऽध्यायः}\resetShloka

\uvacha{सुमतिरुवाच}

\twolineshloka
{एवं तया क्रीडमानः सर्वत्र धरणीतले}
{नाबुध्यत गतानब्दाञ्छतसङ्ख्या परीमितान्}% १

\twolineshloka
{ततो ज्ञात्वाथव तद्विप्रः स्वकालपरिवर्तिनीम्}
{मनोरथैश्च सम्पूर्णां स्वस्यप्रियतमां वराम्}% २

\twolineshloka
{न्यवर्तताश्रमं श्रेष्ठं पयोष्णीतीरसंस्थितम्}
{निर्वैरजं तु जनतासङ्कुलं मृगसेवितम्}% ३

\twolineshloka
{तत्रावसत्स सुतपाः शिष्यैर्वेदसमन्वितैः}
{सेविताङ्घ्रियुगो नित्यं तताप परमं तपः}% ४

\twolineshloka
{कदाचिदथ शर्यातिर्यष्टुमैच्छत देवताः}
{तदा च्यवनमानेतुं प्रेषयामास सेवकान्}% ५

\twolineshloka
{तैराहूतो द्विजवरस्तत्रागच्छन्महातपाः}
{सुकन्यया धर्मपत्न्या स्वाचार परिनिष्ठया}% ६

\twolineshloka
{आगतं तं मुनिवरं पत्न्या पुत्र्या महायशाः}
{ददर्श दुहितुः पार्श्वे पुरुषं सूर्यवर्चसम्}% ७

\twolineshloka
{राजा दुहितरं प्राह कृतपादाभिवन्दनाम्}
{आशिषो न प्रयुञ्जानो नातिप्रीतमना इव}% ८

\fourlineindentedshloka
{चिकीर्षितं ते किमिदं पतिस्त्वया}
{प्रलम्भितो लोकनमस्कृतो मुनिः}
{त्वया जराग्रस्तमसम्मतं पतिं}
{विहाय जारं भजसेऽमुमध्वगम्}% ९

\twolineshloka
{कथं मतिस्तेऽवगतान्यथासतां कुलप्रसूतेः कुलदूषणं त्विदम्}
{बिभर्षि जारं यदपत्रपाकुलं पितुः स्वभर्तुश्च नयस्यधस्तमाम्}% १०

\twolineshloka
{एवं ब्रुवाणं पितरं स्मयमाना शुचिस्मिता}
{उवाच तात जामाता तवैष भृगुनन्दनः}% ११

\twolineshloka
{शशंस पित्रे तत्सर्वं वयोरूपाभिलम्भनम्}
{विस्मितः परमप्रीतस्तनयां परिषस्वजे}% १२

\twolineshloka
{सोमेनायाजयद्वीरं ग्रहं सोमस्य चाग्रहीत्}
{असोमपोरप्यश्विनोश्च्यवनः स्वेन तेजसा}% १३

\twolineshloka
{ग्रहं तु ग्राहयामास तपोबलसमन्वितः}
{वज्रं गृहीत्वा शक्रस्तु हन्तुं ब्राह्मणसत्तमम्}% १४

\twolineshloka
{अपङ्क्तिपावनौ देवौ कुर्वाणं पङ्क्तिगोचरौ}
{शक्रं वज्रधरं दृष्ट्वा मुनिः स्वहननोद्यतम्}% १५

\twolineshloka
{हुङ्कारमकरोद्धीमान्स्तम्भयामास तद्भुजम्}
{इन्द्रः स्तब्धभुजस्तत्र दृष्टः सर्वैश्च मानवैः}% १६

\twolineshloka
{कोपेन श्वसमानोऽहिर्यथा मन्त्रनियन्त्रितः}
{तुष्टाव स मुनिं शक्रः स्तब्धबाहुस्तपोनिधिम्}% १७

\twolineshloka
{अश्विभ्यां भागदानं च कुर्वन्तं निर्भयान्तरम्}
{कथयामास भोः स्वामिन्दीयतामश्विनोर्बलि}% १८

\twolineshloka
{मया न वार्यते तात क्षमस्वाघं महत्कृतम्}
{इत्युक्तः स मुनिः कोपं जहौ तूर्णं कृपानिधिः}% १९

\twolineshloka
{इन्द्रो मुक्तभुजोऽप्यासीत्तदानीं पुरुषर्षभ}
{एतद्वीक्ष्य जनाः सर्वे कौतुकाविष्टमानसाः}% २०

\twolineshloka
{शशंसुर्ब्राह्मणबलं ते तु देवादिदुर्ल्लभम्}
{ततो राजा बहुधनं ब्राह्मणेभ्योऽददन्महान्}% २१

\twolineshloka
{चक्रे चावभृथस्नानं यागान्ते शत्रुतापनः}
{त्वया पृष्टं यदाचक्ष्व च्यवनस्य महोदयम्}% २२

\twolineshloka
{स मया कथितः सर्वस्तपोयोगसमन्वितः}
{नमस्कृत्वा तपोमूर्तिमिमं प्राप्य जयाशिषः}% २३

\onelineshloka*
{प्रेषय त्वं सपत्नीकं रामयज्ञे मनोरमे}

\uvacha{शेष उवाच}

\onelineshloka
{एवं तु कुर्वतोर्वार्तां हयः प्रापाश्रमं प्रति}% २४

\twolineshloka
{विदधद्वायुवेगेन पृथ्वीं खुरविलक्षिताम्}
{दूर्वाङ्कुरान्मुखाग्रेण चरंस्तत्र महाश्रमे}% २५

\twolineshloka
{मुनयो यावदादाय दर्भान्स्नातुं गता नदीम्}
{शत्रुघ्नः शत्रुसेनायास्तापनः शूरसम्मतः}% २६

\twolineshloka
{तावत्प्राप मुनेर्वासं च्यवनस्यातिशोभितम्}
{गत्वा तदाश्रमे वीरो ददर्श च्यवनं मुनिम्}% २७

\twolineshloka
{सुकन्यायाः समीपस्थं तपोमूर्तिमिवस्थितम्}
{ववन्दे चरणौ तस्य स्वाभिधां समुदाहरन्}% २८

\onelineshloka
{शत्रुघ्नोहं रघुपतेर्भ्राता वाहस्य पालकः}% २९

\twolineshloka
{नमस्करोमि युष्मभ्यं महापापोपशान्तये}
{इति वाक्यं समाकर्ण्य जगाद मुनिसत्तमः}% ३०

\twolineshloka
{शत्रुघ्न तव कल्याणं भूयान्नरवरर्षभ}
{यज्ञं पालयमानस्य कीर्तिस्ते विपुला भवेत्}% ३१

\twolineshloka
{चित्रं पश्यत भो विप्रा रामोऽपि मखकारकः}
{यन्नामस्मरणादीनि कुर्वन्ति पापनाशनम्}% ३२

\twolineshloka
{महापातकसंयुक्ताः परदाररता नराः}
{यन्नामस्मरणोद्युक्ता मुक्ता यान्ति परां गतिम्}% ३३

\twolineshloka
{पादपद्मसमुत्थेन रेणुना ग्रावमूर्तिभृत्}
{तत्क्षणाद्गौतमार्धाङ्गी जाता मोहनरूपधृक्}% ३४

\twolineshloka
{मामकीयस्य रूपस्य ध्यानेन प्रेमनिर्भरा}
{सर्वपातकराशिं सा दग्ध्वा प्राप्ता सुरूपताम्}% ३५

\twolineshloka
{दैत्या यस्य मनोहारिरूपं प्रधनमण्डले}
{पश्यन्तः प्रापुरेतस्य रूपं विकृतिवर्जितम्}% ३६

\twolineshloka
{योगिनो ध्याननिष्ठा ये यं ध्यात्वा योगमास्थिताः}
{संसारभयनिर्मुक्ताः प्रयाताः परमं पदम्}% ३७

\twolineshloka
{धन्योऽहमद्य रामस्य मुखं द्रक्ष्यामि शोभनम्}
{पयोजदलनेत्रान्तं सुनसं सुभ्रुसून्नतम्}% ३८

\twolineshloka
{सा जिह्वा रघुनाथस्य नामकीर्तनमादरात्}
{करोति विपरीता या फणिनो रसना समा}% ३९

\twolineshloka
{अद्य प्राप्तं तपःपुण्यमद्य पूर्णा मनोरथाः}
{यद्द्रक्ष्ये रामचन्द्रस्य मुखं ब्रह्मादिदुर्ल्लभम्}% ४०

\twolineshloka
{तत्पादरेणुना स्वाङ्गं पवित्रं विदधाम्यहम्}
{विचित्रतरवार्ताभिः पावये रसनां स्वकाम्}% ४१

\fourlineindentedshloka
{इत्यादि रामचरणस्मरणप्रबुद्ध-}
{प्रेमव्रजप्रसृतगद्गदवागुदश्रुः}
{श्रीरामचन्द्र रघुपुङ्गवधर्ममूर्ते}
{भक्तानुकम्पकसमुद्धर संसृतेर्माम्}% ४२

\twolineshloka
{जल्पन्नश्रुकलापूर्णो मुनीनां पुरतस्तदा}
{नाज्ञासीत्तत्र पारक्यं निजं ध्यानेन संस्थितः}% ४३

\twolineshloka
{शत्रुघ्नस्तं मुनिं प्राह स्वामिन्नो मखसत्तमः}
{क्रियतां भवता पादरजसा सुपवित्रितः}% ४४

\twolineshloka
{महद्भाग्यं रघुपतेर्यद्युष्मन्मानसान्तरे}
{तिष्ठत्यसौ महाबाहुः सर्वलोकसुपूजितः}% ४५

\twolineshloka
{इत्युक्तः सपरीवारः सर्वाग्निपरिसंवृतः}
{जगाम च्यवनस्तत्र प्रमोदह्रदसम्प्लुतः}% ४६

\twolineshloka
{हनूमांस्तं पदायान्तं रामभक्तमवेक्ष्य ह}
{शत्रुघ्नं निजगादासौ वचो विनयसंयुतः}% ४७

\twolineshloka
{स्वामिन्कथयसि त्वं चेन्महापुरुषसुन्दरम्}
{रामभक्तं मुनिवरं नयामि स्वपुरीमहम्}% ४८

\twolineshloka
{इति श्रुत्वा महद्वाक्यं कपिवीरस्य शत्रुहा}
{आदिदेश हनूमन्तं गच्छ प्रापयतं मुनिम्}% ४९

\twolineshloka
{हनूमांस्तं मुनिं स्वीये पृष्ठ आरोप्य वेगवान्}
{सकुटुम्बं निनायाशु वायुः ख इव सर्वगः}% ५०

\twolineshloka
{आगतं तं मुनिं दृष्ट्वा रामो मतिमतां वरः}
{अर्घ्यपाद्यादिकं चक्रे प्रीतः प्रणयविह्वलः}% ५१

\twolineshloka
{धन्योऽस्मि मुनिवर्यस्य दर्शनेन तवाधुना}
{पवित्रितो मखो मह्यं सर्वसम्भारसम्भृतः}% ५२

\twolineshloka
{इति वाक्यं समाकर्ण्य च्यवनो मुनिसत्तमः}
{उवाच प्रेमनिर्भिन्न पुलकाङ्गोऽतिनिर्वृतः}% ५३

\twolineshloka
{स्वामिन्ब्रह्मण्यदेवस्य तव वाडवपूजनम्}
{युक्तमेव महाराज धर्ममार्गं प्ररक्षितुः}% ५४

{॥इति श्रीपद्मपुराणे पातालखण्डे शेषवात्स्यायनसंवादे रामाश्वमेधे च्यवनाश्रमे हयगमनो नाम षोडशोऽध्यायः॥१६॥}

\dnsub{सप्तदशोऽध्यायः}\resetShloka

\uvacha{शेष उवाच}

\twolineshloka
{शत्रुघ्नश्च्यवनस्याथ दृष्ट्वाऽचिन्त्यं तपोबलम्}
{प्रशशंस तपो ब्राह्मं सर्वलोकैकवन्दितम्}% १

\twolineshloka
{अहो पश्यत योगस्य सिद्धिं ब्राह्मणसत्तमे}
{यः क्षणादेव दुष्प्रापं तद्विमानमचीकरत्}% २

\twolineshloka
{क्व भोगसिद्धिर्महती मुनीनाममलात्मनाम्}
{क्व तपोबलहीनानां भोगेच्छा मनुजात्मनाम्}% ३

\twolineshloka
{इति स्वगतमाशंसञ्छत्रुघ्नश्च्यवनाश्रमे}
{क्षणं स्थित्वा जलं पीत्वा सुखसम्भोगमाप्तवान्}% ४

\twolineshloka
{हयस्तस्याः पयोष्ण्याख्या नद्याः पुण्यजलात्मनः}
{पयः पीत्वा ययौ मार्गे वायुवेगगतिर्महान्}% ५

\twolineshloka
{योधास्तन्निर्गमं दृष्ट्वा पृष्ठतोऽनुययुस्तदा}
{हस्तिभिः पत्तिभिः केचिद्रथैः केचन वाजिभिः}% ६

\twolineshloka
{शत्रुघ्नोऽमात्यवर्येण सुमत्याख्येन संयुतः}
{पृष्ठतोऽनुजगामाशु रथेन हयशोभिना}% ७

\twolineshloka
{गच्छन्वाजीपुरं प्राप्तो विमलाख्यस्य भूपतेः}
{रत्नातटाख्यं च जनैर्हृष्टपुष्टैः समाकुलम्}% ८

\twolineshloka
{स सेवकादुपश्रुत्य रघुनाथ हयोत्तमम्}
{पुरोन्तिके हि सम्प्राप्तं सर्वयोधसमन्वितम्}% ९

\twolineshloka
{तदा गजानां सप्तत्या चन्द्रवर्णसमानया}
{अश्वानामयुतैः सार्धं रथानां काञ्चनत्विषाम्}% १०

\twolineshloka
{सहस्रेण च संयुक्तः शत्रुघ्नं प्रति जग्मिवान्}
{शत्रुघ्नं स नमस्कृत्य सर्वान्प्राप्तान्महारथान्}% ११

\twolineshloka
{वसुकोशं धनं सर्वं राज्यं तस्मै निवेद्य च}
{किं करोमीति राजा तं जगाद पुरतः स्थितः}% १२

\fourlineindentedshloka
{राजापि तं स्वीयपदे प्रणम्रं}
{दोर्भ्यां दृढं सम्परिषस्वजे महान्}
{जगाम साकं तनये स्वराज्यं}
{निक्षिप्य सर्वं बहुधन्विभिर्वृतः}% १३

\twolineshloka
{रामचन्द्राभिधां श्रुत्वा सर्वश्रुतिमनोहराम्}
{सर्वे प्रणम्य तं वाहं ददुर्वसुमहाधनम्}% १४

\twolineshloka
{राजानं पूजयित्वा तु शत्रुघ्नः परया मुदा}
{सेनया सहितोऽगच्छद्वाजिनः पृष्ठतस्तदा}% १५

\twolineshloka
{एवं स गच्छंस्तन्मार्गे पर्वताग्र्यं ददर्श ह}
{स्फाटिकैः कानकै रौप्यै राजितं प्रस्थराजिभिः}% १६

\twolineshloka
{जलनिर्झरसंह्रादं नानाधातुकभूतलम्}
{गैरिकादिकसद्धातु लाक्षारङ्गविराजितम्}% १७

\twolineshloka
{यत्र सिद्धाङ्गनाः सिद्धैः सङ्क्रीडन्त्यकुतोभयाः}
{गन्धर्वाप्सरसो नागा यत्र क्रीडन्ति लीलया}% १८

\twolineshloka
{गङ्गातरङ्गसंस्पर्श शीतवायुनिषेवितम्}
{वीणारणद्धंसशुकक्वणसुन्दरशोभितम्}% १९

\twolineshloka
{पर्वतं वीक्ष्य शत्रुघ्न उवाच सुमतिं त्विदम्}
{तद्दर्शनसमुद्भूत विस्मयाविष्टमानसः}% २०

\twolineshloka
{कोऽयं महागिरिवरो विस्मापयति मे मनः}
{महारजतसत्प्रस्थो मार्गे राजति मेऽद्भुतः}% २१

\twolineshloka
{अत्र किं देवतावासो देवानां क्रीडनस्थलम्}
{यदेतन्मनसः क्षोभं करोति श्रीसमुच्चयैः}% २२

\twolineshloka
{इति वाक्यं समाकर्ण्य जगाद सुमतिस्तदा}
{वक्ष्यमाणगुणागार रामचन्द्र पदाब्जधीः}% २३

\twolineshloka
{नीलोऽयं पर्वतो राजन्पुरतो भाति भूमिप}
{मनोहरैर्महाशृङ्गैः स्फाटिकाग्रैः समन्ततः}% २४

\twolineshloka
{एनं पश्यन्ति नो पापाः परदाररता नराः}
{विष्णोर्गुणगणान्ये वै न मन्यन्ते नराधमाः}% २५

\twolineshloka
{श्रुतिस्मृतिसमुत्थं ये धर्मं सद्भिः सुसाधितम्}
{न मन्यन्ते स्वबुद्धिस्थ हेतुवादविचारणाः}% २६

\twolineshloka
{नीलीविक्रयकर्तारो लाक्षाविक्रयकारकाः}
{यो ब्राह्मणो घृतादीनि विक्रीणाति सुरापकः}% २७

\twolineshloka
{कन्यां रूपेण सम्पन्नां न दद्यात्कुलशीलिने}
{विक्रीणाति द्रव्यलोभात्पिता पापेन मोहितः}% २८

\twolineshloka
{पत्नीं दूषयते यस्तु कुलशीलवतीं नरः}
{स्वयमेवात्ति मधुरं बन्धुभ्यो न ददाति यः}% २९

\twolineshloka
{भोजने ब्राह्मणार्थे च पाकभेदं करोति यः}
{कृसरं पायसं वापि नार्थिनं दापयेत्कुधीः}% ३०

\twolineshloka
{अतिथीनवमन्यन्ते सूर्यतापादितापितान्}
{अन्तरिक्षभुजो ये च ये च विश्वासघातकाः}% ३१

\twolineshloka
{न पश्यन्ति महाराज रघुनाथ पराङ्मुखाः}
{असौ पुण्यो गिरिवरः पुरुषोत्तम शोभितः}% ३२

\twolineshloka
{पवित्रयति सर्वान्नो दर्शनेन मनोहरः}
{अत्र तिष्ठति देवानां मुकुटैरर्चिताङ्घ्रिकः}% ३३

\twolineshloka
{पुण्यवद्भिर्दर्शनार्हः पुण्यदः पुरुषोत्तमः}
{श्रुतयो नेतिनेतीति ब्रुवाणा न विदन्ति यम्}% ३४

\twolineshloka
{यत्पादरज इन्द्रादिदेवैर्मृग्यं सुदुर्ल्लभम्}
{वेदान्तादिभिरन्यूनैर्वाक्यैर्विदन्ति यं बुधाः}% ३५

\twolineshloka
{सोऽत्र श्रीमान्नीलशैले वसते पुरुषोत्तमः}
{आरुह्य तं नमस्कृत्य सम्पूज्य सुकृतादिना}% ३६

\twolineshloka
{नैवेद्यं भक्षयित्वा वै भूप भूयाच्चतुर्भुजः}
{अत्राप्युदाहरन्तीममितिहासं पुरातनम्}% ३७

\twolineshloka
{तं शृणुष्व महाराज सर्वाश्चर्यसमन्वितम्}
{रत्नग्रीवस्य नृपतेर्यद्वृत्तं सकुटुम्बिनः}% ३८

\twolineshloka
{चतुर्भुजादिकं प्राप्तं देवदानवदुर्लभम्}
{आसीत्काञ्ची महाराज पुरी लोकेषु विश्रुता}% ३९

\twolineshloka
{महाजनपरीवारसमृद्धबलवाहना}
{यस्यां वसन्ति विप्राग्र्याः षट्कर्मनिरता भृशम्}% ४०

\twolineshloka
{सर्वभूतहिते युक्ता रामभक्तिषु लालसाः}
{क्षत्रिया रणकर्तारः सङ्ग्रामेऽप्यपलायिनः}% ४१

\twolineshloka
{परदार परद्रव्य परद्रोहपराङ्मुखाः}
{वैश्याः कुसीदकृष्यादिवाणिज्यशुभवृत्तयः}% ४२

\twolineshloka
{कुर्वन्ति रघुनाथस्य पदाम्भोजे रतिं सदा}
{शूद्रा ब्राह्मणसेवाभिर्गतरात्रिदिनान्तराः}% ४३

\twolineshloka
{कुर्वन्ति कथनं रामरामेति रसनाग्रतः}
{प्राकृताः केऽपि नो पापं कुर्वन्ति मनसात्र वै}% ४४

\twolineshloka
{दानं दया दमः सत्यं तत्र तिष्ठन्ति नित्यशः}
{वदते न पराबाधं वाक्यं कोऽपि नरोऽनघः}% ४५

\twolineshloka
{न पारक्ये धने लोभं कुर्वन्ति न हि पातकम्}
{एवं प्रजा महाराज रत्नग्रीवेण पाल्यते}% ४६

\twolineshloka
{षष्ठांशं तत्र गृह्णाति नान्यं लोभविवर्जितः}
{एवं पालयमानस्य प्रजाधर्मेण भूपतेः}% ४७

\twolineshloka
{गतानि बहुवर्षाणि सर्वभोगविलासिनः}
{विशालाक्षीं महाराज एकदा ह्यूचिवानिदम्}% ४८

\twolineshloka
{पतिव्रतां धर्मपत्नीं पतिव्रतपरायणाम्}
{पुत्रा जाता विशालाक्षि प्रजारक्षा धुरन्धराः}% ४९

\twolineshloka
{परीवारो महान्मह्यं वर्तते विगतज्वरः}
{हस्तिनो मम शैलाभा वाजिनः पवनोपमाः}% ५०

\twolineshloka
{रथाश्च सुहयैर्युक्ता वर्तन्ते मम नित्यशः}
{महाविष्णुप्रसादेन किञ्चिन्न्यूनं ममास्ति न}% ५१

\twolineshloka
{एवं मनोरथस्त्वेकस्तिष्ठते मानसे मम}
{परं तीर्थं मया नाद्य कृतं परमशोभने}% ५२

\twolineshloka
{गर्भवासविरामाय क्षमं गोविन्दशोभितम्}
{वृद्धो जातोऽस्म्यहं तावद्वलीपलितदेहवान्}% ५३

\twolineshloka
{करिष्यामि मनोहारि तीर्थसेवनमादृतः}
{यो नरो जन्मपर्यन्तं स्वोदरस्य प्रपूरकः}% ५४

\twolineshloka
{न करोति हरेः पूजां स नरो गोवृषः स्मृतः}
{तस्माद्गच्छामि भो भद्रे तीर्थयात्रां प्रति प्रिये}% ५५

\twolineshloka
{सकुटुम्बः सुते न्यस्य धुरं राज्यस्य निर्भृताम्}
{इति व्यवस्य सन्ध्यायां हरिं ध्यायन्निशान्तरे}% ५६

\twolineshloka
{अद्राक्षीत्स्वप्नमप्येकं ब्राह्मणं तापसं वरम्}
{प्रातरुत्थाय राजासौ कृत्वा सन्ध्यादिकाः क्रियाः}% ५७

\twolineshloka
{सभां मन्त्रिजनैः सार्द्धं सुखमासेदिवान्महान्}
{तावद्विप्रं ददर्शाथ तापसं कृशदेहिनम्}% ५८

\twolineshloka
{जटावल्कलकौपीनधारिणं दण्डपाणिनम्}
{अनेकतीर्थसेवाभिः कृतपुण्यकलेवरम्}% ५९

\twolineshloka
{राजा तं वीक्ष्य शिरसा प्रणनाम महाभुजः}
{अर्घ्यपाद्यादिकं चक्रे प्रहृष्टात्मा महीपतिः}% ६०

\twolineshloka
{सुखोपविष्टं विश्रान्तं पप्रच्छ विदितं द्विजम्}
{स्वामिंस्त्वद्दर्शनान्मेऽद्य गतं देहस्य पातकम्}% ६१

\twolineshloka
{महान्तः कृपणान्पातुं यान्ति तद्गेहमादरात्}
{तस्मात्कथय भो विप्र वृद्धस्य मम सम्प्रति}% ६२

\twolineshloka
{को देवो गर्भनाशाय किं तीर्थं च क्षमं भवेत्}
{यूयं सर्वगताः श्रेष्ठाः समाधिध्यानतत्पराः}% ६३

\twolineshloka
{सर्वतीर्थावगाहेन कृतपुण्यात्मनोऽमलाः}
{यथावच्छृण्वते मह्यं श्रद्दधानाय विस्तरात्}% ६४

\onelineshloka*
{कथयस्व प्रसादेन सर्वतीर्थविचक्षण}

\uvacha{ब्राह्मण उवाच}

\onelineshloka
{शृणु राजेन्द्र वक्ष्यामि यत्पृष्टं तीर्थसेवनम्}% ६५

\twolineshloka
{कस्य देवस्य कृपया गर्भनिर्वारणं भवेत्}
{सेव्यः श्रीरामचन्द्रोऽसौ संसारज्वरनाशकः}% ६६

\twolineshloka
{पूज्यः स एव भगवान्पुरुषोत्तमसंज्ञकः}
{नाना पुर्यो मया दृष्टाः सर्वपापक्षयङ्कराः}% ६७

\twolineshloka
{अयोध्या सरयूस्तापी तथा द्वारं हरेः परम्}
{अवन्ती विमला काञ्ची रेवा सागरगामिनी}% ६८

\twolineshloka
{गोकर्णं हाटकाख्यं च हत्याकोटिविनाशनम्}
{मल्लिकाख्यो महाशैलो मोक्षदः पश्यतां नृणाम्}% ६९

\twolineshloka
{यत्राङ्गेषु नृणां तोयं श्यामं वा निर्मलं भवेत्}
{पातकस्यापहारीदं मया दृष्टं तु तीर्थकम्}% ७०

\twolineshloka
{मया द्वारवती दृष्टा सुरासुर निषेविता}
{गोमती यत्र वहति साक्षाद्ब्रह्मजला शुभा}% ७१

\twolineshloka
{यत्र स्वापो लयः प्रोक्तो मृतिर्मोक्ष इति श्रुतिः}
{यस्यां संवसतां नॄणां न कलि प्रभवेत्क्वचित्}% ७२

\twolineshloka
{चक्राङ्का यत्र पाषाणा मानवा अपि चक्रिणः}
{पशवः कीटपक्ष्याद्याः सर्वे चक्रशरीरिणः}% ७३

\twolineshloka
{त्रिविक्रमो वसेद्यस्यां सर्वलोकैकपालकः}
{सा पुरी तु महापुण्यैर्मया दृग्गोचरीकृता}% ७४

\twolineshloka
{कुरुक्षेत्रं मया दृष्टं सर्वहत्यापनोदनम्}
{स्यमन्तपञ्चकं यत्र महापातकनाशनम्}% ७५

\twolineshloka
{वाराणसी मया दृष्टा विश्वनाथकृतालया}
{यत्रोपदिशते मन्त्रं तारकं ब्रह्मसंज्ञितम्}% ७६

\fourlineindentedshloka
{यस्यां मृताः कीटपतङ्गभृङ्गाः}
{पश्वादयो वा सुरयोनयो वा}
{स्वकर्मसम्भोगसुखं विहाय}
{गच्छन्ति कैलासमतीतदुःखाः}% ७७

\twolineshloka
{मणिकर्णिर्यत्र तीर्थं यस्यामुत्तरवाहिनी}
{करोति संसृतेर्बन्धच्छेदं पापकृतामपि}% ७८

\twolineshloka
{कपर्दिनः कुण्डलिनः सर्पभूषाधरावराः}
{गजचर्मपरीधाना वसन्ति गतदुःखकाः}% ७९

\twolineshloka
{कालभैरवनामात्र करोति यमशासनम्}
{न करोति नृणां वार्तां यमो दण्डधरः प्रभुः}% ८०

\twolineshloka
{एतादृशी मया दृष्टा काशी विश्वेश्वराङ्किता}
{अनेकान्यपि तीर्थानि मया दृष्टानि भूमिप}% ८१

\twolineshloka
{परमेकं महच्चित्रं यद्दृष्टं नीलपर्वते}
{पुरुषोत्तमसान्निध्ये तन्न क्वाप्यक्षिगोचरम्}% ८२

{॥इति श्रीपद्मपुराणे पातालखण्डे शेषवात्स्यायनसंवादे रामाश्वमेधे ब्राह्मणसमागमो नाम सप्तदशोऽध्यायः॥१७॥}

\dnsub{अष्टादशोऽध्यायः}\resetShloka

\uvacha{ब्राह्मण उवाच}

\twolineshloka
{राजंस्त्वं शृणु यद्वृत्तं नीले पर्वतसत्तमे}
{यच्छ्रद्दधानाः पुरुषा यान्ति ब्रह्म सनातनम्}% १

\twolineshloka
{मया पर्यटता तत्र गतं नीलाभिधे गिरौ}
{गङ्गासागरतोयेन क्षालितप्राङ्गणे मुहुः}% २

\twolineshloka
{तत्र भिल्ला मया दृष्टाः पर्वताग्रे धनुर्भृतः}
{चतुर्भुजा मूलफलैर्भक्ष्यैर्निर्वाहितक्लमाः}% ३

\twolineshloka
{तदा मे मनसि क्षिप्रं संशयः सुमहानभूत्}
{चतुर्भुजाः किमेते वै धनुर्बाणधरा नराः}% ४

\twolineshloka
{वैकुण्ठवासिनां रूपं दृश्यते विजितात्मनाम्}
{कथमेतैरुपालब्धं ब्रह्माद्यैरपि दुर्ल्लभम्}% ५

\twolineshloka
{शङ्खचक्रगदाशार्ङ्गपद्मोल्लसितपाणयः}
{वनमालापरीताङ्गा विष्णुभक्ता इवान्तिके}% ६

\twolineshloka
{संशयाविष्टचित्तेन मया पृष्टं तदा नृप}
{यूयं के बत युष्माभिर्लब्धं चातुर्भुजं कथम्}% ७

\twolineshloka
{तदा तैर्बहु हास्यं तु कृत्वा मां प्रतिभाषितम्}
{ब्राह्मणोऽयं न जानाति पिण्डमाहात्म्यमद्भुतम्}% ८

\twolineshloka
{इति श्रुत्वाऽवदं चाहं कः पिण्डः कस्य दीयते}
{तन्मम ब्रूत धर्मिष्ठाश्चतुर्भुजशरीरिणः}% ९

\twolineshloka
{तदा मद्वाक्यमाकर्ण्य कथितं तैर्महात्मभिः}
{सर्वं तत्र तु यद्वृत्तं चतुर्भुजभवादिकम्}% १०

\uvacha{किराता ऊचुः}

\twolineshloka
{शृणु ब्राह्मण वृत्तान्तमस्माकं पृथुकः शिशुः}
{नित्यं जम्बूफलादीनि भक्षयन्क्रीडया चरन्}% ११

\twolineshloka
{एकदा रममाणस्तु गिरिशृङ्गं मनोरमम्}
{समारुरोह शिशुभिः समन्तात्परिवारितः}% १२

\twolineshloka
{तदा तत्र ददर्शाहं देवायतनमद्भुतम्}
{गारुत्मतादिमणिभिः खचितं स्वर्णभित्तिकम्}% १३

\twolineshloka
{स्वकान्त्यातिमिरश्रेणीं दारयद्रविवद्भृशम्}
{दृष्ट्वा विस्मयमापेदे किमिदं कस्य वै गृहम्}% १४

\twolineshloka
{गत्वा विलोकयामीति किमिदं महतां पदम्}
{इति सञ्चिन्त्य गेहान्तर्जगाम बहुभाग्यतः}% १५

\twolineshloka
{ददर्श तत्र देवेशं सुरासुरनमस्कृतम्}
{किरीटहारकेयूरग्रैवेयाद्यैर्विराजितम्}% १६

\twolineshloka
{मनोहरावतंसौ च धारयन्तं सुनिर्मलौ}
{पादपद्मे तुलसिका गन्धमत्तषडङ्घ्रिके}% १७

\twolineshloka
{शङ्खचक्रगदाशार्ङ्ग पद्माद्यैर्मूर्तिसंयुतैः}
{उपासिताङ्घ्रिं श्रीमूर्तिं नारदाद्यैः सुसेवितम्}% १८

\twolineshloka
{केचिद्गायन्ति नृत्यन्ति हसन्ति परमाद्भुतम्}
{प्रीणयन्ति महाराजं सर्वलोकैकवन्दितम्}% १९

\twolineshloka
{हरिं वीक्ष्य मदीयोर्भस्तत्र सञ्जग्मिवान्मुने}
{देवास्तत्र विधायोच्चैः पूजां धूपादिसंयताम्}% २०

\twolineshloka
{नैवेद्यं श्रीप्रियस्यार्थे कृत्वा नीराजनं ततः}
{जग्मुः स्वं स्वं गृहं राजन्कृपां पश्यन्त आदरात्}% २१

\twolineshloka
{महाभाग्यवशात्तेन प्राप्तं नैवेद्यसिक्थकम्}
{पतितं ब्रह्मदेवाद्यैर्दुर्ल्लभं सुरमानुषैः}% २२

\twolineshloka
{तद्भक्षणं च कृत्वाथो श्रीमूर्तिमवलोक्य च}
{चतुर्भुजत्वमाप्तं वै पृथुकेन सुशोभिना}% २३

\twolineshloka
{तदास्माभिर्गृहं प्राप्तो बालको वीक्षितो मुहुः}
{चतुर्भुजत्वं सम्प्राप्तः शङ्खचक्रादिधारकः}% २४

\twolineshloka
{अस्माभिः पृष्टमेतस्य किमेतज्जातमद्भुतम्}
{तदा प्रोवाच नः सर्वान्बालकः परमाद्भुतम्}% २५

\twolineshloka
{शिखराग्रे गतः पूर्वं तत्र दृष्टः सुरेश्वरः}
{तत्र नैवेद्यसिक्थं तु मया प्राप्तं मनोहरम्}% २६

\twolineshloka
{तस्य भक्षणमात्रेण कारणेन तु साम्प्रतम्}
{चतुर्भुजत्वं सम्प्राप्तो विस्मयेन समन्वितः}% २७

\twolineshloka
{तच्छ्रुत्वा तु वचस्तस्य सद्यः सम्प्राप्तविस्मयैः}
{अस्माभिरप्यसौ दृष्टो देवः परमदुर्ल्लभः}% २८



{अन्नादिकं तत्र भुक्तं सर्वस्वादसमन्वितम्}
{वयं चतुर्भुजा जाता देवस्य कृपया पुनः}
{गत्वा त्वमपि देवस्य दर्शनं कुरु सत्तम}% २९

\twolineshloka
{भुक्त्वा तत्रान्नसिक्थं तु भव विप्र चतुर्भुजः}
{त्वया पृष्टं यदाश्चर्यं तदुक्तं वाडवर्षभ}% ३०

{॥इति श्रीपद्मपुराणे पातालखण्डे शेषवात्स्यायनसंवादे रामाश्वमेधे ब्राह्मणोपदेशोनामाष्टादशोऽध्यायः॥१८॥}

\dnsub{ऊनविंशोऽध्यायः}\resetShloka

\uvacha{ब्राह्मण उवाच}

\twolineshloka
{इति श्रुत्वा तु तद्वाक्यं भिल्लानामहमद्भुतम्}
{अत्याश्चर्यमिदं मत्वा प्रहृष्टोऽभवमित्युत}% १

\twolineshloka
{गङ्गासागरसंयोगे स्नात्वा पुण्यकलेवरः}
{शृङ्गमारुरुहे तत्र मणिमाणिक्यचित्रितम्}% २

\twolineshloka
{तत्रापश्यं महाराज देवं देवादिवन्दितम्}
{नमस्कृत्वा कृतार्थोऽहं जातोन्नप्राशनेन च}% ३

\twolineshloka
{चतुर्भुजत्वं सम्प्राप्तः शङ्खचक्रादिचिह्नितम्}
{पुरुषोत्तमदर्शनेन न पुनर्गर्भमाविशम्}% ४

\twolineshloka
{राजंस्त्वमपि तत्राशु गच्छ नीलाभिधं गिरिम्}
{कृतार्थं कुरु चात्मानं गर्भदुःखविवर्जितम्}% ५

\twolineshloka
{इत्याकर्ण्य वचस्तस्य वाडवाग्र्यस्य धीमतः}
{पप्रच्छ हृष्टगात्रस्तु तीर्थयात्राविधिं मुनिम्}% ६

\uvacha{राजोवाच}

\twolineshloka
{साधु विप्राग्र्य हे साधो त्वया प्रोक्तं ममानघ}
{पुरुषोत्तममाहात्म्यं शृण्वतां पापनाशनम्}% ७

\twolineshloka
{ब्रूहि तत्तीर्थयात्रायां विधिं श्रुतिसमन्वितम्}
{विधिना केन सम्पूर्ण फलप्राप्तिर्नृणां भवेत्}% ८

\uvacha{ब्राह्मण उवाच}

\twolineshloka
{शृणु राजन्प्रवक्ष्यामि तीर्थयात्राविधिं शुभम्}
{येन सम्प्राप्यते देवः सुरासुरनमस्कृतः}% ९

\twolineshloka
{वलीपलितदेहो वा यौवनेनान्वितोऽपि वा}
{ज्ञात्वा मृत्युमनिस्तीर्यं हरिं शरणमाव्रजेत्}% १०

\twolineshloka
{तत्कीर्तने तच्छ्रवणे वन्दने तस्य पूजने}
{मतिरेव प्रकर्तव्या नान्यत्र वनितादिषु}% ११

\twolineshloka
{सर्वं नश्वरमालोक्य क्षणस्थायि सुदुःखदम्}
{जन्ममृत्युजरातीतं भक्तिवल्लभमच्युतम्}% १२

\twolineshloka
{क्रोधात्कामाद्भयाद्द्वेषाल्लोभाद्दम्भान्नरः पुनः}
{यथाकथञ्चिद्विभजन्न स दुःखं समश्नुते}% १३

\twolineshloka
{स हरिर्जायते साधुसङ्गमात्पापवर्जितात्}
{येषां कृपातः पुरुषा भवन्त्यसुखवर्जिताः}% १४

\twolineshloka
{ते साधवः शान्तरागाः कामलोभविवर्जिताः}
{ब्रुवन्ति यन्महाराज तत्संसारनिवर्तकम्}% १५

\twolineshloka
{तीर्थेषु लभ्यते साधू रामचन्द्र परायणः}
{यद्दर्शनं नृणां पापराशिदाहाशुशुक्षणिः}% १६

\twolineshloka
{तस्मात्तीर्थेषु गन्तव्यं नरैः संसारभीरुभिः}
{पुण्योदकेषु सततं साधुश्रेणिविराजिषु}% १७

\twolineshloka
{तानि तीर्थानि विधिना दृष्टानि प्रहरन्त्यघम्}
{तं विधिं नृपशार्दूल कुरुष्व श्रुतिगोचरम्}% १८

\twolineshloka
{विरागं जनयेत्पूर्वं कलत्रादि कुटुम्बके}
{असत्यभूतं तज्ज्ञात्वा हरिं तु मनसा स्मरेत्}% १९

\twolineshloka
{क्रोशमात्रं ततो गत्वा रामरामेति च ब्रुवन्}
{तत्र तीर्थादिषु स्नात्वा क्षौरं कुर्याद्विधानवित्}% २०

\twolineshloka
{मनुष्याणां च पापानि तीर्थानि प्रति गच्छताम्}
{केशानाश्रित्य तिष्ठन्ति तस्माद्वपनमाचरेत्}% २१

\twolineshloka
{ततो दण्डं तु निर्ग्रन्थिं कमण्डलुमथाजिनम्}
{बिभृयाल्लोभनिर्मुक्तस्तीर्थवेषधरो नरः}% २२

\twolineshloka
{विधिना गच्छतां नॄणां फलावाप्तिर्विशेषतः}
{तस्मात्सर्वप्रयत्नेन तीर्थयात्राविधिं चरेत्}% २३

\twolineshloka
{यस्य हस्तौ च पादौ च मनश्चैव सुसंहितम्}
{विद्या तपश्च कीर्तिश्च स तीर्थफलमश्नुते}% २४

\twolineshloka
{हरेकृष्ण हरेकृष्ण भक्तवत्सल गोपते}
{शरण्य भगवन्विष्णो मां पाहि बहुसंसृतेः}% २५

\twolineshloka
{इति ब्रुवन्रसनया मनसा च हरिं स्मरन्}
{पादचारी गतिं कुर्यात्तीर्थं प्रति महोदयः}% २६

\twolineshloka
{यानेन गच्छन्पुरुषः समभागफलं लभेत्}
{उपानद्भ्यां चतुर्थांशं गोयाने गोवधादिकम्}% २७

\twolineshloka
{व्यवहर्ता तृतीयांशं सेवयाष्टमभागभाक्}
{अनिच्छया व्रजंस्तत्र तीर्थमर्धफलं लभेत्}% २८

\twolineshloka
{यथायथं प्रकर्तव्या तीर्थानामभियात्रिका}
{पापक्षयो भवत्येव विधिदृष्ट्या विशेषतः}% २९

\twolineshloka
{तत्र साधून्नमस्कुर्यात्पादवन्दनसेवनैः}
{तद्द्वारा हरिभक्तिर्हि प्राप्यते पुरुषोत्तमे}% ३०

\twolineshloka
{इति तीर्थविधिः प्रोक्तः समासेन न विस्तरात्}
{एवं विधिं समाश्रित्य गच्छ त्वं पुरुषोत्तमम्}% ३१

\twolineshloka
{तुभ्यं तुष्टो महाराज दास्यते भक्तिमच्युतः}
{यथा संसारनिर्वाहः क्षणादेव भविष्यति}% ३२

\twolineshloka
{तीर्थयात्राविधिं श्रुत्वा सर्वपातकनाशनम्}
{मुच्यते सर्वपापेभ्य उग्रेभ्यः पुरुषर्षभ}% ३३

\uvacha{सुमतिरुवाच}

\twolineshloka
{इति वाक्यं समाकर्ण्य ववन्दे चरणौ महान्}
{तत्तीर्थदर्शनौत्सुक्य विह्वलीकृतमानसः}% ३४

\twolineshloka
{आदिदेश निजामात्यं मन्त्रवित्तममुत्तमम्}
{तीर्थयात्रेच्छया सर्वान्सह नेतुं मनो दधत्}% ३५

\twolineshloka
{मन्त्रिन्पौरजनान्सर्वानादिश त्वं ममाज्ञया}
{पुरुषोत्तमपादाब्जदर्शनप्रीतिहेतवे}% ३६

\twolineshloka
{ये मदीये पुरे लोका ये च मद्वाक्यकारकाः}
{सर्वे निर्यान्तु मत्पुर्या मया सह नरोत्तमाः}% ३७

\twolineshloka
{ये तु मद्वाक्यमुल्लङ्घ्य स्थास्यन्ति पुरुषा गृहे}
{ते दण्ड्या यमदण्डेन पापिनोऽधर्महेतवः}% ३८

\twolineshloka
{किं तेन सुतवृन्देन बान्धवैः किं सुदुर्नयैः}
{यैर्नदृष्टः स्वचक्षुर्भ्यां पुण्यदः पुरुषोत्तमः}% ३९

\twolineshloka
{सूकरीयूथवत्तेषां प्रसूतिर्विट्प्रभक्षिका}
{येषां पुत्राश्च पौत्रा वा हरिं न शरणं गताः}% ४०

\twolineshloka
{यो देवो नाममात्रेण सर्वान्पावयितुं क्षमः}
{तं नमस्कुरुत क्षिप्रं मदीयाः प्रकृतिव्रजाः}% ४१

\twolineshloka
{इति वाक्यं मनोहारि भगवद्गुणगुम्फितम्}
{प्रजहर्ष महामात्य उत्तमः सत्यनामधृक्}% ४२

\twolineshloka
{हस्तिनं वरमारोप्य पटहेन व्यघोषयत्}
{यदादिष्टं नृपेणेह तीर्थयात्रां समिच्छता}% ४३

\twolineshloka
{गच्छन्तु त्वरिता लोका राज्ञा सह महागिरिम्}
{दृश्यतां पापसंहारी पुरुषोत्तमनामधृक्}% ४४

\twolineshloka
{क्रियतां सर्वसंसारसागरो गोष्पदं पुनः}
{भूष्यतां शङ्खचक्रादिचिह्नैः स्वस्व तनुर्नरैः}% ४५

\twolineshloka
{इत्यादिघोषयामास राज्ञादिष्टं यदद्भुतम्}
{सचिवो रघुनाथाङ्घ्रि ध्याननिर्वारितश्रमः}% ४६

\twolineshloka
{तच्छ्रुत्वा ताः प्रजाः सर्वा आनन्दरससम्प्लुताः}
{मनो दधुः स्वनिस्तारे पुरुषोत्तमदर्शनात्}% ४७

\twolineshloka
{निर्ययुर्ब्राह्मणास्तत्र शिष्यैः सह सुवेषिणः}
{आशिषं वरदानाढ्यां ददतो भूमिपं प्रति}% ४८

\twolineshloka
{क्षत्त्रिया धन्विनो वीरा वैश्या वस्तुक्रयाञ्चिताः}
{शूद्राः संसारनिस्तारहर्षित स्वीयविग्रहाः}% ४९

\twolineshloka
{रजकाश्चर्मकाः क्षौद्राः किराता भित्तिकारकाः}
{सूचीवृत्त्या च जीवन्तस्ताम्बूलक्रयकारकाः}% ५०

\twolineshloka
{तालवाद्यधरा ये च ये च रङ्गोपजीविनः}
{तैलविक्रयिणश्चैव वस्त्रविक्रयिणस्तथा}% ५१

\twolineshloka
{सूता वदन्तः पौराणीं वार्तां हर्षसमन्विताः}
{मागधा बन्दिनस्तत्र निर्गता भूमिपाज्ञया}% ५२

\twolineshloka
{भिषग्वृत्त्या च जीवन्तस्तथा पाशककोविदाः}
{पाकस्वादुरसाभिज्ञा हास्यवाक्यानुरञ्जकाः}% ५३

\twolineshloka
{ऐन्द्रजालिकविद्याध्रास्तथा वार्तासुकोविदाः}
{प्रशंसन्तो महाराजं निर्ययुः पुरमध्यतः}% ५४

\twolineshloka
{राजापि तत्र निर्वर्त्य प्रातःसन्ध्यादिकाः क्रियाः}
{ब्राह्मणं तापसश्रेष्ठमानिनाय सुनिर्मलम्}% ५५

\twolineshloka
{तदाज्ञया महाराजो निर्जगाम पुराद्बहिः}
{लोकैरनुगतो राजा बभौ चन्द्र इवोडुभिः}% ५६

\twolineshloka
{क्रोशमात्रं स गत्वाथ क्षौरं कृत्वा विधानतः}
{दण्डं कमण्डलुं बिभ्रन्मृगचर्म तथा शुभम्}% ५७

\twolineshloka
{शुभवेषेण संयुक्तो हरिध्यानपरायणः}
{कामक्रोधादिरहितं मनो बिभ्रन्महायशाः}% ५८

\twolineshloka
{तदा दुन्दुभयो भेर्य आनकाः पणवास्तथा}
{शङ्खवीणादिकाश्चैवाध्मातास्तद्वादकैर्मुहुः}% ५९

\twolineshloka
{जय देवेश दुःखघ्न पुरुषोत्तमसंज्ञित}
{दर्शयस्व तनुं मह्यं वदन्तो निर्ययुर्जनाः}% ६०

{॥इति श्रीपद्मपुराणे पातालखण्डे शेषवात्स्यायनसंवादे रामाश्वमेधे रत्नग्रीवस्य तीर्थप्रयाणन्नामैकोनविंशोऽध्यायः॥१९॥}

\dnsub{विंशोऽध्यायः}\resetShloka

\uvacha{सुमतिरुवाच}

\twolineshloka
{अथ प्रयाते भूपाले सर्वलोकसमन्विते}
{महाभागैर्वैष्णवैश्च गायकैः कृष्णकीर्तनम्}% १

\twolineshloka
{शुश्रावासौ महाराजो मार्गे गोविन्दकीर्तनम्}
{जय माधव भक्तानां शरण्य पुरुषोत्तम}% २

\twolineshloka
{मार्गे तीर्थान्यनेकानि कुर्वन्पश्यन्महोदयम्}
{तापसब्राह्मणात्तेषां महिमानमथा शृणोत्}% ३

\twolineshloka
{विचित्रविष्णुवार्ताभिर्विनोदितमना नृपः}
{मार्गेमार्गे महाविष्णुं गापयामास गायकान्}% ४

\twolineshloka
{दीनान्धकृपणानां च पङ्गूनां वासनोचितम्}
{दानं ददौ महाराजो बुद्धिमान्विजितेन्द्रियः}% ५

\twolineshloka
{अनेकतीर्थविरजमात्मानं भव्यतां गतम्}
{कुर्वन्ययौ स्वीयलोकैर्हरिध्यानपरायणः}% ६

\twolineshloka
{नृपो गच्छन्ददर्शाग्रे नदीं पापप्रणाशिनीम्}
{चक्राङ्कितग्रावयुतां मुनिमानस निर्मलाम्}% ७

\twolineshloka
{अनेकमुनिवृन्दानां बहुश्रेणिविराजिताम्}
{सारसादिपतत्रीणां कूजितैरुपशोभिताम्}% ८

\twolineshloka
{दृष्ट्वा पप्रच्छ विप्राग्र्यं तापसं धर्मकोविदम्}
{अनेकतीर्थमाहात्म्य विशेषज्ञानजृम्भितम्}% ९

\twolineshloka
{स्वामिन्केयं नदी पुण्या मुनिवृन्दनिषेविता}
{करोति मम चित्तस्य प्रमोदभरनिर्भरम्}% १०

\twolineshloka
{इति श्रुत्वा वचस्तस्य राजराजस्य धीमतः}
{वक्तुं प्रचक्रमे विद्वांस्तीर्थमाहात्म्यमुत्तमम्}% ११

\uvacha{ब्राह्मण उवाच}

\twolineshloka
{गण्डकीयं नदी राजन्सुरासुरनिषेविता}
{पुण्योदकपरीवाह हतपातकसञ्चया}% १२

\twolineshloka
{दर्शनान्मानसं पापं स्पर्शनात्कर्मजं दहेत्}
{वाचिकं स्वीय तोयस्य पानतः पापसञ्चयम्}% १३

\twolineshloka
{पुरा दृष्ट्वा प्रजानाथः प्रजाः सर्वा विपावनीः}
{स्वगण्डविप्रुषोनेक पापघ्नीं सृष्टवानिमाम्}% १४

\twolineshloka
{एनां नदीं ये पुण्योदां स्पृशन्ति सुतरङ्गिणीम्}
{ते गर्भभाजो नैव स्युरपि पापकृतो नराः}% १५

\twolineshloka
{अस्यां भवा ये चाश्मानश्चक्रचिह्नैरलङ्कृताः}
{ते साक्षाद्भगवन्तो हि स्वस्वरूपधराः पराः}% १६

\twolineshloka
{शिलां सम्पूजयेद्यस्तु नित्यं चक्रयुतां नरः}
{न जातु स जनन्या वै जठरं समुपाविशेत्}% १७

\twolineshloka
{पूजयेद्यो नरो धीमाञ्छालग्रामशिलां वराम्}
{तेनाचारवता भाव्यं दम्भलोभवियोगिना}% १८

\twolineshloka
{परदार परद्रव्यविमुखेन नरेण हि}
{पूजनीयः प्रयत्नेन शालग्रामः सचक्रकः}% १९

\twolineshloka
{द्वारवत्यां भवं चक्रं शिला वै गण्डकीभवा}
{पुंसां क्षणाद्धरत्येव पापं जन्मशतार्जितम्}% २०

\twolineshloka
{अपि पापसहस्राणां कर्ता तावन्नरो भवेत्}
{शालग्रामशिलातोयं पीत्वा पूतो भवेत्क्षणात्}% २१

\twolineshloka
{ब्राह्मणः क्षत्रियो वैश्यः शूद्रो वेदपथि स्थितः}
{शालग्रामं पूजयित्वा गृहस्थो मोक्षमाप्नुयात्}% २२

\twolineshloka
{न जातु चित्स्त्रिया कार्यं शालग्रामस्य पूजनम्}
{भर्तृहीनाथ सुभगा स्वर्गलोकहितैषिणी}% २३

\twolineshloka
{मोहात्स्पृष्ट्वापि महिला जन्मशीलगुणान्विता}
{हित्वा पुण्यसमूहं सा सत्वरं नरकं व्रजेत्}% २४

\twolineshloka
{स्त्रीपाणिमुक्तपुष्पाणि शालग्रामशिलोपरि}
{पवेरधिकपातानि वदन्ति ब्राह्मणोत्तमाः}% २५

\twolineshloka
{चन्दनं विषसङ्काशं कुसुमं वज्रसन्निभम्}
{नैवेद्यं कालकूटाभं भवेद्भगवतः कृतम्}% २६

\twolineshloka
{तस्मात्सर्वात्मना त्याज्यं स्त्रिया स्पर्शः शिलोपरि}
{कुर्वती याति नरकं यावदिन्द्राश्चतुर्दश}% २७

\twolineshloka
{अपि पापसमाचारो ब्रह्महत्यायुतोऽपि वा}
{शालग्रामशिलातोयं पीत्वा याति परां गतिम्}% २८

\twolineshloka
{तुलसीचन्दनं वारि शङ्खो घण्टाथ चक्रकम्}
{शिला ताम्रस्य पात्रं तु विष्णोर्नामपदामृतम्}% २९

\twolineshloka
{पदामृतं तु नवभिः पापराशिप्रदाहकम्}
{वदन्ति मुनयः शान्ताः सर्वशास्त्रार्थकोविदाः}% ३०

\twolineshloka
{सर्वतीर्थपरिस्नानात्सर्वक्रतुसमर्चनात्}
{पुण्यं भवति यद्राजन्बिन्दौ बिन्दौ तदद्भुतम्}% ३१

\twolineshloka
{शालग्रामशिला यत्र पूज्यते पुरुषोत्तमैः}
{तत्र योजनमात्रं तु तीर्थकोटिसमन्वितम्}% ३२

\twolineshloka
{शालग्रामाः समाः पूज्याः समेषु द्वितयं नहि}
{विषमा एव सम्पूज्या विषमेषु त्रयं नहि}% ३३

\twolineshloka
{द्वारावती भवं चक्रं तथा वै गण्डकीभवम्}
{उभयोः सङ्गमो यत्र तत्र गङ्गा समुद्रगा}% ३४

\twolineshloka
{रूक्षाः कुर्वन्ति पुरुषा नायुः श्रीबलवर्जितान्}
{तस्मात्स्निग्धा मनोहारि रूपिण्यो ददति श्रियम्}% ३५

\twolineshloka
{आयुष्कामो नरो यस्तु धनकामो हि यः पुमान्}
{पूजयन्सर्वमाप्नोति पारलौकिकमैहिकम्}% ३६

\twolineshloka
{प्राणान्तकाले पुंसस्तु भवेद्भाग्यवतो नृप}
{वाचि नाम हरेः पुण्यं शिला हृदि तदन्तिके}% ३७

\twolineshloka
{गच्छत्सु प्राणमार्गेषु यस्य विश्रम्भतोऽपि चेत्}
{शालग्रामशिला स्फूर्तिस्तस्य मुक्तिर्न संशयः}% ३८

\twolineshloka
{पुरा भगवता प्रोक्तमम्बरीषाय धीमते}
{ब्राह्मणा न्यासिनः स्निग्धाः शालग्रामशिलास्तथा}% ३९

\twolineshloka
{स्वरूपत्रितयं मह्यमेतद्धि क्षितिमण्डले}
{पापिनां पापनिर्हारं कर्तुं धृतमुदं च ता}% ४०

\twolineshloka
{निन्दन्ति पापिनो ये वा शालग्रामशिलां सकृत्}
{कुम्भीपाके पचन्त्याशु यावदाभूतसम्प्लवम्}% ४१

\twolineshloka
{पूजां समुद्यतं कर्तुं यो वारयति मूढधीः}
{तस्य मातापिताबन्धुवर्गा नरकभागिनः}% ४२

\twolineshloka
{यो वा कथयति प्रेष्ठं शालग्रामार्चनं कुरु}
{सकृतार्थो नयत्याशु वैकुण्ठं स्वस्य पूर्वजान्}% ४३

\twolineshloka
{अत्रैवोदाहरन्तीममितिहासं पुरातनम्}
{मुनयो वीतरागाश्च कामक्रोधविवर्जिताः}% ४४

\twolineshloka
{पुरा कीकटसंज्ञे वै देशे धर्मविवर्जिते}
{आसीत्पुल्कसजातीयो नरः शबरसंज्ञितः}% ४५

\twolineshloka
{नित्यं जन्तुवधोद्युक्तः शरासनधरो मुहुः}
{तीर्थं प्रति यियासूनां बलाद्धरति जीवितम्}% ४६

\twolineshloka
{अनेकप्राणिहत्याकृत्परस्वे निरतः सदा}
{सदा रागादिसंयुक्तः कामक्रोधादिसंयुतः}% ४७

\twolineshloka
{विचरत्यनिशं भीमे वने प्राणिवधङ्करः}
{विषसंसक्तबाणाग्र रूढचापगुणोद्धुरः}% ४८

\twolineshloka
{सैकदा पर्यटन्व्याधः प्राणिमात्रभयङ्करः}
{कालं प्राप्तं न जानाति समीपेऽप्युग्रमानसः}% ४९

\twolineshloka
{यमदूतास्तु सम्प्राप्ताः पाशमुद्गरपाणयः}
{ताम्रकेशा दीर्घनखा लम्बदंष्ट्रा भयानकाः}% ५०

\twolineshloka
{श्यामा लोहस्यनिगडान्बिभ्रतो मोहकारकाः}
{बध्नन्तु पापिनं ह्येनं प्राणिमात्रभयङ्करम्}% ५१

\twolineshloka
{कदाचिन्मनसा नायं प्राणिमात्रोपकारकः}
{परदार परद्रव्य परद्रोहपरायणः}% ५२

\twolineshloka
{एतस्य जिह्वां महतीमहं निष्कासयाम्यतः}
{एको वदति चैतस्य चक्षुरुत्पाटयाम्यहम्}% ५३

\twolineshloka
{एको वदति चैतस्य करौ कृन्तामि पापिनः}
{अन्यो वदत्यहं कर्णौ कर्तयामि दुरात्मनः}% ५४

\twolineshloka
{एवं वदन्तः सुभृशं दन्तैर्दन्तनिपीडकाः}
{आगत्य तं दुरात्मानं सायुधास्तस्थुरुन्मदाः}% ५५

\twolineshloka
{एको दूतस्तदा सर्परूपं धृत्वादशत्पदे}
{स दष्टमात्रः सहसा गतासुः पर्यजायत}% ५६

\twolineshloka
{तदा तं लोहपाशेन बद्ध्वा ते यमकिङ्कराः}
{कशाभिस्ताडयामासुर्मुद्गरैः प्राहरन्क्रुधा}% ५७

\twolineshloka
{अहो दुष्ट दुरात्मंस्त्वं कदाचिन्नाचरः शुभम्}
{मनसापि यतस्त्वां वै क्षेप्स्यामो रौरवेषु च}% ५८

\twolineshloka
{त्वङ्मांसं वायसा रौद्रा भक्षयिष्यन्ति वै क्रुधा}
{आजन्मतस्तु भवता न कृतं हरिसेवनम्}% ५९

\twolineshloka
{त्वया कलत्रपुत्राद्या द्रोहं कृत्वा सुपोषिताः}
{न कदाचित्स्मृतो देवः पापहारी जनार्दनः}% ६०

\twolineshloka
{तस्मात्त्वां लोहशङ्कौ वा कुम्भीपाके च रौरवे}
{धर्मराजाज्ञया सर्वे नेष्यामो बहुताडनैः}% ६१

\twolineshloka
{एवमुक्त्वा यदानेतुं समैच्छन्यमकिङ्कराः}
{तावत्प्राप्तो महाविष्णुचरणाब्जपरायणः}% ६२

\twolineshloka
{यमदूतास्तदा दृष्टा वैष्णवेन महात्मना}
{पाशमुद्गरदण्डादिदुष्टायुधधरा गणाः}% ६३

\twolineshloka
{पुल्कसं लोहनिगडैर्बद्ध्वा यातुं समुद्यताः}
{बन्ध बन्ध ग्रसच्छिन्धि भिन्धि भिन्धीति वादिनः}% ६४

\twolineshloka
{तदा कृपालुस्तं प्रेक्ष्य पद्मनाभपरायणः}
{अत्यन्तकृपयायुक्तं चेतस्तत्र तदाकरोत्}% ६५

\twolineshloka
{असौ महादुष्ट पीडां मा यातु मम सन्निधौ}
{मोचयाम्यहमद्यैव यमदूतेभ्य एव च}% ६६

\twolineshloka
{इति कृत्वा मतिं तस्मिन्कृपायुक्तो मुनीश्वरः}
{शालग्रामशिलां हस्ते गृहीत्वास्य गतोऽन्तिके}% ६७

\twolineshloka
{तस्य पादोदकं पुण्यं तुलसीदलमिश्रितम्}
{मुखे विनिक्षिपन्कर्णे रामनाम जजाप ह}% ६८

\twolineshloka
{तुलसीं मस्तके तस्य धारयामास वैष्णवः}
{शिलां हृदि महाविष्णोर्धृत्वा प्राह स वैष्णवः}% ६९

\twolineshloka
{गच्छन्तु यमदूता वै यातनासु परायणाः}
{शालग्रामशिलास्पर्शो दहतात्पातकं महत्}% ७०

\twolineshloka
{इत्युक्तवति तस्मिन्वै गणा विष्णोर्महाद्भुताः}
{आययुस्तस्य सविधे शिलास्पर्शाद्गतांहसः}% ७१

\twolineshloka
{पीतवस्त्राः शङ्खचक्रगदापद्मविराजिताः}
{आगत्य मोचयामासुर्लोहपाशाद्दुरासदात्}% ७२

\twolineshloka
{मोचयित्वा महापापकारकं पुल्कसं नरम्}
{ऊचुः किमर्थं बद्धोऽयं वैष्णवः पूज्यदेहभृत्}% ७३

\twolineshloka
{कस्याज्ञाकारका यूयं यदधर्मप्रकारकाः}
{मुञ्चन्तु वैष्णवं त्वेनं किमर्थं विधृतो ह्ययम्}% ७४

\twolineshloka
{इति वाक्यं समाकर्ण्य जगदुर्यमकिङ्कराः}
{धर्मराजाज्ञया प्राप्ता नेतुं पापिनमुद्यताः}% ७५

\twolineshloka
{नासौ कदाचिन्मनसा प्राणिमात्रोपकारकः}
{प्राणिहत्या महापापकारी दुष्टशरीरभृत्}% ७६

\twolineshloka
{नॄन्बहूंस्तीर्थयात्रायां गच्छतोऽसौ व्यलुण्ठयत्}
{परदाररतो नित्यं सर्वपापाधिकारकः}% ७७

\twolineshloka
{तस्मान्नेतुं वयं प्राप्ताः पापिनं पुल्कसं नरम्}
{भवद्भिर्मोचितः कस्मादकस्मादागतैर्भटैः}% ७८

\uvacha{विष्णुदूता ऊचुः}

\twolineshloka
{ब्रह्महत्यादिकं पापं प्राणिकोटिवधोद्भवम्}
{शालग्रामशिलास्पर्शः सर्वं दहति तत्क्षणात्}% ७९

\twolineshloka
{रामेति नाम यच्छ्रोत्रे विश्रम्भादागतं यदि}
{करोति पापसन्दाहं तूलं वह्निकणो यथा}% ८०

\twolineshloka
{तुलसी मस्तके यस्य शिला हृदि मनोहरा}
{मुखे कर्णेऽथवा राम नाम मुक्तस्तदैव सः}% ८१

\twolineshloka
{तस्मादनेन तुलसी मस्तके विधृता पुरा}
{श्रावितं रामनामाशु शिला हृदि सुधारिता}% ८२

\twolineshloka
{तस्मात्पापसमूहोऽस्य दग्धः पुण्यकलेवरः}
{यास्यते परमं स्थानं पापिनां यत्सुदुर्ल्लभम्}% ८३

\twolineshloka
{वर्षायुतं तत्र भुक्त्वा भोगान्सर्वमनोहरान्}
{भारते जन्म सम्प्राप्य समाराध्य जगद्गुरुम्}% ८४

\twolineshloka
{प्राप्स्यते परमं स्थानं सुरासुरसुदुर्ल्लभम्}
{न ज्ञातो महिमा सम्यक्छिलायाः परमेष्ठिनः}% ८५

\twolineshloka
{दृष्टा स्पृष्टार्चिता वापि सर्वपापहरा क्षणात्}
{इत्युक्त्वा विरताः सर्वे महाविष्णोर्गणा मुदा}% ८६

\twolineshloka
{याम्यास्ते किङ्करा राज्ञे कथयामासुरद्भुतम्}
{वैष्णवो हर्षमापेदे रघुनाथपरायणः}% ८७

\twolineshloka
{मुक्तोऽसौ यमपाशाच्च गमिष्यति परं पदम्}
{तदाजगाम विमलं किङ्किणीजालमण्डितम्}% ८८

\twolineshloka
{विमानं देवलोकात्तु मनोहारि महाद्भुतम्}
{तत्रारुह्य गतः स्वर्गं महापुण्यैर्निषेवितम्}% ८९

\twolineshloka
{भोगान्भुक्त्वा स विपुलानाजगाम महीतलम्}
{काश्यां जन्म समासाद्य शुचिवाडवसत्कुले}% ९०

\twolineshloka
{आराध्य जगतामीशं गतवान्परमं पदम्}
{स पापी साधुसङ्गत्या शालग्रामशिलां स्पृशन्}% ९१

\twolineshloka
{महापीडाविनिर्मुक्तो गतवान्परमं पदम्}
{मया तेऽभिहितं राजन्गण्डकीचरितं महत्}% ९२

\onelineshloka
{श्रुत्वा विमुच्यते पापैर्भुक्तिं मुक्तिं च विन्दति}% ९३

{॥इति श्रीपद्मपुराणे पातालखण्डे रामाश्वमेधे शेषवात्स्यायनसंवादे गण्डकीमाहात्म्यं नाम विंशोऽध्यायः॥२०॥}

\dnsub{एकविंशोऽध्यायः}\resetShloka

\uvacha{सुमतिरुवाच}

\twolineshloka
{एतन्माहात्म्यमतुलं गण्डक्याः कर्णगोचरम्}
{कृत्वा कृतार्थमात्मानममन्यत नृपोत्तमः}% १

\twolineshloka
{स्नात्वा तीर्थे पितॄन्सर्वान्सन्तर्प्य जहृषे महान्}
{शालग्रामशिलापूजां कुर्वन्वाडववाक्यतः}% २

\twolineshloka
{चतुर्विंशच्छिलास्तत्र गृहीत्वा स नृपोत्तमः}
{पूजयामास प्रेम्णा च चन्दनाद्युपचारकैः}% ३

\twolineshloka
{तत्र दानानि दत्त्वा च दीनान्धेभ्यो विशेषतः}
{गन्तुं प्रचक्रमे राजा पुरुषोत्तममन्दिरम्}% ४

\twolineshloka
{एवं क्रमेण सम्प्राप्तो गङ्गासागरसङ्गमम्}
{कृत्वाक्षिगोचरं तं च ब्राह्मणं पृष्टवान्मुदा}% ५

\twolineshloka
{स्वामिन्वद कियद्दूरे नीलाख्यः पर्वतो महान्}
{पुरुषोत्तमसंवासः सुरासुरनमस्कृतः}% ६

\twolineshloka
{तदा श्रुत्वा महद्वाक्यं रत्नग्रीवस्य भूपतेः}
{उवाच विस्मयाविष्टो राजानं प्रति सादरम्}% ७

\twolineshloka
{राजन्नेतत्स्थलं नीलपर्वतस्य नमस्कृतम्}
{किमर्थं दृश्यते नैव महापुण्यफलप्रदम्}% ८

\twolineshloka
{पुनःपुनरुवाचेदं स्थलं नीलस्य भूभृतः}
{कथं न दृश्यते राजन्पुरुषोत्तमवासभृत्}% ९

\twolineshloka
{अत्र स्नातं मया सम्यगत्र भिल्लाक्षिगोचराः}
{अनेनैव पथा राजन्नारूढं पर्वतोपरि}% १०

\twolineshloka
{इति तद्वाक्यमाकर्ण्य विव्यथे मानसे नृपः}
{नीलभूधरदर्शाय कुर्वन्नुत्कण्ठितं मनः}% ११

\twolineshloka
{उवाच तत्कथं विप्र दृश्येत पुरुषोत्तमः}
{कथं वा दृश्यते नीलस्तदुपायं वदस्व नः}% १२

\twolineshloka
{तदा वाक्यं समाकर्ण्य रत्नग्रीवस्य भूपतेः}
{तापसो ब्राह्मणो वाक्यमुवाच नृप विस्मितः}% १३

\twolineshloka
{गङ्गासागरसंयोगे स्नात्वास्माभिर्महीपते}
{स्थातव्यं तावदेवात्र यावन्नीलो न दृश्यते}% १४

\twolineshloka
{गीयते पापहा देवः पुरुषोत्तमसंज्ञितः}
{करिष्यते कृपामाशु भक्तवत्सलनामधृक्}% १५

\twolineshloka
{त्यजत्यसौ न वै भक्तान्देवदेवशिरोमणिः}
{अनेके रक्षिता भक्तास्तद्गायस्व महामते}% १६

\twolineshloka
{इति वाक्यं समाकर्ण्य राजा व्यथितचेतसा}
{स्नात्वा गङ्गाब्धिसंयोगे ततोनशनमादधात्}% १७

\twolineshloka
{करिष्यति कृपां यर्हि दर्शने पुरुषोत्तमः}
{पूजां कृत्वाशनं कुर्यामन्यथानशनं व्रतम्}% १८

\twolineshloka
{इति कृत्वा स नियमं गङ्गासागररोधसि}
{गायन्हरिगुणग्राममुपवासमथाचरत्}% १९

\uvacha{राजोवाच}

\twolineshloka
{जय दीनदयाकरप्रभो जय दुःखापह मङ्गलाह्वय}
{जय भक्तजनार्तिनाशन कृतवर्ष्मञ्जयदुष्टघातक}% २०

\twolineshloka
{अम्बरीषमथ वीक्ष्य दुःखितं विप्रशापहतसर्वमङ्गलम्}
{धारयन्निजकरे सुदर्शनं संररक्ष जठराधिवासतः}% २१

\twolineshloka
{दैत्यराज पितृकारितव्यथः शूलपाशजलवह्निपातनैः}
{श्रीनृसिंहतनुधारिणा त्वया रक्षितः सपदि पश्यतः पितुः}% २२

\twolineshloka
{ग्राहवक्त्रपतिताङ्घ्रिमुद्भटं वारणेन्द्रमतिदुःखपीडितम्}
{वीक्ष्य साधुकरुणार्द्रमानसस्त्वं गरुत्मति कृतारुहक्रियः}% २३

\twolineshloka
{त्यक्तपक्षिपतिरात्तचक्रको वेगकम्पयुतमालिकाम्बरः}
{गीयसे सुभिरमुष्य न क्रतो मोचकः सपदि तद्विनाशकः}% २४

\twolineshloka
{यत्रयत्र तव सेवकार्दनं तत्र तत्र बत देहधारिणा}
{पाल्यते च भवता निजः प्रभो पापहारिचरितैर्मनोहरैः}% २५

\twolineshloka
{दीननाथ सुरमौलिहीरकाघृष्टपादतल भक्तवल्लभ}
{पापकोटिपरिदाहक प्रभो दर्शयस्व निजदर्शनं मम}% २६

\twolineshloka
{पापकृद्यदि जनोयमागतो मानसे तव तथा हि दर्शय}
{तावका वयमघौघनाशनं विस्मृतं नहि सुरासुरार्चित}% २७

\twolineshloka
{ये वदन्ति तव नाम निर्मलं ते तरन्ति सकलाघसागरम्}
{संस्मृतिर्यदि कृता तदा मया प्राप्यतां सकलदुःखवारक}% २८

\uvacha{सुमतिरुवाच}

\twolineshloka
{एवं गायन्गुणान्रात्रौ दिवा वापि महीपतिः}
{क्षणमात्रं न विश्रान्तो निद्रामाप न वै सुखम्}% २९

\twolineshloka
{गायन्गच्छन्गृणंस्तिष्ठन्वदत्येतदहर्निशम्}
{दर्शयस्व कृपानाथ स्वतनुं पुरुषोत्तम}% ३०

\twolineshloka
{एवं राज्ञः पञ्चदिनं गतं गङ्गाब्धिसङ्गमे}
{तदा कृपाब्धिः कृपया चिन्तयामास गोपतिः}% ३१

\twolineshloka
{असौ राजा मदीयेन गानेन विगताघकः}
{पश्य तान्मामकीं प्रेष्ठां सुरासुरनमस्कृताम्}% ३२

\twolineshloka
{इति सञ्चिन्त्य भगवान्कृपापूरितमानसः}
{सन्न्यासिवेषमास्थाय ययौ राज्ञोऽन्तिकं विभुः}% ३३

\twolineshloka
{तत्र गत्वा महाराज त्रिदण्डियतिवेषधृक्}
{भक्तानुकम्पया प्राप्तो वीक्षितस्तापसेन हि}% ३४

\twolineshloka
{ॐनमो विष्णवेत्युक्त्वा नमश्चक्रे नृपोत्तमः}
{अर्घ्यपाद्यासनैः पूजां चकार हरिमानसः}% ३५

\twolineshloka
{उवाच भाग्यमतुलं यद्भवानक्षिगोचरः}
{अतः परं दास्यते मे गोविन्दो निजदर्शनम्}% ३६

\twolineshloka
{इति श्रुत्वा तु तद्वाक्यं सन्न्यासी निजगाद तम्}
{राजञ्छृणुष्व कथितं मम वाक्यविनिःसृतम्}% ३७

\twolineshloka
{अहं ज्ञानेन जानामि भूतं भव्यं भवच्च यत्}
{तस्मादहं ब्रुवे किञ्चिच्छृणुष्वैकाग्रमानसः}% ३८

\twolineshloka
{श्वो मध्याह्ने हरिर्दाता दर्शनं ब्रह्मदुर्ल्लभम्}
{पञ्चभिः स्वजनैः साकं यास्यसे परमं पदम्}% ३९

\twolineshloka
{त्वममात्यश्च महिला तव तापस वाडवः}
{पुरे तव करम्बाख्यः साधुश्च तं तु वायकः}% ४०

\twolineshloka
{एतैः पञ्चभिरेतस्मिन्नीले पर्वतसत्तमे}
{यास्यसे ब्रह्मदेवेन्द्र वन्दितं सुरपूजितम्}% ४१

\twolineshloka
{इत्युक्त्वाऽदृश्यतां प्राप्तो यतिः क्वापि न दृश्यते}
{तदाकर्ण्य नृपो हर्षं प्राप चाशु सविस्मयम्}% ४२

\uvacha{राजोवाच}

\twolineshloka
{स्वामिन्कोऽसौ समागत्य सन्न्यासी मां यदूचिवान्}
{न दृश्यते पुनः कुत्र गतोऽसौ चित्तहर्षदः}% ४३

\uvacha{तापस उवाच}

\twolineshloka
{राजंस्तव महाप्रेम्णा कृष्टचित्तः समभ्यगात्}
{पुरुषोत्तमनामायं सर्वपापप्रणाशनः}% ४४

\twolineshloka
{श्वोमध्याह्ने तव पुरो भविष्यति महान्गिरिः}
{तमारुह्य हरिं दृष्ट्वा कृतार्थस्त्वं भविष्यसि}% ४५

\twolineshloka
{इतिवाक्यसुधापूर नाशितस्वान्त सञ्ज्वरः}
{हर्षं यमाप स नृपो ब्रह्मापि न हि वेत्ति तम्}% ४६

\twolineshloka
{तदा दुन्दुभयो नेदुर्वीणापणवगोमुखाः}
{महानन्दस्तदा ह्यासीद्राजराजस्य चेतसि}% ४७

\twolineshloka
{गायन्हरिं क्षणं तिष्ठन्हसञ्जल्पन्ब्रुवन्नमन्}
{आनन्दं प्राप सुघनं सर्वसन्तापनाशनम्}% ४८

{॥इति श्रीपद्मपुराणे पातालखण्डे शेषवात्स्यायनसंवादे रामाश्वमेधे सन्न्यासिदर्शनं नाम एकविंशोऽध्यायः॥२१॥}

\dnsub{द्वाविंशोऽध्यायः}\resetShloka

\uvacha{सुमतिरुवाच}

\twolineshloka
{अथ सर्वं दिनं नीत्वा हरिस्मरणकीर्तनैः}
{रात्रौ सुष्वाप गङ्गाया रोधस्युरुफलप्रदे}% १

\twolineshloka
{ददर्श स्वप्नमध्ये तु स स्वात्मानं चतुर्भुजम्}
{शङ्खचक्रगदापद्मशार्ङ्गकोदण्डधारिणम्}% २

\fourlineindentedshloka
{नृत्यन्तं पुरुषोत्तमस्य पुरतः शर्वादि देवैः सह}
{श्रीमद्भिः स्वतनूयुतैररिगदाम्बूत्थाब्जहेत्यादिभिः}
{विष्वक्सेनवरैर्गणैः सुतनुभिः श्रीशंसदोपासितं}
{दृष्ट्वा विस्मयमाप लोकविषयं हर्षं तथात्यद्भुतम्}% ३

\twolineshloka
{ददतं मनसोऽभीष्टं पुरुषोत्तमसंज्ञितम्}
{आत्मानं च कृपापात्रममन्यत महामतिः}% ४

\twolineshloka
{इत्येवं स्वप्नविषये ददर्श नृपसत्तमः}
{प्रातः प्रबुद्धो विप्राय जगाद स्वप्नमीक्षितम्}% ५

\twolineshloka
{तच्छ्रुत्वा वाडवो धीमान्कथयामास विस्मितः}
{राजंस्त्वयासौ दृष्टो यः पुरुषोत्तमसंज्ञितः}% ६

\twolineshloka
{दास्यते शङ्खचक्रादिचिह्नितां स्वतनुं हरिः}
{इति श्रुत्वा तु तद्वाक्यं रत्नग्रीवो महामनाः}% ७

\twolineshloka
{दापयामास दानानि दीनानां मानसोचितम्}
{स्नात्वा गङ्गाब्धिसंयोगे तर्पयित्वा पितॄन्सुरान्}% ८

\twolineshloka
{गायन्हरिगुणग्रामं प्रत्यैक्षत च दर्शनम्}
{ततो मध्याह्नसमये दिविदुन्दुभयो मुहुः}% ९

\twolineshloka
{जघ्नुः सुरकराघात बहुशब्दसुशब्दिताः}
{अकस्मात्पुष्पवृष्टिश्च बभूव नृपमस्तके}% १०

\onelineshloka
{धन्योसि नृपवर्यस्त्वं नीलं पश्याक्षिगोचरम्}% ११

\twolineshloka
{शृणोतीति यदा वाक्यं नृपो देवप्रणोदितम्}
{तदा स सूर्यकोटीनामधिकान्ति धरोद्भुतः}% १२

\twolineshloka
{राज्ञोऽक्षिगोचरो जातो नीलनामा महागिरिः}
{राजतैः कानकैः शृङ्गैः समन्तात्परिराजितः}% १३

\twolineshloka
{किमग्निः प्रज्वलत्येष द्वितीयः किमु भास्करः}
{किमयं वैद्युतः पुञ्जो ह्यकस्मात्स्थिरकान्तिधृक्}% १४

\twolineshloka
{तापस ब्राह्मणो दृष्ट्वा नीलप्रस्थं सुशोभितम्}
{राज्ञे निवेदयामास एष पुण्यो महागिरिः}% १५

\twolineshloka
{तच्छ्रुत्वा नृपतिश्रेष्ठः शिरसा प्रणनाम ह}
{धन्योऽस्मि कृतकृत्योऽस्मि नीलो मे दृष्टिगोचरः}% १६

\twolineshloka
{अमात्यो राजपत्नी च करम्बस्तन्तुवायकः}
{नीलदर्शनसंहृष्टा बभूवुः पुरुषर्षभ}% १७

\twolineshloka
{पञ्चैते विजये काले नीलपर्वतमारुहन्}
{महादुन्दुभिनिर्घोषाञ्च्छृण्वन्तो ह्यमरैः कृतान्}% १८

\twolineshloka
{तस्योपरितने शृङ्गे चित्रपादपराजिते}
{ददर्श हाटकाबद्धं देवालयमनुत्तमम्}% १९

\twolineshloka
{ब्रह्मागत्य सदा पूजां करोति परमेष्ठिनः}
{नैवेद्यं कुरुते यत्र हरिसन्तोषकारकम्}% २०

\twolineshloka
{दृष्ट्वाथ तत्र विमलं देवायतनमुत्तमम्}
{प्रविवेश परीवारैः पञ्चभिः सह संवृतः}% २१

\twolineshloka
{तत्र दृष्ट्वा जातरूपे महामणिविचित्रिते}
{सिंहासने विराजन्तं चतुर्भुजमनोहरम्}% २२

\twolineshloka
{चण्ड प्रचण्ड विजय जयादिभिरुपासितम्}
{प्रणनाम सपत्नीको राजा सेवकसंयुतः}% २३

\twolineshloka
{प्रणम्य परमात्मानं महाराजः सुरोत्तमम्}
{स्नापयामास विधिवद्वेदोक्तैः स्नानमन्त्रकैः}% २४

\twolineshloka
{अर्घ्यपाद्यादिकं चक्रे प्रीतेन मनसा नृपः}
{चन्दनेन विलिप्यैनं सुवस्त्रे विनिवेद्य च}% २५

\twolineshloka
{धूपमारार्तिकं कृत्वा सर्वस्वादुमनोहरम्}
{नैवेद्यं भगवन्मूर्त्यै न्यवेदयदथो नृपः}% २६

\twolineshloka
{प्रणम्य च स्तुतिं चक्रे तापसब्राह्मणेन च}
{यथामतिगुणग्रामगुम्फितस्तोत्रसञ्चयैः}% २७

\uvacha{राजोवाच}

\twolineshloka
{एकस्त्वं पुरुषः साक्षाद्भगवान्प्रकृतेः परः}
{कार्यकारणतो भिन्नो महत्तत्त्वादिपूजितः}% २८

\twolineshloka
{त्वन्नाभिकमलाज्जज्ञे ब्रह्मा सृष्टिविचक्षणः}
{तथा संहारकर्ता च रुद्रस्त्वन्नेत्रसम्भवः}% २९

\twolineshloka
{त्वयाज्ञप्तः करोत्यस्य विश्वस्य परिचेष्टितम्}
{त्वत्तो जातं पुराणाद्यज्जगत्स्थास्नु चरिष्णु च}% ३०


\threelineshloka
{चेतनाशक्तिमाविश्य त्वमेनं चेतयस्यहो}
{तव जन्म तु नास्त्येव नान्तस्तव जगत्पते}
{वृद्धिक्षयपरीणामास्त्वयि सन्त्येव नो विभो}% ३१

\twolineshloka
{तथापि भक्तरक्षार्थं धर्मस्थापनहेतवे}
{करोषि जन्मकर्माणि ह्यनुरूपगुणानि च}% ३२

\twolineshloka
{त्वया मात्स्यं वपुर्धृत्वा शङ्खस्तु निहतोसुरः}
{वेदाः सुरक्षिता ब्रह्मन्महापुरुषपूर्वज}% ३३

\twolineshloka
{शेषो न वेत्ति महि ते भारत्यपि महेश्वरी}
{किमुतान्ये महाविष्णो मादृशास्तु कुबुद्धयः}% ३४

\twolineshloka
{मनसा त्वां न चाप्नोति वागियं परमेश्वरी}
{तस्मादहं कथं त्वां वै स्तोतुं स्यामीश्वरः प्रभो}% ३५

\twolineshloka
{इति स्तुत्वा स शिरसा प्रणाममकरोन्मुहुः}
{गद्गदस्वरसंयुक्तो रोमहर्षाङ्किताङ्गकः}% ३६

\twolineshloka
{इति स्तुत्या प्रहृष्टात्मा भगवान्पुरुषोत्तमः}
{उवाच वचनं सत्यं राजानं प्रति सार्थकम्}% ३७

\uvacha{श्रीभगवानुवाच}

\twolineshloka
{तव स्तुत्या प्रहर्षोऽभून्मम राजन्महामते}
{जानीहि त्वं महाराज मां च प्रकृतितः परम्}% ३८

\twolineshloka
{नैवेद्यभक्षणं त्वं हि शीघ्रं कुरु मनोहरम्}
{चतुर्भुजत्वमाप्तः सन्गन्तासि परमं पदम्}% ३९

\twolineshloka
{त्वत्कृत्स्तुतिरत्नेन यो मां स्तोष्यति मानवः}
{तस्यापि दर्शनं दास्ये भुक्तिमुक्तिवरप्रदम्}% ४०

\twolineshloka
{इत्येवं वचनं राजा श्रुत्वा भगवतोदितम्}
{नैवेद्यभक्षणं चक्रे चतुर्भिः सह सेवकैः}% ४१

\twolineshloka
{ततो विमानं सम्प्राप्तं किङ्किणीजालमण्डितम्}
{अप्सरोवृन्दसंसेव्यं सर्वभोगसमन्वितम्}% ४२

\twolineshloka
{पुरुषोत्तमसंज्ञं च पश्यन्राजा स धार्मिकः}
{ववन्दे चरणौ तस्य कृपापात्रकृतात्मकः}% ४३

\twolineshloka
{तदाज्ञया विमाने स आरुह्य महिलायुतः}
{जगाम पश्यतस्तस्य दिवि वैकुण्ठमद्भुतम्}% ४४

\twolineshloka
{मन्त्री धर्मपरो राज्ञः सर्वधर्मविदुत्तमः}
{ययौ साकं विमानेन ललनावृन्दसेवितः}% ४५

\twolineshloka
{तापसब्राह्मणस्तत्र सर्वतीर्थावगाहकः}
{चतुर्भुजत्वं सम्प्राप्तो ययौ देवैर्विमानिभिः}% ४६

\twolineshloka
{करम्बोऽपि महाराज गानपुण्येन दर्शनम्}
{प्राप्तो ययौ सुरावासं सर्वदेवादिदुर्ल्लभम्}% ४७

\twolineshloka
{सर्वे प्रचलिता विष्णुलोकं परममद्भुतम्}
{चतुर्भुजाः शङ्खचक्रगदापाथोजधारिणः}% ४८

\twolineshloka
{सर्वे मेघश्रियः शुद्धा लसदम्भोजपाणयः}
{हारकेयूरकटकैर्भूषिताङ्गा ययुर्दिवम्}% ४९

\twolineshloka
{तद्विमानावलीर्दृष्ट्वा लोकैः प्रकृतिभिस्तदा}
{दुन्दुभीनां तु निर्घोषस्तैः कृतः कर्णगोचरः}% ५०

\twolineshloka
{तदैको ब्राह्मणो ह्यासीद्विष्णुपादाब्जवल्लभः}
{गतस्तद्विरहाकृष्टचेता जातश्चतुर्भुजः}% ५१

\twolineshloka
{तच्चित्रं वीक्ष्य लोकास्ते प्रशंसन्तो महोदयम्}
{गङ्गासागरसंयोगे स्नात्वाऽगुस्तं पुरं प्रति}% ५२

\twolineshloka
{अहो भाग्यं भूमिपते रत्नग्रीवस्य सन्मतेः}
{जगामानेन देहेन तद्विष्णोः परमं पदम्}% ५३

\twolineshloka
{राजन्नसौ नीलगिरिः पुरुषोत्तमसत्कृतः}
{यं वीक्ष्यैव व्रजन्त्यद्धा वैकुण्ठं परमायनम्}% ५४

\twolineshloka
{एतन्नीलस्य माहात्म्यं यः शृणोति स भाग्यवान्}
{यः श्रावयति लोकान्वै तौ गच्छेतां परं पदम्}% ५५

\twolineshloka
{एतच्छ्रुत्वा तु दुःस्वप्नो नश्यति स्मृतिमात्रतः}
{प्रान्ते संसारनिस्तारं ददाति पुरुषोत्तमः}% ५६

\twolineshloka
{योऽसौ नीलाधिवासी च स रामः पुरुषोत्तमः}
{सीतासाक्षान्महालक्ष्मीः सर्वकारणकारणम्}% ५७

\twolineshloka
{हयमेधं चरित्वा स लोकान्वै पावयिष्यति}
{यन्नामब्रह्महत्यायाः प्रायश्चित्ते प्रदिश्यते}% ५८

\twolineshloka
{इदानीं त्वद्धयः प्राप्तो नीलेपर्वतसत्तमे}
{पुरुषोत्तमदेवं त्वं नमस्कुरु महामते}% ५९

\twolineshloka
{तत्र निष्पापिनो भूत्वा यास्यामः परमं पदम्}
{यस्य प्रसादाद्बहवो निस्तीर्णा भवसागरात्}% ६०

\twolineshloka
{एवं प्रवदतस्तस्य प्राप्तोऽश्वो नीलपर्वतम्}
{वायुवेगेन पृथिवीं कुर्वन्सङ्क्षुण्णमण्डलाम्}% ६१

\twolineshloka
{तदा राजापि तत्पृष्ठचारी नीलाभिधं गिरिम्}
{प्राप्तो गङ्गाब्धिसंयोगे स्नात्वागात्पुरुषोत्तमम्}% ६२

\twolineshloka
{स्तुत्वा नत्वा च देवेशं सुरासुरनमस्कृतम्}
{जातं कृतार्थमात्मानममन्यत स शत्रुहा}% ६३

{॥इति श्रीपद्मपुराणे पातालखण्डे शेषवात्स्यायनसंवादे रामाश्वमेधे नीलगिरिमहिमवर्णनं नाम द्वाविंशोऽध्यायः॥२२॥}

\dnsub{त्रयोविंशोऽध्यायः}\resetShloka

\uvacha{शेष उवाच}

\twolineshloka
{क्षणं स्थित्वा तृणान्यत्त्वा ययौ वाजी मनोजवः}
{वीरश्रेणीवृतः पत्रं भाले धृत्वा सचामरः}% १

\twolineshloka
{शत्रुघ्नेन सुवीरेण लक्ष्मीनिधि नृपेण च}
{पुष्कलेनोग्रवाहेन प्रतापाग्र्येण रक्षितः}% २

\twolineshloka
{ययौ पुरीं स चक्राङ्कां सुबाहुपरिरक्षिताम्}
{अनेकवीरकोटीभी रक्षितोऽनुगतः प्रभो}% ३

\twolineshloka
{तदा पुत्रोस्य दमनो मृगयामास्थितो महान्}
{ददर्शाश्वं भालपत्रं चन्दनादिकचर्चितम्}% ४

\twolineshloka
{विलोक्य सेवकं प्राह कस्याश्वो मेऽक्षिगोचरः}
{भाले पत्रं धृतं किं नु चामरं किन्तु शोभनम्}% ५

\twolineshloka
{इति राज्ञोवचः श्रुत्वा सेवकः प्रययौ ततः}
{यत्रासौ वर्तते वाजी भालपत्रः सुशोभनः}% ६

\twolineshloka
{गृहीत्वा तं केशसङ्घे रत्नमालाविभूषितम्}
{निनाय चाग्रे भूपस्य सुबाहुकुलधारिणः}% ७

\twolineshloka
{स पत्रं वाचयामास सुन्दराक्षरशोभितम्}
{अयोध्याधिपतिश्चासीद्राजा दशरथो बली}% ८

\twolineshloka
{तस्यात्मजो रामभद्रः सर्वशूरशिरोमणिः}
{नान्योस्ति तत्समः पृथ्व्यां धनुर्धरणविक्रमः}% ९

\twolineshloka
{तेनासौ मोचितो वाजी चन्दनादिकचर्चितः}
{तं पालयति धर्मात्मा शत्रुघ्नः परवीरहा}% १०

\twolineshloka
{ये च शूरा वयं वीरा धनुर्हस्ता इमे वयम्}
{ते गृह्णन्तु बलाद्वाहं रत्नमालाविभूषितम्}% ११

\twolineshloka
{तं च मोक्ष्यति शत्रुघ्नः सर्ववीरशिरोमणिः}
{अन्यथा पादयोस्तस्य प्रणतिं यान्तु धन्विनः}% १२

\twolineshloka
{इत्यभिप्रायमालोक्य जगाद नृपनन्दनः}
{राम एव धनुर्धारी न वयं क्षत्त्रियाः स्मृताः}% १३

\twolineshloka
{ताते मेऽवस्थिते पृथ्व्यां कोऽयं गर्वो महान्भुवि}
{प्राप्नोतु गर्वस्य फलं मम निर्मुक्तसायकैः}% १४

\twolineshloka
{अद्य मे निशिता बाणाः शत्रुघ्नं किंशुकं यथा}
{पुष्पितं विदधत्वद्धा क्षतावृतशरीरकम्}% १५

\twolineshloka
{दारयन्तु कपोलांश्च सायका मम दन्तिनाम्}
{अश्वान्पश्यन्तु शतशो रुधिरौघपरिप्लुतान्}% १६

\twolineshloka
{पिबन्तु योगिनीसङ्घा रुधिराणि नृमस्तकैः}
{शिवा भवन्तु सन्तुष्टा मद्वैरिक्रव्यभक्षणैः}% १७

\twolineshloka
{पश्यन्तु सुभटास्तस्य मम बाहुबलं महत्}
{कोदण्डदण्डनिर्मुक्ताः शरकोटीर्विमुञ्चतः}% १८

\twolineshloka
{इत्थमुक्त्वा महीपस्य तनुजो दमनाभिधः}
{स्वपुरं प्रेषयित्वा तं प्रहृष्टोऽभवदुद्भटः}% १९

\twolineshloka
{सेनापतिमुवाचेदं सज्जीकुरु महामते}
{सेनां परिमितां मह्यं वैरिवृन्दनिवारणे}% २०

\twolineshloka
{सज्जां सेनां विधायाशु सम्मुखो रणमण्डले}
{स्थितवान्या वदत्युग्रस्तावत्प्राप्ता हयानुगाः}% २१

\twolineshloka
{क्वासौ हयो महाराज्ञो भालपत्रेण चिह्नितः}
{पप्रच्छुस्ते तु चान्योन्यमतिव्याकुलिता मुहुः}% २२

\twolineshloka
{तावद्ददर्श पुरतः प्रतापाग्र्यः परन्तपः}
{सज्जीभूतं तु कटकं वीरशब्दनिनादितम्}% २३

\twolineshloka
{तत्रावदञ्जनाः केचिन्नीतोऽश्वोऽनेन भूपते}
{अन्यथा सम्मुखस्तिष्ठेत्कथं वीरो बलानुगः}% २४

\twolineshloka
{इत्याकर्ण्य प्रतापाग्र्यः प्रेषयामास सेवकम्}
{स गत्वा तत्र पप्रच्छ कुत्राश्वो रामभूपतेः}% २५

\twolineshloka
{केन नीतः कुतो नीतो रामं जानाति नो कुधीः}
{यं शक्रप्रमुखा देवा बलिमादाय सन्नताः}% २६

\twolineshloka
{तस्य वै धर्मराजस्य कुपितं तु बलं महत्}
{सर्वथा हि ग्रसिष्येत प्रणतिं चेन्न यास्यति}% २७

\twolineshloka
{इत्थमुक्तं समाकर्ण्य तदा राजसुतो बली}
{तं वै धिक्कारयामास वाग्जालेन सुदुर्मनाः}% २८

\twolineshloka
{मयानीतो यज्ञहयः पत्रचिह्नाद्यलङ्कृतः}
{ये शूरास्ते तु मां जित्वा मोचयन्तु बलादिह}% २९

\twolineshloka
{सेवकस्तद्वचः श्रुत्वा रोषपूर्णो हसन्ययौ}
{राज्ञे निवेदयामास यथावदुपवर्णितम्}% ३०

\twolineshloka
{तच्छ्रुत्वा रोषताम्राक्षः प्रतापाग्र्यो महाबलः}
{ययौ योद्धुं राजपुत्रं महावीरपुरस्कृतम्}% ३१

\twolineshloka
{रथेन कनकाङ्गेन चतुर्वाजिसुशोभिना}
{सुकूबरेण सर्वास्त्रपूरितेन ययौ बली}% ३२

\twolineshloka
{धनुष्टङ्कारयामास महाबलसमन्वितः}
{पुनःपुनर्जहासोच्चैः कोपादुद्गमिताश्रुकः}% ३३

\twolineshloka
{अश्ववाहा गजारूढाः खड्गोल्लसितपाणयः}
{अन्वयुस्ते प्रतापाग्र्यं रोषपूर्णाकुलेक्षणम्}% ३४

\twolineshloka
{हस्तिनः पत्तयश्चैव कोटिशः प्रधनोद्यताः}
{चिरकालमभीप्सन्तो रणं वीरेणकारितम्}% ३५

\twolineshloka
{तदोद्यतं समाज्ञाय रिपुसैन्यं नृपात्मजः}
{प्रत्युज्जगाम वीराग्र्यो महाबलपरीवृतः}% ३६

\twolineshloka
{सन्नद्धः कवची खड्गी शरासनधरो युवा}
{लीलयैव ययौ योद्धुं मृगराड्गजतामिव}% ३७

\twolineshloka
{तदा योधाः प्रकुपिताः परस्परवधैषिणः}
{छिन्धि भिन्धीति भाषन्तो रणकार्यविशारदाः}% ३८

\twolineshloka
{पत्तयः पत्तिसङ्घेन गजारूढाश्च सादिभिः}
{रथारूढा रथस्थैश्च वाहारूढाश्वसंस्थितैः}% ३९

\twolineshloka
{गजा भिन्ना द्विधा जाता हयाश्च द्विदलीकृताः}
{अनेकनरमस्तिष्कैर्मेदिनीपूरिता ह्यभूत्}% ४०

\twolineshloka
{तदा प्रकुपितो राजा प्रतापाग्र्यो महाबलः}
{स्वसैन्यकदनोद्युक्तं राजपुत्रं समीक्ष्य च}% ४१

\twolineshloka
{उवाच सारथिं तत्र प्रापयाश्वान्यतो मम}
{सैन्यस्य कदनासक्तो राजपुत्रो महारथः}% ४२

\twolineshloka
{अथ वीरशिरोरत्न नमिताङ्घ्रिर्नृपात्मजः}
{ययौ सम्मुखमेवास्य प्रतापाग्र्यस्य वीर्यवान्}% ४३

\twolineshloka
{सारथिः प्रापयामास प्रतापाग्र्यस्य वाजिनः}
{यत्रासौ दमनो वीरः सर्वशूरशिरोमणिः}% ४४

\twolineshloka
{गत्वा तमाह्वयामास राजपुत्रं रणोद्यतम्}
{रथे पुरटनिर्णिक्ते तिष्ठन्कोदण्डदण्डभृत्}% ४५

\twolineshloka
{रे राजपुत्र क शिशो त्वया बद्धोऽश्वसत्तमः}
{न ज्ञातोसि महाराजः सर्ववीरेन्द्र सेवितः}% ४६

\twolineshloka
{यस्य प्रतापं दैत्येन्द्रो न शक्तः सोढुमद्भुतम्}
{तस्य त्वं वाजिनं नीत्वा गतोऽसि पुटभेदनम्}% ४७

\twolineshloka
{मां जानीहि पुरः प्राप्तं कालरूपं तु वैरिणम्}
{मुञ्चाश्वमर्भ गच्छाशु बालक्रीडनकं कुरु}% ४८

\twolineshloka
{कस्यात्मजस्त्वं कुत्रत्यः कथं नोऽदीर्घदर्शिना}
{धृतोऽश्वस्त्वथ सञ्जाता घृणा मम शिशो त्वयि}% ४९

\twolineshloka
{इत्थमाकर्ण्य दमनः स्मितं चक्रे महामनाः}
{उवाच च प्रतापाग्र्यं तृणीकुर्वंश्च तद्बलम्}% ५०

\uvacha{दमन उवाच}

\twolineshloka
{मया बद्धो बलादश्वो नीतः स्वपुटभेदनम्}
{नार्पयिष्येऽद्य सप्राणः कुरु युद्धं महाबल}% ५१

\twolineshloka
{त्वया यदुक्तं बालस्त्वं गत्वा क्रीडनकं कुरु}
{तन्मे पश्य महाराज क्रीडनं रणमूर्धनि}% ५२

\uvacha{शेष उवाच}

\twolineshloka
{इत्युक्त्वा सगुणं चापं विधाय सुभुजां गजः}
{शराणां शतमाधत्त प्रतापाग्र्यस्य वक्षसि}% ५३

\twolineshloka
{सन्धाय बाणशतकं शङ्खं दध्मौ प्रतापवान्}
{तेन शङ्खनिनादेन कातराणां भयं ह्यभूत्}% ५४

\twolineshloka
{ताडयामास हृदये बाणानां शतकेन सः}
{प्रतापाग्र्यः प्रचिच्छेद लघुहस्तः सुपर्वणः}% ५५

\twolineshloka
{स बाणच्छेदनं दृष्ट्वा कुपितो व्यसृजच्छरान्}
{कङ्कपक्षान्वितांस्तीक्ष्णभल्लान्राजात्मजो बली}% ५६

\twolineshloka
{आकाशे भुवि मध्ये च बाणा ददृशिरेऽञ्चिताः}
{स्वनामचिह्नितास्तीक्ष्णधारापातसुशोभिताः}% ५७

\twolineshloka
{शरास्तद्बाहु हृदये लग्ना वह्निकणान्बहून्}
{सृजन्तः कुर्वते सैन्यदाहनं तदभून्महत्}% ५८

\twolineshloka
{प्रतापाग्र्यः प्रकुपितस्तिष्ठतिष्ठेति च ब्रुवन्}
{शरेण दशसङ्ख्येन ताडयामास मूर्धनि}% ५९

\twolineshloka
{ते बाणा राजपुत्रस्य ललाटे परिनिष्ठिताः}
{विराजन्ते स्म च मुने दशशाखास्तरोरिव}% ६०

\twolineshloka
{तेन बाणप्रहारेण विव्यथेन महामनाः}
{यष्टिकाप्रहतो यद्वत्कुञ्जरः सप्तवर्षकः}% ६१

\twolineshloka
{बाणान्धनुषि सन्धाय मुमोच त्रिशताञ्छुभान्}
{सुवर्णपुङ्खरचितान्महाकालानलोपमान्}% ६२

\twolineshloka
{ते बाणास्तु प्रतापाग्र्य वक्षो भित्त्वा गता ह्यधः}
{शोणिताक्ता यथा रामचन्द्र भक्ति पराङ्मुखाः}% ६३

\twolineshloka
{प्रतापाग्र्यः प्रकुपितः शरान्मुञ्चन्सहस्रशः}
{अकरोद्विरथं सूनुं सुबाहोस्तत्क्षणाद्द्रुतम्}% ६४

\twolineshloka
{चतुर्भिश्च तुरो वाहान्द्वाभ्यां ध्वजमशातयत्}
{एकेन सारथेः कायाच्छिरो मह्यामपातयत्}% ६५

\twolineshloka
{चतुर्भिस्ताडयामास तं सूनुं नृपतेः पुनः}
{तत्क्षणाच्चापमेकेन गुणयुक्तं तु चिच्छिदे}% ६६

\twolineshloka
{सोऽन्यरथं समारुह्य हयरत्नसुशोभितम्}
{धनुः करे समादाय सज्यं चक्रे महामनाः}% ६७

\twolineshloka
{प्रत्युवाच प्रतापाग्र्यं त्वया विक्रान्तमद्भुतम्}
{पश्येदानीं पराक्रान्तिं धनुषो मम सद्भट}% ६८

\twolineshloka
{एवमुक्त्वा तु दमनो बाणान्दश समाददे}
{चतुर्भिश्चतुरो वाहान्निनाय यमसादनम्}% ६९

\twolineshloka
{चतुर्भिस्तिलशः कृत्तो रथश्चक्रसमन्वितः}
{एकेन हृदि विव्याध बाणेनैकेन सारथिम्}% ७०

\twolineshloka
{जगर्ज शङ्खमापूर्य शङ्खशब्दसमन्वितः}
{तत्कर्म पूजयामास साधु वीर महाबल}% ७१

\twolineshloka
{इति विक्रान्तमालोक्य प्रतापाग्र्यो रुषान्वितः}
{अन्यं रथं समास्थाय ययौ योद्धुं नृपात्मजम्}% ७२

\twolineshloka
{उवाच वीर पश्य त्वं मम विक्रान्तमद्भुतम्}
{इत्युक्त्वाशु मुमोचौघाञ्छराणां शितपर्वणाम्}% ७३

\twolineshloka
{शराः सर्वत्र दृश्यन्ते कुञ्जरेषु हयेषु च}
{परब्रह्मेव सर्वत्र व्याप्ताश्चान्तरगोचराः}% ७४

\fourlineindentedshloka
{तं राजपुत्रं शितबाणकोटिभि-}
{र्व्याप्तं विधायाशु जगर्ज विक्रमी}
{संहर्षयन्स्वीयगणान्परान्महान्}
{कुर्वन्हृदा शून्यतमान्गतासुकान्}% ७५

\fourlineindentedshloka
{स राजपुत्रः शितसायकव्रजैः}
{सम्पूर्णमात्मानमवेक्ष्य रोषितः}
{जग्राह शस्त्राणि दुरन्तविक्रमो}
{धनुश्च धुन्वन्भुजदण्डयोर्महान्}% ७६

\twolineshloka
{चकर्त सर्वाण्यस्त्राणि शस्त्राणि च महाबलः}
{रोषताम्रेक्षणो मुञ्चञ्छरान्वैरिविदारिणः}% ७७

\twolineshloka
{तच्छस्त्रजालं निर्धूय राजपुत्रो जगाद तम्}
{क्षमस्वैकं प्रहारं मे यदि शूरोसि मारिष}% ७८

\twolineshloka
{यद्यनेन भवन्तं वै रथाच्चेत्पातयामि न}
{प्रतिज्ञां शृणु मे वीर मम गर्वेण निर्मिताम्}% ७९

\twolineshloka
{वेदं निन्दन्ति ये मत्ता हेतुवादविचक्षणाः}
{तेषां पापं ममैवास्तु नरकार्णवमज्जकम्}% ८०

\twolineshloka
{इत्युक्त्वा बाणमासाद्य कोदण्डे कालसन्निभम्}
{ज्वालामालाकुलं तीक्ष्णं निषङ्गादुद्धृतं वरम्}% ८१

\twolineshloka
{स मुक्तो नृपवर्येण हृदि लक्ष्यीकृतः शरः}
{जगाम तरसा तं वै कालानलसमप्रभः}% ८२

\twolineshloka
{प्रतापाग्र्यः शरं दृष्ट्वा स्वपातनसमुद्यतम्}
{बाणान्धनुष्यथाधत्त शरच्छेदायवै शितान्}% ८३

\twolineshloka
{स बाणः सर्वबाणांस्तांश्छिन्दन्मध्यत एव हि}
{जगाम वै प्रतापाग्र्यहृदयं धैर्यसंयुतम्}% ८४

\twolineshloka
{संलग्नो हृदि नालीकः प्रविवेश तदन्तरम्}
{राजाकृतप्रहारस्तु पपात धरणीतले}% ८५

\twolineshloka
{मूर्च्छितं चेतनाहीनं रथोपस्थाद्गतं भुवि}
{सारथिस्तं समादायापोवाह रणमण्डलात्}% ८६

\twolineshloka
{हाहाकारोमहानासीद्बलं भग्नं गतं ततः}
{यत्र शत्रुघ्ननामासौ वीरकोटिपरीवृतः}% ८७

\twolineshloka
{राजात्मजो जयं प्राप्य प्रतापाग्र्यं विजित्य सः}
{प्रतीक्षां तु चकारास्य शत्रुघ्नस्य च भूपतेः}% ८८

{॥इति श्रीपद्मपुराणे पातालखण्डे शेषवात्स्यायनसंवादे रामाश्वमेधे राजपुत्रयुद्धकथनं नाम त्रयोविंशोऽध्यायः॥२३॥}

\dnsub{चतुर्विंशतितमोऽध्यायः}\resetShloka

\uvacha{शेष उवाच}

\twolineshloka
{शत्रुघ्नस्तु क्रुधाविष्टो दन्तान्दन्तैर्विनिष्पिषन्}
{हस्तौ धुन्वंल्लेलिहानमधरं जिह्वया सकृत्}% १

\twolineshloka
{पुनः पुनस्तान्पप्रच्छ केनाश्वो नीयते मम}
{प्रतापाग्र्यः केन जितः सर्वशूरशिरोमणिः}% २

\twolineshloka
{सेवकास्ते तदा प्रोचुर्दमनो नाम शत्रुहन्}
{सुबाहुजः प्रतापाग्र्यं जितवान्हयमाहरत्}% ३

\twolineshloka
{इति श्रुत्वा हयं नीतं दमनेन स्ववैरिणा}
{आजगाम स वेगेन यत्राभूद्रणमण्डलम्}% ४

\twolineshloka
{तत्रापश्यत्स शत्रुघ्नो गजान्दीर्णकपोलकान्}
{पर्वतानिव रक्तोदे मज्जमानान्मदोद्धतान्}% ५

\twolineshloka
{हयास्तत्र निजारोहकर्तृभिः सहिताः क्षताः}
{मृता वीरेण ददृशे शत्रुघ्नेन सुकोपिना}% ६

\twolineshloka
{नरान्रथान्गजान्भग्नान्वीक्षमाणः स शत्रुहा}
{अतीव चुक्रुधे यद्वत्प्रलये प्रलयार्णवः}% ७

\twolineshloka
{पुरतो दमनं वीक्ष्य हयनेतारमुद्भटम्}
{प्रतापाग्र्यस्य जेतारं तृणीकृत्य निजं बलम्}% ८

\twolineshloka
{तदा राजा प्रत्युवाच योधान्कोपाकुलेक्षणः}
{कोऽसौ दमन जेताऽत्र सर्वशस्त्रास्त्रधारकः}% ९

\twolineshloka
{यो वै राजसुतं वीरं रणकर्मविशारदम्}
{जेष्यत्यस्त्रेण निर्भीतः सज्जीभूतो भवत्वयम्}% १०

\twolineshloka
{इति वाक्यं समाकर्ण्य पुष्कलः परवीरहा}
{दमनं जेतुमुद्युक्तो जगाद वचनं त्विदम्}% ११

\twolineshloka
{स्वामिन्क्वायं दमनकः क्व तेऽपरिमितं बलम्}
{जेष्येऽहं त्वत्प्रतापेन गच्छाम्येष महामते}% १२

\twolineshloka
{सेवके मयि युद्धाय स्थिते कैर्नीयते हयः}
{रघुनाथप्रतापोऽयं सर्वं कृत्यं करिष्यति}% १३

\twolineshloka
{स्वामिञ्छृणु प्रतिज्ञां मे तव मोदप्रदायिनीम्}
{विजेष्ये दमनं युद्धे रणकर्मविचक्षणम्}% १४

\twolineshloka
{रामचन्द्रपदाम्भोजमध्वास्वादवियोगिनाम्}
{यदघं तु भवेत्तन्मे दमनं न जयेयदि}% १५

\twolineshloka
{पुत्रो यो मातृपादान्यत्तीर्थं मत्वा तया सह}
{विरुद्ध्येत्तत्तमो मह्यं न जयेदमनं यदि}% १६

\twolineshloka
{अद्य मद्बाणनिर्भिन्न महोरस्को नृपाङ्गजः}
{अलङ्करोतु प्रधने भूतलं शयनेन हि}% १७

\uvacha{शेष उवाच}

\twolineshloka
{इति प्रतिज्ञामाकर्ण्य पुष्कलस्य रघूद्वहः}
{जहर्ष चित्ते तेजस्वी निदिदेश रणं प्रति}% १८

\twolineshloka
{आज्ञप्तोऽसौ ययौ सैन्यैर्बहुभिः परिवारितः}
{यत्रास्ते दमनो राजपुत्रः शूरकुलोद्भवः}% १९

\twolineshloka
{दमनोऽपि तमाज्ञाय ह्यागतं रणमण्डले}
{प्रत्युज्जगाम वीराग्र्यः स्वसैन्यपरिवारितः}% २०

\twolineshloka
{अन्योन्यं तौ सम्मिलितौ रथस्थौ रथशोभिनौ}
{समरे शक्रदैत्यौ किं युद्धार्थं रणमागतौ}% २१

\twolineshloka
{उवाच पुष्कलस्तं वै राजपुत्रं महाबलम्}
{राजपुत्र दमनक मां जानीहि समागतम्}% २२

\twolineshloka
{स प्रतिज्ञं तु युद्धाय भरतात्मजमुद्भटम्}
{पुष्कलेन स्वनाम्ना च लक्षितं विद्धिसत्तम}% २३

\twolineshloka
{रघुनाथपदाम्भोज नित्यसेवामधुव्रतम्}
{त्वां जेष्ये शस्त्रसङ्घेनसज्जीभव महामते}% २४

\twolineshloka
{इति वाक्यं समाकर्ण्य दमनः परवीरहा}
{प्रत्युवाच हसन्वाग्मी निर्भयोद्दृष्टविक्रमः}% २५

\twolineshloka
{सुबाहुपुत्रं दमनं पितृभक्ति हृताघकम्}
{विद्धि मामश्वनेतारं शत्रुघ्नस्य महीपतेः}% २६

\twolineshloka
{जयो दैवविसृष्टोऽयं यस्य चालङ्करिष्यति}
{स प्राप्नोति निरीक्षस्व बलं मे रणमूर्धनि}% २७

\twolineshloka
{इत्युक्त्वा स शरं चापं विधायाकर्णपूरितम्}
{मुमोच बाणान्निशितान्वैरिप्राणापहारिणः}% २८

\twolineshloka
{ते बाणास्त्वाविलीभूताश्छादयामासुरम्बरम्}
{सूर्यभानुप्रभा यत्र बाणच्छायानिवारिता}% २९

\twolineshloka
{गजानां कटभित्त्योघे लग्ना सायकसन्ततिः}
{अलङ्करोति धातूनां रागा इव विचित्रिताः}% ३०

\twolineshloka
{पतितास्तत्र दृश्यन्ते नरा वाहा गजा रथाः}
{शरव्रातेन नृपतेः सुतेन परिताडिताः}% ३१

\twolineshloka
{तद्विक्रान्तं समालोक्य पुष्कलः परवीरहा}
{शराणां छायया व्याप्तं रणमण्डलमीक्ष्य च}% ३२

\twolineshloka
{शरासने समाधत्त बाणं वह्न्यभिमन्त्रितम्}
{आचम्य सम्यग्विधिवन्मोचयामास सायकम्}% ३३

\twolineshloka
{ततोऽग्निप्रादुरभवत्तत्र सङ्ग्राममूर्धनि}
{ज्वालाभिर्विलिहन्व्योम प्रलयाग्निरिवोत्थितः}% ३४

\twolineshloka
{ततोऽस्य सैन्यं निर्दग्धं त्रासं प्राप्तं रणाङ्गणे}
{पलायनपरं जातं वह्निज्वालाभिपीडितम्}% ३५

\twolineshloka
{छत्राणि तु प्रदग्धानि चन्द्राकाराणि धन्विनाम्}
{दृश्यन्ते जातरूपाभ कान्तिधारीणि तत्र ह}% ३६

\twolineshloka
{हया दग्धाः पलायन्ते केसरेषु च वैरिणाम्}
{रथा अपि गता दाहं सुकूबरसमन्विताः}% ३७

\twolineshloka
{मणिमाणिक्यरत्नानि वहन्तः करभास्ततः}
{पलायन्ते दहनभू ज्वालामालाभिपीडिताः}% ३८

\twolineshloka
{कुत्रचिद्दन्तिनो नष्टाः कुत्रचिद्धयसादिनः}
{कुत्रचित्पत्तयो नष्टा वह्निदग्धकलेवराः}% ३९

\twolineshloka
{शराः सर्वे नृपसुतप्रमुक्ताः प्रलयं गताः}
{आशुशुक्षणिकीलाभिर्भस्मीभूताः समन्ततः}% ४०

\twolineshloka
{तदा स्वसैन्ये दग्धे च दमनो रोषपूरितः}
{सर्वास्त्रवित्तच्छान्त्यर्थं वारुणास्त्रमथा ददे}% ४१

\twolineshloka
{वारुणं वह्निशान्त्यर्थं मुक्तं तेन महीभृता}
{आप्लावयद्बलं तस्य रथवाजिसमाकुलम्}% ४२

\twolineshloka
{रथा विप्लावितास्तोये दृश्यन्ते परिपन्थिनाम्}
{गजाश्चापि परिप्लुष्टाः स्वीयाः शान्तिमुपागताः}% ४३

\twolineshloka
{वह्निश्च शान्तिमगमदग्न्यस्त्र परिमोचितः}
{शान्तिमाप बलं स्वीयं वह्निज्वालाभिपीडितम्}% ४४

\twolineshloka
{कम्पिताः शीततोयेन सीत्कुर्वन्ति च वैरिणः}
{करकावृष्टिभिः क्षिप्ता वायुना च प्रपीडिताः}% ४५

\twolineshloka
{तदा स्वबलमालोक्य तोयपूरेण पीडितम्}
{कम्पितं क्षुभितं नष्टं वारुणेन विनिर्हृतम्}% ४६

\twolineshloka
{तदातिकोपताम्राक्षः पुष्कलो भरतात्मजः}
{वायव्यास्त्रं समाधत्त धनुष्येकं महाशरम्}% ४७

\twolineshloka
{ततो वायुर्महानासीद्वायव्यास्त्रप्रचोदितः}
{नाशयामास वेगेन घनानीकमुपस्थितम्}% ४८

\twolineshloka
{वायुना स्फालिता नागाः परस्परसमाहताः}
{अश्वाश्च संहतान्योन्यं स्वस्वारोहसमन्विताः}% ४९

\twolineshloka
{नराः प्रभञ्जनोद्धूता मुक्तकेशा निरोजसः}
{पतन्तोऽत्र समीक्ष्यन्ते वेताला इव भूगताः}% ५०

\twolineshloka
{वायुना स्वबलं सर्वं परिभूतं विलोक्य सः}
{राजपुत्रः पर्वतास्त्रं धनुष्येवं समादधे}% ५१

\twolineshloka
{तदा तु पर्वताः पेतुर्मस्तकोपरि युध्यताम्}
{वायुः सञ्च्छादितस्तैस्तु न प्रचक्राम कुत्रचित्}% ५२

\twolineshloka
{पुष्कलो वज्रसंज्ञं तु समाधत्त शरासने}
{वज्रेण कृत्तास्ते सर्वे जाताश्च तिलशः क्षणात्}% ५३

\twolineshloka
{वज्रं नगान्रजः शेषान्कृत्वा बाणाभिमन्त्रितम्}
{राजपुत्रोरसि प्रोच्चैः पपात स्वनवद्भृशम्}% ५४

\twolineshloka
{सत्वाकुलितचेतस्को हृदि विद्धः क्षतो भृशम्}
{विव्यथे बलवान्वीरः कश्मलं परमाप सः}% ५५

\twolineshloka
{तं वै कश्मलितं दृष्ट्वा सारथिर्नयकोविदः}
{अपोवाह रणात्तस्मात्क्रोशमात्रं नरेन्द्रजम्}% ५६

\twolineshloka
{ततो योधा राजसूनोः प्रणष्टाः प्रपलायिताः}
{गत्वा पुरीं समाचख्युः कश्मलस्थं नृपात्मजम्}% ५७

\twolineshloka
{पुष्कलो जयमाप्यैवं रणमूर्धनि धर्मवित्}
{न प्रहर्तुं पुनः शक्तो रघुनाथवचः स्मरन्}% ५८

\twolineshloka
{ततो दुन्दुभिनिर्घोषो जयशब्दो महानभूत्}
{साधुसाध्विति वाचश्च प्रावर्तन्त मनोहराः}% ५९

\twolineshloka
{हर्षं प्राप स शत्रुघ्नो जयिनं वीक्ष्य पुष्कलम्}
{प्रशशंस सुमत्यादि मन्त्रिभिः परिवारितः}% ६०

{॥इति श्रीपद्मपुराणे पातालखण्डे शेषवात्स्यायनसंवादे रामाश्वमेधे पुष्कलविजयो नाम चतुर्विंशतितमोऽध्यायः॥२४॥}

\dnsub{पञ्चविंशोऽध्यायः}\resetShloka

\uvacha{शेष उवाच}

\twolineshloka
{अथ वीक्ष्य भटान्निजान्नृपो रुधिरौघेण परिप्लुताङ्गकान्}
{सुखमाप न वै शुशोच तान्परिपप्रच्छ सुतस्य चेष्टितम्}% १

\twolineshloka
{गदताखिलकर्म तस्य वै स कथं चाहरदश्ववर्यकम्}
{कथयन्तु पुनः कियद्बलं बत वीराः कति योद्धुमागताः}% २

\twolineshloka
{अथ शत्रुबलोन्मुखः कथं मम वीरो दमनो रणं व्यधात्}
{विजयं च विधाय दुर्जयं किल वीरं बत कोऽप्यशातयत्}% ३

\twolineshloka
{इत्याकर्ण्य वचो राज्ञः प्रत्यूचुस्तेऽस्य सेवकाः}
{क्षतजेन परिक्लिन्न गात्रवस्त्रादिधारिणः}% ४

\twolineshloka
{राजन्नश्वं समालोक्य पत्रचिह्नाद्यलङ्कृतम्}
{ग्राहयामास गर्वेण तृणीकृत्य रघूत्तमम्}% ५

\twolineshloka
{ततो हयानुगः प्राप्तः स्वल्पसैन्यसमावृतः}
{तेन साकमभूद्युद्धं तुमुलं रोमहर्षणम्}% ६

\twolineshloka
{तं मूर्च्छितं ततः कृत्वा तव पुत्रः स्वसायकैः}
{यावत्तिष्ठत्यथायातः शत्रुघ्नः स्वबलैर्वृतः}% ७

\twolineshloka
{ततो युद्धं महदभूच्छस्त्रास्त्रपरिबृंहितम्}
{बहुशो जयमापेदे तव पुत्रो महाबलः}% ८

\twolineshloka
{इदानीं तेन मुक्त्वास्त्रं शत्रुघ्नभ्रातृसूनुना}
{मूर्च्छितः प्रधने राजन्कृतो वीरः सुतस्तव}% ९

\twolineshloka
{इति वाक्यं समाकर्ण्य रोषशोकसमन्वितः}
{स्थगिताङ्ग इवासीत्स समुद्र इव पर्वणि}% १०

\twolineshloka
{उवाच सेनाधिपतिं रोषप्रस्फुरिताधरः}
{दन्तैर्दताँल्लिहन्नोष्ठं जिह्वया शोककर्शितः}% ११

\twolineshloka
{सेनापते कुरुष्वारान्मम सेनां तु सज्जिताम्}
{योत्स्ये रामस्य सुभटैर्ममपुत्रोपघातकैः}% १२

\twolineshloka
{अद्याहं मम पुत्रस्य दुःखदं निशितैः शरैः}
{पातयिष्ये यदि ह्येनं रक्षितापि महेश्वरः}% १३

\twolineshloka
{सेनापतिरिदं वाक्यं प्रोक्तं सुभुजभूपतेः}
{निशम्य च तथा कृत्वा सज्जीभूतो भवत्स्वयम्}% १४

\twolineshloka
{राज्ञे निवेदयामास ससज्जां चतुरङ्गिणीम्}
{सेनां कालबलप्रख्यां हतदुर्जनकोटिकाम्}% १५

\twolineshloka
{श्रुत्वा सेनापतेर्वाक्यं सुबाहुः परवीरहा}
{निर्जगाम ततो यत्र शत्रुघ्नः स्वसुतार्दनः}% १६

\twolineshloka
{कुञ्जरैश्च मदोन्मत्तैर्हयैश्चापि मनोजवैः}
{रथैश्च सर्वशस्त्रास्त्रपूरितै रिपुजेतृभिः}% १७

\twolineshloka
{भूश्चकम्पे तदा तत्र सैन्यभारेण भूरिणा}
{सम्मर्दः सुमहानासीत्तत्र सैन्ये विसर्पति}% १८

\twolineshloka
{राजानं निर्गतं दृष्ट्वा रथेन कनकाङ्गिना}
{शत्रुघ्नबलमुद्युक्तं सर्ववैरिप्रहारकम्}% १९

\twolineshloka
{सुकेतुस्तस्य वै भ्राता गदायुद्धविशारदः}
{रथेनाश्वा जगामायं सर्वशस्त्रास्त्रपूरितः}% २०

\twolineshloka
{चित्राङ्गस्तु सुतो राज्ञः सर्वयुद्धविचक्षणः}
{जगाम स्वरथेनाशु शत्रुघ्नबलमुन्मदम्}% २१

\twolineshloka
{तस्यानुजो विचित्राख्यो विचित्ररणकोविदः}
{ययौ रथेन हैमेन भ्रातृदुःखेन पीडितः}% २२

\twolineshloka
{अन्ये शूरा महेष्वासाः सर्वशस्त्रास्त्रकोविदाः}
{ययुर्नृपसमादिष्टाः प्रधनं वीरपूरितम्}% २३

\twolineshloka
{राजा सुबाहुः संरोषादागतः प्रधनाङ्गणे}
{विलोकयामास सुतं मूर्च्छितं शरपीडितम्}% २४

\twolineshloka
{रथोपस्थस्थितं मूढं स्वसुतं दमनाभिधम्}
{वीक्ष्य दुःखं मुहुः प्राप वीजयामास पल्लवैः}% २५

\twolineshloka
{जलेन सिक्तः संस्पृष्टो राज्ञा कोमलपाणिना}
{संज्ञामाप शनैर्वीरो दमनः परमास्त्रवित्}% २६

\twolineshloka
{उत्थितः क्व धनुर्मेऽस्ति क्व पुष्कल इतो गतः}
{संसज्य समरं त्यक्त्वा मद्बाणव्रणपीडितः}% २७

\twolineshloka
{इति वाक्यं समाकर्ण्य सुबाहुः पुत्रभाषितम्}
{परमं हर्षमापेदे परिरभ्य सुतं स्वकम्}% २८

\twolineshloka
{दमनो वीक्ष्य जनकं नृपं नम्रशिरोधरः}
{पपात पादयोर्भक्त्या क्षतदेहोऽस्त्रराजिभिः}% २९

\twolineshloka
{स्वसुतं रथसंस्थं तु विधाय नृपतिः पुनः}
{जगाद सेनाधिपतिं रणकर्मविशारदः}% ३०

\twolineshloka
{व्यूहं रचय सङ्ग्रामे क्रौञ्चाख्यं रिपुदुर्जयम्}
{यमाविश्य जये सैन्यं शत्रुघ्नस्य महीपतेः}% ३१

\fourlineindentedshloka
{तद्वाक्यमाकर्ण्य सुबाहुभूपतेः}
{क्रौञ्चाख्यसद्व्यूहविशेषमादधात्}
{यन्नो विशन्ते सहसा रिपोर्गणा}
{महाबलाः शस्त्रसमूहधारिणः}% ३२

\twolineshloka
{मुखे सुकेतुस्तस्यासीद्गले चित्राङ्गसंज्ञकः}
{पक्षयो राजपुत्रौ द्वौ पुच्छे राजा प्रतिष्ठितः}% ३३

\twolineshloka
{मध्ये सैन्यं महत्तस्य चतुरङ्गैस्तु शोभितम्}
{कृत्वा न्यवेदयद्राज्ञे क्रौञ्चव्यूहं विचित्रितम्}% ३४

\twolineshloka
{राजा दृष्ट्वा सुसन्नद्धं क्रौञ्चव्यूहं सुनिर्मितम्}
{रणाय स्वमतिं चक्रे शत्रुघ्नकटके स्थितैः}% ३५

{॥इति श्रीपद्मपुराणे पातालखण्डे शेषवात्स्यायनसंवादे रामाश्वमेधे सुबाहुसैन्यसमागमो नाम पञ्चविंशोऽध्यायः॥२५॥}

\dnsub{षड्विंशतितमोऽध्यायः}\resetShloka

\uvacha{शेष उवाच}

\twolineshloka
{शत्रुघ्नस्तद्बलं दृष्ट्वा भीषणाकृतिमेघवत्}
{हस्त्यश्वरथपादातैर्बहुभिः परिवारितम्}% १

\twolineshloka
{सुमतिं प्रत्युवाचेदं वचोगम्भीरशब्दयुक्}
{नानावाक्यविचारज्ञैः पण्डितैः परिसेवितः}% २

\uvacha{शत्रुघ्न उवाच}

\twolineshloka
{सुमते कस्य नगरं प्राप्तो मे हयसत्तमः}
{बलमेतन्निरीक्षेहं पयोदधितरङ्गवत्}% ३

\twolineshloka
{कस्यैतद्बलमुद्धर्षं चतुरङ्गसमन्वितम्}
{पुरतो भाति युद्धाय समुपस्थितमादरात्}% ४

\twolineshloka
{एतत्सर्वं समाचक्ष्व यथावत्पृच्छतो मम}
{यज्ज्ञात्वा युद्धसंस्थायै निर्दिशामि स्वकान्भटान्}% ५

\twolineshloka
{इति वाक्यं समाकर्ण्य सुमतिः शुभबुद्धिमान्}
{उवाच वचनं प्रीतः शत्रुघ्नं वैरितापनम्}% ६

\uvacha{सुमतिरुवाच}

\twolineshloka
{चक्राङ्का नगरी राजन्वर्तते सविधे शुभा}
{यस्यां सन्ति नराः पापरहिता विष्णुभक्तितः}% ७

\twolineshloka
{तस्याः पुर्याः पतिरयं सुबाहुर्धर्मवित्तमः}
{तवायं पुरतो भाति पुत्रपौत्रसमावृतः}% ८

\twolineshloka
{स्वदारनिरतो नित्यं परदारपराङ्मुखः}
{विष्णोः कथास्य कर्णस्थाना परार्थप्रकाशिनी}% ९

\twolineshloka
{परस्वं न समादत्ते षष्ठांशादधिकं नृपः}
{ब्राह्मणा विष्णुभक्त्यैव पूज्यन्ते तेन धर्मिणा}% १०

\twolineshloka
{नित्यं सेवारतो विष्णुपादपद्ममधुव्रतः}
{एष स्वधर्मनिरतः परधर्मपराङ्मुखः}% ११

\twolineshloka
{एतस्य बलतुल्यं हि न वीराणां बलं क्वचित्}
{पुत्रस्य पतनं श्रुत्वा रोषशोकसमाकुलः}% १२

\twolineshloka
{चतुरङ्गसमेतोऽयं युद्धाय समुपस्थितः}
{तवापि वीरा बहवो लक्ष्मीनिधिमुखा अमून्}% १३

\twolineshloka
{जेष्यन्ति शस्त्रसङ्घेन निर्दिशाशु परं हि तान्}
{शत्रुघ्नस्तद्वचः श्रुत्वा प्रोवाच स्वभटान्वरान्}% १४

\twolineshloka
{रणप्राप्तिभवोद्धर्षपूरपूरितमानसान्}
{क्रौञ्चव्यूहोऽद्य रचितः सुबाहुपरिसैनिकैः}% १५

\twolineshloka
{मुखपक्षस्थिता योधास्तान्को भेत्स्यति शस्त्रवित्}
{यस्य भेदे निजा शक्तिर्यो वीर विजयोद्यतः}% १६

\twolineshloka
{स गृह्णातु मदीयाद्धि पाणिपद्माच्च वीटकम्}
{तदा लक्ष्मीनिधिर्वीरो जग्राह क्रौञ्चभेदने}% १७

\twolineshloka
{सर्वशस्त्रास्त्रविद्वीरैर्बहुभिः परिवारितः}
{उवाच वचनं राजन्यास्येऽहं क्रौञ्चभेदने}% १८

\twolineshloka
{भार्गवः पूर्वमेवासीत्क्रौञ्चभेत्ता तथा ह्यहम्}
{तथान्यं वीरमावोचत्कोऽस्य सार्धं गमिष्यति}% १९

\twolineshloka
{पुष्कलः पृष्ठतस्तस्य यातुं चक्रे मतिं ततः}
{रिपुतापो नीलरत्न उग्राश्वो वीरमर्दनः}% २०

\twolineshloka
{सर्वे शत्रुघ्नसन्देशाद्ययुस्तत्क्रौञ्चभेदने}
{शत्रुघ्नोऽपि रथस्थश्च सर्वायुधधरः परः}% २१

\twolineshloka
{पृष्ठतोऽस्य परीयाय बहुभिः सैनिकैर्वृतः}
{तदा प्रचलितौ दृष्टावन्योन्यबलवारिधी}% २२

\twolineshloka
{प्रलयं कर्तुमुद्युक्तौ जगतः सुतरङ्गिणौ}
{तदा भेर्यः समाजघ्नुरुभयोः सेनयोर्दृढाः}% २३

\twolineshloka
{रणभेर्यः शङ्खनादाः श्रूयन्ते तत्र तत्र ह}
{हेषन्ते वाजिनस्तत्र गर्जन्ति द्विरदा भृशम्}% २४

\twolineshloka
{हुं हुं कुर्वन्ति वीराग्र्या नदन्ति रथनेमयः}
{तत्र प्रकुपिताः शूराः सुबाहुबलदर्पिताः}% २५

\twolineshloka
{छिन्धि भिन्धीति भाषन्तो दृश्यन्ते बहवो रणे}
{एवम्भूते रणोद्युक्ते सैन्ये शत्रुघ्नवैरिणोः}% २६

\onelineshloka*
{मुखसंस्थं सुकेतुं तं लक्ष्मीनिधिरुवाच ह}

\uvacha{लक्ष्मीनिधिरुवाच}

\onelineshloka
{जनकस्य सुतं विद्धि लक्ष्मीनिधिरिति स्मृतम्}% २७

\twolineshloka
{सर्वशस्त्रास्त्रकुशलं सर्वयुद्धविशारदम्}
{मुञ्चाश्वं रामचन्द्रस्य सर्वदानवदंशितुः}% २८

\twolineshloka
{नोचेन्मद्बाणनिर्भिन्नो यास्यसे यमसादनम्}
{इति ब्रुवन्तं वीराग्र्यं सुकेतुः सहसा त्वरन्}% २९

\twolineshloka
{सज्यं चापं विधायाशु बाणान्मुञ्चन्स्थिरोऽभवत्}
{ते बाणाः शितपर्वाणः स्वर्णपुङ्खाः समन्ततः}% ३०

\onelineshloka*
{दृश्यन्ते व्यापिनस्तत्र रणमध्ये सुदुर्भराः}

\fourlineindentedshloka
{तद्बाणजालं तरसा निहत्य}
{लक्ष्मीनिधिश्चापमथा ततज्यम्}
{विधाय तस्योरसि बाणषट्कं}
{मुमोच तीक्ष्णं शितपर्वशोभितम्}% ३१

\twolineshloka
{तद्बाणाः सुभुजभ्रातुर्हृदयं संविदार्य च}
{गतास्ते भुवि दृश्यन्ते रुधिराक्ता मलीमसाः}% ३२

\twolineshloka
{तद्बाणभिन्नहृदयः सुकेतुः कोपपूरितः}
{जघानशरविंशत्या तीक्ष्णया नतपर्वया}% ३३

\twolineshloka
{उभौ बाणविभिन्नाङ्गावुभौ क्षतजविप्लुतौ}
{सैनिकैः परिदृश्यन्ते किंशुकाविव पुष्पितौ}% ३४

\twolineshloka
{मुञ्चन्तौ बाणकोटीश्च सन्दधन्तौ त्वरा शरान्}
{न केनापि विलक्ष्येते लघुहस्तौ महाबलौ}% ३५

\twolineshloka
{कुण्डलीकृत सच्चापौ वर्षन्तौ बाणधारया}
{नवाम्बुदाविव दिवि शक्रनिर्देशकारिणौ}% ३६

\twolineshloka
{तयोर्बाणा गजान्वाहान्नराञ्छूरान्विमस्तकान्}
{कुर्वन्तः केवलं दृष्टा न च सन्धानमोक्षयोः}% ३७

\twolineshloka
{पृथिवी सुभटैः पूर्णा सकिरीटैः सकुण्डलैः}
{धनुर्बाणकरै रोषसन्दष्टाधरयुग्मकैः}% ३८

\twolineshloka
{तयोः प्रयुद्ध्यतोर्दर्पात्सर्वशस्त्रास्त्रवेदिनोः}
{युद्धं समभवद्घोरं देवविस्मापनं महत्}% ३९

\twolineshloka
{सम्मर्दोऽभवदत्यन्तं वीरकोटिविदारणः}
{न केनचित्क्वचिद्दृष्टं शरजालान्तरेऽम्बरम्}% ४०

\twolineshloka
{तस्मिंस्तु समये लक्ष्मीनिधिर्वीरोऽरिमर्दनः}
{बाणांश्चापे समाधत्त वसुसङ्ख्यान्दृढाञ्छितान्}% ४१

\twolineshloka
{चतुर्भिस्तुरगान्वीरः सुकेतोरनयत्क्षयम्}
{एकेन ध्वजमत्युग्रं चिच्छेद तरसा हसन्}% ४२

\twolineshloka
{एकेन सारथेः कायाच्छिरोभूमावपातयत्}
{एकेन चापं सगुणमच्छिनद्रोषपूरितः}% ४३

\twolineshloka
{एकेन हृदि विव्याध सुकेतोर्वेगवान्नृपः}
{तत्कर्माद्भुतमुद्वीक्ष्य वीरा विस्मयमाययुः}% ४४

\twolineshloka
{सच्छिन्नधन्वा विरथो हताश्वो हतसारथिः}
{महतीं स गदां धृत्वा योद्घुकामोऽभ्युपेयिवान्}% ४५

\twolineshloka
{तमायान्तं समालक्ष्य गदायुद्धविशारदम्}
{महत्या गदया युक्तं रथादवततार सः}% ४६

\twolineshloka
{गदामादाय महतीं सर्वायसविनिर्मिताम्}
{जातरूपविचित्राङ्गीं सर्वशोभापुरस्कृताम्}% ४७

\twolineshloka
{लक्ष्मीनिधिर्भृशं क्रुद्धः सुकेतोर्वक्षसि त्वरन्}
{ताडयामास हृदये गदां वज्राग्निसन्निभाम्}% ४८

\twolineshloka
{गदया ताडितो वीरो नाकम्पत महामुने}
{मदोन्मत्तो यथा दन्ती बालेन स्रग्भिराहतः}% ४९

\twolineshloka
{उवाच तं स वीराग्र्यो नृपं लक्ष्मीनिधिं तदा}
{सहस्वैकं प्रहारं मे यदि शूरः परन्तप}% ५०

\twolineshloka
{इत्युक्त्वा ताडयामास ललाटे गदया भृशम्}
{गदया ताडितो भालेऽसृग्वमन्कुपितो भृशम्}% ५१

\twolineshloka
{मूर्ध्नि तं ताडयामास गदया कालरूपया}
{सुकेतुरपि तं स्कन्धे ताडयामास धर्मवित्}% ५२

\twolineshloka
{एवं भृशं प्रकुपितौ गदायुद्धविशारदौ}
{गदायुद्धं प्रकुर्वाणौ परस्परजयैषिणौ}% ५३

\twolineshloka
{अन्योन्याघातविमतौ परस्परवधोद्यतौ}
{न कोपि तत्र हीयेत न को जीयेत संयुगे}% ५४

\twolineshloka
{मूर्ध्नि भाले तथा स्कन्धे हृदि गात्रेषु सर्वतः}
{रुधिरौघ परिक्लिन्नौ महाबलपराक्रमौ}% ५५

\twolineshloka
{तदा लक्ष्मीनिधिः क्रुद्धो गदामुद्यम्य वेगवान्}
{जगाम प्रबलं हन्तुं हृदि राजानुजं बली}% ५६

\twolineshloka
{तमायान्तमथालोक्य स्वगदां महतीं दधत्}
{ययौ तं तरसा हन्तुं राजभ्राता बलाद्बलम्}% ५७

\twolineshloka
{गदां तेन विनिक्षिप्तां स्वकरे धृतवानयम्}
{तयैव गदया तस्य हृदि जघ्ने महाबलः}% ५८

\twolineshloka
{स्वगदां तेन वै नीतां दृष्ट्वा लक्ष्मीनिधिर्नृपः}
{बाहुयुद्धेन तं योद्धुमियेष बलवत्तमम्}% ५९

\twolineshloka
{तदा राजानुजः क्रुद्धो बाहुभ्यामुपगृह्य तम्}
{युयुधे सर्वयुद्धस्य ज्ञातावीरेषु सत्तमः}% ६०

\twolineshloka
{तदा लक्ष्मीनिधिस्तस्य हृदि जघ्ने स्वमुष्टिना}
{तदा सोपि शिरस्येनं मुष्टिमुद्यम्य चाहनत्}% ६१

\twolineshloka
{मुष्टिभिर्वज्रसङ्काशैस्तलस्फोटैश्च दारुणैः}
{अन्योन्यं जघ्नतुः क्रुद्धौ सन्दष्टाधरपल्लवौ}% ६२

\twolineshloka
{मुष्टी मुष्टि दन्ता दन्ति कचा कचि नखा नखि}
{उभयोरभवद्युद्धं तुमुलं रोमहर्षणम्}% ६३

\twolineshloka
{तदा प्रकुपितो भ्राता नृपतेश्च रणे नृपम्}
{गृहीत्वा भ्रामयित्वाथ पातयामास भूतले}% ६४

\twolineshloka
{लक्ष्मीनिधिः करे गृह्य तं नृपानुजमुच्चकैः}
{भ्रामयित्वा शतगुणं गजोपस्थे जघान तम्}% ६५

\twolineshloka
{स तदा पतितो भूमौ संज्ञां प्राप्य क्षणादनु}
{तथैव भ्रामयामास व्योम्नि वेगेन विक्रमी}% ६६

\twolineshloka
{एवं प्रयुध्यमानौ तौ बाहुयुद्धं गतौ पुनः}
{पादे पादं करे पाणिं हृदि हृद्वदने मुखम्}% ६७

\twolineshloka
{एवं परस्परं श्लिष्टौ परस्परवधैषिणौ}
{उभावपि पराक्रान्तावुभावपि मुमूर्च्छतुः}% ६८

\twolineshloka
{तद्दृष्ट्वा विस्मयं प्राप्ताः प्रशशंसुः सहस्रशः}
{धन्यो लक्ष्मीनिधिर्भूपो धन्यो राजानुजो बली}% ६९

{॥इति श्रीपद्मपुराणे पातालखण्डे शेषवात्स्यायनसंवादे रामाश्वमेधे गदायुद्धं नाम षड्विंशतितमोऽध्यायः॥२६॥}

\dnsub{सप्तविंशतितमोऽध्यायः}\resetShloka

\uvacha{शेष उवाच}

\twolineshloka
{चित्राङ्गः क्रौञ्चकण्ठस्थो रथस्थो वीरशोभितः}
{गाहयामास तत्सैन्यं वाराह इव वारिधिम्}% १

\twolineshloka
{धनुर्विस्फार्य सुदृढं मेघनादनिनादितम्}
{मुमोच बाणान्निशितान्वैरिकोटिविदाहकान्}% २

\twolineshloka
{तद्बाणभिन्नसर्वाङ्गाः शेरते सुभटा भृशम्}
{सकिरीटतनुत्राणाः सन्दष्टदशनच्छदाः}% ३

\twolineshloka
{एवं प्रवृत्ते सङ्ग्रामे ययौ योद्धुं स पुष्कलः}
{मणिचित्रितमादाय चापं वैरिप्रतापनम्}% ४

\twolineshloka
{तयोः सङ्गतयोरूपं दृश्यतेऽतिमनोहरम्}
{पुरा तारकसङ्ग्रामे स्कन्दतारकयोर्यथा}% ५

\twolineshloka
{विस्फारयन्धनुः शीघ्रं सव्यसाची तु पुष्कलः}
{ताडयामास तं क्षिप्रं शरैः सन्नतपर्वभिः}% ६

\twolineshloka
{चित्राङ्गोऽपि रुषाक्रान्तः शरासन इषूञ्छितान्}
{दधद्व्यमुञ्चद्बहुशो रणमण्डलमूर्धनि}% ७

\twolineshloka
{नादानं न च सन्धानं न मोचनमथापि वा}
{दृष्टं तावेव सन्दृष्टौ कुण्डलीकृतचापिनौ}% ८

\twolineshloka
{तदासौ पुष्कलः क्रुद्धः शराणां शतकेन तम्}
{विव्याध वक्षःस्थलके महायोद्धारमुद्भटम्}% ९

\twolineshloka
{चित्राङ्गस्ताञ्शरान्सर्वांश्चिच्छेद तिलशः क्षणात्}
{ताडयामास चाङ्गेषु पुष्कलं शितसायकैः}% १०

\twolineshloka
{पुष्कलस्तद्रथं दिव्यं भ्रामकास्त्रेण शोभिना}
{नभसि भ्रामयामास तदद्भुतमिवाभवत्}% ११

\twolineshloka
{भ्रान्त्वा मुहूर्तमात्रं तु सरथो हयसंयुतः}
{स्थितिर्लेभेतिकष्टेन सन्धृतो रणमण्डले}% १२

\twolineshloka
{स चास्य विक्रमं दृष्ट्वा चित्राङ्गः कुपितो भृशम्}
{उवाच पुष्कलं धीमान्सर्वास्त्रेषु विशारदः}% १३

\uvacha{चित्राङ्ग उवाच}

\twolineshloka
{त्वया साधुकृतं कर्म सुभटैर्युधिसम्मतम्}
{मद्रथो वाजिसंयुक्तो भ्रामितो नभसि क्षणम्}% १४

\twolineshloka
{पराक्रमं समीक्षस्व ममापि सुभटेरितम्}
{आकाशचारी तु भवान्भवत्वमरपूजितः}% १५

\twolineshloka
{इत्युक्त्वा स मुमोचास्त्रं रणे परमदारुणम्}
{धनुषा परमास्त्रज्ञः सर्वधर्मविदुत्तमः}% १६

\twolineshloka
{तेन बाणेन संविद्धः खे बभ्राम पतङ्गवत्}
{सरथः सहयः सङ्ख्ये सध्वजश्च ससारथिः}% १७

\twolineshloka
{भ्रान्त्वा सरथवर्यस्तु नभसि त्वरयान्वितः}
{यावत्स्थितिं न लभते तावन्मुक्तोऽपरः शरः}% १८

\twolineshloka
{पुनश्च परिबभ्राम रथः सूतसमन्वितः}
{तत्कर्मवीक्ष्य पुत्रस्य राज्ञो विस्मयमाप सः}% १९

\twolineshloka
{कथञ्चित्स्थितिमप्याप पुष्कलः परवीरहा}
{रथं जघान बाणैश्च ससूतहयमस्य च}% २०

\twolineshloka
{सभग्नस्यन्दनो वीरः पुनरन्यं समाश्रितः}
{सोऽपि भग्नः शरैराशु पुष्कलेन रणाङ्गणे}% २१

\twolineshloka
{पुनरन्यं समास्थाय यावदायाति सम्मुखम्}
{तावद्बभञ्ज निशितैः सायकैस्तद्रथं पुनः}% २२

\twolineshloka
{एवं दश रथा भग्ना नृपतेरात्मजस्य हि}
{पुष्कलेन तु वीरेण महासंयुगशालिना}% २३

\twolineshloka
{तदा चित्राङ्गकः सङ्ख्ये रथे स्थित्वा विचित्रिते}
{आजगाम ह वेगेन पुष्कलं प्रति योधितुम्}% २४

\twolineshloka
{पुष्कलं पञ्चभिर्बाणैस्ताडयामास संयुगे}
{तैर्बाणैर्निहतोऽत्यतं विव्यथे भरतात्मजः}% २५

\twolineshloka
{सक्रुद्धश्चापमुद्यम्य बाणान्दश शितान्महान्}
{मुमोच हृदये तस्य स्वर्णपुङ्खसुशोभितान्}% २६

\twolineshloka
{ते बाणाः पपुरेतस्य रुधिरं बहुदारुणाः}
{पीत्वा पेतुः क्षितौ कूटसाक्षिणः पूर्वजा इव}% २७

\twolineshloka
{तदा चित्राङ्गकः क्रुद्धो भल्लान्पञ्च समाददे}
{मुमोच भाले पुत्रस्य भरतस्य महौजसः}% २८

\twolineshloka
{तैर्भल्लैराहतः क्रुद्धः शरासनवरे शरम्}
{दधत्प्रतिज्ञामकरोच्चित्राङ्गनिधनं प्रति}% २९

\twolineshloka
{शृणु वीर मम क्षिप्रं प्रतिज्ञां त्वद्वधाश्रिताम्}
{तज्ज्ञात्वा सावधानेन योद्धव्यं च त्वयात्र हि}% ३०

\twolineshloka
{बाणेनानेन चेत्त्वां वै न कुर्यां प्राणवर्जितम्}
{सतीं सन्दूष्य वनितां शीलाचारसुशोभिताम्}% ३१

\twolineshloka
{यो लोकः प्राप्यते लोकैर्यमस्य वशवर्तिभिः}
{स लोको मम वै भूयात्सत्यं मम प्रतिश्रुतम्}% ३२

\twolineshloka
{इति श्रेष्ठं वचः श्रुत्वा जहास परवीरहा}
{उवाच मतिमान्वीरः पुष्कलं वचनं शुभम्}% ३३

\twolineshloka
{मृत्युर्वै प्राणिनां भाव्यः सर्वत्रैव च सर्वदा}
{तस्मान्मे निधने दुःखं नास्ति शूरशिरोमणे}% ३४

\twolineshloka
{प्रतिज्ञा या कृता वीर त्वया वीरत्वशालिना}
{सा सत्यैव पुनर्मेऽद्य श्रूयतां व्याहृतं महत्}% ३५

\twolineshloka
{त्वद्बाणं मद्वधोद्युक्तं न च्छिन्द्यां यदि चेदहम्}
{तदा प्रतिज्ञां शृणु मे सर्ववीराभिमानिनः}% ३६

\twolineshloka
{तीर्थं जिगमिषोर्यो वै कुर्यात्स्वान्तविखण्डनम्}
{एकादशीव्रतादन्यज्जानाति व्रतमुच्चकैः}% ३७

\twolineshloka
{तस्य पापं ममैवास्तु प्रतिज्ञापरिघातिनः}
{इति वाक्यमुदीर्यैव तूष्णीम्भूतो धनुर्दधे}% ३८

\twolineshloka
{तदानेन निषङ्गात्स्वादुद्धृत्य सायकं वरम्}
{कथयामास विशदं वाक्यं शत्रुवधावहम्}% ३९

\uvacha{पुष्कल उवाच}

\twolineshloka
{यदि रामाङ्घ्रियुगुलं निष्कापट्येन चेतसा}
{उपासितं मया तर्हि मम वाक्यमृतं भवेत्}% ४०

\twolineshloka
{यदि स्वमहिलां भुक्त्वा नान्यां जानामिचेतसा}
{तेन सत्येन मे वाक्यं सत्यं भवतु सङ्गरे}% ४१

\twolineshloka
{इति वाक्यमुदीर्याशु बाणं धनुषि सन्धितम्}
{कालानलोपमं वीरशिरश्छेदनमाक्षिपत्}% ४२

\twolineshloka
{तं बाणं मुक्तमालोक्य स तु राजसुतो बली}
{बाणं शरासने धत्त तीक्ष्णं कालानलोपमम्}% ४३

\twolineshloka
{तेन बाणेन सञ्छिन्नो बाणः स्ववधउद्यतः}
{हाहाकारो महानासीच्छिन्ने तस्मिञ्छरे तदा}% ४४

\twolineshloka
{परार्धं पतितं भूमौ पूर्वार्धं फलसंयुतम्}
{शिरोधरां चकर्ताशु पद्मनालमिव क्षणात्}% ४५

\twolineshloka
{तदा भूमौ पतन्तं तु दृष्ट्वा तत्तस्यसैनिकाः}
{हाहाकृत्वा भृशं सर्वे पलायनपरागताः}% ४६

\twolineshloka
{पृथ्व्यां तन्मस्तकं श्रेष्ठं सकिरीटं सकुण्डलम्}
{शुशुभेऽतीव पतितं चन्द्रबिम्बं दिवो यथा}% ४७

\twolineshloka
{तं वीक्ष्य पतितं वीरः पुष्कलो भरतात्मजः}
{व्यगाहत व्यूहमिमं सर्ववीरैकशोभितम्}% ४८

{॥इति श्रीपद्मपुराणे पातालखण्डे शेषवात्स्यायनसंवादे रामाश्वमेधे चित्राङ्गवधो नाम सप्तविंशतितमोऽध्यायः॥२७॥}

\dnsub{अष्टाविंशतितमोऽध्यायः}\resetShloka

\uvacha{शेष उवाच}

\twolineshloka
{अथ पुत्रं समालोक्य पतितं व्यसुमुद्धतम्}
{विललाप भृशं राजा सुतदुःखेन दुःखितः}% १

\twolineshloka
{मूर्ध्नि सन्ताडयामास पाणिभ्यामतिदुःखितः}
{कम्पमानो भृशं चाश्रूण्यमुञ्चन्नयनाब्जयोः}% २

\twolineshloka
{गृहीत्वा पतितं वक्त्रं चन्द्रबिम्बमनोरमम्}
{पुष्कलेषु क्षतासृग्भिः क्लिन्नं कुण्डलशोभितम्}% ३

\twolineshloka
{कुटिलभ्रूयुगं श्रेष्ठं सन्दष्टाधरपल्लवम्}
{स चुम्बन्मुखपद्मेन विलपन्निदमब्रवीत्}% ४

\fourlineindentedshloka
{हा पुत्र वीर कथमुत्सुकचेतसं मां}
{किं नेक्षसे विशदनेत्रयुगेन शूर}
{किं मद्विनोदकथयारहितस्त्वमेव}
{रोषोदधिप्लुतमनाः किल लक्ष्यसे च}% ५

\twolineshloka
{वद पुत्र कथं मां त्वं प्रब्रूषे न हसन्पुनः}
{अमृतैर्मधुरास्वादैर्विनोदयसि पुत्रक}% ६

\twolineshloka
{शत्रुघ्नाश्वं गृहाण त्वं सितचामरशोभितम्}
{स्वर्णपत्रेण शोभाढ्यं त्यज निद्रां महामते}% ७

\twolineshloka
{एष प्रतापविशदः प्रतापाग्र्यः परन्तपः}
{धनुर्बिभ्रत्पुरो भाति पुष्कलः परवीरहा}% ८

\twolineshloka
{एनं वारय सत्तीक्ष्णैर्बाणैः कोदण्डनिर्गतैः}
{कथं त्वं रणमध्ये वै शेते वीरविमोहितः}% ९

\twolineshloka
{हस्तिनः पत्तयश्चैव रथारूढा भयार्दिताः}
{शरणं त्वां समायान्ति तानीक्षस्व महामते}% १०

\twolineshloka
{पुत्र त्वया विना सोढुं कथं शक्तो रणाङ्गणे}
{शत्रुघ्नसायकांस्तीक्ष्णांश्चण्डकोदण्डनिर्गतान्}% ११

\twolineshloka
{अतो मां तु त्वया हीनं को वा पालयितुं क्षमः}
{यदि त्यक्ष्यसि निद्रा त्वं जयायाहं क्षमस्तदा}% १२

\twolineshloka
{इत्थं विलप्य सुभृशं तताड हृदयं स्वकम्}
{बहुशः पाणिना राजा पुत्रदुःखेन दुःखितः}% १३

\twolineshloka
{तदा विचित्र दमनौ स्व स्व स्यन्दनसंस्थितौ}
{पितुश्चरणयोर्नत्वा ऊचतुः समयोचितम्}% १४

\twolineshloka
{राजन्नस्मासु जीवत्सु किं दुःखं हृदि तद्वद}
{वीराणां प्रधने मृत्युर्वाञ्च्छितो जायते महान्}% १५

\twolineshloka
{धन्योऽयं बत चित्राङ्गो यो वीर भुवि शोभते}
{सकिरीटस्तु सन्दष्टदन्तच्छदयुगः प्रभुः}% १६

\twolineshloka
{कथयाशु किमद्यैव कुर्वस्ते कार्यमीप्सितम्}
{शत्रुघ्नवाहिनीं सर्वां हन्व आवामनाथिनीम्}% १७

\twolineshloka
{अद्यैव पुष्कलं भ्रातुर्वधकारिणमाहवे}
{पातयावो रथाच्छित्त्वा शिरोमुकुटमण्डितम्}% १८

\twolineshloka
{त्यज शोकं सुदुःखार्तः कथं भासि महामते}
{आज्ञापयावां मानार्ह कुरु युद्धे मतिं तथा}% १९

\twolineshloka
{इति वाक्यं समाकर्ण्य पुत्रयोर्वीरमानिनोः}
{शोकं त्यक्त्वा महाराजो युद्धाय मतिमादधात्}% २०

\twolineshloka
{तावपि प्रतियोद्धारं वाञ्च्छन्तौ रणदुर्मदौ}
{जग्मतुः कटके शत्रोरनन्तभटपूरिते}% २१

\twolineshloka
{रिपुतापेन दमनो नीलरत्नेन चेतरः}
{युयुधाते रणे वीरौ प्रावृषीव बलाहकौ}% २२

\twolineshloka
{राजा कनकसन्नद्धे रथे मणिविचित्रिते}
{रत्नकूबरशोभाढ्ये तिष्ठंश्चापधरो बली}% २३

\twolineshloka
{ययौ योद्धुं तु शत्रुघ्नं वीरकोटभिरावृतम्}
{तृणीकुर्वन्महावीरान्धनुर्विद्याविशारदान्}% २४

\twolineshloka
{तं योद्धुमागतं दृष्ट्वा सुबाहुं रोषपूरितम्}
{पुत्रनाशेन कुर्वन्तं सर्वसैन्यवधादिकम्}% २५

\twolineshloka
{शत्रुघ्नपार्श्वसञ्चारी हनूमांस्तमुपाद्रवत्}
{नखायुधो महानादं कुर्वन्मेघ इवाहवे}% २६

\twolineshloka
{सुबाहुस्तं हनूमन्तमागच्छन्तं महारवम्}
{उवाच प्रहसन्वाक्यं रोषपूरितलोचनः}% २७

\twolineshloka
{क्व गतः पुष्कलो हत्वा मत्पुत्रं रणमण्डले}
{पातयाम्यद्य तस्याशु शिरो ज्वलितकुण्डलम्}% २८

\twolineshloka
{क्व शत्रुघ्नो वाहपालः क्व च रामः कुतो भटाः}
{प्राणहन्तारमायान्तं पश्यन्तु प्रधने तु माम्}% २९

\twolineshloka
{इति तद्वाक्यमाकर्ण्य हनूमान्निजगाद तम्}
{शत्रुघ्नो लवणच्छेत्ता वर्तते सैन्यपालकः}% ३०

\twolineshloka
{स कथं प्रधने युध्येत्सेवकेऽग्रे स्थिते नृप}
{मां विजित्य रणे तं च त्वं गन्तासि नरर्षभ}% ३१

\twolineshloka
{इत्युक्तवन्तं तरसा विव्याध दशसायकैः}
{हृदि वानरमत्युग्रं पर्वताग्र्यमिवस्थितम्}% ३२

\twolineshloka
{सबाणानागतांस्तांश्च गृहीत्वा करसम्पुटे}
{चूर्णयामास तिलशः शितान्वैरिविदारणान्}% ३३

\twolineshloka
{चूर्णयित्वा शरांस्तांस्तु निनदन्घनगर्जितैः}
{पुच्छेनावेष्ट्य तस्योच्चै रथं निन्ये महाबलः}% ३४

\twolineshloka
{तदा तं नृपवर्योऽसावाकाशे स्थित एव सः}
{लाङ्गूलं ताडयामास शिताग्रैः सायकैर्मुहुः}% ३५

\twolineshloka
{स ताडितस्तु पुच्छाग्रे शरैः सन्नतपर्वभिः}
{मुमोच तद्रथं दिव्यं कनकेन विचित्रितम्}% ३६

\twolineshloka
{स मुक्तस्तेन तरसा शरैस्तीक्ष्णैर्जघान तम्}
{हनूमन्तं कपिवरं रोषसम्पूरितेक्षणः}% ३७

\twolineshloka
{हनूमान्बाणविच्छिन्नः सर्वत्ररुधिराप्लुतः}
{महारोषं समाधत्त नृपोपरि कपीश्वरः}% ३८

\twolineshloka
{गृहीत्वा तस्य दंष्ट्राभी रथं हयसमन्वितम्}
{चूर्णयामस वेगेन तदद्भुतमिवाभवत्}% ३९

\twolineshloka
{स्वरथं भज्यमानं तु दृष्ट्वा राजा त्वरन्बली}
{अन्यं रथं समास्थाय युयुधे तं महाबलम्}% ४०

\twolineshloka
{पुच्छे मुखेऽथोरसि च भुजे चरणयोर्नृपः}
{जघान शरसन्धानकोविदः परमास्त्रवित्}% ४१

\twolineshloka
{तदा क्रुद्धः कपिवरस्ताडयामास वक्षसि}
{पादेनोत्प्लुत्य वेगेन राज्ञः सुभटशोभिनः}% ४२

\twolineshloka
{स पदा प्रहतो भूमौ पपात किल मूर्च्छितः}
{मुखाद्वमन्नसृक्चोष्णं श्वासपूरप्रवेपितः}% ४३

\twolineshloka
{तदा प्रकुपितोऽत्यन्तं हनूमान्प्रधनाङ्गणे}
{अश्वान्वीरान्गजांश्चापि चूर्णयामास वेगतः}% ४४

\twolineshloka
{तदा सुकेतुस्तद्भ्राता तथा लक्ष्मीनिधिर्नृपः}
{उभावपि सुसन्नद्धौ युद्धाय समुपस्थितौ}% ४५

\twolineshloka
{राजानं मूर्च्छितं दृष्ट्वा प्रपलाय्य गता नराः}
{इतस्ततो बाणसङ्घैः क्षताः पुष्कलवर्षितैः}% ४६

\twolineshloka
{तद्भज्यमानं स्वबलं वीक्ष्य राजात्मजो बली}
{दमनः स्तम्भयामास सेतुर्वार्धिमिवोच्चलम्}% ४७

\twolineshloka
{तदा तु मूर्च्छितो राजा स्वप्नमेकं ददर्श ह}
{रणमध्ये कपिवरप्रपदाघातताडितः}% ४८

\twolineshloka
{रामचन्द्रस्त्वयोध्यायां सरयूतीरमण्डपे}
{ब्राह्मणैर्याज्ञिकश्रेष्ठैर्बहुभिः परिवारितः}% ४९

\twolineshloka
{तत्र ब्रह्मादयो देवास्तत्र ब्रह्माण्डकोटयः}
{कृतप्राञ्जलयस्तं वै स्तुवन्ति स्तुतिभिर्मुहुः}% ५०

\twolineshloka
{रामं श्यामं सुनयनं मृगशृङ्गपरिग्रहम्}
{गायन्ति नारदाद्याश्च वीणोल्लसितपाणयः}% ५१

\twolineshloka
{नृत्यन्त्यप्सरसस्तत्र घृताची मेनकादयः}
{वेदा मूर्तिधरा भूत्वा उपतिष्ठन्ति राघवम्}% ५२

\twolineshloka
{यच्च किञ्चिद्वस्तुजातं सर्वशोभासमन्वितम्}
{तस्य दातारमखिलं भक्तानां भोगदायकम्}% ५३

\twolineshloka
{इत्येवमादिसम्पश्यञ्जाग्रत्संज्ञामवाप सः}
{ब्रह्मशापहतज्ञानः किं दृष्टमिति वै वदन्}% ५४

\twolineshloka
{उत्थाय प्रययौ पद्भ्यां शत्रुघ्नचरणं प्रति}
{भृत्यकोटिपरीवारो रथकोटिपरीवृतः}% ५५

\twolineshloka
{सुकेतुं तु समाहूय विचित्रं दमनं तथा}
{युद्धं कर्तुं समुद्युक्तान्वारयामास धर्मवित्}% ५६

\twolineshloka
{उवाच तान्महाराजो धर्मात्मा धर्मसंयुतः}
{भ्रातःपुत्रौ शृणुत मे वाक्यं धर्मसमन्वितम्}% ५७

\twolineshloka
{मा युद्धं कुरुत क्षिप्रमनयस्तु महानभूत्}
{यद्रामचन्द्रवाहं त्वमगृह्णाद मनोर्ज्जितम्}% ५८

\twolineshloka
{एष रामः परम्ब्रह्म कार्यकारणतः परम्}
{चराचरजगत्स्वामी न मानुषवपुर्धरः}% ५९

\twolineshloka
{एतद्धि ब्रह्मविज्ञानमधुना ज्ञातवानहम्}
{पुरासिताङ्गशापेन हृतज्ञानधनोऽनघाः}% ६०

\twolineshloka
{अहं पुरा तीर्थयात्रां गतस्तत्त्वविवित्सया}
{तत्रानेके मया दृष्टा मुनयो धर्मवित्तमाः}% ६१

\twolineshloka
{असिताङ्गं मुनिमहं गतवाञ्ज्ञातुमिच्छया}
{तदा प्रोवाच मां विप्रः कृपां कृत्वा ममोपरि}% ६२

\twolineshloka
{योऽसावयोध्याधिपतिः स परब्रह्मशब्दितः}
{तस्य या जानकी देवी साक्षात्सा चिन्मयी स्मृता}% ६३

\twolineshloka
{एनं तु योगिनः साक्षादुपासते यमादिभिः}
{दुस्तरा पारसंसारवारिधिं सन्तितीर्षवः}% ६४

\twolineshloka
{स्मृतमात्रो महापापहारी स गरुडध्वजः}
{य एनं सेवते विद्वान्स संसारं तरिष्यति}% ६५

\twolineshloka
{तदाहमहसं विप्रं कोऽयं रामस्तु मानुषः}
{केयं सा जानकी देवी हर्षशोकसमाकुला}% ६६

\twolineshloka
{अजन्मनः कथं जन्म अकर्तुः कृत्यमत्र किम्}
{जन्मदुःखजरातीतं कथयस्व मुने मम}% ६७

\twolineshloka
{इत्युक्तवन्तं मां क्रुद्धः शशाप स मुनीश्वरः}
{अज्ञात्वा तत्स्वरूपं त्वं प्रतिब्रूषे ममाधम}% ६८

\twolineshloka
{एनं निन्दसि रामं त्वं मानुषोऽयमिदं हसन्}
{तस्मात्त्वं तत्त्वसम्मूढो भविष्यस्युदरम्भरिः}% ६९

\twolineshloka
{तदाहं तस्य चरणावगृह्णं सदया युतः}
{दृष्ट्वा मे विनयं मां तु प्रावोचत्करुणानिधिः}% ७०

\twolineshloka
{त्वं रामस्य मखे विघ्नं करिष्यसि यदा नृप}
{तदा हनूमानङ्घ्रिं त्वां ताडयिष्यति वेगतः}% ७१

\twolineshloka
{तदा त्वं ज्ञास्यसे राजन्नान्यथा स्वमनीषया}
{पुराहमुक्तस्तेनैवं तद्दृष्टमधुना मया}% ७२

\twolineshloka
{यदा मां हनुमान्क्रुद्धस्ताडयामास वक्षसि}
{तदाऽदर्शं रमानाथं पूर्णब्रह्मस्वरूपिणम्}% ७३

\twolineshloka
{तस्मादश्वं तु शोभाढ्यमानयन्तु महाबलाः}
{धनानि चैव वासांसि राज्यं चेदं समर्पये}% ७४

\twolineshloka
{रामं दृष्ट्वा कृतार्थः स्यामहं यज्ञेति पुण्यदे}
{आनयन्तु हयं मह्यं रोचते तु तदर्पणम्}% ७५

{॥इति श्रीपद्मपुराणे पातालखण्डे शेषवात्स्यायनसंवादे रामाश्वमेधे सुबाहुपराजयो नाम अष्टाविंशतितमोऽध्यायः॥२८॥}

\dnsub{एकोनत्रिंशत्तमोऽध्यायः}\resetShloka

\uvacha{शेष उवाच}

\twolineshloka
{ते तु तातवचः श्रुत्वा हर्षिताः सम्प्रहारिणः}
{तथेत्यूचुर्महाराजं रामदर्शनलालसम्}% १

\uvacha{पुत्रा ऊचुः}

\twolineshloka
{राजन्भवत्पदाम्भोजान्नान्यं जानीमहे वयम्}
{यत्तव स्वान्ततो जातं तद्भवत्वद्य वेगतः}% २

\twolineshloka
{अश्वोऽयं नीयतां तत्र सितचामरभूषितः}
{रत्नमालातिशोभाढ्यश्चन्दनादिकचर्चितः}% ३

\twolineshloka
{राज्यमाज्ञाफलं स्वामिन्कोशा बहुसमृद्धयः}
{वासांसि सुमहार्हाणि सूक्ष्माणि सुगुणानि च}% ४

\twolineshloka
{चन्दनं चन्द्रकं चैव वाजिनः सुमनोहराः}
{हस्तिनस्तु मदोद्धूता रथाः काञ्चनकूबराः}% ५

\twolineshloka
{विचित्रतरवर्णादि नानाभूषणभूषिताः}
{दास्यः शतसहस्रं च दासाश्च सुमनोरमाः}% ६

\twolineshloka
{मणयः सूर्यसङ्काशा रत्नानि विविधानि च}
{मुक्ताफलानि शुभ्राणि गजकुम्भभवानि च}% ७

\twolineshloka
{विद्रुमाः शतसाहस्रा यद्यद्वस्तुमहोदयम्}
{तत्सर्वं रामचन्द्राय देहि राजन्महामते}% ८

\twolineshloka
{सुतानस्मान्किङ्करान्नः सर्वानर्पय भूपते}
{कथं न कुरुषेराजंस्तदधीनं नृपासनम्}% ९

\uvacha{शेष उवाच}

\twolineshloka
{इति पुत्रवचः श्रुत्वा हर्षितोऽभून्महीपतिः}
{उवाच च सुतान्वीरान्स्ववाक्यकरणोद्यतान्}% १०

\uvacha{राजोवाच}

\twolineshloka
{आनयन्तु हयं सर्वे सन्नद्धाः शस्त्रपाणयः}
{नानारथपरीवारास्ततो यास्ये नृपं प्रति}% ११

\uvacha{शेष उवाच}

\twolineshloka
{इति राज्ञोवचः श्रुत्वा विचित्रो दमनस्तथा}
{सुकेतुः समरे शूरा जग्मुस्तस्याज्ञयोद्यताः}% १२

\twolineshloka
{ते गत्वाथ पुरीं शूरा वाजिनं सुमनोरमम्}
{सितचामरसंयुक्तं स्वर्णपत्राद्यलङ्कृतम्}% १३

\twolineshloka
{रत्नमालाविभूषाढ्यं चित्रपत्रेणशोभितम्}
{विचित्रमणिभूषाढ्यं मुक्ताजालस्वलङ्कृतम्}% १४

\twolineshloka
{रज्ज्वा धृतं महावीरैः पूर्वतः पृष्ठतो भटैः}
{महाशस्त्रास्त्रसंयुक्तैः सर्वशोभासमन्वितैः}% १५

\twolineshloka
{सितातपत्रमस्योच्चैर्भाति मूर्धनि वाजिनः}
{सुचामरद्वयं यस्य ध्रियते पुरतो मुहुः}% १६

\twolineshloka
{कृष्णागर्वादिधूपैश्च धूपितं वायुवेगिनम्}
{राज्ञः पुरो निनायाश्वं हयमेधस्य सत्क्रतोः}% १७

\twolineshloka
{तमानीतं हयं दृष्ट्वा रत्नमालाविभूषितम्}
{मनोजवं कामरूपं जहर्ष मतिमान्नृपः}% १८

\twolineshloka
{जगाम पद्भ्यां शत्रुघ्नं राजचिह्नाद्यलङ्कृतः}
{स्वपुत्रपौत्रैः संयुक्तो राजा परमधार्मिकः}% १९

\twolineshloka
{ययौ कर्तुं धनानां स सद्व्ययं चलगामिनाम्}
{एतद्विनश्वरं मत्वा दुःखदं सक्तचेतसाम्}% २०

\twolineshloka
{शत्रुघ्नं स ददर्शाथ सितच्छत्रेण शोभितम्}
{चामरैर्वीज्यमानञ्च सेवकैः पुरतः स्थितैः}% २१

\twolineshloka
{सुमतिं परिपृच्छन्तं रामचन्द्रकथानकम्}
{भयवार्ताविनिर्मुक्तं वीरशोभास्वलङ्कृतम्}% २२

\twolineshloka
{वीरैः कोटिभिराकीर्णं वाजिपालनकाङ्क्षिभिः}
{वानराणां सहस्रैश्च समन्तात्परिवारितम्}% २३

\twolineshloka
{दृष्ट्वा शत्रुघ्नचरणौ प्रणनाम सपुत्रकः}
{धन्योऽहमिति संहृष्टो वदन्रामैकमानसः}% २४

\twolineshloka
{शत्रुघ्नस्तं प्रणयिनं दृष्ट्वा राजानमुद्भटम्}
{उत्थायासनतः सर्वैर्भटैर्दोर्भ्यां स सस्वजे}% २५

\twolineshloka
{दृढं सम्पूज्य राजा तं शत्रुघ्नं परवीरहा}
{उवाच हर्षमापन्नो गद्गदस्वरया गिरा}% २६

\uvacha{सुबाहुरुवाच}

\twolineshloka
{अद्य धन्योस्मि ससुतः सकुटुम्बः सवाहनः}
{यद्युष्मच्चरणौ द्रक्ष्ये नृपकोटिभिरीडितौ}% २७

\twolineshloka
{अज्ञानिना सुतेनायं गृहीतो वाजिनां वरः}
{दमनेनानयं त्वस्य क्षमस्व करुणानिधे}% २८

\twolineshloka
{न जानाति रघूत्तंसं सर्वदेवाधिदैवतम्}
{लीलया विश्वस्रष्टारं हन्तारमपि पालकम्}% २९

\twolineshloka
{इदं राज्यं समृद्धाङ्गं समृद्धबलवाहनम्}
{इमे कोशा धनैः पूर्णा इमे पुत्रा इमे वयम्}% ३०

\twolineshloka
{सर्वे वयं रामनाथास्त्वदाज्ञा प्रतिपालकाः}
{गृहाण सर्वं सफलं न मेऽस्ति क्वचिदुन्मतम्}% ३१

\twolineshloka
{क्वासौ हनूमान्रामस्य चरणाम्भोजषट्पदः}
{यत्प्रसादादहं प्राप्स्ये राजराजस्य दर्शनम्}% ३२

\twolineshloka
{साधूनां सङ्गमे किं किं प्राप्यते न महीतले}
{यत्प्रसादादहं मूढो ब्रह्मशापमतीतरम्}% ३३

\twolineshloka
{दृष्ट्वा त्वद्य महाराजं पद्मपत्रनिभेक्षणम्}
{प्राप्स्यामि जन्मनः सर्वं फलं दुर्लभमत्र च}% ३४

\twolineshloka
{मम तावद्गतं चायुर्बहुरामवियोगिनः}
{स्वल्पमुर्वरितं तत्र कथं द्रक्ष्ये रघूत्तमम्}% ३५

\twolineshloka
{मह्यं दर्शयतं रामं यज्ञकर्मविचक्षणम्}
{यदङ्घ्रिरजसापूता शिलाभूता मुनिप्रिया}% ३६

\twolineshloka
{काकः परं पदं प्राप्तो यद्बाणस्पर्शनात्खगः}
{अनेके यस्य वक्त्राब्जं वीक्ष्य सङ्ख्ये पदं गताः}% ३७

\twolineshloka
{ये त्वस्य रघुनाथस्य नाम गृह्णन्ति सादराः}
{ते यान्ति परमं स्थानं योगिभिर्यद्विचिन्त्यते}% ३८

\twolineshloka
{धन्यायोध्याभवा लोका ये राममुखपञ्जम्}
{स्वलोचनपुटैः पीत्वा सुखं यान्ति महोदयम्}% ३९

\twolineshloka
{इति सम्भाष्य नृपतिं वाहं राज्यं धनानि च}
{सर्वं समर्प्य चावोचत्किङ्करोस्मि महीपते}% ४०

\twolineshloka
{इति वाक्यं समाकर्ण्य राज्ञः परपुरञ्जयः}
{प्रत्युवाचेति तं भूपं वाग्मी वाक्यविशारदः}% ४१

\uvacha{शत्रुघ्न उवाच}

\twolineshloka
{कथं राजन्निदं ब्रूषे त्वं वृद्धो मम पूजितः}
{सर्वं त्वदीयं त्वद्राज्यं दमनो विदधात्वयम्}% ४२

\twolineshloka
{क्षत्त्रियाणामिदं कृत्यं यत्सङ्ग्रामविधायकम्}
{सर्वं राज्यं धनं चेदं प्रतियातु ममाज्ञया}% ४३

\twolineshloka
{यथा मे रघुनाथस्तु पूज्यो वाङ्मनसा सदा}
{तथा त्वमपि मत्पूज्यो भविष्यसि महीपते}% ४४

\twolineshloka
{भवान्सज्जो भवत्वद्य हयस्यानुगमं प्रति}
{सन्नद्धः कवची खड्गी गजाश्वरथसंयुतः}% ४५

\twolineshloka
{इति वाक्यं समाकर्ण्य शत्रुघ्नस्य महीपतिः}
{पुत्रं राज्येऽभिषेच्यैव शत्रुघ्नेन सुपूजितः}% ४६

\twolineshloka
{महारथैः परिवृतो निजं पुत्रं रणाङ्गणे}
{पुष्कलेन हतं भूपः संस्कृत्य विधिपूर्वकम्}% ४७

\twolineshloka
{क्षणं शुशोच तत्त्वज्ञो लोकदृष्ट्या महारथः}
{ज्ञानेनानाशयच्छोकं रघुनाथमनुस्मरन्}% ४८

\twolineshloka
{सज्जीभूतो रथे तिष्ठन्महासैन्यसमावृतः}
{आजगाम स शत्रुघ्नं महारथिपुरस्कृतः}% ४९

\twolineshloka
{राजा तमागतं दृष्ट्वा सर्वसैन्यसमन्वितम्}
{गन्तुं चकार धिषणां हयवर्यस्य पालने}% ५०

\twolineshloka
{सोऽश्वो विमोचितस्तेन भाले पत्रेण चिह्नितः}
{वामावर्तं भ्रमन्प्रायात्पौर्वाञ्जनपदान्बहून्}% ५१

\twolineshloka
{तत्रतत्रत्य भूपालैर्महाशूराभिपूजितैः}
{प्रणतिः क्रियते तस्य न कोपि तमगृह्णत}% ५२

\twolineshloka
{केचिद्वासांसि चित्राणि केचिद्राज्यं स्वकं महत्}
{केचिद्धनं जनं केचिदानीय प्रणमन्ति तम्}% ५३

{॥इति श्रीपद्मपुराणे पातालखण्डे शेषवात्स्यायनसंवादे रामाश्वमेधे शत्रुघ्नस्य सुबाहुना सह निर्याणं नाम एकोनत्रिंशत्तमोऽध्यायः॥२९॥}

\dnsub{त्रिंशोऽध्यायः}\resetShloka

\uvacha{शेष उवाच}

\twolineshloka
{अथ तेजःपुरं प्राप्तस्तुरगः पत्रशोभितः}
{यस्यां पालयते राजा प्रजाः सत्येन सत्यवान्}% १

\twolineshloka
{अथ कोटिपरीवारो रघुनाथानुजस्ततः}
{हयानुगो ययौ तस्य पुरतः पुरधर्षणः}% २

\twolineshloka
{तद्दृष्ट्वा नगरं रम्यं चित्रप्राकारशोभितम्}
{काञ्चनैः कलशैस्तत्र परितः प्रतिभासितम्}% ३

\twolineshloka
{देवायतनसाहस्रैः सर्वतश्च विराजितम्}
{यतीनां तु मठास्तत्र शोभन्ते यतिपूरिताः}% ४

\twolineshloka
{वहत्यत्र महादेवी शिखिलोचनमूर्धगा}
{हंसकारण्डवाकीर्णामुनिवृन्दनिषेविता}% ५

\twolineshloka
{ब्राह्मणानां प्रत्यगारमग्निहोत्रभवः पुनः}
{धूमस्तत्र पुनात्यङ्ग पातकाप्लुतमानसान्}% ६

\twolineshloka
{उवाच सुमतिं राजा शत्रुघ्नः शत्रुतापनः}
{तत्पुरप्रेक्षणोद्भूतहर्षविस्मितमानसः}% ७

\uvacha{शत्रुघ्न उवाच}

\twolineshloka
{मन्त्रिन्कथय कस्येदं पुरं मे दृष्टिगोचरम्}
{करोति मानसाह्लादं धर्मेण प्रतिपालितम्}% ८

\uvacha{शेष उवाच}

\twolineshloka
{इति वाक्यं समाकर्ण्य शत्रुघ्नस्य महीपतेः}
{उवाच सुमतिः सर्वं यथातथमनुद्धतम्}% ९

\uvacha{सुमतिरुवाच}

\twolineshloka
{शृणुष्वावहितः स्वामिन्वैष्णवस्य कथाः शुभाः}
{याः श्रुत्वा मुच्यते पापाद्ब्रह्महत्यासमादपि}% १०

\twolineshloka
{जीवन्मुक्तो वरीवर्ति रामाङ्घ्र्यम्बुजषट्पदः}
{सत्यवान्यज्ञयज्ञाङ्ग ज्ञाता कर्ताऽविता महान्}% ११

\twolineshloka
{धेनुं प्रसाद्य बहुभिर्व्रतैर्यं प्राप तत्पिता}
{ऋतम्भराख्यो जगति ख्यातः परमधार्मिकः}% १२

\twolineshloka
{गौः प्रसन्ना ददौ पुत्रमनेकगुणसंस्कृतम्}
{सत्यवन्तं सुशोभाढ्यं तं जानीहि नृपोत्तमम्}% १३

\uvacha{शत्रुघ्न उवाच}

\twolineshloka
{को वा ऋतम्भरो राजा किमर्थं धेनुपूजनम्}
{कथं प्राप्तः सुतस्तस्य वैष्णवो विष्णुसेवकः}% १४

\twolineshloka
{सर्वमेतत्समाचक्ष्व वैष्णवस्य कथानकम्}
{श्रुतं हरति जन्तूनां महापातकपर्वतम्}% १५

\uvacha{शेष उवाच}

\twolineshloka
{इति वाक्यं समाकर्ण्य शत्रुघ्नस्य महार्थकम्}
{कथयामास विशदं तदुत्पत्तिकथानकम्}% १६

\twolineshloka
{ऋतम्भरो नरपतिरनपत्यः पुराऽभवत्}
{कलत्राणि बहून्यस्य न पुत्रं प्राप तेषु वै}% १७

\twolineshloka
{तदा जाबालिनामानं मुनिं दैवादुपागतम्}
{प्रपच्छ कुशलोद्युक्तः सपुत्रोत्पत्तिकारणम्}% १८

\uvacha{ऋतम्भर उवाच}

\twolineshloka
{स्वामिन्वन्ध्यस्य मे ब्रूहि पुत्रोत्पत्तिकरं वचः}
{यत्कृत्वा जायतेऽपत्यं मम वंशधरं वरम्}% १९

\twolineshloka
{तज्ज्ञात्वा भवतो भव्यं प्रकुर्यां निश्चितं वचः}
{दानं व्रतं वा तीर्थं वा मखं वा मुनिसत्तम}% २०

\twolineshloka
{इति राज्ञोवचः श्रुत्वा जगाद मुनिसत्तमः}
{सुतोत्पत्तिकरं वाक्यं प्रणतस्य सुतार्थिनः}% २१

\twolineshloka
{अपत्यप्राप्तिकामस्य सन्त्युपायास्त्रयः प्रभो}
{विष्णोः प्रसादो गोश्चापि शिवस्याप्यथवा पुनः}% २२

\twolineshloka
{तस्मात्त्वं कुरु वै पूजां धेनोर्देवतनोर्नृप}
{यस्याः पुच्छे मुखे शृङ्गे पृष्ठे देवाः प्रतिष्ठिताः}% २३

\twolineshloka
{सा तुष्टा दास्यति क्षिप्रं वाञ्छितं धर्मसंयुतम्}
{एवं विदित्वा गोपूजां विधेहि त्वमृतम्भर}% २४

\twolineshloka
{यो वै नित्यं पूजयति गां गेहे यवसादिभिः}
{तस्य देवाश्च पितरो नित्यं तृप्ता भवन्ति हि}% २५

\twolineshloka
{यो वै गवाह्निकं दद्यान्नियमेन शुभव्रतः}
{तेन सत्येन तस्य स्युः सर्वे पूर्णा मनोरथाः}% २६

\twolineshloka
{तृषिता गौर्गृहे बद्धा गेहे कन्या रजस्वला}
{देवता च सनिर्माल्या हन्ति पुण्यं पुराकृतम्}% २७

\twolineshloka
{यो वै गां प्रतिषिद्ध्येत चरन्तीं स्वं तृणं नरः}
{तस्य पूर्वे च पितरः कम्पन्ते पतनोन्मुखाः}% २८

\twolineshloka
{यो वै यष्ट्या ताडयति धेनुं मर्त्यो विमूढधीः}
{धर्मराजस्य नगरं स याति करवर्जितः}% २९

\twolineshloka
{यो वै दंशान्वारयति तस्य पूर्वे ह्यधोगताः}
{नृत्यन्ति मत्सुतो ह्यस्मांस्तारयिष्यति भाग्यवान्}% ३०

\twolineshloka
{अत्रैवोदाहरन्तीममितिहासं पुरातनम्}
{जनकस्य पुरावृत्तं धर्मराजपुरेऽद्भुतम्}% ३१

\twolineshloka
{एकदा जनको राजा योगेनासून्समत्यजत्}
{तदा विमानं सम्प्राप्तं किङ्किणीजालभूषितम्}% ३२

\twolineshloka
{तदारुह्य गतो राजा सेवकैरूढदेहवान्}
{मार्गे जगाम धर्मस्य संयमिन्याः पुरोऽन्तिके}% ३३

\twolineshloka
{तदा नरककोटीषु पीड्यन्ते पापकारिणः}
{जनकस्याङ्गपवनं प्राप्य सौख्यं प्रपेदिरे}% ३४

\twolineshloka
{निरये दाहजापीडा जातैषां सुखकारिणी}
{महादुःखं तदा नष्टं जनकस्याङ्गवायुना}% ३५

\twolineshloka
{तदा तं निर्गतं दृष्ट्वा जन्तवः पापपीडिताः}
{अत्यन्तं चुक्रुशुर्भीतास्तद्वियोगमनिच्छवः}% ३६

\twolineshloka
{ऊचुस्ते करुणां वाचं मा गच्छ सुकृतिन्नितः}
{त्वदङ्गवायुसंस्पर्शात्सुखिनः स्यामपीडिताः}% ३७

\twolineshloka
{इति वाक्यं समाकर्ण्य राजा परमधार्मिकः}
{मानसे चिन्तयामास करुणापूरपूरितः}% ३८

\twolineshloka
{चेन्मत्तः प्राणिनां सौख्यं भवेदिह तदा पुनः}
{अत्रैव च पुरे स्थास्ये स्वर्ग एष मनोरमः}% ३९

\twolineshloka
{एवं कृत्वा नृपस्तस्थौ तत्रैव निरयाग्रतः}
{विदधत्प्राणिनां सौख्यमनुकम्पितमानसः}% ४०

\twolineshloka
{तत्र धर्मस्तु सम्प्राप्तो निरयद्वारि दुःखदे}
{कारयन्यातनास्तीव्रा नानापातककारिणाम्}% ४१

\twolineshloka
{तदा ददर्श राजानं जनकं द्वारिसंस्थितम्}
{विमानेन महापुण्यकारिणं दययायुतम्}% ४२

\twolineshloka
{तमुवाच प्रेतपतिर्जनकं सहसन्गिरा}
{राजन्कुतस्त्वं सम्प्राप्तः सर्वधर्मशिरोमणिः}% ४३

\twolineshloka
{एतत्स्थानं पातकिनां दुष्टानां प्राणघातिनाम्}
{नायान्ति पुरुषा भूप त्वादृशाः पुण्यकारिणः}% ४४

\twolineshloka
{अत्रायान्ति नरास्ते वै ये परद्रोहतत्पराः}
{परापवादनिरताः परद्रव्यपरायणाः}% ४५

\twolineshloka
{यो वै कलत्रं धर्मिष्ठं निजसेवापरायणम्}
{अपराधादृते जह्यात्सनरोऽत्र समाव्रजेत्}% ४६

\twolineshloka
{मित्रं वञ्चयते यस्तु धनलोभेन लोभितः}
{आगत्यात्र नरः पीडां मत्तः प्राप्नोति दारुणाम्}% ४७

\twolineshloka
{यो रामं मनसा वाचा कर्मणा दम्भतोऽपि वा}
{द्वेषाद्वाचोपहासाद्वा न स्मरत्येव मूढधीः}% ४८

\twolineshloka
{तं बध्नामि पुनस्त्वेषु निक्षिप्य श्रपयामि च}
{यैः स्मृतो न रमानाथो नरकक्लेशवारकः}% ४९

\twolineshloka
{तावत्पापं मनुष्याणामङ्गेषु नृप तिष्ठति}
{यावद्रामं न रसना गृणाति कलि दुर्मतेः}% ५०

\twolineshloka
{महापापकरा राजन्ये भवन्ति महामते}
{तानानयन्ति मद्भृत्यास्त्वादृशान्द्रष्टुमक्षमाः}% ५१

\twolineshloka
{तस्माद्गच्छ महाराज भुङ्क्ष्व भोगाननेकशः}
{विमानवरमारुह्य भुङ्क्ष्व पुण्यमुपार्जितम्}% ५२

\twolineshloka
{इति वाक्यं समाकर्ण्यध र्मराजस्य तत्पतेः}
{उवाच धर्मराजानं करुणापूरपूरितः}% ५३

\uvacha{जनक उवाच}

\twolineshloka
{अहं गच्छामि नो नाथ जीवानामनुकम्पया}
{मदङ्गवायुना ह्येते सुखं प्राप्ताः स्म संस्थिताः}% ५४

\twolineshloka
{एतान्मुञ्चसि चेद्राजन्सर्वान्वै निरयस्थितान्}
{ततो गच्छामि सुखितः स्वर्गं पुण्यजनाश्रितम्}% ५५

\uvacha{जाबालिरुवाच}

\twolineshloka
{इति वाक्यमथाश्रुत्य जनकं प्रत्युवाच सः}
{प्रत्येकं निर्दिशञ्जीवान्निरयस्थाननेकशः}% ५६

\uvacha{धर्म उवाच}

\twolineshloka
{अयं मित्र कलत्रं वै विश्वस्तमनुजग्मिवान्}
{तस्मादेनं लोहशङ्कौ वर्षायुतमपीपचम्}% ५७

\twolineshloka
{पश्चादेनं सूकराणां योनौ निक्षिप्य दोषिणम्}
{मानुषेष्ववतार्योऽयं षण्ढचिह्नेन चिह्नितः}% ५८

\twolineshloka
{अनेन परदाराश्च बलादालिगिता मुहुः}
{तस्मादयं पच्यतेऽत्र रौरवे शतहायनम्}% ५९

\twolineshloka
{अयं तु परकीयं स्वं मुषित्वा बुभुजे कुधीः}
{तस्मादस्य करौ छित्त्वा पचेयं पूयशोणिते}% ६०

\twolineshloka
{अयं सायन्तने प्राप्तमतिथिं क्षुधयार्दितम्}
{वाण्यापि नाकरोत्तस्य पूजनं स्वागतं न च}% ६१

\twolineshloka
{तस्मादयं पातनीयस्तामिस्रेन्धनपूरिते}
{भ्रमरैः पीडितो यातु यातनां शतहायनम्}% ६२

\twolineshloka
{अयं तावत्परस्योच्चैर्निन्दां कुर्वन्नलज्जितः}
{अयमप्यशृणोत्कर्णौ प्रेरयन्बहुशस्तु ताम्}% ६३

\twolineshloka
{तस्मादिमावन्धकूपे पतितौ दुःखदुःखितौ}
{अयं मित्रध्रुगुद्विग्नः पच्यते रौरवे भृशम्}% ६४

\twolineshloka
{तस्मादेतान्पापभोगान्कारयित्वा विमोचये}
{त्वं गच्छ नरशार्दूल पुण्यराशिविधायकः}% ६५

\uvacha{जाबालिरुवाच}

\twolineshloka
{एवं स निर्दिशञ्जीवांस्तूष्णीमासाघकारिणः}
{प्रोवाच रामभक्तोऽसौ करुणापूरितेक्षणः}% ६६

\uvacha{जनक उवाच}

\twolineshloka
{कथं निरयनिर्मुक्तिर्जीवानां दुःखिनां भवेत्}
{तदाशु कथय त्वं वै यत्कृत्वा सुखमाप्नुयुः}% ६७

\uvacha{धर्म उवाच}

\twolineshloka
{नैभिराराधितो विष्णुर्नैभिस्तस्य कथाः श्रुताः}
{कथं निरयनिर्मुक्तिर्भवेद्वै पापकारिणाम्}% ६८

\twolineshloka
{यदि त्वं मोचयस्येतान्महापापकरानपि}
{तर्ह्यर्पय महाराज पुण्यं तत्कथयामि यत्}% ६९

\twolineshloka
{एकदा प्रातरुत्थाय शुद्धभावेन चेतसा}
{ध्यातः श्रीरघुनाथोऽसौ महापापहराभिधः}% ७०

\twolineshloka
{रामरामेति यच्चोक्तं त्वया शुद्धेन चेतसा}
{तत्पुण्यमर्पयैतेभ्यो येन स्यान्निरयाच्च्युतिः}% ७१

\uvacha{जाबालिरुवाच}

\twolineshloka
{एतच्छ्रुत्वा वचस्तस्य धर्मराजस्य धीमतः}
{पुण्यं ददौ महाराज आजन्मसमुपार्जितम्}% ७२

\twolineshloka
{यदा जन्मकृतैः पुण्यै रघुनाथार्चनोद्भवैः}
{एतेषां निरयान्मुक्तिर्भवत्वत्र मनोरमा}% ७३

\twolineshloka
{एवं कथयतस्तस्य जीवा निरयसंस्थिताः}
{तत्क्षणान्निरयान्मुक्ता जाता दिव्यवपुर्धराः}% ७४

\twolineshloka
{ऊचुस्ते जनकं राजंस्त्वत्प्रसादाद्वयं क्षणात्}
{दुःखदान्निरयान्मुक्ता यास्यामः परमं पदम्}% ७५

\twolineshloka
{तान्दृष्ट्वा सूर्यसङ्काशान्नरान्निरयनिःसृतान्}
{तुतोष चित्ते सुभृशं सर्वभूतदयापरः}% ७६

\twolineshloka
{ते सर्वे प्रययुर्लोकं दिवं देवैरलङ्कृतम्}
{जनकं तु प्रशंसन्तो महाराजं दयानिधिम्}% ७७

{॥इति श्रीपद्मपुराणे पातालखण्डे शेषवात्स्यायनसंवादे रामाश्वमेधे सत्यवदाख्याने जनकेन नरकस्थप्राणिमोचनं नाम त्रिंशोऽध्यायः॥३०॥}

\dnsub{एकत्रिंशोऽध्यायः}\resetShloka

\uvacha{जाबालिरुवाच}

\twolineshloka
{अथ तेषु प्रयातेषु नरकस्थेषु वै नृषु}
{राजा पप्रच्छ कीनाशं सर्वधर्मविदांवरम्}% १

\uvacha{राजोवाच}

\twolineshloka
{धर्मराज त्वया प्रोक्तं यत्पातककरा नराः}
{आयान्ति तव संस्थानं न च धर्मकथारताः}% २

\twolineshloka
{मदागमनमत्राभूत्केनपापेन धार्मिक}
{तद्वै कथय सर्वं मे पापकारणमादितः}% ३

\twolineshloka
{इति श्रुत्वा तु तद्वाक्यं धर्मराजः परन्तप}
{कथयामास तस्यैवं यमपुर्यागमं तदा}% ४

\uvacha{धर्मराज उवाच}

\twolineshloka
{राजंस्तव महत्पुण्यं नैतादृक्कस्य भूतले}
{रघुनाथपदद्वन्द्वमकरन्द मधुव्रत}% ५

\twolineshloka
{त्वत्कीर्ति स्वर्धुनी सर्वान्पापिनो मलसंयुतान्}
{पुनाति परमाह्लादकारिणी दुष्टतारिणी}% ६

\twolineshloka
{तथापि पापलेशस्ते वर्तते नृपसत्तम}
{येन संयमिनीपार्श्वमागतः पुण्यपूरितः}% ७

\twolineshloka
{एकदा तु चरन्तीं गां वारयामास वै भवान्}
{तेन पापविपाकेन निरयद्वारदर्शनम्}% ८

\twolineshloka
{इदानीं पापनिर्मुक्तो बहुपुण्यसमन्वितः}
{भुङ्क्ष्व भोगान्सुविपुलान्निजपुण्यार्जितान्बहून्}% ९

\twolineshloka
{एतेषां करुणावार्धी रघुनाथो सुखं हरन्}
{संयमिन्या महामार्गे प्रेरयामास वैष्णवम्}% १०

\twolineshloka
{नागमिष्यो यदि त्वं वै मार्गेणानेन सुव्रत}
{अभविष्यत्कथं तेषां निरयात्परिमोचनम्}% ११

\twolineshloka
{त्वादृशाः परदुःखेन दुःखिताः करुणालयाः}
{प्राणिनां दुःखविच्छेदं कुर्वन्त्येव महामते}% १२

\uvacha{जाबालिरुवाच}

\twolineshloka
{एवं वदन्तं शमनं प्रणम्य स दिवङ्गतः}
{दिव्येन सुविमानेन अप्सरोगणशोभिना}% १३

\twolineshloka
{तस्माद्गावोऽनिशं पूज्या मनसापि न गर्हयेत्}
{गर्हयन्निरयं याति यावदिन्द्राश्चतुर्दश}% १४

\twolineshloka
{तस्मात्त्वं नृपतिश्रेष्ठ गोपूजां वै समाचर}
{सा तुष्टा दास्यति क्षिप्रं पुत्रं धर्मपरायणम्}% १५

\uvacha{सुमतिरुवाच}

\twolineshloka
{तच्छ्रुत्वा धेनुपूजां स पप्रच्छ कथमादरात्}
{पूजनीया प्रयत्नेन कीदृशं कुरुते नरम्}% १६

\twolineshloka
{जाबालि कथयामास धेनुपूजां यथाविधि}
{प्रत्यहं विपिनं गच्छेच्चारणार्थं व्रती तु गोः}% १७

\twolineshloka
{गवे यवांस्तु सम्भोज्य गोमयस्थान्समाहरेत्}
{भक्षणीया यवास्ते तु पुत्रकामेन भूपते}% १८

\twolineshloka
{सा यदा पिबते तोयं तदा पेयं जलं शुचि}
{सोच्चैः स्थाने यदा तिष्ठेत्तदानीं चासनस्थितः}% १९

\twolineshloka
{दंशान्निवारयेन्नित्यं यवसं स्वयमाहरेत्}
{एवं प्रकुर्वतः पुत्रं दास्यते धर्मतत्परम्}% २०

\uvacha{सुमतिरुवाच}

\twolineshloka
{इति वाक्यं समाकर्ण्य पुत्रकाम ऋतम्भरः}
{व्रतं चकार धर्मात्मा धेनुपूजां समाचरन्}% २१

\twolineshloka
{प्रत्यहं कुरुते गां वै यवसाद्येन तोषिताम्}
{दंशान्न्यवारयद्धीमान्यवभक्षकृतादरः}% २२

\twolineshloka
{एवं धेनुं पूजयतो गतास्तु दिवसा घनाः}
{वनमध्ये तृणादींश्च चरन्तीमकुतोभयाम्}% २३

\twolineshloka
{एकदा नृपतिस्तस्य वनस्य श्रीनिरीक्षणे}
{न्यस्तदृष्टिः सपरितो बभ्राम स कुतूहली}% २४

\twolineshloka
{तदागत्याहनद्गां वै पञ्चास्यः काननान्तरात्}
{क्रोशन्तीं बहुधा दीनां सिंहभारेणदुःखिताम्}% २५

\twolineshloka
{तदा नृपः समागत्य विलोक्य निजमातरम्}
{सिंहेन निहतां पश्यन्रुरोदातीव विह्वलः}% २६

\twolineshloka
{स दुःखितः समागत्य जाबालिमुनिसत्तमम्}
{निष्कृतिं तस्य पप्रच्छ गोवधस्य प्रमादतः}% २७

\uvacha{ऋतम्भर उवाच}

\twolineshloka
{स्वामिंस्त्वदाज्ञया धेनुं पालयन्वनमास्थितः}
{कुतोप्यागत्य तां सिंहो जघानादृष्टिगोचरः}% २८

\twolineshloka
{तस्य पापस्य निष्कृत्यै किं करोमि त्वदाज्ञया}
{कथं वा व्रतसम्पूर्तिर्मम पुत्रप्रदायिनी}% २९

\twolineshloka
{इत्युक्तवन्तं तं भूपं जगाद मुनिसत्तमः}
{सन्त्युपाया महीपाल पापस्यास्यापनुत्तये}% ३०

\twolineshloka
{ब्रह्मघ्नस्य कृतघ्नस्य सुरापस्य महामते}
{प्रायश्चित्तानि वर्तन्ते सर्वपापहराणि च}% ३१

\twolineshloka
{कृच्छ्रैश्चान्द्रायणैर्दानैर्व्रतैः सनियमैर्यमैः}
{पापानि प्रलयं यान्ति नियमादनुतिष्ठतः}% ३२

\twolineshloka
{द्वयोश्च निष्कृतिर्नास्ति पापपुञ्जकृतोस्तयोः}
{मत्या गोवधकर्तुश्च नारायणविनिन्दितुः}% ३३

\twolineshloka
{गवां यो मनसा दुःखं वाञ्च्छत्यधमसत्तमः}
{स याति निरयस्थानं यावदिन्द्राश्चतुर्दश}% ३४

\twolineshloka
{योऽपि देवं हरिं निन्देत्सकृद्दुर्भाग्यवान्नरः}
{स चापि नरकं पश्येत्पुत्रपौत्रपरीवृतः}% ३५

\twolineshloka
{तस्माज्ज्ञात्वा हरिं निन्दन्गोषु दुःखं समाचरन्}
{कदापि नरकान्मुक्तिं न प्राप्नोति नरेश्वर}% ३६

\twolineshloka
{अज्ञानप्राप्तगोहत्या प्रायश्चित्तं तु विद्यते}
{रामभक्तं तु धीमन्तं याहि त्वमृतुपर्णकम्}% ३७

\twolineshloka
{स वै समदृशः सर्वाञ्छत्रून्मित्राणि पश्यति}
{तुभ्यं कथिष्यति क्षिप्रं गोवधस्यास्य निष्कृतिम्}% ३८

\twolineshloka
{तस्य देशांस्त्वमाक्रामंस्तेन निर्वासितः पुरा}
{वैरिभावं परित्यज्य गच्छ त्वमृतुपर्णकम्}% ३९

\twolineshloka
{स यद्वदिष्यति क्षिप्रं तत्कुरुष्व समाहितः}
{यथा त्वत्कृतपापस्य निष्कृतिर्हि भविष्यति}% ४०

\twolineshloka
{स तु तद्वचनं श्रुत्वा जगाम ऋतुपर्णकम्}
{रामभक्तं रिपौ मित्रे समदृष्ट्या समञ्जसम्}% ४१

\twolineshloka
{स तस्मै कथयामास यज्जातं गोवधादिकम्}
{तस्य पापस्य निष्कृत्यै ह्युपायं सोऽप्यचिन्तयत्}% ४२

\twolineshloka
{क्षणं ध्यात्वाथ तं राजा ऋतुपर्ण ऋतम्भरम्}
{उवाच प्रहसन्वाक्यं बुद्धिमान्धर्मकोविदः}% ४३

\twolineshloka
{कोऽहं राजन्मुनीनां वै पुरतः शास्त्रवेदिनाम्}
{तान्हित्वा किं तु मां प्राप्तो मूर्खम्पण्डितमानिनम्}% ४४

\twolineshloka
{तव मय्यस्ति चेच्छ्रद्धा तदा किञ्चिद्ब्रवीम्यहम्}
{शृणुष्व नरशार्दूल गदितं मम सादरः}% ४५

\twolineshloka
{भज श्रीरघुनाथं त्वं कर्मणा मनसा गिरा}
{नैष्कापट्येन लोकेशं तोषयस्व महामते}% ४६

\twolineshloka
{स तुष्टो दास्यते सर्वं त्वद्धृदिस्थं मनोरथम्}
{अज्ञानकृत गोहत्यापापनाशं करिष्यति}% ४७

\twolineshloka
{रामं स्मरंस्त्वं धर्मात्मन्धेनुं पालय सत्तम}
{दत्त्वा द्विजाय कनकं पापनिष्कृतिमाप्स्यसि}% ४८

\uvacha{सुमतिरुवाच}

\twolineshloka
{एतच्छ्रुत्वा तु तद्वाक्यमृतम्भरनृपस्तदा}
{विधाय रामस्मरणं पूतात्मा व्रतमाचरत्}% ४९

\twolineshloka
{पूर्ववत्पालयन्धेनुं जगाम विपिनं महत्}
{रामनामस्मरन्नित्यं सर्वभूतहिते रतः}% ५०

\twolineshloka
{तस्मै तुष्टा तु सुरभिः प्रोवाच परितोषिता}
{राजन्वरय मत्तो वै वरं हृत्स्थं मनोरथम्}% ५१

\twolineshloka
{तदा प्रोवाच वै राजा पुत्रं देहि मनोरमम्}
{रामभक्तं पितृरतं स्वधर्मप्रतिपालकम्}% ५२

\twolineshloka
{तुष्टा दत्त्वा वरं सापि तस्मै राज्ञे सुतार्थिने}
{जगामादर्शनं देवी कामधेनुः कृपावती}% ५३

\twolineshloka
{स काले प्राप्तवान्पुत्रं वैष्णवं रामसेवकम्}
{सत्यवत्संज्ञयायुक्तमकरोत्तत्र तत्पिता}% ५४

\twolineshloka
{सत्यवन्तं सुतं लब्ध्वा पितृभक्तिपरं महान्}
{परमं हर्षमापेदे शक्रतुल्यपराक्रमम्}% ५५

\twolineshloka
{स राजा धार्मिकं पुत्रं प्राप्य हर्षेणनिर्भरः}
{राज्यं तस्मिन्महन्न्यस्य जगाम तपसे वनम्}% ५६

\twolineshloka
{तत्राराध्य हृषीकेशं भक्तियुक्तेन चेतसा}
{निर्धूतपापः सतनुरगाद्धरिपदं नृपः}% ५७

{॥इति श्रीपद्मपुराणे पातालखण्डे शेषवात्स्यायनसंवादे रामाश्वमेधे सत्यवदाख्याने धेनुव्रतवर्णनं नाम एकत्रिंशोऽध्यायः॥३१॥}

\dnsub{द्वात्रिंशोऽध्यायः}\resetShloka

\uvacha{सुमतिरुवाच}

\twolineshloka
{असावपि नृपः सौम्य सत्यवान्नाम विश्रुतः}
{निजधर्मेण लोकेशं रघुनाथमतोषयत्}% १

\twolineshloka
{अस्मै तुष्टो रमानाथो ददौ भक्तिमचञ्चलाम्}
{निजाङ्घ्रिपद्मे यजतां दुर्लभां पुण्यकोटिभिः}% २

\twolineshloka
{नित्यं श्रीरघुनाथस्य कथानकमनातुरः}
{कुरुते सर्वलोकानां पावनं कृपयायुतः}% ३

\twolineshloka
{यो न पूजयते देवं रघुनाथं रमापतिम्}
{स तेन ताड्यते दण्डैर्यमस्यापि भयावहैः}% ४

\twolineshloka
{अष्टमाद्वत्सरादूर्ध्वमशीतिवत्सरो भवेत्}
{तावदेकादशी सर्वैर्मानुषैः कारिताऽमुना}% ५

\twolineshloka
{तुलसी वल्लभा यस्य कदाचिद्यच्छिरोधराम्}
{न मुञ्चति रमानाथ पादपद्मस्रगुत्तमा}% ६

\twolineshloka
{ऋषीणामपि पूज्योयमितरेषां कथं नहि}
{रघुनाथस्मृतिप्रीतिर्धूतपाप्मा हताशुभः}% ७

\twolineshloka
{ज्ञात्वायं रामचन्द्रस्य वाजिनं परमाद्भुतम्}
{आगत्य तुभ्यं सन्दास्यत्येतद्राज्यमकण्टकम्}% ८

\twolineshloka
{त्वया यद्गदितं राजंस्तत्ते कथितमुत्तमम्}
{पुनः किं पृच्छसे स्वामिन्नाज्ञापय करोमि तत्}% ९

\uvacha{शेष उवाच}

\twolineshloka
{गतोऽश्वस्तत्पुरान्तस्तु नानाश्चर्यसमन्वितः}
{तं दृष्ट्वा जनताः सर्वा राज्ञे गत्वा न्यवेदयन्}% १०

\uvacha{जनता ऊचुः}

\twolineshloka
{कोऽप्यश्वः सितवर्णेन गङ्गाजलसमेन वै}
{भाले सौवर्णपत्रेण राजमानः समागतः}% ११

\twolineshloka
{तच्छ्रुत्वा वचनं रम्यं जनानां हृद्यमीरितम्}
{ताः प्रत्याह हसन्भूपो ज्ञायतां कस्य वै हयः}% १२

\twolineshloka
{ताश्चैनं कथयामासुः शत्रुघ्नेन प्रपालितः}
{आयात्यश्वो महीभर्तू रामस्य पुरमध्यतः}% १३

\twolineshloka
{रामस्य नाम स श्रुत्वा द्व्यक्षरं सुमनोरमम्}
{जहर्ष चित्ते सुभृशं गद्गदस्वरचिन्हितः}% १४

\twolineshloka
{मयायोध्यापतिर्नित्यं यो रामश्चिन्त्यते हृदि}
{तस्याश्वः सहशत्रुघ्नः समायातः पुरं मम}% १५

\twolineshloka
{हनूमांस्तत्र रामाङ्घ्रिसेवाकर्ता भविष्यति}
{कदाचिदपि यो रामं न विस्मरति मानसे}% १६

\twolineshloka
{गच्छामि यत्र शत्रुघ्नो यत्र मारुतनन्दनः}
{अन्येऽपि यत्र पुरुषा रामपादाब्जसेवकाः}% १७

\twolineshloka
{अमात्यमादिदेशाथ सर्वं राजधनं महत्}
{गृहीत्वा तु मया सार्द्धमागच्छ त्वरया युतः}% १८

\twolineshloka
{यास्येऽहं रघुनाथस्य हयं पालयितुं वरम्}
{कर्तुं च रामपादाब्जपरिचर्यां सुदुर्लभाम्}% १९

\twolineshloka
{इत्युक्त्वा निर्जगामाथ शत्रुघ्नं प्रति सैनिकैः}
{तावत्पुरीमथ प्राप्तो रामभ्राता ससैनिकः}% २०

\twolineshloka
{वीरा गर्जन्ति प्रबला रथाः सुनिनदन्ति च}
{जयशङ्खस्वनास्तत्र वेणुनादाश्च सर्वतः}% २१

\twolineshloka
{आगत्य सत्यवान्राजा मन्त्रिभिः सुसमन्वितः}
{चरणे प्रणिपत्यास्मै राज्यं प्रादान्महाधनम्}% २२

\twolineshloka
{शत्रुघ्नस्तं तु राजानं ज्ञात्वा राममनुव्रतम्}
{तद्राज्यं तस्य पुत्राय रुक्मनाम्ने ददौ महत्}% २३

\twolineshloka
{हनूमन्तं परीरभ्य सुबाहुं रामसेवकम्}
{अन्यान्वै रामभक्तांश्च परिरभ्य महायशाः}% २४

\twolineshloka
{कृतार्थमिव चात्मानं मेने सत्यसमन्वितः}
{ननन्द चेतसि तदा शत्रुघ्नेन समन्वितः}% २५

\twolineshloka
{हयस्तावद्गतो दूरं वीरैः सुपरिरक्षितः}
{शत्रुघ्नस्तेन भूपेन ययौ वीरसमन्वितः}% २६

{॥इति श्रीपद्मपुराणे पातालखण्डे शेषवात्स्यायनसंवादे रामाश्वमेधे सत्यवत्समागमो नाम द्वात्रिंशोऽध्यायः॥३२॥}

\dnsub{त्रयस्त्रिंशत्तमोऽध्यायः}\resetShloka

\uvacha{शेष उवाच}

\twolineshloka
{गच्छत्सु रथिवर्येषु शत्रुघ्नादिषु भूरिषु}
{महाराजेषु सर्वेषु रथकोटियुतेषु च}% १

\twolineshloka
{अकस्मादभवन्मार्गे तमः परमदारुणम्}
{यस्मिन्स्वीयो न पारक्यो लक्ष्यते ज्ञातिभिर्नरैः}% २

\twolineshloka
{रजसा व्यावृतं व्योम विद्युत्स्तनितसङ्कुलम्}
{एतादृशे तु सम्मर्दे महाभयकरे ततः}% ३

\twolineshloka
{मेघा वर्षन्ति रुधिरं पूयामेध्यादिकं बहु}
{अत्याकुला बभूवुस्ते वीराः परमवैरिणः}% ४

\twolineshloka
{आकुलीकृतलोके तु किमिदं किमिति स्थितिः}
{तमोव्याप्तानि लोकानां चक्षूंषि प्रथितौजसाम्}% ५

\twolineshloka
{जहाराश्वं रावणस्य सुहृत्पातालसंस्थितः}
{विद्युन्मालीति विख्यातो राक्षसश्रेणिसंवृतः}% ६

\twolineshloka
{कामगे सुविमाने तु सर्वायसनिषेविणि}
{आरूढोऽश्वं तु वीराणां भयं कुर्वञ्जहार ह}% ७

\twolineshloka
{मुहूर्तात्तत्तमो नष्टमाकाशं विमलं बभौ}
{वीराः शत्रुघ्नमुख्यास्ते प्रोचुः कुत्र हयोऽस्ति सः}% ८

\twolineshloka
{ते सर्वे हयराजं तु लोकयन्तः परस्परम्}
{ददृशुर्न यदा वाहं हाहाकारस्तदाभवत्}% ९

\twolineshloka
{कुत्राश्वो हयमेधस्य केन नीतः कुबुद्धिना}
{इति वाचमवोचंस्ते तावत्स दनुजेश्वरः}% १०

\twolineshloka
{ददृशे सुभटैः सर्वै रथस्थैः शौर्यशोभितैः}
{विमानवरमारूढै राक्षसाग्र्यैः समावृतः}% ११

\twolineshloka
{दुमुर्खा विकरालास्या लम्बदंष्ट्रा भयानकाः}
{राक्षसास्तत्र दृश्यन्ते सैन्यग्रासाय चोद्यताः}% १२

\twolineshloka
{तदा तं वेदयामासुः शत्रुघ्नं नृवरोत्तमम्}
{हयो नीतो न जानीमः खे विमानविलासिना}% १३

\twolineshloka
{तमसा व्याकुलान्कृत्वा वीरानस्मान्समाययौ}
{जग्राह नृपशार्दूल हयं कुरु यथोचितम्}% १४

\twolineshloka
{शत्रुघ्नस्तद्वचः श्रुत्वा महारोषसमावृतः}
{कोऽस्त्येष राक्षसो यो मे हयं जग्राह वीर्यवान्}% १५

\twolineshloka
{विमानं तत्पतत्वद्य मद्बाणव्रजनिर्हतम्}
{पतत्वद्य शिरस्तस्य क्षुरप्रैर्मम वैरिणः}% १६

\twolineshloka
{सज्जीयन्तां रथाः सर्वैर्महाशस्त्रास्त्रपूरिताः}
{यान्तु तं प्रतिसंहर्तुं योद्धारो वाजिहारिणम्}% १७

\twolineshloka
{इत्युक्त्वा रोषताम्राक्ष उवाच निजमन्त्रिणम्}
{नयानयविदं शूरं युद्धकार्यविशारदम्}% १८

\uvacha{शत्रुघ्न उवाच}

\twolineshloka
{मन्त्रिन्कथय के योज्या राक्षसस्य वधोद्यताः}
{महाशस्त्रा महाशूराः परमास्त्रविदुत्तमाः}% १९

\twolineshloka
{कथयाशु विचार्यैवं तत्करोमि भवद्वचः}
{वीरान्कथय तस्यैवं योग्यान्सर्वास्त्रकोविदान्}% २०

\twolineshloka
{एतच्छ्रुत्वा तु सचिवः प्राह वाक्यं यथोचितम्}
{वीरान्रणवरे योग्यान्दर्शयंस्तरसा नतान्}% २१

\uvacha{सुमतिरुवाच}

\twolineshloka
{जेतुं गच्छतु तद्रक्षः समरे विजयोद्यतः}
{महाशस्त्रास्त्रसंयुक्तः पुष्कलः परतापनः}% २२

\twolineshloka
{तथा लक्ष्मीनिधिर्यातु शस्त्रसङ्घसमन्वितः}
{करोतु तस्य यानस्य भङ्गं तीक्ष्णैः स्वसायकैः}% २३

\twolineshloka
{हनूमान्धृष्टकर्मात्र राक्षसैर्योधितुं क्षमः}
{करोतु मुखपुच्छाभ्यां ताडनं रक्षसां प्रभो}% २४

\twolineshloka
{वानरा अपि ये वीरा रणकर्मविशारदाः}
{गच्छन्तु तेऽखिला योद्धुं तववाक्यप्रणोदिताः}% २५

\twolineshloka
{सुमदश्च सुबाहुश्च प्रतापाग्र्यश्च सत्तमाः}
{गच्छन्तु सायकैस्तीक्ष्णैस्तान्योद्धुं राक्षसाधमान्}% २६

\twolineshloka
{भवानपि महाशस्त्रपरिवारो रथे स्थितः}
{करोतु युद्धे विजयं राक्षसं हन्तुमुद्यतः}% २७

\twolineshloka
{एतन्मम मतं राजन्ये योधास्तत्प्रमर्दनाः}
{ते गच्छन्तु रणे शूराः किमन्यैर्बहुभिर्भटैः}% २८

\twolineshloka
{इत्युक्तवति वीराग्र्येऽमात्ये सुमतिसंज्ञिके}
{शत्रुघ्नः कथयामास वीरान्सङ्ग्रामकोविदान्}% २९

\twolineshloka
{भो वीराः पुष्कलाद्या ये सर्वशस्त्रास्त्रकोविदाः}
{ते वदन्तु प्रतिज्ञां वै मत्पुरो राक्षसार्दने}% ३०

\twolineshloka
{कृत्वा प्रतिज्ञां विपुलां स्वपराक्रमशोभिनीम्}
{गच्छन्तु रणमध्ये हि भवन्तो बलसंयुताः}% ३१

\twolineshloka
{इति वाक्यं समाकर्ण्य शत्रुघ्नस्य महाबलाः}
{स्वां स्वां प्रतिज्ञां महतीं चक्रुस्ते तेजसान्विताः}% ३२

\twolineshloka
{तत्रादौ पुष्कलो वीरः श्रुत्वा वाक्यं महीपतेः}
{परमोत्साहसम्पन्नः प्रतिज्ञामूचिवानिमाम्}% ३३

\uvacha{पुष्कल उवाच}

\twolineshloka
{शृणुष्व नृपशार्दूल मत्प्रतिज्ञां पराक्रमात्}
{विहितां सर्वलोकानां शृण्वतां परमाद्भुताम्}% ३४

\twolineshloka
{चेन्न कुर्यां क्षुरप्राग्रैस्तीक्ष्णैः कोदण्डनिर्गतैः}
{दैत्यं मूर्च्छासमाक्रान्तं कीर्णकेशाकुलाननम्}% ३५

\twolineshloka
{कन्या स्वभोक्तुर्यत्पापं यत्पापं देवनिन्दने}
{तत्पापं मम वै भूयाच्चेत्कुर्यां स्ववचोऽनृतम्}% ३६

\twolineshloka
{यदिमद्बाणनिर्भिन्नाः सैनिकाः सुमहाबलाः}
{न पतन्ति महाराज प्रतिज्ञां तत्र मे शृणु}% ३७

\twolineshloka
{विष्ण्वीशयोर्विभेदं यः शिवशक्त्योः करोत्यपि}
{तत्पापं मम वै भूयाच्चेन्न कुर्यामृतं वचः}% ३८

\twolineshloka
{सर्वं मद्वाक्यमित्युक्तं रघुनाथपदाम्बुजे}
{भक्तिर्मे निश्चला यास्ति सैव सत्यं करिष्यति}% ३९

\twolineshloka
{पुष्कलस्य प्रतिज्ञां तां श्रुत्वा लक्ष्मीनिधिर्नृपः}
{प्रतिज्ञां व्यदधात्सत्यां स्वपराक्रमशोभिताम्}% ४०

\uvacha{लक्ष्मीनिधिरुवाच}

\twolineshloka
{वेदानां निन्दनं श्रुत्वा आस्ते यो मौनिवन्नरः}
{मानसे रोचयेद्यस्तु सर्वधर्मबहिष्कृतः}% ४१

\twolineshloka
{ब्राह्मणो यो दुराचारो रसलाक्षादिविक्रयी}
{विक्रीणाति च गां मूढो धनलोभेन मोहितः}% ४२

\twolineshloka
{म्लेच्छकूपोदकं पीत्वा प्रायश्चित्तं तु नाचरेत्}
{तत्पापं मम वै भूयाद्विमुखश्चेद्भवाम्यहम्}% ४३

\twolineshloka
{तत्प्रतिज्ञामथाश्रुत्य हनूमान्रणकोविदः}
{रामाङ्घ्रिस्मरणं कृत्वा प्रोवाच वचनं शुभम्}% ४४

\twolineshloka
{मत्स्वामीहृदये नित्यं ध्येयो वै योगिभिर्मुहुः}
{यं देवाः सासुराः सर्वे नमन्ति मणिमौलिभिः}% ४५

\twolineshloka
{रामः श्रीमानयोध्यायाः पतिर्लोकेशपूजितः}
{तं स्मृत्वा यद्ब्रुवे वाक्यं तद्वै सत्यं भवष्यिति}% ४६

\twolineshloka
{राजन्कोयं लघुर्दैत्यो दुर्बलः कामगे स्थितः}
{कथयाशु मया कार्यमेकेन विनिपातनम्}% ४७

\twolineshloka
{मेरुं देवेन्द्रसहितं लाङ्गूलाग्रेण तोलये}
{जलधिं शोषये सर्वं सांवर्तं वा पिबाम्यहम्}% ४८

\twolineshloka
{राज्ञः श्रीरघुनाथस्य जानक्याः कृपया मम}
{तन्नास्ति भूतले राजन्यदसाध्यं कदा भवेत्}% ४९

\twolineshloka
{एतद्वाक्यं मया प्रोक्तमनृतं स्याद्यदि प्रभो}
{तदैव रघुनाथस्य भक्तिदूरो भवाम्यहम्}% ५०

\twolineshloka
{यः शूद्रः कपिलां गां वै पयोबुद्ध्यानुपालयेत्}
{तस्य पापं ममैवास्तु चेत्कुर्यामनृतं वचः}% ५१

\twolineshloka
{ब्राह्मणीं गच्छते मोहाच्छूद्रः कामविमोहितः}
{तस्य पापं ममैवास्तु चेत्कुर्यामनृतं वचः}% ५२

\twolineshloka
{यद्घ्राणान्नरकं गच्छेत्स्पर्शनाच्चापि रौरवम्}
{तां पिबेन्मदिरां यो वा जिह्वास्वादेन लोलुपः}% ५३

\twolineshloka
{तस्य यज्जायते पापं तन्ममैवास्तु निश्चितम्}
{चेन्न कुर्यां प्रतिज्ञातं सत्यं रामकृपाबलात्}% ५४

\twolineshloka
{एवमुक्ते महावीरैर्योद्धारस्तरसा युताः}
{चक्रुः प्रतिज्ञां महतीं स्वपराक्रमशालिनीम्}% ५५

\twolineshloka
{शत्रुघ्नोऽपि व्यधात्तत्र प्रतिज्ञां पश्यतां नृणाम्}
{साधुसाधु प्रशंसन्वै तान्वीरान्युद्धकोविदान्}% ५६

\twolineshloka
{कथयामि पुरो वः स्वां प्रतिज्ञां सत्त्वशोभिताम्}
{तच्छृण्वन्तु महाभागा युद्धोत्साहसमन्विताः}% ५७

\twolineshloka
{चेत्तस्य शिर आहत्य पातयामि न सायकैः}
{विमानाच्च कबन्धाच्च भिन्नं छिन्नं च भूतले}% ५८

\twolineshloka
{यत्पापं कूटसाक्ष्येण यत्पापं स्वर्णचौर्यतः}
{यत्पापं ब्रह्मनिन्दायां तन्ममास्त्वद्य निश्चयात्}% ५९

\twolineshloka
{इति शत्रुघ्नसद्वाक्यं श्रुत्वा ते वीरपूजिताः}
{धन्योसि राघवभ्रातः कस्त्वदन्यो परो भवेत्}% ६०

\twolineshloka
{त्वया वै निहतो दैत्यो देवदानवदुःखदः}
{लवणो नाम लोकेश मधुपुत्रो महाबलः}% ६१

\twolineshloka
{कोयं वै राक्षसो दुष्टः क्व चास्य बलमल्पकम्}
{करिष्यसि क्षणादेव तस्य नाशं महामते}% ६२

\twolineshloka
{इत्युक्त्वा ते महावीराः सज्जीभूता रणाङ्गणे}
{प्रतिज्ञां स्वामृतां कर्तुं ययुस्ते राक्षसं मुदा}% ६३

{॥इति श्रीपद्मपुराणे पातालखण्डे शेषवात्स्यायनसंवादे रामाश्वमेधे वीरप्रतिज्ञाकथनं नाम त्रयस्त्रिंशत्तमोऽध्यायः॥३३॥}

\dnsub{चतुस्त्रिंशत्तमोऽध्यायः}\resetShloka

\uvacha{शेष उवाच}

\twolineshloka
{रथैः सदश्वैः शोभाढ्यैः सर्वशस्त्रास्त्रपूरितैः}
{नानारत्नसमायुक्तैर्ययुस्ते राक्षसाधमम्}% १

\twolineshloka
{तान्दृष्ट्वा कामगे याने स्थितः प्रोवाच राक्षसः}
{मेघगम्भीरया वाचा तर्जयन्निव भूरिशः}% २

\twolineshloka
{मायां तु सुभटा योद्धुं गच्छन्तु निजमन्दिरम्}
{मा त्यजन्तु स्वकान्प्राणान्न मोक्ष्ये वाजिनं वरम्}% ३

\twolineshloka
{विद्युन्मालीति विख्यातो रावणस्य सुहृत्सखा}
{मत्सख्युः प्रेतभूतस्य निष्कृतिं कर्तुमेयिवान्}% ४

\twolineshloka
{क्वासौ रामो य आहत्य सखायं रावणं गतः}
{तस्य भ्रातापि कुत्रास्ते सर्वशूरशिरोमणिः}% ५

\twolineshloka
{तं हत्वा निष्कृतिं तस्य प्राप्स्ये रामस्य चानुजम्}
{पिबन्रुधिरमुद्भूतं कण्ठनालस्य बुद्बुदैः}% ६

\twolineshloka
{इति वाक्यं समाकर्ण्य योधानां प्रवरोत्तमः}
{पुष्कलो निजगादैनं वीर्यशौर्यसमन्वितम्}% ७

\uvacha{पुष्कल उवाच}

\twolineshloka
{विकत्थनं न कुर्वन्ति सङ्ग्रामे सुभटा नराः}
{पराक्रमं दर्शयन्ति निजशस्त्रास्त्रवर्षणैः}% ८

\twolineshloka
{रावणो निहतो येन ससुहृत्स्वजनैर्वृतः}
{तस्य वाजिनमाहृत्य कुत्र गन्तासि दुर्मद}% ९

\twolineshloka
{पतिष्यसि त्वं शत्रुघ्नबाणैः कोदण्डनिर्गतैः}
{त्वामत्स्यन्ति शिवा भूमौ पतितं प्राणवर्जितम्}% १०

\twolineshloka
{मा गर्ज दुष्ट रामस्य सेवके मयि संस्थिते}
{गर्जन्ति सुभटा युद्धे शत्रुं जित्वा महोदयात्}% ११

\uvacha{शेष उवाच}

\twolineshloka
{एवं ब्रुवन्तं तं वीरं पुष्कलं रणदुर्मदम्}
{जघान शक्त्या सुभृशं हृदि राक्षससत्तमः}% १२

\twolineshloka
{आयान्तीं तां महाशक्तिमायसीं काञ्चनाश्रिताम्}
{चिच्छेद त्रिभिरत्युग्रैः शितैर्बाणैः स पुष्कलः}% १३

\twolineshloka
{सा त्रिधा ह्यपतद्भूमौ विशिखैर्निष्प्रभीकृता}
{पतन्ती विरराजासौ विष्णोः शक्तित्रयीव किम्}% १४

\twolineshloka
{तां छिन्नां शक्तिकां दृष्ट्वा राक्षसः परतापनः}
{जग्राह शूलं तरसा त्रिशिखं लोहनिर्मितम्}% १५

\twolineshloka
{तीक्ष्णाग्रं ज्वलनप्रख्यं राक्षसेन्द्रो व्यमोचयत्}
{आयान्तं तिलशश्चक्रे बाणैः पुष्कलसंज्ञितः}% १६

\twolineshloka
{छित्त्वा त्रिशूलं तरसा राघवस्य हि सेवकः}
{पुष्कलश्चाप आधत्त बाणांस्तीक्ष्णान्मनोजवान्}% १७

\twolineshloka
{ते बाणा हृदि तस्याशु लग्ना रागं बतासृजन्}
{वैष्णवस्य यथा स्वान्ते गुणा विष्णोर्मनोहराः}% १८

\twolineshloka
{तद्बाणवेधदुःखार्तो विद्युन्माली सुदुर्मदः}
{जग्राह मुद्गरं घोरं पुष्कलं हन्तुमुद्यतः}% १९

\twolineshloka
{मुद्गरः प्रहितस्तेन विद्युन्माल्यभिधेन हि}
{हृदि लग्नोसृजच्छीघ्रं कश्मलं तदकारयत्}% २०

\twolineshloka
{मुद्गरप्रहतो वीरः कम्पमानः सवेपथुः}
{पपात स्यन्दनोपस्थे पुष्कलः शत्रुतापनः}% २१

\twolineshloka
{उग्रदंष्ट्रोऽथ तद्भ्राता लक्ष्मीनिधिमयोधयत्}
{शस्त्रास्त्रैर्बहुधा मुक्तैर्वीरप्राणहृतिङ्करैः}% २२

\twolineshloka
{पुष्कलस्तत्क्षणात्प्राप्य संज्ञां राक्षसमब्रवीत्}
{धन्योसि राक्षसश्रेष्ठ महीयांस्ते पराक्रमः}% २३

\twolineshloka
{पश्येदानीं ममाप्युच्चैः प्रतिज्ञां शूरमानिताम्}
{विमानात्पातयाम्यद्य भूमौ त्वां शितसायकैः}% २४

\twolineshloka
{इत्युक्त्वा निशितं बाणं समगृह्णाद्दुरासदम्}
{ज्वलन्तमग्नितेजस्कं महौदार्यसमन्वितम्}% २५

\twolineshloka
{स यावत्तत्प्रतीकर्तुं विधत्ते स्वपराक्रमम्}
{तावद्धृदिगतोऽत्युग्रस्तीक्ष्णधारः ससायकः}% २६

\twolineshloka
{तेन बाणेन विभ्रान्तो भ्रमच्चित्तः स राक्षसः}
{पपात कामगोपस्थाद्भूमौ विगतचेतनः}% २७

\twolineshloka
{उग्रदंष्ट्रेण वै दृष्टः पतमानो निजाग्रजः}
{गृहीत्वा तं विमानान्तर्निनाय रिपुशङ्कितः}% २८

\twolineshloka
{प्राह चारिं महारोषात्पुष्कलं बलिनां वरम्}
{मद्भ्रातरं पातयित्वा कुत्र यास्यसि दुर्मते}% २९

\twolineshloka
{मां वै युधि विनिर्जित्य गन्तासि जयमुत्तमम्}
{स्थिते मयि तव स्वान्ते जयाशा विनिवर्त्य ताम्}% ३०

\twolineshloka
{एवं ब्रुवन्तं तरसा जघान दशभिः शरैः}
{हृदये तस्य दुष्टस्य रोषपूरितलोचनः}% ३१

\twolineshloka
{स ताडितो दशशरैः पुष्कलेन महात्मना}
{चुक्रोध हृदि दुर्बुद्धिस्तं हन्तुमुपचक्रमे}% ३२

\twolineshloka
{दन्तान्निष्पिष्य सक्रोधो मुष्टिमुद्यम्य चाहनत्}
{व्यनदद्वज्रनिर्घातपातशङ्कां सृजन्हृदि}% ३३

\twolineshloka
{मुष्टिनाभिहतो वीरः पुष्कलः परमास्त्रवित्}
{नाकम्पत विनिष्पेषं वाञ्छंस्तस्य दुरात्मनः}% ३४

\twolineshloka
{वत्सदन्तान्महातीक्ष्णान्मुमोच हृदि सायकान्}
{तैर्बाणैर्व्यथितो दैत्यस्त्रिशूलं तु समाददे}% ३५

\twolineshloka
{जाज्वल्यमानं त्रिशिखं ज्वालामालातिभीषणम्}
{लग्नं हृदि महावीर पुष्कलस्य तु दारुणम्}% ३६

\twolineshloka
{मूर्च्छितस्तेन शूलेन निहतो धन्विसत्तमः}
{कश्मलं परमं प्राप्तः पपात स्यन्दनोपरि}% ३७

\twolineshloka
{मूर्च्छां प्राप्तं तमाज्ञाय हनूमान्पवनात्मजः}
{कोपव्याकुलितस्वान्तो बभाषे तं तु राक्षसम्}% ३८

\twolineshloka
{कुत्र गच्छसि दुर्बुद्धे मयि योद्धरि संस्थिते}
{त्वां हन्मि चरणाघातैर्वाजिहर्तारमागतम्}% ३९

\twolineshloka
{एवमुक्त्वा महादैत्याञ्जघान परसैनिकान्}
{विमानस्थान्नखाग्रेण दारयन्नभसि स्थितः}% ४०

\twolineshloka
{लाङ्गूलेनाहताः केचित्केचित्पादतला हताः}
{बाहुभ्यां दारिताः केचित्पवनस्य तनूभुवा}% ४१

\twolineshloka
{नश्यन्ति केचिन्निहताः केचिन्मूर्च्छन्ति संहताः}
{पलायन्ते पदाघातभयपीडाहतास्ततः}% ४२

\twolineshloka
{अनेके निहतास्तत्र राक्षसाश्चातिदारुणाः}
{छिन्ना भिन्ना द्विधा जाताः पवनस्य सुतेन वै}% ४३

\twolineshloka
{कामगन्तुविमानं तद्भिन्नप्राकारतोरणम्}
{हाहा कुर्वद्भिरसुरैः समन्तात्परिवारितम्}% ४४

\twolineshloka
{हनूमति महाशूरे क्षणं भूमौ क्षणं दिवि}
{इतस्ततः प्रदृश्येत कामयानं दुरासदम्}% ४५

\twolineshloka
{यत्रयत्र विमानं तत्तत्रतत्र समीरजः}
{प्रहरन्नेव दृश्येत कामरूपधरः कपिः}% ४६

\twolineshloka
{एवं तदाकुलीभूते विमानस्थे महाजने}
{उग्रदंष्ट्रस्तु दैत्येन्द्रो हनूमन्तमुपेयिवान्}% ४७

\twolineshloka
{कपे त्वया महत्कर्म कृतं यद्भटपातनम्}
{क्षणं तिष्ठसि चेत्कुर्वे तव प्राणवियोजनम्}% ४८

\twolineshloka
{एवमुक्त्वा हनूमन्तं प्रजघान स दुर्मतिः}
{त्रिशूलेन सुतीक्ष्णेन ज्वलत्पावककान्तिना}% ४९

\twolineshloka
{तदागतं त्रिशूलं च मुखे जग्राह वीर्यवान्}
{चूर्णयामास सकलं सर्वलोहविनिर्मितम्}% ५०

\twolineshloka
{चूर्णयित्वा त्रिशूलं तदायसं दैत्यमोचितम्}
{जघान तं चपेटाभिर्बह्वीभिर्हनुमान्बली}% ५१

\twolineshloka
{स आहतः कपीन्द्रेण चपेटाभिरितस्ततः}
{व्यथितो व्यसृजन्मायां सर्वलोकभयङ्करीम्}% ५२

\twolineshloka
{तदा तमोभवत्तीव्रं यत्र को वा न लक्ष्यते}
{यत्र स्वीयो न पारक्यो विदामास जनान्बहून्}% ५३

\twolineshloka
{शिलाः पर्वतशृङ्गाभाः पतन्ति सुभटोपरि}
{ताभिर्हतास्तु ते सर्वे व्याकुला अथ जज्ञिरे}% ५४

\twolineshloka
{विद्युतो विलसन्त्यत्र गर्जन्ति जलदा घनम्}
{वर्षन्ति पूयरुधिरं मुञ्चन्ति समलं जलम्}% ५५

\twolineshloka
{आकाशात्पतमानानि कबन्धानि बहूनि च}
{दृश्यन्ते छिन्नशीर्षाणि सकुण्डलयुतानि च}% ५६

\twolineshloka
{नग्ना विरूपाः सुभृशं कीर्णकेशाः सुदुर्मुखाः}
{दृश्यन्ते सर्वतो दैत्या दारुणा भयकारिणः}% ५७

\twolineshloka
{तदा व्याकुलिता लोकाः परस्परभयाकुलाः}
{पलायनपरा जाता महोत्पातममंसत}% ५८

\twolineshloka
{तदा शत्रुघ्न आयातो रथे स्थित्वा महायशाः}
{श्रीरामस्मरणं कृत्वा चापे सन्धाय सायकान्}% ५९

\twolineshloka
{तां मायां स विधूयाथ मोहनास्त्रेण वीर्यवान्}
{शरधाराः किरन्व्योम्नि ववर्ष समरेसुरम्}% ६०

\twolineshloka
{तदादिशः प्रसेदुस्ता रविस्त्वपरिवेषवान्}
{मेघा यथागतं याता विद्युतः शान्तिमागताः}% ६१

\twolineshloka
{तदा विमानं पुरतो दृश्यते राक्षसैर्युतम्}
{छिन्धि भिन्धीति भाषाभिर्व्याकुलं सुतरां महत्}% ६२

\twolineshloka
{बाणाश्च शतसाहस्राः स्वर्णपुङ्खैश्च शोभिताः}
{पेतुर्विमाने नभसि स्थिते कामगमे मुहुः}% ६३

\twolineshloka
{तदा भग्नं विमानं हि दृश्यते न तदुच्चकैः}
{स्वपुरी खण्डमेकत्र भग्नाङ्गमिव भूतले}% ६४

\twolineshloka
{तदा प्रकुपितो दैत्यो बाणान्धनुषि सन्दधे}
{तैर्बाणैर्विकिरन्रामभ्रातरं चाभिगर्जितः}% ६५

\twolineshloka
{ते बाणाः शतशस्तस्य लग्ना वपुषि भूरिशः}
{शोभामापुः शोणितौघान्वहन्तस्तीक्ष्णवक्त्रिणः}% ६६

\twolineshloka
{शत्रुघ्नः परया शक्त्या संयुक्तो वायुदैवतम्}
{अस्त्रं धनुषि चाधत्त राक्षसानां प्रकम्पनम्}% ६७

\twolineshloka
{तेनास्त्रेण विमानात्खात्पतन्तो मुक्तमूर्धजाः}
{दृश्यन्ते भूतवेतालसङ्घा इव नभश्चराः}% ६८

\twolineshloka
{तदस्त्रं रघुनाथस्य भ्रात्रा मुक्तं विलोक्य सः}
{अस्त्रं च पाशुपत्यं स चापे धाद्दनुजात्मजः}% ६९

\twolineshloka
{ततः प्रवृत्ता वेताला भूताः प्रेतनिशाचराः}
{कपालकर्तरीयुक्ताः पिबन्तः शोणितं बहु}% ७०

\twolineshloka
{ते वै शत्रुघ्नवीराणां रुधिराणि पपुर्मुदा}
{जीवतामपि दुर्वाराः कर्तरीपाणिशोभिताः}% ७१

\twolineshloka
{तदस्त्रं व्याप्नुवद्दृष्ट्वा सर्ववीरप्रभञ्जनम्}
{मुमोच तन्निरासाय चास्त्रं नारायणाभिधम्}% ७२

\twolineshloka
{नारायणास्त्रं तान्सर्वान्वारयामास तत्क्षणात्}
{ते सर्वे विलयं प्रापुर्निशाचरप्रणोदिताः}% ७३

\twolineshloka
{तदा क्रुद्धो निशाचारी विद्युन्माली समाददे}
{त्रिशूलं निशितं घोरं शत्रुघ्नं हन्तुमुल्बणम्}% ७४

\twolineshloka
{शूलहस्तं समायान्तं विद्युन्मालिनमाहवे}
{सायकैः प्राहरत्तस्य भुजे त्वर्धशशिप्रभैः}% ७५

\twolineshloka
{तैर्बाणैश्छिन्नहस्तः स शिरसा हन्तुमुद्यतः}
{हतोसि याहि शत्रुघ्न कस्त्वां त्राता भविष्यति}% ७६

\twolineshloka
{इति ब्रुवाणं तरसा चिच्छेद शितसायकैः}
{मस्तकं तस्य बलिनः शूरस्य सह कुण्डलम्}% ७७

\twolineshloka
{तं छिन्नशिरसं दृष्ट्वा उग्रदंष्ट्रः प्रतापवान्}
{मुष्टिना हन्तुमारेभे शत्रुघ्नं शूरसेवितम्}% ७८

\twolineshloka
{शत्रुघ्नस्तु क्षुरप्रेण सायकेनाच्छिनच्छिरः}
{प्रधावतो रणे वीरान्सर्वशस्त्रास्त्रकोविदान्}% ७९

\twolineshloka
{हतशेषा ययुः सर्वे राक्षसा नाथवर्जिताः}
{शत्रुघ्नं प्रणिपत्याथ ददुर्वाजिनमाहृतम्}% ८०

\twolineshloka
{ततो वीणानिनादाश्च शङ्खनादाः समन्ततः}
{श्रूयन्ते शूरवीराणां जयनादा मनोहराः}% ८१

{॥इति श्रीपद्मपुराणे पातालखण्डे शेषवात्स्यायनसंवादे रामाश्वमेधे विद्युन्मालिनामराक्षसपराजयो नाम चतुस्त्रिंशत्तमोऽध्यायः॥३४॥}

\dnsub{पञ्चत्रिंशत्तमोऽध्यायः}\resetShloka

\uvacha{शेष उवाच}

\twolineshloka
{प्राप्य तं वाजिनं राजा शत्रुघ्नो राक्षसैर्हृतम्}
{अत्यन्तं हर्षमापेदे पुष्कलेन समन्वितः}% १

\twolineshloka
{रुधिरैः सिक्तगात्रास्ते योधा लक्ष्मीनिधिस्तथा}
{रणोत्साहेन संयुक्तं प्रशशंसुर्महानृपम्}% २

\twolineshloka
{हते तस्मिन्महादैत्ये विद्युन्मालिनि दुर्जये}
{सुराः सर्वे भयं त्यक्त्वा सुखमापुर्मुनेमहत्}% ३

\twolineshloka
{नद्यस्तु विमला जाता रविस्तु विमलोऽभवत्}
{वाता ववुः सुगन्धोद सिक्ता विमलशुष्मिणः}% ४

\twolineshloka
{सन्नद्धास्ते महावीरा रथस्था विमलाङ्गकाः}
{राजानमूचुस्ते सर्वे जयलक्ष्म्या समन्विताः}% ५

\uvacha{वीरा ऊचुः}

\twolineshloka
{दिष्ट्या हतस्त्वया दैत्यो विद्युन्माली महामते}
{यद्भयात्त्रासमापन्नाः सुराः स्वर्गान्निराकृताः}% ६

\twolineshloka
{दिष्ट्या प्राप्तो महावाजी रघुनाथस्य शोभनः}
{दिष्ट्या गन्तासि सर्वत्र जयं तु क्षितिमण्डले}% ७

\twolineshloka
{स्वामिन्मुञ्चत्विमं वाहं मनोवेगं मनोरमम्}
{समयस्य विलम्बो मा भवत्वत्र महामते}% ८

\uvacha{शेष उवाच}

\twolineshloka
{इति श्रुत्वा तु तद्वाक्यं वीराणां समयोचितम्}
{साधु साधु प्रशस्यैतान्मुमोच हयसत्तमम्}% ९

\twolineshloka
{स मुक्तश्चोत्तरामाशां बभ्रामाथ सुरक्षितः}
{रथपत्तिहयश्रेष्ठैः सर्वशस्त्रास्त्रकोविदैः}% १०

\twolineshloka
{तत्र यद्वृत्तमेतस्य शत्रुघ्नस्य मनोहरम्}
{वात्स्यायन शृणुष्वैतत्पापराशिप्रदाहकम्}% ११

\twolineshloka
{रेवातीरमथ प्राप्तो मुनिवृन्दनिषेवितम्}
{नीलरत्नसमूहस्य रसः किं तु पयो मिषात्}% १२

\twolineshloka
{तांस्तान्मुनिवरान्सर्वान्प्रणमञ्छूरसेवितः}
{जगाम हयरत्नस्य पृष्ठतः कामगामिनः}% १३

\twolineshloka
{गच्छंस्तत्राश्रमं जीर्णं पलाशपर्णनिर्मितम्}
{रेवायाजलकल्लोलैः सिक्तं पापहराश्रयम्}% १४

\twolineshloka
{तं दृष्ट्वा सुमतिं प्राह सर्वज्ञं नयकोविदम्}
{शत्रुघ्नः सर्वधर्मार्थकर्मकर्तव्यकोविदः}% १५

\uvacha{राजोवाच}

\twolineshloka
{मन्त्रिन्कथय कस्यायमाश्रमः पुण्यदर्शनः}
{विचारचतुरश्रेष्ठ वदैतन्मम पृच्छतः}% १६

\uvacha{शेष उवाच}

\twolineshloka
{इति वाक्यं समाकर्ण्य सुमतिः प्राह तं नृपम्}
{विशद स्मेरया वाचा दर्शयन्नात्मसौहृदम्}% १७

\uvacha{सुमतिरुवाच}

\twolineshloka
{एनं दृष्ट्वा महाराज धूतपापा वयं खलु}
{भविष्यामो मुनिश्रेष्ठं सर्वशास्त्रपरायणम्}% १८

\twolineshloka
{तस्मान्नत्वा तमापृच्छ सर्वं ते कथयिष्यति}
{रघुनाथपदाम्भोजमकरन्दाति लोलुपः}% १९

\twolineshloka
{नाम्ना त्वारण्यकं ख्यातं रघुनाथाङ्घ्रिसेवकम्}
{अत्युग्रतपसा पूर्णं सर्वशास्त्रार्थकोविदम्}% २०

\twolineshloka
{इति श्रुत्वा तु तद्वाक्यं धर्मार्थपरिबृंहितम्}
{जगाम तमथो द्रष्टुं स्वल्पसेवकसंयुतः}% २१

\twolineshloka
{हनूमान्पुष्कलो वीरः सुमतिर्मन्त्रिसत्तमः}
{लक्ष्मीनिधिः प्रतापाग्र्यः सुबाहुः सुमदस्तथा}% २२

\twolineshloka
{एतैः परिवृतो राजा शत्रुघ्नः प्रापदाश्रमम्}
{नमस्कर्तुं द्विजवरमारण्यकमुदारधीः}% २३

\twolineshloka
{गत्वा तं तापसश्रेष्ठं नमस्कारमथाकरोत्}
{सर्वैस्तैः सहितो वीरैर्विनयानतकन्धरैः}% २४

\twolineshloka
{तान्दृष्ट्वा सन्नतान्सर्वाञ्छत्रुघ्नप्रमुखान्नृपान्}
{अर्घ्यपाद्यादिकं चक्रे फलमूलादिभिस्तदा}% २५

\twolineshloka
{उवाच तान्नृपान्सर्वान्भवन्तः कुत्र सङ्गताः}
{कथमत्र समायातास्तत्सर्वं वदतानघाः}% २६

\twolineshloka
{तच्छ्रुत्वा वाक्यमेतस्य मुनिवर्यस्य वाडव}
{सुमतिः कथयामास वाक्यं वादविचक्षणः}% २७

\uvacha{सुमतिरुवाच}

\twolineshloka
{रघुवंशनृपस्यायमश्वो वै पाल्यतेऽखिलैः}
{यागं करिष्यते वीरः सर्वसम्भारसम्भृतम्}% २८

\twolineshloka
{तच्छ्रुत्वा वचनं तेषां जगाद मुनिसत्तमः}
{दन्तकान्त्याखिलं घोरं तमोनिर्वारयन्निव}% २९

\uvacha{आरण्यक उवाच}

\twolineshloka
{किं यागैर्विविधैरन्यैः सर्वसम्भारसम्भृतैः}
{स्वल्पपुण्यप्रदैर्नूनं क्षयिष्णुपददातृभिः}% ३०

\twolineshloka
{मूढो लोको हरिं त्यक्त्वा करोत्यन्यसमर्चनम्}
{रघुवीरं रमानाथं स्थिरैश्वर्यपदप्रदम्}% ३१

\twolineshloka
{यो नरैः स्मृतमात्रोपि हरते पापपर्वतम्}
{तं मुक्त्वा क्लिश्यते मूढो यागयोगव्रतादिभिः}% ३२

\twolineshloka
{अहो पश्यत मूढत्वं लोकानामतिवञ्चितम्}
{सुलभं रामभजनं मुक्त्वा दुर्ल्लभमाचरेत्}% ३३

\twolineshloka
{सकामैर्योगिभिर्वापि चिन्त्यते कामवर्जितैः}
{अपवर्गप्रदं नॄणां स्मृतमात्राखिलाघहम्}% ३४

\twolineshloka
{पुराहं तत्त्ववित्सायां ज्ञानिनं सुविचारयन्}
{अगमं बहुतीर्थानि तत्त्वं कोपि न मेऽदिशत्}% ३५

\twolineshloka
{तदैकं हि महद्भाग्यात्प्राप्तं वै लोमशं मुनिम्}
{स्वर्गलोकात्समायातं तीर्थयात्राचिकीर्षया}% ३६

\twolineshloka
{तमहं प्रणिपत्याथ पर्यपृच्छं महामुनिम्}
{महायुषं महायोगिसंसेवितपदद्वयम्}% ३७

\twolineshloka
{स्वामिन्मयाद्य मानुष्यं प्राप्तमद्भुतदुर्ल्लभम्}
{संसारघोरजलधिं किं कर्तव्यं तितीर्षुणा}% ३८

\twolineshloka
{विचार्य कथय त्वं तद्व्रतं दानं जपो मखः}
{देवो वा विद्यते यो वै संसृत्यम्भोधितारकः}% ३९

\twolineshloka
{यज्ज्ञात्वा संसृतिं घोरां तरामि त्वत्कृपाब्धितः}
{तन्मे कथय योगेश सर्वशास्त्रार्थपारग}% ४०

\twolineshloka
{इति मद्वाक्यमाकर्ण्य जगाद मुनिसत्तमः}
{शृणुष्वैकमना विप्र श्रद्धया परया युतः}% ४१

\twolineshloka
{सन्ति दानानि तीर्थानि व्रतानि नियमा यमाः}
{योगा यज्ञास्तथानेके वर्तन्ते स्वर्गदायकाः}% ४२

\twolineshloka
{परं गुह्यं प्रवक्ष्यामि सर्वपापप्रणाशनम्}
{तच्छृणुष्व महाभाग संसाराम्भोधितारकम्}% ४३

\twolineshloka
{नास्तिकाय न वक्तव्यं न चाऽश्रद्धालवे पुनः}
{निन्दकाय शठायापि न देयं भक्तिवैरिणे}% ४४

\twolineshloka
{रामभक्ताय शान्ताय कामक्रोधवियोगिने}
{वक्तव्यं सर्वदुःखस्य नाशकारकमुत्तमम्}% ४५

\twolineshloka
{रामान्नास्ति परो देवो रामान्नास्ति परं व्रतम्}
{न हि रामात्परो योगो न हि रामात्परो मखः}% ४६

\twolineshloka
{तं स्मृत्वा चैव जप्त्वा च पूजयित्वा नरः परम्}
{प्राप्नोति परमामृद्धिमैहिकामुष्मिकीं तथा}% ४७

\twolineshloka
{संस्मृतो मनसा ध्यातः सर्वकामफलप्रदः}
{ददाति परमां भक्तिं संसाराम्भोधितारिणीम्}% ४८

\twolineshloka
{श्वपाकोपि हि संस्मृत्य रामं याति परां गतिम्}
{ये वेदशास्त्रनिरतास्त्वादृशास्तत्र किं पुनः}% ४९

\twolineshloka
{सर्वेषां वेदशास्त्राणां रहस्यं ते प्रकाशितम्}
{समाचर तथा त्वं वै यथा स्यात्ते मनीषितम्}% ५०

\twolineshloka
{एको देवो रामचन्द्रो व्रतमेकं तदर्चनम्}
{मन्त्रोऽप्येकश्च तन्नाम शास्त्रं तद्ध्येव तत्स्तुतिः}% ५१

\twolineshloka
{तस्मात्सर्वात्मना रामचन्द्रं भजमनोहरम्}
{यथा गोष्पदवत्तुच्छो भवेत्संसारसागरः}% ५२

\twolineshloka
{श्रुत्वा मया तु तद्वाक्यं पुनः प्रश्नमकारिषम्}
{कथं वा ध्यायते देवः कथं वा पूज्यते नरैः}% ५३

\twolineshloka
{कथयस्व महाबुद्धे सर्वज्ञ मम विस्तरात्}
{यज्ज्ञात्वाहं कृतार्थः स्यां त्रिलोक्यां मुनिसत्तम}% ५४

\twolineshloka
{एतच्छ्रुत्वा तु मद्वाक्यं विचार्य स तु लोमशः}
{कथयामास मे सर्वं रामध्यानपुरःसरम्}% ५५

\twolineshloka
{शृणु विप्रेन्द्र वक्ष्यामि यत्पृष्टं तु त्वयानघ}
{यथा तुष्येद्रमानाथः संसारज्वरदाहकः}% ५६

\twolineshloka
{अयोध्यानगरे रम्ये चित्रमण्डपशोभिते}
{ध्यायेत्कल्पतरोर्मूले सर्वकामसमृद्धिदे}% ५७

\twolineshloka
{महामरकतस्वर्णनीलरत्नादिशोभितम्}
{सिंहासनं चित्तहरं कान्त्या तामिस्रनाशनम्}% ५८

\twolineshloka
{तस्योपरि समासीनं रघुराजं मनोरमम्}
{दूर्वादलश्यामतनुं देवदेवेन्द्रपूजितम्}% ५९

\twolineshloka
{राकायां पूर्णशीतांशुकान्तिधिक्कारिवक्त्रिणम्}
{अष्टमीचन्द्रशकलसमभालाधिधारिणम्}% ६०

\twolineshloka
{नीलकुन्तलशोभाढ्यं किरीटमणिरञ्जितम्}
{मकराकारसौन्दर्यकुण्डलाभ्यां विराजितम्}% ६१

\twolineshloka
{विद्रुमच्छवि सत्कान्तिरदच्छदविराजितम्}
{तारापतिकराकार द्विजराजि सुशोभितम्}% ६२

\twolineshloka
{जपापुष्पाभया माध्व्या जिह्वया शोभिताननम्}
{यस्यां वसन्ति निगमा ऋगाद्याः शास्त्रसंयुताः}% ६३

\twolineshloka
{कम्बुकान्तिधरग्रीवा शोभया समलङ्कृतम्}
{सिंहवदुच्चकौ स्कन्धौ मांसलौ बिभ्रतं वरम्}% ६४

\twolineshloka
{बाहू दधानं दीर्घाङ्गौ केयूरकटकाङ्कितौ}
{मुद्रिकाहीरशोभाभिर्भूषितौ जानुलम्बिनौ}% ६५

\twolineshloka
{वक्षो दधानं विपुलं लक्ष्मीवासेन शोभितम्}
{श्रीवत्सादिविचित्राङ्कैरङ्कितं सुमनोहरम्}% ६६

\twolineshloka
{महोदरं महानाभिं शुभकट्याविराजितम्}
{काञ्च्या वै मणिमत्या च विशेषेण श्रियान्वितम्}% ६७

\twolineshloka
{ऊरुभ्यां विमलाभ्यां वै जानुभ्यां शोभितं श्रिया}
{चरणाभ्यां वज्ररेखा यवाङ्कुशसुरेखया}% ६८

\twolineshloka
{युताभ्यां योगिध्येयाभ्यां कोमलाभ्यां विराजितम्}
{ध्यात्वा स्मृत्वा च संसारसागरं त्वं तरिष्यसि}% ६९

\twolineshloka
{तमेव पूजयन्नित्यं चन्दनादिभिरिच्छया}
{प्राप्नोति परमामृद्धिमैहिकामुष्मिकीं पराम्}% ७०

\twolineshloka
{त्वया पृष्टं महाराज रामस्य ध्यानमुत्तमम्}
{तत्ते कथितमेतद्वै संसारजलधिं तर}% ७१

{॥इति श्रीपद्मपुराणे पातालखण्डे शेषवात्स्यायनसंवादे रामाश्वमेधे आरण्यको पाख्याने लोमशारण्यकसंवादो नाम पञ्चत्रिंशत्तमोऽध्यायः॥३५॥}

\dnsub{षट्त्रिंशत्तमोऽध्यायः}\resetShloka

\uvacha{शेषउवाच}

\twolineshloka
{एतच्छ्रुत्वा तु विप्रेन्द्रो लोमशात्परमं महत्}
{पुनः पप्रच्छ तमृषिं सर्वज्ञं योगिनां वरम्}% १

\uvacha{आरण्यक उवाच}

\twolineshloka
{मुनिश्रेष्ठ वदैतन्मे पृच्छामि त्वां महामते}
{गुरवः कृपया युक्ता भाषन्ते सेवकेऽखिलम्}% २

\twolineshloka
{कोऽसौ रामो महाभाग यो नित्यं ध्यायते त्वया}
{तस्य कानि चरित्राणि वदस्व त्वं द्विजर्षभ}% ३

\twolineshloka
{किमर्थमवतीर्णोऽसौ कस्मान्मानुषतां गतः}
{तत्सर्वं कथयाशु त्वं मम संशयनुत्तये}% ४

\uvacha{शेष उवाच}

\twolineshloka
{इति वाक्यं समाकर्ण्य मुनेः परमशोभनम्}
{लोमशः कथयामास रामचारित्रमद्भुतम्}% ५

\twolineshloka
{लोकान्निरयसम्मग्नांज्ञात्वा योगेश्वरेश्वरः}
{कीर्तिं प्रथयितुं लोके यया घोरं तरिष्यति}% ६

\twolineshloka
{एवं ज्ञात्वा दयावार्धिः परमेशो मनोहरः}
{अवतारं चकारात्र चतुर्धा सश्रियान्वितः}% ७

\twolineshloka
{पुरा त्रेतायुगे प्राप्ते पूर्णांशो रघुनन्दनः}
{सूर्यवंशे समुत्पन्नो रामो राजीवलोचनः}% ८

\twolineshloka
{स रामो लक्ष्मणसखः काकपक्षधरो युवा}
{तातस्य वचनात्तौ तु विश्वामित्रमनुव्रतौ}% ९

\twolineshloka
{यज्ञसंरक्षणार्थाय राज्ञा दत्तौ कुमारकौ}
{दान्तौ धनुर्धरौ वीरौ विश्वामित्रमनुव्रतौ}% १०

\twolineshloka
{पथि प्रव्रजतोस्तत्र ताटका नाम राक्षसी}
{सङ्गता च वने घोरे तयोर्वै विघ्नकारणात्}% ११

\twolineshloka
{ऋषेरनुज्ञया रामस्ताटकां यमयातनाम्}
{प्रावेशयद्धनुर्वेदविद्याभ्यासेन राघवः}% १२

\twolineshloka
{यस्य पादतलस्पर्शाच्छिला वासवयोगजा}
{अहल्या गौतमवधूः पुनर्जाता स्वरूपिणी}% १३

\twolineshloka
{विश्वामित्रस्य यज्ञे तु सुप्रवृत्ते रघूत्तमः}
{मारीचं च सुबाहुं च जघान परमेषुभिः}% १४

\twolineshloka
{ईश्वरस्य धनुर्भग्नं जनकस्य गृहे स्थितम्}
{रामः पञ्चदशे वर्षे षड्वर्षामथ मैथिलीम्}% १५

\twolineshloka
{उपयेमे विवाहेन रम्यां सीतामयोनिजाम्}
{कृतकृत्यस्तदा जातः सीतां सम्प्राप्य राघवः}% १६

\twolineshloka
{ततो द्वादश वर्षाणि रेमे रामस्तया सह}
{सप्तविंशतिमे वर्षे यौवराज्यमकल्पयत्}% १७

\twolineshloka
{राजानमथ कैकेयी वरद्वयमयाचत}
{तयोरेकेन रामस्तु ससीतः सह लक्ष्मणः}% १८

\twolineshloka
{जटाधरः प्रव्रजतुवर्षाणीह चतुर्दश}
{भरतस्तु द्वितीयेन यौवराज्याधिपोऽस्तु मे}% १९

\twolineshloka
{जानकी लक्ष्मणसखं रामं प्राव्राजयन्नृपः}
{त्रिरात्रमुदकाहारश्चतुर्थेऽह्नि फलाशनः}% २०

\twolineshloka
{पञ्चमे चित्रकूटे तु रामस्थानमकल्पयत्}
{अथ त्रयोदशे वर्षे पञ्चवट्यां महामुने}% २१

\twolineshloka
{रामो विरूपयामास शूर्पणखां निशाचरीम्}
{वने विचरतस्तस्य जानक्या सहितस्य च}% २२

\twolineshloka
{आगतो राक्षसस्तां तु हर्तुं पापविपाकतः}
{ततो माघासिताष्टम्यां मुहूर्ते वृन्दसंज्ञिते}% २३

\twolineshloka
{राघवाभ्यां विना सीतां जहार दशकन्धरः}
{तेनैवं ह्रियमाणा सा चक्रन्द कुररी यथा}% २४

\twolineshloka
{रामरामेति मां रक्ष रक्ष मां रक्षसा हृताम्}
{यथा श्येनः क्षुधाक्रान्तः क्रन्दन्तीं वर्तिकां नयेत्}% २५

\twolineshloka
{तथा कामवशं प्राप्तो रावणो जनकात्मजाम्}
{नयत्येवं जनकजां जटायुः पक्षिराट्तदा}% २६

\twolineshloka
{युयुधे राक्षसेन्द्रेण स रावणहतोऽपतत्}
{मार्गशुक्लनवम्यां तु वसन्तीं रावणालये}% २७

\twolineshloka
{सम्पातिर्दशमे मास आचख्यौ वानरेषु ताम्}
{एकादश्यां महेन्द्राद्रे पुःप्लुवे शतयोजनम्}% २८

\twolineshloka
{हनूमान्निशि तस्यां तु लङ्कायां पर्यकालयत्}
{तद्रात्रिशेषे सीताया दर्शनं हि हनूमतः}% २९

\twolineshloka
{द्वादश्यां शिंशपावृक्षे हनूमान्पर्यवस्थितः}
{तस्यां निशायां जानक्या विश्वासाय च सङ्कथा}% ३०

\twolineshloka
{अक्षादिभिस्त्रयोदश्यां ततो युद्धमवर्तत}
{ब्रह्मास्त्रेण चतुर्दश्यां बद्धः शक्रजिता कपिः}% ३१

\twolineshloka
{वह्निना पुच्छयुक्तेन लङ्काया दहनं कृतम्}
{पूर्णिमायां महेन्द्राद्रौ पुनरागमनं कपेः}% ३२

\twolineshloka
{मार्गासितप्रतिपदः पञ्चभिः पथिवासरैः}
{पुनरागत्य षष्ठेऽह्नि ध्वस्तं मधुवनं किल}% ३३

\twolineshloka
{सप्तम्यां प्रत्यभिज्ञानदानं सर्वनिवेदनम्}
{अष्टम्युत्तरफल्गुन्यां मुहूर्ते विजयाभिधे}% ३४

\twolineshloka
{मध्यं प्राप्ते सहस्रांशौ प्रस्थानं राघवस्य च}
{रामः कृत्वा प्रतिज्ञां तु प्रयातो दक्षिणां दिशम्}% ३५

\twolineshloka
{तीर्त्वाहं सागरमपि हनिष्ये राक्षसेश्वरम्}
{दक्षिणाशां प्रयातस्य सुग्रीवोऽप्यभवत्सखा}% ३६

\twolineshloka
{वासरैः सप्तभिः सिन्धोः स्कन्धावारनिवेशनम्}
{पौषशुक्लप्रतिपदस्तृतीयायावदम्बुधेः}% ३७

\twolineshloka
{उपस्थानं ससैन्यस्य राघवस्य बभूव ह}
{बिभीषणश्चतुर्थ्यां तु रामेण सह सङ्गतः}% ३८

\twolineshloka
{समुद्रतरणार्थाय पञ्चम्यां मन्त्र उद्यतः}
{प्रायोपवेशनं चक्रे रामो दिनचतुष्टयम्}% ३९

\twolineshloka
{समुद्रवरलाभश्च सहोपायप्रदर्शनम्}
{ततो दशम्यामारम्भस्त्रयोदश्यां समापनम्}% ४०

\twolineshloka
{चतुर्दश्यां सुवेलाद्रौ रामः सैन्यं न्यवेशयत्}
{पौर्णमास्यां द्वितीयां तं त्रिदिनैः सैन्यतारणम्}% ४१

\twolineshloka
{तीर्त्वा तोयनिधिं रामो वानरेश्वरसैन्यवान्}
{रुरोध च पुरीं लङ्कां सीतार्थं सह लक्ष्मणः}% ४२

\twolineshloka
{तृतीयादि दशम्यन्तं निवेशश्च दिनाष्टकम्}
{शुकसारणयोस्तत्र प्राप्तिरेकादशे दिने}% ४३

\twolineshloka
{पौषासिताख्यद्वादश्यां सैन्यसङ्ख्यानमेव च}
{शार्दूलेन कपीन्द्राणां सहसारोपवर्णनम्}% ४४

\twolineshloka
{त्रयोदश्या अमावास्यां लङ्कायां दिवसैस्त्रिभिः}
{रावणः सैन्यसङ्ख्यानं रणोत्साहं तदाकरोत्}% ४५

\twolineshloka
{प्रययावङ्गदो दौत्यं माघशुक्लाद्यवासरे}
{सीतायाश्च ततो भर्तुर्मायामूर्द्धादिदर्शनम्}% ४६

\twolineshloka
{माघद्वितीयादि दिनैः सप्तभिर्यावदष्टमी}
{रक्षसां वानराणां च युद्धमासीच्च सङ्कुलम्}% ४७

\twolineshloka
{माघशुक्लनवम्यां तु रात्राविन्द्रजिता रणे}
{रामलक्ष्मणयोर्नागपाशबन्धः कृतः किल}% ४८

\twolineshloka
{आकुलेषु कपीशेषु निरुत्साहेषु सर्वशः}
{नागपाशविमोक्षार्थं दशम्यां पवनोऽजपत्}% ४९

\twolineshloka
{कर्णे स्वरूपं रामस्य गरुडागमनं ततः}
{एकादश्यां च द्वादश्यां धूम्राक्षस्य वधः कृतः}% ५०

\twolineshloka
{त्रयोदश्यां तु तेनैव निहतः कम्पनो रणे}
{माघशुक्लचतुर्दश्या यावत्कृष्णादिवासरम्}% ५१

\twolineshloka
{त्रिदिनेन प्रहस्तस्य नीलेन विहितो वधः}
{माघकृष्णद्वितीयायाश्चतुर्थ्यं तं त्रिभिर्दिनैः}% ५२

\twolineshloka
{रामेण तुमुले युद्धे रावणो द्रावितो रणात्}
{पञ्चम्या अष्टमीयावद्रावणेन प्रबोधितः}% ५३

\twolineshloka
{कुम्भकर्णस्तदा चक्रेऽभ्यवहारं चतुर्दिनम्}
{कुम्भकर्णो दिनैः षड्भिर्नवम्यास्तु चतुर्दशीम्}% ५४

\twolineshloka
{रामेण निहतो युद्धे बहुवानरभक्षकः}
{अमावास्यादिने शोकादवहारो बभूव ह}% ५५

\twolineshloka
{फाल्गुनादिप्रतिपदश्चतुर्थ्यन्तं चतुर्दिनैः}
{बिसतन्तुप्रभृतयो निहताः पञ्चराक्षसाः}% ५६

\twolineshloka
{पञ्चम्याः सप्तमी यावदतिकायवधस्तथा}
{अष्टम्याद्वादशी यावन्निहतौ दिनपञ्चकात्}% ५७

\twolineshloka
{निकुम्भकुम्भावूर्ध्वं तु मकराक्षस्त्रिभिर्दिनैः}
{फाल्गुनासितद्वितीयायां दिने शक्रजिता जितम्}% ५८

\twolineshloka
{तृतीयादिसप्तम्यन्तं दिनपञ्चकमेव च}
{ओषध्यानयनव्यग्रादवहारो बभूव ह}% ५९

\twolineshloka
{ततस्त्रयोदशीयावद्दिनैः पञ्चभिरिन्द्रजित्}
{लक्ष्मणेन हतो युद्धे विख्यातबलपौरुषः}% ६०

\twolineshloka
{चतुर्दश्यां दशग्रीवो दीक्षां प्रापावहारतः}
{अमावास्यादिने प्रायाद्युद्धाय दशकन्धरः}% ६१

\twolineshloka
{चैत्रशुक्लप्रतिपदः पञ्चमीदिनपञ्चकैः}
{रावणे युद्ध्यमाने तु प्रचुरो रक्षसां वधः}% ६२

\twolineshloka
{चैत्रषष्ठ्याष्टमी यावन्महापार्श्वादि मारणम्}
{चैत्रशुक्लनवम्यां तु सौमित्रेः शक्तिभेदनम्}% ६३

\twolineshloka
{कोपाविष्टेन रामेण द्रावितो दशकन्धरः}
{द्रोणाद्रिराञ्जनेयेन लक्ष्मणार्थमुपाहृतः}% ६४

\twolineshloka
{दशम्यामवहारोभूद्रात्रौ युद्धे तु रक्षसाम्}
{एकादश्यां तु रामाय रथं मातलिसारथिः}% ६५

\twolineshloka
{प्रेरितो वासवेनाजावर्पयामास भक्तितः}
{कोपवानथ द्वादश्या यावत्कृष्णचतुर्दशी}% ६६

\twolineshloka
{अष्टादशदिनै रामो रावणं द्वैरथेऽवधीत्}
{सङ्ग्रामे तुमुले जाते रामो जयमवाप्तवान्}% ६७

\twolineshloka
{माघशुक्लद्वितीयायाश्चैत्रकृष्ण चतुर्दशीम्}
{सप्ताशीतिदिनेष्वेव मध्यं पञ्चदशाहकम्}% ६८

\twolineshloka
{युद्धावहारः सङ्ग्रामो द्वासप्तति दिनान्यभूत्}
{संस्कारो रावणादीनाममावस्या दिनेऽभवत्}% ६९

\twolineshloka
{वैशाखादि तिथौ राम उवास रणभूमिषु}
{अभिषिक्तो द्वितीयायां लङ्काराज्ये विभीषणः}% ७०

\twolineshloka
{सीताशुद्धिस्तृतीयायां देवेभ्यो वरलम्भनम्}
{हत्वा चिरेण लङ्केशं लक्ष्मणाग्रज एव सः}% ७१

\twolineshloka
{गृहीत्वा जानकीं पुण्यां दुःखितां राक्षसेन तु}
{आदाय परया प्रीत्या जानकीं स न्यवर्तत}% ७२

\twolineshloka
{वैशाखस्य चतुर्थ्यां तु रामः पुष्पकमाश्रितः}
{विहायसा निवृत्तस्तु भूयोऽयोध्यां पुरीं प्रति}% ७३

\twolineshloka
{पूर्णे चतुर्दशे वर्षे पञ्चम्यां माधवस्य तु}
{भरद्वाजाश्रमे रामः सगणः समुपाविशत्}% ७४

\twolineshloka
{नन्दिग्रामे तु षष्ठ्यां स भरतेन समागतः}
{सप्तम्यामभिषिक्तोऽसावयोध्यायां रघूद्वहः}% ७५

\twolineshloka
{दशैकाधिकमासांस्तुचतुर्दशाहानि मैथिली}
{उवास राम रहिता रावणस्य निवेशने}% ७६

\twolineshloka
{द्विचत्वारिंशक वर्षे रामो राज्यमकारयत्}
{सीतायाश्च त्रयस्त्रिंशद्वत्सराश्च तदाभवन्}% ७७

\twolineshloka
{स चतुर्दशवर्षान्ते प्रविश्य च पुरीं प्रभुः}
{अयोध्यां मुदितो रामो हत्वा रावणमाहवे}% ७८

\twolineshloka
{भ्रातृभिः सहितस्तत्र रामो राज्यमथाकरोत्}
{राज्यं प्रकुर्वतस्तस्य पुरोधा वदतां वरः}% ७९

\twolineshloka
{अगस्त्यः कुम्भसम्भूतिस्तमागन्ता रघोः पतिम्}
{तद्वाक्याद्रघुनाथोऽसौ करिष्यति हयक्रतुम्}% ८०

\twolineshloka
{तस्यागमिष्यति हयो ह्याश्रमे तव सुव्रत}
{तस्य योधाः प्रमुदिता आयास्यन्ति तवाश्रमम्}% ८१

\twolineshloka
{तेषामग्रे रामकथाः करिष्यसि मनोहराः}
{तैः साकं त्वमयोध्यायां गन्तासि वै द्विजर्षभ}% ८२

\twolineshloka
{दृष्ट्वा राममयोध्यायां पद्मपत्रनिभेक्षणम्}
{तत्क्षणादेव संसारवार्धिनिस्तारवान्भव}% ८३

\twolineshloka
{इत्युक्त्वा मां मुनिवरो लोमशः सर्वबुद्धिमान्}
{उवाच ते किं प्रष्टव्यं तदाहमवदं हि तम्}% ८४

\twolineshloka
{ज्ञातं त्वत्कृपया सर्वं रामचारित्रमद्भुतम्}
{त्वत्प्रसादादवाप्स्येऽहं रामस्य चरणाम्बुजम्}% ८५

\twolineshloka
{मया नमस्कृतः पश्चाज्जगाम स मुनीश्वरः}
{तत्प्रसादान्मयावाप्तं रामस्य चरणार्चनम्}% ८६

\twolineshloka
{सोऽहं स्मरामि रामस्य चरणावन्वहं मुहुः}
{गायामि तस्य चरितं मुहुर्मुहुरतन्द्रितः}% ८७

\twolineshloka
{पावयामि जनानन्यान्गानेन स्वान्तहारिणा}
{हृष्यामि तन्मुनेर्वाक्यं स्मारंस्मारं तदीक्षया}% ८८

\twolineshloka
{धन्योऽहं कृतकृत्योऽहं सभाग्योऽहं महीतले}
{रामचन्द्र पदाम्भोज दिदृक्षा मे भविष्यति}% ८९

\twolineshloka
{तस्मात्सर्वात्मना रामो भजनीयो मनोहरः}
{वन्दनीयो हि सर्वेषां संसाराब्धितितीर्षया}% ९०

\twolineshloka
{तस्माद्यूयं किमर्थं वै प्राप्ताः को वानराधिपः}
{यागं करोति धर्मात्मा हयमेधं महाक्रतुम्}% ९१

\twolineshloka
{तत्सर्वं कथयन्त्वत्र यां तु वाहस्य पालने}
{स्मरन्तु रघुनाथाङ्घ्रिं स्मृत्वा स्मृत्वा पुनः पुनः}% ९२

\twolineshloka
{इति वाक्यं समाकर्ण्य मुनेर्विस्मयमागताः}
{रघुनाथं स्मरन्तस्ते प्रोचुरारण्यकं मुनिम्}% ९३

{॥इति श्रीपद्मपुराणे पातालखण्डे शेषवात्स्यायनसंवादे रामाश्वमेधे लोमशारण्यकसंवादे रामचरित्रकथनं नाम षट्त्रिंशत्तमोऽध्यायः॥३६॥}

\dnsub{सप्तत्रिंशत्तमोऽध्यायः}\resetShloka

\uvacha{शेष उवाच}

\twolineshloka
{ते पृष्टा मुनिवर्येण रामचारित्रमद्भुतम्}
{धन्यं सभाग्यं मन्वानाः प्रोचुरात्मानमादरात्}% १

\uvacha{जना ऊचुः}

\twolineshloka
{पवित्रिता वयं सर्वे दर्शनेन तवाधुना}
{यद्रामकथयास्मान्वै पावयस्यधुना जनान्}% २

\twolineshloka
{शृणुष्व वचनं तथ्यं भवान्ब्रह्मर्षिसत्तमः}
{त्वया पृष्टं यदस्मभ्यं सर्वं तत्कथयाम वै}% ३

\twolineshloka
{अगस्त्यवाक्याच्छ्रीरामो विप्रहत्यापनुत्तये}
{यागं करोति सुमहान्सर्वसम्भारसम्भृतम्}% ४

\twolineshloka
{तं पालयानाः सर्वे वै त्वदाश्रममुपागताः}
{अश्वेन सहिता विप्र तज्जानीहि महामते}% ५

\twolineshloka
{इति वाक्यं समाकर्ण्य मनोहारि रसायनम्}
{अत्यन्तं हर्षमापेदे ब्राह्मणो रामभक्तिमान्}% ६

\twolineshloka
{अद्य मे फलितो वृक्षो मनोरथश्रियान्वितः}
{अद्य मे जननी धन्या जातं मां सुषुवे तु या}% ७

\twolineshloka
{अद्य राज्यं मया प्राप्तं कण्टकेन विवर्जितम्}
{अद्य कोशाः सुसम्पन्ना अद्य देवाः सुतोषिताः}% ८

\twolineshloka
{अग्निहोत्रफलं त्वद्य प्राप्तं मे हविषा हुतम्}
{यद्द्रक्ष्ये रामचन्द्रस्य चरणाम्भोरुहोर्युगम्}% ९

\twolineshloka
{यो नित्यं ध्यायते स्वान्ते अयोध्यायाः पतिः प्रभुः}
{स मे दृग्गोचरो नूनं भविष्यति मनोहरः}% १०

\twolineshloka
{हनूमान्मां समालिङ्ग्य प्रक्ष्यते कुशलं मम}
{भक्तिं मे महतीं दृष्ट्वा तोषं प्राप्स्यति सत्तमः}% ११

\twolineshloka
{इति वाक्यं समाकर्ण्य हनूमान्कपिसत्तमः}
{जग्राह पादयुगलं मुनेरारण्यकस्य हि}% १२

\twolineshloka
{स्वामिन्हनूमान्विप्रर्षे सेवकोऽहं पुरःस्थितः}
{जानीहि रामदासस्य रेणुकल्पं मुनीश्वर}% १३

\twolineshloka
{इत्युक्तवति तस्मिन्वै मुनिः परमहर्षितः}
{आलिलिङ्ग हनूमन्तं रामभक्त्या सुशोभितम्}% १४

\twolineshloka
{उभौ प्रेमविनिर्भिन्नावुभावपि सुधाप्लुतौ}
{स्थगितौ चित्रलिखिताविव तत्र बभूवतुः}% १५

\twolineshloka
{उपविष्टौ कथास्तत्र चक्रतुः सुमनोहराः}
{रघुनाथपदाम्भोजप्रीतिनिर्भरमानसौ}% १६

\twolineshloka
{हनूमांस्तमुवाचेदं वचो विविधशोभनम्}
{आरण्यकं मुनिवरं रामाङ्घ्रिध्याननिर्भृतम्}% १७

\twolineshloka
{स्वामिन्नयं दशरथकुलहीराङ्कुरो महान्}
{रामभ्राता महाशूरः शत्रुघ्नः प्रणमत्यसौ}% १८

\twolineshloka
{लवणो येन निहतः सर्वलोकभयङ्करः}
{कृताश्च सुखिनः सर्वे मुनयः सुतपोधनाः}% १९

\twolineshloka
{एष पुष्कलनामा त्वां नमत्युद्भटसेवितः}
{येनाधुना महावीरा जिताः समरमण्डले}% २०

\twolineshloka
{जानीह्येनं बहुगुणं रामामात्यं महाबलम्}
{प्राणप्रियं रघुपतेः सर्वज्ञं धर्मकोविदम्}% २१

\twolineshloka
{सुबाहुरयमत्युग्रो वैरिवंशदवानलः}
{रामपादाब्जरोलम्बो नमति त्वां महायशाः}% २२

\twolineshloka
{सुमदोऽप्येष पार्वत्या दत्तरामाङ्घ्रिसेवया}
{प्राप्तोऽधुनासौ संसारवार्धिनिस्तरणं महत्}% २३

\twolineshloka
{सत्यवानयमश्वं यः प्राप्तमाश्रुत्य सेवकात्}
{राज्यं निवेदयामास स त्वां प्रणमति क्षितौ}% २४

\twolineshloka
{इति वाक्यं समाकर्ण्य समालिङ्ग्य समादरात्}
{चकारारण्यक ऋषिः स्वागतं फलकादिना}% २५

\twolineshloka
{ते हृष्टास्तत्र वसतिं चक्रुर्मुनिवराश्रमे}
{प्रातर्नित्यक्रियां कृत्वा रेवायां ते महोद्यमाः}% २६

\twolineshloka
{नरयानमथारोप्य सेवकैः सहितं मुनिम्}
{शत्रुघ्नः प्रापयामासायोध्यां रामकृतालयाम्}% २७

\twolineshloka
{स दूरान्नगरीं दृष्ट्वा सूर्यवंशनृपोषिताम्}
{पदातिरभवद्वेगाद्रघुनाथदिदृक्षया}% २८

\twolineshloka
{सम्प्राप्य नगरीं रम्यामयोध्यां जनशोभिताम्}
{मनोरथसहस्रेण संरूढो रामदर्शने}% २९

\twolineshloka
{ददर्श तत्र सरयूतीरे मण्डपशोभिते}
{रामं दूर्वादलश्यामं कञ्जकान्तिविलोचनम्}% ३०

\twolineshloka
{मृगशृङ्गं कटौ रम्यं धारयन्तं श्रियान्वितम्}
{ऋषिवृन्दैर्व्यासमुख्यैर्वृतं शूरैः सुसेवितम्}% ३१

\twolineshloka
{भरतेन सुमित्रायास्तनूजेन परीवृतम्}
{ददतं दीनसन्धेभ्यो दानानि प्रार्थितानि तम्}% ३२

\twolineshloka
{विलोक्यारण्यकाख्योऽसौ कृतार्थ इत्यमन्यत}
{मल्लोचने पद्मदलसमाने रामलोकके}% ३३

\twolineshloka
{अद्य मे सर्वशास्त्रस्य ज्ञातृत्वं बहुसार्थकम्}
{येन श्रीराममाज्ञाय प्राप्तोऽयोध्यापुरीमिमाम्}% ३४

\fourlineindentedshloka
{इत्येवमादिवचनानि बहूनि हृष्टो}
{रामाङ्घ्रिदर्शनसुहर्षित गात्रशोभी}
{प्रायाद्रमेश्वरसमीपमगम्यमन्यै-}
{र्योगेश्वरैरपि विचारपरैः सुदूरम्}% ३५

\twolineshloka
{धन्योऽहमद्य रामस्य चरणावक्षिगोचरौ}
{करिष्यामि वचो रम्यं वदन्राममवेक्षयन्}% ३६

\twolineshloka
{रामोऽपि वाडवश्रेष्ठं ज्वलन्तं स्वेन तेजसा}
{तपोमूर्तिधरं वीक्ष्य प्रत्युत्थानमथाकरोत्}% ३७

\twolineshloka
{रामचन्द्रस्तस्य पादौ सुचिरं नतवान्महान्}
{ब्रह्मण्यदेवपावित्र्यं कृतमद्यतनोर्मम}% ३८

\twolineshloka
{इति वाक्यं वदंस्तस्य पादयोः पतितः प्रभुः}
{सुरासुरनमन्मौलिमणिनीराजिताङ्घ्रिकः}% ३९

\twolineshloka
{प्रणतं तं नृपश्रेष्ठं वाडवेन्द्रो महातपाः}
{गृहीत्वा भुजयोर्मध्यमालिलिङ्ग प्रियं प्रभुम्}% ४०

\twolineshloka
{कौसल्यातनयस्तं वा उच्चैर्मणिमयासने}
{संस्थाप्य च पदोर्युग्मं जलेनाक्षालयत्प्रभुः}% ४१

\twolineshloka
{पादावनेजनोदं तु मस्तकेऽधाद्धरिः स्वयम्}
{पवित्रितोऽद्य सगणः सकुटुम्ब इति ब्रुवन्}% ४२

\twolineshloka
{चन्दनेन विलिप्याथ गां च प्रादात्पयस्विनीम्}
{उवाच च वचो रम्यं देवदेवेन्द्र सेवितः}% ४३

\twolineshloka
{स्वामिन्मखो मया वाजिमेधसंज्ञः क्रियेत ह}
{सोयं त्वच्चरणा यातादद्यपूर्णो भविष्यति}% ४४

\twolineshloka
{अद्य मे ब्रह्महत्योत्थ पापहानिं करिष्यति}
{अश्वमेधः क्रतुर्युष्मच्चरणेन पवित्रितः}% ४५

\twolineshloka
{इति वाक्यं ब्रुवाणं तं राजराजेन्द्रसेवितम्}
{आरण्यक उवाचेदं हसन्माध्व्या गिरा मुनिः}% ४६

\twolineshloka
{स्वामिंस्तव तु युक्तं हि वचो ब्रह्मण्यभूमिप}
{त्वन्मूर्तयो महाराज ब्राह्मणा वेदपारगाः}% ४७

\twolineshloka
{त्वं यदा ब्रह्मपूजादि शुभं कर्म करिष्यसि}
{ततोऽखिला नृपा विप्रं पूजयिष्यन्ति भूमिप}% ४८

\twolineshloka
{त्वयोक्तं यन्महाराज विप्रहत्यापनुत्तये}
{यागं करोमि विमलं तत्तु हास्यकरं वचः}% ४९

\twolineshloka
{त्वन्नामस्मरणान्मूढः सर्वशास्त्रविवर्जितः}
{सर्वपापाब्धिमुत्तीर्य स गच्छेत्परमं पदम्}% ५०

\twolineshloka
{सर्ववेदेतिहासानां सारार्थोऽयमिति स्फुटम्}
{यद्रामनामस्मरणं क्रियते पापतारकम्}% ५१

\twolineshloka
{तावद्गर्जन्ति पापानि ब्रह्महत्यासमानि च}
{न यावत्प्रोच्यते नाम रामचन्द्र तव स्फुटम्}% ५२

\twolineshloka
{त्वन्नामगर्जनं श्रुत्वा महापातककुञ्जराः}
{पलायन्ते महाराज कुत्रचित्स्थानलिप्सया}% ५३

\twolineshloka
{तस्मात्तव कथं हत्या महापुण्यददर्शन}
{राम त्वत्सुकथां श्रुत्वा पूतः सद्यो भविष्यति}% ५४

\twolineshloka
{मया पूर्वं कृतयुगे गङ्गायास्तीरवासिनाम्}
{ऋषीणां मुखतो वाक्यं श्रुतमेतत्पुराविदाम्}% ५५

\twolineshloka
{तावत्पापभियः पुंसां कातराणां सुपापिनाम्}
{यावन्न वदते वाचा रामनाममनोहरम्}% ५६

\twolineshloka
{तस्माद्धन्योऽहमधुना मम संसृतिनाशनम्}
{साम्प्रतं सुलभं रामचन्द्र त्वद्दर्शनादभूत्}% ५७

\twolineshloka
{इत्युक्तवन्तं स मुनिं पूजयामास तत्र वै}
{सर्वे मुनिजनाः साधु साधु वाक्यमिति ब्रुवन्}% ५८

\uvacha{शेष उवाच}

\twolineshloka
{अत्याश्चर्यमभूत्तत्र तन्मे निगदतः शृणु}
{वात्स्यायनमुनिश्रेष्ठ रामभक्तिपरायण}% ५९

\twolineshloka
{रामं दृष्ट्वा महाराजं यादृशं ध्यानगोचरम्}
{अत्यन्तं हर्षमापन्नो जगाद स मुनीश्वरान्}% ६०

\twolineshloka
{मुनीश्वराः संशृणुत मद्वाक्यं सुमनोहरम्}
{मादृशः को न भूलोके भविष्यति सुभाग्यवान्}% ६१

\twolineshloka
{नास्ति मत्सदृशः कोपि न जातो न भविष्यति}
{यद्रामभद्रो मां नत्वा स्वागतं परिपृष्टवान्}% ६२

\twolineshloka
{यत्पादपङ्कजरजः श्रुतिमृग्यं सदैव हि}
{सोऽद्य मत्पादयोः पाथः पीत्वा पूतममन्यत}% ६३

\twolineshloka
{एवं प्रवदतस्तस्य ब्रह्मस्फोटोऽभवत्तदा}
{निर्गतं तद्भवं तेजो विवेश रघुनायके}% ६४

\twolineshloka
{पश्यतां सर्वलोकानां सरयूतीरमण्डपे}
{सायुज्यमुक्तिं सम्प्राप दुर्ल्लभां योगिभिर्जनैः}% ६५

\twolineshloka
{दिवि तूर्यनिनादोऽभूद्वीणानादोऽभवत्तदा}
{पुष्पवृष्टिः पपाताग्रे पश्यतां चित्रमद्भुतम्}% ६६

\twolineshloka
{मुनयोऽप्येतदीक्षित्वा प्रशंसन्तो मुनीश्वरम्}
{कृतार्थोयं मुनिश्रेष्ठो यद्रामवपुषीक्षितः}% ६७

{॥इति श्रीपद्मपुराणे पातालखण्डे शेषवात्स्यायनसंवादे रामाश्वमेधे आरण्यकमुनेर्विष्णुलोकगमनं नाम सप्तत्रिंशत्तमोऽध्यायः॥३७॥}

\dnsub{अष्टत्रिंशत्तमोऽध्यायः}\resetShloka

\uvacha{सूत उवाच}

\twolineshloka
{एतदाख्यानकं श्रुत्वा वात्स्यायन उदारधीः}
{परमं हर्षमापेदे जगाद च फणीश्वरम्}% १

\uvacha{वात्स्यायन उवाच}

\twolineshloka
{कथां संशृण्वते मह्यं तृप्तिर्नास्ति फणीश्वर}
{रघुनाथस्य भक्तार्तिहारिकीर्तिकरस्य वै}% २

\twolineshloka
{धन्य आरण्यको नाम मुनिर्वेदधरः परः}
{रघुनाथं समालोक्य देहं तत्याज नश्वरम्}% ३

\twolineshloka
{ततो राज्ञो हयः कुत्र गतः केन नियन्त्रितः}
{कथं तत्र रमानाथ कीर्तिर्जाता फणीश्वर}% ४

\twolineshloka
{सर्वं कथय मे तथ्यं सर्वज्ञोऽस्ति यतो भवान्}
{धराधरवपुर्धारी साक्षात्तस्य स्वरूपधृक्}% ५

\uvacha{व्यास उवाच}

\twolineshloka
{इति वाक्यं समाकर्ण्य प्रहृष्टेनान्तरात्मना}
{उवाच रामचारित्रं तत्तद्गुणकथोदयम्}% ६

\uvacha{शेष उवाच}

\twolineshloka
{साधु पृच्छसि विप्रर्षे रघुनाथगुणान्मुहुः}
{श्रुता न श्रुतवत्कृत्वा तेषु लोलुपतां दधत्}% ७

\twolineshloka
{ततो निरगमद्वाहः सैनिकैर्बहुभिवृतः}
{रेवातीरे मनोज्ञे तु मुनिवृन्दनिषेविते}% ८

\twolineshloka
{सेनाचरास्ततः सर्वे यत्र वाहस्ततस्ततः}
{प्रसर्पन्ति निरीक्षन्तस्तन्मार्गं रणकोविदाः}% ९

\twolineshloka
{वाजी गतोऽथ रेवाया ह्रदेऽगाधजलान्विते}
{भाले स्वर्णभवं पत्रं धारयन्पूजिताङ्गकः}% १०

\twolineshloka
{ततो जले ममज्जासौ रामचन्द्र हयो वरः}
{तदा सर्वे महाशूरास्तत्र विस्मयमागताः}% ११

\twolineshloka
{तैः परस्परमेवोचे कथं हयसमागमः}
{कोऽत्र गन्ता जले वाहमानेतुं तं महोदयम्}% १२

\twolineshloka
{इति यावत्समुद्विग्ना मन्त्रयन्ते परस्परम्}
{तावद्वीरशतैः सार्धमाजगाम रघोः पतिः}% १३

\twolineshloka
{तान्सर्वान्विमनस्कान्स दृष्ट्वा शत्रुघ्नसंज्ञितः}
{पप्रच्छ मेघगम्भीरवाचा वीरशिरोमणिः}% १४

\twolineshloka
{किं स्थितं निखिलैरद्य युष्माभिः सङ्घशो जले}
{कुत्राश्वो रघुनाथस्य स्वर्णपत्रेण शोभितः}% १५

\twolineshloka
{जले किं विनिमग्नोऽसौ हृतो वा केन मानिना}
{तन्मे कथयत क्षिप्रं कथं यूयं विमोहिताः}% १६

\uvacha{शेष उवाच}

\twolineshloka
{इति वाक्यं समाकर्ण्य राज्ञो रघुवरस्य हि}
{कथयामासुस्ते सर्वं वीराः शूरशिरोमणिम्}% १७

\uvacha{जना ऊचुः}

\twolineshloka
{स्वामिन्वयं न जानीमो मुहूर्तमभवज्जले}
{निममज्ज ततो नायाद्धयस्तव मनोहरः}% १८

\twolineshloka
{त्वमेव तत्र गत्वेमं वाहमानय वेगतः}
{अस्माभिस्तत्र गन्तव्यं त्वया सार्द्धं महामते}% १९

\twolineshloka
{इति श्रुत्वा वचस्तेषां सैनिकानां रघूद्वहः}
{खेदं प्राप जनान्पश्यञ्जलसन्तरणोद्यतान्}% २०

\twolineshloka
{उवाच मन्त्रिमुख्यं स किं कर्तव्यमतः परम्}
{कथं वाहस्य सम्प्राप्तिर्भविष्यति वदस्व तत्}% २१

\twolineshloka
{के तत्र शूराः संयोज्या जलेऽन्वेषयितुं हयम्}
{को वा नयिष्यते वाहं केनोपायेन तद्वद}% २२

\twolineshloka
{इति राज्ञोवचः श्रुत्वा सुमतिर्मन्त्रिसत्तमः}
{उवाच समये योग्यं शत्रुघ्नं हर्षयन्निव}% २३

\twolineshloka
{स्वामिन्नस्ति तव श्रीमञ्छक्तिरद्भुतकर्मणः}
{पातालगमने शक्तिर्जलमध्यादिह स्फुटम्}% २४

\twolineshloka
{अन्यच्च पुष्कलस्यापि शक्तिरस्ति महात्मनः}
{हनूमतोऽपि रामस्य पादसेवापरस्य च}% २५

\twolineshloka
{तस्माद्यूयं त्रयो गत्वा हयमानयत ध्रुवम्}
{यतो भवेद्वाहमेधो रघुनाथस्य धीमतः}% २६

\uvacha{शेष उवाच}

\twolineshloka
{इति वाक्यं समाश्रुत्य शत्रुघ्नः परवीरहा}
{स्वयं विवेश तोयान्तर्हनुमत्पुष्कलान्वितः}% २७

\twolineshloka
{यावज्जलं विवेशासौ तावत्पुरमदृश्यत}
{अनेकोद्यानशोभाढ्यममेयं पुटभेदनम्}% २८

\twolineshloka
{तत्र माणिक्यरचिते स्तम्भे स्वर्णमये हयम्}
{बद्धं ददर्श रामस्य स्वर्णपत्रसुशोभितम्}% २९

\twolineshloka
{स्त्रियस्तत्र मनोहारि रूपधारिण्य उत्तमाः}
{सेवन्ते सुन्दरीमेकां पर्यङ्के सुखमास्थिताम्}% ३०

\twolineshloka
{तान्दृष्ट्वा ताः स्त्रियः सर्वाः प्रावोचन्स्वामिनीं प्रति}
{एतेऽल्पवर्ष्मवयसो मांसपुष्टकलेवराः}% ३१

\twolineshloka
{भविष्यन्ति तव श्रेष्ठमाहारस्य फलं महत्}
{एतेषां शोणितं स्वादु पुरुषाणां गतायुषाम्}% ३२

\twolineshloka
{एतद्वचः समाकर्ण्य सेवकीनां वराङ्गना}
{जहास किञ्चिद्वदनं नर्तयन्ती भ्रुवानघा}% ३३

\twolineshloka
{तावत्त्रयस्ते सम्प्राप्ताः सन्नाहश्री विशोभिताः}
{शिरस्त्राणानि दधतः शौर्यवीर्यसमन्विताः}% ३४

\twolineshloka
{ता दृष्ट्वा महिलास्तत्र सौन्दर्यश्रीसमन्विताः}
{प्रोचुस्ते विस्मयं विप्र किमिदं दृश्यते महत्}% ३५

\twolineshloka
{नमश्चक्रुर्महात्मानः सर्वे देववराङ्गनाः}
{किरीटमणिविद्योतद्योतिताङ्घ्रियुगास्ततः}% ३६

\twolineshloka
{सा तान्पप्रच्छ पुरुषान्सर्वश्रेष्ठा तु भामिनी}
{के यूयमत्र सम्प्राप्ताः कथं चापधरा नराः}% ३७

\twolineshloka
{मत्स्थलं सर्वदेवानामगम्यं मोहनं महत्}
{अत्र प्राप्तस्य तु क्वापि निवृत्तिर्न भवत्युत}% ३८

\twolineshloka
{अश्वोऽयं कस्य राज्ञो वै कथं चामरवीजितः}
{स्वर्णपत्रेण शोभाढ्यः कथयन्तु ममाग्रतः}% ३९

\uvacha{शेष उवाच}

\twolineshloka
{इति तस्या वचः श्रुत्वा मोहनाचारसंयुतम्}
{हनूमांस्तां प्रत्युवाच गतभीः प्रहसन्निव}% ४०

\twolineshloka
{वयं वै किङ्करा राज्ञस्त्रैलोक्यस्य शिखामणेः}
{त्रिलोकीयं प्रणमते सर्वदेवशिरोमणिम्}% ४१

\twolineshloka
{रामभद्रस्य जानीहि हयमेधप्रवर्तितुः}
{प्रमुञ्च वाहमस्माकं कथं बद्धो वराङ्गने}% ४२

\twolineshloka
{वयं सर्वास्त्रकुशलाः सर्वशस्त्रास्त्रकोविदाः}
{नयिष्यामो बलाद्वाहं हत्वा तत्प्रतिरोधकान्}% ४३

\twolineshloka
{इति वाक्यं समाकर्ण्य प्लवङ्गस्य वराङ्गना}
{विवरस्था प्रत्युवाच हसन्ती वाक्यकोविदा}% ४४

\twolineshloka
{मयानीतमिमं वाहं न कोमोचयितुं क्षमः}
{वर्षायुतेन निशितैर्बाणकोटिभिरुच्छिखैः}% ४५

\twolineshloka
{परं रामस्य पादाब्जसैवकी कर्मकारिणी}
{न ग्रहीष्यामि तद्वाहं राजराजस्य धीमतः}% ४६

\twolineshloka
{महान विनयो जातो मम नेत्र्याः सुवाजिनः}
{क्षमताद्रामचन्द्रस्तच्छरण्यो भक्तवत्सलः}% ४७

\twolineshloka
{यूयं क्लिष्टास्तत्पुरुषा हयार्थं तस्य रक्षितुः}
{याचध्वं वरमप्राप्यं देवानामपि सत्तमाः}% ४८

\twolineshloka
{यथा मेमीवमत्युग्रं क्षमेत पुरुषोत्तमः}
{व्रीडां त्यक्त्वाखिलां यूयं वृणुध्वं वरमुत्तमम्}% ४९

\twolineshloka
{तस्या वचः परं श्रुत्वा हनूमान्नि जगाद ताम्}
{रघुनाथप्रसादेन सर्वमस्माकमूर्जितम्}% ५०

\fourlineindentedshloka
{तथापि याचे वरमेकमुत्तमं}
{विधेहि तन्मे मनसः समीहितम्}
{भवे भवे नो रघुनायकः पति-}
{र्वयं च तत्कर्मकराश्च किङ्कराः}% ५१

\twolineshloka
{एतद्वचनमाकर्ण्य प्लवगस्य तदाङ्गना}
{उवाच वाक्यं मधुरं प्रहस्य गुणपूजितम्}% ५२

\twolineshloka
{भवद्भिः प्रार्थितं यद्वै दुर्ल्लभं सर्वदैवतैः}
{तद्भविष्यत्यसन्देहः सेवकास्तद्रघोः पतेः}% ५३

\twolineshloka
{अथापि वरमेकं वै दास्यामि कृतहेलना}
{रघुनाथस्य तुष्ट्यर्थं तदृतं मे भवेद्वचः}% ५४

\twolineshloka
{अग्रे वीरमणिर्भूपो महावीरसमन्वितः}
{ग्रहीष्यति भवद्वाहं शिवेन परिरक्षितः}% ५५

\twolineshloka
{तज्जयार्थे महास्त्रं मे गृह्णीत सुमहाबलाः}
{द्वैरथे स तु योद्धव्यः शत्रुघ्नेन त्वया महान्}% ५६

\twolineshloka
{इदमस्त्रं यदा त्वं तु क्षेपयिष्यसि सङ्गरे}
{अनेन पूतो रामस्य स्वरूपं ज्ञास्यते पुनः}% ५७

\twolineshloka
{ज्ञात्वा तं वाजिनं दत्वा चरणे प्रपतिष्यति}
{तस्माद्गृह्णीध्वमस्त्रं तन्मम वैरिविदारणम्}% ५८

\twolineshloka
{तच्छ्रुत्वा रघुनाथस्य भ्राता जग्राह चास्त्रकम्}
{उदङ्मुखः पवित्राङ्गो योगिन्या दत्तमद्भुतम्}% ५९

\twolineshloka
{तत्प्राप्यास्त्रं महातेजा बभूव रिपुकर्शनः}
{दुष्प्रधर्ष्यो दुराराध्यो वैरिवारणसत्सृणिः}% ६०

\twolineshloka
{तां नत्वा राघवश्रेष्ठः शत्रुघ्नो हयसत्तमम्}
{गृहीत्वागाज्जलात्तस्माद्रेवातीरे सुखोचिते}% ६१

\twolineshloka
{तं दृष्ट्वा सैनिकाः सर्वे प्रहृष्टाङ्गा मुदान्विताः}
{साधुसाधु प्रशंसन्तः पप्रच्छुर्हयनिर्गमम्}% ६२

\twolineshloka
{हनूमान्कथयामास हयस्यागमनं महत्}
{वरप्राप्तिं च ताभ्यो वै तेऽपि श्रुत्वा मुदं गताः}% ६३

{॥इति श्रीपद्मपुराणे पातालखण्डे शेषवात्स्यायनसंवादे रामाश्वमेधे शत्रुघ्नस्य योगिनीदर्शनजलमध्याद्वाहप्राप्तिर्नाम अष्टत्रिंशत्तमोऽध्यायः॥३८॥}

\dnsub{एकोनचत्वारिंशत्तमोऽध्यायः}\resetShloka

\uvacha{शेष उवाच}

\twolineshloka
{निनदत्सुमृदङ्गेषु वीणानादेषु सर्वतः}
{मुक्तो वाहस्ततो देव पुरं देवविनिर्मितम्}% १

\twolineshloka
{यत्र स्फाटिक कुड्यानां रचनाभिर्गृहा नृणाम्}
{हसन्ति विन्ध्यं विमलं पर्वतं नागसेवितम्}% २

\twolineshloka
{राजतानि गृहाण्यत्र दृश्यन्ते प्रकृतेरपि}
{विचित्रमणिसन्नद्धा नानामाणिक्यगोपुराः}% ३

\twolineshloka
{पद्मिन्यो यत्र लोकानां गेहे गेहे मनोहराः}
{हरन्ति चित्तानि नृणां मुखपद्मकलेक्षिताः}% ४

\twolineshloka
{पद्मरागमणिर्यत्र गेहे गेहे सुभूमिषु}
{बद्धः संलक्ष्यते विप्र तदोष्ठस्पर्धया नु किम्}% ५

\twolineshloka
{क्रीडाशैलाः प्रत्यगारं नीलरत्नविनिर्मिताः}
{कुर्वन्ति शङ्कां मेघस्य मयूराणां कलापिनाम्}% ६

\twolineshloka
{हंसा यत्र नृणां गेहे स्फाटिकेषु नियन्त्रिताः}
{कुर्वन्ति मेघान्नो भीतिं मानसं न स्मरन्ति च}% ७

\twolineshloka
{निरन्तरं शिवस्थाने ध्वस्तं चन्द्रिकया तमः}
{शुक्लकृष्णविभेदो न पक्षयोस्तत्र वै नृणाम्}% ८

\twolineshloka
{तत्र वीरमणी राजा धार्मिकेष्वग्रणीर्महान्}
{राज्यं करोति विपुलं सर्वभोगसमन्वितम्}% ९

\twolineshloka
{तस्य पुत्रो महाशूरो नाम्ना रुक्माङ्गदो बली}
{वनिताभिर्गतो रम्यदेहाभिः क्रीडितुं वनम्}% १०

\twolineshloka
{तासां मञ्जीरसंरावः कङ्कणानां रवस्तथा}
{मनो हरति कामस्य किमन्यस्य कथात्र भोः}% ११

\twolineshloka
{वनं जगाम सुमहत्सुपुष्पनगसंयुतम्}
{सदाशिवकृतस्थानमृतुषट्कैर्विराजितम्}% १२

\twolineshloka
{चम्पका यत्र बहुशः फुल्लकोरकशोभिताः}
{कुर्वन्ति कामिनां तत्र हृच्छयार्तिं विलोकिताः}% १३

\twolineshloka
{चूताः फलादिभिर्नम्रा मञ्जरीकोटिसंयुताः}
{नागाः पुन्नागवृक्षाश्च शालास्तालास्तमालकाः}% १४

\twolineshloka
{कोकिलानां समारावा यत्र च श्रुतिगोचराः}
{सदा मधुपझङ्कार गतनिद्राः सुमल्लिकाः}% १५

\twolineshloka
{दाडिमानां समूहाश्च कर्णिकारैः समन्विताः}
{केतकीकानकीवन्यवृक्षराजिविराजिताः}% १६

\twolineshloka
{तस्मिन्वने प्रमदसंयुतचित्तवृत्तिर्गायन्कलं मधुरवाग्विचिकीर्षयोच्चैः}
{उद्यत्कुचाभिरभितो वनिताभिरागाच्छोभानिधान वपुरुद्गतभीर्विवेश}% १७

\twolineshloka
{काश्चित्तं नृत्यविद्याभिस्तोषयन्ति स्म शोभनम्}
{काश्चिद्गानकलाभिश्च काश्चिद्वाक्चतुरोचितैः}% १८

\twolineshloka
{भ्रूसंज्ञया पराः काश्चित्तोषयामासुरुन्मदाः}
{परिरम्भणचातुर्यैस्तं हृष्टं विदधुः स्त्रियः}% १९

\twolineshloka
{ताभिः पुष्पोच्चयं कृत्वा भूषयामास ताः स्त्रियः}
{वाण्या कोमलया शंसन्रेमे कामवपुर्धरः}% २०

\twolineshloka
{एवं प्रवृत्ते समये राजराजस्य धीमतः}
{प्रायात्तद्वनदेशं स हयः परमशोभनः}% २१

\fourlineindentedshloka
{तं स्वर्णपत्ररचितैकललाटदेशं}
{गङ्गासमं घुसृणकुङ्कुम पिञ्जराङ्गम्}
{गत्यासमं पवनवेगतिरस्करिण्या}
{दृष्ट्वा स्त्रियः परमकौतुकधामदेहम्}% २२

\fourlineindentedshloka
{ऊचुः पतिं कमलमध्यपिशङ्गवर्णा-}
{स्ताम्राधरप्रतिभयाहतविद्रुमाभाः}
{दन्तव्रजप्रमितहास्यसुशोभिवक्त्राः}
{कामस्य बाणनयनादिविमोहनाभाः}% २३

\uvacha{स्त्रिय ऊचुः}

\twolineshloka
{कान्तकोयं महानर्वास्वर्णपत्रैकशोभितः}
{कस्य वा भाति शोभाढ्यो गृहाण स्वबलादिमम्}% २४

\uvacha{शेष उवाच}

\twolineshloka
{तदुक्तं वच आकर्ण्य लीलाललितलोचनः}
{जग्राह हयमेकेन करपद्मेन लीलया}% २५

\twolineshloka
{वाचयित्वा स्वर्णपत्रं स्पष्टवर्णसमन्वितम्}
{जहास महिलामध्ये जगाद वचनं पुनः}% २६

\uvacha{रुक्माङ्गद उवाच}

\twolineshloka
{पृथिव्यां नास्ति मे पित्रा समः शौर्येण च श्रिया}
{तस्मिन्कथं विधत्ते स उत्सेकं रामभूमिपः}% २७

\twolineshloka
{यस्य रक्षां प्रकुरुते सदा रुद्रः पिनाकधृक्}
{यं देवा दानवा यक्षा नमन्ति मणिमौलिभिः}% २८

\twolineshloka
{कुरुताद्वाजिमेधं वै जनको मे महाबलः}
{या त्वेष वाजिशालायां बध्नन्तु मम वै भटाः}% २९

\twolineshloka
{इति वाक्यं समाकर्ण्य महिलास्ता मनोहराः}
{प्रहर्षवदना जाताः कान्तं तु परिरेभिरे}% ३०

\twolineshloka
{गृहीत्वा च हयं पुत्रो राज्ञो वीरमणेर्महान्}
{पुरं पत्नीसमायुक्तो महोत्साहमवीविशत्}% ३१

\twolineshloka
{मृदङ्गध्वनिषु प्रोच्चैराहतेषु समन्ततः}
{बन्दिभिः संस्तुतः प्रागात्स्वपितुर्मन्दिरं महत्}% ३२

\twolineshloka
{तस्मै स कथयामास हयं नीतं रघोः पतेः}
{वाजिमेधाय निर्मुक्तं स्वच्छन्दगतिमद्भुतम्}% ३३

\twolineshloka
{रक्षितं शत्रुसूदेन महाबलसमेतिना}
{तच्छ्रुत्वा वचनं तस्य नृपो वीरमणिर्महान्}% ३४

\twolineshloka
{नातिप्रशंसयामास तत्कर्म सुमहामतिः}
{नीत्वा पुनः समायान्तं चौरस्येव विचेष्टितम्}% ३५

\twolineshloka
{कथयामास जामात्रे शिवायाद्भुतकर्मणे}
{अर्धाङ्गनाधरायाङ्गभूषाय चन्द्रधारिणे}% ३६

\twolineshloka
{तेन सम्मन्त्रयामास नृपो वीरमणिर्महान्}
{पुत्रसृष्टं महत्कर्म विनिन्द्यं महतां मतः}% ३७

\uvacha{शिव उवाच}

\twolineshloka
{राजन्पुत्रेण भवतः कृतं कर्म महाद्भुतम्}
{यो जहार महावाहंरामचन्द्रस्य धीमतः}% ३८

\twolineshloka
{अद्य युद्धं महद्भाति सुरासुरविमोहनम्}
{शत्रुघ्नेन महाराज्ञा वीरकोट्येकरक्षिणा}% ३९

\twolineshloka
{मया यो ध्रियते स्वान्ते जिह्वया प्रोच्यते हि यः}
{तस्य रामस्ययज्ञाङ्गं जहार तव पुत्रकः}% ४०

\twolineshloka
{परमत्र महाँल्लाभो भविष्यतितरां रणे}
{यद्रामचरणाम्भोजं द्रक्ष्यामः स्वीयसेवितम्}% ४१

\twolineshloka
{अत्र यत्नो महान्कार्यो हयस्य परिरक्षणे}
{नयिष्यन्ति बलाद्वाहं मया रक्षितमप्यमुम्}% ४२

\twolineshloka
{तस्मादिमं महाराज राज्येन सह सन्नतः}
{वाजिनं भोजनं दत्वा प्रेक्षस्वाङ्घ्रियुगं ततः}% ४३

\twolineshloka
{इति वाक्यं समाकर्ण्य शिवस्य स नृपोत्तमः}
{उवाच तं सुरेन्द्रादिवन्द्यपादाम्बुजद्वयम्}% ४४

\uvacha{वीरमणिरुवाच}

\twolineshloka
{क्षत्रियाणामयं धर्मो यत्प्रतापस्य रक्षणम्}
{तदसौ क्रान्तुमुद्युक्तः क्रतुना हयसंज्ञिना}% ४५

\twolineshloka
{तस्माद्रक्ष्यः स्वप्रतापो येनकेनापि मानिना}
{यावच्छक्यं कर्म कृत्वा शरीरव्ययकारिणा}% ४६

\twolineshloka
{सर्वं कृतं सुतेनेदं गृहीतोऽश्व पुनर्यतः}
{कोपितं रामभूपालं समयार्हं कुरु प्रभो}% ४७

\twolineshloka
{क्षत्त्रियाणामिदं कर्म कर्तव्यार्हं भवेन्नहि}
{यदकस्माद्रिपोः पादौ प्रणमेद्भयविह्वलः}% ४८

\twolineshloka
{रिपवो विहसन्त्येनं कातरोऽयं नृपाधमः}
{क्षुद्रः प्राकृतवन्नीचो नतवान्भयविह्वलः}% ४९

\twolineshloka
{तस्माद्भवान्यथायोग्यं योद्धव्ये समुपस्थिते}
{यद्विधेयं विचार्यैव कर्तव्यं भक्तरक्षणम्}% ५०

\uvacha{शेष उवाच}

\twolineshloka
{इति वाक्यं समाकर्ण्य चन्द्रचूडोवदद्वचः}
{प्रहसन्मेघगम्भीरवाण्या सम्मोहयन्मनः}% ५१

\twolineshloka
{यदि देवास्त्रयस्त्रिंशत्कोटयः समुपस्थिताः}
{तथापि त्वत्तः केनाश्वो गृह्यते मम रक्षितुः}% ५२

\twolineshloka
{यदि रामः समागत्य स्वात्मानं दर्शयिष्यति}
{तदाहं चरणौ तस्य प्रणमामि सुकोमलौ}% ५३

\twolineshloka
{स्वामिना न हि योद्धव्यं महान नय उच्यते}
{अन्ये वीरास्तृणप्रायाः किञ्चित्कर्तुं न वै क्षमाः}% ५४

\twolineshloka
{तस्माद्युद्ध्यस्व राजेन्द्र रक्षके मयि सुस्थिते}
{को गृह्णाति बलाद्वाहं त्रिलोकी यदि सङ्गता}% ५५

\uvacha{शेष उवाच}

\twolineshloka
{एतद्वचः परं श्रुत्वा चन्द्रचूडस्य भूमिपः}
{जहर्ष मानसेऽत्यन्तं युद्धकर्मणि कौतुकी}% ५६

{॥इति श्रीपद्मपुराणे पातालखण्डे शेषवात्स्यायनसंवादे रामाश्वमेधे वीरमणिपुत्रेण हयग्रहणं नाम एकोनचत्वारिंशत्तमोऽध्यायः॥३९॥}

\dnsub{चत्वारिंशत्तमोऽध्यायः}\resetShloka

\uvacha{शेष उवाच}

\twolineshloka
{सेनाचरा महाराज्ञो महाबलसमन्विताः}
{समागतास्तं पश्यन्तो हयं रामस्य भूपतेः}% १

\twolineshloka
{क्वा सावश्वः केन नीतः कथं वा दृश्यते न सः}
{को गन्ता यमपुर्यां वै वाहं हृत्वा सुमन्दधीः}% २

\twolineshloka
{विलोकयन्तस्तन्मार्गं यावत्सेनाचरा रघोः}
{तावत्प्राप्तो महाराजो महासैन्यपरीवृतः}% ३

\twolineshloka
{पप्रच्छ सेवकान्सर्वान्कुत्राश्वो मम साम्प्रतम्}
{न दृश्यते कथं वाहः स्वर्णपत्रसुशोभितः}% ४

\twolineshloka
{इति तद्वचनं श्रुत्वा सेवकास्ते हयानुगाः}
{प्रोचुर्नाथ मनोवेगो वाहः केनापि कानने}% ५

\twolineshloka
{हृतो न लक्ष्यते तस्मादस्माभिर्मार्गकोविदैः}
{तदत्र यत्नः कर्तव्यो हयप्राप्तिं प्रति प्रभो}% ६

\twolineshloka
{तेषां वचनमाकर्ण्य पप्रच्छ सुमतिं नृपः}
{शत्रुघ्नः शत्रुसंहारकारीमोहनरूपधृक्}% ७

\uvacha{शत्रुघ्न उवाच}

\twolineshloka
{कोऽत्र राजा निवसति कथं वाहस्य सङ्गमः}
{कियद्बलं भूमिपतेर्येन मेऽद्य हृतो हयः}% ८

\uvacha{सुमतिरुवाच}

\twolineshloka
{राजन्देवपुरं ह्येतद्देवेनैव विनिर्मितम्}
{कैलासमिव दुर्गम्यं वैरिसङ्घैः सुसंहतैः}% ९

\twolineshloka
{अस्मिन्वीरमणी राजा महाशूरः प्रतापवान्}
{राज्यं करोति धर्मेण शिवेन परिरक्षितः}% १०

\twolineshloka
{योऽसौ प्रलयकारी स आस्ते भक्त्या वशीकृतः}
{चन्द्रचूडः स्वभक्तस्य पक्षपातं सृजन्सदा}% ११

\twolineshloka
{तस्मादत्र महद्युद्धं गृहीतश्चेद्भविष्यति}
{यत्ताः सन्तः प्रकुर्वन्तु रक्षणं कटकस्य हि}% १२

\twolineshloka
{एवं श्रुत्वा स शत्रुघ्नः सर्वभूपशिरोमणिः}
{सैन्यव्यूहं रचित्वासौ तिष्ठति स्म महायशाः}% १३

\twolineshloka
{अथ तं सुखमासीनं मन्त्रयन्तं सुमन्त्रिणा}
{आजगाम स देवर्षिर्युद्धकौतुकसंयुतः}% १४

\twolineshloka
{तमागतं मुनिं दृष्ट्वा शत्रुघ्नस्तपसां निधिम्}
{अभ्युत्थायासने स्थाप्य मधुपर्कमथार्पयत्}% १५

\twolineshloka
{स्वागतेन च सन्तुष्टं नारदं मुनिसत्तमम्}
{उवाच प्रीणयन्वाचा वाक्यवादविशारदः}% १६

\uvacha{शत्रुघ्न उवाच}

\twolineshloka
{मदीयोऽश्व कुत्र विप्र कथयस्व महामते}
{न लक्ष्यते गतिस्तस्य सेवकैर्मम कोविदैः}% १७

\twolineshloka
{शंस तं येन वा नीतं क्षत्त्रियेण च मानिना}
{कथमत्र हयप्राप्तिर्भविष्यति तपोधन}% १८

\twolineshloka
{इति वाक्यं समाकर्ण्य शत्रुघ्नस्य स नारदः}
{उवाच वीणां रणयन्गायन्रामकथां मुहुः}% १९

\uvacha{नारद उवाच}

\twolineshloka
{एतद्देवपुरं राजन्भूपो वीरमणिर्महान्}
{तत्पुत्रेण वनस्थेन गृहीतस्तव वाजिराट्}% २०

\twolineshloka
{तत्र युद्धं महत्तेऽद्य भविष्यति सुदारुणम्}
{अत्र वीराः पतिष्यन्ति बलशौर्यसमन्विताः}% २१

\twolineshloka
{तस्मादत्र महायत्नात्स्थातव्यं ते महाबल}
{रचय व्यूहरचनां दुर्गमां परसैनिकैः}% २२

\twolineshloka
{जयस्ते भविता राजन्कृच्छ्रेणास्मान्नृपोत्तमात्}
{रामं को नु पराजीयाद्भुवने सकले ह्यपि}% २३

\twolineshloka
{इत्युक्त्वान्तर्दधे विप्रो नभसि स्थितवांस्ततः}
{युद्धं सुदारुणं द्रक्ष्यन्देवदानवयोरिव}% २४

\uvacha{शेष उवाच}

\twolineshloka
{अथ राजा वीरमणिः सर्वशूरशिरोमणिः}
{पटहं घोषितुं स्वीये पुरमध्ये महारवम्}% २५

\twolineshloka
{आह्वयामास सेनान्यं रिपुवीरं महोन्नतम्}
{कथयामास च क्षिप्रं मेघगम्भीरया गिरा}% २६

\uvacha{वीरमणिरुवाच}

\twolineshloka
{सेनानीः पटहस्याज्ञां देहि मे शोभने पुरे}
{तच्छ्रुत्वा मे सुसन्नद्धाः शत्रुघ्नं प्रति यान्तु ते}% २७

\twolineshloka
{इति वाक्यं समाकर्ण्य राज्ञो वीरमणेस्तदा}
{कारयामास पटहं महारवनिनादितम्}% २८

\twolineshloka
{गेहे गेहे च रथ्यायां श्रूयते पटहध्वनिः}
{शत्रुघ्नं यान्तु ये सर्वे वीरा राजपुरे स्थिताः}% २९

\twolineshloka
{ये वै राज्ञः समुल्लङ्घ्य शासनं वीरमानिनः}
{पुत्रा वा भ्रातरो वापि ते वध्याः स्युर्नृपाज्ञया}% ३०

\twolineshloka
{शृण्वन्तु वीराः पुनरप्याह ते पटहे रवम्}
{श्रुत्वा विधीयतामाशु कर्तव्यं मा विलम्बितम्}% ३१

\uvacha{शेष उवाच}

\fourlineindentedshloka
{इति पटहरवं स्वकर्णगोचरं}
{नरवरवीरवरा ययुर्नृपोत्तमम्}
{कनककवचभूषितस्वदेहाः}
{समरमहोत्सव हृष्टचित्तकोशाः}% ३२

\twolineshloka
{केचिद्ययुः शिरस्त्राणं धृत्वा शिरसि शोभनम्}
{कवचेन सुशोभाढ्याः शतकोटिसुशोभिताः}% ३३

\twolineshloka
{रथेन हययुग्मेन मणिकाञ्चनशोभिना}
{ययुस्ते राजसन्देशान्नृवरालयमुन्मदाः}% ३४

\twolineshloka
{केचिन्मतङ्गजैर्मत्तैः केचिद्वाहैः सुशोभनैः}
{ययुर्नपगृहं सर्वे राजसन्देशकारकाः}% ३५

\twolineshloka
{विविक्तस्वर्णकवचशिरस्त्राणसुशोभितः}
{रुक्माङ्गदोऽपि च निजे रथे तिष्ठन्मनोजवे}% ३६

\twolineshloka
{शुभाङ्गदोऽनुजस्तस्य महारत्नमयं दधत्}
{कवचं वपुषि श्रेष्ठं निजं प्रायाद्रणोत्सवे}% ३७

\twolineshloka
{राजभ्राता वीरसिंहः सर्वशस्त्रास्त्रकोविदः}
{ययौ नृपाज्ञया तत्र शासनं भूमिपस्य हि}% ३८

\twolineshloka
{जामेयस्तस्य राज्ञोऽपि बलमित्र इति स्मृतः}
{सन्नद्धः कवची खड्गी जगाम नृपमन्दिरम्}% ३९

\twolineshloka
{सेनानी रिपुवारोऽपि सेनां तां चतुरङ्गिणीम्}
{सज्जां विधाय भूपाय न्यवेदयदथो महान्}% ४०

\twolineshloka
{अथ राजा वीरमणिः सर्वशस्त्रास्त्रपूरितम्}
{मणिसृष्टोच्चचक्रोच्चमारोहत्स्यन्दनोत्तमम्}% ४१

\twolineshloka
{ततो वीरार्णवे शङ्खनिनादश्च समन्ततः}
{श्रूयते कातरान्वीरान्प्रेरयन्निव सङ्गरे}% ४२

\twolineshloka
{भेर्यः समन्ततो जघ्नुः शुभवादकवादिताः}
{अनीकान्यत्र तस्यासन्सङ्ग्रामाय प्रतस्थुषः}% ४३

\twolineshloka
{सर्वे कृतस्वस्त्ययनाः सर्वाभरणभूषिताः}
{सर्वशस्त्रास्त्रसम्पूर्णा ययुः समरमण्डलम्}% ४४

\twolineshloka
{भेरीशङ्खनिनादेन पूरिताश्च नगा गुहाः}
{आकारितुं गतः किं नु तद्रवः स्वर्गसंस्थितान्}% ४५

\twolineshloka
{तस्मिन्कोलाहले वृत्ते राजा वीरमणिर्महान्}
{रणोत्साहेन संयुक्तो ययौ प्रधनमण्डलम्}% ४६

\twolineshloka
{आगत्य संस्थितस्तावद्रथपत्तिसमाकुलम्}
{समुद्र इव तत्स्थानात्प्लावितुं पुरुषानयात्}% ४७

\twolineshloka
{तदागतं बलं दृष्ट्वा रथिभिः शस्त्रकोविदैः}
{कोलाहलीकृतं सर्वमुवाच सुमतिं नृपः}% ४८

\uvacha{शत्रुघ्न उवाच}

\twolineshloka
{समागतो वीरमणिर्मम वाजिधरो बली}
{योद्धुं मां महता भूयः सैन्येन चतुरङ्गिणा}% ४९

\twolineshloka
{कथं युद्धं प्रकर्तव्यं के योत्स्यन्ति बलोत्कटाः}
{तान्सर्वान्दिश मे वीरान्यथा स्याज्जय ईप्सितः}% ५०

\uvacha{सुमतिरुवाच}

\twolineshloka
{स्वामिन्नसौ महाराजो महासैन्यपरीवृतः}
{समागतः स युद्धार्थं शिवभक्तिसमन्वितः}% ५१

\twolineshloka
{साम्प्रतं युद्ध्यतां वीरः पुष्कलः परमास्त्रवित्}
{अन्येपि नीलरत्नाद्या योद्धारो युद्धकोविदाः}% ५२

\twolineshloka
{शिवेन सह योद्धव्यं राज्ञा वा भवतानघ}
{द्वन्द्वयुद्धेन जेतव्यो महाबलपराक्रमः}% ५३

\twolineshloka
{अनेन विधिना राजञ्जयस्तेऽत्र भविष्यति}
{पश्चाद्यद्रोचते स्वामिंस्तत्कुरुष्व महामते}% ५४

\uvacha{शेष उवाच}

\twolineshloka
{इति वाक्यं समाकर्ण्य शत्रुघ्नः परवीरहा}
{सुभटानादिदेशाथ युद्धाय कृतनिश्चयः}% ५५

\twolineshloka
{सर्वैः ससैन्यैर्युद्धार्थं राजभिः शस्त्रकोविदैः}
{यथा स्यान्मे जयः क्षिप्रं यतितव्यं तथा पुनः}% ५६

\twolineshloka
{जयार्थं राघवस्यैव श्रुत्वा ते रणकोविदाः}
{महोत्साहेन संयुक्ता ययुर्योद्धुं तु सैनिकैः}% ५७

{॥इति श्रीपद्मपुराणे पातालखण्डे शेषवात्स्यायनसंवादे रामाश्वमेधे वीरमणिना सह युद्धनिश्चयो नाम चत्वारिंशत्तमोऽध्यायः॥४०॥}

\dnsub{एकचत्वारिंशत्तमोऽध्यायः}\resetShloka

\uvacha{शेष उवाच}

\twolineshloka
{युद्धाय ते सुसन्नद्धाः शत्रुघ्नस्य महाबलाः}
{ययुर्वीरमणेः सैन्यमध्ये शौर्यसमन्विताः}% १

\twolineshloka
{शरान्विमुञ्चमानास्ते भिन्दन्तः सैनिकान्बहून्}
{व्यदृश्यन्त रणान्तःस्थाः शरासनधरा नराः}% २

\twolineshloka
{अनेके निहतास्तत्र गजा मणिमया रथाः}
{भग्ना वाहसमेताश्च दृश्यन्ते रणमण्डले}% ३

\twolineshloka
{विहितं कदनं तेषां श्रुत्वा रुक्माङ्गदो बली}
{रथे मणिमये तिष्ठन्ययौ योद्धुं ससैनिकान्}% ४

\twolineshloka
{शरासने शरान्धास्यन्निषुधी अक्षयौ दधत्}
{शोणनेत्रान्तरो भीमो महाकोपसमन्वितः}% ५

\twolineshloka
{अनेकबाणसंविग्नान्कुर्वञ्छूरान्सहस्रशः}
{हाहाकारं कारयंस्तद्ययौ रुक्माङ्गदो बली}% ६

\twolineshloka
{राजपुत्रः स्वसदृशं बलेन यशसाश्रिया}
{आह्वयामास शत्रुघ्नं भारतिं पुष्कलं बली}% ७

\uvacha{रुक्माङ्गद उवाच}

\twolineshloka
{आगच्छ वीरकर्मा त्वं महाबलपराक्रम}
{मया योद्धुं तु बलिना राजपुत्रेण भास्वता}% ८

\twolineshloka
{किमन्यैस्त्रासितैर्वीर निहतैः कोटिभिर्नरैः}
{मया समं महायुद्धं विधाय जयमाप्नुहि}% ९

\twolineshloka
{इत्युक्तवं तं तरसा प्रहसन्पुष्कलो बली}
{जघान विपुले मध्ये वक्षसस्तीक्ष्णपर्वभिः}% १०

\twolineshloka
{तदमृष्यन्राजपुत्रो महाचापे दधच्छरान्}
{जघान दशभिर्वीरं पुष्कलं वक्षसोऽन्तरे}% ११

\twolineshloka
{उभौ समरसंरब्धावुभावपि जयैषिणौ}
{रेजाते समरे तौ हि कुमारतारकौ यथा}% १२

\twolineshloka
{बाणान्धनुषि सन्धाय दशसङ्ख्यान्महाशितान्}
{अकरोत्पुष्कलो वीरो विरथं राजपुत्रकम्}% १३

\twolineshloka
{चतुर्भिश्चतुरोवाहान्द्वाभ्यां सूतमपातयत्}
{एकेन ध्वजमेतस्य द्वाभ्यां स्यन्दनरक्षकौ}% १४

\twolineshloka
{एकेन हृदि विव्याध राजपुत्रस्य वेगवान्}
{तदद्भुतं कर्म दृष्ट्वा सर्वे वीराः प्रतोषिताः}% १५

\twolineshloka
{सच्छिन्नधन्वा विरथो हताश्वो हतसारथिः}
{अत्यन्तं कोपमापन्नः स्यन्दनं परमाविशत्}% १६

\twolineshloka
{स स्थित्वा स्यन्दनवरे हयरत्नेन भूषिते}
{शरासनं महद्धृत्वा सुदृढं गुणपूरितम्}% १७

\twolineshloka
{उवाच पुष्कलं वीरं रुक्माङ्गद इदं वचः}
{महत्पराक्रमं कृत्वा क्व यास्यसि परन्तप}% १८

\twolineshloka
{पश्य मेऽद्यपराक्रान्तिं यद्बलेन विनिर्मिताम्}
{यत्नात्तिष्ठस्व भो वीर नयामि त्वद्रथं नभः}% १९

\twolineshloka
{इत्युक्त्वा शरमत्युग्रं दधार स्वशरासने}
{मन्त्रयित्वा ततश्चास्त्रं भ्रामकं पौष्कले रथे}% २०

\twolineshloka
{मुमोच निशितं बाणं स्वर्णपङ्खैकशोभितम्}
{तेन बाणेन नीतोऽस्य रथो योजनमात्रकम्}% २१

\twolineshloka
{धृतः कृच्छ्रेण सूतेन रथो बभ्राम भूतले}
{कृच्छ्रेण प्राप्य स्वस्थानं पुष्कलः परमास्त्रवित्}% २२

\twolineshloka
{जगाद वचनं तं वै बाणं बिभ्रच्छरासने}
{स्वर्गं प्राप्नुहि वीराग्र्य सर्वदेवैश्च सेवितम्}% २३

\twolineshloka
{त्वादृशाः पृथिवीयोग्या न भवन्ति नृपोत्तम}
{शतक्रतुसभायोग्यास्तद्गच्छ त्वं सुरालयम्}% २४

\twolineshloka
{इत्युक्त्वा स मुमोचास्त्रमाकाशप्रापकं महत्}
{तेन बाणेन सरथो ययौ खमनुलोमतः}% २५

\twolineshloka
{सर्वांल्लोकानतिक्रामन्ययौ सूर्यस्य मण्डलम्}
{तज्ज्वालया रथो दग्धो हयसूतसमन्वितः}% २६

\twolineshloka
{तत्करैर्दग्धभूयिष्ठ कलेवरः सुदुःखितः}
{पपात चन्द्रचूडं स धृत्वा हृद्यसुखार्दनम्}% २७

\twolineshloka
{भूमौ निपतितस्तत्र करदग्धकलेवरः}
{अत्यन्तं दुःखमापन्नो मुमूर्च्छ रणमण्डले}% २८

\twolineshloka
{तस्मिन्निपतिते भूमौ मूर्च्छिते राजपुत्रके}
{हाहाकारो महानासीत्तत्र सङ्ग्राममूर्धनि}% २९

\twolineshloka
{वैरिणो जयलक्ष्मीं ते प्रापुः पुष्कलमुख्यकाः}
{पलायनपरा जाता वैरिणो हयरक्षकाः}% ३०

\twolineshloka
{तदा पुत्रस्य वै मूर्च्छां दृष्ट्वा वीरमणिर्नृपः}
{प्रायात्समरमध्यस्थं पुष्कलं कोपपूरितः}% ३१

\twolineshloka
{तदा भूमिश्चचालेयं सपर्वतवनोत्तमा}
{शूरा वै हर्षमापन्नाः कातरा भयपीडिताः}% ३२

\twolineshloka
{चापं महद्दधानः स इषुधी अक्षयावपि}
{रोषान्निःश्वासमामुञ्चन्नाह्वयामास वैरिणम्}% ३३

{॥इति श्रीपद्मपुराणे पातालखण्डे शेषवात्स्यायनसंवादे रामाश्वमेधे रुक्माङ्गदपराजय पुष्कलविजयो नाम एकचत्वारिंशत्तमोऽध्यायः॥४१॥}

\dnsub{द्विचत्वारिंशत्तमोऽध्यायः}\resetShloka

\uvacha{शेष उवाच}

\twolineshloka
{आह्वयन्तं महासैन्यवारिधौ पुष्कलं नृपम्}
{समालक्ष्य कपीन्द्रोऽपि हनूमांस्तमधावत}% १

\fourlineindentedshloka
{लाङ्गूलमुद्यम्य विशालदेहं}
{सरावमातत्य पयोदघोषम्}
{रणस्थितान्वीरवरान्कपीन्द्रो}
{जगाम तं वीरमणिं नृपेन्द्रम्}% २

\twolineshloka
{आयान्तं तं हनूमन्तं वीक्ष्य पुष्कल उद्भटः}
{विलोकयामासदृशा वैरिक्रोध सुशोणया}% ३

\twolineshloka
{जगाद तं हनूमन्तं पुष्कलः परमास्त्रवित्}
{मेघगम्भीरया वाचा नादयन्रणमण्डलम्}% ४

\uvacha{पुष्कल उवाच}

\twolineshloka
{कथं त्वं समरे योद्धुमागतोसि महाकपे}
{कियद्बलं स्वल्पमेतद्राज्ञो वीरमणेर्महत्}% ५

\twolineshloka
{यत्र त्रिजगती सर्वा सम्मुखे समुपागता}
{तत्र त्वं लीलया योद्धुं यातुमिच्छसि वा न वा}% ६

\twolineshloka
{कोयं राजा वीरमणिः कियद्बलमथाल्पकम्}
{अत्रागमनमत्युग्रं तव वीर न भाव्यते}% ७

\twolineshloka
{रघुनाथकृपापाङ्गादहं निस्तीर्य दुस्तरम्}
{क्षणान्निर्यामि कीशेन्द्र मा चित्तं कुरु सङ्गरे}% ८

\twolineshloka
{त्वया राक्षसपाथोधिस्तीर्णो रामकृपाव्रजात्}
{तथा रामं सुसंस्मृत्य निस्तरिष्यामि दुस्तरम्}% ९

\twolineshloka
{ये केचिद्दुस्तरं प्राप्य रघुनाथं स्मरन्ति च}
{तेषां दुःखोदधिः शुष्को भविष्यति न संशयः}% १०

\twolineshloka
{तस्माद्व्रज महावीर शत्रुघ्नसविधे बलिन्}
{एष आयामि निर्जित्य भूपं वीरमणिं क्षणात्}% ११

\uvacha{शेष उवाच}

\twolineshloka
{इति धीरां समाकर्ण्य वाणीं पुष्कलभाषिताम्}
{जगाद वचनं भूयः पुष्कलं परवीरहा}% १२

\uvacha{हनुमानुवाच}

\twolineshloka
{पुत्र मा साहसं कार्षीर्भूपं वीरमणिं प्रति}
{एष दाता शरण्यश्च बलशौर्यसमन्वितः}% १३

\twolineshloka
{त्वं बालः स्थविरो भूपोऽखिलशस्त्रास्त्रवित्तमः}
{अनेके विजिताः सङ्ख्ये वीराः शौर्यसुशोभिनः}% १४

\twolineshloka
{जानीहि पार्श्वे तस्य त्वं रक्षितारं सदाशिवम्}
{भक्त्या वशीकृतं स्थाणुं सोमं चैतत्पुरिस्थितम्}% १५

\twolineshloka
{तस्मादहमनेनैव योत्स्ये भूपेन पुष्कल}
{अन्यान्वीरान्विजित्वा त्वं कीर्तिमाप्नुहि पुष्कलाम्}% १६

\uvacha{पुष्कल उवाच}

\twolineshloka
{शिवो भक्त्या वशीकृत्य स्वपुरे स्थापितोऽमुना}
{परमस्याशु हृदयेन तिष्ठति महेश्वरः}% १७

\twolineshloka
{सदाशिवोयमाराध्य परमं स्थानमागतः}
{स रामो मन्मनस्त्यक्त्वा न क्वापि परिगच्छति}% १८

\twolineshloka
{यत्र रामस्तत्र विश्वं सर्वं स्थास्नु चरिष्णु च}
{तस्मादहं जयिष्यामि रणे वीरमणिं नृपम्}% १९

\twolineshloka
{व्रज त्वं समरे योद्धुमन्यान्मानिवरान्नृपान्}
{वीरसिंहमुखान्कीश मच्चिन्तां मा कुरु प्रभो}% २०

\twolineshloka
{वाचमित्थं समाकर्ण्य हनूमान्धीरसेविताम्}
{जगाम समरे योद्धुं वीरसिंहं नृपानुजम्}% २१

\twolineshloka
{लक्ष्मीनिधिः सुतेनास्य शुभाङ्गदसुसंज्ञिना}
{द्वैरथेन प्रयुयुधे महाशस्त्रास्त्रवेदिना}% २२

\twolineshloka
{बलमित्रेण सुमदः स्वप्रतापबलोर्जितः}
{योद्धुं सशस्त्रः सङ्ग्रामं विचचार नृपात्मजः}% २३

\twolineshloka
{आह्वयन्तं नृपं दृष्ट्वा द्वैरथे युद्धकोविदः}
{पुष्कलो रुक्मखचिते रथे तिष्ठन्ययौ हि तम्}% २४

\twolineshloka
{राजा तमागतं दृष्ट्वा पुष्कलं युद्धकोविदम्}
{उवाच निर्भिया वाण्या रणमध्ये सुभाषितः}% २५

\uvacha{वीरमणिरुवाच}

\twolineshloka
{बालमायाहि मां क्रुद्धं सङ्ग्रामे चण्डकोपनम्}
{गच्छ प्राणपरीप्सायै मा युद्धं कुरु मे सह}% २६

\twolineshloka
{त्वादृशान्बालकान्भूपा मादृशाः कृपयन्ति हि}
{प्रहरन्ति न चैतान्वै तस्माद्गच्छ रणाद्बहिः}% २७

\twolineshloka
{यावत्त्वं न मया दृष्टश्चक्षुर्भ्यां तावदुन्मनाः}
{साम्प्रतं त्वां प्रहर्तुं न मनः समभिकाङ्क्षति}% २८

\twolineshloka
{यत्त्वया मत्सुतो बाणैर्भिन्नो मूर्च्छीकृतः पुनः}
{सर्वं मया क्षान्तमद्य तवबालधियो महत्}% २९

\onelineshloka*
{इति वाक्यं समाकर्ण्य पुष्कलो निजगाद तम्}

\uvacha{पुष्कल उवाच}

\onelineshloka
{बालोऽहं त्वं महावृद्धः सर्वशस्त्रास्त्रकोविदः}% ३०

\twolineshloka
{क्षत्रियाणां मतं चैव ये बलाधिक्यसंयुताः}
{त एव वृद्धा भूपाग्र्य न वयोवृद्धतां गताः}% ३१

\twolineshloka
{मया ते मूर्च्छितः पुत्रः सशौर्यबलदर्पितः}
{इदानीं त्वामहं शस्त्रैः पातयिष्यामि सङ्गरे}% ३२

\twolineshloka
{तस्मात्त्वं यत्नतस्तिष्ठ राजन्सङ्ग्राममूर्धनि}
{रामभक्तं न मां कश्चिज्जयतीन्द्रपदे स्थितः}% ३३

\twolineshloka
{इत्थं भाषितमाश्रुत्य पुष्कलस्य नृपाग्रणीः}
{जहास बालं संवीक्ष्य कोपं च व्यदधात्पुनः}% ३४

\twolineshloka
{तं वै कुपितमालक्ष्य भरतात्मज उन्मदः}
{जघान शरविंशत्या राजानं हृदि तीक्ष्णया}% ३५

\twolineshloka
{राजा तानागतान्दृष्ट्वा बाणांस्तेन विमोचितान्}
{चिच्छेद परमक्रुद्धः शरैस्तीक्ष्णैरनेकधा}% ३६

\twolineshloka
{तद्बाणच्छेदनं दृष्ट्वा भारतिः परवीरहा}
{चुकोप हृदयेऽत्यन्तं राजानं च त्रिभिः शरैः}% ३७

\twolineshloka
{विव्याध भाले भूपाल पुत्रः पुष्कलसंज्ञकः}
{तत्र लग्ना विरेजुस्ते त्रिकूटशिखराणि किम्}% ३८


\threelineshloka
{तैर्बाणैर्व्यथितो राजा जघान नवभिः शरैः}
{हृदये पुष्कलं वीरं महाकोपसमन्वितः}
{तैर्वत्सदन्तैर्बह्वस्रं पीतं रामानुजाङ्गजम्}% ३९

\twolineshloka
{सर्पा आशीविषा यद्वत्क्रुद्धास्तद्वपुषि स्थिताः}
{परमं कोपमापन्नः पुष्कलो भूमिपं पुनः}% ४०

\twolineshloka
{बाणानां शतकेनाशु बिभेद शितपर्वणाम्}
{तैर्बाणैः कवचं भिन्नं किरीटः सशिरस्त्रकः}% ४१

\twolineshloka
{रथो धनुर्महत्सज्यं छिन्नं कोपपरिप्लवात्}
{क्षतजेन परिप्लुष्टो बाणभिन्नकलेवरः}% ४२

\twolineshloka
{अन्यं स्यन्दनमारुह्य जगाम भरतात्मजम्}
{धन्योसि वीर रामस्य चरणाब्जमधुव्रत}% ४३

\twolineshloka
{महत्कृतं कर्म तेऽद्य यदहं विरथीकृतः}
{प्राणान्रक्षस्व भो वीर साम्प्रतं मयि युद्ध्यति}% ४४

\twolineshloka
{सुलभा न तव प्राणाः कालरूपे मयि स्थिते}
{इत्युक्त्वा व्यहनद्बाणैरसङ्ख्यैः शस्त्रकोविदः}% ४५

\twolineshloka
{भूमौ दिशि च तद्बाणा नान्यद्दृश्येत तत्र ह}
{अनेके गजसाहस्रा भिन्ना अश्वाः समन्ततः}% ४६

\twolineshloka
{रथारथियुतास्तेन छिन्ना भिन्ना द्विधाकृताः}
{शोणितौघा सरित्तत्र प्रसुस्राव रणाङ्गणे}% ४७

\twolineshloka
{यत्रोन्मदा हि मातङ्गा दृश्यन्ते शैलशृङ्गवत्}
{केशाः शैवाललक्ष्यास्ते मुहुः प्राणिशिरः स्थिताः}% ४८

\twolineshloka
{अनेके पाणयश्छिन्ना वीराणां मुद्रिकाश्रियः}
{दृश्यन्ते अहिवत्तत्र चन्दनादिकरूषिताः}% ४९

\twolineshloka
{शिरांसि च भटाग्र्याणां कच्छपाभां वहन्ति वै}
{मांसानि पङ्का यत्रासन्वीराणां महतां ततः}% ५०

\twolineshloka
{एवं व्यतिकरे वृत्ते योगिन्यः शतशो रणे}
{पपुः पात्रेण रुधिरं प्राणिनां रणपातिनाम्}% ५१

\twolineshloka
{मांसानि बुभुजुस्ता वै हर्षकौतुकसंयुताः}
{पीत्वा तु शोणितं तत्र भक्षित्वा मांसमुन्मदाः}% ५२

\twolineshloka
{ननृतुर्जहसुः प्रोच्चैरुज्जगुः प्रधनाङ्गणे}
{पिशाचास्तत्र समरे प्राणिनां मस्तकानि वै}% ५३

\twolineshloka
{धृत्वा कराभ्यां मत्ताङ्गास्तालवद्वादनोद्यताः}
{शिवास्तत्र महामांसं पतितानां रणाङ्गणे}% ५४

\twolineshloka
{भक्षित्वा व्यनदन्मत्ताः कातराणां भयप्रदाः}
{कातरास्त्राः समापन्ना गताः कुञ्जरकोटरे}% ५५

\twolineshloka
{भक्षिता योगिनीभिस्ते पापिनां क्वापि न स्थितिः}
{एतत्कदनमालक्ष्य स्वसैन्यस्य रथाग्रणीः}% ५६

\twolineshloka
{पुष्कलोऽपि चकारात्र कदनं रणमण्डले}
{भिद्यन्ते गजशीर्षाणि पतन्ति मौक्तिकानि तु}% ५७


\threelineshloka
{दृश्यते लोमभिः पूर्णा ताम्रपर्णीव तन्नदी}
{पुष्कलप्रहिता बाणा नृणामङ्गेषु सङ्गताः}
{कुर्वन्ति प्राणविच्छेदं वीराणामपि सर्वतः}% ५८

\twolineshloka
{सर्वे रुधिरसिक्ताङ्गाः सर्वे च्छिन्ननिजाङ्गकाः}
{दृश्यन्ते किंशुका यद्वत्सुभटाः प्रधनाङ्गणे}% ५९

\twolineshloka
{एतस्मिन्समये क्रुद्धः समाभाष्य महीपतिम्}
{जघान बहुबाणैस्तं रोषपूरपरिप्लुतः}% ६०

\twolineshloka
{तद्बाणवेधभिन्नाङ्गो विशीर्णकवचो नृपः}
{महाबलं तं मन्वानः प्राहरच्छरकोटिभिः}% ६१

\twolineshloka
{तैर्बाणैः कवचान्मुक्तं सुस्राव बहुशोणितम्}
{वपुर्बभूव रुचिरं शरपञ्जरगोचरम्}% ६२

\twolineshloka
{शरपञ्जरमध्यस्थो विह्वलीकृतमानसः}
{शरान्नेतुं च सन्धातुं न क्षमः स च भारतिः}% ६३

\twolineshloka
{रामं स्मृत्वा धनुर्धृत्वा करे सज्जं महद्दृढम्}
{मुमोच बाणान्निशितान्वैरिवृन्दनिवारणान्}% ६४

\twolineshloka
{तैर्बाणैः शरजालं तद्विधूय मुनिपुङ्गव}
{शङ्खं प्रध्माय समरे जगाद गतभीर्नृपम्}% ६५

\uvacha{पुष्कल उवाच}

\twolineshloka
{त्वया कृतं महत्कर्म यन्मां बाणस्य पञ्जरे}
{गोचरं कृतवान्वीर वीरतापनमुद्भटम्}% ६६

\twolineshloka
{वृद्धत्वान्मम मान्योसि साम्प्रतं रणमण्डले}
{पश्यमेऽद्य पराक्रान्तं राजन्वीरमणे महत्}% ६७

\twolineshloka
{बाणत्रयेण भो वीर मूर्च्छितं करवै नहि}
{तर्हि प्रतिज्ञां शृणु वै सर्ववीरविमोहिनीम्}% ६८

\twolineshloka
{गङ्गां प्राप्यापि यो वै तां निन्दित्वा पापहारिणीम्}
{न मज्जति महापापो महामूढविचेष्टितः}% ६९

\twolineshloka
{तस्य पापं ममैवास्तु चेन्न त्वां रणमण्डले}
{पातये मूर्च्छया वीर सन्नद्धो भव भूपते}% ७०

\twolineshloka
{इति वाक्यं समाकर्ण्य पुष्कलस्य नृपोत्तमः}
{चुकोप भृशमुद्विग्नः सन्दधे निशिताञ्छरान्}% ७१

\twolineshloka
{ते शरा हृदयं भित्त्वा गतास्ते भारतेर्महत्}
{पेतुः क्षितावधो यद्वद्रामभक्तिपराङ्मुखाः}% ७२

\twolineshloka
{ततः शरं मुमोचास्मै निशितं वह्निसप्रभम्}
{लक्षीकृत्य महद्वक्षः कपाटतटविस्तृतम्}% ७३

\twolineshloka
{स बाणो भूमिपतिना द्विधा छिन्नः शरेण हि}
{पपात रथमध्ये स रविमण्डलवज्ज्वलन्}% ७४

\twolineshloka
{अपरं बाणमाधत्त मातृभक्तिभवं ततः}
{निधाय पुण्यं सोऽप्येष चिच्छेद महता पुनः}% ७५

\twolineshloka
{तदा खिन्नः स हृदये किङ्कर्तव्यमिति स्मरन्}
{रामं हृदि निजार्तिघ्नं मुमोच परमास्त्रवित्}% ७६

\twolineshloka
{स बाणस्तस्य हृदये लग्न आशीविषोपमः}
{मूर्च्छामप्रापयत्तं वै ज्वलन्सूर्यसमप्रभः}% ७७

\twolineshloka
{ततो हाहाकृतं सर्वं पलायनपरायणम्}
{राज्ञि सम्मूर्च्छिते जाते पुष्कलो जयमाप्तवान्}% ७८

{॥इति श्रीपद्मपुराणे पातालखण्डे शेषवात्स्यायनसंवादे रामाश्वमेधे वीरमणेः पराभवो नाम द्विचत्वारिंशत्तमोऽध्यायः॥४२॥}

\dnsub{त्रिचत्वारिंशत्तमोऽध्यायः}\resetShloka

\uvacha{शेष उवाच}

\twolineshloka
{हनूमान्वीरसिंहं तु समागत्याब्रवीद्वचः}
{तिष्ठ यासि कुतो वीर जेष्यामि त्वां क्षणादिह}% १

\twolineshloka
{एवमुक्तं समाकर्ण्य प्लवगस्य वचो महत्}
{कोपपूरपरिप्लुष्टः कार्मुकं जलदस्वनम्}% २

\twolineshloka
{विनद्य घोरान्निशितान्बाणान्मुञ्चन्बभौ रणे}
{आषाढे जलदस्येव धारासारे मनोहरः}% ३

\twolineshloka
{तान्दृष्ट्वा निशितान्बाणान्स्वदेहे सुविलग्नकान्}
{चुकोप हृदयेऽत्यतं हनूमानञ्जनी सुतः}% ४

\twolineshloka
{मुष्टिना ताडयामास हृदये वज्रसारिणा}
{समुष्टिना हतो वीरः पपात धरणीतले}% ५

\twolineshloka
{मूर्च्छितं तं समालोक्य पितृव्यं स शुभाङ्गदः}
{रुक्माङ्गदोऽपि सम्मूर्च्छां त्यक्त्वागाद्रणमण्डलम्}% ६

\twolineshloka
{बाणान्समभिवर्षन्तौ मेघाविव महास्वनौ}
{कुर्वन्तौ कदनं घोरं प्लवङ्गं प्रति जग्मतुः}% ७

\twolineshloka
{तौ दृष्ट्वा समरे वीरौ समायातौ कपीश्वरः}
{लाङ्गूलेन च संवेष्ट्य सरथौ चापधारकौ}% ८

\twolineshloka
{स्फोटयामास भूदेशे तत्क्षणान्मूर्च्छितावुभौ}
{निश्चेष्टौ समभूतां तौ रुधिरारक्तदेहकौ}% ९

\twolineshloka
{बलमित्रश्चिरं युद्धं विधाय सुमदेन हि}
{मूर्च्छामप्रापयत्तं वै बाणैः सुशितपर्वभिः}% १०

\twolineshloka
{पुष्कलेन क्षणान्नीतो मूर्च्छां चैतन्यवर्जिताम्}
{जयमाप्तं तु कटकं शत्रुघ्नस्य भटार्दनम्}% ११

\twolineshloka
{एतस्मिन्समये साम्बः स्यन्दनं वरमास्थितः}
{विस्फारयन्धनुर्दिव्यमुपाधावद्भटानिमान्}% १२

\twolineshloka
{जटाजूटान्तरगतां चन्द्ररेखां वहन्महान्}
{सर्पाभूषां मनःस्पर्शां दधदाजगवं धनुः}% १३

\twolineshloka
{मूर्च्छितान्स्वजनान्दृष्ट्वा भक्तार्तिघ्नो महेश्वरः}
{योद्धुं प्रायान्महासैन्यैः शत्रुघ्नस्य भटानिमान्}% १४

\twolineshloka
{सगणः सपरीवारः कम्पयन्पृथिवीतलम्}
{भक्तरक्षार्थमागच्छंस्त्रिपुरं तु पुरा यथा}% १५

\twolineshloka
{कोपाच्छोणतरे नेत्रे वहन्प्रलयकारकः}
{पश्यन्वीरान्बहुमतीन्पिनाकी देववन्दितः}% १६

\twolineshloka
{तमागतं महेशानं वीक्ष्य रामानुजो बली}
{जगाम समरे योद्धुं सर्वदेवशिरोमणिम्}% १७

\twolineshloka
{अथागतं तु शत्रुघ्नं रुद्रो वीक्ष्य पिनाकधृक्}
{उवाच परमापन्नः कोपं सगुणचापधृक्}% १८

\twolineshloka
{पुष्कलेन महत्कर्म कृतं रामाङ्घ्रिसेविना}
{मद्भक्तं यो रणे हत्वा गतः समरमण्डलम्}% १९

\twolineshloka
{अद्य क्वास्ति परो वीरः पुष्कलः परमास्त्रवित्}
{तं हत्वा सुखमाप्स्यामि समरे भक्तपीडनम्}% २०

\uvacha{शेष उवाच}

\twolineshloka
{इत्युक्त्वा वीरभद्रं स प्रेषयामास पुष्कलम्}
{याहि त्वं समरे योद्धुं पुष्कलं सेवकार्दनम्}% २१

\twolineshloka
{नन्दिनं प्रेषयामास हनूमन्तं महाबलम्}
{कुशध्वजं प्रचण्डं तु भृङ्गिणं च सुबाहुकम्}% २२

\twolineshloka
{सुमदं चण्डनामानं गणं स्वीयं समादिशत्}
{पुष्कलस्तु समायान्तं वीरभद्रं महागणम्}% २३

\twolineshloka
{महारुद्रस्य संवीक्ष्य योद्धुं प्रायान्महामनाः}
{पुष्कलः पञ्चभिर्बाणैस्ताडयामास संयुगे}% २४

\twolineshloka
{तैर्बाणैः क्षतगात्रस्तु त्रिशूलं स समादिशत्}
{स त्रिशूलं क्षणाच्छित्त्वा व्यगर्जत महाबलः}% २५

\twolineshloka
{छिन्नं स्वीयं त्रिशूलं वै वीक्ष्य रुद्रानुगो बली}
{खट्वाङ्गेन जघानाशु मस्तके भारतिं द्विज}% २६

\twolineshloka
{खट्वाङ्गाभिहतः सोऽथ मुमूर्च्छ क्षणमुद्भटः}
{विहाय मूर्च्छां सद्वीरः पुष्कलः परमास्त्रवित्}% २७

\twolineshloka
{शरैश्चिच्छेद खट्वाङ्गं करस्थं तस्य तत्क्षणात्}
{वीरभद्रः स्वकेच्छिन्ने खट्वाङ्गे करसंस्थिते}% २८

\twolineshloka
{परमक्रोधमापन्नो बभञ्ज रथिनो रथम्}
{भङ्क्त्वा रथं तु वीरस्य पदातिं च विधाय सः}% २९

\twolineshloka
{बाहुयुद्धेन युयुधे पुष्कलेन महात्मना}
{स पुष्कलो रथं त्यक्त्वा चूर्णितं तेन वेगतः}% ३०

\twolineshloka
{मुष्टिना ताडयामास वीरभद्रं महाबलः}
{अन्योन्यं मुष्टिभिर्घ्नन्तावूरुभिर्जानुभिस्तथा}% ३१

\twolineshloka
{परस्परवधोद्युक्तौ परस्परजयैषिणौ}
{एवं चतुर्दिनमभूद्रात्रिं दिवमपीशयोः}% ३२

\twolineshloka
{न कोपि तत्र हीयेत न जीयेत महाबलः}
{पञ्चमे तु दिने वृत्ते वीरभद्रो महाबलः}% ३३

\twolineshloka
{गृहीत्वा नभ उड्डीनो महावीरं तु पुष्कलम्}
{तत्र युद्धं तयोरासीद्देवासुरविमोहनम्}% ३४

\twolineshloka
{मुष्टिना चरणाघातैर्बाहुभिः सुखुरैर्महत्}
{तदात्यन्तं प्रकुपितः पुष्कलो वीरभद्रकम्}% ३५

\twolineshloka
{गृहीत्वा कण्ठदेशे तु ताडयामास भूतले}
{तत्प्रहारेण व्यथितो वीरभद्रो महाबलः}% ३६

\twolineshloka
{गृहीत्वा पुष्कलं पादे जघानास्फालयन्मुहुः}
{ताडयित्वा महीदेशे पुष्कलं सुमहाबलः}% ३७

\twolineshloka
{त्रिशूलेन चकर्ताशु शिरो ज्वलितकुण्डलम्}
{जगर्ज पुष्कलं हत्वा वीरभद्रो महाबलः}% ३८

\twolineshloka
{गर्जता तेन शार्वेण प्रापितास्त्रा समुद्भटाः}
{हाहाकारो महानासीत्पुष्कले पतिते रणे}% ३९

\twolineshloka
{त्रासं प्रापुर्जनाः सर्वे रणमध्ये सुकोविदाः}
{ते शशंशुश्च शत्रुघ्नं पुष्कलं पतितं रणे}% ४०

\twolineshloka
{निहतं वीरभद्रेण महेश्वरगणेन वै}
{इत्याश्रुत्य महावीरः पुष्कलस्य वधं तदा}% ४१

\twolineshloka
{दुःखं प्राप्तो रणेऽत्यतं कम्पमानः शुचा महान्}
{तं दुःखितं च शत्रुघ्नं ज्ञात्वा रुद्रो ऽब्रवीद्वचः}% ४२

\twolineshloka
{शत्रुघ्नं समरे वीरं शोचन्तं पुष्कले हते}
{रे शत्रुघ्न रणे शोकं मा कृथाः सुमहाबल}% ४३

\twolineshloka
{वीराणां रणमध्ये तु पातनं कीर्तये स्मृतम्}
{धन्यो वीरः पुष्कलाख्यो यश्च वै दिनपञ्चकम्}% ४४

\twolineshloka
{युयुधे वीरभद्रेण महाप्रलयकारिणा}
{येन क्षणाद्विनिहतो दक्षो मदपमानकृत्}% ४५

\twolineshloka
{क्षणाद्विनिहता येन दैत्यास्त्रिपुरसैनिकाः}
{तस्माद्युद्ध्स्व राजेन्द्र शोकं त्यक्त्वा महाबल}% ४६

\twolineshloka
{यत्नात्तिष्ठाद्य वीराग्र्य मयि योद्धरि संस्थिते}
{शोकं सन्त्यज्य शत्रुघ्नो वीरश्चुक्रोध शङ्करम्}% ४७

\twolineshloka
{आत्तसज्जधनुर्बाणैः प्रचच्छाद महेश्वरम्}
{ते बाणाः सुरशीर्षण्य वपुषं क्षतविक्षतम्}% ४८

\twolineshloka
{अकुर्वत महच्चित्रं भक्तरक्षार्थमागतम्}
{ते बाणाः शङ्करस्यापि बाणा नभसि संस्थिताः}% ४९

\twolineshloka
{व्याप्यैतत्सकलं विश्वं चित्रकारि मुनेरपि}
{तद्बाणयोर्युद्धबलं वीक्ष्य सर्वत्र मेनिरे}% ५०

\twolineshloka
{प्रलयं लोकसंहारकारकं सर्वमोहकम्}
{आकाशे तु विमानानि संश्रित्य स्वपुरस्थिताः}% ५१

\twolineshloka
{विलोकयितुमागत्य प्रशंसन्ति तयोर्भृशम्}
{अयं लोकत्रयस्यास्य प्रलयोत्पत्तिकारकः}% ५२

\twolineshloka
{असावपि महाराज रामचन्द्रस्य चानुजः}
{किमिदं भविता को वा जेष्यति क्षितिमण्डले}% ५३

\twolineshloka
{पराजयं वा को वीरः प्राप्स्यते रणमूर्धनि}
{एवमेकादशाहानि वृत्तं युद्धं परस्परम्}% ५४

\twolineshloka
{द्वादशे दिवसे प्राप्ते मुमोचास्त्रं नराधिपः}
{ब्रह्मसंज्ञं महादेवं हन्तुं क्रोधसमन्वितः}% ५५

\twolineshloka
{सविज्ञाय महास्त्रं तन्मुक्तं शत्रुघ्नवैरिणा}
{हसन्नप्यपिबत्तेन मुक्तं ब्रह्मशिरो महत्}% ५६

\twolineshloka
{अत्यन्तं विस्मयं प्राप्य किं कर्तव्यमतः परम्}
{एवं विचारयुक्तस्य हृदये ज्वलनोपमम्}% ५७

\twolineshloka
{शरं वै निचखानाशु देवदेव शिरोमणिः}
{तेन बाणेन शत्रुघ्नो मूर्च्छितो रणमण्डले}% ५८

\twolineshloka
{हाहाभूतमभूत्सर्वं कटकं भटसेवितम्}
{वीराः सर्वे रुद्रगणैः पातिताः पृथिवीतले}% ५९

\twolineshloka
{सुबाहुसुमदोन्मुख्याः स्वबाहुबलदर्पिताः}
{पतितं मूर्च्छया वीक्ष्य शत्रुघ्नं शरपीडितम्}% ६०

\twolineshloka
{पुष्कलं तु रथे स्थाप्य सेवकैः परिरक्षितुम्}
{हनूमानागतो योद्धुं शिवं संहारकारकम्}% ६१

\twolineshloka
{श्रीरामस्मरणं योधान्स्वीयान्विप्र प्रहर्षितान्}
{प्रकुर्वन्रोषितस्तीव्रं लाङ्गूलं च प्रकम्पयन्}% ६२

{॥इति श्रीपद्मपुराणे पातालखण्डे शेषवात्स्यायनसंवादे रामाश्वमेधे पुष्कलशत्रुघ्नपराजयो नाम त्रिचत्वारिंशत्तमोऽध्यायः॥४३॥}

\dnsub{चतुश्चत्वारिंशत्तमोऽध्यायः}\resetShloka

\uvacha{शेष उवाच}

\twolineshloka
{आगत्य सविधे रुद्रं समराङ्गणमूर्धनि}
{जगाद हनुमान्वीरः सञ्जिहीर्षुः सुराधिपम्}% १

\uvacha{हनूमानुवाच}

\twolineshloka
{त्वं यदाचरसे रुद्र धर्मस्य प्रतिकूलनम्}
{तस्मात्त्वां शास्तुमिच्छामि रामभक्तवधोद्यतम्}% २

\twolineshloka
{मया पुरा श्रुतं देव ऋषिभिर्बहुधोदितम्}
{रघुनाथपदस्मारी नित्यं रुद्रः पिनाकभृत्}% ३

\twolineshloka
{तत्सर्वं तु मृषा जातं शत्रुघ्नं प्रति युध्यतः}
{पुष्कलो मे हतः शूरः शत्रुघ्नोऽपि विमूर्च्छितः}% ४

\twolineshloka
{तस्मात्त्वां पातयाम्यद्य त्रैलोक्यप्रलयोद्यतम्}
{यत्नात्तिष्ठस्व भोः शर्व रामभक्तिपराङ्मुखः}% ५

\uvacha{शेष उवाच}

\twolineshloka
{इत्युक्तवन्तं प्लवगं प्रोवाच स महेश्वरः}
{धन्योऽसि वीरवर्यस्त्वं भवान्वदति नो मृषा}% ६

\twolineshloka
{मत्स्वामी रामचन्द्रोऽयं सुरासुरनमस्कृतः}
{तदश्वमानयामास शत्रुघ्नः परवीरहा}% ७

\twolineshloka
{तद्रक्षार्थं समायातस्तद्भक्त्या तु वशीकृतः}
{यथाकथञ्चिद्भक्तोऽसौ रक्ष्यः स्वात्मा इति स्थितिः}% ८

\twolineshloka
{रघुनाथः कृपां कृत्वा विलोकय तु निस्त्रपम्}
{मां स्वभक्त सुदुःखेन किञ्चित्कोपं दधन्महान्}% ९

\uvacha{शेष उवाच}

\twolineshloka
{एवं वदति चण्डीशे हनूमान्कुपितो भृशम्}
{शिलामादाय महतीं ताडयामास तद्रथम्}% १०

\twolineshloka
{शिलया ताडितस्तस्य रथः शकलतां गतः}
{ससूतः सहयः केतुपताकाभिः समन्वितः}% ११

\twolineshloka
{नभःस्था देवताः सर्वाः प्रशशंसुः कपीश्वरम्}
{धन्योसि प्लवगाधीश महत्कर्म त्वया कृतम्}% १२

\twolineshloka
{श्रीशिवं विरथं दृष्ट्वा नन्दी तं समुपाद्रवत्}
{उवाच श्रीमहादेवं मत्पृष्ठं गम्यतामिति}% १३

\twolineshloka
{वृषस्थितं तु भूतेशं हनूमान्कुपितो भृशम्}
{शिलामुत्पाट्य तरसा प्राहनद्धृदये तदा}% १४

\twolineshloka
{तदाहतो भूतपतिः शूलं तीक्ष्णं समाददे}
{जाज्वल्यमानं त्रिशिखं वह्निज्वालासमप्रभम्}% १५

\twolineshloka
{आयातं तन्महद्दृष्ट्वा शूलं प्रज्वलनप्रभम्}
{हस्ते गृहीत्वा तरसा बभञ्ज तिलशः क्षणात्}% १६

\twolineshloka
{भग्ने त्रिशूले तरसा कपीन्द्रेण क्षणाच्छिवः}
{शक्तिं करे समाधत्त सर्वलोहविनिर्मिताम्}% १७

\twolineshloka
{सा शक्तिः शिवनिर्मुक्ता हृदये तस्य धीमतः}
{लग्ना क्षणादभूत्तत्र विक्लवः प्लवगाधिपः}% १८

\twolineshloka
{क्षणाच्च तद्व्यथां तीर्त्वा गृहीत्वा वृक्षमुल्बणम्}
{ताडयामास हृदये महाव्यालविभूषिते}% १९

\twolineshloka
{ताडितास्तेन वीरेण फणीन्द्रास्त्रा समागताः}
{इतस्ततस्ते तं मुक्त्वा गताः पातालमुज्जवाः}% २०

\twolineshloka
{शिवस्तस्मिन्नगे मुक्ते वक्षसि स्वे निरीक्ष्य च}
{कुपितो व्यदधाद्घोरं मुसलं करयुग्मके}% २१

\twolineshloka
{हतोसि गच्छ सङ्ग्रामात्पलाय्य प्लवगाधम}
{एष ते प्राणहन्ताहं मुसलेन क्षणादिह}% २२

\twolineshloka
{मुसलं वीक्ष्य निर्मुक्तं शिवेन कुपितेन वै}
{कीशस्तद्वञ्चयामास महावेगाद्धरिं स्मरन्}% २३

\twolineshloka
{मुसलं तत्पपाताधः शिवमुक्तं महायसम्}
{विदार्य पृथिवीं सर्वां जगाम च रसातलम्}% २४

\twolineshloka
{तदा प्रकुपितोऽत्यतं हनूमान्रामसेवकः}
{गृहीत्वा पर्वतं हस्ते ताडयामास वक्षसि}% २५

\twolineshloka
{स यावत्पर्वतं छेत्तुं मतिं चक्रे सतीपतिः}
{तावद्धतः कपीन्द्रेण शालेन बहुशाखिना}% २६

\twolineshloka
{तमपिच्छेत्तुमुद्युक्तो यावत्तावच्छिलाहतः}
{शिलास्ता भेदितुं स्वान्तं चकार मृड उद्यतः}% २७

\twolineshloka
{तावद्वृष्टिं चकारायं शिलाभिर्नगपर्वतैः}
{लाङ्गूलेन च संवेष्ट्य ताडयत्येष भूतपम्}% २८

\twolineshloka
{शिलाभिः पर्वतैर्वृक्षैः पुच्छास्फोटेन भूरिशः}
{नन्दी प्राप्तो महात्रासं चन्द्रोऽपि शकलीकृतः}% २९

\twolineshloka
{अत्यन्तं विह्वलो जातो महेशानः प्रकोपनः}
{क्षणेक्षणे प्रहारेण विह्वलं कुर्वतं भृशम्}% ३०

\twolineshloka
{जगाद प्लवगाधीशं धन्योसि रघुपानुग}
{महत्कर्म कृतं तेऽद्य यत्तेहं सुप्रतोषितः}% ३१

\twolineshloka
{न दानेन न यज्ञेन नाल्पेन तपसा ह्यहम्}
{सुलभोऽस्मि महावेग तस्मात्प्रार्थय मे वरम्}% ३२

\uvacha{शेष उवाच}

\twolineshloka
{एवं ब्रुवन्तं तं दृष्ट्वा हनूमान्निजगाद तम्}
{प्रहसन्निर्भिया वाण्या महेशानं सुतोषितम्}% ३३

\uvacha{हनूमानुवाच}

\twolineshloka
{रघुनाथप्रसादेन सर्वं मेऽस्ति महेश्वर}
{तथापि याचे हि वरं त्वत्तः समरतोषितात्}% ३४

\twolineshloka
{एष पुष्कलसंज्ञो नः समरे पतितो हतः}
{तथैव रामावरजः शत्रुघ्नो मूर्च्छितो रणे}% ३५

\twolineshloka
{अन्ये च वीरा बहवः पतिताः शरविक्षताः}
{मूर्च्छिताः पतिताः केचित्तान्रक्षस्व गणैः सह}% ३६

\twolineshloka
{यथा चैतान्महाभूता वेतालाश्च पिशाचकाः}
{न हरन्ति न खादन्ति श्वशृगालादयस्तथा}% ३७

\twolineshloka
{एतेषां वपुषो भेदो न भवेत्त्वं तथाचर}
{यावदिन्द्रगणं जित्वा न यामि द्रोणपर्वतम्}% ३८

\twolineshloka
{तत्रस्था औषधीर्वापि नीत्वा संस्थापितान्भटान्}
{जीवयामि बलात्सर्वांस्तावत्त्वं रक्ष सर्वशः}% ३९

\twolineshloka
{एष गच्छामि तं नेतुं द्रोणं पर्वतसत्तमम्}
{यस्मिन्वसन्त्योषधयः प्राणिसञ्जीवनङ्कराः}% ४०

\twolineshloka
{एतद्वचः समाकर्ण्य तथेति निजगाद तम्}
{याहि शीघ्रं नगं तं तु रक्षामि त्वद्भटान्मृतान्}% ४१

\twolineshloka
{तच्छ्रुत्वा वाक्यमीशस्य जगाम द्रोणपर्वतम्}
{द्वीपान्सर्वानतिक्रम्य जगाम क्षीरसागरम्}% ४२

\twolineshloka
{अत्र तु स्वगणैश्चायं रक्षति स्म शिवो महान्}
{श्मशानं तद्गणैः स्वीयैर्महाबलपराक्रमैः}% ४३

\twolineshloka
{हनूमान्द्रोणमासाद्य द्रोणं नाम महागिरिम्}
{लाङ्गूले तं निधायाशु प्रतस्थे रणमण्डलम्}% ४४

\twolineshloka
{तं नेतुमुद्यते विप्र चकम्पे स च पर्वतः}
{कम्पमानं तु तं दृष्ट्वा तत्पाला देवतागणाः}% ४५

\twolineshloka
{हाहेति कृत्वा प्रोचुस्ते किमिदं भविता गिरौ}
{को ह्येनं नयते वीरो महाबलपराक्रमः}% ४६

\twolineshloka
{एवं कृत्वा सुराः सर्वे संहता ददृशुः कपिम्}
{मुञ्चैनमिति तं प्रोच्य जघ्नुः शस्त्रास्त्रकोटिभिः}% ४७

\twolineshloka
{तान्सर्वान्निघ्नतो दृष्ट्वा हनूमान्कुपितो भृशम्}
{जघान तान्क्षणाद्वीरः शक्रः सर्वासुरान्यथा}% ४८

\twolineshloka
{केचित्पदाहतास्तत्र केचित्करविमर्दिताः}
{लाङ्गूलेन हताः केचित्केचिच्छृङ्गेण चाहताः}% ४९

\twolineshloka
{सर्वे ते नाशमापन्नाः क्षणात्कीशेन ताडिताः}
{केचिन्निपतिता भूमौ रुधिरेण परिप्लुताः}% ५०

\twolineshloka
{केचित्कीशभयात्त्रस्ता जग्मुः शक्रं सुराधिपम्}
{क्षतेन च परिप्लुष्टा रुधिरक्षतदेहिनः}% ५१

\twolineshloka
{तान्दृष्ट्वा भयसंविग्नान्रुधिरेण परिप्लुतान्}
{सुराञ्जगाद विमनाः शक्रः सर्वसुरोत्तमः}% ५२

\twolineshloka
{कथं यूयं भयत्रस्ताः कथं रुधिरविप्लुताः}
{केन दैत्येन निहता राक्षसेनाधमेन वा}% ५३

\twolineshloka
{सर्वं शंसत मे तथ्यं यथा ज्ञात्वा व्रजामि तम्}
{निहत्य बद्ध्वा चायामि युष्मद्घातकमुन्मदम्}% ५४

\twolineshloka
{इति वाक्यं समाकर्ण्य तुरासाहं सुरोत्तमाः}
{जगदुर्दीनया वाचा सुरासुरनमस्कृतम्}% ५५

\uvacha{देवा ऊचुः}

\twolineshloka
{इहागत्य न जानीमः कश्चिद्वानररूपधृक्}
{नेतुं द्रोणं समुद्युक्तो लाङ्गूले वेष्ट्य तं गिरिम्}% ५६

\twolineshloka
{गन्तुं कृतमतिस्तावद्वयं सर्वे सुसंहताः}
{युद्धं चक्रुः सुसन्नद्धाः सर्वशस्त्रास्त्रवर्षिणः}% ५७

\twolineshloka
{तेन सर्वे वयं युद्धे निर्जिता बलशालिना}
{अनेके निहतास्तत्र भूमौ पेतुः सुरोत्तमाः}% ५८

\twolineshloka
{वयं तु बहुभिः पुण्यैर्जीविता इह चागताः}
{शोणितेन सुसिक्ताङ्गाः क्षतपीडासमन्विताः}% ५९

\twolineshloka
{एतद्वाक्यं समाकर्ण्य सुराणां स पुरन्दरः}
{आदिदेश सुरान्सर्वान्महाबलसमन्वितान्}% ६०

\twolineshloka
{यात महाद्रोणगिरिं कपिं बद्धुं महाबलम्}
{बद्ध्वा नयत यूयं वै सुराणां रणपातकम्}% ६१

\twolineshloka
{इत्याज्ञप्ता ययुस्ते वै द्रोणं पर्वतसत्तमम्}
{यत्रास्ते बलवान्वीरो हनूमान्कपिसत्तमः}% ६२

\twolineshloka
{गत्वा ते प्राहरन्सर्वे हनूमन्तं महाबलम्}
{हनूमता ते निहता मुष्टिभिः करताडनैः}% ६३

\twolineshloka
{पतितास्ते क्षणात्तत्र रुधिरक्षतविग्रहाः}
{अन्ये पलायनपरा जग्मुस्ते त्रिदिवेश्वरम्}% ६४

\twolineshloka
{तच्छ्रुत्वा कुपितः शक्रः सर्वानमरसत्तमान्}
{आदिदेश महावीरं वानरेन्द्रं सुरोत्तमः}% ६५

\twolineshloka
{तदाज्ञप्ता ययुस्ते वै यत्र कीशेश्वरो बली}
{तान्सर्वानागतान्दृष्ट्वा जगाद कपिसत्तमः}% ६६

\twolineshloka
{मायां तु वीराः समरे संहर्तारं हि मां बलात्}
{नेष्यामि युष्मानधुना संयमिन्याः पुरोऽन्तिके}% ६७

\twolineshloka
{इत्युक्ता अपि ते सर्वे सन्नद्धाः प्राहरन्कपिम्}
{शस्त्रास्त्रैर्बहुधा मुक्तैर्महाबलसमन्विताः}% ६८

\twolineshloka
{केचिच्छूलैः परशुभिः केचित्खड्गैश्च पट्टिशैः}
{मुसलैः शक्तिभिः केचित्क्रोधेन कलुषीकृताः}% ६९

\twolineshloka
{स आहतोऽमरवरैर्विविधैरायुधैर्बली}
{शिलाभिस्ताञ्जघानाशु सर्वानमरसत्तमान्}% ७०

\twolineshloka
{केचित्पलाय्य आहुस्ते गताः शक्रसमीपकम्}
{तदुक्तं वाक्यमाकर्ण्य भयं प्राप सुराधिपः}% ७१

\twolineshloka
{बृहस्पतिं सुराध्यक्षं मन्त्रिणं स्वर्गवासिनाम्}
{पप्रच्छ सविधे गत्वा नत्वा सुरगुरुं वरम्}% ७२

\uvacha{इन्द्र उवाच}

\twolineshloka
{कोऽसौ यो वानरो द्रोणं नेतुं स्वामिन्समागतः}
{येन मे निहता वीरा अमराः शस्त्रधारिणः}% ७३

\uvacha{शेष उवाच}

\twolineshloka
{एतच्छ्रुत्वा तु तद्वाक्यमुक्तमाङ्गिरसो महान्}
{जगाद भयसंविग्नं तुरासाहं सुराधिपम्}% ७४

\uvacha{बृहस्पतिरुवाच}

\twolineshloka
{यो रावणमहन्सङ्ख्ये कुम्भकर्णमदीदहत्}
{येन ते वैरिणः सर्वे हतास्तस्यैव सेवकः}% ७५

\twolineshloka
{येन लङ्का सत्रिकूटा निर्दग्धा पुच्छवह्निना}
{अक्षश्च निहतो येन हनूमन्तमवेहि तम्}% ७६

\twolineshloka
{तेन सर्वे विनिहता द्रोणार्थमयमुद्यतः}
{हयमेधं महाराजः करोति बलिसत्तमः}% ७७

\twolineshloka
{तस्याश्वं शिवभक्तस्तु नृपो वीरमणिर्महान्}
{जहार तत्र समभूद्रणं सुरविमोहनम्}% ७८

\twolineshloka
{शिवेन निहताः सङ्ख्ये वीरा रामस्य भूरिशः}
{तान्वै जीवयितुं द्रोणं नेष्यत्येव महाबलः}% ७९

\twolineshloka
{नायं वर्षशतैर्जेयो भवता बलसंयुतः}
{तस्मात्प्रसादय कपिं देहि तत्रत्यमौषधम्}% ८०

{॥इति श्रीपद्मपुराणे पातालखण्डे शेषवात्स्यायनसंवादे रामाश्वमेधे द्रोणगिरौ देवानां पराजयो नाम चतुश्चत्वारिंशत्तमोऽध्यायः॥४४॥}

\dnsub{पञ्चचत्वारिंशत्तमोऽध्यायः}\resetShloka

\uvacha{शेष उवाच}

\twolineshloka
{गुरुभाषितमाकर्ण्य वृषपर्वरिपुः स्वराट्}
{ज्ञात्वा रामस्य कार्यार्थमागतं पवनात्मजम्}% १

\twolineshloka
{भयं तत्याज मनसि वानरात्समुपस्थितम्}
{जहर्ष चित्ते च भृशं वाचस्पतिमुवाच ह}% २

\uvacha{इन्द्र उवाच}

\twolineshloka
{कथं कार्यं सुराधीश द्रोणोऽयं नीयते यदि}
{देवानां जीवनं भूयः कथं स्यादिति मे वद}% ३

\twolineshloka
{इदानीं पवनोद्भूतं प्रसादय यथातथम्}
{रामः प्रीतिं परां याति देवानां च सुखं भवेत्}% ४

\twolineshloka
{देवाधिपस्य वचनं श्रुत्वा वाचस्पतिस्तदा}
{शक्रं तु पुरतः कृत्वा सर्वदेवैः परीवृतम्}% ५

\twolineshloka
{जगाम तत्र यत्रास्ते हनूमान्निर्भयः कपिः}
{गर्जति प्रसभं जित्वा सुरान्सर्वान्सुखासिनः}% ६

\twolineshloka
{ते गत्वा सन्निधौ तस्य बृहस्पतिपुरोगमाः}
{पेतुस्ते चरणौ नत्वा समीरतनुजस्य हि}% ७

\twolineshloka
{बृहस्पतिश्च तं वीरं जगाद प्रेरितोऽमुना}
{सुराधीशेन लोकस्य गुरुणा वदतां वरः}% ८

\twolineshloka
{अजानद्भिः कृतं कर्म देवैस्तव पराक्रमम्}
{श्रीरामचरणस्य त्वं सेवकोऽसि महामते}% ९

\twolineshloka
{किमर्थमयमारम्भः कथमत्र समागमः}
{तत्करिष्यामहे सर्वे सन्नतास्तव भाषितम्}% १०

\twolineshloka
{रोषं त्यक्त्वा कृपां कृत्वा देवाधीशं विलोकय}
{पवनात्मज दैत्यानां भयङ्करवपुर्दधत्}% ११

\uvacha{शेष उवाच}

\twolineshloka
{इत्थं भाषितमाकर्ण्य देवानां स गुरोर्वचः}
{उवाच देवान्सकलान्गुरुं चैव महयशाः}% १२

\twolineshloka
{राज्ञो वीरमणेः सङ्ख्ये हताः शर्वेण भूरिशः}
{भटास्तान्वै जीवयितुं द्रोणं नेष्यामि पर्वतम्}% १३

\twolineshloka
{तं ये निवारयिष्यन्ति स्ववीर्यबलदर्पिताः}
{तान्नेष्यामि क्षणादेव यमस्य सदनं प्रति}% १४

\twolineshloka
{तस्माद्वदत मे यूयं द्रोणं वाथ तदौषधम्}
{येन सञ्जीवयिष्यामि मृतान्वीरान्रणाङ्गणे}% १५

\uvacha{शेष उवाच}

\twolineshloka
{इति वाक्यं समाकर्ण्य वायुसूनोर्महात्मनः}
{ते सर्वे प्रणतिं गत्वा ददुः सञ्जीवनौषधम्}% १६

\twolineshloka
{ते प्रहृष्टा भयं त्यक्त्वा सुराः स्वर्गौकसः स्वयम्}
{ययुः सुरपतिं कृत्वा पुरः सौख्य समन्विताः}% १७

\twolineshloka
{हनुमान्भेषजं तत्तु समादायागतो रणम्}
{स्तुतः सर्वैः सुरगणैर्महाकर्मसमुत्सुकैः}% १८

\twolineshloka
{तमागतं हनूमन्तं वीक्ष्य सर्वेऽपि वैरिणः}
{साधुसाधुप्रशंसन्तमद्भुतं मेनिरे कपिम्}% १९

\fourlineindentedshloka
{कपिः समागत्य महामुदायुतः}
{पुरो भटं पुष्कलमागतं मृतम्}
{शिवेन संरक्षितमुग्रमण्डले}
{श्रीरामचित्तं सविधे जगाम ह}% २०

\twolineshloka
{सुमतिं च समाहूय मन्त्रिणं महतां मतम्}
{उवाच जीवयाम्यद्य सर्वान्वीरान्रणे मृतान्}% २१

\twolineshloka
{एवमुक्त्वा भेषजं तत्पुष्कलस्य महोरसि}
{शिरः कायेन सन्धाय जगाद वचनं शुभम्}% २२

\twolineshloka
{यद्यहं मनसा वाचा कर्मणा राघवं पतिम्}
{जानामि तर्हि एतेन भेषजेनाशु जीवतु}% २३

\twolineshloka
{इति वाक्यं यदा वक्ति तावत्पुष्कल उत्थितः}
{रणाङ्गणेऽदशद्रोषाद्दन्तान्वीरशिरोमणिः}% २४

\twolineshloka
{क्व गतो वीरभद्रोऽसौ मां सम्मूर्च्छ्य रणाङ्गणे}
{सद्योऽहं पातयाम्येनं क्वास्ति मे धनुरुत्तमम्}% २५

\twolineshloka
{इति तं भाषमाणं वै प्राह वीरं कपीन्द्रकः}
{धन्योऽसि वीर यद्भूयो वदस्येनं रणाङ्गणे}% २६

\twolineshloka
{त्वं हतो वीरभद्रेण रघुनाथप्रसादतः}
{पुनः सञ्जीवितोऽस्येहि शत्रुघ्नं याम मूर्च्छितम्}% २७

\twolineshloka
{इत्युक्त्वा प्रययौ तत्र सङ्ग्रामवरमूर्धनि}
{श्वसन्नास्ते स शत्रुघ्नः शिवबाणप्रपीडितः}% २८

\twolineshloka
{तत्र गत्वा समीपं तच्छत्रुघ्नस्य महात्मनः}
{निधाय भेषजं तस्य वक्षसि श्वासमागते}% २९

\twolineshloka
{उवाच हनुमांस्तं वै जीव शत्रुघ्नसत्तम}
{मूर्च्छितोऽसि रणे कस्मान्महाबलपराक्रम}% ३०

\twolineshloka
{यद्यहं ब्रह्मचर्यं च जन्मपर्यन्तमुद्यतः}
{पालयामि तदा वीरः शत्रुघ्नो जीवतु क्षणात्}% ३१

\twolineshloka
{उक्तमात्रेण तेनेदं जीवितः क्षणमात्रतः}
{क्व शिवः क्व शिवो यातो विहायरणमण्डलम्}% ३२

\twolineshloka
{अनेके निहताः सङ्ख्ये श्रीरुद्रेण पिनाकिना}
{ते सर्वे जीविता वीराः कपीन्द्रेण महात्मना}% ३३

\twolineshloka
{तदा सर्वे सुसन्नद्धा रोषपूरितमानसाः}
{स्वेस्वे रथे स्थिताः शत्रून्प्रययुः क्षतविग्रहाः}% ३४

\twolineshloka
{पुष्कलो वीरभद्रं तु चण्डं चैव कुशध्वजः}
{नन्दिनं हनुमान्वीरः शत्रुघ्नः सङ्गरे शिवम्}% ३५

\twolineshloka
{धनुर्विस्फारयन्तं तं शत्रुघ्नं बलिनां वरम्}
{सङ्ग्रामे शिवमाहूय तिष्ठन्तं प्रययौ नृपः}% ३६

\twolineshloka
{राजा वीरमणिर्वीरः शत्रुघ्नः समरे बली}
{अन्योन्यं चक्रतुर्युद्धं मुनिविस्मयकारकम्}% ३७

\twolineshloka
{राज्ञा च वीरमणिना रथा भग्नाः शताधिकाः}
{शत्रुघ्नस्य नरेन्द्रस्य तिलशः क्षणतो द्विज}% ३८

\twolineshloka
{तदा प्रकुपितोऽत्यन्तं शत्रुघ्नो रणमण्डले}
{आग्नेयास्त्रं मुमोचामुं दग्धुं सैन्यसमन्वितम्}% ३९

\twolineshloka
{दाहकं तन्महद्दृष्ट्वा महास्त्रं शत्रुमोचितम्}
{अत्यन्तं कुपितो राजा वारुणास्त्रं समाददे}% ४०

\twolineshloka
{वारुणास्त्रेण शीतार्तं वीक्ष्य रामानुजो बली}
{वायव्यास्त्रं मुमोचास्मै तेन वायुर्महानभूत्}% ४१

\twolineshloka
{वायुना संहता मेघा ययुस्ते सर्वतोदिशम्}
{इतस्ततो गताः सर्वे सैन्यं तत्सुखितं बभौ}% ४२

\twolineshloka
{सैन्ये पवनपीडार्ते नृपो वीरमतिर्महान्}
{पर्वतास्त्रं रिपूद्धारि जग्राह च शरासने}% ४३

\twolineshloka
{पर्वतैः स्तम्भितो वायुर्न चासर्पत सङ्गरे}
{तद्वीक्ष्य रामावरजो वज्रास्त्रं तु समाददे}% ४४

\twolineshloka
{वज्रास्त्रेण हताः सर्वे नगास्तु तिलशः कृताः}
{चूर्णतां प्रापुरेतस्मिन्रणे वीरवरार्चिते}% ४५

\twolineshloka
{वज्रास्त्रेण विदीर्णाङ्गा वीराः शोणितशोभिताः}
{बभूवुः समरप्रान्ते चित्रं समभवद्रणम्}% ४६

\twolineshloka
{तदा प्रकुपितोऽत्यन्तं राजा वीरमणिर्महान्}
{ब्रह्मास्त्रं चाप आधत्त वैरिदाहकमद्भुतम्}% ४७

\twolineshloka
{शत्रुघ्नः शरमादाय सस्मार सुमनोहरम्}
{अस्त्रं तद्योगिनीदत्तं सर्ववैरिविमोहनम्}% ४८

\twolineshloka
{ब्रह्मास्त्रं तत्करभ्रष्टमागतं वैरिणं प्रति}
{तावच्छत्रुघ्ननाम्ना तु तन्मुक्तं मोहनास्त्रकम्}% ४९

\twolineshloka
{मोहनास्त्रेण तद्ब्राह्मं द्विधाछिन्नं क्षणादिह}
{लग्नं राज्ञो हृदि क्षिप्रं मूर्च्छां सम्प्रापयन्नृपम्}% ५०

\twolineshloka
{ते बाणाः शतशो मुक्ताः शत्रुघ्नेन महीभृता}
{सर्वेपि मूर्च्छिता वीरा गणा रुद्रस्य ये पुनः}% ५१

\twolineshloka
{शिवस्य चरणोपस्थे मूढाः पेतुर्महीतले}
{तदा शिवः प्रकुपितो रथे तिष्ठन्ययौ नृपम्}% ५२

\twolineshloka
{शिवेन सहसा योद्धुं समायातो रणाङ्गणे}
{शत्रुघ्नः सज्जमात्तज्यं धनुः कृत्वा व्ययुद्ध्यत}% ५३

\twolineshloka
{तयोः समभवद्युद्धं घोरं वैरिविदारणम्}
{शस्त्रास्त्रैर्बहुधामुक्तैरादीपित दिगन्तरम्}% ५४

\twolineshloka
{अस्त्रप्रत्यस्त्रसङ्घातैस्ताडनप्रतिताडनैः}
{देवानामपि दैत्यानां नैतादृग्रणमण्डलम्}% ५५

\twolineshloka
{तदा व्याकुलितोऽत्यन्तं शत्रुघ्नः शिवसङ्गरे}
{सस्मार स्वामिनं तत्र पावनेरुपदेशतः}% ५६

\twolineshloka
{हा नाथ भ्रातरत्युग्रः शिवः प्राणापहारणम्}
{करोति धनुरुद्यम्य त्रायस्व रणमण्डले}% ५७

\twolineshloka
{अनेके दुःखपाथोधिं तीर्णा राम तवाख्यया}
{मामप्युद्धर दुःखस्थं रामराम कृपानिधे}% ५८

\twolineshloka
{इत्थं वक्ति यदा तावद्वीक्षितो रणमण्डले}
{नीलोत्पलदलश्यामो रामो राजीवलोचनः}% ५९

\twolineshloka
{मृगशृङ्गं कटौ धृत्वा दीक्षितं वपुरुद्वहन्}
{तं दृष्ट्वा विस्मयं प्राप शत्रुघ्नः समराङ्गणे}% ६०

{॥इति श्रीपद्मपुराणे पातालखण्डे शेषवात्स्यायनसंवादे रामाश्वमेधे श्रीरामसमागमो नाम पञ्चचत्वारिंशत्तमोऽध्यायः॥४५॥}

\dnsub{षट्चत्वारिंशत्तमोऽध्यायः}\resetShloka

\uvacha{शेष उवाच}

\twolineshloka
{आगतं वीक्ष्य श्रीरामं शत्रुघ्नः प्रणतार्तिहम्}
{भ्रातरं सकलाद्दुःखान्मुक्तोऽभूद्द्विजसत्तम}% १

\twolineshloka
{हनूमान्वीक्ष्य विभ्रान्तो रामस्य चरणौ मुदा}
{ववन्दे भक्तरक्षार्थमागतं निजगाद च}% २

\twolineshloka
{स्वामिंस्तवैतद्युक्तं तु स्वभक्तपरिपालनम्}
{यत्सङ्ग्रामे जितं सर्वं पाशबद्धममोचयः}% ३

\twolineshloka
{वयं त्विदानीं धन्या वै यद्द्रक्ष्यामो भवत्पदे}
{जेष्यामोऽरीन्क्षणादेव त्वत्कृपातो रघूद्वह}% ४

\uvacha{शेष उवाच}

\twolineshloka
{स्थाणुस्तदागतं रामं योगिनां ध्यानगोचरम्}
{पतित्वा पादयोर्विप्र जगाद प्रणताभयम्}% ५

\twolineshloka
{एकस्त्वं पुरुषः साक्षात्प्रकृतेः पर ईर्यसे}
{यः स्वांशकलया विश्वं सृजस्यवसि हंसि च}% ६

\twolineshloka
{अरूपस्त्वमशेषस्य जगतः कारणं परम्}
{एक एव त्रिधारूपं गृह्णासि कुहकान्वितः}% ७

\twolineshloka
{सृष्टौ विधातृरूपेण पालने स्वयमास च}
{प्रलये जगतः साक्षादहं शर्वाख्यतां गतः}% ८

\twolineshloka
{तव यत्परमेशस्य हयमेधक्रतुक्रिया}
{ब्रह्महत्यापनोदाय तद्विडम्बनमद्भुतम्}% ९

\twolineshloka
{यत्पादशौचममलं गङ्गाख्यं शिरसोऽन्तरा}
{वहामि पापशान्त्यर्थं तस्य ते पातकं कुतः}% १०

\twolineshloka
{मया बह्वपकाराय कृतं कर्म तव स्फुटम्}
{क्षम्यतां तत्कृपालो हि भवतो व्यवधायकम्}% ११

\twolineshloka
{किं करोमि मया सत्यपालनार्थमिदं कृतम्}
{जानन्प्रभावं भवतो भक्तरक्षार्थमागतः}% १२

\twolineshloka
{असौ पुरा उज्जयिन्यां महाकालनिकेतने}
{स्नात्वा शिप्राख्य सरिति तपस्तेपे महाद्भुतम्}% १३

\twolineshloka
{ततः प्रसन्नो जातोऽहं जगाद भूमिपं प्रति}
{याचस्वेति महाराज स वव्रे राज्यमद्भुतम्}% १४

\twolineshloka
{मया प्रोक्तं देवपुरे तव राज्यं भविष्यति}
{यावद्रामहयः पुर्यामागमिष्यति याज्ञिकः}% १५

\twolineshloka
{तावत्प्रभृत्यहं स्थास्ये तव रक्षार्थमुद्यतः}
{एतद्दत्तवरो राम किं करोमि स्वसत्यतः}% १६

\twolineshloka
{घृणितोऽस्म्यधुना राज्ञा सपुत्रपशुबान्धवः}
{हयं समर्प्यते पादसेवां राजा विधास्यति}% १७

\uvacha{शेष उवाच}

\twolineshloka
{इति वाक्यं समाकर्ण्य महेशस्य रघूत्तमः}
{उवाच धीरया वाचा कृपया पूर्णलोचनः}% १८

\uvacha{राम उवाच}

\twolineshloka
{देवानामयमेवास्ति धर्मो भक्तस्य पालनम्}
{त्वया साधुकृतं कर्म यद्भक्तो रक्षितोऽधुना}% १९

\twolineshloka
{ममासि हृदये शर्व भवतो हृदये त्वहम्}
{आवयोरन्तरं नास्ति मूढाः पश्यन्ति दुर्धियः}% २०

\twolineshloka
{ये भेदं विदधत्यद्धा आवयोरेकरूपयोः}
{कुम्भीपाकेषु पच्यन्ते नराः कल्पसहस्रकम्}% २१

\twolineshloka
{ये त्वद्भक्तास्त एवासन्मद्भक्ता धर्मसंयुताः}
{मद्भक्ता अपि भूयस्या भक्त्या तव नतिङ्कराः}% २२

\uvacha{शेष उवाच}

\twolineshloka
{इत्थं भाषितमाकर्ण्य शर्वो वीरमणिं नृपम्}
{मूर्च्छितं जीवयामास करस्पर्शादिना प्रभुः}% २३

\twolineshloka
{अन्यानपि सुतानस्य मूर्च्छिताञ्छरपीडितान्}
{जीवयामास स मृडः समर्थः प्रभुरीश्वरः}% २४

\twolineshloka
{सज्जं विधाय तं भूपं श्रीरामपदयोर्नतिम्}
{कारयामास भूतेशः पुत्रपौत्रैः परीवृतम्}% २५

\twolineshloka
{धन्यो राजा वीरमणिर्यो ददर्श रघूत्तमम्}
{योगिभिर्योगनिष्ठाभिर्दुष्प्रापमयुतायुतैः}% २६

\twolineshloka
{ते नत्वा रघुनाथं तं कृतार्थी कृतविग्रहाः}
{ब्रह्मादिभिः पूज्यतमा अभूवन्द्विजसत्तम}% २७

\twolineshloka
{शत्रुघ्न हनुमद्भ्यां च पुष्कलादिभिरुद्भटैः}
{परिष्टुताय रामाय ददौ राजा हयोत्तमम्}% २८

\twolineshloka
{राज्येन सहितं सर्वं सपुत्रपशुबान्धवम्}
{शर्वेण प्रेरितः प्रादाद्भूपो वीरमणिस्तदा}% २९

\twolineshloka
{ततो रामो नुतः सर्वैर्वैरिभिर्निजसेवकैः}
{शत्रुघ्नादिभिरत्यन्तमुत्सुकैश्च विशेषतः}% ३०

\twolineshloka
{रथे मणिमये तिष्ठन्बभूव स तिरोहितः}
{अन्तर्हिते रामभद्रे सर्वे प्रापुः सुविस्मयम्}% ३१

\twolineshloka
{मा जानीहि मनुष्यं तं रामं लोकैकवन्दितम्}
{जले स्थले च सर्वत्र वर्तते संस्थितः सदा}% ३२

\twolineshloka
{ततो वीरा अलं हृष्टा अन्योन्यं परिरेभिरे}
{तूर्यमङ्गलवादित्रैः सुमहानुत्सवोऽभवत्}% ३३

\twolineshloka
{ततो मुक्तो हयः सर्वैर्वीरैः शस्त्रास्त्रकोविदैः}
{सर्वैरनुगतः प्रीतैर्विस्मयेन समन्वितैः}% ३४

\twolineshloka
{शर्वः सत्यप्रतिज्ञश्च तमनुज्ञाप्य सेवकम्}
{श्रीरामं शरणं प्रोच्य याहि लोकैकदुर्ल्लभम्}% ३५

\twolineshloka
{स्वयमन्तर्हितस्तत्र प्रलयोत्पत्तिकारकः}
{कैलासमगमच्छर्वः सेवकैः परिशोभितः}% ३६

\twolineshloka
{भूपो वीरमणिर्ध्यायञ्छ्रीरामचरणोदजम्}
{शत्रुघ्नेन ययौ साकं बलिना बलसंयुतः}% ३७

\twolineshloka
{एतद्रामस्य चरितं ये शृण्वन्ति नरोत्तमाः}
{तेषां संसारजं दुःखं न भविष्यति कर्हिचित्}% ३८

{॥इति श्रीपद्मपुराणे पातालखण्डे शेषवात्स्यायनसंवादे रामाश्वमेधे हयप्रस्थानं नाम षट्चत्वारिंशत्तमोऽध्यायः॥४६॥}

\dnsub{सप्तचत्वारिंशत्तमोऽध्यायः}\resetShloka

\uvacha{शेष उवाच}

\twolineshloka
{हयो गतो हेमकूटं भारतान्ते ततो द्विज}
{अनेकभटसाहस्रै रक्षितो बद्धचामरः}% १

\twolineshloka
{यो वै विस्तरतो दैर्घ्याद्योजनानां समं ततः}
{अयुतेन सुशृङ्गैश्च राजतैः काञ्चनादिभिः}% २

\twolineshloka
{तत्रोद्यानं महच्छ्रेष्ठं पादपैः परिशोभितम्}
{शालैस्तालैस्तमालैश्च कर्णिकारैः समन्ततः}% ३

\twolineshloka
{हिन्तालैर्नागपुन्नागैः कोविदारैः सबिल्वकैः}
{चम्पकैर्बकुलैर्मेघैर्मदनैः कुटजादिभिः}% ४

\twolineshloka
{जातिकाभिर्यूथिकाभिर्नवमालिकया तथा}
{आम्रैर्माधवद्राक्षाभिर्दाडिमैः शोभितं वनम्}% ५

\twolineshloka
{अनेकपक्षिसङ्घुष्टं भ्रमरैर्निनदीकृतम्}
{मयूरकेकारवितं सर्वर्तुसुखदं हयः}% ६

\twolineshloka
{प्रविवेश स शत्रुघ्नो मनोवेगसमन्वितः}
{स्वर्णपत्रं विशाले स्वे भाले बिभ्रन्मनोहरम्}% ७

\twolineshloka
{गच्छतस्तस्य वाहस्य हयमेधक्रतोस्तदा}
{अकस्मादभवच्चित्रं तच्छृणुष्व द्विजोत्तम}% ८

\twolineshloka
{गात्रस्तम्भोऽभवत्तस्य न चचाल पथिस्थितः}
{हेमकूटइवाचाल्यो बभूव हयसत्तमः}% ९

\twolineshloka
{तदा तद्रक्षकाः सर्वे कशाघातान्वितेनिरे}
{तदाहतेऽपि न ययौ स्तब्धगात्रो हयोत्तमः}% १०

\twolineshloka
{शत्रुघ्नं सविधे गत्वा चुक्रुशुर्वाहरक्षकाः}
{स्वामिन्वयं न जानीमः किमभूद्धयसत्तमे}% ११

\twolineshloka
{गच्छतो वाहवर्यस्य मनोवेगस्य भूपते}
{आकस्मिकोऽभवत्तस्य गात्रस्तम्भो महामते}% १२

\twolineshloka
{कशाभिस्ताडितोऽस्माभिः परं तत्र चचाल न}
{एवं विचार्य यत्कर्म तत्कुरुष्व नृपोत्तम}% १३

\twolineshloka
{तदा विस्मयमापन्नो भूपतिः सह सैनिकैः}
{जगाम सहितः सर्वैर्हयस्य महतोऽन्तिके}% १४

\twolineshloka
{पुष्कलो बाहुना धृत्वा चरणौ तस्य भूतलात्}
{उत्पाटयामास तदा परं नो चेलतुस्ततः}% १५

\twolineshloka
{बलेन बलिनाक्रान्तो नाकम्पत हयस्तदा}
{हनूमांस्तं समुद्धर्तुं मतिं चक्रे महामनाः}% १६

\twolineshloka
{लाङ्गूलेन समावेष्ट्य बलेन बलिनां वरः}
{आचकर्ष बलाद्वाहं न चचाल तथापि सः}% १७

\twolineshloka
{तदोवाच कपिश्रेष्ठो हनूमान्विस्मयान्वितः}
{शत्रुघ्नं बलिनां श्रेष्ठं वीराणां परिशृण्वताम्}% १८

\twolineshloka
{मया द्रोणो लाङ्गुलेन लीलयोत्पाटितोऽधुना}
{परमत्र महाश्चर्यं कम्पते न हयोऽल्पकः}% १९

\twolineshloka
{दृष्टमत्र निदानं हि वीरैर्बलिभिरुद्धतैः}
{आकृष्टोऽपि न च स्थानाच्चचाल तिलमात्रतः}% २०

\twolineshloka
{कपिभाषितमाकर्ण्य शत्रुघ्नो विस्मयान्वितः}
{सुमतिं मन्त्रिणां श्रेष्ठमुवाच वदतां वरः}% २१

\uvacha{शत्रुघ्न उवाच}

\twolineshloka
{मन्त्रिन्किमभवद्वाहे स्तम्भनं वपुषोऽनघ}
{कोऽत्रोपायो विधेयः स्याद्येन वाहगतिर्भवेत्}% २२

\uvacha{सुमतिरुवाच}

\twolineshloka
{स्वामिन्कश्चिन्मुनिर्मृग्योऽखिलज्ञानविचक्षणः}
{देशोद्भवमहं जाने प्रत्यक्षं न परोक्षजम्}% २३

\uvacha{शेष उवाच}

\twolineshloka
{इति वाक्यं समाकर्ण्य सुमतेर्धर्मकोविदः}
{अन्वेषयामास मुनिं सेवकैः सह शोभनम्}% २४

\twolineshloka
{ते सर्वे सर्वतो गत्वा मुनिं धर्मविदं भटाः}
{व्यालोकयन्तः सर्वत्र न चापश्यन्मुनीश्वरम्}% २५

\twolineshloka
{एकस्त्वनुचरो विप्र गतो योजनमात्रतः}
{पूर्वस्यां दिशि चोद्युक्तः पश्यति स्म महाश्रमम्}% २६

\twolineshloka
{यत्र निर्वैरिणः सर्वे पशवो जनतास्तथा}
{गङ्गास्नानहताशेषकिल्बिषाः सुमनोहराः}% २७

\twolineshloka
{यत्र केचित्तपः श्रेष्ठं कुर्वन्ति स्म हुताशनैः}
{धूमैरधोमुखाः पत्रैर्वायुभिः स्वोदरम्भराः}% २८

\twolineshloka
{यत्राग्निहोत्रजो धूमः पवित्रयति सर्वदा}
{अनेकमुनिसंहृष्टो मुक्तपत्रलतोत्तमः}% २९

\twolineshloka
{तमाश्रमं मुनेर्ज्ञात्वा शौनकस्य मनोहरम्}
{न्यवेदयन्नृपायासौ विस्मयाविष्टचेतसे}% ३०

\twolineshloka
{तच्छ्रुत्वा हर्षितोऽत्यन्तं शत्रुघ्नः सह सेवकैः}
{हनूमत्पुष्कलाद्यैश्च सयुतोऽगात्तदाश्रमम्}% ३१

\twolineshloka
{तत्र वीक्ष्य मुनिश्रेष्ठं सम्यग्घुतहुताशनम्}
{प्रणम्य दण्डवत्तस्य चरणौ पापहारिणौ}% ३२

\twolineshloka
{तमागतं नृपं ज्ञात्वा शत्रुघ्नं बलिनां वरम्}
{अर्घ्यपाद्यादिकं चक्रे प्रीतस्तद्दर्शनादभूत्}% ३३

\twolineshloka
{सुखोपविष्टं विश्रान्तं नृपं प्राह मुनीश्वरः}
{किमर्थमटनं देव महत्पर्यटनं तव}% ३४

\twolineshloka
{त्वादृशाः पृथिवीं सर्वां नृपा वै न भ्रमन्ति चेत्}
{तदा दुष्टजनाः साधून्बाधन्ते विगतज्वरान्}% ३५

\twolineshloka
{कथयस्व महीपाल शत्रुघ्न बलिनां वर}
{सर्वं शुभायनो भूयात्तव पर्यटनादिकम्}% ३६

\uvacha{शेष उवाच}

\twolineshloka
{इत्युक्तवन्तं भूदेवं प्रत्युवाच महीश्वरः}
{गद्गद स्वरया वाण्या हर्षित स्वीयविग्रहः}% ३७

\uvacha{शत्रुघ्न उवाच}

\twolineshloka
{अकस्मादभवच्चित्रं रामाश्वस्य मनोहृतः}
{नातिदूरे त्वदावासात्तच्छृणुष्व विदांवर}% ३८

\twolineshloka
{उद्याने तव शोभाढ्ये यदृच्छातो हयो गतः}
{तत्प्रान्ते तस्य वाहस्य गात्रस्तम्भोऽभवत्क्षणात्}% ३९

\twolineshloka
{तदा मे बलिनो वीराः पुष्कलाद्या मदोत्कटाः}
{बलादाचकृषुर्वाहं न चचाल तथाप्यसौ}% ४०

\twolineshloka
{अस्मानपारदुःखाब्धौ मग्नान्प्रतितरिः स्मृतः}
{दैवाद्दृष्टः सुभाग्यैस्त्वं कथयस्व निदानकम्}% ४१

\uvacha{शेष उवाच}

\twolineshloka
{एवं पृष्टो मुनिवरः क्षणं दध्यौ महामतिः}
{ततः कारणसंयुक्तं विचारेण दधद्धयम्}% ४२

\twolineshloka
{क्षणात्तज्ज्ञानतां प्राप्य विस्मयोत्फुल्ललोचनः}
{जगाद स महीपालं दुःखितं संशयान्वितम्}% ४३

\uvacha{शौनक उवाच}

\twolineshloka
{शृणु राजन्प्रवक्ष्यामि हयस्तम्भस्य कारणम्}
{यच्छ्रुत्वा मुच्यते दुःखादतिचित्रकथानकम्}% ४४

\twolineshloka
{गौडदेशे महारण्ये कावेरीतीरभूषिते}
{वाडवः सात्वको नाम्ना चचार परमं तपः}% ४५

\twolineshloka
{एकाहं पयसः प्राशी दिनैकं वायुभक्षकः}
{दिनैकं तु निराहार एवं त्रिदिनमुन्नयेत्}% ४६

\twolineshloka
{एवं व्रते प्रवृत्तस्य कालः सर्वक्षयङ्करः}
{जग्राह स्वस्य दंष्ट्रायां मृतिं प्राप महाव्रती}% ४७

\twolineshloka
{विमाने सर्वशोभाढ्ये सर्वरत्नविभूषिते}
{अप्सरोभिः सह क्रीडन्ययौ मेरोः शिखास्थितौ}% ४८

\twolineshloka
{जम्बूनाममहावृक्षस्तत्र सेव्यरसोऽभवत्}
{नदी जाम्बवती संज्ञा स्वर्णद्रवसमन्विता}% ४९

\twolineshloka
{तस्यां मुनयइच्छाभिः क्रीडन्ते कुतुकान्विताः}
{अनेकतपसा पुण्याः सर्वसौख्यसमन्विताः}% ५०

\twolineshloka
{तत्रासौ स्वेच्छया क्रीडन्नप्सरोभिर्मुदान्वितः}
{प्रतीपमाचरत्तेषां स्वाभिमानमदोद्धतः}% ५१

\twolineshloka
{ततः शप्तः स मुनिभी राक्षसो भव दुर्मुखः}
{ततोऽतिदुःखितः प्राह मुनीन्विद्यातपोधनान्}% ५२

\twolineshloka
{अनुगृह्णन्तु मां सर्वे विप्रा यूयं कृपालवः}
{तदा तैरनुगृहीतो यदा रामहयं भवान्}% ५३

\twolineshloka
{स्तम्भयिष्यति वेगेन ततो रामकथाश्रुतिः}
{पश्चान्मुक्तिर्भवित्री ते शापादस्मात्सुदारुणात्}% ५४

\twolineshloka
{स प्रोक्तो मुनिभिर्देवो राक्षसत्वमितः प्रभो}
{स्तम्भयामास रामाश्वं मोचयानघकीर्तनैः}% ५५

{॥इति श्रीपद्मपुराणे पातालखण्डे शेषवात्स्यायनसंवादे रामाश्वमेधे शापकीर्तनं नाम सप्तचत्वारिंशत्तमोऽध्यायः॥४७॥}

\dnsub{अष्टचत्वारिंशत्तमोऽध्यायः}\resetShloka

\uvacha{शेष उवाच}

\twolineshloka
{इति प्रोक्तं तु मुनिना संश्रुत्य परवीरहा}
{विस्मयं मानयामास हृदि शौनकमब्रवीत्}% १

\uvacha{शत्रुघ्न उवाच}

\twolineshloka
{कर्मणो गहना वार्ता यया सात्वकनामधृत्}
{दिवं प्राप्तोऽपि महता कर्मणा राक्षसीकृतः}% २

\twolineshloka
{स्वामिन्वद महर्षे त्वं कर्मणां स्वगतिर्यथा}
{येन कर्मविपाकेन यादृशं नरकं भवेत्}% ३

\uvacha{शौनक उवाच}

\twolineshloka
{धन्योसि राघवश्रेष्ठ यत्ते मतिरियं शुभा}
{जानन्नपि हितार्थाय लोकानां त्वं ब्रवीषि भोः}% ४

\twolineshloka
{कथयामि विचित्राणां कर्मणां विविधा गतीः}
{ताः शृणुष्व महाराज यच्छ्रुत्वा मोक्षमाप्नुयात्}% ५

\twolineshloka
{परवित्तं परापत्यं कलत्रं पारकं च यः}
{बलात्कारेण गृह्णाति भोगबुद्ध्या च दुर्मतिः}% ६

\twolineshloka
{कालपाशेन सम्बद्धो यमदूतैर्महाबलैः}
{तामिस्रे पात्यते तावद्यावद्वर्षसहस्रकम्}% ७

\twolineshloka
{तत्र ताडनमुद्धूताः कुर्वन्ति यमकिङ्कराः}
{पापभोगेन सन्तप्तस्ततो योनिं तु शौकरीम्}% ८

\twolineshloka
{तत्र भुक्त्वा महादुःखं मानुषत्वं गमिष्यति}
{रोगादिचिह्नितं तत्र दुर्यशो ज्ञापकं स्वकम्}% ९

\twolineshloka
{भूतद्रोहं विधायैव केवलं स्वकुटुम्बकम्}
{पुष्णाति पापनिरतः सोऽन्धतामिस्रके पतेत्}% १०

\twolineshloka
{ये नरा इह जन्तूनां वधं कुर्वन्ति वै मृषा}
{ते रौरवे निपात्यन्ते भिद्यन्ते रुरुभी रुषा}% ११

\twolineshloka
{यः स्वोदरार्थे भूतानां वधमाचरति स्फुटम्}
{महारौरवसंज्ञे तु पात्यते स यमाज्ञया}% १२

\twolineshloka
{यो वै निजं तु जनकं ब्राह्मणं द्वेष्टि पापकृत्}
{कालसूत्रे महादुष्टे योजनायुतविस्तृते}% १३

\twolineshloka
{यावन्ति पशुरोमाणि गवां द्वेषं करोति यः}
{तावद्वर्षसहस्राणि पच्यते यमकिङ्करैः}% १४

\twolineshloka
{यो भूमौ भूपतिर्भूत्वा दण्डायोग्यं तु दण्डयेत्}
{करोति ब्राह्मणस्यापि देहदण्डं च लोलुपः}% १५

\twolineshloka
{स सूकरमुखैर्दुष्टैः पीड्यते यमकिङ्करैः}
{पश्चाद्दुष्टासु योनीषु जायते पापमुक्तये}% १६

\twolineshloka
{ब्राह्मणानां गवां ये तु द्रव्यं वृत्तं तथाल्पकम्}
{वृत्तिं वा गृह्णते मोहाल्लुम्पन्ति स्वबलान्नराः}% १७

\twolineshloka
{ते परत्रान्धकूपे च पात्यन्ते च महार्दिताः}
{योऽन्नं स्वयमुपाहृत्य मधुरं चात्तिलोलुपः}% १८

\twolineshloka
{न देवाय न सुहृदे ददाति रसनापरः}
{स पतत्येव नरके कृमिभोजनसंज्ञिते}% १९

\twolineshloka
{अनापदि नरो यस्तु हिरण्यादीन्यपाहरेत्}
{ब्रह्मस्वं वा महादुष्टे सन्दंशे नरके पतेत्}% २०

\twolineshloka
{यः स्वदेहं प्रपुष्णाति नान्यं जानाति मूढधीः}
{स पात्यते तैलतप्ते कुम्भीपाकेऽतिदारुणे}% २१

\twolineshloka
{यो नागम्यां स्त्रियं मोहाद्योषिद्भावाच्च कामयेत्}
{तं तया किङ्कराः सूर्म्या परिरम्भं च कुर्वते}% २२

\twolineshloka
{ये बलाद्वेदमर्यादां लुम्पन्ति स्वबलोद्धताः}
{ते वैतरण्यां पतिता मांसशोणितभक्षकाः}% २३

\twolineshloka
{वृषलद्यं यः स्त्रियं कृत्वा तया गार्हस्थ्यमाचरेत्}
{पूयोदे निपतत्येव महादुःखसमन्वितः}% २४

\twolineshloka
{ये दम्भमाश्रयन्ते वै धूर्ता लोकस्य वञ्चने}
{वैशसे नरके मूढाः पतन्ति यमताडिताः}% २५

\twolineshloka
{ये सवर्णां स्त्रियं मूढा रेतः स्वं पाययन्ति च}
{रेतःकुल्यासु ते पापा रेतःपानेषु तत्पराः}% २६

\twolineshloka
{ये चौरा वह्निदा दुष्टा गरदा ग्रामलुण्ठकाः}
{सारमेयादने ते वै पात्यन्ते पातकान्विताः}% २७

\twolineshloka
{कूटसाक्ष्यं वदत्यद्धा पुरुषः पापसम्भृतः}
{परकीयं तु द्रव्यं यो हरति प्रसभं बली}% २८

\twolineshloka
{सोऽवीचिनरके पापी अवाग्वक्त्रः पतत्यधः}
{तत्र दुःखं महद्भुक्त्वा पापिष्ठां योनिमाव्रजेत्}% २९

\twolineshloka
{यो नरो रसनास्वादात्सुरां पिबति मूढधीः}
{तं पाययन्ति लोहस्य रसं धर्मस्य किङ्कराः}% ३०

\twolineshloka
{यो गुरूनवमन्येत स्वविद्याचारदर्पितः}
{स मृतः पात्यते क्षारनरकेऽधोमुखः पुमान्}% ३१

\twolineshloka
{विश्वासघातं कुर्वन्ति ये नरा धर्मनिष्कृताः}
{शूलप्रोते च नरके पात्यन्ते बहुयातने}% ३२

\twolineshloka
{पिशुनो यो नरान्सर्वानुद्वेजयति वाक्यतः}
{दन्दशूके च पतितो दन्दशूकैः स दश्यते}% ३३

\twolineshloka
{एवं राजन्ननेके वै नरकाः पापकारिणाम्}
{पापं कृत्वा प्रयान्त्येते पीडां यान्ति सुदारुणाम्}% ३४

\twolineshloka
{यैर्न श्रुता रामकथा न परोपकृतिः कृता}
{तेषां सर्वाणि दुःखानि भवन्ति नरकान्तरे}% ३५

\twolineshloka
{अत्र यस्य सुखं स्वर्गे भूयात्तस्य इतीर्यते}
{ये दुःखिनो रोगयुता नरकादागताश्च ते}% ३६

\uvacha{शेष उवाच}

\twolineshloka
{एतच्छ्रुत्वा महीपालः कम्पमानः क्षणे क्षणे}
{पप्रच्छ भूयस्तं विप्रं सर्वसंशयनुत्तये}% ३७

\twolineshloka
{तत्तत्पापस्य चिह्नानि कथयस्व महामुने}
{केन पापेन किं चिह्नं भूलोके उपजायते}% ३८

\twolineshloka
{इति श्रुत्वा तु तद्वाक्यं मुनिः प्रोवाच भूपतिम्}
{शृणु राजन्प्रवक्ष्यामि चिह्नानि पापकारिणाम्}% ३९

\uvacha{शौनक उवाच}

\twolineshloka
{सुरापः श्यामदन्तश्च नरकान्ते प्रजायते}
{अभक्ष्यभक्षकारी च जायते गुल्मकोदरः}% ४०

\twolineshloka
{उदक्यावीक्षितं भुक्त्वा जायते कृमिलोदरः}
{श्वमार्जारादिसंस्पृष्टं भुक्त्वा दुर्गन्धिमान्भवेत्}% ४१

\twolineshloka
{अनिवेद्य सुरादिभ्यो भुञ्जानो जायते नरः}
{उदरे रोगवान्दुःखी महारोगप्रपीडितः}% ४२

\twolineshloka
{परान्नविघ्नकरणादजीर्णमभिजायते}
{मन्दोदराग्निर्भवति सति द्रव्ये कदन्नदः}% ४३

\twolineshloka
{विषदश्छर्दिरोगी स्यान्मार्गहा पादरोगवान्}
{पिशुनो नरकस्यान्ते जायते श्वासकासवान्}% ४४

\twolineshloka
{धूर्तोऽपस्माररोगी स्याच्छूली च परतापनः}
{दावाग्निदायकश्चैव रक्तातीसारवान्भवेत्}% ४५

\twolineshloka
{सुरालये जले वापि शकृत्क्षेपं करोति यः}
{गुदरोगो भवेत्तस्य पापरूपः सुदारुणः}% ४६

\twolineshloka
{गर्भपातनजा रोगाः क्षयमेहजलोदराः}
{प्रतिमा भङ्गकारी च अप्रतिष्ठश्च जायते}% ४७

\twolineshloka
{दुष्टवादी खण्डितः स्यात्खल्वाटः परनिन्दकः}
{सभायां पक्षपाती च जायते पक्षघातवान्}% ४८

\twolineshloka
{परोक्तहास्यकृत्काणः कुनखी विप्रहेमहृत्}
{तुन्दीवरी ताम्रचौरः कांस्यहृत्पुण्डरीकिकः}% ४९

\twolineshloka
{त्रपुहारी च पुरुषो जायते पिङ्गमूर्द्धजः}
{शीसहारी च पुरुषो जायते शीर्षरोगवान्}% ५०

\twolineshloka
{घृतचौरस्तु पुरुषो जायते नेत्ररोगवान्}
{लोहहारी च पुरुषो बर्बराङ्गः प्रजायते}% ५१

\twolineshloka
{चर्महारी च पुरुषो जायते मेदसा वृतः}
{मधुचौरस्तु पुरुषो जायते बस्तिगन्धवान्}% ५२

\twolineshloka
{तैलचौर्येण भवति नरः कण्ड्वातिपीडितः}
{आमान्नहरणाच्चैव दन्तहीनः प्रजायते}% ५३

\twolineshloka
{पक्वान्नहरणाच्चैव जिह्वारोगयुतो भवेत्}
{मातृगामी च पुरुषो जायते लिङ्गवर्जितः}% ५४

\twolineshloka
{गुरुजायाभिगमनान्मूत्रकृच्छ्रः प्रजायते}
{भगिनीं चैव गमने पीतकुष्ठः प्रजायते}% ५५

\twolineshloka
{स्वसुतागमने चैव रक्तकुष्ठः प्रजायते}
{भ्रातृभार्याभिगमने गुल्मकुष्ठः प्रजायते}% ५६

\twolineshloka
{स्वामिगम्यादिगमने जायते दद्रुमण्डलम्}
{विश्वस्तभार्यागमने गजचर्मा प्रजायते}% ५७

\twolineshloka
{पितृष्वस्रभिगमने दक्षिणाङ्गे व्रणी भवेत्}
{मातुलान्यास्तु गमने वामाङ्गे व्रणवान्भवेत्}% ५८

\twolineshloka
{पितृव्यपत्नीगमने कटौ कुष्ठः प्रजायते}
{मित्रभार्याभिगमने मृतभार्यः प्रजायते}% ५९

\twolineshloka
{स्वगोत्रस्त्रीप्रसङ्गेन जायते च भगन्दरः}
{तपस्विनीप्रसङ्गेन प्रमेहो जायते नरे}% ६०

\twolineshloka
{श्रोत्रियस्त्रीप्रसङ्गेन जायते नासिकाव्रणी}
{दीक्षितस्त्रीप्रसङ्गेन जायते दुष्टरक्तसृक्}% ६१

\twolineshloka
{स्वजातिजायागमने जायते हृदयव्रणी}
{जात्युन्नतस्त्रीगमने जायते मस्तकव्रणी}% ६२

\twolineshloka
{पशुयोनौ च गमनान्मूत्रघातः प्रजायते}
{एते दोषा नराणां स्युर्नरकान्ते न संशयः}% ६३

\twolineshloka
{स्त्रीणामपि भवन्त्येते तत्तत्पुरुषसङ्गमात्}
{एवं राजन्हि चिह्नानि कीर्तितानि सुपापिनाम्}% ६४

\twolineshloka
{दानपुण्यप्रसङ्गेन तीर्थादिक्रियया तथा}
{रामस्य चरितं श्रुत्वा तपसा वाक्षयं व्रजेत्}% ६५

\twolineshloka
{सर्वेषामेव पापानां हरिकीर्तिधुनी नृणाम्}
{क्षालयेत्पापिनां पङ्कं नात्र कार्या विचारणा}% ६६

\twolineshloka
{यो नावमन्येत हरिं तस्य यागाविधि श्रुताः}
{तीर्थान्यपि सुपुण्यानि पावितुं न क्षमाणि तम्}% ६७

\twolineshloka
{हसते कीर्त्यमानं यश्चरित्रं ज्ञानदुर्बलः}
{न तस्य नरकान्मुक्तिः कल्पान्तेऽपि भविष्यति}% ६८

\twolineshloka
{या हि राजन्विमोक्षार्थं हयस्यानुचरैः सह}
{श्रावय श्रीशचरितं यतो वाहगतिर्भवेत्}% ६९

\uvacha{शेष उवाच}

\twolineshloka
{इति श्रुत्वा प्रहृष्टोऽभूच्छत्रुघ्नः परवीरहा}
{प्रणम्य तं परिक्रम्य ययौ सेवकसंयुतः}% ७०

\twolineshloka
{तत्र गत्वा स हनुमान्हयवर्यस्य पार्श्वतः}
{उवाच रामचरितं महादुर्गतिनाशकम्}% ७१

\twolineshloka
{याहि देव विमानं स्वं रामकीर्तनपुण्यतः}
{स्वैरं चर स्वलोके त्वं मुक्तो भव कुयोनितः}% ७२

\twolineshloka
{इति वाक्यं समाकर्ण्य शत्रुघ्नो यावदास्थितः}
{तावद्ददर्श विमलं देवं वैमानिकं वरम्}% ७३

\twolineshloka
{स उवाच विमुक्तोऽहं रामकीर्तनसंश्रुतेः}
{यामि स्वं भवनं राजन्नाज्ञापय महामते}% ७४

\twolineshloka
{इत्युक्त्वा प्रययौ देवो विमाने स्वे परिस्थितः}
{तदा विस्मयमापुस्ते शत्रुघ्नेन सहानुगाः}% ७५

\twolineshloka
{ततो वाहो विनिर्मुक्तो गात्रस्तम्भाच्च भूतलात्}
{ययौ तद्विपिनं सर्वं भ्रमन्पक्षिसमाकुलम्}% ७६

{॥इति श्रीपद्मपुराणे पातालखण्डे रामाश्वमेधे शेषवात्स्यायनसंवादे हयनिर्मुक्तिर्नामाष्टचत्वारिंशत्तमोऽध्यायः॥४८॥}

\dnsub{एकोनपञ्चाशत्तमोऽध्यायः}\resetShloka

\uvacha{शेष उवाच}

\twolineshloka
{मासाः सप्ताभवंस्तस्य हयवर्यस्य हेलया}
{चरतो भारतं वर्षमनेकनृपपूरितम्}% १

\twolineshloka
{स पूजितो भूपवरैः परीत्य वरभारतम्}
{परीवृतो वीरवरैः शत्रुघ्नादिभिरुद्भटैः}% २

\twolineshloka
{स बभ्राम बहून्देशान्हिमालयसमीपतः}
{न कोपि तं निजग्राह हयं रामबलं स्मरन्}% ३

\twolineshloka
{अङ्गवङ्गकलिङ्गानां राजभिः संस्तुतो हयः}
{जगाम राज्ञो नगरे सुरथस्य मनोहरे}% ४

\twolineshloka
{कुण्डलं नाम नगरमदितेर्यत्र कुण्डलम्}
{कर्णयोः पतितं भूमौ हर्षभयसुकम्पयोः}% ५

\twolineshloka
{यत्र धर्मव्यतिक्रान्तिं न करोति कदापिना}
{श्रीरामस्मरणं प्रेम्णा करोति जनतान्वहम्}% ६

\twolineshloka
{अश्वत्थानां तु यत्रार्चा तुलस्याः प्रत्यहं नृभिः}
{क्रियते रघुनाथस्य सेवकैः पापवर्जितैः}% ७

\twolineshloka
{यत्र देवालया रम्या राघवप्रतिमायुताः}
{पूज्यन्ते प्रत्यहं शुद्धचित्तैः कपटवर्जितैः}% ८

\twolineshloka
{वाचि नाम हरेर्यत्र न वै कलहसङ्कथा}
{हृदि ध्यानं तु तस्यैव न च कामफलस्मृतिः}% ९

\twolineshloka
{देवनं यत्र रामस्य वार्त्ताभिः पूतदेहिनाम्}
{न जातुचिन्नृणामस्ति सत्यव्यसनमानिनाम्}% १०

\twolineshloka
{तस्मिन्वसति धर्मात्मा सुरथः सत्यवान्बली}
{रघुनाथपदस्मारहृष्टचित्तः परोन्मदः}% ११

\twolineshloka
{किं वर्णयामि रामस्य सेवकं सुरथं वरम्}
{यस्याशेषगुणा भूमौ विस्तृताः पावयन्त्यघम्}% १२

\twolineshloka
{सेवकास्तस्य भूपस्य पर्यटन्तः कदाचन}
{अपश्यन्हयमेधस्य हयं चन्दनचर्चितम्}% १३

\twolineshloka
{ते दृष्ट्वा विस्मयं प्राप्ता हयपत्रमलोकयन्}
{स्पष्टाक्षरसमायुक्तं चन्दनादिकचर्चितम्}% १४

\twolineshloka
{ज्ञात्वा रामेण सम्मुक्तं हयं नेत्रमनोहरम्}
{हृष्टा राज्ञे सभास्थाय कथयामासुरुत्सुकाः}% १५

\twolineshloka
{स्वामिन्नयोध्यानगरीपतिस्तस्यास्तु राघवः}
{हयमेधक्रतोर्योग्यो हयो मुक्तः परिभ्रमन्}% १६

\twolineshloka
{स ते पुरस्य निकटे प्राप्तः सेवकसंयुतः}
{गृहाण त्वं महाराज हयं तं सुमनोहरम्}% १७

\uvacha{शेष उवाच}

\twolineshloka
{इति श्रुत्वा निजप्रोक्तं वाक्यं हर्षपरिप्लुतः}
{उवाच वीरान्बलिनो मेघगम्भीरया गिरा}% १८

\uvacha{सुरथ उवाच}

\twolineshloka
{धन्या वयं राममुखं पश्यामः सह सेवकाः}
{ग्रहीष्यामि हयं तस्य भटकोटिपरीवृतम्}% १९

\twolineshloka
{तदा मोक्ष्यामि वाहं तं यदा रामः समाव्रजेत्}
{कृतार्थं मम भक्तस्य चिरं ध्यानरतस्य वै}% २०

\uvacha{शेष उवाच}

\twolineshloka
{इत्थमुक्त्वा महीपालः सेवकान्स्वयमादिशत्}
{गृह्णन्तु वाहं प्रसभं मोच्यो नाश्वोऽक्षिगोचरः}% २१

\twolineshloka
{अनेन सुमहाँल्लाभो भविष्यति तु मे मतम्}
{यद्रामचरणौ प्रेक्षे ब्रह्मशक्रादिदुर्ल्लभौ}% २२

\twolineshloka
{स एव धन्यः स्वजनः पुत्रो वा बान्धवोऽथवा}
{पशुर्वा वाहनं वापि रामाप्तिर्येन सम्भवेत्}% २३

\twolineshloka
{तस्माद्गृहीत्वा क्रत्वश्वं स्वर्णपत्रेण शोभितम्}
{बध्नन्तु वाजिशालायां कामवेगं मनोरमम्}% २४

\twolineshloka
{इत्युक्तास्ते ततो गत्वा वाहं रामस्य शोभितम्}
{गृहीत्वा तरसा राज्ञे ददुः सर्वं शुभाङ्गिनम्}% २५

\twolineshloka
{राजा प्राप्य मुदा चाश्वं रामस्य दनुजार्दनः}
{सेवकान्प्राह बलिनो धर्मकृत्यविचक्षणः}% २६

\twolineshloka
{वात्स्यायन महाबुद्धे शृणुष्वैकाग्रमानसः}
{न तस्य विषये कश्चित्परदाररतो नरः}% २७

\twolineshloka
{न परद्रव्यनिरतो न च कामेषु लम्पटः}
{न जिह्वाभिरतोन्मार्गे कीर्त्तयेद्रामकीर्तनात्}% २८

\twolineshloka
{यः सेवकान्नृपो वक्ति यूयं सेवार्थमागताः}
{कथयन्तु भवच्चेष्टां धर्मकर्मविशारदाः}% २९

\twolineshloka
{एकपत्नीव्रतधरा न परद्रव्यलोलुपाः}
{परापवादानिरता न च वेदोत्पथं गताः}% ३०

\twolineshloka
{श्रीरामस्मरणादीनि कुर्वन्ति प्रत्यहं भटाः}
{तानहं रामसेवार्थं रक्षाम्यन्तक कोपवान्}% ३१

\twolineshloka
{एतद्विरुद्धधर्माणो ये नराः पापसंयुताः}
{तानहं विषये मह्यं वासयामि न दुर्मतीन्}% ३२

\twolineshloka
{तस्य देशे न पापिष्ठाः पापं कुर्वन्ति मानसे}
{हरिध्यानहताशेष पातकामोदसंयुताः}% ३३

\twolineshloka
{यदैवमभवद्देशो राजा धर्मेण संयुतः}
{तदा तत्स्था नराः सर्वे मृता गच्छन्ति निर्वृतिम्}% ३४

\twolineshloka
{यमानुचरनिर्वेशो नाभवत्सौरथे पुरे}
{तदा यमो मुनेरूपं धृत्वा प्रागान्महीश्वरम्}% ३५

\twolineshloka
{वल्कलाम्बरधारी च जटाशोभितशीर्षकः}
{सुरथं तु सभामध्ये ददर्श हरिसेवकम्}% ३६

\twolineshloka
{तुलसीमस्तके यस्य वाचि नाम हरेः परम्}
{धर्मकर्मरतां वार्त्तां श्रावयन्तं निजाञ्जनान्}% ३७

\twolineshloka
{तदा मुनिं नृपो दृष्ट्वा तपोमूर्तिमिव स्थितम्}
{ववन्दे चरणौ तस्य पाद्यादिकमथाकरोत्}% ३८

\twolineshloka
{सुखोपविष्टं विश्रान्तं मुनिं प्राह नृपाग्रणीः}
{धन्यमद्य जनुर्मह्यं धन्यमद्य गृहं मम}% ३९

\twolineshloka
{कथाः कथयतान्मह्यं रामस्य विविधा वराः}
{याः शृण्वतां पापहानिर्भविष्यति पदे पदे}% ४०

\twolineshloka
{इत्थमुक्तं समाकर्ण्य जहास स मुनिर्भृशम्}
{दन्तान्प्रदर्शयन्सर्वांस्तालास्फालितपाणिकः}% ४१

\twolineshloka
{हसन्तं तं मुनिं प्राह हसने कारणं किमु}
{कथयस्व प्रसादेन यथा स्यान्मनसः सुखम्}% ४२

\twolineshloka
{ततो मुनिर्नृपं प्राह शृणु राजन्धियायुतः}
{यदहं तेऽभिधास्यामि स्मिते कारणमुत्तमम्}% ४३

\twolineshloka
{त्वया प्रोक्तं हरेः कीर्तिं कथयस्व ममाग्रतः}
{को हरिः कस्य वा कीर्तिः सर्वे कर्मवशा नराः}% ४४

\twolineshloka
{कर्मणा प्राप्यते स्वर्गः कर्मणा नरकं व्रजेत्}
{कर्मणैव भवेत्सर्वं पुत्रपौत्रादिकं बहु}% ४५

\twolineshloka
{शक्रः शतं क्रतूनां तु कृत्वागात्परमं पदम्}
{ब्रह्मापि कर्मणा लोकं प्राप्य सत्याख्यमद्भुतम्}% ४६

\twolineshloka
{अनेके कर्मणा सिद्धा मरुदादय ईशिनः}
{कुर्वन्ति भोगसौख्यं च अप्सरोगणसेविताः}% ४७

\twolineshloka
{तस्मात्कुरुष्व यज्ञादीन्यजस्व किल देवताः}
{यथा ते विमलाकीर्तिर्भविष्यति महीतले}% ४८

\twolineshloka
{इति श्रुत्वा तु तद्वाक्यं कोपक्षुभितमानसः}
{उवाच रामैकमना विप्रं कर्मविशारदम्}% ४९

\twolineshloka
{मा ब्रूहि कर्मणो वार्तां क्षयिष्णुफलदायिनीम्}
{गच्छ मन्नगरोपान्ताद्बहिर्लोकविगर्हितः}% ५०

\twolineshloka
{इन्द्रः पतिष्यति क्षिप्रं पतिष्यत्यपि पद्मजः}
{न पतिष्यन्ति मनुजा रामस्य भजनोत्सुकाः}% ५१

\twolineshloka
{पश्य ध्रुवं च प्रह्लादं बिभीषणमथाद्भुतम्}
{ये चान्ये रामभक्ता वै कदापि न पतन्ति ते}% ५२

\twolineshloka
{ये रामनिन्दका दुष्टास्तानि मे यमकिङ्कराः}
{ताडयिष्यन्ति लोहस्य मुद्गरैः पाशबन्धनैः}% ५३

\twolineshloka
{ब्राह्मणत्वाद्देहदण्डं न कुर्यां ते द्विजाधम}
{गच्छ गच्छ मदालोकात्ताडयिष्यामि चान्यथा}% ५४

\twolineshloka
{इत्थमुक्तवति श्रेष्ठे भूपे सुरथसंज्ञिते}
{सेवका बाहुना धृत्वा निष्कासयितुमुद्यताः}% ५५

\twolineshloka
{तदा यमो निजं रूपं धृत्वा लोकैकवन्दितम्}
{प्राह भूपं प्रतुष्टोऽस्मि याचस्व हरिसेवक}% ५६

\twolineshloka
{मया प्रलोभितो वाग्भिर्बह्वीभिरपि सुव्रत}
{चलितोसि न रामस्य सेवायाः साधुसेवितः}% ५७

\twolineshloka
{तदा प्रोवाच भूमीशो यमं दृष्ट्वा सुतोषितम्}
{उवाच यदि तुष्टोसि देहि मे वरमुत्तमम्}% ५८

\twolineshloka
{तावन्मम न वै मृत्युर्यावद्रामसमागमः}
{न भयं मे भवत्तो हि कदाचन हि धर्मराट्}% ५९

\twolineshloka
{तदोवाच यमो भूपमिदं तव भविष्यति}
{सर्वं त्वदीप्सितं तथ्यं करिष्यति रघोःपतिः}% ६०

\twolineshloka
{इत्युक्त्वान्तर्हितो धर्मो जगाम स्वपुरं प्रति}
{प्रशस्य तस्य चरितं हरिभक्तिपरात्मनः}% ६१

\twolineshloka
{स राजा धार्मिको रामसेवकः परया मुदा}
{गृहीत्वाश्वं प्रत्युवाच सेवकान्हरिसेवकान्}% ६२

\twolineshloka
{मया गृहीतो वाहोऽसौ राघवस्य महीपतेः}
{सज्जी भवन्तु सर्वत्र यूयं रणविशारदाः}% ६३

\twolineshloka
{इति प्रोक्तास्तु ते सर्वे भटा राज्ञो महाबलाः}
{सज्जीभूताः क्षणादेव सभायां जग्मुरुत्सुकाः}% ६४

\twolineshloka
{राज्ञो वीरा दशसुताश्चम्पको मोहकस्तथा}
{रिपुञ्जयोऽतिदुर्वारः प्रतापीबलमोदकः}% ६५

\twolineshloka
{हर्यक्षः सहदेवश्च भूरिदेवः सुतापनः}
{इति राज्ञो दश सुताः सज्जीभूता रणाङ्गणे}% ६६

\twolineshloka
{यातुमिच्छामकुर्वंस्ते महोत्साहसमन्विताः}
{राजापि स्वरथं चित्रं हेमशोभाविनिर्मितम्}% ६७

\twolineshloka
{आह्वयामास सुजवैर्वाजिभिः समलङ्कृतम्}
{रणोत्साहेन संयुक्तः सर्वसैन्यपरीवृतः}% ६८

\onelineshloka
{सभायां सेवकान्सर्वान्दिशन्नास्ते महीपतिः}% ६९

{॥इति श्रीपद्मपुराणे पातालखण्डे शेषवात्स्यायनसंवादे रामाश्वमेधे सुरथराज्ञा हयग्रहणं नाम एकोनपञ्चाशत्तमोऽध्यायः॥४९॥}

\dnsub{पञ्चाशत्तमोऽध्यायः}\resetShloka

\uvacha{शेष उवाच}

\twolineshloka
{अथ रामानुजो वेगात्समागत्य स्वसेवकान्}
{पप्रच्छ कुत्र वाहोऽसौ याज्ञिकः सुमनोहरः}% १

\twolineshloka
{तदा ते वचनं प्रोचुः शत्रुघ्नं सुमहाबलाः}
{न जानीमो भटाः केचिद्धयं नीत्वा गताः पुरे}% २

\twolineshloka
{वयं च धिक्कृताः सर्वे बलिभी राजसेवकैः}
{अत्र प्रमाणं भगवानिति कर्तव्य तां प्रति}% ३

\twolineshloka
{तच्छ्रुत्वा वचनं तेषां शत्रुघ्नः कुपितो भृशम्}
{दशन्रोषात्स्वदशनाञ्जिह्वया लेलिहन्मुहुः}% ४

\twolineshloka
{उवाच वीरो मद्वाहं हृत्वा कुत्र गमिष्यसि}
{इदानीं पातये बाणैः पुरञ्जनसमन्वितम्}% ५

\twolineshloka
{इत्युक्त्वा सुमतिं प्राह कस्येदं पुटभेदनम्}
{को वर्ततेऽस्याधिपतिर्यो मे वाहमजीहरत्}% ६

\uvacha{शेष उवाच}

\twolineshloka
{इति वाक्यं समाकर्ण्य भूपतेः कोपसंयुतम्}
{जगाद मन्त्री सुगिरा स्फुटाक्षरसमन्वितम्}% ७

\twolineshloka
{विद्धीदं कुण्डलं नाम नगरं सुमनोहरम्}
{अस्मिन्वसति धर्मात्मा सुरथः क्षत्त्रियो बली}% ८

\twolineshloka
{नित्यं धर्मपरो रामचरणद्वन्द्वसेवकः}
{मनसा कर्मणा वाचा हनूमानिव सेवकः}% ९

\twolineshloka
{चरितान्यस्य शतशो वर्तन्ते धर्मकारिणः}
{महाबलपरीवारः सुरथः सर्वशोभनः}% १०

\twolineshloka
{महद्युद्धं भवेदत्र हृतश्चेद्वाहसत्तमः}
{अनेके प्रपतिष्यन्ति वीरा रणविशारदाः}% ११

\twolineshloka
{एवमुक्तं समाश्रुत्य शत्रुघ्नः सचिवं प्रति}
{उवाच पुनरप्येवं वचनं वदतां वरः}% १२

\uvacha{शत्रुघ्न उवाच}

\twolineshloka
{कथमत्र प्रकर्तव्यं रामाश्वोऽनेन चेद्धृतः}
{नायाति योद्धुं प्रबलं कटकं वीरसेवितम्}% १३

\uvacha{सुमतिरुवाच}

\twolineshloka
{दूतः प्रेष्यो महाराज राजानं प्रति वाग्मिकः}
{यद्वाक्येन समायाति बलेन बलिनां वरः}% १४

\twolineshloka
{नोचेदज्ञानतो वाहो धृतः केनापि मानिना}
{अर्पयिष्यति नः साधुमश्वं क्रतुवरं शुभम्}% १५

\twolineshloka
{इति श्रुत्वातु तद्वाक्यं शत्रुघ्नो बुद्धिमान्बली}
{अङ्गदं प्रत्युवाचेदं वचनं विनयान्वितम्}% १६

\uvacha{शत्रुघ्न उवाच}

\twolineshloka
{याहि त्वं निकटस्थे वै सुरथस्य महापुरे}
{दूतत्वेन ततो गत्वा प्रब्रूहि नृपतिं प्रति}% १७

\twolineshloka
{त्वया धृतो रामवाहो ज्ञानतोऽज्ञानतोपि वा}
{अर्पयतु न वा यातु प्रधनं वीरसंयुतम्}% १८

\twolineshloka
{रामस्य दौत्यं लङ्कायां रावणं प्रति यत्कृतम्}
{तथैव कुरु भूयिष्ठ बलसंयुतबुद्धिमन्}% १९

\uvacha{शेष उवाच}

\twolineshloka
{एतच्छ्रुत्वाङ्गदो वीर ओमिति प्रोच्य भूमिपम्}
{जगाम संसदो मध्ये वीरश्रेणिसमन्विते}% २०

\twolineshloka
{ददर्श सुरथं भूपं तुलसीमञ्जरीधरम्}
{रामभद्रं रसनया ब्रुवन्तं सेवकान्निजान्}% २१

\twolineshloka
{राजापि दृष्ट्वा प्लवगं मनोहरवपुर्धरम्}
{शत्रुघ्नदूतं मत्वापि वालिजं प्रत्यभाषत}% २२

\uvacha{सुरथ उवाच}

\twolineshloka
{प्लवगाधिप कस्मात्त्वमागतोऽत्र कथं भवान्}
{ब्रूहि मे कारणं सर्वं यथा ज्ञात्वा करोमि तत्}% २३

\uvacha{शेष उवाच}

\twolineshloka
{इति सम्भाषमाणं तं प्रत्युवाच कपीश्वरः}
{विस्मयंश्चेतसि भृशं रामसेवाकरं नृपम्}% २४

\twolineshloka
{जानीहि मां नृपश्रेष्ठ वालिपुत्रं हरीश्वरम्}
{शत्रुघ्नेन च दूतत्वे प्रेषितो भवतोऽन्तिकम्}% २५

\twolineshloka
{सेवकैः कैश्चिदागत्य धृतोऽश्वो मम साम्प्रतम्}
{अज्ञानतो महान्याय्यं कुर्वद्भिः सहसा नृप}% २६

\twolineshloka
{तमश्वं सह राज्येन सहपुत्रैर्मुदान्वितः}
{शत्रुघ्नं याहि चरणे पतित्वाशु प्रदेहि च}% २७

\twolineshloka
{नोचेच्छत्रुघ्ननिर्मुक्तनाराचैः क्षतविग्रहः}
{पृथ्वीतलमलं कुर्वञ्छयिष्यसि विशीर्षकः}% २८

\twolineshloka
{येन लङ्कापतिर्नाशं प्रापितो लीलया क्षणात्}
{तस्याश्वं यागयोग्यं तु हृत्वा कुत्र गमिष्यसि}% २९

\uvacha{शेष उवाच}

\twolineshloka
{इत्यादिभाषमाणं तं प्रत्युवाच महीश्वरः}
{सर्वं तथ्यं ब्रवीषि त्वं नानृतं तव भाषितम्}% ३०

\twolineshloka
{परं शृणुष्व मद्वाक्यं शत्रुघ्नपदसेवक}
{मया धृतो महानश्वो रामचन्द्रस्य धीमतः}% ३१

\twolineshloka
{न मोक्ष्ये सर्वथा वाहं शत्रुघ्नादिभयादहम्}
{चेद्रामः स्वयमागत्य दर्शनं दास्यते मम}% ३२

\twolineshloka
{तदाहं चरणौ नत्वा दास्यामि सुतसंयुतः}
{सर्वं राज्यं कुटुम्बं च धनं धान्यं बलं बहु}% ३३

\twolineshloka
{क्षत्त्रियाणामयं धर्मः स्वामिनापि विरुद्ध्यते}
{धर्मेण युद्धं तत्रापि रामदर्शनमिच्छता}% ३४

\twolineshloka
{शत्रुघ्नादीन्प्रवीरांस्तानधुनाहं क्षणादपि}
{जित्वा बध्नामि मद्गेहे नोचेद्रामः समाव्रजेत्}% ३५

\uvacha{शेष उवाच}

\twolineshloka
{इति श्रुत्वाङ्गदो धीमाञ्जहास नृपतिं तदा}
{उवाच च महद्वाक्यं महाधैर्यसमन्वितम्}% ३६

\uvacha{अङ्गद उवाच}

\twolineshloka
{बुद्धिहीनः प्रवदसि वृद्धत्वात्सागता तव}
{यत्त्वं शत्रुघ्ननृपतिं धिक्करोषि धिया बली}% ३७

\twolineshloka
{यो मान्धातृरिपुं दैत्यं लवणं लीलयावधीत्}
{येनानेके जिताः सङ्ख्ये वैरिणः प्रबलोद्धताः}% ३८

\twolineshloka
{विद्युन्माली हतो येन राक्षसः कामगे स्थितः}
{त्वं तं बध्नासि वीरेन्द्रं मतिहीनः प्रभासि मे}% ३९

\twolineshloka
{भ्रातृजो यस्य सुबली पुष्कलः परमास्त्रवित्}
{येन रुद्रगणः सङ्ख्ये वीरभद्रः सुतोषितः}% ४०

\twolineshloka
{वर्णयामि किमेतस्य पराक्रान्तिं बलोर्जिताम्}
{येन नास्ति समः पृथ्व्यां बलेन यशसा श्रिया}% ४१

\twolineshloka
{हनूमान्यस्य निकटे रघुनाथपदाब्जधीः}
{यस्यानेकानि कर्माणि भविष्यन्ति श्रुतानि ते}% ४२

\twolineshloka
{सत्रिकूटा राक्षसपूर्दग्धा येन क्षणाद्बलात्}
{अक्षो येन हतः पुत्रो राक्षसेन्द्रस्य दुर्मतेः}% ४३

\twolineshloka
{द्रोणो नाम गिरिर्येन पुच्छाग्रेण सदैवतः}
{आनीतो जीवनार्थं तु सैनिकानां मुहुर्मुहुः}% ४४

\twolineshloka
{जानाति रामश्चारित्रं नान्यो जानाति मूढधीः}
{यं कपीन्द्रं मनाक्स्वान्तान्न विस्मरति सेवकम्}% ४५

\twolineshloka
{सुग्रीवाद्याः कपीन्द्रा ये पृथ्वीं सर्वां ग्रसन्ति ये}
{ते शत्रुघ्नं नृपं सर्वे सेवन्ते प्रेक्षणोत्सुकाः}% ४६

\twolineshloka
{कुशध्वजो नीलरत्नो रिपुतापो महास्त्रवित्}
{प्रतापाग्र्यः सुबाहुश्च विमलः सुमदस्तथा}% ४७

\twolineshloka
{राजा वीरमणिः सत्ययुतो रामस्य सेवकः}
{एतेऽन्येपि नृपा भूमेः पतयः पर्युपासते}% ४८

\twolineshloka
{तत्र त्वं वीर जलधौ मशकः को भवानिति}
{तज्ज्ञात्वा गच्छ शत्रुघ्नं कृपालुं पुत्रकैर्युतः}% ४९

\twolineshloka
{वाहं समर्प्य गन्तासि रामं राजीवलोचनम्}
{दृष्ट्वा कृतार्थी कुरुषे स्वाङ्गानि जनुषा सह}% ५०

\uvacha{शेष उवाच}

\twolineshloka
{राजा प्रोवाच तं दूतं प्रब्रुवन्तमनेकधा}
{एतान्दर्शयसि क्षिप्रं सर्वे न ममगोचराः}% ५१

\twolineshloka
{यादृशं मद्बलं दूत तादृशं न हनूमतः}
{यो रामं पृष्ठतः कृत्वा प्रागाद्यागस्य पालने}% ५२

\twolineshloka
{यद्यहं मनसा वाचा कर्मणा कुतुकान्वितः}
{भजामि रामं तर्ह्याशु दर्शयिष्यति स्वां तनुम्}% ५३

\twolineshloka
{अन्यथा हनुमन्मुख्या वीरा बध्नन्तु मां बलात्}
{गृह्णन्तु वाहं तरसा रामभक्तिसमन्विताः}% ५४

\twolineshloka
{गच्छ त्वं नृप शत्रुघ्नं कथयस्व ममोदितम्}
{सज्जीभवन्तु सुभटा एष यामि रणे बली}% ५५

\twolineshloka
{स विचार्य यथायुक्तं करिष्यति रणाङ्गणे}
{मोचयन्तु महावाहं न वामा मा ददन्तु ते}% ५६

\uvacha{शेष उवाच}

\twolineshloka
{इति श्रुत्वास्मि तं कृत्वा ययौ वीरो यतो नृपः}
{गत्वा निवेदयामास यथोक्तं सुरथेन वै}% ५७

{॥इति श्रीपद्मपुराणे पातालखण्डे शेषवात्स्यायनसंवादे रामाश्वमेधे सुरथदूतयोः संवादो नाम पञ्चाशत्तमोऽध्यायः॥५०॥}

\dnsub{एकपञ्चाशत्तमोऽध्यायः}\resetShloka

\uvacha{शेष उवाच}

\twolineshloka
{तच्छ्रुत्वा भाषितं तस्य सुरथस्याङ्गदाननात्}
{सज्जीभूता रणे सर्वे रथस्था रणकोविदाः}% १

\twolineshloka
{पटहानां निनादोऽभूद्भेरीनादस्तथैव च}
{वीराणां गर्जनानादाः प्रादुर्भूता रणाङ्गणे}% २

\twolineshloka
{रथचीत्कारशब्देन गजानां बृंहितेन च}
{व्याप्तं तत्सकलं विश्वं दिवं यातो महारवः}% ३

\twolineshloka
{रणोत्साहेन संयुक्ता वीरा रणविशारदाः}
{कुर्वन्ति विविधान्नादान्कातरस्य भयङ्करान्}% ४

\twolineshloka
{एवं कोलाहले वृत्ते सुरथो नाम भूमिपः}
{स्वसुतैः सैनिकैश्चाथ वृतः प्रायाद्रणाङ्गणे}% ५

\twolineshloka
{गजैरथैर्हयैः पत्तिव्रजैः पूर्णां तु मेदिनीम्}
{कुर्वन्समुद्रइव तां प्लावयन्ददृशे भटैः}% ६

\twolineshloka
{शङ्खनादेन सङ्घुष्टं जयनादैस्तथैव च}
{वीक्ष्य तं प्रधनोद्युक्तं सुमतिं प्राह भूमिपः}% ७

\uvacha{शत्रुघ्न उवाच}

\twolineshloka
{एष राजा समायातो महासैन्यपरीवृतः}
{अत्र यत्कृत्यमस्माकं तद्वदस्व महामते}% ८

\uvacha{सुमतिरुवाच}

\twolineshloka
{योद्धव्यमत्र बहुभिर्वीरै रणविशारदैः}
{पुष्कलादिभिरत्युग्रैः सर्वशस्त्रास्त्रकोविदैः}% ९

\twolineshloka
{राज्ञा सह समीरस्य पुत्रः परमशौर्यवान्}
{युद्धं करोतु सुबलः परयुद्धविशारदः}% १०

\uvacha{शेष उवाच}

\twolineshloka
{इति ब्रूते महामात्यो यावत्तावन्नृपात्मजाः}
{रणाङ्गणे धनूंष्यद्धा स्फारयामासुरुद्धताः}% ११

\twolineshloka
{तान्वीक्ष्य योधाः सुबलाः पुष्कलाद्या रणोत्कटाः}
{अभिजग्मुः स्यन्दनैः स्वैर्धनुर्बाणकरा मताः}% १२

\twolineshloka
{चम्पकेन महावीरः पुष्कलः परमास्त्रवित्}
{द्वैरथेनैव युयुधे महावीरेण शालिना}% १३

\twolineshloka
{मोहकं योधयामास जानकिः स कुशध्वजः}
{रिपुञ्जयेन विमलो दुर्वारेण सुबाहुकः}% १४

\twolineshloka
{प्रतापिना प्रतापाग्र्यो बलमोदेन चाङ्गदः}
{हर्यक्षेण नीलरत्नः सहदेवेन सत्यवान्}% १५

\twolineshloka
{राजा वीरमणिर्भूरि देवेन युयुधे बली}
{असुतापेन चोग्राश्वो युयुधे बलसंयुतः}% १६

\twolineshloka
{द्वैरथं तु महद्युद्धमकुर्वन्युद्धकोविदाः}
{सर्वे शस्त्रास्त्रकुशलाः सर्वे युद्धविशारदाः}% १७

\twolineshloka
{एवं प्रवृत्ते सङ्ग्रामे सुरथस्य सुतैस्तदा}
{अत्यन्तं कदनं तत्र बभूव मुनिसत्तम}% १८

\twolineshloka
{पुष्कलश्चम्पकं प्राह किं नामासि नृपात्मज}
{धन्योसि यो मया सार्धं रणमध्यमुपेयिवान्}% १९

\twolineshloka
{इदानीं तिष्ठ किं यासि कथं ते जीवितं भवेत्}
{एहि युद्धं मया सार्धं सर्वशस्त्रास्त्रकोविद}% २०

\twolineshloka
{इत्यभिव्याहृतं तस्य श्रुत्वा राजात्मजो बली}
{जगाद पुष्कलं वीरो मेघगम्भीरया गिरा}% २१

\uvacha{चम्पक उवाच}

\twolineshloka
{न नाम्ना न कुलेनेदं युद्धमत्र भविष्यति}
{तथापि तव वक्ष्येऽहं स्वनामबलपूर्वकम्}% २२

\twolineshloka
{मम माता राघवेशो मत्पिता राघवः स्मृतः}
{मम बन्धू रामचन्द्र स्वःजनो मम राघवः}% २३

\twolineshloka
{मन्नाम रामदासश्च सदा रामस्य सेवकः}
{तारयिष्यति मां युद्धे रामो भक्तकृपाकरः}% २४

\twolineshloka
{लोकानां मतमास्थाय प्रब्रवीमि तवाधुना}
{सुरथस्य सुतश्चाहं माता वीरवतीमम}% २५

\twolineshloka
{मन्नामयो मधौ सर्वाञ्छोभनान्विदधाति वै}
{मधुपायंरसावा सन्त्यजन्ति मधुमोहिताः}% २६

\twolineshloka
{वर्णेन स्वर्णसदृशो मध्ये लिङ्गवपुर्धरः}
{तदाख्ययाभिधां वीर जानीहि मम मोहिनीम्}% २७

\twolineshloka
{युध्यस्व बाणैः प्रधनेन को जेतुं हि मां क्षमः}
{इदानीं दर्शयिष्यामि स्वपराक्रममद्भुतम्}% २८

\uvacha{शेष उवाच}

\twolineshloka
{इति श्रुत्वा महद्वाक्यं पुष्कलो हृदि तोषितः}
{तं दुर्जयं मन्यमानः शरान्मुञ्चन्रणेऽभवत्}% २९

\twolineshloka
{शरसङ्घं प्रमुञ्चन्तं कोटिधा पुष्कलं ययौ}
{चम्पकः कोपसंयुक्तो धनुः सज्यमथाकरोत्}% ३०

\twolineshloka
{मुमोच निशितान्बाणान्वैरिवृन्दविदारणान्}
{स्वनामचिह्नितान्स्वर्णपुङ्खभागसमन्वितान्}% ३१

\twolineshloka
{तांश्चिच्छेद महावीरः पुष्कलः प्रधनाङ्गणे}
{शरान्धकारं सर्वत्र मुञ्चन्बाणाञ्छिलाशितान्}% ३२

\twolineshloka
{स्वबाणच्छेदनं दृष्ट्वा कृतं वीरेण चम्पकः}
{आह्वयामास बलिनं पुष्कलं कोपपूरितः}% ३३

\twolineshloka
{मा प्रयाहि रणं त्यक्त्वेति ब्रुवन्समरे पुनः}
{पुष्कलं हृदये बाणैर्विव्याध दशभिस्त्वरन्}% ३४

\twolineshloka
{ते बाणाः पुष्कलस्याहो हृदये तीव्रवेगिनः}
{आगत्य सुभृशं लग्नाः शोणितं पपुरूर्जितम्}% ३५

\twolineshloka
{तैर्बाणैर्व्यथितो वीरः शरान्पञ्च समाददे}
{सुतीक्ष्णाग्रान्महाकोपाद्वारयन्पर्वतानिव}% ३६

\twolineshloka
{ते बाणास्तस्य बाणाश्च परस्परमथोर्जिताः}
{आकाशे रचिताश्छिन्नाः शतधा राजसूनुना}% ३७

\twolineshloka
{छित्त्वा बाणान्सुतीक्ष्णाग्रान्सुरथाङ्गोद्भवो बली}
{बाणाञ्छतं समाधत्त पुष्कलं ताडितुं हृदि}% ३८

\twolineshloka
{ते बाणाः शतधाच्छिन्नाः पुष्कलेन महात्मना}
{अपतन्समरोपान्ते शरवेगप्रपीडिताः}% ३९

\twolineshloka
{तदा तत्सुमहत्कर्म दृष्ट्वा राज्ञः सुतो बली}
{सहस्रेण शराणां च ताडयन्वक्षसि स्फुटम्}% ४०

\twolineshloka
{तानप्याशु प्रचिच्छेद पुष्कलः परमास्त्रवित्}
{पुनरप्याशु स्वे चापे समाधत्तायुतं शरान्}% ४१

\twolineshloka
{तानप्याशु प्रचिच्छेद पुष्कलः परमास्त्रवित्}
{ततोऽत्यतं प्रकुपितः शरवृष्टिमथाकरोत्}% ४२

\twolineshloka
{शरवृष्टिं समायान्तीं मत्वा चम्पक वीरहा}
{साधुसाधुप्रशंसन्तं पुष्कलं समताडयत्}% ४३

\twolineshloka
{पुष्कलश्चम्पकं दृष्ट्वा महावीर्यसमन्वितम्}
{ब्रह्मणोऽस्त्रसमाधत्त स्वे चापे सर्वशस्त्रवित्}% ४४

\twolineshloka
{तेन मुक्तं महाशस्त्रं प्रजज्वाल दिशो दश}
{खं रोदसी व्याप्य विश्वं प्रलयं कर्तुमुद्यतम्}% ४५

\twolineshloka
{चम्पको मुक्तमस्त्रं तद्दृष्ट्वा सर्वास्त्रकोविदः}
{तत्संहर्तुं तदेवास्त्रं मुमोच रिपुमुद्यतम्}% ४६

\twolineshloka
{द्वयोरेकतमं तेजः प्रलयं मेनिरे जनाः}
{सञ्जहार तदास्त्रास्त्रमेकीभूतं परास्त्रकम्}% ४७

\twolineshloka
{तत्कर्मचाद्भुतं दृष्ट्वा पुष्कलस्तिष्ठतिष्ठ च}
{ब्रुवञ्छरानमोघांस्तु चम्पकं स क्रुधाहनत्}% ४८

\twolineshloka
{चम्पकस्ताञ्छरान्मुक्तानगणय्य महामनाः}
{रामास्त्रं प्रमुमोचाथ पुष्कलं प्रति दारुणम्}% ४९

\twolineshloka
{तन्मुक्तमस्त्रमालोक्य चम्पकेन महात्मना}
{छेत्तुं यावन्मनश्चक्रे तावद्ग्रस्तः शरेण सः}% ५०

\twolineshloka
{बद्धश्चम्पकवीरेण रथे स्वे स्थापितः पुनः}
{पुरं प्रेषयितुं तावन्मनश्चक्रे महामनाः}% ५१

\twolineshloka
{हाहाकारो महानासीद्बद्धे पुष्कलसंज्ञिके}
{शत्रुघ्नं प्रययुर्योधाः पलायनपरायणाः}% ५२

\twolineshloka
{भग्नांस्तान्वीक्ष्य शत्रुघ्नो हनूमन्तमुवाच ह}
{केन वीरेण मे भग्नं बलं वीरैरलङ्कृतम्}% ५३

\twolineshloka
{तदोवाच महीनाथ पुष्कलं परवीरहा}
{बद्ध्वा नयति वीरोऽसौ चम्पकः स्वपदोद्धुरः}% ५४

\twolineshloka
{तस्येदृग्वाक्यमाकर्ण्य शत्रुघ्नः कोपसंयुतः}
{उवाच पवनोद्भूतं मोचयाशु नृपात्मजात्}% ५५

\twolineshloka
{महाबलः सुतश्चास्य बद्ध्वा यः पुष्कलं भटम्}
{तस्मान्मोचय वीराग्र्य कथं तिष्ठसि चाहवे}% ५६

\twolineshloka
{एतद्वाक्यं समाकर्ण्य हनूमानोमिति ब्रुवन्}
{जगाम तं मोचयितुं पुष्कलं चम्पकाद्भटात्}% ५७

\twolineshloka
{हनूमन्तमथालोक्य तं मोचयितुमागतम्}
{बाणैः शतैश्च साहस्रैर्जघान परकोपनः}% ५८

\twolineshloka
{बाणांस्तान्स बभञ्जाशु मुक्तांस्तेन महात्मना}
{पुनरप्येनमेवाशु बाणान्मुञ्चन्महानभूत्}% ५९

\twolineshloka
{तान्सर्वांश्चूर्णयामास नाराचान्वैरिमोचितान्}
{शालं करे समाधृत्य जघान नृपनन्दनम्}% ६०

\twolineshloka
{शालं तेन विनिर्मुक्तं तिलशः कृतवान्बली}
{गजो हनूमता मुक्तो नृपनन्दन मस्तके}% ६१

\twolineshloka
{सोऽप्याहतश्चम्पकेन मृतो भूमौ पपातसः}
{शिलाः सम्मोचयामास हनूमान्परमास्त्रवित्}% ६२

\twolineshloka
{चम्पकस्ताः शिलाः सर्वाः क्षणाच्चूर्णितवान्भृशम्}
{बाणयन्त्रिकया ब्रह्मन्महच्चित्रमभूदिदम्}% ६३

\twolineshloka
{स्वमुक्तास्ताः शिलाः सर्वाश्चूर्णिता वीक्ष्य मारुतिः}
{चुकोप हृदयेऽत्यतं बहुवीर्यमिति स्मरन्}% ६४

\twolineshloka
{आगत्य च करे धृत्वा नभस्युत्पतितः कपिः}
{तावद्ययौ नेत्रपथादुपरि क्षिप्रवेगवान्}% ६५

\twolineshloka
{चम्पकस्तं हनूमन्तं युयुधे नभसि स्थितः}
{बाहुयुद्धेन महता ताडितः कपिपुङ्गवः}% ६६

\twolineshloka
{चुकोप मानसे वीरो गर्वपर्वतदारुणः}
{पदा धृत्वा चम्पकं तं ताडयामास भूतले}% ६७

\twolineshloka
{ताडितोऽसौ कपीन्द्रेण क्षणादुत्थाय वेगवान्}
{हनूमन्तं तु लाङ्गूले धृत्वा बभ्राम सर्वतः}% ६८

\twolineshloka
{कपीन्द्रस्तद्बलं वीक्ष्य हसन्पादेऽग्रहीत्पुनः}
{भ्रामयित्वा शतगुणं गजोपस्थे ह्यपातयत्}% ६९

\twolineshloka
{पपात भूमौ सुबलो राजसूनुः स चम्पकः}
{मूर्च्छितो वीरभूषाढ्यमलङ्कुर्वन्रणाङ्गणम्}% ७०

\twolineshloka
{तदा हाहेति वै लोकाश्चुक्रुशुश्चम्पकानुगाः}
{पुष्कलं मोचयामास बद्धं चम्पकपाशतः}% ७१

{॥इति श्रीपद्मपुराणे पातालखण्डे शेषवात्स्यायनसंवादे रामाश्वमेधे पुष्कलमोचनं नामैकपञ्चाशत्तमोऽध्यायः॥५१॥}

\dnsub{द्विपञ्चाशत्तमोऽध्यायः}\resetShloka

\uvacha{शेष उवाच}

\twolineshloka
{चम्पकं पतितं दृष्ट्वा सुरथः क्षत्रियो बली}
{पुत्रदुःखपरीताङ्गो जगाम स्यन्दने स्थितः}% १

\twolineshloka
{कपीन्द्रमाजुहावाथ सुरथः कोपसंयुतः}
{निःश्वासवेगं सम्मुञ्चन्महाबलसमन्वितः}% २

\twolineshloka
{आह्वयानं नृपं दृष्ट्वा निजं वीरः कपीश्वरः}
{जगाम तं महावीरो महावेगसमन्वितः}% ३

\twolineshloka
{तमागतं हनूमन्तं तृणीकुर्वं तमुद्भटान्}
{उवाच सुरथो राजा मेघगम्भीरसुस्वरः}% ४

\uvacha{सुरथ उवाच}

\twolineshloka
{धन्योसि कपिवर्य त्वं महाबलपराक्रमः}
{येन राममहत्कृत्यं कृतं राक्षसके पुरे}% ५

\twolineshloka
{त्वं रामचरणस्यासि सेवको भक्तिसंयुतः}
{त्वया वीरेण मत्पुत्रः पातितश्चम्पको बली}% ६

\twolineshloka
{इदानीं त्वां तु सम्बध्य गन्तास्मि नगरेमम}
{यत्नात्तिष्ठ कपीशेशसत्यमुक्तं मया स्मृतम्}% ७

\twolineshloka
{इति भाषितमाकर्ण्य सुरथस्य कपीश्वरः}
{उवाच धीरया वाण्या रणे वीरैकभूषिते}% ८

\uvacha{हनूमानुवाच}

\twolineshloka
{त्वं रामचरणस्मारी वयं रामस्य सेवकाः}
{बध्नासि चेन्मां प्रसभं मोचयिष्यति मत्प्रभुः}% ९

\twolineshloka
{कुरु वीर भवत्स्वान्तस्थितं सत्यं प्रतिश्रुतम्}
{रामं स्मरन्वै दुःखान्तं याति वेदा वदन्त्यदः}% १०

\uvacha{शेष उवाच}

\twolineshloka
{इति ब्रुवन्तं सुरथः प्रशस्य पवनात्मजम्}
{विव्याध बाणैर्बहुभिः शितैः शाणेन दारुणैः}% ११

\twolineshloka
{तान्मुक्तानगणय्याथ बाणाञ्छोणितपातिनः}
{करे जग्राह कोदण्डं सज्यं शरसमन्वितम्}% १२

\twolineshloka
{गृहीत्वा करयोश्चापं बभञ्ज कुपितः कपिः}
{चीत्कुर्वंस्त्रासयन्वीरान्नखैर्दीर्णान्सृजन्भटान्}% १३

\twolineshloka
{तेन भग्नं धनुर्दृष्ट्वा स्वकीयं गुणसंयुतम्}
{अपरं धनुरादत्त महद्गुणविशोभितम्}% १४

\twolineshloka
{तच्चापि जगृहे रोषात्कपिश्चापं बभञ्ज तत्}
{अन्यच्चापं समादत्त तद्बभञ्ज महाबलः}% १५

\twolineshloka
{तस्मिंश्चापे प्रभग्नेऽपि सोऽन्यद्धनुरुपाददत्}
{सोपि चापं बभञ्जाशु महावेगसमन्वितः}% १६

\twolineshloka
{एवं राज्ञस्तु चापानामशीतिर्विदलीकृता}
{क्षणे क्षणे महारोषात्कुर्वन्नादाननेकधा}% १७

\twolineshloka
{तदात्यन्तं प्रकुपितः शक्तिमुग्रामथाददे}
{शक्त्या स ताडितो वीरः पपात क्षणमुत्सुकः}% १८

\twolineshloka
{उत्थाय स्यन्दनं राज्ञो जग्राह कुपितो भृशम्}
{उड्डीनस्तं गृहीत्वा तु समुद्रमतिवेगतः}% १९

\twolineshloka
{तमुड्डीनं समालक्ष्य सुरथः परवीरहा}
{ताडयामास परिघैर्हृदि मारुतिमुद्यतम्}% २०

\twolineshloka
{मुक्तस्तेन रथो दूराच्चूर्णीभूतोऽभवत्क्षणात्}
{सोऽन्यरथं समारुह्य ययौ वेगात्समीरजम्}% २१

\twolineshloka
{हनूमांस्तद्रथं पुच्छे संवेष्ट्य प्रधनाङ्गणे}
{हयसारथिसंयुक्तं बभञ्ज सपताकिनम्}% २२

\twolineshloka
{अन्यं रथं समास्थाय ययौ राजा महाबलः}
{बभञ्ज तं रथं वेगान्मारुतिः कुपिताङ्गकः}% २३

\twolineshloka
{भग्नं तं स्यन्दनं वीक्ष्य सुरथोऽन्यसमाश्रितः}
{भग्नः स तेन सहसा हयसारथिसंयुतः}% २४

\twolineshloka
{एवमेकोनपञ्चाशद्रथा भग्ना हनूमता}
{तत्कर्म वीक्ष्य राजापि विसिस्माय ससैनिकः}% २५

\twolineshloka
{कुपितः प्राह कीशेन्द्रं धन्योसि पवनात्मज}
{पराक्रमन्निदं कर्म न कर्ता न करिष्यति}% २६

\twolineshloka
{क्षणमेकं प्रतीक्षस्व यावत्सज्यं धनुस्त्वहम्}
{करोमि पवनोद्भूत रामपादाब्जषट्पद}% २७

\twolineshloka
{इत्युक्त्वा चापमात्तज्यं कृत्वा रोषपरिप्लुतः}
{अस्त्रं पाशुपतं नाम सन्दधे शर उल्बणे}% २८

\twolineshloka
{ततो भूताश्च वेतालाः पिशाचा योगिनीमुखाः}
{प्रादुर्बभूवुः सहसा भीषयन्तः समीरजम्}% २९

\twolineshloka
{कपिः पाशुपतैरस्त्रैर्बद्धो लोकैरभीक्षितः}
{हाहेति च वदन्त्येते यावत्तावत्समीरजः}% ३०

\twolineshloka
{स्मृत्वा रामं स्वमनसा त्रोटयामास तत्क्षणात्}
{स मुक्तगात्रः सहसा युयुधे सुरथं नृपम्}% ३१

\twolineshloka
{तं मुक्तगात्रं संवीक्ष्य सुरथः परमास्त्रवित्}
{महाबलं मन्यमानो ब्राह्ममस्त्रं समाददे}% ३२

\twolineshloka
{मारुतिर्ब्राह्ममस्त्रं तु निजगाल हसन्बली}
{तन्निगीर्णं नृपो दृष्ट्वा रामं सस्मार भूमिपः}% ३३

\twolineshloka
{स्मृत्वा दाशरथिं रामं रामास्त्रं स्वशरासने}
{सन्धाय तं जगादेदं बद्धोसि कपिपुङ्गव}% ३४

\twolineshloka
{श्रुत्वा तत्प्रक्रमेद्यावत्तावद्बद्धो रणाङ्गणे}
{राज्ञा रामास्त्रतो वीरो हनूमान्रामसेवकः}% ३५

\twolineshloka
{उवाच भूपं हनुमान्किङ्करोमि महीभुज}
{मत्स्वाम्यस्त्रेण सम्बद्धो नान्येन प्राकृतेन वै}% ३६

\twolineshloka
{तन्मानयामिभूपालनयस्वस्वपुरम्प्रति}
{मोचयिष्यति मत्स्वामी आगत्य स दयानिधिः}% ३७

\twolineshloka
{बद्धे समीरजे क्रुद्धः पुष्कलो भूमिपं ययौ}
{तं पुष्कलं समायान्तं विव्याध वसुभिः शरैः}% ३८

\twolineshloka
{अनेकबाणसाहस्रैर्निजघान नृपं बली}
{राज्ञानेके शरास्तस्य च्छिन्नाः प्रधनमण्डले}% ३९

\twolineshloka
{एवं समरसङ्क्रुद्धे सुरथे पुष्कले तथा}
{बाणैर्व्याप्तं जगत्सर्वं स्थास्नुभूयश्चरिष्णु च}% ४०

\twolineshloka
{तेषां रणोद्यमं वीक्ष्य मुमुहुः सुरसैनिकाः}
{मानवानां तु का वार्ता क्षणात्त्रासं समीयुषाम्}% ४१

\twolineshloka
{अस्त्रप्रत्यस्त्रविगमैर्महामन्त्रपरिस्तुतैः}
{बभूव तुमुलं युद्धं वीराणां रोमहर्षणम्}% ४२

\twolineshloka
{तदा प्रकुपितो राजा नाराचं तु समाददे}
{छिन्नः स तु क्रुधा मुक्तैर्वत्सदन्तैः सभारतैः}% ४३

\twolineshloka
{छिन्ने तस्मिञ्छरे राजा कोपादन्यं समाददे}
{छिनत्ति यावत्स शरं तावल्लग्नो हृदि क्षतः}% ४४

\twolineshloka
{मूर्च्छां प्राप महातेजाः पुष्कलो महदद्भुतम्}
{युद्धं विधाय सुमहद्राज्ञा सह महामतिः}% ४५

\twolineshloka
{पुष्कले पतिते राजा शत्रुघ्नः शत्रुतापनः}
{सुरथं प्रति सङ्क्रुद्धो जगाम स्यन्दनस्थितः}% ४६

\twolineshloka
{उवाच सुरथं भूपं रामभ्राता महाबलः}
{त्वया महत्कृतं कर्म यद्बद्धः पवनात्मजः}% ४७

\twolineshloka
{पुष्कलोऽपि महावीरस्तथान्ये मम सैनिकाः}
{पातिताः प्रधने घोरे महाबलपराक्रमाः}% ४८

\twolineshloka
{इदानीं तिष्ठ मद्वीरान्पातयित्वा रणाङ्गणे}
{कुत्र यास्यसि भूमीश सहस्व मम सायकान्}% ४९

\twolineshloka
{इत्थमाश्रुत्य वीरस्य भाषितं सुरथो बली}
{जगाद रामपादाब्जं दधच्चेतसि शोभनम्}% ५०

\twolineshloka
{मया ते पातिताः सङ्ख्ये वीरा मारुतजोन्मुखाः}
{इदानीं पातयिष्यामि त्वामपि प्रधनाङ्गणे}% ५१

\twolineshloka
{स्मरस्व रामं यो वीरः स्वमागत्य प्ररक्षति}
{अन्यथा जीवितं नास्ति मत्पुरः शत्रुतापन}% ५२

\twolineshloka
{इत्युक्त्वा बाणसाहस्रैस्तं जघान महीपतिः}
{शत्रुघ्नं शरसङ्घातपञ्जरे न्यदधात्परम्}% ५३

\twolineshloka
{शत्रुघ्नः शरसङ्घातं मुञ्चन्तं वह्निदैवतम्}
{अस्त्रं मुमोच दाहार्थं शराणां नतपर्वणाम्}% ५४

\twolineshloka
{तदस्त्रं मुक्तमालोक्य राजा वै सुरथो महान्}
{वारुणास्त्रेण शमयन्विव्याध शरकोटिभिः}% ५५

\twolineshloka
{तदा तद्योगिनीदत्तमस्त्रं धनुषि सन्दधे}
{मोहनं सर्ववीराणां निद्राप्रापकमद्भुतम्}% ५६

\twolineshloka
{तन्मोहनं महास्त्रं स वीक्ष्य राजा हरिंस्मरन्}
{जगाद शत्रुघ्नमयं सर्वशस्त्रास्त्रकोविदः}% ५७

\twolineshloka
{मोहितस्य मम श्रीमद्रामस्य स्मरणेन ह}
{नान्यन्मोहनमाभाति ममापि भयतापदम्}% ५८

\twolineshloka
{इत्युक्तवति वीरे तु मुमोच स महास्त्रकम्}
{तेन बाणेन सञ्छिन्नं पपात रणमण्डले}% ५९

\twolineshloka
{तन्मोहनं महास्त्रं तु निष्फलं वीक्ष्य भूमिपे}
{अत्यन्तं विस्मयं प्राप्य बाणं धनुषि सन्दधे}% ६०

\twolineshloka
{लवणो येन निहतो महासुरविमर्दनः}
{तं बाणं चाप आधत्त घोरं कान्त्यानलप्रभम्}% ६१

\twolineshloka
{तं वीक्ष्य राजा प्रोवाच बाणोऽयमसतां हृदि}
{लगते रामभक्तस्य सम्मुखेऽपि न भात्यसौ}% ६२

\twolineshloka
{इत्येवं भाषमाणं तु बाणेनानेन शत्रुहा}
{विव्याध हृदये क्षिप्रं वह्निज्वालासमप्रभम्}% ६३

\twolineshloka
{तेन बाणेन दुःखार्तो महापीडासमन्वितः}
{रथोपस्थे क्षणं मूर्च्छामवाप परतापनः}% ६४

\twolineshloka
{स क्षणात्तां व्यथां तीर्त्वा जगाद रिपुमग्रतः}
{सहस्वैकं प्रहारं मे कुत्र यासि ममाग्रतः}% ६५

\twolineshloka
{एवमुक्त्वा महासङ्ख्ये बाणमाधत्त सायके}
{ज्वालामालापरीताङ्गं स्वर्णपुङ्खसमन्वितम्}% ६६

\twolineshloka
{स बाणो धनुषो मुक्तः शत्रुघ्नेन पथिस्थितः}
{छिन्नोऽप्यग्रफलेनाशु हृदये समपद्यत}% ६७

\twolineshloka
{तेन बाणेन सम्मूर्छ्य पपात स्यन्दनोपरि}
{ततो हाहाकृतं सैन्यं भग्नं सर्वं पराद्रवत्}% ६८

\twolineshloka
{सुरथो जयमापेदे सङ्ग्रामे रामसेवकः}
{दशवीरा दशसुतैर्मूर्च्छिताः पतिताः क्वचित्}% ६९

{॥इति श्रीपद्मपुराणे पातालखण्डे शेषवात्स्यायनसंवादे रामाश्वमेधे सुरथविजयो नाम द्विपञ्चाशत्तमोऽध्यायः॥५२॥}

\dnsub{त्रिपञ्चाशत्तमोऽध्यायः}\resetShloka

\uvacha{शेष उवाच}

\twolineshloka
{सुग्रीवस्तु तत्कटकं भग्नं वीक्ष्य रणाङ्गणे}
{स्वामिनं मूर्च्छितं वापि ययौ योद्धुं नृपं प्रति}% १

\twolineshloka
{आगच्छ भूप सर्वान्नो मूर्च्छयित्वा कुतो भवान्}
{गच्छति क्षिप्रं मां देहि युद्धं रणविशारद}% २

\twolineshloka
{एवमुक्त्वा नगं कञ्चिद्विशालं शाखया युतम्}
{उत्पाट्य प्राहरत्तस्य मस्तके बलसंयुतः}% ३

\fourlineindentedshloka
{तेन प्रहारेण महाबलो नृपः}
{संवीक्ष्यसु ग्रीवमथो स्वचापे}
{बाणान्समाधाय शितान्सरोषा}
{ज्जघान वक्षस्यतिपौरुषो बली}% ४

\twolineshloka
{तान्बाणान्व्यधमत्सर्वान्सुग्रीवः सहसा हसन्}
{ताडयामास हृदये सुरथं सुमहाबलः}% ५

\twolineshloka
{पर्वतैः शिखरैश्चैव नगैर्द्विरदवर्ष्मभिः}
{वेगात्सन्ताडयामास दारयन्सुरथं नखैः}% ६

\twolineshloka
{तमप्याशु बबन्धास्त्राद्रामसंज्ञात्सुदारुणात्}
{बद्धः कपिवरो मेने सुरथं रामसेवकम्}% ७

\twolineshloka
{गजो यथायसमयीं शृङ्खलां पादलम्बिताम्}
{प्राप्य किञ्चिन्न वै कर्तुं शक्नोति स तथा ह्यभूत्}% ८

\twolineshloka
{जितं तेन महाराज्ञा सुरथेन सुपुत्रिणा}
{सर्वान्वीरान्रथे स्थाप्य ययौ स्वनगरं प्रति}% ९

\twolineshloka
{गत्वा सभायां सुमहान्बद्धं मारुतिमब्रवीत्}
{स्मर श्रीरघुनाथं त्वं दयालुं भक्तपालकम्}% १०

\twolineshloka
{यथा त्वां बन्धनात्सद्यो मोचयिष्यति सुष्ठुधीः}
{नान्यथायुतवर्षेण मोचयिष्यामि बन्धनात्}% ११

\fourlineindentedshloka
{इत्युक्तमाकर्ण्य समीरजस्तदा}
{सुबद्धमात्मानमवेक्ष्य वीरान्}
{सम्मूर्च्छिताञ्छत्रुशराभिघात-}
{पीडायुतान्बन्धनमुक्तये स्मरत्}% १२

\twolineshloka
{श्रीरामचन्द्रं रघुवंशजातं सीतापतिं पङ्कजपत्रनेत्रम्}
{स्वमुक्तये बन्धनतः कृपालुं सस्मार सर्वैः करणैर्विशोकैः}% १३

\uvacha{हनूमानुवाच}

\fourlineindentedshloka
{हा नाथ हा नरवरोत्तम हा दयालो}
{सीतापते रुचिरकुन्तलशोभिवक्त्र}
{भक्तार्तिदाहक मनोहररूपधारिन्}
{मां बन्धनात्सपदि मोचय मा विलम्बम्}% १४

\fourlineindentedshloka
{सम्मोचितास्तु भवता गजपुङ्गवाद्याः}
{देवाश्च दानवकुलाग्नि सुदह्यमानाः}
{तत्सुन्दरीशिरसिसंस्थितकेशबन्धः}
{सम्मोचितस्तु करुणालय मां स्मरस्व}% १५

\fourlineindentedshloka
{त्वं यागकर्मनिरतोऽसि मुनीश्वरेन्द्रै}
{र्धर्मं विचारयसि भूमिपतीड्यपाद}
{अत्राहमद्य सुरथेन विगाढपाश-}
{बद्धोस्मि मोचय महापुरुषाशु देव}% १६

\fourlineindentedshloka
{नो मोचयस्यथ यदि स्मरणातिरेकात्}
{त्वं सर्वदेववरपूजितपादपद्म}
{लोको भवन्तमिदमुल्लसितोऽहसिष्य-}
{त्तस्माद्विलम्बमिह माचर मोचयाशु}% १७

\twolineshloka
{इति श्रुत्वा जगन्नाथो रघुवीरः कृपानिधिः}
{भक्तं मोचयितुं प्रागात्पुष्पकेणाशुवेगिना}% १८

\twolineshloka
{लक्ष्मणेनानुगेनाथ भरतेन सुशोभितम्}
{मुनिवृन्दैर्व्यासमुख्यैः समेतं ददृशे कपिः}% १९

\twolineshloka
{तमागतं निजं नाथं वीक्ष्य भूपं समब्रवीत्}
{पश्य राजन्निजं मोक्तुमायातं कृपया हरिम्}% २०

\twolineshloka
{अनेके मोचिताः पूर्वं स्मरणात्सेवका निजाः}
{तथा मां पाशतो बद्धं सम्मोचयितुमागतः}% २१

\twolineshloka
{श्रीरामभद्रमायान्तं वीक्ष्यासौ सुरथः क्षणात्}
{नतीश्च शतशश्चक्रे भक्तिपूरपरिप्लुतः}% २२

\twolineshloka
{श्रीरामस्तं निजैर्दोर्भिः परिरेभे चतुर्भुजः}
{मूर्ध्नि सिञ्चन्नश्रुजलैर्हर्षाद्भक्तं स्वकं मुहुः}% २३

\twolineshloka
{उवाच धन्यदेहोऽसि महत्कर्म कृतं त्वया}
{कपीश्वरस्त्वया बद्धो हनूमान्सर्वतो बलः}% २४

\twolineshloka
{श्रीरामः कपिवर्यं तं मोचयामास बन्धनात्}
{मूर्छितांस्तान्भटान्सर्वान्वीक्ष्य दृष्ट्या स्वजीवयत्}% २५

\twolineshloka
{ते मूर्च्छां तत्यजुर्दृष्टा रामेण सुरसेविना}
{उत्थिता ददृशुः श्रीमद्रामचन्द्रं मनोरमम्}% २६

\twolineshloka
{प्रणतास्ते रघुपतिं तेन पृष्टा अनामयम्}
{सुखीभूता नृपं प्रोचुः सर्वं स्वकुशलं नृपाः}% २७

\twolineshloka
{सुरथो वीक्ष्य रामं च कृपार्थं सेवकात्मनः}
{आगतं सकलं राज्यं सहयं सुमुदार्पयत्}% २८

\threelineshloka
{अनेकवरिवस्याभिः श्रीरामं समतोषयत्}
{कथयामास मेऽन्याय्यं कृतं ते क्षम राघव}
{श्रीरामस्तमुवाचाथ कृतं ते वाहरक्षणम्}% २९

\twolineshloka
{क्षत्त्रियाणामयं धर्मः स्वामिना सह युद्ध्यते}
{त्वया साधुकृतं कर्म रणे वीराः प्रतोषिताः}% ३०

\twolineshloka
{इत्युक्तवन्तं नृहरिं पूजयन्ससुतोऽभवत्}
{श्रीरामस्त्रिदिनं स्थित्वा ययौ तमनुमन्त्र्य च}% ३१

\twolineshloka
{कामगेन विमानेन मुनिभिः सहितो महान्}
{तं दृष्ट्वा विस्मितास्तस्य कथाश्चक्रुर्मनोहराः}% ३२

\twolineshloka
{चम्पकं स्वपुरे स्थाप्य सुरथः क्षत्रियो बली}
{शत्रुघ्नेन समं यातुं मनश्चक्रे महाबलः}% ३३

\twolineshloka
{शत्रुघ्नः स्वहयं प्राप्य भेरीनादानकारयत्}
{शङ्खनादान्बहुविधान्सर्वत्र समवादयत्}% ३४

\twolineshloka
{सुरथेन समं वीरो यज्ञवाहममूमुचत्}
{स बभ्रामापरान्देशान्न कोपि जगृहे बली}% ३५

\twolineshloka
{यत्रयत्र गतो वाहस्तत्रतत्र परिभ्रमन्}
{सैन्येन महता यातः शत्रुघ्नः सुरथेन च}% ३६

\twolineshloka
{कदाचिज्जाह्नवीतीरे वाल्मीकेराश्रमं वरम्}
{गतो मुनिवरैर्जुष्टं प्रातर्धूमेन चिह्नितम्}% ३७

{॥इति श्रीपद्मपुराणे पातालखण्डे शेषवात्स्यायनसंवादे रामाश्वमेधे रघुनाथसमागमो नाम त्रिपञ्चाशत्तमोऽध्यायः॥५३॥}

\dnsub{चतुःपञ्चाशत्तमोऽध्यायः}\resetShloka

\uvacha{शेष उवाच}

\twolineshloka
{गतः प्रातःक्रियां कर्तुं समिधस्तत्क्रियार्हकाः}
{आनेतुं जानकीसूनुर्वृतो मुनिसुतैर्लवः}% १

\twolineshloka
{ददर्श तत्र यज्ञाश्वं स्वर्णपत्रेण चिह्नितम्}
{कुङ्कुमागरुकस्तूरी दिव्यगन्धेन वासितम्}% २

\twolineshloka
{विलोक्य जातकुतुको मुनिपुत्रानुवाच सः}
{अर्वा कस्य मनोवेगः प्राप्तो दैवान्मदाश्रमम्}% ३

\twolineshloka
{आगच्छन्तु मया सार्धं प्रेक्षन्तां मा भयं कृथाः}
{इत्युक्त्वा स लवस्तूर्णं वाहस्य निकटे गतः}% ४

\twolineshloka
{स रराज समीपस्थो वाहस्य रघुवंशजः}
{धनुर्बाणधरः स्कन्धे जयन्त इव दुर्जयः}% ५

\twolineshloka
{गत्वा मुनिसुतैः सार्धं वाचयामास पत्रकम्}
{भालस्थितं स्पष्टवर्णराजिराजितमुत्तमम्}% ६

\twolineshloka
{विवस्वतो महान्वंशः सर्वलोकेषु विश्रुतः}
{यत्र कोपि पराबाधी न परद्रव्यलम्पटः}% ७

\twolineshloka
{सूर्यवंशध्वजो धन्वी धनुर्दीक्षा गुरुर्गुरुः}
{यं देवाः सामराः सर्वे नमन्ति मणिमौलिभिः}% ८

\twolineshloka
{तस्यात्मजो वीर बलदर्पहारी रघूद्वहः}
{रामचन्द्रो महाभागः सर्वशूरशिरोमणिः}% ९

\twolineshloka
{तन्माता कोशलनृपपुत्रीरत्नसमुद्भवा}
{तस्याः कुक्षिभवं रत्नं रामः शत्रुक्षयङ्करः}% १०

\twolineshloka
{करोति हयमेधं स ब्राह्मणेन सुशिक्षितः}
{रावणाभिधविप्रेन्द्र वधपापापनुत्तये}% ११

\twolineshloka
{मोचितस्तेन वाहानां मुख्योऽसौ याज्ञिको हयः}
{महाबलपरीवारो परिखाभिः सुरक्षितः}% १२

\twolineshloka
{तद्रक्षकोऽस्ति मद्भ्राता शत्रुघ्नो लवणान्तकः}
{हस्त्यश्वरथपादातसङ्घसेनासमन्वितः}% १३

\twolineshloka
{यस्य राज्ञ इति श्रेष्ठो मानो जायेत्स्वकान्मदात्}
{शूरा वयं धनुर्धारि श्रेष्ठा वयमिहोत्कटाः}% १४

\twolineshloka
{ते गृह्णन्तु बलाद्वाहं रत्नमालाविभूषितम्}
{मनोवेगं कामजवं सर्वगत्याधिभास्वरम्}% १५

\twolineshloka
{ततो मोचयिता भ्राता शत्रुघ्नो लीलया हठात्}
{शरासनविनिर्मुक्त वत्सदन्तकृतव्यथात्}% १६

\fourlineindentedshloka
{ये क्षत्रियाः क्षत्रियकन्यकायां}
{जाताश्च सत्क्षेत्रकुलेषु सत्सु}
{गृह्णन्तु ते तद्विपरीतदेहा}
{नमन्तु राज्यं रघवे निवेद्य}% १७

\twolineshloka
{इति संवाच्य कुपितो लवः शस्त्रधनुर्धरः}
{उवाच मुनिपुत्रांस्तान्रोषगद्गदभाषितः}% १८

\twolineshloka
{पश्यत क्षिप्रमेतस्य धृष्टत्वं क्षत्रियस्य वै}
{लिलेख यो भालपत्रे स्वप्रतापबलं नृपः}% १९

\twolineshloka
{कोऽसौ रामः कः शत्रुघ्नः कीटाः स्वल्पबलाश्रिताः}
{क्षत्रियाणां कुले जाता एते न वयमुत्तमाः}% २०

\twolineshloka
{एतस्य वीरसूर्माता जानकी न कुशप्रसूः}
{या रत्नं कुशसंज्ञं तु दधाराग्निमिवारणिः}% २१

\twolineshloka
{इदानीं क्षत्रियत्वादि दर्शयिष्यामि सर्वतः}
{यदि क्षत्रियभूरेष भविष्यति च शत्रुहा}% २२

\twolineshloka
{गृहीष्यति मया बद्धं वाहं यज्ञक्रियोचितम्}
{नोचेत्क्षत्रत्वमुन्मुच्य कुशस्य चरणार्चकः}% २३

\twolineshloka
{अधुना मद्धनुर्मुक्तैः शरैः सुप्तो भविष्यति}
{अन्ये ये च महावीरा रणमण्डलभूषणाः}% २४

\twolineshloka
{इत्यादिवाक्यमुच्चार्य लवो जग्राह तं हयम्}
{तृणीकृत्य नृपान्सर्वांश्चापबाणधरो वरः}% २५

\twolineshloka
{तदा मुनिसुताः प्रोचुर्लवं हयजिहीर्षकम्}
{अयोध्यानृपती रामो महाबलपराक्रमः}% २६

\twolineshloka
{तस्य वाहं न गृह्णाति शक्रोऽपि स्वबलोद्धतः}
{मा गृहाण शृणुष्वेदं मद्वाक्यं हितसंयुतम्}% २७

\twolineshloka
{इत्युक्तं स श्रुतौ धृत्वा जगाद स द्विजात्मजान्}
{यूयं बलं न जानीथ क्षत्रियाणां द्विजोत्तमाः}% २८

\twolineshloka
{क्षत्रिया वीर्यशौण्डीर्या द्विजा भोजनशालिनः}
{तस्माद्यूयं गृहे गत्वा भुञ्जन्तु जननी हृतम्}% २९

\twolineshloka
{इत्युक्तास्तेऽभवंस्तूष्णीं प्रेक्षन्तस्तत्पराक्रमम्}
{लवस्य मुनिपुत्रास्ते सन्तस्थुर्दूरतो बहिः}% ३०

\twolineshloka
{एवं व्यतिकरे वृत्ते सेवकास्तस्य भूपतेः}
{आयातास्तं हयं बद्धं दृष्ट्वा प्रोचुस्तदा लवम्}% ३१

\twolineshloka
{बबन्ध को हयमहो रुष्टः कस्य च धर्मराट्}
{को बाणव्रजमध्यस्थः प्राप्स्यते परमां व्यथाम्}% ३२

\twolineshloka
{तदा लवो जगादाशु मया बद्धोऽश्व उत्तमः}
{यो मोचयति तस्याशु रुष्टो भ्राता कुशो महान्}% ३३

\twolineshloka
{यमः करिष्यति किमु ह्यागतोऽपि स्वयं प्रभुः}
{नत्वा गमिष्यति क्षिप्रं शरवृष्ट्या सुतोषितः}% ३४

\uvacha{शेष उवाच}

\twolineshloka
{इति वाक्यं समाकर्ण्य बालोयमिति तेब्रुवन्}
{समागता मोचयितुं हयं बद्धं तु ये हरेः}% ३५

\twolineshloka
{तान्वैमोचयितुं प्राप्ताञ्छत्रुघ्नस्य च सेवकान्}
{कोदण्डं करयोर्धृत्वा क्षुरप्रान्सममूमुचत्}% ३६

\twolineshloka
{ते छिन्नबाहवः शोकाच्छत्रुघ्नं प्रतिसङ्गताः}
{पृष्टास्ते जगदुः सर्वे लवात्स्वभुजकृन्तनम्}% ३७

{॥इति श्रीपद्मपुराणे पातालखण्डे शेषवात्स्यायनसंवादे रामाश्वमेधे लवेन हयबन्धनं नाम चतुःपञ्चाशत्तमोऽध्यायः॥५४॥}

\dnsub{पञ्चपञ्चाशत्तमोऽध्यायः}\resetShloka

\uvacha{व्यास उवाच}

\twolineshloka
{एतां श्रुत्वा कथां रम्यां लवस्य बलिनो मुनिः}
{संशयानः पर्यपृच्छन्नागं दशशताननम्}% १

\uvacha{श्रीवात्स्यायन उवाच}

\twolineshloka
{त्वयोक्तं तु पुरा रामः सीतामेकाकिनीं वने}
{रजकस्य दुरुक्त्यासौ तत्याज महि लोलुपः}% २

\twolineshloka
{जानक्यां क्व सुतौ जातौ क्व धनुर्धरतां गतौ}
{कथं च शिक्षिता विद्या यो रामहयमाहरत्}% ३

\uvacha{व्यास उवाच}

\twolineshloka
{इति श्रुत्वा मुनेर्वाक्यं शेषो नागो महामतिः}
{प्रशस्य विप्रं जगदे रामचारित्रमद्भुतम्}% ४

\uvacha{शेष उवाच}

\twolineshloka
{रामो राज्यमयोध्यायां भ्रातृभिः सहितोऽकरोत्}
{धर्मेण पालयन्सर्वं क्षितिखण्डं स्वया स्त्रिया}% ५

\twolineshloka
{सीता दधार तद्वीर्यं मासाः पञ्चाभवंस्तदा}
{अत्यन्तं शुशुभे देवी त्रयीव पुरुषं धरा}% ६

\twolineshloka
{कदाचित्समये रामः पप्रच्छ च विदेहजाम्}
{कीदृशो दोहदः साध्वि मया ते साध्यते हि सः}% ७

\twolineshloka
{रहस्येव तु सा पृष्टा त्रपमाणा पतिं सती}
{लज्जा गद्गद वाग्रामं निजगाद वचोऽमृतम्}% ८

\uvacha{सीतोवाच}

\twolineshloka
{त्वत्कृपातो मया सर्वं भुक्तं भोक्ष्यामि शोभनम्}
{न कश्चिन्मानसे कान्त विषयो ह्यतिरिच्यते}% ९

\twolineshloka
{यस्याभवादृशः स्वामी देवसंस्तुतसत्पदः}
{तस्याः सर्वं वरीवर्ति न किञ्चिदवशिष्यते}% १०

\twolineshloka
{त्वमाग्रहात्पृच्छसि मां दोहदं मनसि स्थितम्}
{ब्रवीमि पुरतः सत्यं तव स्वामिन्मनोहर}% ११

\twolineshloka
{चिरं जातं मया सत्यो लोपामुद्रादिकाः स्त्रियः}
{दृष्ट्वा स्वामिन्मनो द्रष्टुं ता उत्सुकति सुन्दरीः}% १२

\twolineshloka
{राज्यं प्राप्ता त्वया सार्द्धमनेकसुखमास्थिता}
{कृतघ्नाहं कदापीह ता नमस्कर्तुमानसा}% १३

\twolineshloka
{तत्र गत्वा तपःकोशान्वस्त्राद्यैः परिपूजये}
{रत्नानि चैव भास्वन्ति भूषा अपि समर्पये}% १४

\twolineshloka
{यथा मे तोषिताः सत्यो ददत्याशीर्मनोहराः}
{एष मे दोहदः कान्त परिपूरय मानसः}% १५

\twolineshloka
{इत्थमाकर्ण्य वचनं सीतायाः सुमनोहरम्}
{जगाद परमप्रीतो रामचन्द्रः प्रियां प्रति}% १६

\twolineshloka
{धन्यासि जानकी प्रातर्गमिष्यसि तपोधनाः}
{प्रेक्ष्यतास्तु कृतार्था त्वमागमिष्यसि मेऽन्तिकम्}% १७

\twolineshloka
{इति रामवचः श्रुत्वा परमां प्रीतिमाप सा}
{प्रातर्मम भवत्यद्धा तापसीनां समीक्षणम्}% १८

\twolineshloka
{अथ तन्निशि रामेण चाराः कीर्तिं निजां श्रुताम्}
{प्रेक्षितुं प्रेषितास्ते तु निशीथे ह्यगमनञ्छनैः}% १९

\twolineshloka
{ते प्रत्यहं रामकथाः शृण्वन्तः सुमनोहराः}
{तद्दिने गतवन्तस्तु धनाढ्यस्य गृहं महत्}% २०

\twolineshloka
{दीपं वीक्ष्य प्रज्वलन्तं वचनं वीक्ष्य मानुषम्}
{स्थितास्तत्र क्षणं चाराः समशृण्वन्यशो भृशम्}% २१

\twolineshloka
{तत्र काचन वामाक्षी बालकं प्रति हर्षिता}
{स्तनं धयन्तं निजगौ वाक्यं तु सुमनोहरम्}% २२

\twolineshloka
{पिब पुत्र यथेष्टं त्वं स्तन्यं मम मनोहरम्}
{पश्चात्तव सुदुष्प्रापं भविष्यति ममात्मज}% २३

\twolineshloka
{एतत्पुर्याः पती रामो नीलोत्पलदलप्रभः}
{तत्पुरीस्थजनानां तु न भविष्यति वै जनुः}% २४

\twolineshloka
{जन्माभावात्कथं पानं स्तन्यस्य भुवि जायते}
{तस्मात्पिब मुहुः स्तन्यं दुर्ल्लभं हृदि मन्य च}% २५

\twolineshloka
{ये श्रीरामं स्मरिष्यन्ति ध्यायन्ति च वदन्ति ये}
{तेषामपि पयःपानं न भविष्यति जातुचित्}% २६

\twolineshloka
{इत्यादिवाक्यं संश्रुत्य श्रीरामयशसोऽमृतम्}
{हर्षिताः प्रययुर्गेहमन्यद्भाग्यवतो महत्}% २७

\twolineshloka
{तावदन्यश्चरस्तत्र मनोरममिदं गृहम्}
{मत्वा तिष्ठन्हि रामस्य क्षणं शुश्रूषया यशः}% २८

\twolineshloka
{तत्र काचिन्निजं कान्तं पर्यङ्कोपरि सुस्थितम्}
{ताम्बूलं चर्वती दत्तं भर्त्तास्नेहेन सुन्दरी}% २९

\twolineshloka
{कङ्कणस्वरशोभाढ्या कर्पूरागरुधूपिता}
{कान्तं वीक्ष्य चलन्नेत्रा कामरूपमवोचत}% ३०

\twolineshloka
{नाथ त्वं तादृशो मह्यं भासि यादृग्रघोः पतिः}
{अत्यन्तं सुन्दरतरं वपुर्बिभ्रत्सुकोमलम्}% ३१

\twolineshloka
{पद्मप्रान्तं नेत्रयुग्मं वक्षो मोहनविस्तृतम्}
{भुजौ च साङ्गदौ बिभ्रत्साक्षाद्राम इवासि मे}% ३२

\twolineshloka
{इति वाक्यं समाकर्ण्य कान्तायाः सुमनोहरम्}
{उवाच नेत्रयोः प्रान्तं नर्तयन्कामसुन्दरः}% ३३

\twolineshloka
{शृणु कान्ते त्वया प्रोक्तं साध्व्या तु सुमनोहरम्}
{पतिव्रतानां तद्योग्यं स्वकान्तो राम एव हि}% ३४

\twolineshloka
{परं क्वाहं मन्दभाग्यः क्व रामो भाग्यवान्महान्}
{क्व चाहं कीटवत्तुच्छः क्व ब्रह्मादिसुरार्चितः}% ३५

\twolineshloka
{खद्योतः क्व नभोरत्नं शलभः क्व नु पामरः}
{गजारिः क्व मृगेन्द्रोऽसौ शशकः क्व नु मन्दधीः}% ३६

\twolineshloka
{क्व च सा जाह्नवी देवी क्व रथ्या जलमुत्पथम्}
{क्व मेरुः सुरसंवासः क्व गुञ्जापुञ्जकोल्पकः}% ३७

\twolineshloka
{तथाहं क्व क्व रामोऽसौ यत्पादरजसाङ्गना}
{शिलीभूता क्षणाज्जाता ब्रह्ममोहनरूपधृक्}% ३८

\twolineshloka
{इति वाक्यं प्रब्रुवाणं परिरेभे निजं पतिम्}
{जातकामा हृतप्रेम्णा नर्तित भ्रू धनुर्धरा}% ३९

\twolineshloka
{इत्यादि वाक्यं संश्रुत्य गतश्चान्यनिवेशनम्}
{तावदन्यश्चरो वाक्यं शुश्राव यशसान्वितम्}% ४०

\twolineshloka
{काचित्पुष्पमयीं शय्यां चन्दनं सह चन्द्रकम्}
{सर्वं विधाय कामार्हं जगाद वचनं पतिम्}% ४१

\twolineshloka
{पते कुरुष्व भोगार्हे शयनं पुष्पमञ्चके}
{चन्दनादिकलेपं च तथा भोगमनेकधा}% ४२

\twolineshloka
{त्वादृशा एव भोगार्हा न च रामपराङ्मुखाः}
{सर्वं रामकृपाप्राप्तमुपभुङ्क्ष्व यथातथम्}% ४३

\twolineshloka
{मत्सदृशी कामिनी ते चन्दनं तापहारकम्}
{पर्यङ्कः पुष्परचितः सर्वं रामकृपाभवम्}% ४४

\twolineshloka
{ये रामं न भजिष्यन्ति ते नरा जठरं स्वयम्}
{न भर्तुं शक्नुवन्त्येव वस्त्रभोगादि वर्जिताः}% ४५

\twolineshloka
{इति ब्रुवन्तीं महिलां हर्षितः पतिरब्रवीत्}
{सर्वं तथ्यं ब्रवीषि त्वं मम रामकृपाभवम्}% ४६

\twolineshloka
{इत्येवं रामभद्रस्य यशः श्रुत्वा गतश्चरः}
{तावदन्यस्य वेश्मस्थश्चरोऽन्य शुश्रुवे वचः}% ४७

\twolineshloka
{काचित्कान्तेन पर्यङ्के वीणावादनतत्परा}
{कान्तेन रामसत्कीर्तिं गायमाना पतिं जगौ}% ४८

\twolineshloka
{स्वामिन्वयं धन्यतमा येषां पुर्याः पतिः प्रभुः}
{श्रीरामः स्वप्रजाः पुत्रानिव पाति च रक्षकः}% ४९

\twolineshloka
{यो महत्कर्मदुःसाध्यं कृतवान्सुलभं न तत्}
{समुद्रं यो निजग्राह सेतुं तत्र बबन्ध च}% ५०

\twolineshloka
{रावणं यो रिपुं हत्वा लङ्कां सम्भज्य वानरैः}
{जानकीमाजहारात्र महदाचारमाचरत्}% ५१

\twolineshloka
{इति प्रोक्तं समाकर्ण्य वचः सुमधुराक्षरम्}
{पतिः स्मितं चकारेमां वाक्यं पुनरथाब्रवीत्}% ५२

\twolineshloka
{मुग्धेनेदं महत्कर्म रामचन्द्रस्य भामिनी}
{दशाननवधादीनि समुद्र दमनानि च}% ५३

\twolineshloka
{लीलयायोऽवनिं प्राप्तो ब्रह्मादिप्रार्थितो महान्}
{करोति सच्चरित्राणि महापापहराणि च}% ५४

\twolineshloka
{मा जानीहि नरं रामं कौसल्यानन्ददायकम्}
{सृजत्यवति हन्त्येतद्विश्वं लीलात्तमानुषः}% ५५

\twolineshloka
{धन्या वयं ये रामस्य पश्यामो मुखपङ्कजम्}
{ब्रह्मादिसुरदुर्दर्शं महत्पुण्यकृतो वयम्}% ५६

\twolineshloka
{अशृणोद्रामचन्द्रस्य चरित्रं श्रुतिसौख्यदम्}
{इत्यादिवाक्यं शुश्राव चारो द्वारिस्थितो मुहुः}% ५७

\twolineshloka
{अन्यो ह्यन्यं गृहं गत्वा तस्थौ श्रोतुं हरेर्यशः}
{तत्रापि रामभद्रस्य यशः शुश्राव शोभनम्}% ५८

\twolineshloka
{खेलन्ती स्वामिना सार्धं द्यूतेन सुमनोहरा}
{उवाच वाक्यं मधुरं नर्तयन्तीव कङ्कणे}% ५९

\fourlineindentedshloka
{जितं मया कान्त जवेन सर्वं}
{करिष्यसि त्वं किमु हारिमानसः}
{इत्यादि वाक्यं परिहासपूर्वकं}
{कृत्वा स्वकान्तं परिषस्वजे मुदा}% ६०

\twolineshloka
{उवाच कान्तश्चार्वङ्गि जितमेव सुशोभने}
{रामं मे स्मरतो नित्यं न कुत्रापि पराजयः}% ६१

\twolineshloka
{इदानीं त्वां तु जेष्यामि रामं स्मृत्वा मनोहरम्}
{देवा यथा पुरा स्मृत्वा दितिजानजयन्क्षणात्}% ६२

\twolineshloka
{एवमुक्त्वा पाशकानां परिवर्तनमाकरोत्}
{तावज्जयं प्रपेदेऽसौ हर्षितो वाक्यमब्रवीत्}% ६३

\twolineshloka
{मम प्रोक्तमृतं जातं जिता त्वं नवयौवना}
{रामस्मारी कदाप्येव न भवेद्रिपुतो भयी}% ६४

\twolineshloka
{इत्येवं तौ वदन्तौ च परस्परमथोत्सुकौ}
{परिरभ्य दृढं प्रेम्णा ततश्चारो गतो गृहम्}% ६५

\twolineshloka
{एवं पञ्चमहाचारा राज्ञः संश्रुत्य वै यशः}
{परस्परं प्रशंसन्तो गेहं स्वं स्वं ययुर्मुदा}% ६६

\twolineshloka
{एकः षष्ठश्चरः कारुगेहानालोक्य तत्र ह}
{जगाम श्रोतुकामोऽसौ यशो राज्ञो महीपतेः}% ६७

\twolineshloka
{रजकः क्रोधसंस्पृष्टो भार्यामन्यगृहोषिताम्}
{पदा सन्ताडयामास धिक्कुर्वञ्छोणनेत्रवान्}% ६८

\twolineshloka
{गच्छ त्वं मद्गृहात्तस्य गेहं यत्रोषिता दिनम्}
{नाहं गृह्णामि भवतीं दुष्टां वचनलङ्घिनीम्}% ६९

\twolineshloka
{तदास्य माता प्रोवाच मा त्यजैनां गृहागताम्}
{अपराधेन रहितां दुष्टकर्मविवर्जिताम्}% ७०

\twolineshloka
{मातरं प्रत्युवाचाथ रजकः क्रोधसंयुतः}
{नाहं रामइव प्रेष्ठां गृह्णाम्यन्यगृहोषिताम्}% ७१

\twolineshloka
{स राजा यत्करोत्येव तत्सर्वं नीतिमद्भवेत्}
{अहं गृह्णामि नो भार्यां परवेश्मनि संस्थिताम्}% ७२

\twolineshloka
{पुनःपुनरुवाचेदं नाहं रामो महीश्वरः}
{यः परस्य गृहे संस्थां जानकीं वै ररक्ष सः}% ७३

\twolineshloka
{इति वाक्यं समाश्रुत्य चारः क्रोधपरिप्लुतः}
{खड्गं गृहीत्वा स्वकरे तं हन्तुं विदधे मनः}% ७४

\twolineshloka
{स रामोक्तं च सस्मार न वध्यः कोपि मे जनः}
{इति ज्ञात्वा सरोषं तु सञ्जहार महामनाः}% ७५

\twolineshloka
{तदा श्रुत्वा सुदुःखार्तः पञ्चचारा यतः स्थिताः}
{ततो गतः प्रकुपितो निःश्वसन्मुहुरुच्छ्वसन्}% ७६

\twolineshloka
{ते वै परस्परं तत्र मिलितास्तु समब्रुवन्}
{स्वश्रुतं रामचरितं सर्वलोकैकपूजितम्}% ७७

\twolineshloka
{ते तद्भाषितमाकर्ण्य परस्परममन्त्रयन्}
{न वाच्यं रघुनाथाया वाच्यं दुष्टजनोदितम्}% ७८

\twolineshloka
{इति सम्मन्त्र्य ते गेहं गत्वा सुषुपुरुत्सुकाः}
{प्राता राज्ञे प्रशंसाम इति बुद्ध्या व्यवस्थिताः}% ७९

{॥इति श्रीपद्मपुराणे पातालखण्डे शेषवात्स्यायनसंवादे रामाश्वमेधे चारनिरीक्षणं नाम पञ्चपञ्चाशत्तमोऽध्यायः॥५५॥}

\dnsub{षट्पञ्चाशत्तमोऽध्यायः}\resetShloka

\uvacha{शेष उवाच}

\twolineshloka
{प्रातर्नित्यं विधायासौ ब्राह्मणान्वेदवित्तमान्}
{हिरण्यदानैः सन्तर्प्य विधिवत्संसदं ययौ}% १

\twolineshloka
{लोकाः सर्वे नमस्कर्तुं रघुनाथं महीपतिम्}
{पुत्रवत्स्वप्रजाः सर्वाः पालयन्तं ययुः सभाम्}% २

\twolineshloka
{लक्ष्मणेनातपत्रं तु धृतं मूर्धनि भूपतेः}
{तदा भरतशत्रुघ्नौ चामरद्वन्द्व धारिणौ}% ३

\twolineshloka
{वसिष्ठप्रमुखास्तत्र मुनयः पर्युपासत}
{सुमन्त्रप्रमुखास्तत्र मन्त्रिणो न्यायकर्तृकाः}% ४

\twolineshloka
{एवं प्रवृत्ते समये षट्चारास्ते स्वलङ्कृताः}
{समाजग्मुर्नरपतिं नमस्कर्तुं सभास्थितम्}% ५

\twolineshloka
{तान्वक्तुकामान्संवीक्ष्य चारान्नृपतिसत्तमः}
{सभायामन्तरावेश्म रहः प्राविशदुत्सुकः}% ६

\twolineshloka
{एकान्ते तांश्चरान्सर्वान्पप्रच्छ सुमतिर्नृपः}
{कथयन्तु चरा मह्यं यथातथ्यमरिन्दमाः}% ७

\twolineshloka
{लोका ब्रुवन्ति मां कीदृग्भार्याया मम कीदृशम्}
{मन्त्रिणां कीदृशं लोका वदन्ति चरितं कथम्}% ८

\twolineshloka
{इति वाक्यं समाकर्ण्य चारा राममथाब्रुवन्}
{मेघगम्भीरया वाचा पृच्छन्तं रघुनायकम्}% ९

\uvacha{चारा ऊचुः}

\twolineshloka
{नाथ कीर्तिर्जनान्सर्वान्पावयत्यधुना भुवि}
{गृहेगृहे श्रुतास्माभिः पुरुषैः स्त्रीभिरीडिता}% १०

\twolineshloka
{विवस्वतो महान्वंशो भवता परमेष्ठिना}
{अलङ्कर्तुं गतं भूमौ कीर्तिर्विस्तारिता बहुः}% ११

\twolineshloka
{अनेके सगराद्याश्च कीर्तिमन्तो महाबलाः}
{अभवंस्तादृशी कीर्तिस्तेषां नाभूद्यथा तव}% १२

\twolineshloka
{त्वया नाथेन सकलाः कृतार्थाश्च प्रजाः कृताः}
{यासां नाकालमरणं न च रागाद्युपद्रुतिः}% १३

\twolineshloka
{यादृशश्चन्द्रमालोके यादृशी जाह्नवी सरित्}
{तादृक्तव च सत्कीर्तिः प्रकाशयति भूतलम्}% १४

\twolineshloka
{ब्रह्मादिका भवत्कीर्तिमाकर्ण्य त्रपिता भृशम्}
{नाथ सर्वत्र ते कीर्तिः पावयत्यधुना जनान्}% १५

\twolineshloka
{वयं धन्यतमाः सर्वे ये चारास्तव भूपते}
{क्षणेक्षणे तव मुखं लोकयाम मनोहरम्}% १६

\twolineshloka
{इत्यादिवाक्यं चाराणां पञ्चानां वीक्ष्य राघवः}
{षष्ठं पप्रच्छ चारं तं विलक्षणमुखाङ्कितम्}% १७

\uvacha{राम उवाच}

\twolineshloka
{सत्यं वद महाबुद्धे यच्छ्रुतं लोकसङ्करे}
{तादृक्प्रशंस मे सर्वमन्यथा पातकादिकृत्}% १८

\twolineshloka
{पुनः पुनश्च तं रामः पप्रच्छाशु सविस्तरम्}
{तथापि न ब्रवीत्येव रामं लौकिकभाषितम्}% १९

\twolineshloka
{तदा रामः प्रत्यवोचच्चारं मुखविलक्षितम्}
{शपामि त्वां तु सत्येन शंस सर्वं यथातथम्}% २०

\twolineshloka
{तदा रामं प्रत्युवाच चारो वाक्यं शनैः शनैः}
{अकथ्यमपि ते वाच्यं वाक्यं कारुजनोदितम्}% २१

\uvacha{चार उवाच}

\twolineshloka
{स्वामिन्सर्वत्र ते कीर्तिर्दशाननवधादिका}
{अन्यत्र राक्षसगृहे स्थितायास्ते स्त्रिया अहो}% २२

\twolineshloka
{कारुरेकस्तु रजको निशीथे महिलां स्वकाम्}
{अन्यगेहोषितां दृष्ट्वा धिक्कुर्वन्समताडयत्}% २३

\twolineshloka
{तन्माता प्रत्युवाचेमां कथं ताडयसेऽनघाम्}
{गृहाण मा कृथा निन्दां स्त्रियं मद्वाक्यमाचर}% २४

\twolineshloka
{तदावोचत्स रजको नाहं रामो महीपतिः}
{यद्राक्षसगृहेध्युष्टां सीतामङ्गीचकार सः}% २५

\twolineshloka
{सर्वं राज्ञः कृतं कर्म नीतिमद्भवति प्रभो}
{अन्येषां पुण्यकर्तॄणामपि कृत्यमनीतिमत्}% २६

\twolineshloka
{पुनः पुनरुवाचासौ नाहं रामो महीपतिः}
{चुक्रुधे समये राजन्मया वाक्यं तव स्मृतम्}% २७

\twolineshloka
{तदानीं शिर आच्छिद्य पातयामि महीतले}
{कृतः पुनर्विचारोमे क्व रामो रजकः क्व नु}% २८

\twolineshloka
{अयं दुष्टोऽनृतं वक्ति न हीदं तथ्यमुच्यते}
{आज्ञापयसि चेद्राम साम्प्रतं मारयामि तम्}% २९

\twolineshloka
{अवाच्यमपि ते प्रोक्तं त्वदाग्रहत उन्नयम्}
{राजा प्रमाणमत्रेदं विचारयतु सङ्गतम्}% ३०

\uvacha{शेष उवाच}

\twolineshloka
{इति वाक्यं समाकर्ण्य महावज्रनिभाक्षरम्}
{निःश्वसन्मुहुरुच्छ्वासमाचरन्मूर्च्छितोऽपतत्}% ३१

\twolineshloka
{तं मूर्च्छितं नृपं दृष्ट्वा चारा दुःखसमन्विताः}
{वीजयामासुर्वासोग्रैर्दुःखापनय हेतवे}% ३२

\twolineshloka
{स लब्धसंज्ञो नृपतिर्मुहूर्तेन जगाद तान्}
{गच्छन्तु भरतं शीघ्रं प्रेषयन्तु च मां प्रति}% ३३

\twolineshloka
{ते दुःखिताश्चरास्तूर्णं भरतस्य गृहं गताः}
{कथयामासु रामस्य सन्देशं नयहारकाः}% ३४

\twolineshloka
{भरतो रामसन्देशं श्रुत्वा धीमान्ययौ सदः}
{रामं प्रति रहःसंस्थं श्रुत्वा तं त्वरया युतः}% ३५

\twolineshloka
{आगत्य तं प्रतीहारं प्रत्युवाच महामनाः}
{कुत्रास्ते रामभद्रोऽसौ मम भ्राता कृपानिधिः}% ३६

\twolineshloka
{तन्निर्दिष्टं गृहं वीरो ययौ रत्नमनोहरम्}
{रामं विलोक्य विक्लान्तं भयमाप स मानसे}% ३७

\twolineshloka
{किं वासौ कुपितो रामः किं वा दुःखमिदं विभोः}
{तदा प्रोवाच नृपतिं निःश्वसन्तं मुहुर्मुहुः}% ३८

\twolineshloka
{स्वामिन्सुखसमाराध्यं वक्त्रं ते कथमानतम्}
{अश्रुभिर्लक्ष्यते राहुग्रस्तदेहः शशीव ते}% ३९

\twolineshloka
{सर्वं मे कारणं तथ्यं ब्रूहि मां किं करोमि ते}
{त्यज दुःखं महाराज कथं दुःखस्य भाजनम्}% ४०

\twolineshloka
{एवं भ्रात्रा प्रोच्यमानो गद्गदस्वरया गिरा}
{प्रोवाच भ्रातरं वीरो रामचन्द्रश्च धार्मिकः}% ४१

\twolineshloka
{शृणु भ्रातर्वचो मह्यं मम दुःखस्य कारणम्}
{तन्मार्जनं कुरुष्वाद्य भ्रातः प्रातर्महामते}% ४२

\twolineshloka
{वंशे वैवस्वते राजा न कश्चिदयशः क्षतः}
{मत्कीर्तिरद्य कलुषा गङ्गायमुनया गता}% ४३

\twolineshloka
{येषां यशो नृणां भूमौ तेषामेव सुजीवितम्}
{अपकीर्तिक्षतानां तु जीवितं मृतकैः समम्}% ४४

\twolineshloka
{येषां यशो भवेद्भूमौ तेषां लोकाः सनातनाः}
{अपकीर्त्युरगी दष्टास्तेषां भूयादधोगतिः}% ४५

\twolineshloka
{अद्य मे कलुषा कीर्तिः स्वर्धुनी लोकविश्रुता}
{तच्छृणुष्व वचो मेऽद्य रजकेन यथोदितम्}% ४६

\twolineshloka
{अस्मिन्पुरेऽद्य रजक उक्तवाञ्जानकीभवम्}
{किञ्चिद्वाच्यं ततो भ्रातः किं करोमि महीतले}% ४७

\twolineshloka
{किमात्मानं जहाम्यद्य किमेनां जानकीं स्त्रियम्}
{उभयोः किं मया कार्यं तत्तथ्यं ब्रूहि मे भवान्}% ४८

\twolineshloka
{इत्युक्त्वा निर्गलद्बाष्पो वेपथु क्षुभिताङ्गकः}
{पपात भूमौ विरजो धार्मिकाणां शिरोमणिः}% ४९

\twolineshloka
{भ्रातरं पतितं दृष्ट्वा भरतो दुःखसंयुतः}
{संवीक्ष्य शनकै रामं प्राप्तसंज्ञं चकार सः}% ५०

\twolineshloka
{संज्ञां प्राप्तं तु संवीक्ष्य रामचन्द्रं सुदुःखितम्}
{उवाच दुःखनाशाय वाक्यं तु सुमनोहरम्}% ५१

\twolineshloka
{कोऽयं वै रजकः किन्तु वाच्यं वाक्यं यथाब्रवीत्}
{जिह्वाच्छेदं करिष्यामि जानकीवाच्यकारिणः}% ५२

\twolineshloka
{तदा रामोऽब्रवीद्वाक्यं रजकस्य मुखोद्गतम्}
{श्रुतं चारेण तत्सर्वं भरताय महात्मने}% ५३

\twolineshloka
{तच्छ्रुत्वा भरतः प्राह भ्रातरं दुःखशोकिनम्}
{जानकीवह्निशुद्धाभूल्लङ्कायां वीरपूजिता}% ५४

\twolineshloka
{ब्रह्माब्रवीदियं शुद्धा पिता दशरथस्तव}
{कथं सा रजकोक्तित्वाद्धातव्या लोकपूजिता}% ५५

\twolineshloka
{ब्रह्मादिसंस्तुता कीर्तिस्तवलोकान्पुनाति हि}
{सा कथं रजकोक्त्या वै कलुषाद्य भविष्यति}% ५६

\twolineshloka
{तस्मात्त्यज महादुःखं सीतावाच्यसमुद्भवम्}
{कुरु राज्यं तया सार्धमन्तर्वत्न्या सुभाग्यया}% ५७

\twolineshloka
{त्वं कथं स्वशरीरं तु हातुमिच्छसि शोभनम्}
{वयं हताः स्म सर्वेऽद्य त्वां विना दुःखनाशकम्}% ५८

\twolineshloka
{क्षणं सीता न जीवेत त्वां विना सुमहोदया}
{तस्मात्पतिव्रता साकं भुनक्तु विपुलां श्रियम्}% ५९

\twolineshloka
{इति वाक्यं समाकर्ण्य भरतस्य च धार्मिकः}
{पुनरेव जगादेमं वाक्यं वाक्यविदां वरः}% ६०

\twolineshloka
{यत्त्वं कथयसि भ्रातस्तत्सर्वं धर्मसंयुतम्}
{परं यद्वच्म्यहं वाक्यं तत्कुरुष्व ममाज्ञया}% ६१

\twolineshloka
{जानाम्येनां वह्निशुद्धां पवित्रां लोकपूजिताम्}
{लोकापवादाद्भीतोऽहं त्यजामि स्वां तु जानकीम्}% ६२

\twolineshloka
{तस्मात्करे शितं धृत्वा करवालं सुदारुणम्}
{शिरश्छिन्ध्यथवा जायां जानकीं मुञ्च वै वने}% ६३

\twolineshloka
{इति वाक्यं समाकर्ण्य रामस्य भरतोऽपतत्}
{मूर्च्छितः सन्क्षितौ देहे कम्पयुक्तः सबाष्पकः}% ६४

{॥इति श्रीपद्मपुराणे पातालखण्डे शेषवात्स्यायनसंवादे रामाश्वमेधे भरतवाक्यं नाम षट्पञ्चाशत्तमोऽध्यायः॥५६॥}

\dnsub{सप्तपञ्चाशत्तमोऽध्यायः}\resetShloka

\uvacha{वात्स्यायन उवाच}

\twolineshloka
{जगत्पवित्रसत्कीर्ति जानक्या वाच्यवाचनम्}
{कथं समकरोत्स्वामी तन्मे कथय सुव्रत}% १


\threelineshloka
{यथा मे मनसः सौख्यं भविष्यति सुशोभनम्}
{तथा कुरुष्व शेषाद्य त्वन्मुखान्निःसृतामृतम्}
{पिबतस्तृप्तिरेव स्याद्यया संसृतिकृं तनम्}% २

\uvacha{शेष उवाच}

\twolineshloka
{मिथिलायां महापुर्यां जनको नाम भूपतिः}
{तस्यां करोति सद्राज्यं धर्मेणाराधयन्प्रजाः}% ३

\twolineshloka
{तस्य सङ्कर्षतो भूमिं सीतया दीर्घमुख्यया}
{सीरध्वजस्य निरगात्कुमारी ह्यतिसुन्दरी}% ४

\twolineshloka
{तदात्यन्तं मुदं प्राप्तः सीरकेतुर्महीपतिः}
{सीता नामाकरोत्तस्या मोहिन्या जगतः श्रियः}% ५

\twolineshloka
{सैकदोद्यानविपिने खेलन्ती सुमनोहरा}
{अपश्यत्स्वमनःकान्तं शुकशुक्योर्युगं वदत्}% ६

\twolineshloka
{अत्यन्तं हर्षमापन्नमत्यन्तं कामलोलुपम्}
{परस्परं भाषमाणं स्नेहेन शुभया गिरा}% ७

\twolineshloka
{रममाणं तदा युग्मं नभसि क्षिप्रवेगतः}
{समुत्पतन्नगोपस्थे स्थितं शब्दं चकार तत्}% ८

\twolineshloka
{रामो महीपतिर्भूमौ भविष्यति मनोहरः}
{तस्य सीतेति नाम्ना तु भविष्यति महेलिका}% ९

\twolineshloka
{स तया सह वर्षाणां सहस्राण्येकयुग्दश}
{राज्यं करिष्यते धीमान्कर्षन्भूमिपतीन्बली}% १०

\twolineshloka
{धन्या सा जानकी देवी धन्योऽसौ रामसंज्ञितः}
{यौ परस्परमापन्नौ पृथिव्यां रंस्यतो मुदा}% ११

\twolineshloka
{इति सम्भाष्यमाणं तु शुकयुग्मं तु मैथिली}
{ज्ञात्वेदं देवतायुग्मं वाणीं तस्य विलोक्य च}% १२

\twolineshloka
{मदीयास्तु कथा रम्याः कुरुते शुकयोर्युगम्}
{एतद्गृहीत्वा पृच्छामि सर्वं वाक्यं गतार्थकम्}% १३

\twolineshloka
{एवं विचार्य सा स्वीयाः सखीः प्रतिजगाद ह}
{गृह्णन्तु शनकैरेतत्पक्षियुग्मं मनोहरम्}% १४

\twolineshloka
{सख्यस्तास्तद्गिरिं गत्वा गृह्णन्पक्षियुगं वरम्}
{निवेदयामासुरिदं स्वसख्याः प्रियकाम्यया}% १५

\twolineshloka
{बहुधा विविधाञ्छब्दान्कुर्वद्वीक्ष्य मनोहरम्}
{आश्वासयामास तदा पप्रच्छ तदिदं वचः}% १६

\uvacha{सीतोवाच}

\twolineshloka
{मा भैषातां युवां रम्यौ कौ वां कुत्र समागतौ}
{को रामः का च सा सीता तज्ज्ञानं तु कुतः स्मृतम्}% १७

\twolineshloka
{तत्सर्वं शंसतं क्षिप्रं मत्तो वां व्येतु यद्भयम्}
{इति पृष्टं तया पक्षियुगं सर्वमशंसत}% १८

\uvacha{पक्षियुग्ममुवाच}

\twolineshloka
{वाल्मीकिरास्ते सुमहानृषिर्धर्मविदुत्तमः}
{आवां तदाश्रमस्थाने सर्वदा सुमनोहरे}% १९

\twolineshloka
{सशिष्यान्गापयामास भावि रामायणं मुनिः}
{प्रत्यहं तत्पदस्मारी सर्वभूतहिते रतः}% २०

\twolineshloka
{तदावाभ्यां श्रुतं सर्वं भावि रामायणं महत्}
{मुहुर्मुहुर्गीयमानमायातं परिपाठतः}% २१

\twolineshloka
{शृण्वावां तेऽभिधास्यावो यो रामो या च जानकी}
{यद्यद्भविष्यते तस्या रामेण क्रीडितात्मना}% २२

\twolineshloka
{ऋष्यशृङ्गकृतेष्ट्यां च चतुर्धा त्वङ्गतो हरिः}
{प्रादुर्भविष्यति श्रीमान्सुरस्त्रीगीतसत्कथः}% २३

\twolineshloka
{स कौशिकेन मिथिलां प्राप्स्यते भ्रातृसंयुतः}
{धनुष्पाणिस्तदा दृष्ट्वा दुर्ग्राह्यमन्यभूभुजाम्}% २४

\twolineshloka
{धनुर्भङ्क्त्वा जनकजां प्राप्स्यते सुमनोहराम्}
{तया सह महद्राज्यं करिष्यति श्रुतं वरे}% २५

\twolineshloka
{एतदन्यच्च तत्रस्थैः श्रुतमस्माभिरुद्गतैः}
{कथितं तव चार्वङ्गि मुञ्चावां गन्तुकामुकौ}% २६

\twolineshloka
{इति वाक्यं तयोर्धृत्वा श्रोत्रयोः सुमनोहरम्}
{पुनरेवजगादेदं वाक्यं पक्षियुगं प्रति}% २७

\twolineshloka
{स रामः कुत्र वर्तेत कस्य पुत्रः कथं तु ताम्}
{परिग्रहीष्यति वरः कीदृग्रूपधरो नरः}% २८

\twolineshloka
{मया पृष्टमिदं सर्वं कथयन्तु यथातथम्}
{पश्चात्सर्वं करिष्यामि प्रियं युष्मन्मनोहरम्}% २९

\twolineshloka
{तच्छ्रुत्वा तां तु कामेन पीडितां वीक्ष्य सा शुकी}
{जानकीं हृदये ज्ञात्वा पपाठ पुरतस्ततः}% ३०

\twolineshloka
{सूर्यवंशध्वजो धीमान्राजा पङ्क्तिरथो बली}
{यं देवाः श्रित्य सर्वारीन्विजेष्यन्ति च सर्वशः}% ३१

\twolineshloka
{तस्य भार्यात्रयं भावि शक्रमोहनरूपधृक्}
{तस्मिन्नपत्य चातुष्कं भविष्यति बलोन्नतम्}% ३२

\twolineshloka
{सर्वेषामग्रजो रामो भरतस्तदनुस्मृतः}
{लक्ष्मणस्तदनु श्रीमाञ्छत्रुघ्नः सर्वतो बली}% ३३

\twolineshloka
{रघुनाथ इति ख्यातिं गमिष्यति महामनाः}
{तेषामनन्तनामानि रामस्य बलिनः सखि}% ३४

\fourlineindentedshloka
{पद्मकोश इव शोभनं मुखं}
{पङ्कजाभनयने सुदीर्घके}
{उन्नता पृथुमनोहरा नसा}
{वल्गुसङ्गत मनोहरे भ्रुवौ}% ३५

\fourlineindentedshloka
{जानुलम्बित मनोहरौ भुजौ}
{कम्बुशोभिगलकोऽतिह्रस्वकः}
{सत्कपाटतलविस्तृतश्रिकं}
{वक्ष एतदमलं सलक्ष्मकम्}% ३६

\fourlineindentedshloka
{शोभनोरुकटिशोभया युतं}
{जानुयुग्मममलं स्वसेवितम्}
{पादपद्ममखिलैर्निजैः सदा}
{सेवितं रघुपतिं सुशोभनम्}% ३७

\twolineshloka
{एतद्रूपधरो रामो मया किं तु स वर्ण्यते}
{शताननोपि नो याति पक्षिणः किमु मादृशाः}% ३८

\twolineshloka
{यद्रूपं वीक्ष्य ललिता मनोहरवपुर्धरा}
{लक्ष्मीर्मुमोह भुविका वर्तते या न मोहति}% ३९

\twolineshloka
{महाबलो महावीर्यो महामोहनरूपधृक्}
{किं वर्णयामि श्रीरामं सर्वैश्वर्यगुणान्वितम्}% ४०

\twolineshloka
{धन्या सा जानकीदेवी महामोहनरूपधृक्}
{रंस्यते येन सहिता वर्षाणामयुतं मुदा}% ४१

\twolineshloka
{त्वं कासि किं तु नामासि बत सुन्दरि यत्तु माम्}
{परिपृच्छसि वैदग्ध्याद्रामकीर्तनमादरात्}% ४२

\twolineshloka
{एतद्वाक्यं समाकर्ण्य जानकी पक्षिणोर्युगम्}
{उवाच जन्मललितं शंसन्ती स्वस्य मोहनम्}% ४३

\twolineshloka
{या त्वया जानकी प्रोक्ता साहं जनकपुत्रिका}
{स रामो मां यदाभ्येत्य प्राप्स्यते सुमनोहरः}% ४४

\twolineshloka
{तदा वां मोचयाम्यद्धानान्यथा वाक्यलोभिता}
{लालयामि सुखेनास्तां मद्गेहे मधुराक्षरौ}% ४५

\twolineshloka
{इत्युक्तं तत्समाकर्ण्य वेपतुर्भयतां गतौ}
{परस्परं प्रक्षुभितौ जानकीमित्यवोचताम्}% ४६

\twolineshloka
{वयं वै पक्षिणः साध्वि वनस्था वृक्षगोचराः}
{परिभ्रमाम सर्वत्र नो सुखं नो भवेद्गृहे}% ४७

\twolineshloka
{अन्तर्वत्नी स्वके स्थाने गत्वा संसूय पुत्रकान्}
{त्वत्स्थानमागमिष्यामि सत्यं मे ह्युदितं वचः}% ४८

\twolineshloka
{एवं प्रोक्ता तया सा तु न मुमोच शुकीं स्वयम्}
{तदापतिस्तां प्रोवाच विनीतवदनुत्सुकः}% ४९

\twolineshloka
{सीते मुञ्च कथं भार्यां रक्षसे मे मनोहरीम्}
{आवां गच्छाव विपिने विचरावः सुखं वने}% ५०

\twolineshloka
{अन्तर्वत्नी तु वर्तेत भार्या मम मनोरमा}
{तस्याः प्रसूतिं कृत्वा त्वामागमिष्यामि शोभने}% ५१

\twolineshloka
{इत्युक्ता निजगादेमं सुखं गच्छ महामते}
{एतां रक्षामि सुखिनीं मत्पार्श्वे प्रियकारिणीम्}% ५२

\twolineshloka
{इत्युक्तो दुःखितः पक्षी तामूचे करुणान्वितः}
{योगिभिः प्रोच्यते यद्वै तद्वचस्तथ्यमेव हि}% ५३

\twolineshloka
{न वक्तव्यं न वक्तव्यं मौनमाश्रित्य तिष्ठताम्}
{नोचेत्स वाक्यदोषेण प्राप्नोत्यालानमुन्मदः}% ५४

\twolineshloka
{वयं चेदत्र वाक्यं नाकरिष्याम नगोपरि}
{बन्धनं कथमावां स्यात्तस्मान्मौनं समाचरेत्}% ५५

\twolineshloka
{इत्युक्त्वा तां प्रत्युवाच नाहं जीवामि सुन्दरि}
{एतया भार्यया सीते तस्मान्मुञ्च मनोहरे}% ५६

\twolineshloka
{अनेकविधवाक्यैः सा बोधिता नामुचत्तदा}
{कुपिता दुःखिता भार्या शशाप जनकात्मजाम्}% ५७

\twolineshloka
{यथा त्वं पतिना सार्धं वियोजयसि मामितः}
{तथा त्वमपि रामेण वियुक्ता भव गर्भिणी}% ५८

\twolineshloka
{इत्युक्तवत्यां तस्यां तु दुःखितायां पुनः पुनः}
{प्राणा निरगमन्दुःखात्पतिदुःखेन पूरितात्}% ५९

\twolineshloka
{रामं रामं स्मरन्त्याश्च वदन्त्यांश्च पुनः पुनः}
{विमानमागतं सुष्ठु पक्षिणी स्वर्गता बभौ}% ६०

\twolineshloka
{तस्यां मृतायां दुःखार्तो भर्ता तस्याः स पक्षिराट्}
{परमं क्रोधमापन्नो जाह्नव्यां दुःखितोऽपतत्}% ६१

\twolineshloka
{तथा भवामि रामस्य नगरे जनपूरिते}
{मद्वाक्यादियमुद्विग्ना वियोगेन सुदुःखिता}% ६२

\twolineshloka
{इत्युक्त्वा स पपातोदे जाह्नव्या भ्रमशोभिते}
{दुःखितः कुपितो भीतस्तद्वियोगेन कम्पितः}% ६३

\twolineshloka
{क्रुद्धत्वाद्दुःखितत्वाच्च सीताया अपमाननात्}
{अन्त्यजत्वं परं प्राप्तो रजकः क्रोधनाभिधः}% ६४

\twolineshloka
{यः क्रोधाच्च स्वकान्प्राणान्महतां दुष्टमाचरन्}
{सन्त्यजेत्स मृतो याति अन्त्यजत्वं द्विजोत्तमः}% ६५

\twolineshloka
{तज्जातं रजकोक्त्यासौ निन्दिता च वियोगिता}
{रजकस्य च शापेन वियुक्ता सा वनं गता}% ६६

\twolineshloka
{एतत्ते कथितं विप्र यत्ते पृष्टं विदेहजाम्}
{पुनरत्र परं वृत्तं शृणुष्व निगदामि तत्}% ६७


{॥इति श्रीपद्मपुराणे पातालखण्डे शेषवात्स्यायनसंवादे रामाश्वमेधे रजकप्राग्जन्मकथनन्ना सप्तपञ्चाशत्तमोऽध्यायः॥५७॥}

\dnsub{अष्टपञ्चाशत्तमोऽध्यायः}\resetShloka

\uvacha{शेष उवाच}

\twolineshloka
{भरतं मूर्च्छितं दृष्ट्वा रघुनाथः सुदुःखितः}
{प्रतीहारमुवाचेदं शत्रुघ्नं प्रापयाशु माम्}% १

\twolineshloka
{तद्वाक्यं प्रोक्तमाकर्ण्य क्षणाच्छत्रुघ्नमानयत्}
{यत्र रामो निजभ्राता भरतेन सह स्थितः}% २

\twolineshloka
{भरतं मूर्च्छितं दृष्ट्वा रघुनाथं च दुःखितम्}
{प्रणम्य दुःखितोऽवोचत्किमिदं दारुणं महत्}% ३

\twolineshloka
{तदा रामोऽन्त्यजप्रोक्तं वाक्यं लोकविगर्हितम्}
{तं प्रत्युवाच रामोऽसौ शत्रुघ्नं पदसेवकम्}% ४

\twolineshloka
{अधोमुखो दीनरवो गद्गदाक्षरवेपथुः}
{शृणु भ्रातर्वचो मेऽद्य कुरु तत्क्षिप्रमादरात्}% ५

\twolineshloka
{यथा स्याद्विमलाकीर्तिर्गङ्गेव पृथिवीं गता}
{सीताया वाच्यमतुलं लोके श्रुत्वान्त्यजोदितम्}% ६

\twolineshloka
{हातुमिच्छामि देहं स्वमेनां वा जानकीं किल}
{इति वाक्यं समाकर्ण्य रामस्य किल शत्रुहा}% ७

\twolineshloka
{सवेपथुः पपातोर्व्यां दुःखितः परदारणः}
{संज्ञां प्राप्य मुहूर्तेन रघुनाथमवोचत}% ८

\uvacha{शत्रुघ्न उवाच}

\twolineshloka
{किमेतदुच्यते स्वामिञ्जानकीं प्रति दारुणम्}
{पाखण्डैर्दुष्टचित्तैश्च सर्वधर्मबहिष्कृतैः}% ९

\twolineshloka
{निन्दिता श्रुतिरग्राह्या भवति त्वग्र्यजन्मनाम्}
{जाह्नवी सर्वलोकानां पापघ्नी दुरितापहा}% १०

\twolineshloka
{निस्पृष्टा पापिभिः पुम्भिः सा स्पर्शेनार्हिता सताम्}
{सूर्यो जगत्प्रकाशाय समुदेति जगत्यहो}% ११

\twolineshloka
{उलूकानां रुचिकरो न भवेत्तत्र का क्षतिः}
{तस्मात्त्वमेनां गृह्णीष्व मा त्यजानिन्दितां स्त्रियम्}% १२

\twolineshloka
{श्रीरामभद्रकृपया कुरुष्व वचनं मम}
{एतच्छ्रुत्वा वचस्तस्य शत्रुघ्नस्य महात्मनः}% १३

\twolineshloka
{पुनःपुनर्जगादेदं यदुक्तं भरतं प्रति}
{तन्निशम्य वचो भ्रातुर्दुःखपूरपरिप्लुतः}% १४

\twolineshloka
{पपात मूर्च्छितो भूमौ छिन्नमूल इव द्रुमः}
{भ्रातरं पतितं वीक्ष्य शत्रुघ्नं दुःखितो भृशम्}% १५

\twolineshloka
{प्रतीहारमुवाचेदं लक्ष्मणं त्वानयान्तिकम्}
{स लक्ष्मणगृहं गत्वा न्यवेदयदिदं वचः}% १६

\uvacha{प्रतीहार उवाच}

\twolineshloka
{स्वामिन्रामो भवन्तं तु समाह्वयति वेगतः}
{स तच्छ्रुत्वा समाह्वानं रामचन्द्रेण वेगतः}% १७

\twolineshloka
{जगाम तरसा तत्र यत्र स भ्रातृकोऽनघः}
{भरतं मूर्च्छितं दृष्ट्वा शत्रुघ्नमपि मूर्च्छितम्}% १८

\twolineshloka
{श्रीरामचन्द्रं दुःखार्तं दुःखितो वाक्यमब्रवीत्}
{किमेतद्दारुणं राजन्दृश्यते मूर्च्छनादिकम्}% १९

\twolineshloka
{तदाशु शंस मां सर्वं कारणं मुख्यतोऽनघ}
{एवं वदन्तं नृपतिर्वृत्तान्तं सर्वमादितः}% २०

\twolineshloka
{शशंस लक्ष्मणं क्षिप्रं दुःखपूरपरिप्लुतम्}
{लक्ष्मणस्तद्वचः श्रुत्वा सीतायास्त्यागसम्भवम्}% २१

\twolineshloka
{निःश्वसन्मुहुरुच्छ्वासं स्तब्धगात्र इवाभवत्}
{भ्रातरं स्तब्धगात्रं च कम्पमानं मुहुर्मुहुः}% २२

\twolineshloka
{न किञ्चन वदन्तं तं वीक्ष्य शोकार्दितोऽब्रवीत्}
{किं करिष्याम्यहं भूमौ स्थित्वा दुर्यशसाङ्कितः}% २३

\twolineshloka
{त्यजामीदं वपुः श्रीमल्लोकभीत्या च शोकवान्}
{सर्वदा भ्रातरो मह्यं मद्वाक्यकरणोत्सुकाः}% २४

\twolineshloka
{इदानीं तेपि दैवेन प्रतिकूलवचः कराः}
{कुत्र गच्छामि कं यामि हसिष्यन्ति नृपा भुवि}% २५

\twolineshloka
{दुर्यशो लाञ्छितं मां वै कुष्ठिनं रूपवन्नराः}
{मनोर्वंशे पुरा भूपा जाता जाता गुणाधिकाः}% २६

\twolineshloka
{इदानीं मयि जाते तु विपरीतं बभूव तत्}
{इति सम्भाषमाणं तं रामभद्रं समीक्ष्य सः}% २७

\twolineshloka
{संस्तभ्याश्रूणि विपुलान्युवाच विकल स्वरः}
{स्वामिन्विषादं मा कार्षीः कथं तव मतिर्हृता}% २८

\twolineshloka
{सीतामनिन्दितां को नु त्यजति श्रुतवान्भवान्}
{आकारयामि रजकं परिपृच्छामि तं प्रति}% २९

\twolineshloka
{कथं त्वयानिन्दिता सा जानकी योषितां वरा}
{तव देशे बलात्कश्चिद्बाध्यते न जनोऽल्पकः}% ३०

\twolineshloka
{तस्मात्तस्य यथास्वान्ते प्रतीतिः स्यात्तथाचर}
{किमर्थं त्यज्यते भीरुः पतिव्रतपरायणा}% ३१

\twolineshloka
{मनसा वचसा नान्यं जानाति जनकात्मजा}
{तस्मादेनां गृहाण त्वमेतां मा त्यज जानकीम्}% ३२

\twolineshloka
{ममोपरि कृपां कृत्वा मदुक्तं संश्रयाशु तत्}
{एवं वदन्तं प्रत्यूचे रामः शोकेन कर्षितः}% ३३

\onelineshloka
{लक्ष्मणं धर्मवाक्येन बोधयंस्त्यजनोद्यमः}% ३४

\uvacha{राम उवाच}

\twolineshloka
{कथं तु मां ब्रवीषि त्वं मा त्यजैनामनिन्दिताम्}
{लोकापवादात्त्यक्ष्येऽहं जानन्नपि विपापिनीम्}% ३५

\twolineshloka
{स्वयशः कारणेऽहं स्वं देहं त्यक्ष्याम्यशोभनम्}
{त्वामपि भ्रातरं त्यक्ष्ये लोकवादाद्विगर्हितम्}% ३६

\twolineshloka
{किमुतान्ये गृहाः पुत्रा मित्राणि वसुशोभनम्}
{स्वयशःकारणे सर्वं त्यजामि किमु मैथिलीम्}% ३७

\twolineshloka
{न तथा मे प्रियो भ्राता न कलत्रं न बान्धवाः}
{यथा मे विमलाकीर्तिर्वल्लभा लोकविश्रुता}% ३८

\twolineshloka
{इदानीं रजको नैव प्रष्टव्यो भवति ध्रुवम्}
{कालेन सर्वं भविता लोकचित्तस्य रञ्जनम्}% ३९

\twolineshloka
{आमयो यद्वदामस्तु न चिकित्स्यो भवेत्क्षितौ}
{सकालेन परीपाकाद्भेषजादेव नश्यति}% ४०

\twolineshloka
{तथा कालेन सम्भावि साम्प्रतं मा विलम्बय}
{त्यजैनां विपिने साध्वीं मां वा खड्गेन घातय}% ४१

\twolineshloka
{इत्युक्तं वाक्यमाकर्ण्य दुःखितोऽभूत्तदा महान्}
{चिन्तयामास च स्वान्ते लक्ष्मणः शोककर्षितः}% ४२

\twolineshloka
{पित्राज्ञप्तो जामदग्न्यो मातरं चाप्यघातयत्}
{गुरोराज्ञा नैव लङ्घ्या युक्ताऽयुक्तापि सर्वथा}% ४३

\twolineshloka
{तस्मादेनां त्यजाम्येव रामस्य प्रियकाम्यया}
{इति सञ्चिन्त्य मनसि भ्रातरं प्रत्युवाच सः}% ४४

\uvacha{लक्ष्मणउवाच}

\twolineshloka
{अकृत्यमपि कार्यं वै गुर्वाज्ञां नैव लङ्घयेत्}
{तस्मात्कुर्वे भवद्वाक्यं यत्त्वं वदसि सुव्रत}% ४५

\twolineshloka
{इत्येवं भाषमाणं तं लक्ष्मणं प्रत्युवाच सः}
{साधुसाधु महाप्राज्ञ त्वया मे तोषितं मनः}% ४६

\twolineshloka
{अद्यैव रात्रौ जानक्या दोहदस्तापसी क्षणे}
{तन्मिषेण रथे स्थाप्य मोचयैनां महावने}% ४७

\twolineshloka
{इत्थं भाषितमाकर्ण्य विशुष्यद्वदनोऽभितः}
{रुदन्बाष्पकलां मुञ्चञ्जगाम स्वं निवेशनम्}% ४८

\twolineshloka
{सुमन्त्रं तु समाहूय वचनं तमथाब्रवीत्}
{रथं मे कुरु सज्जं वै सदश्ववरभूषितम्}% ४९

\twolineshloka
{स तद्वाक्यं समाकर्ण्य रथमानीतवांस्तदा}
{आनीतं तं रथं दृष्ट्वा लक्ष्मणः शोककर्षितः}% ५०

\twolineshloka
{परमं दुःखमापन्नः संरुह्य स्यन्दनं वरम्}
{निःश्वसञ्जानकीगेहं प्रतस्थे भ्रातृसेवकः}% ५१

\twolineshloka
{गत्वा चान्तःपुरे भ्राता रामस्य मिथिलात्मजाम्}
{प्रत्यूचे निःश्वसन्वाक्यं दुःखपूरपरिप्लुतः}% ५२

\twolineshloka
{मातर्जानकि रामेण प्रेषितो भवनं तव}
{तापसीः प्रति याहि त्वं दोहदप्राप्तिहेतवे}% ५३

\twolineshloka
{इति वाक्यं समाकर्ण्य लक्ष्मणस्य विदेहजा}
{परमं हर्षमापन्ना लक्ष्मणं प्रत्यभाषत}% ५४

\uvacha{जानक्युवाच}

\twolineshloka
{धन्याहं मैथिली राज्ञी रामस्य चरणस्मरा}
{यस्या दोहदपूर्त्यर्थं प्रेषयामास लक्ष्मणम्}% ५५

\twolineshloka
{अद्याहं ता वनचरीस्तापसीः पतिदेवताः}
{नमस्कुर्यां च वासोभिः पूजयामि मनोहराः}% ५६

\twolineshloka
{इत्युक्त्वा रम्यवस्त्राणि महार्हाभरणानि च}
{मणीन्विमलमुक्ताश्च कर्पूरादिसुगन्धिमत्}% ५७

\twolineshloka
{चन्दनादिकवस्तूनि विचित्राणि सहस्रधा}
{जग्राह रघुनाथस्य पत्नी स्वप्रियकाम्यया}% ५८

\twolineshloka
{सीता गृहीत्वा सर्वाणि दासीनां करयोर्मुहुः}
{लक्ष्मणं प्रतिगच्छन्ती देहल्यां चास्खलत्तदा}% ५९

\twolineshloka
{अविचार्य तदौत्सुक्याल्लक्ष्मणं प्रियकारिणम्}
{उवाच कुत्र सरथो येन मां प्रापयिष्यसि}% ६०

\twolineshloka
{स निःश्वसन्रथं हैमं जानक्या सह निर्विशत्}
{सुमन्त्रं प्रत्युवाचासौ चालयाश्वान्मनोजवान्}% ६१

\twolineshloka
{स सुयुक्तं रथं वाक्याल्लक्ष्मणस्य तु चाह्वयत्}
{अश्रुपूर्णमुखं पश्यँल्लक्ष्मणं स मुहुर्मुहुः}% ६२

\twolineshloka
{आहतास्तेन कशया वाहास्तस्यापतन्पथि}
{न चलन्ति यदा वाहास्तदा लक्ष्मणमब्रवीत्}% ६३

\uvacha{सुमन्त्र उवाच}

\twolineshloka
{स्वामिंश्चलन्ति नो वाहा यत्नेन परिचालिताः}
{किं करोमि न जानेऽत्र कारणं वाहपातने}% ६४

\twolineshloka
{एवं ब्रुवन्तं प्रत्यूचे लक्ष्मणो गद्गदस्वरः}
{सारथिं धैर्यमास्थाय ताडयैतान्कशादिभिः}% ६५

\twolineshloka
{एतच्छ्रुत्वोदितं यन्ता कथञ्चित्समचालयत्}
{तदा स्फुरद्दक्षनेत्रं जानक्या दुःखशंसकम्}% ६६

\twolineshloka
{तदैव हृदये शोकः समभूद्दुःखशंसकः}
{तदैव पक्षिणः पुण्याः कुर्वन्ति परिवर्तनम्}% ६७

\twolineshloka
{एवं वीक्ष्यैव वैदेही प्रत्युवाचाथ देवरम्}
{कथं मे तापसीक्षायै यातुमिच्छा रघूद्वह}% ६८

\twolineshloka
{रामे भूयाद्धि कल्याणं भरते वा तथानुजे}
{तत्प्रजासु च सर्वत्र मा भवन्तु विपर्ययाः}% ६९

\twolineshloka
{एवं ब्रुवन्तीं संवीक्ष्य जानकीं च स लक्ष्मणः}
{न किञ्चिदुक्तवान्रुद्ध कण्ठो बाष्पप्रपूरितः}% ७०

\twolineshloka
{सा गच्छन्ती मृगान्वामं परिवर्तनकारकान्}
{अपश्यद्दुःखसङ्घातकारकान्समभाषत}% ७१

\twolineshloka
{अद्य यन्मे मृगा वामं वर्तयन्ति तदिष्यते}
{श्रीरामचरणौ मुक्त्वा गच्छन्त्यायुक्तमेव तत्}% ७२

\twolineshloka
{महिलानां परोधर्मः स्वभर्तृचरणार्चनम्}
{तन्मुक्त्वान्यत्र यान्त्या मे यद्भवेद्युक्तमेव तत्}% ७३

\twolineshloka
{एवं पथि विचारं तु कुर्वन्त्या परमार्थतः}
{जाह्नवी ददृशे देव्या मुनिवृन्दैकसेविता}% ७४

\twolineshloka
{यस्यां जलस्य कल्लोला दृश्यन्ते दुग्धसन्निभाः}
{तरङ्गो दृश्यते यत्र स्वर्गसोपानमूर्तिभृत्}% ७५

\twolineshloka
{यस्या वारिकणस्पर्शान्महापातकसञ्चयः}
{पलायते न कुत्रापि स्थानमीक्षन्समन्ततः}% ७६

\twolineshloka
{गङ्गां प्राप्याथ सौमित्रिर्जानकीं स्यन्दने स्थिताम्}
{उवाच निर्गलद्बाष्प एहि सीते रथाद्भुवि}% ७७

\twolineshloka
{सीता तद्वाक्यमाकर्ण्य क्षणादवततार सा}
{लक्ष्मणेन धृता बाहौ स्खलन्ती पथि कण्टकैः}% ७८

{॥इति श्रीपद्मपुराणे पातालखण्डे शेषवात्स्यायनसंवादे रामाश्वमेधे जानक्या गङ्गादर्शनं नाम अष्टपञ्चाशत्तमोऽध्यायः॥५८॥}

\dnsub{एकोनषष्टितमोऽध्यायः}\resetShloka

\uvacha{शेष उवाच}

\twolineshloka
{अथ नावा समुत्तीर्य जाह्नवीं लक्ष्मणस्तदा}
{जानकीं परतस्तीरे हस्ते धृत्वा वनं ययौ}% १

\twolineshloka
{सा चलन्ती पथि तदा शुष्यद्वदनलक्षिता}
{कण्टकक्षतसत्पादा स्खलन्ती च पदे पदे}% २

\twolineshloka
{लक्ष्मणस्तां महाघोरे विपिने दुःखदायिनि}
{प्रवेशयामास तदा राघवाज्ञाविधायकः}% ३

\twolineshloka
{यत्र वृक्षा महाघोरा बर्बूलाः खदिरा घनाः}
{श्लेष्मातकाश्चिञ्चिणीकाः शुष्का दावेन वह्निना}% ४

\twolineshloka
{कोटरस्था महासर्पाः फूत्कुर्वन्ति सुकोपिताः}
{घूका घूत्कुर्वते यत्र लोकचित्तभयङ्कराः}% ५

\twolineshloka
{व्याघ्राः सिंहाः सृगालाश्च द्वीपिनोऽतिभयङ्कराः}
{दृश्यन्ते यत्र सरला मनुष्यादाः सुकोपनाः}% ६

\twolineshloka
{महिषाः सूकरा दुष्टा दंष्ट्रा द्वयविलक्षिताः}
{कुर्वन्ति प्राणिनां तापं मानसस्य मदोद्धुराः}% ७

\twolineshloka
{ईदृग्वनं प्रपश्यन्ती भयेनोपगतज्वरा}
{कण्टकैर्दष्टचरणा लक्ष्मणं वाक्यमब्रवीत्}% ८

\uvacha{जानक्युवाच}

\twolineshloka
{वीरर्षिमुनिसंसेव्या नाश्रमान्नेत्रसौख्यदान्}
{नाहं पश्यामि नो तेषां पत्नीश्च सुतपोधनाः}% ९

\twolineshloka
{पश्यामि केवलं घोरान्पक्षिणः शुष्कवृक्षकान्}
{दावानलेन तत्सर्वं दह्यमानमिदं वनम्}% १०

\twolineshloka
{त्वां च पश्यामि दुःखार्तमश्रुपूर्णाकुलेक्षणम्}
{शकुनेतरसाहस्रं भवेन्मम पदे पदे}% ११

\twolineshloka
{तन्मे कथय वीराग्र्य कथं मुक्ता महात्मना}
{रामेण दुष्टहृदया क्षिप्रं कथय मे हि तत्}% १२

\twolineshloka
{इति वाक्यं समाकर्ण्य लक्ष्मणः शोककर्शितः}
{संरुद्धबाष्पवदनो न किञ्चित्प्रोक्तवांस्तदा}% १३

\twolineshloka
{तदेव विपिनं घोरं गच्छन्ती लक्ष्मणान्विता}
{पुनरप्याह तं वीरं दुःखार्ता पश्यती मुखम्}% १४

\twolineshloka
{तदापि स न तां वक्ति किमपि प्रेक्षुलोलुपः}
{तदा सात्यन्तनिर्बन्धं चकार परिपृच्छती}% १५

\twolineshloka
{आग्रहेण यदा पृष्टो लक्ष्मणः सीतया तदा}
{रुद्धकण्ठो मुहुः शोचन्नवदत्त्यागसम्भवम्}% १६

\twolineshloka
{तद्वाक्यं पविना तुल्यं निशम्य मुनिसत्तम}
{सुलताकृत्तमूलेव बभूवाकल्पवर्जिता}% १७

\twolineshloka
{तदैव पृथिवी तां न जग्राह तनयामिमाम्}
{रामो विपापिनीं सीतां न जह्यादिति शङ्किनी}% १८

\twolineshloka
{पतितां तां तु वैदेहीं दृष्ट्वा सौमित्रिरुत्सुकः}
{पल्लवाग्र्यसमीरेण संज्ञितां तु चकार सः}% १९

\twolineshloka
{संज्ञां प्राप्ता ह्युवाचेदं मा हास्यं कुरु देवर}
{कथं मां पापरहितां त्यजते स रघूद्वहः}% २०

\twolineshloka
{एवं बहुविलप्याथ लक्ष्मणं दुःखसंयुतम्}
{संवीक्ष्यमू र्च्छिता भूमौ पपात परिदुःखिता}% २१

\twolineshloka
{मुहूर्तेनापि संज्ञां सा प्राप्य दुःखपरिप्लुता}
{जगाद रामचरणौ स्मंरती शोकविक्षता}% २२

\twolineshloka
{रघुनाथो महाबुद्धिस्त्यजते मां कथं महान्}
{यो मदर्थे पयोराशिं बद्धवान्वानरैर्युतः}% २३

\twolineshloka
{स कथं मां महावीरो निष्पापां रजकोक्तितः}
{त्यजिष्यति ममैवात्र दैवं तु प्रतिकूलितम्}% २४

\twolineshloka
{एवं वदन्ती पुनरपि मूर्च्छां प्राप्ता विदेहजा}
{मूर्च्छितां तां समीक्ष्याथ रुरोद विकृतस्वरः}% २५

\twolineshloka
{पुनः संज्ञामवाप्यैवं सौमित्रिं निजगाद सा}
{दुःखातुरं वीक्षमाणा रुद्धकण्ठं सुदुःखिता}% २६

\twolineshloka
{सौमित्रे गच्छ रामं त्वं धर्ममूर्तिं यशोनिधिम्}
{मद्वाक्यमेवैतद्ब्रूयाः समक्षं तपसां निधेः}% २७

\twolineshloka
{मां तत्याज भवान्यद्वै जानन्नपि विपापिनीम्}
{कुलस्य सदृशं किं वा शास्त्रज्ञानस्य तत्फलम्}% २८

\twolineshloka
{नित्यं तव पदे रक्तां त्वदुच्छिष्टभुजं हि माम्}
{भवांस्तत्याज तत्सर्वं मम दैवं तु कारणम्}% २९

\twolineshloka
{कल्याणं तव सर्वत्र भूयाद्वीरवरोत्तम}
{अहं तावद्वने त्वां हि स्मरन्ती प्राणधारिका}% ३०

\twolineshloka
{मनसा कर्मणा वाचा भवानेव ममोत्तमः}
{अन्ये तुच्छीकृताः सर्वे मनसा रघुवंशज}% ३१

\twolineshloka
{भवेभवे भवानेव पतिर्भूयान्महीश्वर}
{त्वत्पदस्मरणानेक हतपापा सतीश्वरी}% ३२

\twolineshloka
{श्वश्रूजनं ब्रूहि सर्वं मत्सन्देशं रघूत्तम}
{त्यक्ता वने महाघोरे रामेण निरघा सती}% ३३

\twolineshloka
{स्मरामि चरणौ युष्मद्वने मृगगणैर्युते}
{अन्तर्वत्नी वने त्यक्ता रामेण सुमहात्मना}% ३४

\twolineshloka
{सौमित्रे शृणु मद्वाक्यं भद्रं भूयाद्रघूत्तमे}
{इदानीं सन्त्यजे प्राणान्रामवीर्यं सुरक्षती}% ३५

\twolineshloka
{त्वं रामवचनं तथ्यं यत्करोषि शुभं तव}
{परतन्त्त्रेण तत्कार्यं रामपादाब्जसेविना}% ३६

\twolineshloka
{गच्छ त्वं राम सविधे शिवाः पन्थान एव ते}
{ममोपरि कृपा कार्या स्मर्तव्याहं कदा कदा}% ३७

\twolineshloka
{इत्युक्त्वा मूर्च्छिता भूमौ पपात पुरतस्तदा}
{लक्ष्मणो दुःखमापेदे वीक्ष्य मूर्च्छितजानकीम्}% ३८

\twolineshloka
{वीजयामास वासोग्रैः संज्ञां प्राप्तां प्रकृत्य च}
{सौमित्रिः सान्त्वयामास वचनैर्मधुरैर्मुहुः}% ३९

\uvacha{लक्ष्मण उवाच}

\twolineshloka
{एष गच्छामि रामं वै गत्वा शंसामि सर्वशः}
{समीपे ते मुनेरस्ति वाल्मीकेराश्रमो महान्}% ४०

\twolineshloka
{इत्युक्त्वा तां परिक्रम्य दुःखितो बाष्पपूरितः}
{मुञ्चन्नश्रुकलां दुःखाद्ययौ रामं महीपतिम्}% ४१

\twolineshloka
{जानकी देवरं यान्तं वीक्ष्य विस्मितलोचना}
{हसत्ययं महाभागो लक्ष्मणो देवरो मम}% ४२

\twolineshloka
{कथं मां प्राणतः प्रेष्ठां विपापां राघवोऽत्यजत्}
{इति सञ्चिन्तयन्ती सा तमैक्षदनिमेषणा}% ४३

\twolineshloka
{जाह्नवीं सर्वथोत्तीर्णां ज्ञात्वा सत्यं स्वहापनम्}
{पतिता प्राणसन्देहं प्राप्ता मूर्च्छां गता तदा}% ४४

\twolineshloka
{तदा हंसाः स्वपक्षाभ्यां जलमानीय सर्वतः}
{सिषिचुर्मधुरो वायुर्ववौ पुष्पसुगन्धिमान्}% ४५

\twolineshloka
{करिणः पुष्करैः स्वीयैर्जलपूर्णैः समन्ततः}
{व्याप्तं शरीरं रजसा क्षालयन्त इवागताः}% ४६

\twolineshloka
{मृगास्तदन्तिकं प्राप्य सन्तस्थुर्विस्मितेक्षणाः}
{नगाः पुष्पयुता आसंस्तत्कालं मधुना विना}% ४७

\twolineshloka
{एतस्मिन्समये वृत्ते संज्ञां प्राप्य तदा सती}
{विललाप सुदुःखार्ता रामरामेति जल्पती}% ४८

\twolineshloka
{हा नाथ दीनबन्धो हे करुणामयसन्निधे}
{अपराधादृते मां त्वं कथं त्यजसि वै वने}% ४९

\twolineshloka
{इत्येवमादिभाषन्ती विलपन्ती मुहुर्मुहुः}
{इतस्ततः प्रपश्यन्ती सम्मूर्च्छन्ती पुनःपुनः}% ५०

\twolineshloka
{तदा स्वशिष्यैर्भगवान्वाल्मीकिः सङ्गतो वनम्}
{शुश्राव रुदितं तत्र करुणास्वरभाषितम्}% ५१

\twolineshloka
{शिष्यान्प्रति जगादाथ पश्यन्तु वनमध्यतः}
{को रोदिति महाघोरे विपिने दुःखितस्वरः}% ५२

\twolineshloka
{ते प्रयुक्ताश्च मुनिना सञ्जग्मुर्यत्र जानकी}
{रामरामेति भाषन्ती बाष्पपूरपरिप्लुता}% ५३

\twolineshloka
{तां दृष्ट्वा स्त्रियमौत्सुक्याद्वाल्मीकिं प्रत्यगुर्मुनिम्}
{श्रुत्वा तदीरितं वाक्यं जगामासौ ततो मुनिः}% ५४

\twolineshloka
{दृष्ट्वा तं तपसां राशिं जानकी पतिदेवता}
{नमोस्तु मुनये वेदमूर्तये व्रतवार्धये}% ५५

\twolineshloka
{इत्युक्तवन्तीं तां सीतामाशीर्भिरभ्यनन्दयत्}
{भर्त्रा सह चिरञ्जीव पुत्रौ प्राप्नुहि शोभनौ}% ५६

\twolineshloka
{कासि त्वं किं वने घोरे सङ्गतासि किमीदृशी}
{सर्वं मे शंस जानीयां तव दुःखस्य कारणम्}% ५७

\twolineshloka
{सा तदा प्रत्युवाचेमं रामस्य महिला मुनिम्}
{निःश्वसन्ती करुणया गिरासञ्जातवेपथुः}% ५८

\twolineshloka
{शृणु मे वाक्यमर्थोक्तं सर्वदुःखस्य कारणम्}
{जानीहि मां भूमिपते रघुनाथस्य सेवकीम्}% ५९

\twolineshloka
{अपराधं विना त्यक्ता न जाने तत्र कारणम्}
{लक्ष्मणो मां विमुच्यात्र गतवान्राघवाज्ञया}% ६०

\twolineshloka
{इत्युक्त्वाश्रुकलापूर्णं बिभ्रतीं मुखपङ्कजम्}
{वाल्मीकिः सान्त्वयन्प्राह जानकीं कमलेक्षणाम्}% ६१

\twolineshloka
{वाल्मीकिं मां विजानीहि पितुस्तव गुरुं मुनिम्}
{दुःखं मा कुरु वैदेहि ह्यागच्छ मम चाश्रमम्}% ६२

\twolineshloka
{भिन्नस्थाने पितुर्गेहं जानीहि पतिदेवते}
{ईदृशे कर्मणि मम रोषोस्त्वेव महीपतेः}% ६३

\twolineshloka
{एवं वचः समाकर्ण्य जानकी पतिदेवता}
{दुःखपूर्णाश्रुवदना किञ्चित्सुखमवाप सा}% ६४

\uvacha{शेष उवाच}

\twolineshloka
{वाल्मीकिः सान्त्वयित्वैनां दुःखपूराकुलेक्षणाम्}
{निनाय स्वाश्रमं पुण्यं तापसीवृन्दपूरितम्}% ६५

\twolineshloka
{सा गच्छन्ती पृष्ठतोऽस्य वाल्मीकेस्तपसां निधेः}
{रराजेन्दोः पृष्ठतो वै तारकेव मनोहरा}% ६६

\twolineshloka
{वाल्मीकिः प्राप्य च स्वीयमाश्रमं मुनिपूरितम्}
{तापसीः प्रतिसञ्चख्यौ जानकीं स्वाश्रमं गताम्}% ६७

\twolineshloka
{वैदेही तापसीः सर्वा नमश्चक्रे महामनाः}
{परस्परं प्रहृषिताः परिरम्भं समाचरन्}% ६८

\twolineshloka
{वाल्मीकिर्निजशिष्यान्स प्रत्युवाच तपोनिधिः}
{रच्यतां बत जानक्याः पर्णशाला मनोरमा}% ६९

\twolineshloka
{इत्युक्तं वाक्यमाकर्ण्य वाल्मीकेः सुमनोरमम्}
{व्यरचन्पत्रकैः शालां दारुभिः सुमनोहराम्}% ७०

\twolineshloka
{तत्रावसद्धि वैदेही पतिव्रतपरायणा}
{वाल्मीकेः परिचर्यां च कुर्वन्ती फलभक्षिका}% ७१

\twolineshloka
{जपन्ती रामरामेति मनसा वचसा स्वयम्}
{निनाय दिवसांस्तत्र जानकी पतिदेवता}% ७२

\twolineshloka
{काले सासूत पूत्रौ द्वौ मनोहरवपुर्धरौ}
{रामचन्द्र प्रतिनिधी ह्यश्विनाविव जानकी}% ७३

\twolineshloka
{तच्छ्रुत्वा तु मुनिर्हृष्टो जानक्याः पुत्रसम्भवम्}
{चकार जातकर्मादि संस्कारान्मन्त्रवित्तमः}% ७४

\twolineshloka
{कुशैर्लवैश्च वाल्मीकिर्मुनिः कर्माणि चाचरत्}
{तन्नाम्ना पुत्रयोराख्या कुशो लव इति स्फुटा}% ७५

\twolineshloka
{वाल्मीकिर्यत्र विरजा मङ्गलं तद्यथाचरत्}
{अत्यन्तं हृष्टचित्ता सा बभूव कमलेक्षणा}% ७६

\twolineshloka
{तद्दिने लवणं हत्वा शत्रुघ्नः स्वल्पसैनिकः}
{आगमच्चाश्रमे चास्य वाल्मीकेर्निशि शोभने}% ७७

\twolineshloka
{तदा वाल्मीकिना शिष्टः शत्रुघ्नो रघुनायकम्}
{मा शंस जानकीपुत्रौ कथयिष्याम्यहं पुरः}% ७८

\twolineshloka
{जानकीपुत्रकौ तत्र ववृधाते मनोरमौ}
{कन्दमूलफलैः पुष्टौ व्यदधादुन्मदौ वरौ}% ७९

\twolineshloka
{शुक्लप्रतिपदायाश्च शशीव सुमनोहरौ}
{कालेन संस्कृतौ जातावुपनीतौ मनोहरौ}% ८०

\twolineshloka
{उपनीयमुनिर्वेदं साङ्गमध्यापयत्सुतौ}
{सरहस्यं धनुर्वेदं रामायणमपाठयत्}% ८१

\twolineshloka
{वाल्मीकिना च धनुषी दत्ते स्वर्णसुभूषिते}
{अभेद्ये सगुणे श्रेष्ठे वैरिवृन्दविदारणे}% ८२

\twolineshloka
{इषुधी बाणसम्पूर्णौ अक्षये करवालके}
{चर्माण्यभेद्यानि ददौ जानक्यात्मजयोस्तदा}% ८३

\twolineshloka
{धनुर्धरौ धनुर्वेदपारगावाश्रमे मुदा}
{चरन्तौ तत्र रेजाते अश्विनाविव शोभनौ}% ८४

\twolineshloka
{जानकी वीक्ष्य पुत्रौ द्वौ खड्गचर्मधरौ वरौ}
{परमं हर्षमापन्ना विरहोद्भवमत्यजत्}% ८५

\twolineshloka
{एष ते कथितो विप्र जानक्याः पुत्रसम्भवः}
{अतः शृणुष्व यद्वृत्तं वीरबाहुविकृन्तनम्}% ८६

{॥इति श्रीपद्मपुराणे पातालखण्डे शेषवात्स्यायनसंवादे रामाश्वमेधे कुशलवोत्पत्तिकथानकन्नामैकोनषष्टितमोऽध्यायः॥५९॥}

\dnsub{षष्टितमोऽध्यायः}\resetShloka

\uvacha{शेष उवाच}

\twolineshloka
{शत्रुघ्नो निजवीराणां छिन्नान्बाहून्निरीक्षयन्}
{उवाच तान्सुकुपितो रोषसन्दंशिताधरः}% १

\twolineshloka
{केन वीरेण वो बाहुकृन्तनं समकारि भोः}
{तस्याहं बाहू कृन्तामि देवगुप्तस्य वै भटाः}% २

\twolineshloka
{न जानाति महामूढो रामचन्द्र बलं महत्}
{इदानीं दर्शयिष्यामि पराक्रान्त्या बलं स्वकम्}% ३

\twolineshloka
{स कुत्र वर्तते वीरो हयः कुत्र मनोरमः}
{को वाऽगृह्णात्सुप्तसर्पान्मूढो ज्ञात्वा पराक्रमम्}% ४

\twolineshloka
{इति ते कथिता वीरा विस्मिता दुःखिता भृशम्}
{रामचन्द्र प्रतिनिधिं बालकं समशंसत}% ५

\twolineshloka
{स श्रुत्वा रोषताम्राक्षो बालकेन हयग्रहम्}
{सेनान्यं वै कालजितमाज्ञापयद्युयुत्सुकः}% ६

\twolineshloka
{सेनानीः सकलां सेनां व्यूहयस्व ममाज्ञया}
{रिपुः सम्प्रति गन्तव्यो महाबलपराक्रमः}% ७

\twolineshloka
{नायं बालो हरिर्नूनं भविष्यति हयन्धरः}
{अथवा त्रिपुरारिः स्यान्नान्यथा मद्धयापहृत्}% ८

\twolineshloka
{अवश्यं कदनं भाविसैन्यस्य बलिनो महत्}
{स्वच्छन्दचरितैः खेलन्नास्ते निर्भयधीः शिशुः}% ९

\twolineshloka
{तत्र गन्तव्यमस्माभिः सन्नद्धै रिपुदुर्जयैः}
{एतन्निशम्य वचनं शत्रुघ्नस्य ससैन्यपः}% १०

\twolineshloka
{सज्जीचकार सेनां तां दुर्व्यूढां चतुरङ्गिणीम्}
{सज्जां तां शत्रुजिद्दृष्ट्वा चतुरङ्गयुतां वराम्}% ११

\twolineshloka
{आज्ञापयत्ततो गन्तुं यत्र बालो हयन्धरः}
{सा चचाल तदा सेना चतुरङ्गसमन्विता}% १२

\twolineshloka
{कम्पयन्ती महीभागं त्रासयन्ती रिपून्बलात्}
{सेनानीस्तं ददर्शाथ बालकं रामरूपिणम्}% १३

\twolineshloka
{विचार्य रामप्रतिममब्रवीद्वचनं हितम्}
{बाल मुञ्च हयश्रेष्ठं रामस्य बलशालिनः}% १४

\twolineshloka
{सेनानीः कालजिन्नाम तस्य भूपस्य दुर्मदः}
{त्वां रामप्रतिमं दृष्ट्वा कृपा मे हृदि जायते}% १५

\twolineshloka
{अन्यथा तव मे दौस्थ्याज्जीवितं न भविष्यति}
{एतद्वाक्यं समाकर्ण्य शत्रुघ्नस्य भटस्य हि}% १६

\twolineshloka
{जहास किञ्चिदाकोपादुवाच च वचोद्भुतम्}
{गच्छ मुक्तोसि तं रामं कथयस्व हयग्रहम्}% १७

\twolineshloka
{त्वत्तो बिभेमि नो शूर वाक्येन नयशालिना}
{ममात्र गणना नास्ति त्वादृशाः कोटयो यदि}% १८

\twolineshloka
{मातृपादप्रसादेन तूलीभूता न संशयः}
{कालजित्तव यन्नाम मात्राकारि मनोज्ञया}% १९

\twolineshloka
{पक्वबिम्बफलस्येव वर्णतो न च वीर्यतः}
{दर्शयस्वाधुना वीर्यं स्वनामबलचिह्नितः}% २०

\onelineshloka*
{मां कालं तव सञ्जित्य सत्यनामा भविष्यसि}

\uvacha{शेष उवाच}

\onelineshloka
{स वाक्यैः पविनातुल्यैर्भिन्नः सुभटशेखरः}% २१


\threelineshloka
{चुकोप हृदयेऽत्यतं जगाद वचनं पुनः}
{कस्मिन्कुले समुत्पत्तिः किं नामासि च बालक}
{त्वन्नाम नाभिजानामि कुलं शीलं वयस्तथा}% २२

\twolineshloka
{पादचारं रथस्थोऽहमधर्मेण कथं जये}
{तदात्यन्तं प्रकुपितो जगाद वचनं पुनः}% २३

\twolineshloka
{कुलेन किं च शीलेन नाम्ना वा सुमनोहृदा}
{लवोऽहं लवतः सर्वाञ्जेष्यामि रिपुसैनिकान्}% २४

\twolineshloka
{इदानीं त्वामपि भटं करिष्ये पादचारिणम्}
{इत्थमुक्त्वा धनुः सज्यं चकार स लवो बली}% २५

\twolineshloka
{टङ्कारयामास तदा वीरानाकम्पयन्हृदि}
{वाल्मीकिं प्रथमं स्मृत्वा जानकीं मातरं लवः}% २६

\twolineshloka
{मुमोच बाणान्निशितान्सद्यः प्राणापहारिणः}
{कालजित्स्वधनुः कृत्वा सज्यं कोपसमन्वितः}% २७

\twolineshloka
{ताडयामास जवनो लवं रणविशारदः}
{तद्बाणाञ्छतधा छित्त्वा क्षणाद्वेगात्कुशानुजः}% २८

\twolineshloka
{सेनान्यं विरथं चक्रे वसुभिः स्वशरोत्तमैः}
{विरथो गजमानीतमारुरोह भटैर्निजैः}% २९

\twolineshloka
{मदोन्मत्तं महावेगं सप्तधा प्रस्रवान्वितम्}
{गजारूढं तु तं दृष्ट्वा दशभिर्धनुषोगतैः}% ३०

\twolineshloka
{बाणैर्विव्याध विहसन्सर्वान्रिपुगणाञ्जयी}
{कालजित्तस्य वीर्यं तु दृष्ट्वा विस्मितमानसः}% ३१

\twolineshloka
{गदां मुमोच महतीं महायस विनिर्मिताम्}
{आपतन्तीं गदां वेगाद्भारायुतविनिर्मिताम्}% ३२

\twolineshloka
{त्रिधा चिच्छेद तरसा क्षुरप्रैः सकुशानुजः}
{परिघं निशितं घोरं वैरिप्राणहरोदितम्}% ३३

\twolineshloka
{मुक्तं पुनस्तेन लवश्चिच्छेद तरसान्वितः}
{छित्त्वा तत्परिघं घोरं कोपादारक्तलोचनः}% ३४

\twolineshloka
{गजोपस्थे समारूढं मन्यमानश्चुकोप ह}
{तत्क्षणादच्छिनत्तस्य शुण्डां खड्गेन दन्तिनः}% ३५

\twolineshloka
{दन्तयोश्चरणौ धृत्वा रुरोह गजमस्तके}
{मुकुटं शतधा कृत्वा कवचं तु सहस्रधा}% ३६

\twolineshloka
{केशेष्वाकृष्य सेनान्यं पातयामास भूतले}
{पातितः स गजोपस्थात्सेनानीः कुपितः पुनः}% ३७

\twolineshloka
{हृदये ताडयामास मुष्टिना वज्रमुष्टिना}
{स आहतो मुष्टिभिस्तु क्षुरप्रान्निशिताञ्छरान्}% ३८

\twolineshloka
{मुमोच हृदये क्षिप्रं कुण्डलीकृतधन्ववान्}
{स रराज रणोपान्ते कुण्डलीकृत चापवान्}% ३९

\twolineshloka
{शिरस्त्रं कवचं बिभ्रदभेद्यं शरकोटिभिः}
{स विद्धः सायकैस्तीक्ष्णैस्तं हन्तुं खड्गमाददे}% ४०

\twolineshloka
{दशन्रोषात्स्वदशनान्निःश्वसन्नुच्छ्वसन्मुहुः}
{खड्गहस्तं समायान्तं शूरं सेनापतिं लवः}% ४१

\twolineshloka
{चिच्छेद भुजमध्यं तु स खड्गः पाणिरापतत्}
{छिन्नं खड्गधरं हस्तं वीक्ष्य कोपाच्चमूपतिः}% ४२

\twolineshloka
{वामेन गदया हन्तुं प्रचक्राम भुजेन तम्}
{सोऽपि च्छिन्नो भुजस्तस्य साङ्गदस्तीक्ष्णसायकैः}% ४३

\twolineshloka
{तदा प्रकुपितो वीरः पादाभ्यामहनल्लवम्}
{लवः पादाहतस्तस्य न चचाल रणाङ्गणे}% ४४

\twolineshloka
{स्रजाहतो द्विप इव चरणच्छेदनं व्यधात्}
{तदापि तं मौलिनासौ प्रहर्तुमुपचक्रमे}% ४५

\twolineshloka
{तदा लवश्चमूनाथं मन्यमानोऽधिपौरुषम्}
{करवालं समादाय करे कालानलोपमम्}% ४६

\twolineshloka
{अच्छिनच्छिर एतस्य महामुकुटशोभितम्}
{हाहाकारो महानासीच्चमूनाथे निपातिते}% ४७

\twolineshloka
{सैनिकाः परिसङ्क्रुद्धा लवं हन्तुं समागताः}
{लवस्तान्स्वशराघातैः पलायनपरान्व्यधात्}% ४८

\twolineshloka
{छिन्नाभिन्नाङ्गकाः केचिद्गता केचिद्रणाङ्गणात्}
{स निवार्याखिलान्योधान्विजगाह चमूं मुदा}% ४९

\twolineshloka
{वाराह इव निःश्वस्य प्रलये सुमहार्णवम्}
{गजा भिन्ना द्विधा जाता मौक्तिकैः पूरिता मही}% ५०

\twolineshloka
{दुर्गमाभूद्भटाग्र्याणां पर्वतैर्व्यापृता यथा}
{अश्वाः कनकपल्याणा रुचिरारत्नराजिताः}% ५१

\twolineshloka
{अपतन्रुधिराप्लुष्टे ह्रदे बल सुशोभिताः}
{रथिनः करमध्यस्थ धनुर्दण्डसुशोभिनः}% ५२

\twolineshloka
{रथोपस्थे निपतिताः स्वर्गगा इव वै सुराः}
{सन्दष्टौष्ठपुटा वक्त्र भ्रमल्लक्ष्मीविलक्षिताः}% ५३

\twolineshloka
{पतितास्तत्र दृश्यन्ते वीरा रणविशारदाः}
{सुस्राव शोणितसरिद्धयमस्तककच्छपा}% ५४

\twolineshloka
{महाप्रवाहललिता वैरिणां भयकारिका}
{केषाञ्चिद्बाहविश्छिन्नाः केषां पादा विकर्तिता}% ५५

\twolineshloka
{केषां कर्णाश्च नासाश्च केषां कवचकुण्डले}
{एवं तु कदनं जातं सेनान्यां पतिते रणे}% ५६

\twolineshloka
{सर्वे निपतिता वीरा न केचिज्जीवितास्ततः}
{लवो जयं रणे प्राप्य वैरिवृन्दं विजित्य च}% ५७

\twolineshloka
{अन्यागमनशङ्कायां मनः कुर्वन्नवैक्षत}
{केचिदुर्वरिता युद्धाद्भाग्येन न रणे मृताः}% ५८

\twolineshloka
{शत्रुघ्नं सविधे जग्मुः शंसितुं वृत्तमद्भुतम्}
{गत्वा ते कथयामासुर्यथावृत्तं रणाङ्गणे}% ५९

\twolineshloka
{कालजिन्निधनं बालाच्चित्रकारि रणोद्यमम्}
{तच्छ्रुत्वा विस्मयं प्राप्तः शत्रुघ्नस्तानुवाच ह}% ६०

\twolineshloka
{हसन्रोषाद्दशन्दन्तान्बालग्राह हयं स्मरन्}
{रे वीराः किं मदोन्मत्ता यूयं किं वा छलग्रहाः}% ६१

\twolineshloka
{किं वा वैकल्यमायातं कालजिन्मरणं कथम्}
{यः सङ्ख्ये वैरिवृन्दानां दारुणः समितिञ्जयः}% ६२

\twolineshloka
{तं कथं बालको जीयाद्यमस्यापि दुरासदम्}
{शत्रुघ्नवाक्यं संश्रुत्य वीराः प्रोचुरसृक्प्लुताः}% ६३

\twolineshloka
{नास्माकं मदमत्तादि न च्छलो न च देवनम्}
{कालजिन्मरणं सत्यं लवाज्जानीहि भूपते}% ६४

\twolineshloka
{बलं च कृत्स्नं मथितं बालेनातुलशौण्डिना}
{अतः परं तु यत्कार्यं ये प्रेष्या नृवरोत्तमाः}% ६५

\twolineshloka
{बालं ज्ञात्वा भवान्नात्र करोतु बलसाहसम्}
{इति श्रुत्वा वचस्तेषां वीराणां शत्रुहा तदा}% ६६

\onelineshloka
{सुमतिं च मतिश्रेष्ठमुवाच रणकारणे}% ६७

{॥इति श्रीपद्मपुराणे पातालखण्डे शेषवात्स्यायनसंवादे रामाश्वमेधे कुशलवयुद्धे सैन्यपराजय कालजित्सेनानीमरणं नाम षष्टितमोऽध्यायः॥६०॥}

\dnsub{एकषष्टितमोऽध्यायः}\resetShloka

\uvacha{शत्रुघ्न उवाच}

\twolineshloka
{जानासि किं महामन्त्रिन्को बालो हयमाहरत्}
{येन मे क्षपितं सर्वं बलं वारिधिसन्निभम्}% १

\uvacha{सुमतिरुवाच}

\twolineshloka
{स्वामिन्नयं मुनिश्रेष्ठ वाल्मीकेराश्रमो महान्}
{क्षत्त्रियाणामत्र वासो नास्त्येव परतापन}% २

\twolineshloka
{इन्द्रो भविष्यति परममर्षी हयमाहरत्}
{पुरारिर्वान्यथा वाहं तव कः समुपाहरेत्}% ३

\twolineshloka
{कालजिद्येन नाशं वै प्राप्तः परमदारुणः}
{तं प्रति श्रीमहाराज गन्ता कः पुष्कलान्यतः}% ४

\twolineshloka
{त्वं च वीरैर्भटैः सर्वैराजभिः परिवारितः}
{तत्र गच्छस्व सैन्येन महता शत्रुकृन्तन}% ५

\twolineshloka
{गत्वा स जीवितं वीरं बद्ध्वा तु कुतुकार्थिने}
{दर्शयिष्यामि रामाय मतं मे त्विदमादृतम्}% ६

\twolineshloka
{इति वाक्यं समाकर्ण्य वीरान्सर्वान्समादिशत्}
{सैन्येन महता यात यूयमायामि पृष्ठतः}% ७

\twolineshloka
{निर्दिष्टास्ते क्षणाद्वीरा जग्मुर्यत्र लवो बली}
{धनुर्विस्फारयंस्तत्र सुदृढं गुणपूरितम्}% ८

\twolineshloka
{आयातं तन्महद्दृष्ट्वा बलं वीरप्रपूरितम्}
{न किञ्चिन्मनसा बिभ्येलवेन बलशालिना}% ९

\twolineshloka
{लवः सिंह इवोत्तस्थौ मृगान्मत्वाऽखिलान्भटान्}
{धनुर्विस्फारयन्रोषाच्छरान्मुञ्चन्सहस्रशः}% १०

\twolineshloka
{ते शरैः पीड्यमानास्तु महारोषेण पूरिताः}
{वीरं बालं मन्यमानाः सम्मुखं प्राद्रवंस्तदा}% ११

\twolineshloka
{वीरान्सहस्रशो दृष्ट्वा भ्रमिभिः पर्यवस्थितान्}
{लवो जवेन सन्धाय शरान्रोषप्रपूरितः}% १२

\twolineshloka
{भ्रमिराद्या सहस्रेण द्वितीयायुतसङ्ख्यया}
{तृतीयायुतयुग्मेन तुरीयायुतपञ्चभिः}% १३

\twolineshloka
{पञ्चमी लक्षयोधानां षष्ठी योधायुताधिकैः}
{सप्तमी लक्षयुग्मेन सप्तभिर्भ्रमिभिर्वृतः}% १४

\twolineshloka
{मध्ये लवो भ्रमिव्याप्तः स चरन्वह्निवत्तदा}
{दाहयामास सर्वान्वै सैनिकान्भ्रमिकारकान्}% १५

\twolineshloka
{काचित्खङ्गैः शरैः काचित्काचित्प्रासैश्च कुन्तलैः}
{पट्टिशैः परिघैः सर्वा भ्रमिर्भग्ना महात्मना}% १६

\twolineshloka
{सप्तभिर्भ्रमिभिर्मुक्तो रराज स कुशानुजः}
{मेघवृन्दविनिर्मुक्तः शशीव शरदागमे}% १७

\twolineshloka
{प्राहरत्सर्वथा योधान्भिन्दन्गजकरान्बहून्}
{छिन्दञ्छिरांसि वीराणां चक्रभ्रूणि महान्ति च}% १८

\twolineshloka
{अनेके पतिता वीरा लवबाणप्रपीडिताः}
{मुमुहुः समरेऽथान्ये नष्टा अन्ये सुकातराः}% १९

\twolineshloka
{पलायनपरं सैन्यं लवबाणप्रपीडितम्}
{वीक्ष्य वीरो रणे योद्धुं प्रायात्पुष्कलसंज्ञकः}% २०

\twolineshloka
{तिष्ठतिष्ठेति च वदन्रोषपूरितलोचनः}
{रथे सुहयशोभाढ्ये तिष्ठन्प्रायाल्लवं बली}% २१

\twolineshloka
{स लवं प्रत्युवाचाथ पुष्कलः परमास्त्रवित्}
{तिष्ठ दत्ते मया सङ्ख्ये रथे सुहयशोभिते}% २२


\threelineshloka
{पदातिना त्वया युद्धं करोमि किमथाहवे}
{तस्मात्तिष्ठ रथे पश्चाद्युद्ध्येऽहं भवता सह}
{एतद्वाक्यं निशम्यासौ लवः पुष्कलमब्रवीत्}% २३

\twolineshloka
{त्वया दत्ते रथे स्थित्वा युद्धं कुर्यामहं रणे}
{तदा मे पापमेव स्याज्जयः सन्दिग्ध एव हि}% २४

\twolineshloka
{न वयं ब्राह्मणा वीर प्रतिग्रहपरायणाः}
{वयं तु क्षत्रिया नित्यं दानकर्मक्रियारताः}% २५

\twolineshloka
{इदानीं त्वद्रथं कोपाद्भनज्मि प्रत्यहं भवान्}
{पादचारी भवत्येव पश्चाद्युद्धं करिष्यति}% २६

\twolineshloka
{पुष्कलो वाक्यमाकर्ण्य धर्मधैर्यसमन्वितम्}
{विसिस्माय चिरं चित्ते धनुः सज्यमथाकरोत्}% २७

\twolineshloka
{तमात्तधनुषं दृष्ट्वा लवः कोपसमन्वितः}
{चापं चिच्छेद पाणिस्थं शरसन्धानमाचरन्}% २८

\twolineshloka
{स यावत्स गुणं चापं कुरुते तावदुद्धतः}
{रथभङ्गं चकारास्य समरे प्रहसन्बली}% २९

\twolineshloka
{भग्नं रथं स्वकं वीक्ष्य धनुश्छिन्नं महात्मना}
{महावीरं मन्यमानः पदातिः प्राद्रवद्रणे}% ३०

\twolineshloka
{उभौ धनुर्धरौ वीरावुभावपि शरोद्धतौ}
{उभौ क्षतजविप्लुष्टौ छिन्नसन्नाहितावुभौ}% ३१

\twolineshloka
{परस्परं बाणघातविशीर्णावपुलक्षितौ}
{जयाकाङ्क्षां प्रकुर्वन्तौ परस्परवधैषिणौ}% ३२

\twolineshloka
{जयन्तकार्तिकेयौ वा पुरारिः पुरभिद्यथा}
{एवं परस्परं युद्धं प्रकुर्वाणौ रणाङ्गणे}% ३३

\twolineshloka
{पुष्कलः प्रत्युवाचाथ बालं शूरशिरोमणे}
{त्वादृशो न मया दृष्टः कश्चिद्वीरशिरोमणिः}% ३४

\twolineshloka
{शिरस्ते पातयाम्यद्य बाणैः शितसुपर्वभिः}
{मा पलायस्व समरे प्राणान्रक्षस्व संयतः}% ३५

\twolineshloka
{एवमुक्त्वा लवं वीरं चकार शरपञ्जरे}
{पुष्कलस्य शरा भूमौ नभसि व्याप्य संस्थिताः}% ३६

\twolineshloka
{शरपञ्जरमध्यस्थो लवः पुष्कलमब्रवीत्}
{महत्कर्म कृतं वीर यन्मां बाणैरपीडयत्}% ३७

\twolineshloka
{इत्युक्त्वा बाणसङ्घातं प्रच्छिद्य वचनं पुनः}
{जगाद पुष्कलं वीरः शरसन्धानकोविदः}% ३८

\twolineshloka
{पालयात्मानमाजिस्थं मच्छराघातपीडितः}
{पतिष्यसि महीपृष्ठे रुधिरेण परिप्लुतः}% ३९

\twolineshloka
{एवमुक्तं समाकर्ण्य पुष्कलः कोपसंयुतः}
{रणे संयोधयामास लवं वीरं महाबलम्}% ४०

\twolineshloka
{लवः प्रकुपितो बाणं तीक्ष्णं वैरिविदारणम्}
{जग्राह लवतः कोशादाशीविषमिव क्रुधा}% ४१

\twolineshloka
{जाज्वल्यमानः सशरश्चापमुक्तो लवस्य च}
{हृदयं भेत्तुमुद्युक्तश्छिन्नो भारतिनाशु सः}% ४२

\twolineshloka
{छिन्ने भारतिना सङ्ख्ये शरेण प्राणहारिणा}
{अत्यन्तं कुपितो घोरं शरमन्यं समाददे}% ४३

\twolineshloka
{आकर्णाकृष्टचापेन स मुक्तो निशितः शरः}
{बिभेद हृदयं तस्य पुष्कलस्य महारणे}% ४४

\twolineshloka
{भिन्नो वक्षसि वीरेण सायकेनाशुगामिना}
{पपात धरणीपृष्ठे महाशूरशिरोमणिः}% ४५

\twolineshloka
{पतितं तं समालोक्य पुष्कलं पवनात्मजः}
{गृहीत्वा राघवभ्रात्रे ददौ मूर्च्छासमन्वितम्}% ४६

\twolineshloka
{मूर्च्छितं तं समालोक्य शोकविह्वलमानसः}
{हनूमन्तं लवं हन्तुं निदिदेश क्रुधान्वितः}% ४७

\twolineshloka
{हनूमान्क्रोधसम्प्लुष्टो लवं सङ्ख्ये महाबलम्}
{विजेतुं तरसा प्रागाद्वृक्षमुद्यम्य शाल्मलिम्}% ४८

\twolineshloka
{वृक्षेण हतवान्मूर्ध्नि लवस्य हनुमान्बली}
{तमापतन्तं तरसा चिच्छेद शतधा लवः}% ४९

\twolineshloka
{छिन्ने नगे पुनः कोपाद्वृक्षानुत्पाट्य मूलतः}
{ताडयामास हृदये मस्तके च महाबलः}% ५०

\twolineshloka
{यान्यान्वृक्षान्समाहृत्याताडयत्पवनात्मजः}
{तांस्तांश्चिच्छेद तरसा बलवान्नतपर्वभिः}% ५१

\twolineshloka
{तदा शिलाः समुत्पाट्य गण्डशैलोपमाः कपिः}
{पातयामास शिरसि क्षिप्रवेगेन मारुतिः}% ५२

\twolineshloka
{स आहतः शिलासङ्घैः सङ्ख्ये कोदण्डमुन्नयन्}
{बाणैस्ताश्चूर्णयामास यन्त्रिकाभिर्यथा कणाः}% ५३

\twolineshloka
{तदात्यन्तं प्रकुपितो मारुतिः पुच्छवेष्टनम्}
{चकार समरोपान्ते लवस्य बलिनः कृती}% ५४

\twolineshloka
{स्वं पुच्छेन समाविद्धं वीक्ष्य स्वाम्बां हृदि स्मरन्}
{मुष्टिना ताडयामास लाङ्गूलं मारुतेर्बली}% ५५

\twolineshloka
{तन्मुष्टिघातव्यथितो मारुतिस्तममूमुचत्}
{स मुक्तः पुच्छतो युद्धे शरान्मुञ्चन्नभूद्बली}% ५६

\twolineshloka
{दुर्वारशरघातेन सम्पीडिततनुः कपिः}
{बाणवर्षं मन्यमानो दुःसहं समरे बहु}% ५७

\twolineshloka
{किङ्कर्तव्यमितोऽस्माभिः पलाय्य यदि गम्यते}
{तदा मे स्वामिनो लज्जा ताडयेद्बालकोऽत्र माम्}% ५८

\twolineshloka
{ब्रह्मदत्तवरत्वात्तु मूर्च्छा न मरणं नहि}
{दुःसहा बाणपीडात्र किं कर्तव्यं मयाधुना}% ५९

\twolineshloka
{शत्रुघ्नः समरे गत्वा जयं प्राप्नोतु बालकात्}
{अहं तावज्जयाकाङ्क्षी शये कपटमूर्च्छया}% ६०

\twolineshloka
{इत्येवं मानसे कृत्वा पपात रणमण्डले}
{पश्यतां सर्ववीराणां कपटेन विमूर्च्छितः}% ६१

\twolineshloka
{तं मूर्च्छितं समाज्ञाय हनूमन्तं महाबलम्}
{जघान सर्वान्नृपतीञ्छरमोक्षविचक्षणः}% ६२

{॥इति श्रीपद्मपुराणे पातालखण्डे शेषवात्स्यायनसंवादे रामाश्वमेधे हनुमत्पतनन्नामैकषष्टितमोऽध्यायः॥६१॥}

\dnsub{द्विषष्टितमोऽध्यायः}\resetShloka

\uvacha{शेष उवाच}

\twolineshloka
{मूर्च्छितं मारुतिं श्रुत्वा शत्रुघ्नः शोकमाययौ}
{किङ्कर्तव्यं मया सङ्ख्ये बालकोऽयं महाबलः}% १

\twolineshloka
{स्वयं रथे हेममये तिष्ठन्वीरवरैः सह}
{योद्धुं प्रागाल्लवो यत्र विचित्ररणकोविदः}% २

\twolineshloka
{लवं ददर्श शिशुतां प्राप्तं राममिव क्षितौ}
{धनुर्बाणकरं वीरान्क्षिपन्तं रणमूर्धनि}% ३

\twolineshloka
{विचारयामास तदा कोऽयं रामस्वरूपधृक्}
{नीलोत्पलदलश्यामं वपुर्बिभ्रन्मनोहरम्}% ४

\twolineshloka
{एष वै देहतनुजा सुतो भवति नान्यथा}
{अस्मान्विजित्य समरे यास्यते मृगराडिव}% ५

\twolineshloka
{अस्माकं नो जयो भाव्यः शक्त्या विरहितात्मनाम्}
{अशक्ताः किं करिष्यामः समरे रणकोविदाः}% ६

\twolineshloka
{इत्येवं स विचार्याथ बालकं तु वचोऽब्रवीत्}
{रणे कुतुककर्तारं वीरकोटिनिपातकम्}% ७

\twolineshloka
{कस्त्वं बाल रणेऽस्माकं वीरान्पातयसि क्षितौ}
{न जानीषे बलं राज्ञो रामस्य दनुजार्दिनः}% ८

\twolineshloka
{का ते माता पिता कस्ते सुभाग्यो जयमाप्तवान्}
{नाम किं विश्रुतं लोके जानीयां ते महाबल}% ९

\twolineshloka
{मुञ्च वाहः कथं बद्धः शिशुत्वात्तत्क्षमामि ते}
{आयाहि रामं वीक्षस्व दास्यते बहुलं तव}% १०

\twolineshloka
{इत्युक्तो बालको वीरो वचः शत्रुघ्नमावदत्}
{किं ते नाम्नाथ पित्रा वा कुलेन वयसा तथा}% ११

\twolineshloka
{युध्यस्व समरे वीर चेत्त्वं बलयुतो भवेः}
{कुशं वीरं नमस्कृत्य पादयोर्याहि नान्यथा}% १२

\twolineshloka
{भ्राता रामस्य वीरो भूर्नावयोर्बलिनां वरः}
{वाहं विमोचय बलाच्छक्तिस्ते विद्यते यदि}% १३

\twolineshloka
{इत्युक्त्वा शरसन्धानं कृत्वा प्राहरदुद्भटः}
{हृदये मस्तके चैव भुजयो रणमण्डले}% १४

\twolineshloka
{तदा प्रकुपितो राजा धनुः सज्यमथाकरोत्}
{नादयन्मेघगम्भीरं त्रासयन्निव बालकम्}% १५

\twolineshloka
{बाणानपरिसङ्ख्यातान्मुमोच बलिनां वरः}
{बालो बलेन चिच्छेद सर्वांस्तान्सायकव्रजान्}% १६

\twolineshloka
{लवस्यानेकधा मुक्तैर्बाणैर्व्याप्तं महीतलम्}
{व्यतीपाते प्रदत्तस्य दानस्येवाक्षयं गताः}% १७

\twolineshloka
{ते बाणा व्योमसकलं व्याप्नुवँल्लवसन्धिताः}
{सूर्यमण्डलमासाद्य प्रवर्तन्ते समन्ततः}% १८

\twolineshloka
{मारुतो नाविशद्यत्र बाणपञ्जरगोचरे}
{मनुष्याणां तु का वार्ता क्षणजीवितशंसिनाम्}% १९

\twolineshloka
{तद्बाणान्विस्तृतान्दृष्ट्वा शत्रुघ्नो विस्मयं गतः}
{अच्छिनच्छतसाहस्रं बाणमोचनकोविदः}% २०

\twolineshloka
{ताञ्छिन्नान्सायकान्सर्वान्स्वीयान्दृष्ट्वा कुशानुजः}
{धनुश्चिच्छेद तरसा शत्रुघ्नस्य महीपतेः}% २१

\twolineshloka
{सोऽन्यद्धनुरुपादाय यावन्मुञ्चति सायकान्}
{तावद्बभञ्ज सरथं सायकैः शितपर्वभिः}% २२

\twolineshloka
{करस्थमच्छिनच्चापं सुदृढं गुणपूरितम्}
{तत्कर्मापूजयन्वीरा रणमण्डलवर्तिनः}% २३

\twolineshloka
{सच्छिन्नधन्वा विरथो हताश्वो हतसारथिः}
{अन्यं रथं समास्थाय ययौ योद्धुं लवं बलात्}% २४

\twolineshloka
{अनेकबाणनिर्भिन्नः स्रवद्रक्तकलेवरः}
{पुष्पितः किंशुक इव शुशुभे रणमध्यगः}% २५

\twolineshloka
{शत्रुघ्नबाणप्रहतः परं कोपमुपागमत्}
{बाणसन्धानचतुरः कुण्डलीकृत चापवान्}% २६

\twolineshloka
{विशीर्णकवचं देहं शिरोमुकुटवर्जितम्}
{स्रवद्रक्तपरिप्लुष्टं शत्रुघ्नस्य चकार सः}% २७

\twolineshloka
{तदा रामानुजः क्रुद्धो दशबाणाञ्छिताग्रकान्}
{मुमोच प्राणसंहारकारकान्कुपितो भृशम्}% २८

\twolineshloka
{स तांस्तांस्तिलशः कृत्वा बाणैर्निशितपर्वभिः}
{ताडयामास हृदये शत्रुघ्नस्याष्टभिः शरैः}% २९

\twolineshloka
{अत्यन्तं बाणपीडार्तो लवं बलिनमुत्स्मरन्}
{दुःसहं मन्यमानस्तं शरान्मुञ्चन्नभूत्तदा}% ३०

\twolineshloka
{तदा लवेन तीक्ष्णेन हृदि भिन्नो विशालके}
{अर्धचन्द्रसमानेन तीक्ष्णपर्वसुशोभिना}% ३१

\twolineshloka
{स विद्धो हृदि बाणेन पीडां प्राप्तः सुदारुणाम्}
{पपात स्यन्दनोपस्थे धनुःपाणिः सुशोभितः}% ३२

\twolineshloka
{शत्रुघ्नं मूर्छितं दृष्ट्वा नृपाश्च सुरथादयः}
{दुद्रुवुर्लवमुद्युक्ता जयप्राप्त्यै रणे तदा}% ३३

\twolineshloka
{सुरथो विमलो वीरो राजा वीरमणिस्तथा}
{सुमदो रिपुतापाद्याः परिवव्रुश्च संयुगे}% ३४

\twolineshloka
{केचित्क्षुरप्रैर्मुसलैः केचिद्बाणैः सुदारुणैः}
{प्रासैः परशुभिः केचित्सर्वतः प्राहरन्नृपाः}% ३५

\twolineshloka
{तानधर्मेण युद्धोत्कान्दृष्ट्वा वीरशिरोमणिः}
{दशभिर्दशभिर्बाणैस्ताडयामास संयुगे}% ३६

\twolineshloka
{ते बाणवर्षविहता रणमध्ये सुकोपनाः}
{केचित्पलायिताः केचिन्मुमुहुर्युद्धमण्डले}% ३७

\twolineshloka
{तावत्स राजा शत्रुघ्नो मूर्च्छां सन्त्यज्य सङ्गरे}
{लवं प्रायान्महावीरं योद्धुं बलसमन्वितः}% ३८

\twolineshloka
{आगत्य तं लवं प्राह धन्योसि शिशुसन्निभः}
{न बालस्त्वं सुरः कश्चिच्छलितुं मां समागतः}% ३९

\twolineshloka
{केनापि नहि वीरेण पातितो रणमण्डले}
{त्वयाहं प्रापितो मूर्च्छां समक्षं मम पश्यतः}% ४०

\twolineshloka
{इदानीं पश्य मे वीर्यं त्वां सङ्ख्ये पातयाम्यहम्}
{सहस्व बाणमेकं त्वं मापलायस्व बालक}% ४१

\twolineshloka
{इत्युक्त्वा समरे बालं शरमेकं समाददे}
{यमवक्त्रसमं घोरं लवणो येन घातितः}% ४२

\twolineshloka
{सन्धाय बाणं जाज्वल्यं हृदि भेत्तुं मनो दधत्}
{लवं वीरसहस्राणां वह्निवत्सर्वदाहकम्}% ४३

\twolineshloka
{तं बाणं प्रज्वलन्तं स द्योतयन्तं दिशो दश}
{दृष्ट्वा सस्मार बलिनं कुशं वैरिनिपातिनम्}% ४४

\twolineshloka
{यद्यस्मिन्समये वीरो भ्राता स्याद्बलवान्मम}
{तदा शत्रुघ्नवशता न मे स्याद्भयमुल्बणम्}% ४५

\twolineshloka
{एवं तर्कयतस्तस्य लवस्य च महात्मनः}
{हृदि लग्नो महाबाणो घोरः कालानलोपमः}% ४६

\twolineshloka
{मूर्च्छां प्राप तदा वीरो भूपसायकसंहतः}
{सङ्गरे सर्ववीराणां शिरोभिः समलङ्कृते}% ४७

{॥इति श्रीपद्मपुराणे पातालखण्डे शेषवात्स्यायनसंवादे रामाश्वमेधे लवमूर्च्छा नाम द्विषष्टितमोऽध्यायः॥६२॥}

\dnsub{त्रिषष्टितमोऽध्यायः}\resetShloka

\uvacha{शेष उवाच}

\twolineshloka
{लवं विमूर्च्छितं दृष्ट्वा बलिवैरिविदारणम्}
{शत्रुघ्नो जयमापेदे रणमूर्ध्नि महाबलः}% १

\twolineshloka
{लवं बालं रथे स्थाप्य शिरस्त्राणाद्यलङ्कृतम्}
{रामप्रतिनिधिं मूर्त्या ततो गन्तुमियेष सः}% २

\twolineshloka
{स्वमित्रं शत्रुणा ग्रस्तमिति दुःखसमन्विताः}
{बालामात्रेऽस्य सीतायै त्वरिताः सन्न्यवेदयन्}% ३

\uvacha{बाला ऊचुः}

\twolineshloka
{मातर्जानकि ते पुत्रो बलाद्वाहमपाहरत्}
{कस्यचिद्भूपवर्यस्य बलयुक्तस्य मानिनः}% ४

\twolineshloka
{ततो युद्धमभूद्घोरं तस्य सैन्येन जानकि}
{तदा वीरेण पुत्रेण तव सर्वं निपातितम्}% ५

\twolineshloka
{पश्चादपि जयं प्राप्तः सुतस्तव मनोहरः}
{तं भूपं मूर्छितं कृत्वा जयमाप रणाङ्गणे}% ६

\twolineshloka
{ततो मूर्च्छां विहायैष राजा परमदारुणः}
{सङ्कुप्य पातयामास तव पुत्रं रणाङ्गणे}% ७

\twolineshloka
{अस्माभिर्वारितः पूर्वं मा गृहाण हयोत्तमम्}
{अस्मान्सर्वांश्च धिक्कृत्य ब्राह्मणान्वेदपारगान्}% ८

\twolineshloka
{इति वाक्यं शिशूनां सा समाकर्ण्य सुदारुणम्}
{पपात भूतलोपस्थे दुःखयुक्ता रुरोद ह}% ९

\uvacha{सीतोवाच}

\twolineshloka
{कथं नृपो दयाहीनो बालेन सह युध्यति}
{अधर्मकृतदुर्बुद्धिर्यो मद्बालं न्यपातयत्}% १०

\twolineshloka
{लव वीरभवान्कुत्र वर्ततेऽति बलान्वितः}
{कथं त्वं निष्कृपस्याहो राज्ञोऽहार्षीद्धयोत्तमम्}% ११

\twolineshloka
{त्वं बालस्ते दुराक्रान्ताः सर्वशस्त्रविशारदाः}
{रथस्था विरथस्त्वं वै कथं युद्धं समं भवेत्}% १२

\twolineshloka
{ताताहं तु त्वया सार्द्धं रामत्यागासुखं जहौ}
{इदानीं रहिता युष्मत्कथं जीवामि कानने}% १३

\twolineshloka
{एहि मां मुञ्च यज्ञाश्वं गच्छत्वेष महीपतिः}
{मद्दुःखं नाभिजानासि मम दुःखप्रमार्जकः}% १४

\twolineshloka
{कुशो यद्यभविष्यत्स रणे वीरशिरोमणिः}
{अमोचयिष्यदधुना भवन्तं भूपपार्श्वतः}% १५

\twolineshloka
{सोऽपि मद्दैवतो नास्ति समीपे किं करोम्यतः}
{दैवमेव ममाप्यत्र कारणं दुःखसम्भवे}% १६

\twolineshloka
{एवमादि बहुश्रीमत्येषा वै विललाप ह}
{पादाङ्गुष्ठेन लिखती भूमिं नेत्रद्वयाश्रुभिः}% १७

\twolineshloka
{बालान्प्रति जगादासौ पृथुकः स च भूपतिः}
{कथं मत्सुतमापात्य रणे कुत्र गमिष्यति}% १८

\twolineshloka
{इति वाक्यं वदत्येषा जानकी पतिदेवता}
{तावत्कुशस्तु सम्प्राप्त उज्जयिन्या महर्षिभिः}% १९

\twolineshloka
{माघासितचतुर्दश्यां महाकालं समर्च्य च}
{प्राप्य भूरिवरांस्तस्मादागमन्मातृसन्निधौ}% २०

\twolineshloka
{जानकीं विह्वलां दृष्ट्वा नेत्रोद्भूताश्रु विक्लवाम्}
{शोकविह्वलदीनाङ्गीं बभाषे यावदुत्सुकः}% २१

\twolineshloka
{तदा स्वबाहुरवदत्स्फुरद्युद्धाभिशंसनः}
{हृदये चरणोत्साहो बभूवातिरथस्य हि}% २२

\twolineshloka
{स प्रत्युवाच जननीं दीनगद्गदभाषिणीम्}
{मातस्तव गतं दुःखं मयि पुत्र उपस्थिते}% २३

\twolineshloka
{मयि जीवति ते नेत्रादश्रूणि भुवि नो पतन्}
{प्रसूमुवाचाश्रुखिन्नां दीनगद्गदभाषिणीम्}% २४

\twolineshloka
{कुशो दुःखमितः सद्यो दुःखितां धीरमानसः}
{मम भ्राता लवः कुत्र वर्तते वैरिमर्दनः}% २५

\twolineshloka
{सदा मामागतं ज्ञात्वा प्रहर्षन्सन्निधावियात्}
{न दृश्यते कथं वीरः कुत्र रन्तुं गतो बली}% २६

\twolineshloka
{केन वा सह बालत्वाद्गतो मां वै निरीक्षितुम्}
{किं त्वं रोदिषि मे मातर्लवः कुत्र स वर्तते}% २७

\twolineshloka
{तन्मे कथय सर्वं यत्तव दुःखस्य कारणम्}
{तच्छ्रुत्वा पुत्रवाक्यं सा दुःखिता कुशमब्रवीत्}% २८

\twolineshloka
{लवो धृतो नृपेणात्र केनचिद्धयरक्षिणा}
{बबन्ध बालको मेत्र हयं यागक्रियोचितम्}% २९

\twolineshloka
{तद्रक्षकान्बहूञ्जिग्ये एकोऽनेकान्रिपून्बली}
{राजा तं मूर्च्छितं कृत्वा बबन्ध रणमूर्धनि}% ३०

\twolineshloka
{बालका इति मामूचुः सहगन्तार एव हि}
{ततोऽहं दुःखिता जाता निशम्य लवमाधृतम्}% ३१


\threelineshloka
{त्वं मोचय बलात्तस्मात्काले प्राप्तो नृपोत्तमात्}
{निशम्य मातुर्वचनं कुशः कोपसमन्वितः}
{जगाद तां दशन्नोष्ठं दन्तैर्दन्तान्विनिष्पिषन्}% ३२

\uvacha{कुश उवाच}

\twolineshloka
{मातर्जानीहि तं मुक्तं लवं पाशस्य बन्धनात्}
{इदानीं हन्मि तं बाणैः समग्रबलवाहनम्}% ३३

\twolineshloka
{यदि देवोऽमरो वापि यदि शर्वः समागतः}
{तथापि मोचये तस्माद्बाणैर्निशितपर्वभिः}% ३४

\twolineshloka
{मा रोदिषि मातरिह वीराणां रणमूर्छितम्}
{कीर्तयेऽत्र भवत्येव पलायनमकीर्तये}% ३५

\twolineshloka
{देहि मे कवचं दिव्यं धनुर्गुणसमन्वितम्}
{शिरस्त्राणं च मे मातः करवालं तथाशितम्}% ३६

\twolineshloka
{इदानीं यामि समरे पातयामि बलं महत्}
{मोचयामि भ्रातरं स्वं रणमध्याद्विमूर्छितम्}% ३७

\twolineshloka
{न मोचयाम्यद्य पुत्रं तव मातर्महारणात्}
{तदा तौ मे भवत्पादौ संरुष्टौ भवतां क्षितौ}% ३८

\uvacha{शेष उवाच}

\twolineshloka
{इति वाक्येन सन्तुष्टा जानकी शुभलक्षणा}
{सर्वं प्रादादस्त्रवृन्दं जयाशीर्भिर्नियुज्यतम्}% ३९

\twolineshloka
{प्रयाहि पुत्र सङ्ग्रामं लवं मोचय मूर्च्छितम्}
{इत्याज्ञप्तः कुशः सङ्ख्ये कवची कुण्डली बली}% ४०

\twolineshloka
{मुकुटी करवाली च चर्मधारी धनुर्धरः}
{अक्षयाविषुधी कृत्वा स्कन्धयोः सिंहवीर्ययोः}% ४१

\twolineshloka
{जगाम तरसा नत्वा मातृपादौ रथाग्रणीः}
{वेगेन यावद्युद्धाय गच्छति क्षिप्रमाहवे}% ४२

\twolineshloka
{तावद्ददर्श स्वलवं वैरिवृन्दनिपातकम्}
{आयान्तं तं कुशं वीरा ददृशुः सुमहाभटाः}% ४३

\twolineshloka
{कृतान्तमिव संहर्तुं सर्वं विश्वमुपस्थितम्}
{लवो महाबलं दृष्ट्वा कुशं भ्रातरमागतम्}% ४४

\twolineshloka
{अत्यन्तं वह्निवद्युद्धे दिदीपे वायुना समम्}
{रथादुन्मुच्य चात्मानं युद्धाय स विनिर्गतः}% ४५

\twolineshloka
{कुशः सर्वान्रणस्थान्वै वीरान्पूर्वदिशि क्षिपत्}
{पश्चिमायां दिशि लवः कोपात्सर्वान्समैरयत्}% ४६

\twolineshloka
{कुशबाणव्यथाव्याप्ता लवसायकपीडिताः}
{सैन्ये जना मुने सर्वे उत्कल्लोलाम्बुधिभ्रमाः}% ४७

\twolineshloka
{कुशेन च लवेनाथ शरव्रातैः प्रपीडितम्}
{न शर्म लेभे सकलं सैन्यं वीरप्रपूरितम्}% ४८

\twolineshloka
{इतस्ततः प्रभग्नं तद्बलं त्रस्तं पुनः पुनः}
{न कुत्रचिद्रणे स्थित्वा युद्धमैच्छद्बलान्वितः}% ४९

\twolineshloka
{एतस्मिन्समये राजा शत्रुघ्नः परतापनः}
{कुशं वीरं ययौ योद्धुं तादृशं लवसन्निभम्}% ५०

\twolineshloka
{कुशं दृष्ट्वा बलाक्रान्तं राममूर्तिसमप्रभम्}
{रथे तिष्ठन्हेममये जगाद परवीरहा}% ५१

\uvacha{शत्रुघ्न उवाच}

\twolineshloka
{कोऽसि त्वं सन्निभो भ्रात्रा लवेन सुमहाबलः}
{किं नामासि महावीर कस्ते तातः क्व ते प्रसूः}% ५२

\twolineshloka
{कथं वने द्विजैर्जुष्टे तिष्ठसि त्वं नरर्षभ}
{सर्वं शंस यथायुध्ये त्वया सह महाबल}% ५३

\twolineshloka
{इति वाक्यं समाकर्ण्य कुशः प्रोवाच भूमिपम्}
{मेघगम्भीरया वाचा नादयन्रणमण्डलम्}% ५४

\twolineshloka
{केवलं सुषुवे सीता पतिव्रतपरायणा}
{वने वसावो वाल्मीकेश्चरणार्चनतत्परौ}% ५५

\twolineshloka
{मातृसेवासमुद्युक्तौ सर्वविद्याविशारदौ}
{कुशो लव इति प्रख्यामागतौ भूपतेऽनघ}% ५६

\twolineshloka
{कस्त्वं वीरो रणश्लाघी किमर्थं हयसत्तमः}
{मुक्तोऽस्ति समरे त्वद्य जेतासि बलसंयुतः}% ५७

\twolineshloka
{युध्यस्व त्वं मया सार्द्धं यदि वीरोऽसि भूमिप}
{इदानीं पातयिष्यामि भवन्तं रणमूर्धनि}% ५८

\twolineshloka
{शत्रुघ्नस्तं सुतं ज्ञात्वा सीताया रामसम्भवम्}
{विसिष्माय स्वयं चित्ते कोपाद्धनुरुपाददत्}% ५९

\twolineshloka
{तमात्तधनुषं दृष्ट्वा कुशः कोपसमन्वितः}
{विस्फारयामास धनुः स्वीयं सुदृढमुत्तमम्}% ६०

\twolineshloka
{मुमोच बाणान्निशिताञ्छत्रुघ्नः सर्वशस्त्रवित्}
{तांश्चिच्छेद कुशः सर्वांल्लीलया प्रहसन्रणे}% ६१

\twolineshloka
{बाणाश्च शतसाहस्राः कुशस्य च नृपस्य च}
{भुवनं व्याप्नुवन्सर्वं तच्चित्रमभवन्मुने}% ६२

\twolineshloka
{अग्न्यस्त्रेण कुशः सर्वान्ददाह तरसा बली}
{शमयामास तं भूपः पर्जन्यास्त्रेण वीर्यवान्}% ६३


\threelineshloka
{शमयामास तं भूपो वायव्येनातिविक्रमः}
{तदा वायुरभूत्तीव्रः सर्वतो रणमण्डले}
{पर्वतास्त्रेण तं वायुं क्षोभयन्तं समावृणोत्}% ६४

\threelineshloka
{वज्रास्त्रेण नृपः सङ्ख्ये चिच्छेद सनगोपलान्}
{तदा नारायणास्त्रं स मुमोच कुश उद्भटः}
{नारायणं तदा भूपं नाशकत्परिबाधितुम्}% ६५

\twolineshloka
{तदा प्रकुपितोऽत्यतं कुशः कोपपरायणः}
{उवाच भूपं शत्रुघ्नं महाबलपराक्रमम्}% ६६

\twolineshloka
{जानामि त्वां महावीरं सङ्ग्रामे जयकारिणम्}
{यत्त्वां नारायणास्त्रं मे न बबाधे भयानकम्}% ६७

\twolineshloka
{इदानीं पातयाम्यद्य भूमौ त्वां नृपते शरैः}
{त्रिभिश्चेन्नकरोम्येतत्प्रतिज्ञां तर्हि मे शृणु}% ६८

\twolineshloka
{यो मनुष्यवपुः प्राप्य दुर्लभं पुण्यकोटिभिः}
{तन्नाद्रियेत सम्मोहात्तस्य मेस्त्वत्र पातकम्}% ६९

\twolineshloka
{सावधानो भवानत्र भवतु प्रधनाङ्गणे}
{पातयामि क्षितौ सद्य इत्युक्त्वा स्वशरासने}% ७०

\twolineshloka
{शरं संरोपयामास घोरं कालानलप्रभम्}
{लक्षीकृत्य रिपोर्वक्षो विपुलं कठिनं महत्}% ७१

\twolineshloka
{तं सन्धितं शरं दृष्ट्वा शत्रुघ्नः कोपमूर्च्छितः}
{मुमोच बाणान्निशितान्कुशत्वग्भेदकारकान्}% ७२

\twolineshloka
{स बाणो हृदयं तस्य भेत्तुं तत्प्रचचाल वै}
{घोररूपो वह्निसमआशीविषवदुच्छ्वसन्}% ७३

\twolineshloka
{स बाणो नृपवर्येण रामं स्मृत्वाशुलक्षितः}
{चिच्छेद कुशमुक्तं स सायकं शितपर्वकम्}% ७४

\twolineshloka
{तदात्यन्तं प्रकुपितः कुशो बाणस्य कृन्तनात्}
{अपरं सायकं चापे दधार शितपर्वकम्}% ७५

\twolineshloka
{स यावत्तदुरो भेत्तुं करोति च बलोद्धुरः}
{तं तावदच्छिनत्तस्य शरं कालानलप्रभम्}% ७६

\twolineshloka
{तदा कुशो मातृपादौ स्मृत्वा रोषसमन्वितः}
{तृतीयं चापके स्वीये दधार शरमद्भुतम्}% ७७

\twolineshloka
{शत्रुघ्नस्तमपि क्षिप्रं च्छेत्तुं बाणं समाददे}
{तावद्विद्धः शरेणासौ पपात धरणीतले}% ७८

\twolineshloka
{हाहाकारो महानासीच्छत्रुघ्ने विनिपातिते}
{जयमापकुशस्तत्र स्वबाहुबलदर्पितः}% ७९

{॥इति श्रीपद्मपुराणे पातालखण्डे शेषवात्स्यायनसंवादे रामाश्वमेधे शत्रुघ्नमूर्च्छने कुशजयो नाम त्रिषष्टितमोऽध्यायः॥६३॥}

\dnsub{चतुःषष्टितमोऽध्यायः}\resetShloka

\uvacha{शेष उवाच}

\twolineshloka
{शत्रुघ्नं पतितं वीक्ष्य सुरथः प्रवरो नृपः}
{प्रययौ मणिना सृष्टे रथे तिष्ठन्महाद्भुते}% १

\twolineshloka
{पुष्कलस्तु रणे पूर्वं पातितः स विचारयन्}
{लवं ययौ तदा योद्धुं महावीरबलोन्नतम्}% २

\twolineshloka
{सुरथः कुशमासाद्य बाणान्मुञ्चन्ननेकधा}
{व्यथयामास समरे महावीरशिरोमणिः}% ३

\twolineshloka
{सुरथं विरथं चक्रे बाणैर्दशभिरुच्छिखैः}
{धनुश्चिच्छेद तरसा सुदृढं गुणपूरितम्}% ४

\twolineshloka
{अस्त्रप्रत्यस्त्रसंहारैः क्षैपणैः प्रतिक्षेपणैः}
{अभवत्तुमुलं युद्धं वीराणां रोमहर्षणम्}% ५

\twolineshloka
{अत्यन्तं समरोद्युक्ते सुरथे दुर्जये नृपे}
{कुशः सञ्चिन्तयामास किङ्कर्तव्यं रणे मया}% ६

\twolineshloka
{विचार्य निशितं घोरं सायकं समुपाददे}
{हननाय नृपस्यास्य महाबलसमन्वितः}% ७

\twolineshloka
{तमागतं शरं दृष्ट्वा कालानलसमप्रभम्}
{छेत्तुं मतिं चकाराशु तावल्लग्नो महाशरः}% ८

\twolineshloka
{मुमूर्च्छ समरे वीरो महावीरबलस्ततः}
{पपात स्यन्दनोपस्थे सारथिस्तमुपाहरत्}% ९

\twolineshloka
{सुरथे पतिते दृष्ट्वा कुशं जयसमन्वितम्}
{त्रासयन्तं वीरगणानियाय पवनात्मजः}% १०

\twolineshloka
{समीरसूनुं प्रबलमायान्तं वीक्ष्य वानरम्}
{जहास दर्शयन्दन्तान्कोपयन्निव तं क्रुधा}% ११

\twolineshloka
{उवाच च हनूमन्तमेहि त्वं मम सम्मुखम्}
{भेत्स्ये बाणसहस्रेण मृतो यास्यसि यामिनीम्}% १२

\twolineshloka
{इत्युक्तो हनुमांज्ञात्वा रामसूनुं महाबलम्}
{स्वामिकार्यं प्रकर्तव्यमिति कृत्वा प्रधावितः}% १३

\twolineshloka
{शालमुत्पाट्य तरसा विशालं शतशाखिनम्}
{कुशं वक्षसि संलक्ष्य ययौ योद्धुं महाबलः}% १४

\twolineshloka
{शालहस्तं समायान्तं हनूमन्तं महाबलम्}
{त्रिभिः क्षुरप्रैर्विव्याध हृदि चन्द्रोपमैर्बली}% १५

\twolineshloka
{स बाणविद्धस्तरसा कुशेन बलशालिना}
{शालेन हृदि सञ्जघ्ने दन्तान्निष्पिष्य मारुतिः}% १६

\twolineshloka
{शालाहतस्तदा बालः किञ्चिन्नाकम्पत स्मयात्}
{तदा वीराः प्रशंसां तु प्रचक्रुस्तस्य बाल्यतः}% १७

\twolineshloka
{स शालेन हतो वीरः संहारास्त्रं समाददे}
{संहन्तुं वैरिणं कोपात्कुशः स परमास्त्रवित्}% १८

\twolineshloka
{संहारास्त्रं समालोक्य दुर्जयं कुशमोचितम्}
{दध्यौ रामं स्वमनसा भक्तविघ्नविनाशकम्}% १९

\twolineshloka
{तदा मुक्तं कुशेनाशु तदस्त्रं हृदि मारुतेः}
{लग्नं महाव्यथाकारि तेन मूर्च्छामितः पुनः}% २०

\twolineshloka
{मूर्च्छां प्राप्तं तु तं दृष्ट्वा प्लवङ्गं बलसंयुतः}
{विव्याध सायकैस्तीक्ष्णैः सैन्यं तत्सकलं महत्}% २१

\twolineshloka
{तस्य बाणायुतैर्भग्नं बलं सर्वं रणाङ्गणे}
{पलायनपरं जातं चतुरङ्गसमन्वितम्}% २२

\twolineshloka
{तदा कपिपतिः कोपात्सुग्रीवो रक्षको महान्}
{अभ्यधावन्नगान्नैकानुत्पाट्य कुशमुद्भटम्}% २३

\twolineshloka
{कुशः सर्वान्प्रचिच्छेद लीलया प्रहसन्नगान्}
{पुनरप्यागतान्वृक्षांश्चिच्छेद तरसा बली}% २४

\twolineshloka
{अनेकबाणव्यथितः सुग्रीवः समराङ्गणे}
{जग्राह पर्वतं घोरं कुशमस्तकमध्यतः}% २५

\twolineshloka
{कुशस्तं नगमायान्तं वीक्ष्य बाणैरनेकधा}
{निष्पिपेष चकाराशु महारुद्राङ्गयोग्यताम्}% २६

\twolineshloka
{सुग्रीवस्तन्महत्कर्म दृष्ट्वा बालेन निर्मितम्}
{जयाशाप्रतिनिर्वृत्तो बभूव समराङ्गणे}% २७

\twolineshloka
{रणमध्ये दुराक्रान्तं कुशं लाङ्गूलताडकम्}
{अत्यमर्षीरुषाक्रान्तस्तं हन्तुं नगमाददे}% २८

\twolineshloka
{आत्मानं हन्तुमुद्युक्तं वीक्ष्य सुग्रीवमादरात्}
{ताडयामास बहुभिः सायकैः शितपर्वभिः}% २९

\twolineshloka
{स ताडितो बहुविधैः शरैः पीडासमन्वितः}
{कुशं हन्तुं समारब्धो ययौ शालं समाददे}% ३०

\twolineshloka
{तदापि च कुशो वीरो वारुणास्त्रं समाददे}
{बबन्ध तं च पाशेन दृढेन स लवाग्रजः}% ३१

\twolineshloka
{स बद्धः पाशकैः स्निग्धैः कुशेन बलशालिना}
{पपात रणमध्ये वै महावीरैरलङ्कृते}% ३२

\twolineshloka
{सुग्रीवं पतितं दृष्ट्वा वीराः सर्वत्र दुद्रुवुः}
{जयमाप लवभ्राता महावीरशिरोमणिः}% ३३

\twolineshloka
{तावल्लवो भटाञ्जित्वा पुष्कलं चाङ्गदं तथा}
{प्रतापाग्र्यं वीरमणिं तथान्यानपि भूभुजः}% ३४

\twolineshloka
{जयं प्राप्य रणे वीरो लवो भ्रातरमागमत्}
{सङ्ग्रामे जयकर्तारं वैरिकोटिनिपातकम्}% ३५

\twolineshloka
{परस्परं प्रहृषितौ परिरम्भं प्रकुर्वतः}
{जयं प्राप्तौ तदा वार्तां मुने चक्रतुरुन्मदौ}% ३६

\uvacha{लव उवाच}

\twolineshloka
{भ्रातस्तव प्रसादेन निस्तीर्णो रणतोयधिः}
{इदानीं वीररणकं शोधयावः सुशोभितम्}% ३७

\twolineshloka
{इत्युक्त्वा त्वरितं वीरो जग्मतुस्तौ कुशीलवौ}
{राज्ञो मौलिमणिं चित्रं जग्राह कनकाचितम्}% ३८

\twolineshloka
{पुष्कलस्य लवो वीरो जग्राह मुकुटं शुभम्}
{अङ्गदे च महानर्घ्ये शत्रुघ्नस्यापरस्य च}% ३९

\twolineshloka
{गृहीत्वा शस्त्रसङ्घातं हनूमन्तं कपीश्वरम्}
{सुग्रीवं सविधे गत्वा उभावपि बबन्धतुः}% ४०

\twolineshloka
{पुच्छे वायुसुतस्यायं गृहीत्वा तु कुशानुजः}
{भ्रातरं प्रत्युवाचेदं नेष्यामि स्वकमन्दिरम्}% ४१

\twolineshloka
{आवयोर्जननी प्रीत्यै गृहीत्वा पुच्छके त्वहम्}
{क्रीडार्थमृषिपुत्राणां कौतुकार्थं ममैव च}% ४२

\twolineshloka
{एतच्छ्रुत्वा ततो वाक्यमुवाच च कुशो लवम्}
{अहमेनं ग्रहीष्यामि वानरं बलिनं दृढम्}% ४३

\twolineshloka
{इत्येवं भाषमाणौ तौ बद्ध्वा तौ बलिनां वरौ}
{पुच्छयोर्बलिनौ धृत्वा जग्मतुः स्वाश्रमं प्रति}% ४४

\twolineshloka
{स्वाश्रमाय प्रगच्छन्तौ वीक्ष्य तौ कपिसत्तमौ}
{कम्पमानौ जगदतुरन्योन्यं भीतया गिरा}% ४५

\twolineshloka
{हनूमान्कपिराजानं प्रत्युवाच भयार्द्रधीः}
{एतौ रामसुतावस्मान्नेष्यतः स्वाश्रमं प्रति}% ४६

\twolineshloka
{मया पूर्वं कृतं कर्म जानकीं प्रतिगच्छता}
{तत्र मे जानकी देवी सम्मुखाभून्मनोहरा}% ४७

\twolineshloka
{सा मां द्रक्ष्यति वैदेही बद्धं पाशेन वैरिणा}
{तदा हसिष्यति वरा त्रपा मेऽत्र भविष्यति}% ४८

\twolineshloka
{मया किमत्र कर्तव्यं प्राणत्यागो भविष्यति}
{महद्दुःखं चापतितं स रामः किं करिष्यति}% ४९

\twolineshloka
{सुग्रीवस्तद्वचः श्रुत्वा ममाप्येवं महाकपे}
{नेष्यते यदि मामेवं निधनं तु भविष्यति}% ५०

\twolineshloka
{एवं कथयतोरेव ह्यन्योन्यं भयभीतयोः}
{कुशो लवश्च भवनं मातुः प्रापतुरोजसा}% ५१

\twolineshloka
{तावायातौ समीक्ष्यैव जहर्ष जननी तयोः}
{अन्योन्यं परमप्रीत्या परिरेभे निजौ सुतौ}% ५२

\twolineshloka
{ताभ्यां पुच्छगृहीतौ तौ वानरौ वीक्ष्य जानकी}
{हनूमन्तं च सुग्रीवं सर्ववीरं कपीश्वरम्}% ५३

\twolineshloka
{जहास पाशबद्धौ तौ वीक्षमाणा वराङ्गना}
{उवाच च विमोक्षार्थं वदन्ती वचनं वरम्}% ५४

\twolineshloka
{पुत्रौ प्रमुञ्चतं कीशौ महावीरौ महाबलौ}
{ईक्षन्तौ मां यदि स्फीतौ प्राणत्यागं करिष्यतः}% ५५

\twolineshloka
{अयं वै हनुमान्वीरो यो ददाह दनोः पुरीम्}
{अयमप्यृक्षराजो हि सर्ववानरभूमिपः}% ५६

\twolineshloka
{किमर्थं विधृतौ कुत्र किं वा कृतमनादरात्}
{पुच्छे युवाभ्यां विधृतौ स महान्विस्मयोऽस्ति मे}% ५७

\twolineshloka
{इति मातुर्वचः श्लक्ष्णं वीक्ष्यतां पुत्रकौ तदा}
{ऊचतुर्विनयश्रेष्ठौ महाबलसमन्वितौ}% ५८

\twolineshloka
{मातः कश्चन भूपालो रामो दाशरथिर्बली}
{तेन मुक्तो हयः स्वर्णभालपत्रः सुशोभितः}% ५९

\twolineshloka
{तत्रैवं लिखितं मातरेकवीराप्रसूर्मम}
{ये क्षत्रियास्ते गृह्णन्तु नोचेत्पादतलार्चकाः}% ६०

\twolineshloka
{तदा मया विचारो वै कृतः स्वान्ते पतिव्रते}
{भवती क्षत्रिया किं न वीरसूः किं न वा भवेत्}% ६१

\twolineshloka
{धार्ष्ट्यं तद्वीक्ष्य भूपस्य गृहीतोऽश्वो मया बलात्}
{जितं कुशेन वीरेण सैन्यं तत्पातितं रणे}% ६२

\twolineshloka
{मुकुटोऽयं भूमिपतेर्जानीहि पतिदेवते}
{अयमप्यन्यवीरस्य पुष्कलस्य महात्मनः}% ६३

\twolineshloka
{जानीहि मुकुटं त्वन्यं मणिमुक्ताविराजितम्}
{अश्वोऽयं मे मनोहारी कामयानो हि भूपतेः}% ६४

\twolineshloka
{आरोहणाय मद्भ्रातुर्जानीहि बलिनो वरे}
{इमौ कीशौ मया रन्तुमानीतौ बलिनां वरौ}% ६५

\twolineshloka
{कौतुकार्थं तवैवैतौ सङ्ग्रामे युद्धकारकौ}
{इति वाक्यं समाकर्ण्य जानकी पतिदेवता}% ६६

\onelineshloka*
{जगाद पुत्रौ तौ वीरौ मोचयेथां पुनः पुनः}

\uvacha{सीतोवाच}

\onelineshloka
{युवाभ्यामनयः सृष्टो हृतो रामहयो महान्}% ६७

\twolineshloka
{अनेके पातिता वीरा इमौ बद्धौ कपीश्वरौ}
{पितुस्तव हयो वीरो यागार्थं मोचितोऽमुना}% ६८

\twolineshloka
{तस्यापि हृतवन्तौ किं वाजिनं मखसत्तमे}
{मुञ्चतं प्लवगावेतौ मुञ्चतं वाजिनां वरम्}% ६९

\twolineshloka
{क्षाम्यतां भूपतेर्भ्राता शत्रुघ्नः परकोपनः}
{जनन्यास्तद्वचः श्रुत्वा ऊचतुस्तां बलान्वितौ}% ७०

\twolineshloka
{क्षात्रधर्मेण तं भूपं जितवन्तौ बलान्वितम्}
{नास्माकमनयोर्भावि क्षात्रधर्मेण युध्यताम्}% ७१

\onelineshloka
{वाल्मीकिना पुरा प्रोक्तमस्माकं पठतां पुरः}% ७२

\twolineshloka
{कण्वस्याश्रमकेवाहं धृत्वा यागक्रियोचितम्}
{तस्मात्सुतः स्वपित्रापि युध्येद्भ्रात्रापि चानुजः}% ७३

\twolineshloka
{गुरुणा शिष्य एवापि तस्मान्नो पापसम्भवः}
{त्वदाज्ञातो ऽधुना चावां दास्यावो हयमुत्तमम्}% ७४

\twolineshloka
{मोक्ष्यावः कीशावेतौ हि करिष्यावो वचस्तव}
{इत्युक्त्वा मातरं वीरौ गतौ रणे कपीश्वरौ}% ७५

\twolineshloka
{अमुञ्चतां हयं चापि हयमेधक्रियोचितम्}
{सीतादेवी स्वपुत्राभ्यां श्रुत्वा सैन्यं निपातितम्}% ७६

\twolineshloka
{श्रीरामं मनसा ध्यात्वा भानुमैक्षत साक्षिणम्}
{यद्यहं मनसा वाचा कर्मणा रघुनायकम्}% ७७

\twolineshloka
{भजामि नान्यं मनसा तर्हि जीवेदयं नृपः}
{सैन्यं चापि महत्सर्वं यन्नाशितमिदं बलात्}% ७८

\twolineshloka
{पुत्राभ्यां तत्तु जीवेत मत्सत्याज्जगताम्पते}
{इति यावद्वचो ब्रूते जानकीपतिदेवता}% ७९

\onelineshloka
{तावद्बलं च तत्सर्वं जीवितं रणमूर्द्धनि}% ८०

{॥इति श्रीपद्मपुराणे पातालखण्डे शेषवात्स्यायनसंवादे रामाश्वमेधे सैन्यजीवनं नाम चतुःषष्टितमोऽध्यायः॥६४॥}

\dnsub{पञ्चषष्टितमोऽध्यायः}\resetShloka

\uvacha{शेष उवाच}

\twolineshloka
{क्षणान्मूर्च्छां जहौ वीरः शत्रुघ्नः समराङ्गणे}
{अन्येऽपि वीराबलिनो मूर्च्छां प्राप्ताः सुजीविताः}% १

\twolineshloka
{शत्रुघ्नो वाजिनां श्रेष्ठं ददर्श पुरतः स्थितम्}
{आत्मानं च शिरस्त्राणरहितं सैन्यजीवितम्}% २

\twolineshloka
{वीक्ष्य चित्रमिदं स्वान्ते चकारच जगाद ह}
{सुमतिं मन्त्रिणां श्रेष्ठं मूर्च्छाविरहितं तदा}% ३

\twolineshloka
{कृपां कृत्वा हयं प्रादाद्बालो यज्ञस्य पूर्तये}
{गच्छाम रामं तरसा हयागमनकाङ्क्षिणम्}% ४

\twolineshloka
{इत्युक्त्वा स रथे स्थित्वा हयमादाय वेगतः}
{ययौ तदाश्रमाद्दूरं भेरीशङ्खविवर्जितः}% ५

\twolineshloka
{तत्पृष्ठतो महासैन्यं चतुरङ्गसमन्वितम्}
{चचाल कुर्वन्सम्भग्नं स्वभारेण फणीश्वरम्}% ६

\twolineshloka
{जवेन जाह्नवीं तीर्त्वा कल्लोलजलशालिनीम्}
{जगाम विषये स्वीये स्वकीयजनशोभिते}% ७

\twolineshloka
{पुष्कलेन युतो राजा सुरथेन समन्वितः}
{रथे मणिमये तिष्ठन्महत्कोदण्डधारकः}% ८

\twolineshloka
{हयं तं पुरतः कृत्वा रत्नमालाविभूषितम्}
{श्वेतातपत्रं तस्यैव मूर्ध्नि चामरभूषितम्}% ९

\twolineshloka
{अनेकरथसाहस्रैः परितो बलिभिर्नृपैः}
{उद्यत्कोदण्डललितैर्वीरनादविभूषितैः}% १०

\twolineshloka
{क्रमेण नगरीं प्राप सूर्यवंशविभूषिताम्}
{अनेकैः केतुभिः श्रेष्ठैर्भूषितां दुर्गराजिताम्}% ११

\twolineshloka
{रामः श्रुत्वा हयं प्राप्तं शत्रुघ्नेन सहामुना}
{पुष्कलेन च वीरेण ययौ हर्षमनेकधा}% १२

\twolineshloka
{कटकं निर्दिदेशासौ चतुरङ्गं महाबलम्}
{लक्ष्मणं प्रेषयामास भ्रातरं बलिनां वरम्}% १३

\twolineshloka
{लक्ष्मणः सैन्यसहितो गत्वा भ्रातरमागतम्}
{परिरेभे मुदाक्रान्तः क्षतशोभितगात्रकम्}% १४

\twolineshloka
{सर्वत्र कुशलं पृष्टो वार्तां चात्र चकार सः}
{परमं हर्षमापन्नः शत्रुघ्नः सङ्गतो मुदा}% १५

\twolineshloka
{सौमित्रिः स्वरथे स्थित्वा भ्रात्रा सह महामनाः}
{सैन्येन महता वीरो ययौ स्वनगरीं प्रति}% १६

\twolineshloka
{सरयूः पुण्यसलिला पवित्रित जगत्त्रया}
{रामपादरजः पूता शरच्चन्द्रसमप्रभा}% १७

\twolineshloka
{हंसकारण्डवाकीर्णा चक्रवाकोपशोभिता}
{विचित्रतरवर्णैश्च पक्षिभिर्नादिता भृशम्}% १८

\twolineshloka
{मण्डपास्तत्र बहुशो रामचन्द्रेणकारिताः}
{ब्राह्मणानां वेदविदां पृथक्पाठनिनादकाः}% १९

\twolineshloka
{क्षत्रियास्तत्र बहवो धनुःपाणि सुशोभिताः}
{ज्याटङ्कारेण बहुना नादयन्तो महीतलम्}% २०

\twolineshloka
{भुञ्जते ब्राह्मणा यत्र विचित्रान्नैर्मनोहरैः}
{परस्परं प्रपश्यन्तो वार्तां चक्रुर्मनोहराम्}% २१

\twolineshloka
{पायसान्नानि शुभ्राणि चन्द्रकान्तिसमानि च}
{क्षीराज्यबहुयुक्तानि शर्करामिश्रितानि च}% २२

\twolineshloka
{अपूपास्तत्र बहुलाश्चन्द्रबिम्बसमाः श्रिया}
{कर्पूरादिसुगन्धेन वासिताः सुमनोहराः}% २३

\twolineshloka
{फेनिकाघटकाः स्निग्धाः शतच्छिद्रा विरन्ध्रकाः}
{शष्कुल्यो मण्डकामृष्टा मधुरान्नसमन्विताः}% २४

\twolineshloka
{भक्तं कुमुदसङ्काशं मुद्गदालिविमिश्रितम्}
{सुगन्धेन समायुक्तमत्यन्तं प्रीतिदायकम्}% २५

\twolineshloka
{ओदनो दधिना युक्तो भीमसेनसमन्वितः}
{स्वादुपाककरैः सृष्टः पात्रे मुक्तः प्रवेषकैः}% २६

\twolineshloka
{तत्र केचिद्द्विजाः पात्रे निक्षिप्तं वीक्ष्य पायसम्}
{परस्परं ते प्रत्यूचुः किमिदं दृश्यतेऽद्भुतम्}% २७

\twolineshloka
{किं चन्द्रबिम्बं नभसः पतितं तमसो भयात्}
{अमृतं तु भवत्यत्र मृत्युनाशकमद्भुतम्}% २८

\twolineshloka
{तच्छ्रुत्वा रोषताम्राक्षः प्रोवाचान्यो द्विजोत्तमः}
{नभवत्येव चन्द्रस्य बिम्बं त्वमृतविप्लुतम्}% २९

\twolineshloka
{एकमिन्दोर्वपुस्त्वेतद्दृश्यते सदृशं कथम्}
{ब्राह्मणानां सहस्रस्य पात्रे पात्रे पृथक्पृथक्}% ३०

\twolineshloka
{ततो जानीहि कुमुदं कर्पूरं वा भविष्यति}
{मा जानीहि मृगाङ्कस्य बिम्बं शुभ्रश्रियान्वितम्}% ३१

\twolineshloka
{तावदन्यो रुषाक्रान्तो धुन्वन्स्वं मस्तकं तथा}
{न जानन्ति द्विजा मूढाः स्वादुज्ञाना विचक्षणाः}% ३२

\twolineshloka
{इदं तु क्षौद्रकन्दस्यरसेन परिपाचितम्}
{जानीहि शतपत्रस्य पुष्पाणि मधुराणि च}% ३३

\twolineshloka
{एवं परस्परं विप्राः कन्दमूलफलाशिनः}
{तर्कयन्ति मुने प्रीता रसज्ञानेऽतिलोलुपाः}% ३४

\twolineshloka
{तावदन्यो द्विजः प्राह क्षत्त्रियाणां वरं जनुः}
{भोक्ष्यन्ते तादृशं त्वन्नं महत्पुण्यैरुपस्कृतम्}% ३५

\twolineshloka
{तदा तं प्राब्रवीद्विप्रो दत्तस्य फलमीदृशम्}
{ये ददत्यग्रजन्मभ्यः प्राप्नुवन्ति त ईप्सितम्}% ३६

\twolineshloka
{यैरर्चितो नैव हरिर्नैवेद्यैर्विविधैर्मुहुः}
{तेषामेतादृशं भोज्यं न भवेदक्षिगोचरम्}% ३७

\twolineshloka
{यैर्नरैरग्रजन्मानो भोजिता विविधै रसैः}
{भुञ्जते ते स्वादुरसं पापिनां चक्षुरुज्झितम्}% ३८

\twolineshloka
{एवंविधैरसैर्मिष्टैर्भोजिता द्विजसत्तमाः}
{मण्डपे विपठन्तस्ते शब्दब्रह्मविचक्षणाः}% ३९

\twolineshloka
{नृत्यन्त्येके हसन्त्येके नदन्त्येके प्रहर्षिताः}
{उत्सवो बहुरुद्भाति तत्र शत्रुघ्न आगमत्}% ४०

\twolineshloka
{रामः शत्रुघ्नमायान्तं पुष्कलेन समन्वितम्}
{निरीक्ष्यमुदमुद्भूतां रक्षितुं नाशकत्तदा}% ४१

\twolineshloka
{यावदुत्तिष्ठते रामो भ्रातरं हयपालकम्}
{तावद्रामपदेलग्नः शत्रुघ्नो भ्रातृवत्सलः}% ४२

\twolineshloka
{पादयोः पतितं वीक्ष्य भ्रातरं विनयान्वितम्}
{परिरेभे दृढं प्रीतः क्षतसंशोभिताङ्गकम्}% ४३

\twolineshloka
{अश्रूणि बहुधा मुञ्चन्हर्षाच्छिरसि राघवः}
{अत्यन्तं परमां प्राप मुदं वचनदूरगाम्}% ४४

\twolineshloka
{पुष्कलं स्वीयपदयोर्नम्रं विनयविह्वलः}
{सुदृढं भुजयोर्मध्ये विनीयापीडयद्भृशम्}% ४५

\twolineshloka
{हनूमन्तं तथा वीरं सुग्रीवं चाङ्गदं तथा}
{लक्ष्मीनिधिं जनकजं प्रतापाग्र्यं रिपुञ्जयम्}% ४६

\twolineshloka
{सुबाहुं सुमदं वीरं विमलं नीलरत्नकम्}
{सत्यवन्तं वीरमणिं सुरथं रामसेवकम्}% ४७

\twolineshloka
{अन्यानपि महाभागान्रघुनाथः स्वयं तदा}
{परिरेभे दृढं स्निग्धान्पादयोः प्रणतान्नृपान्}% ४८

\twolineshloka
{सुमतिः श्रीरघुपतिं भक्तानुग्रहकारकम्}
{परिरभ्य दृढं प्रीतः सम्मुखे तिष्ठदुन्नतः}% ४९

\twolineshloka
{तदा रामो निजामात्यं वीक्ष्य सान्निध्यमागतम्}
{उवाच परमप्रीत्या मन्त्रिणं वदतां वरः}% ५०

\twolineshloka
{सुमते मन्त्रिणां श्रेष्ठ शंश मे वाग्मिनां वर}
{क एते भूमिपाः सर्वे कथमत्र समागताः}% ५१

\twolineshloka
{कुत्रकुत्र हयः प्राप्तः केनकेन नियन्त्रितः}
{कथं वै मोचितो भ्रात्रा महाबलसुशालिना}% ५२

\uvacha{शेष उवाच}

\twolineshloka
{इत्युक्तो मन्त्रिणां श्रेष्ठः सुमतिः प्राह राघवम्}
{प्रहसन्मेघगम्भीर नादेन च सुबुद्धिमान्}% ५३

\uvacha{सुमतिरुवाच}

\twolineshloka
{सर्वज्ञस्य पुरस्तेऽद्य मया कथमुदीर्यते}
{पृच्छसि त्वं लोकरीत्या सर्वं जानासि सर्वदृक्}% ५४

\twolineshloka
{तथापि तव निर्देशं शिरस्याधाय सर्वदा}
{ब्रवीमि तच्छृणुष्वाद्य सर्वराजशिरोमणे}% ५५

\twolineshloka
{त्वत्प्रसादादहो स्वामिन्सर्वत्र जगतीतले}
{परिबभ्राम ते वाहो भालपत्रसुशोभितः}% ५६

\twolineshloka
{न कश्चित्तं निजग्राह स्वनामबलदर्पितः}
{स्वं स्वं राज्यं समर्प्याथ प्रणेमुस्ते पदाम्बुजम्}% ५७

\twolineshloka
{को वा रावण दैत्येन्द्र निहन्तुर्वाजिसत्तमम्}
{गृह्णाति विजयाकाङ्क्षी जरामरणवर्जितः}% ५८

\twolineshloka
{अहिच्छत्रां गतस्तावत्तव वाजी मनोरमः}
{तद्राजा सुमदः श्रुत्वा हयं प्राप्तं तव प्रभो}% ५९

\twolineshloka
{सपुत्रः प्रबलः सर्वसैन्येन बलिना वृतः}
{सर्वं समर्पयामास राज्यं निहतकण्टकम्}% ६०

\twolineshloka
{यो राजा जगतां नेत्रीं मातरं जगदम्बिकाम्}
{प्रसाद्य चिरमायुष्यं लेभे राज्यमकण्टकम्}% ६१

\twolineshloka
{स एष त्वां प्रणमति सुमदः प्रभुसेवितम्}
{तं गृहाण कृपादृष्ट्या चिराद्दर्शनकाङ्क्षकम्}% ६२

\twolineshloka
{ततः सुबाहुभूपस्य नगरे बलपूरिते}
{दमनस्तस्य वै पुत्रः प्रजग्राह हयोत्तमम्}% ६३

\twolineshloka
{तेन साकं महद्युद्धं बभूव दमनेन च}
{पुष्कलो जयमापेदे सम्मूर्छ्य सुभुजात्मजम्}% ६४

\twolineshloka
{ततः सुबाहुः सङ्क्रुद्धो रणे पवनजं बलात्}
{युयुधे तव पादाब्जसेवकं बलिनां वरम्}% ६५

\twolineshloka
{तस्य पादाहतो ज्ञानं प्राप्य शापतिरस्कृतम्}
{तुभ्यं समर्प्य सकलं वाजिनः पालकोऽभवत्}% ६६

\twolineshloka
{एष त्वां सुभुजो राजा प्रणमत्युन्नताङ्गकः}
{कृपादृष्ट्याभिषिञ्च त्वं सुबाहुं नयकोविदम्}% ६७

\twolineshloka
{ततो मुक्तो हयो रेवाह्रदे स निममज्ज ह}
{तत्र प्राप्तं मोहनास्त्रं शत्रुघ्नेन बलीयसा}% ६८

\twolineshloka
{ततो देवपुरे प्रागाच्छिववासविभूषिते}
{तत्रत्यं तु विजानासि यतस्त्वं तत्र चागतः}% ६९

\twolineshloka
{विद्युन्माली हतो दैत्यः सत्यवान्सङ्गतस्ततः}
{सुरथेन समं युद्धं जानासि त्वं महामते}% ७०

\twolineshloka
{ततः कुण्डलकान्मुक्तो हयो बभ्राम सर्वतः}
{न कश्चित्तं निजग्राह स्ववीर्यबलदर्पितः}% ७१

\twolineshloka
{वाल्मीकेराश्रमे रम्ये हयः प्राप्तो मनोरमः}
{तत्र यत्कुतुकं जातं तच्छृणुष्व नरोत्तम}% ७२

\twolineshloka
{तत्रार्भस्तव सारूप्यं बिभ्रत्षोडशवार्षिकः}
{जग्राह वीक्ष्य पत्राङ्कं वाजिनं बलवत्तमः}% ७३

\twolineshloka
{तत्र कालजिता युद्धं महज्जातं नरोत्तम}
{निहतस्तेन वीरेण शितधारेण हेतिना}% ७४

\twolineshloka
{अनेके निहताः सङ्ख्ये पुष्कलाद्या महाबलाः}
{मूर्च्छितं चापि शत्रुघ्नं चक्रे वीरशिरोमणिः}% ७५

\twolineshloka
{तदा राजा महद्दुःखं विचार्य हृदिसंयुगे}
{कोपेन मूर्च्छितं चक्रे वीरो हि बलिनां वरः}% ७६

\twolineshloka
{स यावन्मूर्च्छितो राज्ञा तावदन्यः समागतः}
{तेनैतेन च सञ्जीव्य नाशितं कटकं तव}% ७७

\twolineshloka
{सर्वेषां मूर्च्छितानां तु शस्त्राण्याभरणानि च}
{गृहीत्वा वानरौ बद्धौ जग्मतुः स्वाश्रमं प्रति}% ७८

\twolineshloka
{कृपां कृत्वा पुनस्तेन दत्तोऽश्वो यज्ञियो महान्}
{जीवनं प्रापितं सर्वं कटकं नष्टजीवितम्}% ७९

\twolineshloka
{वयं गृहीत्वा तं वाहं प्राप्तास्तव समीपतः}
{एतदेव मया ज्ञातं तदुक्तं ते पुरोवचः}% ८०

{॥इति श्रीपद्मपुराणे पातालखण्डे शेषवात्स्यायनसंवादे रामाश्वमेधे सुमतिनिवेदनं नाम पञ्चषष्टितमोऽध्यायः॥६५॥}

\dnsub{षट्षष्टितमोऽध्यायः}\resetShloka

\uvacha{शेष उवाच}

\twolineshloka
{कथितौ वै सुमतिना वाल्मीकेराश्रमे शिशू}
{पुत्रौ स्वीयाविति ज्ञात्वा वाल्मीकिं प्रति सञ्जगौ}% १

\uvacha{श्रीराम उवाच}

\twolineshloka
{कौ शिशू मम सारूप्यधारकौ बलिनां वरौ}
{किमर्थं तिष्ठतस्तत्र धनुर्विद्याविशारदौ}% २

\twolineshloka
{अमात्यकथितौ श्रुत्वा विस्मयो मम जायते}
{यौ शत्रुघ्नं हनूमन्तं लीलयाङ्ग बबन्धतुः}% ३

\twolineshloka
{तस्माच्छंस मुने सर्वं बालयोश्च विचेष्टितम्}
{यथा मे परमा प्रीतिर्भवत्येवमभीप्सिता}% ४

\twolineshloka
{इति तत्कथितं श्रुत्वा राजराजस्य धीमतः}
{उवाच परमं वाक्यं स्पष्टाक्षरसमन्वितम्}% ५

\uvacha{वाल्मीकिरुवाच}

\twolineshloka
{तवान्तर्यामिणो नॄणां कथं ज्ञानं च नो भवेत्}
{तथापि कथयाम्यत्र तव सन्तोषहेतवे}% ६

\twolineshloka
{राजन्यौ बालकौ मह्यमाश्रमे बलिनां वरौ}
{त्वत्सारूप्यधरौ स्वाङ्गमनोहरवपुर्धरौ}% ७

\twolineshloka
{त्वया यदा वने त्यक्ता जानकी वै निरागसी}
{अन्तर्वत्नी वने घोरे विलपन्ती मुहुर्मुहुः}% ८

\twolineshloka
{कुररीमिव दुःखार्तां वीक्ष्याहं तव वल्लभाम्}
{जनकस्य सुतां पुण्यामाश्रमे त्वानयं तदा}% ९

\twolineshloka
{तस्याः पर्णकुटीरम्या रचिता मुनिपुत्रकैः}
{तस्यामसूत पुत्रौ द्वौ भासयन्तौ दिशो दश}% १०

\twolineshloka
{तयोरकरवं नाम कुशो लव इति स्फुटम्}
{ववृधातेऽनिशं तत्र शुक्लपक्षे यथा शशी}% ११

\twolineshloka
{कालेनोपनयाद्यानि सर्वाणि कृतवानहम्}
{वेदान्साङ्गानहं सर्वान्ग्राहयामास भूपते}% १२

\twolineshloka
{सर्वाणि सरहस्यानि शृणुष्व मुखतो मम}
{आयुर्वेदं धनुर्विद्यां शस्त्रविद्यां तथैव च}% १३

\twolineshloka
{विद्यां जालन्धरीं चाथ सङ्गीतकुशलौ कृतौ}
{गङ्गाकूले गायमानौ लताकुञ्जवनेषु च}% १४

\twolineshloka
{चञ्चलौ चलचित्तौ तौ सर्वविद्याविशारदौ}
{तदाहमतिसन्तोषं प्राप्तश्चाहं रघूत्तम}% १५


\threelineshloka
{दत्त्वा सर्वाणि चास्त्राणि मस्तके निहितः करः}
{अतीवगानकुशलौ दृष्ट्वा लोका विसिष्मिरे}
{षड्जमध्यमगान्धारस्वरभेदविशारदौ}% १६

\twolineshloka
{तथाविधौ विलोक्याहं गापयामि मनोहरम्}
{भविष्यज्ञानयोगाच्च कृतं रामायणं शुभम्}% १७

\twolineshloka
{मृदङ्गपणवाद्यादि यन्त्रवीणाविशारदौ}
{वनेवने च गायन्तौ मृगपक्षिविमोहकौ}% १८

\twolineshloka
{अद्भुतं गीतमाधुर्यं तव रामकुमारयोः}
{श्रोतुं तौ वरुणो बाला वा निनाय विभावरीम्}% १९

\twolineshloka
{मनोहरवयोरूपौ गानविद्याब्धिपारगौ}
{कुमारौ जगदुस्तत्र लोकेशादेशतः कलम्}% २०

\twolineshloka
{परमं मधुरं रम्यं पवित्रं चरितं तव}
{शुश्राव वरुणः सार्द्धं कुटुम्बेन च गायकैः}% २१

\twolineshloka
{शृण्वन्नैव गतस्तृप्तिं मित्रेण वरुणः सह}
{सुधातोऽपि परं स्वादुचरितं रघुनन्दन}% २२

\twolineshloka
{गानानन्दमहालोभ हृतप्राणेन्द्रियक्रियः}
{प्रत्यागन्तुं दिदेशासौ कुमारौ न हि तावकौ}% २३

\twolineshloka
{रमणीय महाभोगैर्लोभितावपि बालकौ}
{चलितौ न गुरोश्चात्ममातुः पादाम्बुजस्मृतेः}% २४

\twolineshloka
{अहं चापि गतः पश्चाद्वरुणालयमुत्तमम्}
{वरुणः प्रेमसहितः पूजां चक्रे मम प्रभो}% २५

\twolineshloka
{पृच्छते जन्मकर्मादि सर्वज्ञायापि बालयोः}
{वरुणायाब्रुवं सर्वं जन्मविद्याद्युपागमम्}% २६

\twolineshloka
{श्रुत्वा सीतासुतौ देवः स चक्रेम्बरभूषणैः}
{देवदत्तमिति ग्राह्यमिति मद्वाक्यगौरवात्}% २७


\threelineshloka
{आहृतं राजपुत्राभ्यां यद्दत्तं वरुणेन तत्}
{प्रसन्नेन तयोर्वाद्यगानविद्यावयोगुणैः}
{ततो मामब्रवीत्सीतामुद्दिश्य वरुणः कृती}% २८

\twolineshloka
{सीतापति व्रताधुर्या रूपशीलवयोन्विता}
{वीरपुत्रा महाभागा त्यागं नार्हति कर्हिचित्}% २९

\twolineshloka
{महती हानिरेतस्यास्त्यागे हि रघुनन्दन}
{सिद्धीनां परमासिद्धिरेषा ते ह्यनपायिनी}% ३०

\twolineshloka
{पामरैर्महिमानास्या ज्ञायते यदि दूषितैः}
{का हानिस्तावता राम पुण्यश्रवणकीर्तन}% ३१

\twolineshloka
{अस्मत्साक्षिकमेतस्याः पावनं चरितं सदा}
{सद्यस्ते सिद्धिमायान्ति ये सीतापदचिन्तकाः}% ३२

\twolineshloka
{यस्याः सङ्कल्पमात्रेण जन्मस्थितिलयादिकाः}
{भवन्ति जगतां नित्यं व्यापारा ऐश्वरा अमी}% ३३

\twolineshloka
{सीता मृत्युःसुधा चेयं तपत्येषा च वर्षति}
{स्वर्गो मोक्षस्तपो योगो दानं च तव जानकी}% ३४

\twolineshloka
{ब्रह्माणं शिवमन्यांश्च लोकपालान्मदादिकान्}
{करोत्येषा करोत्येव नान्या सीता तव प्रिया}% ३५

\twolineshloka
{त्वं पिता सर्वलोकानां सीता च जननीत्यतः}
{कुदृष्टिरत्र तु क्षेमयोग्या न तव कर्हिचित्}% ३६

\twolineshloka
{वेत्ति सीतां सदा शुद्धां सर्वज्ञो भगवान्स्वयम्}
{भवानपि सुतां भूमेः प्राणादपि गरीयसीम्}% ३७

\twolineshloka
{आदर्तव्या त्वया तस्मात्प्रिया शुद्धेति जानकी}
{न च शापपराभूतिः सीतायां त्वयि वा विभो}% ३८

\twolineshloka
{इमानि मम वाक्यानि वाच्यानि जगतां पतिम्}
{रामं प्रति त्वया साक्षाद्वाल्मीके मुनिसत्तम}% ३९

\twolineshloka
{इत्युक्तो वरुणेनाहं सीतासङ्ग्रहकारणात्}
{एवमेव हि सर्वैश्च लोकपालैरपि प्रभो}% ४०

\twolineshloka
{श्रुतं रामायणोद्गानं पुत्राभ्यां ते सुरासुरैः}
{गन्धर्वैरपि सर्वैश्च कौतुकाविष्टमानसैः}% ४१

\twolineshloka
{प्रसन्ना एव सर्वेऽपि प्रशशंसुः सुतौ च ते}
{त्रैलोक्यं मोहितं ताभ्यां रूपगानवयोगुणैः}% ४२

\twolineshloka
{दत्तं यल्लोकपालैस्तु सुताभ्यां स्वीकृतं हि तत्}
{ऋषिभिश्च वरा आभ्यामन्येभ्यः कीर्तिरेव च}% ४३

\twolineshloka
{एकरामं जगत्सर्वं पूर्वं मुनिविलोकितम्}
{त्रिराममधुना जान्तं सुताभ्यां तेखिलेक्षितम्}% ४४

\twolineshloka
{एककामपरामूर्तिर्लोके पूर्वमवेक्षिता}
{कामैश्चतुर्भिरद्यायं जायते च यतस्ततः}% ४५

\twolineshloka
{सर्वत्रान्यत्र राजेन्द्र रामपुत्रौ कुशीलवौ}
{गीयते अत्र सङ्कोचः किं कृतो विदुषि त्वयि}% ४६

\twolineshloka
{कृतेषु तव सर्वेषु श्रूयते महती स्तुतिः}
{त्यागादन्यत्र सीतायाः पुण्यश्लोकशिरोमणे}% ४७

\twolineshloka
{त्वया त्रैलोक्यनाथेन गार्हस्थ्यमनुकुर्वता}
{अङ्गीकार्यौ सुतौ रामविद्याशीलगुणान्वितौ}% ४८

\twolineshloka
{न तौ स्वां मातरं हित्वा स्थास्यतोऽभवदन्तिके}
{जनन्या सहितौ तस्मादाकार्यौ भवता सुतौ}% ४९

\twolineshloka
{दत्त एव तयेदानीं सेनासञ्जीवनात्पुनः}
{प्रत्ययः सर्वलोकानां पावनः पततामपि}% ५०

\twolineshloka
{नाज्ञातं तेन चास्माकं नामराणां च मानद}
{शुद्धौ तस्यास्तु लोकानां यन्नष्टं तदिह ध्रुवम्}% ५१

\uvacha{शेष उवाच}

\twolineshloka
{इति वाल्मीकिना रामः सर्वज्ञोऽप्यवबोधितः}
{स्तुत्वा नत्वा च वाल्मीकिं प्रत्युवाच स लक्ष्मणम्}% ५२

\twolineshloka
{गच्छ ताताधुना सीतामानेतुं धर्मचारिणीम्}
{सपुत्रां रथमास्थाय सुमन्त्रसहितः सखे}% ५३

\twolineshloka
{श्रावयित्वा ममेमानि मुनेश्च वचनान्यपि}
{सम्बोध्य च पुरीमेतां सीतां प्रत्यानयाशु ताम्}% ५४

\uvacha{लक्ष्मण उवाच}

\twolineshloka
{यास्यामि तव सन्देशात्सर्वेषां नः प्रभोर्विभो}
{देव्या यास्यति चेद्देव यात्रा मे सफला ततः}% ५५

\twolineshloka
{मयि सामाभ्यसूयैव पूर्वदोषवशात्सती}
{अनागतायां तस्यां तु क्षमस्वागन्तुकं मम}% ५६

\twolineshloka
{इत्युक्त्वा लक्ष्मणो रामं रथे स्थित्वा नृपाज्ञया}
{सुमित्रमुनिशिष्याभ्यां युतोऽगाद्भूमिजाश्रमम्}% ५७

\twolineshloka
{कथं प्रसादनीया स्यात्सीता भगवती मया}
{पूर्वदोषं विजानन्ती रामाधीनस्य मे सदा}% ५८

\twolineshloka
{एवं सञ्चिन्तयन्नन्तर्हर्षसङ्कोच मध्यगः}
{लक्ष्मणः प्राप सीताया आश्रमं श्रमनाशनम्}% ५९

\twolineshloka
{रथात्सोथावरुह्यारादश्रुरुद्धविलोचनः}
{आर्ये पूज्ये भगवति शुभे इति वदन्मुहुः}% ६०

\twolineshloka
{पपात पादयोस्तस्या वेपमानाखिलाङ्गकः}
{उत्थापितस्तया देव्या प्रीतिविह्वलया स च}% ६१

\twolineshloka
{किमर्थमागतः सौम्य वनं मुनिजनप्रियम्}
{आस्ते स कुशली देवः कौसल्याशुक्तिमौक्तिकः}% ६२

\twolineshloka
{अरोषो मयि कश्चित्स कीर्त्या केवलयादृतः}
{कीर्त्यते सर्वलोकैश्च कल्याणगुणसागरः}% ६३

\twolineshloka
{अकीर्तिभीतिमापन्नस्त्यक्तुं मां त्वां नियुक्तवान्}
{यदि ततश्च लोकेषु कीर्तिस्तस्यामलाभवत्}% ६४

\twolineshloka
{मृत्वापि पतिसत्कीर्तिं कुर्वन्त्या मे हि सुस्थिरा}
{पतिसामीप्यमेवाशु भूयादेव हि देवर}% ६५

\twolineshloka
{त्यक्तयापि मया तेन नासौ त्यक्तो मनागपि}
{फलं हि साधनायत्तं हेतुः फलवशो न तु}% ६६

\twolineshloka
{कौसल्याशल्यशून्यासौ कृपापूर्णा सदा मयि}
{आस्ते कुशलिनी यस्याः पुत्रस्त्रैलोक्यपालकः}% ६७

\twolineshloka
{सर्वे कुशलिनः सन्ति भरताद्याश्च बान्धवाः}
{सुमित्रा च महाभागा यस्याः प्राणादहं प्रिया}% ६८

\twolineshloka
{मद्वत्किं त्वमपि त्यक्तः सर्वलोकेषु कीर्तये}
{राज्ञः किं दुस्त्यजं तस्य स्वात्मापि यस्य न प्रियः}% ६९

\twolineshloka
{इत्येवं बहुधा पृष्टस्तया रामानुजः सताम्}
{उवाच कुशली देवः कुशलं त्वयि पृच्छति}% ७०

\twolineshloka
{कौसल्या च सुमित्रा च याश्चान्या राजयोषितः}
{पप्रच्छुः कुशलं देवि प्रीत्या त्वामाशिषा सह}% ७१

\twolineshloka
{कुशलप्रश्नपूर्वं हि तव पादाभिवन्दनम्}
{निवेदयामि शत्रुघ्न भरताभ्यां कृतं शुभे}% ७२

\twolineshloka
{गुरुभिर्गुरुपत्नीभिः सर्वाभिरपि ते शुभे}
{दत्ताशीः कुशलप्रश्नः कृतश्च त्वयि जानकि}% ७३

\twolineshloka
{आकारयति देवस्त्वां निर्व्यलीकेन चात्मवान्}
{अलभ्यान्यरतिस्त्वत्तोऽन्यत्र सर्वत्र भामिनि}% ७४

\twolineshloka
{शून्या एव दिशः सर्वास्त्वां विना जनकात्मजे}
{पश्यन्रोदिति नाथो नो रोदयन्नितरानपि}% ७५

\twolineshloka
{यत्र देवि स्थितासि त्वं नित्यं स्मरति राघवः}
{अशून्यं तु तमेवासौ मन्यमानो विदेहजे}% ७६

\twolineshloka
{धन्योऽयमाश्रमो जातो वाल्मीकेर्यत्र जानकी}
{कालं क्षपति वार्ताभिर्मदीयाभिर्वदन्निति}% ७७

\twolineshloka
{उक्तवान्यद्रुदन्किञ्चित्स्वामी नस्त्वयि तच्छृणु}
{व्यक्तीभवति वक्तुर्यद्धृद्गतं तदसंशयम्}% ७८

\twolineshloka
{लोका वदन्ति मामेव सर्वेषामीश्वरेश्वरम्}
{अहं त्वदृष्टमेवैषां स्वतन्त्रं कारणं ब्रुवे}% ७९

\twolineshloka
{अदृष्टमेव कार्येषु सर्वेशोऽप्यनुगच्छति}
{ईशनीयाः कुतो नैतदन्वीयुः सुखदुःखयोः}% ८०

\twolineshloka
{धनुर्भङ्गे मतिभ्रंशे कैकय्या मरणे पितुः}
{अरण्यगमने तत्र हरणे तव वारिधेः}% ८१

\twolineshloka
{तरणे रक्षसां भर्तुर्मारणेऽपि रणेरणे}
{सहायीभवने मह्यमृक्षवानररक्षसाम्}% ८२

\twolineshloka
{लाभे तव प्रतिज्ञायाः सत्यत्वे च सतीमणे}
{पुनः स्वबन्धुसम्बन्धे राज्यप्राप्तौ च भामिनि}% ८३

\twolineshloka
{पुनः प्रियावियोगे च कारणं यदवारणम्}
{प्रसीदति तदेवाद्य संयोगे पुनरावयोः}% ८४

\twolineshloka
{वेदोऽन्यथा कृतो येन लोकोत्पत्ति लयौ यतः}
{लोकाननुगतस्तस्मात्कारणं प्रथमं त्वहम्}% ८५

\twolineshloka
{अदृष्टमनुवर्तन्ते लोकाः सम्प्रतिबोधकाः}
{भोगेन जीर्यतेऽदृष्टं तत्तद्भुक्तं त्वया वने}% ८६

\twolineshloka
{स्नेहोऽकारणकः सीते वर्धमानो मम त्वयि}
{लोकादृष्टे तिरस्कृत्य त्वामाह्वयत आदरात्}% ८७

\twolineshloka
{शङ्कितेनापि दोषेण स्नेहनैर्मल्यमज्जनम्}
{भवतीति स वै शुद्ध आस्वाद्यो विबुधैः सदा}% ८८

\twolineshloka
{स्नेहशुद्धिरियं भद्रे कृता मे त्वयि नान्यथा}
{मन्तव्यं रक्षितोऽप्येष लोकः शिष्टानुवर्तिना}% ८९

\twolineshloka
{आवयोर्निन्दया देवि सर्वावस्था सुशुद्धये}
{लोको नश्येद्धि सम्मूढश्चरितैर्महतामयम्}% ९०

\twolineshloka
{आवयोरुज्ज्वला कीर्तिरावयोरुज्ज्वलो रसः}
{आवयोरुज्ज्वलौ वंशावावयोरुज्ज्वलाः क्रियाः}% ९१

\twolineshloka
{भवेयुरावयोः कीर्तिर्गायका उज्ज्वला भुवि}
{आवयोर्भक्तिमन्तो ये ते यान्त्यन्ते भवाम्बुधेः}% ९२

\twolineshloka
{इत्युक्ता भवती तेन प्रीयमाणेन ते गुणैः}
{पत्युः पादाम्बुजे द्रष्टुं करोतु सदयं मनः}% ९३

\twolineshloka
{वासांसि रमणीयानि भूषणानि महान्ति च}
{अङ्गरागस्तथा गन्धा मनोज्ञास्त्वयि योजिताः}% ९४

\twolineshloka
{रथो दास्यश्च रामेण प्रेषिता उत्सवायते}
{छत्रं च चामरे शुभ्रे गजा अश्वाश्च शोभने}% ९५

\twolineshloka
{स्तूयमाना द्विजश्रेष्ठैः सूतमागधबन्दिभिः}
{वन्द्यमाना पुरस्त्रीभिः सेव्यमाना च योद्धृभिः}% ९६

\twolineshloka
{पुष्पैः सञ्छाद्यमाना च देवीदेवाङ्गनादिभिः}
{धनानि ददती तेभ्यो द्विजातिभ्यो यथेप्सितम्}% ९७

\twolineshloka
{गजारूढौ कुमारौ च पुरस्कृत्य जनेश्वरी}
{मयानुगम्यमाना च गच्छायोध्यां निजां पुरीम्}% ९८

\twolineshloka
{त्वयि तत्र गतायां तु सङ्गतायां प्रियेण ते}
{सर्वासां राजनारीणामागतानां च सर्वशः}% ९९

\twolineshloka
{सर्वासामृषिपत्नीनां कौसलानां तथैव च}
{मङ्गलैर्वाद्यगीताद्यैर्भवत्वद्य महोत्सवः}% १००

\uvacha{शेष उवाच}

\twolineshloka
{इतिविज्ञापनां देवी श्रुत्वा सीता तमाह सा}
{नाहं कीर्तिकरी राज्ञो ह्यपकीर्तिः स्वयं त्वहम्}% १०१

\twolineshloka
{किं मया तस्य साध्यं स्याद्धर्मकामार्थशून्यया}
{सत्येवं भवतां भूपे को विश्वासो निरङ्कुशे}% १०२

\twolineshloka
{प्रत्यक्षा वा परोक्षा वा भर्तुर्दोषा मनःस्थिताः}
{न वाच्या जातु मादृश्या कल्याणकुलजातया}% १०३

\twolineshloka
{पाणिग्रहणकाले मे यद्रूपो हृदये स्थितः}
{तद्रूपो हृदयान्नासौ कदाचिदपसर्पति}% १०४

\twolineshloka
{लक्ष्मणेमौ कुमारौ मे तत्तेजोंशसमुद्भवौ}
{वंशाङ्कुरौ महाशूरौ धनुर्विद्याविशारदौ}% १०५

\twolineshloka
{नीत्वा पितुः समीपं तु लालनीयौ प्रयत्नतः}
{तपसाराधयिष्यामि रामं काममिह स्थिता}% १०६

\twolineshloka
{वाच्यं त्वया महाभाग पूज्यपादाभिवन्दनम्}
{सर्वेभ्यः कुशलं चापि गत्वेतो मदपेक्षया}% १०७

\twolineshloka
{पुत्रौ समादिशत्सीता गच्छतं पितुरन्तिकम्}
{शुश्रूषणीय एवासौ भवद्भ्यां स्वपदप्रदः}% १०८

\twolineshloka
{आज्ञप्तावप्यनिच्छन्तौ तौ कुमारौ कुशीलवौ}
{वाल्मीकिवचनात्तत्र जग्मतुश्च सलक्ष्मणौ}% १०९

\twolineshloka
{वाल्मीकेरेव पादाब्जसमीपं तत्सुतौ गतौ}
{लक्ष्मणोऽपि ववन्दे तं गत्वा बालकसंयुतः}% ११०

\twolineshloka
{वाल्मीकिर्लक्ष्मणस्तौ तु कुमारौ मिलिता अमी}
{सभायां संस्थितं रामं ज्ञात्वा ते जग्मुरुत्सुकाः}% १११

\twolineshloka
{लक्ष्मणः प्रणिपत्याथ सीतावाक्यादिसर्वशः}
{कथयामास रामाय हर्षशोकयुतः सुधीः}% ११२

\twolineshloka
{सीतासन्देशवाक्येभ्यो रामो मूर्च्छां समन्वभूत्}
{संज्ञामवाप्य चोवाच लक्ष्मणं नयकोविदम्}% ११३

\twolineshloka
{गच्छ मित्र पुनस्तत्र यत्नेन महता च ताम्}
{शीघ्रमानय भद्रं ते मद्वाक्यानि निवेद्य च}% ११४

\twolineshloka
{अरण्ये किं तपस्यन्त्या गतिरन्या विचिन्तिता}
{श्रुता दृष्टाथ वा मत्तो यन्नागच्छसि जानकि}% ११५

\twolineshloka
{त्वदिच्छया त्वमेवेतो गतारण्यं मुनिप्रियम्}
{पूजिता मुनिपत्न्यस्ता दृष्टा मुनिगणास्त्वया}% ११६

\twolineshloka
{पूर्णो मनोरथस्तेऽद्य किं नागच्छसि भामिनि}
{न दोषं मयि पश्येस्त्वं स्वात्मेच्छाया विलोकनात्}% ११७

\twolineshloka
{गत्वा गत्वाथ वामोरु पतिरेव गतिः स्त्रियाः}
{निर्गुणोपि गुणाम्भोधिः किम्पुनर्मनसेप्सितः}% ११८

\twolineshloka
{याया क्रियाकुलस्त्रीणां सासा पत्युः प्रतुष्टये}
{पूर्वमेवप्रतुष्टोऽहमिदानीं सुतरां त्वयि}% ११९

\twolineshloka
{यागो जपस्तपोदानं व्रतं तीर्थं दयादिकम्}
{देवाश्च मयि सन्तुष्टे तुष्टमेतदसंशयम्}% १२०

\uvacha{शेष उवाच}

\twolineshloka
{इति सन्देशमादाय सीतां प्रति जगत्पतेः}
{आह लक्ष्मण आत्मेशमानतः प्रणयाद्धरौ}% १२१

\twolineshloka
{सीतानयनमुद्दिश्य प्रसन्नस्त्वं यदूचिवान्}
{कथयिष्यामि तद्वाक्यं विनयेन समन्वितम्}% १२२

\twolineshloka
{इत्युक्त्वा पादयोर्नत्वा रघुनाथस्य लक्ष्मणः}
{जगाम त्वरितः सीतां रथे तिष्ठन्महाजवे}% १२३

\twolineshloka
{वाल्मीकिः श्रीयुतौ वीक्ष्य रामपुत्रौ महौजसौ}
{उवाच स्मितमाधाय मुखं कृत्वा मनोहरम्}% १२४

\twolineshloka
{युवां प्रगायतां पुत्रौ रामचारित्रमद्भुतम्}
{वीणां प्रवादयन्तौ च कलगानेन शोभितम्}% १२५

\twolineshloka
{इत्यक्तौ तौ सुतौ रामचारित्रं बहुपुण्यदम्}
{अगायतां महाभागौ सुवाक्यपदचित्रितम्}% १२६

\twolineshloka
{यस्मिन्धर्मविधिः साक्षात्पातिव्रत्यं तु यत्स्थितम्}
{भ्रातृस्नेहो महान्यत्र गुरुभक्तिस्तथैव च}% १२७

\twolineshloka
{स्वामिसेवकयोर्यत्र नीतिर्मूर्तिमती किल}
{अधर्मकरशास्तिं वै यत्र साक्षाद्रघूद्वहात्}% १२८

\twolineshloka
{तद्गानेन जगद्व्याप्तं दिवि देवा अपि स्थिताः}
{किन्नरा अपि यद्गानं श्रुत्वा मूर्च्छामिताः क्षणात्}% १२९

\twolineshloka
{वीणायारणितं श्रुत्वा तालमानेन शोभितम्}
{निखिला परिषत्तत्र शालभं जीवचित्रिता}% १३०

\twolineshloka
{हर्षादश्रूणिमुञ्चन्तो रामाद्या भूमिपास्तथा}
{तद्गानपञ्चमालापमोहिताश्चित्रितोपमाः}% १३१

\twolineshloka
{तत्र रामः सुतौ दृष्ट्वा महागानविमोहकौ}
{अदात्ताभ्यां सुवर्णस्य लक्षं लक्षं पृथक्पृथक्}% १३२

\twolineshloka
{तदा दानपरं दृष्ट्वा वाल्मीकिं मुनिसत्तमम्}
{अब्रूतां प्रहसन्तौ तौ किञ्चिद्वक्रभ्रुवौ ततः}% १३३

\twolineshloka
{मुने महानयोनेन क्रियते भूमिपेन वै}
{यदावाभ्यां सुवर्णानि दातुमिच्छति लोभयन्}% १३४

\twolineshloka
{प्रतिग्रहो ब्राह्मणानां शस्यते नेतरेषु वै}
{प्रतिग्रहपरो राजा नरकायैव कल्पते}% १३५

\twolineshloka
{आवयोः कृपया मुक्तं राज्यं भुङ्क्ते महीपतिः}
{कथं दातुं सुवर्णानि वाञ्छति श्रेयसाञ्चितः}% १३६

\twolineshloka
{इत्युक्तवन्तौ तौ दृष्ट्वा वाल्मीकिः कृपयायुतः}
{अशंसद्युष्मत्पितरं जानीथां नीतिवित्तमौ}% १३७

\twolineshloka
{इति श्रुत्वा मुनेर्वाक्यं बालकौ नृपपादयोः}
{लग्नौ विनयसंयुक्तौ मातृभक्त्यातिनिर्मलौ}% १३८

\twolineshloka
{रामो बालौ दृढं स्वाङ्गे परिरभ्य मुदान्वितः}
{मेने स्त्रियास्तदा धर्मौ मूर्तिमन्तावुपस्थितौ}% १३९

\twolineshloka
{सभापि रामसुतयोर्वीक्ष्य वक्त्रे मनोरमे}
{जानकीपतिभक्तित्वं सत्यं मेने मुनीश्वर}% १४०

\twolineshloka
{इति शेषमुखप्रोक्तं श्रुत्वा वात्स्यायनोऽब्रवीत्}
{रामायणं श्रोतुमनाः सर्वधर्मसमन्वितम्}% १४१

\uvacha{वात्स्यायन उवाच}

\twolineshloka
{कस्मिन्काले कृतं स्वामिन्रामायणमिदं महत्}
{कस्माच्चकार किन्तत्र वर्णनं तद्वदस्व मे}% १४२

\uvacha{शेष उवाच}

\twolineshloka
{एकदा गतवान्विप्रो वाल्मीकिर्विपिनं महत्}
{यत्र तालास्तमालाश्च किंशुका यत्र पुष्पिताः}% १४३


\threelineshloka
{केतकी यत्र रजसा कुर्वती सौरभं वनम्}
{शशिप्रभेव महती दृश्यते शुभ्रकर्णभृत्}
{चम्पकोबकुलश्चापि कोविदारः कुरण्टकः}% १४४

\twolineshloka
{अनेके पुष्पिता यत्र पादपाः शोभने वने}
{कोकिलानां विरावेण षट्पदानां च शब्दितैः}% १४५

\twolineshloka
{सङ्घुष्टं सर्वतो रम्यं मनोहरवयोन्वितम्}
{तत्र क्रौञ्चयुगं रम्यं कामबाणप्रपीडितम्}% १४६

\twolineshloka
{परस्परं प्रहृषितं रेमे स्निग्धतया स्थितम्}
{तदा व्याधः समागत्य तयोरेकं मनोहरम्}% १४७

\twolineshloka
{अवधीन्निर्दयः कश्चिन्मांसास्वादनलोलुपः}
{तदा क्रौञ्ची व्याधहतं स्वपतिं वीक्ष्य दुःखिता}% १४८

\twolineshloka
{विललाप भृशं दुःखान्मुञ्चन्ती रावमुच्चकैः}
{तदा मुनिः प्रकुपितो निषादं क्रौञ्चघातकम्}% १४९

\twolineshloka
{शशाप वार्युपस्पृश्य सरितः पावनं शुभम्}
{मा निषाद प्रतिष्ठां त्वमगमः शाश्वतीः समाः}% १५०

\twolineshloka
{यत्क्रौञ्चपक्षिणोरेकमवधीः काममोहितम्}
{तदा प्रबन्धं श्लोकस्य जातं मत्वा ह्यनुद्विजाः}% १५१

\twolineshloka
{ऊचुर्मुनिं प्रहृष्टास्ते शंसन्तः साधुसाध्विति}
{स्वामिञ्छापोदिते वाक्ये भारतीश्लोकमातनोत्}% १५२


\threelineshloka
{अत्यन्तं मोहनो जातः श्लोकोऽयं मुनिसत्तम}
{तदा मुनिः प्रहृष्टात्मा बभूव वाडवर्षभ}
{तस्मिन्काले समागत्य ब्रह्मा पुत्रैः समन्वितः}% १५३

\twolineshloka
{वचो जगाद वाल्मीकिं धन्योसि त्वं मुनीश्वर}
{भारती त्वन्मुखे स्थित्वा श्लोकत्वं समपद्यत}% १५४

\twolineshloka
{तस्माद्रामायणं रम्यं कुरुष्व मधुराक्षरम्}
{येन ते विमला कीर्तिराकल्पान्तं भविष्यति}% १५५

\twolineshloka
{धन्या सैव मुखे वाणी रामनाम्ना समन्विता}
{अन्या कामकथा नॄणां जनयत्येव पातकम्}% १५६

\twolineshloka
{तस्मात्कुरुष्व रामस्य चरितं लोकविश्रुतम्}
{येन स्यात्पापिनां पापहानिरेव पदेपदे}% १५७

\twolineshloka
{इत्युक्त्वान्तर्दधे स्रष्टा सर्वदेवैः समन्वितः}
{ततः सचिन्तयामास कथं रामायणं भवेत्}% १५८

\twolineshloka
{तदा ध्यानपरो जातो नदीतीरे मनोरमे}
{तस्य चेतस्यथो रामः प्रादुर्भूतो मनोहरः}% १५९

\twolineshloka
{नीलोत्पलदलश्यामं रामं राजीवलोचनम्}
{निरीक्ष्य तस्य चरितं भूतम्भाविभवच्च यत्}% १६०

\twolineshloka
{तदात्यन्तं मुदं प्राप्तो रामायणमथासृजत्}
{मनोरमपदैर्युक्तं वृत्तैर्बहुविधैरपि}% १६१

\twolineshloka
{षट्काण्डानि सुरम्याणि यत्र रामायणेऽनघ}
{बालमारण्यकं चान्यत्किष्किन्धा सुन्दरं तथा}% १६२

\twolineshloka
{युद्धमुत्तरमन्यच्च षडेतानि महामते}
{शृणुयाद्यो नरः पुण्यात्सर्वपापैः प्रमुच्यते}% १६३

\twolineshloka
{तत्र बाले तु सन्तुष्टः पुत्रेष्ट्या चतुरस्सुतान्}
{प्राप पङ्क्तिरथः साक्षाद्धरिं ब्रह्मसनातनम्}% १६४

\twolineshloka
{स कौशिकमखं गत्वा सीतामुद्वाह्य भार्गवम्}
{आगत्य पुरमुत्कृष्टो यौवराज्यप्रकल्पनम्}% १६५

\twolineshloka
{मातृवाक्याद्वनं प्रागाद्गङ्गामुत्तीर्य पर्वतम्}
{चित्रकूटं महिलया लक्ष्मणेन समन्वितः}% १६६

\twolineshloka
{भरतस्तं वने श्रुत्वा जगाम भ्रातरं नयी}
{तमप्राप्य स्वयं नन्दिग्रामे वासमचीकरत्}% १६७

\twolineshloka
{बालमेतच्छृणुष्वान्यदारण्यकसमुद्भवम्}
{मुनीनामाश्रमे वासस्तत्र तत्रोपवर्णनम्}% १६८

\twolineshloka
{नासाच्छेदः शूर्पणख्याः खरदूषणनाशनम्}
{मायामारीचहननं दैत्याद्रामापहारणम्}% १६९

\twolineshloka
{वने विरहिणा भ्रान्तं मनुष्यचरितं भृतम्}
{कबन्धप्रेक्षणं तत्र पम्पायां गमनं तथा}% १७०

\twolineshloka
{हनूमता सङ्गमनमित्येतद्वनसंज्ञितम्}
{अपरं च शृणु मुने सङ्क्षिप्य कथयाम्यहम्}% १७१

\twolineshloka
{सप्ततालप्रभेदश्च वालेर्मारणमद्भुतम्}
{सुग्रीवे राज्यदानं च नगवर्णनमित्युत}% १७२

\twolineshloka
{लक्ष्मणात्कर्मसन्देशः सुग्रीवस्य विवासनम्}
{तथा सैन्यसमुद्देशः सीतान्वेषणमप्युत}% १७३

\twolineshloka
{सम्पातिप्रेक्षणं तत्र वारिधेर्लङ्घनं तथा}
{परतीरे कपिप्राप्तिः कैष्किन्धं काण्डमद्भुतम्}% १७४

\twolineshloka
{सुन्दरं शृणु काण्डं वै यत्र रामकथाद्भुता}
{प्रतिगेहे प्रति भ्रान्तिः कपेश्चित्रस्य दर्शनम्}% १७५

\twolineshloka
{सीतासन्दर्शनं तत्र जानक्याभाषणं तथा}
{वनभङ्गः प्रकुपितैर्बन्धनं वानरस्य च}% १७६

\twolineshloka
{ततो लङ्काप्रज्वलनं वानरैः सङ्गतिस्ततः}
{रामाभिज्ञानदानं च सैन्यप्रस्थानमेव च}% १७७

\twolineshloka
{समुद्रे सेतुकरणं शुकसारणसङ्गतिः}
{इति सुन्दरमाख्यातं युद्धे सीतासमागमः}% १७८

\twolineshloka
{उत्तरे ऋषिसंवादो यज्ञप्रारम्भ एव च}
{तत्रानेका रामकथाः शृण्वतां पापनाशकाः}% १७९

\twolineshloka
{इति षट्काण्डमाख्यातं ब्रह्महत्यापनोदनम्}
{सङ्क्षेपतो मया तुभ्यमाख्यातं सुमनोहरम्}% १८०

\twolineshloka
{चतुर्विंशतिसाहस्रं षट्काण्डपरिचिह्नितम्}
{तद्वै रामायणं प्रोक्तं महापातकनाशनम्}% १८१

\twolineshloka
{तच्छ्रुत्वा राघवः प्रीतः पुत्रावाधाय चासने}
{दृढं तौ परिरभ्याथ सीतां सस्मार वल्लभाम्}% १८२

{॥इति श्रीपद्मपुराणे पातालखण्डे शेषवात्स्यायनसंवादे रामाश्वमेधे रामायणगानं नाम षट्षष्टितमोऽध्यायः॥६६॥}

\dnsub{सप्तषष्टितमोऽध्यायः}\resetShloka

\uvacha{शेष उवाच}

\twolineshloka
{अथ सौमित्रिरागत्य जानकीं नतवान्मुहुः}
{प्रेमगद्गदया शंसन्वाचं रामप्रणोदिताम्}% १

\twolineshloka
{सीता समागतं दृष्ट्वा लक्ष्मणं विनयान्वितम्}
{तन्मुखाद्रामसन्देशं श्रुत्वोवाच विलज्जिता}% २

\twolineshloka
{सौमित्रे कथमागच्छे रामत्यक्ता महावने}
{तिष्ठामि रामं स्मरन्ती वाल्मीकेराश्रमे त्वहम्}% ३

\twolineshloka
{तस्या मुखोदितं वाक्यं श्रुत्वा सौमित्रिरब्रवीत्}
{मातः पतिव्रते रामस्त्वामाकारयते मुहुः}% ४

\twolineshloka
{पतिव्रता पतिकृतं दोषं नानयते हृदि}
{तस्मादागच्छ हि मया स्थित्वा स्यन्दन उत्तमे}% ५

\twolineshloka
{इत्यादि वचनं श्रुत्वा जानकी पतिदेवता}
{मनोरोषं परित्यज्य तस्थौ सौमित्रिणा रथे}% ६

\twolineshloka
{तापसीः सकला नत्वा मुनींश्च निगमोज्ज्वलान्}
{रामं स्मरन्ती मनसा रथे स्थित्वागमत्पुरीम्}% ७

\twolineshloka
{क्रमेण नगरीं प्राप्ता महार्हाभरणान्विता}
{सरयूं सरितं प्राप यत्र रामः स्वयं स्थितः}% ८

\twolineshloka
{रथादुत्तीर्य ललिता लक्ष्मणेन समन्विता}
{रामस्य पादयोर्लग्ना पतिव्रतपरायणा}% ९

\twolineshloka
{रामस्तामागतां दृष्ट्वा जानकीं प्रेमविह्वलाम्}
{साध्वि त्वया सहेदानीं कुर्वे यज्ञसमापनम्}% १०

\twolineshloka
{वाल्मीकिं सा नमस्कृत्य तथान्यान्विप्रसत्तमान्}
{जगाम मातृपदयोः सन्नतिं कर्तुमुत्सुका}% ११

\twolineshloka
{कौशल्या तामथायान्तीं वीरसूं जानकीं प्रियाम्}
{आशीर्भिरभिसंयुज्य ययौ हर्षमनेकधा}% १२

\twolineshloka
{कैकेयीपदयोर्नम्रां वीक्ष्य वैदेहपुत्रिकाम्}
{भर्त्रा सह चिरं जीव सपुत्रेत्याशिषं व्यधात्}% १३

\twolineshloka
{सुमित्रा स्वपदेनम्रां वीक्ष्य वैदेहपुत्रिकाम्}
{आशिषं व्यदधात्तस्याः पुत्रपौत्रप्रदायिनीम्}% १४

\twolineshloka
{सीता ताः सर्वतो नत्वा रामचन्द्र प्रिया सती}
{परमं हर्षमापन्ना बभूव किल वाडव}% १५

\twolineshloka
{समागतां वीक्ष्य पत्नीं रामचन्द्रस्य कुम्भजः}
{सुवर्णपत्नीं धिक्कृत्य तामधाद्धर्मचारिणीम्}% १६

\twolineshloka
{रामस्तदा यज्ञमध्ये शुशुभे सीतया सह}
{तारयानुगतो यद्वच्छशीव शरदुत्प्रभः}% १७

\twolineshloka
{प्रयोगमकरोत्तत्र काले प्राप्ते मनोरमे}
{वैदेह्या धर्मचारिण्या सर्वपापापनोदनम्}% १८

\twolineshloka
{सीतया सहितं रामं प्रसक्तं यज्ञकर्मणि}
{निरीक्ष्य जहृषुस्तत्र कौतुकेन समन्विताः}% १९

\twolineshloka
{वसिष्ठं प्राह सुमतिं रामस्तत्र क्रतौ वरे}
{किं कर्तव्यं मया स्वामिन्नतः परमवश्यकम्}% २०

\twolineshloka
{रामस्य वचनं श्रुत्वा गुरुः प्राह महामतिः}
{ब्राह्मणानां प्रकर्तव्या पूजा सन्तोषकारिका}% २१

\twolineshloka
{मरुत्तेन क्रतुः सृष्टः पूर्वं सम्भारसम्भृतः}
{ब्राह्मणास्तत्र वित्ताद्यैस्तोषिता अभवंस्तदा}% २२

\twolineshloka
{अत्यन्तं वित्तसम्भारं नेतुं विप्राशकन्नहि}
{प्राक्षिपन्हिमवद्देशे वित्तभारासहा द्विजाः}% २३

\twolineshloka
{तस्मात्त्वमपि राजाग्र्य लक्ष्मीवान्नृपसत्तम}
{देहि दानादि विप्रेभ्यो यथा स्यात्प्रीतिरुत्तमा}% २४

\twolineshloka
{एतच्छ्रुत्वा स राजाग्र्यः पूज्यं मत्वा घटोद्भवम्}
{प्रथमं पूजयामास ब्रह्मपुत्रं तपोधनम्}% २५

\twolineshloka
{अनेकरत्नसम्भारैः स्वर्णभारैरनेकधा}
{देशैर्जनैः परिवृतैरत्यन्तप्रीतिदायकैः}% २६

\twolineshloka
{अगस्त्यं पूजयामास सपत्नीकं मनोरमम्}
{तथैव रत्नैः स्वर्णैश्च देशैश्च विविधैरपि}% २७

\twolineshloka
{व्यासं सत्यवतीपुत्रं तथैव समपूजयत्}
{च्यवनं भार्यया साकं सुरत्नैः समपूजयत्}% २८

\twolineshloka
{अन्यानपि मुनीन्सर्वानृत्विजस्तपसां निधीन्}
{पूजयामास रत्नौघैः स्वर्णभारैरनेकधा}% २९

\twolineshloka
{अदात्तदा क्रतौ रामो विप्रेभ्यो भूरिदक्षिणाम्}
{लक्षंलक्षं सुवर्णस्य प्रत्येकं त्वग्रजन्मने}% ३०

\twolineshloka
{दीनान्धकृपणेभ्यश्च ददौ दानमनेकधा}
{यथासन्तोषविहितैर्वित्तै रत्नैर्मनोहरैः}% ३१

\twolineshloka
{वासांसि च विचित्राणि भोजनानि मृदूनि च}
{तत्र प्रादाद्यथाशास्त्रं सर्वेषां प्रीतिदायकम्}% ३२

\twolineshloka
{हृष्टपुष्टजनाकीर्णं सर्वसत्त्वोपबृंहितम्}
{अत्यन्तमभवद्धृष्टं पुरं पुंस्त्रीसमावृतम्}% ३३

\twolineshloka
{दानं ददन्तं सर्वेषां वीक्ष्य कुम्भोद्भवो मुनिः}
{अत्यन्तपरमप्रीतिं ययौ क्रतुवरे द्विजः}% ३४

\twolineshloka
{तदाभिषेकस्नानार्थं पानीयममृतोपमम्}
{आनेतुं च चतुःषष्टि नृपान्सस्त्रीन्समाह्वयत्}% ३५

\twolineshloka
{रामस्तु सीतया सार्द्धमानेतुमुदकं ययौ}
{घटेन स्वर्णवर्णेन सर्वालङ्कारशोभया}% ३६

\twolineshloka
{सौमित्रिरप्यूर्मिलया माण्डव्या भरतो नृपः}
{शत्रुघ्नः श्रुतकीर्त्या च कान्तिमत्या च पुष्कलः}% ३७

\twolineshloka
{सुबाहुः सत्यवत्या च सत्यवान्वीरभूषया}
{सुमदस्तत्र सत्कीर्त्या राज्ञ्या च विमलो नृपः}% ३८

\twolineshloka
{राजावीरमणिस्तत्र श्रुतवत्या मनोज्ञया}
{लक्ष्मीनिधिः कोमलया रिपुतापोङ्गसेनया}% ३९

\twolineshloka
{विभीषणो महामूर्त्या प्रतापाग्र्यः प्रतीतया}
{उग्राश्वः कामगमया नीलरत्नोधिरम्यया}% ४०

\twolineshloka
{सुरथः सुमनोहार्या तथा मोहनया कपिः}
{इत्यादीन्नृपतीन्विप्रो वसिष्ठः प्राहिणोन्मुनिः}% ४१

\twolineshloka
{वसिष्ठः सरयूं गत्वा शिवपुण्यजलाप्लुताम्}
{उदकं मन्त्रयामास वेदमन्त्रेण मन्त्रवित्}% ४२

\twolineshloka
{पयः पुनीह्यमुं वाहमुदकेन मनोहृता}
{यज्ञार्थं रामचन्द्रस्य सर्वलोकैकरक्षितुः}% ४३

\twolineshloka
{उदकं तन्मुनिस्पृष्टं सर्वे रामादयो नृपाः}
{आजह्रुर्मण्डपतले विप्रवर्यैरुपस्तुते}% ४४

\twolineshloka
{पयोभिर्निर्मलैः स्नाप्य वाजिनं क्षीरसन्निभम्}
{मन्त्रेण मन्त्रयामास राम हस्तेन कुम्भजः}% ४५

\twolineshloka
{पुनीहि मां महावाह अस्मिन्ब्रह्मसमाकुले}
{त्वन्मेधेनाखिला देवाः प्रीणन्तु परितोषिताः}% ४६

\twolineshloka
{इत्युक्त्वा स नृपो रामः सीतया सममस्पृशत्}
{तदा सर्वे द्विजाश्चित्रममन्यन्त कुतूहलात्}% ४७

\twolineshloka
{परस्परमवोचंस्ते यन्नामस्मरणान्नराः}
{महापापात्प्रमुच्यन्ते स रामः किं वदत्यहो}% ४८

\twolineshloka
{इत्युक्तवति भूमीशे रामे कुम्भोद्भवो मुनिः}
{करवालं चाभिमन्त्र्य ददौ रामकरे मुनिः}% ४९

\twolineshloka
{करवाले धृते स्पृष्टे रामेण स हयः क्रतौ}
{पशुत्वं तु विहायाशु दिव्यरूपमपद्यत}% ५०

\twolineshloka
{विमानवरमारूढश्चाप्सरोभिः समन्ततः}
{चामरैर्वीज्यमानश्च वैजयन्त्या विभूषितः}% ५१

\twolineshloka
{तदा तं वाजितां त्यक्त्वा दिव्यरूपधरं वरम्}
{वीक्ष्य लोकाः क्रतौ सर्वे विस्मयं प्राप्नुवंस्तदा}% ५२

\twolineshloka
{तदा रामः स्वयं जानंज्ञापयन्सर्वतो नरान्}
{पप्रच्छ दिव्यरूपं तं सुरं परमधार्मिकः}% ५३

\twolineshloka
{कस्त्वं दिव्यवपुः प्राप्तः कस्मात्त्वं वाजितां गतः}
{कथं सुरस्त्रीसहितः किं चिकीर्षसि तद्वद}% ५४

\twolineshloka
{रामस्य वचनं श्रुत्वा देवः प्रोवाच भूमिपम्}
{हसन्मेघरवां वाणीमवदत्सुमनोहराम्}% ५५

\twolineshloka
{तवाज्ञातं न सर्वत्र बाह्याभ्यन्तरचारिणः}
{तथापि पृच्छते तुभ्यं कथयामि यथातथम्}% ५६

\twolineshloka
{अहं पुराभवे राम द्विजः परमधार्मिकः}
{अचरं प्रतिकूलं वै वेदस्य रिपुतापन}% ५७

\twolineshloka
{कदाचिद्धुतपापायास्तीरेऽहं गतवान्पुरा}
{अनेकवृक्षललिते सर्वत्रसुमनोरमे}% ५८

\twolineshloka
{तत्र स्नात्वा पितॄंस्तृप्त्वा दानं दत्त्वा यथाविधि}
{ध्यानं तव महाबाहो कृतवान्वेदसम्मितम्}% ५९

\twolineshloka
{तदा जनाः समायाता बहवस्तत्र भूपते}
{तेषां प्रवञ्चनार्थाय दम्भमेनमकारिषम्}% ६०

\twolineshloka
{अनेकक्रतुसम्भारैः पूर्णमजिरमुत्तमम्}
{वासोभिश्छादितं रम्यं चषालादियुतं महत्}% ६१

\twolineshloka
{अग्निहोत्रोद्भवोधूमः सर्वतो नभसोङ्गणम्}
{चकार रम्यमतुलं चित्रकारिवपुर्धरः}% ६२

\twolineshloka
{अनेकतिलकश्रीभिः शोभिताङ्गो महत्तपाः}
{दर्भशोभः समित्पाणिर्दम्भो मूर्तिधरः किमु}% ६३

\twolineshloka
{दुर्वासास्तत्र स्वच्छन्दं पर्यटञ्जगतीतलम्}
{प्राप तत्र महातेजा धूतपापसरित्तटे}% ६४

\twolineshloka
{ददर्श मां दम्भकरं मौनधारिणमग्रतः}
{अनर्घ्यकरमुन्मत्तमस्वागतवचः करम्}% ६५

\twolineshloka
{दृष्ट्वातीव क्रुधाक्रान्तः समुद्र इव पर्वणि}
{शशापासौ मुनिस्तीव्रो दम्भिनं मां महामतिः}% ६६

\twolineshloka
{दम्भं करोषि चेत्तीरे सरितस्त्वं सुदुर्मते}
{तस्मात्प्राप्नुहि निर्वाच्यं पशुत्वं तापसाधम}% ६७

\twolineshloka
{शापं प्रदत्तं संश्रुत्य दुःखितोऽहं तदाभवम्}
{अग्राहिषं पदे तस्य मुनेर्दुर्वाससः किल}% ६८

\twolineshloka
{तदा मे कृतवान्राम द्विजोऽनुग्रहमुत्तमम्}
{वाजितां प्राप्नुहि मखे राजराजस्य तापस}% ६९

\twolineshloka
{पश्चात्तद्धस्तसम्पर्काद्याहि तत्परमं पदम्}
{दिव्यं वपुर्मनोहारि धृत्वा दम्भविवर्जितम्}% ७०

\twolineshloka
{तेन शापोपिसन्दिष्टो ममानुग्रहतां गतः}
{यदहं तव हस्तस्य स्पर्शं प्राप्तो मनोरमम्}% ७१

\twolineshloka
{यदेव राम देवादिदुर्लभं बहुजन्मभिः}
{तत्तेऽहं करजस्पर्शं प्राप्तवानिह दुर्लभम्}% ७२

\twolineshloka
{आज्ञापय महाराज त्वत्प्रसादादहं महत्}
{गच्छामि शाश्वतं स्थानं तव दुःखादिवर्जितम्}% ७३

\twolineshloka
{न यत्र शोको न जरा न मृत्युः कालविभ्रमः}
{तत्स्थानं देव गच्छामि त्वत्प्रसादान्नराधिप}% ७४

\twolineshloka
{इत्युक्त्वा तं परिक्रम्य विमानवरमारुहत्}
{अनेकरत्नखचितं सर्वदेवाधिवन्दितम्}% ७५

\twolineshloka
{गतोऽसौ शाश्वतस्थानं रामपादप्रसादतः}
{पुनरावृत्तिरहितं शोकमोहविवर्जितम्}% ७६

\twolineshloka
{तेन तत्कथितं श्रुत्वा रामं ज्ञात्वेतरे जनाः}
{विस्मयं प्रापिरे सर्वे परस्परमुदुन्मदाः}% ७७

\twolineshloka
{शृणु द्विजमहाबुद्धे दम्भेनापि स्मृतो हरिः}
{ददाति मोक्षं सुतरां किं पुनर्दम्भवर्जनात्}% ७८

\twolineshloka
{यथाकथञ्चिद्रामस्य कर्तव्यं स्मरणं परम्}
{येन प्राप्नोति परमं पदं देवादिदुर्लभम्}% ७९

\twolineshloka
{तच्चित्रं वीक्ष्य मुनयः कृतार्थं मेनिरे निजम्}
{यद्रामचरणप्रेक्षा करस्पर्शपवित्रितम्}% ८०

\twolineshloka
{गते तस्मिन्सुरे स्वर्गं हयरूपधरे पुरा}
{उवाच रामस्तपसां निधीन्वेदविदुत्तमान्}% ८१

\twolineshloka
{किं कर्तव्यं मयाब्रह्मन्हयो नष्टो गतः सुखम्}
{होमः कथं पुरोभावी सर्वदैवततर्पकः}% ८२

\twolineshloka
{यथा स्यात्सुरसन्तृप्तिर्यथा मे मख उत्तमः}
{तथा कुर्वन्तु मुनयो यथा मे स्याद्विधिश्रुतम्}% ८३

\twolineshloka
{इति वाक्यं समाश्रुत्य जगाद मुनिसत्तमः}
{वसिष्ठः सर्वदेवानां चित्ताभिज्ञानकोविदः}% ८४

\twolineshloka
{कर्पूरमाहर क्षिप्रं येन देवाः स्वयं पुरा}
{प्राप्य हव्यं ग्रहीष्यन्ति मद्वाक्यप्रेरिताधुना}% ८५

\twolineshloka
{इति वाक्यं समाकर्ण्य रामः क्षिप्रमुपाहरत्}
{कर्पूरं बहुदेवानां प्रीत्यर्थं बहुशोभनम्}% ८६

\twolineshloka
{तदा मुनिः प्रहृष्टात्मा देवानाह्वयदद्भुतान्}
{ते सर्वे तत्क्षणात्प्राप्ताः स्वपरीवारसंवृताः}% ८७

{॥इति श्रीपद्मपुराणे पातालखण्डे शेषवात्स्यायनसंवादे रामाश्वमेधे यज्ञप्रारम्भो नाम सप्तषष्टितमोऽध्यायः॥६७॥}

\dnsub{अष्टषष्टितमोऽध्यायः}\resetShloka

\uvacha{शेष उवाच}

\onelineshloka
{परिस्वादन्क्रतौ तृप्तिं न प्राप सुरसंयुतः}% १

\twolineshloka
{नारायणो महादेवो ब्रह्मा तत्र चतुर्मुखः}
{वरुणश्च कुबेरश्च तथान्ये लोकपालकाः}% २

\twolineshloka
{तत्रास्वाद्य हविः स्निग्धं वसिष्ठेन परिष्कृतम्}
{तत्र पुनर्हि विप्रेन्द्राः क्षुधार्ताइव भोजनात्}% ३

\twolineshloka
{सर्वान्देवांश्च सन्तर्प्य हविषा करुणानिधिः}
{वसिष्ठप्रेरितः सर्वमिति कर्तव्यमाचरत्}% ४

\twolineshloka
{ब्राह्मणादानसन्तुष्टा हविस्तुष्टाः सुरावराः}
{तृप्ताः सर्वे स्वकं भागं गृहीत्वा स्वालयं ययुः}% ५

\twolineshloka
{ऋषिभ्यो होतृमुख्येभ्यः प्रादाद्राज्यं चतुर्दिशम्}
{सन्तुष्टास्ते द्विजाराममाशीर्भिरददुः शुभम्}% ६

\twolineshloka
{पूर्णाहुतिं ततः कृत्वा वसिष्ठः प्राह सुस्त्रियः}
{वर्धापयन्तु भूमीशं यागपूर्तिकरं परम्}% ७

\twolineshloka
{तद्वाक्यं ताः स्त्रियः श्रुत्वा लाजैरवाकिरन्मुदा}
{लावण्यजितकन्दर्पं महामणिविभूषितम्}% ८

\twolineshloka
{ततोऽवभृथस्नानार्थं प्रेरयामास भूमिपम्}
{ययौ रामः सहस्वीयैः सरयूतीरमुत्तमम्}% ९

\twolineshloka
{अनेकराजकोटीभिः परीतः पादचारिभिः}
{जगाम स सरिच्छ्रेष्ठां पक्षिवृन्दसमाकुलाम्}% १०

\twolineshloka
{तारापतिरिव स्वाभिर्भार्याभिर्वृत उत्प्रभः}
{विरोचते तथा तद्वद्रामो राजगणैर्वृतः}% ११

\twolineshloka
{तदुत्सवं समाज्ञाय ययुर्लोकास्त्वरायुताः}
{सीतापतिमुखालोकनिश्चलीभूतलोचनाः}% १२

\twolineshloka
{राजेन्द्रं सीतया साकं गच्छन्तं सरितं प्रति}
{विलोक्य मुदिता लोकाश्चिरं दर्शनलालसाः}% १३

\twolineshloka
{अनेक नटगन्धर्वा गायन्तो यश उज्ज्वलम्}
{अनुजग्मुर्महीशानं सर्वलोकनमस्कृतम्}% १४

\twolineshloka
{नर्तक्यस्तत्र नृत्यन्त्यः क्षोभयन्त्यः पतेर्मनः}
{जलयन्त्रैश्च सिञ्चन्त्यो ययुः श्रीरामसेवनम्}% १५

\twolineshloka
{महाराजं विलिपन्त्यो हरिद्रा कुङ्कुमादिभिः}
{परस्परं प्रलिपन्त्यो मुदं प्रापुर्महत्तराम्}% १६

\twolineshloka
{कुचयुग्मोपरिन्यस्तमुक्ताहारसुशोभिताः}
{श्रवणद्वन्द्वसम्मृष्टस्वर्णकुण्डललक्षिताः}% १७

\twolineshloka
{अनेकनरनारीभिः सङ्कीर्णं मार्गमाचरन्}
{यथावत्सरितं प्राप शिवपुण्यजलाप्लुताम्}% १८

\twolineshloka
{तत्र गत्वा स वैदेह्या रामः कमललोचनः}
{प्रविवेश जलं पुण्यं वसिष्ठादिभिरन्वितः}% १९

\twolineshloka
{अनुप्रविविशुः सर्वे राजानो जनतास्तथा}
{तत्पादरजसा पूतजलं लोकैकवन्दितम्}% २०

\twolineshloka
{परस्परं प्रसिञ्चन्तो जलयन्त्रैर्मनोरमैः}
{सुशोणनयनाः सर्वे हर्षं प्रापुर्मनोधिकम्}% २१

\twolineshloka
{स रामः सीतया सार्धं चिरं पुण्यजलप्लवे}
{क्रीडित्वा जलकल्लोलैर्निरगाद्धर्मसंयुतः}% २२

\twolineshloka
{दुकूलवासाः सकिरीटकुण्डलः केयूरशोभावरकङ्कणान्वितः}
{कन्दर्पकोटिश्रियमुद्वहन्नृपो राजाग्र्यवर्यैरुपसंस्तुतो बभौ}% २३

\twolineshloka
{सयागयूपं वरवर्णशोभितं कृत्वा सरित्तीरवरे महामनाः}
{त्रैलोक्यलोकश्रियमाप ह्यद्भुतामन्यैर्दुरापां नृपतिर्भुजैर्निजैः}% २४

\twolineshloka
{एवं जनकपुत्र्यासौ हयमेधत्रयं चरन्}
{त्रैलोक्ये कीर्तिमतुलां प्राप देवैः सुदुर्लभाम्}% २५

\twolineshloka
{एवं ते वर्णितं तात यत्पृष्टो रामसत्कथाम्}
{विस्तृतः कथितो मेधो भूयः किं पृच्छसे द्विज}% २६

\twolineshloka
{यः शृणोति हरेर्भक्त्या रामचन्द्रस्य सन्मखम्}
{ब्रह्महत्यां क्षणात्तीर्त्वा ब्रह्मशाश्वतमाप्नुयात्}% २७

\twolineshloka
{अपुत्रो लभते पुत्रान्निर्धनो धनमाप्नुयात्}
{रोगार्तो मुच्यते रोगाद्बद्धो मुच्येत बन्धनात्}% २८

\twolineshloka
{यत्कथाश्रवणाद्दुष्टः श्वपचोऽपि परं पदम्}
{प्राप्नोति किमु विप्राग्र्यो रामभक्तिपरायणः}% २९

\twolineshloka
{रामं स्मृत्वा महाभागं पापिनः परमं पदम्}
{प्राप्नुयुः परमं स्वर्गं शक्रदेवादिदुर्लभम्}% ३०

\twolineshloka
{ते धन्या मानवा लोके ये स्मरन्ति रघूत्तमम्}
{ते क्षणात्संसृतिं तीर्त्वा गच्छन्ति सुखमव्ययम्}% ३१

\twolineshloka
{प्रत्येकमक्षरं ब्रह्महत्यावंशदवानलः}
{तं यः श्रावयते धीमांस्तं गुरुं सम्प्रपूजयेत्}% ३२

\twolineshloka
{श्रुत्वा कथां वाचकाय गवां द्वन्द्वं प्रदापयेत्}
{सपत्नीकाय सम्पूज्य वस्त्रालङ्कारभोजनैः}% ३३

\twolineshloka
{कुण्डलाभ्यां विराजन्त्यौ मुद्रिकाभिरलङ्कृते}
{रामसीते स्वर्णमय्यौ प्रतिमे शोभने वरे}% ३४

\twolineshloka
{कृत्वा तु वाचकायैव दीयते भो द्विजोत्तम}
{तस्य देवाश्च पितरो वैकुण्ठं प्राप्नुयुस्तदा}% ३५

\twolineshloka
{त्वया पृष्टा रामकथा मया ते कथिता पुरा}
{किमन्यत्कथ्यतां ब्रह्मन्पुरतस्तव धीमतः}% ३६

\twolineshloka
{शृण्वन्ति ये कथामेतां ब्रह्महत्यौघनाशिनीम्}
{ते यान्ति परमं स्थानं यच्च देवैः सुदुर्लभम्}% ३७

\twolineshloka
{गोघ्नश्चापि सुतघ्नश्च सुरापो गुरुतल्पगः}
{क्षणात्पूतो भवत्येव नात्र संशयितुं क्षमम्}% ३८

{॥इति श्रीपद्मपुराणे पातालखण्डे शेषवात्स्यायनसंवादे रामाश्वमेधे श्रवणपठनपुण्यवर्णनं नामाष्टषष्टितमोऽध्यायः॥६८॥}

{॥इति रामाश्वमेधप्रकरणं समाप्तम्॥}
