\chapt{पद्म-पुराणम्}

\sect{मार्कण्डेयाश्रमदर्शनम्}

\src{पद्म-पुराणम्}{}{अध्यायः १२३}{}
% \tags{concise, complete}
\notes{This chapter describes}
\textlink{https://sa.wikisource.org/wiki/पद्मपुराणम्/खण्डः_१_(सृष्टिखण्डम्)/अध्यायः_३३}
\translink{}

\storymeta


\uvacha{भीष्म उवाच}

\twolineshloka
{मार्कण्डेयेन वै रामः कथमत्र प्रबोधितः}
{कथं समागमो भूतः कस्मिन्काले कदा मुने} %॥१।

\twolineshloka
{मार्कण्डेयः कस्य सुतः कथं जातो महातपाः}
{नाम्नोऽस्य निगमं ब्रूहि यथाभूतं महामुने} %॥२।
\uvacha{पुलस्त्य उवाच}

\twolineshloka
{अथ ते सम्प्रवक्ष्यामि मार्कण्डेयोद्भवं पुनः}
{पुराकल्पे मुनिः पूर्वं मृकण्डुर्नाम विश्रुतः} %॥३।

\twolineshloka
{भृगोः पुत्रो महाभागः सभार्यस्तप्तवांस्तपः}
{तस्य पुत्रस्तदा जातो वसतस्तु वनान्तरे} %॥४।

\twolineshloka
{सपञ्चवार्षिको भूतो बाल एव गुणाधिकः}
{ज्ञानिना स तदा दृष्टो भ्रमन्बालस्तदाङ्गणे} %॥५।

\twolineshloka
{स्थित्वा स सुचिरं कालं भाव्यर्थं प्रत्यबुध्यत}
{तस्य पित्रा स वै पृष्टः कियदायुः सुतस्य मे} %॥६।

\twolineshloka
{सङ्ख्यायाचक्ष्व वर्षाणि तस्याल्पान्यधिकानि वा}
{मृकण्डुनैवमुक्तस्तु स ज्ञानी वाक्यमब्रवीत्} %॥७।

\twolineshloka
{षण्मासमायुः पुत्रस्य धात्रा सृष्टं मुनीश्वर}
{नैव शोकस्त्वया कार्यः सत्यमेतदुदाहृतम्} %॥८।

\twolineshloka
{स तच्छ्रुत्वा वचो भीष्म ज्ञानिना यदुदाहृतम्}
{अथोपनयनं चक्रे बालकस्य पिता तदा} %॥९।

\twolineshloka
{आह चैनं पितापुत्रमृषींस्त्वमभिवादय}
{एवमुक्तः स वै पित्रा प्रहृष्टश्चाभिवादने} %॥१०।

\twolineshloka
{न वर्णा वर्णतां वेत्ति सर्ववर्णाभिवादनः}
{पञ्चमासास्त्वतिक्रान्ता दिवसाः पञ्चविंशतिः} %॥११।

\twolineshloka
{मार्गेणाथ समायाता ऋषयस्तत्र सप्त वै}
{बालेन तेन ते दृष्टाः सर्वे चाप्यभिवादिताः} %॥१२।

\twolineshloka
{आयुष्मान्भव तैरुक्तः स बालो दण्डमेखली}
{उक्त्वैवं ते पुनर्बालमपश्यन्क्षीणजीवितम्} %॥१३।

\twolineshloka
{दिनानि पञ्च तस्यायुर्ज्ञात्वा भीताश्च ते नृप}
{तं गृहीत्वा बालकं च गतास्ते ब्रह्मणोन्तिकम्} %॥१४।

\twolineshloka
{प्रतिमुच्य च तं राजन्प्रणिपेतुः पितामहम्}
{अयमावेदितस्तैस्तु तेन ब्रह्माभिवादितः} %॥१५।

\twolineshloka
{चिरायुर्ब्रह्मणा बालः प्रोक्तः स ऋषिसन्निधौ}
{ततस्ते मुनयः प्रीताः श्रुत्वा वाक्यं पितामहात्} %॥१६।

\twolineshloka
{पितामह ऋषीन्दृष्ट्वा प्रोवाच विस्मयान्वितः}
{कार्येण येन चायातः कोयं बालो निवेद्यताम्} %॥१७।

\twolineshloka
{ततस्त ऋषयो राजन्सर्वं तस्मै न्यवेदयन्}
{पुत्रो मृकण्डोः क्षीणायुः सायुषं कुरु बालकम्} %॥१८।

\twolineshloka
{अल्पायुषस्त्वस्य मुनिर्बध्वेमां चापि मेखलाम्}
{यज्ञोपवीतं दण्डं च दत्वा चैनमबोधयत्} %॥१९।

\twolineshloka
{यं कञ्चित्पश्यसे बाल भ्रमन्तं भूतले जनम्}
{तस्याभिवादः कर्तव्य एवमाह पिता वचः} %॥२०।

\twolineshloka
{अभिवादनशीलोयं क्षितौ दृष्टः परिभ्रमन्}
{तीर्थयात्राप्रसङ्गेन दैवयोगात्पितामह} %॥२१।

\twolineshloka
{चिरायुर्भव पुत्रेति प्रोक्तोसौ तत्र बालकः}
{कथं वचो भवेत्सत्यमस्माकं भवता सह} %॥२२।

\twolineshloka
{एवमुक्तस्तदा तैस्तु ब्रह्मा लोकपितामहः}
{ऋतवाक्यादियं भूमिः संस्थिता सर्वतोभया} %॥२३।


\uvacha{ब्रह्मोवाच}

\twolineshloka
{मत्समश्चायुषा बालो मार्कण्डेयो भविष्यति}
{कल्पस्यादौ तथाचान्ते मतो मे मुनिसत्तमः} %॥२४।

\twolineshloka
{एवं ते मुनयो बालं ब्रह्मलोके पितामहात्}
{संसाध्य प्रेषयामासुर्भूयोप्येनं धरातलम्} %॥२५।

\twolineshloka
{तीर्थयात्रां गता विप्रा मार्कण्डेयो निजं गृहम्}
{जगाम तेषु यातेषु पितरं स्वमथाब्रवीत्} %॥२६।

\twolineshloka
{ब्रह्मलोकमहं नीतो मुनिभिर्ब्रह्मवादिभिः}
{दीर्घायुश्च कृतश्चास्मि वरान्दत्वा विसर्जितः} %॥२७।

\twolineshloka
{एतदन्यच्च मे दत्तं गतं चिन्ताकरं तव}
{कल्पस्यादौ तथा चान्ते भविष्ये समनन्तरे} %॥२८।

\twolineshloka
{लोककर्तुर्ब्रह्मणोहं प्रसादात्तस्य वै पितः}
{पुष्करं वै गमिष्यामि तपस्तप्तुं समुद्यतः} %॥२९।

\twolineshloka
{तत्राहं देवदेवेशमुपासिष्ये पितामहम्}
{सर्वकामावाप्तिकरं सर्वारातिनिबर्हणम्} %॥३०।

\twolineshloka
{सर्वसौख्यप्रदं देवमिन्द्रादीनां परायणम्}
{ब्रह्माणं तोषयिष्यामि सर्वलोकपितामहम्} %॥३१।

\twolineshloka
{मार्कण्डेयवचः श्रुत्वा मृकण्डुर्मुनिसत्तमः}
{जगाम परमं हर्षं क्षणमेकं समुच्छ्वसन्} %॥३२।

\twolineshloka
{धैर्यं सुमनसा स्थाय इदं वचनमब्रवीत्}
{अद्य मे सफलं जन्म जीवितं च सुजीवितम्} %॥३३।

\twolineshloka
{सर्वस्य जगतां स्रष्टा येन दृष्टः पितामहः}
{त्वया दायादवानस्मि पुत्रेण वंशधारिणा} %॥३४।

\twolineshloka
{त्वं गच्छ पश्य देवेशं पुष्करस्थं पितामहम्}
{दृष्टे तस्मिन्जगन्नाथे न जरामृत्युरेव च} %॥३५।

\twolineshloka
{नृणां भवति सौख्यानि तथैश्वर्यं तपोऽक्षयम्}
{त्रीणि शृङ्गाणि शुभ्राणि त्रीणि प्रस्रवणानि च} %॥३६।

\twolineshloka
{पुष्कराणि तथा त्रीणि नविद्मस्तत्र कारणम्}
{कनीयांसं मध्यमं च तृतीयं ज्येष्ठपुष्करम्} %॥३७।

\twolineshloka
{शृङ्गशब्दाभिधानानि शुभप्रस्रवणानि च}
{ब्रह्माविष्णुस्तथा रुद्रो नित्यं सन्निहितास्त्रयः} %॥३८।

\twolineshloka
{पुष्करेषु महाराजा नातः पुण्यतमं भुवि}
{विरजं विमलं तोयं त्रिषु लोकेषु विश्रुतम्} %॥३९।

\twolineshloka
{ब्रह्मलोकस्य पन्थानं धन्याः पश्यन्ति पुष्करं}
{यस्तु वर्षशतं साग्रमग्निहोत्रमुपासते} %॥४०।

\twolineshloka
{कार्तिकीं वा वसेदेकां पुष्करे सममेव च}
{कर्तुम्मया न शकितं कर्मणा नैव साधितम्} %॥४१।

\twolineshloka
{तदयत्नात्त्वया तात मृत्युस्सर्वहरो जितः}
{तत्र दृष्टस्स देवेशो ब्रह्मा लोकपितामहः} %॥४२।

\twolineshloka
{नान्यो मर्त्यस्त्वया तुल्यो भविता जगतीतले}
{अहं वै तोषितो येन पञ्चवार्षिकजन्मना} %॥४३।

\twolineshloka
{वरेण त्वं मदीयेन उपमां चिरजीविनाम्}
{गमिष्यसि न सन्देहस्तथाशीर्वचनम्मम} %॥४४।

\twolineshloka
{एवं वदन्ति ते सर्वे व्रज लोकान्यथेप्सितान्}
{एवं लब्धप्रसादेन मृकण्डुतनयेन च} %॥४५।

\twolineshloka
{आश्रमः स्थापितस्तेन मार्कण्डाश्रम इत्युत}
{तत्र स्नात्वा शुचिर्भूत्वा वाजपेयफलं लभेत्} %॥४६।

\onelineshloka*
{सर्वपापविशुद्धात्मा चिरायुर्जायते नरः}


\uvacha{पुलस्त्य उवाच}

\onelineshloka
{तथान्यं ते प्रवक्ष्यामि इतिहासं पुरातनम्} %॥४७।

\twolineshloka
{यथा रामेण वै तीर्थं पुष्करं तु विनिर्मितम्}
{चित्रकूटात्पुरा रामो मैथिल्या लक्ष्मणेन च} %॥४८।

\onelineshloka*
{अत्रेराश्रममासाद्य पप्रच्छ मुनिसत्तमम्}


\uvacha{राम उवाच}

\onelineshloka
{कानि पुण्यानि तीर्थानि किं वा क्षेत्रं महामुने} %॥४९।

\twolineshloka
{यत्र गत्वा नरो योगिन्वियोगं सह बन्धुभिः}
{नैव प्राप्नोति भगवन्तन्ममाचक्ष्व सुव्रत} %॥५०।

\twolineshloka
{अनेन वनवासेन राज्ञस्तु मरणेन च}
{भरतस्य वियोगेन परितप्ये ह्यहं त्रिभिः} %॥५१।

\twolineshloka
{तद्वाक्यं राघवेणोक्तं श्रुत्वा विप्रर्षभस्तदा}
{ध्यात्वा च सुचिरं कालमिदं वचनमब्रवीत्} %॥५२।


\uvacha{अत्रिरुवाच}

\twolineshloka
{साधु पृष्टं त्वया वीर रघूणां वंशवर्धन}
{मम पित्रा कृतं तीर्थं पुष्करं नाम विश्रुतम्} %॥५३।

\twolineshloka
{पर्वतौ द्वौ च विख्यातौ मर्यादा यज्ञपर्वतौ}
{कुण्डत्रयं तयोर्मध्ये ज्येष्ठमध्यकनिष्ठकम्} %॥५४।

\twolineshloka
{तेषु गत्वा दशरथं पिण्डदानेन तर्पय}
{तीर्थानां प्रवरं तीर्थं क्षेत्राणामपि चोत्तमम्} %॥५५।

\twolineshloka
{अवियोगा च सुरसा वापी रघुकुलोद्वह}
{तथा सौभाग्यकूपोन्यः सुजलो रघुनन्दन} %॥५६।

\twolineshloka
{तेषु पिण्डप्रदानेन पितरो मोक्षमाप्नुयुः}
{आभूतसम्प्लवं कालमेतदाह पितामहः} %॥५७।

\twolineshloka
{तत्र राघव गच्छस्व भूयोप्यागमनं क्रियाः}
{तथेति चोक्त्वा रामोपि गमनाय मनो दधे} %॥५८।

\twolineshloka
{ऋक्षवन्तमभिक्रम्य नगरं वैदिशं तथा}
{चर्मण्वतीं समुत्तीर्य प्राप्तोसौ यज्ञपर्वतम्} %॥५९।

\twolineshloka
{तमतिक्रम्य वेगेन मध्यमे पुष्करे स्थितः}
{पितॄन्सन्तर्पयामास अद्भिर्देवांश्च सर्वशः} %॥६०।

\twolineshloka
{स्नानावसाने रामेण मार्कण्डो मुनिपुङ्गवः}
{आगच्छन्शिष्यसंयुक्तो दृष्टस्तत्रैव धीमता} %॥६१।

\twolineshloka
{गत्वा वै सम्मुखं तस्य प्रणिपत्य च सादरम्}
{पृष्टोऽवियोगदः कूपः कतमस्यां दिशि प्रभो} %॥६२।

\twolineshloka
{सुतो दशरथस्याहं रामो नाम जनैः स्मृतः}
{सौभाग्यवापीं तां द्रष्टुमहं प्राप्तोत्रिशासनात्} %॥६३।

\twolineshloka
{तत्स्थानं तौ च वै कूपौ भगवान्प्रब्रवीतु मे}
{एवमुक्तश्च रामेण मार्कण्डः प्रत्युवाच ह} %॥६४।
\uvacha{मार्कण्डेय उवाच}

\twolineshloka
{साधु राघव भद्रं ते सुकृतं भवता कृतम्}
{तीर्थयात्राप्रसङ्गेन यत्प्राप्तोसीह साम्प्रतम्} %॥६५।

\twolineshloka
{एह्यागच्छस्व पश्य स्ववापीं तामवियोगदाम्}
{अवियोगश्च सर्वैश्च कूप एवात्र जायते} %॥६६।

\twolineshloka
{आमुष्मिके चैहिके च जीवतोपि मृतस्य वा}
{एतद्वाक्यं मुनीन्द्रस्य श्रुत्वा लक्ष्मणपूर्वजः} %॥६७।

\twolineshloka
{सस्मार रामो राजानं तदा दशरथं नृप}
{भरतं सह शत्रुघ्न्रं भातॄनन्यांश्चनागरान्} %॥६८।

\twolineshloka
{एवं चिन्तयतस्तस्य सन्ध्याकालो व्यजायत}
{उपास्य पश्चिमां सन्ध्यां मुनिभिः सह राघवः} %॥६९।

\twolineshloka
{सुष्वाप तां निशां तत्र भ्रातृभार्यासमन्वितः}
{विभावर्यवसाने तु स्वप्नान्ते रघुनन्दनः} %॥७०।

\twolineshloka
{पित्रा मात्रा तथा चान्यैरयोध्यायां स्थितः किल}
{विवाहमङ्गले वृत्ते बहुभिर्बान्धवैः सह} %॥७१।

\twolineshloka
{समासीनः सभार्योऽसावृषिभिः परिवारितः}
{लक्ष्मणेनाप्येवमेव दृष्टोऽसौ सीतया तथा} %॥७२।

\twolineshloka
{प्रभाते तु मुनीनां तत्सर्वमेव प्रकीर्तितम्}
{ऋषिभिश्च तथेत्युक्तः सत्यमेतद्रघूत्तम} %॥७३।

\twolineshloka
{मृतस्य दर्शने श्राद्धं कार्यमावश्यकं स्मृतम्}
{वृद्धिकामास्तु पितरस्तथा चैवान्नकाङ्क्षिणः} %॥७४।

\twolineshloka
{ददन्ति दर्शनं स्वप्ने भक्तियुक्तस्य राघव}
{अवियोगस्तु ते भ्रात्रा पित्रा च भरतेन च} %॥७५।

\twolineshloka
{चतुर्दशानां वर्षाणां भविता राघव ध्रुवम्}
{कुरु श्राद्धं तथा वीर राज्ञो दशरथस्य च} %॥७६।

\twolineshloka
{अमी च ऋषयः सर्वे तव भक्ताः कृतक्षणाः}
{अहं च जमदग्निश्च भारद्वाजश्च लोमशः} %॥७७।

\twolineshloka
{देवरातः शमीकश्च षडेते वै द्विजोत्तमाः}
{श्राद्धे च ते महाबाहो सम्भारांस्त्वमुपाहर} %॥७८।

\twolineshloka
{मुख्यं चेङ्गुदिपिण्याकं बदरामलकैः सह}
{श्रीफलानि च पक्वानि मूलं चोच्चावचं बहु} %॥७९।

\twolineshloka
{मार्गेण चाथ मांसेन धान्येन विविधेन च}
{तृप्तिं प्रयच्छ विप्राणां श्राद्धदानेन सुव्रत} %॥८०।

\twolineshloka
{पुष्करारण्यमासाद्य नियतो नियताशनः}
{पितॄंस्तर्पयते यस्तु सोश्वमेधमवाप्नुयात्} %॥८१।

\twolineshloka
{स्नानार्थं तु वयं राम गच्छामो ज्येष्ठपुष्करम्}
{इत्युक्त्वा ते गताः सर्वे मुनयो राघवं नृप} %॥८२।

\twolineshloka
{लक्ष्मणं चाब्रवीद्रामो मेध्यमाहर मे मृगम्}
{शुद्धेक्षणं च शशकं कृष्णशाकं तथा मधु} %॥८३।

\twolineshloka
{जम्बीराणि च मुख्यानि मूलानि विविधानि च}
{पक्वानि च कपित्थानि फलान्यन्यानि यानि च} %॥८४।

\twolineshloka
{तान्याहरस्व वै श्राद्धे क्षिप्रमेवास्तु लक्ष्मण}
{तथा तत्कृतवान्सर्वं रामादेशाच्च राघवः} %॥८५।

\twolineshloka
{बदरेङ्गुदिशाकानि मूलानि विविधानि च}
{तत्राहृत्य च रामेण कूटाकारः कृतो महान्} %॥८६।

\twolineshloka
{परिपक्वं च जानक्या सिद्धं रामे निवेदितम्}
{स्नात्वा रामो योगवाप्यां मुनींस्ताननुपालयन्} %॥८७।

\twolineshloka
{मध्याह्नाच्चलिते सूर्ये काले कुतपके तथा}
{आयाता ऋषयः सर्वे ये रामेणानुमन्त्रिताः} %॥८८।

\twolineshloka
{तानागतान्मुनीन्दृष्ट्वा वैदेही जनकात्मजा}
{रामान्तिकं परित्यज्य व्रीडिताऽन्यत्र संस्थिता} %॥८९।

\twolineshloka
{विस्मयोत्फुल्लनयना चिन्तयाना च वेपती}
{ब्राह्मणा नेह जानन्ति श्राद्धकाले ह्युपस्थिताः} %॥९०।

\twolineshloka
{रामेण भोजिता विप्राः स्मृत्युक्तेन यथाविधि}
{वैदिक्यश्च कृतास्सर्वाः सत्क्रिया यास्समीरिताः} %॥९१।

\twolineshloka
{पुराणोक्तो विधिश्चैव वैश्वदेविकपूर्वकः}
{भुक्तवत्सु च विप्रेषु दत्वा पिण्डान्यथाक्रमम्} %॥९२।

\twolineshloka
{प्रेषितेषु यथाशक्ति दत्वा तेषु च दक्षिणाम्}
{गतेषु विप्रमुख्येषु प्रियां रामोऽब्रवीदिदम्} %॥९३।

\twolineshloka
{किमर्थं सुभ्रु नष्टासि मुनीन्दृष्ट्वा त्विहागतान्}
{तत्सर्वं त्वमिदं तत्वं कारणं वद माचिरम्} %॥९४।

\twolineshloka
{भवितव्यं कारणेन तच्च गोप्यं न मे कुरु}
{शापितासि मम प्राणैर्लक्ष्मणस्य शुचिस्मिते} %॥९५।

\twolineshloka
{एवमुक्ता तदा भर्त्रा त्रपयाऽवाङ्मुखी स्थिता}
{विमुञ्चन्ती साऽश्रुपातं राघवं वाक्यमब्रवीत्} %॥९६।

\twolineshloka
{शृणु त्वं नाथ यद्दृष्टमाश्चर्यमिह यादृशम्}
{राम त्वयाऽचिन्त्यमानो राजेन्द्रस्त्विह चागतः} %॥९७।

\twolineshloka
{सर्वाभरणसंयुक्तौ द्वौ चान्यौ च तथाविधौ}
{द्विजानां देहसंयुक्तास्त्रयस्ते रघुनन्दन} %॥९८।

\twolineshloka
{पितरस्तु मया दृष्टा ब्राह्मणाङ्गेषु राघव}
{दृष्ट्वा त्रपान्विता चाहमपक्रान्ता तवान्तिकात्} %॥९९।

\twolineshloka
{त्वया वै भोजिता विप्राः कृतं श्राद्धं यथाविधि}
{वल्कलाजिनसंवीता कथं राज्ञः पुरःसरा} %॥१००।

\twolineshloka
{भवामि रिपुवीरघ्न सत्यमेतदुदाहृतम्}
{कौशेयानि च वस्त्राणि कैकेय्यापहृतानि च} %॥१०१।

\twolineshloka
{ततः प्रभृति चैवाहं चीरिणी तु वनाश्रयम्}
{ज्ञात्वाहं न वदे किञ्चिन्मा ते दुःखं भवत्विति} %॥१०२।

\twolineshloka
{नाहं स्मरामि वै मातुर्न पितुश्च परन्तप}
{कदा भविष्यतीहान्तो वनवासस्य राघव} %॥१०३।

\twolineshloka
{एतदेवानिशं राम चिन्तयन्त्याः पुनः पुनः}
{व्रजन्ति दिवसा नाथ तव पद्भ्यां शपाम्यहम्} %॥१०४।

\twolineshloka
{स्वहस्तेन कथं राज्ञो दास्ये वै भोजनं त्विदम्}
{दासानामपि यो दासो नोपभुञ्जीतयत्क्वचित्} %॥१०५।

\twolineshloka
{एतादृशी कथं त्वस्मै सम्प्रदातुं समुत्सहे}
{याहं राज्ञा पुरा दृष्टा सर्वालङ्कारभूषिता} %॥१०६।

\twolineshloka
{बालव्यजनहस्ता च वीजयन्ती नराधिपम्}
{सा स्वेदमलदिग्धाङ्गी कथं पश्यामि भूमिपम्} %॥१०७।

\twolineshloka
{व्यक्तं त्रिविष्टपं प्राप्तस्त्वया पुत्रेण तारितः}
{दृष्ट्वा मां दुःखितां बालां वने क्लिष्टामनागसम्} %॥१०८।

\twolineshloka
{शोकः स्यात्पार्थिवस्यास्य तेन नष्टास्मि राघव}
{भवान्प्राणसमो राम न ते गोप्यं ममत्विह} %॥१०९।

\twolineshloka
{सत्येन तेन चैवाथ स्पृशामि चरणौ तव}
{तच्छ्रुत्वा राघवः प्रीतः प्रियां तां प्रियवादिनीम्} %॥११०।

\twolineshloka
{अङ्कमानीय सुदृढं परिष्वज्य च सादरम्}
{भुक्तौ भोज्यं तदा वीरौ पश्चाद्भुक्ता च जानकी} %॥१११।

\twolineshloka
{एवं स्थितौ तदा सा च तां रात्रिं तत्र राघवौ}
{उदिते च सहस्रांशौ गमनाय मनो दधुः} %॥११२।

\twolineshloka
{प्रत्यङ्मुखं गतः क्रोशं ज्येष्ठं यावच्च पुष्करम्}
{पूर्वभागे पुष्करस्य यावत्तिष्ठति राघवः} %॥११३।

\twolineshloka
{शुश्राव च ततो वाचं देवदूतेन भाषितम्}
{भो भो राघव भद्रं ते तीर्थमेतत्सुदुर्लभम्} %॥११४।

\twolineshloka
{अस्मिन्स्थाने स्थितो वीर आत्मनः पुण्यतां कुरु}
{देवकार्यं त्वया कार्यं हन्तव्या देवशत्रवः} %॥११५।

\twolineshloka
{ततो हृष्टमना वीरो ह्यब्रवील्लक्ष्मणं वचः}
{सौमित्रेऽनुगृहीतोहं देवदेवेन ब्रह्मणा} %॥११६।

\twolineshloka
{अत्राश्रमपदं कृत्वा मासमेकं च लक्ष्मण}
{व्रतं चरितुमिच्छामि कायशोधनमुत्तमम्} %॥११७।

\twolineshloka
{तथेति लक्ष्मणेनोक्ते व्रतं परिसमाप्यतु}
{पिण्डदानादिभिर्दानैः श्राद्धैश्चैव पितामहान्} %॥११८।

\twolineshloka
{पुष्करे तु तदा रामोऽतर्पयद्विधिवत्तदा}
{कनका सुप्रभा चैव नन्दा प्राची सरस्वती} %॥११९।

\twolineshloka
{पञ्चस्रोताः पुष्करेषु पितॄणां तुष्टिदायिनी}
{दैनन्दिनीं पितॄणां तु पूजां तां पितृपूर्विकाम्} %॥१२०।

\twolineshloka
{रचयित्वा तदा रामो लक्ष्मणं वाक्यमब्रवीत्}
{एहि लक्ष्मण शीघ्रं त्वं पुष्कराज्जलमानय} %॥१२१।

\twolineshloka
{पादप्रक्षालनं कृत्वा शयनं कुरु संस्तरे}
{विभावर्यां निवृत्तायां यास्यामो दक्षिणां दिशम्} %॥१२२।

\twolineshloka
{लक्ष्मणस्त्वब्रवीद्वाक्यं सीतयानीय तां पयः}
{नाहं राम सर्वकाले दासभावं करोमि ते} %॥१२३।

\twolineshloka
{इयम्पुष्टाचसुभृशम्पीवरीचममाप्युत}
{किं त्वं करिष्यस्यनया भार्यया वद साम्प्रतम्} %॥१२४।

\twolineshloka
{किं वा मृतस्य वै पृष्ठ इयं यास्यति ते प्रिया}
{रक्षसे त्वं सदा कालं सुपुष्टां चैव सर्वदा} %॥१२५।

\twolineshloka
{हृष्टा चैषा क्लेशयति सततं मां रघूत्तम}
{त्वं च क्लेशयसे राम परत्र जायते क्षतिः} %॥१२६।

\twolineshloka
{त्वत्कृते च सदा चाहं पिपासां क्षुधया सह}
{संसहामि न सन्देहः परत्र च निशामय} %॥१२७।

\twolineshloka
{मृतानां पृष्ठतः कश्चिद्गतो नैव च दृश्यते}
{भार्य्या पुत्रो धनं चापि एवमाहुर्मनीषिणः} %॥१२८।

\twolineshloka
{मृतश्च ते पिता राम त्यक्त्वा राज्यमकण्टकम्}
{विनिक्षिप्य वने त्वां च कैकेय्याः प्रियकाम्यया} %॥१२९।

\twolineshloka
{इहस्थिता सा कैकेयी धनं सर्वे च बान्धवाः}
{महाराजो दशरथ एक एव गतो गतिम्} %॥१३०।

\twolineshloka
{मन्येहं न त्वया सार्धं सीता यास्यति वै ध्रुवम्}
{करिष्यसे किमनया वद राघव साम्प्रतम्} %॥१३१।

\twolineshloka
{श्रुत्वा चाश्रुतपूर्वं हि वाक्यं लक्ष्मणभाषितम्}
{विमना राघवस्तस्थौ सीता चापि वरानना} %॥१३२।

\twolineshloka
{यदुक्तं लक्ष्मणेनाथ सीता सर्वं चकार ह}
{स्नात्वा भुक्त्वा ततो वीरौ पुष्करे पुष्करेक्षणौ} %॥१३३।

\twolineshloka
{नीत्वा विभावरीं तत्र गमनाय मनो दधुः}
{एह्युत्तिष्ठ च सौमित्रे व्रजामो दक्षिणां दिशम्} %॥१३४।

\twolineshloka
{सौमित्रिरब्रवीद्राम नाहं यास्ये कथञ्चन}
{व्रज त्वमनया सार्धं भार्यया कमलेक्षण} %॥१३५।

\twolineshloka
{नान्यद्वनं गमिष्यामि नैवायोध्यां च राघव}
{अस्मिन्वने वसिष्यामि वर्षाणीह चतुर्दश} %॥१३६।

\twolineshloka
{मया विना त्वयोध्यायां यदि त्वं न गमिष्यसि}
{अनेन वर्त्मना भूप आगन्तव्यं त्वया विभो} %॥१३७।

\twolineshloka
{यदि जीवामि तत्कालं पुनर्यास्ये पितुः पुरम्}
{तपस्सम्भावयिष्यामि मया त्वं किं करिष्यसि} %॥१३८।

\twolineshloka
{व्रज सौम्य शिवः पन्थामा च ते परिपन्थिनः}
{पश्यामि त्वां पुनः प्राप्तं सभार्यं कमलेक्षणम्} %॥१३९।

\twolineshloka
{पितृपैतामहं राज्यमयोध्यायां नराधिप}
{शत्रुघ्नभरतौ चोभौ त्वदाज्ञाकरणे स्थितौ} %॥१४०।

\twolineshloka
{अहं ते प्रतिकूलस्तु वनवासे विशेषतः}
{अनारतं दिवा चाहं रात्रौ चैव परन्तप} %॥१४१।

\twolineshloka
{कर्मकर्तुं न शक्रोमि व्रज सौम्य यथासुखम्}
{एवं ब्रुवाणं सौमित्रिमुवाच रघुनन्दनः} %॥१४२।

\twolineshloka
{कथं पूर्वमयोध्याया निर्गतोसि मया सह}
{वने वत्स्याम्यहं राम नववर्षाणि पञ्च च} %॥१४३।

\twolineshloka
{न तु त्वया विरहितः स्वर्गेपि निवसे क्वचित्}
{या गतिस्ते नरव्याघ्र मम सापि भविष्यति} %॥१४४।

\twolineshloka
{प्रसादः क्रियतां मह्यं नय मामपि राघव}
{इदानीमर्धमार्गे त्वं कथं स्थास्यसि शत्रुहन्} %॥१४५।

\twolineshloka
{लक्ष्मणस्त्वब्रवीद्रामं नाहं गन्ता वने पुनः}
{लक्ष्मणं संस्थितं ज्ञात्वा रामो वचनमब्रवीत्} %॥१४६।

\twolineshloka
{मामनुव्रज सौमित्र एको यास्यामि काननम्}
{द्वितीया मे त्वियं सीता रामेणोक्तस्तु लक्ष्मणः} %॥१४७।

\twolineshloka
{गृहीत्वाऽथ समुत्तस्थौ रामवाक्यं स लक्ष्मणः}
{मर्यादापर्वतं प्राप्तौ क्षेत्रसीमां परन्तपौ} %॥१४८।

\twolineshloka
{अजगन्धं च देवेशं देवदेवं पिनाकिनम्}
{अष्टाङ्गप्रणिपातेन नत्वा रामस्त्रिलोचनम्} %॥१४९।

\twolineshloka
{तुष्टाव प्रयतः स्थित्वा शङ्करं पार्वतीप्रियम्}
{कृताञ्जलिपुटो भूत्वा रोमाञ्चितशरीरकः} %॥१५०।

\twolineshloka
{सात्विकं भावमापन्नो विनिर्धूतरजस्तमाः}
{लोकानां कारणं देवं बुबुधे विबुधाधिपम्} %॥१५१।
\uvacha{राम उवाच}

\twolineshloka
{कृत्स्नस्य योऽस्य जगतः स चराचरस्य कर्ता कृतस्य च पुनः सुखदुःखदश्च}
{संहारहेतुरपि यः पुनरन्तकाले तं शङ्करं शरणदं शरणं व्रजामि} %॥१५२।

\twolineshloka
{योऽयं सकृद्विमलचारुविलोलतोयां गङ्गां महोर्मिविषमां गगनात्पतन्तीम्}
{मूर्ध्ना दधेऽस्रजमिव प्रविलोलपुष्पां तं शङ्करं शरणदं शरणं व्रजामि} %१५३ ।

\twolineshloka
{कैलासशैलशिखरं परिकम्प्यमानं कैलासशृङ्गसदृशेन दशाननेन}
{यत्पादपद्मविधृतं स्थिरतां दधार तं शङ्करं शरणदं शरणं व्रजामि} %॥१५४।

\twolineshloka
{येनासकृद्दनुसुताः समरे निरस्ता विद्याधरोरगगणाश्च वरैः समग्रैः}
{संयोजिता मुनिवराः फलमूलभक्षास्तं शङ्करं शरणदं शरणं व्रजामि} %॥१५५।

\twolineshloka
{दक्षाध्वरे च नयने च तथा भगस्य पूष्णस्तथा दशनपङ्क्तिमपातयच्च}
{तस्तम्भयः कुलिशयुक्तमथेन्द्रहस्तं तं शङ्करं शरणदं शरणं व्रजामि} %॥१५६।

\twolineshloka
{एनःकृतोपिविषयेष्वपिसक्तचित्ताज्ञानान्वयश्रुतगुणैरपिनैवयुक्ताः}
{यं संश्रिताः सुखभुजः पुरुषा भवन्ति तं शङ्करं शरणदं शरणं व्रजामि} %॥१५७।

\twolineshloka
{अत्रिप्रसूतिरविकोटिसमानतेजाः सन्त्रासनं विबुधदानवसत्तमानाम्}
{यः कालकूटमपिबत्प्रसभं सुदीप्तं तं शङ्करं शरणदं शरणं व्रजामि} %॥१५८।

\twolineshloka
{ब्रह्मेन्द्ररुद्रमरुतां च सषण्मुखानां दद्याद्वरं सुबहुशो भगवान्महेशः}
{नन्दिं च मृत्युवदनात्पुनरुज्जहार तं शङ्करं शरणदं शरणं व्रजामि} %॥१५९।

\twolineshloka
{आराधितः सुतपसा हिमवन्निकुञ्जे धूमव्रतेन मनसापि परैरगम्ये}
{सञ्जीवनीमकथयद्भृगवे महात्मा तं शङ्करं शरणदं शरणं व्रजामि} %॥१६०।

\twolineshloka
{नानाविधैर्गजबिडालसमानवक्त्रैर्दक्षाध्वरप्रमथनैर्बलिभिर्गणैन्द्रैः }
{योभ्यर्चितोमरगणैश्च सलोकपालैस्तं शङ्करं शरणदं शरणं व्रजामि} %॥१६१।

\twolineshloka
{शङ्खेन्दुकुन्दधवलं वृषभं प्रवीरमारुह्य यः क्षितिधरेन्द्रसुतानुयातः}
{यात्यम्बरं प्रलयमेघविभूषितं च तं शङ्करं शरणदं शरणं व्रजामि} %॥१६२।

\twolineshloka
{शान्तं मुनिं यमनियोगपरायणैस्तैर्भीमैर्महोग्रपुरुषैः प्रतिनीयमानम्}
{भक्त्यानतं स्तुतिपरं प्रसभं ररक्ष तं शङ्करं शरणदं शरणं व्रजामि} %॥१६३।

\twolineshloka
{यः सव्यपाणि कमलाग्रनखेन देवस्तत्पञ्चमं प्रसभमेव पुरस्सुराणाम्}
{ब्राह्मं शिरस्तरुणपद्मनिभं चकर्त्त तं शङ्करं शरणदं शरणं व्रजामि} %॥१६४।

\twolineshloka
{यस्य प्रणम्य चरणौ वरदस्य भक्त्या स्तुत्वा च वाग्भिरमलाभिरतन्द्रितात्मा}
{दीप्तस्तमांसि नुदते स्वकरैर्विवस्वांस्तं शङ्करं शरणदं शरणं व्रजामि} %॥१६५।

\twolineshloka
{ये त्वां सुरोत्तमगुरुं पुरुषा विमूढा जानन्ति नास्य जगतः सचराचरस्य}
{ऐश्वर्यमाननिगमानुशयेन पश्चात्ते यातनामनुभवन्त्यविशुद्धचित्ताः} %॥१६६।

\twolineshloka
{तस्यैवं स्तुवतोऽवोचच्छूलपाणिर्वृषध्वजः}
{उवाच वचनं हृष्टो राघवं तुष्टमानसः} %॥१६७।
\uvacha{रुद्र उवाच}

\twolineshloka
{राम हृष्टोस्मि भद्रं ते जातस्त्वं निर्मले कुले}
{त्वं चापि जगतां वन्द्यो देवो मानुषरूपधृत्} %॥१६८।

\twolineshloka
{त्वया नाथेन वै देवाः सुखिनः शाश्वतीः समा}
{सेविष्यन्ते चिरं कालं गते वर्षे चतुर्दशे} %॥१६९।

\twolineshloka
{अयोध्यामागतं त्वां ये द्रक्ष्यन्ति भुवि मानवाः}
{सुखं तेऽत्र भजिष्यन्ति स्वर्गे वासन्तथाक्षयम्} %॥१७०।

\twolineshloka
{देवकार्यं महत्कृत्वा आगच्छेथाः पुनः पुरीम्}
{राघवस्तु तथा देवं नत्वा शीघ्रं विनिर्गतः} %॥१७१।

\twolineshloka
{इन्द्रमार्गां नदीं प्राप्य जटाजूटं नियम्य च}
{अब्रवील्लक्ष्मणं राम इदमर्पय मे धनुः} %॥१७२।

\twolineshloka
{रामवाक्यं तु तच्छ्रुत्वा सीतां वै लक्ष्मणोऽब्रवीत्}
{किमर्थं देवि रामेण त्यक्तोहं कारणं विना} %॥१७३।

\twolineshloka
{अपराधं न जानामि कुपितो यन्महाभुजः}
{रामेणाहं परित्यक्तः प्राणांस्त्यक्ष्याम्यसंशयम्} %॥१७४।

\twolineshloka
{नैव मे जीवितेनार्थो धिग्धिङ्मां कुलपांसनम्}
{आर्यस्य येन वै मन्युर्जनितः पापकारिणा} %॥१७५।

\twolineshloka
{कांस्तु लोकान्गमिष्यामि अपध्यातो महात्मना}
{उभौ हस्तौ मुखे कृत्वा साश्रुकण्ठोऽब्रवीदिदम्} %॥१७६।

\twolineshloka
{नापराध्यामि रामस्य कर्मणा मनसा गिरा}
{स्पृष्टौ ते चरणौ देवि मम नान्या गतिर्भवेत्} %॥१७७।

\twolineshloka
{ततः सीताऽब्रवीद्रामं त्यक्तः किमनुजस्त्वया}
{वैषम्यं त्यज्यतां बाले लक्ष्मणे लक्ष्मिवर्धने} %॥१७८।

\twolineshloka
{राघवस्त्वब्रवीत्सीतां नाहं त्यक्ष्यामि लक्ष्मणम्}
{न कदाचिदपि स्वप्ने लक्ष्मणस्य मतं प्रिये} %॥१७९।

\twolineshloka
{श्रुतपूर्वं च सुश्रोणि क्षेत्रस्यास्य विचेष्टितम्}
{अत्र क्षेत्रे जनास्सत्यं सर्वे हि स्वार्थतत्पराः} %॥१८०।

\twolineshloka
{परस्परं न पश्यन्ति स्वात्मनश्च हितं वचः}
{न शृण्वन्ति पितुः पुत्राः पुत्राणां पितरस्तथा} %॥१८१।

\twolineshloka
{न शिष्या हि गुरोर्वाक्यं शिष्यस्यापि तथा गुरुः}
{अर्थानुबन्धिनीप्रीतिर्न कश्चित्कस्यचित्प्रियः} %॥१८२।

\twolineshloka
{इत्येवं कथयन्नेव प्राप्तो रेवां महानदीम्}
{चक्रेभिषेकं काकुत्स्थः सानुजः सह सीतया} %॥१८३।

\twolineshloka
{तर्पयित्वा च सलिलैः स्वान्पितॄन्दैवतान्यपि}
{उदीक्ष्य च मुहुः सूर्यं देवताश्च समाहितः} %॥१८४।

\twolineshloka
{कृताभिषेकस्तु रराज रामः सीता द्वितीयः सह लक्ष्मणेन}
{कृताभिषेकः सह शैलपुत्र्या गुहेन सार्धं भगवानिवेशः} %॥१८५।

॥इति श्रीपाद्मपुराणे प्रथमे सृष्टिखण्डे मार्कण्डेयाश्रमदर्शनं नाम त्र्यस्त्रिंशोऽध्यायः॥३३॥