\sect{सप्तत्रिंशोऽध्यायः --- यज्ञनिवारणम्}

\src{पद्म-पुराणम्}{सृष्टिखण्डम्}{अध्यायः ३७}{१--१७१}
% \tags{concise, complete}
\notes{This chapter describes Rāma’s Abstaining from the Performance of Rājasūya yajna.}
\textlink{https://sa.wikisource.org/wiki/पद्मपुराणम्/खण्डः_१_(सृष्टिखण्डम्)/अध्यायः_३७}
\translink{https://www.wisdomlib.org/hinduism/book/the-padma-purana/d/doc364160.html}

\storymeta


\uvacha{पुलस्त्य उवाच}

\twolineshloka
{तदद्भुततमं वाक्यं श्रुत्वा च रघुनन्दनः}
{गौरवाद्विस्मयाच्चापि भूयः प्रष्टुं प्रचक्रमे}% १

\uvacha{राम उवाच}

\twolineshloka
{भगवंस्तद्वनं घोरं यत्रासौ तप्तवांस्तपः}
{श्वेतो वैदर्भको राजा तदद्भुतमभूत्कथम्}% २

\twolineshloka
{विषमं तद्वनं राजा शून्यं मृगविवर्जितम्}
{प्रविष्टस्तप आस्थातुं कथं वद महामुने}% ३

\twolineshloka
{समन्ताद्योजनशतं निर्मनुष्यमभूत्कथम्}
{भवान्कथं प्रविष्टस्तद्येन कार्येण तद्वद}% ४

\uvacha{अगस्त्य उवाच}

\twolineshloka
{पुरा कृतयुगे राजा मनुर्दण्डधरः प्रभुः}
{तस्य पुत्रोथ नाम्नासीदिक्ष्वाकुरमितद्युतिः}% ५

\twolineshloka
{तं पुत्रं पूर्वजं राज्ये निक्षिप्य भुविसम्मतम्}
{पृथिव्यां राजवंशानां भव राजेत्युवाच ह}% ६

\twolineshloka
{तथेति च प्रतिज्ञातं पितुः पुत्रेण राघव}
{ततःपरमसंहृष्टः पुनस्तं प्रत्यभाषत}% ७

\twolineshloka
{प्रीतोस्मि परमोदार कर्मणा ते न संशयः}
{दण्डेन च प्रजा रक्ष न च दण्डमकारणम्}% ८

\twolineshloka
{अपराधिषु यो दण्डः पात्यते मानवैरिह}
{स दण्डो विधिवन्मुक्तः स्वर्गं नयति पार्थिवम्}% ९

\twolineshloka
{तस्माद्दण्डे महाबाहो यत्नवान्भव पुत्रक}
{धर्मस्ते परमो लोके कृत एवं भविष्यति}% १०

\twolineshloka
{इति तं बहुसन्दिश्य मनुः पुत्रं समाधिना}
{जगाम त्रिदिवं हृष्टो ब्रह्मलोकमनुत्तमम्}% ११

\twolineshloka
{जनयिष्ये कथं पुत्रानिति चिन्तापरोऽभवत्}
{कर्मभिर्बहुभिस्तैस्तैस्ससुतैस्संयुतोऽभवत्}% १२

\twolineshloka
{तोषयामास पुत्रैस्स पितॄन्देवसुतोपमैः}
{सर्वेषामुत्तमस्तेषां कनीयान्रघुनन्दन}% १३

\twolineshloka
{शूरश्च कृतविद्यश्च गुरुश्च जनपूजया}
{नाम तस्याथ दण्डेति पिता चक्रे स बुद्धिमान्}% १४

\twolineshloka
{भविष्यद्दण्डपतनं शरीरे तस्य वीक्ष्य च}
{सम्पश्यमानस्तं दोषं घोरं पुत्रस्य राघव}% १५

\twolineshloka
{स विन्ध्यनीलयोर्मध्ये राज्यमस्य ददौ प्रभुः}
{स दण्डस्तत्र राजाभूद्रम्ये पर्वतमूर्द्धनि}% १६

\twolineshloka
{पुरं चाप्रतिमं तेन निवेशाय तथा कृतम्}
{नाम तस्य पुरस्याथ मधुमत्तमिति स्वयम्}% १७

\twolineshloka
{तथादेशेन सम्पन्नः शूरो वासमथाकरोत्}
{एवं राजा स तद्राज्यं चकार सपुरोहितः}% १८

\twolineshloka
{प्रहृष्ट सुप्रजाकीर्णं देवराजो यथा दिवि}
{ततः स दण्डः काकुत्स्थ बहुवर्षगणायुतम्}% १९

\twolineshloka
{अकारयत्तु धर्मात्मा राज्यं निहतकण्टकम्}
{अथ काले तु कस्मिंश्चिद्राजा भार्गवमाश्रमम्}% २०

\twolineshloka
{रमणीयमुपाक्रामच्चैत्रमासे मनोरमे}
{तत्र भार्गवकन्यां तु रूपेणाप्रतिमां भुवि}% २१

\twolineshloka
{विचरन्तीं वनोद्देशे दण्डोऽपश्यदनुत्तमाम्}
{उत्तुङ्गपीवरीं श्यामां चन्द्राभवदनां शुभाम्}% २२

\twolineshloka
{सुनासां चारुसर्वाङ्गीं पीनोन्नतपयोधराम्}
{मध्ये क्षामां च विस्तीर्णां दृष्ट्वा तां कुरुते मुदम्}% २३

\twolineshloka
{एकवस्त्रां वने चैकां प्रथमे यौवने स्थिताम्}
{स तां दृष्ट्वात्वधर्मेण अनङ्गशरपीडितः}% २४

\twolineshloka
{अभिगम्य सुविश्रान्तां कन्यां वचनमब्रवीत्}
{कुतस्त्वमसि सुश्रोणि कस्य चासि सुशोभने}% २५

\twolineshloka
{पीडतोहमनङ्गेन पृच्छामि त्वां सुशोभने}
{त्वया मेऽपहृतं चित्तं दर्शनादेव सुन्दरि}% २६

\twolineshloka
{इदं ते वदनं रम्यं मुनीनां चित्तहारकम्}
{यद्यहं न लभे भोक्तुं मृतं मामवधारय}% २७

\twolineshloka
{त्वया हृता मम प्राणा मां जीवय सुलोचने}
{दासोस्मि ते वरारोहे भक्तं मां भज शोभने}% २८

\twolineshloka
{तस्यैवं तु ब्रुवाणस्य मदोन्मत्तस्य कामिनः}
{भार्गवी प्रत्युवाचेदं वचः सविनयं नृपम्}% २९

\twolineshloka
{भार्गवस्य सुतां विद्धि शुक्रस्याक्लिष्टकर्मणः}
{अरजां नाम राजेन्द्र ज्येष्ठामाश्रमवासिनः}% ३०

\twolineshloka
{शुक्रः पिता मे राजेन्द्र त्वं च शिष्यो महात्मनः}
{धर्मतो भगिनी चाहं भवामि नृपनन्दन}% ३१

\twolineshloka
{एवंविधं वचो वक्तुं न त्वमर्हसि पार्थिव}
{अन्येभ्योपि सुदुष्टेभ्यो रक्ष्या चाहं सदा त्वया}% ३२

\twolineshloka
{क्रोधनो मे पिता रौद्रो भस्मत्वं त्वां समानयेत्}
{अथवा राजधर्मेणासम्बन्धं कुरुषे बलात्}% ३३

\twolineshloka
{पितरं याचयस्व त्वं धर्मदृष्टेन कर्मणा}
{वरयस्व नृपश्रेष्ठ पितरं मे महाद्युतिम्}% ३४

\twolineshloka
{अन्यथा विपुलं दुःखं तव घोरं भवेद्ध्रुवम्}
{क्रुद्धो हि मे पिता सर्वं त्रैलोक्यमभिनिर्दहेत्}% ३५

\twolineshloka
{ततोऽशुभं महाघोरं श्रुत्वा दण्डः सुदारुणम्}
{प्रत्युवाच मदोन्मत्तः शिरसाभिनतः पुनः}% ३६

\twolineshloka
{प्रसादं कुरु सुश्रोणि कामोन्मत्तस्य कामिनि}
{त्वया रुद्धा मम प्राणा विशीर्यन्ति शुभानने}% ३७

\twolineshloka
{त्वां प्राप्य वैरं मेऽत्रास्तु वधो वापि महत्तरः}
{भक्तं भजस्व मां भीरु त्वयि भक्तिर्हि मे परा}% ३८

\twolineshloka
{एवमुक्त्वा तु तां कन्यां बलात्सङ्गृह्य बाहुना}
{अन्येन राज्ञा हस्तेन विवस्त्रा सा तथा कृता}% ३९

\twolineshloka
{अङ्गमङ्गे समाश्लेष्य मुखे चैव मुखं कृतम्}
{विस्फुरन्तीं यथाकामं मैथुनायोपचक्रमे}% ४०

\twolineshloka
{तमनर्थं महाघोरं दण्डः कृत्वा सुदारुणम्}
{नगरं स्वं जगामाशु मदोन्मत्त इव द्विपः}% ४१

\twolineshloka
{भार्गवी रुदती दीना आश्रमस्याविदूरतः}
{प्रत्यपालयदुद्विग्ना पितरं देवसम्मितम्}% ४२

\twolineshloka
{स मुहूर्तादुपस्पृश्य देवर्षिरमितद्युतिः}
{स्वमाश्रमं शिष्यवृतं क्षुधार्तः सन्यवर्तत}% ४३

\twolineshloka
{सोपश्यदरजां दीनां रजसा समभिप्लुताम्}
{चन्द्रस्य घनसंयुक्तां ज्योत्स्नामिव पराजिताम्}% ४४

\twolineshloka
{तस्य रोषः समभवत्क्षुधार्तस्य महात्मनः}
{निर्दहन्निव लोकांस्त्रींस्तान्शिष्यान्समुवाच ह}% ४५

\twolineshloka
{पश्यध्वं विपरीतस्य दण्डस्यादीर्घदर्शिनः}
{विपत्तिं घोरसङ्काशां दीप्तामग्निशिखामिव}% ४६

\twolineshloka
{यन्नाशं दुर्गतिं प्राप्तस्सानुगश्च न संशयः}
{यस्तु दीप्तहुताशस्य अर्चिः संस्पृष्टवानिह}% ४७

\twolineshloka
{यस्मात्स कृतवान्पापमीदृशं घोरसम्मितम्}
{तस्मात्प्राप्स्यति दुर्मेधाः पांसुवर्षमनुत्तमम्}% ४८

\twolineshloka
{कुराजा देशसंयुक्तः सभृत्यबलवाहनः}
{पापकर्मसमाचारो वधं प्राप्स्यति दुर्मतिः}% ४९

\twolineshloka
{समन्ताद्योजनशतं विषयं चास्य दुर्मतेः}
{धुनोतु पांसुवर्षेण महता पाकशासनः1.37.}% ५०

\twolineshloka
{सर्वसत्वानि यानीह जङ्गमस्थावराणि वै}
{सर्वेषां पांसुवर्षेण क्षयः क्षिप्रं भविष्यति}% ५१

\twolineshloka
{दण्डस्य विषयो यावत्तावत्सवनमाश्रमम्}
{पांसुवर्षमिवाकस्मात्सप्तरात्रं भविष्यति}% ५२

\twolineshloka
{इत्युक्त्वा क्रोधसन्तप्तस्तमाश्रमनिवासिनम्}
{जनं जनपदस्यान्ते स्थीयतामित्युवाच ह}% ५३

\twolineshloka
{उक्तमात्रे उशनसा आश्रमावसथो जनः}
{क्षिप्रं तु विषयात्तस्मात्स्थानं चक्रे च बाह्यतः}% ५४

\twolineshloka
{तं तथोक्त्वा मुनिजनमरजामिदमब्रवीत्}
{आश्रमे त्वं सुदुर्मेधे वस चेह समाहिता}% ५५

\twolineshloka
{इदं योजनपर्यन्तमाश्रमं रुचिरप्रभम्}
{अरजे विरजास्तिष्ठ कालमत्र समाश्शतम्}% ५६

\twolineshloka
{श्रुत्वा नियोगं विप्रर्षेररजा भार्गवी तदा}
{तथेति पितरं प्राह भार्गवं भृशदुःखिता}% ५७

\twolineshloka
{इत्युक्त्वा भार्गवो वासं तस्मादन्यमुपाक्रमत्}
{सप्ताहे भस्मसाद्भूतं यथोक्तं ब्रह्मवादिना}% ५८

\twolineshloka
{तस्माद्दण्डस्य विषयो विन्ध्यशैलस्य मानुष}
{शप्तो ह्युशनसा राम तदाभूद्धर्षणे कृते}% ५९

\twolineshloka
{ततःप्रभृति काकुत्स्थ दण्डकारण्यमुच्यते}
{एतत्ते सर्वमाख्यातं यन्मां पृच्छसि राघव}% ६०

\twolineshloka
{सन्ध्यामुपासितुं वीर समयो ह्यतिवर्तते}
{एते महर्षयो राम पूर्णकुम्भाः समन्ततः}% ६१

\twolineshloka
{कृतोदका नरव्याघ्र पूजयन्ति दिवाकरम्}
{सर्वैरॄषिभिरभ्यस्तैः स्तोत्रैर्ब्रह्मादिभिः कृतैः}% ६२

\twolineshloka
{रविरस्तङ्गतो राम गत्वोदकमुपस्पृश}
{ॠषेर्वचनमादाय रामः सन्ध्यामुपासितुम्}% ६३

\twolineshloka
{उपचक्राम तत्पुण्यं ससरोरघुनन्दनः}
{अथ तस्मिन्वनोद्देशे रम्ये पादपशोभिते}% ६४

\twolineshloka
{नदपुण्ये गिरिवरे कोकिलाशतमण्डिते}
{नानापक्षिरवोद्याने नानामृगसमाकुले}% ६५

\twolineshloka
{सिंहव्याघ्रसमाकीर्णे नानाद्विजसमावृते}
{गृध्रोलूकौ प्रवसितौ बहून्वर्षगणानपि}% ६६

\twolineshloka
{अथोलूकस्य भवनं गृध्रः पापविनिश्चयः}
{ममेदमिति कृत्वाऽसौ कलहं तेन चाकरोत्}% ६७

\twolineshloka
{राजा सर्वस्य लोकस्य रामो राजीवलोचनः}
{तं प्रपद्यावहै शीघ्रं कस्यैतद्भवनं भवेत्}% ६८

\twolineshloka
{गृध्रोलूकौ प्रपद्येतां जातकोपावमर्षिणौ}
{रामं प्रपद्यतौ शीघ्रं कलिव्याकुलचेतसौ}% ६९

\twolineshloka
{तौ परस्परविद्वेषौ स्पृशतश्चरणौ तथा}
{अथ दृष्ट्वा राघवेन्द्रं गृध्रो वचनमब्रवीत्}% ७०

\twolineshloka
{सुराणामसुराणां च त्वं प्रधानो मतो मम}
{बृहस्पतेश्च शुक्राच्च त्वं विशिष्टो महामतिः}% ७१

\twolineshloka
{परावरज्ञो भूतानां मर्त्ये शक्र इवापरः}
{दुर्निरीक्षो यथा सूर्यो हिमवानिव गौरवे}% ७२

\twolineshloka
{सागरश्चासि गाम्भीर्ये लोकपालो यमो ह्यसि}
{क्षान्त्या धरण्या तुल्योसि शीघ्रत्वे ह्यनिलोपमः}% ७३

\twolineshloka
{गुरुस्त्वं सर्वसम्पन्नो विष्णुरूपोसि राघव}
{अमर्षी दुर्जयो जेता सर्वास्त्रविधिपारगः}% ७४

\twolineshloka
{शृणु त्वं मम देवेश विज्ञाप्यं नरपुङ्गव}
{ममालयं पूर्वकृतं बाहुवीर्येण वै प्रभो}% ७५

\twolineshloka
{उलूको हरते राजंस्त्वत्समीपे विशेषतः}
{ईदृशोयं दुराचारस्त्वदाज्ञा लङ्घको नृप}% ७६

\twolineshloka
{प्राणान्तिकेन दण्डेन राम शासितुमर्हसि}
{एवमुक्ते तु गृध्रेण उलूको वाक्यमब्रवीत्}% ७७

\twolineshloka
{शृणु देव मम ज्ञाप्यमेकचित्तो नराधिप}
{सोमाच्छक्राच्च सूर्याच्च धनदाच्च यमात्तथा}% ७८

\twolineshloka
{जायते वै नृपो राम किञ्चिद्भवति मानुषः}
{त्वं तु सर्वमयो देवो नारायणपरायणः}% ७९

\twolineshloka
{प्रोच्यते सोमता राजन्सम्यक्कार्ये विचारिते}
{सम्यग्रक्षसि तापेभ्यस्तमोघ्नो हि यतो भवान्}% ८०

\twolineshloka
{दोषे दण्डात्प्रजानां त्वं यतः पापभयापहः}
{दाता प्रहर्ता गोप्ता च तेनेन्द्र इव नो भवान्}% ८१

\twolineshloka
{अधृष्यः सर्वभूतेषु तेजसा चानलो मतः}
{अभीक्ष्णं तपसे पापांस्तेन त्वं राम भास्करः}% ८२

\twolineshloka
{साक्षाद्वित्तेशतुल्यस्त्वमथवा धनदाधिकः}
{चित्तायत्ता तु पत्नीश्रीर्नित्यं ते राजसत्तम}% ८३

\twolineshloka
{धनदस्य तु कोशेन धनदस्तेन वैभवान्}
{समः सर्वेषु भूतेषु स्थावरेषु चरेषु च}% ८४

\twolineshloka
{शत्रौ मित्रे च ते दृष्टिः समन्ताद्याति राघव}
{धर्मेण शासनं नित्यं व्यवहारविधिक्रमैः}% ८५

\twolineshloka
{यस्य रुष्यसि वै राम मृत्युस्तस्याभिधीयते}
{गीयसे तेन वै राजन्यम इत्यभिविश्रुतः}% ८६

\twolineshloka
{यश्चासौ मानुषो भावो भवतो नृपसत्तम}
{आनृशंस्यपरो राजा सर्वेषु कृपयान्वितः}% ८७

\twolineshloka
{दुर्बलस्य त्वनाथस्य राजा भवति वै बलम्}
{अचक्षुषो भवेच्चक्षुरमतेषु मतिर्भवेत्}% ८८

\twolineshloka
{अस्माकमपि नाथस्त्वं श्रूयतां मम धार्मिक}
{भवता तत्र मन्तव्यं यथैते किल पक्षिणः}% ८९

\twolineshloka
{योस्मन्नाथः स पक्षीन्द्रो भवतो विनियोज्यकः}
{अस्वाम्यं देव नास्माकं सन्निधौ भवतः प्रभो}% ९०

\twolineshloka
{भवतैव कृतं पूर्वं भूतग्रामं चतुर्विधम्}
{ममालयप्रविष्टस्तु गृध्रो मां बाधते नृप}% ९१

\twolineshloka
{भवान्देवमनुष्येषु शास्ता वै नरपुङ्गव}
{एतच्छ्रुत्वा तु वै रामः सचिवानाह्वयत्स्वयम्}% ९२

\twolineshloka
{विष्टिर्जयन्तो विजयः सिद्धार्थो राष्ट्रवर्धनः}
{अशोको धर्मपालश्च सुमन्त्रश्च महाबलः}% ९३

\twolineshloka
{एते रामस्य सचिवा राज्ञो दशरथस्य च}
{नीतियुक्ता महात्मानः सर्वशास्त्रविशारदाः}% ९४

\twolineshloka
{सुशान्ताश्च कुलीनाश्च नये मन्त्रे च कोविदाः}
{तानाहूय स धर्मात्मा पुष्पकादवरुह्य च}% ९५

\twolineshloka
{गृध्रोलूकौ विवदन्तौ पृच्छति स्म रघूत्तमः}
{कति वर्षाणि भो गृध्र तवेदं निलयं कृतम्}% ९६

\twolineshloka
{एतन्मे कौतुकं ब्रूहि यदि जानासि तत्त्वतः}
{एतच्छ्रुत्वा वचो गृध्रो बभाषे राघवं स्थितम्}% ९७

\twolineshloka
{इयं वसुमती राम मानुषैर्बहुबाहुभिः}
{उच्छ्रितैराचिता सर्वा तदाप्रभृति मद्गृहम्}% ९८

\twolineshloka
{उलूकस्त्वब्रवीद्रामं पादपैरुपशोभिता}
{यदैव पृथिवी राजंस्तदाप्रभृति मे गृहम्}% ९९

\twolineshloka
{एतच्छ्रुत्वा तु रामो वै सभासद उवाचह}
{न सा सभा यत्र न सन्ति वृद्धा वृद्धा न ते ये न वदन्ति धर्मम्}% १००

\twolineshloka
{नासौ धर्मो यत्र न चास्ति सत्यं न तत्सत्यं यच्छलमभ्युपैति}
{ये तु सभ्याः सभां गत्वा तूष्णीं ध्यायन्त आसते}% १०१

\twolineshloka
{यथाप्राप्तं न ब्रुवते सर्वे तेऽनृतवादिनः}
{न वक्ति च श्रुतं यश्च कामात्क्रोधात्तथा भयात्}% १०२

\twolineshloka
{सहस्रं वारुणाः पाशाः प्रतिमुञ्चन्ति तं नरम्}
{तेषां संवत्सरे पूर्णे पाश एकः प्रमुच्यते}% १०३

\twolineshloka
{तस्मात्सत्यं तु वक्तव्यं जानता सत्यमञ्जसा}
{एतच्छ्रुत्वा तु सचिवा राममेवाब्रुवंस्तदा}% १०४

\twolineshloka
{उलूकः शोभते राजन्न तु गृध्रो महामते}
{त्वं प्रमाणं महाराज राजा हि परमा गतिः}% १०५

\twolineshloka
{राजमूलाः प्रजाः सर्वा राजा धर्मः सनातनः}
{शास्ता राजा नृणां येषां न ते गच्छन्ति दुर्गतिम्}% १०६

\twolineshloka
{वैवस्वतेन मुक्ताश्च भवन्ति पुरुषोत्तमाः}
{सचिवानां वचः श्रुत्वा रामो वचनमब्रवीत्}% १०७

\twolineshloka
{श्रूयतामभिधास्यामि पुराणं यदुदाहृतम्}
{द्यौः सचन्द्रार्कनक्षत्रा सपर्वतमहीद्रुमम्}% १०८

\twolineshloka
{सलिलार्णवसम्मग्नं त्रैलोक्यं सचराचरम्}
{एकमेव तदा ह्यासीत्सर्वमेकमिवाम्बरम्}% १०९

\twolineshloka
{पुनर्भूः सह लक्ष्म्या च विष्णोर्जठरमाविशत्}
{तां निगृह्य महातेजाः प्रविश्य सलिलार्णवम्}% ११०

\twolineshloka
{सुष्वाप हि कृतात्मा स बहुवर्षशतान्यपि}
{विष्णौ सुप्ते ततो ब्रह्मा विवेश जठरं ततः}% १११

\twolineshloka
{बहुस्रोतं च तं ज्ञात्वा महायोगी समाविशत्}
{नाभ्यां विष्णोः समुद्भूतं पद्मं हेमविभूषितम्}% ११२

\twolineshloka
{स तु निर्गम्य वै ब्रह्मा योगी भूत्वा महाप्रभुः}
{सिसृक्षुः पृथिवीं वायुं पर्वतांश्च महीरुहान्}% ११३

\twolineshloka
{तदन्तराः प्रजाः सर्वा मानुषांश्च सरीसृपान्}
{जरायुजाण्डजान्सर्वान्ससर्ज स महातपाः}% ११४

\twolineshloka
{तस्य गात्रसमुत्पन्नः कैटभो मधुना सह}
{दानवौ तौ महावीर्यौ घोरौ लब्धवरौ तदा}% ११५

\twolineshloka
{दृष्ट्वा प्रजापतिं तत्र क्रोधाविष्टावुभौ नृप}
{वेगेन महता भोक्तुं स्वयम्भुवमधावताम्}% ११६

\twolineshloka
{दृष्ट्वा सत्वानि सर्वाणि निस्सरन्ति पृथक्पृथक्}
{ब्रह्मणा संस्तुतो विष्णुर्हत्वा तौ मधुकैटभौ}% ११७

\twolineshloka
{पृथिवीं वर्धयामास स्थित्यर्थं मेदसा तयोः}
{मेदोगन्धा तु धरणी मेदिनीत्यभिधां गता}% ११८

\twolineshloka
{तस्माद्गृध्रस्त्वसत्यो वै पापकर्मापरालयम्}
{स्वीयं करोति पापात्मा दण्डनीयो न संशयः}% ११९

\twolineshloka
{ततोऽशरीरिणीवाणी अन्तरिक्षात्प्रभाषते}
{मा वधी राम गृध्रं त्वं पूर्वन्दग्धं तपोबलात्}% १२०

\twolineshloka
{पुरा गौतम दग्धोऽयं प्रजानाथो जनेश्वर}
{ब्रह्मदत्तस्तु नामैष शूरः सत्यव्रतः शुचिः}% १२१

\twolineshloka
{गृहमागत्य विप्रर्षेर्भोजनं प्रत्ययाचत}
{साग्रं वर्षशतं चैव भुक्तवान्नृपसत्तम}% १२२

\twolineshloka
{ब्रह्मदत्तस्य वै तस्य पाद्यमर्घ्यं स्वयं ततः}
{आत्मनैवाकरोत्सम्यग्भोजनार्थं महाद्युते}% १२३

\twolineshloka
{समाविश्य गृहं तस्य आहारे तु महात्मनः}
{नारीं पूर्णस्तनीं दृष्ट्वा हस्तेनाथ परामृशत्}% १२४

\twolineshloka
{अथ क्रुद्धेन मुनिना शापो दत्तः सुदारुणः}
{गृध्रत्वं गच्छ वै मूढ राजा मुनिमथाब्रवीत्}% १२५

\twolineshloka
{कृपां कुरु महाभाग शापोद्धारो भविष्यति}
{दयालुस्तद्वचः श्रुत्वा पुनराह नराधिप}% १२६

\twolineshloka
{उत्पत्स्यते रघुकुले रामो नाम महायशाः}
{इक्ष्वाकूणां महाभागो राजा राजीवलोचनः}% १२७

\twolineshloka
{तेन दृष्टो विपापस्त्वं भविता नरपुङ्गव}
{दृष्टो रामेण तच्छ्रुत्वा बभूव पृथिवीपतिः}% १२८

\twolineshloka
{गृध्रत्वं त्यज्य वै शीघ्रं दिव्यगन्धानुलेपनः}
{पुरुषो दिव्यरूपोऽसौ बभाषे तं नराधिपम्}% १२९

\twolineshloka
{साधु राघव धर्मज्ञ त्वत्प्रसादादहं विभो}
{विमुक्तो नरकाद्घोरादपापस्तु त्वया कृतः}% १३०

\twolineshloka
{विसर्जितं मया गार्ध्यं नररूपी महीपतिः}
{उलूकं प्राह धर्मज्ञ स्वगृहं विश कौशिक}% १३१

\twolineshloka
{अहं सन्ध्यामुपासित्वा गमिष्ये यत्र वै मुनिः}
{अथोदकमुपस्पृश्य सन्ध्यामन्वास्य पश्चिमाम्}% १३२

\twolineshloka
{आश्रमं प्राविशद्रामः कुम्भयोनेर्महात्मनः}
{तस्यागस्त्यो बहुगुणं फलमूलं च सादरम्}% १३३

\twolineshloka
{रसवन्ति च शाकानि भोजनार्थमुपाहरत्}
{सभुक्तवान्नरव्याघ्रस्तदन्नममृतोपमम्}% १३४

\twolineshloka
{प्रीतश्च परितुष्टश्च तां रात्रिं समुपावसत्}
{प्रभाते काल्यमुत्थाय कृत्वाह्निकमरिन्दम}% १३५

\twolineshloka
{ॠषिं समभिचक्राम गमनाय रघूत्तमः}
{अभिवाद्याब्रवीद्रामो महर्षिं कुम्भसम्भवम्}% १३६

\twolineshloka
{आपृच्छे साधये ब्रह्मन्ननुज्ञातुं त्वमर्हसि}
{धन्योस्म्यनुगृहीतोस्मि दर्शनेन महामुने}% १३७

\twolineshloka
{दिष्ट्या चाहं भविष्यामि पावनात्मा महात्मनः}
{एवं ब्रुवति काकुत्स्थे वाक्यमद्भुतदर्शनम्}% १३८

\twolineshloka
{उवाच परमप्रीतो बाष्पनेत्रस्तपोधनः}
{अत्यद्भुतमिदं वाक्यं तव राम शुभाक्षरम्}% १३९

\twolineshloka
{पावनं सर्वभूतानां त्वयोक्तं रघुनन्दन}
{मुहूर्तमपि राम त्वां मैत्रेणेक्षन्ति ये नराः}% १४०

\twolineshloka
{पावितास्सर्वसूक्तैस्ते कथ्यन्ते त्रिदिवौकसः}
{ये च त्वां घोरचक्षुर्भिरीक्षन्ते प्राणिनो भुवि}% १४१

\twolineshloka
{ते हता ब्रह्मदण्डेन सद्यो नरकगामिनः}
{ईदृशस्त्वं रघुश्रेष्ठ पावनः सर्वदेहिनाम्}% १४२

\twolineshloka
{कथयन्तश्च लोकास्त्वां सिद्धिमेष्यन्ति राघव}
{गच्छस्वानातुरोऽविघ्नं पन्थानमकुतोभयः}% १४३

\twolineshloka
{प्रशाधि राज्यं धर्मेण गतिस्तु जगतां भवान्}
{एवमुक्तस्तु मुनिना प्राञ्जलि प्रग्रहो नृपः}% १४४

\twolineshloka
{अभिवादयितुं चक्रे सोऽगस्त्यमृषिसत्तमम्}
{अभिवाद्य मुनिश्रेष्ठंस्तांश्च सर्वांस्तपोधिकान्}% १४५

\twolineshloka
{अथारोहत्तदाव्यग्रः पुष्पकं हेमभूषितम्}
{तं प्रयान्तं मुनिगणा आशीर्वादैस्समन्ततः}% १४६

\twolineshloka
{अपूपुजन्नरेन्द्रं तं सहस्राक्षमिवामराः}
{ततोऽर्धदिवसे प्राप्ते रामः सर्वार्थकोविदः}% १४७

\twolineshloka
{अयोध्यां प्राप्य काकुत्स्थः पद्भ्यां कक्षामवातरत्}
{ततो विसृज्य रुचिरं पुष्पकं कामवाहितम्}% १४८

\twolineshloka
{कक्षान्तराद्विनिष्क्रम्य द्वास्थान्राजाऽब्रवीदिदम्}
{लक्ष्मणं भरतं चैव गच्छध्वं लघुविक्रमाः}% १४९

\twolineshloka
{ममागमनमाख्याय समानयत मा चिरम्}
{श्रुत्वाथ भाषितं द्वास्था रामस्याक्लिष्टकर्मणः1.37.}% १५०

\twolineshloka
{गत्वा कुमारावाहूय राघवाय न्यवदेयन्}
{द्वास्थैः कुमारावानीतौ राघवस्य निदेशतः}% १५१

\twolineshloka
{दृष्ट्वा तु राघवः प्राप्तौ प्रियौ भरतलक्ष्मणौ}
{समालिङ्ग्य तु रामस्तौ वाक्यं चेदमुवाच ह}% १५२

\twolineshloka
{कृतं मया यथातथ्यं द्विजकार्यमनुत्तमम्}
{धर्महेतुमतो भूयः कर्तुमिच्छामि राघवौ}% १५३

\twolineshloka
{भवद्भ्यामात्मभूताभ्यां राजसूयं क्रतूत्तमम्}
{सहितो यष्टुमिच्छामि यत्र धर्मश्च शाश्वतः}% १५४

\twolineshloka
{पुष्करस्थेन वै पूर्वं ब्रह्मणा लोककारिणा}
{शतत्रयेण यज्ञानामिष्टं षष्ट्याधिकेन च}% १५५

\twolineshloka
{इष्ट्वा हि राजसूयेन सोमो धर्मेण धर्मवित्}
{प्राप्तः सर्वेषु लोकेषु कीर्तिस्थानमनुत्तमम्}% १५६

\twolineshloka
{इष्ट्वा हि राजसूयेन मित्रः शत्रुनिबर्हणः}
{मुहूर्तेन सुशुद्धेन वरुणत्वमुपागतः}% १५७

\onelineshloka*
{तस्माद्भवन्तौ सञ्चिन्त्य कार्येस्मिन्वदतं हि तत्}

\uvacha{भरत उवाच}

\onelineshloka
{त्वं धर्मः परमः साधो त्वयि सर्वा वसुन्धरा}% १५८

\twolineshloka
{प्रतिष्ठिता महाबाहो यशश्चामितविक्रम}
{महीपालाश्च सर्वे त्वां प्रजापतिमिवामराः}% १५९

\twolineshloka
{निरीक्षन्ते महात्मानो लोकनाथ तथा वयम्}
{प्रजाश्च पितृवद्राजन्पश्यन्ति त्वां महामते}% १६०

\twolineshloka
{पृथिव्यां गतिभूतोसि प्राणिनामिह राघव}
{सत्वमेवंविधं यज्ञं नाहर्त्तासि परन्तप}% १६१

\twolineshloka
{पृथिव्यां सर्वभूतानां विनाशो दृश्यते यतः}
{श्रूयते राजशार्दूल सोमस्य मनुजेश्वर}% १६२

\twolineshloka
{ज्योतिषां सुमहद्युद्धं सङ्ग्रामे तारकामये}
{तारा बृहस्पतेर्भार्या हृता सोमेनकामतः}% १६३

\twolineshloka
{तत्र युद्धं महद्वृत्तं देवदानवनाशनम्}
{वरुणस्य क्रतौ घोरे सङ्ग्रामे मत्स्यकच्छपाः}% १६४

\twolineshloka
{निवृत्ते राजशार्दूल सर्वे नष्टा जलेचराः}
{हरिश्चन्द्रस्य यज्ञान्ते राजसूयस्य राघव}% १६५

\twolineshloka
{आडीबकम्महद्युद्धं सर्वलोकविनाशनम्}
{पृथिव्यां यानि सत्वानि तिर्यग्योनिगतानि वै}% १६६

\twolineshloka
{दिव्यानां पार्थिवानां च राजसूये क्षयः श्रुतः}
{स त्वं पुरुषशार्दूल बुद्ध्या सञ्चिन्त्य पार्थिव}% १६७

\twolineshloka
{प्राणिनां च हितं सौम्यं पूर्णधर्मं समाचर}
{भरतस्य वचः श्रुत्वा राघवः प्राह सादरम्}% १६८

\twolineshloka
{प्रीतोस्मि तव धर्मज्ञ वाक्येनानेन शत्रुहन्}
{निवर्तिता राजसूयान्मतिर्मे धर्मवत्सल}% १६९

\twolineshloka
{पूर्णं धर्मं करिष्यामि कान्यकुब्जे च वामनम्}
{स्थापयिष्याम्यहं वीर सा मे ख्यातिर्दिवं गता}% १७०

\onelineshloka
{भविष्यति न सन्देहो यथा गङ्गा भगीरथात्}% १७१

{॥इति श्रीपाद्मपुराणे प्रथमे सृष्टिखण्डे यज्ञनिवारणं नाम सप्तत्रिंशोऽध्यायः॥३७॥}
