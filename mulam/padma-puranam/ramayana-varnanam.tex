\sect{द्विचत्वारिंशदधिक-द्विशततमोऽध्यायः --- रामस्यायोध्याप्रवेशः}

\src{पद्म-पुराणम्}{सृष्टिखण्डम्}{अध्यायः २४२--२४४}{}
% \tags{concise, complete}
\notes{}
\textlink{https://sa.wikisource.org/wiki/पद्मपुराणम्/खण्डः_५_(पातालखण्डः)/अध्यायः_००१}
\translink{https://www.wisdomlib.org/hinduism/book/the-padma-purana/d/doc365826.html}

\storymeta


\uvacha{रुद्र उवाच}

\twolineshloka
{स्वायम्भुवो मनुः पूर्वं द्वाशार्णं महामनुम्}
{जजाप गोमतीतीरे नैमिषे विमले शुभे}% १

\twolineshloka
{तेन वर्षसहस्रेण पूजितः कमलापतिः}
{मत्तो वरं वृणीष्वेति तं प्राह भगवान्हरिः}% २

\onelineshloka*
{ततः प्रोवाच हर्षेण मनुः स्वायम्भुवो हरिम्}

\uvacha{मनुरुवाच}
\onelineshloka
{पुत्रत्वं भज देवेश त्रीणि जन्मानि चाच्युत}% ३

\onelineshloka*
{त्वां पुत्रलालसत्वेन भजामि पुरुषोत्तमम्}

\uvacha{रुद्र उवाच}
\onelineshloka
{इत्युक्तस्तेन लक्ष्मीशः प्रोवाच सुमहागिरा}% ४

\uvacha{विष्णुरुवाच}

\twolineshloka
{भविष्यति नृपश्रेष्ठ यत्ते मनसि काङ्क्षितम्}
{ममैव च महत्प्रीतिस्तव पुत्रत्वहेतवे}% ५

\twolineshloka
{स्थितिप्रयोजने काले तत्र तत्र नृपोत्तम}
{त्वयि जाते त्वहमपि जातोस्मि तव सुव्रत}% ६

\twolineshloka
{परित्राणाय साधूनां विनाशाय च दुष्कृताम्}
{धर्म्मसंस्थापनार्थाय सम्भवामि तवानघ}% ७

\uvacha{रुद्र उवाच}

\twolineshloka
{एवं दत्वा वरं तस्मै तत्रैवान्तर्दधे हरिः}
{अस्याभूत्प्रथमं जन्म मनोः स्वायम्भुवस्य च}% ८

\twolineshloka
{रघूणामन्वये पूर्वं राजा दशरथो ह्यभूत्}
{द्वितीयो वसुदेवोऽभूद्वृष्णीनामन्वये विभुः}% ९

\twolineshloka
{कलेर्दिव्यसहस्राब्दप्रमाणस्यान्त्यपादयोः}
{शम्भलग्रामकं प्राप्य ब्राह्मणः सञ्जनिष्यति}% १०

\twolineshloka
{कौशल्या समभूत्पत्नी राज्ञो दशरथस्य हि}
{यदोर्वंशस्य सेवार्थं देवकी नाम विश्रुता}% ११

\twolineshloka
{हरिव्रतस्य विप्रस्य भार्य्या देवप्रभा पुनः}
{एवं मातृत्वमापन्ना त्रीणि जन्मानि शार्ङ्गिणः}% १२

\twolineshloka
{पूर्वं रामस्य चरितं वक्ष्यामि तव सुव्रते}
{यस्य स्मरणमात्रेण विमुक्तिः पापिनामपि}% १३

\twolineshloka
{हिरण्यकहिरण्याक्षौ द्वितीयं जन्मसंश्रितौ}
{कुम्भकर्ण दशग्रीवावजायेतां महाबलौ}% १४

\twolineshloka
{पुलस्त्यस्य सुतो विप्रो विश्रवा नाम धार्मिकः}
{तस्य पत्नी विशालाक्षी राक्षसेन्द्र सुताऽनघे}% १५

\twolineshloka
{सुकेशितनया सा स्यात्सुमाली दानवस्य च}
{केकसी नाम कन्यासीत्तस्य भार्या दृढव्रता}% १६

\twolineshloka
{कामोद्रिक्ता तु सा देवी सन्ध्याकाले महामुनिम्}
{रमयामास तन्वङ्गी यथेष्टं शुभदर्शना}% १७

\twolineshloka
{कामात्सन्ध्याभवाद्यत्वात्तस्यां जातौ महाबलौ}
{रावणः कुम्भकर्णश्च राक्षसौ लोकविश्रुतौ}% १८

\twolineshloka
{कन्या शूर्पणखा नाम जातातिविकृतानना}
{कस्यचित्त्वथ कालस्य तस्यां जातो विभीषणः}% १९

\twolineshloka
{सुशीलो भगवद्भक्तः सत्यवाग्धर्म्मवाञ्शुचिः}
{रावणः कुम्भकर्णश्च हिमवत्पर्वतोत्तमे}% २०

\twolineshloka
{महोग्रतपसा मां वै पूजयामासतुर्भृशम्}
{रावणस्त्वथ दुष्टात्मा स्वशिरःकमलैः शुभैः}% २१

\twolineshloka
{पूजयामास मां देवि दारुणेनैव कर्म्मणा}
{ततस्तमब्रुवं सुभ्रूः प्रहृष्टेनान्तरात्मना}% २२

\twolineshloka
{वरं वृणीष्व मे वत्स यत्ते मनसि वर्त्तते}
{ततः प्रोवाच दुष्टात्मा देवदानव रक्षसाम्}% २३

\twolineshloka
{अवध्यत्वं प्रदेहीति सर्वलोकजिगीषया}
{ततोऽहं दत्तवांस्तस्मै राक्षसाय दुरात्मने}% २४

\twolineshloka
{देवदानवयक्षाणामवध्यत्वं वरानने}
{राक्षसोऽसौ महावीर्यो वरदानात्तु गर्वितः}% २५

\twolineshloka
{त्रींल्लोकान्पीडयामास देवदानवमानुषान्}
{तेन सम्बाध्यमानाश्च देवा ब्रह्मपुरोगमाः}% २६

\twolineshloka
{भयार्त्ताः शरणं जग्मुरीश्वरं कमलापतिम्}
{ज्ञात्वाथ वेदनां तेषामभयाय सनातनः}% २७

\onelineshloka*
{उवाच त्रिदशान्सर्वान्ब्रह्मरुद्रपुरोगमान्}

\uvacha{श्रीभगवानुवाच}
\onelineshloka
{राज्ञो दशरथस्याहमुत्पत्स्यामि रघोः कुले}% २८

\twolineshloka
{वधिष्यामि दुरात्मानं रावणं सह बान्धवम्}
{मानुषं वपुरास्थाय हन्मि दैवतकण्टकम्}% २९

\twolineshloka
{नन्दिशापाद्भवन्तोऽपि वानरत्वमुपागताः}
{कुरुध्वं मम साहाय्यं गन्धर्वाप्सरसोत्तमाः}% ३०

\uvacha{रुद्र उवाच}

\twolineshloka
{इत्युक्ता देवतास्सर्वा देवदेवेन विष्णुना}
{वानरत्वमुपागम्य जज्ञिरे पृथिवीतले}% ३१

\twolineshloka
{भार्गवेण प्रदत्ता तु महीसागरमेखला}
{दत्ता महर्षिभिः पूर्वं रघूणां सुमहात्मनाम्}% ३२

\twolineshloka
{वैवस्वतमनोः पुत्रो राज्ञां श्रेष्ठो महाबलः}
{इक्ष्वाकुरिति विख्यातस्सर्वधर्म्मविदांवरः}% ३३

\twolineshloka
{तदन्वये महातेजा राजा दशरथो बली}
{अजस्य नृपतेः पुत्रः सत्यवान्शीलवान्शुचिः}% ३४

\twolineshloka
{स राजा पृथिवीं सर्वां पालयामास वीर्य्यतः}
{राज्येषु स्थापयामास सर्वान्पार्थिवसत्तमान्}% ३५

\twolineshloka
{कोशलस्य नृपस्याथ पुत्री सर्वाङ्गशोभना}
{कौशल्या नाम तां कन्यामुपयेमे स पार्थिवः}% ३६

\twolineshloka
{मागधस्य नृपस्याथ तनया च शुचिस्मिता}
{सुमित्रा नाम नाम्ना च द्वितीया तस्य भामिनी}% ३७

\twolineshloka
{तृतीया केकयस्याथ नृपतेर्दुहिता तथा}
{भार्य्याभूत्पद्मपत्राक्षी केकयी नाम नामतः}% ३८

\twolineshloka
{ताभिः स्म राजा भार्याभिस्तिसृभिर्धर्मसंयुतः}
{रमयामास काकुत्स्थः पृथिवीं चानुपालयन्}% ३९

\twolineshloka
{अयोध्या नाम नगरी सरयूतीर संस्थिता}
{सर्वरत्नसुसम्पूर्णा धनधान्यसमाकुला}% ४०

\twolineshloka
{प्राकारगोपुरैर्जुष्टा हेमप्राकारसङ्कुला}
{उत्तमैर्नागतुरगैर्महेन्द्रस्य यथा पुरी}% ४१

\twolineshloka
{तस्यां राजा स धर्मात्मा उवास मुनिसत्तमैः}
{पुरोहितेन विप्रेण वसिष्ठेन महात्मना}% ४२

\twolineshloka
{राज्यं चकारयामास सर्वं निहतकण्टकम्}
{यस्मादुत्पत्स्यते तस्यां भगवान्पुरुषोत्तमः}% ४३

\twolineshloka
{तस्मात्तु नगरी पुण्या साप्ययोध्येति कीर्तिता}
{नगरस्य परं धाम्नो नाम तस्याप्यभूच्छुभे}% ४४

\twolineshloka
{यत्रास्ते भगवान्विष्णुस्तदेव परमं पदम्}
{तत्र सद्यो भवेन्मोक्षः सर्वकर्म्मनिकृन्तनः}% ४५

\twolineshloka
{जाते तत्र महाविष्णौ नराः सर्वे मुदं ययुः}
{स राजा पृथिवीं सर्वां पालयित्वा शुभानने}% ४६

\twolineshloka
{अयजद्वैष्णवेष्ट्या च पुत्रार्थी हरिमच्युतम्}
{तेन सम्पूजितः श्रीशो राजा सर्वगतो हरिः}% ४७

\twolineshloka
{वैष्णवेन तु यज्ञेन वरदः प्राह केशवः}
{तस्मिन्नाविरभूदग्नौ यज्ञरूपो हरिस्तदा}% ४८

\twolineshloka
{शुद्धजाम्बूनदप्रख्यः शङ्खचक्रगदाधरः}
{शुक्लाम्बरधरः श्रीमान्सर्वभूषणभूषितः}% ४९

\twolineshloka
{श्रीवत्सकौस्तुभोरस्को वनमालाविभूषितः}
{पद्मपत्रविशालाक्षश्चतुर्बाहुरुदारधीः}% ५०

\twolineshloka
{सव्याङ्कस्थ श्रिया सार्द्धमाविरासीद्रमेश्वरः}
{वरदोस्मीति तं प्राह राजानं भक्तवत्सलः}% ५१

\twolineshloka
{तं दृष्ट्वा सर्वलोकेशं राजा हर्षसमाकुलः}
{ववन्दे भार्य्यया सार्द्धं प्रहृष्टेनान्तरात्मना}% ५२

\twolineshloka
{प्राञ्जलिः प्रणतो भूत्वा हर्षगद्गदया गिरा}
{पुत्रत्वं मे भजेत्याह देवदेवं जनार्दनम्}% ५३

\onelineshloka*
{ततः प्रसन्नो भगवान्प्राह राजानमच्युतः}

\uvacha{विष्णुरुवाच}
\onelineshloka
{उत्पत्स्येऽहं नृपश्रेष्ठ देवलोकहिताय वै}% ५४

\twolineshloka
{परित्राणाय साधूनां राक्षसानां वधाय च}
{मुक्तिं प्रदातुं लोकानां धर्म्मसंस्थापनाय च}% ५५

\uvacha{महादेव उवाच}

\twolineshloka
{इत्युक्त्वा पायसं दिव्यं हेमपात्रस्थितं शृतम्}
{लक्ष्म्याहस्तस्थितं शुभ्रं पार्थिवाय ददौ हरिः}% ५६

\uvacha{विष्णुरुवाच}

\twolineshloka
{इदं वै पायसं राजन्पत्नीभ्यस्तव सुव्रत}
{देहि ते तनयास्तासु उत्पत्स्यन्ते मदङ्गजाः}% ५७

\uvacha{महादेव उवाच}

\twolineshloka
{इत्युक्त्वा मुनिभिः सर्वैः स्तूयमानो जनार्दनः}
{स्वात्मानं दर्शयित्वाथ तथैवान्तरधीयत}% ५८

\twolineshloka
{स राजा तत्र दृष्ट्वा च पत्नीं ज्येष्ठां कनीयसीम्}
{विभज्य पायसं दिव्यं प्रददौ सुसमाहितः}% ५९

\twolineshloka
{एतस्मिन्नन्तरे पत्नी सुमित्रा तस्य मध्यमा}
{तत्समीपं प्रयाता सा पुत्रकामा सुलोचना}% ६०

\twolineshloka
{तां दृष्ट्वा तत्र कौशल्या कैकेयी च सुमध्यमा}
{अर्द्धमर्द्धं प्रददतुस्ते तस्यै पायसं स्वकम्}% ६१

\twolineshloka
{तत्प्राश्य पायसं दिव्यं राजपत्न्यः सुमध्यमाः}
{सम्पन्नगर्भाः सर्वास्ता विरेजुः शुभ्रवर्च्चसः}% ६२

\twolineshloka
{तासां स्वप्नेषु देवेशः पीतवासा जनार्दनः}
{शङ्खचक्रगदापाणिराविर्भूतस्तदा हरिः}% ६३

\twolineshloka
{अस्मिन्काले मनोरम्ये मधुमासि शुचिस्मिते}
{शुक्ले नवम्यां विमले नक्षत्रेऽदितिदैवते}% ६४

\twolineshloka
{मध्याह्नसमये लग्ने सर्वग्रहशुभान्विते}
{कौसल्या जनयामास पुत्रं लोकेश्वरं हरिम्}% ६५

\twolineshloka
{इन्दीवरदलश्यामं कोटिकन्दर्प्पसन्निभम्}
{पद्मपत्रविशालाक्षं सर्वाभरणशोभितम्}% ६६

\twolineshloka
{श्रीवत्सकौस्तुभोरस्कं सर्वाभरणभूषितम्}
{उद्यद्दिनकरप्रख्यकुण्डलाभ्यां विराजितम्}% ६७

\twolineshloka
{अनेकसूर्य्यसङ्काशं तेजसा महता वृतम्}
{परेशस्य तनो रम्यं दीपादुत्पन्नदीपवत्}% ६८

\twolineshloka
{ईशानं सर्वलोकानां योगिध्येयं सनातनम्}
{सर्वोपनिषदामर्थमनन्तं परमेश्वरम्}% ६९

\twolineshloka
{जगत्सर्गस्थितिलये हेतुभूतमनामयम्}
{शरण्यं सर्वभूतानां सर्वभूतमयं विभुम्}% ७०

\twolineshloka
{समुत्पन्ने जगन्नाथे देवदुन्दुभयो दिवि}
{विनेदुः पुष्पवर्षाणि ववर्षुः सुरसत्तमाः}% ७१

\twolineshloka
{प्रजापतिमुखा देवा विमानस्था नभस्तले}
{तुष्टुवुर्मुनिभिः सार्द्धं हर्षपूर्णाङ्गविह्वलाः}% ७२

\twolineshloka
{जगुर्गन्धर्वपतयो ननृतुश्चाप्सरोगणाः}
{ववुः पुण्यशिवा वाताः सुप्रभोभूद्दिवाकरः}% ७३

\twolineshloka
{जज्वलुश्चाग्नयः शान्ता विमलाश्च दिशोदश}
{ततस्स राजा हर्षेण पुत्रं दृष्ट्वा सनातनम्}% ७४

\twolineshloka
{पुरोधसा वसिष्ठेन जातकर्म्मतदाऽकरोत्}
{नाम चास्मै ददौ रम्यं वसिष्ठो भगवांस्तदा}% ७५

\twolineshloka
{श्रियः कमलवासिन्या रमणोऽयं महान्प्रभुः}
{तस्माच्छ्रीराम इत्यस्य नाम सिद्धं पुरातनम्}% ७६

\twolineshloka
{सहस्रनाम्नां श्रीशस्य तुल्यं मुक्तिप्रदं नृणाम्}
{विष्णुना स समुत्पन्नो विष्णुरित्यभिधीयते}% ७७

\twolineshloka
{एवं नामास्य दत्वाथ वसिष्ठो भगवानृषिः}
{परिणीय नमस्कृत्य स्तुत्वा स्तुतिभिरेव च}% ७८

\twolineshloka
{सङ्कीर्त्य नामसाहस्रं मङ्गलार्थं महात्मनः}
{विनिर्ययौ महातेजास्तस्मात्पुण्यतमाद्गृहात्}% ७९

\twolineshloka
{राजाथ विप्रमुख्येभ्यो ददौ बहुधनं मुदा}
{गवामयुतदानं च कारयामास धर्म्मतः}% ८०

\twolineshloka
{ग्रामाणां शतसाहस्रं ददौ रघुकुलोत्तमः}
{वस्त्रैराभरणैर्दिव्यैरसङ्ख्येयैर्धनैरपि}% ८१

\twolineshloka
{विष्णोस्सन्तुष्टये तत्र तर्प्पयामास भूसुरान्}
{कौसल्या च सुतं दृष्ट्वा रामं राजीवलोचनम्}% ८२

\twolineshloka
{फुल्लहस्तारविन्दाभं पद्महस्ताम्बुजान्वितम्}
{तस्य श्रीपादकमले पद्माब्जे च वरानने}% ८३

\twolineshloka
{शङ्खचक्रगदापद्मध्वजवस्त्रादिचिह्निते}
{दृष्ट्वा वक्षसि श्रीवत्सं कौस्तुभं वनमालया}% ८४

\twolineshloka
{तस्याङ्गे सा जगत्सर्वं सदेवासुरमानुषम्}
{स्मितवक्त्रे विशालाक्षी भुवनानि चतुर्दश}% ८५

\twolineshloka
{निश्वासे तस्य वेदांश्च सेतिहासान्महात्मनः}
{द्वीपानब्धीन्गिरींस्तस्य जघने वरवर्णिनि}% ८६

\twolineshloka
{नाभ्यां ब्रह्मशिवौ तस्य कर्णयोश्च दिशः शुभाः}
{नेत्रयोर्वह्निसूर्यौ च घ्राणे वायुं महाजवम्}% ८७


\threelineshloka
{सर्वोपनिषदामर्थं दृष्ट्वा तस्य विभूतयः}
{कृत्स्ना भीता वरारोहा प्रणम्य च पुनः पुनः}
{हर्षाश्रुपूर्णनयना प्राञ्जलिर्वाक्यमब्रवीत्}% ८८

\uvacha{कौशल्योवाच}

\twolineshloka
{धन्यास्मि देवदेवेश लब्ध्वा त्वां तनयं प्रभो}
{प्रसीद मे जगन्नाथ पुत्रस्नेहं प्रदर्शय}% ८९

\uvacha{ईश्वर उवाच}

\twolineshloka
{एवमुक्तो हृषीकेशो मात्रा सर्वगतो हरिः}
{मायामानुषतां प्राप्य शिशुभावाद्रुरोद सः}% ९०

\twolineshloka
{अथ प्रमुदिता देवी कौशल्या शुभलक्षणा}
{पुत्रमालिङ्ग्य हर्षेण स्तन्यं प्रादात्सुमध्यमा}% ९१

\twolineshloka
{तस्याः स्तन्यं पपौ देवो बालभावात्सनातनः}
{उवास मातुरुत्सङ्गे जगद्भर्ता महाविभुः}% ९२

\twolineshloka
{देशे तस्मिञ्छुशुभे रम्ये सर्वकामप्रदे नृणाम्}
{उत्सवं चक्रिरे पौरा हृष्टा जनपदा नराः}% ९३

\twolineshloka
{कैकेय्यां भरतो जज्ञे पाञ्चजन्यांशचोदितः}
{सुमित्रा जनयामास लक्ष्मणं शुभलक्षणम्}% ९४

\twolineshloka
{शत्रुघ्नं च महाभागा देवशत्रुप्रतापनम्}
{अनन्तांशेन सम्भूतो लक्ष्मणः परवीरहा}% ९५

\twolineshloka
{सुदर्शनांशाच्छत्रुघ्नः सञ्जज्ञेऽमितविक्रमः}
{ते सर्वे ववृधुस्तत्र वैवस्वतमनोः कुले}% ९६

\twolineshloka
{संस्कृतास्ते सुताः सम्यग्वसिष्ठेन महौजसा}
{अधीतवेदास्ते सर्वे श्रुतवन्तस्तथा नृपाः}% ९७

\twolineshloka
{सर्वशास्त्रार्थतत्वज्ञा धनुर्वेदे च निष्ठिताः}
{बभूवुः परमोदारा लोकानां हर्षवर्द्धनाः}% ९८

\twolineshloka
{युग्मं बभूवतुस्तत्र राजानौ रामलक्ष्मणौ}
{तथा भरतशत्रुघ्नौ तयोर्युग्मं बभूव ह}% ९९

\twolineshloka
{अथ लोकेश्वरी लक्ष्मीर्जनकस्य निवेशने}
{शुभक्षेत्रे हलोद्धाते सुनासीरे शुभेक्षणे}% १००

\twolineshloka
{बालार्ककोटिसङ्काशा रक्तोत्पलकराम्बुजा}
{सर्वलक्षणसम्पन्ना सर्वाभरणभूषिता}% १०१

\twolineshloka
{धृत्वा वक्षसि चार्वङ्गी मालामम्लानपङ्कजाम्}
{सीतामुखे समुत्पन्ना बालभावेन सुन्दरी}% १०२

\twolineshloka
{तां दृष्ट्वा जनको राजा कन्यां वेदमयीं शुभाम्}
{उद्धृत्यापत्यभावेन पुपोष मिथिलापतिः}% १०३

\twolineshloka
{जनकस्य गृहे रम्ये सर्वलोकेश्वरप्रिया}
{ववृधे सर्वलोकस्य रक्षणार्थं सुरेश्वरी}% १०४

\twolineshloka
{एतस्मिन्नन्तरे देवि कौशिको लोकविश्रुतः}
{सिद्धाश्रमे महापुण्ये भागीरथ्यास्तटे शुभे}% १०५

\twolineshloka
{क्रतुप्रवरमारेभे यष्टुं तत्र महामुनिः}
{वर्त्तमानस्य तस्यास्य यज्ञस्याथ द्विजन्मनः}% १०६

\twolineshloka
{क्रतुविध्वंसिनोऽभूवन्रावणस्य निशाचराः}
{कौशिकश्चिन्तयित्वाथ रघुवंशोद्भवं हरिम्}% १०७

\twolineshloka
{आनेतुमैच्छद्धर्मात्मा लोकानां हितकाम्यया}
{स गत्वा नगरीं रम्यामयोध्यां रघुपालिताम्}% १०८

\twolineshloka
{नृपश्रेष्ठं दशरथं ददर्श मुनिसत्तमः}
{राजापि कौशिकं दृष्ट्वा प्रत्युत्थाय कृताञ्जलिः}% १०९

\twolineshloka
{पुत्रैः सह महातेजा ववन्दे मुनिसत्तमम्}
{धन्योऽहमस्मीति वदन्हर्षेण रघुनन्दनम्}% ११०

\twolineshloka
{अर्चयामास विधिना निवेश्य परमासने}
{परिणीय नमस्कृत्य किं करोमीत्युवाच तम्}% १११

\onelineshloka*
{ततः प्रोवाच हृष्टात्मा विश्वामित्रो महातपाः}

\uvacha{विश्वामित्र उवाच}
\onelineshloka
{देहि मे राघवं राजन्रक्षणार्थं क्रतोर्मम}% ११२

\twolineshloka
{साफल्यमस्तु मे यज्ञे राघवस्य समीपतः}
{तस्माद्रामं रक्षणार्थं दातुमर्हसि भूपते}% ११३

\uvacha{ईश्वर उवाच}

\twolineshloka
{तच्छ्रुत्वा मुनिवर्य्यस्य वाक्यं सर्वविदां वरः}
{प्रददौ मुनिवर्य्याय राघवं सह लक्ष्मणम्}% ११४

\twolineshloka
{आदाय राघवं तत्र विश्वामित्रो महातपाः}
{स्वमाश्रममभिप्रीतः प्रययौ द्विजसत्तमः}% ११५

\twolineshloka
{ततः प्रहृष्टास्त्रिदशाः प्रयाते रघुसत्तमे}
{ववृषुः पुष्पवर्षाणि तुष्टुवुश्च महौजसः}% ११६

\twolineshloka
{अथाजगाम हृष्टात्मा वैनतेयो महाबलः}
{अदृश्यभूतो भूतानां सम्प्राप्य रघुसत्तमम्}% ११७

\twolineshloka
{ताभ्यां दिव्ये च धनुषी तूणौ चाक्षयसायकौ}
{दिव्यान्यस्त्राणि शस्त्राणि दत्वा च प्रययौ द्विजः}% ११८

\twolineshloka
{तौ रामलक्ष्मणौ वीरौ कौशिकेन महात्मना}
{गच्छन्ती ज्ञापितारण्ये राक्षसी घोरदर्शना}% ११९

\twolineshloka
{नाम्ना तु ताडका देवि भार्या सुन्दस्य रक्षसः}
{जघ्नतुस्तां महावीरौ बाणैर्दिव्यधनुश्च्युतैः}% १२०

\twolineshloka
{निहता राघवेणाथ राक्षसी घोरदर्शना}
{त्यक्त्वा तनुं घोररूपां दिव्यरूपा बभूव सा}% १२१

\twolineshloka
{जाज्वल्यमानावपुषा सर्वाभरणविभूषिता}
{प्रययौ वैष्णवं लोकं प्रणम्य च रघूत्तमौ}% १२२

\twolineshloka
{तां हत्वा राघवः श्रीमान्कौशिकस्याश्रमं शुभम्}
{प्रविवेश महातेजा लक्ष्मणेन महात्मना}% १२३

\twolineshloka
{ततः प्रहृष्टा मुनयः प्रत्युद्गम्य रघूत्तमम्}
{निवेश्य पूजयामासुरर्घाद्यैः परमात्मने}% १२४

\twolineshloka
{कौशिकः कृतदीक्षस्तु यंष्टुं यज्ञमनुत्तमम्}
{आरेभे मुनिभिः सार्द्धं विधिना मुनिसत्तमः}% १२५

\twolineshloka
{वर्त्तमाने महायज्ञे मारीचो नाम राक्षसः}
{भ्रात्रा सुबाहुना तत्र विघ्नं कर्तुमवस्थितः}% १२६

\twolineshloka
{दृष्ट्वा तौ राक्षसौ घोरौ राघवः परवीरहा}
{जघानैकेन बाणेन सुबाहुं राक्षसेश्वरम्}% १२७

\twolineshloka
{पवनास्त्रेण महता मारीचं स निशाचरम्}
{सागरे पातयामास शुष्कपर्णमिवानिलः}% १२८

\twolineshloka
{स रामस्य महावीर्य्यं दृष्ट्वा राक्षससत्तमः}
{न्यस्तशस्त्रस्तपस्तप्तुं प्रययौ महादाश्रमम्}% १२९

\twolineshloka
{विश्वामित्रो महातेजाः समाप्ते महति क्रतौ}
{प्रहृष्टमनसा तत्र पूजयामास राघवम्}% १३०

\twolineshloka
{समाश्लिष्य महात्मानं काकपक्षधरं हरिम्}
{नीलोत्पलदलश्यामं पद्मपत्रायतेक्षणम्}% १३१

\twolineshloka
{उपाघ्राय तदा मूर्ध्नि तुष्टाव मुनिसत्तमः}
{एतस्मिन्नन्तरे राजा मिथिलाया अधीश्वरः}% १३२

\twolineshloka
{वाजपेयं क्रतुं यष्टुमारेभे मुनिसत्तमैः}
{तं द्रष्टुं प्रययुस्सर्वे विश्वामित्रपुरोगमाः}% १३३

\twolineshloka
{मुनयो रघुशार्दूल सहिताः पुण्यचेतसः}
{गच्छतस्तस्य रामस्य पदाब्जेन महात्मनः}% १३४

\twolineshloka
{अभूत्सा पावनी भूमिः समाक्रान्ता महाशिला}
{सापि शप्ता पुरा भर्त्रा गौतमेन द्विजन्मना}% १३५

\twolineshloka
{अहल्या रघुनाथस्य पादस्पर्शाच्छुभाऽभवत्}
{अथ सम्प्राप्य नगरीं मिथिलां मुनिसत्तमाः}% १३६

\twolineshloka
{राघवाभ्यां तु सहिता बभूवुः प्रीतमानसाः}
{समागतान्महाभागान्दृष्ट्वा राजा महाबलः}% १३७

\twolineshloka
{प्रत्युद्गम्य प्रणम्याथ पूजयामास मैथिलः}
{रामं पद्मविशालाक्षमिन्दीवरदलप्रभम्}% १३८

\twolineshloka
{पीताम्बरधरं श्लक्ष्णं कोमलावयवोज्ज्वलम्}
{अवधीरित कन्दर्प्पकोटिलावण्यमुत्तमम्}% १३९

\twolineshloka
{सर्वलक्षणसम्पन्नं सर्वाभरणभूषितम्}
{स्वस्य हृत्पद्ममध्ये यः परेशस्य तनुर्हरिः}% १४०

\twolineshloka
{उत्पन्नो दीपवद्दीपात्सौशील्यादिगुणैः परैः}
{तं दृष्ट्वा रघुनाथं स जनको हृष्टमानसः}% १४१

\twolineshloka
{परेशमेव तं मेने रामं दशरथात्मजम्}
{पूजयामास काकुत्स्थं धन्योस्मीति ब्रुवन्नृपः}% १४२

\twolineshloka
{प्रसादं वासुदेवस्य विष्णोर्मेने नरेश्वरः}
{प्रदातुं दुहितां तस्मै मनसा चिन्तयन्प्रभुः}% १४३

\twolineshloka
{आत्मजौ रघुवंशस्य ज्ञात्वा तत्र नृपोत्तमः}
{पूजयामास धर्मेण वस्त्रैराभरणैः शुभैः}% १४४

\twolineshloka
{ऋषीन्समर्चयामास मधुपर्कादिपूजनैः}
{ततोऽवसाने यज्ञस्य रामो राजीवलोचनः}% १४५

\twolineshloka
{भङ्क्त्वा शैवं धनुर्दिव्यं जितवान्जनकात्मजाम्}
{अथासौ वीर्यशुल्केन महता परितोषितः}% १४६

\twolineshloka
{मुदा धरणिजां तस्मै प्रददौ मिथिलाधिपः}
{केशवाय श्रियमिव यथापूर्वं महार्णवः}% १४७

\twolineshloka
{स दूतं प्रेषयामास राघवं मिथिलाधिपः}
{पुत्राभ्यां सह धर्मात्मा मिथिलायां विवेश ह}% १४८

\twolineshloka
{वसिष्ठवामदेवाद्यैः प्रीतैः सह महीपतिः}
{उवास नगरे रम्ये जनकस्य रघूत्तमः}% १४९

\twolineshloka
{तस्मिन्नेव शुभे काले रामस्य धरणीसुताम्}
{विवाहमकरोद्राजा मिथिलेन समर्चितः}% १५०

\twolineshloka
{लक्ष्मणस्योर्मिलां नाम कन्यां जनकसम्भवाम्}
{जनकस्यानुजस्याथ तनये शुभवर्चसी}% १५१

\twolineshloka
{माण्डवी श्रुतकीर्त्तिश्च सर्वलक्षणलक्षिते}
{भरतस्य च सौमित्रेर्विवाहमकरोन्नृपः}% १५२

\twolineshloka
{निर्वर्त्यौद्वाहिकं तत्र राजा दशरथो बली}
{अयोध्यां प्रस्थितः श्रीमान्पौरैर्जनपदैर्वृतः}% १५३

\twolineshloka
{पारिबर्हं समादाय मैथिलेन च पूजितः}
{ससुतः सस्नुषः साश्वः सगजः सबलानुगः}% १५४

\twolineshloka
{तदध्वनि महावीर्य्यो जामदग्निः प्रतापवान्}
{गृहीत्वा परशुं चापं सङ्क्रुद्ध इव केसरी}% १५५

\twolineshloka
{अभ्यधावच्च काकुत्स्थं योद्धुकामो नृपान्तकः}
{सम्प्राप्य राघवं दृष्ट्वा वचनं प्राह भार्गवः}% १५६

\uvacha{परशुराम उवाच}

\twolineshloka
{रामराम महाबाहो शृणुष्व वचनं मम}
{बहुशः पार्थिवान्हत्वा संयुगे भूरिविक्रमान्}% १५७

\twolineshloka
{ब्राह्मणेभ्यो महीं दत्वा तपस्तप्तुमहं गतः}
{तव वीर्यबलं श्रुत्वा त्वया योद्धुमिहागतः}% १५८

\twolineshloka
{इक्ष्वाकवो न वध्या मे मातामहकुलोद्भवाः}
{वीर्य्यं क्षत्रबलं श्रुत्वा न शक्यं सहितुं मम}% १५९

\twolineshloka
{रौद्रं चापं दुराधर्षं भज्यमानां त्वया नृप}
{तस्माद्वदान्य युद्धं मे दीयतां रघुसत्तम}% १६०

\twolineshloka
{इदं तु वैष्णवं चापं तेन तुल्यमरिन्दम}
{आरोपय स्ववीर्येण निर्जितोस्मि त्वयैव हि}% १६१

\twolineshloka
{अथवा त्यज शस्त्राणि पुरस्ताद्बलिनो मम}
{शरणं भज काकुत्स्थ कातरोस्यथ चेतसी}% १६२

\uvacha{ईश्वर उवाच}

\twolineshloka
{एवमुक्तस्तु काकुत्स्थो भार्गवेण प्रतापवान्}
{तच्चापं तस्य जग्राह तच्छक्तिं वैष्णवीमपि}% १६३

\twolineshloka
{शक्त्या वियुक्तस्स तदा जामदग्निः प्रतापवान्}
{निर्वीर्यो नष्टतेजाभूत्कर्म्महीनो यथा द्विजः}% १६४

\twolineshloka
{विनष्टतेज सन्दृष्ट्वा भार्गवं नृपसत्तमाः}
{साधुसध्विति काकुत्स्थं प्रशशंसुर्मुहुर्मुहुः}% १६५

\twolineshloka
{काकुत्स्थस्तन्महच्चापं गृहीत्वारोप्य लीलया}
{सन्धाय बाणं तच्चापे भार्गवं प्राह विस्मितम्}% १६६

\uvacha{राम उवाच}

\twolineshloka
{अनेन शरमुख्येन किं कर्त्तव्यं तव द्विज}
{छेद्मि लोकमिमं चाधः स्वर्गं वा हन्मि ते तपः}% १६७

\uvacha{ईश्वर उवाच}

\twolineshloka
{तन्दृष्ट्वा घोरसङ्काशं बाणं रामस्य भार्गवः}
{ज्ञात्वा तं परमात्मानं प्रहृष्टो राममब्रवीत्}% १६८

\uvacha{परशुराम उवाच}

\twolineshloka
{रामराम महाबाहो न वेद्मि त्वां सनातनम्}
{जानाम्यद्यैव काकुत्स्थ तव वीर्य्यगुणादिभिः}% १६९

\twolineshloka
{त्वमादिपुरुषः साक्षात्परब्रह्मपरोऽव्ययः}
{त्वमनन्तो महाविष्णुर्वासुदेवः परात्परः}% १७०

\twolineshloka
{नारायणस्त्वं श्रीशस्त्वमीश्वरस्त्वं त्रयीमयः}
{त्वं कालस्त्वं जगत्सर्वमकाराख्यस्त्वमेव च}% १७१

\twolineshloka
{स्रष्टा धाता च संहर्त्ता त्वमेव परमेश्वरः}
{त्वमचिन्त्यो महद्भूतरूपस्त्वं तु मनुर्महान्}% १७२

\twolineshloka
{चतुःषट्पञ्चगुणवांस्त्वमेव पुरुषोत्तमः}
{त्वं यज्ञस्त्वं वषट्कारस्त्वमोङ्कारस्त्रयीमयः}% १७३

\twolineshloka
{व्यक्ताव्यक्तस्वरूपस्त्वं गुणभृन्निर्ग्गुणः परः}
{स्तोतुं त्वाहमशक्तश्च वेदानामप्यगोचरम्}% १७४

\twolineshloka
{यच्चापलत्वं कृतवांस्त्वां युयुत्सुतया प्रभो}
{तत्क्षन्तव्यं त्वया नाथ कृपया केवलेन तु}% १७५

\twolineshloka
{तव शक्त्या नृपान्सर्वाञ्जित्वा दत्वा महीं द्विजान्}
{त्वत्प्रसादवशादेव शान्तिमाप्नोति नैष्ठिकीम्}% १७६

\uvacha{ईश्वर उवाच}

\twolineshloka
{एवमुक्त्वा तु काकुत्स्थं जामदग्निर्महातपाः}
{परिणीय नमस्कृत्वा राघवं लोकरक्षकम्}% १७७

\twolineshloka
{शतक्रतुकृतं स्वर्गं तदस्त्राय न्यवेदयत्}
{राघवोऽथ महातेजा ववन्दे तं महामुनिम्}% १७८

\twolineshloka
{विधिवत्पूजयामास पाद्यार्घाचमनादिभिः}
{तेन सम्पूजितस्तत्र जामदग्निर्महातपाः}% १७९

\twolineshloka
{तपस्तप्तुं ययौ रम्यं नरनारायणाश्रमम्}
{राजा दशरथः सोऽथ पुत्रैर्दारसमन्वितैः}% १८०

\twolineshloka
{स्वां पुरीं सुमुहूर्त्तेन प्रविवेश महाबलः}
{राघवो लक्ष्मणश्चैव शत्रुघ्नो भरतस्तथा}% १८१

\twolineshloka
{स्वान्स्वान्दारानुपागम्य रेमिरे हृष्टमानसाः}
{तत्र द्वादश वर्षाणि सीतया सह राघवः}% १८२

\twolineshloka
{रमयामास धर्मात्मा नारायण इव श्रिया}
{तस्मिन्नेव तु राजाथ काले दशरथः सुतम्}% १८३

\twolineshloka
{ज्येष्ठं राज्येन संयोक्तुमैच्छत्प्रीत्या महीपतिः}
{तस्य भार्याथ कैकेयी पुरा दत्तवरा प्रिया}% १८४

\twolineshloka
{अयाचत नृपश्रेष्ठं भरतस्याभिषेचनम्}
{विवासनं च रामस्य वत्सराणि चतुर्दश}% १८५

\twolineshloka
{स राजा सत्यवचनाद्रामं राज्यादथोः सुतम्}
{विवासयामास तदा दुःखेन हतचेतनः}% १८६

\twolineshloka
{शक्तोऽपि राघवस्तस्मिन्राज्यं सन्त्यज्य धर्मतः}
{दशग्रीववधार्थाय पितुर्वचनहेतुना}% १८७

\twolineshloka
{वनं जगाम काकुत्स्थो लक्ष्मणेन च सीतया}
{राजा पुत्रवियोगार्त्तः शोकेन च ममार सः}% १८८

\twolineshloka
{नियुज्यमानो भरतस्तस्मिन्राज्ये समन्त्रिभिः}
{नैच्छद्राज्यं स धर्म्मात्मा सौभ्रात्रमनुदर्शयन्}% १८९

\twolineshloka
{वनमागम्य काकुत्स्थमयाचद्भ्रातरं ततः}
{रामस्तु पितुरादेशान्नैच्छद्राज्यमरिन्दमः}% १९०

\twolineshloka
{स्वपादुके ददौ तस्मै भक्त्या सोऽप्यग्रहीत्तथा}
{रामस्य पादुके राज्यमवाप्य भरतः शुभे}% १९१

\twolineshloka
{प्रत्यहं गन्धपुष्पैश्च पूजयन्कैकयीसुतः}
{तपश्चरणयुक्तेन तस्मिंस्तस्थौ नृपोत्तमः}% १९२

\twolineshloka
{यावदागमनं तस्य राघवस्य महात्मनः}
{तावद्व्रतपराः सर्वे बभूवुः पुरवासिनः}% १९३

\twolineshloka
{राघवश्चित्रकूटाद्रौ भरद्वाजाश्रमे शुभे}
{रमयामास वैदेह्या मन्दाकिन्या जले शुभे}% १९४

\twolineshloka
{कदाचिदङ्के वैदेह्याः शेते रामो महामनाः}
{ऐन्द्रिः काकस्समागम्य तस्मिन्नेव चचार ह}% १९५

\twolineshloka
{स दृष्ट्वा जानकीं तत्र कन्दर्प्पशरपीडितः}
{विददार नखैस्तीक्ष्णैः पीनोन्नतपयोधरम्}% १९६

\twolineshloka
{तं दृष्ट्वा वायसं रामः कुशं जग्राह पाणिना}
{ब्रह्मणास्त्रेण संयोज्य चिक्षेप धरणीधरः}% १९७

\twolineshloka
{तं तृणं घोरसङ्काशं ज्वालारचितविग्रहम्}
{दृष्ट्वा काकः प्रदुद्राव विमुञ्चन्कातरं स्वरम्}% १९८

\twolineshloka
{तं काकं प्रत्यनुययौ रामस्यास्त्रं सुदारुणम्}
{वायसस्त्रिषुलोकेषु बभ्राम भयपीडितः}% १९९

\twolineshloka
{यत्र यत्र ययौ काकः शरणार्थी स वायसः}
{तत्र तत्र तदस्त्रं तु प्रविवेश भयावहम्}% २००

\twolineshloka
{ब्रह्माणमिन्द्रं रुद्रं च यमं वरुणमेव च}
{शरणार्थी जगामाशु वायसः शस्त्रपीडितः}% २०१


\threelineshloka
{तं दृष्ट्वा वायसं सर्वे रुद्राद्या देव दानवाः}
{न शक्ताः स्म वयं त्रातुमिति प्राहुर्मनीषिणः}
{अथ प्रोवाच भगवान्ब्रह्मा त्रिभुवनेश्वरः}% २०२

\uvacha{ब्रह्मोवाच}

\twolineshloka
{भो भो बलिभुजां श्रेष्ठ तमेव शरणं व्रज}
{स एव रक्षकः श्रीमान्सर्वेषां करुणानिधिः}% २०३

\twolineshloka
{रक्षत्येव क्षमासारो वत्सलं शरणागतान्}
{ईश्वरः सर्वभूतानां सौशील्यादिगुणान्वितः}% २०४

\twolineshloka
{रक्षिता जीवलोकस्य पिता माता सखा सुहृत्}
{शरणं व्रज देवेशं नान्यत्र शरणं द्विज}% २०५

\uvacha{महादेव उवाच}

\twolineshloka
{इत्युक्तस्तेन बलिभुग्ब्रह्मणा रघुनन्दनम्}
{उपेत्य सहसा भूमौ निपपात भयातुरः}% २०६

\twolineshloka
{प्राणसंशयमापन्नं दृष्ट्वा सीताथ वायसम्}
{त्राहित्राहीति भर्तारमुवाच विनयाद्विभुम्}% २०७

\twolineshloka
{पुरतः पतितं देवी धरण्यां वायसं तदा}
{तच्छिरः पादयोस्तस्य योजयामास जानकी}% २०८

\twolineshloka
{समुत्थाप्य करेणाथ कृपापीयूषसागरः}
{ररक्ष रामो गुणवान् वायसं दययार्दितः}% २०९

\twolineshloka
{तमाह वायसं रामो मा भैरिति दयानिधिः}
{अभयं ते प्रदास्यामि गच्छ गच्छ यथासुखम्}% २१०

\twolineshloka
{प्रणम्य राघवायाथ सीतायै च मुहुर्मुहुः}
{स्वर्ल्लोकं प्रययावाशु राघवेण च रक्षितः}% २११

\twolineshloka
{ततो रामस्तु वैदेह्या लक्ष्मणेन च धीमता}
{उवास चित्रकूटाद्रौ स्तूयमानो महर्षिभिः}% २१२

\twolineshloka
{तस्मिन्सम्पूज्यमानस्तु भरद्वाजेन राघवः}
{जगामात्रेस्सुविपुलमाश्रमं रघुसत्तमः}% २१३

\twolineshloka
{समागतं रघुवरं दृष्ट्वा मुनिवरोत्तमः}
{भार्यया सह धर्म्मात्मा प्रत्युद्गम्य मुदा युतः}% २१४

\twolineshloka
{आसने सुशुभे मुख्ये निवेश्य सह सीतया}
{अर्घ्यपाद्याचमनीयं च वस्त्राणि विविधानि च}% २१५

\twolineshloka
{मधुपर्कन्ददौ प्रीत्या भूषणं चानुलेपनम्}
{तस्य पत्न्यनसूया तु दिव्याम्बरमनुत्तमम्}% २१६

\twolineshloka
{सीतायै प्रददौ प्रीत्या भूषणानि द्युमन्ति च}
{दिव्यान्नपानभक्षाद्यैर्भोजयामास राघवम्}% २१७

\twolineshloka
{तेन सम्पूजितस्तत्र भक्त्या परमया नृपः}
{उवास दिवसं तत्र प्रीत्या रामस्सलक्ष्मणः}% २१८

\twolineshloka
{प्रभाते विमले रामः समुत्थाय महामुनिम्}
{परिणीय प्रणम्याथ गमनायोपचक्रमे}% २१९

\twolineshloka
{अनुज्ञातस्ततस्तेन रामो राजीवलोचनः}
{प्रययौ दण्डकारण्यं महर्षिकुलसङ्कुलम्}% २२०

\twolineshloka
{तत्रातिभीषणं घोरं विराधं नाम राक्षसम्}
{हत्वाथ शरभङ्गस्य प्रविवेशाश्रमं शुभम्}% २२१

\twolineshloka
{स तु दृष्ट्वाथ काकुत्स्थं सद्यः सङ्क्षीणकल्मषः}
{प्रययौ ब्रह्मलोकं तु गन्धर्वाप्सरसान्वितम्}% २२२

\twolineshloka
{सुतीक्ष्णस्याप्यगस्त्यस्य ह्यगस्त्यभ्रातुरेव च}
{क्रमेण प्रययौ रामस्तैश्च सम्पूजितस्तथा}% २२३

\twolineshloka
{पञ्चवट्यां ततो रामो गोदावर्यास्तटे शुभे}
{उवास सुचिरं कालं सुखेन परमेण च}% २२४

\twolineshloka
{तत्र गत्वा मुनिश्रेष्ठास्तापसा धर्मचारिणः}
{पूजयामासुरात्मेशं रामं राजीवलोचनम्}% २२५

\twolineshloka
{भयं विज्ञापयामासुस्तं च रक्षोगणेरितम्}
{तानाश्वास्य तु काकुस्थो ददौ चाभयदक्षिणाम्}% २२६

\twolineshloka
{ते तु सम्पूजितास्तेन स्वाश्रमान्सम्प्रपेदिरे}
{तस्मिंस्त्रयोदशाब्दानि रामस्य परिनिर्य्ययुः}% २२७

\twolineshloka
{गोदावर्य्यास्तटे रम्ये पञ्चवट्यां मनोरमे}
{कस्यचित्त्वथ कालस्य राक्षसी घोररूपिणी}% २२८

\twolineshloka
{रावणस्य स्वसा तत्र प्रविवेश दुरासदा}
{सा तु दृष्ट्वा रघुवरं कोटिकन्दर्प्पसन्निभम्}% २२९

\twolineshloka
{इन्दीवरदलश्यामं पद्मपत्रायतेक्षणम्}
{प्रोन्नतांसं महाबाहुं कम्बुग्रीवं महाहनुम्}% २३०

\twolineshloka
{सम्पूर्णचन्द्रसदृशं सस्मिताननपङ्कजम्}
{भृङ्गावलिनिभैः स्निग्धैः कुटिलैः शीर्षजैर्वृतम्}% २३१

\twolineshloka
{रक्तारविन्दसदृशं पद्महस्ततलाङ्कितम्}
{निष्कलङ्केन्दुसदृशं नखपङ्क्तिविराजितम्}% २३२

\twolineshloka
{स्निग्धकोमलदूर्वाभं सौकुमार्य्यनिधिं शुभम्}
{पीतकौशेयवसनं सर्वाभरणभूषितम्}% २३३

\twolineshloka
{युवाकुमारवयसं जगन्मोहनविग्रहम्}
{दृष्ट्वा तं राक्षसी रामं कन्दर्प्पशरपीडिता}% २३४

\onelineshloka*
{अब्रवीत्समुपेत्याथ रामं कमललोचनम्}

\uvacha{राक्षस्युवाच}
\onelineshloka
{कस्त्वं तापसवेषेण वर्त्तसे दण्डके वने}% २३५

\twolineshloka
{आगतोऽसि किमर्थं च राक्षसानां दुरासदे}
{शीघ्रमाचक्ष्व तत्त्वेन नानृतं वक्तुमर्हसि}% २३६

\uvacha{महेश्वर उवाच}

\onelineshloka*
{इत्युक्तः स तदा रामः सम्प्रहस्याब्रवीद्वचः}

\uvacha{राम उवाच}

\twolineshloka
{राज्ञो दशरथस्याहं पुत्रो राम इतीरितः}
{असौ ममानुजो धन्वी लक्ष्मणो नाम चानघः}% २३७

\twolineshloka
{पत्नी चेयं च मे सीता जनकस्यात्मजा प्रिया}
{पितुर्वचननिर्देशादहं वनमिहागतः}% २३८

\twolineshloka
{विचरामो महारण्यमृषीणां हितकाम्यया}
{आगतासि किमर्थं त्वमाश्रमं मम सुन्दरि}% २३९

\onelineshloka*
{का त्वं कस्य कुले जाता सर्वं सत्यं वदस्व मे}

\uvacha{महेश्वर उवाच}
\onelineshloka
{इत्युक्ता सा तु रामेण प्राह वाक्यमशङ्किता}% २४०

\uvacha{राक्षस्युवाच}

\twolineshloka
{अहं विश्रवसः पुत्री रावणस्य स्वसा नृप}
{नाम्ना शूर्पणखा नाम त्रिषु लोकेषु विश्रुता}% २४१

\twolineshloka
{इदं च दण्डकारण्यं भ्रात्रा दत्तं मम प्रभो}
{भक्षयन्नृषिसङ्घान्वै विचरामि महावने}% २४२

\twolineshloka
{त्वां तु दृष्ट्वा मुनिवरं कन्दर्पशरपीडिता}
{रन्तुकामा त्वया सार्द्धमागतास्मि सुनिर्भया}% २४३

\twolineshloka
{मम त्वं नृपशार्दूल भर्ता भवितुमर्हसि}
{इमां तव सतीं सीतां ग्रसितुं भूप कामये}% २४४

\onelineshloka*
{वनेषु गिरिमुख्येषु रमयामि त्वया सह}

\uvacha{महेश्वर उवाच}

\onelineshloka
{इत्युक्त्वा राक्षसी सीतां ग्रसितुं वीक्ष्य चोद्यताम्}% २४५

\onelineshloka
{श्रीरामः खड्गमुद्यम्य नासाकर्णौ प्रचिच्छिदे}% २४६

\twolineshloka
{रुदन्ती सभयं शीघ्रं राक्षसी विकृतानना}
{खरालयं प्रविश्याह तस्य रामस्य चेष्टितम्}% २४७

\twolineshloka
{स तु राक्षससाहस्रैर्दूषणत्रिशिरो वृतः}
{आजगाम भृशं योद्धुं राघवं शत्रुसूदनः}% २४८

\twolineshloka
{तान्रामः कानने घोरे बाणः कालान्तकोपमैः}
{निजघान महाकायान्राक्षसांस्तत्र लीलया}% २४९

\twolineshloka
{खरं त्रिशिरसं चैव दूषणं तु महाबलम्}
{रणे निपातयामास बाणैराशीविषोपमैः}% २५०

\twolineshloka
{निहत्य राक्षसान्सर्वान्दण्डकारण्यवासिनः}
{पूजितः सुरसङ्घैश्च स्तूयमानो महर्षिभिः}% २५१

\twolineshloka
{उवास दण्डकारण्ये सीतया लक्ष्मणेन च}
{राक्षसानां वधं श्रुत्वा रावणः क्रोधमूर्च्छितः}% २५२

\twolineshloka
{आजगाम जनस्थानं मारीचेन दुरात्मना}
{सम्प्राप्य पञ्चवट्यां तु दशग्रीवः स राक्षसः}% २५३

\twolineshloka
{मायाविना मरीचेन मृगरूपेण रक्षसः}
{अपहृत्याश्रमाद्दूरे तौ तु दशरथात्मजौ}% २५४

\twolineshloka
{जहार सीतां रामस्य भार्यां स्ववधकाङ्क्षया}
{ह्रियमाणां तु तां दृष्ट्वा जटायुर्गृध्रराड्बली}% २५५

\twolineshloka
{रामस्य सौहृदात्तत्र युयुधे तेन रक्षसा}
{तं हत्वा बाहुवीर्येण रावणं शत्रुवारणः}% २५६

\twolineshloka
{प्रविवेश पुरीं लङ्कां राक्षसैर्बहुभिर्वृताम्}
{अशोकवनिकामध्ये निःक्षिप्य जनकात्मजाम्}% २५७

\twolineshloka
{निधनं रामबाणेन काङ्क्षयन्स्वगृहं विशत्}
{रामस्तु राक्षसं हत्वा मारीचं मृगरूपिणम्}% २५८

\twolineshloka
{पुनराविश्य तत्राथ भ्रात्रा सौमित्रिणा ततः}
{राक्षसापहृतां भार्यां ज्ञात्वा दशरथात्मजः}% २५९

\twolineshloka
{प्रभूतशोकसन्तप्तो विललाप महामतिः}
{मार्गमाणो वने सीतां पथि गृध्रं महाबलम्}% २६०

\twolineshloka
{विच्छिन्नपादपक्षं च पतितं धरणीतले}
{रुधिरापूर्णसर्वाङ्गं दृष्ट्वा विस्मयमागतः}% २६१

\twolineshloka
{पप्रच्छ राघवं श्रीमान्केन किं त्वं जिघांसितः}
{गृध्रस्तु राघवं दृष्ट्वा मन्दमन्दमुवाच ह}% २६२

\uvacha{गृध्र उवाच}

\twolineshloka
{रावणेन हृता राम तव भार्यां बलीयसा}
{तेन राक्षसमुख्येन सङ्ग्रामे निहतोस्म्यहम्}% २६३

\uvacha{महेश्वर उवाच}

\twolineshloka
{इत्युक्त्वा राघवस्याग्रे सहसा त्यक्तजीवितः}
{संस्कारमकरोद्रामस्तस्य ब्रह्मविधानतः}% २६४

\twolineshloka
{स्वपदं च ददौ तस्मै योगिगम्यं सनातनम्}
{राघवस्य प्रसादेन स गृध्रः परमं पदम्}% २६५

\twolineshloka
{हरेः सामान्यरूपेण मुक्तिं प्राप खगोत्तमः}
{माल्यवन्तं ततो गत्वा मतङ्गस्याश्रमे शुभे}% २६६

\twolineshloka
{अभिगम्य महाभागां शबरीं धर्मचारिणीम्}
{सा तु भागवतश्रेष्ठा दृष्ट्वा तौ रामलक्ष्मणौ}% २६७

\twolineshloka
{प्रत्युद्गम्य नमस्कृत्वा निवेश्य कुशविष्टरे}
{पादप्रक्षालनं कृत्वा वन्यैः पुष्पैः सुगन्धिभिः}% २६८

\twolineshloka
{अर्चयामास भक्त्या वै हर्षनिर्भरमानसा}
{फलानि च सुगन्धीनि मूलानि मधुराणि च}% २६९

\twolineshloka
{निवेदयामास तदा राघवाभ्यां दृढव्रता}
{फलान्यास्वाद्य काकुत्स्थस्तस्यै मुक्तिं ददौ पराम्}% २७०

\twolineshloka
{ततः पम्पासरो गत्वा राघवः शत्रुसूदनः}
{जघान राक्षसं तत्र कबन्धं घोररूपिणम्}% २७१

\twolineshloka
{तं निहत्य महावीर्यो ददाह स्वर्गतश्च सः}
{ततो गोदावरीं गत्वा रामो राजीवलोचनः}% २७२

\twolineshloka
{पप्रच्छ सीतां गङ्गे त्वं किं तां जानासि मे प्रियाम्}
{न शशंस तदा तस्मै सा गङ्गा तमसावृता}% २७३

\twolineshloka
{शशाप राघवः क्रोधाद्रक्ततोया भवेति ताम्}
{ततो भयात्समुद्विग्ना पुरस्कृत्य महामुनीन्}% २७४

\twolineshloka
{कृताञ्जलिपुटा दीना राघवं शरणं गता}
{ततो महर्षयस्सर्वे रामं प्राहुस्सनातनम्}% २७५

\uvacha{ऋषय ऊचुः}

\twolineshloka
{त्वत्पादकमलोद्भूता गङ्गा त्रैलोक्यपावनी}
{त्वमेव हि जगन्नाथ तां शापान्मोक्तुमर्हसि}% २७६

\uvacha{महेश्वर उवाच}

\onelineshloka*
{ततः प्रोवाच धर्मात्मा रामः शरणवत्सलः}

\uvacha{राम उवाच}

\twolineshloka
{शबर्याः स्नानमात्रेण सङ्गता शुभवारिणा}
{मुक्ता भवतु मच्छापाद्गङ्गेयं पापनाशिनी}% २७७

\twolineshloka
{एवमुक्त्वा तु काकुत्स्थः शबरीतीर्थमुत्तमम्}
{गङ्गा गयासमं चक्रे शार्ङ्गकोट्या महाबलः}% २७८

\twolineshloka
{महाभागवतानां च तीर्थं यस्योदकेऽभवत्}
{तच्छरीरं जगद्वन्द्यं भविष्यति न संशयः}% २७९

\twolineshloka
{एवमुक्त्वा तु काकुत्स्थ ऋष्यमूकं गिरिं ययौ}
{ततः पम्पासरस्तीरे वानरेण हनूमता}% २८०

\twolineshloka
{सङ्गतस्तस्य वचनात्सुग्रीवेण समागतः}
{सुग्रीववचनाद्धत्वा वालिनं वानरेश्वरम्}% २८१

\twolineshloka
{सुग्रीवमेव तद्राज्ये रामोसावभ्यषेचयत्}
{स तु सम्प्रेषयामास दिदृक्षुर्जनकात्मजाम्}% २८२

\twolineshloka
{हनुमत्प्रमुखान्वीरान्वानरान्वानराधिपः}
{स लङ्घयित्वा जलधिं हनूमान्मारुतात्मजः}% २८३

\twolineshloka
{प्रविश्य नगरीं लङ्कां दृष्ट्वा सीतां दृढव्रताम्}
{उपवासकृशां दीनां भृशं शोकपरायणाम्}% २८४

\twolineshloka
{मलपङ्केन दिग्धाङ्गीं मलिनाम्बरधारिणीम्}
{निवेदयित्वाऽभिज्ञानं प्रवृत्तिं च निवेद्य ताम्}% २८५

\twolineshloka
{सप्तमन्त्रिसुतांस्तत्र रावणस्य सुतं तथा}
{तोरणस्तम्भमुत्पाट्य निजघान स्वयं कपिः}% २८६

\twolineshloka
{समाश्वास्य च वैदेहीं बभञ्जोपवनं तदा}
{वनपालान्किङ्करांश्च पञ्चसेनाग्रनायकान्}% २८७

\twolineshloka
{रावणस्य सुतेनाथ निगृहीतो यदृच्छया}
{दृष्ट्वा च राक्षसेन्द्रं तु सम्भाषित्वा तथैव च}% २८८

\twolineshloka
{ददाह नगरीं लङ्कां स्वलाङ्गूलाग्निना कपिः}
{तया दत्तमभिज्ञानं गृहीत्वा पुनरागमत्}% २८९

\twolineshloka
{सोऽभिगम्य महातेजा रामं कमललोचनम्}
{न्यवेदयद्वानरेन्द्रो दृष्टा सीतेति तत्वतः}% २९०

\twolineshloka
{सुग्रीवसहितो रामो वानरैर्बहुभिर्वृतः}
{महोदधेस्तटं गत्वा तत्रानीकं न्यवेशयत्}% २९१

\twolineshloka
{रावणस्यानुजो भ्राता विभीषण इतीरितः}
{धर्मात्मा सत्यसन्धश्च महाभागवतोत्तमः}% २९२

\twolineshloka
{ज्ञात्वा समागतं रामं परित्यज्य स्वपूर्वजम्}
{राज्यं सुतांश्च दारांश्च राघवं शरणं ययौ}% २९३

\twolineshloka
{परिगृह्य च तं रामो मारुतेर्वचनात्प्रभुः}
{तस्मै दत्वाऽभयं सौम्यं रक्षो राज्येऽभ्यषेचयत्}% २९४

\twolineshloka
{ततस्समुद्रं काकुत्स्थस्तर्तुकामः प्रपद्य वै}
{सुप्रसन्नजलं तं तु दृष्ट्वा रामो महाबलः}% २९५

\twolineshloka
{शार्ङ्गमादाय बाणौघैः शोषयामास वारिधिम्}
{ततस्तु सरितामीशः काकुत्स्थं करुणानिधिम्}% २९६

\twolineshloka
{प्रपद्य शरणं देवमर्चयामास वारिधिः}
{पुनरापूर्य जलधिं वरुणास्त्रेण राघवः}% २९७

\twolineshloka
{उदधेर्वचनात्सेतुं सागरे मकरालये}
{गिरिभिर्वानरानीतैर्नलः सेतुमकारयत्}% २९८

\twolineshloka
{ततो गत्वा पुरीं लङ्कां सन्निवेश्य महाबलम्}
{सम्यगायोधनं चक्रे वानराणां च रक्षसाम्}% २९९

\twolineshloka
{ततो दशास्यतनयः शक्रजिद्राक्षसो बली}
{बबन्ध नागपाशैश्च तावुभौ रामलक्ष्मणौ}% ३००

\twolineshloka
{वैनतेयः समागत्य तान्यस्त्राणि प्रमोचयत्}
{राक्षसा निहतास्सर्वे वानरैश्च महाबलैः}% ३०१

\twolineshloka
{रावणस्यानुजं वीरं कुम्भकर्णं महाबलम्}
{निजघान रणे रामो बाणैरग्निशिखोपमैः}% ३०२

\twolineshloka
{ब्रह्मास्त्रेणेन्द्रजित्क्रुद्धः पातयामास वानरान्}
{हनूमता समानीतो महौषधि महीधरः}% ३०३

\twolineshloka
{तस्यानीतस्य च स्पर्शात्सर्व एव समुत्थिताः}
{ततो रामानुजो वीरः शक्रजेतारमाहवे}% ३०४

\twolineshloka
{निपातयामास शरैर्वृत्रं वज्रधरो यथा}
{निर्ययावथ पौलस्त्यो योद्धुं रामेण संयुगे}% ३०५

\twolineshloka
{चतुरङ्गबलैः सार्द्धं मन्त्रिभिश्च महाबलः}
{समन्ततोभवद्युद्धं वानराणां च रक्षसाम्}% ३०६

\twolineshloka
{रामरावणयोश्चैव तथा सौमित्रिणा सह}
{शक्त्या निपातयामास लक्ष्मणं राक्षसेश्वरः}% ३०७

\twolineshloka
{ततः क्रुद्धो महातेजा राघवो राक्षसान्तकः}
{जघान राक्षसान्वीराञ्शरैः कालान्तकोपमैः}% ३०८

\twolineshloka
{प्रदीप्तैर्बाणसाहस्रैः कालदण्डोपमैर्भृशम्}
{छादयामास काकुत्स्थो दशग्रीवं च राक्षसम्}% ३०९

\twolineshloka
{स तु निर्भिन्नसर्वाङ्गो राघवास्त्रैर्निशाचरः}
{भयात्प्रदुद्राव रणाल्लङ्कां प्रति निशाचरः}% ३१०

\twolineshloka
{जगद्राममयं पश्यन्निर्वेदाद्गृहमाविशत्}
{ततो हनूमता नीतो महौषधिमहागिरिः}% ३११

\twolineshloka
{तेन रामानुजस्तूर्णं लब्धसंज्ञोऽभवत्तदा}
{दशग्रीवस्ततो होममारेभे जयकाङ्क्षया}% ३१२

\twolineshloka
{ध्वंसितं वानरेन्द्रैस्तदभिचारात्मकं रिपोः}
{पुनर्युद्धाय पौलस्त्यो रामेण सह निर्ययौ}% ३१३

\twolineshloka
{दिव्यस्यन्दनमारुह्य राक्षसैर्बहुभिर्युतः}
{ततः शतमखो दिव्यं रथं हर्यश्वसंयुतम्}% ३१४

\twolineshloka
{राघवाय ससूतं हि प्रेषयामास बुद्धिमान्}
{रथं मातलिना नीतं समारुह्य रघूत्तमः}% ३१५

\twolineshloka
{स्तूयमानं सुरगणैर्युयुधे तेन रक्षसा}
{ततो युद्धमभूद्धोरं रामरावणयोर्महत्}% ३१६

\twolineshloka
{सप्ताह्निकमहोरात्रं शस्त्रास्त्रैरतिभीषणम्}
{विमानस्थाः सुरास्सर्वे ददृशुस्तत्र संयुगम्}% ३१७

\twolineshloka
{दशग्रीवस्य चिच्छेद शिरांसि रघुसत्तमः}
{समुत्थितानि बहुशो वरदानात्कपर्दिनः}% ३१८

\twolineshloka
{ब्राह्ममस्त्रं महारौद्रं वधायास्य दुरात्मनः}
{ससर्ज राघवस्तूर्णं कालाग्निसदृशप्रभम्}% ३१९

\twolineshloka
{तदस्त्रं राघवोत्सृष्टं रावणस्य स्तनान्तरम्}
{विदार्य धरणीं भित्त्वा रसातलतले गतम्}% ३२०

\twolineshloka
{सम्पूज्यमानं भुजगै राघवस्य करं ययौ}
{स गतासुर्महादैत्यः पपात च ममार च}% ३२१

\twolineshloka
{ततो देवगणास्सर्वे हर्षनिर्भरमानसाः}
{ववृषुः पुष्पवर्षाणि महात्मनि जगद्गुरौ}% ३२२

\twolineshloka
{जगुर्गन्धर्वपतयो ननृतुश्चाप्सरोगणाः}
{ववुः पुण्यास्तथा वाताः सुप्रभोऽभूद्दिवाकरः}% ३२३

\twolineshloka
{तुष्टुवुर्मुनयः सिद्धा देवगन्धर्वकिन्नराः}
{लङ्कायां राक्षसश्रेष्ठमभिषिच्य विभीषणम्}% ३२४

\twolineshloka
{कृतकृत्यमिवात्मानं मेने रघुकुलोत्तमः}
{रामस्तत्राब्रवीद्वाक्यमभिषिच्य विभीषणम्}% ३२५

\uvacha{राम उवाच}

\twolineshloka
{यावच्चन्द्रश्च सूर्यश्च यावत्तिष्ठति मेदिनी}
{यावन्ममकथालोके तावद्राज्यं विभीषणे}% ३२६

\twolineshloka
{गत्वा मम पदं दिव्यं योगिगम्यं सनातनम्}
{सपुत्रपौत्रः सगणः सम्प्राप्नुहि महाबलः}% ३२७

\uvacha{ईश्वर उवाच}

\twolineshloka
{एवं दत्वा वरं तस्मै राक्षसाय महाबलः}
{सम्प्राप्य मैथिलीं तत्र परुषं जनसंसदि}% ३२८

\twolineshloka
{उवाच राघवः सीतां गर्हितं वचनं बहु}
{सा तेन गर्हिता साध्वी विवेश चानलं महत्}% ३२९


\threelineshloka
{ततो देवगणास्सर्वे शिवब्रह्मपुरोगमाः}
{दृष्ट्वा तु मातरं वह्नौ प्रविशन्तीं भयातुराः}
{समागम्य रघुश्रेष्ठं सर्वे प्राञ्जलयोऽब्रुवन्}% ३३०

\uvacha{देवा ऊचुः}

\twolineshloka
{रामराम महाबाहो शृणु त्वं चातिविक्रम}
{सीतातिविमला साध्वी तव नीत्यानपायिनी}% ३३१

\twolineshloka
{अत्याज्या तु वृथा सा हि भास्करेण प्रभा यथा}
{सेयं लोकहितार्थाय समुत्पन्ना महीतले}% ३३२

\twolineshloka
{माता सर्वस्य जगतः समस्तजगदाश्रया}
{रावणः कुम्भकर्णश्च भृत्यौ पूर्वपरायणौ}% ३३३

\twolineshloka
{शापात्तौ सनकादीनां समुत्पन्नौ महीतले}
{तयोर्विमुक्त्यै वैदेही गृहीता दण्डके वने}% ३३४

\twolineshloka
{तावुभौ वै वधं प्राप्तौ त्वया राक्षसपुङ्गवौ}
{तौ विमुक्तौ दिवं यातौ पुत्रपौत्रसहानुगौ}% ३३५

\twolineshloka
{त्वं विष्णुस्त्वं परं ब्रह्म योगिध्येयः सनातनः}
{त्वमेव सर्वदेवानामनादिनिधनोऽव्ययः}% ३३६

\twolineshloka
{त्वं हि नारायणः श्रीमान्सीता लक्ष्मीः सनातनी}
{माता सा सर्वलोकानां पिता त्वं परमेश्वर}% ३३७

\twolineshloka
{नित्यैवैष जगन्माता तव नित्यानपायिनी}
{यथा सर्वगतस्त्वं हि तथा चेयं रघूत्तम}% ३३८

\twolineshloka
{तस्माच्छुद्धसमाचारां सीतां साध्वीं दृढव्रताम्}
{गृहाण सौम्य काकुत्स्थ क्षीराब्धेरिव मा चिरम्}% ३३९

\uvacha{ईश्वर उवाच}


\threelineshloka
{एतस्मिन्नन्तरे तत्र लोकसाक्षी स पावकः}
{आदाय सीतां रामाय प्रददौ सुरसन्निधौ}
{अब्रवीत्तत्र काकुत्स्थं वह्निः सर्वशरीरगः}% ३४०

\uvacha{वह्निरुवाच}

\twolineshloka
{इयं शुद्धसमाचारा सीता निष्कल्मषा विभो}
{गृहाण मा चिरं राम सत्यं सत्यं तवाब्रुवन्}% ३४१

\uvacha{ईश्वर उवाच}

\twolineshloka
{ततोऽग्निवचनात्सीतां परिगृह्य रघूद्वहः}
{बभूव रामः संहृष्टः पूज्यमानः सुरोत्तमैः}% ३४२

\twolineshloka
{राक्षसैर्निहता ये तु सङ्ग्रामे वानरोत्तमाः}
{पितामहवरात्तूर्णं जीवमानाः समुत्थिताः}% ३४३

\twolineshloka
{ततस्तु पुष्पकं नाम विमानं सूर्यसन्निभम्}
{भ्रात्रा गृहीतं सङ्ग्रामे कौबेरं राक्षसेश्वरः}% ३४४

\twolineshloka
{तद्राघवाय प्रददौ वस्त्राण्याभरणानि च}
{तेन सम्पूजितः श्रीमान्रामचन्द्रः प्रतापवान्}% ३४५

\twolineshloka
{आरुरोह विमानाग्र्यं वैदेह्या भार्यया सह}
{लक्ष्मणेन च शूरेण भ्रात्रा दशरथात्मजः}% ३४६

\twolineshloka
{ऋक्षवानरसङ्घातैः सुग्रीवेण महात्मना}
{विभीषणेन शूरेण राक्षसैश्च महाबलैः}% ३४७

\twolineshloka
{यथाविमाने वैकुण्ठे नित्यमुक्तैर्महात्मभिः}
{तथा सर्वे समारुह्य ऋक्षवानरराक्षसाः}% ३४८

\twolineshloka
{अयोध्यां प्रस्थितो रामः स्तूयमानः सुरोत्तमैः}
{भरद्वाजाश्रमं गत्वा रामः सत्यपराक्रमः}% ३४९

\twolineshloka
{भरतस्यान्तिके तत्र हनूमन्तं व्यसर्जयत्}
{स निषादालयं गत्वा गुहं दृष्ट्वाऽथ वैष्णवम्}% ३५०

\twolineshloka
{राघवागमनं तस्मै प्राह वानरपुङ्गवः}
{नन्दिग्रामं ततो गत्वा दृष्ट्वा तं राघवानुजम्}% ३५१

\twolineshloka
{न्यवेदयत्तथा तस्मै रामस्यागमनोत्सवम्}
{भरतश्चागतं श्रुत्वा वानरेण रघूत्तमम्}% ३५२

\twolineshloka
{प्रर्हर्षमतुलं लेभे सानुजः ससुहृज्जनः}
{पुनरागत्य काकुत्स्थं हनूमान्मारुतात्मजः}% ३५३

\twolineshloka
{सर्वं शशंस रामाय भरतस्य च वर्तितम्}
{राघवस्तु विमानाग्र्यादवरुह्य सहानुजः}% ३५४

\twolineshloka
{ववन्दे भार्यया सार्द्धं भारद्वाजं तपोनिधिम्}
{स तु सम्पूजयामास काकुत्स्थं सानुजं मुनिः}% ३५५

\twolineshloka
{पक्वान्नैः फलमूलाद्यैर्वस्त्रैराभरणैरपि}
{तेन सम्पूजितस्तत्र प्रणम्य मुनिसत्तमम्}% ३५६

\twolineshloka
{अनुज्ञातः समारुह्य पुष्पकं सानुगस्तदा}
{नन्दिग्रामं ययौ रामः पुष्पकेण सुहृद्वृतः}% ३५७

\twolineshloka
{मन्त्रिभिः पौरमुख्यैश्च सानुजः केकयीसुतः}
{प्रत्युद्ययौ नृपवरैः सबलैः पूर्वजं मुदा}% ३५८

\twolineshloka
{सम्प्राप्य रघुशार्दूलं ववन्दे सानुगैर्वृतः}
{पुष्पकादवरुह्याथ राघवः शत्रुतापनः}% ३५९

\twolineshloka
{भरतं चैव शत्रुघ्नमुपसम्परिषस्वजे}
{पुरोहितं वसिष्ठं च मातृवृद्धांश्च बान्धवान्}% ३६०

\twolineshloka
{प्रणनाम महातेजाः सीतया लक्ष्मणेन च}
{विभीषणं च सुग्रीवं जाम्बवन्तं तथाङ्गदम्}% ३६१

\twolineshloka
{हनुमन्तं सुषेणं च भरतः परिषस्वजे}
{भ्रातृभिः सानुगैस्तत्र मङ्गलस्नानपूर्वकम्}% ३६२

\twolineshloka
{दिव्यमाल्याम्बरधरो दिव्यगन्धानुलेपनः}
{आरुरोह रथं दिव्यं सुमन्त्राधिष्ठितं शुभम्}% ३६३

\twolineshloka
{संस्तूयमानस्त्रिदशैर्वैदेह्या लक्ष्मणेन च}
{भरतश्चैव सुग्रीवः शत्रुघ्नश्च विभीषणः}% ३६४

\twolineshloka
{अङ्गदश्च सुषेणश्च जाम्बवान्मारुतात्मजः}
{नीलो नलश्च सुभगः शरभो गन्धमादनः}% ३६५

\twolineshloka
{अन्ये च कपयः शूरा निषादाधिपतिर्गुहः}
{राक्षसाश्च महावीर्याः पार्थिवेन्द्रा महाबलाः}% ३६६

\twolineshloka
{गजानश्वानथो सम्यगारुह्य बहुशः शुभान्}
{नानामङ्गलवादित्रैः स्तुतिभिः पुष्कलैस्तथा}% ३६७

\twolineshloka
{ऋक्षवानररक्षोभिर्निषादवरसैनिकैः}
{प्रविवेश महातेजाः साकेतं पुरमव्ययम्}% २६८

\twolineshloka
{आलोक्य राजनगरीं पथि राजपुत्रो राजानमेव पितरं परिचिन्तयानः}
{सुग्रीवमारुतिविभीषणपुण्यपादसञ्चारपूतभवनं प्रविवेश रामः}% ३६९

{॥इति श्रीपाद्मे महापुराणे पञ्चपञ्चाशत्साहस्र्यां संहितायामुत्तरखण्डे उमामहेश्वरसंवाद रामस्यायोध्याप्रवेशो नाम द्विचत्वारिंशदधिकद्विशततमोऽध्यायः॥२४२॥}

\sect{त्रिचत्वारिंशदधिक-द्विशततमोऽध्यायः --- विश्वदर्शनम्}

\uvacha{शङ्कर उवाच}

\twolineshloka
{अथ तस्मिन्दिने पुण्ये शुभलग्ने शुभान्विते}
{मङ्गलस्याभिषेकार्थं मङ्गलं चक्रिरे जनाः}% १

\twolineshloka
{वसिष्ठो वामदेवश्च जाबालिरथ कश्यपः}
{मार्कण्डेयश्च मौद्गल्यः पर्वतो नारदस्तथा}% २

\twolineshloka
{एते महर्षयस्तत्र जपहोमपुरस्सरम्}
{अभिषेकं शुभं चक्रुर्मुनयो राजसत्तमम्}% ३

\twolineshloka
{नानारत्नमये दिव्ये हेमपीठे शुभान्विते}
{निवेश्य सीतया सार्द्धं श्रिया इव जनार्दनम्}% ४

\twolineshloka
{सौवर्णकलशैर्दिव्यैर्नानारत्नमयैः शुभैः}
{सर्वतीर्थोदकैः पुण्यैर्माङ्गल्यद्रव्यसंयुतैः}% ५

\twolineshloka
{दूर्वाग्रतुलसीपत्रपुष्पगन्धसमन्वितैः}
{मन्त्रपूतजलैः शुद्धैर्मुनयः संशितव्रताः}% ६

\twolineshloka
{अजपन्वैष्णवान्सूक्तान्चतुर्वेदमयान्शुभान्}
{अभिषेकं शुभं चक्रुः काकुत्स्थं जगतः पतिम्}% ७

\twolineshloka
{तस्मिन्शुभतमे लग्ने देवदुन्दुभयो दिवि}
{विनेदुः पुष्पवर्षाणि ववृषुश्च समन्ततः}% ८

\twolineshloka
{दिव्याम्बरैर्भूषणैश्च दिव्यगन्धानुलेपनैः}
{पुष्पैर्नानाविधैर्दिव्यैर्देव्या सह रघूद्वहः}% ९

\twolineshloka
{अलङ्कृतश्च शुशुभे मुनिभिर्वेदपारगैः}
{छत्रं च चामरं दिव्यं धृतवान्लक्ष्मणस्तदा}% १०

\twolineshloka
{पार्श्वे भरतशत्रुघ्नौ तालवृन्तौ विरेजतुः}
{दर्पणं प्रददौ श्रीमान्राक्षसेन्द्रो विभीषणः}% ११

\twolineshloka
{दधार पूर्णकलशं सुग्रीवो वानरेश्वरः}
{जाम्बवांश्च महातेजाः पुष्पमालां मनोहराम्}% १२

\twolineshloka
{वालिपुत्रस्तु ताम्बूलं सकर्पूरं ददौ हरेः}
{हनुमान्दीपकां दिव्यां सुषेणश्च ध्वजं शुभम्}% १३

\twolineshloka
{परिवार्य महात्मानं मन्त्रिणः समुपासिरे}
{सृष्टिर्जयन्तो विजयः सौराष्ट्रो राष्ट्रवर्द्धनः}% १४

\twolineshloka
{अकोपो धर्मपालश्च सुमन्त्रो मन्त्रिणः स्मृताः}
{राजानश्च नरव्याघ्रा नानाजनपदेश्वराः}% १५

\twolineshloka
{पौराश्च नैगमा वृद्धा राजानं पर्युपासत}
{ऋक्षैश्च वानरेन्द्रैश्च मन्त्रिभिः पृथिवीश्वरैः}% १६

\twolineshloka
{राक्षसैर्द्विजमुख्यैश्च किङ्करैश्च समावृतः}
{परे व्योम्नि यथा लीनो दैवतैः कमलापतिः}% १७

\twolineshloka
{तथा नृपवरः श्रीमान्साकेते शुशुभे तदा}
{इन्दीवरदलश्यामं पद्मपत्रनिभेक्षणम्}% १८

\twolineshloka
{आजानुबाहुं काकुत्स्थं पीतवस्त्रधरं हरिम्}
{कम्बुग्रीवं महोरस्कं विचित्राभरणैर्युतम्}% १९

\twolineshloka
{देव्या सह समासीनमभिषिक्तं रघूत्तमम्}
{विमानस्थाः सुरगणा हर्षनिर्भरमानसाः}% २०

\twolineshloka
{तुष्टुवुर्जयशब्देन गन्धर्वाप्सरसां गणाः}
{अभिषिक्तस्ततो रामो वसिष्ठाद्यैर्महर्षिभिः}% २१

\twolineshloka
{शुशुभे सीतया देव्या नारायण इव श्रिया}
{अतिमर्त्यतयाभीत उपासितुं पदाम्बुजम्}% २२

\threelineshloka
{दृष्ट्वा तुष्टाव हृष्टात्मा शङ्करो हृष्टमागतः}
{कृताञ्जलिपुटो भूत्वा सानन्दो गद्गदाकुलः}
{हर्षयन्सकलान्देवान्मुनीनपि च वानरान्}% २३

\uvacha{महादेव उवाच}

\twolineshloka
{नमो मूलप्रकृतये नित्याय परमात्मने}
{सच्चिदानन्दरूपाय विश्वरूपाय वेधसे}% २४

\twolineshloka
{नमो निरन्तरानन्द कन्दमूलाय विष्णवे}
{जगत्त्रयकृतानन्द मूर्त्तये दिव्यमूर्त्तये}% २५

\twolineshloka
{नमो ब्रह्मेन्द्रपूज्याय शङ्कराभयदाय च}
{नमो विष्णुस्वरूपाय सर्वरूपनमोनमः}% २६

\twolineshloka
{उत्पत्तिस्थितिसंहारकारिणे त्रिगुणात्मने}
{नमोस्तु निर्गतोपाधिस्वरूपाय महात्मने}% २७

\twolineshloka
{अनया विद्यया देव्या सीतयोपाधिकारिणे}
{नमः पुम्प्रकृतिभ्यां च युवाभ्यां जगतां कृते}% २८

\twolineshloka
{जगन्मातापितृभ्यां च जनन्यै राघवाय च}
{नमः प्रपञ्चरूपिण्यै निष्प्रपञ्चस्वरूपिणे}% २९

\twolineshloka
{नमो ध्यानस्वरूपिण्यै योगिध्येयात्ममूर्त्तये}
{परिणामापरीणामरिक्ताभ्यां च नमोनमः}% ३०

\twolineshloka
{कूटस्थबीजरूपिण्यै सीतायै राघवाय च}
{सीता लक्ष्मीर्भवान्विष्णुः सीता गौरी भवान्शिवः}% ३१

\twolineshloka
{सीता स्वयं हि सावित्रि भवान्ब्रह्मा चतुर्मुखः}
{सीता शची भवान्शक्रः सीता स्वाहानलो भवान्}% ३२

\twolineshloka
{सीता संहारिणी देवी यमरूपधरो भवान्}
{सीता हि सर्वसम्पत्तिः कुबेरस्त्वं रघूत्तम}% ३३

\twolineshloka
{सीता देवी च रुद्राणी भवान्रुद्रो महाबलः}
{सीता तु रोहिणी देवी चन्द्रस्त्वं लोकसौख्यदः}% ३४

\twolineshloka
{सीता संज्ञा भवान्सूर्यः सीता रात्रिर्दिवा भवान्}
{सीतादेवी महाकाली महाकालो भवान्सदा}% ३५

\twolineshloka
{स्त्रीलिङ्गेषु त्रिलोकेषु यत्तत्सर्वं हि जानकी}
{पुन्नाम लाञ्छितं यत्तु तत्सर्वं हि भवान्प्रभो}% ३६

\twolineshloka
{सर्वत्र सर्वदेवेश सीता सर्वत्र धारिणी}
{तदात्वमपिचत्रातुन्तच्छक्तिर्विश्वधारिणी}% ३७

\twolineshloka
{तस्मात्कोटिगुणं पुण्यं युवाभ्यां परिचिह्नितम्}
{चिह्नितं शिवशक्तिभ्यां चरितं तव शान्तिदम्}% ३८

\twolineshloka
{आवां राम जगत्पूज्यौ मम पूज्यौ सदा युवाम्}
{त्वन्नामजापिनी गौरी त्वन्मन्त्रजपवानहम्}% ३९

\twolineshloka
{मुमूर्षोर्मणिकर्ण्यां तु अर्द्धोदकनिवासिनः}
{अहं दिशामि ते मन्त्रं तारकं ब्रह्मदायकम्}% ४०

\twolineshloka
{अतस्त्वं जानकीनाथ परब्रह्मासि निश्चितम्}
{त्वन्मायामोहितास्सर्वे न त्वां जानन्ति तत्वतः}% ४१

\uvacha{ईश्वर उवाच}

\twolineshloka
{इत्युक्तः शम्भुना रामः प्रसादप्रवणोऽभवत्}
{दिव्यरूपधरः श्रीमानद्भुताद्भुतदर्शनः}% ४२

\twolineshloka
{तथा तं रूपमालोक्य नरवानरदेवताः}
{न द्रष्टुमपिशक्तास्ते तेजसं महदद्भुतम्}% ४३


\threelineshloka
{भयाद्वै त्रिदशश्रेष्ठाः प्रणेमुश्चातिभक्तितः}
{भीता विज्ञाय रामोऽपि नरवानरदेवताः}
{मायामानुषतां प्राप्य स देवानब्रवीत्पुनः}% ४४

\uvacha{रामचन्द्र उवाच}

\twolineshloka
{शृणुध्वं देवता यो मां प्रत्यहं संस्तुविष्यति}
{स्तवेन शम्भुनोक्तेन देवतुल्यो भवेन्नरः}% ४५

\twolineshloka
{विमुक्तः सर्वपापेभ्यो मत्स्वरूपं समश्नुते}
{रणे जयमवाप्नोति न क्वचित्प्रतिहन्यते}% ४६

\twolineshloka
{भूतवेतालकृत्याभिर्ग्रहैश्चापि न बाध्यते}
{अपुत्रो लभते पुत्रं पतिं विन्दति कन्यका}% ४७

\twolineshloka
{दरिद्रः श्रियमाप्नोति सत्ववाञ्शीलवान्भवेत्}
{आत्मतुल्यबलः श्रीमाञ्जायते नात्र संशयः}% ४८

\twolineshloka
{निर्विघ्नं सर्वकार्येषु सर्वारम्भेषु वै नृणाम्}
{यंयं कामयते मर्त्यः सुदुर्लभमनोरथम्}% ४९


\threelineshloka
{षण्मासात्सिद्धिमाप्नोति स्तवस्यास्य प्रसादतः}
{यत्पुण्यं सर्वतीर्थेषु सर्वयज्ञेषु यत्फलम्}
{तत्फलं कोटिगुणितं स्तवेनानेन लभ्यते}% ५०

\uvacha{ईश्वर उवाच}

\twolineshloka
{इत्युक्त्वा रामचन्द्रोऽसौ विससर्ज महेश्वरम्}
{ब्रह्मादि त्रिदशान्सर्वान्विससर्ज समागतान्}% ५१

\twolineshloka
{अर्चिता मानवाः सर्वे नरवानरदेवताः}
{विसृष्टा रामचन्द्रेण प्रीत्या परमया युताः}% ५२

\twolineshloka
{इत्थं विसृष्टाः खलु ते च सर्वे सुखं तदा जग्मुरतीवहृष्टाः}
{परं पठन्तः स्तवमीश्वरोक्तं रामं स्मरन्तो वरविश्वरूपम्}% ५३

{॥इति श्रीपाद्मे महापुराणे पञ्चपञ्चाशत्साहस्र्यां संहितायामुत्तरखण्डे उमामहेश्वर संवादे विश्वदर्शनं नाम त्रिचत्वारिंशदधिकद्विशततमोऽध्यायः॥२४३॥}

\sect{चतुश्चत्वारिंशदधिक-द्विशततमोऽध्यायः --- श्रीरामचरितकथनम्}

\uvacha{शङ्कर उवाच}

\twolineshloka
{अथ रामस्तु वैदेह्या राज्यभोगान्मनोरमान्}
{बुभुजे वर्षसाहस्रं पालयन्सर्वतोदिशः}% १

\twolineshloka
{अन्तःपुरजनास्सर्वे राक्षसस्य गृहे स्थिताम्}
{गर्हयन्ति स्म वैदेहीं तथा जानपदा जनाः}% २

\twolineshloka
{लोकापवादभीत्या च रामः शत्रुनिवारकः}
{दर्शयन्मानुषं धर्ममन्तर्वत्नीं नृपात्मजाम्}% ३

\twolineshloka
{वाल्मीकेराश्रमे पुण्ये गङ्गातीरे महावने}
{विससर्ज महातेजा गर्भिणीं मुनिसंसदि}% ४

\twolineshloka
{सा भर्तुः परतन्त्रा हि उवास मुनिवेश्मनि}
{अर्चिता मुनिपत्नीभिर्वाल्मीकमुनि रक्षिता}% ५

\twolineshloka
{तत्रैवासूत यमलौ नाम्ना कुशलवौ सुतौ}
{तौ च तत्रैव मुनिना संस्कृतौ च ववर्धतुः}% ६

\twolineshloka
{रामोऽपि भ्रातृभिस्सार्द्धं पालयामास मेदिनीम्}
{यमादिगुणसम्पन्नस्सर्वभोगविवर्जितः}% ७

\twolineshloka
{अर्चयन्सततं विष्णुमनादिनिधनं हरिम्}
{ब्रह्मचर्यपरो नित्यं शशास पृथिवीं नृपः}% ८

\twolineshloka
{शत्रुघ्नो लवणं हत्वा मथुरां देवनिर्मिताम्}
{पालयामास धर्मात्मा पुत्राभ्यां सह राघवः}% ९

\twolineshloka
{गन्धर्वान्भरतो हत्वा सिन्धोरुभयपार्श्वतः}
{स्वात्मजौ स्थापयामास तस्मिन्देशे महाबलौ}% १०

\twolineshloka
{पश्चिमे मद्रदेशे तु मद्रान्हत्वा च लक्ष्मणः}
{स्वसुतौ च महावीर्यौ अभिषिच्य महाबलः}% ११

\twolineshloka
{गत्वा पुनरयोध्यां तु रामपादावुपस्पृशत्}
{ब्राह्मणस्य मृतं बालं कालधर्ममुपागतम्}% १२

\twolineshloka
{जीवयामास काकुत्स्थः शूद्रं हत्वा च तापसम्}
{ततस्तु गौतमीतीरे नैमिषे जनसंसदि}% १३

\twolineshloka
{इयाज वाजिमेधं च राघवः परवीरहा}
{काञ्चनीं जानकीं कृत्वा तया सार्द्धं महाबलः}% १४

\twolineshloka
{चकार यज्ञान्बहुशो राघवः परमार्थवित्}
{अयुतान्यश्वमेधानि वाजपेयानि च प्रभुः}% १५

\twolineshloka
{अग्निष्टोमं विश्वजितं गोमेधं च शतक्रतुम्}
{चकार विविधान्यज्ञान्परिपूर्णसदक्षिणान्}% १६

\twolineshloka
{एतस्मिन्नन्तरे तत्र वाल्मीकिः सुमहातपाः}
{सीतामानीय काकुत्स्थमिदं वचनमब्रवीत्}% १७

\uvacha{वाल्मीकिरुवाच}


\threelineshloka
{अपापां मैथिलीं राम त्यक्तुं नार्हसि सुव्रत}
{इयं तु विरजा साध्वी भास्करस्य प्रभा यथा}
{अनन्या तव काकुत्स्थ कस्मात्त्यक्ता त्वयानघ}% १८

\uvacha{राम उवाच}

\twolineshloka
{अपापां मैथिलीं ब्रह्मन्जानामि वचनात्तव}
{रावणेन हृता साध्वी दण्डके विजने पुरा}% १९

\twolineshloka
{तं हत्वा समरे सीतां शुद्धामग्निमुखागताम्}
{पुनर्यातोस्म्ययोध्यायां सीतामादाय धर्मतः}% २०

\twolineshloka
{लोकापवादः सुमहानभूत्पौरजनेषु च}
{त्यक्ता मया शुभाचारा तद्भयात्तव सन्निधौ}% २१

\twolineshloka
{तस्माल्लोकस्य सन्तुष्ट्यै सीता मम परायणा}
{पार्थिवानां महर्षीणां प्रत्ययं कर्तुमर्हति}% २२

\uvacha{महेश्वर उवाच}

\twolineshloka
{एवमुक्ता तदा सीता मुनिपार्थिवसंसदि}
{चकारप्रत्ययं देवी लोकाश्चर्यकरं सती}% २३

\twolineshloka
{दर्शयंस्तस्य लोकस्य रामस्यानन्यतां सती}
{अब्रवीत्प्राञ्जलिः सीता सर्वेषां जनसंसदि}% २४

\uvacha{सीतोवाच}

\twolineshloka
{यथाऽहं राघवादन्यं मनसापि न चिन्तये}
{तथा मे धरणी देवी विवरन्दातुमर्हति}% २५

\twolineshloka
{यथैव सत्यमुक्तं मे वेद्मि रामात्परं न च}
{तथा स्वपुत्र्यां वैदेह्यां धरणी सहसा इयात्}% २६

\uvacha{महेश्वर उवाच}

\twolineshloka
{ततो रत्नमयं पीठं पृष्ठे धृत्वा खगेश्वरः}
{रसातलात्तदा वीरो विज्ञाय जननीं तदा}% २७

\twolineshloka
{ततस्तु धरणीदेवी हस्ताभ्यां गृह्य मैथिलीम्}
{स्वागतेनाभिनन्द्यैनामासने सन्न्यवेशयत्}% २८

\twolineshloka
{सीतां समागतां दृष्ट्वा दिवि देवगणा भृशम्}
{पुष्पवृष्टिमविच्छिन्नां दिव्यां सीतामवाकिरन्}% २९

\twolineshloka
{सापि दिव्याप्सरोभिस्तु पूज्यमाना सनातनी}
{वैनतेयं समारुह्य तस्मान्मार्गाद्दिवं ययौ}% ३०

\twolineshloka
{दासीगणैः पूर्वभागे संवृता जगदीश्वरी}
{सम्प्राप्य परमं धाम योगिगम्यं सनातनम्}% ३१

\twolineshloka
{रसातलप्रविष्टां तु तां दृष्ट्वा सर्वमानुषाः}
{साधुसाध्विति सीतेयमुच्चैः सर्वे प्रचुक्रुशुः}% ३२

\twolineshloka
{रामः शोकसमाविष्टः सङ्गृह्य तनयावुभौ}
{मुनिभिः पार्थिवेन्द्रैश्च साकेतं प्रविवेश ह}% ३३

\twolineshloka
{अथ कालेन महता मातरः संशितव्रताः}
{कालधर्मं समापन्ना भर्तुः स्वर्गं प्रपेदिरे}% ३४

\twolineshloka
{दशवर्षसहस्राणि दशवर्षशतानि च}
{चकार राज्यं धर्मेण राघवः संशितव्रतः}% ३५

\twolineshloka
{कस्यचित्त्वथकालस्य राघवस्य निवेशनम्}
{कालस्तापसरूपेण सम्प्राप्तो वाक्यमब्रवीत्}% ३६

\uvacha{काल उवाच}

\twolineshloka
{राम राम महाबाहो धात्रा सम्प्रेषितोऽस्म्यहम्}
{यद्ब्रवीमि रघुश्रेष्ठ तच्छृणुष्व महामते}% ३७

\twolineshloka
{द्वन्द्वमेव हि कार्यं स्यादावयोः परिभाषितम्}
{तदन्तरे प्रविष्टोयस्स वद्ध्यो हि भविष्यति}% ३८

\uvacha{महेश्वर उवाच}


\threelineshloka
{तथेति च प्रतिश्रुत्य रामो राजीवलोचनः}
{द्वास्थं कृत्वा तु सौमित्रिं कालो वाक्यमभाषत}
{वैवस्वतोऽब्रवीद्वाक्यं रामं दशरथात्मजम्}% ३९

\uvacha{काल उवाच}

\twolineshloka
{शृणु राम यथावृत्तं समागमनकारणात्}
{दशवर्षसहस्राणि दशवर्षशतानि च}% ४०

\twolineshloka
{वसामि मानुषे लोके हत्वा राक्षसपुङ्गवौ}
{एवमुक्तः सुरगणैरवतीर्णोसि भूतले}% ४१

\twolineshloka
{तदयं समयः प्राप्तः स्वर्लोकं गमितुं त्वया}
{सनाथा हि सुरास्सर्वे भवन्त्वद्य त्वयानघ}% ४२

\uvacha{महेश्वर उवाच}

\twolineshloka
{एवमस्त्विति काकुत्स्थो रामः प्राह महामुनिम्}
{एतस्मिन्नन्तरे तत्र दुर्वासास्तु महातपाः}% ४३

\onelineshloka*
{राजद्वारमुपागम्य लक्ष्मणं वाक्यमब्रवीत्}

\uvacha{दुर्वासा उवाच}
\onelineshloka
{मां निवेदय काकुत्स्थं शीघ्रं गत्वा नृपात्मज}% ४४

\uvacha{महेश्वर उवाच}

\twolineshloka
{तमब्रवील्लक्ष्मणस्तु असान्निध्यमिति द्विज}
{ततः क्रोधसमाविष्टः प्राह तं मुनिसत्तमः}% ४५

\uvacha{दुर्वासा उवाच}

\onelineshloka*
{शापं दास्यामि काकुत्स्थं रामं न यदि दर्शये}

\uvacha{महेश्वर उवाच}

\twolineshloka
{तस्माच्छापभयाद्विप्रं राघवाय न्यवेदयत्}
{तत्रैवान्तर्दधे कालः सर्वभूतभयावहः}% ४६

\twolineshloka
{पूजयामास तं प्राप्तमृषिं दुर्वाससं नृपः}
{अग्रजस्य प्रतिज्ञा तं विज्ञाय रघुसत्तमः}% ४७

\twolineshloka
{तत्याज मानुषं रूपं लक्ष्मणः सरयूजले}
{विसृज्य मानुषं रूपं प्रविवेश स्वकां तनुम्}% ४८

\twolineshloka
{फणासहस्रसंयुक्तः कोटीन्दुसमवर्चसः}
{दिव्यमाल्याम्बरधरो दिव्यगन्धानुलेपनः}% ४९

\twolineshloka
{नागकन्यासहस्रैस्तु संवृतः समलङ्कृतः}
{विमानं दिव्यमारुह्य प्रययौ वैष्णवं पदम्}% ५०

\twolineshloka
{लक्ष्मणस्य गतिं सर्वां विदित्वा रघुसत्तमः}
{स्वयमप्यथ काकुत्स्थः स्वर्गं गन्तुमभीप्सितः}% ५१

\twolineshloka
{अभिषिच्याथ काकुत्स्थः स्वात्मजौ च कुशीलवौ}
{विभज्य रथनागाश्वं सधनं प्रददौ तयोः}% ५२

\twolineshloka
{कुशवत्यां कुशं तं च शरवत्यां लवं तथा}
{स्थापयामास धर्मेण राज्ये स्वे रघुसत्तमः}% ५३

\twolineshloka
{अभिप्रायं तु विज्ञाय रामस्य विदितात्मनः}
{आजग्मुर्वानराः सर्वे राक्षसाः सुमहाबलाः}% ५४

\twolineshloka
{विभीषणोऽथ सुग्रीवो जाम्बवान्मारुतात्मजः}
{नीलो नलः सुषेणश्च निषादाधिपतिर्गुहः}% ५५

\twolineshloka
{अभिषिच्य सुतौ वीरौ शत्रुघ्नश्च महामनाः}
{सर्व एते समाजग्मुरयोध्यां रामपालिताम्}% ५६

\onelineshloka*
{ते प्रणम्य महात्मानमूचुः प्राञ्जलयस्तथा}

\uvacha{वानरप्रभृतय ऊचुः}

\onelineshloka
{स्वर्लोकं गन्तुमुद्युक्तं ज्ञात्वा त्वां रघुसत्तम}% ५७


\threelineshloka
{आगताः स्म वयं सर्वे तवानुगमनं प्रति}
{न शक्ताः स्म क्षणं राम जीवितुं त्वां विना प्रभो}
{तस्मात्त्वया विशालाक्ष गच्छामस्त्रिदशालयम्}% ५८

\uvacha{महेश्वर उवाच}

\twolineshloka
{तैरेवमुक्तः काकुत्स्थो बाढमित्यब्रवीत्ततः}
{अथोवाच महातेजा राक्षसेन्द्रं विभीषणम्}% ५९

\uvacha{राम उवाच}

\onelineshloka*
{राज्यं प्रशास धर्मेण मा प्रतिज्ञां वृथा कृथाः}

\twolineshloka
{यावच्चन्द्रश्च सूर्यश्च यावत्तिष्ठति मेदिनी}
{तावद्रमस्व सुप्रीतो काले मम पदं व्रज}% ६०

\uvacha{महेश्वर उवाच}

\twolineshloka
{इत्युक्त्वाथ स काकुत्स्थः स्वाड्गं विष्णुं सनातनम्}
{श्रीरङ्गशायिनं सौम्यमिक्ष्वाकुकुलदैवतम्}% ६१

\twolineshloka
{सम्प्रीत्या प्रददौ तस्मै रामो राजीवलोचनः}
{हनुमन्तमथोवाच राघवः शत्रुसूदनः}% ६२

\uvacha{राम उवाच}

\twolineshloka
{मत्कथाः प्रचरिष्यन्ति यावल्लोके हरीश्वर}
{तावत्त्वमास मेदिन्यां काले मां व्रज सुव्रत}% ६३

\uvacha{महेश्वर उवाच}

\onelineshloka*
{तमेवमुक्त्वा काकुत्स्थो जाम्बवन्तमथाब्रवीत्}

\uvacha{राम उवाच}
\onelineshloka
{द्वापरे समनुप्राप्ते यदूनामन्वये पुनः}% ६४

\twolineshloka
{भूभारस्य विनाशाय समुत्पत्स्याम्यहं भुवि}
{करिष्ये तत्र सङ्ग्रामं स्वयं भल्लूकसत्तम}% ६५

\uvacha{महेश्वर उवाच}

\twolineshloka
{तमेवमुक्त्वा काकुत्स्थः सर्वांस्तानृक्षवानरान्}
{उवाच वाचा गच्छध्वमिति रामो महाबलः}% ६६

\twolineshloka
{मन्त्रिणो नैगमाश्चैव भरतः कैकयीसुतः}
{राघवस्यानुगमने निश्चितास्ते समाययुः}% ६७

\twolineshloka
{ततः शुक्लाम्बरधरो ब्रह्मचारी ययौ परम्}
{कुशान्गृहीत्वा पाणिभ्यां संसक्तः प्रययौ परम्}% ६८

\twolineshloka
{रामस्य दक्षिणे पार्श्वे पद्महस्ता रमा गता}
{तथैव धरणीदेवी दक्षिणेतरगा तथा}% ६९

\twolineshloka
{वेदाः साङ्गाः पुराणानि सेतिहासानि सर्वतः}
{ॐकारोऽथ वषट्कारः सावित्री लोकपावनी}% ७०

\twolineshloka
{अस्त्रशस्त्राणि च तदा धनुराद्यानि पार्वति}
{अनुजग्मुस्तथा रामं सर्वे पुरुषविग्रहाः}% ७१

\twolineshloka
{भरतश्चैव शत्रुघ्नः सर्वे पुरनिवासिनः}
{सपुत्रदाराः काकुत्स्थमनुजग्मुः सहानुगाः}% ७२

\twolineshloka
{मन्त्रिणो भृत्यवर्गाश्च किङ्करा नैगमास्तथा}
{वानराश्चैव ऋक्षाश्च सुग्रीवसहितास्तदा}% ७३

\twolineshloka
{सपुत्रदाराः काकुत्स्थमन्वगच्छन्महामतिम्}
{पशवः पक्षिणश्चैव सर्वे स्थावरजङ्गमाः}% ७४

\twolineshloka
{अनुजग्मुर्महात्मानं समीपस्था नरोत्तमाः}
{ये च पश्यन्ति काकुत्स्थं स्वपथान्तर्गतं प्रभुम्}% ७५

\twolineshloka
{ते तथानुगता रामं निवर्त्तन्ते न केचन}
{अथ त्रियोजनं गत्वा नदीं पश्चान्मुखीं स्थिताम्}% ७६

\twolineshloka
{सरयूं पुण्यसलिलां प्रविवेश सहानुगः}
{ततः पितामहो ब्रह्मा सर्वदेवगणावृतः}% ७७

\twolineshloka
{तुष्टाव रघुशार्दूलमृषिभिः सार्द्धमक्षरैः}
{अब्रवीत्तत्र काकुत्स्थं प्रविष्टं सरयूजले}% ७८

\uvacha{ब्रह्मोवाच}

\twolineshloka
{आगच्छ विष्णो भद्रं ते दिष्ट्या प्राप्तोऽसि मानद}
{भ्रातृभिस्सहदेवाभैः प्रविशस्व निजां तनुम्}% ७९

\twolineshloka
{वैष्णवीं तां महातेजां देवाकारां सनातनीम्}
{त्वं हि लोकगतिर्देव न त्वां केचित्तु जानते}% ८०

\twolineshloka
{त्वामचिन्त्यं महात्मानमक्षरं सर्वसङ्ग्रहम्}
{यमिच्छसि महातेजस्तां तनुं प्रविशस्व भोः}% ८१

\uvacha{महेश्वर उवाच}

\twolineshloka
{तस्मिन्सूर्यकराकीर्णे पुष्पवृष्टिनिपातिते}
{उत्सृज्य मानुषं रूपं स्वां तनुं प्रविवेश ह}% ८२

\twolineshloka
{अंशाभ्यां शङ्खचक्राभ्यां शत्रुघ्नभरतावुभौ}
{तदा तेन महात्मानौ दिव्यतेजस्समन्वितौ}% ८३

\twolineshloka
{शङ्खचक्रगदाशार्ङ्गपद्महस्तश्चतुर्भुजः}
{दिव्याभरणसम्पन्नो दिव्यगन्धानुलेपनः}% ८४

\twolineshloka
{दिव्यपीताम्बरधरः पद्मपत्रनिभेक्षणः}
{युवा कुमारः सौम्याङ्गः कोमलावयवोज्ज्वलः}% ८५

\twolineshloka
{सुस्निग्धनीलकुटिलकुन्तलः शुभलक्षणः}
{नवदूर्वाङ्कुरः श्यामः पूर्णचन्द्र निभाननः}% ८६

\twolineshloka
{देवीभ्यां सहितः श्रीमान्विमानमधिरुह्य च}
{तस्मिन्सिंहासने दिव्ये मूले कल्पतरोः प्रभुः}% ८७

\twolineshloka
{निषसाद महातेजाः सर्वदेवैरभिष्टुतः}
{राघवानुगता ये च ऋक्षवानरमानुषाः}% ८८

\twolineshloka
{स्पृष्ट्वैव सरयूतोयं सुखेन त्यक्तजीविताः}
{रामप्रसादात्ते सर्वे दिव्यरूपधराः शुभाः}% ८९

\twolineshloka
{दिव्यमाल्याम्बरधरा दिव्यमङ्गलवर्चसः}
{आरुरोह विमानं तदसङ्ख्यास्तत्र देहिनः}% ९०

\twolineshloka
{सर्वैः परिवृतः श्रीमान्रामो राजीवलोचनः}
{पूजितः सुरसिद्धौघैर्मुनिभिस्तु महात्मभिः}% ९१

\twolineshloka
{आययौ शाश्वतं दिव्यमक्षरं स्वपदं विभुः}
{यः पठेद्रामचरितं श्लोकं श्लोकार्धमेव वा}% ९२

\twolineshloka
{शृणुयाद्वा तथा भक्त्या स्मरेद्वा शुभदर्शने}
{कोटिजन्मार्जितात्पापाज्ज्ञानतोऽज्ञानतः कृतात्}% ९३

\twolineshloka
{विमुक्तो वैष्णवं लोकं पुत्रदारसबान्धवैः}
{समाप्नुयाद्योगगम्यमनायासेन वै नरः}% ९४


\onelineshloka
{एतत्ते कथितं देवि रामस्य चरितं महत्}
{धन्योऽस्म्यहं त्वया देवि रामचन्द्रस्य कीर्त्तनात्}
{किमन्यच्छ्रोतुकामासि तद्ब्रवीमि वरानने}% ९५

{॥इति श्रीपाद्मे महापुराणे पञ्चपञ्चाशत्साहस्र्यां संहितायामुत्तरखण्डे उमामहेश्वर संवादे श्रीरामचरितकथनं नाम चतुश्चत्वारिंशदधिकद्विशततमोऽध्यायः॥२४४॥}


