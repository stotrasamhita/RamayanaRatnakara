\sect{अष्टात्रिंशोऽध्यायः --- वामनप्रतिष्ठा}

\src{पद्म-पुराणम्}{सृष्टिखण्डम्}{अध्यायः ३८}{१--१९४}
% \tags{concise, complete}
\notes{This chapter describes the conversation between Rama and Agastya. It narrates how Rama, after defeating Ravana, meets Agastya in the forest. Agastya explains the significance of the divine ornaments given to Rama.}
\textlink{https://sa.wikisource.org/wiki/पद्मपुराणम्/खण्डः_१_(सृष्टिखण्डम्)/अध्यायः_३८}
\translink{https://www.wisdomlib.org/hinduism/book/the-padma-purana/d/doc364161.html}

\storymeta


\uvacha{भीष्म उवाच}

\twolineshloka
{कथं रामेण विप्रर्षे कान्यकुब्जे तु वामनः}
{स्थापितः क्व च लब्धोसौ विस्तरान्मम कीर्तय}% १

\twolineshloka
{तथा हि मधुरा चैषा या वाणी रामकीर्तने}
{कीर्तिता भगवन्मह्यं हृता कर्णसुखावह}% २

\twolineshloka
{अनुरागेण तं लोकाः स्नेहात्पश्यन्ति राघवम्}
{धर्मज्ञश्च कृतज्ञश्च बुद्ध्या च परिनिष्ठितः}% ३

\twolineshloka
{प्रशास्ति पृथिवीं सर्वां धर्मेण सुसमाहितः}
{तस्मिन्शासति वै राज्यं सर्वकामफलाद्रुमाः}% ४

\twolineshloka
{रसवन्तः प्रभूताश्च वासांसि विविधानि च}
{अकृष्टपच्या पृथिवी निःसपत्ना महात्मनः}% ५

\twolineshloka
{देवकार्यं कृतं तेन रावणो लोककण्टकः}
{सपुत्रोमात्यसहितो लीलयैव निपातितः}% ६

\twolineshloka
{तस्यबुद्धिस्समुत्पन्ना पूर्णे धर्मे द्विजोत्तम}
{तस्याहं चरितं सर्वं श्रोतुमिच्छामि वै मुने}% ७

\uvacha{पुलस्त्य उवाच}

\twolineshloka
{कस्यचित्त्वथ कालस्य रामो धर्मपथे स्थितः}
{यच्चकार महाबाहो शृणुष्वैकमना नृप}% ८

\twolineshloka
{सस्मार राक्षसेन्द्रं तं कथं राजा विभीषणः}
{लङ्कायां संस्थितो राज्यं करिष्यति च राक्षसः}% ९

\twolineshloka
{गीर्वाणेषु प्रातिकूल्यं विनाशस्य तु लक्षणम्}
{मया तस्य तु तद्दत्तं राज्यं चन्द्रार्ककालिकम्}% १०

\twolineshloka
{तस्याविनाशतः कीर्तिः स्थिरा मे शाश्वती भवेत्}
{रावणेन तपस्तप्तं विनाशायात्मनस्त्विह}% ११

\twolineshloka
{विध्वस्तः स च पापिष्ठो देवकार्ये मयाधुना}
{तदिदानीं मयान्वेष्यः स्वयं गत्वा विभीषणः}% १२

\twolineshloka
{सन्देष्टव्यं हितं तस्य येन तिष्ठेत्स शाश्वतम्}
{एवं चिन्तयतस्तस्य रामस्यामिततेजसः}% १३

\twolineshloka
{आजगामाथ भरतो रामं दृष्ट्वाब्रवीदिदम्}
{किं त्वं चिन्तयसे देव न रहस्यं वदस्व मे}% १४

\twolineshloka
{देवकार्ये धरायां वा स्वकार्ये वा नरोत्तम}
{एवं ब्रुवन्तं भरतं ध्यायमानमवस्थितम्}% १५

\twolineshloka
{अब्रवीद्राघवो वाक्यं रहस्यं तु न वै तव}
{भवान्बहिश्चरः प्राणो लक्ष्मणश्च महायशाः}% १६

\twolineshloka
{अवेद्यं भवतो नास्ति मम सत्यं विधारय}
{एषा मे महती चिन्ता कथं देवैर्विभीषणः}% १७

\twolineshloka
{वर्तते यद्धितार्थं वै दशग्रीवो निपातितः}
{गमिष्ये तदहं लङ्कां यत्र चासौ विभीषणः}% १८

\twolineshloka
{तं च दृष्ट्वा पुरीं तां तु कार्यमुक्त्वा च राक्षसम्}
{आलोक्य सर्ववसुधां सुग्रीवं वानरेश्वरम्}% १९

\twolineshloka
{महाराजं च शत्रुघ्नं भातृपुत्रांश्च सर्वशः}
{एवं वदति काकुत्स्थे भरतः पुरतः स्थितः}% २०

\twolineshloka
{उवाच राघवं वाक्यं गमिष्ये भवता सह}
{एवं कुरु महाबाहो सौमित्रिरिह तिष्ठतु}% २१

\twolineshloka
{इत्युक्त्वा भरतं रामः सौमित्रं चाह वै पुरे}
{रक्षाकार्या त्वया वीर यावदागमनं हि नौ}% २२

\twolineshloka
{एवं लक्ष्मणमादिश्य ध्यात्वा वै पुष्पकं नृप}
{आरुरोह स वै यानं कौसल्यानन्दवर्धनः}% २३

\twolineshloka
{पुष्पकं तु ततः प्राप्तं गान्धारविषयो यतः}
{भरतस्य सुतौ दृष्ट्वा जगन्नीतिं निरीक्ष्य च}% २४

\twolineshloka
{पूर्वां दिशं ततो गत्वा लक्ष्मणस्य सुतौ यतः}
{पुरेषु तेषु षड्रात्रमुषित्वा रघुनन्दनौ}% २५

\twolineshloka
{गतौ तेन विमानेन दक्षिणामभितो दिशम्}
{गङ्गायामुनसम्भेदं प्रयागमृषिसेवितम्}% २६

\twolineshloka
{अभिवाद्य भरद्वाजमत्रेराश्रममीयतुः}
{सम्भाष्य च मुनींस्तत्र जनस्थानमुपागतौ}% २७

\uvacha{राम उवाच}

\twolineshloka
{अत्र पूर्वं हृता सीता रावणेन दुरात्मना}
{हत्वा जटायुषं गृध्रं योसौ पितृसखो हि नौ}% २८

\twolineshloka
{अत्रास्माकं महद्युद्धं कबन्धेन कुबुद्धिना}
{हतेन तेन दग्धेन सीतास्ते रावणालये}% २९

\twolineshloka
{ॠष्यमूके गिरिवरे सुग्रीवो नाम वानरः}
{स ते करिष्यते साह्यं पम्पां व्रज सहानुजः}% ३०

\twolineshloka
{पम्पासरः समासाद्य शबरीं गच्छ तापसीम्}
{इत्युक्तो दुःखितो वीर निराशो जीविते स्थितः}% ३१

\twolineshloka
{इयं सा नलिनी वीर यस्यां वै लक्ष्मणोवदत्}
{मा कृथाः पुरुषव्याघ्र शोकं शत्रुविनाशन}% ३२

\twolineshloka
{आज्ञाकारिणि भृत्ये च मयि प्राप्स्यसि मैथिलीम्}
{अत्र मे वार्षिका मासा गता वर्षशतोपमाः}% ३३

\twolineshloka
{अत्रैव निहतो वाली सुग्रीवार्थे परन्तप}
{एषा सा दृश्यते नूनं किष्किन्धा वालिपालिता}% ३४

\twolineshloka
{यस्यां वै स हि धर्मात्मा सुग्रीवो वानरेश्वरः}
{वानरैः सहितो वीर तावदास्ते समाः शतम्}% ३५

\twolineshloka
{वानरैस्सह सुग्रीवो यावदास्ते सभां गतः}
{तावत्तत्रागतौ वीरौ पुर्यां भरतराघवौ}% ३६

\twolineshloka
{दृष्ट्वा स भ्रातरौ प्राप्तौ प्रणिपत्याब्रवीदिदम्}
{क्व युवां प्रस्थितौ वीरौ कार्यं किं नु करिष्यथः}% ३७

\twolineshloka
{विनिवेश्यासने तौ च ददावर्घ्ये स्वयं तदा}
{एवं सभास्थिते तत्र धर्मिष्टे रघुनन्दने}% ३८

\twolineshloka
{अङ्गदोथ हनूमांश्च नलो नीलश्च पाटलः}
{गजो गवाक्षो गवयः पनसश्च महायशाः}% ३९

\twolineshloka
{पुरोधसो मन्त्रिणश्च दैवज्ञो दधिवक्रकः}
{नीलश्शतबलिर्मैन्दो द्विविदो गन्धमादनः}% ४०

\twolineshloka
{वीरबाहुस्सुबाहुश्च वीरसेनो विनायकः}
{सूर्याभः कुमुदश्चैव सुषेणो हरियूथपः}% ४१

\twolineshloka
{ॠषभो विनतश्चैव गवाख्यो भीमविक्रमः}
{ॠक्षराजश्च धूम्रश्च सहसैन्यैरुपागताः}% ४२

\twolineshloka
{अन्तःपुराणि सर्वाणि रुमा तारा तथैव च}
{अवरोधोङ्गदस्यापि तथान्याः परिचारिकाः}% ४३

\twolineshloka
{प्रहर्षमतुलं प्राप्य साधुसाध्विति चाब्रुवन्}
{वानराश्च महात्मानः सुग्रीवसहितास्तदा}% ४४

\twolineshloka
{वानर्यश्च महाभागास्ताराद्यास्तत्र राघवम्}
{अभिप्रेक्ष्याश्रुकण्ठ्यश्च प्रणिपत्येदमब्रुवन्}% ४५

\twolineshloka
{क्व सा देवी त्वया देव या विनिर्जित्यरावणम्}
{शुद्धिं कृत्वा हि ते वह्नौ पितुरग्र उमापतेः}% ४६

\twolineshloka
{त्वयानीता पुरीं राम न तां पश्यामि तेग्रतः}
{न विना त्वं तया देव शोभसे रघुनन्दन}% ४७

\twolineshloka
{त्वया विनापि साध्वी सा क्व नु तिष्ठति जानकी}
{अन्यां भार्यां न ते वेद्मि भार्याहीनो न शोभसे}% ४८

\twolineshloka
{क्रौञ्चयुग्मं मिथो यद्वच्चक्रवाकयुगं यथा}
{एवं वदन्तीं तां तारां ताराधिपसमाननाम्}% ४९

\twolineshloka
{प्राह प्रवचसां श्रेष्ठो रामो राजीवलोचनः}
{चारुदंष्ट्रे विशालाक्षि कालो हि दुरतिक्रमः1.38.}% ५०

\twolineshloka
{सर्वं कालकृतं विद्धि जगदेतच्चराचरम्}
{विसृज्यताः स्त्रियः सर्वाः सुग्रीवोभिमुखः स्थितः}% ५१

\uvacha{सुग्रीव उवाच}

\twolineshloka
{भवन्तौ येन कार्येण इहायातौ नरेश्वरौ}
{तच्चापि कथ्यतां शीघ्रं कृत्यकालो हि वर्तते}% ५२


\threelineshloka
{ब्रुवाणमेवं सुग्रीवं भरतो रामचोदितः}
{आचचक्षे च गमनं लङ्कायां राघवस्य तु}
{तौ चाब्रवीच्च सुग्रीवो भवद्भ्यां सहितः पुरीम्}% ५३

\twolineshloka
{गमिष्ये राक्षसं देव द्रष्टुं तत्र विभीषणम्}
{सुग्रीवेणैवमुक्ते तु गच्छस्वेत्याह राघवः}% ५४

\twolineshloka
{सुग्रीवो राघवौ तौ च पुष्पके तु स्थितास्त्रयः}
{तावत्प्राप्तं विमानं तु समुद्रस्योत्तरं तटम्}% ५५

\twolineshloka
{अब्रवीद्भरतं रामो ह्यत्र मे राक्षसेश्वरः}
{चतुर्भिः सचिवैः सार्धं जीवितार्थे विभीषणः}% ५६

\twolineshloka
{प्राप्तस्ततो लक्ष्मणेन लङ्काराज्येभिषेचितः}
{अत्र चाहं समुद्रस्य परेपारे स्थितस्त्र्यहम्}% ५७

\twolineshloka
{दर्शनं दास्यते मेऽसौ ज्ञातिकार्यं भविष्यति}
{तावन्न दर्शनं मह्यं दत्तमेतेन शत्रुहन्}% ५८

\twolineshloka
{ततः कोपः सुमद्भूतश्चतुर्थेहनि राघव}
{धनुरायम्य वेगेन दिव्यमस्त्रं करे धृतम्}% ५९

\twolineshloka
{दृष्ट्वा मां शरणान्वेषी भीतो लक्ष्मणमाश्रितः}
{सुग्रीवेणानुनीतोऽस्मि क्षम्यतां राघव त्वया}% ६०

\twolineshloka
{ततो मयोत्क्षिप्तशरो मरुदेशे ह्यपाकृतः}
{ततस्समुद्रराजेन भृशं विनयशालिना}% ६१

\twolineshloka
{उक्तोहं सेतुबन्धेन लङ्कां त्वं व्रज राघव}
{लङ्घयित्वा नरव्याघ्र वारिपूर्णं महोदधिम्}% ६२

\twolineshloka
{एष सेतुर्मया बद्धः समुद्रे वरुणालये}
{त्रिभिर्दिनैः समाप्तिं वै नीतो वानरसत्तमैः}% ६३

\twolineshloka
{प्रथमे दिवसे बद्धो योजनानि चतुर्दश}
{द्वितीयेहनि षट्त्रिंशत्तृतीयेर्धशतं तथा}% ६४

\twolineshloka
{इयं सा दृश्यते लङ्का स्वर्णप्राकारतोरणा}
{अवरोधो महानत्र कृतो वानरसत्तमैः}% ६५

\twolineshloka
{अत्र युद्धं महद्वृत्तं चैत्राशुक्लचतुर्दशि}
{अष्टचत्वारिंशद्दिनं यत्रासौ रावणो हतः}% ६६

\twolineshloka
{अत्र प्रहस्तो नीलेन हतो राक्षसपुङ्गवः}
{हनूमता च धूम्राक्षो ह्यत्रैव विनिपातितः}% ६७

\twolineshloka
{महोदरातिकायौ च सुग्रीवेण महात्मना}
{अत्रैव मे कुम्भकर्णो लक्ष्मणेनेन्द्रजित्तथा}% ६८

\twolineshloka
{मया चात्र दशग्रीवो हतो राक्षसपुङ्गवः}
{अत्र सम्भाषितुं प्राप्तो ब्रह्मा लोकपितामहः}% ६९

\twolineshloka
{पार्वत्या सहितो देवः शूलपाणिर्वृषध्वजः}
{महेन्द्राद्याः सुरगणाः सगन्धर्वास्स किन्नराः}% ७०

\twolineshloka
{पिता मे च समायातो महाराजस्त्रिविष्टपात्}
{वृतश्चाप्सरसां सङ्घैर्विद्याधरगणैस्तथा}% ७१

\twolineshloka
{तेषां समक्षं सर्वेषां जानकी शुद्धिमिच्छता}
{उक्ता सीता हव्यवाहं प्रविष्टा शुद्धिमागता}% ७२

\twolineshloka
{लङ्काधिपैः सुरैर्दृष्टा गृहीता पितृशासनात्}
{अथाप्युक्तोथ राज्ञाहमयोध्यां गच्छ पुत्रकम्}% ७३

\twolineshloka
{न मे स्वर्गो बहुमतस्त्वया हीनस्य राघव}
{तारितोहं त्वया पुत्र प्राप्तोऽस्मीन्द्रसलोकताम्}% ७४

\twolineshloka
{लक्ष्मणं चाब्रवीद्राजा पुत्र पुण्यं त्वयार्जितम्}
{भ्रात्रासममथो दिव्यांल्लोकान्प्राप्स्यसि चोत्तमान्}% ७५

\twolineshloka
{आहूय जानकीं राजा वाक्यं चेदमुवाच ह}
{न च मन्युस्त्वया कार्यो भर्तारं प्रति सुव्रते}% ७६

\twolineshloka
{ख्यातिर्भविष्यत्येवाग्र्या भर्तुस्ते शुभलोचने}
{एवं वदति रामे तु पुष्पके च व्यवस्थिते}% ७७

\twolineshloka
{तत्र ये राक्षसवरास्ते गत्वाशु विभीषणम्}
{प्राप्तो रामः ससुग्रीवश्चारा इत्थं तदाऽवदन्}% ७८

\twolineshloka
{विभीषणस्तु तच्छ्रुत्वा रामागमनमन्तिके}
{चारांस्तान्पूजयामास सर्वकामधनादिभिः}% ७९

\twolineshloka
{अलङ्कृत्य पुरीं तां तु निष्क्रान्तः सचिवैः सह}
{दृष्ट्वा रामं विमानस्थं मेराविव दिवाकरम्}% ८०

\twolineshloka
{अष्टाङ्गप्रणिपातेन नत्वा राघवमब्रवीत्}
{अद्य मे सफलं जन्म प्राप्ताः सर्वे मनोरथाः}% ८१

\twolineshloka
{यद्दृष्टौ देवचरणौ जगद्वन्द्यावनिन्दितौ}
{कृतः श्लाघ्योस्म्यहं देव शक्रादीनां दिवौकसाम्}% ८२

\twolineshloka
{आत्मानमधिकं मन्ये त्रिदशेशात्पुरन्दरात्}
{रावणस्य गृहे दीप्ते सर्वरत्नोपशोभिते}% ८३

\twolineshloka
{उपविष्टे तु काकुत्स्थे अर्घं दत्वा विभीषणः}
{उवाच प्राञ्जलिर्भूत्वा सुग्रीवं भरतं तथा}% ८४

\twolineshloka
{इहागतस्य रामस्य यद्दास्ये न तदस्ति मे}
{इयं च लङ्का रामेण रिपुं त्रैलोक्यकण्टकम्}% ८५

\twolineshloka
{हत्वा तु पापकर्माणं दत्ता पूर्वं पुरी मम}
{इयं पुरी इमे दारा अमी पुत्रास्तथा ह्यहम्}% ८६

\twolineshloka
{सर्वमेतन्मया दत्तं सर्वमक्षयमस्तु ते}
{ततः प्रकृतयः सर्वा लङ्कावासिजनाश्च ये}% ८७

\twolineshloka
{आजग्मू राघवं द्रष्टुं कौतूहलसमन्विताः}
{उक्तो विभीषणस्तैस्तु रामं दर्शय नः प्रभो}% ८८

\twolineshloka
{विभीषणेन कथिता राघवाय महात्मने}
{तेषामुपायनं सर्वं भरतो रामचोदितः}% ८९

\twolineshloka
{जग्राह वानरेन्द्रश्च धनरत्नौघसञ्चयम्}
{एवं तत्र त्र्यहं रामो ह्यवसद्राक्षसालये}% ९०

\twolineshloka
{चतुर्थेहनि सम्प्राप्ते रामे चापि सभास्थिते}
{केकसी पुत्रमाहेदं रामं द्रक्ष्यामि पुत्रक}% ९१

\twolineshloka
{दृष्टे तस्मिन्महत्पुण्यं प्राप्यते मुनिसत्तमैः}
{विष्णुरेष महाभागश्चतुर्मूर्तिस्सनातनः}% ९२

\twolineshloka
{सीता लक्ष्मीर्महाभाग न बुद्धा साग्रजेन ते}
{पित्रा ते पूर्वमाख्यातं देवानां दिविसङ्गमे}% ९३

\twolineshloka
{कुले रघूणां वै विष्णुः पुत्रो दशरथस्य तु}
{भविष्यति विनाशाय दशग्रीवस्य रक्षसः}% ९४

\uvacha{विभीषण उवाच}

\twolineshloka
{एवं कुरुष्व वै मातर्गृहाण नवमं वरम्}
{पात्रं चन्दनसंयुक्तं दधिक्षौद्राक्षतैः सह}% ९५

\twolineshloka
{दूर्वयार्घं सह कुरु राजपुत्रस्य दर्शनम्}
{सरमामग्रतः कृत्वा याश्चान्या देवकन्यकाः}% ९६

\twolineshloka
{व्रजस्व राघवाभ्याशं तस्मादग्रे व्रजाम्यहम्}
{एवमुक्त्वा गतं रक्षो यत्र रामो व्यवस्थितः}% ९७

\twolineshloka
{उत्सार्य दानवान्सर्वान्रामं द्रष्टुं समागतान्}
{सभां तां विमलां कृत्वा रामं स्वाभिमुखे स्थितम्}% ९८

\uvacha{विभीषण उवाच}

\twolineshloka
{विज्ञाप्यं शृणु मे देव वदतश्च विशाम्पते}
{दशग्रीवं कुम्भकर्णं या च मां चाप्यजीजनत्}% ९९

\twolineshloka
{इयं सा देवमाता नः पादौ ते द्रष्टुमिच्छति}
{तस्यास्तु त्वं कृपां कृत्वा दर्शनं दातु मर्हसि1.38.}% १००

\uvacha{राम उवाच}

\twolineshloka
{अहं तस्याः समीपं तु मातृदर्शनकाङ्क्षया}
{गमिष्ये राक्षसेन्द्र त्वं शीघ्रं याहि ममाग्रतः}% १०१

\twolineshloka
{प्रतिज्ञाय तु तं वाक्यमुत्तस्थौ च वरासनात्}
{मूर्ध्नि चाञ्जलिमाधाय प्रणाममकरोद्विभुः}% १०२

\twolineshloka
{अभिवादयेहं भवतीं माता भवसि धर्मतः}
{महता तपसा चापि पुण्येन विविधेन च}% १०३

\twolineshloka
{इमौ ते चरणौ देवि मानवो यदि पश्यति}
{पूर्णस्स्यात्तदहं प्रीतो दृष्ट्वेमौ पुत्रवत्सले}% १०४

\twolineshloka
{कौसल्या मे यथा माता भवती च तथा मम}
{केकसी चाब्रवीद्रामं चिरं जीव सुखी भव}% १०५

\twolineshloka
{भर्त्रा मे कथितं वीर विष्णुर्मानुषरूपधृत्}
{अवतीर्णो रघुकुले हितार्थेत्र दिवौकसाम्}% १०६

\twolineshloka
{दशग्रीव विनाशाय भूतिं दातुं विभीषणे}
{वालिनो निधनं चैव सेतुबन्धं च सागरे}% १०७

\twolineshloka
{पुत्रो दशरथस्यैव सर्वं स च करिष्यति}
{इदानीं त्वं मया ज्ञातः स्मृत्वा तद्भर्तृभाषितम्}% १०८

\twolineshloka
{सीता लक्ष्मीर्भवान्विष्णुर्देवा वै वानरास्तथा}
{गृहं पुत्र गमिष्यामि स्थिरकीर्तिमवाप्नुहि}% १०९

\uvacha{सरमोवाच}

\twolineshloka
{इहैव वत्सरं पूर्णमशोकवनिकास्थिता}
{सेविता जानकी देव सुखं तिष्ठति ते प्रिया}% ११०

\twolineshloka
{नित्यं स्मरामि वै पादौ सीतायास्तु परन्तप}
{कदा द्रक्ष्यामि तां देवीं चिन्तयाना त्वहर्निशम्}% १११

\twolineshloka
{किमर्थं देवदेवेन नानीता जानकी त्विह}
{एकाकी नैव शोभेथा योषिता च तया विना}% ११२

\twolineshloka
{समीपे शोभते सीता त्वं च तस्याः परन्तप}
{एवं ब्रुवन्त्यां भरतः केयमित्यब्रवीद्वचः}% ११३

\twolineshloka
{ततश्चेङ्गितविद्रामो भरतं प्राह सत्वरम्}
{विभीषणस्य भार्या वै सरमा नाम नामतः}% ११४

\twolineshloka
{प्रिया सखी महाभागा सीतायास्सुदृढं मता}
{सर्वङ्कालकृतं पश्य न जाने किं करिष्यति}% ११५

\twolineshloka
{गच्छ त्वं सुभगे भर्तृगेहं पालय शोभने}
{मां त्यक्त्वा हि गता देवी भाग्यहीनं गतिर्यथा}% ११६

\twolineshloka
{तया विरहितः सुभ्रु रतिं विन्दे न कर्हिचित्}
{शून्या एव दिशः सर्वाः पश्यामीह पुनर्भ्रमन्}% ११७

\twolineshloka
{विसृज्यतां च सरमां सीतायास्तु प्रियां सखीम्}
{गतायामथ केकस्यां रामः प्राह विभीषणम्}% ११८

\twolineshloka
{दैवतेभ्यः प्रियं कार्यं नापराध्यास्त्वया सुराः}
{आज्ञया राजराजस्य वर्तितव्यं त्वयानघ}% ११९

\twolineshloka
{लङ्कायां मानुषो यो वै समागच्छेत्कथञ्चन}
{राक्षसैर्न च हन्तव्यो द्रष्टव्योसौ यथा त्वहम्}% १२०

\uvacha{विभीषण उवाच}

\twolineshloka
{आज्ञयाहं नरव्याघ्र करिष्ये सर्वमेव तु}
{विभीषणे हि वदति वायू राममुवाच ह}% १२१

\twolineshloka
{इहास्तिवैष्णवी मूर्तिः पूर्वं बद्धो बलिर्यया}
{तां नयस्व महाभाग कान्यकुब्जे प्रतिष्ठय}% १२२

\twolineshloka
{विदित्वा तदभिप्रायं वायुना समुदाहृतम्}
{विभीषणस्त्वलङ्कृत्य रत्नैः सर्वैश्च वामनम्}% १२३

\twolineshloka
{आनीय चार्पयद्रामे वाक्यं चेदमुवाच ह}
{यदा वै निर्जितः शक्रो मेघनादेन राघव}% १२४

\twolineshloka
{तदा वै वामनस्त्वेष आनीतो जलजेक्षण}
{नयस्व तमिमं देव देवदेवं प्रतिष्ठय}% १२५

\twolineshloka
{तथेति राघवः कृत्वा पुष्पकं च समारुहत्}
{धनं रत्नमसङ्ख्येयं वामनं च सुरोत्तमम्}% १२६

\twolineshloka
{गृह्य सुग्रीवभरतावारूढौ वामनादनु}
{व्रजन्नेवाम्बरे रामस्तिष्ठेत्याह विभीषणम्}% १२७

\twolineshloka
{राघवस्य वचः श्रुत्वा भूयोप्याह स राघवम्}
{करिष्ये सर्वमेतद्धि यदाज्ञप्तं विभो त्वया}% १२८

\twolineshloka
{सेतुनानेन राजेन्द्र पृथिव्यां सर्वमानवाः}
{आगत्य प्रतिबाधेरन्नाज्ञाभङ्गो भवेत्तव}% १२९

\twolineshloka
{कोत्र मे नियमो देव किन्नु कार्यं मया विभो}
{श्रुत्वैतद्राघवो वाक्यं राक्षसोत्तमभाषितम्}% १३०

\twolineshloka
{कार्मुकं गृह्य हस्तेन रामः सेतुं द्विधाच्छिनत्}
{त्रिर्विभज्य च वेगेन मध्ये वै दशयोजनम्}% १३१

\twolineshloka
{छित्वा तु योजनं चैकमेकं खण्डत्रयं कृतम्}
{वेलावनं समासाद्य रामः पूजां रमापतेः}% १३२

\twolineshloka
{कृत्वा रामेश्वरं नाम्ना देवदेवं जनार्दनम्}
{अभिषिच्याथ सङ्गृह्य वामनं रघुनन्दनः}% १३३

\twolineshloka
{दक्षिणादुदधेश्चैव निर्जगाम त्वरान्वितः}
{अन्तरिक्षादभूद्वाणी मेघगम्भीरनिःस्वना}% १३४

\uvacha{रुद्र उवाच}

\twolineshloka
{भो भो रामास्तु भद्रं ते स्थितोऽहमिह साम्प्रतम्}
{यावज्जगदिदं राम यावदेषा धरा स्थिता}% १३५

\twolineshloka
{तावदेव च ते सेतु तीर्थं स्थास्यति राघव}
{श्रुत्वैवं देवदेवस्य गिरं ताममृतोपमाम्}% १३६

\uvacha{राम उवाच}

\twolineshloka
{नमस्ते देवदेवेश भक्तानामभयङ्कर}
{गौरीकान्त नमस्तुभ्यं दक्षयज्ञविनाशन}% १३७

\twolineshloka
{नमो भवाय शर्वाय रुद्राय वरदाय च}
{पशूनाम्पतये नित्यं चोग्राय च कपर्दिने}% १३८

\twolineshloka
{महादेवाय भीमाय त्र्यम्बकाय दिशाम्पते}
{ईशानाय भगघ्नाय नमोस्त्वन्धकघातिने}% १३९

\twolineshloka
{नीलग्रीवाय घोराय वेधसे वेधसा स्तुत}
{कुमारशत्रुनिघ्नाय कुमारजननाय च}% १४०

\twolineshloka
{विलोहिताय धूम्राय शिवाय क्रथनाय च}
{नमो नीलशिखण्डाय शूलिने दैत्यनाशिने}% १४१

\twolineshloka
{उग्राय च त्रिनेत्राय हिरण्यवसुरेतसे}
{अनिन्द्यायाम्बिकाभर्त्रे सर्वदेवस्तुताय च}% १४२

\twolineshloka
{अभिगम्याय काम्याय सद्योजाताय वै नमः}
{वृषध्वजाय मुण्डाय जटिने ब्रह्मचारिणे}% १४३

\twolineshloka
{तप्यमानाय तप्याय ब्रह्मण्याय जयाय च}
{विश्वात्मने विश्वसृजे विश्वमावृत्य तिष्ठते}% १४४

\twolineshloka
{नमो नमोस्तु दिव्याय प्रपन्नार्तिहराय च}
{भक्तानुकम्पिने देव विश्वतेजो मनोगते}% १४५

\uvacha{पुलस्त्य उवाच}

\twolineshloka
{एवं संस्तूयमानस्तु देवदेवो हरो नृप}
{उवाच राघवं वाक्यं भक्तिनम्रं पुरास्थितम्}% १४६

\uvacha{रुद्र उवाच}

\twolineshloka
{भो भो राघव भद्रं ते ब्रूहि यत्ते मनोगतम्}
{भवान्नारायणो नूनं गूढो मानुषयोनिषु}% १४७

\twolineshloka
{अवतीर्णो देवकार्यं कृतं तच्चानघ त्वया}
{इदानीं स्वं व्रजस्थानं कृतकार्योसि शत्रुहन्}% १४८

\twolineshloka
{त्वया कृतं परं तीर्थं सेत्वाख्यं रघुनन्दन}
{आगत्य मानवा राजन्पश्येयुरिह सागरे}% १४९

\twolineshloka
{महापातकयुक्ता ये तेषां पापं विलीयते}
{ब्रह्मवध्यादिपापानि यानि कष्टानि कानिचित्1.38.}% १५०

\twolineshloka
{दर्शनादेव नश्यन्ति नात्र कार्या विचारणा}
{गच्छ त्वं वामनं स्थाप्य गङ्गातीरे रघूत्तम}% १५१

\twolineshloka
{पृथिव्यां सर्वशः कृत्वा भागानष्टौ परन्तप}
{श्वेतद्वीपं स्वकं स्थानं व्रज देव नमोस्तु ते}% १५२

\twolineshloka
{प्रणिपत्य ततो रामस्तीर्थं प्राप्तश्च पुष्करम्}
{विमानं तु न यात्यूर्ध्वं वेष्टितं तत्तु राघवः}% १५३

\twolineshloka
{किमिदं वेष्टितं यानं निरालम्बेऽम्बरे स्थितम्}
{भवितव्यं कारणेन पश्येत्याह स्म वानरम्}% १५४

\twolineshloka
{सुग्रीवो रामवचनादवतीर्य धरातले}
{स च पश्यति ब्रह्माणं सुरसिद्धसमन्वितम्}% १५५

\twolineshloka
{ब्रह्मर्षिसङ्घसहितं चतुर्वेदसमन्वितम्}
{दृष्ट्वाऽऽगत्याब्रवीद्रामं सर्वलोकपितामहः}% १५६

\twolineshloka
{सहितो लोकपालैश्च वस्वादित्यमरुद्गणैः}
{तं देवं पुष्पकं नैव लङ्घयेद्धि पितामहम्}% १५७

\twolineshloka
{अवतीर्य ततो रामः पुष्पकाद्धेमभूषितात्}
{नत्वा विरिञ्चनं देवं गायत्र्या सह संस्थितम्}% १५८

\twolineshloka
{अष्टाङ्गप्रणिपातेन पञ्चाङ्गालिङ्गितावनिः}
{तुष्टाव प्रणतो भूत्वा देवदेवं विरिञ्चनम्}% १५९

\uvacha{राम उवाच}

\twolineshloka
{नमामि लोककर्तारं प्रजापतिसुरार्चितम्}
{देवनाथं लोकनाथं प्रजानाथं जगत्पतिम्}% १६०

\twolineshloka
{नमस्ते देवदेवेश सुरासुरनमस्कृत}
{भूतभव्यभवन्नाथ हरिपिङ्गललोचन}% १६१

\twolineshloka
{बालस्त्वं वृद्धरूपी च मृगचर्मासनाम्बरः}
{तारणश्चासि देवस्त्वं त्रैलोक्यप्रभुरीश्वरः}% १६२

\twolineshloka
{हिरण्यगर्भः पद्मगर्भः वेदगर्भः स्मृतिप्रदः}
{महासिद्धो महापद्मी महादण्डी च मेखली}% १६३

\twolineshloka
{कालश्च कालरूपी च नीलग्रीवो विदांवरः}
{वेदकर्तार्भको नित्यः पशूनां पतिरव्ययः}% १६४

\twolineshloka
{दर्भपाणिर्हंसकेतुः कर्ता हर्ता हरो हरिः}
{जटी मुण्डी शिखी दण्डी लगुडी च महायशाः}% १६५

\twolineshloka
{भूतेश्वरः सुराध्यक्षः सर्वात्मा सर्वभावनः}
{सर्वगः सर्वहारी च स्रष्टा च गुरुरव्ययः}% १६६

\twolineshloka
{कमण्डलुधरो देवः स्रुक्स्रुवादिधरस्तथा}
{हवनीयोऽर्चनीयश्च ॐकारो ज्येष्ठसामगः}% १६७

\twolineshloka
{मृत्युश्चैवामृतश्चैव पारियात्रश्च सुव्रतः}
{ब्रह्मचारी व्रतधरो गुहावासी सुपङ्कजः}% १६८

\twolineshloka
{अमरो दर्शनीयश्च बालसूर्यनिभस्तथा}
{दक्षिणे वामतश्चापि पत्नीभ्यामुपसेवितः}% १६९

\twolineshloka
{भिक्षुश्च भिक्षुरूपश्च त्रिजटी लब्धनिश्चयः}
{चित्तवृत्तिकरः कामो मधुर्मधुकरस्तथा}% १७०

\twolineshloka
{वानप्रस्थो वनगत आश्रमी पूजितस्तथा}
{जगद्धाता च कर्त्ता च पुरुषः शाश्वतो ध्रुवः}% १७१

\twolineshloka
{धर्माध्यक्षो विरूपाक्षस्त्रिधर्मो भूतभावनः}
{त्रिवेदो बहुरूपश्च सूर्यायुतसमप्रभः}% १७२

\twolineshloka
{मोहकोवन्धकश्चैवदानवानांविशेषतः}
{देवदेवश्च पद्माङ्कस्त्रिनेत्रोऽब्जजटस्तथा}% १७३

\twolineshloka
{हरिश्मश्रुर्धनुर्धारी भीमो धर्मपराक्रमः}
{एवं स्तुतस्तु रामेण ब्रह्मा ब्रह्मविदांवरः}% १७४

\twolineshloka
{उवाच प्रणतं रामं करे गृह्य पितामहः}
{विष्णुस्त्वं मानुषे देहेऽवतीर्णो वसुधातले}% १७५

\twolineshloka
{कृतं तद्भवता सर्वं देवकार्यं महाविभो}
{संस्थाप्य वामनं देवं जाह्नव्या दक्षिणे तटे}% १७६

\twolineshloka
{अयोध्यां स्वपुरीं गत्वा सुरलोकं व्रजस्व च}
{विसृष्टो ब्रह्मणा रामः प्रणिपत्य पितामहम्}% १७७

\twolineshloka
{आरूढः पुष्पकं यानं सम्प्राप्तो मधुरां पुरीम्}
{समीक्ष्य पुत्रसहितं शत्रुघ्नं शत्रुघातिनम्}% १७८

\twolineshloka
{तुतोष राघवः श्रीमान्भरतः स हरीश्वरः}
{शत्रुघ्नो भ्रातरौ प्राप्तौ शक्रोपेन्द्राविवागतौ}% १७९

\twolineshloka
{प्रणिपत्य ततो मूर्ध्ना पञ्चाङ्गालिङ्गितावनिः}
{उत्थाप्य चाङ्कमारोप्य रामो भ्रातरमञ्जसा}% १८०

\twolineshloka
{भरतश्च ततः पश्चात्सुग्रीवस्तदनन्तरम्}
{उपविष्टोऽथ रामाय सोऽर्घमादाय सत्वरम्}% १८१

\twolineshloka
{राज्यं निवेदयामास चाष्टाङ्गं राघवे तदा}
{श्रुत्वा प्राप्तं ततो रामं सर्वो वै माथुरो जनः}% १८२

\twolineshloka
{वर्णा ब्राह्मणभूयिष्ठा द्रष्टुमेनं समागताः}
{सम्भाष्य प्रकृतीः सर्वा नैगमान्ब्राह्मणैः सह}% १८३

\twolineshloka
{दिनानि पञ्चोषित्वाऽत्र रामो गन्तुं मनो दधे}
{शत्रुघ्नश्च ततो रामे वाजिनोथ गजांस्तथा}% १८४

\twolineshloka
{कृताकृतं च कनकं तत्रोपायनमाहरत्}
{रामस्त्वाह ततः प्रीतः सर्वमेतन्मया तव}% १८५

\twolineshloka
{दत्तं पुत्रौ तेऽभिषिञ्च राजानौ माथुरे जने}
{एवमुक्त्वा ततो रामः प्राप्तो मध्यन्दिने रवौ}% १८६

\twolineshloka
{महोदयं समासाद्य गङ्गातीरे स वामनम्}
{प्रतिष्ठाप्य द्विजानाह भाविनः पार्थिवांस्तथा}% १८७

\twolineshloka
{मया कृतोऽयं धर्मस्य सेतुर्भूतिविवर्धनः}
{प्राप्ते काले पालनीयो न च लोप्यः कथञ्चन}% १८८

\twolineshloka
{प्रसारितकरेणैवं प्रार्थनैषा मया कृता}
{नृपाः कृते मयार्थित्वे यत्क्षेमं क्रियतामिह}% १८९

\twolineshloka
{नित्यं दैनन्दिनीपूजा कार्या सर्वैरतन्द्रितैः}
{ग्रामान्दत्वा धनं तच्च लङ्काया आहृतं च यत्}% १९०

\twolineshloka
{प्रेषयित्वा च किष्किन्धां सुग्रीवं वानरेश्वरम्}
{अयोध्यामागतो रामः पुष्पकं तमथाब्रवीत्}% १९१

\twolineshloka
{नागन्तव्यं त्वया भूयस्तिष्ठ यत्र धनेश्वरः}
{कृतकृत्यस्ततो रामः कर्तव्यं नाप्यमन्यत}% १९२

\uvacha{पुलस्त्य उवाच}

\twolineshloka
{एवन्ते भीष्म रामस्य कथायोगेन पार्थिव}
{उत्पत्तिर्वामनस्योक्ता किं भूयः श्रोतुमिच्छसि}% १९३

\twolineshloka
{कथयामि तु तत्सर्वं यत्र कौतूहलं नृप}
{सर्वं ते कीर्त्तयिष्यामि येनार्थी नृपनन्दन}% १९४

{॥इति श्रीपाद्मपुराणे प्रथमे सृष्टिखण्डे वामनप्रतिष्ठानामाष्टत्रिंशोऽध्यायः॥३८॥}

