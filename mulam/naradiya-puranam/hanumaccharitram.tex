\sect{हनुमच्चरित्रम् --- एकोनाशीतितमोऽध्यायः}

\uvacha{सनत्कुमार उवाच}

\twolineshloka
{अथापरं वायुसूनोश्चरितं पापनाशनम्}
{यदुक्तं स्वासु रामेण आनन्दवनवासिना}% १

\twolineshloka
{सद्योजाते महाकल्पे श्रुतवीर्ये हनूमति}
{मम श्रीरामचन्द्र स्य भक्तिरस्तु सदैव हि}% २

\twolineshloka
{शृणुष्व गदतो मत्तः कुमारस्य कुमारक}
{चरितं सर्वपापघ्नं शृण्वतां पठतां सदा}% ३

\twolineshloka
{वाञ्छाम्यहं सदा विप्र सङ्गमं कीशरूपिणा}
{रहस्यं रहसि स्वस्य ममानन्दवनोत्तमे}% ४

\twolineshloka
{परीतेऽत्र सखायो मे सख्यश्च विगतज्वराः}
{क्रीडन्ति सर्वदा चात्र प्राकट्येऽपि रहस्यपि}% ५

\twolineshloka
{कस्मिंश्चिदवतारे तु यद्वृत्तं च रहो मम}
{तदत्र प्रकटं तुभ्यं करोमि प्रीतमानसः}% ६

\twolineshloka
{आविर्भूतोऽस्म्यहं पूर्वं राज्ञो दशरथक्षये}
{चतुर्व्यूहात्मकस्तत्र तस्य भार्यात्रये मुने}% ७

\twolineshloka
{ततः कतिपयैरब्दैरागतो द्विजपुङ्गवः}
{क्श्विमित्रोऽथयामास पितरं मम भूपतिम्}% ८

\twolineshloka
{यक्षरक्षोविघातार्थं लक्ष्मणेन सहैव माम्}
{प्रेषयामास धर्मात्मा सिद्धाश्रममरण्यकम्}% ९

\twolineshloka
{तत्र गत्वाश्रममृषेर्दूषयन्तौ निशाचरौ}
{ध्वस्तौ सुबाहुमारीचौ प्रसन्नोऽभूत्तदा मुनिः}% १०

\twolineshloka
{अस्त्रग्रामं ददौ मह्यं मासं चावासयत्तथा}
{ततो गाधिसुतो धीमान् ज्ञात्वा भाव्यर्थमादरात्}% ११

\twolineshloka
{मिथिलामनयत्तत्र रौद्रं चादर्शयद्धनुः}
{तस्य कन्यां पणीभूतां सीतां सुरसुतोपमाम्}% १२

\twolineshloka
{धनुर्विभज्य समिति लब्धवान्मानिनोऽस्य च}
{ततो मार्गे भृगुपतेर्दर्प्पमूढं चिरं स्मयन्}% १३

\twolineshloka
{व्यपनीयागमं पश्चादयोध्यां स्वपितुः पुरीम्}
{ततो राज्ञाहमाज्ञाय प्रजाशीलनमानसः}% १४

\twolineshloka
{यौवराज्ये स्वयं प्रीत्या सम्मन्त्र्याप्तैर्विकल्पितः}
{तच्छ्रुत्वा सुप्रिया भार्या कैकेयी भूपतिं मुने}% १५

\twolineshloka
{देवकार्यविधानार्थं विदूषितमतिर्जगौ}
{पुत्रो मे भरतो नाम यौवराज्येऽभिषिच्यताम्}% १६

\twolineshloka
{रामश्चतुर्दशसमा दण्डकान्प्रविवास्यताम्}
{तदाकर्ण्याहमुद्युक्तोऽरण्यं भार्यानुजान्वितः}% १७

\twolineshloka
{गन्तुं नृपतिनानुक्तोऽप्यगमं चित्रकूटकम्}
{तत्र नित्यं वन्यफलैर्मांसैश्चावर्तितक्रियः}% १८

\twolineshloka
{निवसन्नेव राज्ञस्तु निधनं चाप्यवागमम्}
{ततो भरतशत्रुघ्नौ भ्रातरौ मम मानदौ}% १९

\twolineshloka
{मातृवर्गयुतौ दीनौ साचार्यामात्यनागरौ}
{व्यजिज्ञपतमागत्य पञ्चवट्यां निजाश्रमम्}% २०

\twolineshloka
{अकल्पयं भ्रातृभार्यासहितश्च त्रिवत्सरम्}
{ततस्त्रयोदशे वर्षे रावणो नाम राक्षसः}% २१

\twolineshloka
{मायया हृतवान्सीतां प्रियां मम परोक्षतः}
{ततोऽहं दीनवदनः ऋष्यमूकं हि पर्वतम्}% २२

\twolineshloka
{भार्यामन्वेषयन्प्राप्तः सख्यं हर्यधिपेन च}
{अथ वालिनमाहत्य सुग्रीवस्तत्पदे कृतः}% २३

\twolineshloka
{सह वानरयूथैश्च साहाय्यं कृतवान्मम}
{विरुध्य रावणेनालं मम भक्तो विभीषणः}% २४

\twolineshloka
{आगतो ह्यभिषिच्याशु लङ्केशो हि विकल्पितः}
{हत्वा तु रावणं सङ्ख्ये सपुत्रामात्यबान्धवम्}% २५

\twolineshloka
{सीतामादाय संशुद्धामयोध्यां समुपागतः}
{ततः कालान्तरे विप्र सुग्रीवश्च विभीषणः}% २६

\twolineshloka
{निमन्त्रितौ पितुः श्राद्धे षट्कुलाश्च द्विजोत्तमाः}
{अयोध्यायां समाजग्मुस्ते तु सर्वे निमन्त्रिताः}% २७

\twolineshloka
{ऋते विभीषणं तत्र चिन्तयाने रघूत्तमे}
{शम्भुर्बाह्मणरूपेण षट्कुलैश्च सहागतः}% २८

\twolineshloka
{अथ पृष्टो मया शम्भुर्विभीषणसमागमे}
{नीत्वा मां द्र विडे देशे मोचय द्विजबन्धनात्}% २९

\twolineshloka
{मया निमन्त्रिताः श्राद्धे ह्यगस्त्याद्या मुनीश्वराः}
{सम्भोजितास्तु प्रययुः स्वस्वमाश्रममण्डलम्}% ३०

\twolineshloka
{ततः कालान्तरे विप्रा देवा दैत्या नरेश्वराः}
{गौतमेन समाहूताः सर्वे यज्ञसभाजिताः}% ३१

\twolineshloka
{ते सर्वे स्फाटिकं लिङ्गं त्र्! यम्बकाद्रौ निवेशितम्}
{सम्पूज्य न्यवंसस्तत्र देवदैत्यनृपाग्रजाः}% ३२

\twolineshloka
{तस्मिन्समाजे वितते सर्वैर्लिङ्गे समर्चिते}
{गौतमोऽप्यथ मध्याह्ने पूजयामास शङ्करम्}% ३३

\twolineshloka
{सर्वे शुक्लाम्बरधरा भस्मोद्धूलितविग्रहाः}
{सितेन भस्मना कृत्वा सर्वस्थाने त्रिपुण्ड्रकम्}% ३४

\twolineshloka
{नत्वा तु भार्गवं सर्वे भूतशुद्धिं प्रचक्रमुः}
{हृत्पद्ममध्ये सुषिरं तत्रैव भूतपञ्चकम्}% ३५

\twolineshloka
{तेषां मध्ये महाकाशमाकाशे निर्मलामलम्}
{तन्मध्ये च महेशानं ध्यायेद्दीप्तिमयं शुभम्}% ३६

\twolineshloka
{अज्ञानसंयुतं भूतं समलं कर्मसङ्गतः}
{तं देहमाकाशदीपे प्रदहेज्ज्ञानवह्निना}% ३७

\twolineshloka
{आकाशस्यावृतिं चाहं दग्ध्वाकाशमथो दहेत्}
{दग्ध्वाकाशमथो वायुमग्निभूतं तथा दहेत्}% ३८

\twolineshloka
{अब्भूतं च ततो दग्ध्वा पृथिवीभूतमेव च}
{तदाश्रितान्गुणान्दग्ध्वा ततो देहं प्रदाहयेत्}% ३९

\twolineshloka
{एवं प्रदग्ध्वा भूतादि देही तज्ज्ञानवह्निना}
{शिखामध्यस्थितं विष्णुमानन्दरसनिर्भरम्}% ४०

\twolineshloka
{निष्पन्नचन्द्र किरणसङ्काशकिरणं शिवम्}
{शिवाङ्गोत्पन्नकिरणैरमृतद्र वसंयुतैः}% ४१
सुशीतला ततो ज्वाला प्रशान्ता चन्द्र रश्मिवत्

\twolineshloka
{प्रसारितसुधारुग्भिः सान्द्री भूतश्च सम्प्लवः}
{अनेन प्लावितं भूतग्रामं सञ्चिन्तयेत्परम्}% ४२

\twolineshloka
{इत्थं कृत्वा भूतशुद्धिं क्रियार्हो मर्त्यः शुद्धो जायते ह्येव सद्यः}
{पूजां कर्तुं जप्यकर्मापि पश्चादेवं ध्यायेद्ब्रह्महत्यादिशुद्ध्यै}% ४३

\twolineshloka
{एवं ध्यात्वा चन्द्र दीप्तिप्रकाशं ध्यानेनारोप्याशु लिङ्गे शिवस्य}
{सदाशिवं दीपमध्ये विचिन्त्य पञ्चाक्षरेणार्चनमव्ययं तु}% ४४

\twolineshloka
{आवाहनादीनुपचारांस्तथापि कृत्वा स्नानं पूर्ववच्छङ्करस्य}
{औदुम्बरं राजतं स्वर्णपीठं वस्त्रादिच्छन्नं सर्वमेवेह पीठम्}% ४५

\twolineshloka
{अन्ते कृत्वा बुद्बुदाभ्यां च सृष्टिं पीठे पीठे नागमेकं पुरस्तात्}
{कुर्यात्पीठे चोर्द्ध्वके नागयुग्मं देवाभ्याशे दक्षिणे वामतश्च}% ४६

\twolineshloka
{जपापुष्पं नागमध्ये निधाय मध्ये वस्त्रं द्वादशप्रातिगुण्ये}
{सुश्वेतेन तस्य मध्ये महेशं लिङ्गाकारं पीठयुक्तं प्रपूज्यम्}% ४७

\twolineshloka
{एवं कृत्वा साधकास्ते तु सर्वे दत्त्वा दत्त्वा पञ्चगन्धाष्टगन्धम्}
{पुष्पैः पत्रैः श्रीतिलैरक्षतैश्च तिलोन्मिश्रैः केवलैश्च प्रपूज्य}% ४८

\twolineshloka
{धूपं दत्त्वा विधिवत्सम्प्रयुक्तं दीपं दत्त्वा चोक्तमेवोपहारम्}
{पूजाशेषं ते समाप्याथ सर्वे गीतं नृत्यं तत्र तत्रापि चक्रुः}% ४९}

\onelineshloka
{काले चास्मिन्सुव्रते गौतमस्य शिष्यः प्राप्तः शङ्करात्मेति नाम्ना}% ५०

\twolineshloka
{उन्मत्तवेषो दिग्वासा अनेकां वृत्तिमास्थितः}
{क्वचिद्द्विजातिप्रवरः क्वचिच्चण्डालसन्निभः}% ५१

\twolineshloka
{क्वच्छ्द्र समो योगी तापसः क्वचिदप्युत}
{गर्जत्युत्पतते चैव नृत्यति स्तौति गायति}% ५२

\twolineshloka
{रोदिति शृणुतेऽत्युक्तं पतत्युत्तिष्ठति क्वचित्}
{शिवज्ञानैकसम्पन्नः परमानन्दनिर्भरः}% ५३

\twolineshloka
{सम्प्राप्तो भोज्यवेलायां गौतमस्यान्तिकं ययौ}
{बुभुजे गुरुणा साकं क्वचिदुच्छिष्टमेव च}% ५४

\twolineshloka
{क्वचिल्लिहति तत्पात्रं तूष्णीमेवाभ्यगात्क्वचित्}
{हस्तं गृहीत्वैव गुरोः स्वयमेवाभुनक्क्वचित्}% ५५

\twolineshloka
{क्वचिद् गृहान्तरे मूत्रं क्वचित्कर्दमलेपनम्}
{सर्वदा तं गुरुर्दृष्ट्वा करमालम्ब्य मन्दिरम्}% ५६

\twolineshloka
{प्रविश्य स्वीयपीठे तमुपवेश्याप्यभोजयत्}
{स्वयं तदस्य पात्रेण बुभुजे गौतमो मुनिः}% ५७

\threelineshloka
{तस्य चित्तं परिज्ञातुं कदाचिदथ सुन्दरी}
{अहल्या शिष्यमाहूय भुङ्क्ष्वेति प्राह तं मुदा}
{निर्दिष्टो गुरुपत्न्या तु बुभुजे सोऽविशेषतः}% ५८

\twolineshloka
{यथा पपौ हि पानीयं तथा वह्निमपि द्विज}
{कन्टकानन्नवद्भुक्त्वा यथापूर्वमतिष्ठत}% ५९

\twolineshloka
{पुरो हि मुनिकन्याभिराहूतो भोजनाय च}
{दिने दिने तत्प्रदत्तं लोष्टमम्बु च गोमयम्}% ६०

\twolineshloka
{कर्दमं काष्ठदण्डं च भुक्त्वा पीत्वाथ हर्षितः}
{एतादृशो मुनिरसौ चण्डालसदृशाकृतिः}% ६१

\twolineshloka
{सुजीर्णोपानहौ हस्ते गृहीत्वा प्रलपन्हसन्}
{अन्त्यजोचितवेषश्च वृषपर्वाणमभ्यगात्}% ६२

\twolineshloka
{वृषपर्वेशयोर्मध्ये दिग्वासाः समतिष्ठत}
{वृषपर्वा तमज्ञात्वा पीडयित्वा शिरोऽच्छिनत्}% ६३

\twolineshloka
{हते तस्मिन्द्विजश्रेष्ठे जगदेतच्चराचरम्}
{अतीव कलुषं ह्यासीत्तत्रस्था मुनयस्तथा}% ६४

\twolineshloka
{गौतमस्य महाशोकः सञ्जातः सुमहात्मनः}
{निर्ययौ चक्षुषो वारि शोकं सन्दर्शयन्निव}% ६५

\twolineshloka
{गौतमः सर्वदैत्यानां सन्निधौ वाक्यमुक्तवान्}
{किमनेन कृतं पापं येन च्छिन्नमिदं शिरः}% ६६

\twolineshloka
{मम प्राणाधिकस्येह सर्वदा शिवयोगिनः}
{ममापि मरणं सत्यं शिष्यच्छद्मा यतो गुरुः}% ६७

\twolineshloka
{शैवानां धर्मयुक्तानां सर्वदा शिववर्तिनाम्}
{मरणं यत्र दृष्टं स्यात्तत्र नो मरणं ध्रुवम्}% ६८

\twolineshloka
{तच्छ्रुत्वा ह्यसुराचार्यः शुक्रः प्राह विदांवरः}
{एनं सञ्जीवयिष्यामि भार्गवं शङ्करप्रियम्}% ६९

\twolineshloka
{किमर्थं म्रियते ब्रह्मन्पश्य मे तपसो बलम्}
{इति वादिनि विप्रेन्द्रे गौतमोऽपि ममार ह}% ७०

\twolineshloka
{तस्मिन्मृतेऽथ शुक्रोऽपि प्राणांस्तत्याज योगतः}
{तस्यैवं हतिमाज्ञाय प्रह्लादाद्या दितीश्वराः}% ७१

\twolineshloka
{देवा नृपा द्विजाः सर्वे मृता आसंस्तदद्भुतम्}
{मृतमासीदथ बलं तस्य बाणस्य धीमतः}% ७२

\twolineshloka
{अहल्याशोकसन्तप्ता रुरोदोच्चैः पुनः पुनः}
{गौतमेन महेशस्य पूजया पूजितो विभुः}% ७३

\twolineshloka
{वीरभद्रो महायोगी सर्वं दृष्ट्वा चुकोप ह}
{अहो कष्टमहो कष्टं महेशा बहवो हताः}% ७४

\twolineshloka
{शिवं विज्ञापयिष्यामि तेनोक्तं करवाण्यथ}
{इति निश्चित्य गतवान्मन्दराचलमव्ययम्}% ७५

\twolineshloka
{नमस्कृत्वा विरूपाक्षं वृत्तं सर्वमथोक्तवान्}
{ब्रह्माणं च हरिं तत्र स्थितौ प्राह शिवो वचः}% ७६

\twolineshloka
{मद्भक्तैः साहसं कर्म कृतं ज्ञात्वा वरप्रदम्}
{गत्वा पश्यामि हे विष्णो सर्वं तत्कृतसाहसम्}% ७७

\twolineshloka
{इत्युक्त्वा वृषमारुह्य वायुना धूतचामरः}
{नन्दिकेन सुवेषेण धृते छत्रेऽतिशोभने}% ७८

\twolineshloka
{सुश्वेते हेमदण्डे च नान्ययोग्ये धृते विभो}
{महेशानुमतिं लब्ध्वा हरिर्नागान्तके स्थितः}% ७९

\twolineshloka
{आरक्तनीलच्छत्राभ्यां शुशुभे लक्ष्मकौस्तुभः}
{शिवानुमत्या ब्रह्मापि हंसारूढोऽभवत्तदा}% ८०

\twolineshloka
{इन्द्र गोपप्रभाकारच्छत्राभ्यां शुशुभे विधिः}
{इन्द्रा दिसर्वदेवाश्च स्वस्ववाहनसंयुताः}% ८१

\twolineshloka
{अथ ते निर्ययुः सर्वे नानावाद्यानुमोदिताः}
{कोटिकोटिगणाकीर्णा गौतमस्याश्रमं गताः}% ८२

\twolineshloka
{ब्रह्मविष्णुमहेशाना दृष्ट्वा तत्परमाद्भुतम्}
{स्वभक्तं जीवयामास वामकोणनिरीक्षणात्}% ८३

\twolineshloka
{शङ्करो गौतमं प्राह तुष्टोऽहं ते वरं वृणु}
{तदाकर्ण्य वचस्तस्य गौतमः प्राह सादरम्}% ८४

\twolineshloka
{यदि प्रसन्नो देवेश यदि देयो वरो मम}
{त्वल्लिङ्गार्चनसामर्थ्यं नित्यमस्तु ममेश्वर}% ८५

\twolineshloka
{वृतमेतन्मया देव त्रिनेत्र शृणु चापरम्}
{शिष्योऽय मे महाभागो हेयादेयादिवर्जितः}% ८६

\twolineshloka
{प्रेक्षणीयं ममत्वेन न च पश्यति चक्षुषा}
{न घ्राणग्राह्यं देवेश न पातव्यं न चेतरम्}% ८७

\twolineshloka
{इति बुद्ध्या तथा कुर्वन्स हि योगी महायशाः}
{उन्मत्तविकृताकारः शङ्करात्मेति कीर्तितः}% ८८

\twolineshloka
{न कश्चित्तं प्रति द्वेषी न च तं हिंसयेदपि}
{एतन्मे दीयतां देव मृतानाममृतिस्तथा}% ८९

\twolineshloka
{तच्छ्रुत्वोमापतिः प्रीतो निरीक्ष्य हरिमव्ययः}
{स्वांशेन वायुना देहमाविशज्जगदीश्वरः}% ९०

\twolineshloka
{हरिरूपः शङ्करात्मा मारुतिः कपिसत्तमः}
{पर्यायैरुच्यतेऽधीशः साक्षाद्विष्णुः शिवः परः}% ९१

\twolineshloka
{आकल्पमेष प्रत्येकं कामरूपमुपाश्रितः}
{ममाज्ञाकारको रामभक्तः पूजितविग्रहः}% ९२

\twolineshloka
{अनन्तकल्पमीशानः स्थास्यति प्रीतमानसः}
{त्वया कृतमिदं वेश्म विस्तृतं सुप्रतिष्ठितम्}% ९३

\twolineshloka
{नित्यं वै सर्वरूपेण तिष्ठामः क्षणमादरात्}
{समर्चिताः प्रयास्यामः स्वस्ववासं ततः परम्}% ९४

\twolineshloka
{अथाबभाषे विश्वेशं गौतमो मुनिपुङ्गवः}
{अयोग्यं प्रार्थयामीश ह्यर्थी दोषं न पश्यति}% ९५

\twolineshloka
{ब्रह्माद्यलभ्यं देवेश दीयतां यदि रोचते}
{अथेशो विष्णुमालोक्य गृहीत्वा तत्करं करे}% ९६

\twolineshloka
{प्रहसन्नम्बुजाभाक्षमित्युवाच सदाशिवः}
{क्षामोदरोऽसि गोविन्द देयं ते भोजनं किमु}% ९७

\twolineshloka
{स्वयं प्रविश्य यदि वा स्वयं भुङ्क्ष्व स्वगेहवत्}
{गच्छ वा पार्वतीगेहं या कुक्षिं पूरयिष्यति}% ९८

\twolineshloka
{इत्युक्त्वा तत्करालम्बी ह्येकान्तमगमद्विभुः}
{आदिश्य नन्दिनं देवो द्वाराध्यक्षं यथोक्तवत्}% ९९

\twolineshloka
{स गत्वा गौतमं वाथ ह्युक्तवान्विष्णुभाषणम्}
{सम्पादयान्नं देवेशा भोक्तुकामा वयं मुने}% १००

\twolineshloka
{इत्युक्त्वैकान्तमगमद्वासुदेवेन शङ्करः}
{मृदुशय्यां समारुह्य शयितौ देवतोत्तमौ}% १०१

\twolineshloka
{अन्योन्यं भाषणं कृत्वा प्रोत्तस्थतुरुभावपि}
{गत्वा तडागं गम्भीरं स्नास्यन्तौ देवसत्तमौ}% १०२

\twolineshloka
{कराम्बुपातमन्योन्यं पृथक्कृत्वोभयत्र च}
{मुनयो राक्षसाश्चैव जलक्रीडां प्रचक्रिरे}% १०३

\twolineshloka
{अथ विष्णुर्महेशश्च जलपानानि शीघ्रतः}
{चक्रतुः शङ्करः पद्मकिञ्जल्काञ्जलिना हरेः}% १०४

\twolineshloka
{अवाकिरन्मुखे तस्य पद्मोत्फुल्लविलोचने}
{नेत्रे केशरसम्पातात्प्रमीलयत केशवः}% १०५

\twolineshloka
{अत्रान्तरे हरेः स्कन्धमारुरोह महेश्वरः}
{हर्युत्तमाङ्गं बाहुभ्यां गृहीत्वा सन्न्यमज्जयत्}% १०६

\twolineshloka
{उन्मज्जयित्वा च पुनः पुनश्चापि पुनः पुनः}
{पीडितं स हरिः सूक्ष्मं पातयामास शङ्करम्}% १०७

\twolineshloka
{अथ पादौ गृहीत्वा तं भ्रामयन्विचकर्ष ह}
{अताडयद्धरेर्वक्षः पातयामास चाच्युतम्}% १०८

\twolineshloka
{अथोत्थितो हरिस्तोयमादायाञ्जलिना ततः}
{शीर्षे चैवाकिरच्छम्भुमथ शम्भुरथो हरेः}% १०९

\twolineshloka
{जलक्रीडैवमभवदथ चर्षिगणान्तरे}
{जलक्रीडासम्भ्रमेण विस्रस्तजटबन्धनाः}% ११०

\twolineshloka
{अथ सम्भ्रमतां तेषामन्योन्यजटबन्धनम्}
{इतरेतरबद्धासु जटासु च मुनीश्वराः}% १११

\twolineshloka
{शक्तिमन्तोऽशक्तिमत आकर्षन्ति च सव्यथम्}
{पातयन्तोऽन्यतश्चापि क्रोशन्तो रुदतस्तथा}% ११२

\twolineshloka
{एवं प्रवृत्ते तुमुले सम्भूते तोयकर्मणि}
{आकाशे वानरेशस्तु ननर्त च ननाद च}% ११३

\twolineshloka
{विपञ्चीं वादयन्वाद्यं ललितां गीतिमुज्जगौ}
{सुगीत्या ललिता यास्तु अगायत विधा दश}% ११४

\twolineshloka
{शुश्राव गीतिं मधुरां शङ्करो लोकभावतः}
{स्वयं गातुं हि ललितं मन्दं मन्दं प्रचक्रमे}% ११५

\twolineshloka
{स्वयं गायति देवेशे विश्रामं गलदेशिकम्}
{स्वरं ध्रुवं समादाय सर्वलक्षणसंयुतम्}% ११६

\twolineshloka
{स्वधारामृतसंयुक्तं गानेनैवमपोनयन्}
{वासुदेवो मर्दलं च कराभ्यामप्यवादयत्}% ११७

\twolineshloka
{अम्बुजाङ्गश्चतुर्वक्त्रस्तुम्बुरुर्मुखरो बभौ}
{तानका गौतमाद्यास्तु गायको वायुजोऽभवत्}% ११८

\twolineshloka
{गायके मधुरं गीतं हनूमति कपीश्वरे}
{म्लानमम्लानमभवत्कृशाः पुष्टास्तदाभवन्}% ११९

\twolineshloka
{स्वां स्वां गीतिमतः सर्वे तिरस्कृत्यैव मूर्च्छिताः}
{तूष्णीभूतं समभवद्देवर्षिगणदानवम्}% १२०

\threelineshloka
{एकः स हनुमान् गाता श्रोतारः सर्व एव ते}
{मध्याह्नकाले वितते गायमाने हनूमति}
{स्वस्ववाहनमारुह्य निर्गताः सर्वदेवताः}% १२१

\twolineshloka
{गानप्रियो महेशस्तु जग्राह प्लवगेश्वरम्}
{प्लवग त्वं मयाज्ञप्तो निःशङ्को वृषमारुह}% १२२

\twolineshloka
{मम चाभिमुखो भूत्वा गायस्वानेकगायनम्}
{अथाह कपिशार्दूलो भगवन्तं महेश्वरम्}% १२३

\twolineshloka
{वृषभारोहसामर्थ्यं तव नान्यस्य विद्यते}
{तव वाहनमारुह्य पातकी स्यामहं विभो}% १२४

\twolineshloka
{मामेवारुह देवेश विहङ्गः शिवधारणः}
{तव चाभिमुखं गानं करिष्यामि विलोकय}% १२५

\twolineshloka
{अथेश्वरो हनूमन्तमारुरोह यथा बृषम्}
{आरूढे शङ्करे देवे हनुमत्कन्धरां शिवः}% १२६

\twolineshloka
{छित्वा त्वचं परावृत्य सुखं गायति पूर्ववत्}
{शृण्वन्गीतिसुधां शम्भुर्गौतमस्य गृहं ततः}% १२७

\twolineshloka
{सर्वे चाप्यागतास्तत्र देवर्षिगणदानवाः}
{पूजिता गौतमेनाथ भोजनावसरे सति}% १२८

\twolineshloka
{यच्छुष्कं दारुसम्भूतं गृहोपकरणादिकम्}

\twolineshloka
{प्ररूढमभवत्सर्वं गायमाने हनूमति}% १२९}
{तस्मिन्गाने समस्तानां चित्रदृष्टिरतिष्ठत}% १३०

\twolineshloka
{द्विबाहुरीशस्य पदाभिवन्दनः समस्तगात्राभरणोपपन्नः}
{प्रसन्नमूर्तिस्तरुणः सुमध्ये विन्यस्तमूर्द्धाञ्जलिभिः शिरोभिः}% १३१

\twolineshloka
{शिरः कराभ्यां परिगृह्य शङ्करो हनुमतं पूर्वमुखं चकार}
{पद्मासनासीनहनूमतोऽञ्जलौ निधाय पादं त्वपरं मुखे च}% १३२

\twolineshloka
{पादाङ्गुलीभ्यामथ नासिकां विभुः स्नेहेन जग्राह च मन्दमन्दम्}
{स्कन्धे मुखे त्वंसतले च कण्ठे वक्षस्थले च स्तनमध्यमे हृदि}% १३३

\twolineshloka
{ततश्च कुक्षावथ नाभिमण्डले पादं द्वितीयं विदधाति चाञ्जलौ}
{शिरो गृहीत्वाऽवनमय्य शङ्करः पस्पर्श पृष्ठं चिबुकेन सोऽध्वनि}% १३४}

\onelineshloka
{हारं च मुक्तापरिकल्पितं शिवो हनूमतः कण्ठगतं चकार}% १३५

\twolineshloka
{अथ विष्णुर्महेशानमिदं वचनमुक्तवान्}
{हनूमता समो नास्ति कृत्स्नब्रह्माण्डमण्डले}% १३६

\twolineshloka
{श्रुतिदेवाद्यगम्यं हि पदं तव कपिस्थितम्}
{सर्वोपनिषदव्यक्तं त्वत्पदं कपिसर्वयुक्}% १३७

\twolineshloka
{यमादिसाधनैर्योगैर्न क्षणं ते पदं स्थिरम्}
{महायोगिहृदम्भोजे परं स्वस्थं हनूमति}% १३८

\twolineshloka
{वर्षकोटिसहस्रं तु सहस्राब्दैरथान्वहम्}
{भक्त्या सम्पूज्!ऽपीश पादो नो दर्शितस्त्वया}% १३९

\twolineshloka
{लोके वादो हि सुमहाञ्छम्भुर्नारायणप्रियः}
{हरिप्रियस्तथा शम्भुर्न तादृग्भाग्यमस्ति मे}% १४०

\twolineshloka
{तच्छ्रुत्वा वचनं शम्भुर्विष्णोः प्राह मुदान्वितः}
{न त्वया सदृशो मह्यं प्रियोऽन्योऽस्ति हरे क्वचित्}% १४१

\twolineshloka
{पार्वती वा त्वया तुल्या वर्तते नैव भिद्यते}
{अथ देवाय महते गौतमः प्रणिपत्य च}% १४२

\twolineshloka
{व्यजिज्ञपदमेयात्मन्देवैर्हि करुणानिधे}
{मध्याह्नोऽय व्यतिक्रान्तो भुक्तिवेलाखिलस्य च}% १४३

\twolineshloka
{अथाचम्य महादेवो विष्णुना सहितो विभुः}
{प्रविश्य गौतमगृहं भोजनायोपचक्रमे}% १४४

\twolineshloka
{रत्नाङ्गुलीयैरथ नूपुराभ्यां दुकूलबन्धेन तडित्सुकाञ्च्या}
{हारैरनेकैरथ कण्ठनिष्कयज्ञोपवीतोत्तरवाससी च}% १४५

\twolineshloka
{विलम्बिचञ्चन्मणिकुडण्लेन सुपुष्पधम्मिल्लवरेण चैव}
{पञ्चागगन्धस्य विलेपनेन बाह्वङ्गदैः कङ्कणकाङ्गुलीयैः}% १४६

\twolineshloka
{अथो विभूषितः शिवो निविष्ट उत्तमासने}
{स्वसम्मुखं हरिं तथा न्यवेशयद्वरासने}% १४७

\twolineshloka
{देव श्रेष्ठौ हरीशौ तावन्योन्याभिमुखस्थितौ}
{सुवर्णभाजनस्थान्नं ददौ भक्त्या स गौतमः}% १४८

\twolineshloka
{त्रिंशत्प्रभेदान्भक्ष्यांस्तु पायसं च चतुर्विधम्}
{सुपक्वं पाकजातं च कल्पितं यच्छतद्वयम्}% १४९

\twolineshloka
{अपक्वं मिश्रकं तद्वत्त्रिंशतं परिकल्पितम्}
{शतं शतं सुकन्दानां शाकानां च प्रकल्पितम्}% १५०

\twolineshloka
{पञ्चविंशतिधा सर्पिःसंस्कृतं व्यञ्जनं तथा}
{शर्कराद्यं तथा चूतमोचाखर्जूरदाडिमम्}% १५१

\twolineshloka
{द्रा क्षेक्षुनागरङ्गं च मिष्टं पक्वं फलोत्करम्}
{प्रियालकं जम्बुफलं विकङ्कतफलं तथा}% १५२

\twolineshloka
{एवमादीनि चान्यानि द्र व्याणीशे समर्प्य च}
{दत्त्वापोशानकं विप्रो भुञ्जध्वमिति चाब्रवीत्}% १५३

\twolineshloka
{भुञ्जानेषु च सर्वेषु व्यजनं सूक्ष्मविस्तृतम्}
{गौतमः स्वयमादाय शिवविष्णू अवीजयत्}% १५४

\twolineshloka
{परिहासमथो कर्तुमियेष परमेश्वरः}
{पश्य विष्णो हनूमन्तं कथं भुङ्क्ते स वानरः}% १५५

\twolineshloka
{वानरं पश्यति हरौ मण्डकं विष्णुभाजने}
{चिक्षेप मुनिसङ्घेषु पश्यत्स्वपि महेश्वरः}% १५६

\twolineshloka
{हनूमते दत्तवांश्च स्वोच्छिष्टं पायसादिकम्}
{त्वदुच्छिष्टमभोज्यं तु तवैव वचनाद्विभो}% १५७

\twolineshloka
{अनर्हं मम नैवेद्यं पत्रं पुष्पं फलादिकम्}
{मह्यं निवेद्यसकलं कूप एव विनिःक्षिपेत्}% १५८

\twolineshloka
{अभुक्ते त्वद्वचो नूनं भुक्ते चापि कृपा तव}
{बाणलिङ्गे स्वयम्भूते चन्द्र कान्ते हृदि स्थिते}% १५९

\twolineshloka
{चान्द्रा यणसमं ज्ञेयं शम्भोर्नैवैद्यभक्षणम्}
{भुक्तिवेलेयमधुना तद्वैरस्यं कथान्तरात्}% १६०

\twolineshloka
{भुक्त्वा तु कथयिष्यामि निर्विशङ्कं विभुङ्क्ष्व तत्}
{अथासौ जलसंस्कारं कृतवान् गौतमो मुनिः}% १६१

\twolineshloka
{आरक्तसुस्निग्धसुसूक्ष्मगात्राननेकधाधौतसुशोभिताङ्गान्}
{तडागतोयैः कतबीजघर्षितैर्विशौधितैस्तैः करकानपूरयत्}% १६२

\fourlineindentedshloka*
{नद्याः सैकतवेदिकां नवतरां सञ्छाद्य सूक्ष्माम्बरैः}
{शुद्धैः श्वेततरैरथोपरि घटांस्तोयेन पूर्णान्क्षिपेत्}
{लिप्त्वा नालकजातिमास्तपुटकं तत्कौलकं कारिका}
{चूर्णं चन्दनचन्द्र रश्मिविशदां मालां पुटान्तं क्षिपेत्}

\twolineshloka
{यामस्यापि पुनश्च वारिवसनेनाशोध्य कुम्भेन तच्चन्द्र ग्रन्थिमथो-}
{निधाय बकुलं क्षिप्त्वा तथा पाटलम्}% १६३
शेफालीस्तबकमथो जलं च तत्रै
विन्यस्य प्रथमत एव तोयशुद्धिम्

\twolineshloka
{कृत्वाथो मृदुतरसूक्ष्मवस्त्रखण्डे-}
{नावेष्टेत्सृणिकमुखं च सूक्ष्मचन्द्र म्}% १६४

\twolineshloka
{अनातपप्रदेशे तु निधाय करकानथ}
{मन्दवातसमोपेते सूक्ष्मव्यजनवीजिते}% १६५

\twolineshloka
{सिञ्चेच्छीतैर्जलैश्चापि वासितैः सृणिकामपि}
{संस्कृताः स्वायतास्तत्र नरा नार्योऽथवा नृपाः}% १६६

\twolineshloka
{तत्कन्या वा क्षालिताङ्गा धौतपादास्सुवाससः}
{मधुपिङ्गमनिर्यासमसान्द्र मगुरूद्भवम्}% १६७

\twolineshloka
{बाहुमूले च कण्ठे च विलिप्यासान्द्र मेव च}
{मस्तकेजापकं न्यस्य पञ्चगन्धविलेपनम्}% १६८

\twolineshloka
{पुष्पनद्धसुकेशास्तु ताः शुभाः स्युः सुनिर्मलाः}
{एवमेवार्चिता नार्य आप्तकुङ्कुमविग्रहाः}% १६९

\twolineshloka
{युवत्यश्चारुसर्वाङ्ग्यो नितरां भूषणैरपि}
{एतादृग्वनिताभिर्वा नरैर्वा दापयेज्जलम्}% १७०

\twolineshloka
{तेऽपि प्रादानसमये सूक्ष्मवस्त्राल्पवेष्टनम्}
{अथ वामकरे न्यस्य करकं प्रेक्ष्य तत्र हि}% १७१

\twolineshloka
{दोरिकान्यस्तमुन्मुच्य ततस्तोयं प्रदापयेत्}
{एवं स कारयामास गौतमो भगवान्मुनिः}% १७२

\twolineshloka
{महेशादिषु सर्वेषु भुक्तवत्सु महात्मसु}
{प्रक्षालिताङिघ्रहस्तेषु गन्धोद्वर्तितपाणिषु}% १७३

\twolineshloka
{उच्चासनसमासीने देवदेवे महेश्वरे}
{अथ नीचसमासीना देवाः सर्षिगणास्तथा}% १७४

\twolineshloka
{मणिपात्रेषु संवेष्ट्य पूगखण्डान्सुधूपितान्}
{अकोणान्वर्तुलान्स्थूलानसूक्ष्मानकृशानपि}% १७५

\twolineshloka
{श्वेतपत्राणि संशोध्य क्षिप्त्वा कर्पूरखण्डकम्}
{चूर्णं च शङ्करायाथ निवेदयति गौतमे}% १७६

\twolineshloka
{गृहाण देव ताम्बूलमित्युक्तवचने मुनौ}
{कपे गृहाण ताम्बूलं प्रयच्छ मम खण्डकान्}% १७७

\twolineshloka
{उवाच वानरो नास्ति मम शुद्धिर्महेश्वर}
{अनेकफलभोक्तृत्वाद्वानरस्तु कथं शुचिः}% १७८

\twolineshloka
{तच्छ्रुत्वा तु विरूपाक्षः प्राह वानरसत्तमम्}
{मद्वाक्यादखिलं शुद्ध्य्न्मेद्वाक्यादमृतं विषम्}% १७९

\twolineshloka
{मद्वाक्यादखिला वेदा मद्वाक्याद्देवतादयः}
{मद्वाक्याद्धर्मविज्ञानं मद्वाक्यान्मोक्ष उच्यते}% १८०

\twolineshloka
{पुराणान्यागमाश्चैव स्मृतयो मम वाक्यतः}
{अतो गृहाण ताम्बूलं मम देहि सुखण्डकान्}% १८१

\twolineshloka
{हरिर्वामकरेणाधात्ताम्बूलं पूगखण्डकम्}
{ततः पत्राणि सङ्गृह्य तस्मै खण्डान्समर्पयत्}% १८२

\twolineshloka
{कर्पूरमग्रतो दत्तं गृहीत्वाभक्षयच्छिवः}
{देवे तु कृतताम्बूले पार्वती मन्दराचलात्}% १८३

\twolineshloka
{जयाविजययोर्हस्तं गृहीत्वायान्मुनेर्गृहम्}
{देवपादौ ततो नत्वा विनम्रवदनाभवत्}% १८४

\twolineshloka
{उन्नमय्यमुखे तस्या इदमाह त्रिलोचनः}
{त्वदर्थं देवदेवेशि अपराधः कृतो मया}% १८५

\twolineshloka
{यत्त्वां विहाय भुक्तं हि तथान्यच्छृणु सुन्दरी}
{यत्त्वां स्वमन्दिरे त्यक्त्वा महदेनो मया कृतम्}% १८६

\twolineshloka
{क्षन्तुमर्हसि देवेशि त्यक्तकोपा विलोकय}
{न बभाषेऽप्येवमुक्ता सारुन्धत्या विनिर्ययौ}% १८७

\twolineshloka
{निर्गच्छन्तीं मुनिर्ज्ञात्वा दण्डवत्प्रणनाम ह}
{अथोवाच शिवा तं च गौतम त्वं किमिच्छसि}% १८८

\twolineshloka
{अथाह गौतमो देवीं पार्वतीं प्रेक्ष्य संस्मिताम्}
{कृतकृत्यो भवेयं वै भुक्तायां मद्गृहे त्वयि}% १८९

\twolineshloka
{ततः प्राह शिवा विप्रं गौतमं रचिताञ्जलिम्}
{भोक्ष्यामि त्वद्गृहे विप्र शङ्करानुमतेन वै}% १९०

\twolineshloka
{अथ गत्वा शिवं विंशे लब्धानुज्ञस्त्वरागतः}
{भोजयामास गिरिजां देवीं चारुन्धतीं तथा}% १९१

\twolineshloka
{भुक्त्वाथ पार्वती सर्वगन्धपुष्पाद्यलङ्कृता}
{सहानुचरकन्याभिः सहस्राभिर्हरं ययौ}% १९२

\twolineshloka
{अथाह शङ्करो देवीं गच्छ गौतममन्दिरम्}
{सन्ध्योपास्तिमहं कृत्वा ह्यागमिष्ये तवान्तिकम्}% १९३

\twolineshloka
{इत्युक्त्वा प्रययौ देवी गौतमस्यैव मन्दिरम्}
{सन्ध्यावन्दनकामास्तु सर्व एव विनिर्गताः}% १९४

\twolineshloka
{कृतसन्ध्यास्तडागे तु महेशाद्याश्च कृत्स्नशः}
{अथोत्तरमुखः शम्भुर्न्यास कृत्वा जजाप ह}% १९५

\twolineshloka
{अथ विष्णुर्महातेजा महेशमिदमब्रवीत्}
{सर्वैर्नमस्यते यस्तु सर्वैरेव समर्च्यते}% १९६

\twolineshloka
{हूयते सर्वयज्ञेषु स भवान्किं जपिष्यति}
{रचिताञ्जलयः सर्वे त्वामेवैकमुपासते}% १९७

\twolineshloka
{स भवान्देवदेवेशः कस्मै विरचिताञ्जलि}
{नमस्कारादिपुण्यानां फलदस्त्वं महेश्वर}% १९८

\twolineshloka
{तव कः फलदो वन्द्यः को वा त्वत्तोऽधिको वद}
{तच्छ्रुत्वा शङ्करः प्राह देवदेवं जनार्दनम्}% १९९

\twolineshloka
{ध्याये न किञ्चिद्गोविन्द नमस्ये ह न किञ्चन}
{किन्तु नास्तिकजन्तूनां प्रवृत्त्यर्थमिदं मया}% २००

\twolineshloka
{दर्शनीयं हरे चैतदन्यथा पापकारिणः}
{तस्माल्लोकोपकारार्थमिदं सर्वं कृतं मया}% २०१

\twolineshloka
{ओमित्युक्त्वा हरिरथ तं नत्वा समतिष्ठत}
{अथ ते गौतमगृहं प्राप्ता देवर्षयस्तदा}% २०२

\twolineshloka
{सर्वे पूजामथो चक्रुर्देवदेवाय शूलिने}
{देवो हनूमता सार्द्धं गायन्नास्ते मुनीश्वर}% २०३

\twolineshloka
{पञ्चाक्षरीं महाविद्यां सर्व एव तदाऽजपन्}
{हनुमत्करमालम्ब्य देवाभ्यां सङ्गतो हरः}% २०४

\twolineshloka
{एकशय्यासमासीनौ तावुभौ देवदम्पती}
{गायन्नास्ते च हनुमांस्तुम्बुरुप्रमुखास्तथा}% २०५

\twolineshloka
{नानाविधविलासांश्च चकार परमेश्वरः}
{आहूय पार्वतीमीश इदं वाक्यमुवाच ह}% २०६

\twolineshloka
{रचयिष्यामि धम्मिल्लमेहि मत्पुरतः शुभे}
{देव्याह न च युक्तं तद्भर्त्रा शुश्रूषणं स्त्रियः}% २०७

\twolineshloka
{केशप्रसाधनकृतावनर्थान्तरमापतेत्}
{केशप्रसाधने देव तत्त्वं सर्वं न चेप्सितम्}% २०८

\twolineshloka
{अथ बन्धेकृते पश्चादंसप्रान्तप्रमार्जनम्}
{ततश्चरमसंलग्नकेशपुष्पादिमार्जनम्}% २०९

\twolineshloka
{एतस्मिन्वर्तमाने तु महात्मानो यदागमन्}
{तदा किमुत्तरं वाच्यं तव देवादिवन्दित}% २१०

\twolineshloka
{नायान्ति चेदथ विभो भीतिर्नाशमुपैष्यति}
{एवं हि भाषमाणां तां करेणाकृष्य शङ्करः}% २११

\twolineshloka
{स्वोर्वोः संस्थापयित्वैव विस्रस्य कचबन्धनम्}
{विभज्य च कराभ्यां स प्रससार नखैरपि}% २१२

\twolineshloka
{विष्णुदत्तां पारिजातस्रजं कचगतामपि}
{कृत्वा धम्मिल्लमकरोदथ मालां करागताम्}% २१३

\twolineshloka
{मल्लिकास्रजमादाय बबन्ध कचबन्धने}
{कल्पप्रसूनमालां च ब्रह्मदत्तां महेश्वरः}% २१४

\twolineshloka
{पार्वतीवसने गूढगन्धाढ्ये च समाददात्}
{अथांसपृष्टसंलग्नमार्जनं कृतवान् विभुः}% २१५

\twolineshloka
{श्लथन्नीवेरधो देव्या वस्त्रवेष्टादधोगतः}
{किमिदं देवि चेत्युक्त्वा नीवीबन्धं चकार ह}% २१६

\twolineshloka
{नासा भूषणमेतत्ते सत्यमेव वदामि ते}
{ततः प्राह शिवा शम्भुं स्मित्वा पर्वतनन्दिनी}% २१७

\twolineshloka
{अहो त्वन्मन्दिरे शम्भो सर्ववस्तु समृद्धिमत्}
{पूर्वमेव मया सर्वं ज्ञातप्रायमभूत्किल}% २१८

\twolineshloka
{सर्वद्र विणसम्पत्तिर्भूषणैरवगम्यते}
{शिरो विभूषितं देव ब्रह्मशीर्षस्य मालया}% २१९

\twolineshloka
{नरकस्य तथा माला वक्षस्थलविभूषणम्}
{शेषश्च वासुकिश्चैव सविषौ तव कङ्कणौ}% २२०

\twolineshloka
{दिशोऽम्बरं जटाः केशा भसितं चाङ्गरागकम्}
{महोक्षो वाहनं गोत्रं कुलं चाज्ञातमेव च}% २२१

\twolineshloka
{ज्ञायेते पितरौ नैव विरूपाक्षं तथा वपुः}
{एवं वदन्तीं गिरिजां विष्णुः प्राहातिकोपनः}% २२२

\twolineshloka
{किमर्थं निन्दसे देवि देवदेवं जगत्पतिम्}
{दुष्प्राणा न प्रिया भद्रे तव नूनमसंयमात्}% २२३

\twolineshloka
{यत्रेशनिन्दनं भद्रे तत्र नो मरणव्रतम्}
{इत्युक्त्वाथ नखाभ्यां हि हरिश्छेत्तुं शिरो गतः}% २२४

\twolineshloka
{महेशस्तत्करं गृह्य प्राह मा साहसं कृथाः}
{पार्वतीवचनं सर्वं प्रियं मम न चाप्रियम्}% २२५

\twolineshloka
{ममाप्रियं हृषीकेश कर्तुं यत्किञ्चिदिष्यते}
{ओमित्युक्त्वाथ भगवांस्तूष्णीम्भूतोऽभवद्धरिः}% २२६

\twolineshloka
{हनुमानथ देवाय व्यज्ञापयदिदं वचः}
{अर्थयामि विनिष्क्रामं मम पूजाव्रतं तथा}% २२७

\twolineshloka
{पूजार्थमप्यहं गच्छे मामनुज्ञातुमर्हसि}
{तच्छ्रुत्वा शङ्करो देवः स्मित्वा प्राह कपीश्वरम्}% २२८

\twolineshloka
{कस्य पूजा क्व वा पूजा किं पुष्पं किं दलं वद}
{को गुरुः कश्च मन्त्रस्ते कीदृशं पूजनं तथा}% २२९

\twolineshloka
{एवं वदति दैवेशे हनुमान्नीतिसंयुतः}
{वेपमानसमस्ताङ्गः स्तोतुमेवं प्रचक्रमे}% २३०

\twolineshloka
{नमो देवाय महते शङ्करायामितात्मने}
{योगिने योगधात्रे च योगिनां गुरवे नमः}% २३१

\twolineshloka
{योगगम्याय देवाय ज्ञानिनां पतये नमः}
{वेदानां पतये तुभ्यं देवानां पतये नमः}% २३२

\twolineshloka
{ध्यानाय ध्यानगम्याय ध्यातॄणां गुरवे नमः}
{अष्टमूर्ते नमस्तुभ्यं पशूनां पतये नमः}% २३३

\twolineshloka
{अम्बकाय त्रिनेत्राय सोमसूर्याग्निचक्षुषे}
{सुभृङ्गराजधत्तूरद्रो णपुष्पप्रियस्य ते}% २३४

\twolineshloka
{बृहतीपूगपुन्नागचम्पकादिप्रियाय च}
{नमस्तेऽस्तु नमस्तेऽस्तु भूय एव नमो नमः}% २३५

\twolineshloka
{शिवो हरिमथ प्राह मा भैषीर्वद मेऽखिलम्}
{ततस्त्यक्त्वा भयं प्राह हनुमान् वाक्यकोविदः}% २३६

\twolineshloka
{शिवलिङ्गार्चनं कार्यं भस्मोद्धूलितदेहिना}
{दिवा सम्पादितैस्तोयैः पुष्पाद्यैरपि तादृशैः}% २३७

\twolineshloka
{देव विज्ञापयिष्यामि शिवपूजाविधिं शुभम्}
{सायङ्काले तु सम्प्राप्ते आशिरःस्नानमाचरेत्}% २३८

\twolineshloka
{क्षालितं वसनं शुष्कं धृत्वाचम्य त्रिरन्यधीः}
{अथ भस्म समादाय आग्नेयं स्नानमाचरेत्}% २३९

\twolineshloka
{प्रणवेन समामम्त्र्य अष्टवारमथापि वा}
{पञ्चाक्षरेण मन्त्रेण नाम्ना वा येन केनचित्}% २४०

\twolineshloka
{सप्ताभिमन्त्रितं भस्म दर्भपाणिः समाहरेत्}
{ईशानं सर्वविद्यानामुक्त्वा शिरसि पातयेत्}% २४१

\twolineshloka
{तत्पुरुषाय विद्महे मुखे भस्म प्रसेचयेत्}
{अघोरेभ्योऽथ घोरेभ्यो भस्म वक्षसि निक्षिपेत्}% २४२

\twolineshloka
{वामदेवाय नमः इति गुह्यस्थाने विनिक्षिपेत्}
{सद्योजातं प्रपद्यामि निक्षिपेदथ पादयोः}% २४३

\twolineshloka
{उद्धूलयेत्समस्ताङ्गं प्रणवेन विचक्षणः}
{त्रैवर्णिकानामुदितः स्नानादिविधिरुत्तमः}% २४४

\twolineshloka
{शूद्रा दीनां प्रवक्ष्यामि यदुक्तं गुरुणा तथा}
{शिवेति पदमुच्चार्य भस्म सम्मन्त्रयेत्सुधीः}% २४५

\twolineshloka
{सप्त वारमथादाय शिवायेति शिरस्यथ}
{शङ्कराय मुखे प्रोक्तं सर्वज्ञाय हृदि क्षिपेत्}% २४६

\twolineshloka
{स्थाणवे नम इत्युक्त्वा मुखे चापि स्वयम्भुवे}
{उच्चार्य पादयोः क्षिप्त्वा भस्म शुद्धमतः परम्}% २४७

\twolineshloka
{नमः शिवायेत्युच्चार्य सर्वाङ्गोद्धूलनं स्मृतम्}
{प्रक्षाल्य हस्तावाचम्य दर्भपाणिः समाहितः}% २४८

\twolineshloka
{दर्भाभावे सुवर्णं स्यात्तदभावे गवालुकाः}
{तदभावे तु दूर्वाः स्युस्तदभावे तु राजतम्}% २४९

\twolineshloka
{सन्ध्योपास्तिं जपं देव्याः कृत्वा देवगृहं व्रजेत्}
{देववेदीमथो वापि कल्पितं स्थण्डिलं तु वा}% २५०

\twolineshloka
{मृण्मयं कल्पितं शुद्धं पद्मादिरचनायुतम्}
{चातुर्वर्णकरङ्गैश्च श्वेतेनैकेन वा पुनः}% २५१

\twolineshloka
{विचित्राणि च पद्मानि स्वस्तिकादि तथैव च}
{उत्पलादिगदाशङ्खत्रिशूलडमरूंस्तथा}% २५२

\twolineshloka
{शरोक्तपञ्च प्रासादं शिवलिङ्गमथैव च}
{सर्वकामफलं वृक्षं कुलकं कोलकं तथा}% २५३

\twolineshloka
{षट्कोणं च त्रिकोणं च नवकोणमथापि वा}
{कोणे द्वादशकान्दोलापादुकाव्यजनानि च}% २५४

\twolineshloka
{चामरच्छत्र्रयुगलं विष्णुब्रह्मादिकांस्तथा}
{चूर्णैर्विरचयेद्वेद्यां धीमान्देवालयेऽपि वा}% २५५

\twolineshloka
{यत्रापि देवपूजा स्यात् तत्रैव कल्पयेद्बुधः}
{स्वहस्तरचितं मुख्यं क्रीतं चैव तु मध्यमम्}% २५६

\twolineshloka
{याचितं तु कनिष्ठं स्याद्बलात्कारमथोऽधमम्}
{अर्हेषु यत्त्वनर्हेषु बलात्कारात्तु निष्फलम्}% २५७

\twolineshloka
{रक्तशालिजपाशाणकलमासितरक्तकैः}
{तन्दुलैर्वीहिमात्रोत्थैः कणैश्चैव यथाक्रमम्}% २५८

\twolineshloka
{उत्तमैर्मध्यमैश्चैव कनिष्ठैरधमैस्तथा}
{पद्मादिस्थापनैरेव तत्सम्यग्यागमाचरेत्}% २५९

\twolineshloka
{प्रागुत्तरमुखो वापि यदि वा प्राङ्मुखो भवेत्}
{आसनं च प्रवक्ष्यामि यथादृष्टं यथा श्रुतम्}% २६०

\twolineshloka
{कौशं चार्मं चैलतल्पे दारवं तालपत्रकम्}
{काम्बलं काञ्चनं चैव राजतं ताम्रमेव च}% २६१

\twolineshloka
{गोकरीषार्कजैर्वापि ह्यासनं परिकल्पयेत्}
{वैयाघ्रं रौरवं चैव हारिणं मार्गमेव च}% २६२

\twolineshloka
{चार्मं चतुर्विधं ज्ञेयमथ बन्धुकमेव च}
{यथासम्भवमेतेषु ह्यासनं परिकल्पयेत्}% २६३

\twolineshloka
{कृतपद्मासनो वापि स्वस्तिकासन एव च}
{दर्भभस्मसमासीनः प्राणानायम्य वाग्यतः}% २६४

\twolineshloka
{तावत्स देवतारूपो ध्यानं चान्तः समाचरेत्}
{शिखान्ते द्वादशाङ्गुल्ये स्थितं सूक्ष्मतनुं शिवम्}% २६५

\twolineshloka
{अन्तश्चरन्तं भूतेषु गुहायां विश्वतोमुखम्}
{सर्वाभरणसंयुक्तमणिमादिगुणान्वितम्}% २६६

\twolineshloka
{ध्यात्वा तं धारयेच्चिते तद्दीप्त्या पूरयेत्तनुम्}
{तया दीप्त्या शरीरस्थं पापं नाशमुपागतम्}% २६७

\twolineshloka
{स्वर्णपादैरसम्पर्काद्र क्तं श्वेतं यथा भवेत्}
{तद्द्वादशदलावृत्तमष्ट पञ्च त्रिरेव वा}% २६८

\twolineshloka
{परिकल्प्यासनं शुद्धं तत्र लिङ्ग निधाय च}
{गुहास्थितं महेशानं लिङ्गेशं चिन्तयेत्तथा}% २६९

\twolineshloka
{शोधिते कलशे तोयं शोधितं गन्धवासितम्}
{सुगन्धपुष्पं निक्षिप्य प्रणवेनाभिमन्त्रितम्}% २७०

\twolineshloka
{प्राणायामश्च प्रणवः शूद्रे षु न विधीयते}
{प्राणायामपदे ध्यानं शिवेत्यॐकारमन्त्रितम्}% २७१

\twolineshloka
{गन्धपुष्पाक्षतादीनि पूजाद्र व्याणि यानि च}
{तानि स्थाप्य समीपे तु ततः सङ्कल्पमाचरेत्}% २७२

\twolineshloka
{शिवपूजां करिष्यामि शिवतुष्ट्यर्थमेव च}
{इति सङ्कल्पयित्वा तु तत आवाहनादिकम्}% २७३

\twolineshloka
{कृत्वा तु स्नानपर्यन्तं ततः स्नानं प्रकल्पयेत्}
{नमस्तेत्यादिमन्त्रेण शतरुद्र विधानतः}% २७४

\twolineshloka
{अविच्छिन्ना तु या धारा मुक्तिधारेति कीर्तिता}
{तया यः स्नापयेन्मासं जपन् रुद्र मुखांश्च वा}% २७५

\twolineshloka
{एकवारं त्रिवारं च पञ्च सप्त नवापि वा}
{एकादश तथा वारमथैकादशधान्वितम्}% २७६

\twolineshloka
{मुक्तिस्नानमिदं ज्ञेयं मासं मोक्षप्रदायकम्}
{शैवया विद्यया स्नानं केवलं प्रणवेन वा}% २७७

\twolineshloka
{मृण्मयैर्नालिकेरस्य शकलैश्चोर्मिभिस्तथा}
{कांस्येन मुक्ताशुक्त्या च पुष्पादिकेसरेण वा}% २७८

\twolineshloka
{स्नापयेद्देवदेवेशं यथासम्भवमीरितैः}
{शृङ्गस्य च विधिं वक्ष्ये स्नानयोग्यं यथा भवेत्}% २७९

\twolineshloka
{पूर्वमन्तस्तु संशोध्य बहिरन्तस्तु शोधयेत्}
{सुस्निग्धं लघु कृत्वाथ नाङ्गं छिन्द्यात्कथञ्चन}% २८०

\twolineshloka
{नीचैकदेशविन्यस्तद्वारद्रो ण्या सुहृत्तया}
{कृशानुयुक्तं स्नानं तु देवाय परिकल्पयेत्}% २८१

\twolineshloka
{एवं गवयशृङ्गस्य जलपूर्तिरथोच्यते}
{द्वारे निषिद्धलोहार्द्धसन्धिद्वारासमन्विते}% २८२

\twolineshloka
{योगवक्रं नागदण्डं नागाकारं प्रकल्पयेत्}
{फलस्थाने तु चषकं दण्डेन समरन्ध्रकम्}% २८३

\twolineshloka
{तत्रैव पातयेत्तोयं मूर्द्धयन्त्रघटे स्थितम्}
{पातयेदथ चान्येन वामेनैव करेण वा}% २८४

\twolineshloka
{मुक्तिधारा कृता तेन पवित्रं पापनाशनम्}
{एवं संस्नाप्य देवेशं पञ्चगव्यैस्तथैव च}% २८५

\twolineshloka
{पञ्चामृतैरथ स्नाप्य मधुरत्रितयेन च}
{विभूष्य भूषणैर्देवं पुनः स्नाप्यमहेश्वरम्}% २८६

\twolineshloka
{शीतोपचारं कृत्वाथ तत आचमनादिकम्}
{वस्त्रं तथोपवीतं च गन्धद्र व्यकमेव च}% २८७


\twolineshloka*
{कर्पूरमगरुं चापि पाटीरमथवा भवेत्}
{उभयमिश्रितं चापि शिवलिङ्गं प्रपूजयेत्}

\twolineshloka
{कृत्स्नं पीठं गन्धपूर्णं यद्वा विभवसारतः}
{तूष्णीमथोपचारं वा कालीयं पुष्पमेव च}% २८८

\twolineshloka
{श्रीपत्रमरुचित्याज्यं यथाशक्त्यखिलं यथा}
{अनेकद्र व्यधूपं च गुग्गुलं केवलं तथा}% २८९

\twolineshloka
{कपिलाघृतसंयुक्तं सर्वधूपं प्रशस्यते}
{धूपं दत्वा यथाशक्ति कपिलाघृतदीपकान्}% २९०

\twolineshloka
{अथवा पूजामात्रेण दीपान्दत्वोपहारकम्}
{नैवेद्यमुपपन्नं च दत्वा पुष्पसमन्वितम्}% २९१

\twolineshloka
{मुखशुद्धिं ततः कृत्वा दत्त्वा ताम्बूलमादरात्}
{प्रदक्षिणानमस्कारौ पूजैवं हि समाप्यते}% २९२

\twolineshloka
{गीत्यङ्गपञ्चकं पश्चात्तानि विज्ञापयामि ते}
{गीतिर्वाद्यं पुराणं च नृत्यं हासोक्तिरेव च}% २९३

\twolineshloka
{नीराजनं च पुष्पाणामञ्जलिश्चाखिलार्पणम्}
{क्षमापनं चोद्वसनं प्रोक्तं पञ्चोपचारकम्}% २९४

\twolineshloka
{भूषणं च तथा छत्रं चामरव्यजने अपि}
{उपवीतं च कैकर्यं षोडशानुपचारकान्}% २९५

\twolineshloka
{द्वात्रिंशदुपचारैस्तु यः समाराधयेच्छिवम्}
{एकेनाह्ना समस्तानां पातकानां क्षयो भवेत्}% २९६

\twolineshloka
{एतच्छ्रुत्वा हनुमतो वचनं प्राह शङ्करः}
{एवमेतत्कपिश्रेष्ठः यदुक्तं पूजनं मम}% २९७

\twolineshloka
{सारभूतमहं तुभ्यमुपदेक्ष्यामि साम्प्रतम्}
{आराधनं यथा लिङ्गे विस्तरेण त्वयोदितम्}% २९८

\twolineshloka
{मत्पादयुगलं प्रार्च्य पूजाफलमवाप्नुहि}
{ततः प्राह कपिश्रेष्ठो देवदेवमुमापतिम्}% २९९

\twolineshloka
{गुरुणा लिङ्गपूजैव नियता परिकल्पिता}
{तां करोमि पुरा देव पश्चात्त्वत्पादपूजनम्}% ३००

\twolineshloka
{इत्युक्त्वेशं नमस्कृत्य शिवलिङ्गार्चनाय च}
{सरसस्तीरमागत्य कृत्वा सैकतवेदिकाम्}% ३०१

\twolineshloka
{तालपत्रैर्विरचितमासनं पर्यकल्पयत्}
{प्रक्षाल्य पादौ हस्तौ च समाचम्य समाहितः}% ३०२

\twolineshloka
{भस्मस्नानमथो चक्रे पुनराचम्य वाग्यतः}
{देववेद्यामथो चक्रे पद्मं च सुमनोहरम्}% ३०३

\twolineshloka
{अनन्तरं तालपत्रे पद्मासनगतः कपिः}
{प्राणानायम्य सन्न्यस्य शुक्लध्यानसमन्वितः}% ३०४

\twolineshloka
{प्रणम्य गुरुमीशानं जपन्नासीदतः परम्}
{अथ देवार्चनं कर्त्तुं यत्नमास्थितवान्कपिः}% ३०५

\twolineshloka
{पलाशपत्रपुटकद्वयानीतजलं शुचि}
{शिरः कमण्डलुगतं निधायाग्निनिमन्त्रितम्}% ३०६

\twolineshloka
{आवाहनादि कृत्वाथ स्नानपर्यन्तमेव च}
{अथ स्नापयितुं देवमादाय करसम्पुटे}% ३०७

\twolineshloka
{कृत्वा निरीक्षणं देवपीठं नो दृष्टवान्कपिः}
{लिङ्गमात्रं करगतं दृष्ट्वा भीतिसमन्वितः}% ३०८

\twolineshloka
{इदमाह महायोगी किं वा पापं मया कृतम्}
{यदेतत्पीठरहितं शिवलिङ्गं करस्थितम्}% ३०९

\twolineshloka
{ममाद्य मरणं सिद्धं न पीठं चागमिष्यति}
{अथ रुद्रं जपिष्यामि तदायाति महेश्वरः}% ३१०

\twolineshloka
{इति निश्चित्य मनसा जजाप शतरुद्रि यम्}
{यदा तु न समायातो महेशोऽथ कपीश्वरः}% ३११

\twolineshloka
{रुद्रं न्यपातयद्भूमौ वीरभद्र ः! समागतः}
{किमर्थ रुद्यते भद्र रुदिते कारणं वद}% ३१२

\twolineshloka
{तच्छ्रुत्वा प्राह हनुमान्वीरभद्रं मनोगतम्}
{पीठहीनमिदं लिङ्गं पश्य मे पापसञ्चयम्}% ३१३

\twolineshloka
{वीरभद्र स्ततः प्राह श्रुत्वा कपिसमीरितम्}
{यदि नायाति पीठं ते लिङ्गे मा साहसं कुरु}% ३१४

\twolineshloka
{दाहयिष्याम्यहं लोकान्यदि नायाति पीठकम्}
{पश्य दर्शय मे लिङ्गं पीठं चात्रागतं न वा}% ३१५

\twolineshloka
{अथ दृष्ट्वा वीरभद्रो लिङ्गे पीठमनागतम्}
{दग्धुकामोऽखिलाँल्लोकान्वीरभद्रः प्रतापवान्}% ३१६

\twolineshloka
{अनलं भुवि चिक्षेप क्षणाद्दग्धा मही तदा}
{अथ सप्ततलान्दग्ध्वा पुनरूर्द्ध्वमवर्तत}% ३१७

\twolineshloka
{पञ्चोर्द्ध्वलोकानदहज्जनलोकनिवासिनः}
{ललाटनेत्रसम्भूतं नखेनादाय चानलम्}% ३१८

\twolineshloka
{जम्बीरफलसङ्काशं कृत्वा करतले विभुः}
{तपः सत्यं च सन्दग्धुमुद्यतोऽभून्मुनीश्वरः}% ३१९

\twolineshloka
{ततस्तु मुनयो दृष्ट्वा तपोलोकनिवासिनः}
{दग्धुकामं वीरभद्रं गौतमाश्रममागताः}% ३२०

\twolineshloka
{न दृष्ट्वा तत्र देवेशं शङ्करं स्वात्मनि स्थितम्}
{अस्तुवन्भक्तिसंयुक्ताः स्तोत्रैर्वेदसमुद्भवैः}% ३२१

\twolineshloka
{ॐ वेदवेद्याय देवाय तस्मै शुद्धप्रभाचिन्त्यरूपाय कस्मै}
{ब्रह्माद्यधीशाय सृष्ट्यादिकर्त्रे विष्णुप्रियायार्तिहन्त्रेऽन्तकर्त्रे}% ३२२

\twolineshloka
{नमस्तेऽखिलधीश्वरायाम्बराय नमस्ते चरस्थावरव्यापकाय}
{नमो वेदगुह्याय भक्तप्रियाय नमः पाकभोक्त्रे मखेशाय तुभ्यम्}% ३२३

\twolineshloka
{नमस्ते शिवायादिदेवाय कुर्मो नमो व्यालयज्ञोपवीतप्रधर्त्रे}
{नमस्ते सुराबिन्दुवर्षापनाय त्रयीमूर्तये कालकालाय नाथ}% ३२४

\twolineshloka
{धरित्रीमरुद्व्योमतोयेन्दुवह्निप्रभामण्डलात्माष्टधामूर्तिधर्त्रे}
{शिवायाशिवघ्नाय वीराय भूयात्सदा नः प्रसन्नो जगन्नाथकेज्यः}% ३२५

\twolineshloka
{कलानाथभालाय आत्मा महात्मा मनो ह्यग्रयानो निरूप्यो न वाग्भिः}
{जगज्जाढ्यविध्वंसनो भुक्तिमुक्तिप्रदः स्तात्प्रसन्नः सदा शुद्धकीर्तिः}% ३२६

\twolineshloka
{यतः सम्प्रसूतं जगज्जातमीशात्स्थितं येन रक्षावता भावितं च}
{लयं यास्यते यत्र वाचां विदूरे स वै नः प्रसन्नोऽस्तु कालत्रयात्मा}% ३२७

\twolineshloka
{यदादिं च मध्यं तथान्तं न केऽपि विजानन्ति विज्ञा अपि स्वानुमानाः}
{स वै सर्वमूर्तिः सदा नो विभूत्यै प्रसन्नोऽस्तु किं ज्ञापयामोऽत्र कृत्यम्}% ३२८

\twolineshloka
{एतां स्तुतिमथाकर्ण्य भगनेत्रप्रदः शिवः}
{विष्णुमाह मुनीनेतानानयस्व मदन्तिकम्}% ३२९

\twolineshloka
{अथ विष्णुः समागत्य तपोलोकनिवासिनः}
{मुनीन्सान्त्वय्य विश्वेशं दर्शयामास शङ्करम्}% ३३०

\twolineshloka
{तानाह शङ्करो वाक्यं किमर्थं यूयमागताः}
{तपोलोकाद्भूमिलोकं मुनयो मुक्तकिल्बिषाः}% ३३१

\twolineshloka
{तच्छ्रुत्वा शूलिनो वाक्यं प्रोचुस्ते मुनिसत्तमाः}
{देव द्वादशलोकानां दृश्यन्ते भस्मराशयः}% ३३२

\twolineshloka
{स्थितमेकं वनमिदं पश्य तल्लोकसङ्क्षयम्}
{तच्छ्रुत्वा गिरिशः प्राह तान्मुनीनूर्द्ध्वरेतसः}% ३३३

\twolineshloka
{भूर्लोकस्य तु सन्दाहे पातालानां तथैव च}
{सन्देहो नास्ति मुनयः स्थितानां नो रहः स्थले}% ३३४

\twolineshloka
{ऊर्द्ध्वपञ्चकलोकानां दाहे सन्देह एव नः}
{कथमङ्गारवृष्टिश्च कथं नो वा महाध्वनिः}% ३३५

\twolineshloka
{तदाकर्ण्य विभोर्वाक्यं शङ्करस्य मुनीश्वराः}
{प्रोचुः प्राञ्जलयो देवं ब्रह्मादिसुरसङ्गतम्}% ३३६

\twolineshloka
{भीतिरस्माकमधुना वर्तते वीरभद्र तः}
{स एवाङ्गारवृष्टिं च पिपासुरपिबद्विभोः}% ३३७

\twolineshloka
{देवोऽथ वीरमाहूय किं वीरेत्यब्रवीद्वचः}
{वीरोऽप्याह कपेर्लिङ्गे पीठाभावादिदं कृतम्}% ३३८

\twolineshloka
{तच्छ्रुत्वाह शिवो देवो मुनींस्तान्भयविह्वलान्}
{कपेश्चित्तं परिज्ञातुं मया कृतमिदं द्विजाः}% ३३९

\twolineshloka
{मा भैष्ट भवतां सौख्यं सदा सम्पादयाम्यहम्}
{इत्युक्त्वा तु यथापूर्वं देवदेवः कृपानिधिः}% ३४०

\twolineshloka
{दग्धानप्यखिलाँ ल्लोकान्पूर्वतः शोभनान्विभुः}
{कल्पयामास विश्वात्मा वीरभद्र मथाब्रवीत्}% ३४१

\twolineshloka
{साधु वत्स यतो भद्रं भक्तानामीहसे सदा}
{ततस्ते विपुला कीर्तिर्लोके स्थास्यति शाश्वती}% ३४२

\twolineshloka
{इत्युक्त्वालिङ्ग्य शिरसि समाघ्राय महेश्वरः}
{ताम्बूलं वीरभद्रा य दत्तवान्प्रीतमानसः}% ३४३

\twolineshloka
{अथासौ हनुमानीशपूजनं कृतवान्यथा}
{समाप्तायां तु पूजायां हनुमान्प्रीतमानसः}% ३४४

\twolineshloka
{एकं वनचरं तत्र गन्धर्वं सविपञ्चकम्}
{ददर्श तमथाभ्याह वीणा मे दीयतामिति}% ३४५

\twolineshloka
{गन्धर्वोऽप्याह न मया त्याज्या वीणा प्रिया मम}
{ममापीष्टेह गन्धर्व वीणेत्याह कपीश्वरः}% ३४६

\twolineshloka
{यदा न दत्ते गन्धर्वो वल्लकीं कपये प्रियाम्}
{तदा मुष्टिप्रहारेण गन्धर्वः पातितः क्षितौ}% ३४७

\twolineshloka
{वीणामादाय महतीं स्वरतन्तुसमन्विताम्}
{हनुमान्वानरश्रेष्ठो गायन्प्रागाच्छिवान्तिकम्}% ३४८

\twolineshloka
{ततो गानेन महता प्रसाद्य जगदीश्वरम्}
{बृहतीकुसुमैः शुद्धैर्देवपादावपूजयत्}% ३४९

\twolineshloka
{ततः प्रसन्नो विश्वात्मा मुनीनां सन्निधौ तदा}
{दैत्यानां देवतानां च नृपाणां शङ्करोऽपि च}% ३५०

\twolineshloka
{तस्मै वरमथ प्रादात्कल्पान्तं जीवितं पुनः}
{समुद्र लङ्घने शक्तिं शास्त्रज्ञत्वं बलोन्नतिम्}% ३५१

\twolineshloka
{एवं दत्तं वरं प्राप्य महेशेन महात्मना}
{प्रत्यक्षं मम विप्रेन्द्र हनुमान्हर्षमागतः}% ३५२

\twolineshloka
{समस्तभूषासुविभूषिताङ्गः स्वदीप्तिमन्दीकृतदेवदीप्तिः}
{प्रसन्नमूर्तिस्तरुणः शिवांशः सम्भावयामास समस्तदेवान्}% ३५३

\twolineshloka
{आज्ञप्तो हनुमांस्तत्र मत्सेवायै मुनीश्वरः}
{महेशेनाहमप्येनं शशिमौलिमवैमि च}% ३५४

\twolineshloka
{किं बहूक्तेन विप्रर्षे यादृशो वानरेश्वरः}
{बुद्धौ न्याये च वै धैर्ये तादृगन्योऽस्ति न क्वचित्}% ३५५

\twolineshloka
{इति ते सर्वमाख्यातं चरितं पापनाशनम्}
{पठतां शृण्वतां चैव गच्छ विप्र यथासुखम्}% ३५६

\twolineshloka
{तच्छ्रुत्वा रामभद्र स्य रघुनाथस्य धीमतः}
{वचनं दक्षिणीकृत्य नत्वा चागां यथागतः}% ३५७

\twolineshloka
{एतत्तेऽभिहितं विप्र चरितं च हनूमतः}
{सुखदं मोक्षदं सारं किमन्यच्छ्रोतुमिच्छसि}% ३५८

॥इति श्रीबृहन्नारदीयपुराणे पूर्वभागे तृतीयपादे बृहदुपाख्याने तृतीयपादे हनुमच्चरित्रं नाम एकोनाशीतितमोऽध्यायः॥७९॥
