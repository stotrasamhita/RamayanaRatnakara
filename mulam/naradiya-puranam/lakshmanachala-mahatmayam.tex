\chapt{नारदीय-पुराणम्}
\sect{लक्ष्मणाचलमाहात्म्यम् --- पञ्चसप्ततितमोऽध्यायः}

\src{नारदीय-पुराणम्}{उत्तरभागः}{}{पञ्चसप्ततितमोऽध्यायः}
\tags{concise, complete}
\notes{}
\textlink{https://sa.wikisource.org/wiki/नारदपुराणम्-_उत्तरार्धः/अध्यायः_७५}
\translink{}

\storymeta


\uvacha{मोहिन्युवाच}

\twolineshloka
{श्रुतं गोकर्णमाहात्म्यं वसो पापविनाशनम्}
{लक्ष्मणस्यापि माहात्म्यं वक्तुमर्हसि साम्प्रतम्}% १

\uvacha{वसुरुवाच}

\twolineshloka
{शृणु देवि प्रवक्ष्यामि माहात्म्यं लक्ष्मणस्य च}
{यं दृष्ट्वा मनुजो देवं मुच्यते सर्वपातकैः}% २

\twolineshloka
{चतुर्व्यूहावतारे यो देवः सकर्षणः स्वयय्}
{सर्वभूमण्डलं ह्येतत्सहस्रवदनः स्वराट्}% ३

\twolineshloka
{एकस्मिञ्छिरसि न्यस्तं नावैत्सिद्धार्थकोपमम्}
{देवो नारायणः साक्षाद्रा मो ब्रह्मादिवन्दितः}% ४

\twolineshloka
{प्रद्युम्नो भरतो भद्रे शत्रुघ्नो ह्यनिरुद्धकः}
{लक्ष्मणस्तु महाभागे स्वयं सङ्कर्षणः शिवः}% ५

\twolineshloka
{ब्रह्माद्यैः प्रार्थितः पूर्वं साक्षाद्देवो रमापतिः}
{रामादिनामभिर्जज्ञे चतुर्द्धा दिग्ग्रथान्नृपात्}% ६

\twolineshloka
{ततः कालान्तरे देवि विश्वामित्रो मुनीश्वरः}
{यज्ञरक्षार्थमागत्य प्रार्थयद्रा मलक्ष्मणौ}% ७

\twolineshloka
{ततो राजा दशरथः प्राणेभ्योऽपि प्रियौ सुतौ}
{मुनेः शापभयाद्भीतो ददौ तौ रामलक्ष्मणौ}% ८

\twolineshloka
{गत्वा यज्ञं मुनीन्द्र स्य गाधिपस्य ररक्षतुः}
{सताडकं सुबाहुं तु हत्वा प्रक्षिप्य दूरतः}% ९

\twolineshloka
{मारीचं मानवास्त्रेण विश्वामित्रमतोषयत्}
{ततः प्रीतान्मुनिश्रेष्ठादस्त्रग्राममवाप्य च}% १०

\twolineshloka
{उवाच स कियत्कालं सानुजस्तेन सत्कृतः}
{वैदेहनगरं नीतो विश्वामित्रेण तत्परम्}% ११

\twolineshloka
{ततस्तु राजा जनको विश्वामित्रं सुसत्कृतम्}
{पप्रच्छ बालकावेतौ कस्य क्षत्रकुलेशितुः}% १२

\twolineshloka
{ततस्तस्मै मुनिवरो राज्ञो दशरथस्य तौ}
{पुत्रौ निवेदयामास भ्रातरौ रामलक्ष्मणौ}% १३

\twolineshloka
{ततो विदेहः सप्रीतो दृष्ट्वा रामं च लक्ष्मणम्}
{सीतोर्मिलाख्ययोः पुत्र्! योश्चेतसाकल्पयत्पती}% १४

\twolineshloka
{त्रिकालज्ञस्तु स मुनिर्ज्ञात्वा तस्य मनोगतम्}
{मोदमानोऽथ जनकं प्राह दर्शय तद्धनुः}% १५

\twolineshloka
{सीतास्वयंवरे न्यस्तं न्यासभूतं महेशितुः}
{राजा श्रुत्वा तु तद्वाक्यं विश्वामित्रस्य सत्वरम्}% १६

\twolineshloka
{भृत्यत्रिशत्यानाय्यास्मै दर्शयामास सादरम्}
{रामश्चण्डीशचापं तद्वामदोष्णोद्धरन् क्षणात्}% १७

\twolineshloka
{सज्यं विकृष्य सहसा बभञ्जेक्षुमिवेभराट्}
{ततोऽति मिथिलः प्रीतः स्वे कन्ये रामलक्ष्मणौ}% १८

\twolineshloka
{समभ्यर्च्यार्पयामास ताभ्यां ते विधिपूर्वकम्}
{ज्ञात्वा मुनिवरादन्यौ राज्ञो दशरथस्य तु}% १९

\twolineshloka
{ताभ्यां सह तमाहूय भ्रातृकन्ये अदापयत्}
{ततः स कृतदारैस्तु चतुर्भिस्तनयैः सह}% २०

\twolineshloka
{समर्चितो विदेहेनायोध्यां मुन्याज्ञया ययौ}
{मार्गे भृगुपतेर्दर्पं शमयित्वा स राघवः}% २१

\twolineshloka
{पितृभ्रातृयुतः श्रीमान्मुमुदे बहुवत्सरान्}
{पण्डितैस्तु वसिष्ठाद्यैर्बोधितोऽसौ निजं महः}% २२

\twolineshloka
{ब्रह्माख्यं बुबुधे रामो मानुषत्वं विडम्बयन्}
{ततो दशरथो राजा ज्ञातज्ञेयं निजं सुतम्}% २३

\twolineshloka
{रामं समुद्यतो हृष्टो यौवराज्येऽभिषेचितुम्}
{यज्ज्ञात्वा कैकयी देवी राज्ञः प्रेष्ठा कनीयसी}% २४

\twolineshloka
{सन्निवार्य हठात्तस्य पुत्रस्य तदरोचत}
{ततो रामो मुदे तस्याः पित्राननुमतो ययौ}% २५

\twolineshloka
{सभार्यः सः ससौमित्रिश्चित्रकूटं गिरिं शुभे}
{कियत्कालमुवासासौ तत्रैव मुनिवेषधृक्}% २६

\twolineshloka
{मातामहगृहात्तच्च श्रुत्वाऽयातः पितुर्वधम्}
{स विज्ञाय मृतं तातं हा रामेति विराविणम्}% २७

\twolineshloka
{धिक्कृत्य कैकयीं यातो रामं स विनिवर्तितुम्}
{ततः स्वपादुके दत्वा भरतं विनिवर्त्य च}% २८

\twolineshloka
{रामोऽत्रेश्चाप्यगस्त्यस्य सुतीक्ष्णस्याश्रमेष्वगात्}
{तेषु द्वादश वर्षाणि गमयित्वा रघूद्वहः}% २९

\twolineshloka
{भार्यानुजान्वितः श्रीमांस्ततः पञ्चवटीमगात्}
{तत्रावसज्जनस्थाने त्रिशिरःखरदूषणान्}% ३०

\twolineshloka
{शूर्पणख्या विकृतया प्रेरितान्स व्यनाशयत्}
{ततो रक्षःसहस्रैश्च चतुर्द्दशभिरागतान्}% ३१

\twolineshloka
{विचित्रवाजैर्नाराचैर्यमक्षयमनीनयत्}
{यच्छ्रुत्वा रक्षसां राजा मारीचं काञ्चनं मृगम्}% ३२

\twolineshloka
{दर्शयित्वापवाह्यैतौ सीतां हृत्वा जटायुषम्}
{रुन्धानं मार्गमाहत्य लङ्कायां समुपानयत्}% ३३

\twolineshloka
{आगत्य तौ हृतां सीतां मार्गमाणौ समन्ततः}
{दृष्ट्वा जटायुषं शान्तं दग्ध्वा हत्वा कबन्धकम्}% ३४

\twolineshloka
{शबरीमनुकम्प्याथ ऋष्यमूकमुपागतौ}
{ततस्तु हनुमद्वाक्यात्स्वसख्युः प्लवगेशितुः}% ३५

\twolineshloka
{विद्विषं वालिनं हत्वा सुग्रीवमकरोन्नृपम्}
{तदाज्ञप्तास्तु ते कीशाः सर्वतः समुपागताः}% ३६

\twolineshloka
{हनुमत्प्रमुखाः सीतां मार्गन्तो दक्षिणोदधिम्}
{प्राप्य सम्पातिवचनाल्लङ्कायां निश्चयं गताः}% ३७

\twolineshloka
{ततस्तु हनुमानेकः प्राप्य लङ्कां पुरीं कपिः}
{समुद्र स्य परे पारेऽपश्यद्रा मप्रियां सतीम्}% ३८

\twolineshloka
{दत्त्वा रामाङ्गुलीरत्नं विश्वासमुपपाद्य ताम्}
{तयोः कुशलमाश्राव्य लब्ध्वा चूडामणिं ततः}% ३९

\twolineshloka
{भङ्क्त्वा चाशोकवनिकां हत्वा चाक्षं ससैन्यकम्}
{इन्द्र जिद्बन्धनात्प्राप्य सम्भाष्यापि च रावणम्}% ४०

\twolineshloka
{दग्ध्वा लङ्कां पुरीं कृत्स्नां पुनर्दृष्ट्वा तु मैथिलीम्}
{लब्धाज्ञोऽणवमुल्लङ्घ्य रामायैनां न्यवेदयत्}% ४१

\twolineshloka
{श्रुत्वा रामोऽपि तां सीतां राक्षसस्य निवासगाम्}
{सार्द्धं स कपिसैन्येन सम्प्राप्तो मकरालयम्}% ४२

\twolineshloka
{सागरानुमतेनासौ सेतुं बद्ध्वा महोदधौ}
{अद्रि कूटेः परं तीरं प्राप्य सेनां न्यवेशयत्}% ४३

\twolineshloka
{ततोऽसौ रावणो भ्रात्रा बोधितोऽपि कनीयसा}
{प्रदानं तत्र मैथिल्यास्तद्भर्त्रे न त्वरोचयत्}% ४४

\twolineshloka
{पदा हतस्ततस्तेन रावणेन विभीषणः}
{सम्प्राप्तः शरणं रामं रामो लङ्कामुपारुणत्}% ४५

\twolineshloka
{ततस्तु मन्त्रिणोऽमात्याः पुत्रा भृत्याः प्रचोदिताः}
{युद्धाय ते क्षयं नीतास्ताभ्यां सङ्ख्ये कपीश्वरैः}% ४६

\twolineshloka
{लक्ष्मणः शक्रजेतारं जघ्निवान्निशितैः शरैः}
{रामोऽपि कुम्भश्रवणं रावणं चाप्यजीघनत्}% ४७

\twolineshloka
{विभीषणेन तत्कृत्यं कारयित्वा निजां प्रियाम्}
{वह्नौ संशोध्य दत्वास्मै रामो रक्षोगणेशताम्}% ४८

\twolineshloka
{लङ्कामायुश्च कल्पान्तं ययौ चीर्णव्रतः पुरीम्}
{पुष्पकेण विमानेन ससुग्रीवविभीषणः}% ४९

\twolineshloka
{नन्दिग्रामस्थभरतं नीत्वायोध्यां समाविशत्}
{मातॄः प्रणम्य ताः सर्वा भ्रातरस्ते पुरोधसा}% ५०

\twolineshloka
{वसिष्ठेनानुविज्ञाप्य रामं राज्येऽभ्यषेचयन्}
{ततो रामोऽपि भगवान्प्रजाः शासन्निवौरसान्}% ५१

\twolineshloka
{लोकापवादात्सन्त्रस्तः सीतां तत्याज धर्मवित्}
{सा तु सम्प्राप्य वाल्मीकेराश्रमं न्यवसत्सुखम्}% ५२

\twolineshloka
{पुत्रौ च सुषुवे तत्र नाम्ना ख्यातौ कुशीलवौ}
{वाल्मीकिस्तु तयोः कृत्वा यथा समुदिताः क्रियाः}% ५३

\twolineshloka
{रामायणं विरच्यैतावध्यापयदुदारधीः}
{तौ गायमानौ सत्रेषु मुनीनां ख्यातिमागतौ}% ५४

\twolineshloka
{यज्ञे रामस्य सम्प्राप्तौ वाजिमेधे प्रवर्तिते}
{तत्र ताभ्यां तु तद्गीतं स्वचरित्रं प्रसन्नधीः}% ५५

\twolineshloka
{मुनिमाकारयामास ससीतं तत्र संसदि}
{सा तु रामाय तौ पुत्रौ निवेद्य जगतीजनिः}% ५६

\twolineshloka
{जगत्या विवरं भूयो विवेशासीत्तदद्भुतम्}
{ततः परं ब्रह्मचर्यं यज्ञमेव त्रयोदश}% ५७

\twolineshloka
{सहस्राब्दान्प्रकुवार्णस्तस्थौ भुवि रघूत्तमः}
{ततस्तु काले दुर्वासाः सम्प्राप्तो राघवं प्रति}% ५८

\twolineshloka
{ब्रह्मणा प्रेषितो भद्रे वैकुण्ठगमनाय च}
{स एकान्तगतो रामं प्राह कोऽपीह नाऽव्रजेत्}% ५९

\twolineshloka
{आगतो वध्यतां यातु रामस्तत्प्रतिजज्ञिवान्}
{स लक्ष्मणं समाहूय प्रोवाच रघुनन्दनः}% ६०

\twolineshloka
{द्वारि तिष्ठात्र निर्विष्टो वध्यतां मे प्रयास्यति}
{स तथेति प्रतिज्ञाय रामस्याज्ञां समाचरन्}% ६१

\twolineshloka
{प्रवेशनं न कस्यापि प्रददौ रामसन्निधौ}
{एवमेकान्तगं रामं कालसंविदमास्थितम्}% ६२

\twolineshloka
{ज्ञात्वाथ द्वारि दुर्वासा लक्ष्मणं समुपागमत्}
{तमागतं तु सम्प्रेक्ष्य सौमित्रिः प्रणिपत्य च}% ६३

\twolineshloka
{मुहूर्तं पालयेत्याह मन्त्रव्यग्रोऽस्ति राघवः}
{दुर्वासास्तद्वचः श्रुत्वा कालस्यार्थविधायकः}% ६४

\twolineshloka
{क्रुद्धः प्रोवाच सौमित्रिं देहि मेऽन्तप्रवेशनम्}
{नो चेत्त्वां भस्मसात्सद्यः करिष्यामि विचारय}% ६५

\twolineshloka
{वचो दुर्वाससः श्रुत्वा लक्ष्मणो जातसम्भ्रमः}
{मुनेर्भीतो विवेशान्तर्विज्ञापयितुमग्रजम्}% ६६

\twolineshloka
{दृष्ट्वा तु लक्ष्मणं काल उत्थाय कृतमन्त्रकः}
{प्रतिज्ञां पालयेत्युक्त्वा ययौ रामविसर्ज्जितः}% ६७

\twolineshloka
{ततो निष्क्रम्य भगवान् रामो धर्मभृतां वरः}
{प्रतोष्य तं मुनिं प्रीतो दुर्वाससमभोजयत्}% ६८

\twolineshloka
{भोजयित्वा प्रणम्यैनं विसृज्य प्राह लक्ष्मणम्}
{भ्रातर्लक्ष्मण सम्प्राप्तं सङ्कटं धर्मकारणात्}% ६९

\twolineshloka
{यत्त्वं मे वध्यतां प्राप्तो दैवं हि बलवत्तरम्}
{मया त्यक्तस्ततो वीर यथेच्छं गच्छ साम्प्रतम्}% ७०

\twolineshloka
{ततः प्रणम्य तं रामं सत्यधर्मे व्यवस्थितम्}
{दक्षिणां दिशमाश्रित्य तपश्चक्रे नगोपरि}% ७१

\twolineshloka
{ततो रामोऽपि भगवान्ब्रह्मप्रार्थनया पुनः}
{स्वधामाविशदव्यग्रः ससाकेतः सकोशलः}% ७२

\twolineshloka
{गोप्रतारे सरय्वां ये रामं सञ्चिन्त्य सम्प्लुताः}
{ते रामधाम विविशुर्दिव्याङ्गा योगिदुर्लभम्}% ७३

\twolineshloka
{लक्ष्मणस्तु कियत्कालं तपोयोगबलान्वितः}
{रामानुगमनेनैव स्वधामाविशदव्ययम्}% ७४

\twolineshloka
{सान्निध्यं पर्वते तस्मिन्दत्त्वा सौमित्रिरन्वहम्}
{चक्रं निजाधिकारं स ततस्तत्क्षेत्रमुत्तमम्}% ७५

\twolineshloka
{ये पश्यन्ति नरा भक्त्या लक्ष्मणं लक्ष्मणाचले}
{ते कृतार्था न सन्देहो गच्छन्ति हरिमन्दिरम्}% ७६

\twolineshloka
{तत्र दानं प्रशंसन्ति स्वर्णगोभूमिवाजिनाम्}
{दत्तं तत्राक्षयं सर्वं हुतं जप्तं कृतं तथा}% ७७

\twolineshloka
{बहुना किमिहोक्तेन दर्शनं तस्य दुर्लभम्}
{अगस्त्याज्ञान्तरा देवि दृष्टे मुक्तिर्न संशयः}% ७८

\twolineshloka
{एतद्रामचरित्रं तु लक्ष्मणाख्यानसंयुतम्}
{श्रावयेद्योऽपि शृणुयात्स्यातां तौ रामवल्लभौ}% ७९

॥इति श्रीबृहन्नारदीयपुराणे बृहदुपाख्याने उत्तरभागे वसुमोहिनीसंवादे रामलक्ष्मणचरित्रसहितलक्ष्मणाचलमाहात्म्यं नाम पञ्चसप्ततितमोऽध्यायः॥७५॥
