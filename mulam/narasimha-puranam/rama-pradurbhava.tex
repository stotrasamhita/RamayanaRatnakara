\chapt{नरसिंह-पुराणम्}

\src{नरसिंह-पुराणम्}{अध्यायः २६}{}{}
\vakta{}
\shrota{}
\notes{}
\textlink{https://archive.org/details/narasimha-purana-english/page/171/mode/2up}
\translink{https://archive.org/details/narasimha-purana-english/page/171/mode/2up}

\storymeta

\sect{षडविंशोऽध्यायः --- सूर्यवंशानुचरितम्}

\addtocounter{shlokacount}{8}

\uvacha{सूत उवाच}

दीर्घबाहोरजोऽजाद्दशरथः। तस्य गृहे रावणविनाशार्थं साक्षान्नारायणोऽवतीर्णो रामः॥९॥

स तु पितृवचनाद भ्रातृभार्यासहितो दण्डकारण्यं प्राप्य तपश्चचार।

वने रावणापहतभार्यो भ्रात्रा सह दुःखितोऽनेककोटिवानरनायक सुग्रीवसहायो मदोदधौ

सेतुं निबध्य तैर्गत्वा लङ्कां रावणं देवकण्टकं सबान्धवं हत्वा सीतामादाय पुनरयोध्यां

प्राप्य भरताभिषिक्तो विभीषणाय लङ्काराज्यं विमानं वा दत्त्वा तं प्रेषयामास।

स तु परमेश्वरो विमानस्थो विभीषणेन नीयमानो लङ्कायामपि राक्षसपुर्यां वस्तुमनिच्छन् पुण्यारण्यं तत्र स्थापितवान्॥१०॥

तन्निरीक्ष्य तत्रैव महाहिभोगशयने भगवान् शेते। सोऽपि विभीषणस्ततस्तद्विमानं नेतुमसमर्थः, तद्वचनात् स्वां पुरीं जगाम॥११॥

नारायणसन्निधानान्महद्वैष्णवं क्षेत्रमभवदद्यापि दृश्यते। रामाल्लवो लवात्पद्यः पद्मादृतुपर्ण ऋतुपर्णादस्त्रपाणिः।

अस्त्रपाणेः शुद्धोदनः शुद्धोदनाद्वुधः। बुधाद्वंशो निवर्तते॥१२॥

\twolineshloka
{एते महीपा रविवंशजास्तव प्राधान्यतस्ते कथिता महाबलाः}
{पुरातनैर्यैर्वसुधा प्रपालिता यज्ञक्रियाभिश्च दिवौकसैर्नृपैः} % ॥१३॥

॥इति श्रीनरसिंहपुराणे सूर्यवंशानुचरितं नाम षडविंशोऽध्यायः ॥२६॥


\src{नरसिंह-पुराणम्}{अध्यायः ४७--५२}{}{}
\vakta{}
\shrota{}
\notes{Concise retelling of all the Kandas of Ramayana.}
\textlink{https://archive.org/details/narasimha-purana-english/page/171/mode/2up}
\translink{https://archive.org/details/narasimha-purana-english/page/171/mode/2up}

\storymeta


\sect{सप्तचत्वारिंशोऽध्यायः --- बाल-काण्डः}

\uvacha{मार्कण्डेय उवाच}

\twolineshloka
{श्रुणु राजन् प्रवक्ष्यामि प्रादुर्भावं हरेः शुभम्}
{निहतो रावणो येन सगणो देवकण्टकः} %॥१॥

\twolineshloka
{ब्रह्मणो मानसः पुत्रः पुनस्त्योऽभून्महामुनिः}
{तस्य वै विश्रवा नाम पुत्रोऽभूत्तस्य राक्षसः} %॥२॥

\twolineshloka
{तस्माज्जातो महावीरो रावणो लोकरावणः}
{तपसा महता युक्तः स तु लोकानुपाद्रवत्} %॥३॥

\twolineshloka
{सेन्द्रा देवा जितास्तेन गन्धर्वाः किन्नरास्तथा}
{यक्षाश्च दानवाश्चैव तेन राजन् विनिर्जिताः} %॥४॥

\twolineshloka
{स्त्रियश्चैव सुरुपिण्यो हतास्तेन दुरात्मना}
{देवादीनां नृपश्रेष्ठ रत्नानि विविधानि च} %॥५॥

\twolineshloka
{रणे कुबेरं निर्जित्य रावणो बलदर्पितः}
{तत्पुरीं जगृहे लङ्कां विमानं चापि पुष्पकम्} %॥६॥

\twolineshloka
{तस्यां पुर्यां दशग्रीवो रक्षसामधिपोऽभवत्}
{पुत्राश्च बहवस्तस्य बभूवुरमितौजसः} %॥७॥

\twolineshloka
{राक्षसाश्च तमाश्रित्य महाबलपराक्रमाः}
{अनेककोटयो राजन् लङ्कायां निवसन्ति ये} %॥८॥

\twolineshloka
{देवान् पितृन मनुष्यांश्च विद्याधरगणानपि}
{यक्षांश्चैव ततः सर्वे घातयन्ति दिवाशिनम्} %॥९॥

\twolineshloka
{सन्त्रस्तं तद्भयादेव जगदासीच्चराचरम्}
{दुःखाभिभूतमत्त्यर्थं सम्बभूव नराधिप} %॥१०॥

\twolineshloka
{एतस्मिन्ने व काले तु देवाः सेन्द्रा महर्षयः}
{सिद्धा विद्याधराश्चैव गन्धर्वाः किन्नरास्तथा} %॥११॥

\twolineshloka
{गुह्यका भुजगा यक्षा ये चान्ये स्वर्गवासिनः}
{ब्रह्माणमग्रतः कृत्वा शङ्करं च नराधिप} %॥१२॥

\twolineshloka
{ते ययुर्हतविक्रान्ताः क्षीराब्धेस्तटमुत्तमम्}
{तत्राराध्य हरिं देवतास्तस्थुः प्राञ्जलयस्तदा} %॥१३॥

\twolineshloka
{ब्रह्मा च विष्णुमाराध्य गन्धपुष्पादिभिः शुभैः}
{प्राञ्जलिः प्रणतो भूत्वा वासुदेवमथास्तुवत्} %॥१४॥

\uvacha{ब्रह्मोवाच}

\twolineshloka
{नमः क्षीराब्धिवासाय नागपर्यङ्कशायिने}
{नमः श्रीकरसंस्पृष्टदिव्यपादाय विष्णवे} %॥१५॥

\twolineshloka
{नमस्ते योगनिद्राय योगान्तर्भाविताय च}
{तार्क्ष्यासनाय देवाय गोविन्दाय नमो नमः} %॥१६॥

\twolineshloka
{नमः क्षीराब्धिकल्लोलस्पृष्टमात्राय शार्ङ्गिणे}
{नमोऽरविन्दपादाय पद्मनाभाय विष्णवे} %॥१७॥

\twolineshloka
{भक्तार्चितसुपादाय नमो योगाप्रियाय वै}
{शुभाङ्गाय सुनेत्राय माधवाय नमो नमः} %॥१८॥

\twolineshloka
{सुकेशाय सुनेत्राय सुललाटाय चक्रिणे}
{सुवक्त्राय सुकर्णाय श्रीधराय नमो नमः} %॥१९॥

\twolineshloka
{सुवक्षसे सुनाभाय पद्मनाभाय वै नमः}
{सुभ्रुवे चारुदेहाय चारुदन्ताय शार्ङ्गिणे} %॥२०॥

\twolineshloka
{चारुजङ्घाय दिव्याय केशवाय नमो नमः}
{सुनखाय सुशान्ताय सुविद्याय गदाभृते} %॥२१॥

\twolineshloka
{धर्माप्रियाय देवाय वामनाय नमो नमः}
{असुरघ्नाय चोग्राय रक्षोघ्नाय नमो नमः} %॥२२॥

\twolineshloka
{देवानामार्तिनाशाय भीमर्ककृते नमः}
{नमस्ते लोकनाथाय रावणान्तकृते नमः} %॥२३॥

\uvacha{मार्कण्डेय उवाच}

\twolineshloka
{इति स्तुतो हषीकेशस्तुतोष परमेष्ठिना}
{स्वरुपं दर्शयित्वा तु पितामहमुवाच ह} %॥२४॥

\twolineshloka
{किमर्थं तु सुरैः सार्धमागतस्त्वं पितामह}
{यत्कार्य ब्रूहि मे ब्रह्मन् यदर्थं संस्तुतस्त्वया} %॥२५॥

\twolineshloka
{इत्युक्तो देवदेवेन विष्णुना प्रभविष्णुना}
{सर्वदेवगणैः सार्धं ब्रह्मा प्राह जनार्दनम्} %॥२६॥

\uvacha{ब्रह्मोवाच}

\twolineshloka
{नाशितं तु जगत्सर्वं रावणेन दुरात्मना}
{सेन्द्राः पराजितास्तेन बहुशो रक्षसा विभो} %॥२७॥

\twolineshloka
{राक्षसैर्भक्षिता मर्त्या यज्ञाश्चापि विदूषिताः}
{देवकन्या हतास्तेन बलाच्छतसहस्त्रशः} %॥२८॥

\twolineshloka
{त्वामृते पुण्डरीकाक्ष रावणस्य वधं प्रति}
{न समर्था यतो देवास्त्वमतस्तद्वधं कुरु} %॥२९॥

\twolineshloka
{इत्युक्तो ब्रह्मणा विष्णुर्ब्रह्माणमिदमब्रवीत्}
{श्रृणुष्वावहितो ब्रह्मन् यद्वदामि हितं वचः} %॥३०॥

\twolineshloka
{सूर्यवंशोद्भवः श्रीमान् राजाऽऽसीद्भुवि वीर्यवान्}
{नाम्ना दशरथख्यातस्तस्य पुत्रो भवाम्यहम्} %॥३१॥

\twolineshloka
{रावणस्य वधार्थाय चतुर्धांशेन सत्तम}
{स्वांशैर्वानररुपेण सकला देवतागणाः} %॥३२॥

\twolineshloka
{वतार्यन्तां विश्वकर्तः स्यादेवं रावणक्षयः}
{इत्युक्तो देवदेवेन ब्रह्मा लोकपितामहः} %॥३३॥

\twolineshloka
{देवाश्च ते प्रणम्याथ मेरुपृष्ठं तदा ययुः}
{स्वांशैर्वानररुपेण अवतेरुश्च भूतले} %॥३४॥

\twolineshloka
{अथापुत्रो दशरथो मुनिभिर्वेदपारगैः}
{इष्टिं तु कारयामास पुत्रप्राप्तिकरी नृपः} %॥३५॥

\twolineshloka
{ततः सौवर्णपात्रस्थं हविरादाय पायसम्}
{वह्निः कुण्डात् समुत्तस्थौ नूनं देवेन नोदितः} %॥३६॥

\twolineshloka
{आदाय मुनयो मन्त्राच्चक्रुः पिण्डद्वयं शुभम्}
{दत्ते कौशल्यकैकेय्योर्द्वे पिण्डे मन्त्रमन्त्रिते} %॥३७॥

\twolineshloka
{ते पिण्डप्राशने काले सुमित्राया महामते}
{पिण्डाभ्यामल्पमल्पं तु सुभागिन्याः प्रयच्छतः} %॥३८॥

\twolineshloka
{ततस्ताः प्राशयामासू राजपत्न्यो यथाविधि}
{पिण्डान् देवकृतान् प्राश्य प्रापुर्गर्भाननिन्दितान्} %॥३९॥

\twolineshloka
{एवं विष्णुर्दशरथाज्जातस्तत्पत्निषु त्रिषु}
{स्वांशैर्लोकहितायैव चतुर्धा जगतीपते} %॥४०॥

\twolineshloka
{रामश्च लक्ष्मणश्चैव भरतः शत्रुघ्न एव च}
{जातकर्मादिकं प्राप्य संस्कारं मुनिसंस्कृतम्} %॥४१॥

\twolineshloka
{मन्त्रपिण्डवशाद्योगं प्राप्य चेरुर्यथार्भकाः}
{रामश्च लक्ष्मणश्चैव सह नित्यं विचेरतुः} %॥४२॥

\twolineshloka
{जन्मादिकृतसंस्कारौ पितुः प्रीतिकरौ नृप}
{ववृधाते महावीर्यौ श्रुतिशब्दातिलक्षणौ} %॥४३॥

\twolineshloka
{भरतः कैकयो राजन् भ्रात्रा सह गृहेऽवसत्}
{वेदशास्त्राणि बुबुधे शस्त्रशास्त्रं नृपोत्तम} %॥४४॥

\twolineshloka
{एतस्मिन्नेव काले तु विश्वामित्रो महातपाः}
{यागेन यष्टुमारेभे विधिना मधुसूदनम्} %॥४५॥

\twolineshloka
{स तु विघ्नेन यागोऽभूद्राक्षसैर्बहुशः पुरा}
{नेतुं स यागरक्षार्थं सम्प्रातो रामलक्ष्मणौ} %॥४६॥

\twolineshloka
{विश्वामित्रो नृपश्रेष्ठ तत्पितुर्मन्दिरं शुभम्}
{दशरथस्तु तं दृष्ट्वा प्रत्युत्थाय महामतिः} %॥४७॥

\twolineshloka
{अर्घ्यपाद्यादि विधिना विश्वामित्रमपूजयत्}
{स पूजितो मुनिः प्राह राजानं राजसन्निधौ} %॥४८॥

\twolineshloka
{श्रृणु राजन् दशरथ यदर्थमहमागतः}
{तत्कार्यं नृपशार्दूल कथयामि तवाग्रतः} %॥४९॥

\twolineshloka
{राक्षसैर्नाशितो यागो बहुशो मे दुरासदैः}
{यज्ञस्य रक्षणार्थं मे देहि त्वं रामक्ष्मणौ} %॥५०॥

\twolineshloka
{राजा दशरथः श्रुत्वा विश्वामित्रवचो नृप}
{विषण्णवदनो भूत्वा विश्वामित्रमुवाच ह} %॥५१॥

\twolineshloka
{बालाभ्यां मम पुत्राभ्यां किं ते कार्यं भविष्यति}
{अहं त्वया सहागत्य शक्त्या रक्षामि ते मखम्} %॥५२॥

\twolineshloka
{राज्ञस्तु वचनं श्रुत्वा राजानं मुनिरब्रवीत्}
{रामोऽपि शक्नुते नूनं सर्वान्नशयितुं नृप} %॥५३॥

\twolineshloka
{रामेणैव हि ते शक्या न त्वया राक्षसा नृप}
{अतो मे देहि रामं च न चिन्तां कर्तुमर्हसि} %॥५४॥

\twolineshloka
{इत्युक्तो मुनिना तेन विश्वामित्रेण धीमता}
{तूष्णीं स्थित्वा क्षणं राजा मुनिवर्यमुवाच ह} %॥५५॥

\twolineshloka
{यद्ववीमि मुनिश्रेष्ठ प्रसन्नस्त्वं निबोध मे}
{राजीवलोचनं राममहं दास्ये सहानुजम्} %॥५६॥

\twolineshloka
{किं त्वस्य जननी ब्रह्मन् अदृष्टैनं मरिष्यति}
{अतोऽहं चतुरङ्गेण बलेन सहितो मुने} %॥५७॥

\twolineshloka
{आगत्य राक्षसान् हन्मीत्येबं मे मनसि स्थितम्}
{विश्वामित्रः पुनः प्राह राजानममितौजसम्} %॥५८॥

\twolineshloka
{नाज्ञो रामो नृपश्रेष्ठ स सर्वज्ञः समः क्षमः}
{शेषनारायणावेतौ तव पुत्रौ न संशयः} %॥५९॥

\twolineshloka
{दुष्टानां निग्रहार्थाय शिष्टानां पालनाय च}
{अवतीर्णो न सन्देहो गृहे तव नराधिप} %॥६०॥

\twolineshloka
{न मात्रा न त्वया राजन् शोकः कार्योऽत्र चाण्वपि}
{निः क्षेपे च महाराज अर्पयिष्यामि ते सुतौ} %॥६१॥

\twolineshloka
{इत्युक्तो दशरथस्तेन विश्वामित्रेण धीमता}
{तच्छापभीतो मनसा नीयतामित्यभाषत्} %॥६२॥

\twolineshloka
{कृच्छ्रात्पित्रा विनिर्मुक्तं राममादाय सानुजम्}
{ततः सिद्धाश्रमं राजन् सम्प्रतस्थे स कौशिकः} %॥६३॥

\twolineshloka
{तं प्रस्थितमथालोक्य राजा दशरथस्तदा}
{अनुव्रज्याब्रवीदेतद् वचो दशरथस्तदा} %॥६४॥

\twolineshloka
{अपुत्रोऽहं पुरा ब्रह्मन् बहुभिः काम्यकर्मभिः}
{मुनिप्रसादादधुना पुत्रवानस्मि सत्तम} %॥६५॥

\twolineshloka
{मनसा तद्वियोगं तु न शक्ष्यामि विशेषतः}
{त्वमेव जानासि मुने नीत्वा शीघ्रं प्रयच्छ मे} %॥६६॥

\twolineshloka
{इत्येवमुक्तो राजानं विश्वामित्रोऽब्रवीत्पुनः}
{समाप्तयज्ञश्च पुनर्नेष्ये रामं च लक्ष्मणम्} %॥६७॥

\twolineshloka
{सत्यपूर्वं तु दास्यामि न चिन्तां कर्तुमर्हसि}
{इत्युक्तः प्रेषयामास रामं लक्ष्मणसंयुतम्} %॥६८॥

\twolineshloka
{अनिच्छन्नपि राजासौ मुनिशापभयान्नृपः}
{विश्वामित्रस्तु तौ गृह्य अयोध्याया ययौ शनैः} %॥६९॥

\twolineshloka
{सरय्वास्तीरमासाद्य गच्छन्नेव स कौशिकः}
{तयोः प्रीत्या स राजेन्द्र द्वे विद्ये प्रथमं ददौ} %॥७०॥

\twolineshloka
{बलामतिबलां चैव समन्त्रे च ससङ्ग्रहे}
{क्षुत्पिपासापनयने पुनश्चैव महामतिः} %॥७१॥

\twolineshloka
{अस्त्रग्राममशेषं तु शिक्षयित्वा तु तौ तदा}
{आश्रमाणि च दिव्यानि मुनीनां भावितात्मनाम्} %॥७२॥

\twolineshloka
{दर्शयित्वा उषित्वा च पुण्यस्थानेषु सत्तमः}
{गङ्गामुत्तीर्य शोणस्य तीरमासाद्य पश्चिमम्} %॥७३॥

\twolineshloka
{मुनिधार्मिकसिद्धांश्च पश्यन्तौ रामलक्ष्मणौ}
{ऋषिभ्यश्च वरान् प्राप्य तेन नीतौ नृपात्मजौ} %॥७४॥

\twolineshloka
{ताटकाया वनं घोरं मृत्योर्मुखमिवापरम्}
{गते तत्र नृपश्रेष्ठ विश्वामित्रो महातपाः} %॥७५॥

\twolineshloka
{राममक्लिष्टकर्माणमिदं वचनमब्रवीत्}
{राम राम महाबाहो ताटका नाम राक्षसी} %॥७६॥

\twolineshloka
{रावणस्य नियोगेन वसत्यस्मिन् महावने}
{तया मनुष्या बहवो मुनिपुत्रा मृगास्तथा} %॥७७॥

\twolineshloka
{निहता भक्षिताश्चैव तस्मात्तां वध सत्तम}
{इत्येवमुक्तो मुनिना रामस्तं मुनिमब्रवीत्} %॥७८॥

\twolineshloka
{कथं हि स्त्रीवधं कुर्यामहमद्य महामुने}
{स्त्रीवधे तु महापापं प्रवदन्ति मनीषिणः} %॥७९॥

\twolineshloka
{इति रामवचः श्रुत्वा विश्वामित्र उवाच तम्}
{तस्यास्तु निधनाद्राम जनाः सर्वे निराकुलाः} %॥८०॥

\twolineshloka
{भवन्ति सततं तस्मात् तस्याः पुण्यप्रदो वधः}
{इत्येवं वादिनि मुनौ विश्वामित्रे निशाचरी} %॥८१॥

\twolineshloka
{आगता सुमहाघोरा ताटका विवृतानना}
{मुनिना प्रेरितो रामस्तां दृष्ट्वा विवृताननाम्} %॥८२॥

\twolineshloka
{उद्यतैकभुजयष्टिमायतीं श्रोणिलम्बिपुरुषान्त्रमेखलाम्}
{तां विलोक्य वनितावधे घृणां पत्रिणा सह मुमोच राघवः} %॥८३॥

\twolineshloka
{शरं सन्धाय वेगेन तेन तस्या उरः स्थलम्}
{विपाटितं द्विधा राजन् सा पपात ममार च} %॥८४॥

\twolineshloka
{घातयित्वा तु तामेवं तावानीय मुनिस्तु तौ}
{प्रापयामास तं तत्र नानाऋषिनिषेवितम्} %॥८५॥

\twolineshloka
{नानाद्रुमलताकीर्णं नानापुष्पोपशोभितम्}
{नानानिर्झरतोयाढ्यं विन्ध्यशैलान्तरस्थितम्} %॥८६॥

\twolineshloka
{शकमूलफलोपेतं दिव्यं सिद्धाश्रमं स्वकम्}
{रक्षार्थं तावुभौ स्थाप्य शिक्षयित्वा विशेषतः} %॥८७॥

\twolineshloka
{ततश्चारब्धवान् यागं विश्वामित्रो महातपाः}
{दीक्षां प्रविष्टे च मुनौ विश्वामित्रे महात्मनि} %॥८८॥

\twolineshloka
{यज्ञे तु वितते तत्र कर्म कुर्वन्ति ऋत्विजः}
{मारीचश्च सुबाहुश्च बहवश्चान्यराक्षसाः} %॥८९॥

\twolineshloka
{आगता यागनाशाय रावणेन नियोजिताः}
{तानागतान् स विज्ञाय रामः कमललोचनः} %॥९०॥

\twolineshloka
{शरेण पातयामास सुबाहुं धरणीतले}
{असृक्प्रवाहं वर्षन्तं मारीचं भल्लकेन तु} %॥९१॥

\twolineshloka
{प्रताङ्य नीतवानब्धिं यथा पर्णं तु वायुना}
{शेषांस्तु हतवान् रामो लक्ष्मणश्च निशाचरान्} %॥९२॥

\twolineshloka
{रामेण रक्षितमखो विश्वामित्रो महायशाः}
{समाप्य यागं विधिवत् पूजयामास ऋत्विजान्} %॥९३॥

\twolineshloka
{सदस्यानपि सम्पूज्य यथार्हं च ह्यरिन्दम}
{रामं च लक्ष्मणं चैव पूजयामास भक्तितः} %॥९४॥

\twolineshloka
{ततो देवगणस्तुष्टो यज्ञभागेन सत्तम}
{ववर्ष पुष्पवर्षं तु रामदेवस्य मूर्धनि} %॥९५॥

\twolineshloka
{निवार्य राक्षसभयं कारयित्वा तु तन्मखम्}
{श्रुत्वा नानाकथाः पुण्या रामो भ्रातृसमन्वितः} %॥९६॥

\twolineshloka
{तेन नीतो विनीतात्मा अहल्या यत्र तिष्ठति}
{व्यभिचारान्महेन्द्रेण भर्त्रा शप्ता हि सा पुरा} %॥९७॥

\twolineshloka
{पाषाणभूता राजेन्द्र तस्य रामस्य दर्शनात्}
{अहल्या मुक्तशापा च जगाम गौतमं प्रति} %॥९८॥

\twolineshloka
{विश्वामित्रस्ततस्तत्र चिन्तयामास वै क्षणम्}
{कृतदारो मया नेयो रामः कमललोचनः} %॥९९॥

\twolineshloka
{इति सञ्चिन्त्य तौ गृह्य विश्वामित्रो महातपाः}
{शिष्यैः परिवृतोऽनेकैर्जगाम मिथिलां प्रति} %॥१००॥

\twolineshloka
{नानादेशादथायाता जनकस्य निवेशनम्}
{राजपुत्रा महावीर्याः पूर्वं सीताभिकाङ्क्षिणः} %॥१०१॥

\twolineshloka
{तान् दृष्ट्वा पूजयित्वा तु जनकश्च यथार्हतः}
{यत्सीतायाः समुत्पन्नं धनुर्माहेश्वरं महत्} %॥१०२॥

\twolineshloka
{अर्चितं गन्धमालाभी रम्यशोभासमन्विते}
{रङ्गे महति विस्तीर्णे स्थापयामास तद्धनुः} %॥१०३॥

\twolineshloka
{उवाच च नृपान् सर्वांस्तदोच्चैर्जनको नृपः}
{आकर्षणादिदं येन धनुर्भग्नं नृपात्मजाः} %॥१०४॥

\twolineshloka
{तस्येयं धर्मतो भार्या सीता सर्वाङ्गशोभना}
{इत्येवं श्राविते तेन जनकेन महात्मना} %॥१०५॥

\twolineshloka
{क्रमादादाय ते तत्तु सज्यीकर्तुमथाभवन्}
{धनुषा ताडिताः सर्वे क्रमात्तेन महीपते} %॥१०६॥

\twolineshloka
{विधूय पतिता राजन् विलजास्तत्र पार्थिवाः}
{तेषु भग्नेषु जनकस्तद्धनुस्त्र्यम्बकं नृप} %॥१०७॥

\twolineshloka
{संस्थाप्य स्थितवान् वीरो रामागमनकाङ्क्षया}
{विश्वामित्रस्ततः प्राप्तो मिथिलाधिपतेर्गृहम्} %॥१०८॥

\twolineshloka
{जनकोऽपि च तं दृष्टवा विश्वामित्रं गृहागतम्}
{रामलक्ष्मणसंयुक्तं शिष्यैश्चाभिगतं तदा} %॥१०९॥

\twolineshloka
{तं पूजयित्वा विधिवत्प्राज्ञं विप्रानुयायिनम्}
{रामं रघुपतिं चापि लावण्यादिगुणैर्युतम्} %॥११०॥

\twolineshloka
{शीलाचारगुणोपेतं लक्ष्मणं च महामतिम्}
{पूजयित्वा यथान्यायं जनकः प्रीतमानसः} %॥१११॥

\twolineshloka
{हेमपीठे सुखासीनं शिष्यैः पूर्वापरैर्वृतम्}
{विश्वामित्रमुवाचाथ किं कर्तव्यं मयेति सः} %॥११२॥

\uvacha{मार्कण्डेय उवाच}

\twolineshloka
{इति श्रुत्वा वचस्तस्य मुनिः प्राह महीपतिम्}
{एष रामो महाराज विष्णुः साक्षान्महीपतिः} %॥११३॥

\twolineshloka
{रक्षार्थं विष्टपानां तु जातो दशरथात्मजः}
{अस्मै सीतां प्रयच्छ त्वं देवकन्यामिव स्थिताम्} %॥११४॥

\twolineshloka
{अस्या विवाहे राजेन्द्र धनुर्भङ्गमुदीरितम्}
{तदानय भवधनुरर्चयस्व जनाधिप} %॥११५॥

\twolineshloka
{तथेत्युक्त्वा च राजा हि भवचापं तदद्भुतम्}
{अनेक भूभुजां भङ्गि स्थापयामास पूर्ववत्} %॥११६॥

\twolineshloka
{ततो दशरथसुतो विश्वामित्रेण चोदितः}
{तेषां मध्यात्समुत्थाय रामः कमललोचनः} %॥११७॥

\twolineshloka
{प्रणम्य विप्रान् देवांश्च धनुरादाय तत्तदा}
{सज्यं कृत्वा महाबाहुर्ज्याघोषमकरोत्तदा} %॥११८॥

\twolineshloka
{आकृष्यमाणं तु बलात्तेन भग्नं महद्धनुः}
{सीता च मालामादाय शुभां रामस्य मूर्धनि} %॥११९॥

\twolineshloka
{क्षिप्त्वा संवरयामास सर्वक्षत्रियसन्निधौ}
{ततस्ते क्षत्रियाः क्रुद्धा राममासाद्य सर्वतः} %॥१२०॥

\twolineshloka
{मुमुचुः शरजालानि गर्जयन्तो महाबलाः}
{तान्निरीक्ष्य ततो रामो धनुरादाय वेगवान्} %॥१२१॥

\twolineshloka
{ज्याघोषतलघोषेण कम्पयामास तान्नृपान्}
{चिच्छेद शरजालानि तेषां स्वास्त्रै रथांस्ततः} %॥१२२॥

\twolineshloka
{धनूंषि च पताकाश्च रामश्चिच्छेद लीलया}
{सन्नह्य स्वबलं सर्वं मिथिलाधिपतिस्ततः} %॥१२३॥

\twolineshloka
{जामातरं रणे रक्षन् पार्ष्णिग्राहो बभूव ह}
{लक्ष्मणश्च महावीरो विद्राव्य युधि तान्नृपान्} %॥१२४॥

\twolineshloka
{हस्त्यश्वाञ्जगृहे तेषां स्यन्दनानि बहूनि च}
{वाहनानि परित्यज्य पलायनपरान्नृपान्} %॥१२५॥

\twolineshloka
{तान्निहन्तुं च धावत्स पृष्ठतो लक्ष्मणस्तदा}
{मिथिलाधिपतिस्तं च वारयामास कौशिकः} %॥१२६॥

\twolineshloka
{जितसेनं महावीरं रामं भ्रात्रा समन्वितम्}
{आदाय प्रविवेशाथ जनकः स्वगृहं शुभम्} %॥१२७॥

\twolineshloka
{दूतं च प्रेषयामास तदा दशरथाय सः}
{श्रुत्वा दूतमुखात् सर्वं विदितार्थः स पार्थिवः} %॥१२८॥

\twolineshloka
{सभार्यः ससुतः श्रीमान् हस्त्यश्वरथवाहनः}
{मिथिलामाजगामाशु स्वबलेन समन्वितः} %॥१२९॥

\twolineshloka
{जनकोऽप्यस्य सत्कारं कृत्वा स्वां च सुतां ततः}
{विधिवत्कृतशुल्कां तां ददौ रामाय पार्थिव} %॥१३०॥

\twolineshloka
{अपराश्च सुतास्तिस्त्रो रुपवत्यः स्वलडकृताः}
{त्रिभ्यस्तु लक्ष्मणादिभ्यः स्वकन्या विधिवद्ददौ} %॥१३१॥

\twolineshloka
{एवं कृतविवाहोऽसौ रामः कमललोचनः}
{भ्रातृभिर्मातृभिः सार्धं पित्रा बलवता सह} %॥१३२॥

\threelineshloka
{दिनानि कतिचित्तत्र स्थितो विविधभोजनैः}
{ततोऽयोध्यापुरीं गन्तुमुत्सुकं ससुतं नृपम्}
{दृष्ट्वा दशरथं राजा सीतायाः प्रददौ वसु} %॥१३३॥

\fourlineindentedshloka
{रत्नानि दिव्यानि बहूनि दत्त्वा}
{रामाय वस्त्राण्यतिशोभनानि}
{हस्त्यश्वदासानपि कर्मयोग्यान्}
{दासीजनांश्च प्रवराः स्त्रियश्च} %॥१३४॥

\fourlineindentedshloka
{सीतां सुशीलां बहुरत्नभूषितां}
{रथं समारोप्य सुतां सुरुपाम्}
{वेदादिघोषैर्बहुमङ्गलैश्च}
{सम्प्रेषयामास स पार्थिवो बली} %॥१३५॥

\twolineshloka
{प्रेषयित्वा सुतां दिव्यां नत्वा दशरथं नृपम्}
{विश्वामित्रं नमस्कृत्य जनकः सन्निवृत्तवान्} %॥१३६॥

\twolineshloka
{तस्य पल्यो महाभागाः शिक्षयित्वा सुतां तदा}
{भर्तृभक्तिं कुरु शुभे श्वश्रूणां श्वशुरस्य च} %॥१३७॥

\twolineshloka
{श्वश्रूणामर्पयित्वा तां निवृत्ता विविशुः पुरम्}
{ततस्तु रामं गच्छन्तमयोध्यां प्रबलान्वितम्} %॥१३८॥

\twolineshloka
{श्रुत्वा परशुरामो वै पन्थानं संरुरोध ह}
{तं दृष्ट्वा राजपुरुषाः सर्वे ते दीनमानसाः} %॥१३९॥

\twolineshloka
{आसीद्दशरथश्चापि दुःखशोकपरिप्लुतः}
{सभार्यः सपरीवारो भार्गवस्य भयान्नृप} %॥१४०॥

\twolineshloka
{ततोऽब्रवीज्जनान् सर्वान् राजानं च सुदुः खितम्}
{वसिष्ठश्चोर्जिततपा ब्रह्मचारी महामुनिः} %॥१४१॥

\uvacha{वसिष्ठ उवाच}

\onelineshloka
{युष्माभिरत्र रामार्थं न कार्य दुःखमण्वपि} %॥१४२॥

\twolineshloka
{पित्रा वा मातृभिर्वापि अन्यैर्भृत्यजनैरपि}
{अयं हि नृपते रामः साक्षाद्विष्णुस्तु ते गृहे} %॥१४३॥

\twolineshloka
{जगतः पालनार्थाय जन्मप्राप्तो न संशयः}
{यस्य सकीर्त्य नामपि भवभीतिः प्रणश्चति} %॥१४४॥

\twolineshloka
{ब्रह्म मूर्तं स्वयं यत्र भयादेस्तत्र का कथा}
{यत्र सकीर्त्यते रामकथामात्रमपि प्रभो} %॥१४५॥

\twolineshloka
{नोपसर्गभयं तत्र नाकालमरणं नृणाम्}
{इत्युक्ते भार्गवो रामो राममाहाग्रतः स्थितम्} %॥१४६॥

\twolineshloka
{त्यज त्वं रामसज्ञां तु मया वा सगरं कुरु}
{इत्युक्ते राघवः प्राह भार्गवं तं पथि स्थितम्} %॥१४७॥

\twolineshloka
{रामसज्ञां कुतस्त्यक्ष्ये त्वया योत्स्ये स्थिरो भव}
{इत्युक्त्वा तं पृथक् स्थित्वा रामो राजीवलोचनः} %॥१४८॥

\twolineshloka
{ज्याघोषमकरोद्वीरो वीरस्यैवाग्रतस्तदा}
{ततः परशुरामस्य देहान्निष्क्रम्य वैष्णवम्} %॥१४९॥

\twolineshloka
{पश्यतां सर्वभूतानां तेजो राममुखेऽविशत्}
{दृष्ट्वा तं भार्गवो रामः प्रसन्नवदनोऽब्रवीत्} %॥१५०॥

\twolineshloka
{राम राम महाबाहो रामस्त्वं नात्र संशयः}
{विष्णुरेव भवाञ्जातो ज्ञातोऽस्यद्य मया विभो} %॥१५१॥

\twolineshloka
{गच्छ वीर यथाकामं देवकार्यं च वै कुरु}
{दुष्टानां निधनं कृत्वा शिष्टांश्च परिपालय} %॥१५२॥

\twolineshloka
{याहि त्वं स्वेच्छया राम अहं गच्छे तपोवनम्}
{इत्युक्त्वा पूजितस्तैस्तु मुनिभावेन भार्गवः} %॥१५३॥

\twolineshloka
{महेन्द्राद्रिं जगामाथ तपसे धृतमानसः}
{ततस्तु जातहर्षास्ते जना दशरथश्च ह} %॥१५४॥

\twolineshloka
{पुरीमयोध्यां सम्प्राप्य रामेण सह पार्थिवः}
{दिव्यशोभां पुरीं कृत्वा सर्वतो भद्रशालिनीम्} %॥१५५॥

\twolineshloka
{प्रत्युत्थाय ततः पौराः शङ्खतूर्यादिभिः स्वनैः}
{विशन्तं राममागत्य कृतदारं रणेऽजितम्} %॥१५६॥

\twolineshloka
{तं वीक्ष्य हर्षिताः सन्तो विविशुस्तेन वै पुरीम्}
{तौ दृष्ट्वा स मुनिः प्राप्तौ रामं लक्ष्मणमन्तिके} %॥१५७॥

\threelineshloka
{दशरथाय तत्पित्रे मातृभ्यश्च विशेषतः}
{तौ समर्प्य मुनिश्रेष्ठस्तेन राज्ञा च पूजितः}
{विश्वामित्रश्च सहसा प्रतिगन्तुं मनो दधे} %॥१५८॥

\fourlineindentedshloka
{समर्प्य राम स मुनिः सहानुजं}
{सभार्यमग्ने पितुरेकवल्लभम्}
{पुनः पुनः श्राव्य हसन्महामतिर्-}
{जगाम सिद्धाश्रममेवमात्मनः} %॥१५९॥

॥इति श्रीनरसिंहपुराणे रामप्रादुर्भावे सप्तचत्वारिंशोऽध्यायः॥४७॥

\sect{अष्टचत्वारिंशोऽध्यायः --- अयोध्या-काण्डः}

\uvacha{मार्कण्डेय उवाच}

\twolineshloka
{कृतदारो महातेजा रामः कमललोचनः}
{पित्रे सुमहतीं प्रीतिं जनानामुपपादयन्} %॥१॥

\twolineshloka
{अयोध्यायां स्थितो रामः सर्वभोगसमन्वितः}
{प्रीत्या नन्दत्ययोध्यायां रामे रघुपतौ नृप} %॥२॥

\twolineshloka
{भ्राता शत्रुघ्नसहितो भरतो मातुलं ययौ}
{ततो दशरथो राजा प्रसमीक्ष्य सुशोभनम्} %॥३॥

\twolineshloka
{युवानं बलिनं योग्यं भूपसिद्ध्यै सुतं कविम्}
{अभिषिच्य राज्यभारं रामे संस्थाप्य वैष्णवम्} %॥४॥

\twolineshloka
{पदं प्राप्तुं महद्यत्नं करिष्यामीत्यचिन्तयत्}
{सचिन्त्य तत्परो राजा सर्वदिक्षु समादिशत्} %॥५॥

\twolineshloka
{प्राज्ञान् भृत्यान महीपालान्मन्त्रिणश्च त्वरान्वितः}
{रामाभिषेकद्रव्याणि ऋषिप्रोक्तानि यानि वै} %॥६॥

\twolineshloka
{तानि भृत्याः समाहत्य शीघ्रमागन्तुमर्हथ}
{दूतामात्याः समादेशात्सर्वदिक्षु नराधिपान्} %॥७॥

\twolineshloka
{आहूय तान् समाहत्य शीघ्रमागन्तुमर्हथ}
{अयोध्यापुरमत्यर्थं सर्वशोभासमन्वितम्} %॥८॥

\twolineshloka
{जनाः कुरुत सर्वत्र नृत्यगीतादिनन्दितम्}
{पुरवासिजनानन्दं देशवासिमनः प्रियम्} %॥९॥

\twolineshloka
{रामाभिषेकं विपुलं श्वो भविष्यति जानथ}
{श्रुत्वेत्थं मन्त्रिणः प्राहुस्तं नृपं प्रणिपत्य च} %॥१०॥

\twolineshloka
{शोभनं ते मतं राजन् यदिदं परिभाषितम्}
{रामाभिषेकमस्माकं सर्वेषां च प्रियकरम्} %॥११॥

\twolineshloka
{इत्युक्तो दशरथस्तैस्तान् सर्वान् पुनरब्रवीत्}
{आनीयन्तां द्रुतं सर्वे सम्भारा मम शासनात्} %॥१२॥

\twolineshloka
{सर्वतः सारभूता च पुरी चेयं समन्ततः}
{अद्य शोभान्विता कार्या कर्तव्यं यागमण्डलम्} %॥१३॥

\twolineshloka
{इत्येवमुक्ता राज्ञा ते मन्त्रिणः शीघ्रकारिणः}
{तथैव चक्रुस्ते सर्वे पुनः पुनरुदीरिताः} %॥१४॥

\twolineshloka
{प्राप्तहर्षः स राजा च शुभं दिनमुदीक्षयन्}
{कौशल्या लक्ष्मणश्चैव सुमित्रा नागरो जनः} %॥१५॥

\twolineshloka
{रामाभिषेकमाकर्ण्य मुदं प्राप्यातिहर्षितः}
{श्वश्रूश्वशुरयोः सम्यक् शुश्रूषपणपरा तु सा} %॥१६॥

\twolineshloka
{मुदान्विता सिता सीता भर्तुराकर्ण्य शोभनम्}
{श्वोभाविन्यभिषेके तु रामस्य विदितात्मनः} %॥१७॥

\twolineshloka
{दासी तु मन्थरानाम्नी कैकेय्याः कुब्जरुपिणी}
{स्वां स्वामिनीं तु कैकेयीमिदं वचनमब्रवीत्} %॥१८॥

\twolineshloka
{श्रृणु राज्ञि महाभागे वचनं मम शोभनम्}
{त्वत्पतिस्तु महाराजस्तव नाशाय चोद्यतः} %॥१९॥

\twolineshloka
{रामोऽसौ कौसलीपुत्रः श्वो भविष्यति भूपतिः}
{वसुवाहनकोशादि राज्यं च सकलं शुभे} %॥२०॥

\twolineshloka
{भविष्यत्यद्य रामस्य भरतस्य न किचन}
{भरतोऽपि गतो दूरं मातुलस्य गृहं प्रति} %॥२१॥

\twolineshloka
{हा कष्टं मन्दभाग्यासि सापल्याद्दुःखिता भृशम्}
{सैवमाकर्ण्य कैकेयी कुब्जामिदमथाब्रवीत्} %॥२२॥

\twolineshloka
{पश्य मे दक्षतां कुब्जे अद्यैव त्वं विचक्षणे}
{यथा तु सकलं राज्यं भरतस्य भविष्यति} %॥२३॥

\twolineshloka
{रामस्य वनवासश्च तथा यत्नं करोम्यहम्}
{इत्युक्त्वा मन्थरां सा तु उन्मुच्य स्वाङ्गभूषणम्} %॥२४॥

\twolineshloka
{वस्त्रं पुष्पाणि चोन्मुच्य स्थूलवासोधराभवत्}
{निर्माल्यपुष्पधृक्कष्टा कश्मलाङ्गी विरुपिणी} %॥२५॥

\twolineshloka
{भस्मधूल्यादिनिर्दिग्धा भस्मधूल्या तथा श्रिते}
{भूभागे शान्तदीपे सा सन्ध्याकाले सुदुःखिता} %॥२६॥

\twolineshloka
{ललाटे श्वेतचैलं तु बद्ध्वा सुष्वाप भामिनी}
{मन्त्रिभिः सह कार्याणि सम्मन्त्र्य सकलानि तु} %॥२७॥

\twolineshloka
{पुण्याहः स्वस्तिमाङ्गल्यैः स्थाप्य रामं तु मण्डले}
{ऋषिभस्तु वसिष्ठाद्यैः सार्धं सम्भारमण्डपे} %॥२८॥

\twolineshloka
{वृद्धिजागरणीयैश्च सर्वतस्तूर्यनादिते}
{गीतनृत्यसमाकीर्णे शङ्खकाहलनिः स्वनैः} %॥२९॥

\twolineshloka
{स्वयं दशरथस्तत्र स्थित्वा प्रत्यागतः पुनः}
{कैकेया वेश्मनो द्वारं जरद्भिः परिरक्षितम्} %॥३०॥

\twolineshloka
{रामाभिषेकं कैकेयीं वक्तुकामः स पार्थिवः}
{कैकेयीभवनं वीक्ष्य सान्धकारमथाब्रवीत्} %॥३१॥

\twolineshloka
{अन्धकारमिदं कस्मादद्य ते मन्दिरे प्रिये}
{रामाभिषेकं हर्षाय अन्त्यजा अपि मेनिरे} %॥३२॥

\twolineshloka
{गृहालकरणं कुर्वन्त्यद्य लोका मनोहरम्}
{त्वयाद्य न कृतं कस्मादित्युक्त्वा च महीपतिः} %॥३३॥

\twolineshloka
{ज्वालायित्वा गृहे दीपान् प्रविवेश गृहं नृपः}
{अशोभनाङ्गीं कैकेयीं स्वपन्तीं पतितां भुवि} %॥३४॥

\twolineshloka
{दृष्ट्वा दशरथः प्राह तस्याः प्रियमिदं त्विति}
{आश्लिष्योत्थाय तां राजा श्रृणु मे परमं वचः} %॥३५॥

\twolineshloka
{स्वमातुरधिकां नित्यं यस्ते भक्तिं करोति वै}
{तस्याभिषेकं रामस्य श्वो भविष्यति शोभने} %॥३६॥

\twolineshloka
{इत्युक्ता पार्थिवेनापि किचिन्नोवाच सा शुभा}
{मुञ्चन्ती दीर्घमुष्णं च रोषोस्च्छ्वासं मुहुर्मुहुः} %॥३७॥

\twolineshloka
{तस्थावाश्लिष्य हस्ताभ्यां पार्थिवः प्राह रोषिताम्}
{किं ते कैकेयि दुःखस्य कारणं वद शोभने} %॥३८॥

\twolineshloka
{वस्त्राभरणरत्नादि यद्यदिच्छसि शोभने}
{तत्त्वं गृह्णीष्व निश्शङ्कं भाण्डारात् सुखिनी भव} %॥३९॥

\twolineshloka
{भाण्डारेण मम शुभे श्वोऽर्थसिद्धिर्भविष्यति}
{यदाभिषेकं सम्प्राप्ते रामे राजीवलोचने} %॥४०॥

\twolineshloka
{भाण्डागारस्य मे द्वारं मया मुक्तं निरर्गलम्}
{भविष्यति पुनः पूर्णं रामे राज्यं प्रशासति} %॥४१॥

\twolineshloka
{बहु मानय रामस्य अभिषेकं महात्मनः}
{इत्युक्ता राजवर्य्येण कैकेयी पापलक्षणा} %॥४२॥

\twolineshloka
{कुमतिर्नर्घुणा दुष्टा कुब्जया शिक्षिताब्रवीत्}
{राजानं स्वपतिं वाक्यं क्रूरमत्यन्तनिष्ठुरम्} %॥४३॥

\twolineshloka
{रत्नादि सकलं यत्ते तन्ममैव न संशयः}
{देवासुरमहायुद्धे प्रीत्या यन्मे वरद्वयम्} %॥४४॥

\twolineshloka
{पुरा दत्तं त्वया राजंस्तदिदानीं प्रयच्छ मे}
{इत्युक्तः पार्थिवः प्राह कैकेयीमशुभां तदा} %॥४५॥

\twolineshloka
{अदत्तमप्यहं दास्ये तव नान्यस्य वा शुभे}
{किं मे प्रतिश्रुतं पूर्वं दत्तमेव मया तव} %॥४६॥

\twolineshloka
{शुभाङ्गी भव कल्याणि त्यज कोपमनर्थकम्}
{रामाभिषेकजं हर्षं भजोत्तिष्ठ सुखी भव} %॥४७॥

\twolineshloka
{इत्युक्ता राजवर्येण कैकेयी कलहप्रिया}
{उवाच परुषं वाक्यं राज्ञो मरणकारणम्} %॥४८॥

\twolineshloka
{वरद्वयं पूर्वदत्तं यदि दास्यसि मे विभो}
{श्वोभूते गच्छतु वनं रामोऽयं कोशलात्मजः} %॥४९॥

\twolineshloka
{द्वादशाब्दं निवसतु त्वद्वाक्याद्दण्डके वने}
{अभिषेकं च राज्यं च भरतस्य भविष्यति} %॥५०॥

\twolineshloka
{इत्याकर्ण्य स कैकेया वचनं घोरमप्रियम्}
{पपात भुवि निस्सज्ञो राजा सापि विभूषिता} %॥५१॥

\twolineshloka
{रात्रिशेषं नयित्वा तु प्रभाते सा मुदावती}
{दूतं सुमन्त्रमाहैवं राम आनीयतामिति} %॥५२॥

\twolineshloka
{रामस्तु कृतपुण्याहः कृतस्वस्त्ययनो द्विजैः}
{यागमण्डपमध्यस्थः शङ्खतूर्यरवान्वितः} %॥५३॥

\twolineshloka
{तमासाद्य ततो दूतः प्रणिपत्य पुरः स्थितः}
{राम राम महाबाहो आज्ञापयति ते पिता} %॥५४॥

\twolineshloka
{द्रुतमुत्तिष्ठ गच्छ त्वं यत्र तिष्ठति ते पिता}
{इत्युक्तस्तेन दूतेन शीघ्रमुत्थाय राघवः} %॥५५॥

\twolineshloka
{अनुज्ञाप्य द्विजान् प्राप्तः कैकेय्या भवनं प्रति}
{प्रविशन्तं गृहं रामं कैकेयी प्राह निर्घृणा} %॥५६॥

\twolineshloka
{पितुस्तव मतं वत्स इदं ते प्रब्रवीम्यहम्}
{वने वस महाबाहो गत्वा त्वं द्वादशाब्दकम्} %॥५७॥

\twolineshloka
{अद्यैव गम्यतां वीर तपसे धृतमानसः}
{न चिन्त्यमन्यथा वत्स आदरात् कुरु मे वचः} %॥५८॥

\twolineshloka
{एतच्छुत्वा पितुर्वाक्यं रामः कमललोचनः}
{तथेत्याज्ञां गृहीत्वासौ नमस्कृत्य च तावुभौ} %॥५९॥

\twolineshloka
{निष्क्रम्य तदगृहाद्रामो धनुरादाय वेश्मतः}
{कौशल्यां च नमस्कृत्य सुमित्रां गन्तुमुद्यतः} %॥६०॥

\twolineshloka
{तच्छुत्वा तु ततः पौरा दुःखशोकपरिप्लुताः}
{विव्यथुश्चाथ सौमित्रिः कैकेयीं प्रति रोषितः} %॥६१॥

\twolineshloka
{ततस्तं राघवो दृष्ट्वा लक्ष्मणं रक्तलोचनम्}
{बारयामास धर्मज्ञो धर्मवाग्भिर्महामतिः} %॥६२॥

\twolineshloka
{ततस्तु तत्र ये वृद्धास्तान प्रणम्य मुनींश्च सः}
{रामो रथं खिन्नसूतं प्रस्थानायारुरोह वै} %॥६३॥

\twolineshloka
{आत्मीयं सकलं द्रव्यं ब्राह्मणेभ्यो नृपात्मजः}
{श्रद्धया परया दत्त्वा वस्त्राणि विविधानि च} %॥६४॥

\twolineshloka
{तिस्त्रः श्वश्रूः समामन्त्र्य श्वशुरं च विसज्ञितम्}
{मुञ्चन्तमश्रुधाराणि नेत्रयोः शोकजानि च} %॥६५॥

\twolineshloka
{पश्यती सर्वतः सीता चारुरोह तथा रथम्}
{रथमारुह्य गच्छन्तं सीतया सह राघवम्} %॥६६॥

\twolineshloka
{दृष्ट्वा सुमित्रा वचनं लक्ष्मणं चाह दुःखिता}
{रामं दशरथं विद्धि मां विद्धि जनकात्मजाम्} %॥६७॥

\twolineshloka
{अयोध्यामटर्वी विद्धि व्रज ताभ्यां गुणाकर}
{मात्रैवमुक्तो धर्मात्मा स्तनक्षीरार्द्रदेहया} %॥६८॥

\twolineshloka
{तां नत्वा चारुयानं तमारुरोह स लक्ष्मनः}
{गच्छतो लक्ष्मणो भ्राता सीता चैव पतिव्रताः} %॥६९॥

\twolineshloka
{रामस्य पृष्ठतो यातौ पुराद्धीरौ महामते}
{विधिच्छिन्नाभिषेकं तं रामं राजीवलोचनम्} %॥७०॥

\twolineshloka
{अयोध्याया विनिष्क्रान्तमनुयाताः पुरोहिताः}
{मन्त्रिणः पौरमुख्याश्च दुःखेन महतान्विताः} %॥७१॥

\twolineshloka
{तं च प्राप्य हि गच्छन्तं राममूचुरिदं वचः}
{राम राम महाबाहो गन्तुं नार्हसि शोभन} %॥७२॥

\twolineshloka
{राजन्नत्र निवर्तस्व विहायास्मान् क्व गच्छसि}
{इत्युक्तो राघवस्तैस्तु तानुवाच दृढव्रतः} %॥७३॥

\twolineshloka
{गच्छध्वं मन्त्रिणः पौरा गच्छध्वं च पुरोधसः}
{पित्रादेशं मया कार्यमभियास्यामि वै वनम्} %॥७४॥

\twolineshloka
{द्वादशाब्दं व्रतं चैतन्नीत्वाहं दण्डके वने}
{आगच्छामि पितुः पादं मातृणां द्रष्टुमञ्जसा} %॥७५॥

\twolineshloka
{इत्युक्त्वा ताञ्जगामाथ रामः सत्यपरायणः}
{तं गच्छन्तं पुनर्याताः पृष्ठतो दुःखिता जनाः} %॥७६॥

\twolineshloka
{पुनः प्राह स काकुत्स्थो गच्छध्वं नगरीमिमाम्}
{मातृश्च पितरं चैव शत्रुघ्नं नगरीमिमाम्} %॥७७॥

\twolineshloka
{प्रजाः समस्तास्त्रत्रस्था राज्यं भरतमेव च}
{पालयध्वं महाभागास्तपसे याम्यहं वनम्} %॥७८॥

\twolineshloka
{अथ लक्ष्मणमाहेदं वचनं राघवस्तदा}
{सीतामर्पय राजानं जनकं मिथिलेश्वरम्} %॥७९॥

\twolineshloka
{पितृमातृवशे तिष्ठ गच्छ लक्ष्मण याम्यहम्}
{इत्युक्तः प्राह धर्मात्मा लक्ष्मणो भ्रातृवत्सलः} %॥८०॥

\twolineshloka
{मैवामाज्ञापाय विभो मामद्य करुणाकर}
{गन्तुमिच्छसि यत्र त्वमवश्यं तत्र याम्यहम्} %॥८१॥

\twolineshloka
{इत्युक्तो लक्ष्मणेनासौ सीतां तामाह राघवः}
{सीते गच्छ ममादेशात् पितरं प्रति शोभने} %॥८२॥

\twolineshloka
{सुमित्राया गृहे चापि कौशल्यायाः सुमध्यमे}
{निवर्तस्व हि तावत्त्वं यावदागमनं मम} %॥८३॥

\twolineshloka
{इत्युक्ता राघवेनापि सीता प्राह कृताञ्जलिः}
{यत्र गत्वा वने वासं त्वं करोषि महाभुज} %॥८४॥

\twolineshloka
{तत्र गत्वा त्वया सार्धं वसाम्यहमरिन्दम}
{वियोगं नो सहे राजंस्त्वया सत्यवता क्वचित्} %॥८५॥

\twolineshloka
{अतस्त्वां प्रार्थयिष्यामि दयां कुरु मम प्रभो}
{गन्तुमिच्छसि यत्र त्वमवश्यं तत्र याम्यहम्} %॥८६॥

\twolineshloka
{नानायानैरुपगताञ्जनान् वीक्ष्य स पृष्ठतः}
{योषितां च गणान् रामो वारयामास धर्मवित्} %॥८७॥

\twolineshloka
{निवृत्त्य स्थीयतां स्वैरमयोध्यायां जनाः स्त्रियः}
{गत्वाहं दण्डकारण्यं तपसे धृतमानसः} %॥८८॥

\twolineshloka
{कतिपयाब्दादायास्ये नान्यथा सत्यमीरितम्}
{लक्ष्मणेन सह भ्रात्रा वैदेह्या च स्वभार्यया} %॥८९॥

\twolineshloka
{जनान्निवर्त्य रामोऽसौ जगाम च गुहाश्रमम्}
{गुहस्तु रामभक्तोऽसौ स्वभावादेव वैष्णवः} %॥९०॥

\twolineshloka
{कृताञ्जलिपुटो भूत्वा किं कर्तव्यमिति स्थितः}
{महता तपसाऽऽनीता गुरुणा या हि वः पुरा} %॥९१॥

\twolineshloka
{भगीरथेन या भूमिं सर्वपापहरा शुभा}
{नानामुनिजनैर्जुष्टा कूर्ममत्स्यसमाकुला} %॥९२॥

\twolineshloka
{गङ्गा तुङ्गोर्मिमालाढ्या स्फटिकाभजलावहा}
{गुहोपनीतनावा तु तां गङ्गां स महाद्युतिः} %॥९३॥

\twolineshloka
{उत्तीर्य भगवान् रामो भरद्वाजाश्रमं शुभम्}
{प्रयागे तु ततस्तस्मिन् स्त्रात्वा तीर्थे यथाविधि} %॥९४॥

\twolineshloka
{लक्ष्मणेन सह भ्रात्रा राघवः सीतया सह}
{भरद्वाजाश्रमे तत्र विश्रान्तस्तेन पूजितः} %॥९५॥

\twolineshloka
{ततः प्रभाते विमले तमनुज्ञाप्य राघवः}
{भरद्वाजोक्तमार्गेण चित्रकूटं शनैर्ययौ} %॥९६॥

\twolineshloka
{नानाद्रुमलताकीर्णं पुण्यतीर्थमनुत्तमम्}
{तापसं वेषमास्थाय जह्नुकन्यामतीत्य वै} %॥९७॥

\twolineshloka
{गते रामे सभार्ये तु सह भ्रात्रा ससारथौ}
{अयोध्यामवसन् भूप नष्टशोभां सुदुःखिताः} %॥९८॥

\twolineshloka
{नष्टसज्ञो दशरथः श्रुत्वा वचनमप्रियम्}
{रामप्रवासजननं कैकेय्या मुखनिस्सृतम्} %॥९९॥

\twolineshloka
{लब्धसज्ञः क्षणाद्राजा रामरामेति चुक्रुशे}
{कैकेय्युवाच भूपालं भरतं चाभिषेचय} %॥१००॥

\twolineshloka
{सीतालक्ष्मणसंयुक्तो रामचन्द्रो वनं गतः}
{पुत्रशोकाभिसन्तप्तो राजा दशरथस्तदा} %॥१०१॥

\twolineshloka
{विहाय देहं दुःखेन देवलोकं गतस्तदा}
{ततस्तस्य महापुर्य्यामयोध्यायामरिन्दम} %॥१०२॥

\twolineshloka
{रुरुदुर्दुःखशोकार्त्ता जनाः सर्वे च योषितः}
{कौशल्या च सुमित्रा च कैकेयी कष्टकारिणी} %॥१०३॥

\twolineshloka
{परिवार्य मृतं तत्र रुरुदुस्ताः पतिं ततः}
{ततः पुरोहितस्तत्र वसिष्ठः सर्वधर्मवित्} %॥१०४॥

\twolineshloka
{तैलद्रोण्यां विनिक्षिप्य मृतं राजकलेवरम्}
{दूत वैं प्रेषयामास सहमन्त्रिगणैः स्थितः} %॥१०५॥

\twolineshloka
{स गत्वा यत्र भरतः शत्रुघ्नेन सह स्थितः}
{तत्र प्राप्य तथा वार्तां सन्निवर्त्य नृपात्मजौ} %॥१०६॥

\twolineshloka
{तावानीय ततः शीघ्रमयोध्यां पुनरागतः}
{क्रूराणि दृष्ट्वा भरतो निमित्तानि च वै पथि} %॥१०७॥

\twolineshloka
{विपरीतं त्वयोध्यामिति मेने स पार्थिवः}
{निश्शोभां निर्गतश्रीकां दुःखशोकान्वितां पुरीम्} %॥१०८॥

\twolineshloka
{कैकेय्याग्निविनिर्दग्धामयोध्यां प्रविवेश सः}
{दुःखान्विता जनाः सर्वे तौ दृष्ट्वा रुरुदुर्भृशम्} %॥१०९॥

\twolineshloka
{हा तात राम हा सीते लक्ष्मणेति पुनः पुनः}
{रुरोद भरतस्तत्र शत्रुघ्नश्च सुदुःखितः} %॥११०॥

\twolineshloka
{कैकेय्यास्तत्क्षणाच्छुत्वा चुक्रोध भरतस्तदा}
{दुष्टा त्वं दुष्टचित्ता च यया रामः प्रवासितः} %॥१११॥

\twolineshloka
{लक्ष्मणेन सह भ्रात्रा राघवः सीतया वनम्}
{साहसं किं कृतं दुष्टे त्वया सद्यो‍ऽल्पभाग्यया} %॥११२॥

\twolineshloka
{उद्वास्य सीतया रामं लक्ष्मणेन महात्मना}
{ममैव पुत्रं राजानं करोत्विति मतिस्तव} %॥११३॥

\twolineshloka
{दुष्टाया नष्टभाग्यायाः पुत्रोऽहं भाग्यवर्जितः}
{भ्रात्रा रामेण रहितो नाहं राज्यं करोमि वै} %॥११४॥

\twolineshloka
{यत्र रामो नरव्याध्रः पद्यपत्रायतेक्षणः}
{धर्मज्ञः सर्वशास्त्राज्ञो मतिमान् बन्धुवत्सलः} %॥११५॥

\twolineshloka
{सीता च यत्र वैदेही नियमव्रतचारिणी}
{पतिव्रता महाभागा सर्वलक्षणसंयुता} %॥११६॥

\twolineshloka
{लक्ष्मणश्च महावीर्यो गुणवान् भ्रातृवत्सलः}
{तत्र यास्यामि कैकेयि महत्पापं त्वया कृतम्} %॥११७॥

\twolineshloka
{राम एव मम भ्राता ज्येष्ठो मतिमतां वरः}
{स एव राजा दुष्टात्मे भृत्यो‍ऽहं तस्य वै सदा} %॥११८॥

\twolineshloka
{इत्युक्त्वा मातरं तत्र रुरोद भृशदुःखितः}
{हा राजन् पृथिवीपाल मां विहाय सुदुःखितम्} %॥११९॥

\twolineshloka
{क्व गतोऽस्यद्य वै तात किं करोमीह तद्वद}
{भ्राता पित्रा समः क्वास्ते ज्येष्ठो मे करुणाकरः} %॥१२०॥

\twolineshloka
{सीता च मातृतुल्या मे क्व गतो लक्ष्मणश्चह}
{इत्येवं विलपन्तं तं भरतं मन्त्रिभिः सह} %॥१२१॥

\twolineshloka
{वसिष्ठो भगवानाह कालकर्मविभागवित्}
{उत्तिष्ठोत्तिष्ठ वत्स त्वं न शोकं कर्तुमर्हसि} %॥१२२॥

\twolineshloka
{कर्मकालवशादेव पिता ते स्वर्गमास्थितः}
{तस्य संस्कारकार्याणि कर्माणि कुरु शोभन} %॥१२३॥

\twolineshloka
{रामोऽपि दुष्टनाशाय शिष्टानां पालनाय च}
{अवतीर्णो जगत्स्वामी स्वांशेन भुवि माधवः} %॥१२४॥

\twolineshloka
{प्रायस्तत्रास्ति रामेण कर्तव्यं लक्ष्मणेन च}
{यत्रासौ भगवान् वीरः कर्मणा तेन चोदितः} %॥१२५॥

\twolineshloka
{तत्कृत्वा पुनरायाति रामः कमललोचनः}
{इत्युक्तो भरतस्तेन वसिष्ठेन महात्मना} %॥१२६॥

\twolineshloka
{संस्कारं लम्भयामास विधिदृष्टेन कर्मणा}
{अग्निहोत्राग्निना दग्ध्वा पितुर्देहं विधानतः} %॥१२७॥

\twolineshloka
{स्नात्वा सरय्वाः सलिले कृत्वा तस्योदकक्रियाम्}
{शत्रुघ्नेन सह श्रीमान्तातृभिर्बान्धवैः सह} %॥१२८॥

\twolineshloka
{तस्यौर्ध्वदेहिकं कृत्वा मन्त्रिणा मन्त्रिनायकः}
{हस्त्यश्वरथपत्तीभिः सह प्रायान्महामतिः} %॥१२९॥

\twolineshloka
{भरतो राममन्वेष्टुं राममार्गेण सत्तमः}
{तमायान्तं महासेनं रामस्यानुविरोधिनम्} %॥१३०॥

\twolineshloka
{मत्वा तं भरतं शत्रुं रामभक्तो गुहस्तदा}
{स्वं सैन्यं वर्तुलं कृत्वा सन्नद्धः कवची रथी} %॥१३१॥

\onelineshloka
{महाबलपरीवारो रुरोध भरतं पथि} %॥१३२॥

\twolineshloka
{सभ्रातृकं सभा र्यं मे राम स्वामिनमुत्तमम्}
{प्रापयस्त्वं वनं दुष्टं साम्प्रतं हन्तुमिच्छसि} %॥१३३॥

\twolineshloka
{गमिष्यसि दुरात्मंस्त्वं सेनया सह दुर्मते}
{इत्युक्तो भरतस्तत्र गुहेन नृपनन्दनः} %॥१३४॥

\twolineshloka
{तमुवाच विनीतात्मा रामायाथ कृताञ्जलिः}
{यथा त्वं रामभक्तोऽमि तथाहमपि भक्तिमान्} %॥१३५॥

\twolineshloka
{प्रोषिते मयि कैकेय्या कृतमेतन्महामते}
{रामस्यानयनार्थाय व्रजाम्यद्य महामते} %॥१३६॥

\twolineshloka
{सत्यपूर्वं गमिष्यामि पन्थानं देहि मे गुह}
{इति विश्वासमानीय जाह्नवीं तेन तारितः} %॥१३७॥

\twolineshloka
{नौकावृन्दैरनेकैस्तु स्त्रात्वासौ जाह्नवीजले}
{भरद्वाजाश्रमं प्राप्तो भरतस्तं महामुनिम्} %॥१३८॥

\twolineshloka
{प्रणम्य शिरसा तस्मै यथावृत्तमुवाच ह}
{भरद्वाजोऽपि तं प्राह कालेन कृतमीदृशम्} %॥१३९॥

\twolineshloka
{दुःखं न तावत् कर्तव्यं रामार्थेऽपि त्वयाधुना}
{वर्तते चित्रकूटेऽसौ रामः सत्यपराक्रमः} %॥१४०॥

\twolineshloka
{त्वयि तत्र गते वापि प्रायोऽसौ नागमिष्यति}
{तथापि तत्र गच्छ त्वं यदसौ वक्ति तत्कुरु} %॥१४१॥

\twolineshloka
{रामस्तु सीतया सार्धं वनखण्डे स्थितः शुभे}
{लक्ष्मणस्तु महावीर्यो दुष्टालोकनतत्परः} %॥१४२॥

\twolineshloka
{इत्युक्तो भरतस्तत्र भरद्वाजेन धीमता}
{उत्तीर्य यमुनां यातश्चित्रकूटं महानगम्} %॥१४३॥

\twolineshloka
{स्थितोऽसौ दृष्टवान्दूरात्सधूलीं चोत्तरां दिशम्}
{रामाय कथियित्वाऽऽस तदादेशात्तु लक्ष्मणः} %॥१४४॥

\twolineshloka
{वृक्षमारुह्य मेधावी वीक्षमाणः प्रयत्नतः}
{स ततो दृष्टवान् हष्टामायान्तीं महतीं चमूम्} %॥१४५॥

\twolineshloka
{हस्त्यश्वरथसंयुक्तां दृष्ट्वा राममथाब्रवीत्}
{हे भ्रातस्त्वं महाबाहो सीतापार्श्वे स्थिरो भव} %॥१४६॥

\twolineshloka
{भूपोऽस्ति बलवान् कश्चिद्धस्त्यश्वरथपत्तिभिः}
{इत्याकर्ण्य वचस्तस्य लक्ष्मणस्य महात्मनः} %॥१४७॥

\twolineshloka
{रामस्तब्रवीद्वीरो वीरं सत्यपराक्रमः}
{प्रायेण भरतोऽस्माकं द्रष्टुमायाति लक्ष्मण} %॥१४८॥

\twolineshloka
{इत्येवं वदतस्तस्य रामस्य विदितात्मनः}
{आरात्संस्थाप्य सेनां तां भरतो विनयान्वितः} %॥१४९॥

\twolineshloka
{ब्राह्मणैर्मन्त्रिभिः सार्धं रुदन्नागत्य पादयोः}
{रामस्य निपपाताथ वैदेह्या लक्ष्मणस्य च} %॥१५०॥

\twolineshloka
{मन्त्रिणो मातृवर्गश्च स्निग्धबन्धुसुहज्जनाः}
{परिवार्य ततो रामं रुरुदुः शोककातराः} %॥१५१॥

\twolineshloka
{स्वर्यातं पितरं ज्ञात्वा ततो रामो महामतिः}
{लक्ष्मणेन सह भ्रात्रा वैदोह्याथ समन्वितः} %॥१५२॥

\twolineshloka
{स्त्रात्वा मलापहे तीर्थे दत्त्वा च सलिलाञ्जलिम्}
{मात्रादीनभिवाद्याथ रामो दुःखसमन्वितः} %॥१५३॥

\twolineshloka
{उवाच भरतं राजन् दुःखेन महतान्वितम्}
{अयोध्यां गच्छ भरत इतः शीघ्रं महामते} %॥१५४॥

\twolineshloka
{राज्ञा विहीनां नगरीं अनाथां परिपालय}
{इत्युक्तो भरतः प्राह रामं राजीवलोलचनम्} %॥१५५॥

\twolineshloka
{त्वामृते पुरुषव्याघ्र न यास्येऽहमितो ध्रुवम्}
{यत्र त्वं तत्र यास्यामि वैदेही लक्ष्मणो यथा} %॥१५६॥

\twolineshloka
{इत्याकर्ण्य पुनः प्राह भरतं पुरतः स्थितम्}
{नृणां पितृसमो ज्येष्ठः स्वधर्ममनुवर्तिनाम्} %॥१५७॥

\twolineshloka
{यथा न लङ्ह्यं वचनं मया पितृमुखेरितम्}
{तथा त्वया न लङ्ह्यं स्याद्वचनं मम सत्तम} %॥१५८॥

\twolineshloka
{मत्समीपादितो गत्वा प्रजास्त्वं परिपालय}
{द्वादशाब्दिकमेतन्मे व्रतं पितृमुखेरितम्} %॥१५९॥

\twolineshloka
{तदरण्ये चरित्वा तु आगामिष्यामि तेऽन्तिकम्}
{गच्छ तिष्ठ ममादेशे न दुःखं कर्तुमर्हसि} %॥१६०॥

\twolineshloka
{इत्युक्तो भरतः प्राह बाष्पपर्याकुलेक्षणः}
{यथा पिता तथा त्वं मे नात्र कार्या विचारणा} %॥१६१॥

\twolineshloka
{तवादेशान्मया कार्यं देहि त्वं पादुके मम}
{नन्दिग्रामे वसिष्येऽहं पादुके द्वादशाब्दिकम्} %॥१६२॥

\twolineshloka
{त्वद्वेषमेव मद्वेषं त्वदव्रतं मे महाव्रतम्}
{त्वं द्वादशाब्दिकादूर्ध्वं यदि नायासि सत्तम} %॥१६३॥

\twolineshloka
{ततो हविर्यथा चाग्नौ प्रधक्ष्यामि कलेवरम्}
{इत्येवं शपथं कृत्वा भरतो हि सुदुःखितः} %॥१६४॥

\twolineshloka
{बहु प्रदक्षिणं कृत्वा नमस्कृत्य च राघवम्}
{पादुके शिरसा स्थाप्य भरतः प्रस्थितः शनैः} %॥१६५॥

\twolineshloka
{स कुर्वन् भ्रातुरादेशं नन्दिग्रामे स्थितो वशी}
{तपस्वी नियताहार: शाकमूलफलाशनः} %॥१६६॥

\fourlineindentedshloka
{जटाकलापं शिरसा च बिभ्रत्}
{त्वचश्च वाक्षीः किल वन्यभोजी}
{रामस्य वाक्यादरतो हदि स्थितं}
{बभार भूभारमनिन्दितात्मा} %॥१६७॥

॥इति श्रीनरसिंहपुराणे रामप्रादुर्भावे अष्टचत्वारिंशोऽध्यायः॥४८॥

\sect{एकोनपञ्चाशोऽध्यायः --- अरण्य-काण्डः}

\uvacha{मार्कण्डेय उवाच}

\twolineshloka
{गतेऽथ भरते तस्मिन् रामः कमललोचनः}
{लक्ष्मणेन सह भ्रात्रा भार्यया सीतया सह} %॥१॥

\twolineshloka
{शाकमूलफलाहारो विचचार महावने}
{कदाचिल्लक्ष्मणमृते रामदेवः प्रतापवान्} %॥२॥

\twolineshloka
{चित्रकूटवनोद्देशे वैदेह्युत्सङ्गमाश्रितः}
{सुष्वाप स मुहूर्तं तु ततः काको दुरात्मवान्} %॥३॥

\twolineshloka
{सीताभिमुखमभ्येत्य विददार स्तनान्तरम्}
{विदार्य वृक्षमारुह्य स्थितोऽसौ वायसाधमः} %॥४॥

\twolineshloka
{ततः प्रबुद्धो रामोऽसौ दृष्ट्वा रक्तं स्तनान्तरे}
{शोकाविष्टां तु सीतां तामुवाच कमलेक्षणः} %॥५॥

\twolineshloka
{वद स्तनान्तरे भद्रे तव रक्तस्य कारणम्}
{इत्युक्ता सा च तं प्राह भर्तारं विनयान्विता} %॥६॥

\twolineshloka
{पश्य राजेन्द्र वृक्षाग्रे वायसं दुष्टचेष्टितम्}
{अनेनैव कृतं कर्म सुप्ते त्वयि महामते} %॥७॥

\twolineshloka
{रामोऽपि दृष्टवान् काकं तस्मिन् क्रोधमथाकरोत्}
{इषीकास्त्रं समाधाय ब्रह्मास्त्रेणाभिमन्त्रितम्} %॥८॥

\twolineshloka
{काकमुद्दिश्य चिक्षेप सोऽप्यधावद्भयान्वितः}
{स त्विन्द्रस्य सुतो राजन्निन्द्रलोकं विवेश ह} %॥९॥

\twolineshloka
{रामास्त्रं प्रज्वलद्दीप्तं तस्यानु प्रविवेश वै}
{विदितार्थश्च देवेन्द्रो देवैः सह समन्वितः} %॥१०॥

\twolineshloka
{निष्क्रामयच्च तं दुष्टं राघवस्यापकारिणम्}
{ततोऽसौ सर्वदेवैस्तु देवलोकाद्वहिः कृतः} %॥११॥

\twolineshloka
{पुनः सोऽप्यपतद्रामं राजानं शरणं गतः}
{पाहि राम महाबाहो अज्ञानादपकारिणम्} %॥१२॥

\twolineshloka
{इति ब्रुवन्तं तं प्राह रामः कमललोचनः}
{अमोघं च ममैवास्त्रमङ्गमेकं प्रयच्छ वै} %॥१३॥

\twolineshloka
{ततो जीवसि दुष्ट त्वमपकारो महान् कृतः}
{इत्युक्तोऽसौ स्वकं नेत्रमेकमस्त्राय दत्तवान्} %॥१४॥

\twolineshloka
{अस्त्रं तन्नेत्रमेकं तु भस्मीकृत्य समाययौ}
{ततः प्रभृति काकानां सर्वेषामेकनेत्रता} %॥१५॥

\twolineshloka
{चक्षुषैकेन पश्यन्ति हेतुना तेन पार्थिव}
{उषित्वा तत्र सुचिरं चित्रकूटे स राघवः} %॥१६॥

\twolineshloka
{जगाम दण्डकारण्यं नानामुनिनिषेवितम्}
{सभ्रातृकः सभार्यश्च तापसं वेषमास्थितः} %॥१७॥

\twolineshloka
{धनुः पर्वसुपाणिश्च सेषुधिश्च महाबलः}
{ततो ददर्श तत्रस्थानम्बुभक्षान्महामुनीन्} %॥१८॥

\twolineshloka
{अश्मकुट्टाननेकांश्च दन्तोलूखलिनस्तथा}
{पञ्चाग्निमध्यगानन्यानन्यानुग्रतपश्चरान्} %॥१९॥

\twolineshloka
{तान् दृष्ट्वा प्रणिपत्योच्चै रामस्तैश्चाभिनन्दितः}
{ततोऽखिलं वनं दृष्ट्वा रामः साक्षाज्जनार्दनः} %॥२०॥

\twolineshloka
{भ्रातृभार्यासहायश्च सम्प्रतस्थे महामतिः}
{दर्शयित्वा तु सीतायै वनं कुसुमितं शुभम्} %॥२१॥

\twolineshloka
{नानाश्चर्यसमायुक्तं शनैर्गच्छन् स दृष्टवान्}
{कृष्णाङ्गं रक्तनेत्रं तु स्थूलशैलसमानकम्} %॥२२॥

\twolineshloka
{शुभ्रदंष्ट्रं महाबाहुं सन्ध्याघनशिरोरुहम्}
{मेघस्वनं सापराधं शरं सन्धाय राघवः} %॥२३॥

\twolineshloka
{विव्याध राक्षसं क्रोधाल्लक्ष्मणेन सह प्रभुः}
{अन्यैरवध्यं हत्वा तं गिरिगर्ते महातनुम्} %॥२४॥

\twolineshloka
{शिलाभिश्छाद्य गतवाज्शरभङ्गाश्रमं ततः}
{तं नत्वा तत्र विश्रम्य तत्कथातुष्टमानसः} %॥२५॥

\twolineshloka
{तीक्ष्णाश्रममुपागम्य दुष्टवांस्तं महामुनिम्}
{तेनादिष्टेन मार्गेण गत्वागस्त्यं ददर्श ह} %॥२६॥

\twolineshloka
{खङ्गं तु विमलं तस्मादवाप रघुनन्दनः}
{इषुधिं चाक्षयशरं चापं चैव तु वैष्णवम्} %॥२७॥

\twolineshloka
{ततोऽगस्त्याश्रमाद्रामो भ्रातृभार्यासमन्वितः}
{गोदावर्याः समीपे तु पञ्चवट्यामुवास सः} %॥२८॥

\twolineshloka
{ततो जटायुरभ्येत्य रामं कमललोचनम्}
{नत्वा स्वकुलमाख्याय स्थितवान् गृध्रनायकः} %॥२९॥

\twolineshloka
{रामोऽपि तत्र तं दृष्ट्वा आत्मवृत्तं विशेषतः}
{कथयित्वा तु तं प्राह सीतां रक्ष महामते} %॥३०॥

\twolineshloka
{इत्युक्तोऽसौ जतायुस्तु राममालिङग्य सादरम्}
{कार्यार्थं तु गते रामे भ्रात्रा सह वनान्तरम्} %॥३१॥

\twolineshloka
{अहं रक्ष्यामि ते भार्यां स्थीयतामत्र शोभन}
{इत्युक्त्वा गतवान्रामं गृध्रराजः स्वमाश्रमम्} %॥३२॥

\twolineshloka
{समीपे दक्षिणे भागे नानापक्षिनिषेविते}
{वसन्तं राघवं तत्र सीतया सह सुन्दरम्} %॥३३॥

\twolineshloka
{मन्मथाकारसदृशं कथयन्तं महाकथाः}
{कृत्वा मायामयं रुपं लावण्यगुणसंयुतम्} %॥३४॥

\twolineshloka
{मदनाक्रान्तहदया कदाचिद्रावणानुजा}
{गायन्ती सुस्वरं गीतं शनैरागत्य राक्षसी} %॥३५॥

\twolineshloka
{ददर्श राममासीनं कानने सीतया सह}
{अथ शूर्पणखा घोरा मायारुपधरा शुभा} %॥३६॥

\twolineshloka
{निश्शङ्का दुष्टचित्ता सा राघवं प्रत्यभाषत}
{भज मां कान्त कल्याणीं भजन्तीं कामिनीमिह} %॥३७॥

\twolineshloka
{भजमानां त्यजेद्यस्तु तस्य दोषो महान् भवेत्}
{इत्युक्तः शूर्पणखया रामस्तामाह पार्थिवः} %॥३८॥

\twolineshloka
{कलत्रवानहं बाले कनीयांसं भजस्व मे}
{इति श्रुत्वा ततः प्राह राक्षसी कामरुपिणी} %॥३९॥

\twolineshloka
{अतीव निपुणा चाहं रतिकर्मणि राघव}
{त्यक्त्वैनामनभिज्ञां त्वं सीतां मां भज शोभनाम्} %॥४०॥

\twolineshloka
{इत्याकर्ण्य वचः प्राह रामस्तां धर्मतत्परः}
{परस्त्रियं न गच्छेऽहं त्वमितो गच्छ लक्ष्मणम्} %॥४१॥

\twolineshloka
{तस्य नात्र वने भार्या त्वामसौ सग्रहीष्यति}
{इत्युक्ता सा पुनः प्राह रामं राजीवलोचनम्} %॥४२॥

\twolineshloka
{यथा स्याल्लक्ष्मणो भर्ता तथा त्वं देहि पत्रकम्}
{तथैवमुक्त्वा मतिमान् रामः कमललोचनः} %॥४३॥

\twolineshloka
{छिन्ध्यस्या नासिकामिति मोक्तव्या नात्र संशयः}
{इति रामो महाराजो लिख्य पत्रं प्रदत्तवान्} %॥४४॥

\twolineshloka
{सा गृहीत्वा तु तत्पत्रं गत्वा तस्मान्मुदान्विता}
{गत्वा दत्तवती तद्वल्लक्ष्मणाय महात्मने} %॥४५॥

\twolineshloka
{तां दृष्ट्वा लक्ष्मणः प्राह राक्षसीं कामरुपिणीम्}
{न लङ्घ्यं राघववचो मया तिष्ठात्मकश्मले} %॥४६॥

\twolineshloka
{तां प्रगृह्य ततः खङ्गमुद्यम्य विमलं सुधीः}
{तेन तत्कर्णनासां तु चिच्छेद तिलकाण्डवत्} %॥४७॥

\twolineshloka
{छिन्ननासा ततः सा तु रुरोद भृशदुः खिता}
{हा दशास्य मम भ्रातः सर्वदेवविमर्दक} %॥४८॥

\twolineshloka
{हा कष्टं कुम्भकर्णाद्यायाता मे चापदा परा}
{हा हा कष्टं गुणनिधे विभीषण महामते} %॥४९॥

\twolineshloka
{इत्येवमार्ता रुदती सा गत्वा खरदूषणौ}
{त्रिशिरसं च सा दृष्ट्वा निवेद्यात्मपराभवम्} %॥५०॥

\twolineshloka
{राममाह जनस्थाने भ्रात्रा सह महाबलम्}
{ज्ञात्वा ते राघवं क्रुद्धाः प्रेषयामासुरुर्जितान्} %॥५१॥

\twolineshloka
{चतुर्दशसहस्त्राणि राक्षसानां बलीयसाम्}
{अग्रे निजग्मुस्तेनैव रक्षसां नायकास्त्रयः} %॥५२॥

\twolineshloka
{रावणेन नियुक्तास्ते पुरैव तु महाबलाः}
{महाबलपरीवारा जनस्थानमुपागताः} %॥५३॥

\twolineshloka
{क्रोधेन महताऽऽविष्टा दृष्ट्वा तां छिन्ननासिकाम्}
{रुदतीमश्रुदिग्धाङ्गीं भगिनीं रावणस्य तु} %॥५४॥

\twolineshloka
{रामोऽपि तद्वलं दृष्ट्वा राक्षसानां बलीयसाम्}
{संस्थाप्य लक्ष्मणं तत्र सीताया रक्षणं प्रति} %॥५५॥

\twolineshloka
{गत्वा तु प्रहितैस्तत्र राक्षसैर्बलदर्पितैः}
{चतुर्दशसहस्त्रं तु राक्षसानां महाबलम्} %॥५६॥

\twolineshloka
{क्षणेन निहतं तेन शरैरग्निशिखोपमैः}
{खरश्च निहतस्तेन दूषणश्च महाबलः} %॥५७॥

\twolineshloka
{त्रिशिराश्च महारोषाद रणे रामेण पातितः}
{हत्वा तान् राक्षसान् दुष्टान् रामश्चाश्रममाविशत्} %॥५८॥

\twolineshloka
{शूर्पणखा च रुदती रावणान्तिकमागता}
{छिन्ननासां च तां दृष्ट्वा रावणो भगिनीं तदा} %॥५९॥

\twolineshloka
{मारीचं प्राह दुर्बुद्धिः सीताहरणकर्मणि}
{पुष्पकेण विमानेन गत्वाहं त्वं च मातुल} %॥६०॥

\twolineshloka
{जनस्थानसमीपे तु स्थित्वा तत्र ममाज्ञया}
{सौवर्णमृगरुपं त्वमास्थाय तु शनैः शनैः} %॥६१॥

\twolineshloka
{गच्छ त्वं तत्र कार्यार्थं यत्र सीता व्यवस्थिता}
{दृष्ट्वा सा मृगपोतं त्वां सौवर्णं त्वयि मातुल} %॥६२॥

\twolineshloka
{स्पृहां करिष्यते रामं प्रेषयिष्यति बन्धने}
{तद्वाक्यात्तत्र गच्छन्तं धावस्व गहने वने} %॥६३॥

\twolineshloka
{लक्ष्मणस्यापकर्षार्थं वक्तव्यं वागुदीरणम्}
{ततः पुष्पकमारुह्य मायारुपेण चाप्यहम्} %॥६४॥

\twolineshloka
{तां सीतामहमानेष्ये तस्यामासक्तमानसः}
{त्वमपि स्वेच्छया पश्चादागमिष्यसि शोभन} %॥६५॥

\twolineshloka
{इत्युक्ते रावणेनाथ मारीचो वाक्यमब्रवीत्}
{त्वमेव गच्छ पापिष्ठ नाहं गच्छामि तत्र वै} %॥६६॥

\twolineshloka
{पुरैवानेन रामेण व्यथितोऽहं मुनेर्मखे}
{इत्युक्तवति मारीचे रावणः क्रोधमूर्च्छितः} %॥६७॥

\twolineshloka
{मारीचं हन्तुमारेभे मारीचोऽप्याह रावणम्}
{तव हस्तवधाद्वीर रामेण मरणं वरम्} %॥६८॥

\twolineshloka
{अहं गमिष्यामि तत्र यत्र त्वं नेतुमिच्छसि}
{अथ पुष्पकमारुह्य जनस्थानमुपागतः} %॥६९॥

\twolineshloka
{मारीचस्तत्र सौवर्णं मृगमास्थाय चाग्रतः}
{जगाम यत्र सा सीता वर्तते जनकात्मजा} %॥७०॥

\twolineshloka
{सौवर्णं मृगपोतं तु दृष्ट्वा सीता यशस्विनी}
{भाविकर्मवशाद्रामुवाच पतिमात्मनः} %॥७१॥

\twolineshloka
{गृहीत्वा देहि सौवर्णं मृगपोतं नृपात्मज}
{अयोध्यायां तु मद्रेहे क्रीडनार्थमिदं मम} %॥७२॥

\twolineshloka
{तयैवमुक्तो रामस्तु लक्ष्मणं स्थाप्य तत्र वै}
{रक्षणार्थ तु सीताया गतोऽसौ मृगपृष्ठतः} %॥७३॥

\twolineshloka
{रामेण चानुयातोऽसौ अभ्यधावद्वने मृगः}
{ततः शरेण विव्याध रामस्तं मृगपोतकम्} %॥७४॥

\twolineshloka
{हा लक्ष्मणेति चोक्त्वासौ निपपात महीतले}
{मारीचः पर्वताकारस्तेन नष्टो बभूव सः} %॥७५॥

\twolineshloka
{आकर्ण्य रुदतः शब्दं सीता लक्ष्मणमब्रवीत्}
{गच्छ लक्ष्मण पुत्र त्वं यत्रायं शब्द उत्थितः} %॥७६॥

\twolineshloka
{भ्रातुर्ज्येष्ठस्य तत्त्वं वै रुदतः श्रूयते ध्वनिः}
{प्रायो रामस्य सन्देहं लक्षयेऽहं महात्मनः} %॥७७॥

\twolineshloka
{इत्युक्तः स तथा प्राह लक्ष्मणस्तामनिन्दिताम्}
{न हि रामस्य सन्देहो न भयं विद्यते क्वचित्} %॥७८॥

\twolineshloka
{इति ब्रुवाणं तं सीता भाविकर्मबलाद्भृतम्}
{लक्ष्मणं प्राह वैदेही विरुद्धवचनं तदा} %॥७९॥

\twolineshloka
{मृते रामे तु मामिच्छन्नतस्त्वं न गामिष्यसि}
{इत्युक्तः स विनीतात्म असहन्नप्रियं वचः} %॥८०॥

\twolineshloka
{जगाम राममन्वेष्टुं तदा पार्थिवनन्दनः}
{सन्न्यासवेषमास्थाय रावणोऽपि दुरात्मवान्} %॥८१॥

\twolineshloka
{स सीतापार्श्वमासाद्य वचनं चेदमुक्तवान्}
{आगतो भरतः श्रीमानयोध्याया महामतिः} %॥८२॥

\twolineshloka
{रामेण सह सम्भाष्य स्थितवांस्तत्र कानने}
{मां च प्रेषितवान् रामो विमानमिदमारुह} %॥८३॥

\twolineshloka
{अयोध्यां याति रामस्तु भरतेन प्रसादितः}
{मृगबालं तु वैदेहि क्रीडार्थं ते गृहीतवान्} %॥८४॥

\twolineshloka
{क्लेशितासि महारण्ये बहुकालं त्वमीदृशम्}
{सम्प्राप्तराज्यस्ते भर्ता रामः स रुचिराननः} %॥८५॥

\twolineshloka
{लक्ष्मणश्च विनीतात्मा विमानमिदमारुह}
{इत्युक्ता सा तथा गत्वा नीता तेन महात्मना} %॥८६॥

\twolineshloka
{आरुरोह विमानं तु छद्मना प्रेरिता सती}
{तज्जगाम ततः शीघ्रं विमानं दक्षिणां दिशम्} %॥८७॥

\twolineshloka
{ततः सीता सुदुःखार्ता विललाप सुदुःखिता}
{विमाने खेऽपि रोदन्त्याश्चक्रे स्पर्शं न राक्षसः} %॥८८॥

\twolineshloka
{रावणः स्वेन रुपेण बभूवाथ महातनुः}
{दशग्रीवं महाकायं दृष्ट्वा सीता सुदुःखिता} %॥८९॥

\twolineshloka
{हा राम वञ्चिताद्याहं केनापिच्छद्मरुपिणा}
{रक्षसा घोररुपेण त्रायस्वेति भयार्दिता} %॥९०॥

\twolineshloka
{हे लक्ष्मण महाबाहो मां हि दुष्टेन रक्षसा}
{द्रुतमागत्य रक्षस्व नीयमानामथाकुलाम्} %॥९१॥

\twolineshloka
{एवं प्रलपमानायाः सीतायास्तन्महस्त्स्वनम्}
{आकर्ण्य गृध्रराजस्तु जटायुस्तत्र चागतः} %॥९२॥

\twolineshloka
{तिष्ठ रावण दुष्टात्मन मुञ्च मुञ्चात्र मैथिलीम्}
{इत्युक्त्वा युयुधे तेन जटायुस्तत्र वीर्यवान्} %॥९३॥

\twolineshloka
{पक्षाभ्यां ताडयामास जटायुस्तस्य वक्षसि}
{ताडयन्तं तु तं मत्वा बलवानिति रावणः} %॥९४॥

\twolineshloka
{तुण्डचञ्जुप्रहारैस्तु भृशं तेन प्रपीडितः}
{तत उत्थाप्य वेगेन चन्द्रहासमसिं महत्} %॥९५॥

\twolineshloka
{जघान तेन दुष्टात्मा जटायुं धर्मचारिणम्}
{निपपात महीपृष्ठे जटायुः क्षीणचेतनः} %॥९६॥

\twolineshloka
{उवाच च दशग्रीवं दुष्टात्मन् न त्वया हतः}
{चन्द्रहासस्य वीर्येण हतोऽहं राक्षसाधम} %॥९७॥

\twolineshloka
{निरायुधं को हनेन्मूढ सायुधस्त्वामृते जनः}
{सीतापहरणं विद्धि मृत्युस्ते दुष्ट राक्षस} %॥९८॥

\twolineshloka
{दुष्ट रावण रामस्त्वां वधिष्यति न संशयः}
{रुदती दुःखशोकार्ता जटायुं प्राह मैथिली} %॥९९॥

\twolineshloka
{मत्कृते मरणं यस्मात्त्वया प्राप्तं द्विजोत्तम}
{तस्माद्रामप्रसादेन विष्णुलोकमवाप्स्यसि} %॥१००॥

\twolineshloka
{यावद्रामेण सङ्गस्ते भविष्यति महाद्विज}
{तावत्तिष्ठन्तु ते प्राणा इत्युक्त्वा तु खगोत्तमम्} %॥१०१॥

\twolineshloka
{ततस्तान्यर्पितान्यङ्गाद्भूषणानि विमुच्य सा}
{शीघ्रं निबध्य वस्त्रेण रामहस्तं गमिष्यथ} %॥१०२॥

\twolineshloka
{इत्युक्त्वा पातयामास भूमौ सीता सुदुःखिता}
{एवं हत्वा स सीतां तु जटायुं पात्य भूतले} %॥१०३॥

\twolineshloka
{पुष्पकेण गतः शीघ्रं लङ्कां दुष्टनिशाचरः}
{अशोकवनिकामध्ये स्थापयित्वा स मैथिलीम्} %॥१०४॥

\twolineshloka
{इमामत्रैव रक्षध्वं राक्षस्यो विकृताननाः}
{इत्यादिश्य गृहं यातो रावणो राक्षसेश्वरः} %॥१०५॥

\twolineshloka
{लङ्कानिवासिनश्चोचुरेकान्तं च परस्परम्}
{अस्याः पुर्या विनाशार्थं स्थापितेयं दुरात्मना} %॥१०६॥

\twolineshloka
{राक्षसीभिर्विरुपाभी रक्ष्यमाणा समन्ततः}
{सीता च दुःखिता तत्र स्मरन्ती राममेव सा} %॥१०७॥

\twolineshloka
{उवास सा सुदुःखार्ता दुःखिता रुदती भृशम्}
{यथा ज्ञानखले देवी हंसयाना सरस्वती} %॥१०८॥

\twolineshloka
{सुग्रीवभृत्या हरयश्चतुरश्च यदृच्छया}
{वस्त्रबद्धं तयोत्सृष्टं गृहीत्वा भूषणं द्रुतम्} %॥१०९॥

\twolineshloka
{स्वभर्त्रे विनिवेद्योचुः सुग्रीवाय महात्मने}
{अरण्येऽभून्महायुद्धं जटायो रावणस्य च} %॥११०॥

\twolineshloka
{अथ रामश्च तं हत्वा मारीचं माययाऽऽगतम्}
{निवृत्तो लक्ष्मणं दृष्ट्वा तेन गत्वा स्वमाश्रमम्} %॥१११॥

\twolineshloka
{सीतामपश्यन्दुः खार्तः प्ररुरोद स राघवः}
{लक्ष्मणश्च महातेजा रुरोद भृशदुःखितः} %॥११२॥

\twolineshloka
{बहुप्रकारमस्वस्थं रुदन्तं राघवं तदा}
{भूतले पतितं धीमानुत्थाप्याश्वास्य लक्ष्मणः} %॥११३॥

\twolineshloka
{उवाच वचनं प्राप्तं तदा यत्तच्छृणुष्व मे}
{अतिवेलं महाराज न शोकं कर्तुमर्हसि} %॥११४॥

\twolineshloka
{उत्तिष्ठोत्तिष्ठ शीघ्रं त्वं सीतां मृगयितुं प्रभो}
{इत्येवं वदता तेन लक्ष्मणेन महात्मना} %॥११५॥

\twolineshloka
{उत्थापितो नरपतिर्दुःखितो दुःखितेन तु}
{भ्रात्रा सह जगामाथ सीतां मृगयितुं वनम्} %॥११६॥

\fourlineindentedshloka
{वनानि सर्वाणि विशोध्य राघवो}
{गिरीन् समस्तान् गिरिसानुगोचरान्}
{तथा मुनीनामपि चाश्रमान् बहूंस्-}
{तृणादिवल्लीगहनेषु भूमिषु} %॥११७॥

\fourlineindentedshloka
{नदीतटे भूविवरे गुहायां}
{निरीक्षमाणोऽपि महानुभावः}
{प्रियामपश्यन् भृशदुःखितस्तदा}
{जटायुषं वीक्ष्य च घातितं नृपः} %॥११८॥

\fourlineindentedshloka
{अहो भवान् केन हतस्त्वमीदृशीं}
{दशामवाप्तोऽसि मृतोऽसि जीवसि}
{ममाद्य सर्वं समदुःखितस्य भोः}
{पत्नीवियोगादिह चागतस्य वै} %॥११९॥

\fourlineindentedshloka
{इत्युक्तमात्रे विहगोऽथ कृच्छ्रा-}
{दुवाच वाचं मधुरां तदानीम्}
{श्रृणुष्व राजन् मम वृत्तमत्र}
{वदामि दृष्टं च कृतं च सद्यः} %॥१२०॥

\fourlineindentedshloka
{दशाननस्तामपनीय मायया}
{सीतां समारोप्य विमानमुत्तमम्}
{जगाम खे दक्षिणदिङ्मुखोऽसौ}
{सीता च माता विललाप दुःखिता} %॥१२१॥

\fourlineindentedshloka
{आकर्ण्य सीतास्वनमागतोऽहं}
{सीतां विमोक्तुं स्वबलेन राघव}
{युद्धं च तेनाहमतीव कृत्वा}
{हतः पुनः खङ्गबलेन रक्षसा} %॥१२२॥

\fourlineindentedshloka
{वैदेहिवाक्यादिह जीवता मया}
{दृष्टो भवान् स्वर्गमितो गमिष्ये}
{मा राम शोकं कुरु भूमिपाल}
{जह्यद्य दुष्टं सगणं तु नैऋतम्} %॥१२३॥

\twolineshloka
{रामो जटायुषेत्युक्तः पुनस्तं चाह शोकतः}
{स्वस्त्यस्तु ते द्विजवर गतिस्तु परमास्तु ते} %॥१२४॥

\twolineshloka
{ततो जटायुः स्वं देहं विहाय गतवान्दिवन्}
{विमानेन तु रम्येण सेव्यमानोऽप्सरोगणैः} %॥१२५॥

\twolineshloka
{रामोऽपि दग्ध्वा तद्देहं स्नातो दत्त्वा जलाञ्जलिम्}
{भ्रात्रा सगच्छन् दुःखार्तो राक्षसी पथि दृष्टवान्} %॥१२६॥

\twolineshloka
{उद्वमन्तीं महोल्काभां विवृतास्यां भयकरीम्}
{क्षयं नयन्तीं जन्तून् वै पातयित्वा गतो रुषा} %॥१२७॥

\twolineshloka
{गच्छन् वनान्तरं रामः स कबन्धं ददर्श ह}
{विरुपं जठरमुखं दीर्घबाहुं घनस्तनम्} %॥१२८॥

\twolineshloka
{रुन्धानं राममार्गं तु दृष्ट्वा तं दग्धवाज्शनैः}
{दग्धोऽसौ दिव्यरुपी तु खस्थो राममभाषत} %॥१२९॥

\twolineshloka
{राम राम महाबाहो त्वया मम महामते}
{विरुपं नाशितं वीर मुनिशापाच्चिरागतम्} %॥१३०॥

\twolineshloka
{त्रिदिवं यामि धन्योऽस्मि त्वत्प्रसादान्न संशयः}
{त्वं सीताप्राप्तये सख्यं कुरु सूर्यसुतेन भोः} %॥१३१॥

\twolineshloka
{वानरेन्द्रेण गत्वा तु सुग्रीवे स्वं निवेद्य वै}
{भविष्यति नृपश्रेष्ठ ऋष्यमूकगिरि व्रज} %॥१३२॥

\twolineshloka
{इत्युक्त्वा तु गते तस्मिन् रामो लक्ष्मणसंयुतः}
{सिद्धैस्तु मुनिभिः शून्यमाश्रमं प्रविवेश ह} %॥१३३॥

\twolineshloka
{तत्रस्थां तापसीं दृष्ट्वा तया संलाप्य संस्थितः}
{शबरीं मुनिमुख्यानां सपर्याहतकल्मषाम्} %॥१३४॥

\twolineshloka
{तया सम्पूजितो रामो बदरादिभिरीश्वरः}
{साप्येनं पूजयित्वा तु स्वामवस्थां निवेद्य वै} %॥१३५॥

\twolineshloka
{सीतां त्वं प्राप्स्यसीत्युक्त्वा प्रविश्याग्निं दिवगता}
{दिवं प्रस्थाप्य तां चापि जगामान्यत्र राघवः} %॥१३६॥

\fourlineindentedshloka
{ततो विनीतेन गुणान्वितेन}
{भ्रात्रा समेतो जगदेकनाथः}
{प्रियावियोगेन सुदुःखितात्मा}
{जगाम याम्यां स तु रामदेवः} %॥१३७॥

॥इति श्रीनरसिंहपुराणे रामप्रादुर्भावे एकोनपञ्चाशोऽध्यायः॥४९॥

\sect{पञ्चाशत्तमोऽध्यायः --- किष्किन्धा-काण्डः}

\uvacha{मार्कण्डेय उवाच}

\twolineshloka
{बालिना कृतवैरोऽथ दुर्गवर्ती हरीश्वरः}
{सुग्रीवो दृष्टवान् दूराद् दृष्ट्वाऽऽह पवनात्मजम्} %॥१॥

\twolineshloka
{कस्येमौ सुधनुः पाणी चीरवल्कलधारिणौ}
{पश्यन्तौ सरसीं दिव्यां पद्मोत्पलसमावृताम्} %॥२॥

\twolineshloka
{नानारुपधरावेतौ तापसं वेषमास्थितौ}
{बालिदूताविह प्राप्ताविति निश्चित्य सूर्यजः} %॥३॥

\twolineshloka
{उत्पपात भयत्रस्तः ऋष्यमूकाद् वनान्तरम्}
{वानरैः सहितः सर्वैरगस्त्यश्रममुत्तमम्} %॥४॥

\twolineshloka
{तत्र स्थित्वा स सुग्रीवः प्राह वायुसुतं पुनः}
{हनूमन् पृच्छ शीघ्रं त्वं गच्छ तापसवेषधृक्} %॥५॥

\twolineshloka
{कौ हि कस्य सुतौ जातौ किमर्थं तत्र संस्थितौ}
{ज्ञात्वा सत्यं मम ब्रूहि वायुपुत्र महामते} %॥६॥

\twolineshloka
{इत्युक्तो हनुमान् गत्वा पम्पातटमनुत्तमम्}
{भिक्षुरुपी स तं प्राह रामं भ्रात्रा समन्वितम्} %॥७॥

\twolineshloka
{को भवानिह सम्प्राप्तस्तथ्यं ब्रूहि महामते}
{अरण्ये निर्जने घोरे कुतस्त्वं किं प्रयोजनम्} %॥८॥

\twolineshloka
{एवं वदन्तं तं प्राह लक्ष्मणो भ्रातुराज्ञया}
{प्रवक्ष्यामि निबोध त्वं रामवृत्तान्तमादितः} %॥९॥

\twolineshloka
{राजा दशरथो नाम बभूव भुवि विश्रुतः}
{तस्य पुत्रो महाबुद्धे रामो ज्येष्ठो ममाग्रजः} %॥१०॥

\twolineshloka
{अस्याभिषेक आरब्धः कैकेय्या तु निवारितः}
{पितुराज्ञामयं कुर्वन् रामो भ्राता ममाग्रजः} %॥११॥

\twolineshloka
{मया सह विनिष्क्रम्य सीतया सह भार्यया}
{प्रविष्टो दण्डकारण्यं नानामुनिसमाकुलम्} %॥१२॥

\twolineshloka
{जनस्थाने निवसतो रामस्यास्य महात्मनः}
{भार्या सीता तत्र वने केनापि पाप्पना हता} %॥१३॥

\twolineshloka
{सीतामन्वेषयन् वीरो रामः कमललोचनः}
{इहायातस्त्वया दृष्ट इति वृत्तान्तमीरितम्} %॥१४॥

\twolineshloka
{श्रुत्वा ततो वचस्तस्य लक्ष्मणस्य महात्मनः}
{अव्याञ्जितात्मा विश्वासाद्धनूमान् मारुतात्मजः} %॥१५॥

\twolineshloka
{त्वं मे स्वामी इति वदन् रामं रघुपतिं तदा}
{आश्वास्यानीय सुग्रीवं तयोः सख्यमकारयत्} %॥१६॥

\twolineshloka
{शिरस्यारोप्य पादाब्जं रामस्य विदितात्मनः}
{सुग्रीवो वानरेन्द्रस्तु उवाच मधुराक्षरम्} %॥१७॥

\twolineshloka
{अद्यप्रभृति राजेन्द्र त्वं मे स्वामी न संशयः}
{अहं तु तव भृत्यश्च वानरैः सहितः प्रभो} %॥१८॥

\twolineshloka
{त्वच्छत्रुर्मम शत्रुः स्यादद्यप्रभृति राघव}
{मित्रं ते मम सन्मित्रं त्वददुःखं तन्ममापि च} %॥१९॥

\twolineshloka
{त्वत्प्रीतिरेव मत्प्रीतिरित्युक्त्वा पुनराह तम्}
{वाली नाम मम ज्येष्ठो महाबलपराक्रमः} %॥२०॥

\twolineshloka
{भार्यापहारी दुष्टात्मा मदनासक्तमानसः}
{त्वामृते पुरुषव्याघ्र नास्ति हन्ताद्य वालिनम्} %॥२१॥

\twolineshloka
{युगपत्सप्ततालांस्तु तरुन् यो वै वधिष्यति}
{स तं वधिष्यतीत्युक्तं पुराणज्ञैर्नृपात्मज} %॥२२॥

\twolineshloka
{तत्प्रियार्थं हि रामोऽपि श्रीमांश्छित्त्वा महातरुन्}
{अर्धाकृष्टेन बाणेन युगप्रदघुनन्दनः} %॥२३॥

\twolineshloka
{विदध्वा महातरुन् रामः सुग्रीवं प्राह पार्थिवम्}
{वालिना गच्छ युध्यस्व कृतचिह्नो रवेः सुत} %॥२४॥

\twolineshloka
{इत्युक्तः कृतचिह्नोऽयं युद्धं चक्रेऽथ वालिना}
{रामोऽपि तत्र गत्वाथ शरेणैकेन वालिनम्} %॥२५॥

\twolineshloka
{विव्याध वीर्यवान् वाली पपात च ममार च}
{वित्रस्तं वालिपुत्रं तु अङ्गदं विनयान्वितम्} %॥२६॥

\twolineshloka
{रणशौण्डं यौवराज्ये नियुक्त्वा राघवस्तदा}
{तां च तारां तथा दत्त्वा रामश्च रविसूनवे} %॥२७॥

\twolineshloka
{सुग्रीवं प्राहं धर्मात्मा रामः कमललोचनः}
{राज्यमन्वेषय स्वं त्वं कपीनां पुनराव्रज} %॥२८॥

\twolineshloka
{त्वं सीतान्वेषणे यत्नं कुरु शीघ्रं हरीश्वर}
{इत्युक्तः प्राह सुग्रीवो रामं लक्ष्मणसंयुतम्} %॥२९॥

\twolineshloka
{प्रावृट्कालो महान् प्राप्तः साम्प्रतं रघुनन्दन}
{वानराणां गतिर्नास्ति वने वर्षति वासवे} %॥३०॥

\twolineshloka
{गते तमिंस्तु राजेन्द्र प्राप्ते शरदि निर्मले}
{चारान् सम्प्रेषयिष्यामि वानरान्दिक्षु राघव} %॥३१॥

\twolineshloka
{इत्युक्त्वा रामचन्द्रं स तं प्रणम्य कपीश्वरः}
{पम्पापुरं प्रविश्याथ रेमे तारासमन्वितः} %॥३२॥

\twolineshloka
{रामोऽपि विधिवदभ्रात्रा शैलसानौ महावने}
{निवासं कृतवान् शैले नीलकण्ठे महामतिः} %॥३३॥

\twolineshloka
{प्रावृट्काले गते कृच्छ्रात् प्राप्ते शरदि राघवः}
{सीतावियोगाद्व्यथितः सौमित्रिं प्राह लक्ष्मणम्} %॥३४॥

\twolineshloka
{उल्लङ्घितस्तु समयः सुग्रीवेण ततो रुषा}
{लक्ष्मणं प्राह काकुत्स्थो भ्रातरं भ्रातृवत्सलः} %॥३५॥

\twolineshloka
{गच्छ लक्ष्मण दुष्टोऽसौ नागतः कपिनायकः}
{गते तु वर्षाकालेऽहमागमिष्यामि तेऽन्तिकम्} %॥३६॥

\twolineshloka
{अनेकैर्वानरैः सार्धमित्युक्त्वासौ तदा गतः}
{तत्र गच्छ त्वरा युक्तो यत्रास्ते कपिनायकः} %॥३७॥

\twolineshloka
{तं दुष्टमग्रतः कृत्वा हरिसेनासमन्वितम्}
{रमन्तं तारया सार्धं शीघ्रमानय मां प्रति} %॥३८॥

\twolineshloka
{नात्रागच्छति सुग्रीवो यद्यसौ प्राप्तभूतिकः}
{तदा त्वयैवं वक्तव्यः सुग्रीवोऽनृतभाषकः} %॥३९॥

\twolineshloka
{वालिहन्ता शरो दुष्ट करे मेऽद्यापि तिष्ठति}
{स्मृत्वैतदाचर कपे रामवाक्यं हितं तव} %॥४०॥

\threelineshloka
{इत्युक्तस्तु तथेत्युक्त्वा रामं नत्वा च लक्ष्मणः}
{पम्पापुरं जगामाथ सुग्रीवो यत्र तिष्ठति}
{दृष्ट्वा स तत्र सुग्रीवं कपिराजं बभाष वै} %॥४१॥

\twolineshloka
{ताराभोगविषक्तस्त्वं रामकार्यपराङ्मुखः}
{किं त्वया विस्मृतं सर्वं रामाग्रे समयं कृतम्} %॥४२॥

\twolineshloka
{सीतामन्विष्य दास्यामि यत्र क्वापीति दुर्मते}
{हत्वा तु वालिनं राज्यं येन दत्तं पुरा तव} %॥४३॥

\twolineshloka
{त्वामृते कोऽवमन्येत कपीन्द्र पापचेतस}
{प्रतिश्रुत्य च रामस्य भार्याहीनस्य भूपते} %॥४४॥

\twolineshloka
{साहाय्यं ते करोमिति देवाग्निजलसन्निधौ}
{ये ये च शत्रवो राजंस्ते ते च मम शत्रवः} %॥४५॥

\twolineshloka
{मित्राणि यानि ते देव तानि मित्राणि मे सदा}
{सीतामन्वेषितुं राजन् वानरैर्बहुभिर्वृतः} %॥४६॥

\twolineshloka
{सत्यं यास्यामि ते पार्श्वमित्युक्त्वा कोऽन्यथाकरोत्}
{त्वामृते पापिनं दुष्टं रामदेवस्य सन्निधौ} %॥४७॥

\twolineshloka
{कारयित्वा तु तेनैवं स्वकार्यं दुष्टवानर}
{ऋषीणां सत्यवद्वाक्यं त्वयि दृष्टं मयाधुना} %॥४८॥

\twolineshloka
{सर्वस्य हि कृतार्थस्य मतिरन्या प्रवर्तते}
{वत्सः क्षीरक्षयं दृष्ट्वा परित्यजति मातरम्} %॥४९॥

\twolineshloka
{जनवृत्तविदां लोके सर्वज्ञानां महात्मनाम्}
{न तं पश्यामि लोकेऽस्मिन् कृतं प्रतिकरोति यः} %॥५०॥

\twolineshloka
{शास्त्रेषु निष्कृतिर्दृष्टा महापातकिनामपि}
{कृतघ्नस्य कपे दुष्ट न दृष्टा निष्कृतिः पुरा} %॥५१॥

\twolineshloka
{कृतघ्रना न कार्या ते त्वत्कृतं समयं स्मर}
{एह्येह्यागच्छ शरणं काकुत्स्थं हितपालकम्} %॥५२॥

\twolineshloka
{यदि नायासि च कपे रामवाक्यामिदं श्रृणु}
{नयिष्ये मृत्युसदनं सुग्रीवं वालिनं यथा} %॥५३॥

\twolineshloka
{स शरो विद्यतेऽस्माकं येन वाली हतः कपिः}
{लक्ष्मणेनैवमुक्तोऽसौ सुग्रीवः कपिनायकः} %॥५४॥

\twolineshloka
{निर्गत्य तु नमश्चके लक्ष्मणं मन्त्रिणोदितः}
{उवाच च महात्मानं लक्ष्मणं वानराधिपः} %॥५५॥

\twolineshloka
{अज्ञानकृतपापानामस्माकं क्षन्तुमर्हसि}
{समयः कृतो मया राज्ञा रामेणामिततेजसा} %॥५६॥

\twolineshloka
{यस्तदानीं महाभाग तमद्यापि न लङ्घये}
{यास्यामि निखिलरैद्य कपिभिर्नृपनन्दन} %॥५७॥

\twolineshloka
{त्वया सह महावीर रामपार्श्वं न संशयः}
{मां दृष्ट्वा तत्र काकुत्स्थो यद्वक्ष्यति च मां प्रति} %॥५८॥

\twolineshloka
{तत्सर्वं शिरसा गृह्य करिष्यासि न संशयः}
{सन्ति मे हरयः शूराः सीतान्वेषणकर्मणि} %॥५९॥

\twolineshloka
{तान्यहं प्रेषयिष्यामि दिक्षु सर्वासु पार्थिव}
{इत्युक्तः कपिराजेन सुग्रीवेण स लक्ष्मणः} %॥६०॥

\twolineshloka
{इहि शीघ्रं गमिष्यामो रामपार्श्वमितोऽधुना}
{सेना चाहूयतां वीर ऋक्षाणां हरिणामपि} %॥६१॥

\twolineshloka
{यां दृष्ट्वा प्रीतिमभ्येति राघवस्ते महामते}
{इत्युक्तो लक्ष्मणेनाथ सुग्रीवः स तु वीर्यवान्} %॥६२॥

\twolineshloka
{पार्श्वस्यं युवराजानमङ्गदं सज्ञयाब्रवीत्}
{सोऽपि निर्गत्य सेनानीमाह सेनापतिं तदा} %॥६३॥

\twolineshloka
{तेनाहूताः समागत्य ऋक्षवानरकोटयः}
{गुहास्थाश्च गिरिस्थाश्च वृक्षस्थाश्चैव वानराः} %॥६४॥

\twolineshloka
{तैः सार्धं पर्वताकारैर्वानरैर्भीमविक्रमैः}
{सुग्रीवः शीघ्रमागत्य ववन्दे राघवं तदा} %॥६५॥

\twolineshloka
{लक्ष्मणोऽपि नमस्कृत्य रामं भ्रातरमब्रवीत्}
{प्रसादं कुरु सुग्रीवे विनीते चाधुना नृप} %॥६६॥

\twolineshloka
{इत्युक्तो राघवस्तेन भ्रात्रा सुग्रीवमब्रवीत्}
{आगच्छात्र महावीर सुग्रीव कुशलं तव} %॥६७॥

\twolineshloka
{श्रुत्वेत्थं रामवचनं प्रसन्नं च नराधिपम}
{शिरस्यञ्जलिमाधाय सुग्रीवो राममब्रवीत्} %॥६८॥

\twolineshloka
{तदा मे कुशलं राजन् सीतादेवी तव प्रभो}
{अन्विष्य तु यदा दत्ता मया भवति नान्यथा} %॥६९॥

\twolineshloka
{इत्युक्ते वचने तेन हनूमान्मारुतात्मजः}
{नत्वा रामं बभाषैनं सुग्रीवं कपिनायकम्} %॥७०॥

\twolineshloka
{श्रृणु सुग्रीव मे वाक्यं राजायं दुःखितो भृशम्}
{सीतावियोगेन च सदा नाश्नाति च फलादिकम्} %॥७१॥

\twolineshloka
{अस्य दुःखेन सततं लक्ष्मणोऽयं सुदुःखितः}
{एतयोरत्र यावस्था तां श्रुत्वा भरतोऽनुजः} %॥७२॥

\twolineshloka
{दुःखी भवति तददुः खाददुः खं प्राप्नोति तज्जनः}
{यत एवमतो राजन् सीतान्वेषणमाचर} %॥७३॥

\twolineshloka
{इत्युक्ते वचने तत्र वायुपुत्रेण धीमता}
{जाम्बवानतितेजस्वी नत्वा रामं पुरः स्थितः} %॥७४॥

\twolineshloka
{स प्राह कपिराजं तं नीतिमान् नीतिमद्वचः}
{यदुक्तं वायुपुत्रेण तत्तथेत्यवगच्छ भोः} %॥७५॥

\twolineshloka
{यत्र क्वापि स्थिता सीता रामभार्या यशस्विनी}
{पतिव्रता महाभागा वैदेही जनकात्मजा} %॥७६॥

\twolineshloka
{अद्यापि वृत्तसम्पन्ना इति मे मनसि स्थितम्}
{न हि कल्याणचित्तायाः सीतायाः केनचिद्भुवि} %॥७७॥

\twolineshloka
{पराभवोऽस्ति सुग्रीव प्रेषयाद्यैव वानरान्}
{इत्युक्तस्तेन सुग्रीवः प्रीतामा कपिनायकः} %॥७८॥

\twolineshloka
{पश्चिमायां दिशि तदा प्रेषयामास तान् कपीन्}
{अन्वेष्टुं रामभार्यां तां महाबलपराक्रमः} %॥७९॥

\twolineshloka
{उत्तरस्यां दिशि तदा नियुतान् वानरानसौ}
{प्रेषयामास धर्मात्मा सीतान्वेषणकर्मणि} %॥८०॥

\twolineshloka
{पूर्वस्यां दिशि कर्पीश्च कपिराजः प्रतापवान्}
{प्रेषयामास रामस्य सुभार्यान्वेषणाय वै} %॥८१॥

\twolineshloka
{इति तान् प्रेषयामास वानरान् वानराधिपः}
{सुग्रीवो वालिपुत्रं तमङ्गदं प्राह बुद्धिमान्} %॥८२॥

\twolineshloka
{त्वं गच्छ दक्षिणं देशं सीतान्वेषणकर्मणि}
{जाम्बवांश्च हनूमांश्च मैन्दो द्विविद एव च} %॥८३॥

\twolineshloka
{नीलाद्याश्चैव हरयो महाबलपराक्रमाः}
{अनुयास्यन्ति गच्छन्तं त्वामद्य मम शासनात्} %॥८४॥

\twolineshloka
{अचिरादेव यूयं तां दृष्ट्वा सीतां यशस्विनीम्}
{स्थानतो रुपतश्चैव शीलतश्च विशेषतः} %॥८५॥

\twolineshloka
{केन नीता च कुत्रास्ते ज्ञात्वात्रागच्छ पुत्रक}
{इत्युक्तः कपिराजेन पितृव्येण महात्मना} %॥८६॥

\twolineshloka
{अङ्गदस्तूर्णमुत्थाय तस्याज्ञां शिरसा दधे}
{इत्युक्ते दूरतः स्थाप्य वानरानथ जाम्बवान्} %॥८७॥

\twolineshloka
{रामं च लक्ष्मणं चैव सुग्रीवं मारुतात्मजम्}
{एकतः स्थाप्य तानाह नीतिमान् नीतिमद्वचः} %॥८८॥

\twolineshloka
{श्रूयतां वचनं मेऽद्य सीतान्वेषणकर्मणि}
{श्रुत्वा च तदगृहाण त्वं रोचते यन्नृपात्मज} %॥८९॥

\twolineshloka
{रावणेन जनस्थानान्नीयमाना तपस्विनी}
{जटायुषा तु सा दृष्ट्वा शक्त्या युद्धं प्रकुर्वता} %॥९०॥

\twolineshloka
{भूषणानि च दृष्टानि तया क्षिप्तानि तेन वै}
{तान्यस्माभिः प्रदृष्टानि सुग्रीवायार्पितानि च} %॥९१॥

\twolineshloka
{जटायुवाक्याद्राजेन्द्र सत्यमित्यवधारय}
{एतस्मात्कारणात्सीता नीता तेनैव रक्षसा} %॥९२॥

\twolineshloka
{रावणेन महाबाहो लङ्कायां वर्तते तु सा}
{त्वां स्मरन्ती तु तत्रस्था त्वद्दुःखेन सुदुःखिता} %॥९३॥

\twolineshloka
{रक्षन्ती यत्नतो वृत्तं तत्रपि जनकात्मजा}
{त्वद्ध्यानेनैव स्वान् प्राणान्धारयन्ती शुभानना} %॥९४॥

\twolineshloka
{स्थिता प्रायेण ते देवी सीता दुःखपरायणा}
{हितमेव च ते राजन्नुदधेर्लङ्घने क्षमम्} %॥९५॥

\twolineshloka
{वायुपुत्रं हनूमन्तं त्वमत्रादोष्टुमर्हसि}
{त्वं चाप्यर्हसि सुग्रीव प्रेषितुं मारुतात्मजम्} %॥९६॥

\twolineshloka
{तमृते सागरं गन्तुं वानराणां न विद्यते}
{बलं कस्यापि वा वीर इति मे मनसि स्थितम्} %॥९७॥

\twolineshloka
{क्रियतां मव्दचः क्षिप्रं हितं पथ्यं च नः सदा}
{उक्ते जाम्बवतैवं तु नीतिस्वल्पाक्षरान्विते} %॥९८॥

\twolineshloka
{वाक्ये वानरराजोऽसौ शीघ्रमुत्थाय चासनात्}
{वायुपुत्रसमीपं तु तं गत्वा वाक्यमब्रवीत्} %॥९९॥

\twolineshloka
{श्रृणु मद्वचनं वीर हनुमन्मारुतात्मज}
{अयमिक्ष्वाकुतिलको राजा रामः प्रतापवान्} %॥१००॥

\twolineshloka
{पितुरादेशमादाय भ्रातृभार्यासमन्वितः}
{प्रविष्टो दण्डकारण्यं साक्षाद्धर्मपरायणः} %॥१०१॥

\twolineshloka
{सर्वात्मा सर्वलोकेशो विष्णुर्मानुषरुपवान्}
{अस्य भार्या हता तेन दुष्टेनापि दुरात्मना} %॥१०२॥

\twolineshloka
{तद्वियोगजदुःखार्तो विचिन्वंस्तां वने वने}
{त्वया दृष्टो नृपः पूर्वमयं वीरः प्रतापवान्} %॥१०३॥

\twolineshloka
{एतेन सह सगम्य समयं चापि कारितम्}
{अनेन निहतः शत्रुर्मम वालिर्महाबलः} %॥१०४॥

\twolineshloka
{अस्य प्रसादेन कपे राज्यं प्राप्तं मयाधुना}
{मया च तत्प्रतिज्ञातमस्य साहाय्यकर्मणि} %॥१०५॥

\twolineshloka
{तत्सत्यं कर्तुमिच्छामि त्वद्वलान्मारुतात्मज}
{उत्तीर्य सागरं वीर दृष्टा सीतामनिन्दिताम्} %॥१०६॥

\twolineshloka
{भूयस्तर्तुं बलं नास्ति वानराणां त्वया विना}
{अतस्त्वमेव जानासि स्वामिकार्यं महामते} %॥१०७॥

\twolineshloka
{बलवान्नीतिमांश्चैव दक्षस्त्वं दौत्यकर्मणि}
{तेनैवमुक्तो हनुमान् सुग्रीवेण महात्मना} %॥१०८॥

\twolineshloka
{स्वामिनोऽर्थं न किं कुर्यामीदृशं किं नु भाषसे}
{इत्युक्तो वायुपुत्रेण रामस्तं पुरतः स्थितम्} %॥१०९॥

\twolineshloka
{प्राह वाक्यं महाबाहुर्वाष्पसम्पूर्णलोचनः}
{सीतां स्मृत्वा सुदुःखार्तः कालयुक्तममित्रजित्} %॥११०॥

\twolineshloka
{त्वयि भारं समारोप्य समुद्रतरणादिकम्}
{सुग्रीवः स्थाप्यते ह्यत्र मया सार्धं महामते} %॥१११॥

\twolineshloka
{हनुमंस्तत्र गच्छ त्वं मत्प्रीत्यै कृतनिश्चयः}
{ज्ञातीनां च तथा प्रीत्यै सुग्रीवस्य विशेषतः} %॥११२॥

\twolineshloka
{प्रायेण रक्षसा नीता भार्या मे जनकात्मजा}
{तत्र गच्छ महावीर यत्र सीता व्यवस्थिता} %॥११३॥

\twolineshloka
{यदि पृच्छति सादृश्यं मदाकारमशेषतः}
{अतो निरीक्ष्य मां भूयो लक्ष्मणं च ममानुजम्} %॥११४॥

\twolineshloka
{ज्ञात्वा सर्वाङ्गगं लक्ष्म सकलं चावयोरिह}
{नान्यथा विश्वसेत्सीता इति मे मनसि स्थितम्} %॥११५॥

\twolineshloka
{इत्युक्तो रामदेवेन प्रभञ्जनसुतो बली}
{उत्थाय तत्पुरः स्थित्वा कृताञ्जलिरुवाच तम्} %॥११६॥

\twolineshloka
{जानामि लक्षणं सर्वं युवयोस्तु विशेषतः}
{गच्छामि कपिभिः सार्धं त्वं शोकं मा कुरुष्व वै} %॥११७॥

\twolineshloka
{अन्यच्च देह्यभिज्ञानं विश्वासो येन मे भवेत्}
{सीतायास्तव देव्यास्तु राजन् राजीवलोचन} %॥११८॥

\twolineshloka
{इत्युक्तो वायुपुत्रेण रामः कमललोचनः}
{अङ्गुलीयकमुन्मुच्य दत्तवान् रामचिह्नितम्} %॥११९॥

\twolineshloka
{तदगृहीत्वा तदा सोऽपि हनुमान्मारुतात्मजः}
{रामं प्रदक्षिणीकृत्य लक्ष्मणं च कपीश्वरम्} %॥१२०॥

\twolineshloka
{नत्वा ततो जगामाशु हनुमानञ्जनीसुतः}
{सुग्रीवोऽपि च ताञ्छुत्वा वानरान् गन्तुमुद्यतान्} %॥१२१॥

\twolineshloka
{आज्ञेयानाज्ञापयति वानरान् बलदर्पितान्}
{श्रृण्वन्तु वानराः सर्वे शासनं मम भाषितम्} %॥१२२॥

\twolineshloka
{विलम्बनं न कर्तव्यं युष्माभिः पर्वतादिषु}
{द्रुतं गत्वा तु तां वीक्ष्य आगन्तव्यमनिन्दिताम्} %॥१२३॥

\twolineshloka
{रामपत्नीं महाभागां स्थास्येऽहं रामसन्निधौ}
{कर्तनं वा करिष्यामि अन्यथा कर्णनासयोः} %॥१२४॥

\twolineshloka
{एवं तान् प्रेषयित्वा तु आज्ञापूर्वं कपीश्वरः}
{अथ ते वानरा याताः पश्चिमादिषु दिक्षु वै} %॥१२५॥

\twolineshloka
{ते सानुषु समस्तेषु गिरीणामपि मूर्धसु}
{नदीतीरेषु सर्वेषु मुनीनामाश्रमेषु च} %॥१२६॥

\twolineshloka
{कन्दरेषु च सर्वेषु वनेषूपवनेषु च}
{वृक्षेषु वृक्षगुल्मेषु गुहासु च शिलासु च} %॥१२७॥

\twolineshloka
{सह्यपर्वतपार्श्वेषु विन्ध्यसागरपार्श्वयोः}
{हिमवत्यपि शैले च तथा किम्पुरुषादिषु} %॥१२८॥

\twolineshloka
{मनुदेशेषु सर्वेषु सप्तपातालकेषु च}
{मध्यदेशेषु सर्वेषु कश्मीरेषु महाबलाः} %॥१२९॥

\twolineshloka
{पूर्वदेशेषु सर्वेषु कामरुपेषु कोशले}
{तीर्थस्थानेषु सर्वेषु सप्तकोङ्कणकेषु च} %॥१३०॥

\twolineshloka
{यत्र तत्रैव ते सीतामदृष्ट्वा पुनरागताः}
{आगत्य ते नमस्कृत्य रामलक्ष्मणपादयोः} %॥१३१॥

\twolineshloka
{सुग्रीवं च विशेषेण नास्माभिः कमलेक्षणा}
{दृष्टा सीता महाभागेत्युक्त्वा तांस्तत्र तस्थिरे} %॥१३२॥

\twolineshloka
{ततस्तं दुःखितं प्राह रामदेवं कपीश्वरः}
{सीता दक्षिणदिग्भागे स्थिता द्रष्टुं वने नृप} %॥१३३॥

\twolineshloka
{शक्या वानरसिंहेन वायुपुत्रेण धीमता}
{दृष्टा सीतामिहायाति हनुमान्नात्र संशयः} %॥१३४॥

\twolineshloka
{स्थिरो भव महाबाहो राम सत्यमिदं वचः}
{लक्ष्मणोऽप्याह शकुनं तत्र वाक्यमिदं तदा} %॥१३५॥

\twolineshloka
{सर्वथा दृष्टसीतस्तु हनुमानागमिष्यति}
{इत्याश्वास्य स्थितौ तत्र रामं सुग्रीवलक्ष्मणौ} %॥१३६॥

\twolineshloka
{अथाङ्गदं पुरस्कृत्य ये गता वानरोत्तमाः}
{यत्नादन्वेषणार्थाय रामपत्नीं यशस्विनीम्} %॥१३७॥

\twolineshloka
{अदृष्ट्वा श्रममापन्नाः कृच्छ्रभूतास्तदा वने}
{भक्षणेन विहीनास्ते क्षुधया च प्रपीडिताः} %॥१३८॥

\twolineshloka
{भ्रमद्भिर्गहनेऽरण्ये क्वापि दृष्ट्वा च सुप्रभा}
{गुहानिवासिनी सिद्धा ऋषिपत्नी ह्यनिन्दिता} %॥१३९॥

\twolineshloka
{सा च तानागतान्दृष्ट्वा स्वाश्रमं प्रति वानरान्}
{आगताः कस्य यूयं तु कुतः किं नु प्रयोजनम्} %॥१४०॥

\twolineshloka
{इत्युक्ते जाम्बवानाह तां सिद्धां सुमहामतिः}
{सुग्रीवस्य वयं भृत्या आगता ह्यत्र शोभने} %॥१४१॥

\twolineshloka
{रामभार्यार्थमनघे सीतान्वेषणकर्मणि}
{कां दिग्भूता निराहारा अदृष्टा जनकात्मजाम्} %॥१४२॥

\twolineshloka
{इत्युक्ते जाम्बवत्यत्र पुनस्तानाह सा शुभा}
{जानामि रामं सीतां च लक्ष्मणं च कपीश्वरम्} %॥१४३॥

\twolineshloka
{भुञ्जीध्वमत्र मे दत्तमाहारं च कपीश्वराः}
{रामकार्यागतास्त्वत्र यूयं रामसमा मम} %॥१४४॥

\twolineshloka
{इत्युक्त्वा चामृतं तेषां योगाद्दत्वा तपस्विनी}
{भोजयित्वा यथाकामं भूयस्तानाह तापसी} %॥१४५॥

\twolineshloka
{सीतास्थानं तु जानाति सम्पातिर्नाम पक्षिराट्}
{आस्थितो वै वने सोऽपि महेन्द्रे पर्वते द्विजः} %॥१४६॥

\twolineshloka
{मार्गेणानेन हरयस्तत्र यूयं गमिष्यथ}
{स वक्ति सीतां सम्पातिर्दूरदर्शी तु यः खगः} %॥१४७॥

\twolineshloka
{तेनादिष्टं तु पन्थानं पुनरासाद्य गच्छथ}
{अवश्यं जानकीं सीतां द्रक्ष्यते पवनात्मजः} %॥१४८॥

\twolineshloka
{तयैवमुक्ताः कपयः परां प्रीतिमुपागताः}
{ह्यष्टास्तेजनमापन्नास्तां प्रणम्य प्रतस्थिरे} %॥१४९॥

\twolineshloka
{महेन्द्राद्रिं गता वीरा वानरास्तद्दिदृक्षया}
{तत्र सम्पातिमासीनं दृष्टवन्तः कपीश्वराः} %॥१५०॥

\twolineshloka
{तानुवाचाथ सम्पातिर्वानरानागतान्द्विजः}
{के यूयमिति सम्प्राप्ताः कस्य वा ब्रूत मा चिरम्} %॥१५१॥

\twolineshloka
{इत्युक्ते वानरा ऊचुर्यथावृत्तमनुक्रमात्}
{रामदूता वयं सर्वे सीतान्वेषणकर्मणि} %॥१५२॥

\twolineshloka
{प्रेषिताः कपिराजेन सुग्रीवेण महात्मना}
{त्वां द्रष्टुमिह सम्प्राप्ताः सिद्धाया वचनादद्विज} %॥१५३॥

\twolineshloka
{सीतास्थानं महाभाग त्वं नो वद महामते}
{इत्युक्तो वानरैः श्येनो वीक्षाचक्रे सुदक्षिणाम्} %॥१५४॥

\twolineshloka
{सीतां दृष्ट्वा स लङ्कायामशोकाख्ये महावने}
{स्थितेति कथितं तेज जटायुस्तु मृतस्तव} %॥१५५॥

\twolineshloka
{भ्रातेति चोचुः स स्नात्वा दत्त्वा तस्योदकाञ्जलिम्}
{योगमास्थाय स्वं देहं विससर्ज महामतिः} %॥१५६॥

\twolineshloka
{ततस्तं वानरा दग्ध्वा दत्त्वा तस्योदकाञ्जलिम्}
{गत्वा महेन्द्रश्रृङ्गं ते तमारुह्य क्षणं स्थिताः} %॥१५७॥

\twolineshloka
{सागरं वीक्ष्य ते सर्वे परस्परमथाब्रुवन्}
{रावणेनैव भार्या सा नीता रामस्य निश्चितम्} %॥१५८॥

\twolineshloka
{सम्पातिवचनादद्य सज्ञातं सकलं हि तत्}
{वानराणां तु कश्चात्र उत्तीर्य लवणोदधिम्} %॥१५९॥

\twolineshloka
{लङ्कां प्रविश्य दृष्ट्वा तां रामपत्नीं यशस्विनीम्}
{पुनश्चोदधितरणे शक्तिं ब्रूत हि शोभनाः} %॥१६०॥

\twolineshloka
{इत्युक्तो जाम्बवान् प्राह सर्वे शक्तास्तु वानराः}
{सागरोत्तरणे किन्तु कार्यमन्यस्य सम्भवेत्} %॥१६१॥

\twolineshloka
{तत्र दक्षोऽयमेवात्र हनुमानिति मे मतिः}
{कालक्षेपो न कर्तव्यो मासार्धमधिकं गतम्} %॥१६२॥

\twolineshloka
{यद्यदृष्ट्वा तु गच्छामो वैदेहीं वानरर्षभाः}
{कर्णनासादि नः स्वाङ्गं निकृन्तति कपीश्वरः} %॥१६३॥

\twolineshloka
{तस्मात् प्रार्थ्यः स चास्माभिर्वायुपुत्रस्तु मे मतिः}
{इत्युक्तास्ते तथेत्यूचुर्वानरा वृद्धवानरम्} %॥१६४॥

\twolineshloka
{ततस्ते प्रार्थयामासुर्वानराः पवनात्मजम्}
{हनुमन्तं महाप्राज्ञं दक्षं कार्येषु चाधिकम्} %॥१६५॥

\threelineshloka
{गच्छ त्वं रामभृत्यस्त्वं रावणस्य भयाय च}
{रक्षस्व वानरकुलमस्माकमञ्जनीसुत}
{इत्युक्तस्तांस्तथेत्याह वानरान् पवनात्मजः} %॥१६६॥

\fourlineindentedshloka
{रामप्रयुक्तश्च पुनः स्वभर्तृणा}
{पुनर्महेन्द्रे कपिभिश्च नोदितः}
{गन्तुं प्रचक्रे मतिमञ्जनीसुतः}
{समुद्रमुत्तीर्य निशाचरालयम्} %॥१६७॥

॥इति श्रीनरसिंहपुराणे रामप्रादुर्भावे पञ्चाशत्तमोऽध्यायः॥५०॥

\sect{एकपञ्चाशत्तमोऽध्यायः --- सुन्दर-काण्डः}

\uvacha{मार्कण्डेय उवाच}

\twolineshloka
{स तु रावणनीतायाः सीतायाः परिमार्गणम्}
{इयेष पदमन्वेष्टुं चारणाचरिते पथि} %॥१॥

\twolineshloka
{अञ्जलिं प्राङ्मुखं कृत्वा सगणायात्मयोनये}
{मनसाऽऽवन्द्य रामं च लक्ष्मणं च महारथम्} %॥२॥

\twolineshloka
{सागरं सरितश्चैव प्रणम्य शिरसा कपिः}
{ज्ञातीश्चैव परिष्वज्य कृत्वा चैव प्रदक्षिणाम्} %॥३॥

\twolineshloka
{अरिष्टं गच्छ पन्थानं पुण्यवायुनिषेवितम्}
{पुनरागमनायेति वानरैरभिपूजितः} %॥४॥

\twolineshloka
{अञ्जसा स्वं तथा वीर्यमाविवेशाथ वीर्यवान्}
{मार्गमालोकयन् दूरादूर्ध्वं प्रणिहितेक्षणः} %॥५॥

\twolineshloka
{सम्पूर्णमिव चात्मानं भावयित्वा महाबलः}
{उत्पपात गिरेः श्रृङ्गान्निष्पीड्य गिरिमम्बरम्} %॥६॥

\twolineshloka
{पितुर्मार्गेण यातस्य वायुपुत्रस्य धीमतः}
{रामकार्यपरस्यास्य सागरेण प्रचोदितः} %॥७॥

\twolineshloka
{विश्रामार्थं समुत्तस्थौ मैनाको लवणोदधेः}
{तं निरीक्ष्य निपीड्याथ रयात्सम्भाष्य सादरम्} %॥८॥

\twolineshloka
{उत्पतंश्च वने वीरः सिंहिकास्यं महाकपिः}
{आस्यप्रान्तं प्रविश्याथ वेगेनान्तर्विनिस्सृतः} %॥९॥

\twolineshloka
{निस्सृत्य गतवाञ्शीघ्रं वायुपुत्रः प्रतापवान्}
{लङ्घयित्वा तु तं देशं सागरं पवनात्मजः} %॥१०॥

\twolineshloka
{त्रिकूटशिखरे रम्ये वृक्षाग्रे निपपात ह}
{तस्मिन् स पर्वतश्रेष्ठे दिनं नीत्वा दिनक्षये} %॥११॥

\twolineshloka
{सन्ध्यामुपास्य हनुमान् रात्रौ लङ्कां शनैर्निशि}
{लङ्काभिधां विनिर्जित्य देवतां प्रविवेश ह} %॥१२॥

\twolineshloka
{लङ्कामनेकरत्नाढ्यां बह्वाश्चर्यसमन्विताम्}
{राक्षसेषु प्रसुप्तेषु नीतिमान् पवनात्मजः} %॥१३॥

\twolineshloka
{रावणस्य ततो वेश्म प्रविवेशाथ ऋद्धिमत्}
{शयानं रावणं दृष्ट्वा तल्पे महति वानरः} %॥१४॥

\twolineshloka
{नासापुटैर्घोरकारैर्विशद्भिर्वायुमोचकैः}
{तथैव दशभिर्वक्त्रैर्दंष्टोपेतैस्तु संयुतम्} %॥१५॥

\twolineshloka
{स्त्रीसहस्थैस्तु दृष्ट्वा तं नानाभरणभूषितम्}
{तस्मिन् सीतामदृष्ट्वा तु रावणस्य गृहे शुभे} %॥१६॥

\twolineshloka
{तथा शयानं स्वगृहे राक्षसानां च नायकम्}
{दुःखितो वायुपुत्रस्तु सम्पातेर्वचनं स्मरन्} %॥१७॥

\twolineshloka
{अशोकवनिकां प्राप्तो नानापुष्पसमन्विताम्}
{जुष्टां मलयजातेन चन्दनेन सुगन्धिना} %॥१८॥

\twolineshloka
{प्रविश्य शिंशपावृक्षमाश्रितां जनकात्मजाम्}
{रामपत्नीं समद्राक्षीद राक्षसीभिः सुरक्षिताम्} %॥१९॥

\twolineshloka
{अशोकवृक्षमारुह्य पुष्पितं मधुपल्लवम्}
{आसाचक्रे हरिस्तत्र सेयं सीतेति संस्मरन्} %॥२०॥

\twolineshloka
{सीतां निरीक्ष्य वृक्षाग्रे यावदास्तेऽनिलात्मजः}
{स्त्रीभिः परिवृतस्तत्र रावणस्तावदागतः} %॥२१॥

\twolineshloka
{आगत्य सीतां प्राहाथ प्रिये मां भज कामुकम्}
{भूषिता भव वैदेहि त्यज रामगतं मनः} %॥२२॥

\twolineshloka
{इत्येवं भाषमाणं तमन्तर्धाय तृणं ततः}
{प्राह वाक्यं शनैः सीता कम्पमानाथ रावणम्} %॥२३॥

\twolineshloka
{गच्छ रावण दुष्ट त्वं परदारपरायण}
{अचिराद्रामबाणास्ते पिबन्तु रुधिरं रणे} %॥२४॥

\twolineshloka
{तथेत्यक्तो भर्त्सितश्च राक्षसीराह राक्षसः}
{द्विमासाभ्यन्तरे चैनां वशीकुरुत मानुषीम्} %॥२५॥

\twolineshloka
{यदि नेच्छति मां सीता ततः खादत मानुषीम्}
{इत्युक्त्वा गतवान् दुष्टो रावणः स्वं निकेतनम्} %॥२६॥

\twolineshloka
{ततो भयेन तां प्राहू राक्षस्यो जनकात्मजाम्}
{रावणं भज कल्याणी सधनं सुखिनी भव} %॥२७॥

\twolineshloka
{इत्युक्ता प्राह ताः सीता राघवोऽलघुविक्रमः}
{निहत्य रावणं युद्धे सगणं मां नयिष्यति} %॥२८॥

\twolineshloka
{नाहमन्यस्य भार्या स्यामृते रामं रघूत्तमम्}
{स ह्यागत्य दशग्रीवं हत्वा मां पालयिष्यति} %॥२९॥

\twolineshloka
{इत्याकर्ण्य वचस्तस्या राक्षस्यो ददृशुर्भयम्}
{हन्यतां हन्यतामेषा भक्ष्यतां भक्ष्यतामियम्} %॥३०॥

\twolineshloka
{ततस्तास्त्रिजटा प्राह स्वप्ने दृष्टमनिन्दिता}
{श्रृणुध्वं दुष्टराक्षस्यो रावणस्य विनाशनः} %॥३१॥

\twolineshloka
{रक्षोभिः सह सर्वेस्तु रावणस्य मृतिप्रदः}
{लक्ष्मणेन सह भ्रात्रा रामस्य विजयप्रदः} %॥३२॥

\twolineshloka
{स्वप्नः शुभो मया दृष्टः सीतायाश्च पतिप्रदः}
{त्रिजटावाक्यमाकर्ण्य सीतापार्श्वं विसृज्य ताः} %॥३३॥

\twolineshloka
{राक्षस्यस्ता ययुः सर्वाः सीतामाहाञ्जनीसुतः}
{कीर्तयन् रामवृत्तान्तं सकलं पवनात्मजः} %॥३४॥

\twolineshloka
{तस्यां विश्वासमानीय दत्त्वा रामाङ्गुलीयकम्}
{सम्भाष्य लक्षणं सर्वं रामलक्ष्मणयोस्ततः} %॥३५॥

\twolineshloka
{महत्या सेनया युक्तः सुग्रीवः कपिनायकः}
{तेन सार्धमिहागत्य रामस्तव पतिः प्रभुः} %॥३६॥

\twolineshloka
{लक्ष्मणश्च महावीरो देवरस्ते शुभानने}
{रावणं सगणं हत्वा त्वामितोऽऽदाय गच्छति} %॥३७॥

\twolineshloka
{इत्युक्ते सा तु विश्वस्ता वायुपुत्रमथाब्रवीत्}
{कथमत्रागतो वीर त्वमुत्तीर्य महोदधितम्} %॥३८॥

\twolineshloka
{इत्याकर्ण्य वचस्तस्याः पुनस्तामाह वानरः}
{गोष्पदवन्मयोत्तीर्णः समुद्रोऽयं वरानने} %॥३९॥

\twolineshloka
{जपतो रामरामेति सागरो गोष्पदायते}
{दुःखमग्नासि वैदेहि स्थिरा भव शुभानने} %॥४०॥

\twolineshloka
{क्षिप्रं पश्यसि रामं त्वं सत्यमेतदब्रवीमि ते}
{इत्याश्वास्य सतीं सीतां दुःखितां जनकात्मजाम्} %॥४१॥

\twolineshloka
{ततश्चूडामणिं प्राप्य श्रुत्वा काकपराभवम्}
{नत्वा तां प्रस्थितो वीरो गन्तुं कृतमतिः कपिः} %॥४२॥

\twolineshloka
{ततो विमृश्य तद्भड्क्त्वा क्रीडावनमशेषतः}
{तोरणस्थो ननादोच्चै रामो जयति वीर्यवान्} %॥४३॥

\twolineshloka
{अनेकान् राक्षसान् हत्वा सेनाः सेनापतींश्च सः}
{तदा त्वक्षकुमारं तु हत्वा रावणसैनिकम्} %॥४४॥

\twolineshloka
{साश्वं ससारथिं हत्वा इन्द्रजित्तं गृहीतवान्}
{रावणस्य पुरः स्थित्वा रामं सकीर्त्य लक्ष्मणम्} %॥४५॥

\twolineshloka
{सुग्रीवं च महावीर्यं दग्ध्वा लङ्कामशेषतः}
{निर्भर्त्य्स्य रावण दुष्टं पुनः सम्भाष्य जानकीम्} %॥४६॥

\twolineshloka
{भूयः सागरमुत्तीर्य ज्ञातीनासाद्य वीर्यवान्}
{सीतादर्शनमावेद्य हनूमांश्चैव पूजितः} %॥४७॥

\twolineshloka
{वानरैः सार्धमागत्य हनुमान्मधुवनं महत्}
{निहत्य रक्षपालांस्तु पाययित्वा च तन्मधु} %॥४८॥

\twolineshloka
{सर्वे दधिमुखं पात्य हर्षितो हरिभिः सह}
{खमुत्पत्य च सम्प्राप्य रामलक्ष्मणपादयोः} %॥४९॥

\twolineshloka
{नत्वा तु हनुमांस्तत्र सुग्रीवं च विशेषतः}
{आदितः सर्वमावेद्य समुद्रतरणादिकम्} %॥५०॥

\twolineshloka
{कथयामास रामाय सीता द्रुष्टा मयेति वै}
{अशोकवनिकामध्ये सीता देवी सुदुःखिता} %॥५१॥

\twolineshloka
{राक्षसीभिः परिवृत्ता त्वां स्मरन्ती च सर्वदा}
{अश्रुपूर्णमुखी दीना तव पत्नी वरानना} %॥५२॥

\twolineshloka
{शीलवृत्तसमायुक्ता तत्रापि जनकात्मजा}
{सर्वत्रान्वेषमाणेन मया दुष्टा पतिव्रता} %॥५३॥

\twolineshloka
{मया सम्भाषिता सीता विश्वस्ता रघुनन्दन}
{अलङ्कारश्च सुमणिस्तया ते प्रेषितः प्रभो} %॥५४॥

\twolineshloka
{इत्युक्त्वा दत्तवांस्तस्मै चूडामणिमनुत्तमम्}
{इदं च वचनं तुभ्यं पल्या सम्प्रेषितं श्रृणु} %॥५५॥

\twolineshloka
{चित्रकूटे मदङ्के तु सुप्ते त्वयि महाव्रत}
{वायसाभिभवं राजंस्तत्किल स्मर्तुमर्हसि} %॥५६॥

\twolineshloka
{अल्पापराधे राजेन्द्र त्वया बलिभुजि प्रभो}
{यत्कृतं तन्न कर्तुं च शक्यं देवासुरैरपि} %॥५७॥

\threelineshloka
{ब्रह्मास्त्रं तु तदोत्सृष्टं रावणं किं न जेष्यसि}
{इत्येवमादि बहुशः प्रोक्त्वा सीता रुरोद ह}
{एवं तु दुःखिता सीता तां मोक्तुं यत्नमाचर} %॥५८॥

\fourlineindentedshloka
{इत्येवमुक्ते पवनात्मजेन}
{सीतावचस्तच्छुभभूषणं च}
{श्रुत्वा च दृष्ट्वा च रुरोद रामः}
{कपिं समालिङ्य शनैः प्रतस्थे} %॥५९॥

॥इति श्रीनरसिंहपुराणे रामप्रादुर्भावे एकपञ्चाशत्तमोऽध्यायः॥५१॥

\sect{द्विपञ्चाशोऽध्यायः --- युद्ध-काण्डः}

\uvacha{मार्कण्डेय उवाच}

\twolineshloka
{इति श्रुत्वा प्रियावार्तां वायुपुत्रेण कीर्तिताम्}
{रामो गत्वा समुद्रान्तं वानरैः सह विस्तृतैः} %॥१॥

\twolineshloka
{सागरस्य तटे रम्ये तालीवनविराजिते}
{सुग्रीवो जाम्बवांश्चाथ वानरैरतिहर्षितैः} %॥२॥

\twolineshloka
{सख्यातीतैर्वृतः श्रीमान् नक्षत्रैरिव चन्द्रमाः}
{अनुजेन च धीरेण वीक्ष्य तस्थौ सरित्पतिम्} %॥३॥

\twolineshloka
{रावणेनाथ लङ्कायां स सूक्तौ भर्त्सितोऽनुजः}
{विभीषणो महाबुद्धिः शास्त्रज्ञैर्मन्त्रिभिः सह} %॥४॥

\twolineshloka
{नरसिंहे महादेवे श्रीधरे भक्तवत्सले}
{एवं रामेऽचलां भक्तिमागत्य विनयात्तदा} %॥५॥

\twolineshloka
{कृताञ्जलिरुवाचेदं राममक्लिष्टकारिणम्}
{राम राम महाबाहो देवदेव जनार्दन} %॥६॥

\twolineshloka
{विभीषणोऽस्मि मां रक्ष अहं ते शरणं गतः}
{इत्युक्त्वा निपपाताथ प्राञ्जली रामपादयोः} %॥७॥

\twolineshloka
{विदितार्थोऽथ रामस्तु तमुत्थाप्य महामतिम्}
{समुद्रतोयैस्तं वीरमभिषिच्य विभीषणम्} %॥८॥

\twolineshloka
{लङ्काराज्यं तवैवेति प्रोक्तः सम्भाष्य तस्थिवान्}
{ततो विभीषणेनोक्तं त्वं विष्णुर्भुवनेश्वरः} %॥९॥

\twolineshloka
{अब्धिर्ददातु मार्गं ते देव तं याचयामहे}
{इत्युक्तो वानरैः सार्धं शिश्ये तत्र स राघवः} %॥१०॥

\twolineshloka
{सुप्ते रामे गतं तत्र त्रिरात्रमतितद्युतौ}
{ततः क्रुद्धो जगन्नाथो रामो राजीवलोचनः} %॥११॥

\twolineshloka
{संशोषणमपां कर्तुमस्त्रमाग्नेयमाददे}
{तदोत्थाय वचः प्राह लक्ष्मणश्च रुषान्वितम्} %॥१२॥

\twolineshloka
{क्रोधस्ते लयकर्ता हि एनं जहि महामते}
{भूतानां रक्षणार्थाय अवतारस्त्वया कृतः} %॥१३॥

\twolineshloka
{क्षन्तव्यं देवदेवेश इत्युक्त्वा धृतवान् शरम्}
{ततो रात्रित्रये याते कुद्धं राममवेक्ष्य सः} %॥१४॥

\twolineshloka
{आग्नेयास्त्राच्च सन्त्रस्तः सागरोऽभ्येत्य मूर्तिमान्}
{आह रामं महादेवं रक्ष मामपकारिणम्} %॥१५॥

\twolineshloka
{मार्गो दत्तो मया तेऽद्य कुशलः सेतुकर्मणि}
{नलश्च कथितो वीरस्तेन कारय राघव} %॥१६॥

\twolineshloka
{यावदिष्टं तु विस्तीर्ण सेतुबन्धमुत्तमम्}
{ततो नलमुखैरन्यैर्वानैररमितौजसैः} %॥१७॥

\twolineshloka
{बन्धयित्वा महासेतुं तेन गत्वा स राघवः}
{सुवेलाख्यं गिरिं प्राप्तः स्थितोऽसौ वानरैर्वृतः} %॥१८॥

\twolineshloka
{हर्म्यस्थलास्थितं दुष्टं रावणं वीक्ष्य चाङ्गध}
{रामादेशादथोत्प्लुत्य दूतकर्मसु तत्परः} %॥१९॥

\twolineshloka
{प्रादात्पादप्रहारं तु रोषाद्रावणमूर्धनि}
{विस्मितं तैः सुरगणैर्वीक्षितः सोऽतिवीर्यवान्} %॥२०॥

\twolineshloka
{साधयित्वा प्रतिज्ञां तां सुवेलं पुनरागतः}
{ततो वानरसेनाभिः सख्यातिताभिरच्युतः} %॥२१॥

\twolineshloka
{रुरोध रावणपुरीं लङ्कां तत्र प्रतापवान्}
{रामः समन्तादालोक्य प्राह लक्ष्मणमन्तिके} %॥२२॥

\fourlineindentedshloka
{तीर्णोऽर्णवः कवलितेव कपीश्वरस्य}
{सेनाभटैर्झटिति राक्षसराजधानीम्}
{यत्पौरुषोचितामिहाङ्कुरितं मया तद्}
{दैवस्य वश्यमपरं धनुषोऽथ वास्य} %॥२३॥

\onelineshloka*
{लक्ष्मणः प्राह--- कातरजनमनोऽवलम्बिना किं दैवेन।}

\fourlineindentedshloka
{यावल्ललाटशिखरं भ्रुकुटिर्नयाति}
{यावन्न कार्मुकशिखामधिरोहति ज्या}
{तावन्निशाचरपतेः पटिमानमेतु}
{त्रैलोक्यमूलविभुजेषु दर्पः} %॥२४॥


तदा लक्ष्मणः रामस्य कर्णे लगित्वा पितृवधवैरस्मरणे अथ 
तद्भक्तिवीर्यपरीक्षणाय लक्षणविज्ञानायादिश्यतामङ्गदाय दूत्यम्
रामः साधु इति भणित्वा अङ्गदं सबहुमानमवलोक्य आदिशति॥२५॥

अङ्गद! पिता ते यद्वाली बलिनि दशकण्ठे
कलितवान्नशक्तास्तद्वक्तुं वयमपि मुदा तेन पुलकः
स एव त्वं व्यावर्त्तयसि तनुजत्वेन पितृतां 
ततः किं वक्तव्यं तिलकयति सृष्टार्थपदवीम्॥२६॥

अङ्गदो मौलिमण्डलमिलत्करयुगलेन प्रणम्य---\\
यदाज्ञापयति देवः। अवधार्यताम्॥२७॥

किं प्राकारविहारतोरणवतीं लङ्कामिहैवानये
किं वा सैन्यमहं द्रुतं रघुपते तत्रैव सम्पादये
अत्यल्पं कुलपर्वतैरविरलैर्बध्नामि वा सागरं
देवादेशय किं करोमि सकलं दोर्द्दण्डसाध्यं मम॥२८॥

श्रीरामस्तद्वचनमात्रेणैव तद्भक्तिं सामर्थ्य चावेक्ष्य वदति॥२९॥

अज्ञानादथवाधिपत्यरभसा वास्मत्परोक्षे ह्नता सीतेयं
प्रविमुच्यतामिति वचो गत्वा दशास्यं वद
नो चेल्लोक्ष्मणमुक्तमार्गणगणच्छेदोच्छलच्छोणित-
च्छत्रच्छन्नदिगन्तमन्तकपुरीं पुत्रैर्वृतो यास्यसि॥३०॥

अङ्गदः--- देव!॥३१॥

\addtocounter{shlokacount}{7}
\twolineshloka
{सन्धौ वा विग्रहे वापि मयि दूते दशाननी}
{अक्षता वाक्षता वापि क्षितिपीठे लुठिष्यति} %॥३२॥

\twolineshloka
{तदा श्रीरामचन्द्रेण प्रशस्य प्रहितोऽङ्गदः}
{उक्तिप्रत्युक्तिचात्यर्यैः पराजित्यागतो रिपुम्} %॥३३॥

\twolineshloka
{राघवस्य बलं ज्ञात्वा चारैस्तदनुजस्य च}
{वानराणां च भीतोऽपि निर्भीरिव दशाननः} %॥३४॥

\twolineshloka
{लङ्कापुरस्य रक्षार्थमादिदेश स राक्षसान्}
{आदिश्य सर्वतो दिक्षु पुत्रानाह दशाननः} %॥३५॥

\threelineshloka
{धूम्राक्षं धूम्रपानं च राक्षसा यात मे पुरीम्}
{पाशैर्बध्नीत तौ मर्यौ अमित्रान्तकवीर्यवान्}
{कुम्भकर्णोऽपि मदभ्राता तुर्यनादैः प्रबोधितः} %॥३६॥

\twolineshloka
{राक्षसाश्चैव सन्दिष्टा रावणेन महाबलाः}
{तस्याज्ञां शिरसाऽऽदाय युयुधुर्वानरैः सह} %॥३७॥

\twolineshloka
{युध्यमाना यथाशक्त्या कोटिसख्यास्तु राक्षसाः}
{वानरैर्निधनं प्राप्ताः पुनरन्यान् यथाऽऽदिशत्} %॥३८॥

\twolineshloka
{पूर्वद्वारे दशग्रीवो राक्षसानमितौजसः}
{ते चापि युध्य हरिभिर्नीलाद्यैर्निधनं गताः} %॥३९॥

\twolineshloka
{अथ दक्षिणदिग्भागे रावणेन नियोजिताः}
{ते सर्वे वानरवरैर्दारितास्तु यमं गताः} %॥४०॥

\twolineshloka
{पश्चिमेऽङ्गदमुख्यैश्च वानरैरतिगर्वितैः}
{राक्षसाः पर्वताकाराः प्रापिता यमसादनम्} %॥४१॥

\twolineshloka
{तदुत्तरे तु दिग्भागे रावणेन निवेशिताः}
{पेतुस्ते राक्षसाः क्रूरा मैन्दाद्यैर्वानरैर्हताः} %॥४२॥

\twolineshloka
{ततो वानरसङ्घास्तु लङ्काप्राकारमुच्छ्रितम्}
{उत्प्लुत्याभ्यन्तरस्थांश्च राक्षसान् बलदर्पितान्} %॥४३॥

\twolineshloka
{हत्वा शीघ्रं पुनः प्राप्ताः स्वसेनामेव वानराः}
{एवं हतेषु सर्वेषु राक्षसेषु दशाननः} %॥४४॥

\twolineshloka
{रोदमानासु तस्त्रीषु निर्गतः क्रोधमूर्च्छितः}
{द्वारे स पश्चिमे वीरो राक्षसैर्बहुभिर्वृतः} %॥४५॥

\twolineshloka
{क्वासौ रामेति च वदन् धनुष्पाणीः प्रतापवान्}
{रथस्थः शरवर्षं च विसृजन् वानरेषु सः} %॥४६॥

\twolineshloka
{ततस्तद्वाणछिन्नाङ्गा वानरा दुद्रुवुस्तदा}
{पलायमानांस्तान् दृष्ट्वा वानरान् राघवस्तदा} %॥४७॥

\twolineshloka
{कस्मात्तु वानरा भग्नाः किमेषां भयमागतम्}
{इति रामवचः श्रुत्वा प्राह वाक्यं विभीषणः} %॥४८॥

\twolineshloka
{श्रृणु राजन् महाबाहो रावणो निर्गतोऽधुना}
{तद्वाणाछिन्ना हरयः पलायन्ते महामते} %॥४९॥

\twolineshloka
{इत्युक्तो राघवस्तेन धनुरुद्यम्य रोषितः}
{ज्याघोषतलघोषाभ्यां पूरयामास खं दिशः} %॥५०॥

\twolineshloka
{युयुधे रावणेनाथ रामः कमललोचनः}
{सुग्रीवो जाम्बवांश्चैव हनूमानङ्गदस्तथा} %॥५१॥

\twolineshloka
{विभीषणो वानराश्च लक्ष्मणश्चापि वीर्यवान्}
{उपेत्य रावणीं सेनां वर्षन्तीं सर्वसायकान्} %॥५२॥

\twolineshloka
{हस्त्यश्वरथसंयुक्तां ते निजघ्नुर्महाबलाः}
{रामरावणयोर्युद्धमभूत् तत्रापि भीषणम्} %॥५३॥

\twolineshloka
{रावणेन विसृष्टानि शस्त्रास्त्राणि च यानि वै}
{तानि छित्त्वाथ शस्त्रैस्तु राघवश्च महाबलः} %॥५४॥

\twolineshloka
{शरेण सारथिं हत्वा दशभिश्च महाहयान्}
{रावणस्य धनुश्छित्त्वा भल्लेनैकेन राघवः} %॥५५॥

\twolineshloka
{मुकुटं पञ्चदशभिश्छित्त्वा तन्मस्तकं पुनः}
{सुवर्णपुङ्खैर्दशभिः शरैर्विव्याध वीर्यवान्} %॥५६॥

\twolineshloka
{तदा दशास्यो व्यथितो रामबाणैर्भृशं तदा}
{विवेश मन्त्रिभिर्नीतः स्वपुरीं देवमर्दकः} %॥५७॥

\twolineshloka
{बोधितस्तूर्यनादैस्तु गजयूथक्रमैः शनैः}
{पुनः प्राकारमुल्लङ्घ्य कुम्भकर्णो विनिर्गतः} %॥५८॥

\twolineshloka
{उत्तुङ्गस्थूलदेहोऽसौ भीमदृष्टिर्महाबलः}
{वानरान् भक्षयन् दुष्टो विचचार क्षुधान्वितः} %॥५९॥

\twolineshloka
{तं दृष्टोत्पत्य सुग्रीवः शूलेनोरस्यताडयत्}
{कर्णद्वयं कराभ्यां तुच्छित्त्वा वक्त्रेण नासिकाम्} %॥६०॥

\twolineshloka
{सर्वतो युध्यमानांश्च रक्षोनाथान् रणेऽधिकान्}
{राघवो घातयित्वा तु वानरेन्दैः समन्ततः} %॥६१॥

\twolineshloka
{चकर्त विशिखैस्तीक्ष्णैः कुम्भकर्णस्य कन्धराम्}
{विजित्येन्द्रजितं साक्षादगरुडेनागतेन सः} %॥६२॥

\twolineshloka
{रामो लक्ष्मणसंयुक्तः शुशुभे वानरैर्वृतः}
{व्यर्थं गते चेन्द्रजिति कुम्भकर्णे निपातिते} %॥६३॥

\twolineshloka
{लङ्कानाथस्ततः कुद्धः पुत्रं त्रिशिरसं पुनः}
{अतिकायमहाकायौ देवान्तकनरान्तकौ} %॥६४॥

\twolineshloka
{यूयं हत्वा तु पुत्राद्या तौ नरौ युधि निघ्रत}
{तान्नियुज्य दशग्रीवः पुत्रानेवं पुनर्ब्रवीत्} %॥६५॥

\twolineshloka
{महोदरमहापार्श्वो सार्धमेतैर्महाबलैः}
{सग्रामेऽस्मिन् रिपून हन्तुं युवां व्रजतमुद्यतौ} %॥६६॥

\twolineshloka
{दृष्टा तानागतांश्चैव युध्यमानान् रणे रिपून्}
{अनयल्लक्ष्मणः षड्भिः शरैस्तीक्ष्णैर्यमालयम्} %॥६७॥

\twolineshloka
{वानराणां समूहश्च शिष्टांश्च रजनीचरान्}
{सुग्रीवेण हतः कुम्भो राक्षसो बलदर्पितः} %॥६८॥

\twolineshloka
{निकुम्भो वायुपुत्रेण निहतो देवकण्टकः}
{विरुपाक्षं युध्यमानं गदया तु विभीषणः} %॥६९॥

\twolineshloka
{भीममैन्दौ च श्वपतिं वानरेन्दौ निजघ्रतुः}
{अङ्गदो जाम्बवांश्चाथ हरयोऽन्यान्निशाचरान्} %॥७०॥

\twolineshloka
{युध्यमानस्तु समरे महालक्षं महाचलम्}
{जघान रामोऽथ रणे बाणवृष्टिकरं नृप} %॥७१॥

\twolineshloka
{इन्द्रजिन्मन्त्रलब्धं तु रथमारुह्य वै पुनः}
{वानरेषु च सर्वेषु शरवर्षं ववर्ष सः} %॥७२॥

\twolineshloka
{रात्रौ तद्वाणाभिन्नं तु बलं सर्वं च राघवम्}
{निश्चेष्टमखिलं दृष्ट्वा जाम्बवत्प्रेरितस्तदा} %॥७३॥

\twolineshloka
{वीर्यादौषधमानीय हनुमान मारुतात्मजः}
{भूम्यां शयानमुत्थाप्य रामं हरिगणांस्तथा} %॥७४॥

\twolineshloka
{तैरेव वानरैः सार्धं ज्वलितोल्काकरैर्निशि}
{दाहयामास लङ्कां तां हस्त्यश्वरथरक्षसाम्} %॥७५॥

\twolineshloka
{वर्षन्तं शरजालानि सर्वदिक्षु घनो यथा}
{स भ्रात्रा मेघनादं तं घातयामास राघवः} %॥७६॥

\twolineshloka
{घातितेष्वथ रक्षस्सु पुत्रमित्रादिबन्धुषु}
{कारितेष्वथ विघ्नेषु होमजप्यादिकर्मणाम्} %॥७७॥

\twolineshloka
{ततः क्रुद्धो दशग्रीवो लङ्काद्वारे विनिर्गतः}
{क्वासौ राम इति ब्रूते मानुषस्तापसाकृतिः} %॥७८॥

\twolineshloka
{योद्धा कपिबलीत्युच्चैर्व्याहरद्राक्षसाधिपः}
{वेगवद्भिर्विनीतैश्च अश्वैश्चित्ररथे स्थितः} %॥७९॥

\twolineshloka
{अथायान्तं तु तं दृष्टा रामः प्राह दशाननम्}
{रामोऽहमत्र दुष्टात्मत्रेहि रावण मां प्रति} %॥८०॥

\twolineshloka
{इत्युक्ते लक्ष्मणः प्राह रामं राजीवलोचनम्}
{अनेन रक्षसा योत्स्ये त्वं तिष्ठेति महाबल} %॥८१॥

\twolineshloka
{ततस्तु लक्ष्मणो गत्वा रुरोध शरवृष्टिभिः}
{विंशद्वाहुविसृष्टैस्तु शस्त्रास्त्रैर्लक्ष्मणं युधि} %॥८२॥

\twolineshloka
{रुरोध स दशग्रीवः तयोर्युद्धमभून्महत्}
{देवा व्योम्नि विमानस्था वीक्ष्य तस्थुर्महाहवम्} %॥८३॥

\twolineshloka
{ततो रावणशस्त्राणिच्छित्वा स्वैस्तीक्ष्णसायकैः}
{लक्ष्मणः सारथिं हत्वा स्याश्वानपि भल्लकैः} %॥८४॥

\twolineshloka
{रावणस्य धनुश्छित्तआ ध्वजं च निशितैः शरैः}
{वक्षः स्थलं महावीर्यो विव्याध परवीरहा} %॥८५॥

\twolineshloka
{ततो रथान्निपत्याधः क्षिप्रं राक्षसनायकः}
{शक्तिं जग्राह कुपितो घण्टानादविनादिनीम्} %॥८६॥

\twolineshloka
{अग्निज्वालाज्वलज्जिह्वां महोल्कासदृशद्युतिम्}
{दृढमुष्ट्या तु निक्षिप्ता शक्तिः सा लक्ष्मणोरसि} %॥८७॥

\twolineshloka
{विदार्यान्तः प्रविष्टाथ देवास्त्रस्तास्ततोऽम्बरे}
{लक्ष्मणं पतितं दृष्ट्वा रुदद्भिर्वानरेश्वरैः} %॥८८॥

\twolineshloka
{दुःखितः शीघ्रमागम्य तत्पार्शं प्राह राघवः}
{क्व गतो हनुमान वीरो मित्रो मे पवनात्मजः} %॥८९॥

\twolineshloka
{यदि जीवति मे भ्राता कथचित्पतितो भुवि}
{इत्युक्ते हनुमान राजन् वीरो विख्यातपौरुषः} %॥९०॥

\twolineshloka
{बदध्वाञ्जलिं बभाषेदं देह्यनुज्ञां स्थितोऽस्मि भोः}
{रामः प्राह महावीर विशल्यकरणी मम} %॥९१॥

\twolineshloka
{अनुजं विरुजं शीघ्रं कुरु मित्र महाबल}
{ततो वेगात्समुत्पत्य गत्वा द्रोणागिरिं कपिः} %॥९२॥

\twolineshloka
{बदध्वा च शीघ्रमानीय लक्ष्मणं नीरुजं क्षणात्}
{चकार देवदेवेशां पश्यतां राघवस्य च} %॥९३॥

\twolineshloka
{ततः कुद्धो जगन्नाथो रामः कमललोचनः}
{रावण्यस्य बलं शिष्टं हस्त्यश्वरथराक्षसम्} %॥९४॥

\twolineshloka
{हत्वा क्षणेन रामस्तु तच्छरीरं तु सायकैः}
{तीक्ष्णैर्जर्जरित्म कृत्वा रस्थिवान् वानरैर्वृतः} %॥९५॥

\twolineshloka
{अस्तचेष्टो दशग्रीवः सज्ञां प्राप्य शनैः पुनः}
{उत्थाय रावणः कुद्धः सिंहनादं ननाद च} %॥९६॥

\twolineshloka
{तत्रादश्रवणैर्व्योनि वित्रस्तो देवतागणः}
{एतस्मिन्नेव काले तु रामं प्राप्य महामुनिः} %॥९७॥

\twolineshloka
{रावणे बद्धवैरस्तु अगस्त्यो वै जयप्रदम्}
{आदित्यहदयं नाम मन्त्रं प्रादाज्जयप्रदम्} %॥९८॥

\twolineshloka
{रामोऽपि जप्त्वा तन्मत्रमगस्त्योक्तं जयप्रदम्}
{तद्दत्तं वैष्णवं चापमतुलं सद्गुणं दृढम्} %॥९९॥

\twolineshloka
{पूजायित्वा तदादाय सज्यं कृत्वा महाबलः}
{सौवर्णपुङ्खैस्तीक्ष्णैस्तु शरैर्मर्मविदारणेः} %॥१००॥

\twolineshloka
{युयुधे राक्षसेन्द्रेण रघुनाथः प्रतापवान्}
{तयोस्तु युध्यतोस्तत्र भीमशक्त्योर्महामते} %॥१०१॥

\twolineshloka
{परस्परविसृष्टस्तु व्योम्नि संवर्द्धितोऽनलः}
{समुत्थितो नृपश्रेष्ठ रामरावणयोर्युधि} %॥१०२॥

\twolineshloka
{सगरे वर्तमाने तु रामो दाशरथिस्तदा}
{पदातिर्युयुधे वीरो रामोऽनुक्तपराक्रमः} %॥१०३॥

\twolineshloka
{सहस्त्राश्वयुतं दिव्यं रथं मातलिमेव च}
{प्रेषयामास देवेन्द्रो महान्तं लोकविश्रुतम्} %॥१०४॥

\twolineshloka
{रामस्तं रथमारुह्य पूज्यमानः सुरोत्तमैः}
{मातल्युक्तोपदेशस्तु रामचन्द्रः प्रतापवान्} %॥१०५॥

\twolineshloka
{ब्रह्मदत्तवरं दुष्टं ब्रह्मास्त्रेण दशाननम्}
{जघान वैरिणं क्रूरं रामदेवः प्रतापवान्} %॥१०६॥

\twolineshloka
{रामेण निहते तत्र रावणे सगणे रिपौ}
{इन्द्राद्या देवताः सर्वाः परस्परमथाबुवन्} %॥१०७॥

\twolineshloka
{रामो भूत्वा हरिर्यस्मादस्माकं वैरिणं रणे}
{अन्यैरवध्यमप्येनं जघान युधि रावणम्} %॥१०८॥

\twolineshloka
{तस्मात्तं रामनामानमनन्तमपराजितम्}
{पूजयामोऽवतीर्यैनमित्युक्त्वा ते दिवौकसः} %॥१०९॥

\twolineshloka
{नानाविमानैः श्रीमद्भिरवतीर्य महीतले}
{रुद्रेन्द्रवसुचन्द्राद्या विधातारं सनातनम्} %॥११०॥

\twolineshloka
{विष्णुं जिष्णुं जगन्मूर्तिं सानुजं राममव्ययम्}
{तं पूजयित्वा विधिवत्परिवार्योपतास्थिरे} %॥१११॥

\twolineshloka
{रामोऽयं दृश्यतां देवा लक्ष्मणोऽयं व्यवस्थितः}
{सुग्रीवो रविपुत्रोऽयं वायुपुत्रोऽयमास्थितः} %॥११२॥

\twolineshloka
{अङ्गदाद्या इमे सर्वे इत्यूचुस्ते दिवौकसः}
{गन्धामोदितदिक्चक्रा भ्रमरालिपदानुगा} %॥११३॥

\twolineshloka
{देवस्त्रीकरनिर्मुक्ता राममूर्धनि शोभिता}
{पपात पुष्पवृष्टिस्तु लक्ष्मणस्य च मूर्धनि} %॥११४॥

\twolineshloka
{ततो ब्रह्मा समागत्य हंसयानेन राघवम्}
{अमोघाख्येन स्तोत्रेण स्तुत्वा राममवोचत} %॥११५॥

\uvacha{ब्रह्मोवाच}

\twolineshloka
{त्वं विष्णुरादिर्भूतानामनन्तो ज्ञानदृक्प्रभुः}
{त्वमेव शाश्वतं ब्रह्म वेदान्ते विदितं परम्} %॥११६॥

\twolineshloka
{त्वया यदद्य निहतो रावणो लोकरावणः}
{तदाशु सर्वलोकानां देवानां कर्म साधितम्} %॥११७॥

\twolineshloka
{इत्युक्ते पद्मयोनौ तु शङ्करः प्रीतिमास्थितः}
{प्रणम्य रामं तस्मै तं भूयो दशरथं नृपम्} %॥११८॥

\twolineshloka
{दर्शयित्वा गतो देवः सीता शुद्धेति कीर्तयन्}
{ततो बाहुबलप्राप्तं विमानं पुष्पकं शुभम्} %॥११९॥

\twolineshloka
{पूतामारोप्य सीतां तामादिष्टः पवनात्मजः}
{ततस्तु जानकीं देवीं विशोकां भूषणान्विताम्} %॥१२०॥

\twolineshloka
{वन्दितां वानरेन्दैस्तु सार्धं भ्रात्रा महाबलः}
{प्रतिष्ठाप्य महादेवं सेतुमध्ये स राघवः} %॥१२१॥

\twolineshloka
{लब्धवान् परमां भक्तिं शिवे शम्भोरनुग्रहात्}
{रामेश्वर इति ख्यातो महादेवः पिनाकधृक्} %॥१२२॥

\twolineshloka
{तस्य दर्शनमात्रेण सर्वहत्यां व्यपोहति}
{रामस्तीर्णप्रतिज्ञोऽसौ भरतासक्तमानसः} %॥१२३॥

\threelineshloka
{ततोऽयोध्यां पुरीं दिव्यां गत्वा तस्यां द्विजोत्तमैः}
{अभिषिक्तो वसिष्ठाद्यैर्भरतेन प्रसादितः}
{अकरोद्धर्मतो राज्यं चिरं रामः प्रतापवान्} %॥१२४॥

\sixlineindentedshloka
{यज्ञादिकं कर्म निजं च कृत्वा}{पौरेस्तु रामो दिवमारुरोह}
{राजन्मया ते कथितं समासतो}{रामस्य भूम्यां चरितं महात्मनः}
{इदं सुभक्त्या पठतां च श्रृण्वतां}{ददाति रामः स्वपदं जगत्पतिः} %॥१२५॥

॥इति श्रीनरसिंहपुराणे रामप्रादुर्भावे द्विपञ्चाशोऽध्यायः॥५२॥

\closesection