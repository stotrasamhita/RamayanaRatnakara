\dnsub{मङ्गलश्लोकाः}

\fourlineindentedshloka
{स्वस्ति प्रजाभ्यः परिपालयन्तां}
{न्यायेन मार्गेण महीं महीशाः}
{गोब्राह्मणेभ्यः शुभमस्तु नित्यं}
{लोकाः समस्ताः सुखिनो भवन्तु}

\twolineshloka
{काले वर्षतु पर्जन्यः पृथिवी सस्यशालिनी}
{देशोऽयं क्षोभरहितो ब्राह्मणाः सन्तु निर्भयाः}

\twolineshloka
{अपुत्राः पुत्रिणः सन्तु पुत्रिणः सन्तु पौत्रिणः}
{अधनाः सधनाः सन्तु जीवन्तु शरदां शतम्}

\twolineshloka
{चरितं रघुनाथस्य शतकोटि-प्रविस्तरम्}
{एकैकमक्षरं पुंसां महापातकनाशनम्}

\twolineshloka
{शृण्वन् रामायणं भक्त्या यः पादं पदमेव वा}
{स याति ब्रह्मणः स्थानं ब्रह्मणा पूज्यते सदा}

\twolineshloka
{रामाय रामभद्राय रामचन्द्राय वेधसे}
{रघुनाथाय नाथाय सीतायाः पतये नमः}

\twolineshloka
{यन्मङ्गलं सहस्राक्षे सर्वदेवनमस्कृते}
{वृत्रनाशे समभवत् तत्ते भवतु मङ्गलम्}

\twolineshloka
{यन्मङ्गलं सुपर्णस्य विनताऽकल्पयत् पुरा}
{अमृतं प्रार्थयानस्य तत्ते भवतु मङ्गलम्}

\twolineshloka
{अमृतोत्पादने दैत्यान् घ्नतो वज्रधरस्य यत्}
{अदितिर्मङ्गलं प्रादात् तत्ते भवतु मङ्गलम्}

\twolineshloka
{त्रीन् विक्रमान् प्रक्रमतो विष्णोरमिततेजसः}
{यदासीन्मङ्गलं राम तत्ते भवतु मङ्गलम्}

\twolineshloka
{ऋषयः सागरा द्वीपा वेदा लोका दिशश्च ते}
{मङ्गलानि महाबाहो दिशन्तु तव सर्वदा}

\twolineshloka
{मङ्गलं कोसलेन्द्राय महनीयगुणाब्धये}
{चक्रवर्तितनूजाय सार्वभौमाय मङ्गलम्}

\fourlineindentedshloka*
{कायेन वाचा मनसेन्द्रियैर्वा}
{बुद्‌ध्याऽऽत्मना वा प्रकृतेः स्वभावात्}
{करोमि यद्यत् सकलं परस्मै}
{नारायणायेति समर्पयामि}
