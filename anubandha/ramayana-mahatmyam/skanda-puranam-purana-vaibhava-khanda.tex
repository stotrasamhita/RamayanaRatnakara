 श्रीस्कन्दोपपुराणोक्तम् श्रीरामायणमाहात्म्यम् 


॥ श्रीस्कन्दोपपुराणोक्तं श्रीरामायणमाहात्म्यम् ॥

॥ श्रीरामो जयति ॥

॥ अथ श्रीस्कन्दोपपुराणोक्तम् ॥

॥ श्रीरामायणमाहात्म्यम् ॥

॥ अथ प्रथमोऽध्यायः ॥

नारद उवाच-
पितामहोऽसि लोकानां पिताऽसि च मम प्रभो ।
तत् त्वां पृच्छामि लोकेश लोकानां मम चाब्जज ॥ १॥

हिताय करुणासिन्धो कृपयाऽऽख्याहि सादरम् ।
उक्तं किल त्वया ब्रह्मन् पठनाच्छ्रवणादपि ॥ २॥

श्रीरामायणकाव्यस्य चतुर्वर्गो भवेदिति ।
कथं पाठाद्भवेद्धर्मः कथं चार्थश्च सिध्यति ॥ ३॥

कथं कामो भवेत्पुंसां मोक्षो वाऽपि कथं भवेत् ।
विभवं विस्तरं चैव नियमं च सकल्पनम् ॥ ४॥

प्रब्रूहि जगतां नाथ सर्वज्ञोऽसि मतो मम ।
ब्रह्मोवाच-
श‍ृणु नारद वक्ष्यामि श्रीरामायणवैभवम् ॥ ५॥

यस्य स्मरणमात्रेण नरः पापात्प्रमुच्यते ।
त्वन्मुखेनाखिला लोका उत्तरेयु रघार्णवात् ॥ ६॥

श्रीमद्रामायणं पुण्यमितिहासं पुरातनम् ।
येऽर्चयन्ति रघूत्तंसं स्मरन्तस्ते सुखा नराः ॥ ७॥

ब्रह्मक्षत्रियविट्छूद्रैश्चतुर्वर्गफलेप्सुभिः ।
श्रोतव्यं नियमान्नित्यं श्रीरामायणमादरात् ॥ ८॥

पठनं च प्रवचनं ब्राह्मणाधीनमेव च ।
श्रीमद्रामायणाख्यस्य कल्पवृक्षस्य नारद ॥ ९॥

सप्तकाण्डाह्वयारशाखा सप्तरूपफलप्रदाः ।
पठनाच्छ्रवणं पुण्यमधिकं सम्प्रयच्छति ॥ १०॥

अर्थप्रवचनं तस्मान्महापुण्यतमं स्मृतम् ।
पुत्रपौत्रादिसम्पत्तिं सर्वरोगनिवारणम् ॥ ११॥

पत्नीपुत्त्रादिकलहनाशनं क्षेमवर्द्धनम् ।
करोति बालकाण्डस्य पठनश्रवणादिकम् ॥ १२॥

अभीष्टकार्यप्रत्यूहनाशनं धर्मवर्धनम् ।
पितापुत्रादिवात्सल्यं रामदास्येऽधिकां रुचिम् ॥ १३॥

अयोध्याकाण्डपठनं करोति रुचिरं नृणाम् ।
वियुक्तबन्धुसंश्लेषं पतिभक्तिं च योषिताम् ॥ १४॥

धर्माविरुद्धं कामानामवाप्तिं साधुषु स्पृहाम् ।
लभते वनकाण्डस्य पठनश्रवणान्नरः ॥ १५॥

अर्थावातिं च विपुलां मित्रलाभगुणानपि ।
सत्साहाय्यं सुखप्राप्तिं स्वामिभक्तिं च निर्मलाम् ॥ १६॥

किष्किन्धाकाण्डसर्गाणां पठनात् प्राप्नुयान्नरः ।
यद्यत्सङ्कल्पितं चित्ते सुकरं वाऽपि दुष्करम् ॥ १७॥

शत्रुहस्ताद्विमुक्तिं च कारागाराद्विमोक्षणम् ।
लभते सुन्दराख्यस्य काण्डस्य पठनान्नरः ॥ १८॥

दिनेदिने सप्त सर्गान् तस्य काण्डस्य यो नरः ।
पठेद्भक्त्यातु काकुत्स्थं हृदि कृत्वा नमस्य च ॥ १९॥

तस्य जन्मान्तरशतैस्सञ्चितस्यापि पाप्मनः ।
नाशं कुर्यात् सुप्रसन्नो रामो विष्णुस्सनातनः ॥ २०॥

किम्पुनश्चिन्तितार्थाप्तिं चिन्तयामः पुनः पुनः ।
युद्धकाण्डस्य पठनात् श्रवणाच्च प्रवाचनात् ॥ २१॥

सर्वशत्रुक्षयो भूयात् असाध्यं साध्यतामियात् ।
श्रीमदुत्तरकाण्डस्य पठनात् भक्तितो नरः ॥ २२॥

जीवन्मुक्तस्स विज्ञेयो राघवस्य प्रसादतः ।
न पुनस्तु जायेत कर्म भोक्तुं स्वयं कृतम् ॥ २३॥

उषित्वा स तु वैकुण्ठे कार्यब्रह्मसमीपतः ।
ब्रह्मज्ञानं प्रपन्नोऽथ विष्णोस्सायुज्यमाप्नुयात् ॥ २४॥

येषां गृहेषु तिष्ठेत श्रीरामायणपुस्तकम् ।
यक्षरक्षः पिशाचाद्या भिया धावन्ति तद्गृहात् ॥ २५॥

श्रीरामायणकोशस्य यैः पूजा क्रियतेऽऽदरात् ।
तस्य देवास्तु निखिलाः प्रसीदन्ति स्वयं सुत ॥ २६॥

यैर्नित्यं पठ्यते भक्त्या दिनविच्छित्तिशून्यकम् ।
गच्छन्तमनुगच्छेत्तं श्रीरामः करुणावशः ॥ २७॥

पाठः काम्यश्च नित्यश्च मोक्षार्थश्चेति स त्रिधा ।
सङ्कल्प्यानिष्टनाशं वा इष्टावाप्तिं मनोरथे ॥ २८॥

श्रीरामायणपाठस्तु काम्यस्स परिकीर्तितः ।
काण्डस्य सुन्दराख्यस्य सप्तसर्गीविधानतः ॥ २९॥

सम्पूर्णस्यापि षट्काण्डसम्मितस्य विधानतः ।
पुनर्वसौ समारम्भः सम्पूर्तिश्च पुनर्वसौ ॥ ३०॥

चैत्रे वाऽश्वयुजे मासि नवरात्रे महोत्सवे ।
पठनं कुरुते विप्रः कामानां परिपूर्तये ॥ ३१॥

सङ्गवादौ समारभ्य मुहूर्तत्रयपाठनम् ।
काम्यपाठ इति प्रोक्तो नित्यपाठमथो श‍ृणु ॥ ३२॥

दिने दिने त्वविच्छिच्या मनसा द्रव्यऽतोऽपि वा ।
सम्पूज्य सीतारमणं तं ध्यात्वा पुस्त के धिया ॥ ३३॥

नित्यकर्माविरोधेन प्रातः सङ्गव एव वा ।
पठेद्रामायणं पुण्यमुच्चैः सुस्वरमेव च ॥ ३५॥

अनश्नन्नेव च पठेद्राघवस्य कृपाप्तये ।
पठेत्तु सर्गान्दश वा पञ्च वा त्रीनथापि वा ॥ ३६॥

एकं वाऽपि पठेत्सर्ग प्रयतो नान्यमानसः ।
उत्तमं दश सर्गास्तु मध्यमं सर्गपञ्चकम् ॥ ३७॥

अधर्म तु त्रयःसर्गा एकसर्गोऽधमाधमः ।
न सर्गमध्ये विरमेद्यदि दैवाद्विरम्यते ॥ ३८॥

पुनस्तदादिमारभ्य पठेन्नियमपूर्वकम् ।
एवं कृते फलं यत्स्यात्तद्वदिष्यामि नारद ॥ ३९॥

दारिद्यनाशो रोगाणां द्रावणं सुखवर्धनम् ।
भवेञ्जन्मान्तरे चैव यत्पापं रचितं नृभिः ॥ ४०॥

तच्चापि नश्यति क्षिप्रं ज्ञानं स्यान्मोक्षसाधनम् ।
न पुनर्जायते सोऽयं सवित्रीयोनितो भुवि ॥ ४१॥

मोक्षार्थपाठं वक्ष्यामि श‍ृणु वीणामुने शुभम् ।
दिवा भुक्तत्यन्तरं विप्रस्त्वपराह्वे यतव्रतः ॥ ४२॥

एकावरं तु श्रोतारं कल्पयित्वा शुचिव्रतम् ।
अखादित्वा तु ताम्बूलमपीत्वाऽपि जलं शुचिः ॥ ४३॥

यथाशक्ति च सम्पूज्य श्रीरामायणपुस्तकम् ।
एकावरान् पठेत्सर्गान् विरक्तस्तु त्रिवर्गक ॥ ४४॥

न सम्भाषेत दारैर्वा पुत्रैर्वा पाठमध्यतः ।
आरभ्य नारदप्रोक्तं सर्गं वाल्मीकये पुरा ॥ ४५॥

यावदुत्तरकाण्डान्तं पठेदेवं क्रमेण तु ।
चतुर्विंशतिसाहस्त्र श्लोकमण्डलमादरात् ॥ ४६॥

पठित्वा तु समाप्याथ ब्राह्मणान् भोजयेद्धहून् ।
प्रतिसाहस्रकश्लोकमेकैकं भोजयेद्द्विजम् ॥ ४७॥

श्रीरामजनने चैव सीतायाः पाणिपीडने ।
पादुकाराज्यला मे च जटायो र्मोक्षसङ्गतौ ॥ ४८॥

सुग्रीवराज्यलाभे च तरणेऽब्धेर्हनूमतः ।
अङ्गुलीयकदाने च विभीषणसमागमे ।
विभीषणस्य राज्यार्थं लङ्कायामभिषेचने ॥ ४९॥

रामपट्टाभिषेके च कारयेदुत्सवं प्रमोः ।
श्रीमदुत्तरकाण्डस्य समाप्तौ तु द्विजोत्तम ।
महोत्सवं प्रकुर्वीत रामस्य परमात्मनः ॥ ५०॥

नित्यपाठे च पाठे च मुमुक्षूणां यथाक्रमम् ।
काण्डास्सप्तापि पाठ्यारस्यु रन्यथा निष्फलं भवेत् ॥ ५१॥

काम्यपाठे तु षट्काण्डमितं रामायणं पठेत् ।
न तत्रोत्तरकाण्डञ्च पठने श्रेयसां क्षतिः ॥ ५२॥

रामायणस्य पठनं यदाऽऽरभ्येत हि क्वचित् ।
तद्ग्रन्थस्य पुरस्तात्तु कोऽपि नासीत सर्वथा ॥ ५३॥

श‍ृणोति तत्र ह्यासीनो हनुमान् राधवीं कथाम् ।
स्वस्थाने यस्समासनिस्तं शपेत् स तु कोषतः ॥ ५४॥

सर्वशास्त्रार्थपारज्ञो भक्तियुक्तो रघूत्तमे ।
वाल्मीकिहृदयाभिज्ञश्शब्दसारार्थसारवित् ॥ ५५॥

प्रवक्तृत्वेऽधिक्रियेत ह्यन्यथा हनुमान् शपेत् ।
श्रोतृभिश्च प्रवक्ता तु पूजनीयः प्रयत्नतः ॥ ५६॥

तं च मत्वा रघूत्तंसं आदृताः श‍ृणुयुः कथाम् ।
दुग्धेन मधुनाऽऽज्येन भोजयेतं दिनेदिने ॥ ५७॥

वस्त्रैराभरणैश्चित्रैः स्तुतिभिश्चापि पुष्कलैः ।
सम्मानयेत् प्रवक्तार मादावन्ते च मध्यतः ॥ ५८॥

एतदुद्देशतः प्रोक्तं श्रीरामायणवैभवम् ।
यः पठेच्छृणुयाद्वाऽपि भक्त्याऽध्यायमिमं नरः ।
श्रीरामस्य प्रसादस्य पात्रं स भवति ध्रुवम् ॥ ५९॥

-इति श्रीस्कन्दोपपुराणे पुराणवैभवखण्डे ब्रह्मनारदसंवादे
रामायणपठनश्रवणविधिनिरूपणं नाम त्रयोविंशोऽध्यायः ॥ २३

॥ इति प्रथमोऽध्यायः ॥

॥ श्रीरामो जयति ॥

॥ अथ द्वितीयोऽध्यायः ॥

युधिष्ठिर उवाच-
भगवन् सर्वधर्मज्ञ अपारकरुणानिधे ।
रामायणेतिहासस्य काम्यपाठक्रमं वद ॥ १॥

कथं वा कामिभिः पाठ्यं कं वा नियममाश्रितैः ।
एतदाख्याय मां सद्यः पावय त्वं महामुने ॥ २॥

दाल्भ्य उवाच-
सम्यक्पृष्टं महाराज रामदिव्यकथामृतम् ।
पुनीयावां च मां चैतान् रामायणमहानदी ॥ ३॥

चतुराननवक्त्राब्जात् श्रीरामायणवैभवम् ।
श्रुत्वा जगाद धर्मात्मन् नारदो भगवानृषिः ॥ ४॥

नारद उवाच -
रामस्य चरितं प्रोक्तं मया वाल्मीकये पुरा ।
श्रुत्वा मत्तस्तु निखिलं मुनिः काव्यं चकार हि ॥ ५॥

प्रोच्यते तस्य काव्यस्य माहात्म्यं कथमीदृशम् ।
अन्ये च मुनयः केचिद्रामस्य चरितं शुभम् ॥ ६॥

निबध्नन्ति स्वकैर्वाक्यैर्नतेऽत्र कथमादरः ।
प्रमार्जयैनं सन्देहं चरणाब्जं नतोऽस्मि ते ॥ ७॥

ब्रह्मोवाच -
नतेऽत्त्र संशयः कार्यः वत्स सर्वात्मना मुने ।
सुबहूनि निबध्नन्ति काव्यानि बहवो भुवि ॥ ८॥

श्रीराघवकथाऽऽढ्यानि युक्तियुक्तानि चैव हि ।
न मे तत्रादरं चेतः करोति मधुरेष्वपि ॥ ९॥

वाल्मीकिरुपदेशं तु गृहीत्वा तव बुद्धिमान् ।
भक्तियुक्तो रघूत्तंसे द्रुतचित्तोऽभवत्तदा ॥ १०॥

राघवः परमात्माऽथ रहस्याहूय मां प्रभुः ।
आदिदेश स्वचरितं प्रकटीकर्तुमुद्यतः ॥ ११॥

श्रीराम उवाच -
चतुरास्य तवेदानीं कार्यं किमपि गोपितम् ।
तत्ते कर्तव्यमचिरान्मम प्रियचिकीर्षया ॥ १२॥

संसारसिन्धुतंरणं जनाः कर्तुमनीश्वराः ।
मत्कथासंज्ञितां नावं प्राप्य नन्दन्तु निर्वृताः ॥ १३॥

इति मे बुद्धिरुत्पन्ना तदर्थं चाहुतो भवान् ।
अवतारक्रिया पूर्णा रावणो निहतो रिपुः ॥ १४॥

भोक्ष्ये त्वच्छापमित्येव भृगवेऽहं वरं ह्यदाम् ।
तदर्थं यत्त्रमास्थास्ये इति कार्यद्वयं मतम् ॥ १५॥

श‍ृङ्गारविरळं दुःखशोकायासनिरन्तरम् ।
मच्चरित्रं यथावतु कस्समर्थो निबन्धितुम् ॥ १६॥

वाल्मीकिर्नारदमुने राकर्ण्य चरितं मम ।
द्रवच्चित्तोऽभवत्सद्यः कौमुद्या चन्द्रकान्तवत् ॥ १७॥

मन्येऽहं मच्चरित्रस्य ग्रथने तस्य योग्यताम् ।
खरादिनिधनाद्भीतः शङ्कुकर्णो निशाचरः ॥ १८॥

दण्डकारण्यगो भूत्वा क्रौञ्चरूपसमाश्रयः ।
बाधतेऽद्य मुनींस्तस्मात्तस्य कार्यो वधो मया ॥ १९॥

मन्मायामोहितस्सोऽथ दुष्टो वाल्मीकिसन्निधौ ।
क्रीडिष्यति समं वध्वा रूपे तदनुरूपया ॥ २०॥

तदा निषादरूपोऽहं वधिष्यामि निशाचरम् ।
स्त्रीणामवध्यता यस्मात्तस्मात्तां परिवर्जये ॥ २१॥

वाल्मीकिस्तु स्वया बुध्या द्रुतश्चेदस्य दर्शनात् ।
शप्स्यते मां र्स्त्रीवियोगी भवेति रुषितः किल ॥ २२॥

इष्टं मे कार्यमेतत्स्यात्तस्य दातुं हि गौरवम् ।
अतस्सरस्वतीं त्वं तु प्रेरयास्याऽऽननं प्रति ॥ २३॥

मच्चरित्रं तवादेशा दथासौ ग्रथयिष्यति ।
तच्च श्रोष्यामि कालेन स्वयमेव वृतोऽखिलैः ॥ २४॥

मच्चरित्रपरीक्षायां साधुवाऽसाधुवेति हि ।
प्रवक्तुं कोऽर्हति पुमान् मां विना प्रतिपादितम् ॥ २५॥

प्रायेण चरितं मे हि वनेषु चलितं त्वभूत् ।
तद्गच्छ प्रेरय गिरां देवीमस्याननं प्रति ॥ २६॥

इति प्रतिसमादिष्टो वाणीं सम्प्रैरयं ततः ।
वाल्मीकिर्नियमायाथ तमसामगमन्नदीम् ॥ २७॥

तस्मिन् पश्यति तत्रैव वृक्षे कुत्रापि दम्पती ।
कौञ्चौ चिक्रीडतुर्हर्षात् कामबाणवशानुगौ ॥ २८॥

तयोः पुमांसं वेगेन व्याधो विव्याध बाणतः ।
तदुद्वीक्ष्य महाप्राज्ञो वाल्मीकिस्सुमहातपाः ॥ २९॥

रुवन्त्यां करुणं तत्र क्रौञ्चयां निर्भिण्णमानसः ।
सुरतासक्तयोर्घातो न धर्म्य इति जज्ञिवान् ॥ ३०॥

शशाप समुनिस्तत्र व्याधमाकृष्टकार्मुकम् ।
``मानिषाद प्रतिष्ठां त्वमगमश्शाश्वतीस्समाः ॥ ३१॥

यत्क्रौञ्चमिथुनादेकमवधीः काममोहितं'' ।
शपन्निति मुनिः क्रोधादुत्तरक्षण एव हि ॥ ३२॥

नापश्यत निषादं तं तां क्रौञ्च तं च वृक्षकम् ।
चिन्तयामास च मुनिर्मायेयं कस्य चिद्भवेत् ॥ ३३॥

अन्यथा हि कथं सर्व दृष्टनष्टं भवेदिदम् ।
इति चिन्ताऽऽकुलः स्नात्वा समाप्य नियमांश्च सः ॥ ३४॥

स्वाश्रमं पुनरायातः परिक्लिष्टेन चेतसा ।
ततोऽहमगमं तत्र स्मरन्नाज्ञां रघुप्रभोः ॥ ३५॥

सावित्र्या च सरस्वत्या हंसयुक्तरथे स्थितः ।
अभ्यर्चितश्च तेनाथ ससम्भ्रममहं तदा ॥ ३६॥

वाक्यं वल्मीकजन्मानमब्रवं रचिताञ्जलिम् ।
ब्रह्मोवाच -
ब्रह्मन् वल्मीकसम्भूत माभूश्चिन्तासमाकुलः ॥ ३७॥

कृता माया भगवता परीक्षार्थं हृदस्तव ।
परदुःखासहिष्णुत्वात् द्रुतचित्तोऽसि साम्प्रतम् ॥ ३८॥

मयैव प्रेरिता तुभ्यमियं देवी सरस्वती ।
स्वयं मुखान्निर्गता ते ज्ञातस्ते वचसो रसः ॥ ३९॥

ईन्दृशैरेव सरसैः पद्यैरन्यैश्च भूरिशः ।
कुरु रामकथां पुण्यां यथा ते नारदाच्छ्रुतम् ॥ ४०॥

रामत्वेनावतीर्णस्य हरेरद्भुतकर्मणः ।
जन्मारभ्य स्वधामाप्तिपर्यन्तां सरसां कथाम् ॥ ४१॥

पुण्यां सुललितार्थाञ्च अतीतानागताश्रयाम् ।
आश्रित्य सुमहत्काव्यं सर्गबन्धोज्वलं कुरु ॥ ४२॥

वाल्मीकिरुवाच -
चतुरानन लोकेश सर्वधर्मप्रवर्तक ।
निमग्नोऽस्मि सुधासारे श्रीरामचरिताह्वये ॥ ४३॥

जाने पुरुषमीशानं रामं रघुकुलोद्भवम् ।
आचार्यानुपदिष्टाया विद्याया वीर्यशून्यताम् ॥ ४४॥

विज्ञाय नारदमुनेः प्राप्तं रामकथामृतम् ।
यदि शक्तोऽस्मि रामस्य चरितं पुण्यवत्तमम् ॥ ४५॥

काव्यात्मना सङ्ग्रथितुं यदि चानुग्रहस्तव ।
वक्ष्यामि रघुनाथस्य कथां काव्यात्मना शुभाम् ॥ ४६॥

तवानुग्रहतो भूयात् काव्ये मे नानृतं क्वचित् ।
नच मिथ्याप्रलपितं प्रब्रूयुर्मा बुधा भुवि ॥ ४७॥

श्लोकानामखिलानां च ताप्तर्यं सुमहद्भवेत् ।
रसाः श‍ृङ्गारकारुण्यहास्यवीरादयोऽखिलाः ॥ ४८॥

तत्र तत्र मयोद्दिष्टा भूयासुश्च यथोचितम् ।
न कश्चिदपशब्दो मे न च किञ्चिदृथा पदम् ॥ ४९॥

श्रीरामचरिते भूयान्मम काव्यं प्रमाणकम् ।
नान्यत्किञ्चित्तथा भूयान्मत्काव्यस्यार्चनीयता ॥ ५०॥

पठनश्रवणाभ्यां मे काव्यस्य सकृदप्यहो ।
पुरुषार्थाच सर्वेऽपि भूयासुर्भुवि सर्वतः ॥ ५१॥

ब्रह्मोवाच -
वल्मीकजन्मन् सुमुने रामनामप्रभावतः ।
शापात्तवं नीचवृत्तिस्थोऽप्येवं जातो मुनीश्वरः ॥ ५२॥

नाम्न एवं प्रभावश्चेद्रामस्य परमात्मनः ।
किमु तच्चरिताम्भोधिप्लावनादघनाशने ॥ ५३॥

वाल्मीकिरुवाच -
न ज्ञायते कथं ब्रह्मन् रामनामप्रभावतः ।
एवं मुनिरहं जातश्शापश्च मम कीदृशः ॥ ५४॥

ब्रह्मोवाच-
वर्णने त्वच्चरित्रस्य न कालोऽयं तथाऽप्यहम् ।
तवौत्सुक्यनिवृत्त्यर्थं सङ्क्षेपात् प्रब्रवीमि ते ॥ ५५॥

वरुणस्य सुतः पूर्वं त्वमभूर्हरिताभिधः ।
कदाचित्तव गेहे तु वरुणेन समाहुताः ॥ ५६॥

मुनयस्सनकाद्यास्तु सङ्गता ब्रह्मवादिनः ।
सञ्जजलपुर्मिथस्तत्र श्रुतिस्मृत्याश्रयाः कथाः ॥ ५७॥

प्रसङ्गेन च रामस्य हरेरवतरिष्यतः ।
वधाय दशकण्ठस्य कथां ते चक्रिरे तथा ॥ ५८॥

क्रीडापरवशस्त्वं तु तदाऽभू गृहबर्हिणा ।
तत्केकाभिश्च ते हासैः कथाविघ्नो बभूव ह ॥ ५९॥

सनत्कुमारः कुपितश्शशाप त्वां तदा किल ।
यस्माल्लोकोत्तरकथाविघ्नाय किल दर्पतः ॥ ६०॥

पक्षिणा क्रीडसे तस्माव्द्याधत्वं तु गमिष्यसि ।
इत्युक्ते वचने तेन सम्भ्रान्ता मुनयोऽखिलाः ॥ ६१॥

त्वं तु निश्चेतनो जातो वरुणो दुःखितोऽभवत् ।
प्रसादयामास मुनिं पिता ते शापशान्तये ॥ ६२॥

ततः प्रसन्नस्स मुनिः पुनस्त्वां वाक्यमब्रवीत् ।
तव भ्रातुर्भृगोर्वंशे प्रचेता नाम वै मुनिः ॥ ६३॥

आस्तेऽद्य दण्डकारण्ये तपः कुर्वन् सुदुश्चरम् ।
भवितासि सुतस्तस्य वेदवेदाङ्गपारगः ॥ ६४॥

कालेन व्याधवृत्तिश्च भवितासि गिरा मम ।
सप्तर्षीणां प्रसादेन रामनामप्रभावतः ॥ ६५॥

पूतो भूत्वा तपः कृत्वा मुनिवर्यत्वमेष्यसि ।
इत्युक्त्वा स मुनिर्यातस्त्वं च प्राचेतसोऽभवः ॥ ६६॥

वेदवेदाङ्गनिष्णातो ह्यभूस्त्वं बाल्य एव च ।
समित्कुशादीनाहर्तुं जातु त्वं वनमध्यगः ॥ ६७॥

बहूनानायतो बद्धान् बध्यमानांश्च वै मृगान् ।
पश्यंस्त्वं रममाणोऽभूर्व्याधेन मुनिशापतः ॥ ६८॥

आहूतोऽपि च पित्रा त्वं नाध्यगच्छः स्वमाश्रमम् ।
नैषाद्या वर्तमानस्त्वं वृत्त्या हिंसन् मृगान् खगान् ॥ ६९॥

परिणीय च भिल्लानां कन्यां कामपि संयतः ।
पुत्रानुत्पादयंस्तस्यां भरंस्ताञ्च मृगैः खगैः ॥ ७०॥

कुटुम्बभरणात् खिन्नः बहुवर्षसहस्रतः ।
जातु त्वं वनमध्यस्थ स्तरुस्थं ह्यवधीः खगम् ॥ ७१॥

पलायितस्सच खगस्सच बाणोऽपतत् त्वयि ।
तेन विद्धांसदेशस्त्वं रुदितो वेदनावशात् ॥ ७२॥

तदा सप्तर्षयो मार्गागतास्त्वां वीक्ष्य कम्पिताः ।
दैन्यराशिं त्वामथोचु दययाऽभिपरिप्लुताः ॥ ७३॥

अहो मूर्खत्वमेतत्ते विप्रोभूत्वा कुलोद्भवः ।
ईदृशीं वृत्तिमापन्नोऽस्यरे निष्कृतिवर्जिताम् ॥ ७४॥

यथा त्वं खिद्यसे मूढ बाणेनानेन विध्यता ।
तथा मृगविहङ्गानां व्यथा किं नेति ते मतिः ॥ ७५॥

स्वमांसं परमांसेन यः पुष्णाति नराधमः ।
रुरवो भक्षयिष्यन्ति तस्य मांसं यमालये ॥ ७६॥

न त्वं निर्विद्यसे मूढ अपीमां वेदनां गतः ।
इत्युक्ते वचने तैस्तु करुणार्द्रितमानसैः ॥ ७७॥

कालयोगातु निर्विण्णस्सञ्जातस्त्वं मृगात्ययात् ।
तानब्रवीः प्राञ्जलिस्त्वं तेभ्यो नत्वाऽथ दूरगः ॥ ७८॥

अहो अधन्यता मेस यदहं विश्वंशजः ।
भूत्वाऽपि वृत्तिमापन्नो मृगमांसैः स्वयं हृतैः ॥ ७९॥

पश्चात्तप्तोऽस्म्यहं श्रेष्ठाः प्रायश्चित्तं वदन्तु मे ।
सप्तर्षय ऊचुः -
तुष्टा वयं त्वद्विनया दिच्छामस्त्वत्समुद्धृतिम् ॥ ८०॥

किन्तु त्वं पतितो जातः प्रायश्चित्तं न ते श्रुतम् ।
वक्ष्यामोऽथापि कृपया त्वं कुरुष्व यथोदितम् ॥ ८१॥

पात्रापात्रविचारो न रामनाम्नां प्रकीर्तने ।
श्रवणे चोपदेशे च तस्मात्तत्तेऽद्य तु क्षमम् ॥ ८२॥

रामनाम समुच्चार्य यस्मात् पापान्न मुच्यते ।
त्रिलोक्यामस्ति तन्नैव सत्यमेतत् प्रकीर्तितम् ॥ ८३॥

जातोऽद्यैव रघोर्वंशे रामनाम वहन् हरिः ।
त्वं तदाम्नाय भक्त्याऽद्य मोक्ष्यसे सर्वपातकैः ॥ ८४॥

उपविश्याश्वत्थमूले कामक्रोधौ विसृज्य च ।
आजानुबाहुं दीर्घाक्षं नीलकुञ्चितमूर्द्धजम् ॥ ८५॥

प्रगृह्म वामहस्तेन चापं सव्येन सायकम् ।
प्रावृषेण्यपयोवाहस्वच्छकान्तिमनोहरम् ॥ ८६॥

निष्काभरणविस्तीर्णंवक्षसं वनमालिनम् ।
सुप्रसन्नमुखाम्भोजं सुमृष्टमणिकुण्डलम् ॥ ८७॥

चिन्तयन् राघवं बुध्या रामनाम जपाऽऽदृतः ।
उपकृत्येति कृपया निवृत्तास्ते स्वकं पदम् ॥ ८८॥

त्वं ततः पुत्रदारादिचिन्तामुत्सृज्य दूरतः ।
रामरामेति रामेति जपन् निश्चलतां गतः ॥ ८९॥

कालेनाल्पेन ते देहं वल्मीकः प्रावृणोन्महान् ।
अतीते वत्सरे काले त्वस्थिमात्रावशेषितम् ॥ ९०॥

जपन्तं रामरामेति वल्मीकाच्छादिताङ्गकम् ।
रामनामजपान्मुक्तपातकं त्वां विचिन्तयन् ॥ ९१॥

अहं कमण्डलुजलैरभ्युक्ष्य त्वां न्यबोधयम् ।
कायं च पूर्ववत् कृत्वा तपसे त्वां समैरयम् ॥ ९२॥

स्ववृत्तिमनुचिन्त्य त्वं यतः खिन्नोऽभवः पुरा ।
अतीतानां विस्मरणं ततोऽहमददां तव ॥ ९३॥

ततश्शुद्धात्मतां प्राप्य मुनिश्रेष्ठत्वमागतः ।
यत्त्वं वल्मीकतो जातो वाल्मीकिरिति तद्भवान् ॥ ९४॥

इदमेव नवं जन्म तव कृत्वाऽहमाह्वयम् ।
वाल्मीकिरिति तेन त्वं ख्यातो नाम्नाऽमुनैव ते ॥ ९५॥

ख्यातो लोके सर्वलोकैः रामनाम्नाऽसि सम्मतः ।
तस्मात्वं कुरु रामस्य चरितं काव्यरूपतः ॥ ९६॥

न ते वागनृता काव्ये काचिदत्र भविष्यति ।
रसपुष्टिर्भावपुष्टिः श्लोकानां ते भविष्यति ॥ ९७॥

नचापशब्दो भविता न चार्थगुणविच्युतिः ।
प्रमाणं रामचरिते काव्यं तव भविष्यति ॥ ९८॥

रहस्यं च प्रकाशञ्च रामस्य चरितं महत् ।
तवैव विदितं भूयात् नान्यो ज्ञास्यति तत्वतः ॥ ९९॥

इत्युक्ते च मया तत्र वाक्यमूचे सरस्वती ।
सरस्वत्युवाच -
मा ते भूत् काव्यकरणे भयं शब्दार्थसंश्रयम् ॥ १००॥

यथा यथा कर्षसि मां शब्दतश्चार्थतोऽपि वा ।
तथा तथाऽनुयास्ये त्वां त्वदाशयवशानुगा ॥ १०१॥

तव तं विनैवाऽहं रसं भावं च पोषये ।
त्वन्मूलतः कृतार्था स्यामुक्त्वा रामकथां मुने ॥ १०२॥

त्वन्मार्गालम्बनाद्यै वै कवयो राघवीं कथाम् ।
विवक्षन्तीह तेषां च भविष्यामि वशानुगा ॥ १०३॥

पश्चात् कवीनामाधारं काव्यं तव भविष्यति ।
रञ्जयिष्यन्ति लोकांस्तु कवयस्त्वत्पथानुगाः ॥ १०४॥

इत्युक्तवत्यां वाग्देव्यां सावित्री वाक्यमब्रवीत् ।
सावित्र्युवाच -
कुरु रामकथां पुण्यां काव्यरूपां मनोरमाम् ॥ १०५॥

प्रविष्टा स्याञ्च ते काव्ये प्रतिश्लोकसहस्रकम् ।
विभक्तावयवा भूत्वा चतुर्विंशतिधा ह्यहम् ॥ १०६॥

काव्यचिन्तैकमनसा त्वया चेदनुपस्थिता ।
न ते कुप्यामि बुद्धेश्व प्रसादं ते ददाम्यहम् ॥ १०७॥

त्वन्मार्गालम्बतो ये स्यू रामस्य चरितं शुभम् ।
काव्मात्मनोद्यताः कर्तुं यशसे वा धनाय वा ॥ १०८॥

प्रसीदामि च तेभ्योऽपि तेन मे स्यात् कृतार्थता ।
ब्रह्मोवाच-
इत्येवमनुगृह्यैनं निवृत्ताः स्मो वयं मुने ॥ १०९॥

अथायं ज्ञानतो वीक्ष्य कृत्स्नां रामस्म सत्कथाम् ।
राघवस्य प्रसादेन अस्माकं च विशेषतः ॥ ११०॥

रामायणं महाकाव्यं चकार चरितव्रतः ।
चतुर्विंशतिसाहस्रं श्लोकानामुक्तवानृषिः ॥ १११॥

गायत्र्याद्याक्षरं त्वादौ विन्यस्यारब्धवांस्ततः
प्रतिश्लोकसहस्त्रस्य समाप्तावेकमक्षरम् ॥ ११२॥

विनियुञ्जंश्च गामत्र्या न्यबन्धाद्राघवीं कथाम् ।
त्रिवर्गमात्रशरणशरण्यां प्रथमं मुनिः ॥ ११३॥

श्रीरामस्म महीप्राप्तिपर्यन्तां निर्ममे कथाम् ।
ततश्चतुर्वर्गपरैस्सेवनीयां प्रयत्नतः ॥ ११४॥

स्वधामप्राप्तिपर्यन्तां राघवस्याकरोत्कथाम् ।
तस्मादुक्तं नित्यपाठे मुमुक्षूणां च पाठके ॥ ११५॥

सप्तापि काण्डाः पाठ्यास्स्युष्पट्काण्डाः फल काम्यया ।
तस्मान्नारद वाल्मीकिप्रोक्ता रामस्य सत्कथा ॥ ११६॥

प्रशस्ता सर्वलोकेषु स च तत्करणान्मुनिः ।
उपदेष्टृत्वाच्च तस्याः कथायास्ते यशःकृते ॥ ११७॥

आद्ये सर्गे वर्णितस्त्वं श्लाघ्यस्त्वं तेन भावितः ।
एवं त्वया कृताः प्रश्नास्सम्यगुत्तरिता मया ।(। ११८॥)
मुने रामायणं कृत्स्नं चर गायन्नितः परम् ॥ ११८ ॥

-इति श्रीस्कन्दोपपुराणे पुराणवैभवखण्डे ब्रह्मनारदसंवादे
रामायणप्रणयनकथनं नाम चतुर्विंशोऽध्यायः ॥ २४

॥ इति द्वितीयोऽध्यायः ॥

॥ श्रीरामो जयति ॥

॥ अथ तृतीयोऽध्यायः ॥

नारद उवाच -
तात प्रीतोऽस्मि सञ्जातो रामायणकथाश्रवात् ।
इतःपरं तां गास्यामि वीणाहस्तश्चरन् मुहुः ॥ १॥

मन्ये पूज्यतमं लोके वाल्मीकिं मुनिपुङ्गवम् ।
श्रीरामायणवेदस्य काम्यपाठः कथंविधः ॥ २॥

लोकनाथ मम ब्रूहि शुश्रूषोः प्रणतस्य च ।
ब्रह्मोवाच-
धन्योऽसि नारद मुने वक्ष्यामि श‍ृणु ते हितम् ॥ ३॥

श्रीरामायणवेदस्य काम्यपाठविधिक्रमं
चतुर्विधः काम्यपाठस्तत्तत्कालोचितः स्मृतः ॥ ४॥

श्रीमत्सुन्दरकाण्डस्य सप्तसर्गप्रपाठनम् ।
सर्वाभीष्टप्रदं प्रोक्तममराणामपीहितम् ॥ ५॥

शुभे दिवसनक्षत्रे सङ्कल्प्य विधिपूर्वकम् ।
यथाशक्ति सुवर्णेन रजतेनाथ ताम्रतः ॥ ६॥

रामस्य लक्ष्मणस्यापि वैदेह्याश्च हनूमतः ।
विधाय प्रतिमास्ताश्च प्रतिष्ठाप्य यथाविधि ॥ ७॥

पूजयेद्दर्भपुञ्जे वा त्वरा चेद्रघुनन्दनम् ।
सहस्रनामभिर्विष्णोरभ्यर्चंश्च दिनेदिने ॥ ८॥

पठेत्सुन्दरकाण्डस्य सप्तसर्गी विधानतः ।
यदा तु सप्तमस्सर्गः काण्डस्यान्तिमसर्गकः ॥ ९॥

भविष्यति तदन्तं तु सप्त सर्गान् पठेद्रती ।
अथ श्रीरामचन्द्रस्य सर्गं राज्याभिषेचने ॥ १०॥

पठित्वा पूजयित्वा च पाठनं तु समापयेत् ।
समाप्तिदिवसे कुर्यात् बहुब्राह्मण भोजनम् ॥ ११॥

यद्यद्विचिन्तितं चित्ते तत्सर्वं तु समश्नुते ।
अथ रामायणस्यास्य पाठं वक्ष्ये पुनर्वसौ ॥ १२॥

भुक्तिमुक्तिफलं सूते पाठस्तस्य पुनर्वसौ ।
निर्माय मण्डपं दिव्यं नानालङ्कारशोभितम् ॥ १३॥

रत्रैर्वा मौक्तिकैर्वापि धातुभिर्वा विशेषितैः ।
कृत्वा चित्रा रङ्गवल्लीस्तत्र वेदिं विधाय च ॥ १४॥

अधोव्रिहीन प्रस्थमात्रान् तुषहीनान् प्रसार्य च ।
तावन्मात्रान् चन्द्रशुभ्रान् तदुपर्यपि तण्डुलान् ॥ १५॥

ततश्चणकखण्डांश्च निस्तुषान् प्रस्थमात्रकान् ।
ततस्तिलान् प्रस्थमात्रान् ततो माषांस्तथाविधान् ॥ १६॥

तदुपर्यपि गोधूमैः मुद्गैर्वा हरितैर्भृतम् ।
घटं संस्थाप्य तच्छृङ्गे नारिकेळफलोपरि ॥ १७॥

सुवर्णप्रतिमां सीतारामलक्ष्मणशोभिनीम् ।
निधाय च प्रतिष्ठाप्य सर्वोपकरणैर्युतः ॥ १८॥

पूजां पुरुषसूक्तेन विधाय च यथाविधि ।
अनुज्ञाप्य द्विजानग्रथान् नान्दीश्राद्धं विधाय च ॥ १९॥

हिरण्येनैव पूर्णेन तोषकेण द्विजन्मनाम् ।
श्रीरामायणकोशेऽथ रामपूजां विधाय च ॥ २०॥

पुनर्वसुदिने शुद्धे काले सङ्गवनामके ।
आञ्जनेयस्वरूपांश्च श्रोतॄन् भक्तान् रघूत्तमे ॥ २१॥

वृत्वा यथासम्भवञ्च कृत्वा चित्तं रघूत्तमे ।
नारदेनोपदिष्टार्थसर्गमारभ्य भक्तितः ॥ २२॥

उच्चैः स्वरं रागयुक्तं पठेत् सर्गास्तु विंशतिम् ।
स्वयमेव पठेच्छक्तो यदि सर्गानविक्लवम् ॥ २३॥

नचेद्विप्रमुखेनैव पठेत्तत्स्यात्तु मध्यमम् ।
पठितं यच्च भुक्तेन यच्च व्यग्रेण वा पुनः ॥ २४॥

यच्चास्पष्टपदं भूयात् तत्सर्वं निष्फलं स्मृतम् ।
मध्याह्ने तु समायाते पाठं कृत्वा समाप्य च ॥ २५॥

पञ्चोपचारैस्सम्पूज्य गुडान्नादिनिवेदनैः ।
ब्राह्मणान् भोजयेत्पश्चात् दधिमध्वाज्यसंयुतम् ॥ २६॥

भोजनानन्तरं सर्वैस्सह विश्रम्य किञ्चन ।
अघृत्वाऽपि च ताम्बूलमनाभाष्य च पुल्कसैः ॥ २७॥

वक्तारं ब्राह्मणश्रेष्ठं सर्वशास्त्रार्थपारगम् ।
वाक्यार्थचतुरं दान्तमलुब्धं सत्यवादिनम् ॥ २८॥

नानोपचारैस्सम्पूज्य तोषयित्वा च तं धनैः ।
पूर्वं वृत्वोपवेश्यैनमुच्चैरासनमण्डले ॥ २९॥

उपविश्य ततोऽधस्तात् रामे चित्तं निधाय च ।
पूर्वाह्णे पठितान् सर्गान् प्रार्थयेद्वाचयेति च ॥ ३०॥

तदुक्तार्थांश्च श‍ृणुयान्मङ्गलाशासनान्तकम् ।
ततस्सायं विधिं कृत्वा विष्णोर्नामसहस्रतः ॥ ३१॥

रघुनाथं समभ्यर्च्य ततो भुक्त्वा स्वपेत् क्रमात् ।
एवं दिनेदिने कुर्वन्नुपचारैरनुक्रमात् ॥ ३२॥

पठित्वा विंशतिं सर्गान् वाचयेच्च यथाक्रमम् ।
प्रत्येकदिवसे कुर्यादन्यान्यैस्सम्भवे सुमैः ॥ ३३॥

सहस्रनामभिः पूजां श्रीरामस्य पदाब्जयोः ।
सीतापाणिग्रहो यत्र यत्र सुग्रीवसख्यकम् ॥ ३४॥

अङ्गुलीयकदानश्च रावणस्य वधस्तथा ।
तत्र तत्र दिने कुर्यात्तूर्यलास्यपुरस्सरम् ॥ ३५॥

अभ्यर्चां रघुनाथस्य वक्तारं च प्रपूजयेत् ।
पठित्वैवं क्रमेणैव सप्तविंशतिवासरान् ॥ ३६॥

पुनः पुनर्वसौ प्राप्ते रामस्य विदितात्मनः ।
पट्टाभिषेकसर्गन्तु पठेन्मङ्गळपूर्वकम् ॥ ३७॥

पुनः पुनर्वसुप्राप्तिस्तारावृद्धिक्षयादिभिः ।
सप्तविंशे दिने चेत्स्यात् तदा तत्रैव वासरे ॥ ३८॥

शिष्टान् सर्गान् वाचयेत्तुं यावद्रामाभिषेचनम् ।
अथ वृद्धथा तु सा चेत्स्यादेकोनत्रिंशवासरे ॥ ३९॥

अष्टाविंशदिनं पाठशून्यं कृत्वाऽचयेद्विभुम् ।
एकोनत्रिंशदिवसे पठेत् पट्टाभिषेचनम् ॥ ४०॥

अष्टाविंशे क्रमेणैव सा चेत तत्र तु वासरे ।
पठेत् पट्टाभिषेकाख्यसर्गमात्रं कृतोत्सवः ॥ ४१॥

पूर्वेद्युः पाठसमये तं सर्ग मवशेषयेत् ।
प्रातःकालेऽह्नि यत्त्रातिव्याप्तिर्दृष्टा पुनर्वसोः ॥ ४२॥

आरभेत च तत्रैव तत्रैव च समापयेत् ।
दिनद्वये समा चेत्सा रौद्रविद्धां परित्यजेत् ॥ ४३॥

सप्तविंशावरान् विप्रान् दधिमध्वाज्यपायसैः ।
भोजयेत् परया भक्त्या ध्यात्वा राघवविग्रहान् ॥ ४४॥

वक्तारं पठितारश्च यथाशक्ति प्रपूजयेत् ।
एवमाराधिते रामे सर्वलोकशरण्यके ॥ ४५॥

वन्ध्या नारी प्रसूयेत पुत्रं सर्वगुणान्विम् ।
आपन्नो मुच्यते कृच्छादपमृत्युं व्यपोहति ॥ ४६॥

अतिक्रान्तचिकित्साच्च रोगादुल्लाघतां व्रजेत् ।
भक्तिं च रघुनाथस्य पादयोर्विन्दते दृढाम् ॥ ४७॥

अन्ते ज्ञानेन मुक्तस्स्यादान्तं स्याद्दुःखवर्जितः ।
एष तूद्देशतः प्रोक्तो मया ते फलविस्तरः ॥ ४८॥

नाहं पटुः प्रवक्तुं स्यां सम्पूर्णफलविस्तरम् ।
यं यं कामयते कामं स स हस्ते भविष्यति ॥ ४९॥

मुमुक्षोरपि कार्यं स्यादेतत्पारायणं भुवि ।
एवं कर्तुमशक्तश्चेद्वित्तलोभं विना द्विजः ॥ ५०॥

यथाशक्त्यर्चयित्वा तु रामं सर्वार्थसिद्धिदम् ।
पूर्वोक्तरीत्या सर्गाणां केवलं विंशतिं पठेत ॥ ५१॥

तस्याऽपि फलसामयं दद्याद्धि रघुनन्दनः ।
॥ अथ नवरात्रपारायणविधिः ॥

अतः परं प्रवक्ष्यामि नवरात्रे तु पाठनम् ॥ ५२॥

चैत्रे वाऽऽश्वयुजे मासि प्रथमायां शुभे दिने ।
आरभ्य च नवम्यन्तं पठेत् प्रातरपि क्रमात् ॥ ५३॥

दशम्यां रावणवधं पठित्वा तत्र वै दिने ।
रामपट्टाभिषेकस्य पठन् सर्गं समापयेत् ॥ ५४॥

पूर्वोक्तविश्विमातिष्ठेत् अत्रापि नियतो द्विजः ।
अशक्तः केवलं पाठमाचरेत् भक्तिसंयुतः ॥ ५५॥

सर्वसिद्धिस्ततो भूयात् आधिव्याधी विनश्यतः ।
रघुनाथः प्रसन्नात्मा तस्य स्यादनुगस्सदा ॥ ५६॥

पुत्रमित्रकलत्रेषु न हानिं द्रक्ष्यते कचित् ।
॥ अथ ऐच्छिकपारायणविधिः ॥

अथ रामायणस्यास्य वक्ष्ये मुनिवरोत्तम ॥ ५७॥

पारायणं त्वैच्छिकं यत् कालापेक्षा न तत्र हि ।
महाव्याधिसमुद्भूतौ विश्लेषे बन्धुजाययोः ॥ ५८॥

राष्ट्रक्षोभे च दुर्भिक्षे भूपालाच्च भयागमे ।
यस्यां कस्यां च तारायां जन्मर्क्षे च विशेषतः ॥ ५९॥

श्रीरामायणकाव्यस्य पारायणमथाचरेत् ।
न सर्गसङ्ख्यानियमो न घटस्थापनादिकम् ॥ ६०॥

न चापि दिनसङ्ख्याऽत्र पारायणसमापने ।
मन्द्रस्वरं पठेत् सर्गान् यथासौकर्यमेव च ॥ ६१॥

भुक्त्वा तु न पठेत् किञ्चित् न च प्रवचनं चरेत् ।
पुष्पैर्लब्धैश्च तुळसीदळैर्बिल्वदळैरपि ॥ ६२॥

विष्णोर्नामसहस्रेण प्रयतः पूजयेत् प्रभुम् ।
सर्वसङ्कल्पसिद्धिः स्यादेतावन्मात्रके कृते ॥ ६३॥

जन्मलग्नवशाद्वाऽपि चन्द्रलग्नवशादपि ।
ग्रहाणां क्रूरता यस्मिन्काले भवति दुःखदा ॥ ६४॥

विशेषेणार्कपुत्रस्य चन्द्रपुत्रस्य वा गुरोः ।
दशा चान्तर्दशा यत्र काले भवति दुःखदा ॥ ६५॥

आत्मपीडा दारपुत्रपीडा चापि धनक्षयः ।
तदा रामायणं भक्त्या काम्यपाठक्रमात्पठेत् ॥ ६६॥

प्रसङ्गादेष कथितः पाठो रामायणस्य तु ।
॥ अथ पुनर्वसुत्रयापेक्षपारायणविधिः ॥

ऐच्छिकं तुर्यपाठन्तु काम्यं वक्ष्यामि नारद ॥ ६७॥

पुनर्वसौ समारभ्य तृतीयेऽथ पुनर्वसौ ।
समापयेत् प्रतिदिनं पठेच्च दशसर्गकान् ॥ ६८॥

गीताया एकमद्धयायमपि भक्त्याऽऽदृतः पठेत् ।
गृहे देवालये वाऽपि रामचन्द्रस्य सन्निधौ ॥ ६९॥

पूजयित्वा चम्पकैर्वा पर्वा तुलसीदकैः ।
विष्णोर्नामसहस्रेण पायसापूपमिश्रितम् ॥ ७०॥

निवेदयित्वा दिव्यानं आरभेत ततः परम् ।
अनुज्ञाप्य द्विजान् स्वर्णैः सङ्कल्प्य च यथाविधि ॥ ७१॥

पुण्याहवाचनं कृत्वा कृत्वा शुद्धं च मण्टपम् ।
श्रोतारमेकं वृत्वा च पठेद्वै दश सर्गकान् ॥ ७२॥

सायमर्थप्रवचनं नावश्यकमिहेरितम् ।
विष्णोर्नाम्नां सहस्रेण रामनाम्नां शतेन वा ॥ ७३॥

अभ्यर्च्य रात्रौ श्रीरामं तप्तक्षीरं निवेदयेत् ।
पाठस्य बहुकालेन साद्ध्यत्वाद्दैवयोगतः ॥ ७४॥

आशौचेनाथवा रोगै राजोपल्पवनादिभिः ।
यजमानस्य विघ्नश्वेत् स्वस्थानेऽन्यं प्रकल्पयेत् ॥ ७५॥

यजमानः पठेद्भक्त्या पाठमेनं दिनेदिने ।
अशक्तचेत्तु श‍ृणुयात् पाठकं परिकल्प्य च ॥ ७६॥

चतुःपञ्चाशताऽहोभिर्दशसर्गक्रमेण तु ।
एवं पारायणं कृत्वा तृतीये तु पुनर्वसौ ॥ ७७॥

पूर्वोक्तेन क्रमेणैव श्रीमतो राघवप्रभोः ।
महोत्सवं प्रकुर्वीत सर्गं राज्याभिषेचनम् ॥ ७८॥

पठित्वा पूजयित्वा च पाठं तत्र समापयेत् ।
काम्यपाठस्त्वयं वत्स करणीयः सुखेन च ॥ ७९॥

सुलभस्स्याद्दरिद्राणामपि सर्वार्थसिद्धिदः ।
आवश्यकस्त्रिवर्गस्य मर्त्यानां कामनावताम् ॥ ८०॥

पुनर्वसुमहापाठ उत्तमोत्तम ईरितः ।
अन्ये पाठाश्चोत्तमास्स्युः यस्मात्रैवर्गिका हि ते ॥ ८१॥

इत्युक्तो गुणविस्तारस्सङ्ग्रहान्नारदाद्य ते ।
श्रीरामायणवेदस्य काम्यपाठक्रमस्य तु ॥ ८२॥

विस्तराद्वक्तुमीशस्स्यात् स एव रघुनन्दनः ।
गच्छ कामं चरन् पुत्र गायन् रामायणं सदा ॥ ८३॥

दाल्भ्य उवाच-
श्रुत्वा रामायणस्येत्थं माहात्म्यं नारदो मुनिः ।
नत्वा चतुर्मुखं सद्यो वाल्मीकिमभिगम्य च ॥ ८४॥

तं नमस्कृत्य बहुशः प्रशस्य च पुनः पुनः ।
गीतं कुशलवाभ्यां च श्रुत्वा रामायणं ततः ॥ ८५॥

रामं नन्तुमथागच्छ दयोद्ध्यां मुनिभिस्सह ।
अत्रान्तरे मृतं पुत्रमादायागान्नृपान्तिकम् ॥ ८६॥

विप्रः कोऽपि शपन् रामं तदैनं नारदोऽब्रवीत् ।
प्रभो राघव धर्मज्ञ सर्वभूतहिते रत ॥ ८७॥

किञ्चित्ज्ञ इव भासीह द्विजात्मजमृतिं प्रति ।
शूद्रः करोति च तपो दण्डकायां वने क्वचित् ॥ ८८॥

तं हत्वा जीवय सुतं ब्राह्मणस्य कृपावशात् ।
इति श्रुत्वा रघुवरः पुष्पकेण सहायुधः ॥ ८९॥

शम्बूकाख्यं दण्डकायां शूद्रं हत्वा तपस्विनम् ।
जीवयित्वा द्विजसुत मगस्त्यमभिवाद्य च ॥ ९०॥

गतः पुनरयोद्ध्यां च वाजिमेधेन चेजिवान् ।
॥ अथ नारदकृत पुनर्वसु पारायणघट्टः ॥

निर्माय पर्णशालां तु नारदो मुनिभिस्सह ॥ ९१॥

गङ्गाद्वारे महापुण्ये रामपूजनतत्परः ।
पुनर्वसौ समारभ्य श्रीरामायणमादृतः ॥ ९२॥

सर्गाणां विंशतिं नित्यं पठित्वा विधिपूर्वकम् ।
वक्तारं कल्पयित्वाऽत्र वाल्मीकिं मुनिपुङ्गवम् ॥ ९३॥

व्रतं समापयामास पुनः प्राप्ते पुनर्वसौ ।
अथ दृश्यत्वमापन्न स्साक्षात्तत्रोटजाङ्गणे ॥ ९४॥

स्निग्धेन्दीवरपत्रामः पद्मरक्तान्तलोचनः ।
युक्तः किरीटकटककेयूराङ्गुळिभूषणैः ॥ ९५॥

वनमालायुतोरस्कः पूर्णचन्द्रनिभाननः ।
मन्दस्मितप्रभारूढबिम्बोष्ठविलसन्मुखः ॥ ९६॥

वामहस्तेन कोदण्डं दक्षिणेन च सायकम् ।
विभ्रत्तूणौ च सम्पूर्णौ निशितैस्सायकोत्तर्मैः ॥ ९७॥

पार्श्वस्थया च सर्वाङ्गभूषणोज्वलया पुनः ।
समझकरया देव्या सेव्यमानश्च सीतया ॥ ९८॥

अग्रे धृताञ्जलिं प्रह्वं हनुमन्तं कृपाढ्यया ।
दृष्ट्याऽनुगृह्णन् तमपि प्रसन्नवदने क्षणम् ॥ ९९॥

सेव्यमानो लक्ष्मणेन पार्श्वे प्राञ्जलिना मुदा ।
नारदस्तु तमालोक्य प्रणिपत्य सहस्त्रशः ॥ १००॥

स्तुत्वा बहुविधैः स्तोत्रैर्वैवैदिकैर्लौङ्किकैरपि ।
पाहि पाहीति च ततोऽब्रवीत् प्रेमाश्रुलोचनः ॥ १०१॥

श्रीराम उवाच -
वत्स नारद तुष्टोऽस्मि त्वद्रामायणपाठतः ।
पारायणात्पुनर्वस्वां विशेषानुग्रहो मम ॥ १०२॥

यथाऽनेनास्मि तुष्टोऽद्य तथा न तपसा तव ।
रामायणं यत्र यत्र भक्तितोऽभक्तितोऽपि वा ॥ १०३॥

पठ्यते तत्र सान्निद्वय महं कुर्वे यथेह तु ।
यो वाञ्छति प्रसादं मे स तु रामायणं पठेत् ॥ १०४॥

श्रीमद्वाल्मीकिवचसां माधुर्यं मां प्रकर्षति ।
अहं चाकर्णयिष्यामि त्वया च त्वादृशैस्सह ॥ १०५॥

स च कालो भवेच्छीघ्रं त्वमत्यागच्छ यत्नतः ।
त्रियतामद्य च वरो यथेष्टं प्रददामि ते ॥ १०६॥

नारद उवाच -
किमिहाद्य वरैः कार्यं निस्सङ्गस्य ममेतरैः ।
अयं वरो मे परमो यदेवं त्वां विलोकये ॥ १०७॥

दृष्टा मया शङ्खचक्रगदापीताम्बरान्विता ।
कौस्तुभाभरणग्रीवा लक्ष्मीयुक्तोरसाऽन्विता ॥ १०८॥

तव सा क्षीरपाथोधौ मूर्तिर्नारायणाह्वया ।
वैकुण्ठस्थानगा मूर्तिर्भुजाष्टकविभूषिता ॥ १०९॥

सुनन्दनन्दप्रमुखैस्सेविता नित्यसूरिभिः ।
दृष्टा तपःप्रभावेन कृपया दर्शिता त्वया ॥ ११०॥

नरनारायणाभिख्ये लोकरक्षार्थमादृते ।
तपस्विवेषे ते मूर्ती दृष्टे बदरिकाश्रमे ॥ १११॥

सर्वावतारबीजात्मा वैराजी साऽपि ते तनुः ।
नानाविश्वमयी दृष्टा श्वेतद्वीपे मया पुरा ॥ ११२॥

न तथा हर्षमगमं यथाऽद्य तव दर्शनात् ।
वाणकोदण्डपाणेश्व जानक्या सेवितस्य च ॥ ११३॥

दिव्यतेजोनिधेस्तेऽद्य दर्शनान्निर्वृणोम्यहम् ।
मनो मे मोदतेऽत्यर्थं मन्ये पूर्वं न चेदृशः ॥ ११४॥

तदहं प्रार्थये त्वाऽद्य वरं सर्ववरोत्तमम् ।
यदा यदाऽहं वाञ्छामि द्रक्ष्यामि त्वां तदा तदा ॥ ११५॥

एवं बाणधनुष्पाणिं सीतालल्मणसेवितम् ।
अचञ्चला च त्वद्भक्तिः स्याम्ने कारुण्यतस्तव ॥ ११६॥

किं चैवं काम्यपाठेन श्रीरामायणमादरात् ।
पुनर्वसौ समारभ्य यः पठेद्विधिपूर्वकम् ॥ ११७॥

भक्तितो दम्भतोवाऽपि तस्य त्वं प्रीतमानसः ।
धर्मार्थकाममोक्षाख्यं चतुर्वर्गं दिश प्रभो ॥ ११८॥

पुत्रान् पौत्रान् सम्पदश्च ज्ञानमन्ते च सोऽश्नुताम् ।
श्रीराम उवाच -
सत्यमुक्तं त्वया ब्रह्मन् मद्भक्ता मामकीं तनुम् ॥ ११९॥

कोदण्डपाणि सन्द्रष्टुं समीहन्ते न चान्यथा ।
ईदृशी मे तनुः पूर्वं स्वतेजःपुञ्जसंवृता ॥ १२०॥

कोटिसूर्यसमाभासा रक्तचन्दनसेविता ।
सीतया सेविता पार्श्वे वालव्यजनहस्तया ॥ १२१॥

मायां दर्शनमात्रेण निहन्त्री भेदधीप्रदाम् ।
चिरमाशासमानाय मत्प्रसादाभिकाङ्क्षया ॥ १२२॥

सनत्कुमाराद्विज्ञाय ममाविर्भावमञ्जसा ।
सारथ्यान्मन्त्रकृत्याच्च इक्ष्वाकुकुलसेवने ॥ १२३॥

दर्शिताऽभूत् सुमन्त्राय प्रस्थास्येऽहं यदा वनम् ।
शबर्यै शरभङ्गाय वालिने च मुमूर्षवे ॥ १२४॥

दर्शिताऽभूदिदानिं त्वं दृष्टवान् मत्कथाबलात् ।
नित्यं रामायणं ये वा पठेयुः कामतोऽपि वा ॥ १२५॥

मुक्तये वाऽपि ते सर्वे एवंरूपं तु मां मुने ।
अन्ते दृष्ट्वा प्राप्नुयुर्मां नात्र कार्या विचारणा ॥ १२६॥

य वा पुनर्वसुदिने समारभ्य यथाविधि ।
श्रीमद्रामायणं भक्त्या पठेयुः पुत्रकाम्यया ॥ १२७॥

धनाप्तये वा विपदां तारणायाथवा मुने ।
अन्ते तेषां सुप्रसन्नः पाठस्यास्य यथेप्सितम् ॥ १२८॥

सर्वान् कामान् ददाम्येव न ते शोचन्ति भूतले ।
प्रारब्धमपि दुःखं स्यात् तेषां निर्वीर्यमञ्जसा ॥ १२९॥

नाहं स्वाध्यायतोवाऽपि बाह्याभ्यन्तरपूजनात् ।
तपसा त्यागतो वाऽपि प्रसीदामि तथा मुने ॥ १३०॥

यथा रामायणकथापठनश्रवणादिभिः ।
दर्शयिष्यामि ते मूर्तिमेवंरूपां मनोरमाम् ॥ १३१॥

यदा यदा वाञ्छसि त्वं यत्र क्वापि मुनीश्वर ।
मत्प्रसादाद्दक्षशापो न त्वामभ्यभवन्मुने ॥ १३२॥

अत्र स्थितं मासमात्रं क्षणमात्रमतः परम् ।
नात्र स्थेयमितस्तस्मात् गच्छ लोकान् यथासुखम् ॥ १३३॥

त्वं च नित्यञ्चरन् गायेः श्रीरामायणमादृतः ।
प्रकाशय कथामेतां लोकानां सुखसिद्धये ॥ १३४॥

वाल्मीके त्वं च शिष्याभ्यां गायद्भयां मत्कथां सदा ।
आनन्दमाप्नुहि परं नान्यलभ्यं कथञ्चन ॥ १३५॥

आहरिष्याम्यश्वमेधं स्वर्णसीतासमन्वितः ।
आगच्छ सह शिष्याभ्यां मत्पुत्राभ्यां महाध्वरम् ॥ १३६॥

तदा कर्मान्तरे काव्यं त्वत्कृतं मत्कथाश्रयम् ।
श्रोष्यामि बहुभिस्सार्द्धं सभायां राजभिस्सह ॥ १३७॥

आऽश्वमेधकथाघट्टं मुनिभिश्च भवादृशैः ।
आन्तं ततश्च श्रोष्यामि त्वं यशोभाग्भविष्यसि ॥ १३८॥

पुत्रौ च स्वीकरिष्यामि सर्वलोकस्य सन्निधौ ।
इदं रहस्यं प्रोक्तं ते त्वया प्रोक्तं च काव्यके ॥ १३९॥

त्वं निवत्स्यसि मल्लोके मत्समो बहुकल्पकम् ।
मामेवैष्यसि चान्ते त्वं सत्यं प्रतिश‍ृणोमि ते ॥ १४०॥

दाल्भ्य उवाच-
इत्युक्त्वाऽन्तर्हितो रामस्सीताद्यैरनुगैस्सह ।
अद्यापि गायति मुनिश्वरन रामायणं सदा ॥ १४१॥

स एनमर्थं प्रावोचनद्वयासायामिततेजसे ।
उपादिशत् स्वपुत्राय सोऽपि राजन् शुकाय च ॥ १४२॥

स मे कथितवान् पूर्वं प्रसङ्गाद्भक्तसंसदि ।
मया तवोपदिष्टं च श्रीरामायणवैभवम् ॥ १४३॥

त्वं चादित्ये समायाते मदुक्तविधिपूर्वकम् ।
पठन् रामायणं भक्त्या वक्तारं परिकल्प्य च ॥ १४४॥

श्रोतॄंश्च परमोदारान् तरिष्यसि महापदम् ।
युधिष्ठिर उवाच-
श्रीमन्मुनिगणश्रेष्ठ दाल्भ्य सर्वज्ञसम्मत ॥ १४५॥

धन्योऽस्म्यनुगृहीतोऽस्मि विपदं तारितोऽस्मि च ।
अहं रामायणं भक्त्या पठिष्यामि पुनर्वसौ ॥ १४६॥

वक्ता भव महाभाग नान्योऽस्ति मम सम्मतः ।
श्रीसूत उवाच -
इत्युक्त्वा पाण्डुतनयस्सम्भारान् भातृभिर्मुन्दा ॥ १४७॥

सम्पाद्य च ततो वृत्वा श्रोतॄन् भक्तान् रघूत्तमे ।
दालभ्यं वक्तारमाधाय द्वारकां प्रेष्य चानुजान् ॥ १४८॥

गोविन्दं च समानीय प्रोक्तवान् स्वमनीषितम् ।
स चैनमब्रवीद्राजन् तुष्टोऽस्मि तव कार्यतः ॥ १४९॥

अस्मिन्नप्यवतारेऽहं मम पूर्वांवतारके ।
रामात्मके यच्चालितं तद्योवर्णयते मुदा ॥ १५०॥

तदा परवशो राजन् भवामि प्रीतमानसः ।
किमुताऽद्य महीपाल श्रीरामायणपाठतः ॥ १५१॥

चलत्वयं विधिस्सर्वः श्रोष्यामि च सह द्विजैः ।
श्रीसूत उवाच -
ततो युधिष्ठिरो राजा सङ्कल्प्य विधिपूर्वकम् ॥ १५२॥

द्रौपद्या सह धर्मात्मा पुनर्वसुमहादिने ।
रामायणं यथान्यायं समारभ्य पठन् स्वयम् ॥ १५३॥

समापयच्च विधिना कुत्वा कृष्णं सभेश्वरम् ।
यदा सभां पूजयितुमुत्थितो धर्मनन्दनः ॥ १५४॥

न ददर्श तदा कृष्णं तत्स्थानेऽन्यं ददर्श ह ।
सव्ये पाणौ च कोदण्डं दक्षिणे चापि सायकम् ॥ १५४॥

सीतालक्ष्मणसंसेव्यं पादान्ते च हनूमता ।
श्रीरामं पुष्पवर्षेण कीर्यमाणं नभस्सदाम् ॥ १५६॥

सूर्यकोटिसमाभासं प्रसन्नवदनाम्बुजम् ।
दृष्ट्वा परवशो भूत्वा हर्षेण कुरुनन्दनः ॥ १५७॥

सहस्रकृत्वः प्राणंसीत् तुष्टाव च पुनः पुनः ।
पाद्याध्र्याचमनीयेन सम्यगाराधयन्मुदा ॥ १५८॥

भगवानपि सर्वात्मा राजानमिदमब्रवीत् ।
श्रीभगवानुवाच-
तुष्टोऽस्म्यनेन विधिनां श्रीरामायणपाठतः ॥ १५९॥

अवाप्स्यस्यखिलान् कामानन्ते सायुज्यमेष्यसि ।
शत्रून् हनिष्यस्यखिलान् प्रतिज्ञान्ते नृपोत्तम । १६०॥

राज्यं प्राप्स्यसि भूयोऽपि भ्रातृभिः परिवारितः ।
इत्युक्वाऽऽसीत् पुनः कृष्णस्सर्वालङ्कारशोभितः ॥ १६१॥

सदस्या विस्मयाम्भोधौ मग्नाः श्रीरामदर्शनात् ।
तुष्टर्द्वारकानाथं तन्माहात्म्यवशीकृताः ॥ १६२॥

राजाऽपि कर्म तत् सम्यक्समाप्य विधिपूर्वकम् ।
अन्नैश्च षड्रसोपेतैः सूर्यपात्रविनिर्मितैः ॥ १६३॥

ब्राह्मणान् भोजयाभास सहस्राधिकसङ्ख्यया ।
वक्तारं पूजयामास कृष्णदत्तैधर्नादिभिः ॥ १६४॥

ततो व्यतीते वनवासदुःखे जित्वा रिपून् भीष्ममुखान् महात्मा ।
सहोदरैस्सेवितपादपद्मो महीं प्रपेदे महतीं मुदं च ॥ १६५॥

श्रीसूत उवाच -
यो रामायणमाहात्म्य मिदमुक्तं मुनीश्वराः ।
पठेद्वा श‍ृणुयाद्वापि स रामे भक्तिमान् भवेत् । (। १६६॥)
श्रीरामस्य प्रसादेन सर्वान् कामानवाप्नुयात् ॥ १६६॥

-इति श्रीस्कन्दोपपुराणे पुराणवैभवखण्डे ब्रह्मनारदसंवादे
रामायणमाहात्म्ये काम्यपाठनिरूपणनाम पञ्चविंशोऽध्यायः ॥ २५

॥ इति तृतीयोऽध्यायः ॥

॥ इति श्रीरामायणमाहात्म्यं सम्पूर्णम् ॥

- स्कन्दपुराण । पुराणवैभवखण्ड । अध्याय २३-२५॥