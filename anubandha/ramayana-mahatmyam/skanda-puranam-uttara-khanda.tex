\chapt{रामायणमाहात्म्यम्}

\twolineshloka
{श्रीरामः शरणं समस्तजगतां रामं विना का गती}
{रामेण प्रतिहन्यते कलिमलं रामाय कार्यं नमः}

\sect{प्रथमोऽध्यायः—कल्पानुकीर्तनम्}
 

\twolineshloka
{रामात् त्रस्यति कालभीमभुजगो रामस्य सर्वं वशे}
{रामे भक्तिरखण्डिता भवतु मे राम त्वमेवाश्रयः}%॥१॥

 

\twolineshloka
{चित्रकूटालयं रामं इन्दिरानन्दमन्दिरम्}
{वन्दे च परमानन्दं भक्तानां अभयप्रदम्}%॥२॥

 

\twolineshloka
{ब्रह्मविष्णुमहेशाद्या यस्यांशा लोकसाधकाः}
{नमामि देवं चिद्‌रूपं विशुद्धं परमं भजे}%॥३॥

 

\uvacha{ऋषयः उचुः}

\twolineshloka
{भगवन् सर्वमाख्यातं यत् पृष्ठं विदुषा त्वया}
{संसारपाशबद्धानां दुःखानि सुबहूनि च}%॥४॥

 

\twolineshloka
{एतत् संसारपाशस्य च्छेदकः कतमः स्मृतः}
{कलौ वेदोक्तमार्गाश्च नश्यन्तीति त्वयोदिताः}%॥५॥

 

\twolineshloka
{अधर्मनिरतानां च यातनाश्च प्रकीर्तिताः}
{घोरे कलियुगे प्राप्ते वेदमार्ग बहिष्कृते}%॥६॥

 

\twolineshloka
{पाखंडत्वं प्रसिद्धं वै सर्वैश्च परिकीर्तितम्}
{कामार्त्ता ह्रस्वदेहाश्च लुब्धा अन्योन्यतत्पराः}%॥७॥

 

\twolineshloka
{कलौ सर्वे भविष्यंति स्वल्पायुः बहुपुत्रकाः}
{स्त्रियः स्वपोषणपरा वेश्याचरण तत्पराः}%॥८॥

 

\twolineshloka
{पतिवाक्यं अनादृत्य सदा अन्यगृह तत्पराः}
{दुःशीलेषु करिष्यंति पुरुषेषु सदा स्पृहाम्}%॥९॥

 

\twolineshloka
{असद्वार्त्तां भविष्यन्ति पुरुषेषु कुलाङ्गनाः}
{परुषानृतभाषिण्यो देहसंस्कारवर्जिताः}%॥१०॥

 

\twolineshloka
{वाचालश्च भविष्यन्ति कलौ प्रायेण योषितः}
{भिक्षवश्चापि मित्रादि स्नेहसम्बन्ध यन्त्रिताः}%॥११॥

 

\twolineshloka
{अन्नोपाधिनिमित्तेन शिष्यान् बध्नन्ति लोलुपाः}
{उभाभ्यां अपि पाणिभ्यां शिरः कण्डूयनं स्त्रियः}%॥१२॥

 

\twolineshloka
{कुर्वन्त्यो गृहभर्तृणां आज्ञां भेत्स्यन्त्यतन्द्रिताः}
{पाखण्डालापनिरताः पाखण्डजनसङ्‌गिनः}%॥१३॥

 

\twolineshloka
{यदा द्विजा भविष्यन्ति तदा वृद्धिं गतः कलिः}
{घोरे कलियुगे ब्रह्मन् जनानां पापकर्मिणाम्}%॥१४॥

 

\twolineshloka
{मनः शुद्धि विहीनानां निष्कृतिश्च कथं भवेत्}
{यथा तुष्यति देवेशो देवदेवो जगद्‌गुरुः}%॥१५॥

 

\twolineshloka
{ततो वदस्व सर्वज्ञ सुत धर्मभृतां वर}
{वद सूत मुनिश्रेष्ठ सर्वं एतत् अशेषतः}%॥१६॥

 
\onelineshloka
{कस्य नो जायते तुष्टिः सूत त्वत् वचनामृतात्}%॥१७॥


\uvacha{सूत उवाच}

\twolineshloka
{शृणुध्वं ऋषयः सर्वे यदिष्टं वो वदाम्यहम्}
{गीतं सनत्कुमाराय नारदेन महात्मना}%॥१८॥

 

\twolineshloka
{रामायणं महाकाव्यं सर्ववेदेषु सम्मतम्}
{सर्वपापप्रशमनं दुष्टग्रह निवारणम्}%॥१९॥

 

\twolineshloka
{दुःस्वप्न नाशनं धन्यं भुक्तिमुक्ति फलप्रदम्}
{रामचंद्रकथोपेतं सर्वकल्याणसिद्धिदम्}%॥२०॥

 

\twolineshloka
{धर्मार्थकाममोक्षाणां हेतुभूतं महाफलम्}
{अपूर्वं पुण्यफलदं शृणुध्वं सुसमाहिताः}%॥२१॥

 

\twolineshloka
{महापातक युक्तो वा युक्तो वा सर्वपातकैः}
{शृत्वैतदार्षं दिव्यं हि काव्यं शुद्धिं अवाप्नुयात्}%॥२२॥

 

\twolineshloka
{रामायणेन वर्तन्ते सुतरां ये जगद्धिताः}
{त एव कृतकृत्याश्च सर्वशास्त्रार्थ कोविदाः}%॥२३॥

 

\twolineshloka
{धर्मार्थकाममोक्षाणां साधनं च द्विजोत्तमाः}
{श्रोतव्यं च सदा भक्त्या रामायणपरामृतम्}%॥२४॥

 
\twolineshloka
{पुनर्जितानि पापानि नाशमायांति यस्य वै}
{रामायणे महाप्रीतिः तस्य वै भवति ध्रुवन्}% ॥२५॥

 

\twolineshloka
{रामायणे वर्तमाने पापपाशेन यन्त्रितः}
{अनादृत्य असद्‌गाथा आसक्तबुद्धिः प्रवर्तते}%॥२६॥

 

\twolineshloka
{रामायणं नाम परं तु काव्यं सुपुण्यदं वै शृणुत द्विजेंद्राः}
{यस्मिन् शृते जन्मजरादिनाशो भवत्यदोषः स नरोऽच्युतः स्यात्}%॥२७॥

 

\twolineshloka
{वरं वरेण्यं वरदं तु काव्यं संतारयत्याशु च सर्वलोकम्}
{संकल्पितार्थप्रदमादिकाव्यं शृत्वा च रामस्य पदं प्रयाति}%॥२८॥

 

\twolineshloka
{ब्रह्मेशविष्ण्वाख्य शरीरभेदैः विश्वं सृजत्यत्ति च पाति यश्च}
{तमादिदेवं परमं वरेण्यं आधाय चेतस्युपयाति मुक्तिम्}%॥२९॥

 

\twolineshloka
{यो नामजात्यादि विकल्पहीनः परावराणां परमः परः स्यात्}
{वेदाअंतवेद्यः स्वरुचा प्रकाशः स वीक्ष्यते सर्वपुराणवेदै}%॥३०॥

 

\twolineshloka
{ऊर्जे माहे सिते पक्षे चैत्रे च द्विजसत्तमाः}
{नवाह्ना खलु श्रोतव्यं रामायण कथामृतम्}%॥३१॥

 

\twolineshloka
{इत्येवं शृणुयाद् यस्तु श्रीरामचरितं शुभम्}
{सर्वान् कामान् अवाप्नोति परत्रामुत्र चोत्तमान्}%॥३२॥

 

\twolineshloka
{त्रिसप्तकुलसंयुक्तः सर्वपापविवर्जितः}
{प्रयाति रामभवनं यत्र गत्वा न शोचते}%॥३३॥

 

\twolineshloka
{चैत्रे माघे कार्तिके च सिते पक्षे चा वाचयेत्}
{नवाहस्सु महापुण्यं श्रोतव्यं च प्रयत्‍नतः}%॥३४॥

 

\twolineshloka
{रामायणं आदिकाव्यं स्वर्गमोक्ष प्रदायकम्}
{तस्माद् घोरे कलियुगे सर्वधर्म बहिष्कृते}%॥३५॥

 

\twolineshloka
{नवभिर्दिनैः श्रोतव्यं रामायण कथामृतम्}
{रामनामपरा ये तु घोरे कलियुगे द्विजाः}%॥३६॥

 

\twolineshloka
{त एव कृतकृत्याश्च न कलिर्बाधते हि तान्}
{कथा रामयणस्यापि नित्यं भवति यद्‌गृहे}%॥३७॥

 

\twolineshloka
{तद् गृहं तीर्थरूपं हि दुष्टानां पापनाशनम्}
{तावत्पापानि देहेस्मिन् निवसंति तपोधनाः}%॥३८ ॥॥

 

\twolineshloka
{यावन्न श्रूयते सम्यक् श्रीमद्‌रामायणं नरैः}
{दुर्लभैव कथा लोके श्रीमद्‌रामायणोद्‌भवा}%॥३९॥

 

\twolineshloka
{कोटिजन्मसमुत्थेन पुण्येनैव तु लभ्यते}
{ऊर्जे माघे सिते पक्षे चैत्रे च द्विजसत्तमाः}%॥४०॥

 

\twolineshloka
{यस्य श्रवणमात्रेण सैदासोऽपि विमोचितः}
{गौतमशापतः प्राप्तः सौदासो राक्षसीं तनुम्}%॥४१॥

 

\twolineshloka
{रामायणप्रभावेण विमुक्तिं प्राप्तवान् पुनः}
{यस्त्वेतत् श्रुणुयाद् भक्त्या रामभक्तिपरायणः}%॥४२॥

 
\onelineshloka
{स मुचते पहापापैः पुरुष पातकादिभिः}%॥४३॥

॥इति श्रीस्कान्दे महापुराण उत्तरखण्डे नारद-सनत्कुमार-संवादे रामायणमाहात्म्ये कल्पानुकीर्तनं नाम प्रथमोऽध्यायः॥१॥
 

 
\sect{द्वितीयोध्यायः—राक्षमोक्षणम्}

 

\uvacha{ऋषयः ऊचुः}

\twolineshloka
{कथं सनत्कुमाराय देवर्षिर्नारदो मुनिः}
{प्रोक्तवान् सकलान् धर्मान् कथं तौ मिलितावुभौ}%॥१॥

 

\twolineshloka
{कस्मिन् क्षेत्रे स्थितौ तात तावुभौ ब्रह्मवादिनौ}
{यदुक्तं नारदेनास्मै तत् त्वं ब्रूहि महामुने}%॥२॥

 

\uvacha{सूत उवाच}

\twolineshloka
{सनकाद्या महात्मानो ब्रह्मणस्तनयाः स्मृताः}
{निर्ममा निरहंकारा सर्वे ते ह्यूर्ध्वरेतसः}%॥३॥

 

\twolineshloka
{तेषां नामानि वक्ष्यामि सनकश्च सनन्दनः}
{सनत्कुमारश्च तथा सनातन इति स्मृतः}%॥४॥

 

\twolineshloka
{विष्णुभक्ता महात्मानो ब्रह्मध्यानपरायणाः}
{सहस्रसूर्यसंकाशाः सत्यवन्तो मुमुक्षवः}%॥५॥

 

\twolineshloka
{एकदा ब्रह्मणः पुत्राः सनकाद्या महौजसः}
{मेरुश्रृङ्‍गे समाजग्मुः वीक्षितुं ब्रह्मणः सभाम्}%॥६॥

 

\twolineshloka
{तत्र गङ्‍गां महापुण्यां विष्णुपादोद्‌भवां नदीम्}
{निरीक्ष्य स्नातुमुद्युक्ताः सीताख्यां प्रथितौजसः}%॥७॥

 

\twolineshloka
{एतस्मिन् अंतरे विप्रा देवर्षिनारदो मुनिः}
{अजगामोच्चरन् नाम हरेर्नारायणादिकम्}%॥८॥

 

\twolineshloka
{नारायणाचुतानन्त वासुदेव जनार्दन}
{यज्ञेश जज्ञ्पुरुष राम विष्णो नमोऽस्तु ते}%॥९॥

 

\twolineshloka
{इत्युच्चरन् हरेर्नाम पावयन् अखिलं जगत्}
{आजगाम स्तुवन् गङ्गां मुनिर्लोकैकपावनीम्}%॥१०॥

 

\twolineshloka
{अथयान्तं समुद्‌वीक्ष्य सनकाद्या महौजसः}
{यथार्हमर्हणं चक्रुः ववन्दे सोऽपि तान् मुनीन्}%॥११॥

 

\twolineshloka
{अथ तत्र सभामध्ये नारायणपरायणम्}
{सनत्कुमारः प्रोवाच नारदं मुनिपुङ्गवम्}%॥१२॥

 

\uvacha{सनत्कुमार उवाच}

\twolineshloka
{सर्वज्ञोऽसि महाप्राज्ञ मुनीशानां च नारद}
{हरिभक्तिपरो यस्मात् त्वत्तो नास्त्यपरोऽधिकः}%॥१३॥

 

\twolineshloka
{येनेदं अखिलं जातं जगत् स्थावरजङ्गमम्}
{गङ्गा पादोद्‌भवा यस्य कथं स ज्ञायते हरिः}%॥१४॥

 
\onelineshloka*
{अनुग्रह्योऽस्मि यदि ते तत्त्वतो वक्तुमर्हसि}

\uvacha{नारद उवाच}
\onelineshloka
{नमः पराय देवाय परात्परतराय च}%॥१५॥

 

\twolineshloka
{परात्परनिवासाय सगुणायागुणाय च}
{ज्ञानाज्ञानस्वरूपाय धर्माधर्मस्वरूपिणे}%॥१६॥

 

\twolineshloka
{विद्या अविद्या स्वरूपाय स्वरूपाय ते नमः}
{योदैत्यहन्ता नरकन्तकश्च भुजाग्रमात्रेण च धर्मगोप्ता}%॥१७॥

 

\twolineshloka
{भूभारसन्घात विनोदकामं नमामि देवं रघुवंशदीपम्}
{आविर्भूतश्चतुर्द्धा यः कपिभिः परिवारितः}%॥१८॥

 

\twolineshloka
{हतवान् राक्षसानीकं रामं दाशरथिं भजे}
{एवमादीन्यनेकानि चरितानि महात्मनः}%॥१९॥

 

\twolineshloka
{तेषां नामानि संख्यातुं शक्यन्ते नाब्दकोटिभिः}
{महिमानं तु यन्नाम्नः पारं गन्तुं न शक्यते}%॥२०॥

 

\twolineshloka
{मनुभिश्च मुनीन्द्रैश्च कथं तं क्षुल्लको भजेत्}
{यन्नाम्नः स्मरणेनापि महापातकिनोऽपि ये}%॥२१॥

 

\twolineshloka
{पावनत्वं प्रपद्यन्ते कथं स्तोष्यामि क्षुल्लधीः}
{रामायणपरा ये तु घोरे कलियुगे द्विजाः}%॥२२॥

 

\twolineshloka
{त एव कृतकृत्याश्च तेषां नित्यं नमोऽस्तु ते}
{ऊर्जे मासि सिते पक्षे चैत्रे माघे तथैव च}%॥२३॥

 

\twolineshloka
{नवाह्ना किल श्रोतव्यं रामाण कथामृतमम्}
{गौतमशापतः प्राप्तः सुदासो राक्षसीं तनुम्}%॥२४॥

 

\onelineshloka*
{रामायणप्रभावेण विमुक्तिं प्राप्तवानसौ}

\uvacha{सनत्कुमार उवाच}

\onelineshloka
{रामायणं केन प्रोक्तं सर्वधर्मफलप्रदम्}%॥२५॥

 

\twolineshloka
{प्राप्तः कथं गौतमेन सौदासो मुनिसत्तम}
{रामायणप्रभावेण कथं भूयो विमोक्षितः}%॥२६॥

 

\twolineshloka
{अनुग्रह्योऽस्मि यति ते तत्त्वतो वक्तुमर्हसि}
{सर्वं एतत् अशेषेण मुने नो वक्तुमर्हसि}%॥२७॥

 
\onelineshloka*
{श्रृण्वतां वदतां चैव कथा पापविनाशिनी}

\uvacha{नारद उवाच}

\onelineshloka
{श्रृणु रामायणं विप्र यद वाल्मीकिमुखोद्‌गतम्}%॥२८॥

\twolineshloka
{नवाह्ना खलु श्रोतव्यं रामायण कथामृतम्}
{आस्ते कृतयुगे विप्रो धर्मकर्मविशारदः}%॥२९॥

 

\twolineshloka
{सोमदत्त इति ख्यातो नाम्ना धर्मपरायणः}
{विप्रस्तु गौतमाख्येन मुनिना ब्रह्मवदिना}%॥३०॥

 

\twolineshloka
{श्रावितः सर्वधर्मांश्च गङ्गातीरे मनोरमे}
{पुराणशास्त्र कथनैः तेनासौ बोधितोऽपि च}%॥३१॥

 

\twolineshloka
{श्रुतवान् सर्वधर्मान् वै तेनोक्तान् अखिलानपि}
{कदाचित् परमेशस्य परिचर्यापरोऽभवत्}%॥३२॥

 

\twolineshloka
{उपस्थितायापि तस्मै प्रणामं न चकार सः}
{स तु शान्तो महाबुद्धिः गौतमस्तेजसां निधिः}%॥३३॥

 

\twolineshloka
{शास्त्रोदितानि कर्माणि करोति स मुदं ययौ}
{यस्त्वर्चितो महादेवः शिवः सर्वजगद्‌गुरुः}%॥३४॥

 

\twolineshloka
{गुर्ववज्ञाकृतं पापं राक्षसत्वे नियुक्तवान्}
{उवाच प्राञ्जलिर्भूत्वा विनयेषु च कोविदः}%॥३५॥

 

\uvacha{विप्र उवाच}

\twolineshloka
{भगवन् सर्वधर्मज्ञ सर्वदर्शिन् सुरेश्वरः}
{क्षमस्व भगवन् सर्वं अपराधः कृतो मया}%॥३६॥

 

\uvacha{गौतम उवाच}

\twolineshloka
{ऊर्जे मासे सिते पक्षे रामायण कथामृतम्}
{नवाह्ना चैव श्रोतव्यं भक्तिभावेन सादरम्}%॥३७॥

 
\onelineshloka*
{नात्यन्तिकं भवेद् एतद् द्वादशाब्दं भविष्यति}
\uvacha{विप्र उवाच}

\onelineshloka
{केन रामायणं प्रोक्तं चरितानि तु कस्य वै}%॥३८॥

 

\twolineshloka
{एतत् सर्वं महाप्राज्ञ संक्षेपाद् वक्तुमर्हसि}
{मनसा प्रीतिमापन्नो ववन्दे चरणौ गुरोः}%॥३९॥

 

\uvacha{गौतम उवाच}

\twolineshloka
{श्रृणु रामायणं विप्र वाल्मीकिमुनिना कृतम्}
{येन रामावतारेण राक्षसा रावणादयः}%॥४०॥

 

\twolineshloka
{हतास्तु देवकार्यं हि चरितं तस्य तच्छृणु}
{कार्त्तिके च सिते पक्षे कथा रामायणस्य तु}%॥४१॥

 

\twolineshloka
{नवमेऽहनि श्रोतव्या सर्वपापप्रणाशिनी}
{इत्युक्त्वा चार्थसम्पन्नो गौतमः स्वाश्रमं ययौ}%॥४२॥

 

\twolineshloka
{विप्रोऽपि दुःखमापन्नो राक्षसीं तनुमाश्रितः}
{क्षुत्पीडितः पिपासार्त्तो नित्यं क्रोधपरायणः}%॥४३॥

 

\twolineshloka
{कृष्णक्षपाद्युतिर्भीमो बभ्राम विजने वने}
{मृगांश्च विविधांस्तत्र मनुष्यांश्च सरीसुपान्}%॥४४॥

 

\twolineshloka
{विहगान् प्लवगांश्चैव प्रसभात्तनभक्षयत्}
{अस्थिभिर्बहुभिर्विप्राः पीतरक्त कलेवरैः}%॥४५॥

 

\twolineshloka
{रक्तादप्रेतकैश्चैव तेनासीद् भूर्भयंकरी}
{ऋतुत्रये स पृथिवीं शतयोजन विस्तराम्}%॥४६॥

 

\twolineshloka
{कृत्वादिदुःखितां पश्चाद् वनान्तरमगात् पुनः}
{तत्रापि कृतवान् नित्यं नरमांसाशनं तदा}%॥४७॥

 

\twolineshloka
{जगाम नर्मदातीरे सर्वलोकभयंकरः}
{एतस्मिन् अंतरे प्राप्तः कश्चित् विप्रोऽतिधार्मिकः}%॥४८॥

 

\twolineshloka
{कलिङ्गदेशसंभूतो नाम्ना गर्ग इति स्मृतः}
{वहन् गङ्गाजलं स्कंधे स्तुवन् विश्वेश्वरं प्रभुम्}%॥४९॥

 

\twolineshloka
{गायन् नामानि रामस्य समायातोऽतिहर्षितः}
{तं आयान्तं मुनिं दृष्ट्वा सुदासो नाम राक्षसः}%॥५०॥

 

\twolineshloka
{प्राप्तो न पारणेत्युक्त्वा भुजौ उद्यम्य तं ययौ}
{तेन कीर्तितनामानि श्रुत्वा दूरे व्यवस्थितः}%॥५१॥

 
\onelineshloka*
{अशक्तस्तं द्विजं हन्तुं इदं ऊचे स राक्षसः}

\uvacha{राक्षस उवाच}
\onelineshloka
{अहो भद्र महाभाग नमस्तुभ्यं महात्मने}%॥५२॥

 

\twolineshloka
{नामस्मरणमात्रेण राक्षसा अपि दूरगाः}
{मया प्रभक्षिताः पूर्वं विप्राः कोटिसहस्रशः}%॥५३॥

 

\twolineshloka
{नामप्रावरणं विप्र रक्षति त्वां महाभयात्}
{नामस्मरणमात्रेण राक्षसा अपि भो वयम्}%॥५४॥

 

\twolineshloka
{परां शान्तिं समापन्ना महिमा कोऽच्युतस्य हि}
{सर्वथा त्वं महाभाग रागादिरहितो द्विज}%॥५५॥

 

\twolineshloka
{रामकथाप्रभावेण पाह्यस्मात् पातकाधमात्}
{गुर्ववज्ञा मया पूर्वं कृता च मुनिसत्तम}%॥५६॥

 

\twolineshloka
{कृतश्चानुग्रहः पश्चाद् गुरुणोक्तं इदं वचः}
{वाल्मीकिमुनिना पूर्वं कथा रामायणस्य च}%॥५७॥

 

\twolineshloka
{ऊर्जे मासे सिते पक्षे श्रोतव्या च प्रयत्‍नतः}
{गुरुणापि पुनः प्रोक्तं रम्यं तु शुभदं वचः}%॥५८॥

 

\twolineshloka
{नवाह्ना खलु श्रोतव्यं रामायण कथामृतमम्}
{तस्माद् ब्रह्मन् महाभाग सर्वशास्त्रार्थकोविद}%॥५९॥

 
\onelineshloka*
{कथाश्रवणमात्रेण पाह्यस्मात् पापकर्मणः}

\uvacha{नारद उवाच}

\onelineshloka
{ततो रामायणं ख्यातं राममाहात्म्यमुत्तमम्}%॥६०॥

 

\twolineshloka
{निशम्य विस्मयाविष्टो बभूव द्विजसतमः}
{ततो विप्रः कृपाविष्टो रामनामपरायणः}%॥६१॥

 
\onelineshloka*
{सुदासराक्षसं नाम चेदं वाक्यमथाब्रवीत्}

\uvacha{विप्र उवाच}

\onelineshloka
{राक्षसेन्द्र महाभाग मतिस्ते विमलाभवत्}%॥६२॥

 

\twolineshloka
{अस्मिन् ऊर्जे सिते पक्षे रामायणकथां श्रृणु}
{श्रृणु त्वं राममाहात्म्यं रामभक्तिपरायण}%॥६३॥

 

\twolineshloka
{रामध्यानपराणां च कः समर्थः प्रबाधितुम्}
{रामभक्तिपरो यत्र तत्र ब्रह्मा हरिः शिवः}%॥६४॥

 

\twolineshloka
{तत्र देवाश्च सिद्धाश्च रामायणपरा नराः}
{तस्मात् ऊर्जे सिते पक्षे रामायणकथां श्रृणु}%॥६५॥

 

\twolineshloka
{नवाह्ना खलु श्रोतव्यं सावधानः सदा भव}
{इत्युक्त्वा कथयामास रामायणकथां मुनिः}%॥६६॥

 

\twolineshloka
{कथाश्रवणमात्रेण राक्षसत्वं अपाकृतम्}
{विसृज्य राक्षसं भावं अभवत् देवतोपमः}%॥६७॥

 

\twolineshloka
{कोटिसूर्यप्रतीकाशो नारायण समप्रभः}
{शङ्‍खचक्र गदापाणि हरेः सद्म जगाम सः}%॥६८॥

 
\onelineshloka
{स्तुवन् तं ब्राह्मणं सम्यग् उअगाम हरिमंदिरम्}%॥६९॥

 

\uvacha{नारद उवाच}

\twolineshloka
{तस्मात् श्रूणुध्वं विप्रेन्द्रा रामायण कथामृतम्}
{स तस्य महिमा तत्र ऊर्जे मासि च कीर्त्यते}%॥७०॥

 

\twolineshloka
{यन्नामस्मरणादेव महापातक कोटिभिः}
{विमुक्तः सर्वपापेभ्यो नरो याति परां पगिम्}%॥७१॥

 

\twolineshloka
{रामायणेति यन्नम सकृदप्युच्यते यदा}
{तदैव पापनिर्मुक्तो विष्णुर्लोकं स गच्छति}%॥७२॥

 

\twolineshloka
{ये पठंति सदाऽऽख्यानं भक्त्या श्रृण्वन्ति ये नराः}
{गङ्‍गास्नानच्छतगुणं तेषां संजायते फलम्}%॥७३॥

 

॥इति श्रीस्कान्दे महापुराण उत्तरखण्डे नारद-सनत्कुमार-संवादे रामायणमाहात्म्ये राक्षमोक्षणं नाम द्वितीयोऽध्यायः॥२॥
 

\sect{तृतीयोध्यायः—माघफलानुकीर्तनम्}
 

\uvacha{सनत्कुमार उवाच}

\twolineshloka
{अहो विप्र इदं प्रोक्तं इतिहासं च नारद}
{रामायणस्य माहात्म्यं त्वं पुनर्वद विस्तरात्}%॥१॥

 

\twolineshloka
{अन्यमासस्य माहात्म्यं कथयस्व प्रसादतः}
{कस्य नो जायते तुष्टिः मुने त्वद् वचनामृतात्}%॥२॥

 

\uvacha{नारद उवाच}

\twolineshloka
{सर्वे यूयं महाभागाः कृतार्था नात्र संशयः}
{यतः प्रभावं रामस्य भक्तितः श्रोतुमुद्यताः}%॥३॥

 

\twolineshloka
{माहात्म्यश्रवणं यस्य राघवस्य कृतात्मनाम्}
{दुर्लभं प्राहुरत्यन्तं मुनयो ब्रह्मवादिनः}%॥४॥

 

\twolineshloka
{श्रृणुध्वं ऋषयश्चित्रं इतिहासं पुरातनम्}
{सर्वपापप्रशमनं सर्वरोगविनाशनम्}%॥५॥

 

\twolineshloka
{आसीत् पुरा द्वापरे च सुमतिर्नाम भूपतिः}
{सोमवंशोद्‌भवः श्रीमान् सप्तद्वीपैकनायकः}%॥६॥

 

\twolineshloka
{धर्मात्मा सत्यसम्पन्नः सर्वसम्पद् विभूषितः}
{सदा रामकथासेवी रामपूजापरायणः}%॥७॥

 

\twolineshloka
{रामपूजापराणां च शुश्रूषुरनहंकृतिः}
{पूज्येषु पूजानिरतः समदर्शी गुणान्वितः}%॥८॥

 

\twolineshloka
{सर्वभूतहितः शान्तः कृतज्ञः कीर्त्तिमान् नृपः}
{तस्य भार्या महाभागा सर्वलक्षणसंयुताः}%॥९॥

 

\twolineshloka
{पतिव्रता पतिप्राणा नाम्ना सत्यवती श्रुता}
{तावुभौ दम्पती नित्यं रामायणपरयणौ}%॥१०॥

 

\twolineshloka
{अन्नदानरतौ नित्यं जलदानपरायणौ}
{तडागारामवाप्यादीन् असंफ्यान् वितेनतुः}%॥११॥

 

\twolineshloka
{सोऽपि राजा महाभागो रामायणपरायणः}
{वाचयेच्छृणुयाद वापि भक्तिभावेन भावितः}%॥१२॥

 

\twolineshloka
{एवं रामपरं नित्यं राजानं धर्मकोविदम्}
{तस्य प्रियां सत्यवतीं देवा अपि सदास्तुवन्}%॥१३॥

 

\twolineshloka
{विश्रुतौ त्रिषु लोकेषु दम्पती तौ हि धार्मिकौ}
{आययौ बहुभिः शिष्यैः द्रष्टुकामो विभाण्डकः}%॥१४॥

 

\twolineshloka
{विभाण्डकं मुनिं दृष्ट्वा सुखमाप्तो जनेश्वरः}
{प्रत्युद्ययौ सपत्‍नीकः पूजाभिर्बहुविस्तरम्}%॥१५॥

 

\twolineshloka
{कृतातिथ्यक्रियं शान्तं कृतासनपरिग्रहम्}
{निजासनगतो भूपः प्राञ्जलिर्मुनिमब्रवीत्}%॥१६॥

 

\uvacha{राजोवाच}

\twolineshloka
{भगवन् कृतकृत्योऽद्य त्वदभ्यागमनेन भोः}
{सतमागमनं सन्तः प्रशंसन्ति सुखावहम्}%॥१७॥

 

\twolineshloka
{यत्र स्यान्महतां प्रेम तत्र स्युः सर्वसम्पदः}
{तेजः कीर्तिर्धनं पुत्र इति प्राहुर्विपश्चितः}%॥१८॥

 

\twolineshloka
{तत्र वृद्धिं गमिष्यन्ति श्रेयांस्यनुदिनं मुने}
{यत्र सन्तः प्रकुर्वन्ति महतीं करुणां प्रभो}%॥१९॥

 

\twolineshloka
{यो मूर्ध्नि धारयेद् ब्रह्मन् विप्रपादतलोदकम्}
{स स्नातो सर्वतीर्थेषु पुण्यवान् नात्र संशयः}%॥२०॥

 

\twolineshloka
{मम पुत्राश्च दाराश्च सम्पदश्च समर्पिताः}
{समाज्ञापय शान्तात्मन् वयं किं करवाणि ते}%॥२१॥

 

\twolineshloka
{इत्थं वदन्तं भूपं तं स निरीक्ष्य मुनीश्वरः}
{स्पृशन् करेण राजानं प्रत्युवाचातिहर्षितः}%॥२२॥

 

\uvacha{ऋषिरुवाच}

\twolineshloka
{राजन् यदुक्तं भवता तत्सर्वं त्वत्कुलोचितम्}
{विनयावनताः सर्वे परं श्रेयो भजन्ति हि}%॥२३॥

 

\twolineshloka
{प्रीतोऽस्मि तव भूपाल सन्मार्गपरिवर्तिनः}
{स्वस्ति तेऽस्तु महाभाग यत्पृच्छामि तदुच्यताम्}%॥२४॥

 

\twolineshloka
{हरिसंतोषकान्यासन् पुराणानि बहून्यपि}
{माघे मासि चोद्यतोऽसि रामायणपरायणः}%॥२५॥

 

\twolineshloka
{तव भार्यापि साध्वीयं नित्यं रामपरायणा}
{किमर्थमेतद् वृत्तान्तं यथावद् वक्तुमर्हसि}%॥२६॥

 

\uvacha{राजोवाच}

\twolineshloka
{श्रृणुष्व भगवन् सर्वं यत्पृच्छसि वदामि तत्}
{आश्चर्यं यद्धि लोकानां आवयोश्चरितं मुने}%॥२७॥

 

\twolineshloka
{अहमासं पुरा शुद्रो मालतिर्नाम सत्तम}
{कुमार्गनिरतो नित्यं सर्वलोकाहिते रतः}%॥२८॥

 

\twolineshloka
{पिशुनो धर्मविद्वेषी देवद्रव्यापहारकः}
{महापातकिसंसर्गी देवद्रव्योपजीवकः}%॥२९॥

 

\twolineshloka
{गोघ्नश्च ब्रह्महा चौरो नित्यं प्राणिवधे रतः}
{नित्यं निष्ठुरवक्ता च पापी वेश्यापरायणः}%॥३०॥

 

\twolineshloka
{किञ्चित् काले स्थितो ह्येवं अनादृत्य महद्वचः}
{सर्वबन्धुपरित्यक्तो दुःखी वनमुपागमम्}%॥३१॥

 

\twolineshloka
{मृगमांसाशनं नित्यं तथा मार्गविरोधकृत्}
{एकाकी दुःखबहुलो न्यवसं निर्जने वने}%॥३२॥

 

\twolineshloka
{एकदा क्षुत्परिश्रान्तो निद्राघूर्णः पिपासितः}
{वसिष्ठस्याश्रमं दैवाद् अपश्यं निर्जने वने}%॥३३॥

 

\twolineshloka
{हंसकारण्डवाकीर्णं तत्समीपे महत्सरः}
{पर्यन्ते वनपुष्पौघैः छादितं तन्मुनीश्वर}%॥३४॥

 

\twolineshloka
{अपिबं तत्र पानीयं तत्तटे विगतश्रमः}
{उन्मूल्य वृक्षमूलानि मया क्षुच्च निवारिता}%॥३५॥

 

\twolineshloka
{वसिष्ठस्याश्रमे तत्र निवासं कृतवानहम्}
{शीर्णस्फटिक संधानं तत्र चाहमकारिषम्}%॥३६ ॥॥

 

\twolineshloka
{पर्णैस्तृणैश्च काष्ठैश्च गृहं सम्यक् प्रकल्पितम्}
{तत्राहं व्याधसत्त्वस्थो हत्वा बहुविधान् मृगान्}%॥३७॥

 

\twolineshloka
{आजीविकां च कुर्वाणो वत्सराणां च विंशतिम्}
{अथेयमागता साध्वी विन्ध्यदेशसमुद्‌भवा}%॥३८॥

 

\twolineshloka
{निषादकुलसम्भूता नाम्ना कालीति विश्रुता}
{बन्धुवर्गैः परित्यक्ता दुःखिता जीर्णविग्रहा}%॥३९॥

 

\twolineshloka
{ब्रह्मन् क्षुत्तृट्परिश्रान्ता शोचन्ती भौक्तिकीं क्रियाम्}
{दैवयोगात् समायाता भ्रमन्ती विजने वने}%॥४०॥

 

\twolineshloka
{मासे ग्रीष्मे च तापार्त्ता ह्यन्तस्तापप्रपीडिता}
{इमां दुःखवतीं दृष्ट्वा जाता मे विपुला घृणा}%॥४१॥

 

\twolineshloka
{मया दत्तं जलं चास्यै मांसं वनफलं तथा}
{गतश्रमा तु सा पृष्टा मया ब्रह्मन् यथातथम्}%॥४२॥

 

\twolineshloka
{न्यवेदयत् स्वकर्माणि तानि श्रृणु महामुने}
{इयं काली तु नाम्ना वै निषादकुलसम्भवा}%॥४३॥

 

\twolineshloka
{दाम्भिकस्य सुता विद्वन् न्यवसद् विन्ध्यपर्वते}
{परस्वहारिणी नित्यं सदा पैशुन्यवादिनी}%॥४४॥

 

\twolineshloka
{बन्धुवर्गैः परित्यक्ता यतो हतवती पतिम्}
{कान्तारे विजने ब्रह्मन् मत्समीपं उपागता}%॥४५॥

 

\twolineshloka
{इत्येवं स्वकृत्वं कर्म सर्वं मह्यं न्यवेदयत्}
{वसिष्ठस्याश्रमे पुण्ये अहं चेयं च वै मुने}%॥४६॥

 

\twolineshloka
{दम्पतीभावमाश्रित्य स्थितौ मांसाशिनौ तदा}
{उद्यमार्थे गतौ चैव वसिष्ठस्याश्रमं तदा}%॥४७॥

 

\twolineshloka
{दृष्ट्वा चैव समाजं च देवर्षिणां च सत्तम}
{रामायणपरा विप्रा माघे दृष्ट्वा दिने दिने}%॥४८॥

 

\twolineshloka
{निराहारौ च विक्रान्तौ क्षुत्पिपासाप्रपीडितौ}
{अनिच्छया गतौ तत्र वसिष्ठस्याश्रमं प्रति}%॥४९॥

 

\twolineshloka
{रामायणकथां श्रोतुं नवाह्ना चैव भक्तितः}
{तत्काल एव पञ्चत्वं आवयोरभवन्मुने}%॥५०॥

 

\twolineshloka
{कर्मणा तेन तुष्टात्मा भगवान् मधुसूदनः}
{स्वदूतान् प्रेषयामास मदाहरणकारणात्}%॥५१॥

 

\twolineshloka
{आरोप्य मां विमाने तु जग्मुस्ते च परं पदम्}
{आवां समीपमापन्नौ देवदेवस्य चक्रिणः}%॥५२॥

 

\twolineshloka
{भुक्तवन्तौ महाभोगान् यावत्कालं श्रृणुष्व मे}
{युगकोटिसहस्राणि युगकोटिशतानि च}%॥५३॥

 

\twolineshloka
{उषित्वा रामभवने ब्रह्मलोकमुपागतौ}
{तावत्कालं च तत्रापि स्थित्वैन्द्रपदमागतौ}%॥५४॥

 

\twolineshloka
{तत्रापि तावत्कालं च भुक्त्वा भोगाननुत्तमान्}
{ततः पृथ्वीं वयं प्राप्ताः क्रमेण मुनिसत्तम}%॥५५॥

 

\twolineshloka
{अत्रापि सम्पदतुला रामायणप्रसादतः}
{अनिच्छया कृतेनापि प्राप्तं एवंविधं मुने}%॥५६॥

 

\twolineshloka
{नवाह्ना किल श्रोतव्यं रामायणकथामृतम्}
{भक्तिभावेन धर्मात्मन् जन्ममृत्युजरापहम्}%॥५७॥

 

\twolineshloka
{अवशेनापि यत्कर्म कृतं तु सुमहत्फलम्}
{ददाति श्रृणु विप्रेन्द्र रामायणप्रसादतः}%॥५८॥

 

\uvacha{नारद उवाच}

\twolineshloka
{एतत्सर्वं निशम्यासौ विभाण्डको मुनीश्वरः}
{अभिनन्द्य महीपालं प्रययौ स्वतपोवनम्}%॥५९॥

 

\twolineshloka
{तमाच्छृणुध्वं विप्रेन्द्रा देवदेवस्य चक्रिणः}
{रामायणकथा चैव कामधेनूपमा स्मृता}%॥६०॥

 

\twolineshloka
{माघे मासे सिते पक्षे रामायणं प्रयत्‍नतः}
{नवाह्ना किल श्रोतव्यं सर्वधर्मफलप्रदम्}%॥६१॥

 

\twolineshloka
{य इदं पुण्यमाख्यानं सर्वपापप्रणाशनम्}
{वाचयेत् श्रृणुयाद् वापि रामभक्तश्च जायते}%॥६२॥

 

॥इति श्रीस्कान्दे महापुराण उत्तरखण्डे नारद-सनत्कुमार-संवादे रामायणमाहात्म्ये माघफलानुकीर्तनं नाम तृतीयोऽध्यायः॥३॥
 

\sect{चतुर्थोऽध्यायः—चैत्रमसफलानुकीर्तनम्}
 

\uvacha{नारद उवाच}

\twolineshloka
{अन्यमासं प्रवक्ष्यामि श्रृणुध्वं सुसमाहिताः}
{सर्वपापहरं पुण्यं सर्वदुःख निबर्हणम्}%॥१॥

 

\twolineshloka
{ब्राह्मणक्षत्रियविशां शूद्राणां चैव योषिताम्}
{समस्तकामफलदं सर्वव्रत फलप्रदम्}%॥२॥

 

\twolineshloka
{दुःस्वप्ननाशनं धन्यं भुक्तिमुक्ति फलप्रदम्}
{रामायणस्य माहात्म्यं श्रोतव्यं च प्रयत्‍नतः}%॥३॥

 

\twolineshloka
{अत्रैवोदाहरन्ति इमं इतिहासं पुरातनम्}
{पठतां श्रृण्वतां चैव सर्वपाप प्रणाशनम्}%॥४॥

 

\twolineshloka
{आसीत् पुता कलियुगे कलिको नाम लुब्धकः}
{परदार परद्रव्य हरणे सततं रतः}%॥५॥

 

\twolineshloka
{परनिन्दापरो नित्यं जन्तुपीडाकरस्तथा}
{हतवान् ब्राह्मणान् गावः शतशोऽथ सहस्रशः}%॥६॥

 

\twolineshloka
{देवस्वहरणे नित्यं परस्वहरणे तथा}
{तेन पापान्यनेकानि कृतानि सुमहान्ति च}%॥७॥

 

\twolineshloka
{न तेषां शक्यते वक्तुं संख्या वत्सरकोटिभिः}
{स कदाचिन्महापापो जन्तूनामन्तकोपमः}%॥८॥

 

\twolineshloka
{सौवीरनगरं प्राप्तः सर्वैश्वर्यसमन्वितम्}
{योषिद्‌भिर्भूषिताभिश्च सरोभिर्विमलोदकैः}%॥९॥

 

\twolineshloka
{अलंकृतं विपणिभिः ययौ देवपुरोपमम्}
{तस्योपवनमध्यस्थं रम्यं केशवमन्दिरम्}%॥१०॥

 

\twolineshloka
{छादितं हेमकलशैः दृष्ट्वा व्याधो मुदं ययौ}
{हराम्यत्र सुवर्णानि बहूनीति विनिश्चितः}%॥११॥

 

\twolineshloka
{जगाम रामभवनं कीनाशश्चौर्यलोलुपः}
{तत्रापश्यद् द्विजवरं शान्तं तत्त्वार्थकोविदम्}%॥१२॥

 

\twolineshloka
{परिचर्यापरं विष्णोः उत्तङ्कं तपसां निधिम्}
{एकाकिनं दयालुं च निःस्पृहं ध्यानलोलुपम्}%॥१३॥

 

\twolineshloka
{दृष्ट्वासौ लुब्धको मेने तं चौर्यस्यान्तरायिणम्}
{देवस्य द्रव्यजातं तु समादाय महानिशि}%॥१४॥

 

\twolineshloka
{उत्तङ्कं हन्तुमारेभे उद्यतासिर्मदोद्धतः}
{पादेनाक्रम्य तद्वक्षो गलं संगृह्य पाणिना}%॥१५॥

 

\onelineshloka*
{हन्तुं कृतमतिं व्याधं उत्तङ्कं प्रेक्ष्य चाब्रवीत्}

\uvacha{उत्तङ्क उवाच}

\onelineshloka
{भो भोः साधो वृथा मां त्वं हनिष्यसि निरागसम्}%॥१६॥

 

\twolineshloka
{मया किमपराधं ते तद् वद त्वं च लुब्धक}
{कृतापराधिनो लोके हिंसां कुर्वन्ति यत्‍नतः}%॥१७॥

 

\twolineshloka
{ह हिंसन्ति वृथा सौम्य सज्जना अप्यपापिनम्}
{विरोधिष्वपि मूर्खेषु निरीक्ष्यावस्थितान् गुणान्}%॥१८॥

 

\twolineshloka
{विरोधं नाधिगच्छन्ति सज्जनाः शान्तचेतसः}
{बहुधा वाच्यमानोपि यो नरः क्षमयान्वितः}%॥१९॥

 
\onelineshloka
{तमुत्तमं नरं प्राहुः विष्णोः प्रियतरं तथा}%॥२०॥

 

\twolineshloka
{सुजनो न याति वैरं परहितनिरतो विनाशकालेऽपि}
{छेदेऽपि चंदनतरु सुरभीकरोति मुखं कुठारस्य}%॥२१॥

 

\twolineshloka
{अहो विधिर्वै बलवान् बाधते बहुधा जनान्}
{सर्वसङ्‌‍गविहीनोऽपि बाध्यते तु दुरात्मना}%॥२२॥

 

\twolineshloka
{अहो निष्कारणं लोके बाधन्ते दुर्जना जनान्}
{धीवराः पिशुना व्याधा लोकेऽकारणवैरिणः}%॥२३॥

 

\twolineshloka
{अहो बलवती माया मोहत्यखिलं जगत्}
{पुत्रमित्रकलत्राद्यैः सर्वदुःखेन योज्यते}%॥२४॥

 

\twolineshloka
{परद्रव्यापहारेण कलत्रं पोषितं च यत्}
{अन्ते तत् सर्वमुत्सृज्य एक एव प्रयाति वै}%॥२५॥

 

\twolineshloka
{मम माता मम पिता मम भार्या ममात्मजाः}
{ममेदं इति जन्तूनां ममता बाधते वृथा}%॥२६॥

 

\twolineshloka
{यावद् अर्पयति द्रव्यं तावद् भवति बान्धवः}
{अर्जितं तु धनं सर्वे भुञ्जन्ते बान्धवाः सदा}%॥२७॥

 

\twolineshloka
{दुःखमेकतमो मूढः तत्पाप फलमश्नुते}
{इति ब्रुवाणं तं ऋषिं विमृश्य भयविह्वलः}%॥२८॥

 

\twolineshloka
{कलिकः प्राञ्जलिः प्राह क्षमस्वेति पुनः पुनः}
{तत्सङ्‍गस्य प्रभावेण हरिसंनिधिमात्रतः}%॥२९॥

 

\twolineshloka
{गतपापो लुब्धकश्च सानुतापोऽभवद् ध्रुवम्}
{मया कृतानि पापानि महान्ति सुबहूनि च}%॥३०॥

 

\twolineshloka
{तानि सर्वाणि नष्टानि विप्रेन्द्र तव दर्शनात्}
{अहं वै पापधीर्नित्यं पहापापं समाचरम्}%॥३१॥

 

\twolineshloka
{कथं मे निष्कृतिभूयात् कं यामि शरणं विभो}
{पूर्वजन्मार्जितैः पापैः लुब्धकत्वं अवाप्तवान्}%॥३२॥

 

\twolineshloka
{अत्रापि पापजालानि कृत्वा कां गतिमाप्नुयाम्}
{इति वाक्यं समाकर्ण्य कलिकस्य महात्मनः}%॥३३॥

 
\onelineshloka*
{उत्तङ्को नाम विप्रर्षिः इदं वाक्यं अथाब्रवीत्}

\uvacha{उत्तङ्क उवाच}

\onelineshloka
{साधु साधु महाप्राज्ञ मतिस्ते विमलोज्ज्वला}%॥३४॥

 

\twolineshloka
{यस्मात् संसारदुःखानां नाशोपायं अभीप्ससि}
{चैत्रे मासि सिते पक्षे कथा रामायणस्य च}%॥३५॥

 

\twolineshloka
{नवाह्ना किल श्रोतव्या भक्तिभावेन सादरम्}
{यस्य श्रवणमात्रेण सर्वपापैः प्रमुच्यते}%॥३६॥

 

\twolineshloka
{तस्मिन् क्षणेऽसौ कलिको लुब्धको वीतकल्मषः}
{रामायणकथां श्रुत्वा सद्यः पञ्चत्वमागतः}%॥३७॥

 

\twolineshloka
{उत्तङ्कः पतितः वीक्ष्य लुब्धकं तं दयापरः}
{एतद् दृष्ट्वा विस्मितश्च अस्तौषीत् कमलापतिम्}%॥३८॥

 

\twolineshloka
{कथां रामायणस्यापि श्रुत्वा च वीतकल्मषः}
{दिव्यं विमानमारुह्य मुनिमेतदथाब्रवीत्}%॥३९॥

 

\twolineshloka
{विमुक्तस्त्वत् प्रसादेन महापातक संकटात्}
{तस्मान्नतोऽस्मि ते विद्वन् यत् कृतं तत् क्षमस्व मे}%॥४०॥

 

\uvacha{सूत उवाच}

\twolineshloka
{इत्युक्त्वा देवकुसुमैः मुनिश्रेष्ठमवाकिरत्}
{प्रदक्षिणात्रयं कृत्वा नमस्कारं चकार ह}%॥४१॥

 

\twolineshloka
{ततो विमानमारुह्य सर्वकामसमन्वितम्}
{अप्सरोगणसंकीर्णं प्रपेदे हरिमन्दिरम्}%॥४२॥

 

\twolineshloka
{तस्मात् श्रृणुध्वं विप्रेन्द्राः कथां रामायणस्य च}
{चैत्रे मासि सिते पक्षे श्रोतव्यं च प्रयत्‍नतः}%॥४३॥

 

\twolineshloka
{नवाह्ना किल रामस्य रामायण कथामृताम्}
{तस्माद् ऋतुषु सर्वेषु हितकृत् हरिपूजकः}%॥४४॥

 

\twolineshloka
{ईप्सितं मनसा यद्यत् तदाप्नोति न संशयः}
{सनत्कुमार यत् पृष्टं तत् सर्वं गदितं मया}%॥४५॥

 
\onelineshloka
{रामायणस्य माहात्म्यं किमन्यत् श्रोतुमिच्छसि}%॥४६॥

 

॥इति श्रीस्कान्दे महापुराण उत्तरखण्डे नारद-सनत्कुमार-संवादे रामायणमाहात्म्ये चैत्रमसफलानुकीर्तनं नाम चतुर्थोऽध्यायः॥४॥



\sect{पञ्चमोऽध्यायः—फलानुकीर्तनम्}
 

\uvacha{सूत उवाच}

\twolineshloka
{रामायणस्य माहात्म्यं श्रुत्वा प्रीतो मुनीश्वरः}
{सनत्कुमारः पप्रच्छ नारदं मुनिसत्तमम्}%॥१॥

 

\uvacha{सनत्कुमार उवाच}

\twolineshloka
{रामायणस्य माहात्म्यं कथितं वै मुनिश्वर}
{इदानीं श्रोतुमिच्छामि विधिं रामायणस्य च}%॥२॥

 

\twolineshloka
{एतच्चापि महाभाग मुने तत्त्वार्थकोविद}
{कृपया परयाविष्टो यथावद् वक्तुमर्हसि}%॥३॥

 

\uvacha{नारद उवाच}

\twolineshloka
{रामायणविधिं चैव श्रृणुध्वं सुसमाहिताः}
{सर्वलोकेषु विख्यातं स्वर्गमोक्षविवर्धनम्}%॥४॥

 

\twolineshloka
{विधानं तस्य वक्ष्यामि श्रृणुध्वं गदतो मम}
{रामायणकथां कुर्वन् भक्तिभावेन भावितः}%॥५॥

 

\twolineshloka
{येन चीर्येण पापानां कोटिकोटिः प्रणश्यति}
{चैत्रे माघे कार्त्तिके च पञ्चम्यां अथवाऽऽरभेत्}%॥६॥

 

\twolineshloka
{संकल्पं तु ततः कुर्यात् स्वस्तिवाचनपूर्वकम्}
{अहोभिर्नवभिः श्राव्यं रामायण कथामृतमम्}%॥७॥

 

\twolineshloka
{अद्य प्रभृत्यहं राम श्रृणोमि त्वत्कथामृतम्}
{प्रत्यहं पूर्णतामेतु तव राम प्रसादतः}%॥८॥

 

\twolineshloka
{प्रत्यहं दन्तशुद्धिं च अपामार्गस्य शाखया}
{कृत्वा स्नायीत विधिवत् रामभक्तिपरायणः}%॥९॥

 

\twolineshloka
{स्वयं च बन्धुभिः सार्द्धं श्रृणुयात् प्रयतेन्द्रियः}
{स्नानं कृत्वा यथाचारं दन्तधावनपूर्वकम्}%॥१०॥

 

\twolineshloka
{शुक्लाम्बरधरः शुद्धो गृहमागत्य वाग्यतः}
{प्रक्षाल्य पादौ आचम्य स्मरेन्नारायणं प्रभुम्}%॥११॥

 

\twolineshloka
{नित्यं देवार्चनं कृत्वा पश्चात् संकल्पपूर्वकम्}
{रामायणपुस्तकं च अर्चयेत् भक्तिभावतः}%॥१२॥

 

\twolineshloka
{आवाहन आसनाद्यैश्च गन्धपुष्पादिभिर्व्रती}
{ॐ नमो नारायणायेति पूजयेत् भक्तितत्परः}%॥१३॥

 

\twolineshloka
{एकवारं द्विवारं वा त्रिवारं वापि शक्तितः}
{होमं कुर्यात् प्रयत्‍नेन सर्वपापनिवृत्तये}%॥१४॥

 

\twolineshloka
{एवं यः प्रयतः कुर्याद् रामायणविधिं तथा}
{स याति विष्णुभवनं पुनरावृत्तिदुर्लभम्}%॥१५॥

 

\twolineshloka
{रामायणव्रतधरो धर्मकारी च सत्तमः}
{चाण्डालं पतितं वापि वस्त्रान्नेनापि नार्चयेत्}%॥१६॥

 

\twolineshloka
{नास्तिकान् भिन्नमर्यादान् निन्दकान् पिशुनानपि}
{रामायणव्रतपरो वाङ्‌‍मात्रेणापि नार्चयेत्}%॥१७॥

 

\twolineshloka
{कुण्डाशिनं गायकं च तथा देवलकाशनम्}
{भिषजं काव्यकर्तारं देवद्विज विरोधिनम्}%॥१८॥

 

\twolineshloka
{परान्नलोलुपं चैव परस्त्रीनिरतं तथा}
{रामायणव्रतपरो वाङ्‌मात्रेणापि नार्चयेत्}%॥१९॥

 

\twolineshloka
{इत्येवमादिभिः शुद्धो वशी सर्वहिते रतः}
{रामायणपरो भूत्वा परां सिद्धिं गमिष्यति}%॥२०॥

 

\twolineshloka
{नास्ति गङ्‍गासमं तीर्थं नास्ति मातृसमो गुरुः}
{नास्ति विष्णुसमो देवो नास्ति रामायणात् परम्}%॥२१॥

 

\twolineshloka
{नास्ति वेदसमं शास्त्रं नास्ति शान्तिसमं सुखम्}
{नास्ति शान्तिपरं ज्योतिः नास्ति रामायणात् परम्}%॥२२॥

 

\twolineshloka
{नास्ति क्षमासमं सारं नास्ति कीर्तिसमं धनम्}
{नास्ति ज्ञानसमो लाभो नास्ति रामायणात् परम्}%॥२३॥

 

\twolineshloka
{तदन्ते वेदविदुषे गां दद्याच्च सदक्षिणाम्}
{रामायणं पुस्तकं च वस्त्रालंकरणादिकम्}%॥२४॥

 

\twolineshloka
{रामायणपुस्तकं यो वाचकाय प्रयच्छति}
{स यति विष्णुभवनं यत्र गत्वा न शोचति}%॥२५॥

 

\twolineshloka
{नवाहजफलं कर्तुः श्रृणु धर्मविदां वर}
{पञ्चम्यां तु समारभ्य रामायण कथामृतम्}%॥२६॥

 

\twolineshloka
{कथाश्रवणमात्रेण सर्वपापैः प्रमुच्यते}
{यदि द्वयं कृतं तस्य पुण्डरीकफलं लभेत्}%॥२७॥

 

\twolineshloka
{व्रतधारी तु श्रवणं यः कुर्यात् स जितेन्द्रियः}
{अश्वमेधस्य यज्ञस्य द्विगुणं फलमश्नुते}%॥२८॥

 

\twolineshloka
{चतुःकृत्वः श्रुतं येन कथितं मुनिसत्तमाः}
{स लभेत् परमं पुण्यं अग्निष्टोमाष्टसम्भवम्}%॥२९॥

 

\twolineshloka
{पञ्चकृत्वो व्रतमिदं कृतं येन महात्मना}
{अत्यग्निष्टोमजं पुण्यं द्विगुणं प्राप्नुयान्नरः}%॥३०॥

 

\twolineshloka
{एवं व्रतं च षड्वारं कुर्याद् यस्तु समाहितः}
{अग्निष्टोमस्य यज्ञस्य फलमष्टगुणं लभेत्}%॥३१॥

 

\twolineshloka
{नारी वा पुरुषः कुर्याद् अष्टकृत्वो मुनीश्वराः}
{नरमेधस्य यज्ञस्य फलं पञ्चगुणं लभेत्}%॥३२॥

 

\twolineshloka
{नरो वाप्यथ नारी वा नववारं समाचरेत्}
{गोमेधसवजं पुण्यं स लभेत् त्रिगुणं नरः}%॥३३॥

 

\twolineshloka
{रामायणं तु यः कुर्यात् शान्तात्मा प्रयतेन्द्रियः}
{स याति परमानन्दं यत्र गत्वा न शोचति}%॥३४॥

 

\twolineshloka
{रामायणपरो नित्यं गङ्‌गास्नानपरायणः}
{धर्ममार्ग प्रवक्तारो मुक्ता एवं न संशयः}%॥३५॥

 

\twolineshloka
{यतीनां ब्रह्मचारीणां प्रवीराणां च सतमाः}
{नवाह्ना किल श्रोतव्या कथा रामायणस्य च}%॥३६॥

 

\twolineshloka
{श्रुत्वा नरो रामकथां अतिदीप्तोऽति भक्तितः}
{ब्रह्मणः पदमासाद्य तत्रैव परिमोदते}%॥३७॥

 

\twolineshloka
{तस्मात् श्रृणुध्वं विप्रेन्द्रा रामायण कथामृतम्}
{श्रोतॄणां च परं श्राव्यं पवित्राणां अनुत्तमम्}%॥३८॥

 

\twolineshloka
{दुःस्वप्नप्रनाशनं धन्यं श्रोतव्यं च प्रयत्‍नतः}
{नरोऽत्र श्रद्धया युक्तः श्लोकं श्लोकार्द्धमेव च}%॥३९॥

 

\twolineshloka
{पठते मुच्यते सद्यो हि उपपातककोटिभिः}
{सतामेव प्रयोक्तव्यं गुह्याद् गुह्यतमं तु यत्}%॥४०॥

 

\twolineshloka
{वाचयेत् रामभवने पुण्यक्षेत्रे च संसदि}
{ब्रह्मद्वेषरतानां च दम्भाचार रतात्मनाम्}%॥४१॥

 

\twolineshloka
{लोकवञ्चक वृत्तीनां न ब्रूयाद् इदमुत्तमम्}
{त्यक्त कामादिदोषाणां रामभक्ति रतात्मनाम्}%॥४२॥

 

\twolineshloka
{गुरुभक्तिरतानां च वक्तव्यं मोक्षसाधनम्}
{सर्वदेवमयो रामः स्मृतश्चार्त्तिप्रणाशनः}%॥४३॥

 

\twolineshloka
{सद्‌भक्तवत्सलो देवो भक्त्या तुष्यति नान्यथा}
{अवशेनापि यन्नाम्नि कीर्तिते वा स्मृतेऽपि वा}%॥४४॥

 

\twolineshloka
{विमुक्तपातकः सोऽपि परमं पदमश्नुते}
{संसारघोरकान्तार दावाग्निर्मधुसूदनः}%॥४५॥

 

\twolineshloka
{स्मर्तॄणां सर्वपापानि नाशयत्याशु सत्तमाः}
{तदर्थकमिदं पुण्यं काव्यं श्राव्यं अनुत्तमम्}%॥४६॥

 
\onelineshloka*
{श्रवणाद् पठनाद् वापि सर्वपापविनाशकृत्}

\onelineshloka
{यस्य रामरसे प्रीतिः वर्तते भक्तिसंयुता}%॥४७॥

 

\twolineshloka
{स एव कृतकृत्यश्च सर्व शास्त्रार्थकोविदः}
{तदर्जितं तपः पुण्यं तत्सत्यं सफलं द्विजाः}%॥४८॥

 

\twolineshloka
{यदर्थश्रवणे प्रीतिः अन्यथा न हि वर्तते}
{रामायणपरा ये तु रामनामपरायणाः}%॥४९॥

 

\twolineshloka
{त एव कृतकृत्याश्च घोरे कलियुगे द्विजाः}
{नवाह्ना किल श्रोतव्यं रामायण कथामृतमम्}%॥५०॥

 

\twolineshloka
{ते कृतज्ञा महात्मानः तेभ्यो नित्यं नमो नमः}
{रामनामैव नामैव नामैव मम जीवनम्}%॥५१॥

 

\onelineshloka*
{कलौ नास्त्येव नास्त्येव नास्त्येव गतिरन्यथा}

\uvacha{सूत उवाच}
\onelineshloka
{एवं सनत्कुमारस्तु नारदेन महात्मना}%॥५२॥

 

\twolineshloka
{सम्यक् प्रबोधितः सद्यः परां निर्वृतिमाप ह}
{तस्मात् श्रृणुध्वं विप्रेन्द्रा रामायण कथामृतम्}%॥५३॥

 

\twolineshloka
{नवाह्ना किल श्रोतव्यं सर्वपापैः प्रमुच्यते}
{श्रुत्वा चैतन्महाकाव्यं वाचकं यस्तु पूजयेत्}%॥५४॥

 

\twolineshloka
{तस्य विष्णुः प्रसन्नः स्यात् श्रिया सह द्विजोत्तमाः}
{वाचके प्रीतिमापन्ने ब्रह्मविष्णुमहेश्वराः}%॥५५॥

 

\twolineshloka
{प्रीता भवन्ति विप्रेन्द्रा नात्र कार्या विचारणा}
{रामायाणवाचकाय गावो वासांसि काञ्चनम्}%॥५६॥

 

\twolineshloka
{रामायणं पुस्तकं च दद्यात् वित्तानुसारतः}
{तस्य पुण्यफलं वक्ष्ये श्रृणुध्वं सुसमाहिताः}%॥५७॥

 

\twolineshloka
{न बाधन्ते ग्रहास्तस्य भूतवेतालकादयः}
{तस्यैव सर्वश्रेयांसि वर्द्धन्ते चरिते श्रुते}%॥५८॥

 

\twolineshloka
{न चाग्निर्बाधते तस्य न चौरादिभयं तथा}
{एतज्जन्मार्जितैः पापैः सद्य एव विमुच्यते}%॥५९॥

 

\twolineshloka
{सप्तवंशसमेतस्तु देहान्ते मोक्षमाप्नुयात्}
{इत्येतद्वः समाख्यातं नारदेन प्रभाषितम्}%॥६०॥

 

\twolineshloka
{सनत्कुमारमुनये पृच्छते भक्तितः पुरा}
{रामायणं आदिकाव्यं सर्ववेदार्थ सम्मतम्}%॥६१॥

 

\twolineshloka
{सर्वपापहरं पुण्यं सर्वदुःख निबर्हणम्}
{समस्तपुण्यफलदं सर्वयज्ञ फलप्रदम्}%॥६२॥

 

\twolineshloka
{ये पठन्त्यत्र विबुधाः श्लोकं श्लोकार्द्धमेव च}
{न तेषां पापबन्धस्तु कदाचिदपि जायते}%॥६३॥

 

\twolineshloka
{रामार्पितं इदं पुण्यं काव्यं तु सर्वकामदम्}
{भक्त्य श्रृण्वन्ति विदन्ति तेषां पुणफलं श्रृणु}%॥६४॥

 

\twolineshloka
{शतजन्मार्जितैः पापैः सद्य एव विमोचिताः}
{सहस्रकुलसंयुक्तैः प्रयान्ति परमं पदम्}%॥६५॥

 

\twolineshloka
{किं तीर्थैर्गोपदानैर्वा किं तपोधिः कमध्वरैः}
{अहन्यहनि रामस्य कीर्तनं परिश्रृण्वताम्}%॥६६॥

 

\twolineshloka
{चैत्रे माघे कार्तिके च रामायण कथामृतम्}
{नवैरहोभिः श्रोतव्यं रामायण कथामृतम्}%॥६७॥

 

\twolineshloka
{रामप्रसादजनकं रामभक्तिविवर्धनम्}
{सर्वपापक्षयकरं सर्वसम्पद् विवर्धनम्}%॥६८॥

 

\twolineshloka
{यस्त्वेतत् श्रृणुयाद् वापि पठेद् वा सुसमाहितः}
{सर्वपापविनिर्मुक्तो विष्णुलोकं स गच्छति}%॥६९॥

 

॥इति श्रीस्कान्दे महापुराण उत्तरखण्डे नारद-सनत्कुमार-संवादे रामायणमाहात्म्ये फलानुकीर्तनं नाम पञ्चमोऽध्यायः॥५॥
 

