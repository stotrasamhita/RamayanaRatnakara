% !TeX root = ./ramayana-sangraha-mulam
%! Tex program = latexmk -xelatex
\documentclass[twoside,12pt]{book}
\usepackage{emptypage}
\usepackage{etoolbox}
\newbool{kindle}
\newbool{print}
\setbool{kindle}{false}
\setbool{print}{false}
\ifbool{kindle}{\usepackage[paperwidth=126mm,paperheight=168mm,left=5mm,right=5mm,top=15mm,bottom=20mm]{geometry}}{\usepackage[top=1.5cm, bottom=1.8cm, left=1.5cm, right=1.5cm,paperwidth=148mm,paperheight=210mm]{geometry}}

% !TeX program = XeLaTeX
% !TeX root = ../ramayana-ratnakara-mulam.tex
\usepackage{shloka}
\usepackage{wallpaper}
\usepackage{charter,fbb}

\setmainfont[Script=Devanagari]{Sanskrit 2003}
\setromanfont{fbb}
\setsansfont{fbb}

%%% HEADERS and FOOTERS %%%
\usepackage{fancyhdr}
\pagestyle{fancyplain}
\setlength{\headheight}{28pt}
\lhead[\fancyplain{\rightmark}{\pagenumfont\large\thepage}]
   {\fancyplain{\rightmark}{\leftmark}}
\rhead[\fancyplain{\rightmark}{\leftmark}]
   {\fancyplain{\rightmark}{\pagenumfont\large\thepage}}
\cfoot{}

\fancypagestyle{fancyplain}{ %
\fancyhf{} % remove everything
\renewcommand{\headrulewidth}{0pt} % remove lines as well
\renewcommand{\footrulewidth}{0pt}
\cfoot{\pagenumfont\large\thepage}}

%%% SECTIONS and CHAPTERS %%%
\makeatletter
\renewcommand\section{\resetShloka\@startsection {section}{1}{\z@}%
%{2.3ex \@plus.2ex}%
%{-3.5ex \@plus -1ex \@minus -.2ex}%
%{2.3ex \@plus.2ex}%
{10pt}
{2pt}
{\normalfont\LARGE\bfseries}}

\renewcommand\chapter{\resetShloka\@startsection {chapter}{1}{\z@}%
{10pt}
{2pt}
{\normalfont\LARGE\bfseries}}
\makeatother

\setcounter{secnumdepth}{-1}
%for weird reasons this does not bookmark the section start, but the start of text in the section!!!
%\renewcommand\thesection{}
\renewcommand{\sectionmark}[1]{%
\markboth{\large #1}{\rightmark}
}
\renewcommand{\subsectionmark}[1]{%
\markboth{\large #1}{\rightmark}
}
\renewcommand{\chaptermark}[1]{%
\markboth{\large #1}{\rightmark}
}

\addtolength{\parskip}{4pt}
%\addtolength{\headsep}{10pt}
\setlength{\columnseprule}{1pt}
\setlength{\columnsep}{30pt}

%%% HYPERLINKS %%%
\usepackage[bookmarks=true,bookmarksopen=true,xetex,colorlinks=true,
linkcolor=black,					% colour of internal links
citecolor=cyan,					% colour of links to bibliography
filecolor=magenta,			% colour of file links
urlcolor=black					% colour of external links
]{hyperref}

%%% MISCELLANEOUS %%%
\hbadness=10000
\vbadness=10000
\hfuzz=6pt
%\listfiles

%% MACROS
\usepackage{fontawesome5} % Requires xelatex or lualatex
\usepackage{tcolorbox}
\tcbuselibrary{breakable, skins}

\newcommand{\src}[4]{%
  \def\sourceText{#1}%
  \def\sourceKanda{#2}%
  \def\sourceChapter{#3}%
  \def\sourceVerses{#4}%
  \def\fullSource{\sourceText}%
  %
  \if\relax\detokenize{#2}\relax
    % empty Kanda → do nothing
  \else
    \edef\fullSource{\fullSource / \sourceKanda}%
  \fi
  %
  \if\relax\detokenize{#3}\relax
    % empty Chapter → do nothing
  \else
    \edef\fullSource{\fullSource / \sourceChapter}%
  \fi
  %
  \if\relax\detokenize{#4}\relax
    % empty Verses → do nothing
  \else
    \edef\fullSource{\fullSource / \sourceVerses}%
  \fi
}


\newcommand{\vakta}[1]{\def\vaktaName{#1}}
\newcommand{\shrota}[1]{\def\shrotaName{#1}}
\newcommand{\tags}[1]{\def\tagList{#1}}
\newcommand{\notes}[1]{\def\storyNotes{#1}}
\newcommand{\textlink}[1]{\def\textURL{#1}}
\newcommand{\translink}[1]{\def\transURL{#1}}


\newtcolorbox{StoryMetadataBox}{
  enhanced,
  drop small lifted shadow=red,
  breakable,
  colback=gray!5,
  colframe=gray!5,
  fonttitle=\bfseries,
  % title=\faBookOpen\quad Story Metadata,
  % coltitle=black,
  % sharp corners,
  % boxrule=0.5pt,
  left=1em, right=1em, top=0.5em, bottom=0.5em,
}

\def\extractdomain#1://#2/#3\enddomain{#2}

\def\domainfromurl#1{%
  \expandafter\extractdomain#1/\enddomain
}

\newcommand{\storymeta}{
\begin{StoryMetadataBox}
  \small
\begin{description}
  \item[\faBook]\textbf{\fullSource}
  \ifx\vaktaName\empty \else \item[\faUser\ ~वक्ता ---] \vaktaName\fi
\hspace{2em}\ifx\shrotaName\empty \else \faUserFriends\ \textbf{श्रोता} --- \shrotaName\fi
  % \item[\faCommentDots\ \sffamily\small\footnotesize~Notes:] \textsf{\footnotesize\storyNotes}
  \item[\faCommentDots\sffamily\small\footnotesize]\textsf{\footnotesize\storyNotes}
  % \item[\faTags]\hspace{-0.4ex}\tagList
  \ifx\textURL\empty \else\item[\faLink] \href{\textURL}{Source Text: \domainfromurl{\textURL}} \fi
  \ifx\transURL\empty \else\item[\faGlobe] \href{\transURL}{Translation: \domainfromurl{\transURL}} \fi
\end{description}
\end{StoryMetadataBox}
}

\def\sourceText{}
\def\sourceKanda{}
\def\sourceChapter{}
\def\sourceVerses{}
\def\fullSource{}
\def\vaktaName{}
\def\shrotaName{}
\def\tagList{}
\def\storyNotes{}
\def\textURL{}
\def\transURL{}


\begin{document}
\input{frontmatter/cover-mulam}
% !TeX program = XeLaTeX
% !TeX root = ./ramayana-sangraha-mulam.tex

\begingroup
\fontspec[Script=Devanagari]{Adobe Devanagari}
\fontsize{12pt}{14.4pt}\selectfont
\centerline{\large{ॐ}}
\centerline{॥श्री-गणेशाय नमः॥}
\centerline{॥श्री-गुरुभ्यो नमः॥}
\centerline{॥श्री-सीता-लक्ष्मण-भरत-शत्रुघ्न-हनुमत्-समेत-श्री-रामचन्द्राय नमः॥}

\thispagestyle{empty}

\begin{center}
\chapter*{{प्रस्तावना}}
\end{center}

\twolineshloka*{सदाशिवसमारम्भां शङ्कराचार्यमध्यमाम्}
{अस्मदाचार्यपर्यन्तां वन्दे गुरुपरम्पराम्}

\twolineshloka*
{एष सेतुर्विधरणो लोकासम्भेदहेतवे}
{कोदण्डेन च दण्डेन रामेण गुरुणा कृतः}

रामायण-श्रोतॄणां कदापि तृप्तिर्न जायते! यथा महर्षिः वाल्मीकिः वदति--- ``रामो रामो राम इति प्रजानामभवन् कथाः'', तद्वत् इतिहासपुराणेष्वपि श्रीरामचन्द्रस्य बहवः कथाः लभ्यन्ते। तेषाम् एकत्र प्रस्तुतिं कर्तुम् एषः प्रयासः। सीतादेवी अपि अध्यात्मरामायणे रामस्य अरण्यगमनप्रसङ्गे वदति---

\centerline{``रामायणानि बहुशः श्रुतानि बहुभिर्द्विजैः॥२-४-७७॥''} 

अस्मिन् ग्रन्थे अनेकरामकथाः प्रस्तुताः सन्ति। यद्यपि बहवः कथाः श्रीमाद्वाल्मीकिरामायण\-मनुसृत्य एव वर्तन्ते, काश्चित् कथाः तदतिक्रम्य अपि अन्यकल्पेषु ये केचित्विचित्राः कथाप्रसङ्गाः सन्ति तान् वर्णयन्ति (यथा पद्मपुराणे श्रीरामचन्द्रः स्वयं महादेवं पृच्छति!)। काश्चित् कथाः श्रीमद्वाल्मीकि\-रामायणस्य अन्तर्गत-घट्टानां विस्तृतप्रस्तुतिं कुर्वन्ति। महाभारतेऽपि भीष्मः हनुमान् (स्वानुजं भीमं प्रति) नारदः च विभिन्नेषु प्रसङ्गेषु रामकथां कथयन्ति। विशेषतः वनपर्वणि मारकण्डेयमहर्षिः रामोपाख्यानपर्वणि रामकथां विस्तरेण वर्णयति।

एतासां कथानां वक्तॄन् व्यासं वाल्मीकिं च नमस्कृत्य एतस्य ग्रन्थस्य पारायणम् आरभामहे। रामे अनन्यभक्तिः सदा भवतु नः। 

\twolineshloka*
{नारायणं नमस्कृत्य नरं चैव नरोत्तमम्}
{देवीं सरस्वतीं चैव ततो जयमुदीरयेत्}

\twolineshloka*
{कूजन्तं राम रामेति मधुरं मधुराक्षरम्}
{आरुह्य कविताशाखां वन्दे वाल्मीकिकोकिलम्}


\twolineshloka*
{यत्र यत्र रघुनाथकीर्तनं तत्र तत्र कृतमस्तकाञ्जलिम्}
{बाष्पवारिपरिपूर्णलोचनं मारुतिं नमत राक्षसान्तकम्}

\twolineshloka*
{रामं रामानुजं सीतां भरतं भरतानुजम्}
{सुग्रीवं वायुसूनुं च प्रणमामि पुनः पुनः}

\twolineshloka*
{नमोऽस्तु रामाय सलक्ष्मणाय देव्यै च तस्यै जनकात्मजायै}
{नमोऽस्तु रुद्रेन्द्रयमानिलेभ्यो नमोऽस्तु चन्द्रार्कमरुद्गणेभ्यः}

यथा श्रीमद्भगवद्\-गीतायां भगवान् श्रीकृष्ण आह, ``कथयन्तश्च मां नित्यं तुष्यन्ति च रमन्ति च॥'' तथा वयं सर्वेऽपि रामकथामृतं श्रुत्वा परस्परं च कथयित्वा रामस्य अनन्तकल्याणगुणान् अनुभूय तुष्टिं प्राप्नुयामः! बलं विष्णोः प्रवर्धताम्!\\



\centerline{सर्वम् श्री-सीतारामचन्द्रार्पणमस्तु॥}
\endgroup
\medskip
\noindent\today \hfill कार्तिकः रामसूनूः
\cleardoublepage
\thispagestyle{empty}
\ifbool{print}{%Margin changes for print
\addtolength{\evensidemargin}{-0.5cm}
\addtolength{\oddsidemargin}{0.5cm}}{}
\setmainfont[Script=Devanagari]{Adobe Devanagari}

\setcounter{page}{0}
\pagenumbering{arabic}
\sectionmark{\mbox{}}
\clearpage
% \fontsize{14.4pt}{18pt}\selectfont
\fontsize{16pt}{19.2pt}\selectfont
% \Large
\begin{center}
    \part{राम-चरितानि}
    \input{rama-charitam/valmiki-ramayanam/sankshepa-ramayanam}
    \input{rama-charitam/valmiki-ramayanam/kavya-sankshepa}
    \input{rama-charitam/valmiki-ramayanam/ramavrtta-samshrava}
    \input{rama-charitam/valmiki-ramayanam/hanumajjanaki-samvada}
    \chapt{हनूमदुपदेशः}

\src{श्रीमद्-वाल्मीकि-रामायणम्}{सुन्दरकाण्डः}{अध्यायः ५१}{श्लोकाः १---४६}
\vakta{हनुमान्}
\shrota{रावणादयः}
\tags{concise, part}
\notes{Narration of Rama's story and prowess by Hanuman, in Ravana's court.}
\textlink{}
\translink{}

\storymeta

\twolineshloka
{तं समीक्ष्य महासत्त्वं सत्त्ववान् हरिसत्तमः}
{वाक्यमर्थवदव्यग्रस्तमुवाच दशाननम्}

\twolineshloka
{अहं सुग्रीवसन्देशादिह प्राप्तस्तवालयम्}
{राक्षसेन्द्र हरीशस्त्वां भ्राता कुशलमब्रवीत्}

\twolineshloka
{भ्रातुः शृणु समादेशं सुग्रीवस्य महात्मनः}
{धर्मार्थोपहितं वाक्यमिह चामुत्र च क्षमम्}

\twolineshloka
{राजा दशरथो नाम रथकुञ्जरवाजिमान्}
{पितेव बन्धुर्लोकस्य सुरेश्वरसमद्युतिः}

\twolineshloka
{ज्येष्ठस्तस्य महाबाहुः पुत्रः प्रियकरः प्रभुः}
{पितुर्निदेशान्निष्क्रान्तः प्रविष्टो दण्डकावनम्}

\twolineshloka
{लक्ष्मणेन सह भ्रात्रा सीतया चापि भार्यया}
{रामो नाम महातेजा धर्म्यं पन्थानमाश्रितः}

\twolineshloka
{तस्य भार्या वने नष्टा सीता पतिमनुव्रता}
{वैदेहस्य सुता राज्ञो जनकस्य महात्मनः} 

\twolineshloka
{स मार्गमाणस्तां देवीं राजपुत्रः सहानुजः}
{ऋश्यमूकमनुप्राप्तः सुग्रीवेण समागतः}

\twolineshloka
{तस्य तेन प्रतिज्ञातं सीतायाः परिमार्गणम्}
{सुग्रीवस्यापि रामेण हरिराज्यं निवेदितम्}

\twolineshloka
{ततस्तेन मृधे हत्वा राजपुत्रेण वालिनम्}
{सुग्रीवः स्थापितो राज्ये हर्यृक्षाणां गणेश्वरः}

\twolineshloka
{त्वया विज्ञातपूर्वश्च वाली वानरपुङ्गवः}
{रामेण निहतः सङ्ख्ये शरेणैकेन वानरः}

\twolineshloka
{स सीतामार्गणे व्यग्रः सुग्रीवः सत्यसङ्गरः}
{हरीन् सम्प्रेषयामास दिशः सर्वा हरीश्वरः}

\twolineshloka
{तां हरीणां सहस्राणि शतानि नियुतानि च}
{दिक्षु सर्वासु मार्गन्ते ह्यधश्चोपरि चाम्बरे}

\twolineshloka
{वैनतेयसमाः केचित् केचित्तत्रानिलोपमाः}
{असङ्गगतयः शीघ्रा हरिवीरा महाबलाः}

\twolineshloka
{अहं तु हनुमान्नाम मारुतस्यौरसः सुतः}
{सीतायास्तु कृते तूर्णं शतयोजनमायतम्}

\twolineshloka
{समुद्रं लङ्घयित्वैव तां दिदृक्षुरिहागतः}
{भ्रमता च मया दृष्टा गृहे ते जनकात्मजा}

\twolineshloka
{तद्भवान् दृष्टधर्मार्थस्तपः कृतपरिग्रहः}
{परदारान् महाप्राज्ञ नोपरोद्धुं त्वमर्हसि}

\twolineshloka
{न हि धर्मविरुद्धेषु बह्वपायेषु कर्मसु}
{मूलघातिषु सज्जन्ते बुद्धिमन्तो भवद्विधाः}

\twolineshloka
{कश्च लक्ष्मणमुक्तानां रामकोपानुवर्तिनाम्}
{शराणामग्रतः स्थातुं शक्तो देवासुरेष्वपि}

\twolineshloka
{न चापि त्रिषु लोकेषु राजन् विद्येत कश्चन}
{राघवस्य व्यलीकं यः कृत्वा सुखमवाप्नुयात्}

\twolineshloka
{तत् त्रिकालहितं वाक्यं धर्म्यमर्थानुबन्धि च}
{मन्यस्व नरदेवाय जानकी प्रतिदीयताम्}

\twolineshloka
{दृष्टा हीयं मया देवी लब्धं यदिह दुर्लभम्}
{उत्तरं कर्म यच्छेषं निमित्तं तत्र राघवः}

\twolineshloka
{लक्षितेयं मया सीता तथा शोकपरायणा}
{गृह्य यां नाभिजानासि पञ्चास्यामिव पन्नगीम्}

\twolineshloka
{नेयं जरयितुं शक्या सासुरैरमरैरपि}
{विषसंसृष्टमत्यर्थं भुक्तमन्नमिवौजसा}

\twolineshloka
{तपःसन्तापलब्धस्ते योऽयं धर्मपरिग्रहः}
{न स नाशयितुं न्याय्य आत्मप्राणपरिग्रहः}

\twolineshloka
{अवध्यतां तपोभिर्यां भवान् समनुपश्यति}
{आत्मनः सासुरैर्देवैर्हेतुस्तत्राप्ययं महान्}

\twolineshloka
{सुग्रीवो न हि देवोऽयं नासुरो न च राक्षसः}
{न दानवो न गन्धर्वो न यक्षो न च पन्नगः}

\twolineshloka
{तस्मात् प्राणपरित्राणं कथं राजन् करिष्यसि}
{ननु धर्मोपसंहारमधर्मफलसंहितम्}

\twolineshloka
{तदेव फलमन्वेति धर्मश्चाधर्मनाशनः}
{प्राप्तं धर्मफलं तावद्भवता नात्र संशयः}

\twolineshloka
{फलमस्याप्यधर्मस्य क्षिप्रमेव प्रपत्स्यसे}
{जनस्थानवधं बुद्ध्वा बुद्ध्वा वालिवधं तथा}

\twolineshloka
{रामसुग्रीवसख्यं च बुध्यस्व हितमात्मनः}
{कामं खल्वहमप्येकः सवाजिरथकुञ्जराम्}

\twolineshloka
{लङ्कां नाशयितुं शक्तस्तस्यैष तु न निश्चयः}
{रामेण हि प्रतिज्ञातं हर्यृक्षगणसन्निधौ}

\twolineshloka
{उत्सादनममित्राणां सीता यैस्तु प्रधर्षिता}
{अपकुर्वन् हि रामस्य साक्षादपि पुरन्दरः}

\twolineshloka
{न सुखं प्राप्नुयादन्यः किं पुनस्त्वद्विधो जनः}
{यां सीतेत्यभिजानासि येयं तिष्ठति ते वशे}

\twolineshloka
{कालरात्रीति तां विद्धि सर्वलङ्काविनाशिनीम्}
{तदलं कालपाशेन सीताविग्रहरूपिणा}

\twolineshloka
{स्वयं स्कन्धावसक्तेन क्षममात्मनि चिन्त्यताम्}
{सीतायास्तेजसा दग्धां रामकोपप्रपीडिताम्}

\twolineshloka
{दह्यमानामिमां पश्य पुरीं साट्टप्रतोलिकाम्}
{स्वानि मित्राणि मन्त्रींश्च ज्ञातीन्भ्रातॄन्सुतान्हितान्}

\twolineshloka
{भोगान् दारांश्च लङ्कां च मा विनाशमुपानय}
{सत्यं राक्षसराजेन्द्र शृणुष्व वचनं मम}

\twolineshloka
{रामदासस्य दूतस्य वानरस्य विशेषतः}
{सर्वाँल्लोकान् सुसंहृत्य सभूतान् सचराचरान्}

\twolineshloka
{पुनरेव तदा स्रष्टुं शक्तो रामो महायशाः}
{देवासुरनरेन्द्रेषु यक्षरक्षोगणेषु च}

\twolineshloka
{विद्याधरेषु सर्वेषु गन्धर्वेषूरगेषु च}
{सिद्धेषु किन्नरेन्द्रेषु पतत्रिषु च सर्वतः}

\twolineshloka
{सर्वभूतेषु सर्वत्र सर्वकालेषु नास्ति सः}
{यो रामं प्रतियुध्येत विष्णुतुल्यपराक्रमम्}

\twolineshloka
{सर्वलोकेश्वरस्यैवं कृत्वा विप्रियमीदृशम्}
{रामस्य राजसिंहस्य दुर्लभं तव जीवितम्}

\fourlineindentedshloka
{देवाश्च दैत्याश्च निशाचरेन्द्र}
{गन्धर्वविद्याधरनागयक्षाः}
{रामस्य लोकत्रयनायकस्य}
{स्थातुं न शक्ताः समरेषु सर्वे}

\fourlineindentedshloka
{ब्रह्मा स्वयम्भूश्चतुराननो वा}
{रुद्रस्त्रिनेत्रस्त्रिपुरान्तको वा}
{इन्द्रो महेन्द्रः सुरनायको वा}
{त्रातुं न शक्ता युधि रामवध्यम्}

\fourlineindentedshloka
{स सौष्ठवोपेतमदीनवादिनः}
{कपेर्निशम्याप्रतिमोऽप्रियं वचः}
{दशाननः कोपविवृत्तलोचनः}
{समादिशत्तस्य वधं महाकपेः}

इत्यार्षे श्रीमद्रामायणे वाल्मीकीये आदिकाव्ये चतुर्विंशतिसहस्रिकायां संहितायाम् सुन्दरकाण्डे हनूमदुपदेशो नाम एकपञ्चाशः सर्गः॥

\closesection
    \input{rama-charitam/valmiki-ramayanam/hanumad-bharata-sambhashanam}
    \input{rama-charitam/valmiki-ramayanam/gayatri-ramayanam}
    \chapt{अध्यात्म-रामायणम्}

\sect{रामहृदये रामचरितम्}

\src{अध्यात्म-रामायणम्}{बालकाण्डः}{अध्यायः १}{श्लोकाः ३२--४३}
\vakta{सीता}
\shrota{हनुमान्}
\notes{In the opening chapter of Adhyatma Ramayana, Sita describes the divine nature of Rama and the various events of His life, which She attributes to Her own divine presence. This chapter is often referred to as ``Ramahṛdayam'' or ``The Heart of Rama.''}
\textlink{}
\translink{}

\storymeta


\uvacha{सीतोवाच}

\addtocounter{shlokacount}{31}

\twolineshloka
{रामं विद्धि परं ब्रह्म सच्चिदानन्दमद्वयम्}
{सर्वोपाधिविनिर्मुक्तं सत्तामात्रमगोचरम्} %1-32

\twolineshloka
{आनन्दं निर्मलं शान्तं निर्विकारं निरञ्जनम्}
{सर्वव्यापिनमात्मानं स्वप्रकाशमकल्मषम्} %1-33

\twolineshloka
{मां विद्धि मूलप्रकृतिं सर्गस्थित्यन्तकारिणीम्}
{तस्य सन्निधिमात्रेण सृजामीदमतन्द्रिता} %1-34

\twolineshloka
{तत्सान्निध्यान्मया सृष्टं तस्मिन्नारोप्यतेऽबुधैः}
{अयोध्यानगरे जन्म रघुवंशेऽतिनिर्मले} %1-35

\twolineshloka
{विश्वामित्रसहायत्वं मखसंरक्षणं ततः}
{अहल्याशापशमनं चापभङ्गो महेशितुः} %1-36

\twolineshloka
{मत्पाणिग्रहणं पश्चाद्भार्गवस्य मदक्षयः}
{अयोध्यानगरे वासो मया द्वादशवार्षिकः} %1-37

\twolineshloka
{दण्डकारण्यगमनं विराधवध एव च}
{मायामारीचमरणं मायासीताहृतिस्तथा} %1-38

\twolineshloka
{जटायुषो मोक्षलाभः कबन्धस्य तथैव च}
{शबर्याः पूजनं पश्चात्सुग्रीवेण समागमः} %1-39

\twolineshloka
{वालिनश्च वधः पश्चात्सीतान्वेषणमेव च}
{सेतुबन्धश्च जलधौ लङ्कायाश्च निरोधनम्} %1-40

\twolineshloka
{रावणस्य वधो युद्धे सपुत्रस्य दुरात्मनः}
{विभीषणे राज्यदानं पुष्पकेण मया सह} %1-41

\threelineshloka
{अयोध्यागमनं पश्चाद्राज्ये रामाभिषेचनम्}
{एवमादीनि कर्माणि मयैवाचरितान्यपि}
{आरोपयन्ति रामेऽस्मिन्निर्विकारेऽखिलात्मनि} %1-42

\fourlineindentedshloka
{रामो न गच्छति न तिष्ठति नानुशोच-}
{त्याकाङ्क्षते त्यजति नो न करोति किञ्चित्}
{आनन्दमूर्तिरचलः परिणामहीनो}
{मायागुणाननुगतो हि तथा विभाति} %1-43

॥इति श्रीमदध्यात्मरामायणे उमामहेश्वरसंवादे बालकाण्डे रामहृदयं नाम प्रथमे सर्गे रामचरितं सम्पूर्णम्॥

\closesection
    \input{rama-charitam/adhyatma-ramayanam/hanumat-sita-samvada}
    \chapt{श्रीरामचरितम्}

\sect{दशमोऽध्यायः --- श्रीरामचरितम्}

\src{श्रीमद्-भागवतम्}{नवमः स्कन्धः}{अध्यायः १०}{श्लोकाः १---५६}
\vakta{शुकः}
\shrota{परीक्षितः}
\tags{concise, complete}
\notes{This chapter recounts the appearance of Lord Rāmacandra in the lineage of Mahārāja Khaṭvāṅga and details His divine exploits, including the slaying of Rāvaṇa and His triumphant return to Ayodhyā.}
\textlink{http://stotrasamhita.net/wiki/Bhagavatam/Skandha_09/Adhyaya_10}
\translink{}

\storymeta


\uvacha{श्रीशुक उवाच}

\twolineshloka
{खट्वाङ्गाद्दीर्घबाहुश्च रघुस्तस्मात्पृथुश्रवाः}
{अजस्ततो महाराजस्तस्माद्दशरथोऽभवत्} %1

\threelineshloka
{तस्यापि भगवानेष साक्षाद्ब्रह्ममयो हरि}
{अंशांशेन चतुर्धागात्पुत्रत्वं प्रार्थितः सुरै}
{रामलक्ष्मणभरत शत्रुघ्ना इति संज्ञया} %2

\twolineshloka
{तस्यानुचरितं राजन्नृषिभिस्तत्त्वदर्शिभिः}
{श्रुतं हि वर्णितं भूरि त्वया सीतापतेर्मुहुः} %3

\fourlineindentedshloka
{गुर्वर्थे त्यक्तराज्यो व्यचरदनुवनं पद्मपद्भ्यां प्रियायाः}
{पाणिस्पर्शाक्षमाभ्यां मृजितपथरुजो यो हरीन्द्रानुजाभ्याम्}
{वैरूप्याच्छूर्पणख्याः प्रियविरहरुषारोपितभ्रूविजृम्भ}
{त्रस्ताब्धिर्बद्धसेतुः खलदवदहनः कोसलेन्द्रोऽवतान्नः} %4

\twolineshloka
{विश्वामित्राध्वरे येन मारीचाद्या निशाचराः}
{पश्यतो लक्ष्मणस्यैव हता नैरृतपुङ्गवाः} %5

\fourlineindentedshloka
{यो लोकवीरसमितौ धनुरैशमुग्रं}
{सीतास्वयंवरगृहे त्रिशतोपनीतम्}
{आदाय बालगजलील इवेक्षुयष्टिं}
{सज्ज्यीकृतं नृप विकृष्य बभञ्ज मध्ये} %6

\fourlineindentedshloka
{जित्वानुरूपगुणशीलवयोऽङ्गरूपां}
{सीताभिधां श्रियमुरस्यभिलब्धमानाम्}
{मार्गे व्रजन्भृगुपतेर्व्यनयत्प्ररूढं}
{दर्पं महीमकृत यस्त्रिरराजबीजाम्} %7

\fourlineindentedshloka
{यः सत्यपाशपरिवीतपितुर्निदेशं}
{स्त्रैणस्य चापि शिरसा जगृहे सभार्यः}
{राज्यं श्रियं प्रणयिनः सुहृदो निवासं}
{त्यक्त्वा ययौ वनमसूनिव मुक्तसङ्गः} %8

\fourlineindentedshloka
{रक्षःस्वसुर्व्यकृत रूपमशुद्धबुद्धेस्}
{तस्याः खरत्रिशिरदूषणमुख्यबन्धून्}
{जघ्ने चतुर्दशसहस्रमपारणीय}
{कोदण्डपाणिरटमान उवास कृच्छ्रम्} %9

\fourlineindentedshloka
{सीताकथाश्रवणदीपितहृच्छयेन}
{सृष्टं विलोक्य नृपते दशकन्धरेण}
{जघ्नेऽद्भुतैणवपुषाश्रमतोऽपकृष्टो}
{मारीचमाशु विशिखेन यथा कमुग्रः} %10

\fourlineindentedshloka
{रक्षोऽधमेन वृकवद्विपिनेऽसमक्षं}
{वैदेहराजदुहितर्यपयापितायाम्}
{भ्रात्रा वने कृपणवत्प्रियया वियुक्तः}
{स्त्रीसङ्गिनां गतिमिति प्रथयंश्चचार} %11

\fourlineindentedshloka
{दग्ध्वात्मकृत्यहतकृत्यमहन्कबन्धं}
{सख्यं विधाय कपिभिर्दयितागतिं तैः}
{बुद्ध्वाथ वालिनि हते प्लवगेन्द्रसैन्यैर्}
{वेलामगात्स मनुजोऽजभवार्चिताङ्घ्रिः} %12

\fourlineindentedshloka
{यद्रोषविभ्रमविवृत्तकटाक्षपात}
{सम्भ्रान्तनक्रमकरो भयगीर्णघोषः}
{सिन्धुः शिरस्यर्हणं परिगृह्य रूपी}
{पादारविन्दमुपगम्य बभाष एतत्} %13

\fourlineindentedshloka
{न त्वां वयं जडधियो नु विदाम भूमन्}
{कूटस्थमादिपुरुषं जगतामधीशम्}
{यत्सत्त्वतः सुरगणा रजसः प्रजेशा}
{मन्योश्च भूतपतयः स भवान्गुणेशः} %14

\fourlineindentedshloka
{कामं प्रयाहि जहि विश्रवसोऽवमेहं}
{त्रैलोक्यरावणमवाप्नुहि वीर पत्नीम्}
{बध्नीहि सेतुमिह ते यशसो वितत्यै}
{गायन्ति दिग्विजयिनो यमुपेत्य भूपाः} %15

\fourlineindentedshloka
{बद्ध्वोदधौ रघुपतिर्विविधाद्रिकूटैः}
{सेतुं कपीन्द्रकरकम्पितभूरुहाङ्गैः}
{सुग्रीवनीलहनुमत्प्रमुखैरनीकैर्}
{लङ्कां विभीषणदृशाविशदग्रदग्धाम्} %16

\fourlineindentedshloka
{सा वानरेन्द्रबलरुद्धविहारकोष्ठ}
{श्रीद्वारगोपुरसदोवलभीविटङ्का}
{निर्भज्यमानधिषणध्वजहेमकुम्भ}
{शृङ्गाटका गजकुलैर्ह्रदिनीव घूर्णा} %17

\fourlineindentedshloka
{रक्षःपतिस्तदवलोक्य निकुम्भकुम्भ}
{धूम्राक्षदुर्मुखसुरान्तकनरान्तकादीन्}
{पुत्रं प्रहस्तमतिकायविकम्पनादीन्}
{सर्वानुगान्समहिनोदथ कुम्भकर्णम्} %18

\fourlineindentedshloka
{तां यातुधानपृतनामसिशूलचाप}
{प्रासर्ष्टिशक्तिशरतोमरखड्गदुर्गाम्}
{सुग्रीवलक्ष्मणमरुत्सुतगन्धमाद}
{नीलाङ्गदर्क्षपनसादिभिरन्वितोऽगात्} %19

\fourlineindentedshloka
{तेऽनीकपा रघुपतेरभिपत्य सर्वे}
{द्वन्द्वं वरूथमिभपत्तिरथाश्वयोधैः}
{जघ्नुर्द्रुमैर्गिरिगदेषुभिरङ्गदाद्याः}
{सीताभिमर्षहतमङ्गलरावणेशान्} %20

\fourlineindentedshloka
{रक्षःपतिः स्वबलनष्टिमवेक्ष्य रुष्ट}
{आरुह्य यानकमथाभिससार रामम्}
{स्वःस्यन्दने द्युमति मातलिनोपनीते}
{विभ्राजमानमहनन्निशितैः क्षुरप्रैः} %21

\fourlineindentedshloka
{रामस्तमाह पुरुषादपुरीष यन्नः}
{कान्तासमक्षमसतापहृता श्ववत्ते}
{त्यक्तत्रपस्य फलमद्य जुगुप्सितस्य}
{यच्छामि काल इव कर्तुरलङ्घ्यवीर्यः} %22

\fourlineindentedshloka
{एवं क्षिपन्धनुषि सन्धितमुत्ससर्ज}
{बाणं स वज्रमिव तद्धृदयं बिभेद}
{सोऽसृग्वमन्दशमुखैर्न्यपतद्विमानाद्}
{धाहेति जल्पति जने सुकृतीव रिक्तः} %23

\twolineshloka
{ततो निष्क्रम्य लङ्काया यातुधान्यः सहस्रशः}
{मन्दोदर्या समं तत्र प्ररुदन्त्य उपाद्रवन्} %24

\twolineshloka
{स्वान्स्वान्बन्धून्परिष्वज्य लक्ष्मणेषुभिरर्दितान्}
{रुरुदुः सुस्वरं दीना घ्नन्त्य आत्मानमात्मना} %25

\twolineshloka
{हा हताः स्म वयं नाथ लोकरावण रावण}
{कं यायाच्छरणं लङ्का त्वद्विहीना परार्दिता} %26

\twolineshloka
{न वै वेद महाभाग भवान्कामवशं गतः}
{तेजोऽनुभावं सीताया येन नीतो दशामिमाम्} %27

\twolineshloka
{कृतैषा विधवा लङ्का वयं च कुलनन्दन}
{देहः कृतोऽन्नं गृध्राणामात्मा नरकहेतवे} %28

\dnsub{श्रीशुक उवाच}


\twolineshloka
{स्वानां विभीषणश्चक्रे कोसलेन्द्रानुमोदितः}
{पितृमेधविधानेन यदुक्तं साम्परायिकम्} %29

\twolineshloka
{ततो ददर्श भगवानशोकवनिकाश्रमे}
{क्षामां स्वविरहव्याधिं शिंशपामूलमाश्रिताम्} %30

\twolineshloka
{रामः प्रियतमां भार्यां दीनां वीक्ष्यान्वकम्पत}
{आत्मसन्दर्शनाह्लाद विकसन्मुखपङ्कजाम्} %31

\twolineshloka
{आरोप्यारुरुहे यानं भ्रातृभ्यां हनुमद्युतः}
{विभीषणाय भगवान्दत्त्वा रक्षोगणेशताम्} %32

\twolineshloka
{लङ्कामायुश्च कल्पान्तं ययौ चीर्णव्रतः पुरीम्}
{अवकीर्यमाणः सुकुसुमैर्लोकपालार्पितैः पथि} %33

\twolineshloka
{उपगीयमानचरितः शतधृत्यादिभिर्मुदा}
{गोमूत्रयावकं श्रुत्वा भ्रातरं वल्कलाम्बरम्} %34

\twolineshloka
{महाकारुणिकोऽतप्यज्जटिलं स्थण्डिलेशयम्}
{भरतः प्राप्तमाकर्ण्य पौरामात्यपुरोहितैः} %35

\twolineshloka
{पादुके शिरसि न्यस्य रामं प्रत्युद्यतोऽग्रजम्}
{नन्दिग्रामात्स्वशिबिराद्गीतवादित्रनिःस्वनैः} %36

\twolineshloka
{ब्रह्मघोषेण च मुहुः पठद्भिर्ब्रह्मवादिभिः}
{स्वर्णकक्षपताकाभिर्हैमैश्चित्रध्वजै रथैः} %37

\twolineshloka
{सदश्वै रुक्मसन्नाहैर्भटैः पुरटवर्मभिः}
{श्रेणीभिर्वारमुख्याभिर्भृत्यैश्चैव पदानुगैः} %38

\twolineshloka
{पारमेष्ठ्यान्युपादाय पण्यान्युच्चावचानि च}
{पादयोर्न्यपतत्प्रेम्णा प्रक्लिन्नहृदयेक्षणः} %39

\twolineshloka
{पादुके न्यस्य पुरतः प्राञ्जलिर्बाष्पलोचनः}
{तमाश्लिष्य चिरं दोर्भ्यां स्नापयन्नेत्रजैर्जलैः} %40

\twolineshloka
{रामो लक्ष्मणसीताभ्यां विप्रेभ्यो येऽर्हसत्तमाः}
{तेभ्यः स्वयं नमश्चक्रे प्रजाभिश्च नमस्कृतः} %41

\twolineshloka
{धुन्वन्त उत्तरासङ्गान्पतिं वीक्ष्य चिरागतम्}
{उत्तराः कोसला माल्यैः किरन्तो ननृतुर्मुदा} %42

\twolineshloka
{पादुके भरतोऽगृह्णाच्चामरव्यजनोत्तमे}
{विभीषणः ससुग्रीवः श्वेतच्छत्रं मरुत्सुतः} %43

\twolineshloka
{धनुर्निषङ्गान्छत्रुघ्नः सीता तीर्थकमण्डलुम्}
{अबिभ्रदङ्गदः खड्गं हैमं चर्मर्क्षराण्नृप} %44

\twolineshloka
{पुष्पकस्थो नुतः स्त्रीभिः स्तूयमानश्च वन्दिभिः}
{विरेजे भगवान्राजन्ग्रहैश्चन्द्र इवोदितः} %45

\twolineshloka
{भ्रात्राभिनन्दितः सोऽथ सोत्सवां प्राविशत्पुरीम्}
{प्रविश्य राजभवनं गुरुपत्नीः स्वमातरम्} %46

\twolineshloka
{गुरून्वयस्यावरजान्पूजितः प्रत्यपूजयत्}
{वैदेही लक्ष्मणश्चैव यथावत्समुपेयतुः} %47

\twolineshloka
{पुत्रान्स्वमातरस्तास्तु प्राणांस्तन्व इवोत्थिताः}
{आरोप्याङ्केऽभिषिञ्चन्त्यो बाष्पौघैर्विजहुः शुचः} %48

\twolineshloka
{जटा निर्मुच्य विधिवत्कुलवृद्धैः समं गुरुः}
{अभ्यषिञ्चद्यथैवेन्द्रं चतुःसिन्धुजलादिभिः} %49

\twolineshloka
{एवं कृतशिरःस्नानः सुवासाः स्रग्व्यलङ्कृतः}
{स्वलङ्कृतैः सुवासोभिर्भ्रातृभिर्भार्यया बभौ} %50

\threelineshloka
{अग्रहीदासनं भ्रात्रा प्रणिपत्य प्रसादित}
{प्रजाः स्वधर्मनिरता वर्णाश्रमगुणान्विता}
{जुगोप पितृवद्रामो मेनिरे पितरं च तम्} %51

\twolineshloka
{त्रेतायां वर्तमानायां कालः कृतसमोऽभवत्}
{रामे राजनि धर्मज्ञे सर्वभूतसुखावहे} %52

\twolineshloka
{वनानि नद्यो गिरयो वर्षाणि द्वीपसिन्धवः}
{सर्वे कामदुघा आसन्प्रजानां भरतर्षभ} %53

\twolineshloka
{नाधिव्याधिजराग्लानि दुःखशोकभयक्लमाः}
{मृत्युश्चानिच्छतां नासीद्रामे राजन्यधोक्षजे} %54

\twolineshloka
{एकपत्नीव्रतधरो राजर्षिचरितः शुचिः}
{स्वधर्मं गृहमेधीयं शिक्षयन्स्वयमाचरत्} %55

\twolineshloka
{प्रेम्णाऽनुवृत्त्या शीलेन प्रश्रयावनता सती}
{भिया ह्रिया च भावज्ञा भर्तुः सीताऽहरन्मनः} %56

॥इति श्रीमद्भागवते महापुराणे पारमहंस्यां संहितायां नवमस्कन्धे दशमोऽध्यायः॥

\closesection
    \sect{एकादशोऽध्यायः --- श्रीरामोपाख्यानम्}

\src{श्रीमद्-भागवतम्}{नवमः स्कन्धः}{अध्यायः ११}{श्लोकाः १---३६}
\vakta{शुकः}
\shrota{परीक्षितः}
\tags{concise, complete}
\notes{This chapter summarises how Lord Rāmachandra exemplified supreme dharma through performing extensive yajñas, generous gifts to brāhmaṇas, deep love for His subjects, and painful renunciation of Sītādevī , ultimately culminating in His departure to Vaikuntham after establishing His sons in the kingdom.}
\textlink{http://stotrasamhita.net/wiki/Bhagavatam/Skandha_09/Adhyaya_11}
\translink{}

\storymeta

\uvacha{श्रीशुक उवाच}

\twolineshloka
{भगवानात्मनात्मानं राम उत्तमकल्पकैः}
{सर्वदेवमयं देवमीजेऽथाचार्यवान्मखैः} %1

\twolineshloka
{होत्रेऽददाद्दिशं प्राचीं ब्रह्मणे दक्षिणां प्रभुः}
{अध्वर्यवे प्रतीचीं वा उत्तरां सामगाय सः} %2

\twolineshloka
{आचार्याय ददौ शेषां यावती भूस्तदन्तरा}
{अन्यमान इदं कृत्स्नं ब्राह्मणोऽर्हति निःस्पृहः} %3

\twolineshloka
{इत्ययं तदलङ्कार वासोभ्यामवशेषितः}
{तथा राज्ञ्यपि वैदेही सौमङ्गल्यावशेषिता} %4

\twolineshloka
{ते तु ब्राह्मणदेवस्य वात्सल्यं वीक्ष्य संस्तुतम्}
{प्रीताः क्लिन्नधियस्तस्मै प्रत्यर्प्येदं बभाषिरे} %5

\twolineshloka
{अप्रत्तं नस्त्वया किं नु भगवन्भुवनेश्वर}
{यन्नोऽन्तर्हृदयं विश्य तमो हंसि स्वरोचिषा} %6

\twolineshloka
{नमो ब्रह्मण्यदेवाय रामायाकुण्ठमेधसे}
{उत्तमश्लोकधुर्याय न्यस्तदण्डार्पिताङ्घ्रये} %7

\twolineshloka
{कदाचिल्लोकजिज्ञासुर्गूढो रात्र्यामलक्षितः}
{चरन्वाचोऽशृणोद्रामो भार्यामुद्दिश्य कस्यचित्} %8

\twolineshloka
{नाहं बिभर्मि त्वां दुष्टामसतीं परवेश्मगाम्}
{स्त्रैणो हि बिभृयात्सीतां रामो नाहं भजे पुनः} %9

\twolineshloka
{इति लोकाद्बहुमुखाद्दुराराध्यादसंविदः}
{पत्या भीतेन सा त्यक्ता प्राप्ता प्राचेतसाश्रमम्} %10

\twolineshloka
{अन्तर्वत्न्यागते काले यमौ सा सुषुवे सुतौ}
{कुशो लव इति ख्यातौ तयोश्चक्रे क्रिया मुनिः} %11

\twolineshloka
{अङ्गदश्चित्रकेतुश्च लक्ष्मणस्यात्मजौ स्मृतौ}
{तक्षः पुष्कल इत्यास्तां भरतस्य महीपते} %12

\twolineshloka
{सुबाहुः श्रुतसेनश्च शत्रुघ्नस्य बभूवतुः}
{गन्धर्वान्कोटिशो जघ्ने भरतो विजये दिशाम्} %13

\threelineshloka
{तदीयं धनमानीय सर्वं राज्ञे न्यवेदय}
{शत्रुघ्नश्च मधोः पुत्रं लवणं नाम राक्षस}
{हत्वा मधुवने चक्रे मथुरां नाम वै पुरीम्॥१४} %14

\twolineshloka
{मुनौ निक्षिप्य तनयौ सीता भर्त्रा विवासिता}
{ध्यायन्ती रामचरणौ विवरं प्रविवेश ह} %15

\twolineshloka
{तच्छ्रुत्वा भगवान्रामो रुन्धन्नपि धिया शुचः}
{स्मरंस्तस्या गुणांस्तांस्तान्नाशक्नोद्रोद्धुमीश्वरः} %16

\twolineshloka
{स्त्रीपुम्प्रसङ्ग एतादृक्सर्वत्र त्रासमावहः}
{अपीश्वराणां किमुत ग्राम्यस्य गृहचेतसः} %17

\twolineshloka
{तत ऊर्ध्वं ब्रह्मचर्यं धार्यन्नजुहोत्प्रभुः}
{त्रयोदशाब्दसाहस्रमग्निहोत्रमखण्डितम्} %18

\twolineshloka
{स्मरतां हृदि विन्यस्य विद्धं दण्डककण्टकैः}
{स्वपादपल्लवं राम आत्मज्योतिरगात्ततः} %19

\fourlineindentedshloka
{नेदं यशो रघुपतेः सुरयाच्ञयात्त}
{लीलातनोरधिकसाम्यविमुक्तधाम्नः}
{रक्षोवधो जलधिबन्धनमस्त्रपूगैः}
{किं तस्य शत्रुहनने कपयः सहायाः} %20

\fourlineindentedshloka
{यस्यामलं नृपसदःसु यशोऽधुनापि}
{गायन्त्यघघ्नमृषयो दिगिभेन्द्रपट्टम्}
{तं नाकपालवसुपालकिरीटजुष्ट}
{पादाम्बुजं रघुपतिं शरणं प्रपद्ये} %21

\twolineshloka
{स यैः स्पृष्टोऽभिदृष्टो वा संविष्टोऽनुगतोऽपि वा}
{कोसलास्ते ययुः स्थानं यत्र गच्छन्ति योगिनः} %22

\twolineshloka
{पुरुषो रामचरितं श्रवणैरुपधारयन्}
{आनृशंस्यपरो राजन्कर्मबन्धैर्विमुच्यते} %23

\uvacha{श्रीराजोवाच}


\twolineshloka
{कथं स भगवान्रामो भ्रात्न्वा स्वयमात्मनः}
{तस्मिन्वा तेऽन्ववर्तन्त प्रजाः पौराश्च ईश्वरे} %24

\uvacha{श्रीबादरायणिरुवाच}


\twolineshloka
{अथादिशद्दिग्विजये भ्रात्ंस्त्रिभुवनेश्वरः}
{आत्मानं दर्शयन्स्वानां पुरीमैक्षत सानुगः} %25

\twolineshloka
{आसिक्तमार्गां गन्धोदैः करिणां मदशीकरैः}
{स्वामिनं प्राप्तमालोक्य मत्तां वा सुतरामिव} %26

\twolineshloka
{प्रासादगोपुरसभा चैत्यदेवगृहादिषु}
{विन्यस्तहेमकलशैः पताकाभिश्च मण्डिताम्} %27

\twolineshloka
{पूगैः सवृन्तै रम्भाभिः पट्टिकाभिः सुवाससाम्}
{आदर्शैरंशुकैः स्रग्भिः कृतकौतुकतोरणाम्} %28

\twolineshloka
{तमुपेयुस्तत्र तत्र पौरा अर्हणपाणयः}
{आशिषो युयुजुर्देव पाहीमां प्राक्त्वयोद्धृताम्} %29

\fourlineindentedshloka
{ततः प्रजा वीक्ष्य पतिं चिरागतं}
{दिदृक्षयोत्सृष्टगृहाः स्त्रियो नराः}
{आरुह्य हर्म्याण्यरविन्दलोचनम्}
{अतृप्तनेत्राः कुसुमैरवाकिरन्} %30

\twolineshloka
{अथ प्रविष्टः स्वगृहं जुष्टं स्वैः पूर्वराजभिः}
{अनन्ताखिलकोषाढ्यमनर्घ्योरुपरिच्छदम्} %31

\twolineshloka
{विद्रुमोदुम्बरद्वारैर्वैदूर्यस्तम्भपङ्क्तिभिः}
{स्थलैर्मारकतैः स्वच्छैर्भ्राजत्स्फटिकभित्तिभिः} %32

\twolineshloka
{चित्रस्रग्भिः पट्टिकाभिर्वासोमणिगणांशुकैः}
{मुक्ताफलैश्चिदुल्लासैः कान्तकामोपपत्तिभिः} %33

\twolineshloka
{धूपदीपैः सुरभिभिर्मण्डितं पुष्पमण्डनैः}
{स्त्रीपुम्भिः सुरसङ्काशैर्जुष्टं भूषणभूषणैः} %34

\twolineshloka
{तस्मिन्स भगवान्रामः स्निग्धया प्रिययेष्टया}
{रेमे स्वारामधीराणामृषभः सीतया किल} %35

\twolineshloka
{बुभुजे च यथाकालं कामान्धर्ममपीडयन्}
{वर्षपूगान्बहून्नॄणामभिध्याताङ्घ्रिपल्लवः} % 36


॥इति श्रीमद्भागवते महापुराणे पारमहंस्यां संहितायां नवमस्कन्धे श्रीरामोपाख्याने एकादशोऽध्यायः॥


\closesection
    \chapt{श्रीमन्नारायणीयम्}

\sect{दशकं ३४ --- श्रीरामचरितवर्णनम्}

\src{श्रीमन्नारायणीयम्}{चतुस्त्रिंश-दशकं}{}{श्लोकाः १---10}
\vakta{शुकः}
\shrota{परीक्षितः}
\tags{concise, complete}
\notes{This chapter recounts the appearance of Lord Rāmacandra in the lineage of Mahārāja Khaṭvāṅga and details His divine exploits, including the slaying of Rāvaṇa and His triumphant return to Ayodhyā.}
\textlink{http://stotrasamhita.net/wiki/Narayaniyam/Dashaka_34}
\translink{}

\storymeta

\fourlineindentedshloka
{गीर्वाणैरर्थ्यमानो दशमुखनिधनं कोसलेऽष्वृश्यषृङ्गे}
{पुत्रीयामिष्टिमिष्ट्वा ददुषि दशरथक्ष्माभृते पायसाग्र्यम्}
{तद्भुक्त्या तत्पुरन्ध्रीष्वपि तिसृषु समं जातगर्भासु जातो}
{रामस्त्वं लक्ष्मणेन स्वयमथ भरतेनापि शत्रुघ्ननाम्ना} % ॥१॥

\fourlineindentedshloka
{कोदण्डी कौशिकस्य क्रतुवरमवितुं लक्ष्मणेनानुयातो}
{यातोऽभूस्तातवाचा मुनिकथितमनुद्वन्द्वशान्ताध्वखेदः}
{नॄणां त्राणाय बाणैर्मुनिवचनबलात्ताटकां पाटयित्वा}
{लब्ध्वास्मादस्त्रजालं मुनिवनमगमो देव सिद्धाश्रमाख्यम्} % ॥२॥

\fourlineindentedshloka
{मारीचं द्रावयित्वा मखशिरसि शरैरन्यरक्षांसि निघ्नन्}
{कल्यां कुर्वन्नहल्यां पथि पदरजसा प्राप्य वैदेहगेहम्}
{भिन्दानश्चान्द्रचूडं धनुरवनिसुतामिन्दिरामेव लब्ध्वा}
{राज्यं प्रातिष्ठथास्त्वं त्रिभिरपि च समं भ्रातृवीरैः सदारैः} % ॥३॥

\fourlineindentedshloka
{आरुन्धाने रुषान्धे भृगुकुलतिलके सङ्क्रमय्य स्वतेजो}
{याते यातोऽस्ययोध्यां सुखमिह निवसन्कान्तया कान्तमूर्ते}
{शत्रुघ्नेनैकदाथो गतवति भरते मातुलस्याधिवासम्}
{तातारब्धोऽभिषेकस्तव किल विहतः केकयाधीशपुत्र्या} % ॥४॥

\fourlineindentedshloka
{तातोक्त्या यातुकामो वनमनुजवधूसंयुतश्चापधारः}
{पौरानारूध्य मार्गे गुहनिलयगतस्त्वं जटाचीरधारी}
{नावा सन्तीर्य गङ्गामधिपदवि पुनस्तं भरद्वाजमारा-}
{न्नत्वा तद्वाक्यहेतोरतिसुखमवसश्चित्रकूटे गिरीन्द्रे} % ॥५॥

\fourlineindentedshloka
{श्रुत्वा पुत्रार्तिखिन्नं खलु भरतमुखात् स्वर्गयातं स्वतातम्}
{तप्तो दत्त्वाम्बु तस्मै निदधिथ भरते पादुकां मेदिनीं च}
{अत्रिं नत्वाथ गत्वा वनमतिविपुलां दण्डकां चण्डकायम्}
{हत्वा दैत्यं विराधं सुगतिमकलयश्चारु भोः शारभङ्गीम्} % ॥६॥

\fourlineindentedshloka
{नत्वाऽगस्त्यं समस्ताशरनिकरसपत्राकृतिं तापसेभ्यः}
{प्रत्यश्रौषीः प्रियैषी तदनु च मुनिना वैष्णवे दिव्यचापे}
{ब्रह्मास्त्रे चापि दत्ते पथि पितृसुहृदं दीक्ष्य जटायुम्}
{मोदाद्गोदातटान्ते परिरमसि पुरा पञ्चवट्यां वधूट्या} % ॥७॥

\fourlineindentedshloka
{प्राप्तायाः शूर्पणख्या मदनचलधृतेरर्थनैर्निस्सहात्मा}
{तां सौमित्रौ विसृज्य प्रबलतमरुषा तेन निर्लुननासाम्}
{दृष्ट्वैनां रुष्टचित्तं खरमभिपतितं दुषणं च त्रिमूर्धम्}
{व्याहिंसीराशरानप्ययुतसमधिकांस्तत्क्षणादक्षतोष्मा} % ॥८॥

\fourlineindentedshloka
{सोदर्याप्रोक्तवार्ताविवशदशमुखादिष्टमारीचमाया-}
{सारङ्गं सारसाक्ष्या स्पृहितमनुगतः प्रावधीर्बाणघातम्}
{तन्मायाक्रन्दनिर्यापितभवदनुजां रावणस्तामहार्षीत्}
{तेनार्तोऽपि त्वमन्तः किमपि मुदमधास्तद्वधोपायायलाभात्} % ॥९॥

\fourlineindentedshloka
{भूयस्तन्वीं विचिन्वन्नहृत दशमुखस्त्वद्वधूं मद्वधेने-}
{त्युक्त्वा याते जटायौ दिवमथ सुहृदः प्रातनोः प्रेतकार्यम्}
{गृह्णानं तं कबन्धं जघनिथ शबरीं प्रेक्ष्य पम्पातटे त्वम्}
{सम्प्राप्तो वातसूनुं भृशमुदितमनाः पाहि वातालयेश} % ॥१०॥

॥इति श्रीमन्नारायणीये श्रीरामचरितवर्णनं नाम चतुस्त्रिंश-दशकं सम्पूर्णम्॥

\closesection
    \sect{श्रीरामचरितवर्णनम् - २}

\src{श्रीमन्नारायणीयम्}{पञ्चत्रिंश-दशकं}{}{श्लोकाः १--१०}
\vakta{शुकः}
\shrota{परीक्षितः}
\tags{concise, complete}
\notes{This chapter summarises the events following the death of Vāli, including the alliance with Sugrīva, the search for Sītā, Setubandhanam, the eventual victory over Rāvaṇa, and the establishment of Rāma's rule in Ayodhyā.}
\textlink{http://stotrasamhita.net/wiki/Narayaniyam/Dashaka_35}
\translink{}

\storymeta


\fourlineindentedshloka
{नीतस्सुग्रीवमैत्रीं तदनु दुन्दुभेः कायमुच्चैः}
{क्षिप्त्वाङ्गुष्ठेन भूयो लुलविथ युगपत्पत्रिणा सप्त सालान्}
{हत्वा सुग्रीवघातोद्यतमतुलबलं वालिनं व्याजवृत्त्या}
{वर्षावेलामनैषीर्विरहतरळितस्त्वं मतङ्गाश्रमान्ते} %॥१॥

\fourlineindentedshloka
{सुग्रीवेणानुजोक्त्या सभयमभियता व्यूहितां वाहिनीं ता-}
{मृक्षाणां वीक्ष्य दिक्षु द्रुतमथ दयितामार्गणायावनम्राम्}
{सन्देशं चान्गुलीयं पवनसुतकरे प्रादिशो मोदशाली}
{मार्गे मार्गे ममार्गे कपिभिरपि तदी त्वत्प्रिया सप्रयासैः} %॥२॥

\fourlineindentedshloka
{त्वद्वार्ताकर्णनोद्यद्गरुदुरुजवसम्पातिसम्पातिवाक्य-}
{प्रोत्तीर्णार्णोधिरन्तर्नगरि जनकजां वीक्ष्य दत्त्वाऽङ्गुलीयम्}
{प्रक्षुद्योद्यानमक्षक्षपणचणरणः सोढबन्धो दशास्यम्}
{दृष्ट्वा प्लुष्ट्वा च लङ्कां झटिति स हनुमान्मौलिरत्नं ददौ ते} %॥३॥

\fourlineindentedshloka
{त्वं सुग्रीवाङ्गदादिप्रबलकपिचमूचक्रविक्रान्तभूमी-}
{चक्रोऽभिक्रम्य पारेजलधि निशिचरेन्द्रानुजाश्रीयमाणः}
{तत्प्रोक्तां शत्रुवार्तां रहसि निशमयन्प्रार्थनापार्थ्यरोष-}
{प्रास्ताग्नेयास्त्रतेजस्त्रसदुदधिगिरा लब्धवान्मध्यमार्गम्} %॥४॥

\fourlineindentedshloka
{कीशैराशान्तरोपाहृतगिरिनिकरैः सेतुमाधाप्य यातो}
{यातून्यामर्द्य दंष्ट्रानखशिखरिशिलासालशस्त्रैः स्वसैन्यैः}
{व्याकुर्वन्सानुजस्त्वं समरभुवि परं विक्रमं शक्रजेत्रा}
{वेगान्नागास्त्रबद्धः पतगपतिगरुन्मारुतैर्मोचितोऽभूः} %॥५॥

\fourlineindentedshloka
{सौमित्रिस्त्वत्र शक्तिप्रहृतिगळदसुर्वातजानीतशैल-}
{घ्राणात्प्रणानुपेतो व्यकृणुत कुसृतिश्लाघिनं मेघनादम्}
{मायाक्षोभेषु वैभीषणवचनहृतस्तम्भनः कुम्भकर्णम्}
{सम्प्राप्तं कम्पितोर्वीतलमखिलचमूभक्षिणं व्यक्षिणोस्त्वम्} %॥६॥

\fourlineindentedshloka
{गृह्णन् जम्भारिसम्प्रेषितरथकवचौ रावणेनाभियुध्यन्}
{ब्रह्मास्त्रेणास्य भिन्दन् गळततिमबलामग्निशुद्धां प्रगृह्णन्}
{देव श्रेणीवरोज्जीवितसमरमृतैरक्षतैऱ्क्षसङ्घैर्-}
{लङ्काभर्त्रा च साकं निजनगरमगाः सप्रियः पुष्पकेण} %॥७॥

\fourlineindentedshloka
{प्रीतो दिव्याभिषेकैरयुतसमधिकान्वत्सरान्पर्यरंसी-}
{र्मैथिल्यां पापवाचा शिव शिव किल तां गर्भिणीमभ्यहासीः}
{शत्रुघ्नेनार्दयित्वा लवणनिशिचरं प्रार्दयः शूद्रपाशम्}
{तावद्वाल्मीकिगेहे कृतवसतिरुपासूत सीता सुतौ ते} %॥८॥

\fourlineindentedshloka
{वाल्मीकेस्त्वत्सुतोद्गापितमधुरकृतेराज्ञया यज्ञवाटे}
{सीतां त्वय्याप्तुकामे क्षितिमविशदसौ त्वं च कालार्थितोऽभूः}
{हेतोः सौमित्रिघाती स्वयमथ सरयूमग्ननिश्शेषभृत्यैः}
{साकं नाकं प्रयातो निजपदमगमो देव वैकुण्ठमाद्यम्} %॥९॥

\fourlineindentedshloka
{सोऽयं मर्त्यावतारस्तव खलु नियतं मर्त्यशिक्षार्थमेवं}
{विश्लेषार्तिर्निरागस्त्यजनमपि भवेत्कामधर्मातिसक्त्या}
{नो चेत्स्वात्मानुभूतेः क्वनु तव मनसो विक्रिया चक्रपाणे}
{स त्वं सत्त्वैकमूर्ते पवनपुरपते व्याधुनु व्याधितापान्} %॥१०॥

॥इति श्रीमन्नारायणीये श्रीरामचरितवर्णनं नाम पञ्चत्रिंश-दशकं सम्पूर्णम्॥

\closesection
    \input{rama-charitam/mahabharatam/rama-avatara}
    \chapt{रामोपाख्यान-पर्व}

\src{श्रीमन्महाभारतम्}{वन-पर्व}{श्रीरामोपाख्यानपर्व}{अध्यायाः २७५--२९३}
\vakta{मार्कण्डेयः}
\shrota{युधिष्ठिरः}
\tags{concise, complete}
\notes{This chapter recounts the appearance of Lord Rāmacandra in the lineage of Mahārāja Khaṭvāṅga and details His divine exploits, including the slaying of Rāvaṇa and His triumphant return to Ayodhyā.}
% \textlink{http://stotrasamhita.net/wiki/Narayaniyam/Dashaka_34}
\translink{}

\sect{अध्यायः २७५}

\uvacha{मार्कण्डेय उवाच}

\twolineshloka
{प्राप्तमप्रतिमं दुःखं रामेण भरतर्षभ}
{रक्षसा जानकी तस्य हृता भार्या बलीयसा}


\twolineshloka
{आश्रमाद्राक्षसेन्द्रेण रावणेन दुरात्मना}
{मायामास्थाय तरसा हत्वा गृध्रं जटायुषम्}


\twolineshloka
{प्रत्याजहार तां रामः सुग्रीवबलमाश्रितः}
{बद्ध्वा सेतुं समुद्रस्य दग्ध्वा लङ्कां शितैः शरैः}

\uvacha{युधिष्ठिर उवाच}


\twolineshloka
{कस्मिन् रामः कुले जातः किंवीर्यः किम्पराक्रमः}
{रावणः कस्य पुत्रो वा किं वैरं तस्य तेन ह}


\threelineshloka
{एतन्मे भगवन्सर्वं सम्यगाख्यातुमर्हसि}
{त्वया प्रत्यक्षतो दृष्टं यथासर्वमशेषतः}
{श्रोतुमिच्छामि चरितं रामस्याक्लिष्टकर्मणः}

\uvacha{मार्कण्डेय उवाच}


\twolineshloka
{अजो नामाभवद्राजा महानिक्ष्वाकुवंशजः}
{तस्य पुत्रो दशरथः शश्वत्स्वाध्यायवाञ्छुचिः}


\twolineshloka
{अभवंस्तस्य चत्वारः पुत्रा धर्मार्थकोविदाः}
{रामलक्ष्मणशत्रुघ्ना भरतश्च महाबलः}


\twolineshloka
{रामस्य माता कौसल्या कैकेयी भरतस्य तु}
{सुतौ लक्ष्मणशत्रुघ्नौ सुमित्रायाः परंतपौ}


\twolineshloka
{विदेहराजो जनकः सीता तस्यात्मजा विमो}
{यां चकार स्वयं त्वष्टा रामस्य महिषीं प्रियाम्}


\twolineshloka
{एतद्रामस्य ते जन्म सीतायाश्च प्रकीर्तितम्}
{रावणस्यापि ते जन्म व्याख्यास्यामि रजनेश्वर}


\twolineshloka
{पितामहो रावणस्य साक्षाद्देवः प्रजापतिः}
{स्वयंभूः सर्वलोकानां प्रभुः स्रष्टा महातपाः}


\twolineshloka
{पुलस्त्यो नाम तस्यासीन्मानसो दयितः सुतः}
{तस्य वैश्रवणो नाम गवि पुत्रोऽभवत्प्रभुः}


\twolineshloka
{पितरं स समुत्सृज्य पितामहमुपस्थितः}
{तस्य कोपात्पिता राजन्ससर्जात्मानमात्मना}


\twolineshloka
{स जज्ञे विश्रवा नाम तस्यात्मार्धेन वै द्विजः}
{प्रतीकाराय सक्रोधस्ततो वैश्रवणस्य वै}


\twolineshloka
{पितामहस्तुप्रीतात्मा ददौ वैश्रवणस् ह}
{अमरत्वं धनेशत्वं लोकपालत्वमेव च}


\twolineshloka
{ईशानन तथा सख्यं पुत्रं च नलकूवरम्}
{राजधानीनिवेसं च लङ्कांरक्षोगणान्विताम्}


\twolineshloka
{विमानं पुष्पकं नाम कामगं च ददौ प्रभुः}
{यक्षाणामाधिपत्यंच राजराजत्वमेव च}


\sect{अध्यायः २७६}
\uvacha{मार्कण्डेय उवाच}


\twolineshloka
{पुलस्त्यस्तु यः क्रोधादर्धदेहोऽभवन्मुनिः}
{विश्रवानाम सक्रोधं पितरं राक्षसेश्वरः}


\twolineshloka
{बुबुधे तं तु सक्रोधं पितरं राक्षसेश्वरः}
{कुबेरस्तत्प्रसादार्थं यतते स्म सदा नृप}


\twolineshloka
{स राजराजो लङ्कायां न्यवसन्नरवाहनः}
{राक्षसीः प्रददौ तिस्रः पितुर्वै परिचारिकाः}


\twolineshloka
{ताः सदा तं महात्मानं संतोषयितुमुद्यताः}
{ऋषिं भरतशार्दूल नृत्यगीतविशारदाः}


\twolineshloka
{पुष्पोत्कटा च राका च मालिनी च विशांपते}
{अन्योन्यस्पर्धयाराजञ्श्रेयस्कामाः सुमध्यमाः}


\twolineshloka
{स तासां भगवांस्तुष्टो महात्मा प्रददौ वरान्}
{लोकपालोपमान्पुत्रानकैकस्या यथेप्सितान्}


\twolineshloka
{पुष्पोत्कटायां जज्ञाते द्वौ पुत्रौ राक्षसेश्वरौ}
{कुम्भकर्णदशग्रीवौ बलेनाप्रतिमौ भुवि}


\twolineshloka
{मालिन जनयामास पुत्रमेकं विभीषणम्}
{राकार्या मिथुनं जज्ञे खरः शूर्पणखा तथा}


\twolineshloka
{विभीषणस्तु रूपेण सर्वेभ्योऽभ्यधिकोऽभवत्}
{स बभूव महाभागो धर्मगोप्ता क्रियारतिः}


\twolineshloka
{दशग्रीवस्तु सर्वेषां श्रेष्ठो राक्षसपुङ्गवः}
{महोत्साहो महावीर्यो महासत्वपराक्रमः}


\twolineshloka
{कुम्भकर्णओ बलेनासीत्सर्वेभ्योऽभ्यधिको युधि}
{मायावी रणशौण्डश्चरौद्रश्च रजनीचरः}


\twolineshloka
{खरो धनुषि विक्रान्तो ब्रह्मद्विट् पिशिताशनः}
{सिद्धविघ्नकरी चापि रौद्री शूर्पणखा तदा}


\twolineshloka
{सर्वे वेदविदः शूराः सर्वेसुचरितव्रताः}
{ऊषुः पित्रा सह रता गन्धमादनपर्वते}


\twolineshloka
{ततो वैश्रवणं तत्र ददृशुर्नरवाहनम्}
{पित्रा सार्धं समासीनमृद्ध्या परमया युतम्}


\twolineshloka
{जातामर्षास्ततस्ते तु तपसे धृतनिश्चयाः}
{ब्रह्माणं तोषयामासुर्घोरेण तपसा तदा}


\twolineshloka
{अतिष्ठदेकपादेन सहस्रं परिवत्सरान्}
{वायुभक्षो दशग्रीवः पञ्चाग्निः सुसमाहितः}


\twolineshloka
{अधःशायी कुम्भकर्णो यताहारो यतव्रतः}
{विभीषणः शीर्णपर्णमेकमभ्यवहारयन्}


\twolineshloka
{उपवासरतिर्धीमान्सदा जप्यपरायणः}
{तमेव कालमातिष्ठत्तीव्रं तप उदारधीः}


\twolineshloka
{स्वरः शूर्पणखा चैव तेषां वै तप्यतां तपः}
{परिचर्यां च रक्षां च चक्रतुर्हष्टमानसौ}


\twolineshloka
{पूर्णे वर्षसहस्रेतु शिरश्छित्त्वा दशाननः}
{जुहोत्यग्नौ दुराधर्षस्तेनातुष्यज्जगत्प्रभुः}


\twolineshloka
{ततो ब्रह्मा स्वयं गत्वा तपसस्तान्न्यवारयत्}
{प्रलोभ्यवरदानेन सर्वानेवपृथक्पृथक्}

\uvacha{ब्राह्मोवाच}


\twolineshloka
{प्रीतोऽस्मि वो निवर्तध्वं वरान्वृणुत पुत्रकाः}
{यद्यदिष्टमृते त्वेकममरत्वं तथाऽस्तु तत्}


\twolineshloka
{यद्यदग्नौ हुतं सर्वं शिरस्ते महदीप्सया}
{तथैव तानि ते देहे भविष्यन्ति यथेप्सया}


\twolineshloka
{वैरूप्यं च न ते देहे कामरूपधरस्तथा}
{भविष्यसि रणेऽरीणां विजेता न च संशयः}

\uvacha{रावण उवाच}


\twolineshloka
{गन्धर्वदेवासुरतो यक्षराक्षसतस्तथा}
{सर्पकिंनरभूतेभ्यो न मे भूयात्पराभवः}

\uvacha{ब्रह्मोवाच}


\twolineshloka
{य एते कीर्तिताः सर्वे न तेभ्योस्ति भयं रतव}
{ऋते मनुष्याद्भद्रं ते तथा तद्विहितं मया}

\uvacha{मार्कण्डेय उवाच}


\twolineshloka
{एवमुक्तो दशग्रीवस्तुष्टः समभवत्तदा}
{अवमेने हि दुर्बुद्धिर्मनुष्यान्पुरुषादकः}


\threelineshloka
{कुम्भकर्णमथोवाच तथैव प्रपितामहः}
{वरं वृणीष्व भद्रं ते प्रीतोस्मीति पुनःपुनः}
{स वव्रे महतीं निद्रां तमसा ग्रस्तचेतनः}


\twolineshloka
{तथाभविष्यतीत्युक्त्वा विभीषणमुवाच ह}
{वरं वृणीष्व पुत्र त्वं प्रीतोऽस्मीति पुनःपुनः}

\uvacha{विभीषण उवाच}


\twolineshloka
{परमापद्गतस्यापि नाधर्मे मे मतिर्भवेत्}
{अशिक्षितं च भगवन्ब्रह्मास्त्रं प्रतिभातु मे}

\uvacha{ब्रह्मोवाच}


\twolineshloka
{यस्माद्राक्षसयोनौ ते जातस्यामित्रकर्शन}
{नाधर्मे धीयते बुद्धिरमरत्वं ददानि ते}

\uvacha{मार्कण्डेय उवाच}


\twolineshloka
{राक्षसस्तु वरंलब्ध्वा दशग्रीवो विशांपते}
{लङ्कायाश्च्यावयामास युधि जित्वा धनेश्वरम्}


\twolineshloka
{हित्वास भगवाँल्लङ्कामाविशद्गन्धमादनम्}
{गन्धर्वयक्षानुगतो रक्षःकिंपुरुषैः सह}


\twolineshloka
{विमानं पुष्पकं तस्य जहाराक्रम्य रावणः}
{शशाप तं वैश्रवणो न त्वामेतद्वहिष्यति}


\twolineshloka
{यस्तु त्वां समरे हन्ता तमेवैतद्वहिष्यति}
{अवमत्य गुरुं मां च क्षिप्रं त्वंनभविष्यसि}


\twolineshloka
{विभीषणस्तु धर्मात्मा सतां मार्गमनुस्मरन्}
{अन्वगच्छन्महाराज श्रिया परमया युतः}


\twolineshloka
{तस्मै स भगवांस्तुष्टो भ्राता भ्रात्रे धनेश्वरः}
{सैनापत्यं ददौ धीमान्यक्षराक्षससेनयोः}


\twolineshloka
{राक्षसाः पुरुषादाश्च पिशाचाश्च महाबलाः}
{सर्वे समेत्य राजानमभ्यषिञ्चन्दशाननम्}


\twolineshloka
{दशग्रीवश्चदैत्यानां दानवानां बलोत्कटः}
{आक्रम्य रत्नान्यहरत्कामरूपी विहंगम}


\twolineshloka
{रावयामास लोकान्यत्तस्माद्रावण उच्यते}
{दशग्रीवः कामबलो देवानां भयमादधत्}


\sect{अध्यायः २७७}
\twolineshloka
{मार्कण्डेय उवाच}
{}


\twolineshloka
{ततो ब्रह्मर्षयः सर्वे सिद्धा देवर्षयस्तथा}
{हव्यवाहं पुरस्कृत्य ब्रह्माणं शरणं गताः}

\uvacha{अग्निरुवाच}


\twolineshloka
{योसौ विश्रवसः पुत्रो दशग्रीवो महाबलः}
{अवध्यो वरदानेन कृतो भगवता पुरा}


\twolineshloka
{स बाधते प्रजाः सर्वा विप्रकारैर्महाबलः}
{ततो नस्त्रातु भगवान्नान्यस्त्राता हि विद्यते}

\uvacha{ब्रह्मोवाच}


\twolineshloka
{न स देवासुरैः शक्यो युद्धे जेतुं विभावसो}
{विहितं तत्रयत्कार्यमभितस्तस्य निग्रहः}


\twolineshloka
{तदर्थमवतीर्णोऽसौ मन्नियोगाच्चतुर्भुजः}
{विष्णुः प्रहरतां श्रेष्ठः स तत्कर्म करिष्यति}

\uvacha{मार्कण्डेय उवाच}


\twolineshloka
{पितामहस्ततस्तेषां संनिधौ शक्रमब्रवीत्}
{सर्वैर्देवगणैः सार्धं संभव त्वं महीतले}


\twolineshloka
{विष्णोः सहायानृक्षीषु वानरीषु च सर्वशः}
{जनयध्वं सुतान्वीरान्कामरूपबलान्वितान्}


\twolineshloka
{ते यथोक्ता भगवता तत्प्रतिश्रुत्य शासनम्}
{ससृजुर्देवगन्धर्वाः पुत्रान्वानररूपिणः}


\twolineshloka
{ततो भागानुभागेन देवगन्धर्वदानवाः}
{अवतर्तुं महीं सर्वे मन्त्रयामासुरञ्जसा}


\threelineshloka
{अवतेरुर्महीं स्वर्गादंशैश्च सहिताः सुराः}
{ऋषयश्च महात्मानः सिद्धाश्च सह किन्नरैः}
{चारणाश्चासृजन्घोरान्वानरान्वनचारिणः}


\twolineshloka
{यस्य देवस्य यद्रूपं वेषस्तेजश्च यद्विधम्}
{अजायन्त समास्तेन तस्य तस्य सुतास्तदा}


\twolineshloka
{तेषां समक्षं गन्धर्वी दुन्दुभीं नाम नामतः}
{शशास वरदो देवो गच्छ कार्यार्थसिद्धये}


\twolineshloka
{पितामहवचः श्रुत्वा गन्धर्वी दुन्दुभी ततः}
{मन्थरा मानुषे लोके कुब्जा समभवत्तदा}


\twolineshloka
{शक्रप्रभृतयश्चैव सर्वे ते सुरसत्तमाः}
{वानरर्क्षवरस्त्रीषु जनयामासुरात्मजान्}


\twolineshloka
{तेऽन्ववर्तन्पितॄन्सर्वे यशसा च बलेन च}
{भेत्तारो गिरिशृङ्गाणां सालतालशिलायुधाः}


\twolineshloka
{वज्चसंहननाः सर्वेसर्वेऽमोघवलास्तथा}
{कामवीर्यबलाश्चैवसर्वे बुद्धिविशारदाः}


\twolineshloka
{नागायुतसमप्राणा वायुवेगसमा जवे}
{यथेच्छविनिपाताश्च केचिदत्र वनौकसः}


\twolineshloka
{एवं विधाय तत्सर्वं भगवाँल्लोकभावनः}
{मन्थरां बोधयामास यद्यत्कार्यं त्वया तथा}


\twolineshloka
{सा तद्वच समाज्ञाय तथा चक्रे मनोजवा}
{इतश्चेतश्च गच्छन्ती वैरसंधुक्षणे रता}


\sect{अध्यायः २७८}
\twolineshloka
{युधिष्ठिर उवाच}
{}


\twolineshloka
{उक्तं भगवता जन्म रामादीनां पृथक्पृथक्}
{प्रस्थानकारणं ब्रह्मञ्श्रोतुमिच्छामि कथ्यताम्}


\twolineshloka
{कथं दाशरथी वीरौ भ्रातरौ रामलक्ष्मणौ}
{प्रस्तापितौ वने ब्रह्मन्मैथिली च यशस्विनी}

\uvacha{मार्कण्डेय उवाच}


\twolineshloka
{जातपुत्रो दशरथः प्रीतिमानभवन्नृप}
{क्रियारतिर्धर्मरतः सततं वृद्धसेविता}


\twolineshloka
{क्रमेण चास्य ते पुत्रा व्यवर्धन्त महौजसः}
{वेदेषु सरहस्येषु धनुर्वेदेषु पारगाः}


\twolineshloka
{चरितब्रह्मचर्यास्ते कृतदाराश्च पार्तिव}
{दृष्ट्वा रामं दशरथः प्रीतिमानभवत्सुखी}


\twolineshloka
{ज्येष्ठो रामोऽभवत्तेषां रमयामास हि प्रजाः}
{मनोहरतया धीमान्पितुर्हृदयनन्दनः}


\twolineshloka
{ततः स राजा मतिमान्मत्वाऽऽत्मानं वयोधिकम्}
{मन्त्रयामास सचिवैर्मन्त्रज्ञैश्च पुरोहितैः}


\twolineshloka
{अभिषेकाय रामस्य यावैराज्येन भारत}
{प्राप्तकालं च ते सर्वे मेनिरे मन्त्रिसत्तमाः}


\twolineshloka
{लोहिताक्षं महाबाहुं मत्तमातङ्गगामिनम्}
{कम्बुग्रीवं महोरस्कं नीलकुञ्चितमूर्धजम्}


\twolineshloka
{दीप्यमानं श्रिया वीरं शक्रादनवरं बले}
{पारगं सर्वधर्माणां बृहस्पतिसमं मतौ}


\twolineshloka
{सर्वानुरक्तप्रकृतिं सर्वविद्याविशारदम्}
{जितेन्द्रियममित्राणामपि दृष्टिमनोहरम्}


\twolineshloka
{नियन्तारमसाधूनां गोप्तारं धर्मचारिणाम्}
{धृतिमन्तमनाधृष्यं जेतारमपराजितम्}


\twolineshloka
{पुत्रं राजा दशरथः कौसल्यानन्दवर्धनम्}
{संदृश्यपरमां प्रीतिमगच्छत्कुलनन्दनम्}


\twolineshloka
{चिन्तयंश्च महातेजा गुणान्रामस्य वीर्यवान्}
{अभ्यभाषत भद्रं ते प्रीयमाणः पुरोहितम्}


\twolineshloka
{अद् पुष्यो निशि ब्रह्मन्पुण्यं योगमुपैष्यति}
{संभाराः संभ्रियन्तां मे रामश्चोपनिमन्त्र्यताम्}


\twolineshloka
{श्व एवपुष्यो भविता यत्ररामः सुतो मया}
{यौवराज्येऽभिषेक्तव्यः पौरेषु सहमन्त्रिभिः}


\twolineshloka
{इति तद्राजवचनं प्रतिश्रुत्याथ मन्थरा}
{कैकेयीमभिगम्येदं काले वचनमब्रवीत्}


\twolineshloka
{अद्य कैकेयि दौर्भाग्यं राज्ञा ते ख्यापितं महत्}
{आशीविषस्त्वां संक्रुद्धश्छन्नो दशति दुर्भगे}


\twolineshloka
{सुभगा खलु कौसल्या यस्याः पुत्रोऽभिषेक्ष्यते}
{कुतो हि तव सौभाग्यं यस्याः पुत्रो न राज्यभाक्}


\twolineshloka
{सा तद्वचनमाज्ञाय सर्वाभरणभूषिता}
{वेदी विलग्नमध्येन बिभ्रती रूपमुत्तमम्}


\twolineshloka
{वविक्ते पतिमासाद्य हसन्तीव शुचिस्मिता}
{राजानं तर्जयन्तीव मधुरं वाक्यमब्रवीत्}


\threelineshloka
{सत्यप्रतिज्ञ यन्मे त्वं काममेकं विसृष्टवान्}
{उपाकुरुष्व तद्राजंस्तस्मान्मुञ्चस्व संकटात्}
{तदद्य कुरु सत्यं मे वरं वरद भूपते}

\uvacha{राजोवाच}


\twolineshloka
{वरं ददानि ते हन्त तद्गृहाण यदिच्छसि}
{अवध्यो वध्यतां कोद्य वध्यः कोऽद्य विमुच्यतां}


\twolineshloka
{धनं ददानि कस्याद् ह्रियतां कस्यरवापुनः}
{ब्राह्मणस्वादिहान्यत्रयत्किंचिद्वित्तमस्ति मे}


\twolineshloka
{पृथिव्यां राजराजोस्मि चातुर्वर्ण्यस् रिता}
{यस्तेऽभिलपितः कामो ब्रूहि कल्याणि माचिरं}


\twolineshloka
{सातद्वचनमाज्ञाय परिगृह्य नराधिपम्}
{आत्मनो बलमाज्ञाय तत एनमुवाच ह}


\twolineshloka
{आभिषेचनिकं यत्ते रामार्थमुपकल्पितम्}
{भरतस्तदवाप्नोतु वनं गच्छतु राघवः}


\twolineshloka
{नव पञ्च च वर्षाणि दण्डकारण्यमाश्रितः}
{चीराजिनजटाधारी रामो भवतु तापसः}


\twolineshloka
{स तं राजा वरं श्रुत्वा विप्रियं दारुणोदयम्}
{दुःखार्तो भरतश्रेष्ठ न किंचिद्व्याजहार ह}


\twolineshloka
{ततस्तथोक्तं पितरं रामो विज्ञाय वीर्यवान्}
{वनं प्रतस्थे धर्मात्मा राजा सत्यो भवत्विति}


\twolineshloka
{तमन्वगच्छल्लक्ष्मीवान्धनुष्माँल्लक्ष्मणस्तदा}
{सीता च भार्या भद्रं ते वैदेही जनकात्मजा}


\twolineshloka
{ततो वनं गतेरामे राजा दशरथस्तदा}
{समयुज्यत देहस्य कालपर्यायधर्मणा}


\twolineshloka
{रामं तु गतमाज्ञाय राजानं च तथागतम्}
{अनार्या भरतं देवी कैकेयी वाक्यमब्रवीत्}


\twolineshloka
{गतोदशरथः स्वर्गं वनस्थौ रामलक्ष्मणौ}
{गृहाण राज्यंविपुलं क्षेमं निहतकण्टकम्}


\twolineshloka
{तामुवाच स धर्मात्मा नृशंसं बत ते कृतम्}
{पतिं हत्वाकुलं चेदमुत्साद्य धनलुब्धया}


\twolineshloka
{अयशः पातयित्वा मे मूर्ध्नि त्वं कुलपांसने}
{सकामा भव मे मातरित्युक्त्वा प्ररुरोद ह}


\twolineshloka
{स चारित्रं विशोध्याथ सर्वप्रकृतिसन्निधौ}
{अन्वयाद्धातरं रामं विनिवर्तनलालसः}


\twolineshloka
{कौसल्यां च सुमित्रां च कैकेयीं च सुदुःखितः}
{अग्रे प्रस्थाप्य यानैः स शत्रुघ्नसहितो ययौ}


\twolineshloka
{वसिष्ठवामदेवाभ्यां विप्रैश्चान्यैः सहस्रशः}
{पौरजानपदैः सार्धं रामानयनकाङ्क्षया}


\twolineshloka
{ददर्श चित्रकूटस्थं स रामं सहलक्ष्मणम्}
{तापसानामलंकारं धारयन्तं धनुर्धरम्}


\twolineshloka
{उवाच प्राञ्जलिर्भूत्वाप्रणिपत्य रघूत्तमम्}
{शशंस मरणं राज्ञः सोऽनाथांश्चापि कोसलान्}


% Check verse!
नाथ त्वं प्रतिपद्यस्व स्वराज्यमिति चोक्तवान्
\twolineshloka
{स तस्य वचनं श्रुत्वा रामः परमदुःखितः}
{चकार देवकल्पस्य पितुः स्नात्वोदकक्रियाम्}


\threelineshloka
{अब्रवीत्स तदारामो भ्रातरं भ्रातृवत्सलम्}
{पादुके मे भविष्येते राज्यगोप्त्र्यौ परंतप}
{एवमस्त्विति तं प्राह भरतः प्रणतस्तदा}


\twolineshloka
{विसर्जितः स रामेण पितुर्वचनकारिणा}
{नन्दिग्रामेऽकरोद्राज्यं पुरस्कृत्यास्य पादुके}


\twolineshloka
{रामस्तु पुनराशङ्क्य पौरजानपदागमम्}
{प्रविवेश महारण्यं शरभङ्गाश्रमं प्रति}


\twolineshloka
{सत्कृत्य शरभङ्गं स दण्डकारण्यमाश्रितः}
{नदीं गोदावरीं रम्यामाश्रित्य न्यवसत्तदा}


\twolineshloka
{वसतस्तस्य रामस्य ततः शूर्पणखाकृतम्}
{खरेणासीन्महद्वैरं जनस्थाननिवासिना}


\twolineshloka
{रक्षार्थं तापसानां तु राघवो धर्मवत्सलः}
{चतुर्दशसहस्राणि जघान भुवि राक्षसान्}


\twolineshloka
{दूषणं च स्वरं चैवनिहत्य सुमहाबलौ}
{चक्रे क्षेमं पुनर्धीमान्धर्मारण्यं स राघवः}


\twolineshloka
{हतेषु तेषु रक्षःसु ततः शूर्पणखा पुनः}
{ययौ निकृत्तनासोष्ठी लङ्कां भ्रातुर्निवेशनम्}


\twolineshloka
{ततो रावणमभ्येत्य राक्षसी दुःखमूर्च्छिता}
{पपात पादयोर्भ्रातुः संशुष्करुधिरानना}


\twolineshloka
{तां तथा विकृतां दृष्ट्वा रावणः क्रोधमूर्च्छितः}
{उत्पपातासनात्क्रुद्धो दन्तैर्दन्तानुपस्पृशन्}


\twolineshloka
{स्वानमात्यान्विसृज्याथ विविक्ते तामुवाच सः}
{केनास्येवं कृता भद्रे मामचिन्त्यावमत्य च}


\twolineshloka
{कः शूलं तीक्ष्णमासाद्य सर्वगात्रेषु सेवते}
{कः शिरस्यग्निमाधाय विश्वस्तः स्वपते सुखम्}


\twolineshloka
{आशीविषं घोरतरं पादेन स्पृशतीह कः}
{सिंहं केसरिणं मत्तः स्पृष्ट्वा दंष्ट्रासु तिष्ठति}


\twolineshloka
{इत्येवं ब्रुवतस्तस्य नेत्रेभ्यस्तेजसोऽर्चिषः}
{निश्चेरुर्दह्यतो रात्रौ वृक्षस्येव स्वरन्ध्रतः}


\twolineshloka
{तस्य तत्सर्वमाचख्यौ भगिनी रामविक्रमम्}
{खरदूषणसंयुक्तं राक्षसानां पराभवम्}


\twolineshloka
{ततो ज्ञातिवधं श्रुत्वा रावणः कालचोदितः}
{रामस्य वधमाकाङ्क्षन्मारीचं मनसागमत्}


\twolineshloka
{स निश्चित्यततः कृत्यं सागरं लवणाकरम्}
{ऊर्ध्वमाचक्रमे राजा विधाय नगरे विधिम्}


\twolineshloka
{त्रिकूटं समतिक्रम्य कालपर्वतमेव च}
{ददर्श मकरावासं गम्भीरोदं महोदधिम्}


\twolineshloka
{तमतीत्याथ गोकर्णमभ्यगच्छद्दशाननः}
{दयितं स्तानमव्यग्रं शूलपाणेर्महात्मनः}


\twolineshloka
{तत्राभ्यगच्छन्मारीचं पूर्वामात्यं दशाननः}
{पुरा रामभयादेव तापसं प्रियजीवितम्}


\sect{अध्यायः २७९}
\twolineshloka
{मार्कण्डेय उवाच}
{}


\twolineshloka
{मारीचस्त्वथ संभ्रान्तो दृष्ट्वा रावणमागतम्}
{पूजयामास सत्कारैः फलमूलादिभिस्ततः}


\twolineshloka
{विश्रान्तं चैनमासीनमन्वासीनः स राक्षसः}
{उवाच प्रश्रितं वाक्यं वाक्यज्ञो वाक्यकोविदम्}


\twolineshloka
{न ते प्रकृतिमान्वर्णः कच्चित्क्षेमं पुरे तव}
{कच्चित्प्रकृतयः सर्वा भजन्ते त्वां यथा पुरा}


\twolineshloka
{किमिहागमने चापि कार्यं ते राक्षसेश्वर}
{कृतमित्येव तद्विद्धि यद्यपि स्यात्सुदुष्करम्}


\twolineshloka
{शशंस रावणस्तस्मै तत्सर्वं रामचेष्टितम्}
{समासेनैव कार्याणि क्रोधामर्षसमन्वितः}


\twolineshloka
{मारीचस्त्वब्रवीच्छ्रत्वा समासेनैव रावणम्}
{अलं ते राममासाद्य वीर्यज्ञो ह्यस्मि तस्य वै}


\threelineshloka
{बाणवेगं हि कस्तस्य शक्तः सोढुं महात्मनः}
{प्रव्रज्यायां हि मे हेतुः स एव पुरुषर्षभः}
{विनाशमुखमेतत्ते केनाख्यातं दुरात्मना}


\twolineshloka
{तमुवाचाथ सक्रोधो रावणः परिभर्त्सयन्}
{अकुर्वतोऽस्मद्वचनं स्यान्मृत्युरपि ते ध्रुवम्}


\twolineshloka
{मरीचश्चिन्तयामास विशिष्टान्मरणं वरम्}
{अवश्यं मरणे प्राप्ते करिष्याम्यस्य यन्मतम्}


\twolineshloka
{ततस्तं प्रत्युवाचाथ मारीचो रक्षसांवरम्}
{किं ते साह्यां मया कार्यं करिष्याम्यवशोपि तत्}


\twolineshloka
{तमब्रवीद्दशग्रीवो गच्छ सीतां प्रलोभय}
{रत्नशृङ्गो मृगो भूत्वा रत्नचित्रतनूरुहः}


\twolineshloka
{ध्रुवं सीता समालक्ष्यत्वां रामं चोदयिष्यति}
{अपक्रान्ते च काकुत्स्थे सीता वश्या भविष्यति}


\twolineshloka
{तामादायापनेष्यामि ततः स नभविष्यति}
{भार्यावियोगाद्दुर्बुद्धिरेतत्साह्यं कुरुष्व मे}


\twolineshloka
{इत्येवमुक्तोमारीचः कृत्वोदकमथात्मनः}
{रावणं पुरतो यान्तमन्वगच्छत्सुदुःखितः}


\twolineshloka
{ततस्तस्याश्रमं गत्वारामस्याक्लिष्टकर्मणः}
{चक्रतुस्तद्यथा सर्वमुभौ यत्पूर्वमन्त्रितम्}


\twolineshloka
{रावणस्तु यतिर्भूत्वा मुण्डः कुण्डीत्रिदण्डधृत्}
{मृगश्चभूत्वामारीचस्तं देशमुपजग्मतुः}


\twolineshloka
{दर्शयामास मारीचो वैदेहीं मृगरूपधृत्}
{चोदयामास तस्यार्थे सा रामं विधिचोदिता}


\twolineshloka
{रामस्तस्याः प्रियं कुर्वन्धनुरादाय सत्वरः}
{रक्षार्थे लक्ष्मणं न्यस्य प्रययौ मृगलिप्सया}


\twolineshloka
{स धन्वी बद्धतूणीरः खङ्गगोधाङ्गुलित्रवान्}
{अन्वधावन्मृगं रामो रुद्रस्तारामृगं यथा}


\twolineshloka
{सोऽन्तर्हितः पुनस्तस्य दर्शनं राक्षसो व्रजन्}
{चकर्ष महदध्वानं रामस्तं वुबुधे ततः}


\twolineshloka
{निशाचरं विदित्वा तं राघवः प्रतिभानवान्}
{अमोघं शरमादाय जघान मृगरूपिणम्}


\twolineshloka
{स रामवाणाभिहतः कृत्वा रामस्वरं तदा}
{हा सीते लक्ष्मणेत्येवं चुक्रोशार्तस्वरेण ह}


\twolineshloka
{शुश्राव तस्य वैदेही ततस्तां करुणां गिरम्}
{साप्रापतत्ततः सीता तामुवाचाथ लक्ष्मणः}


\twolineshloka
{अलं ते शङ्कया भीरु को रामं प्रहरिष्यति}
{मुहूर्ताद्द्रक्ष्यसे रामं भर्तारं त्वं शुचिस्मितम्}


\twolineshloka
{इत्युक्ता सा प्ररुदती पर्यशङ्कत लक्ष्मणम्}
{हता वै स्त्रीस्वभावेन शुद्धचारित्रभूषणा}


\twolineshloka
{सा तं परुषमारब्धा वक्तुं साध्वी पतिव्रता}
{नैष कामो भवेन्मूढ यं त्वं प्रार्थयसे हृदा}


\twolineshloka
{अप्यहंशस्त्रमादाय हन्यामात्मानमात्मना}
{पतेयं गिरिशृङ्गाद्वा विशेयं वा हुताशनम्}


\twolineshloka
{रामं भर्तारमुत्सुज्यन त्वहं त्वां कथञ्चन}
{निहीनमुपतिष्ठेयं शार्दूली क्रोष्टुकं यथा}


\twolineshloka
{एतादृशं वचः श्रुत्वा लक्ष्मणः प्रियराघ्नव}
{पिधायकर्णौ सद्वृत्तः प्रस्थितो येन राघवः}


\twolineshloka
{स रामस्य पदंगृह्य प्रससार धनुर्धरः}
{अवीक्षमाणो विम्बोष्ठीं प्रययौ लक्ष्मणस्तदा}


\twolineshloka
{एतस्मिन्नन्तरे रक्षो रावणः प्रत्यदृश्यत}
{अभव्यो भव्यरूपेण भस्मच्छन्न इवानलः}


\twolineshloka
{यतिवेपप्रतिच्छन्नो जिहीर्षुस्तामनिन्दिताम्}
{उपागच्छत्स वैदेहीं रावणः पापनिश्चयः}


\twolineshloka
{सा तमालक्ष्यसंप्राप्तं धर्मज्ञा जनकात्मजा}
{निमन्त्रयामास तदा फलमूलाशनादिभिः}


\twolineshloka
{अवमत्यततः सर्वं स्वं रूपं प्रत्यपद्यत}
{सान्त्वयामास वैदेहीं कामी राक्षसपुङ्गवः}


\twolineshloka
{सीते राक्षसराजोऽहंरावणो नाम विश्रुतः}
{मम लङ्कापुरी नाम्ना रम्या पारे महोदधेः}


\twolineshloka
{तत्र त्वं नरनारीषु शोभिष्यसि मया सह}
{भार्या मे भव सुश्रोणि तापसं त्यज राघवम्}


\twolineshloka
{एवमादीनि वाक्यानि श्रुत्वा तस्याथ जानकी}
{पिधाय कर्णौ सुश्रोणी मैवमित्यब्रवीद्वचः}


\twolineshloka
{प्रपतेद्द्यौः सनक्षत्रा पृथिवी शकलीभवेत्}
{शुष्येत्तोयनिधौ तोयं चन्द्रः शीतांशुतां त्यजेत्}


\twolineshloka
{उष्णांशुत्वमथो जह्यादादित्यो वह्निरुष्णताम्}
{त्यक्त्वाशैत्यं भजेन्नाहं त्यजेयंरघुनन्दनम्}


\twolineshloka
{कथं हि भिन्नकरटं पद्मिनं वनगोचरम्}
{उपस्थाय महानागं करेणुः सूकरं स्पृशेत्}


\twolineshloka
{कथं हि पीत्वा माध्वीकं पीत्वा च मधुमाधवीम्}
{लोभं सौवीरके कुर्यान्नारी काचिदिति स्मरेः}


\twolineshloka
{इति सा तं समाभाष्य प्रविवेशाश्रमं ततः}
{क्रोधात्प्रस्फुरमाणौष्ठी विधुन्वाना करौ मुहुः}


% Check verse!
तामधिद्रुत्य सुश्रोणीं रावणः प्रत्यषेधयत्
\twolineshloka
{भर्त्सयित्वातु रूक्षेण स्वरेण गतचेतनाम्}
{मूर्धजेषु निजग्राह ऊर्ध्वमाचक्रमे ततः}


\twolineshloka
{तां ददर्श ततो गृध्रो जटायुर्गिरिगोचरः}
{रुदतीं रामरामेति हियमाणां तपस्विनीम्}


\sect{अध्यायः २८०}
\twolineshloka
{गार्कण्डेय उवाच}
{}


\twolineshloka
{सखा दशरथस्यासीज्जटायुररुणात्मजः}
{गृध्रराजो महावीरः संपातिर्यस् सोदरः}


\twolineshloka
{स ददर्श तदा सीतां रावणाङ्कगतां स्नुषाम्}
{सक्रोधोऽभ्यद्रवत्पक्षी रावणं राक्षसेश्वरम्}


\twolineshloka
{अथैनमब्रवीद्गृध्रो मुञ्चमुञ्चेति मैथिलीम्}
{ध्रियमाणे मयि कथं हरिष्यसि निशाचर}


\twolineshloka
{न हिमे मोक्ष्यसे जीवन्यदि नोत्सृजसे वधूम्}
{उक्त्वैवं राक्षसेन्द्रं तं चकर्त नखरैर्भृशम्}


\twolineshloka
{पक्षतुण्डप्रहारैश्च शतशो जर्जरीकृतम्}
{चक्षार रुधिरं भूरि गिरिः प्रस्रवणैरिव}


\twolineshloka
{स वध्यमानो गृध्रेण रामप्रियहितैषिणा}
{खङ्गमादाय चिच्छेद भुजौ तस् पतत्रिणः}


\twolineshloka
{निहत्य गृध्रराजं सभिन्नाभ्रशिखरोपमम्}
{ऊर्ध्वमाचक्रमे सीतां गृहीत्वाऽङ्केन राक्षसः}


\twolineshloka
{यत्रयत्रतु वैदेही पश्यत्याश्रममण्डलम्}
{सरोवा सरितो वाऽपि तत्र मुञ्चति भूषणम्}


\twolineshloka
{सा ददर्श गिरिप्रस्थे पञ्च वानरपुङ्गवान्}
{तत्र वासो महद्दिव्यमुत्ससर्ज मनस्विनी}


\twolineshloka
{तत्तेषां वानरेन्द्राणां पपात पवनोद्धतम्}
{मध्ये सुपीतं पञ्चानां विद्युन्मेघान्तरे यथा}


\twolineshloka
{अचिरेणातिचक्राम खेचरः खे चरन्निव}
{ददर्शाथ पुरीं रम्यां बहुद्वारां मनोरमाम्}


\twolineshloka
{प्राकारवप्रसंबाधां निर्मितां विश्वकर्मणा}
{प्रविवेशपुरीं लङ्कां ससीतो राक्षसेश्वरः}


\twolineshloka
{एवं हृतायां वैदेह्यां रामो हत्वा महामृगम्}
{निवृत्तो ददृशे दूराद्भ्रातरं लक्ष्मणं तदा}


\twolineshloka
{कथमुत्सृज्य वैदेहीं वने राक्षससेविते}
{इति तं भ्रातरं दृष्ट्वा प्राप्तोऽसीति व्यगर्हयत्}


\twolineshloka
{मृगरूपधरेणाथ रक्षसासोपकर्षणम्}
{भ्रातुरागमनं चैवचिन्तयन्पर्यतप्यत}


\twolineshloka
{गर्हयन्नेव रामस्तु त्वरितस्तं समासदत्}
{अपि जीवति वैदेहीमिति पश्यामि लक्ष्मण}


\twolineshloka
{तस् तत्सर्वमाचख्यौ सीताया लक्ष्मणो वचः}
{यदुक्तवत्यसदृशं वैदेही पश्चिमं वचः}


\twolineshloka
{दह्यमानेन तु हृदा रामोऽभ्यपतदाश्रमम्}
{स ददर्श रतदा गृध्रं निहतं पर्वतोपमम्}


\twolineshloka
{राक्षसं शङ्कमानस्तं विकृष्य बलवद्धनुः}
{अभ्यधावत काकुत्स्थस्ततस्तं सहलक्ष्मणः}


\twolineshloka
{स तावुवाच तेजस्वी सहितौ रामलक्ष्मणौ}
{गृध्रराजेस्मि भद्रंवां सखा दशरथस् वै}


\twolineshloka
{तस्य तद्वचनं श्रुत्वा संगृह्य धनुषी शुभे}
{कोयं पितरमस्माकं नाम्नाऽऽहेत्यूचतुश्च तौ}


\twolineshloka
{ततो ददृशतुस्तौ तं छिननपक्षद्वयं खगम्}
{तयोः शशंस गृध्रस्तु सीतार्थे रावणाद्वधम्}


\twolineshloka
{अपृच्छद्राघवो गृध्रं रावणः कां दिशं गतः}
{तस् गृध्रः शिरःकम्पैराचचक्षे ममार च}


\twolineshloka
{दक्षिणामिति काकुत्स्थो विदित्वाऽस्य तदिङ्गितम्}
{संस्कारं लम्भयामास सखायं पूजयन्पितुः}


\twolineshloka
{ततो दृष्ट्वाऽऽश्रमपदं व्यपविद्धबृसीकटम्}
{विध्वस्तकलशं शून्यं गोमायुशतसंकुलम्}


\twolineshloka
{दुःखशोकसमाविष्टौ वैदेहीहरणार्दितौ}
{जग्मतुर्दण्डकारण्यं दक्षिणेन परंतपौ}


\twolineshloka
{वने महति तस्मिंस्तु रामः सौमित्रिणा सह}
{ददर्श मृगयूथनि द्रवमाणानि सर्वशः}


\twolineshloka
{शब्दं च घोरं सत्वानां दावाग्नरिववर्धतः}
{अपश्येतां मुहूर्ताच्च कबन्धं घोरदर्शनम्}


\twolineshloka
{मेघपर्वतसंकाशं सालस्कन्धं महाभूजम्}
{उरोगतविशालाक्षं महोदरमहामुखम्}


\twolineshloka
{यदृच्छयाथ तद्रक्षः करे जग्राह लक्ष्मणम्}
{विषादमगमत्सद्यः सौमित्रिरथ भारत}


\twolineshloka
{स राममभिसंप्रेक्ष्य कृष्यते येन तन्मुखम्}
{विषण्णश्चाब्रवीद्रामं पश्यावस्थामिमां मम}


\twolineshloka
{हरणं चैववैदेह्या मम चायमुपप्लवः}
{राज्यभ्रंशश्च भवतस्तातस्य मरणं तथा}


\twolineshloka
{नाहं त्वां मह वैदेह्या समेतं कोसलागतम्}
{द्रक्ष्यामि प्रथिते राज्येपितृपैतामहे स्थितम्}


\twolineshloka
{द्रक्ष्यन्त्यार्यस्य धन्या ये कुशलाजशमीदलैः}
{अभिषिक्तस् वदनं सोमं शान्तघनं यथा}


\twolineshloka
{एवं बहुविधं धीमान्विललाप स लक्ष्मणः}
{तमुवाचाथकाकुत्स्थः संभ्रमेष्वप्यसंभ्रमः}


\threelineshloka
{मा विषीद नरव्याघ्र नैष कश्चिन्मयि स्थिते}
{शक्तो धर्षयितुं वीर सुमित्रानन्दवर्धन}
{छिन्ध्यस्य दक्षिणं बाहुं छिन्नः सव्यो मया भुजः}


\twolineshloka
{इत्येवं वदता तस् भुजो रामेण पातितः}
{खङ्गेन भृशतीक्ष्णेन निकृत्तस्तिलकाण्डवत्}


\twolineshloka
{ततोऽस्य दक्षिणं बाहुं स्वङ्गेनाजघ्निवान्बली}
{सौमित्रिरपि संप्रेक्ष्यभ्रातरं राघवं स्थितम्}


\twolineshloka
{पुनर्जघान पार्श्वे वै तद्रक्षो लक्ष्मणो भृशम्}
{गतासुरपतद्भूमौ कबन्धः सुमहांस्ततः}


\twolineshloka
{तस्य देहाद्विनिःसृत्य पुरुषो दिव्यदर्शनः}
{ददृशे दिवमास्थाय दिवि सूर्य इव ज्वलन्}


\twolineshloka
{पप्रच्छ रामस्तं वाग्मी कस्त्वं प्रब्रूहि पृच्छतः}
{कामया किमिदं चित्रमाश्चर्यं प्रतिभाति मे}


\twolineshloka
{तस्याचचक्षेगन्धर्वोविश्वावसुरहं नृप}
{प्राप्तो ब्राह्मणशपेन योनिं राक्षससेविताम्}


\twolineshloka
{रावणेन हृतासीता लङ्कायां संनिवेशिता}
{सुग्रीवमभिगच्छस्वस ते साह्यं करिष्यति}


\twolineshloka
{एषा पम्पा शिवजला हंसकारण्डवायुता}
{ऋश्यमूकस्य शैलस्य संनिकर्षे तटाकिनी}


\twolineshloka
{वसते तत्रसुग्रीवश्चतुर्भिः सचिवैः सह}
{भ्राता बानरराजस् वालिनो हेममालिनः}


\twolineshloka
{तेन त्वं सहसंगम्य दुःखमूलं निवेदय}
{समानशीलो भवतः साहाय्यं स करिष्यति}


\twolineshloka
{एतावच्छक्यमस्माभिर्वक्तुं द्रष्टासि जानकीम्}
{ध्रुवं वानरराजस् विदितो रावणालयः}


\twolineshloka
{इत्युक्त्वाऽन्तर्हितो दिव्यः पुरुषः स महाप्रभः}
{विस्मयं जग्मतुश्चोभौ प्रवीरौ रामलक्ष्मणौ}


\sect{अध्यायः २८१}
\twolineshloka
{मार्कण्डेय उवाच}
{}


\twolineshloka
{ततोऽविदूरे नलिनीं रप्रभूतकमलोत्पलाम्}
{सीताहरणदुःखार्तः पम्पां रामः समासदत्}


\twolineshloka
{मारुतेन सुशीतेन सुखेनामृतगन्धिना}
{सेव्यमानो वने तस्मिञ्जगाम मनसा प्रियाम्}


\twolineshloka
{विललाप सराजेन्द्रस्तत्रकान्तानुस्मरन्}
{कामबाणाभिसंतप्तं सौमित्रिस्तमथाब्रवीत्}


\twolineshloka
{न त्वामेवंविधो भावः स्प्रष्टुमर्हति मानद}
{आत्मवन्तमिव व्याधिः पुरुषंवृद्धसेविनम्}


\twolineshloka
{प्रवृत्तिरुपलब्धा ते वैदेह्या रावणस्य च}
{तां त्वं पुरुषकारेण बुद्ध्या चैवोपपादय}


\twolineshloka
{अभिगच्छाव सुग्रीवं शैलस्थं हरिपुङ्गवम्}
{मयि शिष्ये च भृत्ये च सहाये च समाश्वस}


\twolineshloka
{एवं बहुविधैर्वाक्यैर्लक्ष्मणेन स राघवः}
{उक्तः प्रकृतिमापेदे कार्ये चानन्तरोऽभवत्}


\twolineshloka
{निषेव्य वारि पम्पायास्तर्पयित्वा पितृनपि}
{प्रतस्थतुरभौ वीरौ भ्रातरौ रामलक्ष्मणौ}


\twolineshloka
{तावृश्यमूकमभ्येत्य बहुमूलफलद्रुमम्}
{गिर्यग्रे वानरान्पञ्वीरौ ददृशतुस्तदा}


\twolineshloka
{सुग्रीवः प्रेषयामास सचिवं वानरं तयोः}
{बुद्धिमन्तं हनूमन्तं हिमवन्तमिव स्थितम्}


\twolineshloka
{तेन संभाष्य पूर्वं तौ सुग्रीवमभिजग्मतुः}
{रसख्यं वानरराजेन चक्रे रामस्तदा नृप}


\twolineshloka
{ततः सीतां हृतां श्रुत्वा सुग्रीवो वालिना कृतम्}
{दुःखमाख्यातवान्सर्वं रामायामिततेजसे}


\twolineshloka
{तद्वासो दर्शयामास तस् कार्ये निवेदिते}
{वानराणां तु यत्सीता ह्रियमाणा व्यपासृजत्}


\twolineshloka
{तत्प्रत्ययकरं लब्ध्वा सुग्रीवं प्लवगाधिपम्}
{पृथिव्यां वानरैश्वर्ये स्वयंरामोऽभ्यषेचयत्}


\twolineshloka
{प्रतिजज्ञे चकाकुत्स्थः समरे वालिनो वधम्}
{सुग्रीवश्चापि वैदेह्याः पुनरानयनं नृप}


\twolineshloka
{इत्येवं समयं कृत्वाविश्वास्य च परस्परम्}
{अभ्येत्य सर्वकिष्किन्धां तस्थुर्युद्धाभिकाङ्क्षिणः}


\twolineshloka
{सुग्रीवः प्राप्यकिष्किन्धां ननादौघनिभस्वनः}
{नसाय् तन्ममृषे वाली तारा तं प्रत्यषेधयत्}


\twolineshloka
{यथानदतिसुग्रीवो बलवानेष वानरः}
{मन्ये चाश्रयवान्प्राप्तो न त्वं निष्क्रान्तुमर्हसि}


\twolineshloka
{हेममाली ततो वाली तारां ताराधिपाननाम्}
{प्रोवाच वचनं वाग्मी तां वानरपतिः पतिः}


% Check verse!
\twolineshloka
{सर्वभूतरुतज्ञा शृणु सर्वं कपीश्वर}
{केन चाश्रयवान्प्राप्तो ममैष भ्रातृगन्धिकः}

\twolineshloka
{चिन्तयित्वा मुहूर्तं तु तारा ताराधिपप्रभा}
{पतिमित्यब्रवीत्प्राज्ञा शृणु सर्वं कपीश्वर}


\twolineshloka
{हृतदारो महासत्वोरामो दशरथात्मजः}
{रतुल्यारिमित्रतां प्राप्तः सुग्रीवेण धनुर्धरः}


\twolineshloka
{भ्राता चास्य महाबाहुः सौमित्रिरपराजितः}
{लक्ष्मणो नाम मेधावी स्थितः कार्यार्थसिद्धये}


\twolineshloka
{मैन्दश्च द्विविदश्चापि हनूमांश्चानिलात्मजः}
{जाम्बवानृक्षराजश्च सुग्रीवसचिवाः स्थिताः}


\twolineshloka
{सर्व एते महात्मानो बुद्धिमन्तो महाबलाः}
{अलं तव विनाशाय रामवीर्यव्यपाश्रयाः}


\twolineshloka
{तस्यास्तदाक्षिप्य वचो हितमुक्तं कपीश्वरः}
{पर्यशङ्कत तामीर्षुः सुग्रीवगतमानसाम्}


\twolineshloka
{तारां परुषमुक्त्वा तु निर्जगाम गुहामुखात्}
{स्थितं माल्यवतोऽभ्याशे सुग्रीवं सोभ्यभाषत}


\twolineshloka
{असकृत्त्वं मया क्लीव निर्जितो जीवितप्रियः}
{मुक्तो गच्छसि दुर्बुद्धे कथंकारं रणे पुनः}


\twolineshloka
{इत्युक्तः प्राहसुग्रीवो भ्रातरं हेतुमद्वचः}
{प्राप्तकालममित्रघ्नं रामं सम्बोधयन्निव}


\twolineshloka
{हृतराज्यस्य मे राजन्हृतदारस्य च त्वया}
{किं मे जीवितसामर्थ्यमिति विद्धि समागतम्}


\twolineshloka
{एवमुक्त्वाबहुविधं ततस्तौ सन्निपेततुः}
{समरे वालिसुग्रीवौ सालतालशिलायुधौ}


\twolineshloka
{उभौ जघ्नतुरन्योन्यमुभौ भूमौ निपेततुः}
{उभौ ववल्गतुश्चित्रं मुष्टिभिश्च निजघ्नतुः}


\twolineshloka
{उभौ रुधिरसंसिक्तौ नखदन्तपरिक्षतौ}
{शुशुभाते तदा वीरौ पुष्पिताविव किंशुकौ}


\twolineshloka
{न विशेषस्तयोर्युद्धे यदा कश्चन दृश्यते}
{सुग्रीवस् तदा मालां हनुमान्कण्ठ आसजत्}


\twolineshloka
{स मालया तदा वीरः शुशुभे कण्ठसक्तया}
{श्रीमानिव महाशैलो मलयो मेघमालया}


\twolineshloka
{कृतचिह्नं तु सुग्रीवं रामो दृष्ट्वा महाधनुः}
{विचकर्ष धनुःश्रेष्ठं वालिमुद्दिश्य लक्षयन्}


\twolineshloka
{विष्फारस्तस् धनुषो यन्त्रस्येव तदा बभौ}
{वितत्रास तदा वाली शरेणाभिहतो हृदि}


\twolineshloka
{स भिन्नहृदयो वाली वक्राच्छोणितमुद्वमन्}
{ददर्शावस्थितं रामं ततः सौमित्रिणा सह}


\twolineshloka
{गर्हयित्वास काकुत्स्थं पपात भुवि मूर्च्छितः}
{तारा ददर्श तं भूमौ तारापतिमिव च्युतम्}


\twolineshloka
{हते वालिनि सुग्रीवः किष्किन्धां प्रत्यपद्यत}
{तां तारापतिमुखीं तारां निपतितेश्वराम्}


\twolineshloka
{रामस्तु चतुरो मासान्पृष्ठे माल्यवतः शुभे}
{निवासमकरोद्धीमान्सुग्रीवेणाभ्युपस्थितः}


\twolineshloka
{रावणोऽपिपुरीं गत्वालङ्कां कामबलात्कृतः}
{सीतां निवेशयामास भवने नन्दनोपमे}


\twolineshloka
{अशोकवनिकाभ्यासे तापसास्रमसन्निभे}
{भर्तृस्मरणतन्वङ्गी तापसीवेषधारिणी}


\twolineshloka
{उपवासतपःशीला ततः सा पृथुलेक्षणा}
{उवास दुःखवसतिं फलमूलकृताशना}


\twolineshloka
{दिदेश राक्षसीस्तत्ररक्षणे राक्षसाधिपः}
{प्रासासिशूलपरशुमुद्गरालातधारिणीः}


\twolineshloka
{द्व्यक्षीं त्र्यक्षीं ललाटक्षीं दीर्घजिह्वामजिह्विकाम्}
{त्रिस्तनीमेकपादां च त्रिजटामेकलोचनाम्}


\twolineshloka
{एताश्चान्याश्च दीप्ताक्ष्यः करभोत्कटमूर्धजाः}
{परिवार्यासते सीतां दिवारात्रमतन्द्रिताः}


\twolineshloka
{तास्तु तामायतापाङ्गीं पिशाच्यो दारुणस्वराः}
{तर्जयन्ति सदा रौद्राः परुषव्यञ्जनस्वराः}


\twolineshloka
{खादाम पाटयामैनां तिलशः प्रविभज्यताम्}
{येयं भर्तारमस्माकमवमत्येह जीवति}


\twolineshloka
{इत्येवं परिभर्त्सन्तीस्त्रासयानाः पुनः पुनः}
{भर्तृशोकसमाविष्टा निःश्वस्येदमुवाच ताः}


\twolineshloka
{आर्याः खादत मां शीघ्रं न मे लोभोस्ति जीविते}
{विना तं पुण्डरीकाक्षं नीलकुञ्चितमूर्धजम्}


\twolineshloka
{अद्यैवाहं निराहारा जीवितप्रियवर्जिता}
{शोषयिष्यामि गात्राणि बल्ली तलगता यथा}


\twolineshloka
{न त्वन्यमभिगच्छेयं पुमांसं राघवादृते}
{इति जानीत सत्यं मेक्रियतां यदनन्तरम्}


\twolineshloka
{तस्यास्तद्वचनं श्रुत्वा राक्षस्यस्ताः खरस्वनाः}
{आख्यातुं राक्षसेन्द्राय जन्मुस्तत्सर्वमादितः}


\twolineshloka
{गतासु तासु सर्वासु त्रिजटा नाम राक्षसी}
{सान्त्वयामास वैदेहीं धर्मज्ञा प्रियवादिनी}


\twolineshloka
{सीते वक्ष्यामि ते किंचिद्विश्वासं करु मे सखि}
{भयं त्वं त्यज वामोरु शृणु चेदं वचो मम}


\twolineshloka
{अविन्ध्यो नाम मेधावी वृद्धो राक्षसपुङ्गवः}
{स रामस्य हितान्वेषी त्वदर्थे मामचूचुदत्}


\twolineshloka
{सीता मद्वचनाद्वाच्या समाश्वास्य प्रसाद्य च}
{भर्ता तेकुशली रामोलक्ष्मणानुगतो बली}


\twolineshloka
{सख्यं वानरराजेन शक्रप्रतिमतेजसा}
{कृतवान्राघवः श्रीमांस्त्वदर्थे च समुद्यतः}


\twolineshloka
{मा च ते भूद्भयं भीरु रावणाल्लोकगर्हितात्}
{नलकूबरशापेन रक्षिता ह्यसि नन्दिनि}


\twolineshloka
{शप्तो ह्येष पुरा पापो वधूं रम्भां परामृशन्}
{न शक्रोत्यवशां नारीमुपैतुमजितेन्द्रियः}


\twolineshloka
{क्षिप्रमेष्यति ते भर्ता सुग्रीवेणाभिरक्षितः}
{सौमित्रिसहितो धीमांस्त्वां चेतो मोक्षयिष्यति}


\twolineshloka
{स्वप्ना हि सुमहाघोरा दृष्टा मेऽनिष्टदर्शनाः}
{विनाशायास्य दुर्बुद्धेः पौलस्त्यस्य कुलस्य च}


\twolineshloka
{दारुणो ह्येष दुष्टात्मा क्षुद्रकर्मा निशाचरः}
{स्वभावाच्छीलदोषेण सर्वेषां भयवर्धनः}


\twolineshloka
{स्पर्धते सर्वदेवैर्यः कालोपहतचेतनः}
{मया विनासलिङ्गानि स्वप्ने दृष्टानि तस्य वै}


\twolineshloka
{तैलाभिषिक्तो विकचो मज्जनप्के दशाननः}
{असकृत्स्वरयुक्ते तु रथे नृत्यन्निव स्थितः}


\twolineshloka
{कुम्भकर्णादयश्चेमे नग्नाः पतितमूर्धजाः}
{गच्छन्ति दक्षिणामाशां रक्तमाल्यानुलेपनाः}


\twolineshloka
{श्वेतातपत्रः सोष्णीषः शुक्लमाल्यानुलेपनः}
{श्वेतपर्वतमारूढ एक एव विभीषणः}


\twolineshloka
{सचिवाश्चास्य चत्वारः शुक्लमाल्यानुलेपनाः}
{श्वेतपर्वतमारूढा मोक्ष्यन्तेऽस्मान्महाभयात्}


\twolineshloka
{रामस्यास्त्रेण पृथिवी परिक्षिप्ता ससागरा}
{यशसा पृथिवीं कृत्स्नां पूरयिष्यति ते पतिः}


\twolineshloka
{हस्तिसक्थिसमारूढो भुञ्जानो मधुपायसम्}
{लक्ष्मणश्च मया दृष्टो दिधक्षुः सर्वतो दिशम्}


\twolineshloka
{रुदती रुधिरार्द्राङ्गी व्याघ्रेण परिरक्षिता}
{असकृत्त्वं मया दृष्टा गच्छन्ती दिशमुत्तराम्}


\twolineshloka
{हर्षमेष्यसि वैदेहि क्षिप्रं भर्त्रा समन्विता}
{राघवेण सहभ्रात्रा सीते त्वमचिरादिव}


\twolineshloka
{इत्येतन्मृगशावाक्षी तच्छ्रुत्वा त्रिजटावचः}
{बभूवाशावती बाला पुनर्भर्तृसमागमे}


\twolineshloka
{तावदभ्यागता रौद्र्यः पिशाच्यस्ताःसुदारुणाः}
{ददृशुस्तां त्रिजटया सहासीनां यथापुरम्}


\sect{अध्यायः २८२}
\twolineshloka
{मार्कण्डेय उवाच}
{}


\twolineshloka
{ततस्तां भर्तृशोकार्तां दीनां मलिनवाससम्}
{मणिशेषाभ्यलंकारां रुदतीं च पतिव्रताम्}


\twolineshloka
{राक्षसीभिरुपास्यन्तीं समासीनां शिलातले}
{रावणःकामबाणार्तो ददर्शोपससर्प च}


\twolineshloka
{देवदानवगन्धर्वयक्षकिंपुरुषैर्युधि}
{अजितोशोकवनिकां ययौ कन्दर्पपीडितः}


\twolineshloka
{दिव्याम्बरधरः श्रीमन्सुमृष्टमणिकुण्डलः}
{विचित्रमाल्यमुकुटो वसन्त इव मूर्तिमान्}


\twolineshloka
{न कल्पवृक्षसदृशोयत्नादपि विभूषितः}
{श्मशानचैत्यद्रुमवद्भूषितोऽपि भयंकरः}


\twolineshloka
{स तस्यास्तनुमध्यायाः समीपे रजनीचरः}
{ददृशे रोहिणीमेत्य शनैश्चर इव ग्रैहः}


\twolineshloka
{स तामामन्त्र्य सुश्रोणीं पुष्पकेतुशराहतः}
{इदमित्यब्रवीद्वाक्यं त्रस्तां रौहीमिवाबलाम्}


\twolineshloka
{सीते पर्याप्तमेतावत्कृतोभर्तुरनुग्रहः}
{प्रसादं कुरु तन्वङ्गि क्रियतां परिकर्म ते}


\twolineshloka
{भजस्वमां वरारोहे महार्हाभरणाम्बरा}
{भवमे सर्वनारीणामुत्तमा वरवर्णिनी}


\threelineshloka
{सन्ति मे देवक्न्याश्च गन्धर्वाणआं च योषितः}
{सन्ति दानवन्याश् दैत्यानां चापि योषितः}
{तासामद्यविशालाक्षि सर्वासां मे भवोत्तमा}


\twolineshloka
{चतुर्दश पिशाचीनां कोट्यो मे वचने स्थिताः}
{द्विस्तावत्पुरुषादानां रक्षसां भीमकर्मणाम्}


\twolineshloka
{ततो मे त्रिगुणा यक्षा ये मद्वचनकारिणः}
{केचिदेव धनाध्यक्षं भ्रातरं मे समाश्रिताः}


\twolineshloka
{गन्दर्वाप्सरसो भद्रे मामापानगतं सदा}
{उपतिष्ठन्ति वामोरु यथैव भ्रातरं मम}


\twolineshloka
{पुत्रोऽहमपि विप्रर्षेः साक्षाद्विश्रवसो मुनेः}
{पञ्चमो लोकपालानामिति मे प्रथितं यशः}


\twolineshloka
{दिव्यानि भक्ष्यभोज्यानि पानानि विविधानि च}
{यथैव त्रिदशेशस्यतथैव मम भामिनि}


\twolineshloka
{क्षीयतां दुष्कृतं कर्म वनवासकृतं तव}
{भार्या मे भवसुश्रोणि यथा मण्डोदरीतथा}


\twolineshloka
{इत्युक्ता तेन वैदेही परिवृत्य सुभानना}
{तृणमनतरतः कृत्वातमुवाच निशाचरम्}


\twolineshloka
{अशिवेनातिवामोरूरजस्रं नेत्रवारिणा}
{स्तनावपतितौ बाला संहतावभिवर्षती}


\twolineshloka
{व्यवस्थाप्यकथंचित्सा विषादादतिमोहिता}
{उवाच वाक्यं तं क्षुद्रं वैदेही पतिदेवता}


\threelineshloka
{असकृद्वदतो वाक्यमीदृशं राक्षसेश्वर}
{विषादयुक्तमेतत्ते मया श्रुतमभाग्यया}
{तद्भद्रमुख भद्रं ते मानसं विनिवर्त्यताम्}


\twolineshloka
{परदाराऽस्म्यलभ्या च सततं च पतिव्रता}
{न चैवौपयिकी भार्य मानुषी तव राक्षस}


\twolineshloka
{विवशां धर्षयित्वच कां त्वं प्रीतिमवाप्स्यसि}
{न च पालयसे धर्मं लोकपालसमः कथम्}


\twolineshloka
{भ्रातरं राजराजं तं महेश्वरसस्वं प्रभुम्}
{धनेश्वरं व्यपदिशन्कथं त्विह न लज्जसे}


\twolineshloka
{इत्युक्त्वा प्रारुदत्सीता कम्पयन्ती पयोधरौ}
{शिरोधरां च तन्वङ्गी मुस्वं प्रच्छाद्यवाससा}


\twolineshloka
{तस्य रुदत्या भामित्या दीर्घा वेणी सुसयता}
{ददृशे स्वसिता स्निग्धा काली व्यालीव मूर्धनि}


\twolineshloka
{श्रुत्वा तद्रावणो वाक्यं सीतयोक्तं सुनिषुरम्}
{प्रत्याख्यातोऽपिदुर्मेधाः पुनरेवाब्रवीद्वचः}


\twolineshloka
{काममङ्गनि मे सीते दुनोतु मकरध्वजः}
{नत्वामकामां सुश्रोणीं समेप्ये चारुहासिनीं}


\twolineshloka
{किंनु शक्यं मया कर्तुं यत्त्वमद्यापिमानुषम्}
{आहारभूतमस्माकं राममेवानुरुध्यसे}


\twolineshloka
{इत्युक्त्वा तामनिन्द्याङ्गीं स राक्षसमहेश्वरः}
{तत्रैवान्तर्हितो भूत्वा जगामाभिमतां दिशम्}


\twolineshloka
{राक्षसीभिः परिवृतावैदेही शोककशिंता}
{सेव्यमाना त्रिजटया तत्रैव न्यवसत्तदा}


\sect{अध्यायः २८३}
\twolineshloka
{मार्कण्डेय उवाच}
{}


\twolineshloka
{राघवः सहसौमित्रिः सुग्रीवेणाभिपालितः}
{वसनमाल्यवतः पृष्ठे ददर्श विमलं नभः}


\twolineshloka
{सदृष्ट्वाविमले व्योम्नि निर्मलं शसलक्षणम्}
{ग्रहनक्षत्रताराभिरनुयान्तममित्रहा}


\twolineshloka
{कुमुदोत्पलपद्मानां गन्धमादाय वायुना}
{महीधरस्थः शीतेन सहसाप्रतिबोधितः}


\twolineshloka
{प्रभाते लक्ष्मणं वीरमभ्यभाषत दुर्मनाः}
{सीतां संस्मृत् यधर्मात्मा रुद्धां राक्षसवेश्मनि}


\twolineshloka
{गच्छ लक्ष्मण जानीहि किष्किंदायां कपीश्वरम्}
{प्रमत्तं ग्राम्यधर्मेषु कृतघ्नं स्वार्थपण्डितम्}


\twolineshloka
{योसौ कुलाधमो मूढो मया राज्येऽभिषेचितः}
{सर्ववानरगोपुच्छा यमृक्षाश्च भजन्ति वै}


\twolineshloka
{यदर्थं निहतो बाली मया रघुकुलोद्वह}
{त्वया सहमहाबाहो किष्किन्धोपवने तदा}


\twolineshloka
{कृतघ्नं तमहं मन्ये वानरापशदं भुवि}
{यो मामेवंगतो मूढो न जानीतेऽद्य लक्ष्मण}


\twolineshloka
{असौ मन्ये न जानीते समयप्रतिपालनम्}
{कृतोपकारं मां नूनमवमत्याल्पया धिया}


\twolineshloka
{यदितावदनुद्युक्तः शेते कामसुखात्मकः}
{नेतव्यो वालिमार्गेण सर्वभूतगतिं त्वया}


\twolineshloka
{अथापि घटतेऽस्माकमर्ते वानरपुङ्गवः}
{तमादायैव काकुत्स्थ त्वरावान्भव माचिरम्}


\twolineshloka
{इत्युक्तो लक्ष्मणो भ्रात्रा गुरुवाक्यहिते रतः}
{प्रतस्थे रुचिरं गृह्य समार्गणगुणं धनुः}


\twolineshloka
{किष्किन्धाद्वारमासाद्यप्रविवेशानिवारितः}
{सक्रोध इतितं मत्वाराजा प्रत्युद्ययौ हरिः}


\twolineshloka
{तं सदारोविनीतात्मा सुग्रीवः प्लवगाधिपः}
{पूजया प्रतिजग्राह प्रीयमाणस्तदर्हया}


\twolineshloka
{तमब्रवीद्रामवचः सौमित्रिरकुतोभयः}
{स तत्सर्वमशेषेण श्रुत्वा प्रह्वः कृताञ्जलिः}


\twolineshloka
{सभृत्यदारो राजेन्द्रसुग्रीवो वानराधिपः}
{इदमाह वचः प्रीतो लक्ष्मणं नरकुञ्जरम्}


\twolineshloka
{नास्मि लक्ष्मण दुर्मेधा नाकृतज्ञो न निर्घृणः}
{श्रूयतां यः प्रयत्नो मे सीतापर्येषणे कृतः}


\twolineshloka
{दिशः प्रस्थापिताः सर्वेविनीता हरयो मया}
{सर्वेषां च कृतः कालो मासेऽभ्यागमने पुनः}


\twolineshloka
{यैरियं सवना साद्रिः सपुरा सागराम्बरा}
{विचेतव्या मही वीर सग्रामनगराकरा}


\twolineshloka
{स मासः पञ्चरात्रेण पूर्णो भवितुमर्हति}
{ततः श्रोष्यसि रामेण सहितः सुमहत्प्रियम्}


\twolineshloka
{इत्युक्तो लक्ष्मणत्तेन कवानरेनद्रेण धीमता}
{त्यक्त्वारोषमदीनात्मा सुग्रीवं प्रत्यपूजयत्}


\twolineshloka
{सरामं सहसुग्रीवो माल्यवत्पुष्ठमास्थितम्}
{भिगम्योदयं तस्य कार्यस्य प्रत्यवेदयत्}


\twolineshloka
{इत्येवंवानरेनद्रास्ते समाजन्मुः सहस्रशः}
{दिशस्तिस्रो विचित्याथ न तु ये दक्षिणां गताः}


\twolineshloka
{आचख्युस्तत्र रामाय महीं सागरमेखलाम्}
{विचितां न तु वैदेह्या दर्शनं रावणस् वा}


\twolineshloka
{गतास्तु दक्षिणामाशां ये वै वानरपुङ्गवाः}
{आशावांस्तेषु काकुत्स्थः प्राणानार्तोऽभ्यधारयत्}


\twolineshloka
{द्विमासोपरमे काले व्यतीते प्लवगास्ततः}
{सुग्रीवमभिगम्येदं त्वरिता वाक्यमब्रुवन्}


\twolineshloka
{रक्षितंवालिना यत्तत्स्फीतं मधुवनं महत्}
{त्वया च प्लवगश्रेष्ठ तद्भुङ्क्ते पवनात्मजः}


\twolineshloka
{वालिपुत्रोऽङ्गदश्चैव ये चान्ये प्लवगर्षभाः}
{विचेतुं दक्षिणामाशां राजन्प्रस्थापितास्त्वया}


\twolineshloka
{तेषामपनयं श्रुत्वा मेने सकृतकृत्यताम्}
{कृतार्थानां हि भृत्यानामेतद्भवति चेष्टितम्}


\twolineshloka
{स तद्रामाय मेधावी शशंस प्लवगर्षभः}
{रामश्चाप्यनुमानेन मेने दृष्टां तु मैथिलीम्}


\twolineshloka
{हनुमत्प्रमुखाश्चापि विश्रान्तास्ते प्लवङ्गमाः}
{अभिजग्मुर्हरीन्द्रं तं रामलक्ष्मणसन्निधौ}


\twolineshloka
{गतिं च मुखवर्णं च दृष्ट्वारामो हनूमतः}
{अगमत्प्रत्ययं भूयो दृष्टा सीतेति भारत}


\twolineshloka
{हनूमत्प्रमुखास्ते तु वानराः पूर्णमानसाः}
{प्रणेमुर्विधिवद्रामं सुग्रीवं लक्ष्मणं तथा}


\twolineshloka
{तानुवाचानतान्रामः प्रगृह्य सशरं धनुः}
{अपि मां जीवयिष्यध्वमपि वः कृतकृत्यता}


\twolineshloka
{अपि राज्यमयोध्यायां कारयिष्याम्यहं पुनः}
{निहत्यसमरे शत्रूनाहृत्यजनकात्मजाम्}


\twolineshloka
{अमोक्षयित्वावैदेहीमहत्वा च रणे रिपून्}
{हृतदारोऽवधूतश्चनाहं जीवितुमुत्सहे}


\twolineshloka
{इत्युक्तवचनं रामं प्रत्युवाचानिलात्मजः}
{प्रियमाख्यामि ते राम दृष्टा सा जानकी मया}


\twolineshloka
{विचित्य दक्षिणामाशां सपर्वतवनाकराम्}
{श्रान्ताः काले व्यतीते स्म दृष्टवन्तो महागुहां}


\twolineshloka
{प्रविशामो वयं तां तु बहुयोजनमायताम्}
{अन्धकारां सुविपिनां गहनां कीटसेविताम्}


\twolineshloka
{गत्वा सुमहदध्वानमादित्यस् प्रभां ततः}
{दृष्टवन्तः स्म तत्रैवभवनं दिव्यमन्तरा}


\twolineshloka
{गयस् किल दैत्यस् तदा सद्वेश्म राघव}
{तत्रप्रभावती नाम तपोऽतप्यत तापसी}


\twolineshloka
{तया दत्तानि भोज्यानिपानानिविविधानि च}
{भुकत््वा लब्धबलाः सन्तस्तयोक्तेन पथा ततः}


\twolineshloka
{निर्याय तस्मादुद्देशात्पश्यामो लवणाम्भसः}
{समीपे सह्यमलयौ दर्दुरं च महागिरिम्}


\twolineshloka
{ततो मलयमारुह्य पश्यन्तो वरुणालयम्}
{कविषण्णा व्यथिताः खिन्ना निराशा जीविते भृशम्}


\twolineshloka
{अनेकशतविस्तीर्णं योजनानां महोदधिम्}
{तिमिनक्रझषावासं चिन्तयन्तः सुदुःखिताः}


\twolineshloka
{तत्रानशनसंकल्पं कृत्वाऽऽसीना वयं तदा}
{ततः कथान्ते गृध्रस्य जटायोरभवत्कथा}


\twolineshloka
{ततः पर्वतशृङ्गाभं घोररूपं भयावहम्}
{पक्षिणं दृष्टवन्तः स्म वैनतियेमिवापरम्}


\twolineshloka
{सोऽस्मानतर्कयद्भोक्तुमथाभ्येत्य वचोऽब्रवीत्}
{भोः क एष मम भ्रातुर्जटायोः कुरुते कथाम्}


\twolineshloka
{संपातिर्नाम तस्याहं ज्येष्ठो भ्राता खगाधिपः}
{अन्योन्यस्पर्धया रूढावावामदित्यसत्पदम्}


\twolineshloka
{ततो दग्धाविमौ पक्षौ न दग्धौ तु जटायुषः}
{तस्मान्मे चिरदृष्टः स भ्राता गृध्रपतः प्रियः}


\twolineshloka
{निर्दग्धपक्षः पतितो ह्यहमस्मिन्महागिरौ}
{द्रष्टुं वीरं न शक्नोमि भ्रातरं वै जटायुषम्}


\twolineshloka
{तस्यैवं वदतोऽस्माभिर्हतो भ्राता निवेदितः}
{व्यसनं भवतश्चेदं संक्षेपाद्वै निवेदितम्}


\twolineshloka
{स सम्पातिस्तदा राजञ्श्रुत्वासुमहदप्रियम्}
{विषण्णचेताः पप्रच्छ पुनरस्मानरिंदम}


\twolineshloka
{कः सरामः कथं सीता जटायुश्च कथं हतः}
{इच्छामि सर्वमेवैतच्छ्रोतुं प्लवगसत्तमाः}


\twolineshloka
{तस्याहं सर्वमेवैतद्भवतो व्यसनागमम्}
{प्रायोपवेशने चैवहेतुं विस्तरशोऽब्रुवम्}


\twolineshloka
{सोऽस्मानाश्वासयामास वाक्येनानेन पक्षिराट्}
{रावणो विदितो मह्यं लङ्का चास्य महापुरी}


\twolineshloka
{दृष्टापारे समुद्रस्य त्रिकूटगिरिकन्दरे}
{भवित्री तत्र वैदेही न रमेऽस्त्यत्रविचारणा}


\twolineshloka
{इतितस्य वचः श्रुत्वा वयमुत्थाय सत्वराः}
{सागरक्रमणे मन्त्रं मन्त्रयामः परंतप}


\threelineshloka
{नाध्यवास्यद्यदा कश्चित्सागरस्य विलङ्घनम्}
{ततः पितरमाविश्य पुप्लुवेऽहंमहार्णवम्}
{शतयोजनविस्तीर्णं निहत्य जलराक्षसीम्}


\twolineshloka
{उपवासतपःशीला भर्तृदर्शनलालसा}
{जटिला मलदिग्धाङ्गीकृश दीना तपस्विन}


\twolineshloka
{निमित्तैस्तामहं सीतामुपलभ्य पृथग्विधैः}
{उपसृत्याब्रवं चार्यामभिगम्य रहोगताम्}


\twolineshloka
{सीते रामस्य दूतोऽहंवानरोमारुतात्मजः}
{त्वद्दर्शनमभिप्रप्सुरिह प्राप्तो विहायसा}


\twolineshloka
{राजपुत्रौ कुशलिनौ भ्रातरौ रामलक्ष्मणौ}
{सर्वशाखामृगेन्द्रेण सुग्रीवेणाभिपालितौ}


\twolineshloka
{कुशलंत्वाब्रवीद्रामःसीते सौमित्रिणा सह}
{सखिभावाच् सुग्रीवः कुशलं त्वाऽनुपृच्छति}


\twolineshloka
{क्षिप्रमेष्यति ते भर्ता सर्वशाखामृगैः सह}
{प्रत्ययं कुरु मे देवि वानरोऽस्मि न राक्षसः}


\twolineshloka
{मुहूर्तमिवच ध्यात्वा सीता मां प्रत्युवाच ह}
{अवैमि त्वांहनूमन्तमविन्ध्यवचनादहम्}


\twolineshloka
{अविन्ध्यो हि महाबाहो राक्षसो वृद्धसंमतः}
{कथितस्तेन सुग्रीवस्त्वद्विधैः सचिवैर्वृतः}


\twolineshloka
{गम्यतामिति चोक्त्वा मां सीता पादादिमं मणिम्}
{घारिता येन वैदेही कालमेतमनिन्दिता}


\twolineshloka
{प्रत्ययार्थं कथां चेमां कथयामास जानकी}
{क्षिप्तामिषीकां काकाय चित्रकूटे महागिरौ}


\twolineshloka
{भवता पुरुषव्याघ्र प्रत्यभिज्ञानकारणात्}
{एकाक्षिविकलः काकः सुदुष्टात्मा कृतश्चवै}


\twolineshloka
{ग्राहयित्वाऽहमात्मानं ततो दग्ध्वाच तां पुरीम्}
{संप्राप्त इतितं रामः प्रियवादिनमार्चयत्}


\sect{अध्यायः २८४}
\twolineshloka
{मार्कण्डेय उवाच}
{}


\twolineshloka
{ततस्तत्रैवरामस्य समासीनस्य तैः सह}
{समाजग्मुः कपिश्रेष्ठाः सुग्रीववचनात्तदा}


\twolineshloka
{वृतः कोटिसहस्रेण वानराणां तरस्विनाम्}
{श्वशुरो वालिनः श्रीमान्सुषेणो राममभ्ययात्}


\twolineshloka
{कोटीशतवृतोवाऽपिगजो गवय एव च}
{वानरेन््रौ महावीर्यौ पृथक्पृथगदृश्यताम्}


\twolineshloka
{षष्टिकोटिसहस्राणि प्रकर्षन्प्रत्यदृश्यत}
{गोलाङ्गूलो महाराज गवाक्षो भीमदर्शनः}


\twolineshloka
{गन्धमादनवासी तु प्रथितो गन्धमादनः}
{कोटीशतसहस्राणि हरीणां समकर्षत}


\twolineshloka
{पनसो नाम मेधावी वानरःसुमहाबलः}
{कोटीर्दश द्वादश च त्रिंशत्पञ्च प्रकर्षति}


\twolineshloka
{श्रीमान्दधिमुखो नाम हरिवृद्धोऽतिवीर्यवान्}
{प्रचकर्ष महासैनयं हरीणां भीमतेजसाम्}


\twolineshloka
{कृषणानां मुखपुण्ड्राणामृक्षाणां भीमकर्मणाम्}
{कोटीर्दश द्वादश च त्रिंशत्पञ्च प्रकर्षति}


\twolineshloka
{एते चान्ये च बहवो हरियूथपयूथपाः}
{असङ्ख्येया महाराज समीयू रामकारणात्}


\twolineshloka
{गिरिकूटनिभाङ्गानां सिंहानामिव गर्जताम्}
{श्रूयते तुमुलः शब्दस्तत्रतत्रप्रधावताम्}


\twolineshloka
{गिरिकूटनिभाः क्नचित्केचिन्महिषसन्निभाः}
{शरदभ्रप्रतीकाशाः केचिद्धिङ्गुलकाननाः}


\twolineshloka
{उत्पतन्तः पतन्तश्च प्लवमानाश्च वानराः}
{उद्धुन्वन्तोऽपरे रेणून्समाजग्मुः समन्ततः}


\twolineshloka
{सवानरमहासैन्यः पूर्णसागरसन्निभः}
{निवेशमकरोत्तत्रसुग्रीवानुमते तदा}


\twolineshloka
{ततस्तेषु हरीन्द्रेषु समावृत्तेषु सर्वशः}
{तिथौ प्रशस्ते नक्षत्रे मुहूर्ते चाभिपूजिते}


\twolineshloka
{तेन व्यूढेन सैन्येन लोकानुद्वर्तयन्निव}
{प्रययौ राघवः श्रीमान्सुग्रीवसहितस्तदा}


\twolineshloka
{मुखमासीत्तु सैन्यस्य हनूमान्मारुतात्मजः}
{जघनं पालयामास सौमित्रिरकुतोभयः}


\twolineshloka
{बद्धगोधाङ्गुलित्रणौ राघवौ तत्रजग्मतुः}
{वृतौ हरिमहामात्रैश्चन्द्रसूर्यौ ग्रहैरिव}


\twolineshloka
{प्रबभौ हरिसैन्यं तत्सालतालशिलायुधम्}
{सुमहच्छालिभवनं यथा सूर्योदयं प्रति}


\twolineshloka
{नलनीलाङ्गदक्राथमैन्दद्विविदपालिता}
{ययौ सुमहती सेना राघवस्यार्थसिद्धये}


\twolineshloka
{विविधेषु प्रशस्तेषु बहुमूलफलेषु च}
{प्रभूतमधुमांसेषु वारिमत्सु विवेषु च}


\twolineshloka
{निवसन्ती निराबाधा तथैवगिरिसानुषु}
{उपायाद्धिरिसेना सा क्षारोदमथ मागरम्}


\twolineshloka
{द्वितीयसागरनिमं तद्बलंबहुलध्वजम्}
{वेलावनं समासाद् निवासमकरोत्तदा}


\twolineshloka
{ततो दाशरथिः श्रीमान्सुग्रीवं प्रत्यभाषत}
{मध्ये वानरमुख्यानां प्राप्तकालमिदं वचः}


\twolineshloka
{उपायः कोनु भवतां मतः सागरलङ्घने}
{इयं हि महती सेना सागरश्चातिदुस्तरः}


\twolineshloka
{तत्रान्ये व्याहरन्ति स्म वानराः पटुमानिनः}
{समर्था लङ्घने सिन्दोर्न तत्कृत्स्नस्य वानराः}


\twolineshloka
{केचिन्नौभिर्व्यवस्यन्ति केचिच्च विविधैः प्लवैः}
{नेति रामस्तु तानसर्वान्सान्त्वयन्प्रत्यभाषत}


\twolineshloka
{शतयोजनविस्तारं न शक्ताः सर्ववानराः}
{क्रान्तुं तोयनिधिं वीरानैषा वो नैष्ठिकी मतिः}


\twolineshloka
{नावो न सन्ति सेनाया बह्व्यस्तारयितुं तथा}
{वणिजामुपघातं च कथमस्मद्विधश्चरेत्}


\twolineshloka
{विस्तीर्णं चैव नः सैन्यं हन्याच्छिद्रेण वै परः}
{प्लवोडुपप्रतारश्चनैवात्रमम रोचते}


\twolineshloka
{अहं त्विमं जलनिधिं समारप्स्याम्युपायतः}
{प्रतिशेष्याम्युपवसन्दर्शयिष्ति मां ततः}


\twolineshloka
{न चेद्दर्शयिता मार्गं धक्ष्याम्यनमहं ततः}
{महास्त्रैरप्रतिहतैरत्यग्निपवनोज्ज्वलैः}


\twolineshloka
{इत्युक्त्वा सहसौमित्रिरुपस्पृश्याथ राघवः}
{प्रतिशिस्ये जलनिधं विधिवत्कुशसंस्तरे}


\twolineshloka
{सागरस्तु ततः स्वप्ने दर्शयामास राघवम्}
{देवो नदनदीमर्ता श्रीमान्यादोगणैर्वृतः}


\twolineshloka
{कौसल्यामातरित्येवमाभाष्य मधुरं वचः}
{इदमित्याह रत्नानामाकरैः शतशो वृतः}


\threelineshloka
{ब्रूहि किं तेकरोम्यत्रसाहाय्यं पुरुषर्षभ}
{ऐक्ष्वाको ह्यस्मि ते ज्ञाती राम सत्यपराक्रमः}
{एवमुक्तः समुद्रेण रामो वाक्यमथाब्रवीत्}


\threelineshloka
{मार्गमिच्छामि सैन्यस्य दत्तं नदनदीपते}
{येन गत्वादशग्रीवं हन्याम कुलपांसनम्}
{राक्षसंसानुबन्धं तं मम भार्यापहारिणम्}


\twolineshloka
{यद्येवं याचतो मार्गं न प्रदास्यति मे भवान्}
{शरैस्त्वां शोषयिष्यामि दिव्यास्त्रयतिमन्त्रितैः}


\twolineshloka
{इत्येवंब्रुवतः श्रुत्वारामस्य वरुणालयः}
{उवाचव्यथितोवाक्यमितिबद्धाञ्जलिःस्थितः}


\twolineshloka
{नेच्छामि प्रतिघातं ते नास्मि विघ्नकरस्तव}
{शृणु चेदं वचोराम श्रुत्वा कर्तव्यमाचर}


\twolineshloka
{यदि दास्यामि ते मार्गं सैन्यस् व्रजतोऽऽज्ञया}
{अन्येऽप्याज्ञापयिष्यन्ति मामेवं धनुषोबलात्}


\twolineshloka
{अस्तित्वत्रनलो नाम वानरः शिल्पिसंमतः}
{त्वष्टुः काकुत्स्थ तनयो बलवान्विश्वकर्मणः}


\twolineshloka
{स यत्काष्ठं तृणं वाऽपिशिलां वा क्षेप्स्यते मयि}
{सर्वं तद्धारयिष्यामि स ते सेतुर्भविष्यति}


\twolineshloka
{इत्युक्त्वाऽन्तर्हिते तस्मिन्रामो नलमुवाच ह}
{कुरु सेतुं समुद्रे त्वंशक्तो ह्यसि मतो मम}


\twolineshloka
{तेनोपायेन काकुत्स्थः सतुबन्धमकारयत्}
{दशयोजनविस्तारमायतं शतयोजनम्}


\twolineshloka
{नलसेतुरिति ख्यातो योऽद्यापि प्रथितो भुवि}
{रामस्याज्ञां पुरस्कृत्य धार्यते गिरिसंनिभः}


\twolineshloka
{तत्रस्थं स तु धर्मात्मा समागच्चद्विभीषणः}
{भ्राता वै राक्षसेन्द्रस्य चतुर्भिः सचिवैः सह}


\twolineshloka
{प्रतिजग्राह रामस्तं स्वागतेन महामनाः}
{सुग्रीवस्य तु शङ्काऽभूत्प्रणिधिः स्यादिति स्मह}


\twolineshloka
{राघवः सत्यचेष्टाभिः सम्यक्व चरितेङ्गितैः}
{यदा तत्त्वेन तुष्टोऽभूत्तत एनमपूजयत्}


\twolineshloka
{सर्वराक्षसराज्येचाप्यभ्यपिञ्चद्विभीषणम्}
{चक्रे च मन्त्रसचिवं सहृदंलक्ष्मणस् च}


\twolineshloka
{विभीषणमते चैव सोऽत्यक्रामन्महार्णवम्}
{ससैन्यः सेतुना तेन मार्गेणैव नराधिपः}


\twolineshloka
{ततो गत्वासमासाद्य लङ्कोद्यानान्यनेकशः}
{भेदयामास कपिभिर्महान्ति च बहूनि च}


\twolineshloka
{तत्रास्तां रावणामात्यौ राक्षसौ शुकसारणौ}
{चरौ वानररूपेण तौ जग्राह विभीषणः}


\twolineshloka
{प्रतिपन्नौ यदा रूपं राक्षसं तौ निशाचरौ}
{दर्शयित्वा ततः सैन्यं रामः पश्चादवासृजत्}


\twolineshloka
{निवेश्योपवने सैन्यं स शूरः प्राज्यवानरम्}
{प्रेषयामास दुत्येन रावणस्य ततोऽङ्गदम्}


\sect{अध्यायः २८५}
\twolineshloka
{मार्कण्डेय उवाच}
{}


\twolineshloka
{प्रभूतान्नोदकेतस्मिन्बहुमूलफले वने}
{सेनां निवेश्य काकुत्स्थो विधिवत्पर्यरक्षत}


\twolineshloka
{रावणः संविधं चक्रे लङ्कायां शास्त्रनिर्मिताम्}
{प्रकृत्यैवदुराधर्षा दृढप्राकारतोरणा}


\twolineshloka
{अगाधतोयाः परिखा मीननक्रसमाकुलाः}
{बभूवुः सप्त दुर्धर्षाः स्वादिरैः शङ्कुभिश्चिताः}


\twolineshloka
{कर्णाटयन्त्रा दुर्धर्षा बभूवुः सहुडोपलाः}
{साशीविषघटायोधाः ससर्जरसपांसवः}


\twolineshloka
{मुसलालातनाराचतोमरासिपरश्वथैः}
{अन्विताश्चशतघ्नीभिः समधूच्छिष्टमुद्गराः}


\twolineshloka
{पुरद्वारेषु सर्वेषु गुल्माः स्थावरजङ्गमाः}
{बभूवुः पत्तिबहुलाः प्रभूतगजवाजिनः}


\twolineshloka
{अङ्गदस्त्वथ लङ्कायां द्वारदेशमुपागतः}
{विदितो रराक्षसेन्द्रस्य प्रविवेशगतव्यथः}


\twolineshloka
{मध्ये राक्षसकोटीनां बह्वीनां सुमहाबलः}
{शुशुभे मेघमालाभिरादित्य इव संवृतः}


\twolineshloka
{ससमासाद्य पौलस्त्यममात्यैरभिसंवृतम्}
{रामसंदेशमामन्त्र्य वाग्मी वक्तुं प्रचक्रमे}


\threelineshloka
{आह त्वां राघवो राजन्कोसलेन्द्रो महायशाः}
{प्राप्तकालमिदं वाक्यं तदादत्स्व सुदुर्मते}
{}


\twolineshloka
{अकृतात्मानमासाद्य राजानमनये रतम्}
{विनश्यन्त्यनयाविष्टा देशाश्च नगराणि च}


\twolineshloka
{त्वयैकेनापराद्धं मे सीतामाहरता बलात्}
{वधायानपराद्धानामन्येषां तद्भविष्यति}


\twolineshloka
{ये त्वया बलदर्पाभ्यामाविष्टेन वनेचराः}
{ऋषयोहिंसिताः पूर्वंदेवाश्चाप्यवमानिताः}


\twolineshloka
{राजर्षयश्च निहता रुदत्यश्चाहृताः स्त्रियः}
{तदिदं समनुप्राप्तं फलंतस्यानयस्य ते}


\twolineshloka
{हन्तास्मि त्वां सहामात्यैर्युध्यस्व पुरुषो भव}
{पश्य मे धनुषो वीर्यं मानुषस् निशाचर}


\twolineshloka
{मुच्यतां जानकी सीता न मे मोक्ष्यमसि कर्हिचित्}
{अराक्षसमिमं लोकंकर्ताऽस्मि निशितैः शरैः}


\twolineshloka
{इतितस् ब्रुवाणस् दूतस् परुषं वचः}
{श्रुत्वा न ममृषे राजा रावणः क्रोधमूर्च्छितः}


\twolineshloka
{हङ्गितज्ञास्ततो भर्तुश्चत्वारो रजनीचराः}
{चतुर्ष्वङ्गेषु जगृहुः शार्दूलमिव पक्षिणः}


\twolineshloka
{तांस्तथाङ्गेषु संसक्तानङ्गदो रजनीचरान्}
{आदायैव खमुत्पत्य प्रासादतलमाविशत्}


\twolineshloka
{वेगेनोत्पततस्तस्य पेतुस्ते रजनीचराः}
{भुवि संभिन्नहृदयाः प्रहारवरपीडिताः}


\twolineshloka
{संसक्तोहर्म्यशिखरात्तस्मात्पुनरवापतत्}
{लङ्घयित्वा पुरं लङ्कां सुवेलस्य समीपतः}


\twolineshloka
{कोसलेन्द्रमथागम्य सर्वमावेद्य वानरः}
{विशश्राम स तेजस्वी राघवेणाभिनन्दितः}


\twolineshloka
{ततः सर्वाभिसारेण हरीणां वातरंहसाम्}
{भेदयामास लङ्कायाः ग्राकारं रघुनन्दनः}


\twolineshloka
{विभीषणर्क्षाधिपती पुरस्कृत्याथ लक्ष्मणः}
{दक्षिणं नगरद्वारमवामृद्गाद्दुरासदम्}


\twolineshloka
{करभारुणगात्राणां हरीणां युद्धशालिनाम्}
{कोटीशतसहस्रेण लङ्कामभ्यपतत्तदा}


\twolineshloka
{प्रलम्बबाहूरुकरजङ्घान्तरविलम्बिनाम्}
{ऋक्षाणआं धूम्रवर्णानां तिस्रः कोठ्यो व्यवस्थिताः}


\twolineshloka
{उत्पतद्भिः पतद्भिश्च निपतद्भिश्च वानरैः}
{नादृश्यत तदा सूर्यो रजसा नाशितप्रभः}


\twolineshloka
{शालिप्रसूनसदृशैः शिरीपकुसुमप्रभैः}
{तरुणादित्यसदृशैः शणगौरैश्च वैनरैः}


\twolineshloka
{प्राकारं ददृशुस्ते तु समन्तात्कपिलीकृतम्}
{राक्षसा विस्मिता राजन्सस्त्रीवृद्धाः समन्ततः}


\twolineshloka
{बिभिदुस्ते मणिस्तम्भान्कर्णाट्टशिखराणि च}
{भग्नोन्मथितशृङ्गाणि यन्त्राणि च विचिक्षिपुः}


\twolineshloka
{परिगृह्य शतघ्नीश्च सचक्राः सगुडोपलाः}
{चिक्षिपुर्भुजवेगेन लङ्कामध्येमहास्वनाः}


\twolineshloka
{प्राकारस्थाश्चये केचिन्निशाचरगणास्तथा}
{प्रदुद्रुवुस्ते शतशः कपिभिः समभिद्रुताः}


\twolineshloka
{ततस्तु राजवचनाद्राक्षसाः कामरूपिणः}
{निर्ययुर्विकृताकाराः सहस्रशतसङ्घशः}


\twolineshloka
{शखवर्षाणि वर्षन्तो द्रावयित्वा वनौकसः}
{प्राकारं शोभयन्तस्ते परं विस्मयमास्थिताः}


\twolineshloka
{स मापराशिसदृशैर्बभूव क्षणादाचरैः}
{कृतो निर्वानरो भूयः प्राकारो भीमदर्शनैः}


\twolineshloka
{पेतुः शलविभिन्नाङ्गा बहवो वानरर्पभाः}
{स्तम्भतोरणभग्नाश्चपेतुस्तत्रनिशाचराः}


\twolineshloka
{केशाकेश्यभवद्युद्धं रक्षसां वानरैः सह}
{नखैर्दन्तैश्च वीराणां खादतां वै परस्परम्}


\twolineshloka
{निष्टनन्तो ह्युभयतस्तत्र वानरराक्षसाः}
{हतानिपतिता भूमौ न मुञ्चन्ति परस्परम्}


\twolineshloka
{रामस्तु शरजालानिववर्ष जलदो यथा}
{तानिलङ्कां समासाद्य जघ्रुस्तान्रजनीचरान्}


\twolineshloka
{सौमित्रिरपि नाराचैर्दृढधन्वा जितक्लमः}
{आदिश्यादिश्य दुर्गस्थान्पातयामास राक्षसान्}


\twolineshloka
{ततः प्रत्यवहारोऽभूत्सैन्यानां राधवाज्ञया}
{कृते विमर्दे लङ्कायां लब्धलक्ष्योजयोत्तरः}


\sect{अध्यायः २८६}
\twolineshloka
{मार्कण्डेय उवाच}
{}


\twolineshloka
{ततो निविशमानांस्तान्सैनिकान्रावणानुगाः}
{अभिजग्मुर्गणाऽनके पिशाचक्षुद्ररक्षसाम्}


\twolineshloka
{पर्वणः पतनो जम्भः खरः क्रोधवशो हरिः}
{प्ररुजश्चारुजश्चैव प्रघसश्चैवमादयः}


\twolineshloka
{ततोऽभिपततां तेषामदृश्यानां दुरात्मनाम्}
{अन्तर्धानवधं तज्ज्ञश्चकार स विभीषणः}


\twolineshloka
{ते दृश्यमाना हरिभिर्बलिभिर्दूरपातिभिः}
{निहताः सर्वशो राजन्महीं जग्मुर्गतासवः}


\twolineshloka
{अमृष्यमाणः सबलो रावणो निर्ययावथ}
{राक्षसानां बलैर्घोरैः पिशाचानांच संवृतः}


\twolineshloka
{युद्धशास्त्रविधानज्ञ उशना इव चापरः}
{व्यूह्यचौशनसं व्यूहं हरीनभ्यवहारयत्}


\twolineshloka
{राघवस्तु विनिर्यान्तं व्यूढानीकं दशाननम्}
{बार्हस्पत्यं विधं कृत्वा प्रतिव्यूह्य ह्यदृश्यत}


\twolineshloka
{समेत्य युयुधे तत्र ततो रामेण रावणः}
{युयुधे लक्ष्मणश्चापि तथैवेन्द्रजिता सह}


\twolineshloka
{विरूपाक्षेण सुग्रीवस्तारेण च निस्वर्वटः}
{पौण्ड्रेण च नलस्तत्र पदुशः पनसेन च}


\twolineshloka
{विषह्यं यं हि यो मेने स स तेन समेयिवान्}
{युयुधे युद्धवेलायां स्वबाहुबलमाश्रितः}


\twolineshloka
{स संप्रहारो ववृधे भीरूणां भयवर्धनः}
{रोमसंहर्षणो घोरः पुरा देवासुरे यथा}


\twolineshloka
{रावणो राममानर्च्छच्छक्तिशूलासिवृष्टिभिः}
{निशितैरायसैस्तीक्ष्णै रावणं चापि राघवः}


\twolineshloka
{तथैवेन्द्रजितं यत्तं लक्ष्मणो मर्मभेदिभिः}
{इन्द्रजिच्चापि सौमित्रिं बिभेद बहुभिः शरैः}


\twolineshloka
{विभीषणः प्रहस्तं च प्रहस्तश्च विभीषणम्}
{खगपत्रैः शरैस्तीक्ष्णैरभ्यवर्षद्गतव्यथः}


\twolineshloka
{तेषां बलवतामासीन्महास्त्राणां समागमः}
{विव्यथुः सकला येन त्रयो लोकाश्चराचराः}


\sect{अध्यायः २८७}
\twolineshloka
{मार्कण्डेय उवाच}
{}


\twolineshloka
{ततः प्रहस्तः सहसा समभ्येत्य विभीषणम्}
{गदया ताडयामास विनद्य रणकर्कशम्}


\twolineshloka
{स तयाऽभिहतो धीमान्गदया भीमवेगया}
{नाकम्पत महाबाहुर्हिमवानिव सुस्थिरः}


\twolineshloka
{ततः प्रगृह्यविपुलां शतघण्टां विभीषणः}
{अनुमन्त्र्य महाशक्तिं चिक्षेपास् शिरः प्रति}


\twolineshloka
{पतन्त्या स तया वेगाद्राक्षसोऽशनिवेगया}
{हृतोत्तामङ्गो ददृशे वातरुग्ण इव द्रुमः}


\twolineshloka
{तं दृष्ट्वा निहतं सङ्ख्ये प्रहस्तं क्षणदाचरम्}
{अभिदुद्राव धूम्राक्षो वेगेन महता कपीन्}


\twolineshloka
{तस् मेघोपमं सैन्यमापतद्भीमदर्शनम्}
{दृष्ट्वैव सहसा दीर्णा रणे वानरपुङ्गवाः}


\twolineshloka
{ततस्तान्सहसा दीर्णान्दृष्ट्वा वानरपुङ्गवान्}
{निर्ययौ कपिशार्दूलो हनूमान्मारुतात्मजः}


\twolineshloka
{तं दृष्ट्वाऽवस्थितं सङ्ख्ये हरयः पवनात्मजम्}
{महत्या त्वरया राजत्संन्यवर्तन्त सर्वशः}


\twolineshloka
{ततः शब्दो महानासीत्तुमुलो रोमहर्षणः}
{रामरावणसैन्यानामन्योन्यमभिधावताम्}


\twolineshloka
{तस्मिन्प्रवृत्ते संग्रामे घोरे रुधिरकर्दमे}
{क्षूम्राक्षः कपिसैन्यं तद्द्रावयामास पत्रिभिः}


\twolineshloka
{तं स रक्षोमहामात्रमापतन्तं सपत्नजित्}
{प्रतिजग्राह हनुमांस्तरसा पवनात्मजः}


\twolineshloka
{तयोर्युद्धमभूदधोरं हरिराक्षसवीरयोः}
{जीगीषतोर्युधाऽन्योन्यमिन्द्रप्रह्लादयोरिवं}


\twolineshloka
{कगदाभिः परिघैश्चैव राक्षसो जघ्निवान्कपिम्}
{कपिश्च जघ्निवान्रः सस्कन्धविटपैर्द्रुमैः}


\twolineshloka
{ततस्तमतिकोपेन साश्वं सरथसारथिम्}
{धूम्राक्षमवधीत्क्रुद्धो हनूमान्मारुतात्मजः}


\twolineshloka
{ततस्तं निहतं दृष्ट्वा धूम्राक्षं राक्षसोत्तमम्}
{हरयो जातविश्रम्भा जघ्नुरन्ये च सैनिकान्}


\twolineshloka
{ते वध्यमाना हरिभिर्बलिभिर्जितकाशिभिः}
{राक्षसा भग्नसंकल्पा लङ्कामभ्यपतन्भयात्}


\twolineshloka
{तेऽभिपत्य पुरं भग्ना हतशेषा निशाचराः}
{सर्वं राज्ञे यथावृत्तं रावणाय न्यवेदयन्}


\twolineshloka
{श्रुत्वा तु रावणस्तेभ्यः प्रहस्तं निहतं युधि}
{धूम्राक्षं च महेष्वासं ससैन्यं सहराक्षसैः}


\twolineshloka
{सुदीर्घमिव निःश्वस्य समुत्पत्य वरासनात्}
{उवाच कुम्भकर्णस्य कर्मकालोऽयमागतः}


\twolineshloka
{इत्येवमुक्त्वा विविधैर्वादित्रैः सुमहास्वनैः}
{शयानमतिनिद्रालुं कुम्भकर्णमबोधयत्}


\threelineshloka
{प्रबोध्य महता चैनं यत्नेनाऽऽगतसाध्वसः}
{स्वस्थमासीनमव्यग्रं विनिद्रं राक्षसाधिपः}
{ततोऽब्रवीद्दशग्रीवः कुम्भकर्णं महाबलम्}


\twolineshloka
{धन्योसि यस्य ते निद्रा कुम्भकर्णेयमीदृशी}
{य इदं दारुणं कालं न जानीषे महाभयम्}


\twolineshloka
{एष तीर्त्वाऽर्णवं रामः सेतुना हरिभिः सह}
{अवमत्येह नः सर्वान्करोति कदनं महत्}


\twolineshloka
{मया त्वपहृता भार्या सीता नामास्य जानकी}
{तां नेतुं स इहायातो बद्ध्वा सेतुं महार्णवे}


\twolineshloka
{तेन चैव प्रहस्तादिर्महान्नः स्वजनो हतः}
{तस्य नान्यो निहन्ताऽस्ति त्वामृतेशत्रुकर्शन}


\twolineshloka
{सदंशितोऽभिनिर्याहि त्वमद्य बलिनांवर}
{रामादीन्समरे सर्वाञ्जहि शत्रूनरिंदम}


\twolineshloka
{दूषणावरजौ चैव वज्रवेगप्रमाथिनौ}
{तौ त्वां बलेन महता सहितावनुयास्यतः}


\twolineshloka
{इत्युक्त्वा राक्षुसपतिः कुम्भकर्णं तरस्विनम्}
{संदिदेशेतिकर्तव्ये वज्रवेगप्रमाथिनौ}


\twolineshloka
{तथ्त्युक्त्वा युतौ वीरौ रावणं दूषाणानुजौ}
{कुम्भकर्णं पुरस्कृत्य तूर्णं निर्ययतुः पुरात्}


\sect{अध्यायः २८८}
\twolineshloka
{मार्कण्डेय उवाच}
{}


\twolineshloka
{ततो निर्याय स्वपुरात्कुम्भकर्णः सहानुगः}
{अपश्यत्कपिसैन्यं रतज्जितकाश्यग्रतः स्थितम्}


\twolineshloka
{स वीक्षमाणस्तत्सैन्यं रामदर्शनकाङ्क्षया}
{अपश्यच्चापि सौमित्रिं धनुष्पाणिं व्यवस्थितम्}


\twolineshloka
{तमभ्येत्याशु हरयः परिवब्रुः समन्ततः}
{शैलवृक्षायुधा नादानमुञ्चन्भीषणास्ततः}


\twolineshloka
{अभ्यघ्नंश्च महाकायैर्बहुभिर्जगतीरुहैः}
{करजैरतुदंश्चान्ये विहाय भयमुत्तमम्}


\twolineshloka
{बहुधा युध्यमानास्ते युद्धमार्गैः प्लवंगमाः}
{नानाप्रहरणैर्भीमै राक्षसेन्द्रमताडयन्}


\twolineshloka
{स ताड्यमानः प्रहसन्भक्षयामास वानरान्}
{बलं चण्डबलाख्यं च वज्रबाहुं च वानरम्}


\twolineshloka
{तद्दृष्ट्वा व्यथनं कर्म कुम्भकर्णस्य रक्षसः}
{उदक्रोशन्परित्रस्तास्तारप्रभृतयस्तदा}


\twolineshloka
{तानुच्चैः क्रोशतः सैन्याञ्श्रुत्वा स हरियूथपान्}
{अभिदुद्राव सुग्रीवः कुम्भकर्णमपेतभीः}


\twolineshloka
{ततो निपत्य वेगेन कुम्भकर्णं महामना}
{सालेन जघ्निवान्मूर्ध्निं बलेन कपिकुञ्जरः}


\twolineshloka
{स महात्मा महावेगः कुम्भकर्णस् मूर्धनि}
{बिभेद सालं सुग्रीवो न चैवाव्यथयत्कपिः}


\twolineshloka
{ततो विनद्यसहसा सालस्पर्शविबोधितः}
{दोर्भ्यामादाय सुग्रीवं कुम्भकर्णोऽहरद्बलात्}


\twolineshloka
{ह्रियमाणं तु सुग्रीवं कुम्भकर्णेन रक्षसा}
{अवेक्ष्याभ्यद्रवद्वीरः सौमित्रिर्मित्रनन्दनः}


\twolineshloka
{सोऽभिपत्य महर्वेगं रुक्मपुङ्खं महाशरम्}
{प्राहिणोत्कुम्भकर्णाय लक्ष्मणः परवीरहा}


\twolineshloka
{स तस्य देहावरणं भित्त्वा देहं च सायकः}
{जगाम दारयन्भूमिं रुधिरेण समुक्षितः}


\twolineshloka
{तथा स भिन्नहृदयः समुत्सृज्य कपीश्वरम्}
{वेगेन महताऽऽविष्टस्तिष्ठतिष्ठेति चाब्रवीत्}


\twolineshloka
{कुम्भकर्णो महेष्वासः प्रगृहीतशिलायुधः}
{अभिदुद्राव सौमित्रिमुद्यम्य महतीं शिलाम्}


\twolineshloka
{तस्याभिपततस्तूर्णं क्षुराभ्यामुच्छितौ करौ}
{चिच्छेद निशिताग्राभ्यां स बभूव चतुर्भुजः}


\twolineshloka
{तानप्यस् भुजान्सर्वान्प्रगृहीतशिलायुधान्}
{क्षुरैश्चिच्छेदलघ्वस्त्रं सौमित्रिः प्रतिदर्शयन्}


\twolineshloka
{स बभूवातिकायश्च बहुपादशिरोभुजः}
{तं ब्रह्मास्त्रेण सौमित्रिर्ददाराद्रिचयोपमम्}


\twolineshloka
{स पपात महावीर्यो दिव्यास्त्राभिहतो रणे}
{महाशनिविनिर्दग्धः पादपोऽङ्कुरवानिव}


\twolineshloka
{तं दृष्ट्वा वृत्रसंकाशं कुम्भकर्णं तरस्विनम्}
{गतासुं पतितं भूमौ राक्षसाः प्राद्रवन्भयात्}


\twolineshloka
{तथातान्द्रवतो योधान्दृष्ट्वा तौ दूषणानुजौ}
{अवस्थाप्याथ सौमित्रिं संक्रुद्धावभ्यधावताम्}


\twolineshloka
{तावाद्रवन्तौ संक्रुद्धौ वज्रवेगप्रमाथिनौ}
{अभिजग्राह सौमित्रिर्विनद्योभौ पतत्रिभिः}


\twolineshloka
{ततः सुतुमुलं युद्धमभवद्रोमहर्षणम्}
{दूषणानुजयोः पार्थ लक्ष्मणस् च धीमतः}


\twolineshloka
{महता शरवर्षेण राक्षसौ सोऽभ्यवर्पत}
{तं चापिवीरौ संक्रुद्धावुभौ तौ समवर्षताम्}


\twolineshloka
{मुहूर्तमेवमभवद्वज्रवेगप्रमाथिनोः}
{सौमित्रेश्च महाबाहोः संप्रहारः सुदारुणः}


\twolineshloka
{अथाद्रिशृङ्गमादाय हनुमान्मारुतात्मजः}
{अभिद्रुत्याददे प्राणान्वज्रवेगस्य रक्षसः}


\twolineshloka
{नीलश्च महता ग्राव्णा दूपणावरजं हरिः}
{प्रमाथिनमभिद्रुत्य प्रममाथ महाबलः}


\twolineshloka
{ततः प्रावर्तत पुनः संग्रामः कटुकोदयः}
{रामरावणसैन्यानामन्योन्यमभिधावताम्}


\twolineshloka
{शतसो नैर्ऋतान्वन्या जघ्नुर्वन्यांश्च नैर्ऋताः}
{नैर्ऋतास्तत्रवध्यन्ते प्रायेण न तु वानराः}


\sect{अध्यायः २८९}
\twolineshloka
{मार्कण्डेय उवाच}
{}


\twolineshloka
{ततः श्रुत्वाहतं सङ्ख्ये कुम्भकर्णं सहानुगम्}
{प्रहस्तं च महेष्वासं धूम्राक्षं चातितेजसम्}


\twolineshloka
{पुत्रमिनद्रजितं वीरं रावणः प्रत्यभाषत}
{जहिरामममित्रघ्न सुग्रीवं च सलक्ष्मणम्}


\twolineshloka
{त्वया हि मम सत्पुत्र यशो दीप्तमुपार्जितम्}
{जित्वावज्रधरं सङ्ख्ये सहस्राक्षं शचीपतिम्}


\twolineshloka
{अन्तर्हितः प्रकाशो वा दिव्यैर्दत्तवरैः शरैः}
{जहि शत्रूनमित्रघ्न मम शस्त्रभृतांवर}


\twolineshloka
{रामलक्ष्मणसुग्रीवाः शरस्पर्शं न तेऽनघ}
{समर्थाः प्रतिसोढुं च कुतस्तदनुयायिनः}


\twolineshloka
{अगता या प्रहस्तेन कुम्भकर्णेन चानघ}
{खरस्यापचितिः सङ्ख्ये तां गच्छ त्वे महाभुज}


\twolineshloka
{त्वमद्य निशितैर्बाणैर्हत्वा शत्रून्ससैनिकान्}
{प्रतिनन्दय मां पुत्र पुरा जित्वेव वासवम्}


\twolineshloka
{इत्युक्तः स तथेत्युक्त्वा रथमास्थाय दंशिथः}
{प्रययाविन्द्रजिद्राजंस्तूर्णमायोधनं प्रति}


\twolineshloka
{ततो विश्राव्य विस्पष्टं नाम राक्षसपुङ्गवः}
{आह्वयामास समरे लक्ष्मणं शुभलक्षणम्}


\twolineshloka
{तं लक्ष्मणोऽभ्यधावच्च प्रगृह्य सशरं धनुः}
{त्रासयंस्तलघोषेण सिंहः क्षुद्रमृगं यथा}


\twolineshloka
{तयोः समभवद्युद्धं सुमहज्जयगृद्धिनोः}
{दिव्यास्त्रविदुपोस्तीव्रमन्योन्यस्पर्धिनोस्तदा}


\twolineshloka
{रावणिस्तु यदा नैनं विशेषयति सायकैः}
{ततो गुरुतरं यत्नमातिष्ठद्बलिनां वरः}


\twolineshloka
{तत एवं महावेगैरर्दयामास तोमरैः}
{तानागतान्स चिच्छेद सौमित्रिर्निशितैः शरैः}


\twolineshloka
{ते निकृत्ताः शरैस्तीक्ष्णैर्न्यपतन्धरणीतले}
{साधका रावणेराजौ शतशः शकलीकृताः}


\twolineshloka
{तमङ्गदो वालिसुतः श्रीमानुद्यम्य पादपम्}
{अभिद्रुत्य महावेगस्ताडयामास मूर्धनि}


\twolineshloka
{तस्येन्द्रजिदसंभ्रान्तः प्रासेनोरसि वीर्यवान्}
{प्रहर्तुमैच्छत्तं चास्य प्रासं चिच्छेद लक्ष्मणः}


\twolineshloka
{तमभ्याशगतं वीरमङ्गदं रावणात्मजः}
{गदयाऽताडयत्सव्ये पार्श्वेवानरपुङ्गवम्}


\twolineshloka
{तमचिन्त्य प्रहारं स बलवान्वालिनः सुतः}
{ससर्जेन्द्रजितः क्रोधात्सालस्कन्धं तथाङ्गदः}


\twolineshloka
{सोऽङ्गदेन रुपोत्सृष्टो वधायेन्द्रजितस्तरुः}
{जघानेन्द्रजितः पार्थ रथं साश्वं ससारथिम्}


\twolineshloka
{ततो हताश्वात्प्रस्कन्द्य रथात्स हतसारथिः}
{तत्रैवान्तर्दधे राजन्मायया रावणात्मजः}


\twolineshloka
{अन्तर्हितं विदित्वा तं बहुमायं च राक्षसम्}
{रामस्तं देशमागम्य तत्सैन्यं पर्यरक्षत}


\twolineshloka
{स राममुद्दिश्य शरैस्ततो दत्तवरैस्तदा}
{विव्याध सर्वगात्रेषु लक्ष्मणं च महाबलम्}


\twolineshloka
{तमदृश्यंशरैः शूरौ माययाऽन्तर्हितं तदा}
{योधयामासतुरुभौ रावणिं रामलक्ष्मणौ}


\twolineshloka
{स रुषा सर्वगात्रेषु तयोः पुरुषसिंहयोः}
{व्यसृजत्सायकान्भूयः शतशोऽथ सहस्रशः}


\twolineshloka
{तमदृश्यं विचिन्वन्तः सृजन्तमनिशं शरान्}
{हरयो विविशुर्व्योम प्रगृह्य महतीः शिलाः}


\twolineshloka
{तांश्च तौ चाप्यदृश्यः सशरैर्विव्याध राक्षसः}
{स भृशं ताडयामास रावणिर्मायया वृतः}


\twolineshloka
{तौ शरैरर्दितौ वीरौ भ्रारौ रामलक्ष्मणौ}
{पेततुर्गगनाद्भूमिं सूर्याचन्द्रमसाविव}


\sect{अध्यायः २९०}
\twolineshloka
{मार्कण्डेय उवाच}
{}


\twolineshloka
{तावुभौ पतितौ दृष्ट्वा भ्रातरौ रामलक्ष्मणौ}
{बबन्ध रावणिर्भूयः शरैर्दत्तवरैस्तदा}


\twolineshloka
{तौ वीरौ शरजालेन बद्धाविन्द्रजिता रणे}
{रेजतुः पुरुषव्याघ्रौ शकुन्ताविव पञ्जरे}


\twolineshloka
{दृष्ट्वा निपतितौ भूमौ सर्वाङ्गेषु शराचितौ}
{सुग्रीवः कपिभिः सार्धं परिवार्योपतस्तिवान्}


\twolineshloka
{सुषेणमैन्दद्विविदैः कुमुदेनाङ्गदेन च}
{हनुमननीलतारैश्च नलेन च कपीश्वरः}


\twolineshloka
{ततस्तं देशमागम्य कृतकर्मा विभीषणः}
{बोधयामास तौ वीरौ प्रज्ञास्त्रेण प्रमोहितौ}


\twolineshloka
{विशल्यौ चापि सुग्रीवः क्षणेनैतौ चकार ह}
{विशल्यया महौषध्या दिव्यमन्त्रप्रयुक्तया}


\twolineshloka
{तौ लब्धसंज्ञौ नृवरौ विशल्यावुदतिष्ठताम्}
{उभौ गतक्लमौ चास्तां णेनैतौ महारथौ}


\twolineshloka
{ततो विभीषणः पार्थ राममिक्ष्वाकुनन्दनम्}
{उवाच विज्वरं दृष्ट्वा कृताञ्जलिरिदं वचः}


\twolineshloka
{अयमम्भो गृहीत्वातु राजराजस् शासनात्}
{गुह्कोऽभ्यागतः श्लेतात्त्वत्सकाशमरिंदम}


\twolineshloka
{इदमम्भः कुबेरस्ते महाराज प्रयच्छति}
{अन्तर्हितानां भूतानां दर्शनार्थं परंतप}


\twolineshloka
{अनेन मृष्टनयनो भूतान्यन्तर्हितान्युत}
{भवान्द्रक्ष्यति यस्मै च भवानेतत्प्रदास्यति}


\twolineshloka
{तथेति रामस्तद्वारि प्रतिगृह्याभिसंस्कृतम्}
{चकार नेत्रयोः शौचं लक्ष्मणश्च महामनाः}


\twolineshloka
{सुग्रीवजाम्बवन्तौ चहनुमानङ्गदस्तथा}
{मैन्दद्विविदनीलाश्च प्रायः प्लवगसत्तमाः}


\twolineshloka
{तथासमभवच्चापि यदुवाच विभीषणः}
{क्षणेनातीन्द्रियाण्येषां चक्षुंष्यासन्युधिष्ठिर}


\twolineshloka
{इन्द्रजित्कृतकर्मा तु पित्रे कर्म तदाऽऽत्मनः}
{निवेद्य पुनरागच्छत्त्वरयाऽऽजिशिरःप्रति}


\twolineshloka
{तमागतं तु संक्रुद्धं पुनरेव युयुत्सया}
{अभिदुद्राव सौमित्रिर्विभीषणमते स्थितः}


\twolineshloka
{अकृताह्निकमेवैनं जिघांसुर्जितकाशिनम्}
{शरैर्जघान संक्रुद्धः कृतसंज्ञोऽथ लक्ष्मणः}


\twolineshloka
{तयोः समभवद्युद्धं तदाऽन्योन्यं जीगीषतोः}
{अतीव चित्रमाश्चर्यं शक्रप्रह्लादयोरिव}


\twolineshloka
{अविध्यदिन्द्रजित्तीक्ष्णैः सौमित्रिं मर्मभेदिभिः}
{सौमित्रिश्चानलस्पर्शैरविध्यद्रावणिं शरैः}


\twolineshloka
{सौमित्रिशरसंस्पर्शाद्रावणिः क्रोधमूर्च्छितः}
{असृजल्लक्ष्मणायाष्टौ शरानाशीविषोपमान्}


\twolineshloka
{तस्येषून्पावकस्पर्शैः सौमित्रिः पत्रिभिस्त्रिभिः}
{वारयामास नाराचैः सौमित्रिर्मित्रनन्दनः}


\twolineshloka
{असृजल्लक्ष्मणश्चाष्टौ राक्षसाय शरान्पुनः}
{तथा तं न्यहनद्वीरस्तन्मे निगदतः शृणु}


\twolineshloka
{एकेनास्य धनुष्मन्तं बाहुं देहादपातयत्}
{द्वितीयेन तु बाणेन भुजमन्यमपातयत्}


\twolineshloka
{तृतीयेन तु बाणेन शितधारेण भास्वता}
{जहार सुनसं चापि शिरो ज्वलितकुण्डलम्}


\twolineshloka
{विनिकृत्तभुजस्कन्धः कबन्धाकृतिदर्शनः}
{पपात वसुधायां तु छिन्नमूल इवद्रुमः}


\twolineshloka
{तं हत्वासूतमप्यस्त्रैर्जघान बलिनंवरः}
{लङ्कां प्रवेशयामासुस्तं रथं वाजिनस्तदा}


\threelineshloka
{ददर्श रावणस्तं च रथं पुत्रविनाकृतम्}
{स पुत्रं निहतं श्रुत्वा त्रासात्संभ्रान्तमानसः}
{}


% Check verse!
रावणः शोकमोहार्तो वैदेहीं हन्तुमुद्यतः
ङ्गमादाय दुष्टात्मा जवेनाभिपपात ह


\twolineshloka
{तं दृष्ट्वातस्य दुर्बुद्देरविन्ध्यः पापनिश्चयम्}
{शमयामास संक्रुद्धं श्रूयतां येन हेतुना}


\twolineshloka
{महाराज्येस्थितो दीप्ते न स्त्रियं हन्तुमर्हसि}
{हतैवैषा यदा स्त्री च कबन्धनस्था च ते वशे}


\twolineshloka
{न चैषा दहभेदेन हतास्यादिति मे मतिः}
{जहि भर्तारमेवास्या हते तस्मिन्हता भवेत्}


\twolineshloka
{न हि ते विक्रमे तुल्यः साक्षादपि शतक्रतुः}
{असकृद्धि त्वया सन्द्रास्त्रासितास्त्रिदसा युधि}


\twolineshloka
{एवं बहुविधैर्वाक्यैरविन्ध्यो रावणं तदा}
{क्रुद्धं संशमयामास जगृहे च स तद्वचः}


\twolineshloka
{निर्याणे स मतिं कृत्वा नियन्तारं क्षपाचरः}
{आज्ञापयामास तदारथो मे कल्प्यतामिति}


\sect{अध्यायः २९१}
\twolineshloka
{मार्कण्डेय उवाच}
{}


\twolineshloka
{ततः क्रुद्धो दशग्रीवः प्रिये पुत्रे निपातिते}
{निर्ययौ रथमास्थाय हेमरत्नविभूषितम्}


\twolineshloka
{संवृतोराक्षसैर्घेरैर्विविधायुधपाणिभिः}
{अभिदुद्राव रामं स पोथयन्हरियूथपान्}


\twolineshloka
{तमाद्रवन्तं संक्रुद्ध मैन्दनीलनलाङ्गदाः}
{हनुमाञ्जाम्बवांश्चैव ससैन्याः पर्यवारयन्}


\twolineshloka
{ते दशग्रीवसैन्यं तदृक्षवानरपुङ्गवाः}
{द्रुमैर्विध्वंसयांचक्रुर्दशग्रीवस्य पश्यतः}


\twolineshloka
{ततः स्वसैन्यमालोक्य वध्यमानमरातिभिः}
{मायावी चासृजन्मायां रावणो राक्षसाधिपः}


\twolineshloka
{तस्य देहविनिष्क्रान्ताः शतशोऽथ सहस्रशः}
{राक्षसाः प्रत्यदृश्यन्त शरशक्त्यृष्टिपाणयः}


\twolineshloka
{तान्रामो जघ्निवान्सर्वान्दिव्येनास्त्रेण राक्षसान्}
{अथ भूयोपि मायां स व्यदधाद्राक्षसाधिपः}


\twolineshloka
{कृत्वा रामस् रूपाणि लक्ष्मणस्य च भारत}
{अभिदुद्राव रामं च लक्ष्मणं च दशाननः}


\twolineshloka
{ततस्ते राममर्च्छन्तो लक्ष्मणं च क्षपाचराः}
{अभिपेतुस्तदा रामं प्रगृहीतशरासनाः}


\twolineshloka
{तां दृष्ट्वाराक्षसेन्द्रस् मायामिक्ष्वाकुनन्दनः}
{उवाच रामः सौमित्रिमसंभ्रान्तो बृहद्वचः}


\twolineshloka
{जहीमान्राक्षसान्पापानात्मनः प्रतिरूपकान्}
{इत्युक्त्वाऽभ्यहनद्रामो लक्ष्मणश्चात्मरूपकान्}


\twolineshloka
{ततो हर्यश्वयुक्तेन रथेनादित्यवर्चसा}
{उपतस्थे रणे रामं मातलिः शक्रसारथिः}

\uvacha{मातलिरुवाच}


\twolineshloka
{अयं हर्यश्वयुग्जैत्रो मघोनः स्यन्दनोत्तमः}
{त्वदर्थमिह संप्राप्तः संदेशाद्वै शतक्रतोः}


\twolineshloka
{अनेन शक्रः काकुत्स्थ समरे दैत्यदानवान्}
{शतशः पुरुषव्याघ्र रथोदारेण जघ्निवान्}


\twolineshloka
{तदनन नरव्याघ्र मया यत्तेन संयुगे}
{स्यन्दनेन जहिक्षिप्रं रावणं मा चिरं कृथाः}


\twolineshloka
{इत्युक्तो राघवस्तथ्यं वचोऽशङ्कत मातलेः}
{मायैषाराक्षसस्येति तमुवाच विबीषणः}


\twolineshloka
{नेयं माया नरव्याघ्ररावणस् दुरात्मनः}
{तदातिष्ठ रथंशीघ्रमिमसैन्द्रं महाद्युते}


\twolineshloka
{ततः प्रहृष्टः काकुत्स्थस्तथेत्युक्त्वा विभीषणम्}
{रथेनाभिपपाताथ दशग्रीवं रुषाऽन्वितः}


\twolineshloka
{हाहाकुतानि भूतानि रावणे समभिद्रुते}
{सिंहनादाः सपटहादिति दिव्यास्तथाऽनदन्}


\twolineshloka
{[दशकन्धरराजसून्वोस्तथा युद्धमभून्महत्}
{अलब्धोपममन्यत्रतयोरेव तथाऽभवत्}


\twolineshloka
{सरामाय महाघोरं विससर्ज निशाचरः}
{शूलमिन्द्राशनिप्रख्यं ब्रह्मदण्डभिवोद्यतम्}


\twolineshloka
{तच्छूलं सत्वरं रामश्चच्छेद निशितैः शरैः}
{तद्दृष्ट्वा दुष्करं कर्म रावणं भयमाविशत्}


\twolineshloka
{ततः क्रुद्धः ससर्जाशु दशग्रीवः शिताञ्छरान्}
{सहस्रायुतशो रामे शस्त्राणि विविधानि च}


\twolineshloka
{ततो भुशुण्डीः शूलानि मुसलानि परश्वथान्}
{शक्तीश्च विविधाकाराः शतघ्नीश्च शितान्क्षुरान्}


\twolineshloka
{तां मायांविविधां दृष्ट्वा दशग्रीवस्य रक्षसः}
{भयात्प्रदुद्रुवुः सर्वे वानराः सर्वतोदिशम्}


\twolineshloka
{ततः सुपत्रं सुमुखंहेमपुङ्गं शरोत्तमम्}
{तूणादादाय काकुत्स्थो ब्रह्मास्त्रेण युयोज ह}


\twolineshloka
{तं प्रेक्ष्यबाणं रामेण ब्रह्मास्त्रेणानुमन्त्रितम्}
{जहृषुर्देवगन्धर्वा दृष्ट्वा शक्रपुरोगमाः}


\twolineshloka
{अल्पावशेषमायुश्च ततोऽमन्यन्त रक्षसः}
{ब्रह्मास्त्रोदीरणाच्छत्रोर्देवदानवकिंनराः}


\twolineshloka
{ततः ससर्ज तं रामः शरमप्रतिमौजसम्}
{रावणान्तकरं घोरं ब्रह्मदण्डमिवोद्यतम्}


\threelineshloka
{मुक्तमात्रेण रामेण दूराकृष्टेन भारत}
{स तेन राक्षसश्रेष्ठः सरथः साश्वसारथिः}
{प्रजज्वाल महाज्वालेनाग्निनाभिपरिप्लुतः}


\twolineshloka
{ततः प्रहृष्टास्त्रिदशाः सहगन्धर्वचारणाः}
{निहतं रावणं दृष्ट्वा रामेणाक्लिष्टकर्मणा}


\twolineshloka
{तत्यजुस्तं महाभागं पञ्चभूतानि रावणम्}
{भ्रंशितः सर्वलोकेषु स हि ब्रह्मास्त्रतेजसा}


\twolineshloka
{शरीरधातवो ह्यस् मासं रुधिरमेव च}
{नेशुर्ब्रह्मास्त्रनिर्दग्दा न च भस्माप्यदृश्यत}


\sect{अध्यायः २९२}
\twolineshloka
{मार्कण्डेय उवाच}
{}


\twolineshloka
{स हत्वा रावयणं क्षुद्रं राक्षसेनद्रं सुरद्विषम्}
{बभूव हृष्टः ससुहृद्रामः सौमित्रिणा सह}


\twolineshloka
{ततो हते दशग्रीवे देवाः सर्षिपुरोगमाः}
{आशीर्भिर्जययुक्ताभिरानर्चुस्तं महाभुजम्}


\twolineshloka
{रामं कमलपत्राक्षं तुष्टुवुः सर्वदेवताः}
{गन्धर्वाः पुष्पवर्षैश्च वाग्भिश्च त्रिदशालयाः}


\twolineshloka
{पूजयित्वा रणे रामं प्रतिजग्मुर्यथागतम्}
{तन्महोत्सवसंकाशमासीदाकाशमच्युत}


\twolineshloka
{ततो हत्वा दशग्रीवं लङ्कां रामो महायशाः}
{विभीषणाय प्रददौ प्रभुः परपुरंजयः}


\twolineshloka
{ततः सीतां पुरस्कृत्य विभीषणपुरस्कृताम्}
{अविन्ध्यो नाम सुप्रज्ञो वृद्धामात्यो विनिर्ययौ}


\twolineshloka
{उवाच च महात्मानं काकुत्स्थं दैन्यमास्थितम्}
{प्रतीच्छ देवीं सद्वृत्तां महात्मञ्जानकीमिति}


\twolineshloka
{एतच्छ्रुत्वा वचस्तस्मादवतीर्य रथोत्तमात्}
{बाष्पेणापिहितां सीतां ददर्शेक्ष्वाकुनन्दनः}


\twolineshloka
{तां दृष्ट्वा चारुसर्वाङ्गीं यानस्थां शोककर्शिताम्}
{मलोपचितसर्वाङ्गीं जटिलां कृष्णवाससम्}


\twolineshloka
{उवाच रामो वैदेहीं परामर्शविशङ्कितः}
{लक्षयित्वेङ्गितं सर्वं प्रियं तस्यै निवेद्य सः}


\threelineshloka
{गच्छ वैदेहि मुक्ता त्वं यत्कार्यं तनमया कृतम्}
{मामासाद्यपतिं भद्रे न त्वं राक्षसवेश्मनि}
{जरां व्रजेथा इतिमे निहतोसौ निशाचरः}


\twolineshloka
{कथं ह्यस्मद्विधो जातु जानन्धर्मविनिश्चयम्}
{परहस्तगतां नारीं मुहूर्तमपि धारयेत्}


\twolineshloka
{सुवृत्तामसुवृत्तां वाऽप्यहं त्वामद्य मैथिलि}
{नोत्सहे परिभोगाय श्वावलीढं हविर्यथा}


\twolineshloka
{ततः सा सहसा बाला तच्छ्रुत्वा दारुणं वचः}
{पपात देवी व्यथिता निकृत्ता कदली यथा}


\twolineshloka
{योप्यस्या हर्षसंभूतो मुखरागः पुराऽभवत्}
{क्षणेन सपुनर्नष्टो निःश्वासादिव दर्पणे}


\twolineshloka
{ततस्ते हरयः सर्वे तच्छ्रुत्वा रामभाषितम्}
{गतासुकल्पा निश्चेष्टा बभूवुः सहलक्ष्मणाः}


\twolineshloka
{ततो देवो विशुद्धात्मा विमानेन चतुर्मुखः}
{पद्मयोनिर्जगत्स्रष्टा दर्शयामास राघवम्}


\twolineshloka
{शक्रश्चाग्निश्च वायुश्चयमो वरुण एव च}
{यक्षाधिपश्च भगवांस्तथा सप्तर्षयोऽमलाः}


\twolineshloka
{राजा दशरथश्चैव दिव्यभास्वरमूर्तिमान्}
{विमानेन महार्हेण हंसयुक्तेन भास्वता}


\twolineshloka
{ततोऽन्तरिक्षं तत्सर्वंदेवगन्धर्वसंकुलम्}
{शुशुभे तारकाचित्रं शरदीव नभस्तलम्}


\twolineshloka
{तत उत्थाय वैदेही तेषां मध्ययशस्विनी}
{उवाच वाक्यं कल्याणी रामं पृथुलवक्षसम्}


\twolineshloka
{राजपुत्र न ते कोपं करोमि विदिताहि मे}
{गतिः स्त्रीणां नराणां च शृणु चदं वचो मम}


\twolineshloka
{अन्तश्चरतिभूतानां मातरिश्वा सदागतिः}
{स मे विमुञ्चतु प्राणान्यदि पापं चराम्यहम्}


\twolineshloka
{अग्निरापस्तथाऽऽकाशं पृथिवी वायुरेव च}
{विमुञ्चन्तु मम प्राणान्यदि पापं चराम्यहम्}


\twolineshloka
{यथाऽहं त्वदृतेवीर नान्यंस्वप्नेऽप्यचिन्तयम्}
{तथा मे देव निर्दिष्टस्त्वमेव हि पतिर्भव}


\twolineshloka
{ततोऽन्तरिक्षे वागारीत्सुभगा लोकसाक्षिणी}
{पुण्यासंहर्षणी तेषां वानराणां महात्मनाम्}

\uvacha{वायुरुवाच}


\twolineshloka
{बोभो राघव सत्यं वै वायुरस्मि सदागतिः}
{अपापा मैथिली राजन्संगच्छसहभार्यया}

\uvacha{अग्निरुवाच}


\twolineshloka
{अहमन्तःशरीरस्थो भूतानां रघुनन्दन}
{सुसूक्ष्ममपि काकुत्स्थ मैथिलीनापराध्यति}

\uvacha{वरुण उवाच}


\twolineshloka
{रसावै मत्प्रसूता हि भूतदेहेषु राघव}
{अहंवै त्वां प्रब्रवीमि मैथिली प्रतिगृह्यताम्}

\uvacha{यम उवाच}


\twolineshloka
{धर्मोऽहमस्मि काकुत्स्थ साक्षी लोकस्य कर्मणाम्}
{शुभाशुभानां सीतेयमपापा प्रतिगृह्यताम्}

\uvacha{ब्र्हमोवाच}


\twolineshloka
{पुत्र नैतदिहाश्चर्यं त्वयि राजर्षिधर्मणि}
{साधो सद्वृत्त काकुत्स्थ शृणु चेदं वचो मम}


\twolineshloka
{शत्रुरेष त्वया वीर देवगनधर्वभोगिनाम्}
{यक्षाणां दानवानां च महर्षीणां च पातितः}


\twolineshloka
{अवध्यः सर्वभूतानां मत्प्रसादात्पुराऽभवत्}
{कस्माच्चित्कारणात्पापः कंचित्कालमुपेक्षितः}


\twolineshloka
{वधार्थमात्मनस्तेन हृता सीता दुरात्मना}
{नलकूबरशापेन रक्षा चास्याः कृता मया}


\twolineshloka
{यदि ह्यकामामासेवेत्स्तरियमन्यामपि ध्रुवम्}
{शतधाऽस्य फलेन्मूर्धा इत्युक्तः सोभवत्पुरा}


\twolineshloka
{नात्रशङ्का त्वया कार्या प्रतीच्छेमां महामते}
{कृतं त्वया महत्कार्यं देवानाममितप्रभ}

\uvacha{दशरथ उवाच}


\twolineshloka
{प्रीतोस्मि वत्स भद्रं ते पिता दशरथोस्मि ते}
{अनुजानामि राज्यं च प्रशाधि पुरुषोत्तम}

\uvacha{राम उवाच}


\twolineshloka
{अभिवादयेत्वां राजेन्द्र यदि त्वं जनको मम}
{गमिष्यामि पुरीं रम्यामयोध्यां शासनात्तव}

\uvacha{मार्कण्डेय उवाच}


\threelineshloka
{तमुवाच पिता भूयः प्रहृष्टो भरतर्षभ}
{गच्छायोध्यां प्रशाधि त्वंराम रक्तान्तलोचन}
{संपूर्णानीहवर्षाणि चतुर्दश महाद्युते}


\twolineshloka
{ततो देवान्नमस्कृत्य मुहृद्भिरभिनन्दितः}
{महेन्द्रइव पौलोम्या भार्यया स समेयिवान्}


\twolineshloka
{ततो वरं ददौ तस्मै ह्यविन्ध्याय परंतपः}
{त्रिजटां चार्थमानाभ्यां योजयामास राक्षसीम्}


\twolineshloka
{तमुवाच ततो ब्रह्मा देवैः शक्रषुरोगमैः}
{कौसल्यामातरिष्टांस्ते वरानद्य ददानि कान्}


\twolineshloka
{वव्रेरामः स्थितिं धर्मे शत्रुभिश्चापराजयम्}
{राक्षसैर्निहतानां च वानराणां समुद्भवम्}


\twolineshloka
{ततस्ते ब्रह्मणा प्रोक्ते तथेतिवचने तदा}
{समुत्तस्थुर्महाराज वानरा लब्धचेतसः}


\twolineshloka
{सीता चापि महाभागा वरं हनुमते ददौ}
{रामकीर्त्या समं पुत्र जीवितं ते भविष्यति}


\twolineshloka
{दिव्यास्त्वामुपभोगाश्च मत्प्रसादकृताः सदा}
{उपस्थास्यन्ति हनुमन्निति स्म हरिलोचन}


\twolineshloka
{ततस्ते प्रेक्षमाणानां तेपामक्लिष्टकर्मणाम्}
{अन्तर्धानं ययुर्देवाः सर्वे शक्रपुरोगमाः}


\twolineshloka
{दृष्ट्वा रामं तु जानक्या संगतं शक्रसारथिः}
{उवाच परमप्रीतसुहृन्मध्य इदं वचः}


\twolineshloka
{देवगन्धर्वयक्षाणां मानुषासुरभोगिनाम्}
{अपनीतं त्वया दुःखमिदं सत्यपराक्रम}


\twolineshloka
{सदेवासुरगनधर्वा यक्षराक्षसपन्नगाः}
{कथयिष्यन्ति लोकास्त्वां यावद्भूमिर्धरिष्यति}


\twolineshloka
{इत्येवमुक्त्वाऽनुज्ञाप्यरामं शस्त्रभृतांवरम्}
{संपूज्यापाक्रमत्तेन रथेनादित्यवर्चसा}


\twolineshloka
{ततःसीतां पुरस्कृत्य रामः सौमित्रिणा सह}
{सुग्रीवप्रमुखैश्चैव सहितः सर्ववानरैः}


\twolineshloka
{विधाय रक्षां लङ्कायां विभीषणपुरस्कृतः}
{संततार पुनस्तेन सेतुना मकरालयम्}


\twolineshloka
{पुष्पकेण विसानन खेचरेण विराजता}
{कामगेन यथामुख्यैरमात्यैः संवृतो वसी}


\twolineshloka
{ततस्तीरे समुद्रस्यं यत्रशिश्य स पार्थिवः}
{तत्रैवोवास धर्मात्मा सहितः सर्ववानरैः}


\twolineshloka
{अथैनान्राघवः काले समानीयाभिपूज्य च}
{विसर्जयामास तदा रत्नैः संतोष्य सर्वशः}


\twolineshloka
{गतेषु वानरेन्द्रेषु गोपुच्छर्क्षेषु तेषु च}
{सुग्रीवसहितो रामः किष्किन्दां पुनरागमत्}


\twolineshloka
{विभीषणेनानुगतः सुग्रीवसहितस्तदा}
{पुष्पकेण विमानेन वैदेह्या दर्शयन्वनम्}


\twolineshloka
{किष्किन्धां तु समासाद्यरामः प्रहरतांवरः}
{अङ्गदं कृतकर्माणं यौवराज्येऽभ्यषेचयत्}


\twolineshloka
{ततस्तैरेव सहितो रामः सौमित्रिणा सह}
{यथागतेन मार्गेण प्रययौ स्वपुरं प्रति}


\twolineshloka
{अयोध्यां स समासाद्यपुरीं राष्ट्रपतिस्ततः}
{भरताय हनूमन्तं दूतं प्रास्थापयद्द्रुतम्}


\twolineshloka
{लक्षयित्वेङ्गितं सर्वंप्रियं तस्मै निवेद्य वै}
{वायुपुत्रे पुनः प्राप्ते नन्दिग्राममुपाविशत्}


\threelineshloka
{सतत्रमलदिग्धाङ्गं भरतं चीरवाससम्}
{नन्दिग्रामगतंरामः सशत्रुघ्नं सराघवः}
{अग्रतःपादुके कृत्वा ददर्शासीनमासने}


\twolineshloka
{समेत्यभरतेनाथ शत्रुघ्नेन च वीर्यवान्}
{राघवः सहसौमित्रिर्मुमुदे भरतर्षभ}


\twolineshloka
{ततो भरतशत्रुघ्नौ समेतौ गुरुणा तदा}
{वैदेह्या दर्शनेनोभौ प्रहर्षं समवापतुः}


\twolineshloka
{तस्मै तद्भरतो राज्यमागतायातिसत्कृतम्}
{न्यासं निर्यातयामास युक्तः परमया मुदा}


\twolineshloka
{ततस्तं वैष्णवे शूरं नक्षत्रेऽभिजितेऽहनि}
{वसिष्ठो वामदेवश्च सहितावभ्यषिञ्चताम्}


\twolineshloka
{सोभिषिक्तः कपिश्रेष्ठं सुग्रीवं ससुहृज्जनम्}
{विभीषणं च पौलस्त्यमन्वजानाद्गृहान्प्रति}


\twolineshloka
{अभ्यर्च्य विविधै रत्नैः प्रीतियुक्तौ मुदा युतौ}
{समाधायेतिकर्तव्यं दुःखेन विससर्ज ह}


\twolineshloka
{पुष्पकं च विमानं तत्पूजयित्वा स राघवः}
{प्रादाद्वैश्रवणायैव प्रीत्या स रघुनन्दनः}


\twolineshloka
{ततो देवर्षिसहितः सरितं गोमतीमनु}
{शताश्वमेधानाजह्रे जारूथ्यान्स निरर्गलान्}


\sect{अध्यायः २९३}
\twolineshloka
{मार्कण्डेय उवाच}
{}


\twolineshloka
{एवमेतन्महाबाहो रामेणामिततेजसा}
{प्राप्तं व्यसनमत्युग्रं वनवासकृतं पुरा}


\twolineshloka
{मा शुचः परुषव्याघ्र क्षत्रियोसि परंतप}
{बाहुवीर्याश्रयेमार्गे वर्तसे दीप्तनिर्णये}


\twolineshloka
{न हि ते वृजिनं किंचिद्दृश्यते परमण्वपि}
{अस्मिन्मार्गे निपीदेयुः सेन्द्रा अपि सुरासुराः}


\twolineshloka
{संहत्य निहतोवृत्रो मरुद्भिर्वज्रपाणिना}
{नमुचिश्चैवदुर्धर्षो दीर्गजिह्वा चराक्षसी}


\twolineshloka
{सहायवति सर्वार्थाः सतिष्ठन्तीह सर्वशः}
{किंनु तस्याजितं सङ्ख्ये यस् भ्राता धनंजयः}


\twolineshloka
{अयं च बलिनांश्रेष्ठो भीमो भीमपराक्रमाः}
{युवानौ च महेष्वासौ वीरौ माद्रवतीसुतौ}


\twolineshloka
{एभिः सहायैः कस्मात्त्वं विषीदसि परंतप}
{य इमे वज्रिणः सेनां जयेयुः समरुद्गणाम्}


\twolineshloka
{त्वमप्येभिर्महेष्वासैः सहायैर्देवरूपिभिः}
{विजेष्यसि रणे सर्वानमित्रान्भरतर्षभ}


\twolineshloka
{इतश्च त्वमिमां पश्यसैन्धवेन दुरात्मना}
{बलिना वीर्यमत्तेन हृतामेभिर्महात्मभिः}


\twolineshloka
{आनीतां द्रौपदीं कृष्णां कृत्वा कर्म सुदुष्करम्}
{जयद्रथं च राजानं विजितं वशमागतम्}


\twolineshloka
{असहायेन रामेण वैदेही पुनराहृता}
{हत्वासङ्ख्ये दशग्रीवं राक्षसं भीमविक्रमम्}


\twolineshloka
{यस् शाखामृगामित्राण्यृक्षाः कालमुखास्तथा}
{जात्यन्तरगता राजन्नेतद्बुद्ध्याऽनुचिन्तय}


\twolineshloka
{तस्मात्सर्वं कुरुश्रेष्ठ मा शुचो भरतर्षभ}
{त्वद्विधा हि महात्मानो न शोचन्ति परंतप}

\uvacha{वैशंपायन उवाच}


\twolineshloka
{एवमाश्वासितो राजामार्कण्डेयेन धीमता}
{त्यक्त्वा दुःखमदीनात्मा पुनरप्येनमब्रवीत्}


    \sect{हनूमता रामकथाकथनम्}

\src{श्रीमन्महाभारतम्}{वन-पर्व}{तीर्थयात्रापर्व}{अध्यायाः १४९--१५०}
\vakta{हनुमान्}
\shrota{भीमः}
\tags{concise, complete}
\notes{After an interesting encounter between Hanuman and Bhima, at Bhima's request, Hanuman narrates Ramayana to Bhima.}
% \textlink{http://stotrasamhita.net/wiki/Narayaniyam/Dashaka_34}
\translink{}

\storymeta

\dnsub{अध्यायः १४९}

\uvacha{हनूमानुवाच}

\addtocounter{shlokacount}{25}

\twolineshloka
{यत्ते मम परिज्ञाने कौतूहलमरिन्दम}
{तत्सर्वमखिलेन त्वं शृणु पाण्डवनन्दन}


\twolineshloka
{अहं केसरिणः क्षेत्रे वायुना जगदायुषा}
{जातः कमलपत्राक्ष हनूमान्नाम वानरः}


\twolineshloka
{सूर्यपुत्रं च सुग्रीवं शक्रपुत्रं च वालिनम्}
{सर्ववानरराजानौ सर्ववानरयूथपाः}


\twolineshloka
{उपतस्थुर्महावीर्या मम चामित्रकर्शन}
{सुग्रीवेणाभवत्प्रीतिरनिलस्याग्निना यथा}


\twolineshloka
{निकृतः स ततो भ्रात्रा कस्मिंश्चित्कारणान्तरे}
{ऋश्यमूके मया सार्धं सुग्रीवो न्यवसच्चिरम्}


\twolineshloka
{अथ दाशरथिर्वीरो रामो नाम महाबलः}
{विष्णुर्मानुषरूपेण चचार वसुधातलम्}


\twolineshloka
{स पितुः प्रियमन्विच्छन्सहभार्यः सहानुजः}
{सधनुर्धन्विनां श्रेष्ठो दण्डकारण्यमाश्रितः}


\twolineshloka
{तस्य भार्या जनस्थानाच्छलेनापहृता बलात्}
{राक्षसेन्द्रेण बलिना रावणेन दुरात्मना}


\twolineshloka
{सुवर्णरत्नचित्रेण मृगरूपेण रक्षसा}
{वञ्चयित्वा नरव्याघ्रं मारीचेन तदाऽनघ}

॥इति श्रीमन्महाभारते अरण्यपर्वणि तीर्थयात्रा-पर्वणि एकोनपञ्चाशदधिकशततमोऽध्यायः॥१४९॥


\dnsub{अध्यायः १५०}

\uvacha{हनूमानुवाच}

\twolineshloka
{हृतदारः सह भ्रात्रा पत्नीं मार्गन्स राघवः}
{दृष्टवाञ्शैलशिखरे सुग्रीवं वानरर्षभम्}


\twolineshloka
{तेन तस्याभवत्सख्यं राघवस्य महात्मनः}
{स हत्वा वालिनं राज्ये सुग्रीवं प्रत्यपादयत्}


\twolineshloka
{स राज्यं प्राप्य सुग्रीवः सीतायाः परिमार्गणे}
{वानरान्प्रेषयामास शतशोऽथ सहस्रशः}


\twolineshloka
{ततो वानरकोटीभिः सहितोऽहं नरर्षभ}
{सीतां मार्गन्महाबाहो प्रस्थितो दक्षिणां दिशम्}


\twolineshloka
{ततः प्रवृत्तिः सीताया गृध्रेण सुमहात्मना}
{सम्पातिना समाख्याता रावणस्य निवेशने}


\twolineshloka
{ततोऽहं कार्यसिद्ध्यर्थं रामस्याक्लिष्टकर्मणः}
{शतयोजनविस्तारमर्णवं सहसा प्लुतः}


\twolineshloka
{अहं स्ववीर्यादुत्तीर्य सागरं मकरालयम्}
{सुतां जनकराजस्य सीतां सुररसुतोपमाम्}


\twolineshloka
{दृष्टवान्भरतश्रेष्ठ रावणस्य निवेशने}
{समेत्य तामहं देवीं वैदेहीं राघवप्रियाम्}


\twolineshloka
{दग्ध्वा लङ्कामशेषेण साट्टप्राकारतोरणाम्}
{प्रत्यागतश्चास्य पुनर्नाम तत्र प्रकाश्य वै}


\threelineshloka
{मद्वाक्यं चावधार्याशु रामो राजीवलोचनः}
{अबद्धपूर्वमन्यैश्च बद्ध्वा सेतुं महोदधौ}
{वृतो वानरकोटीभिः समुत्तीर्णो महार्णवम्}


\twolineshloka
{ततो रामेण वीर्येण हत्वा तान्सर्वराक्षसान्}
{रणे तु राक्षसगणं रावणं लोकरावणम्}


\twolineshloka
{निशाचरेनद्रं हत्वा तु सभ्रातृसुतबान्धवम्}
{राज्येऽभिषिच्य लङ्कायां राक्षसेन्द्रं विभीषणम्}


\twolineshloka
{धार्मिकं भक्तिमन्तं च भक्तानुगतवत्सलः}
{प्रत्याहृत्य ततः सीतां नष्टां वेदश्रुतिं यथा}


\threelineshloka
{तयैव सहितः साध्व्या पत्न्या रामो महायशाः}
{गत्वा ततोऽतित्वरितः स्वां पुरीं रघुनन्दनः}
{अध्यावसत्ततोऽयोध्यामयोध्यां द्विषतां प्रभुः}


\twolineshloka
{ततः प्रतिष्ठितो राज्ये रामो नृपतिसत्तमः}
{वरं मया याचितोऽसौ रामो राजीवलोचनः}


\twolineshloka
{यावद्रामकथेयं ते भवेल्लोकेषु शत्रुहन्}
{तावज्जीवेयमित्येवं तथाऽस्त्विति च सोब्ऽरवीत्}


\twolineshloka
{सीताप्रसादाच्च सदा मामिहस्थमरिन्दम}
{उपतिष्ठन्ति दिव्या हि भोगा भीम यथेप्सिताः}


\twolineshloka
{दशवर्षसहस्राणि दशवर्षशतानि च}
{राज्यं कारितवान्रामस्ततः स्वभवनं गतः}


\twolineshloka
{तदिहाप्सरसस्तात गन्धर्वाश्च सदाऽनघ}
{तस्य वीरस्य चरितं गायन्त्यो रमयन्ति माम्}


\twolineshloka
{अयं च मार्गो मर्त्यानामगम्यः कुरुनन्दन}
{ततोऽहं रुद्धवान्मार्गं तवेमं देवसेवितम्}


\twolineshloka
{त्वामनेन पथा यान्तं यक्षो वा राक्षसोऽपि वा}
{धर्षयेद्वा शपेद्वाऽपि मा कश्चिदिति भारत}


\twolineshloka
{दिव्यो देवपथो ह्येष नात्र गच्छन्ति मानुषाः}
{यदर्थमागतश्चासि अत एव सरश्च तत्}

॥इति श्रीमन्महाभारते अरण्यपर्वणि तीर्थयात्रा-पर्वणि पञ्चाशदधिकशततमोऽध्यायः॥१५०॥

\closesection
    \input{rama-charitam/mahabharatam/narada-srinjaya-samvada}
    \input{rama-charitam/devi-bhagavatam/ramacharita-varnanam}
    \sect{लक्ष्मणकृतरामशोकसान्त्वनम्}

\src{देवी-भागवतम्}{तृतीयः स्कन्धः}{अध्यायः २८}{श्लोकाः १--५५}
\vakta{व्यासः}
\shrota{जनमेजयः}
\tags{concise, complete}
\notes{This chapter describes how Sita rejected Ravana's advances and was forcibly abducted to Lanka despite Jatayu's heroic attempt to stop him, followed by Rama's discovery of Her disappearance and His profound grief, being consoled by Lakshman's encouraging words about the cyclical nature of fortune and Their ability to rescue Her with the help of Their vānara allies.}
\textlink{https://sa.wikisource.org/wiki/देवीभागवतपुराणम्/स्कन्धः_०३/अध्यायः_२९}
\translink{https://www.wisdomlib.org/hinduism/book/devi-bhagavata-purana/d/doc57162.html}

\storymeta

\uvacha{व्यास उवाच}


\twolineshloka
{तदाकर्ण्य वचो दुष्टं जानकी भयविह्वला}
{वेपमाना स्थिरं कृत्वा मनो वाचमुवाच ह}% ॥ १ ॥

\twolineshloka
{पौलस्त्य किमसद्वाक्यं त्वमात्थ स्मरमोहितः}
{नाहं वै स्वैरिणी किन्तु जनकस्य कुलोद्‌भवा}% ॥ २ ॥

\twolineshloka
{गच्छ लङ्कां दशास्य त्वं राम त्वां वै हनिष्यति}
{मत्कृते मरणं तत्र भविष्यति न संशयः}% ॥ ३ ॥

\twolineshloka
{इत्युक्त्वा पर्णशालायां गता सा वह्निसन्निधौ}
{गच्छ गच्छेति वदती रावणं लोकरावणम्}% ॥ ४ ॥

\twolineshloka
{सोऽथ कृत्वा निजं रूपं जगामोटजमन्तिकम्}
{बलाज्जग्राह तां बालां रुदती भयविह्वलाम्}% ॥ ५ ॥

\twolineshloka
{रामरामेति क्रन्दन्ती लक्ष्मणेति मुहुर्मुहुः}
{गृहीत्वा निर्गतः पापो रथमारोप्य सत्वरः}% ॥ ६ ॥

\twolineshloka
{गच्छन्नरुणपुत्रेण मार्गे रुद्धो जटायुषा}
{सङ्ग्रामोऽभून्महारौद्रस्तयोस्तत्र वनान्तरे}% ॥ ७ ॥

\twolineshloka
{हत्वा तं तां गृहीत्वा च गतोऽसौ राक्षसाधिपः}
{लङ्कायां क्रन्दती तात कुररीव दुरात्मनः}% ॥ ८ ॥

\twolineshloka
{अशोकवनिकायां सा स्थापिता राक्षसीयुता}
{स्ववृत्तान्नैव चलिता सामदानादिभिः किल}% ॥ ९ ॥

\twolineshloka
{रामोऽपि तं मृगं हत्वा जगामादाय निर्वृतः}
{आयान्तं लक्ष्मणं वीक्ष्य किं कृतं तेऽनुजासमम्}% ॥ १० ॥

\twolineshloka
{एकाकिनीं प्रियां हित्वा किमर्थं त्वमिहागतः}
{श्रुत्वा स्वनं तु पापस्य राघवस्त्वब्रवीदिदम्}% ॥ ११ ॥

\twolineshloka
{सौ‌मित्रिस्त्वब्रवीद्वाक्यं सीतावाग्बाणपीडितः}
{प्रभोऽत्राहं समायातः कालयोगान्न संशयः}% ॥ १२ ॥

\twolineshloka
{तदा तौ पर्णशालायां गत्वा वीक्ष्यातिदुःखितौ}
{जानक्यन्वेषणे यत्‍नमुभौ कर्तुं समुद्यतौ}% ॥ १३ ॥

\twolineshloka
{मार्गमाणौ तु सम्प्राप्तौ यत्रासौ पतितः खगः}
{जटायुः प्राणशेषस्तु पतितः पृथिवीतले}% ॥ १४ ॥

\twolineshloka
{तेनोक्तं रावणेनाद्य हृता‍‍ऽसौ जनकात्मजा}
{मया निरुद्धः पापात्मा पातितोऽहं मृधे पुनः}% ॥ १५ ॥

\twolineshloka
{इत्युक्त्वाऽसौ गतप्राणः संस्कृतो राघवेण वै}
{कृत्वौर्घ्वदैहिकं रामलक्ष्मणौ निर्गतौ ततः}% ॥ १६ ॥

\twolineshloka
{कबन्धं घातयित्वासौ शापाच्चामोचयत्प्रभुः}
{वचनात्तस्य हरिणा सख्यं चक्रेऽथ राघवः}% ॥ १७ ॥

\twolineshloka
{हत्वा च वालिनं वीरं किष्किन्धाराज्यमुत्तमम्}
{सुग्रीवाय ददौ रामः कृतसख्याय कार्यतः}% ॥ १८ ॥

\twolineshloka
{तत्रैव वार्षिकान्मासांस्तस्थौ लक्ष्मणसंयुतः}
{चिन्तयञ्जानकीं चित्ते दशाननहृतां प्रियाम्}% ॥ १९ ॥

\twolineshloka
{लक्ष्मणं प्राह रामस्तु सीताविरहपीडितः}
{सौ‌मित्रे कैकयसुता जाता पूर्णमनोरथा}% ॥ २० ॥

\twolineshloka
{न प्राप्ता जानकी नूनं नाहं जीवामि तां विना}
{नागमिष्याम्ययोध्यायामृते जनकनन्दिनीम्}% ॥ २१ ॥

\twolineshloka
{गतं राज्यं वने वासो मृतस्तातो हृता प्रिया}
{पीडयन्मां स दुष्टात्मा दैवो‍ऽग्रे किं करिष्यति}% ॥ २२ ॥

\twolineshloka
{दुर्ज्ञेयं भवितव्यं हि प्राणिनां भरतानुज}
{आवयोः का गतिस्तात भविष्यति सुदुःखदा}% ॥ २३ ॥

\twolineshloka
{प्राप्य जन्म मनोर्वंशे राजपुत्रावुभौ किल}
{वनेऽतिदुःखभोक्तारौ जातौ पूर्वकृतेन च}% ॥ २४ ॥

\twolineshloka
{त्यक्त्वा त्वमपि भोगांस्तु मया सह विनिर्गतः}
{दैवयोगाच्च सौ‌मित्रे भुङ्क्ष्व दुःखं दुरत्ययम्}% ॥ २५ ॥

\twolineshloka
{न कोऽप्यस्मत्कुले पूर्वं मत्समो दुःखभाङ्नरः}
{अकिञ्चनोऽक्षमः क्लिष्टो न भूतो न भविष्यति}% ॥ २६ ॥

\twolineshloka
{किं करोम्यद्य सौ‌मित्रे मग्नोऽस्मि दुःखसागरे}
{न चास्ति तरणोपायो ह्यसहायस्य मे किल}% ॥ २७ ॥

\twolineshloka
{न वित्तं न बलं वीर त्वमेकः सहचारकः}
{कोपं कस्मिन्करोम्यद्य भोगेस्मिन्स्वकृतेऽनुज}% ॥ २८ ॥

\twolineshloka
{गतं हस्तगतं राज्यं क्षणादिन्द्रासनोपमम्}
{वने वासस्तु सम्प्राप्तः को वेद विधिनिर्मितम्}% ॥ २९ ॥

\twolineshloka
{बालभावाच्च वैदेही चलिता चावयोः सह}
{नीता दैवेन दुष्टेन श्यामा दुःखतरां दशाम्}% ॥ ३० ॥

\twolineshloka
{लङ्केशस्य गृहे श्यामा कथं दुःखं भविष्यति}
{पतिव्रता सुशीला च मयि प्रीतियुता भृशम्}% ॥ ३१ ॥

\twolineshloka
{न च लक्ष्मण वैदेही सा तस्य वशगा भवेत्}
{स्वैरिणीव वरारोहा कथं स्याज्जनकात्मजा}% ॥ ३२ ॥

\twolineshloka
{त्यजेत्प्राणान्नियन्तृत्वे मैथिली भरतानुज}
{न रावणस्य वशगा भवेदिति सुनिश्चितम्}% ॥ ३३ ॥

\twolineshloka
{मृता चेज्जानकी वीर प्राणांस्त्यक्ष्याम्यसंशयम्}
{मृता चेदसितापाङ्गीं किं मे देहेन लक्ष्मण}% ॥ ३४ ॥

\twolineshloka
{एवं विलपमानं तं रामं कमललोचनम्}
{लक्ष्मणः प्राह धर्मात्मा सान्त्वयन्नृतया गिरा}% ॥ ३५ ॥

\twolineshloka
{धैर्यं कुरु महाबाहो त्यक्त्वा कातरतामिह}
{आनयिष्यामि वैदेहीं हत्वा तं राक्षसाधमम्}% ॥ ३६ ॥

\twolineshloka
{आपदि सम्पदि तुल्या धैर्याद्‌भवन्ति ते धीराः}
{अल्पधियस्तु निमग्नाः कष्टे भवन्ति विभवेऽपि}% ॥ ३७ ॥

\twolineshloka
{संयोगो विप्रयोगश्च दैवाधीनावुभावपि}
{शोकस्तु कीदृशस्तत्र देहेनात्मनि च क्वचित्}% ॥ ३८ ॥

\twolineshloka
{राज्याद्यथा वने वासो वैदेह्या हरणं यथा}
{तथा काले समीचीने संयोगोऽपि भविष्यति}% ॥ ३९ ॥

\twolineshloka
{प्राप्तव्यं सुखदुःखानां भोगान्निर्वर्तनं क्वचित्}
{नान्यथा जानकीजाने तस्माच्छोकं त्यजाधुना}% ॥ ४० ॥

\twolineshloka
{वानराः सन्ति भूयांसो गमिष्यन्ति चतुर्दिशम्}
{शुद्धिं जनकनन्दिन्या आनयिष्यन्ति ते किल}% ॥ ४१ ॥

\twolineshloka
{ज्ञात्वा मार्गस्थितिं तत्र गत्वा कृत्वा पराक्रमम्}
{हत्वा तं पापकर्माणमानयिष्यामि मैथिलीम्}% ॥ ४२ ॥

\twolineshloka
{ससैन्यं भरतं वाऽपि समाहूय सहानुजम्}
{हनिष्यामो वयं शत्रुं किं शोचसि वृथाग्रज}% ॥ ४३ ॥

\twolineshloka
{रघुणैकरथेनैव जिताः सर्वा दिशः पुरा}
{तद्वंशजः कथं शोकं कर्तुमर्हसि राघव}% ॥ ४४ ॥

\twolineshloka
{एकोऽहं सकलाञ्जेतुं समर्थोऽस्मि सुरासुरान्}
{किं पुनः ससहायो वै रावणं कुलपांसनम्}% ॥ ४५ ॥

\twolineshloka
{जनकं वा समानीय साहाय्ये रघुनन्दन}
{हनिष्यामि दुराचारं रावणं सुरकण्टकम्}% ॥ ४६ ॥

\twolineshloka
{सुखस्यानन्तरं दुःखं दुःखस्यानन्तरं सुखम्}
{चक्रनेमिरिवैकं यन्न भवेद्‌रघुनन्दन}% ॥ ४७ ॥

\twolineshloka
{मनोऽतिकातरं यस्य सुखदुःखसमुद्‌भवे}
{स शोकसागरे मग्नो न सुखी स्यात्कदाचन}% ॥ ४८ ॥

\twolineshloka
{इन्द्रेण व्यसनं प्राप्तं पुरा वै रघुनन्दन}
{नहुषः स्थापितो देवैः सर्वैर्मघवतः पदे}% ॥ ४९ ॥

\twolineshloka
{स्थितः पङ्कजमध्ये च बहुवर्षगणानपि}
{अज्ञातवासं मघवा भीतस्त्यक्त्वा निजं पदम्}% ॥ ५० ॥

\twolineshloka
{पुनः प्राप्तं निजस्थानं काले विपरिवर्तिते}
{नहुषः पतितो भूमौ शापादजगराकृतिः}% ॥ ५१ ॥

\twolineshloka
{इन्द्राणीं कामयानस्तु ब्राह्मणानवमन्य च}
{अगस्तिकोपात्सञ्जातः सर्पदेहो महीपतिः}% ॥ ५२ ॥

\twolineshloka
{तस्माच्छोको न कर्तव्यो व्यसने सति राघव}
{उद्यमे चित्तमास्थाय स्थातव्यं वै विपश्चिता}% ॥ ५३ ॥

\twolineshloka
{सर्वज्ञोऽसि महाभाग समर्थोऽसि जगत्पते}
{किं प्राकृत इवात्यर्थं कुरुषे शोकमात्मनि}% ॥ ५४ ॥

\uvacha{व्यास उवाच}


\twolineshloka
{इति लक्ष्मणवाक्येन बोधितो रघुनन्दनः}
{त्यक्त्वा शोकं तथात्यर्थं बभूव विगतज्वरः}% ॥ ५५ ॥


॥इति श्रीदेवीभागवते महापुराणेऽष्टादशसाहस्र्यां संहितायां तृतीयस्कन्धे लक्ष्मणकृतरामशोकसान्त्वनं नामैकोनत्रिंशोऽध्यायः॥

    \input{rama-charitam/devi-bhagavatam/ramaya-devi-varadanam}
    \chapt{अग्नि-पुराणम्}

\src{अग्निपुराणम्}{अध्यायः ५}{}{श्लोकाः १--१४}
\vakta{}
\shrota{}
\notes{}
\textlink{https://sa.wikisource.org/wiki/अग्निपुराणम्/अध्यायः_५}
\translink{}

\storymeta

\sect{पञ्चमोऽध्यायः --- बाल-काण्ड-वर्णनम्}

\uvacha{अग्निरुवाच}
\twolineshloka
{रामायणमहं वक्ष्ये नारदेनोदितं पुरा}
{वाल्मीकये यथा तद्वत् पठितं भुक्तिमुक्तिदम्} %।। १ ।।

\uvacha{नारद उवाच}
\twolineshloka
{विष्णुनाभ्यव्जजो ब्रह्मा मरीचिर्ब्रह्मणः सुतः}
{मरीचेः कश्यपस्तस्मात् सूर्यो वैवस्वतो मनुः} %।। २ ।।

\twolineshloka
{ततस्तस्मात्तथेक्ष्वाकुस्तस्य वंशे ककुत्स्थकः}
{ककुत्स्थस्य रघुस्तस्मादजो दशरथस्ततः} %।। ३ ।।

\twolineshloka
{रावणादेर्वधार्थाय चतुर्द्धाभूत् स्वयं हरिः}
{राज्ञो दशरथाद्रामः कौशल्यायां बभूव ह} %।। ४ ।।

\twolineshloka
{कैकेय्यां भरतः पुत्रः सुमित्रायाञ्च लक्ष्मणः}
{शत्रुघ्नः ऋष्यश्रृङ्गेण तासु सन्दत्तपायसात्} %।। ५ ।।

\twolineshloka
{प्राशिताद्यज्ञसंसिद्धाद्रामाद्याश्च समाः पितुः}
{यज्ञविध्नविनाशाय विश्वामित्रार्थितो नृपः} %।। ६ ।।

\twolineshloka
{रामं सम्प्रेषयामास लक्ष्मणं मुनिना सह}
{रामो गतोऽस्त्रशस्त्राणि शिक्षितस्ताडकान्तकृत्} %।। ७ ।।

\twolineshloka
{मारीचं मानवास्त्रेण मोहितं दूरतोऽनयत्}
{सुबाहुं यज्ञहन्तारं सबलञ्चावधीद् बली} %।। ८ ।।

\twolineshloka
{सिद्धाश्रमनिवासी च विश्वामित्रादिभिः सह}
{गतः क्रतुं मैथिलस्य द्रष्टुञ्चापंसहानुजः} %।। ९ ।।

\twolineshloka
{शतानन्दनिमित्तेन विश्वामित्रप्रभावतः}
{रामाय कथितो राज्ञा समुनिः पूजितः क्रतौ} %।। १० ।।

\twolineshloka
{धनुरापूरयामास लीलया स बभञ्ज तत् }
{वीर्यशुल्कञ्च जनकः सीतां कन्यान्त्वयोनिजाम्} %।। ११ ।।

\twolineshloka
{ददौ रामाय रामोऽपि पित्रादौ हि समागते}
{उपयेमे जानकीन्तामुर्मिलां लक्ष्मणस्तथा} %।। १२ ।।

\twolineshloka
{श्रुतकीर्त्तिं माण्डवीञ्च कुशध्वजसुते तथा}
{जनकस्यानुजस्यैते शत्रुघ्नभरतावुभौ} %।। १३ ।।

\threelineshloka
{कन्ये द्वे उपयेमाते जनकेन सुपूजितः}
{रामोऽगात्सवशिष्ठाद्यैर्जामदग्न्यं विजित्य च}
{अयोध्यां भरतोभ्यागात् सशत्रुघ्नो युधाजितः} %।। १४ ।।

॥इत्यादिमहापुराणे आग्नेये रामायणे बालकाण्डवर्णनं नाम पञ्चमोऽध्यायः॥


\sect{षष्ठोऽध्यायः --- अयोध्या-काण्ड-वर्णनम्}

\uvacha{नारद उवाच}

\twolineshloka
{भरतेऽथ गते रामः पित्रादीनभ्यपूजयत्}
{राजा दशरथो राममुवाच शृणु राघव} % ०१

\twolineshloka
{गुणानुरागाद्राज्ये त्वं प्रजाभिरभिषेचितः}
{मनसाहं प्रभाते ते यौवराज्यं ददामि ह} % ०२

\twolineshloka
{रात्रौ त्वं सीतया सार्धं संयतः सुव्रतो भव}
{राज्ञश्च मन्त्रिणश्चाष्टौ सवसिष्ठास्तथाब्रुवन्} % ०३

\twolineshloka
{सृष्टिर्जयन्तो विजयः सिद्धार्थो राष्ट्रवर्धनः}
{अशोको धर्मपालश्च सुमन्त्रः सवसिष्ठकः} % ०४

\twolineshloka
{पित्रादिवचनं श्रुत्वा तथेत्युक्त्वा स राघवः}
{स्थितो देवार्चनं कृत्वा कौशल्यायै निवेद्य तत्} % ०५

\twolineshloka
{राजोवाच वसिष्ठादीन् रामराज्याभिषेचने}
{सम्भारान् सम्भवन्तु स्म इत्युक्त्वा कैकेयीङ्गतः} % ०६

\twolineshloka
{अयोध्यालङ्कृतिं दृष्ट्वा ज्ञात्वा रामाभिषेचनं}
{भविष्यतीत्याचचक्षे कैकेयीं मन्थरा सखी} % ०७

\twolineshloka
{पादौ गृहीत्वा रामेण कर्षिता सापराधतः}
{तेन वैरेण सा राम वनवासञ्च काङ्क्षति} % ०८

\twolineshloka
{कैकेयि त्वं समुत्तिष्ठ रामराज्याभिषेचनं}
{मरणं तव पुत्रस्य मम ते नात्र संशयः} % ०९

\twolineshloka
{कुब्जयोक्तञ्च तच्छ्रुत्वा एकमाभरणं ददौ}
{उवाच मे यथा रामस्तथा मे भरतः सुतः} % १०

\twolineshloka
{उपायन्तु न पश्यामि भरतो येन राज्यभाक्}
{कैकेयीमब्रवीत्क्रुद्धा हारं त्यक्त्वाथ मन्थरा} % ११

\twolineshloka
{बालिशे रक्ष भरतमात्मानं माञ्च राघवात्}
{भविता राघवो राजा राघवस्य ततः सुतः} % १२

\twolineshloka
{राजवंशस्तु कैकेयि भरतात्परिहास्यते}
{देवासुरे पुरा युद्धे शम्बरेण हताः सुराः} % १३

\twolineshloka
{रात्रौ भर्ता गतस्तत्र रक्षितो विद्यया त्वया}
{वरद्वयं तदा प्रादाद्याचेदानीं नृपं च तत्} % १४

\twolineshloka
{रामस्य च वने वासं नव वर्षाणि पञ्च च}
{यौवराज्यं च भरते तदिदानीं प्रदास्यति} % १५

\twolineshloka
{प्रोत्साहिता कुब्जया सा अनर्थे चार्थदर्शिनी}
{उवाच सदुपायं मे कच्चित्तं कारयिष्यति} % १६

\twolineshloka
{क्रोधागारं प्रविष्टाथ पतिता भुवि मूर्छिता}
{द्विजादीनर्चयित्वाऽथ राजा दशरथस्तदा} % १७

\twolineshloka
{ददर्श केकयीं रुष्टामुवाच कथमीदृशी}
{रोगार्ता किं भयोद्विग्ना किमिच्छसि करोमि तत्} % १८

\twolineshloka
{येन रामेण हि विना न जीवामि मुहूर्तकम्}
{शपामि तेन कुर्यां वै वाञ्छितं तव सुन्दरि} % १९

\twolineshloka
{सत्यं ब्रूहीति सोवाच नृपं मह्यं ददासि चेत्}
{वरद्वयं पूर्वदत्तं सत्यात्त्वं देहि मे नृप} % २०

\twolineshloka
{चतुर्दशसमा रामो वने वसतु संयतः}
{सम्भारैरेभिरद्यैव भरतोऽत्राभिषेच्यताम्} % २१

\twolineshloka
{विषं पीत्वा मरिष्यामि दास्यसि त्वं न चेन्नृप}
{तच्छ्रुत्वा मूर्छितो भूमौ वज्राहत इवापतत्} % २२

\twolineshloka
{मुहूर्ताच्चेतनां प्राप्य कैकेयीमिदमब्रवीत्}
{किं कृतं तव रामेण मया वा पापनिश्चये} % २३

\twolineshloka
{यन्मामेवं ब्रवीषि त्वं सर्वलोकाप्रियङ्करि}
{केवलं त्वत्प्रियं कृत्वा भविष्यामि सुनिन्दितः} % २४

\twolineshloka
{या त्वं भार्या कालरात्री भरतो नेदृशः सुतः}
{प्रशाधि विधवा राज्यं मृते मयि गते सुते} % २५

\twolineshloka
{सत्यपाशनिबद्धस्तु राममाहूय चाब्रवीत्}
{कैकेय्या वञ्चितो राम राज्यं कुरु निगृह्य माम्} % २६

\twolineshloka
{त्वया वने तु वस्तव्यं कैकेयीभरतो नृपः}
{पितरञ्चैव कैकेयीं नमस्कृत्य प्रदक्षिणं} % २७

\twolineshloka
{कृत्वा नत्वा च कौशल्यां समाश्वस्य सलक्ष्मणः}
{सीतया भार्यया सार्धं सरथः ससुमन्त्रकः} % २८

\twolineshloka
{दत्वा दानानि विप्रेभ्यो दीनानाथेभ्य एव सः}
{मातृभिश्चैव विप्राद्यैः शोकार्तैर्निर्गतः पुरात्} % २९

\twolineshloka
{उषित्वा तमसातीरे रात्रौ पौरान् विहाय च}
{प्रभाते तमपश्यन्तोऽयोध्यां ते पुनरागताः} % ३०

\twolineshloka
{रुदन् राजाऽपि कौशल्या गृहमागात्सुदुःखितः}
{पौरा जना स्त्रियः सर्वा रुरुदू राजयोषितः} % ३१

\twolineshloka
{रामो रथस्थश्चीराढ्यः शृङ्गवेरपुरं ययौ}
{गुहेन पूजितस्तत्र इङ्गुदीमूलमाश्रितः} % ३२

\twolineshloka
{लक्ष्मणः स गुहो रात्रौ चक्रतुर्जागरं हि तौ}
{सुमन्त्रं सरथं त्यक्त्वा प्रातर्नावाथ जाह्नवीम्} % ३३

\twolineshloka
{रामलक्ष्मणसीताश्च तीर्णा आपुः प्रयागकम्}
{भरद्वाजं नमस्कृत्य चित्रकूटं गिरिं ययुः} % ३४

\twolineshloka
{वास्तुपूजां ततः कृत्वा स्थिता मन्दाकिनीतटे}
{सीतायै दर्शयामास चित्रकूटं च राघवः} % ३५

\twolineshloka
{नखैर्विदारयन्तं तां काकं तच्चक्षुराक्षिपत्}
{ऐषिकास्त्रेण शरणं प्राप्तो देवान् विहायसः} % ३६

\twolineshloka
{रामे वनं गते राजा षष्ठेऽह्नि निशि चाब्रवीत्}
{कौशल्यां स कथां पौर्वां यदज्ञानाद्धतः पुरा} % ३७

\twolineshloka
{कौमारे सरयूतीरे यज्ञदत्तकुमारकः}
{शब्दभेदाच्च कुम्भेन शब्दं कुर्वंश्च तत्पिता} % ३८

\twolineshloka
{शशाप विलपन्मात्रा शोकं कृत्वा रुदन्मुहुः}
{पुत्रं विना मरिष्यावस्त्वं च शोकान्मरिष्यसि} % ३९

\twolineshloka
{पुत्रं विना स्मरन् शोकात्कौशल्ये मरणं मम}
{कथामुक्त्वाऽथ हा राममुक्त्वा राजा दिवङ्गतः} % ४०

\twolineshloka
{सुप्तं मत्त्वाऽथ कौशल्या सुप्ता शोकार्तमेव सा}
{सुप्रभाते गायनाश्च सूतमागधवन्दिनः} % ४१

\twolineshloka
{प्रबोधका बोधयन्ति न च बुध्यत्यसौ मृतः}
{कौशल्या तं मृतं ज्ञात्वा हा हताऽस्मीति चाब्रवीत्} % ४२

\twolineshloka
{नरा नार्योऽथ रुरुदुरानीतो भरतस्तदा}
{वशिष्ठाद्यैः सशत्रुघ्नः शीघ्रं राजगृहात्पुरीम्} % ४३

\twolineshloka
{दृष्ट्वा सशोकां कैकेयीं निन्दयामास दुःखितः}
{अकीर्तिः पातिता मूर्ध्नि कौशल्यां स प्रशस्य च} % ४४

\twolineshloka
{पितरं तैलद्रोणिस्थं संस्कृत्य सरयूतटे}
{वशिष्ठाद्यैर्जनैरुक्तो राज्यं कुर्विति सोऽब्रवीत्} % ४५

\twolineshloka
{व्रजामि राममानेतुं रामो राजा मतो बली}
{शृङ्गवेरं प्रयागं च भरद्वाजेन भोजितः} % ४६

\twolineshloka
{नमस्कृत्य भरद्वाजं रामं लक्ष्मणमागतः}
{पिता स्वर्गं गतो राम अयोध्यायां नृपो भव} % ४७

\twolineshloka
{अहं वनं प्रयास्यामि त्वदादेशप्रतीक्षकः}
{रामः श्रुत्वा जलं दत्वा गृहीत्वा पादुके व्रज} % ४८

\threelineshloka
{राज्यायाहन्नयास्यामि सत्याच्चीरजटाधरः}
{रामोक्तो भरतश्चायान्नन्दिग्रामे स्थितो बली}
{त्यक्त्वायोध्यां पादुके ते पूज्य राज्यमपालयत्} % ४९

॥इत्यादिमहापुराणे आग्नेये रामायणेऽयोध्याकाण्डवर्णनं नाम षष्ठोऽध्यायः॥


\sect{सप्तमोऽध्यायः --- अरण्य-काण्ड-वर्णनम्}

\uvacha{नारद उवाच}
\twolineshloka
{रामो वशिष्ठं मातॄश्च नत्वाऽत्रिञ्च प्रणम्य सः}
{अनसूयाञ्च तत्पत्नीं शरभङ्गं सुतीक्ष्णकम्}% ।। १ ।।

\twolineshloka
{अगस्त्य भ्रातरं नत्वा अगस्त्यन्तत्प्रसादतः}
{धनुः खङ्गञ्च सम्प्राप्य दण्डकारण्यमागतः}% ।। २ ।।

\twolineshloka
{जनस्थाने पञ्चवट्यां स्थितो गोदावरीं तटे}
{तत्र सूर्पणखायाता भक्षितुं तान् भयङ्करी}% ।। ३ ।।

\twolineshloka
{रामं सुरूपं दृष्ट्वा सा कामिनी वाक्यमब्रवीत्}
{कस्त्वं कस्मात्समायातो भर्त्ता मे भव चार्थितः}% ।। ४ ।।

\twolineshloka
{एतौ च भक्षयिष्यामि इत्युक्त्वा तं समुद्यता }
{तस्या नासाञ्च कर्णौ च रामोक्तो लक्ष्मणोऽच्छिनत्}% ।। ५।।

\twolineshloka
{रक्तं क्षरन्ती प्रययौ खरं भ्रातरमब्रवीत्}
{मरीष्यामि विनासाऽहं खर जीवामि वै तदा}% ।। ६ ।।

\twolineshloka
{रामस्य भार्य्या सीताऽसौ तस्यासील्लक्ष्मणोऽनुजः}
{तेषां यद्रुधिरं सोष्णं पाययिष्यसि मां यदि}% ।। ७ ।।

\twolineshloka
{खरस्तथेति तामुक्त्वा यतुर्दृशसहस्त्रकैः}
{रक्षसां दूषणेनागाद्योद्धु त्रिशिरसा सह}% ।। ८ ।।

\twolineshloka
{रामं रामोऽपि युयुधे शरैर्विव्याध राक्षसान्}
{हस्त्यश्वरथपादातं बलं निन्ये यमक्षयम्}% ।। ९ ।।

\twolineshloka
{त्रिशीर्षाणं खरं रौद्रं युध्यन्तञ्चौव दूषणम्}
{ययौ सूर्पणखा लङ्कां रावणाग्रेपतद् भुवि}% ।। १० ।।

\twolineshloka
{अब्रवीद्रावणं क्रुद्धा न त्वं राजा न रक्षकः}
{खरादिहन्तू रामस्य सीतां भार्यां हरस्व च}% ।। ११ ।।

\twolineshloka
{रामलक्ष्मणरक्तस्य पानाज्जीवामि नान्यथा}
{तथेत्याह च तच्छ्रुत्वा मारीचं प्राह वै व्रज}% ।। १२ ।।

\twolineshloka
{स्वर्णचित्रमृगो भूत्वा रामलक्ष्मणकर्षकः}
{सीताग्रे तां हरिष्यामि अन्यथा मरणं तव}% ।। १३ ।।

\twolineshloka
{मारीचो रावणं प्राह रामो मृत्युर्धनुर्धरः}
{रावणादपि मर्त्तव्यं मर्त्तव्यं राघवादपि}% ।। १४ ।।

\twolineshloka
{अवश्यं यदि मर्त्तव्यं वरं रामो न रावणः}
{इति मत्वा मृगो भूत्वा सीताग्रे व्यचरन्मुहुः}% ।। १५ ।।

\twolineshloka
{सीतया प्रेरितो रामः शरेणाथावधीच्च तम्}
{म्रियमाणो मृगः प्राह हा सीते लक्ष्मणेति च}% ।। १६ ।।

\twolineshloka
{सौमित्रिः सीतयोक्तोऽथ विरुद्धं राममागतः}
{रावणोऽप्यहरत् सीतां हत्वा गृध्रं जटायुषम्}% ।। १७ ।।

\twolineshloka
{जटायुषा स भिन्नाङ्गः अङ्केनादाय जानकीम्}
{गतो लङ्कामशोकाख्ये धारयामास चाब्रवीत्}% ।। १८ ।।

\twolineshloka
{भव भार्य्या ममाग्र्या त्वं राक्षस्यो रक्ष्यतामियम् }
{रामो हत्वा तु मारीचं दृष्ट्वा लक्ष्मणमब्रवीत्}% ।। १९ ।।

\twolineshloka
{मायामृगोऽसौ सौमित्रे यथा त्वमिह चागतः }
{तथा सीता हृता नूनं नापश्यत् स गतोऽथ ताम्}% ।। २० ।।

\twolineshloka
{शुशोच विललापार्त्तो मां त्यक्त्वा क्क गतासि वै}
{लक्ष्मणाश्वासितो रामो मार्गयामास जानकीम्}% ।। २१ ।।

\threelineshloka
{दृष्ट्वा जटायुस्तं प्राह रावणो हृतवांश्च ताम्}
{मृतोऽथ संस्कृतस्तेन कबन्धञ्चावधीत्ततः}
{शापमुक्तोऽब्रवीद्रामं स त्वं सुग्रीवमाव्रज} %।। २२ ।।

॥इत्यादिमहापुराणे अग्नेये रामायणे अरण्यकाण्डवर्णनं नाम सप्तमोऽध्यायः॥


\sect{अष्टमोऽध्यायः --- किष्किन्धा-काण्ड-वर्णनम्}

\uvacha{नारद उवाच}


\twolineshloka
{रामः पम्पासरो गत्वा शोचन् स शर्वरीं ततः}
{हनूमताऽथ सुग्रीवं नीतो मित्रं चकार ह}% ।। १ ।।

\twolineshloka
{सप्त तालन् विनिर्भिद्य शरेणैकेन पश्यतः}
{पादेन दुन्दुभेः कायञ्चिक्षेप दशयोजनम्}% ।। २ ।।

\twolineshloka
{तद्रिपुं बालिनं हत्वा भ्रातरं वैरसारिणम्}
{किष्किन्धां कपिरज्यञ्च रुमान्तारां समर्पयत्}% ।। ३ ।।

\twolineshloka
{ऋष्यमूकेहरीशायकिष्किन्धेशोऽब्रवीत्सच }
{सीतां त्वं प्राश्यसेयद्वत् तथा राम करोमिते}% ।। ४ ।।

\twolineshloka
{तछ्रुत्वा माल्यवत्पृष्ठे चातुर्मास्यं चकारसः}
{किष्किन्धायाञ्च सुग्रीवो यदा नायाति दर्शनम्}% ।। ५ ।।

\twolineshloka
{तदाऽब्रवीत्तं रामोक्तं लक्ष्मणो व्रज राघवम्}
{न स सङ्कुचितः पन्था येन बाली हतो गतः}% ।। ६ ।।

\twolineshloka
{समये तिष्ठ सुग्रीव मा बालिपथमन्वगः}
{सुग्रीव आह संसक्तो गतं कालं न बुद्धवान्}% ।। ७ ।।

\twolineshloka
{इत्युक्त्वा स गतो रामं नत्वोवाच हरीश्वरः}
{आनीता वानराः सर्वे सीतायाश्च गवेषणे}% ।। ८ ।।

\twolineshloka
{त्वन्मतात् प्रेषयिष्यामि विचिन्वन्तु च जानकीम् }
{पूर्वादौ मासमायान्तु मासादूर्ध्वं निहन्मि तान्}% ।। ९ ।।

\twolineshloka
{इत्युक्ता वानराः पूर्वपश्चमोत्तरमार्गगाः}
{जग्मू रामं ससुग्रीवमपश्यन्तस्तु जानकीम्}% ।। १० ।।

\twolineshloka
{रामाङ्गुलीयं सङ्गृह्य हनूमान् वानरैः सह}
{दक्षिणे मागयामास सुप्रभाया गुहान्तिके}% ।। ११ ।।

\twolineshloka
{मासादूर्ध्वञ्च विन्यस्ता अपश्यन्तस्तु जानकीम्}
{ऊचुर्वृथामरिष्यामो जटायुर्द्धन्य एव सः}% ।। १२ ।।

\twolineshloka
{सीतार्थे योऽत्यजत् प्राणान्रावणेन हतो रणे}
{तच्छ्रु त्वा प्राह सम्पातिर्विहाय कपिभक्षणम्}% ।। १३ ।।

\twolineshloka
{भ्राताऽसौ मे जटायुर्वै मयोड्डीनोऽर्कमण्डलम्}
{अर्क तापाद्रक्षितोऽगाद् दग्धपक्षोऽहमभ्रगः}% ।। १४ ।।

\twolineshloka
{रामवार्त्ताश्रवात् पक्षौ जातौ भूयोऽथ जानकीम्}
{पश्याम्यशोकवनिकागतां लङ्कागतां किल}% ।। १५ ।।

\twolineshloka
{शतयोजनविस्तीर्णे लवणाब्धौ त्रिकूटके}
{ज्ञात्वा रामं ससुग्रीवं वानराः कथयन्तु वै}% ।। १६ ।।

॥इत्यादिमहापुराणे आग्नेये रामायणे किष्किन्धाकाण्डर्णनं नाम अष्टमोऽध्यायः॥

\sect{नवमोऽध्यायः --- सुन्दरकाण्ड-वर्णनम्}

\uvacha{नारद उवाच}
\twolineshloka
{सम्पातिवचनं श्रुत्वा हनुमानङ्गदादयः}
{अब्धिं दृष्ट्वाऽब्रुवंस्तेऽब्धिं लङ्घयेत् को नु जीवयेत्}% ।। १ ।।

\twolineshloka
{कपीनां जीवनार्थाय रामकार्य्यप्रसिद्धये}
{शतयोजनविस्तीर्णं पुप्लुवेऽब्धिं स मारुतिः}% ।। २ ।।

\twolineshloka
{दृष्ट्वोत्थितञ्च मैनाकं सिंहिकां विनिपात्य च }
{लङ्कां दृष्ट्वा राक्षसानां गृहाणि वनितागृहे}% ।। ३ ।।

\twolineshloka
{दशग्रीवस्य कुम्भस्य कुम्भकर्णस्य रक्षसः}
{विभीषणस्येन्द्रजितो गृहेऽन्येषां च रक्षसाम्}% ।। ४ ।।

\twolineshloka
{नापश्यत् पानभूम्यादौ सीतां चिन्तापरायणः}
{अशोकवनिकां गत्वा दृष्टवाञ्छिंशपातले}% ।। ५ ।।

\twolineshloka
{राक्षसीरक्षितां सीतां भव भार्येति वादिनम्}
{रावणं शिशपास्थोऽथ नेति सीतान्तु वादिनीम्}% ।। ६ ।।

\twolineshloka
{भव भार्या रावणस्य राक्षसीर्वादिनीः कपिः}
{गते तु रावणे प्राह राजा दशरथोऽभवत्}% ।। ७ ।।

\twolineshloka
{रामोऽस्य लक्ष्ममः पुत्रौ वनवासङ्गतौ वरौ}
{रामपत्नी जानकी त्वं रावणेन हृता बलात्}% ।। ८ ।।

\twolineshloka
{रामः सुग्रीवमित्रस्त्वा मार्गयन् प्रैषयच्च माम् }
{साभिज्ञानं चागुलीयं रामदत्तं गृहाण वै}% ।। ९ ।।

\twolineshloka
{सीताऽङ्गुलीयं जग्रह साऽपश्यन्मारुतिन्तरौ}
{भूयोऽग्रे चोपविष्टं तमुवाच यदि जीवति}% ।। १० ।।

\twolineshloka
{रामः कथं न नयति शङ्कितामब्रवीत् कपिः}
{रामः सीते न जानीते ज्ञात्वा त्वां स नयिष्यति}% ।। ११ ।।

\twolineshloka
{रावणं राक्षसं हत्वा सबलं देविमाशुच}
{साभिज्ञानं देहि मे त्वं मणिं सीताऽददत्कपौ}% ।। १२ ।।

\twolineshloka
{उवाच मां यथा रामो नयेच्छीघ्रं तथा कुरु}
{काकाक्षिपातनकथाम्प्रतियाहि हि शोकह}% ।। १३।।

\twolineshloka
{मणिं कथां गृहीत्वाह हनूमान्नेष्यते पतिः }
{अथवा ते त्वारा काचित् पृष्ठमारुह मे शुभे}% ।। १४ ।।

\twolineshloka
{अद्य त्वां दर्शयिष्यामि ससुग्रीवञ्च राघवम् }
{सीताऽब्रवीद्धनूमन्तं नयतां मां हि राघवः}% ।। १५ ।।

\twolineshloka
{हनूमान् स दशग्रीवदर्शनोपायमाकरोत्}
{वनं बभञ्च तत्पालान् हत्वा दन्तनखादिभिः}% ।। १६ ।।

\twolineshloka
{हत्वा तु किङ्करान् सर्वान् सप्त मन्त्रिसुतानपि}
{पुत्रमक्षं कुमारञ्च शक्रजिच्चबबन्ध तम्}% ।। १७ ।।

\twolineshloka
{नागपाशेन पिङ्गाक्षं दर्शयामास रावणम्}
{उवाच रावणः कस्त्वं मारुतिः प्राह रावणम्}% ।। १८ ।।

\twolineshloka
{रामदूतो राघवाय सीतां देहि मरिष्यसि}
{रामबाणैर्हतः सार्धं लङ्कास्थै राक्षसैर्ध्रुवम्}% ।। १९ ।।

\twolineshloka
{रावणो हन्तुमुद्युक्तो विभीषणनिवारितः}
{दीपयामास लाङ्गूलं दीप्तपुच्छः स मारुतिः}% ।। २० ।।

\twolineshloka
{दग्ध्वा लङ्कां राक्षसाश्च दृष्ट्वा सीतां प्रणम्य ताम्}
{समुद्रपारमागम्य दृष्ट्वा सीतेति चाब्रवीत्}% ।। २१ ।।

\twolineshloka
{अङ्गदादीनङ्गदाद्यैः पीत्वा मधुवने मधु}
{जित्वा दधिमुखादींश्च दृष्ट्वा रामं च तेऽब्रुवन्}% ।। २२ ।।

\twolineshloka
{दृष्टा सीतेति रामोऽपि हृष्टः पप्रच्छ मारुतिम्}
{कथं दृष्टा त्वया सीता किमुवाच च मां प्रति}% ।। २३ ।।

\twolineshloka
{सीताकथामृतेनैव सिञ्च मां कामवह्निगम्}
{हनूमानब्रवीद्रामं लङ्घयित्वाऽब्धिमागतः}% ।। २४ ।।

\twolineshloka
{सीतां दृष्ठ्वा पुरीं दग्ध्वा सीतामणिं गृहाण वै}
{हत्वा त्वं रावणं सीतां प्रास्यसे राम मा शुचः}% ।। २५ ।।

\twolineshloka
{गृहीत्वा तं मणिं रामो रुरोद विरहातुरः }
{मणिं दृष्ट्वा जानकी मे दृष्टा सीता नयस्व माम्}% ।। २६ ।।

\twolineshloka
{तथा विना न जीवामि सुग्रीवाद्यैः प्रबोधितः}
{समुद्रतीरं गतवान् तत्र रामं विभीषणः}% ।। २७ ।।

\twolineshloka
{गतस्तिरस्कृतो भ्रात्रा रावणेन दुरात्मना}
{रामाय देहि सीतां त्वमित्युक्तेनासहायवान्}% ।। २८ ।।

\twolineshloka
{रामो विभीषणं मित्रं लङ्कैवर्येऽभ्यषेचयत्}
{समुद्रं प्रार्थयन्मार्गं यदा नायात्तदा शरैः}% ।। २९ ।।

\twolineshloka
{भेदयामास रामञ्च उवाचाब्धि समागतः}
{नलेन सेतुं बद्‌ध्वाब्धौ लङ्कां व्रज गभीरकः}% ।। ३० ।।

\threelineshloka
{अहं त्वया कृतः पूर्वं रामोऽपि नलसेतुना}
{कृतेन तरुशैलाद्यैर्गतः पारं महोदधेः}
{वानरैः स सुवेलस्थः सह लङ्कां ददर्श वै} %।। ३१ ।।

॥इत्यादिमहापुराणे आग्नये रामायणे सुन्दरकाण्डवर्णनं नाम नवमोऽध्यायः॥

\sect{अष्टमोऽध्यायः --- युद्ध-काण्ड-वर्णनम्}

\uvacha{नारद उवाच}
\twolineshloka
{रामोक्तञ्चाङ्गदौ गत्वा रावणं प्राह जानकी}
{दीयतां राघवायाशु अन्यथा त्वं मरिप्यसि}% ।। १ ।।

\twolineshloka
{रावणो हन्तुमुद्युक्तः सङ्ग्रामोद्धतराक्षसः}
{रामयाह दशग्रीवो युद्धमेकं तु मन्यते}% ।। २ ।।

\twolineshloka
{रामो युद्धाय तच्छ्रुत्वा लङ्कां सकपिराययौ}
{वानरो हनुमान् मैन्दो द्विविदौ जाम्बवान्नलः}% ।। ३ ।।

\twolineshloka
{नीलस्तारोङ्गदो धूभ्रः सुषेणः केशरी गयः}
{पनसो विनतो रम्भः शरभः कथनो बली}% ।। ४ ।।

\twolineshloka
{गवाक्षो दधिवक्त्रश्च गवयो गन्धमादनः}
{एते चान्ये च सुग्रीव एतैर्युक्तो ह्यसङ्ख्यकैः}% ।। ५ ।।

\twolineshloka
{रक्षसां वानराणाञ्च युद्धं सङ्कुलमाबभौ}
{राक्षसा वानराञ्जघ्नुः शरशक्तिगदादिभिः}% ।। ६ ।।

\twolineshloka
{वानरा राक्षसाञ् जघ्नुर्नखदन्तशिलादिभिः}
{हस्त्थश्वरथपादातं राक्षसानां बलं हतम्} %।।७ ।।

\twolineshloka
{हनूमान् गिरिऋङ्गेण धूम्राक्षमवधीद्रिपुम्}
{अकम्पनं प्रहस्तञ्च युध्यन्तं नील आवधीत्}% ।। ८ ।।

\twolineshloka
{इन्द्रजिच्छरबन्धाच्च विमुक्तौ रामलक्ष्मणौ}
{तार्क्ष्यसन्दर्शनाद् बाणैर्जघ्ननू राक्षसं बलम्}% ।। ९ ।।

\twolineshloka
{रामः शरैर्जर्जरितं रावणञ्चाकरोद्रणे}
{रावणः कुम्बकर्णञ्च बौधयामास दुः खितः}% ।। १० ।।

\twolineshloka
{कुम्भकर्णः प्रबुद्धोऽथ पीत्वा घटसहस्त्रकम्}
{मद्यस्य महिषादीनां भक्षयित्वाह रावणम्}% ।। ११ ।।

\twolineshloka
{सीताया हरणं पापं कृतन्त्वं हि गुरुर्यतः}
{अतो गच्छामि युद्धाय रामं हन्मि सवानरम्}% ।। १२ ।।

\twolineshloka
{इत्युक्त्वा वानरान् सर्वान् कुम्भकर्णो ममर्द्द ह}
{गृहीतस्तेन सुग्रीवः कर्णनासं चकर्त्त सः}% ।। १३ ।।

\twolineshloka
{कर्णनासाविहीनोऽसौ भक्षयामास वानरान्}
{अथ कुम्भो निकुम्भश्च मकराक्षश्च राक्षसः}% ।। १४ ।।

\twolineshloka
{ततः पादौ ततश्छित्त्वा शिरो भूमौ व्यपातयत् }
{अथ कुम्भो निकुम्भश्च मकराक्षश्च राक्षसः}% ।। १५ ।।

\twolineshloka
{महोदरो महापार्श्वो मत्त उन्तत्तराक्षसः}
{प्रघसो भासकर्णश्च विरूपाक्षस्छ संयुगे}% ।। १६ ।।

\twolineshloka
{देवान्तको नरान्तश्च त्रिशिराश्चातिकायकः}
{रामेण लक्ष्मणेनैते वानरैः सविभीषणैः}% ।। १७ ।।

\twolineshloka
{युध्यमानास्तथाह्यन्ये राक्षसाभुवि पातिताः}
{इन्द्रजिन्मायया युध्यन् रामादीन् सम्बबन्ध ह}% ।। १८ ।।

\twolineshloka
{वरदत्तैर्नागबाणै रोषध्या तौ विशल्यकौ}
{विशल्ययाव्रणौ कृत्वा मारुत्यानीतपर्वने}% ।। १९ ।।

\twolineshloka
{हनूमान् धारयामास तत्रागं यत्र संश्थितः}
{निकुम्भिलायां होमादि कुर्वन्तं तं हि लक्ष्मणः}% ।। २० ।।

\twolineshloka
{शरैरिन्द्रजितं वीरं युद्धे तं तु व्यशातयत्}
{रावणः शोकसन्तप्तः सीतां हन्तुं समुद्यतः}% ।। २१ ।।

\twolineshloka
{अविन्ध्यवारितो राजरथस्यः सबलौययौ}
{इन्द्रोक्तो मातलीरामं रथस्थं प्रचकार तम्}% ।। २२ ।।

\twolineshloka
{रामरावणयोर्युद्धं रामरावणयोरिव}
{रावणो वानरान् हन्ति मारुत्याद्याश्च रावणम्}% ।। २३ ।।

\twolineshloka
{रामः शस्त्रैस्तमस्त्रैश्च ववर्ष जलदो यथा}
{तस्य ध्वजं स चिच्छेद रथमश्वांश्च सारथिम्}% ।। २४ ।।

\twolineshloka
{धनुर्बाहूञ्छिरांस्येव उत्तिष्ठन्ति शिरांसि हि}
{पैतामहेन हृदयं भित्त्वा रामेण रावणः}% ।। २५ ।।

\twolineshloka
{भूतले पातितः सर्वै राक्षसै रुरुदुः स्त्रियः}
{आश्वास्य तञ्च संस्कृत्य रामज्ञप्तो विभीषणः}% ।। २६ ।।

\twolineshloka
{हनृमतानयद्रामः सीतां शुद्धां गृहीतवान्}
{रामो वह्नौ प्रविष्टान्तां शुद्धामिन्द्रादिभिः स्तुतः}% ।। २७ ।।

\twolineshloka
{ब्रह्मणा दशरथेन त्वं विष्ण् राक्षसमर्द्दनः}
{इन्द्रौर्च्चितोऽमृतवृष्ट्या जीवयामास वानरान्}% ।। २८ ।।

\twolineshloka
{रामेण पूजिता जग्मुर्युद्धं दृष्ट्वा दिवञ्च ते }
{रामो विभीषणायादाल्लङ्कामभ्यर्च्य वानरान्}% ।। २९ ।।

\twolineshloka
{ससीतः पुष्पके स्थित्वा गतमार्गेण वै गतः}
{दर्शयन् वनदुर्गाणि सीतायै हृष्टमानसः}% ।। ३० ।।

\twolineshloka
{भरद्वाजं नमस्कृत्य नन्दिग्रामं समागतः}
{भरतेन नतश्चागादयोध्यान्तत्र संश्थितः}% ।। ३१ ।।

\twolineshloka
{वसिष्ठादीन्नमस्कृत्य कौशल्याञ्चैव केकयीम् }
{सुमित्रां प्राप्तराज्योऽथ द्विजादीन् सोऽभ्यपूजयत्}% ।। ३२ ।।

\twolineshloka
{वासुदेवं स्वमात्मानमश्वमेधैरथायजत्}
{सर्वदानानि स ददौ पालयामास स प्रजाः}% ।। ३३।।

\threelineshloka
{पुत्रवद्धर्म्मकामादीन् दुष्टनिग्रहणे रतः}
{सर्वधर्म्मपरो लोकः सर्वशस्या च मेदिनी}
{नाकालमरणञ्चासीद्रामे राज्यं प्रशासति} %।। ३४ ।।

इत्यादिमहापुराणे आग्नेये रामायणे युद्धकाण्डवर्णनं नाम दशमोऽध्यायः ॥

\sect{एकादशोऽध्यायः --- उत्तर-काण्ड-वर्णनम्}

\uvacha{नारद उवाच}

\twolineshloka
{राज्यस्थं राघवं जग्मुरगस्त्याद्याः सुपूजिताः}
{धन्यस्त्वं विजयी यस्मादिन्द्रजिद्विनिपातितः}% ।। १ ।।

\twolineshloka
{ब्रह्मात्मजः पुलस्त्योभूद् विश्रवास्तस्यनैकषी}
{पुष्पोत्कटाभूत् प्रथमा तत्पुत्रोभूद्धनेश्वरः}% ।। २ ।।

\twolineshloka
{नैकष्यां रावणो जज्ञे विंशद्बाहुर्द्दशाननः}
{तपसा ब्रह्मदत्तेन वरेण जितदैवतः}% ।। ३ ।।

\twolineshloka
{कुम्भकर्णः सनिद्रोऽभूद्धर्म्मिष्ठोऽभूद्धिभीषणः}
{स्वसा शूर्पणखा तेषां रावणान्मेघनादकः}% ।। ४ ।।

\twolineshloka
{इन्द्रं जित्वेन्द्रजिच्चाभूद्रावणादधिको बली}
{हतस्त्वया लक्ष्मणेन देवादेः क्षेममिच्छता}% ।। ५ ।।

\twolineshloka
{इत्युक्त्वा ते गता विप्रा अगस्त्याद्या नमस्कृताः}
{देवप्रार्थितरामोक्तः शत्रुघ्नो लवणार्द्दनः}% ।। ६ ।।

\twolineshloka
{अभूत् पूर्म्मथुरा काचिद् रामोक्तो भरतोऽवधीत्}
{कोटित्रयञ्च शैलूषपुत्राणां निशितैः शरैः}% ।। ७ ।।

\twolineshloka
{शैलूषं दुप्टगन्धर्वं सिन्धुतीरनिवासिनम्}
{तक्षञ्च पुष्करं पुत्रं स्थापयित्वाथ देशयोः}% ।। ८ ।।

\twolineshloka
{भरतोगात्सशत्रुघ्नो राघवं पूजयन् स्थितः}
{रामो दुष्टान्निहत्याजौ शिष्टान् सम्पाल्य मानवः}% ।। ९ ।।

\twolineshloka
{पुत्रौ कुशलवौ जातौ वाल्मीकेराश्रमे वरौ}
{लोकापवादात्त्यक्तायां ज्ञातौ सुचरितश्रवात्}% ।। १० ।।

\twolineshloka
{राज्येभिषिच्य ब्रह्माहमस्मीति ध्यानतत्परः}
{दशवर्षसहस्त्राणि दशवर्षसतानि च}% ।। ११ ।।

\twolineshloka
{राज्यं कृत्वा क्रतून् कृत्वा स्वर्गं देवार्च्चितो ययौ}
{सपौरः सानुजः सीतापुत्रो जनपदान्वितः}% ।। १२ ।।
                     
\uvacha{अग्निरुवाच}
\twolineshloka
{वाल्मीकिर्नारदाच्छ्रु त्वा रामायणमकारयत्}
{सविस्तरं यदेतच्च श्रृणुयात्स दिवं व्रजेत्}% ।। १३ ।।

॥इत्यादिमहापुराणे आग्नेये रामायणे उत्तरकाण्डवर्णनं नाम एकादशोऽध्यायः॥

\closesection
    \chapt{कूर्म-पुराणम्}

\sect{इक्ष्वाकु-वंश-वर्णनम्}

\src{कूर्मपुराणम्}{}{अध्यायः २१}{श्लोकाः १६--}
\vakta{}
\shrota{}
\notes{Brief story of Rama, in the context of Ikshvaku dynasty. Notable is the mention of Rameshvaram temple, and the Shiva linga installed by Rama.}
\textlink{https://sa.wikisource.org/wiki/कूर्मपुराणम्-पूर्वभागः/एकविंशतितमोऽध्यायः}
\translink{}

\storymeta

\uvacha{सूत उवाच}
\twolineshloka
{त्रिधन्वा राजपुत्रस्तु धर्मेणापालयन्महीम्}
{तस्य पुत्रोऽभवद् विद्वांस्त्रय्यारुण इति स्मृतः} %२१.१

\twolineshloka
{तस्य सत्यव्रतो नाम कुमारोऽभून्महाबलः}
{भार्या सत्यधना नाम हरिश्चन्द्रमजीजनत्} %२१.२

\twolineshloka
{हरिश्चन्द्रस्य पुत्रोऽभूद् रोहितो नाम वीर्यवान्}
{रोहितस्य वृकः पुत्रः तस्मात्बाहुरजायत} %२१.३

\threelineshloka*
{हरितो रोहितस्याथ धुन्धुस्तस्य सुतोऽभवत्}
{विजयश्च सुदेवश्च धुन्धुपुत्रौ बभूवतुः}
{विजयस्याभवत् पुत्रः कारुको नाम वीर्यवान्}

\twolineshloka
{सगरस्तस्य पुत्रौऽभूद् राजा परमधार्मिकः}
{द्वे भार्ये सगरस्यापि प्रभा भानुमती तथा} %२१.४

\twolineshloka
{ताभ्यामाराधितः वह्निः प्रादादौ वरमुत्तमम्}
{एकं भानुमती पुत्रमगृह्णादसमञ्जसम्} %२१.५

\twolineshloka
{प्रभा षष्टिसहस्त्रं तु पुत्राणां जगृहे शुभा}
{असमञ्सस्य तनयो ह्यंशुमान् नाम पार्थिवः} %२१.६

\twolineshloka
{तस्य पुत्रो दिलीपस्तु दिलीपात् तु भगीरथः}
{येन भागीरथी गङ्गा तपः कृत्वाऽवतारिता} %२१.७

\twolineshloka
{प्रसादाद् देवदेवस्य महादेवस्य धीमतः}
{भगीरथस्य तपसा देवः प्रीतमना हरः} %२१.८

\twolineshloka
{बभार शिरसा गङ्गां सोमान्ते सोमभूषणः}
{भगीरथसुतश्चापि श्रुतो नाम बभूव ह} %२१.९

\twolineshloka
{नाभागस्तस्य दायादः सिन्धुद्वीपस्ततोऽभवत्}
{अयुतायुः सुतस्तस्य ऋतुपर्णस्तु तत्सुतः} %२१.१०

\twolineshloka
{ऋतुपर्णस्य पुत्रोऽभूत् सुदासो नाम धार्मिकाः}
{सौदासस्तस्य तनयः ख्यातः कल्माषपादकः} %२१.११

\twolineshloka
{वसिष्ठस्तु महातेजाः क्षेत्रे कल्माषपादके}
{अश्मकं जनयामसा तमिक्ष्वाकुकुलध्वजम्} %२१.१२

\twolineshloka
{अश्मकस्योत्कलायां तु नकुलो नाम पार्थिवः}
{स हि रामभयाद् राजा वनं प्राप सुदुः खितः} %२१.१३

\twolineshloka
{विभ्रत् स नारीकवचं तस्माच्छतरथोऽभवत्}
{तस्माद् बिलिबिलिः श्रीमान्‌वृद्धशर्माचतत्सुतः} %२१.१४

\twolineshloka
{तस्माद् विश्वसहस्तस्मात् खट्वाङ्ग इति विश्रुतः}
{दीर्घबाहुः सुतस्तस्य रघुस्तस्मादजायत} %२१.१५

\twolineshloka
{रघोरजः समुत्पन्नो राजा दशरथस्ततः}
{रामो दाशरथिर्वोरो धर्मज्ञो लोकविश्रुतः} %२१.१६

\twolineshloka
{भरतो लक्ष्मणश्चैव शत्रुघ्नश्च महाबलः}
{सर्वे शक्रसमा युद्धे विष्णुशक्तिसमन्विताः} %२१.१७

\twolineshloka
{जज्ञे रावणनाशार्थं विष्णुरंशेन विश्वकृत्}
{रामस्य सुभगा भार्या जनकस्यात्मजा शुभा} %२१.१८

\twolineshloka
{सीता त्रिलोकविख्याता शीलौदार्यगुणान्विता}
{तपसा तोषिता देवी जनकेन गिरीन्द्रजा} %२१.१९

\twolineshloka
{प्रायच्छज्जानकीं सीतां राममेवाश्रितां पतिम्}
{प्रीतश्च भगवानीशस्त्रिशूली नीललोहितः} %२१.२०

\twolineshloka
{प्रददौ शत्रुनाशार्थं जनकायाद्‌भुतं धनुः}
{स राजा जनको विद्वान् दातुकामः सुतामिमाम्} %२१.२१

\twolineshloka
{अघोषयदमित्रघ्नो लोकेऽस्मिन् द्विजपुङ्गवाः}
{इदं धनुः समादातुं यः शक्नोति जगत्त्रये} %२१.२२

\twolineshloka
{देवो वा दानवो वाऽपि स सीतां लब्धुमर्हति}
{विज्ञाय रामो बलवान् जनकस्य गृहं प्रभुः} %२१.२३

\twolineshloka
{भञ्जयामास चादाय गत्वाऽसौ लीलयैव हि}
{उद्ववाह च तां कन्यां पार्वतीमिव शङ्करः} %२१.२४

\twolineshloka
{रामः परमधर्मात्मा सेनामिव च षण्मुखः}
{ततो बहुतिथे काले राजा दशरथः स्वयम्} %२१.२५

\twolineshloka
{रामं ज्येष्ठं सुतं वीरं राजानं कर्तुमारभत्}
{तस्याथ पत्नी सुभगा कैकेयी चारुभाषिणी} %२१.२६

\twolineshloka
{निवारयामास पतिं प्राह सम्भ्रान्तमानसा}
{मत्सुतं भरतं वीरं राजानं कर्त्तुमर्हसि} %२१.२७

\twolineshloka
{पूर्वमेव वरो यस्माद् दत्तो मे भवता यतः}
{स तस्या वचनं श्रुत्वा राजा दुःखितमानसः} %२१.२८

\twolineshloka
{बाढमित्यब्रवीद् वाक्यं तथा रामोऽपि धर्मवित्}
{प्रणम्याथ पितुः पादौ लक्ष्मणेन सहाच्युतः} %२१.२९

\twolineshloka
{ययौ वनं सपत्नीकः कृत्वा समयमात्मवान्}
{संवत्सराणां चत्वारि दश चैव महाबलः} %२१.३०

\twolineshloka
{उवास तत्र मतिमान् लक्ष्मणेन सह प्रभुः}
{कदाचिद् वसतोऽरण्ये रावणो नाम राक्षसः} %२१.३१

\twolineshloka
{परिव्राजकवेषेण सीतां हृत्वा ययौ पुरीम् ।।}
{अदृष्ट्वा लक्ष्मणो रामः सीतामाकुलितेन्द्रियौ} %२१.३२

\twolineshloka
{दुः खशोकाभिसन्तप्तौ बभूवतुररिन्दमौ}
{ततः कदाचित् कपिना सुग्रीवेण द्विजोत्तमाः} %२१.३३

\twolineshloka
{वानराणामभूत् सख्यं रामस्याक्लिष्टकर्मणः ।।}
{सुग्रीवस्यानुगो वीरो हनुमान्नाम वानरः} %२१.३४

\twolineshloka
{वायुपुत्रौ महातेजा रामस्यासीत् प्रियः सदा}
{स कृत्वा परमं धैर्यं रामाय कृतनिश्चयः} %२१.३५

\twolineshloka
{आनयिष्यामि तां सीतामित्युक्त्वा विचचार ह}
{महीं सागरपर्यन्तां सीतादर्शनतत्परः} %२१.३६

\twolineshloka
{जगाम रावणपुरीं लङ्कां सागरसंस्थिताम्}
{तत्राथ निर्जने देशे वृक्ष्मूले शुचिस्मिताम्} %२१.३७

\twolineshloka
{अपश्यदमलां सीतां राक्षसीभिः समावृताम्}
{अश्रुपूर्णेक्षणां हृद्यां संस्मरन्तीमनिन्दिताम्} %२१.३८

\twolineshloka
{राममिन्दीवरश्यामं लक्ष्मणं चात्मसंस्थितम्}
{निवेदयित्वा चात्मानं सीतायै रहसि स्वयम्} %२१.३९

\twolineshloka
{असंशयाय प्रददावस्यै रामाङ्‌गुलीयकम्}
{दृष्ट्वाऽङ्‌गुलीयकं सीता पत्युः परमशोभनम्} %२१.४०

\twolineshloka
{मेने समागतं रामं प्रीतिविस्फारितेक्षणा}
{समाश्वास्य तदा सीतां दृष्ट्वा रामस्य चान्तिकम्} %२१.४१

\twolineshloka
{नयिष्ये त्वां महाबाहुरुक्त्वा रामं ययौ पुनः}
{निवेदयित्वा रामाय सीतादर्शनमात्मवान्} %२१.४२

\twolineshloka
{तस्थौ रामेण पुरतो लक्ष्मणेन च पूजितः}
{ततः स रामो बलवान् सार्द्धं हनुमता स्वयम्} %२१.४३

\twolineshloka
{लक्ष्मणेन च युद्धाय बुद्धिं चक्रे हि रक्षसाम्}
{कृत्वाऽथ वानरशतैर्लङ्कामार्गं महोदधेः} %२१.४४

\twolineshloka
{सेतुं परमधर्मात्मा रावणं हतवान् प्रभुः}
{सपत्नीकं च ससुतं सभ्रातृकमरिन्दमः} %२१.४५

\twolineshloka
{आनयामास तां सीतां वायुपुत्रसहायवान्}
{सेतुमध्ये महादेवमीशानं कृत्तिवाससम्} %२१.४६

\twolineshloka
{स्थापयामास लिङ्गस्थं पूजयामास राघवः}
{तस्य देवो महादेवः पार्वत्या सह शङ्करः} %२१.४७

\twolineshloka
{प्रत्यक्षमेव भगवान् दत्तवान् वरमुत्तमम्}
{यत् त्वया स्थापितं लिङ्गं द्रक्ष्यन्तीह द्विजातयः} %२१४८

\twolineshloka
{महापातकसंयुक्तास्तेषां पापं विनङ्क्ष्यति}
{अन्यानि चैव पापानि स्नातस्यात्र महोदधौ} %२१.४९

\twolineshloka
{दर्शनादेव लिङ्गस्य नाशं यान्ति न संशयः}
{यावत् स्थास्यन्ति गिरयो यावदेषा च मेदिनी} %२१.५०

\twolineshloka
{यावत् सेतुश्च तावच्च स्थास्याम्यत्र तिरोहितः}
{स्नानं दानं जपः श्राद्धं भविष्यत्यक्षयं महत्} %२१.५१

\twolineshloka
{स्मरणादेव लिङ्गस्य दिनपापं प्रणश्यति}
{इत्युक्त्वा भगवाञ्छम्भुः परिष्वज्य तु राघवम्} %२१.५२

\twolineshloka
{सनन्दी सगणो रुद्रस्तत्रैवान्तरधीयत}
{रामोऽपि पालयामास राज्यं धर्मपरायणः} %२१.५३

\twolineshloka
{अभिषिक्तो महातेजा भरतेन महाबलः}
{विशेषाढ् ब्राह्मणान् सर्वान् पूजयामसचेश्वरम्} %२१.५४

\twolineshloka
{यज्ञेन यज्ञहन्तारमश्वमेधेन शङ्करम्}
{रामस्य तनयो जज्ञे कुश इत्यभिविश्रुतः} %२१.५५

\twolineshloka
{लवश्च सुमहाभागः सर्वतत्त्वार्थवित् सुधीः}
{अतिथिस्तु कुशाज्जज्ञे निषधस्तत्सुतोऽभवत्} %२१.५६

\twolineshloka
{नलस्तु निषधस्याभून्नभास्तमादजायत}
{नभसः पुण्डरीकाक्षः क्षेमधन्वा च तत्सुतः} %२१.५७

\twolineshloka
{तस्य पुत्रोऽभवद् वीरो देवानीकः प्रतापवान्}
{अहीनगुस्तस्य सुतो सहस्वांस्तत्सुतोऽभवत्} %२१.५८

\twolineshloka
{तस्माच्चन्द्रावलोकस्तु तारापीडस्तु तत्सुतः}
{तारापीडाच्चन्द्रगिरिर्भानुवित्तस्ततोऽभवत्} %२१.५९

\twolineshloka
{श्रुतायुरभवत् तस्मादेते इक्ष्वाकुवंशजाः}
{सर्वे प्राधान्यतः प्रोक्ताः समासेन द्विजोत्तमाः} %२१.६०

\twolineshloka
{य इमं श्रृणुयान्नित्यमिक्ष्वाकोर्वंशमुत्तमम्}
{सर्वपापविनिर्मुक्तो स्वर्गलोके महीयते} %२१.६१

॥इति श्रीकूर्मपुराणे षट्‌साहस्त्र्यां संहितायां पूर्वविभागे इक्ष्वाकुवंशवर्णनं नाम एकविंशोऽध्यायः॥

\closesection
    \sect{द्विचत्वारिंशदधिक-द्विशततमोऽध्यायः --- रामस्यायोध्याप्रवेशः}

\src{पद्म-पुराणम्}{सृष्टिखण्डम्}{अध्यायः २४२--२४४}{}
% \tags{concise, complete}
\notes{}
\textlink{https://sa.wikisource.org/wiki/पद्मपुराणम्/खण्डः_५_(पातालखण्डः)/अध्यायः_००१}
\translink{https://www.wisdomlib.org/hinduism/book/the-padma-purana/d/doc365826.html}

\storymeta


\uvacha{रुद्र उवाच}

\twolineshloka
{स्वायम्भुवो मनुः पूर्वं द्वाशार्णं महामनुम्}
{जजाप गोमतीतीरे नैमिषे विमले शुभे}% १

\twolineshloka
{तेन वर्षसहस्रेण पूजितः कमलापतिः}
{मत्तो वरं वृणीष्वेति तं प्राह भगवान्हरिः}% २

\onelineshloka*
{ततः प्रोवाच हर्षेण मनुः स्वायम्भुवो हरिम्}

\uvacha{मनुरुवाच}
\onelineshloka
{पुत्रत्वं भज देवेश त्रीणि जन्मानि चाच्युत}% ३

\onelineshloka*
{त्वां पुत्रलालसत्वेन भजामि पुरुषोत्तमम्}

\uvacha{रुद्र उवाच}
\onelineshloka
{इत्युक्तस्तेन लक्ष्मीशः प्रोवाच सुमहागिरा}% ४

\uvacha{विष्णुरुवाच}

\twolineshloka
{भविष्यति नृपश्रेष्ठ यत्ते मनसि काङ्क्षितम्}
{ममैव च महत्प्रीतिस्तव पुत्रत्वहेतवे}% ५

\twolineshloka
{स्थितिप्रयोजने काले तत्र तत्र नृपोत्तम}
{त्वयि जाते त्वहमपि जातोस्मि तव सुव्रत}% ६

\twolineshloka
{परित्राणाय साधूनां विनाशाय च दुष्कृताम्}
{धर्म्मसंस्थापनार्थाय सम्भवामि तवानघ}% ७

\uvacha{रुद्र उवाच}

\twolineshloka
{एवं दत्वा वरं तस्मै तत्रैवान्तर्दधे हरिः}
{अस्याभूत्प्रथमं जन्म मनोः स्वायम्भुवस्य च}% ८

\twolineshloka
{रघूणामन्वये पूर्वं राजा दशरथो ह्यभूत्}
{द्वितीयो वसुदेवोऽभूद्वृष्णीनामन्वये विभुः}% ९

\twolineshloka
{कलेर्दिव्यसहस्राब्दप्रमाणस्यान्त्यपादयोः}
{शम्भलग्रामकं प्राप्य ब्राह्मणः सञ्जनिष्यति}% १०

\twolineshloka
{कौशल्या समभूत्पत्नी राज्ञो दशरथस्य हि}
{यदोर्वंशस्य सेवार्थं देवकी नाम विश्रुता}% ११

\twolineshloka
{हरिव्रतस्य विप्रस्य भार्य्या देवप्रभा पुनः}
{एवं मातृत्वमापन्ना त्रीणि जन्मानि शार्ङ्गिणः}% १२

\twolineshloka
{पूर्वं रामस्य चरितं वक्ष्यामि तव सुव्रते}
{यस्य स्मरणमात्रेण विमुक्तिः पापिनामपि}% १३

\twolineshloka
{हिरण्यकहिरण्याक्षौ द्वितीयं जन्मसंश्रितौ}
{कुम्भकर्ण दशग्रीवावजायेतां महाबलौ}% १४

\twolineshloka
{पुलस्त्यस्य सुतो विप्रो विश्रवा नाम धार्मिकः}
{तस्य पत्नी विशालाक्षी राक्षसेन्द्र सुताऽनघे}% १५

\twolineshloka
{सुकेशितनया सा स्यात्सुमाली दानवस्य च}
{केकसी नाम कन्यासीत्तस्य भार्या दृढव्रता}% १६

\twolineshloka
{कामोद्रिक्ता तु सा देवी सन्ध्याकाले महामुनिम्}
{रमयामास तन्वङ्गी यथेष्टं शुभदर्शना}% १७

\twolineshloka
{कामात्सन्ध्याभवाद्यत्वात्तस्यां जातौ महाबलौ}
{रावणः कुम्भकर्णश्च राक्षसौ लोकविश्रुतौ}% १८

\twolineshloka
{कन्या शूर्पणखा नाम जातातिविकृतानना}
{कस्यचित्त्वथ कालस्य तस्यां जातो विभीषणः}% १९

\twolineshloka
{सुशीलो भगवद्भक्तः सत्यवाग्धर्म्मवाञ्शुचिः}
{रावणः कुम्भकर्णश्च हिमवत्पर्वतोत्तमे}% २०

\twolineshloka
{महोग्रतपसा मां वै पूजयामासतुर्भृशम्}
{रावणस्त्वथ दुष्टात्मा स्वशिरःकमलैः शुभैः}% २१

\twolineshloka
{पूजयामास मां देवि दारुणेनैव कर्म्मणा}
{ततस्तमब्रुवं सुभ्रूः प्रहृष्टेनान्तरात्मना}% २२

\twolineshloka
{वरं वृणीष्व मे वत्स यत्ते मनसि वर्त्तते}
{ततः प्रोवाच दुष्टात्मा देवदानव रक्षसाम्}% २३

\twolineshloka
{अवध्यत्वं प्रदेहीति सर्वलोकजिगीषया}
{ततोऽहं दत्तवांस्तस्मै राक्षसाय दुरात्मने}% २४

\twolineshloka
{देवदानवयक्षाणामवध्यत्वं वरानने}
{राक्षसोऽसौ महावीर्यो वरदानात्तु गर्वितः}% २५

\twolineshloka
{त्रींल्लोकान्पीडयामास देवदानवमानुषान्}
{तेन सम्बाध्यमानाश्च देवा ब्रह्मपुरोगमाः}% २६

\twolineshloka
{भयार्त्ताः शरणं जग्मुरीश्वरं कमलापतिम्}
{ज्ञात्वाथ वेदनां तेषामभयाय सनातनः}% २७

\onelineshloka*
{उवाच त्रिदशान्सर्वान्ब्रह्मरुद्रपुरोगमान्}

\uvacha{श्रीभगवानुवाच}
\onelineshloka
{राज्ञो दशरथस्याहमुत्पत्स्यामि रघोः कुले}% २८

\twolineshloka
{वधिष्यामि दुरात्मानं रावणं सह बान्धवम्}
{मानुषं वपुरास्थाय हन्मि दैवतकण्टकम्}% २९

\twolineshloka
{नन्दिशापाद्भवन्तोऽपि वानरत्वमुपागताः}
{कुरुध्वं मम साहाय्यं गन्धर्वाप्सरसोत्तमाः}% ३०

\uvacha{रुद्र उवाच}

\twolineshloka
{इत्युक्ता देवतास्सर्वा देवदेवेन विष्णुना}
{वानरत्वमुपागम्य जज्ञिरे पृथिवीतले}% ३१

\twolineshloka
{भार्गवेण प्रदत्ता तु महीसागरमेखला}
{दत्ता महर्षिभिः पूर्वं रघूणां सुमहात्मनाम्}% ३२

\twolineshloka
{वैवस्वतमनोः पुत्रो राज्ञां श्रेष्ठो महाबलः}
{इक्ष्वाकुरिति विख्यातस्सर्वधर्म्मविदांवरः}% ३३

\twolineshloka
{तदन्वये महातेजा राजा दशरथो बली}
{अजस्य नृपतेः पुत्रः सत्यवान्शीलवान्शुचिः}% ३४

\twolineshloka
{स राजा पृथिवीं सर्वां पालयामास वीर्य्यतः}
{राज्येषु स्थापयामास सर्वान्पार्थिवसत्तमान्}% ३५

\twolineshloka
{कोशलस्य नृपस्याथ पुत्री सर्वाङ्गशोभना}
{कौशल्या नाम तां कन्यामुपयेमे स पार्थिवः}% ३६

\twolineshloka
{मागधस्य नृपस्याथ तनया च शुचिस्मिता}
{सुमित्रा नाम नाम्ना च द्वितीया तस्य भामिनी}% ३७

\twolineshloka
{तृतीया केकयस्याथ नृपतेर्दुहिता तथा}
{भार्य्याभूत्पद्मपत्राक्षी केकयी नाम नामतः}% ३८

\twolineshloka
{ताभिः स्म राजा भार्याभिस्तिसृभिर्धर्मसंयुतः}
{रमयामास काकुत्स्थः पृथिवीं चानुपालयन्}% ३९

\twolineshloka
{अयोध्या नाम नगरी सरयूतीर संस्थिता}
{सर्वरत्नसुसम्पूर्णा धनधान्यसमाकुला}% ४०

\twolineshloka
{प्राकारगोपुरैर्जुष्टा हेमप्राकारसङ्कुला}
{उत्तमैर्नागतुरगैर्महेन्द्रस्य यथा पुरी}% ४१

\twolineshloka
{तस्यां राजा स धर्मात्मा उवास मुनिसत्तमैः}
{पुरोहितेन विप्रेण वसिष्ठेन महात्मना}% ४२

\twolineshloka
{राज्यं चकारयामास सर्वं निहतकण्टकम्}
{यस्मादुत्पत्स्यते तस्यां भगवान्पुरुषोत्तमः}% ४३

\twolineshloka
{तस्मात्तु नगरी पुण्या साप्ययोध्येति कीर्तिता}
{नगरस्य परं धाम्नो नाम तस्याप्यभूच्छुभे}% ४४

\twolineshloka
{यत्रास्ते भगवान्विष्णुस्तदेव परमं पदम्}
{तत्र सद्यो भवेन्मोक्षः सर्वकर्म्मनिकृन्तनः}% ४५

\twolineshloka
{जाते तत्र महाविष्णौ नराः सर्वे मुदं ययुः}
{स राजा पृथिवीं सर्वां पालयित्वा शुभानने}% ४६

\twolineshloka
{अयजद्वैष्णवेष्ट्या च पुत्रार्थी हरिमच्युतम्}
{तेन सम्पूजितः श्रीशो राजा सर्वगतो हरिः}% ४७

\twolineshloka
{वैष्णवेन तु यज्ञेन वरदः प्राह केशवः}
{तस्मिन्नाविरभूदग्नौ यज्ञरूपो हरिस्तदा}% ४८

\twolineshloka
{शुद्धजाम्बूनदप्रख्यः शङ्खचक्रगदाधरः}
{शुक्लाम्बरधरः श्रीमान्सर्वभूषणभूषितः}% ४९

\twolineshloka
{श्रीवत्सकौस्तुभोरस्को वनमालाविभूषितः}
{पद्मपत्रविशालाक्षश्चतुर्बाहुरुदारधीः}% ५०

\twolineshloka
{सव्याङ्कस्थ श्रिया सार्द्धमाविरासीद्रमेश्वरः}
{वरदोस्मीति तं प्राह राजानं भक्तवत्सलः}% ५१

\twolineshloka
{तं दृष्ट्वा सर्वलोकेशं राजा हर्षसमाकुलः}
{ववन्दे भार्य्यया सार्द्धं प्रहृष्टेनान्तरात्मना}% ५२

\twolineshloka
{प्राञ्जलिः प्रणतो भूत्वा हर्षगद्गदया गिरा}
{पुत्रत्वं मे भजेत्याह देवदेवं जनार्दनम्}% ५३

\onelineshloka*
{ततः प्रसन्नो भगवान्प्राह राजानमच्युतः}

\uvacha{विष्णुरुवाच}
\onelineshloka
{उत्पत्स्येऽहं नृपश्रेष्ठ देवलोकहिताय वै}% ५४

\twolineshloka
{परित्राणाय साधूनां राक्षसानां वधाय च}
{मुक्तिं प्रदातुं लोकानां धर्म्मसंस्थापनाय च}% ५५

\uvacha{महादेव उवाच}

\twolineshloka
{इत्युक्त्वा पायसं दिव्यं हेमपात्रस्थितं शृतम्}
{लक्ष्म्याहस्तस्थितं शुभ्रं पार्थिवाय ददौ हरिः}% ५६

\uvacha{विष्णुरुवाच}

\twolineshloka
{इदं वै पायसं राजन्पत्नीभ्यस्तव सुव्रत}
{देहि ते तनयास्तासु उत्पत्स्यन्ते मदङ्गजाः}% ५७

\uvacha{महादेव उवाच}

\twolineshloka
{इत्युक्त्वा मुनिभिः सर्वैः स्तूयमानो जनार्दनः}
{स्वात्मानं दर्शयित्वाथ तथैवान्तरधीयत}% ५८

\twolineshloka
{स राजा तत्र दृष्ट्वा च पत्नीं ज्येष्ठां कनीयसीम्}
{विभज्य पायसं दिव्यं प्रददौ सुसमाहितः}% ५९

\twolineshloka
{एतस्मिन्नन्तरे पत्नी सुमित्रा तस्य मध्यमा}
{तत्समीपं प्रयाता सा पुत्रकामा सुलोचना}% ६०

\twolineshloka
{तां दृष्ट्वा तत्र कौशल्या कैकेयी च सुमध्यमा}
{अर्द्धमर्द्धं प्रददतुस्ते तस्यै पायसं स्वकम्}% ६१

\twolineshloka
{तत्प्राश्य पायसं दिव्यं राजपत्न्यः सुमध्यमाः}
{सम्पन्नगर्भाः सर्वास्ता विरेजुः शुभ्रवर्च्चसः}% ६२

\twolineshloka
{तासां स्वप्नेषु देवेशः पीतवासा जनार्दनः}
{शङ्खचक्रगदापाणिराविर्भूतस्तदा हरिः}% ६३

\twolineshloka
{अस्मिन्काले मनोरम्ये मधुमासि शुचिस्मिते}
{शुक्ले नवम्यां विमले नक्षत्रेऽदितिदैवते}% ६४

\twolineshloka
{मध्याह्नसमये लग्ने सर्वग्रहशुभान्विते}
{कौसल्या जनयामास पुत्रं लोकेश्वरं हरिम्}% ६५

\twolineshloka
{इन्दीवरदलश्यामं कोटिकन्दर्प्पसन्निभम्}
{पद्मपत्रविशालाक्षं सर्वाभरणशोभितम्}% ६६

\twolineshloka
{श्रीवत्सकौस्तुभोरस्कं सर्वाभरणभूषितम्}
{उद्यद्दिनकरप्रख्यकुण्डलाभ्यां विराजितम्}% ६७

\twolineshloka
{अनेकसूर्य्यसङ्काशं तेजसा महता वृतम्}
{परेशस्य तनो रम्यं दीपादुत्पन्नदीपवत्}% ६८

\twolineshloka
{ईशानं सर्वलोकानां योगिध्येयं सनातनम्}
{सर्वोपनिषदामर्थमनन्तं परमेश्वरम्}% ६९

\twolineshloka
{जगत्सर्गस्थितिलये हेतुभूतमनामयम्}
{शरण्यं सर्वभूतानां सर्वभूतमयं विभुम्}% ७०

\twolineshloka
{समुत्पन्ने जगन्नाथे देवदुन्दुभयो दिवि}
{विनेदुः पुष्पवर्षाणि ववर्षुः सुरसत्तमाः}% ७१

\twolineshloka
{प्रजापतिमुखा देवा विमानस्था नभस्तले}
{तुष्टुवुर्मुनिभिः सार्द्धं हर्षपूर्णाङ्गविह्वलाः}% ७२

\twolineshloka
{जगुर्गन्धर्वपतयो ननृतुश्चाप्सरोगणाः}
{ववुः पुण्यशिवा वाताः सुप्रभोभूद्दिवाकरः}% ७३

\twolineshloka
{जज्वलुश्चाग्नयः शान्ता विमलाश्च दिशोदश}
{ततस्स राजा हर्षेण पुत्रं दृष्ट्वा सनातनम्}% ७४

\twolineshloka
{पुरोधसा वसिष्ठेन जातकर्म्मतदाऽकरोत्}
{नाम चास्मै ददौ रम्यं वसिष्ठो भगवांस्तदा}% ७५

\twolineshloka
{श्रियः कमलवासिन्या रमणोऽयं महान्प्रभुः}
{तस्माच्छ्रीराम इत्यस्य नाम सिद्धं पुरातनम्}% ७६

\twolineshloka
{सहस्रनाम्नां श्रीशस्य तुल्यं मुक्तिप्रदं नृणाम्}
{विष्णुना स समुत्पन्नो विष्णुरित्यभिधीयते}% ७७

\twolineshloka
{एवं नामास्य दत्वाथ वसिष्ठो भगवानृषिः}
{परिणीय नमस्कृत्य स्तुत्वा स्तुतिभिरेव च}% ७८

\twolineshloka
{सङ्कीर्त्य नामसाहस्रं मङ्गलार्थं महात्मनः}
{विनिर्ययौ महातेजास्तस्मात्पुण्यतमाद्गृहात्}% ७९

\twolineshloka
{राजाथ विप्रमुख्येभ्यो ददौ बहुधनं मुदा}
{गवामयुतदानं च कारयामास धर्म्मतः}% ८०

\twolineshloka
{ग्रामाणां शतसाहस्रं ददौ रघुकुलोत्तमः}
{वस्त्रैराभरणैर्दिव्यैरसङ्ख्येयैर्धनैरपि}% ८१

\twolineshloka
{विष्णोस्सन्तुष्टये तत्र तर्प्पयामास भूसुरान्}
{कौसल्या च सुतं दृष्ट्वा रामं राजीवलोचनम्}% ८२

\twolineshloka
{फुल्लहस्तारविन्दाभं पद्महस्ताम्बुजान्वितम्}
{तस्य श्रीपादकमले पद्माब्जे च वरानने}% ८३

\twolineshloka
{शङ्खचक्रगदापद्मध्वजवस्त्रादिचिह्निते}
{दृष्ट्वा वक्षसि श्रीवत्सं कौस्तुभं वनमालया}% ८४

\twolineshloka
{तस्याङ्गे सा जगत्सर्वं सदेवासुरमानुषम्}
{स्मितवक्त्रे विशालाक्षी भुवनानि चतुर्दश}% ८५

\twolineshloka
{निश्वासे तस्य वेदांश्च सेतिहासान्महात्मनः}
{द्वीपानब्धीन्गिरींस्तस्य जघने वरवर्णिनि}% ८६

\twolineshloka
{नाभ्यां ब्रह्मशिवौ तस्य कर्णयोश्च दिशः शुभाः}
{नेत्रयोर्वह्निसूर्यौ च घ्राणे वायुं महाजवम्}% ८७


\threelineshloka
{सर्वोपनिषदामर्थं दृष्ट्वा तस्य विभूतयः}
{कृत्स्ना भीता वरारोहा प्रणम्य च पुनः पुनः}
{हर्षाश्रुपूर्णनयना प्राञ्जलिर्वाक्यमब्रवीत्}% ८८

\uvacha{कौशल्योवाच}

\twolineshloka
{धन्यास्मि देवदेवेश लब्ध्वा त्वां तनयं प्रभो}
{प्रसीद मे जगन्नाथ पुत्रस्नेहं प्रदर्शय}% ८९

\uvacha{ईश्वर उवाच}

\twolineshloka
{एवमुक्तो हृषीकेशो मात्रा सर्वगतो हरिः}
{मायामानुषतां प्राप्य शिशुभावाद्रुरोद सः}% ९०

\twolineshloka
{अथ प्रमुदिता देवी कौशल्या शुभलक्षणा}
{पुत्रमालिङ्ग्य हर्षेण स्तन्यं प्रादात्सुमध्यमा}% ९१

\twolineshloka
{तस्याः स्तन्यं पपौ देवो बालभावात्सनातनः}
{उवास मातुरुत्सङ्गे जगद्भर्ता महाविभुः}% ९२

\twolineshloka
{देशे तस्मिञ्छुशुभे रम्ये सर्वकामप्रदे नृणाम्}
{उत्सवं चक्रिरे पौरा हृष्टा जनपदा नराः}% ९३

\twolineshloka
{कैकेय्यां भरतो जज्ञे पाञ्चजन्यांशचोदितः}
{सुमित्रा जनयामास लक्ष्मणं शुभलक्षणम्}% ९४

\twolineshloka
{शत्रुघ्नं च महाभागा देवशत्रुप्रतापनम्}
{अनन्तांशेन सम्भूतो लक्ष्मणः परवीरहा}% ९५

\twolineshloka
{सुदर्शनांशाच्छत्रुघ्नः सञ्जज्ञेऽमितविक्रमः}
{ते सर्वे ववृधुस्तत्र वैवस्वतमनोः कुले}% ९६

\twolineshloka
{संस्कृतास्ते सुताः सम्यग्वसिष्ठेन महौजसा}
{अधीतवेदास्ते सर्वे श्रुतवन्तस्तथा नृपाः}% ९७

\twolineshloka
{सर्वशास्त्रार्थतत्वज्ञा धनुर्वेदे च निष्ठिताः}
{बभूवुः परमोदारा लोकानां हर्षवर्द्धनाः}% ९८

\twolineshloka
{युग्मं बभूवतुस्तत्र राजानौ रामलक्ष्मणौ}
{तथा भरतशत्रुघ्नौ तयोर्युग्मं बभूव ह}% ९९

\twolineshloka
{अथ लोकेश्वरी लक्ष्मीर्जनकस्य निवेशने}
{शुभक्षेत्रे हलोद्धाते सुनासीरे शुभेक्षणे}% १००

\twolineshloka
{बालार्ककोटिसङ्काशा रक्तोत्पलकराम्बुजा}
{सर्वलक्षणसम्पन्ना सर्वाभरणभूषिता}% १०१

\twolineshloka
{धृत्वा वक्षसि चार्वङ्गी मालामम्लानपङ्कजाम्}
{सीतामुखे समुत्पन्ना बालभावेन सुन्दरी}% १०२

\twolineshloka
{तां दृष्ट्वा जनको राजा कन्यां वेदमयीं शुभाम्}
{उद्धृत्यापत्यभावेन पुपोष मिथिलापतिः}% १०३

\twolineshloka
{जनकस्य गृहे रम्ये सर्वलोकेश्वरप्रिया}
{ववृधे सर्वलोकस्य रक्षणार्थं सुरेश्वरी}% १०४

\twolineshloka
{एतस्मिन्नन्तरे देवि कौशिको लोकविश्रुतः}
{सिद्धाश्रमे महापुण्ये भागीरथ्यास्तटे शुभे}% १०५

\twolineshloka
{क्रतुप्रवरमारेभे यष्टुं तत्र महामुनिः}
{वर्त्तमानस्य तस्यास्य यज्ञस्याथ द्विजन्मनः}% १०६

\twolineshloka
{क्रतुविध्वंसिनोऽभूवन्रावणस्य निशाचराः}
{कौशिकश्चिन्तयित्वाथ रघुवंशोद्भवं हरिम्}% १०७

\twolineshloka
{आनेतुमैच्छद्धर्मात्मा लोकानां हितकाम्यया}
{स गत्वा नगरीं रम्यामयोध्यां रघुपालिताम्}% १०८

\twolineshloka
{नृपश्रेष्ठं दशरथं ददर्श मुनिसत्तमः}
{राजापि कौशिकं दृष्ट्वा प्रत्युत्थाय कृताञ्जलिः}% १०९

\twolineshloka
{पुत्रैः सह महातेजा ववन्दे मुनिसत्तमम्}
{धन्योऽहमस्मीति वदन्हर्षेण रघुनन्दनम्}% ११०

\twolineshloka
{अर्चयामास विधिना निवेश्य परमासने}
{परिणीय नमस्कृत्य किं करोमीत्युवाच तम्}% १११

\onelineshloka*
{ततः प्रोवाच हृष्टात्मा विश्वामित्रो महातपाः}

\uvacha{विश्वामित्र उवाच}
\onelineshloka
{देहि मे राघवं राजन्रक्षणार्थं क्रतोर्मम}% ११२

\twolineshloka
{साफल्यमस्तु मे यज्ञे राघवस्य समीपतः}
{तस्माद्रामं रक्षणार्थं दातुमर्हसि भूपते}% ११३

\uvacha{ईश्वर उवाच}

\twolineshloka
{तच्छ्रुत्वा मुनिवर्य्यस्य वाक्यं सर्वविदां वरः}
{प्रददौ मुनिवर्य्याय राघवं सह लक्ष्मणम्}% ११४

\twolineshloka
{आदाय राघवं तत्र विश्वामित्रो महातपाः}
{स्वमाश्रममभिप्रीतः प्रययौ द्विजसत्तमः}% ११५

\twolineshloka
{ततः प्रहृष्टास्त्रिदशाः प्रयाते रघुसत्तमे}
{ववृषुः पुष्पवर्षाणि तुष्टुवुश्च महौजसः}% ११६

\twolineshloka
{अथाजगाम हृष्टात्मा वैनतेयो महाबलः}
{अदृश्यभूतो भूतानां सम्प्राप्य रघुसत्तमम्}% ११७

\twolineshloka
{ताभ्यां दिव्ये च धनुषी तूणौ चाक्षयसायकौ}
{दिव्यान्यस्त्राणि शस्त्राणि दत्वा च प्रययौ द्विजः}% ११८

\twolineshloka
{तौ रामलक्ष्मणौ वीरौ कौशिकेन महात्मना}
{गच्छन्ती ज्ञापितारण्ये राक्षसी घोरदर्शना}% ११९

\twolineshloka
{नाम्ना तु ताडका देवि भार्या सुन्दस्य रक्षसः}
{जघ्नतुस्तां महावीरौ बाणैर्दिव्यधनुश्च्युतैः}% १२०

\twolineshloka
{निहता राघवेणाथ राक्षसी घोरदर्शना}
{त्यक्त्वा तनुं घोररूपां दिव्यरूपा बभूव सा}% १२१

\twolineshloka
{जाज्वल्यमानावपुषा सर्वाभरणविभूषिता}
{प्रययौ वैष्णवं लोकं प्रणम्य च रघूत्तमौ}% १२२

\twolineshloka
{तां हत्वा राघवः श्रीमान्कौशिकस्याश्रमं शुभम्}
{प्रविवेश महातेजा लक्ष्मणेन महात्मना}% १२३

\twolineshloka
{ततः प्रहृष्टा मुनयः प्रत्युद्गम्य रघूत्तमम्}
{निवेश्य पूजयामासुरर्घाद्यैः परमात्मने}% १२४

\twolineshloka
{कौशिकः कृतदीक्षस्तु यंष्टुं यज्ञमनुत्तमम्}
{आरेभे मुनिभिः सार्द्धं विधिना मुनिसत्तमः}% १२५

\twolineshloka
{वर्त्तमाने महायज्ञे मारीचो नाम राक्षसः}
{भ्रात्रा सुबाहुना तत्र विघ्नं कर्तुमवस्थितः}% १२६

\twolineshloka
{दृष्ट्वा तौ राक्षसौ घोरौ राघवः परवीरहा}
{जघानैकेन बाणेन सुबाहुं राक्षसेश्वरम्}% १२७

\twolineshloka
{पवनास्त्रेण महता मारीचं स निशाचरम्}
{सागरे पातयामास शुष्कपर्णमिवानिलः}% १२८

\twolineshloka
{स रामस्य महावीर्य्यं दृष्ट्वा राक्षससत्तमः}
{न्यस्तशस्त्रस्तपस्तप्तुं प्रययौ महादाश्रमम्}% १२९

\twolineshloka
{विश्वामित्रो महातेजाः समाप्ते महति क्रतौ}
{प्रहृष्टमनसा तत्र पूजयामास राघवम्}% १३०

\twolineshloka
{समाश्लिष्य महात्मानं काकपक्षधरं हरिम्}
{नीलोत्पलदलश्यामं पद्मपत्रायतेक्षणम्}% १३१

\twolineshloka
{उपाघ्राय तदा मूर्ध्नि तुष्टाव मुनिसत्तमः}
{एतस्मिन्नन्तरे राजा मिथिलाया अधीश्वरः}% १३२

\twolineshloka
{वाजपेयं क्रतुं यष्टुमारेभे मुनिसत्तमैः}
{तं द्रष्टुं प्रययुस्सर्वे विश्वामित्रपुरोगमाः}% १३३

\twolineshloka
{मुनयो रघुशार्दूल सहिताः पुण्यचेतसः}
{गच्छतस्तस्य रामस्य पदाब्जेन महात्मनः}% १३४

\twolineshloka
{अभूत्सा पावनी भूमिः समाक्रान्ता महाशिला}
{सापि शप्ता पुरा भर्त्रा गौतमेन द्विजन्मना}% १३५

\twolineshloka
{अहल्या रघुनाथस्य पादस्पर्शाच्छुभाऽभवत्}
{अथ सम्प्राप्य नगरीं मिथिलां मुनिसत्तमाः}% १३६

\twolineshloka
{राघवाभ्यां तु सहिता बभूवुः प्रीतमानसाः}
{समागतान्महाभागान्दृष्ट्वा राजा महाबलः}% १३७

\twolineshloka
{प्रत्युद्गम्य प्रणम्याथ पूजयामास मैथिलः}
{रामं पद्मविशालाक्षमिन्दीवरदलप्रभम्}% १३८

\twolineshloka
{पीताम्बरधरं श्लक्ष्णं कोमलावयवोज्ज्वलम्}
{अवधीरित कन्दर्प्पकोटिलावण्यमुत्तमम्}% १३९

\twolineshloka
{सर्वलक्षणसम्पन्नं सर्वाभरणभूषितम्}
{स्वस्य हृत्पद्ममध्ये यः परेशस्य तनुर्हरिः}% १४०

\twolineshloka
{उत्पन्नो दीपवद्दीपात्सौशील्यादिगुणैः परैः}
{तं दृष्ट्वा रघुनाथं स जनको हृष्टमानसः}% १४१

\twolineshloka
{परेशमेव तं मेने रामं दशरथात्मजम्}
{पूजयामास काकुत्स्थं धन्योस्मीति ब्रुवन्नृपः}% १४२

\twolineshloka
{प्रसादं वासुदेवस्य विष्णोर्मेने नरेश्वरः}
{प्रदातुं दुहितां तस्मै मनसा चिन्तयन्प्रभुः}% १४३

\twolineshloka
{आत्मजौ रघुवंशस्य ज्ञात्वा तत्र नृपोत्तमः}
{पूजयामास धर्मेण वस्त्रैराभरणैः शुभैः}% १४४

\twolineshloka
{ऋषीन्समर्चयामास मधुपर्कादिपूजनैः}
{ततोऽवसाने यज्ञस्य रामो राजीवलोचनः}% १४५

\twolineshloka
{भङ्क्त्वा शैवं धनुर्दिव्यं जितवान्जनकात्मजाम्}
{अथासौ वीर्यशुल्केन महता परितोषितः}% १४६

\twolineshloka
{मुदा धरणिजां तस्मै प्रददौ मिथिलाधिपः}
{केशवाय श्रियमिव यथापूर्वं महार्णवः}% १४७

\twolineshloka
{स दूतं प्रेषयामास राघवं मिथिलाधिपः}
{पुत्राभ्यां सह धर्मात्मा मिथिलायां विवेश ह}% १४८

\twolineshloka
{वसिष्ठवामदेवाद्यैः प्रीतैः सह महीपतिः}
{उवास नगरे रम्ये जनकस्य रघूत्तमः}% १४९

\twolineshloka
{तस्मिन्नेव शुभे काले रामस्य धरणीसुताम्}
{विवाहमकरोद्राजा मिथिलेन समर्चितः}% १५०

\twolineshloka
{लक्ष्मणस्योर्मिलां नाम कन्यां जनकसम्भवाम्}
{जनकस्यानुजस्याथ तनये शुभवर्चसी}% १५१

\twolineshloka
{माण्डवी श्रुतकीर्त्तिश्च सर्वलक्षणलक्षिते}
{भरतस्य च सौमित्रेर्विवाहमकरोन्नृपः}% १५२

\twolineshloka
{निर्वर्त्यौद्वाहिकं तत्र राजा दशरथो बली}
{अयोध्यां प्रस्थितः श्रीमान्पौरैर्जनपदैर्वृतः}% १५३

\twolineshloka
{पारिबर्हं समादाय मैथिलेन च पूजितः}
{ससुतः सस्नुषः साश्वः सगजः सबलानुगः}% १५४

\twolineshloka
{तदध्वनि महावीर्य्यो जामदग्निः प्रतापवान्}
{गृहीत्वा परशुं चापं सङ्क्रुद्ध इव केसरी}% १५५

\twolineshloka
{अभ्यधावच्च काकुत्स्थं योद्धुकामो नृपान्तकः}
{सम्प्राप्य राघवं दृष्ट्वा वचनं प्राह भार्गवः}% १५६

\uvacha{परशुराम उवाच}

\twolineshloka
{रामराम महाबाहो शृणुष्व वचनं मम}
{बहुशः पार्थिवान्हत्वा संयुगे भूरिविक्रमान्}% १५७

\twolineshloka
{ब्राह्मणेभ्यो महीं दत्वा तपस्तप्तुमहं गतः}
{तव वीर्यबलं श्रुत्वा त्वया योद्धुमिहागतः}% १५८

\twolineshloka
{इक्ष्वाकवो न वध्या मे मातामहकुलोद्भवाः}
{वीर्य्यं क्षत्रबलं श्रुत्वा न शक्यं सहितुं मम}% १५९

\twolineshloka
{रौद्रं चापं दुराधर्षं भज्यमानां त्वया नृप}
{तस्माद्वदान्य युद्धं मे दीयतां रघुसत्तम}% १६०

\twolineshloka
{इदं तु वैष्णवं चापं तेन तुल्यमरिन्दम}
{आरोपय स्ववीर्येण निर्जितोस्मि त्वयैव हि}% १६१

\twolineshloka
{अथवा त्यज शस्त्राणि पुरस्ताद्बलिनो मम}
{शरणं भज काकुत्स्थ कातरोस्यथ चेतसी}% १६२

\uvacha{ईश्वर उवाच}

\twolineshloka
{एवमुक्तस्तु काकुत्स्थो भार्गवेण प्रतापवान्}
{तच्चापं तस्य जग्राह तच्छक्तिं वैष्णवीमपि}% १६३

\twolineshloka
{शक्त्या वियुक्तस्स तदा जामदग्निः प्रतापवान्}
{निर्वीर्यो नष्टतेजाभूत्कर्म्महीनो यथा द्विजः}% १६४

\twolineshloka
{विनष्टतेज सन्दृष्ट्वा भार्गवं नृपसत्तमाः}
{साधुसध्विति काकुत्स्थं प्रशशंसुर्मुहुर्मुहुः}% १६५

\twolineshloka
{काकुत्स्थस्तन्महच्चापं गृहीत्वारोप्य लीलया}
{सन्धाय बाणं तच्चापे भार्गवं प्राह विस्मितम्}% १६६

\uvacha{राम उवाच}

\twolineshloka
{अनेन शरमुख्येन किं कर्त्तव्यं तव द्विज}
{छेद्मि लोकमिमं चाधः स्वर्गं वा हन्मि ते तपः}% १६७

\uvacha{ईश्वर उवाच}

\twolineshloka
{तन्दृष्ट्वा घोरसङ्काशं बाणं रामस्य भार्गवः}
{ज्ञात्वा तं परमात्मानं प्रहृष्टो राममब्रवीत्}% १६८

\uvacha{परशुराम उवाच}

\twolineshloka
{रामराम महाबाहो न वेद्मि त्वां सनातनम्}
{जानाम्यद्यैव काकुत्स्थ तव वीर्य्यगुणादिभिः}% १६९

\twolineshloka
{त्वमादिपुरुषः साक्षात्परब्रह्मपरोऽव्ययः}
{त्वमनन्तो महाविष्णुर्वासुदेवः परात्परः}% १७०

\twolineshloka
{नारायणस्त्वं श्रीशस्त्वमीश्वरस्त्वं त्रयीमयः}
{त्वं कालस्त्वं जगत्सर्वमकाराख्यस्त्वमेव च}% १७१

\twolineshloka
{स्रष्टा धाता च संहर्त्ता त्वमेव परमेश्वरः}
{त्वमचिन्त्यो महद्भूतरूपस्त्वं तु मनुर्महान्}% १७२

\twolineshloka
{चतुःषट्पञ्चगुणवांस्त्वमेव पुरुषोत्तमः}
{त्वं यज्ञस्त्वं वषट्कारस्त्वमोङ्कारस्त्रयीमयः}% १७३

\twolineshloka
{व्यक्ताव्यक्तस्वरूपस्त्वं गुणभृन्निर्ग्गुणः परः}
{स्तोतुं त्वाहमशक्तश्च वेदानामप्यगोचरम्}% १७४

\twolineshloka
{यच्चापलत्वं कृतवांस्त्वां युयुत्सुतया प्रभो}
{तत्क्षन्तव्यं त्वया नाथ कृपया केवलेन तु}% १७५

\twolineshloka
{तव शक्त्या नृपान्सर्वाञ्जित्वा दत्वा महीं द्विजान्}
{त्वत्प्रसादवशादेव शान्तिमाप्नोति नैष्ठिकीम्}% १७६

\uvacha{ईश्वर उवाच}

\twolineshloka
{एवमुक्त्वा तु काकुत्स्थं जामदग्निर्महातपाः}
{परिणीय नमस्कृत्वा राघवं लोकरक्षकम्}% १७७

\twolineshloka
{शतक्रतुकृतं स्वर्गं तदस्त्राय न्यवेदयत्}
{राघवोऽथ महातेजा ववन्दे तं महामुनिम्}% १७८

\twolineshloka
{विधिवत्पूजयामास पाद्यार्घाचमनादिभिः}
{तेन सम्पूजितस्तत्र जामदग्निर्महातपाः}% १७९

\twolineshloka
{तपस्तप्तुं ययौ रम्यं नरनारायणाश्रमम्}
{राजा दशरथः सोऽथ पुत्रैर्दारसमन्वितैः}% १८०

\twolineshloka
{स्वां पुरीं सुमुहूर्त्तेन प्रविवेश महाबलः}
{राघवो लक्ष्मणश्चैव शत्रुघ्नो भरतस्तथा}% १८१

\twolineshloka
{स्वान्स्वान्दारानुपागम्य रेमिरे हृष्टमानसाः}
{तत्र द्वादश वर्षाणि सीतया सह राघवः}% १८२

\twolineshloka
{रमयामास धर्मात्मा नारायण इव श्रिया}
{तस्मिन्नेव तु राजाथ काले दशरथः सुतम्}% १८३

\twolineshloka
{ज्येष्ठं राज्येन संयोक्तुमैच्छत्प्रीत्या महीपतिः}
{तस्य भार्याथ कैकेयी पुरा दत्तवरा प्रिया}% १८४

\twolineshloka
{अयाचत नृपश्रेष्ठं भरतस्याभिषेचनम्}
{विवासनं च रामस्य वत्सराणि चतुर्दश}% १८५

\twolineshloka
{स राजा सत्यवचनाद्रामं राज्यादथोः सुतम्}
{विवासयामास तदा दुःखेन हतचेतनः}% १८६

\twolineshloka
{शक्तोऽपि राघवस्तस्मिन्राज्यं सन्त्यज्य धर्मतः}
{दशग्रीववधार्थाय पितुर्वचनहेतुना}% १८७

\twolineshloka
{वनं जगाम काकुत्स्थो लक्ष्मणेन च सीतया}
{राजा पुत्रवियोगार्त्तः शोकेन च ममार सः}% १८८

\twolineshloka
{नियुज्यमानो भरतस्तस्मिन्राज्ये समन्त्रिभिः}
{नैच्छद्राज्यं स धर्म्मात्मा सौभ्रात्रमनुदर्शयन्}% १८९

\twolineshloka
{वनमागम्य काकुत्स्थमयाचद्भ्रातरं ततः}
{रामस्तु पितुरादेशान्नैच्छद्राज्यमरिन्दमः}% १९०

\twolineshloka
{स्वपादुके ददौ तस्मै भक्त्या सोऽप्यग्रहीत्तथा}
{रामस्य पादुके राज्यमवाप्य भरतः शुभे}% १९१

\twolineshloka
{प्रत्यहं गन्धपुष्पैश्च पूजयन्कैकयीसुतः}
{तपश्चरणयुक्तेन तस्मिंस्तस्थौ नृपोत्तमः}% १९२

\twolineshloka
{यावदागमनं तस्य राघवस्य महात्मनः}
{तावद्व्रतपराः सर्वे बभूवुः पुरवासिनः}% १९३

\twolineshloka
{राघवश्चित्रकूटाद्रौ भरद्वाजाश्रमे शुभे}
{रमयामास वैदेह्या मन्दाकिन्या जले शुभे}% १९४

\twolineshloka
{कदाचिदङ्के वैदेह्याः शेते रामो महामनाः}
{ऐन्द्रिः काकस्समागम्य तस्मिन्नेव चचार ह}% १९५

\twolineshloka
{स दृष्ट्वा जानकीं तत्र कन्दर्प्पशरपीडितः}
{विददार नखैस्तीक्ष्णैः पीनोन्नतपयोधरम्}% १९६

\twolineshloka
{तं दृष्ट्वा वायसं रामः कुशं जग्राह पाणिना}
{ब्रह्मणास्त्रेण संयोज्य चिक्षेप धरणीधरः}% १९७

\twolineshloka
{तं तृणं घोरसङ्काशं ज्वालारचितविग्रहम्}
{दृष्ट्वा काकः प्रदुद्राव विमुञ्चन्कातरं स्वरम्}% १९८

\twolineshloka
{तं काकं प्रत्यनुययौ रामस्यास्त्रं सुदारुणम्}
{वायसस्त्रिषुलोकेषु बभ्राम भयपीडितः}% १९९

\twolineshloka
{यत्र यत्र ययौ काकः शरणार्थी स वायसः}
{तत्र तत्र तदस्त्रं तु प्रविवेश भयावहम्}% २००

\twolineshloka
{ब्रह्माणमिन्द्रं रुद्रं च यमं वरुणमेव च}
{शरणार्थी जगामाशु वायसः शस्त्रपीडितः}% २०१


\threelineshloka
{तं दृष्ट्वा वायसं सर्वे रुद्राद्या देव दानवाः}
{न शक्ताः स्म वयं त्रातुमिति प्राहुर्मनीषिणः}
{अथ प्रोवाच भगवान्ब्रह्मा त्रिभुवनेश्वरः}% २०२

\uvacha{ब्रह्मोवाच}

\twolineshloka
{भो भो बलिभुजां श्रेष्ठ तमेव शरणं व्रज}
{स एव रक्षकः श्रीमान्सर्वेषां करुणानिधिः}% २०३

\twolineshloka
{रक्षत्येव क्षमासारो वत्सलं शरणागतान्}
{ईश्वरः सर्वभूतानां सौशील्यादिगुणान्वितः}% २०४

\twolineshloka
{रक्षिता जीवलोकस्य पिता माता सखा सुहृत्}
{शरणं व्रज देवेशं नान्यत्र शरणं द्विज}% २०५

\uvacha{महादेव उवाच}

\twolineshloka
{इत्युक्तस्तेन बलिभुग्ब्रह्मणा रघुनन्दनम्}
{उपेत्य सहसा भूमौ निपपात भयातुरः}% २०६

\twolineshloka
{प्राणसंशयमापन्नं दृष्ट्वा सीताथ वायसम्}
{त्राहित्राहीति भर्तारमुवाच विनयाद्विभुम्}% २०७

\twolineshloka
{पुरतः पतितं देवी धरण्यां वायसं तदा}
{तच्छिरः पादयोस्तस्य योजयामास जानकी}% २०८

\twolineshloka
{समुत्थाप्य करेणाथ कृपापीयूषसागरः}
{ररक्ष रामो गुणवान् वायसं दययार्दितः}% २०९

\twolineshloka
{तमाह वायसं रामो मा भैरिति दयानिधिः}
{अभयं ते प्रदास्यामि गच्छ गच्छ यथासुखम्}% २१०

\twolineshloka
{प्रणम्य राघवायाथ सीतायै च मुहुर्मुहुः}
{स्वर्ल्लोकं प्रययावाशु राघवेण च रक्षितः}% २११

\twolineshloka
{ततो रामस्तु वैदेह्या लक्ष्मणेन च धीमता}
{उवास चित्रकूटाद्रौ स्तूयमानो महर्षिभिः}% २१२

\twolineshloka
{तस्मिन्सम्पूज्यमानस्तु भरद्वाजेन राघवः}
{जगामात्रेस्सुविपुलमाश्रमं रघुसत्तमः}% २१३

\twolineshloka
{समागतं रघुवरं दृष्ट्वा मुनिवरोत्तमः}
{भार्यया सह धर्म्मात्मा प्रत्युद्गम्य मुदा युतः}% २१४

\twolineshloka
{आसने सुशुभे मुख्ये निवेश्य सह सीतया}
{अर्घ्यपाद्याचमनीयं च वस्त्राणि विविधानि च}% २१५

\twolineshloka
{मधुपर्कन्ददौ प्रीत्या भूषणं चानुलेपनम्}
{तस्य पत्न्यनसूया तु दिव्याम्बरमनुत्तमम्}% २१६

\twolineshloka
{सीतायै प्रददौ प्रीत्या भूषणानि द्युमन्ति च}
{दिव्यान्नपानभक्षाद्यैर्भोजयामास राघवम्}% २१७

\twolineshloka
{तेन सम्पूजितस्तत्र भक्त्या परमया नृपः}
{उवास दिवसं तत्र प्रीत्या रामस्सलक्ष्मणः}% २१८

\twolineshloka
{प्रभाते विमले रामः समुत्थाय महामुनिम्}
{परिणीय प्रणम्याथ गमनायोपचक्रमे}% २१९

\twolineshloka
{अनुज्ञातस्ततस्तेन रामो राजीवलोचनः}
{प्रययौ दण्डकारण्यं महर्षिकुलसङ्कुलम्}% २२०

\twolineshloka
{तत्रातिभीषणं घोरं विराधं नाम राक्षसम्}
{हत्वाथ शरभङ्गस्य प्रविवेशाश्रमं शुभम्}% २२१

\twolineshloka
{स तु दृष्ट्वाथ काकुत्स्थं सद्यः सङ्क्षीणकल्मषः}
{प्रययौ ब्रह्मलोकं तु गन्धर्वाप्सरसान्वितम्}% २२२

\twolineshloka
{सुतीक्ष्णस्याप्यगस्त्यस्य ह्यगस्त्यभ्रातुरेव च}
{क्रमेण प्रययौ रामस्तैश्च सम्पूजितस्तथा}% २२३

\twolineshloka
{पञ्चवट्यां ततो रामो गोदावर्यास्तटे शुभे}
{उवास सुचिरं कालं सुखेन परमेण च}% २२४

\twolineshloka
{तत्र गत्वा मुनिश्रेष्ठास्तापसा धर्मचारिणः}
{पूजयामासुरात्मेशं रामं राजीवलोचनम्}% २२५

\twolineshloka
{भयं विज्ञापयामासुस्तं च रक्षोगणेरितम्}
{तानाश्वास्य तु काकुस्थो ददौ चाभयदक्षिणाम्}% २२६

\twolineshloka
{ते तु सम्पूजितास्तेन स्वाश्रमान्सम्प्रपेदिरे}
{तस्मिंस्त्रयोदशाब्दानि रामस्य परिनिर्य्ययुः}% २२७

\twolineshloka
{गोदावर्य्यास्तटे रम्ये पञ्चवट्यां मनोरमे}
{कस्यचित्त्वथ कालस्य राक्षसी घोररूपिणी}% २२८

\twolineshloka
{रावणस्य स्वसा तत्र प्रविवेश दुरासदा}
{सा तु दृष्ट्वा रघुवरं कोटिकन्दर्प्पसन्निभम्}% २२९

\twolineshloka
{इन्दीवरदलश्यामं पद्मपत्रायतेक्षणम्}
{प्रोन्नतांसं महाबाहुं कम्बुग्रीवं महाहनुम्}% २३०

\twolineshloka
{सम्पूर्णचन्द्रसदृशं सस्मिताननपङ्कजम्}
{भृङ्गावलिनिभैः स्निग्धैः कुटिलैः शीर्षजैर्वृतम्}% २३१

\twolineshloka
{रक्तारविन्दसदृशं पद्महस्ततलाङ्कितम्}
{निष्कलङ्केन्दुसदृशं नखपङ्क्तिविराजितम्}% २३२

\twolineshloka
{स्निग्धकोमलदूर्वाभं सौकुमार्य्यनिधिं शुभम्}
{पीतकौशेयवसनं सर्वाभरणभूषितम्}% २३३

\twolineshloka
{युवाकुमारवयसं जगन्मोहनविग्रहम्}
{दृष्ट्वा तं राक्षसी रामं कन्दर्प्पशरपीडिता}% २३४

\onelineshloka*
{अब्रवीत्समुपेत्याथ रामं कमललोचनम्}

\uvacha{राक्षस्युवाच}
\onelineshloka
{कस्त्वं तापसवेषेण वर्त्तसे दण्डके वने}% २३५

\twolineshloka
{आगतोऽसि किमर्थं च राक्षसानां दुरासदे}
{शीघ्रमाचक्ष्व तत्त्वेन नानृतं वक्तुमर्हसि}% २३६

\uvacha{महेश्वर उवाच}

\onelineshloka*
{इत्युक्तः स तदा रामः सम्प्रहस्याब्रवीद्वचः}

\uvacha{राम उवाच}

\twolineshloka
{राज्ञो दशरथस्याहं पुत्रो राम इतीरितः}
{असौ ममानुजो धन्वी लक्ष्मणो नाम चानघः}% २३७

\twolineshloka
{पत्नी चेयं च मे सीता जनकस्यात्मजा प्रिया}
{पितुर्वचननिर्देशादहं वनमिहागतः}% २३८

\twolineshloka
{विचरामो महारण्यमृषीणां हितकाम्यया}
{आगतासि किमर्थं त्वमाश्रमं मम सुन्दरि}% २३९

\onelineshloka*
{का त्वं कस्य कुले जाता सर्वं सत्यं वदस्व मे}

\uvacha{महेश्वर उवाच}
\onelineshloka
{इत्युक्ता सा तु रामेण प्राह वाक्यमशङ्किता}% २४०

\uvacha{राक्षस्युवाच}

\twolineshloka
{अहं विश्रवसः पुत्री रावणस्य स्वसा नृप}
{नाम्ना शूर्पणखा नाम त्रिषु लोकेषु विश्रुता}% २४१

\twolineshloka
{इदं च दण्डकारण्यं भ्रात्रा दत्तं मम प्रभो}
{भक्षयन्नृषिसङ्घान्वै विचरामि महावने}% २४२

\twolineshloka
{त्वां तु दृष्ट्वा मुनिवरं कन्दर्पशरपीडिता}
{रन्तुकामा त्वया सार्द्धमागतास्मि सुनिर्भया}% २४३

\twolineshloka
{मम त्वं नृपशार्दूल भर्ता भवितुमर्हसि}
{इमां तव सतीं सीतां ग्रसितुं भूप कामये}% २४४

\onelineshloka*
{वनेषु गिरिमुख्येषु रमयामि त्वया सह}

\uvacha{महेश्वर उवाच}

\onelineshloka
{इत्युक्त्वा राक्षसी सीतां ग्रसितुं वीक्ष्य चोद्यताम्}% २४५

\onelineshloka
{श्रीरामः खड्गमुद्यम्य नासाकर्णौ प्रचिच्छिदे}% २४६

\twolineshloka
{रुदन्ती सभयं शीघ्रं राक्षसी विकृतानना}
{खरालयं प्रविश्याह तस्य रामस्य चेष्टितम्}% २४७

\twolineshloka
{स तु राक्षससाहस्रैर्दूषणत्रिशिरो वृतः}
{आजगाम भृशं योद्धुं राघवं शत्रुसूदनः}% २४८

\twolineshloka
{तान्रामः कानने घोरे बाणः कालान्तकोपमैः}
{निजघान महाकायान्राक्षसांस्तत्र लीलया}% २४९

\twolineshloka
{खरं त्रिशिरसं चैव दूषणं तु महाबलम्}
{रणे निपातयामास बाणैराशीविषोपमैः}% २५०

\twolineshloka
{निहत्य राक्षसान्सर्वान्दण्डकारण्यवासिनः}
{पूजितः सुरसङ्घैश्च स्तूयमानो महर्षिभिः}% २५१

\twolineshloka
{उवास दण्डकारण्ये सीतया लक्ष्मणेन च}
{राक्षसानां वधं श्रुत्वा रावणः क्रोधमूर्च्छितः}% २५२

\twolineshloka
{आजगाम जनस्थानं मारीचेन दुरात्मना}
{सम्प्राप्य पञ्चवट्यां तु दशग्रीवः स राक्षसः}% २५३

\twolineshloka
{मायाविना मरीचेन मृगरूपेण रक्षसः}
{अपहृत्याश्रमाद्दूरे तौ तु दशरथात्मजौ}% २५४

\twolineshloka
{जहार सीतां रामस्य भार्यां स्ववधकाङ्क्षया}
{ह्रियमाणां तु तां दृष्ट्वा जटायुर्गृध्रराड्बली}% २५५

\twolineshloka
{रामस्य सौहृदात्तत्र युयुधे तेन रक्षसा}
{तं हत्वा बाहुवीर्येण रावणं शत्रुवारणः}% २५६

\twolineshloka
{प्रविवेश पुरीं लङ्कां राक्षसैर्बहुभिर्वृताम्}
{अशोकवनिकामध्ये निःक्षिप्य जनकात्मजाम्}% २५७

\twolineshloka
{निधनं रामबाणेन काङ्क्षयन्स्वगृहं विशत्}
{रामस्तु राक्षसं हत्वा मारीचं मृगरूपिणम्}% २५८

\twolineshloka
{पुनराविश्य तत्राथ भ्रात्रा सौमित्रिणा ततः}
{राक्षसापहृतां भार्यां ज्ञात्वा दशरथात्मजः}% २५९

\twolineshloka
{प्रभूतशोकसन्तप्तो विललाप महामतिः}
{मार्गमाणो वने सीतां पथि गृध्रं महाबलम्}% २६०

\twolineshloka
{विच्छिन्नपादपक्षं च पतितं धरणीतले}
{रुधिरापूर्णसर्वाङ्गं दृष्ट्वा विस्मयमागतः}% २६१

\twolineshloka
{पप्रच्छ राघवं श्रीमान्केन किं त्वं जिघांसितः}
{गृध्रस्तु राघवं दृष्ट्वा मन्दमन्दमुवाच ह}% २६२

\uvacha{गृध्र उवाच}

\twolineshloka
{रावणेन हृता राम तव भार्यां बलीयसा}
{तेन राक्षसमुख्येन सङ्ग्रामे निहतोस्म्यहम्}% २६३

\uvacha{महेश्वर उवाच}

\twolineshloka
{इत्युक्त्वा राघवस्याग्रे सहसा त्यक्तजीवितः}
{संस्कारमकरोद्रामस्तस्य ब्रह्मविधानतः}% २६४

\twolineshloka
{स्वपदं च ददौ तस्मै योगिगम्यं सनातनम्}
{राघवस्य प्रसादेन स गृध्रः परमं पदम्}% २६५

\twolineshloka
{हरेः सामान्यरूपेण मुक्तिं प्राप खगोत्तमः}
{माल्यवन्तं ततो गत्वा मतङ्गस्याश्रमे शुभे}% २६६

\twolineshloka
{अभिगम्य महाभागां शबरीं धर्मचारिणीम्}
{सा तु भागवतश्रेष्ठा दृष्ट्वा तौ रामलक्ष्मणौ}% २६७

\twolineshloka
{प्रत्युद्गम्य नमस्कृत्वा निवेश्य कुशविष्टरे}
{पादप्रक्षालनं कृत्वा वन्यैः पुष्पैः सुगन्धिभिः}% २६८

\twolineshloka
{अर्चयामास भक्त्या वै हर्षनिर्भरमानसा}
{फलानि च सुगन्धीनि मूलानि मधुराणि च}% २६९

\twolineshloka
{निवेदयामास तदा राघवाभ्यां दृढव्रता}
{फलान्यास्वाद्य काकुत्स्थस्तस्यै मुक्तिं ददौ पराम्}% २७०

\twolineshloka
{ततः पम्पासरो गत्वा राघवः शत्रुसूदनः}
{जघान राक्षसं तत्र कबन्धं घोररूपिणम्}% २७१

\twolineshloka
{तं निहत्य महावीर्यो ददाह स्वर्गतश्च सः}
{ततो गोदावरीं गत्वा रामो राजीवलोचनः}% २७२

\twolineshloka
{पप्रच्छ सीतां गङ्गे त्वं किं तां जानासि मे प्रियाम्}
{न शशंस तदा तस्मै सा गङ्गा तमसावृता}% २७३

\twolineshloka
{शशाप राघवः क्रोधाद्रक्ततोया भवेति ताम्}
{ततो भयात्समुद्विग्ना पुरस्कृत्य महामुनीन्}% २७४

\twolineshloka
{कृताञ्जलिपुटा दीना राघवं शरणं गता}
{ततो महर्षयस्सर्वे रामं प्राहुस्सनातनम्}% २७५

\uvacha{ऋषय ऊचुः}

\twolineshloka
{त्वत्पादकमलोद्भूता गङ्गा त्रैलोक्यपावनी}
{त्वमेव हि जगन्नाथ तां शापान्मोक्तुमर्हसि}% २७६

\uvacha{महेश्वर उवाच}

\onelineshloka*
{ततः प्रोवाच धर्मात्मा रामः शरणवत्सलः}

\uvacha{राम उवाच}

\twolineshloka
{शबर्याः स्नानमात्रेण सङ्गता शुभवारिणा}
{मुक्ता भवतु मच्छापाद्गङ्गेयं पापनाशिनी}% २७७

\twolineshloka
{एवमुक्त्वा तु काकुत्स्थः शबरीतीर्थमुत्तमम्}
{गङ्गा गयासमं चक्रे शार्ङ्गकोट्या महाबलः}% २७८

\twolineshloka
{महाभागवतानां च तीर्थं यस्योदकेऽभवत्}
{तच्छरीरं जगद्वन्द्यं भविष्यति न संशयः}% २७९

\twolineshloka
{एवमुक्त्वा तु काकुत्स्थ ऋष्यमूकं गिरिं ययौ}
{ततः पम्पासरस्तीरे वानरेण हनूमता}% २८०

\twolineshloka
{सङ्गतस्तस्य वचनात्सुग्रीवेण समागतः}
{सुग्रीववचनाद्धत्वा वालिनं वानरेश्वरम्}% २८१

\twolineshloka
{सुग्रीवमेव तद्राज्ये रामोसावभ्यषेचयत्}
{स तु सम्प्रेषयामास दिदृक्षुर्जनकात्मजाम्}% २८२

\twolineshloka
{हनुमत्प्रमुखान्वीरान्वानरान्वानराधिपः}
{स लङ्घयित्वा जलधिं हनूमान्मारुतात्मजः}% २८३

\twolineshloka
{प्रविश्य नगरीं लङ्कां दृष्ट्वा सीतां दृढव्रताम्}
{उपवासकृशां दीनां भृशं शोकपरायणाम्}% २८४

\twolineshloka
{मलपङ्केन दिग्धाङ्गीं मलिनाम्बरधारिणीम्}
{निवेदयित्वाऽभिज्ञानं प्रवृत्तिं च निवेद्य ताम्}% २८५

\twolineshloka
{सप्तमन्त्रिसुतांस्तत्र रावणस्य सुतं तथा}
{तोरणस्तम्भमुत्पाट्य निजघान स्वयं कपिः}% २८६

\twolineshloka
{समाश्वास्य च वैदेहीं बभञ्जोपवनं तदा}
{वनपालान्किङ्करांश्च पञ्चसेनाग्रनायकान्}% २८७

\twolineshloka
{रावणस्य सुतेनाथ निगृहीतो यदृच्छया}
{दृष्ट्वा च राक्षसेन्द्रं तु सम्भाषित्वा तथैव च}% २८८

\twolineshloka
{ददाह नगरीं लङ्कां स्वलाङ्गूलाग्निना कपिः}
{तया दत्तमभिज्ञानं गृहीत्वा पुनरागमत्}% २८९

\twolineshloka
{सोऽभिगम्य महातेजा रामं कमललोचनम्}
{न्यवेदयद्वानरेन्द्रो दृष्टा सीतेति तत्वतः}% २९०

\twolineshloka
{सुग्रीवसहितो रामो वानरैर्बहुभिर्वृतः}
{महोदधेस्तटं गत्वा तत्रानीकं न्यवेशयत्}% २९१

\twolineshloka
{रावणस्यानुजो भ्राता विभीषण इतीरितः}
{धर्मात्मा सत्यसन्धश्च महाभागवतोत्तमः}% २९२

\twolineshloka
{ज्ञात्वा समागतं रामं परित्यज्य स्वपूर्वजम्}
{राज्यं सुतांश्च दारांश्च राघवं शरणं ययौ}% २९३

\twolineshloka
{परिगृह्य च तं रामो मारुतेर्वचनात्प्रभुः}
{तस्मै दत्वाऽभयं सौम्यं रक्षो राज्येऽभ्यषेचयत्}% २९४

\twolineshloka
{ततस्समुद्रं काकुत्स्थस्तर्तुकामः प्रपद्य वै}
{सुप्रसन्नजलं तं तु दृष्ट्वा रामो महाबलः}% २९५

\twolineshloka
{शार्ङ्गमादाय बाणौघैः शोषयामास वारिधिम्}
{ततस्तु सरितामीशः काकुत्स्थं करुणानिधिम्}% २९६

\twolineshloka
{प्रपद्य शरणं देवमर्चयामास वारिधिः}
{पुनरापूर्य जलधिं वरुणास्त्रेण राघवः}% २९७

\twolineshloka
{उदधेर्वचनात्सेतुं सागरे मकरालये}
{गिरिभिर्वानरानीतैर्नलः सेतुमकारयत्}% २९८

\twolineshloka
{ततो गत्वा पुरीं लङ्कां सन्निवेश्य महाबलम्}
{सम्यगायोधनं चक्रे वानराणां च रक्षसाम्}% २९९

\twolineshloka
{ततो दशास्यतनयः शक्रजिद्राक्षसो बली}
{बबन्ध नागपाशैश्च तावुभौ रामलक्ष्मणौ}% ३००

\twolineshloka
{वैनतेयः समागत्य तान्यस्त्राणि प्रमोचयत्}
{राक्षसा निहतास्सर्वे वानरैश्च महाबलैः}% ३०१

\twolineshloka
{रावणस्यानुजं वीरं कुम्भकर्णं महाबलम्}
{निजघान रणे रामो बाणैरग्निशिखोपमैः}% ३०२

\twolineshloka
{ब्रह्मास्त्रेणेन्द्रजित्क्रुद्धः पातयामास वानरान्}
{हनूमता समानीतो महौषधि महीधरः}% ३०३

\twolineshloka
{तस्यानीतस्य च स्पर्शात्सर्व एव समुत्थिताः}
{ततो रामानुजो वीरः शक्रजेतारमाहवे}% ३०४

\twolineshloka
{निपातयामास शरैर्वृत्रं वज्रधरो यथा}
{निर्ययावथ पौलस्त्यो योद्धुं रामेण संयुगे}% ३०५

\twolineshloka
{चतुरङ्गबलैः सार्द्धं मन्त्रिभिश्च महाबलः}
{समन्ततोभवद्युद्धं वानराणां च रक्षसाम्}% ३०६

\twolineshloka
{रामरावणयोश्चैव तथा सौमित्रिणा सह}
{शक्त्या निपातयामास लक्ष्मणं राक्षसेश्वरः}% ३०७

\twolineshloka
{ततः क्रुद्धो महातेजा राघवो राक्षसान्तकः}
{जघान राक्षसान्वीराञ्शरैः कालान्तकोपमैः}% ३०८

\twolineshloka
{प्रदीप्तैर्बाणसाहस्रैः कालदण्डोपमैर्भृशम्}
{छादयामास काकुत्स्थो दशग्रीवं च राक्षसम्}% ३०९

\twolineshloka
{स तु निर्भिन्नसर्वाङ्गो राघवास्त्रैर्निशाचरः}
{भयात्प्रदुद्राव रणाल्लङ्कां प्रति निशाचरः}% ३१०

\twolineshloka
{जगद्राममयं पश्यन्निर्वेदाद्गृहमाविशत्}
{ततो हनूमता नीतो महौषधिमहागिरिः}% ३११

\twolineshloka
{तेन रामानुजस्तूर्णं लब्धसंज्ञोऽभवत्तदा}
{दशग्रीवस्ततो होममारेभे जयकाङ्क्षया}% ३१२

\twolineshloka
{ध्वंसितं वानरेन्द्रैस्तदभिचारात्मकं रिपोः}
{पुनर्युद्धाय पौलस्त्यो रामेण सह निर्ययौ}% ३१३

\twolineshloka
{दिव्यस्यन्दनमारुह्य राक्षसैर्बहुभिर्युतः}
{ततः शतमखो दिव्यं रथं हर्यश्वसंयुतम्}% ३१४

\twolineshloka
{राघवाय ससूतं हि प्रेषयामास बुद्धिमान्}
{रथं मातलिना नीतं समारुह्य रघूत्तमः}% ३१५

\twolineshloka
{स्तूयमानं सुरगणैर्युयुधे तेन रक्षसा}
{ततो युद्धमभूद्धोरं रामरावणयोर्महत्}% ३१६

\twolineshloka
{सप्ताह्निकमहोरात्रं शस्त्रास्त्रैरतिभीषणम्}
{विमानस्थाः सुरास्सर्वे ददृशुस्तत्र संयुगम्}% ३१७

\twolineshloka
{दशग्रीवस्य चिच्छेद शिरांसि रघुसत्तमः}
{समुत्थितानि बहुशो वरदानात्कपर्दिनः}% ३१८

\twolineshloka
{ब्राह्ममस्त्रं महारौद्रं वधायास्य दुरात्मनः}
{ससर्ज राघवस्तूर्णं कालाग्निसदृशप्रभम्}% ३१९

\twolineshloka
{तदस्त्रं राघवोत्सृष्टं रावणस्य स्तनान्तरम्}
{विदार्य धरणीं भित्त्वा रसातलतले गतम्}% ३२०

\twolineshloka
{सम्पूज्यमानं भुजगै राघवस्य करं ययौ}
{स गतासुर्महादैत्यः पपात च ममार च}% ३२१

\twolineshloka
{ततो देवगणास्सर्वे हर्षनिर्भरमानसाः}
{ववृषुः पुष्पवर्षाणि महात्मनि जगद्गुरौ}% ३२२

\twolineshloka
{जगुर्गन्धर्वपतयो ननृतुश्चाप्सरोगणाः}
{ववुः पुण्यास्तथा वाताः सुप्रभोऽभूद्दिवाकरः}% ३२३

\twolineshloka
{तुष्टुवुर्मुनयः सिद्धा देवगन्धर्वकिन्नराः}
{लङ्कायां राक्षसश्रेष्ठमभिषिच्य विभीषणम्}% ३२४

\twolineshloka
{कृतकृत्यमिवात्मानं मेने रघुकुलोत्तमः}
{रामस्तत्राब्रवीद्वाक्यमभिषिच्य विभीषणम्}% ३२५

\uvacha{राम उवाच}

\twolineshloka
{यावच्चन्द्रश्च सूर्यश्च यावत्तिष्ठति मेदिनी}
{यावन्ममकथालोके तावद्राज्यं विभीषणे}% ३२६

\twolineshloka
{गत्वा मम पदं दिव्यं योगिगम्यं सनातनम्}
{सपुत्रपौत्रः सगणः सम्प्राप्नुहि महाबलः}% ३२७

\uvacha{ईश्वर उवाच}

\twolineshloka
{एवं दत्वा वरं तस्मै राक्षसाय महाबलः}
{सम्प्राप्य मैथिलीं तत्र परुषं जनसंसदि}% ३२८

\twolineshloka
{उवाच राघवः सीतां गर्हितं वचनं बहु}
{सा तेन गर्हिता साध्वी विवेश चानलं महत्}% ३२९


\threelineshloka
{ततो देवगणास्सर्वे शिवब्रह्मपुरोगमाः}
{दृष्ट्वा तु मातरं वह्नौ प्रविशन्तीं भयातुराः}
{समागम्य रघुश्रेष्ठं सर्वे प्राञ्जलयोऽब्रुवन्}% ३३०

\uvacha{देवा ऊचुः}

\twolineshloka
{रामराम महाबाहो शृणु त्वं चातिविक्रम}
{सीतातिविमला साध्वी तव नीत्यानपायिनी}% ३३१

\twolineshloka
{अत्याज्या तु वृथा सा हि भास्करेण प्रभा यथा}
{सेयं लोकहितार्थाय समुत्पन्ना महीतले}% ३३२

\twolineshloka
{माता सर्वस्य जगतः समस्तजगदाश्रया}
{रावणः कुम्भकर्णश्च भृत्यौ पूर्वपरायणौ}% ३३३

\twolineshloka
{शापात्तौ सनकादीनां समुत्पन्नौ महीतले}
{तयोर्विमुक्त्यै वैदेही गृहीता दण्डके वने}% ३३४

\twolineshloka
{तावुभौ वै वधं प्राप्तौ त्वया राक्षसपुङ्गवौ}
{तौ विमुक्तौ दिवं यातौ पुत्रपौत्रसहानुगौ}% ३३५

\twolineshloka
{त्वं विष्णुस्त्वं परं ब्रह्म योगिध्येयः सनातनः}
{त्वमेव सर्वदेवानामनादिनिधनोऽव्ययः}% ३३६

\twolineshloka
{त्वं हि नारायणः श्रीमान्सीता लक्ष्मीः सनातनी}
{माता सा सर्वलोकानां पिता त्वं परमेश्वर}% ३३७

\twolineshloka
{नित्यैवैष जगन्माता तव नित्यानपायिनी}
{यथा सर्वगतस्त्वं हि तथा चेयं रघूत्तम}% ३३८

\twolineshloka
{तस्माच्छुद्धसमाचारां सीतां साध्वीं दृढव्रताम्}
{गृहाण सौम्य काकुत्स्थ क्षीराब्धेरिव मा चिरम्}% ३३९

\uvacha{ईश्वर उवाच}


\threelineshloka
{एतस्मिन्नन्तरे तत्र लोकसाक्षी स पावकः}
{आदाय सीतां रामाय प्रददौ सुरसन्निधौ}
{अब्रवीत्तत्र काकुत्स्थं वह्निः सर्वशरीरगः}% ३४०

\uvacha{वह्निरुवाच}

\twolineshloka
{इयं शुद्धसमाचारा सीता निष्कल्मषा विभो}
{गृहाण मा चिरं राम सत्यं सत्यं तवाब्रुवन्}% ३४१

\uvacha{ईश्वर उवाच}

\twolineshloka
{ततोऽग्निवचनात्सीतां परिगृह्य रघूद्वहः}
{बभूव रामः संहृष्टः पूज्यमानः सुरोत्तमैः}% ३४२

\twolineshloka
{राक्षसैर्निहता ये तु सङ्ग्रामे वानरोत्तमाः}
{पितामहवरात्तूर्णं जीवमानाः समुत्थिताः}% ३४३

\twolineshloka
{ततस्तु पुष्पकं नाम विमानं सूर्यसन्निभम्}
{भ्रात्रा गृहीतं सङ्ग्रामे कौबेरं राक्षसेश्वरः}% ३४४

\twolineshloka
{तद्राघवाय प्रददौ वस्त्राण्याभरणानि च}
{तेन सम्पूजितः श्रीमान्रामचन्द्रः प्रतापवान्}% ३४५

\twolineshloka
{आरुरोह विमानाग्र्यं वैदेह्या भार्यया सह}
{लक्ष्मणेन च शूरेण भ्रात्रा दशरथात्मजः}% ३४६

\twolineshloka
{ऋक्षवानरसङ्घातैः सुग्रीवेण महात्मना}
{विभीषणेन शूरेण राक्षसैश्च महाबलैः}% ३४७

\twolineshloka
{यथाविमाने वैकुण्ठे नित्यमुक्तैर्महात्मभिः}
{तथा सर्वे समारुह्य ऋक्षवानरराक्षसाः}% ३४८

\twolineshloka
{अयोध्यां प्रस्थितो रामः स्तूयमानः सुरोत्तमैः}
{भरद्वाजाश्रमं गत्वा रामः सत्यपराक्रमः}% ३४९

\twolineshloka
{भरतस्यान्तिके तत्र हनूमन्तं व्यसर्जयत्}
{स निषादालयं गत्वा गुहं दृष्ट्वाऽथ वैष्णवम्}% ३५०

\twolineshloka
{राघवागमनं तस्मै प्राह वानरपुङ्गवः}
{नन्दिग्रामं ततो गत्वा दृष्ट्वा तं राघवानुजम्}% ३५१

\twolineshloka
{न्यवेदयत्तथा तस्मै रामस्यागमनोत्सवम्}
{भरतश्चागतं श्रुत्वा वानरेण रघूत्तमम्}% ३५२

\twolineshloka
{प्रर्हर्षमतुलं लेभे सानुजः ससुहृज्जनः}
{पुनरागत्य काकुत्स्थं हनूमान्मारुतात्मजः}% ३५३

\twolineshloka
{सर्वं शशंस रामाय भरतस्य च वर्तितम्}
{राघवस्तु विमानाग्र्यादवरुह्य सहानुजः}% ३५४

\twolineshloka
{ववन्दे भार्यया सार्द्धं भारद्वाजं तपोनिधिम्}
{स तु सम्पूजयामास काकुत्स्थं सानुजं मुनिः}% ३५५

\twolineshloka
{पक्वान्नैः फलमूलाद्यैर्वस्त्रैराभरणैरपि}
{तेन सम्पूजितस्तत्र प्रणम्य मुनिसत्तमम्}% ३५६

\twolineshloka
{अनुज्ञातः समारुह्य पुष्पकं सानुगस्तदा}
{नन्दिग्रामं ययौ रामः पुष्पकेण सुहृद्वृतः}% ३५७

\twolineshloka
{मन्त्रिभिः पौरमुख्यैश्च सानुजः केकयीसुतः}
{प्रत्युद्ययौ नृपवरैः सबलैः पूर्वजं मुदा}% ३५८

\twolineshloka
{सम्प्राप्य रघुशार्दूलं ववन्दे सानुगैर्वृतः}
{पुष्पकादवरुह्याथ राघवः शत्रुतापनः}% ३५९

\twolineshloka
{भरतं चैव शत्रुघ्नमुपसम्परिषस्वजे}
{पुरोहितं वसिष्ठं च मातृवृद्धांश्च बान्धवान्}% ३६०

\twolineshloka
{प्रणनाम महातेजाः सीतया लक्ष्मणेन च}
{विभीषणं च सुग्रीवं जाम्बवन्तं तथाङ्गदम्}% ३६१

\twolineshloka
{हनुमन्तं सुषेणं च भरतः परिषस्वजे}
{भ्रातृभिः सानुगैस्तत्र मङ्गलस्नानपूर्वकम्}% ३६२

\twolineshloka
{दिव्यमाल्याम्बरधरो दिव्यगन्धानुलेपनः}
{आरुरोह रथं दिव्यं सुमन्त्राधिष्ठितं शुभम्}% ३६३

\twolineshloka
{संस्तूयमानस्त्रिदशैर्वैदेह्या लक्ष्मणेन च}
{भरतश्चैव सुग्रीवः शत्रुघ्नश्च विभीषणः}% ३६४

\twolineshloka
{अङ्गदश्च सुषेणश्च जाम्बवान्मारुतात्मजः}
{नीलो नलश्च सुभगः शरभो गन्धमादनः}% ३६५

\twolineshloka
{अन्ये च कपयः शूरा निषादाधिपतिर्गुहः}
{राक्षसाश्च महावीर्याः पार्थिवेन्द्रा महाबलाः}% ३६६

\twolineshloka
{गजानश्वानथो सम्यगारुह्य बहुशः शुभान्}
{नानामङ्गलवादित्रैः स्तुतिभिः पुष्कलैस्तथा}% ३६७

\twolineshloka
{ऋक्षवानररक्षोभिर्निषादवरसैनिकैः}
{प्रविवेश महातेजाः साकेतं पुरमव्ययम्}% २६८

\twolineshloka
{आलोक्य राजनगरीं पथि राजपुत्रो राजानमेव पितरं परिचिन्तयानः}
{सुग्रीवमारुतिविभीषणपुण्यपादसञ्चारपूतभवनं प्रविवेश रामः}% ३६९

{॥इति श्रीपाद्मे महापुराणे पञ्चपञ्चाशत्साहस्र्यां संहितायामुत्तरखण्डे उमामहेश्वरसंवाद रामस्यायोध्याप्रवेशो नाम द्विचत्वारिंशदधिकद्विशततमोऽध्यायः॥२४२॥}

\sect{त्रिचत्वारिंशदधिक-द्विशततमोऽध्यायः --- विश्वदर्शनम्}

\uvacha{शङ्कर उवाच}

\twolineshloka
{अथ तस्मिन्दिने पुण्ये शुभलग्ने शुभान्विते}
{मङ्गलस्याभिषेकार्थं मङ्गलं चक्रिरे जनाः}% १

\twolineshloka
{वसिष्ठो वामदेवश्च जाबालिरथ कश्यपः}
{मार्कण्डेयश्च मौद्गल्यः पर्वतो नारदस्तथा}% २

\twolineshloka
{एते महर्षयस्तत्र जपहोमपुरस्सरम्}
{अभिषेकं शुभं चक्रुर्मुनयो राजसत्तमम्}% ३

\twolineshloka
{नानारत्नमये दिव्ये हेमपीठे शुभान्विते}
{निवेश्य सीतया सार्द्धं श्रिया इव जनार्दनम्}% ४

\twolineshloka
{सौवर्णकलशैर्दिव्यैर्नानारत्नमयैः शुभैः}
{सर्वतीर्थोदकैः पुण्यैर्माङ्गल्यद्रव्यसंयुतैः}% ५

\twolineshloka
{दूर्वाग्रतुलसीपत्रपुष्पगन्धसमन्वितैः}
{मन्त्रपूतजलैः शुद्धैर्मुनयः संशितव्रताः}% ६

\twolineshloka
{अजपन्वैष्णवान्सूक्तान्चतुर्वेदमयान्शुभान्}
{अभिषेकं शुभं चक्रुः काकुत्स्थं जगतः पतिम्}% ७

\twolineshloka
{तस्मिन्शुभतमे लग्ने देवदुन्दुभयो दिवि}
{विनेदुः पुष्पवर्षाणि ववृषुश्च समन्ततः}% ८

\twolineshloka
{दिव्याम्बरैर्भूषणैश्च दिव्यगन्धानुलेपनैः}
{पुष्पैर्नानाविधैर्दिव्यैर्देव्या सह रघूद्वहः}% ९

\twolineshloka
{अलङ्कृतश्च शुशुभे मुनिभिर्वेदपारगैः}
{छत्रं च चामरं दिव्यं धृतवान्लक्ष्मणस्तदा}% १०

\twolineshloka
{पार्श्वे भरतशत्रुघ्नौ तालवृन्तौ विरेजतुः}
{दर्पणं प्रददौ श्रीमान्राक्षसेन्द्रो विभीषणः}% ११

\twolineshloka
{दधार पूर्णकलशं सुग्रीवो वानरेश्वरः}
{जाम्बवांश्च महातेजाः पुष्पमालां मनोहराम्}% १२

\twolineshloka
{वालिपुत्रस्तु ताम्बूलं सकर्पूरं ददौ हरेः}
{हनुमान्दीपकां दिव्यां सुषेणश्च ध्वजं शुभम्}% १३

\twolineshloka
{परिवार्य महात्मानं मन्त्रिणः समुपासिरे}
{सृष्टिर्जयन्तो विजयः सौराष्ट्रो राष्ट्रवर्द्धनः}% १४

\twolineshloka
{अकोपो धर्मपालश्च सुमन्त्रो मन्त्रिणः स्मृताः}
{राजानश्च नरव्याघ्रा नानाजनपदेश्वराः}% १५

\twolineshloka
{पौराश्च नैगमा वृद्धा राजानं पर्युपासत}
{ऋक्षैश्च वानरेन्द्रैश्च मन्त्रिभिः पृथिवीश्वरैः}% १६

\twolineshloka
{राक्षसैर्द्विजमुख्यैश्च किङ्करैश्च समावृतः}
{परे व्योम्नि यथा लीनो दैवतैः कमलापतिः}% १७

\twolineshloka
{तथा नृपवरः श्रीमान्साकेते शुशुभे तदा}
{इन्दीवरदलश्यामं पद्मपत्रनिभेक्षणम्}% १८

\twolineshloka
{आजानुबाहुं काकुत्स्थं पीतवस्त्रधरं हरिम्}
{कम्बुग्रीवं महोरस्कं विचित्राभरणैर्युतम्}% १९

\twolineshloka
{देव्या सह समासीनमभिषिक्तं रघूत्तमम्}
{विमानस्थाः सुरगणा हर्षनिर्भरमानसाः}% २०

\twolineshloka
{तुष्टुवुर्जयशब्देन गन्धर्वाप्सरसां गणाः}
{अभिषिक्तस्ततो रामो वसिष्ठाद्यैर्महर्षिभिः}% २१

\twolineshloka
{शुशुभे सीतया देव्या नारायण इव श्रिया}
{अतिमर्त्यतयाभीत उपासितुं पदाम्बुजम्}% २२

\threelineshloka
{दृष्ट्वा तुष्टाव हृष्टात्मा शङ्करो हृष्टमागतः}
{कृताञ्जलिपुटो भूत्वा सानन्दो गद्गदाकुलः}
{हर्षयन्सकलान्देवान्मुनीनपि च वानरान्}% २३

\uvacha{महादेव उवाच}

\twolineshloka
{नमो मूलप्रकृतये नित्याय परमात्मने}
{सच्चिदानन्दरूपाय विश्वरूपाय वेधसे}% २४

\twolineshloka
{नमो निरन्तरानन्द कन्दमूलाय विष्णवे}
{जगत्त्रयकृतानन्द मूर्त्तये दिव्यमूर्त्तये}% २५

\twolineshloka
{नमो ब्रह्मेन्द्रपूज्याय शङ्कराभयदाय च}
{नमो विष्णुस्वरूपाय सर्वरूपनमोनमः}% २६

\twolineshloka
{उत्पत्तिस्थितिसंहारकारिणे त्रिगुणात्मने}
{नमोस्तु निर्गतोपाधिस्वरूपाय महात्मने}% २७

\twolineshloka
{अनया विद्यया देव्या सीतयोपाधिकारिणे}
{नमः पुम्प्रकृतिभ्यां च युवाभ्यां जगतां कृते}% २८

\twolineshloka
{जगन्मातापितृभ्यां च जनन्यै राघवाय च}
{नमः प्रपञ्चरूपिण्यै निष्प्रपञ्चस्वरूपिणे}% २९

\twolineshloka
{नमो ध्यानस्वरूपिण्यै योगिध्येयात्ममूर्त्तये}
{परिणामापरीणामरिक्ताभ्यां च नमोनमः}% ३०

\twolineshloka
{कूटस्थबीजरूपिण्यै सीतायै राघवाय च}
{सीता लक्ष्मीर्भवान्विष्णुः सीता गौरी भवान्शिवः}% ३१

\twolineshloka
{सीता स्वयं हि सावित्रि भवान्ब्रह्मा चतुर्मुखः}
{सीता शची भवान्शक्रः सीता स्वाहानलो भवान्}% ३२

\twolineshloka
{सीता संहारिणी देवी यमरूपधरो भवान्}
{सीता हि सर्वसम्पत्तिः कुबेरस्त्वं रघूत्तम}% ३३

\twolineshloka
{सीता देवी च रुद्राणी भवान्रुद्रो महाबलः}
{सीता तु रोहिणी देवी चन्द्रस्त्वं लोकसौख्यदः}% ३४

\twolineshloka
{सीता संज्ञा भवान्सूर्यः सीता रात्रिर्दिवा भवान्}
{सीतादेवी महाकाली महाकालो भवान्सदा}% ३५

\twolineshloka
{स्त्रीलिङ्गेषु त्रिलोकेषु यत्तत्सर्वं हि जानकी}
{पुन्नाम लाञ्छितं यत्तु तत्सर्वं हि भवान्प्रभो}% ३६

\twolineshloka
{सर्वत्र सर्वदेवेश सीता सर्वत्र धारिणी}
{तदात्वमपिचत्रातुन्तच्छक्तिर्विश्वधारिणी}% ३७

\twolineshloka
{तस्मात्कोटिगुणं पुण्यं युवाभ्यां परिचिह्नितम्}
{चिह्नितं शिवशक्तिभ्यां चरितं तव शान्तिदम्}% ३८

\twolineshloka
{आवां राम जगत्पूज्यौ मम पूज्यौ सदा युवाम्}
{त्वन्नामजापिनी गौरी त्वन्मन्त्रजपवानहम्}% ३९

\twolineshloka
{मुमूर्षोर्मणिकर्ण्यां तु अर्द्धोदकनिवासिनः}
{अहं दिशामि ते मन्त्रं तारकं ब्रह्मदायकम्}% ४०

\twolineshloka
{अतस्त्वं जानकीनाथ परब्रह्मासि निश्चितम्}
{त्वन्मायामोहितास्सर्वे न त्वां जानन्ति तत्वतः}% ४१

\uvacha{ईश्वर उवाच}

\twolineshloka
{इत्युक्तः शम्भुना रामः प्रसादप्रवणोऽभवत्}
{दिव्यरूपधरः श्रीमानद्भुताद्भुतदर्शनः}% ४२

\twolineshloka
{तथा तं रूपमालोक्य नरवानरदेवताः}
{न द्रष्टुमपिशक्तास्ते तेजसं महदद्भुतम्}% ४३


\threelineshloka
{भयाद्वै त्रिदशश्रेष्ठाः प्रणेमुश्चातिभक्तितः}
{भीता विज्ञाय रामोऽपि नरवानरदेवताः}
{मायामानुषतां प्राप्य स देवानब्रवीत्पुनः}% ४४

\uvacha{रामचन्द्र उवाच}

\twolineshloka
{शृणुध्वं देवता यो मां प्रत्यहं संस्तुविष्यति}
{स्तवेन शम्भुनोक्तेन देवतुल्यो भवेन्नरः}% ४५

\twolineshloka
{विमुक्तः सर्वपापेभ्यो मत्स्वरूपं समश्नुते}
{रणे जयमवाप्नोति न क्वचित्प्रतिहन्यते}% ४६

\twolineshloka
{भूतवेतालकृत्याभिर्ग्रहैश्चापि न बाध्यते}
{अपुत्रो लभते पुत्रं पतिं विन्दति कन्यका}% ४७

\twolineshloka
{दरिद्रः श्रियमाप्नोति सत्ववाञ्शीलवान्भवेत्}
{आत्मतुल्यबलः श्रीमाञ्जायते नात्र संशयः}% ४८

\twolineshloka
{निर्विघ्नं सर्वकार्येषु सर्वारम्भेषु वै नृणाम्}
{यंयं कामयते मर्त्यः सुदुर्लभमनोरथम्}% ४९


\threelineshloka
{षण्मासात्सिद्धिमाप्नोति स्तवस्यास्य प्रसादतः}
{यत्पुण्यं सर्वतीर्थेषु सर्वयज्ञेषु यत्फलम्}
{तत्फलं कोटिगुणितं स्तवेनानेन लभ्यते}% ५०

\uvacha{ईश्वर उवाच}

\twolineshloka
{इत्युक्त्वा रामचन्द्रोऽसौ विससर्ज महेश्वरम्}
{ब्रह्मादि त्रिदशान्सर्वान्विससर्ज समागतान्}% ५१

\twolineshloka
{अर्चिता मानवाः सर्वे नरवानरदेवताः}
{विसृष्टा रामचन्द्रेण प्रीत्या परमया युताः}% ५२

\twolineshloka
{इत्थं विसृष्टाः खलु ते च सर्वे सुखं तदा जग्मुरतीवहृष्टाः}
{परं पठन्तः स्तवमीश्वरोक्तं रामं स्मरन्तो वरविश्वरूपम्}% ५३

{॥इति श्रीपाद्मे महापुराणे पञ्चपञ्चाशत्साहस्र्यां संहितायामुत्तरखण्डे उमामहेश्वर संवादे विश्वदर्शनं नाम त्रिचत्वारिंशदधिकद्विशततमोऽध्यायः॥२४३॥}

\sect{चतुश्चत्वारिंशदधिक-द्विशततमोऽध्यायः --- श्रीरामचरितकथनम्}

\uvacha{शङ्कर उवाच}

\twolineshloka
{अथ रामस्तु वैदेह्या राज्यभोगान्मनोरमान्}
{बुभुजे वर्षसाहस्रं पालयन्सर्वतोदिशः}% १

\twolineshloka
{अन्तःपुरजनास्सर्वे राक्षसस्य गृहे स्थिताम्}
{गर्हयन्ति स्म वैदेहीं तथा जानपदा जनाः}% २

\twolineshloka
{लोकापवादभीत्या च रामः शत्रुनिवारकः}
{दर्शयन्मानुषं धर्ममन्तर्वत्नीं नृपात्मजाम्}% ३

\twolineshloka
{वाल्मीकेराश्रमे पुण्ये गङ्गातीरे महावने}
{विससर्ज महातेजा गर्भिणीं मुनिसंसदि}% ४

\twolineshloka
{सा भर्तुः परतन्त्रा हि उवास मुनिवेश्मनि}
{अर्चिता मुनिपत्नीभिर्वाल्मीकमुनि रक्षिता}% ५

\twolineshloka
{तत्रैवासूत यमलौ नाम्ना कुशलवौ सुतौ}
{तौ च तत्रैव मुनिना संस्कृतौ च ववर्धतुः}% ६

\twolineshloka
{रामोऽपि भ्रातृभिस्सार्द्धं पालयामास मेदिनीम्}
{यमादिगुणसम्पन्नस्सर्वभोगविवर्जितः}% ७

\twolineshloka
{अर्चयन्सततं विष्णुमनादिनिधनं हरिम्}
{ब्रह्मचर्यपरो नित्यं शशास पृथिवीं नृपः}% ८

\twolineshloka
{शत्रुघ्नो लवणं हत्वा मथुरां देवनिर्मिताम्}
{पालयामास धर्मात्मा पुत्राभ्यां सह राघवः}% ९

\twolineshloka
{गन्धर्वान्भरतो हत्वा सिन्धोरुभयपार्श्वतः}
{स्वात्मजौ स्थापयामास तस्मिन्देशे महाबलौ}% १०

\twolineshloka
{पश्चिमे मद्रदेशे तु मद्रान्हत्वा च लक्ष्मणः}
{स्वसुतौ च महावीर्यौ अभिषिच्य महाबलः}% ११

\twolineshloka
{गत्वा पुनरयोध्यां तु रामपादावुपस्पृशत्}
{ब्राह्मणस्य मृतं बालं कालधर्ममुपागतम्}% १२

\twolineshloka
{जीवयामास काकुत्स्थः शूद्रं हत्वा च तापसम्}
{ततस्तु गौतमीतीरे नैमिषे जनसंसदि}% १३

\twolineshloka
{इयाज वाजिमेधं च राघवः परवीरहा}
{काञ्चनीं जानकीं कृत्वा तया सार्द्धं महाबलः}% १४

\twolineshloka
{चकार यज्ञान्बहुशो राघवः परमार्थवित्}
{अयुतान्यश्वमेधानि वाजपेयानि च प्रभुः}% १५

\twolineshloka
{अग्निष्टोमं विश्वजितं गोमेधं च शतक्रतुम्}
{चकार विविधान्यज्ञान्परिपूर्णसदक्षिणान्}% १६

\twolineshloka
{एतस्मिन्नन्तरे तत्र वाल्मीकिः सुमहातपाः}
{सीतामानीय काकुत्स्थमिदं वचनमब्रवीत्}% १७

\uvacha{वाल्मीकिरुवाच}


\threelineshloka
{अपापां मैथिलीं राम त्यक्तुं नार्हसि सुव्रत}
{इयं तु विरजा साध्वी भास्करस्य प्रभा यथा}
{अनन्या तव काकुत्स्थ कस्मात्त्यक्ता त्वयानघ}% १८

\uvacha{राम उवाच}

\twolineshloka
{अपापां मैथिलीं ब्रह्मन्जानामि वचनात्तव}
{रावणेन हृता साध्वी दण्डके विजने पुरा}% १९

\twolineshloka
{तं हत्वा समरे सीतां शुद्धामग्निमुखागताम्}
{पुनर्यातोस्म्ययोध्यायां सीतामादाय धर्मतः}% २०

\twolineshloka
{लोकापवादः सुमहानभूत्पौरजनेषु च}
{त्यक्ता मया शुभाचारा तद्भयात्तव सन्निधौ}% २१

\twolineshloka
{तस्माल्लोकस्य सन्तुष्ट्यै सीता मम परायणा}
{पार्थिवानां महर्षीणां प्रत्ययं कर्तुमर्हति}% २२

\uvacha{महेश्वर उवाच}

\twolineshloka
{एवमुक्ता तदा सीता मुनिपार्थिवसंसदि}
{चकारप्रत्ययं देवी लोकाश्चर्यकरं सती}% २३

\twolineshloka
{दर्शयंस्तस्य लोकस्य रामस्यानन्यतां सती}
{अब्रवीत्प्राञ्जलिः सीता सर्वेषां जनसंसदि}% २४

\uvacha{सीतोवाच}

\twolineshloka
{यथाऽहं राघवादन्यं मनसापि न चिन्तये}
{तथा मे धरणी देवी विवरन्दातुमर्हति}% २५

\twolineshloka
{यथैव सत्यमुक्तं मे वेद्मि रामात्परं न च}
{तथा स्वपुत्र्यां वैदेह्यां धरणी सहसा इयात्}% २६

\uvacha{महेश्वर उवाच}

\twolineshloka
{ततो रत्नमयं पीठं पृष्ठे धृत्वा खगेश्वरः}
{रसातलात्तदा वीरो विज्ञाय जननीं तदा}% २७

\twolineshloka
{ततस्तु धरणीदेवी हस्ताभ्यां गृह्य मैथिलीम्}
{स्वागतेनाभिनन्द्यैनामासने सन्न्यवेशयत्}% २८

\twolineshloka
{सीतां समागतां दृष्ट्वा दिवि देवगणा भृशम्}
{पुष्पवृष्टिमविच्छिन्नां दिव्यां सीतामवाकिरन्}% २९

\twolineshloka
{सापि दिव्याप्सरोभिस्तु पूज्यमाना सनातनी}
{वैनतेयं समारुह्य तस्मान्मार्गाद्दिवं ययौ}% ३०

\twolineshloka
{दासीगणैः पूर्वभागे संवृता जगदीश्वरी}
{सम्प्राप्य परमं धाम योगिगम्यं सनातनम्}% ३१

\twolineshloka
{रसातलप्रविष्टां तु तां दृष्ट्वा सर्वमानुषाः}
{साधुसाध्विति सीतेयमुच्चैः सर्वे प्रचुक्रुशुः}% ३२

\twolineshloka
{रामः शोकसमाविष्टः सङ्गृह्य तनयावुभौ}
{मुनिभिः पार्थिवेन्द्रैश्च साकेतं प्रविवेश ह}% ३३

\twolineshloka
{अथ कालेन महता मातरः संशितव्रताः}
{कालधर्मं समापन्ना भर्तुः स्वर्गं प्रपेदिरे}% ३४

\twolineshloka
{दशवर्षसहस्राणि दशवर्षशतानि च}
{चकार राज्यं धर्मेण राघवः संशितव्रतः}% ३५

\twolineshloka
{कस्यचित्त्वथकालस्य राघवस्य निवेशनम्}
{कालस्तापसरूपेण सम्प्राप्तो वाक्यमब्रवीत्}% ३६

\uvacha{काल उवाच}

\twolineshloka
{राम राम महाबाहो धात्रा सम्प्रेषितोऽस्म्यहम्}
{यद्ब्रवीमि रघुश्रेष्ठ तच्छृणुष्व महामते}% ३७

\twolineshloka
{द्वन्द्वमेव हि कार्यं स्यादावयोः परिभाषितम्}
{तदन्तरे प्रविष्टोयस्स वद्ध्यो हि भविष्यति}% ३८

\uvacha{महेश्वर उवाच}


\threelineshloka
{तथेति च प्रतिश्रुत्य रामो राजीवलोचनः}
{द्वास्थं कृत्वा तु सौमित्रिं कालो वाक्यमभाषत}
{वैवस्वतोऽब्रवीद्वाक्यं रामं दशरथात्मजम्}% ३९

\uvacha{काल उवाच}

\twolineshloka
{शृणु राम यथावृत्तं समागमनकारणात्}
{दशवर्षसहस्राणि दशवर्षशतानि च}% ४०

\twolineshloka
{वसामि मानुषे लोके हत्वा राक्षसपुङ्गवौ}
{एवमुक्तः सुरगणैरवतीर्णोसि भूतले}% ४१

\twolineshloka
{तदयं समयः प्राप्तः स्वर्लोकं गमितुं त्वया}
{सनाथा हि सुरास्सर्वे भवन्त्वद्य त्वयानघ}% ४२

\uvacha{महेश्वर उवाच}

\twolineshloka
{एवमस्त्विति काकुत्स्थो रामः प्राह महामुनिम्}
{एतस्मिन्नन्तरे तत्र दुर्वासास्तु महातपाः}% ४३

\onelineshloka*
{राजद्वारमुपागम्य लक्ष्मणं वाक्यमब्रवीत्}

\uvacha{दुर्वासा उवाच}
\onelineshloka
{मां निवेदय काकुत्स्थं शीघ्रं गत्वा नृपात्मज}% ४४

\uvacha{महेश्वर उवाच}

\twolineshloka
{तमब्रवील्लक्ष्मणस्तु असान्निध्यमिति द्विज}
{ततः क्रोधसमाविष्टः प्राह तं मुनिसत्तमः}% ४५

\uvacha{दुर्वासा उवाच}

\onelineshloka*
{शापं दास्यामि काकुत्स्थं रामं न यदि दर्शये}

\uvacha{महेश्वर उवाच}

\twolineshloka
{तस्माच्छापभयाद्विप्रं राघवाय न्यवेदयत्}
{तत्रैवान्तर्दधे कालः सर्वभूतभयावहः}% ४६

\twolineshloka
{पूजयामास तं प्राप्तमृषिं दुर्वाससं नृपः}
{अग्रजस्य प्रतिज्ञा तं विज्ञाय रघुसत्तमः}% ४७

\twolineshloka
{तत्याज मानुषं रूपं लक्ष्मणः सरयूजले}
{विसृज्य मानुषं रूपं प्रविवेश स्वकां तनुम्}% ४८

\twolineshloka
{फणासहस्रसंयुक्तः कोटीन्दुसमवर्चसः}
{दिव्यमाल्याम्बरधरो दिव्यगन्धानुलेपनः}% ४९

\twolineshloka
{नागकन्यासहस्रैस्तु संवृतः समलङ्कृतः}
{विमानं दिव्यमारुह्य प्रययौ वैष्णवं पदम्}% ५०

\twolineshloka
{लक्ष्मणस्य गतिं सर्वां विदित्वा रघुसत्तमः}
{स्वयमप्यथ काकुत्स्थः स्वर्गं गन्तुमभीप्सितः}% ५१

\twolineshloka
{अभिषिच्याथ काकुत्स्थः स्वात्मजौ च कुशीलवौ}
{विभज्य रथनागाश्वं सधनं प्रददौ तयोः}% ५२

\twolineshloka
{कुशवत्यां कुशं तं च शरवत्यां लवं तथा}
{स्थापयामास धर्मेण राज्ये स्वे रघुसत्तमः}% ५३

\twolineshloka
{अभिप्रायं तु विज्ञाय रामस्य विदितात्मनः}
{आजग्मुर्वानराः सर्वे राक्षसाः सुमहाबलाः}% ५४

\twolineshloka
{विभीषणोऽथ सुग्रीवो जाम्बवान्मारुतात्मजः}
{नीलो नलः सुषेणश्च निषादाधिपतिर्गुहः}% ५५

\twolineshloka
{अभिषिच्य सुतौ वीरौ शत्रुघ्नश्च महामनाः}
{सर्व एते समाजग्मुरयोध्यां रामपालिताम्}% ५६

\onelineshloka*
{ते प्रणम्य महात्मानमूचुः प्राञ्जलयस्तथा}

\uvacha{वानरप्रभृतय ऊचुः}

\onelineshloka
{स्वर्लोकं गन्तुमुद्युक्तं ज्ञात्वा त्वां रघुसत्तम}% ५७


\threelineshloka
{आगताः स्म वयं सर्वे तवानुगमनं प्रति}
{न शक्ताः स्म क्षणं राम जीवितुं त्वां विना प्रभो}
{तस्मात्त्वया विशालाक्ष गच्छामस्त्रिदशालयम्}% ५८

\uvacha{महेश्वर उवाच}

\twolineshloka
{तैरेवमुक्तः काकुत्स्थो बाढमित्यब्रवीत्ततः}
{अथोवाच महातेजा राक्षसेन्द्रं विभीषणम्}% ५९

\uvacha{राम उवाच}

\onelineshloka*
{राज्यं प्रशास धर्मेण मा प्रतिज्ञां वृथा कृथाः}

\twolineshloka
{यावच्चन्द्रश्च सूर्यश्च यावत्तिष्ठति मेदिनी}
{तावद्रमस्व सुप्रीतो काले मम पदं व्रज}% ६०

\uvacha{महेश्वर उवाच}

\twolineshloka
{इत्युक्त्वाथ स काकुत्स्थः स्वाड्गं विष्णुं सनातनम्}
{श्रीरङ्गशायिनं सौम्यमिक्ष्वाकुकुलदैवतम्}% ६१

\twolineshloka
{सम्प्रीत्या प्रददौ तस्मै रामो राजीवलोचनः}
{हनुमन्तमथोवाच राघवः शत्रुसूदनः}% ६२

\uvacha{राम उवाच}

\twolineshloka
{मत्कथाः प्रचरिष्यन्ति यावल्लोके हरीश्वर}
{तावत्त्वमास मेदिन्यां काले मां व्रज सुव्रत}% ६३

\uvacha{महेश्वर उवाच}

\onelineshloka*
{तमेवमुक्त्वा काकुत्स्थो जाम्बवन्तमथाब्रवीत्}

\uvacha{राम उवाच}
\onelineshloka
{द्वापरे समनुप्राप्ते यदूनामन्वये पुनः}% ६४

\twolineshloka
{भूभारस्य विनाशाय समुत्पत्स्याम्यहं भुवि}
{करिष्ये तत्र सङ्ग्रामं स्वयं भल्लूकसत्तम}% ६५

\uvacha{महेश्वर उवाच}

\twolineshloka
{तमेवमुक्त्वा काकुत्स्थः सर्वांस्तानृक्षवानरान्}
{उवाच वाचा गच्छध्वमिति रामो महाबलः}% ६६

\twolineshloka
{मन्त्रिणो नैगमाश्चैव भरतः कैकयीसुतः}
{राघवस्यानुगमने निश्चितास्ते समाययुः}% ६७

\twolineshloka
{ततः शुक्लाम्बरधरो ब्रह्मचारी ययौ परम्}
{कुशान्गृहीत्वा पाणिभ्यां संसक्तः प्रययौ परम्}% ६८

\twolineshloka
{रामस्य दक्षिणे पार्श्वे पद्महस्ता रमा गता}
{तथैव धरणीदेवी दक्षिणेतरगा तथा}% ६९

\twolineshloka
{वेदाः साङ्गाः पुराणानि सेतिहासानि सर्वतः}
{ॐकारोऽथ वषट्कारः सावित्री लोकपावनी}% ७०

\twolineshloka
{अस्त्रशस्त्राणि च तदा धनुराद्यानि पार्वति}
{अनुजग्मुस्तथा रामं सर्वे पुरुषविग्रहाः}% ७१

\twolineshloka
{भरतश्चैव शत्रुघ्नः सर्वे पुरनिवासिनः}
{सपुत्रदाराः काकुत्स्थमनुजग्मुः सहानुगाः}% ७२

\twolineshloka
{मन्त्रिणो भृत्यवर्गाश्च किङ्करा नैगमास्तथा}
{वानराश्चैव ऋक्षाश्च सुग्रीवसहितास्तदा}% ७३

\twolineshloka
{सपुत्रदाराः काकुत्स्थमन्वगच्छन्महामतिम्}
{पशवः पक्षिणश्चैव सर्वे स्थावरजङ्गमाः}% ७४

\twolineshloka
{अनुजग्मुर्महात्मानं समीपस्था नरोत्तमाः}
{ये च पश्यन्ति काकुत्स्थं स्वपथान्तर्गतं प्रभुम्}% ७५

\twolineshloka
{ते तथानुगता रामं निवर्त्तन्ते न केचन}
{अथ त्रियोजनं गत्वा नदीं पश्चान्मुखीं स्थिताम्}% ७६

\twolineshloka
{सरयूं पुण्यसलिलां प्रविवेश सहानुगः}
{ततः पितामहो ब्रह्मा सर्वदेवगणावृतः}% ७७

\twolineshloka
{तुष्टाव रघुशार्दूलमृषिभिः सार्द्धमक्षरैः}
{अब्रवीत्तत्र काकुत्स्थं प्रविष्टं सरयूजले}% ७८

\uvacha{ब्रह्मोवाच}

\twolineshloka
{आगच्छ विष्णो भद्रं ते दिष्ट्या प्राप्तोऽसि मानद}
{भ्रातृभिस्सहदेवाभैः प्रविशस्व निजां तनुम्}% ७९

\twolineshloka
{वैष्णवीं तां महातेजां देवाकारां सनातनीम्}
{त्वं हि लोकगतिर्देव न त्वां केचित्तु जानते}% ८०

\twolineshloka
{त्वामचिन्त्यं महात्मानमक्षरं सर्वसङ्ग्रहम्}
{यमिच्छसि महातेजस्तां तनुं प्रविशस्व भोः}% ८१

\uvacha{महेश्वर उवाच}

\twolineshloka
{तस्मिन्सूर्यकराकीर्णे पुष्पवृष्टिनिपातिते}
{उत्सृज्य मानुषं रूपं स्वां तनुं प्रविवेश ह}% ८२

\twolineshloka
{अंशाभ्यां शङ्खचक्राभ्यां शत्रुघ्नभरतावुभौ}
{तदा तेन महात्मानौ दिव्यतेजस्समन्वितौ}% ८३

\twolineshloka
{शङ्खचक्रगदाशार्ङ्गपद्महस्तश्चतुर्भुजः}
{दिव्याभरणसम्पन्नो दिव्यगन्धानुलेपनः}% ८४

\twolineshloka
{दिव्यपीताम्बरधरः पद्मपत्रनिभेक्षणः}
{युवा कुमारः सौम्याङ्गः कोमलावयवोज्ज्वलः}% ८५

\twolineshloka
{सुस्निग्धनीलकुटिलकुन्तलः शुभलक्षणः}
{नवदूर्वाङ्कुरः श्यामः पूर्णचन्द्र निभाननः}% ८६

\twolineshloka
{देवीभ्यां सहितः श्रीमान्विमानमधिरुह्य च}
{तस्मिन्सिंहासने दिव्ये मूले कल्पतरोः प्रभुः}% ८७

\twolineshloka
{निषसाद महातेजाः सर्वदेवैरभिष्टुतः}
{राघवानुगता ये च ऋक्षवानरमानुषाः}% ८८

\twolineshloka
{स्पृष्ट्वैव सरयूतोयं सुखेन त्यक्तजीविताः}
{रामप्रसादात्ते सर्वे दिव्यरूपधराः शुभाः}% ८९

\twolineshloka
{दिव्यमाल्याम्बरधरा दिव्यमङ्गलवर्चसः}
{आरुरोह विमानं तदसङ्ख्यास्तत्र देहिनः}% ९०

\twolineshloka
{सर्वैः परिवृतः श्रीमान्रामो राजीवलोचनः}
{पूजितः सुरसिद्धौघैर्मुनिभिस्तु महात्मभिः}% ९१

\twolineshloka
{आययौ शाश्वतं दिव्यमक्षरं स्वपदं विभुः}
{यः पठेद्रामचरितं श्लोकं श्लोकार्धमेव वा}% ९२

\twolineshloka
{शृणुयाद्वा तथा भक्त्या स्मरेद्वा शुभदर्शने}
{कोटिजन्मार्जितात्पापाज्ज्ञानतोऽज्ञानतः कृतात्}% ९३

\twolineshloka
{विमुक्तो वैष्णवं लोकं पुत्रदारसबान्धवैः}
{समाप्नुयाद्योगगम्यमनायासेन वै नरः}% ९४


\onelineshloka
{एतत्ते कथितं देवि रामस्य चरितं महत्}
{धन्योऽस्म्यहं त्वया देवि रामचन्द्रस्य कीर्त्तनात्}
{किमन्यच्छ्रोतुकामासि तद्ब्रवीमि वरानने}% ९५

{॥इति श्रीपाद्मे महापुराणे पञ्चपञ्चाशत्साहस्र्यां संहितायामुत्तरखण्डे उमामहेश्वर संवादे श्रीरामचरितकथनं नाम चतुश्चत्वारिंशदधिकद्विशततमोऽध्यायः॥२४४॥}



    \chapt{नरसिंह-पुराणम्}

\src{नरसिंह-पुराणम्}{अध्यायः २६}{}{}
\vakta{}
\shrota{}
\notes{}
\textlink{https://archive.org/details/narasimha-purana-english/page/171/mode/2up}
\translink{https://archive.org/details/narasimha-purana-english/page/171/mode/2up}

\storymeta

\sect{षडविंशोऽध्यायः --- सूर्यवंशानुचरितम्}

\addtocounter{shlokacount}{8}

\uvacha{सूत उवाच}

दीर्घबाहोरजोऽजाद्दशरथः। तस्य गृहे रावणविनाशार्थं साक्षान्नारायणोऽवतीर्णो रामः॥९॥

स तु पितृवचनाद भ्रातृभार्यासहितो दण्डकारण्यं प्राप्य तपश्चचार।

वने रावणापहतभार्यो भ्रात्रा सह दुःखितोऽनेककोटिवानरनायक सुग्रीवसहायो मदोदधौ

सेतुं निबध्य तैर्गत्वा लङ्कां रावणं देवकण्टकं सबान्धवं हत्वा सीतामादाय पुनरयोध्यां

प्राप्य भरताभिषिक्तो विभीषणाय लङ्काराज्यं विमानं वा दत्त्वा तं प्रेषयामास।

स तु परमेश्वरो विमानस्थो विभीषणेन नीयमानो लङ्कायामपि राक्षसपुर्यां वस्तुमनिच्छन् पुण्यारण्यं तत्र स्थापितवान्॥१०॥

तन्निरीक्ष्य तत्रैव महाहिभोगशयने भगवान् शेते। सोऽपि विभीषणस्ततस्तद्विमानं नेतुमसमर्थः, तद्वचनात् स्वां पुरीं जगाम॥११॥

नारायणसन्निधानान्महद्वैष्णवं क्षेत्रमभवदद्यापि दृश्यते। रामाल्लवो लवात्पद्यः पद्मादृतुपर्ण ऋतुपर्णादस्त्रपाणिः।

अस्त्रपाणेः शुद्धोदनः शुद्धोदनाद्वुधः। बुधाद्वंशो निवर्तते॥१२॥

\twolineshloka
{एते महीपा रविवंशजास्तव प्राधान्यतस्ते कथिता महाबलाः}
{पुरातनैर्यैर्वसुधा प्रपालिता यज्ञक्रियाभिश्च दिवौकसैर्नृपैः} % ॥१३॥

॥इति श्रीनरसिंहपुराणे सूर्यवंशानुचरितं नाम षडविंशोऽध्यायः ॥२६॥


\src{नरसिंह-पुराणम्}{अध्यायः ४७--५२}{}{}
\vakta{}
\shrota{}
\notes{Concise retelling of all the Kandas of Ramayana.}
\textlink{https://archive.org/details/narasimha-purana-english/page/171/mode/2up}
\translink{https://archive.org/details/narasimha-purana-english/page/171/mode/2up}

\storymeta


\sect{सप्तचत्वारिंशोऽध्यायः --- बाल-काण्डः}

\uvacha{मार्कण्डेय उवाच}

\twolineshloka
{श्रुणु राजन् प्रवक्ष्यामि प्रादुर्भावं हरेः शुभम्}
{निहतो रावणो येन सगणो देवकण्टकः} %॥१॥

\twolineshloka
{ब्रह्मणो मानसः पुत्रः पुनस्त्योऽभून्महामुनिः}
{तस्य वै विश्रवा नाम पुत्रोऽभूत्तस्य राक्षसः} %॥२॥

\twolineshloka
{तस्माज्जातो महावीरो रावणो लोकरावणः}
{तपसा महता युक्तः स तु लोकानुपाद्रवत्} %॥३॥

\twolineshloka
{सेन्द्रा देवा जितास्तेन गन्धर्वाः किन्नरास्तथा}
{यक्षाश्च दानवाश्चैव तेन राजन् विनिर्जिताः} %॥४॥

\twolineshloka
{स्त्रियश्चैव सुरुपिण्यो हतास्तेन दुरात्मना}
{देवादीनां नृपश्रेष्ठ रत्नानि विविधानि च} %॥५॥

\twolineshloka
{रणे कुबेरं निर्जित्य रावणो बलदर्पितः}
{तत्पुरीं जगृहे लङ्कां विमानं चापि पुष्पकम्} %॥६॥

\twolineshloka
{तस्यां पुर्यां दशग्रीवो रक्षसामधिपोऽभवत्}
{पुत्राश्च बहवस्तस्य बभूवुरमितौजसः} %॥७॥

\twolineshloka
{राक्षसाश्च तमाश्रित्य महाबलपराक्रमाः}
{अनेककोटयो राजन् लङ्कायां निवसन्ति ये} %॥८॥

\twolineshloka
{देवान् पितृन मनुष्यांश्च विद्याधरगणानपि}
{यक्षांश्चैव ततः सर्वे घातयन्ति दिवाशिनम्} %॥९॥

\twolineshloka
{सन्त्रस्तं तद्भयादेव जगदासीच्चराचरम्}
{दुःखाभिभूतमत्त्यर्थं सम्बभूव नराधिप} %॥१०॥

\twolineshloka
{एतस्मिन्ने व काले तु देवाः सेन्द्रा महर्षयः}
{सिद्धा विद्याधराश्चैव गन्धर्वाः किन्नरास्तथा} %॥११॥

\twolineshloka
{गुह्यका भुजगा यक्षा ये चान्ये स्वर्गवासिनः}
{ब्रह्माणमग्रतः कृत्वा शङ्करं च नराधिप} %॥१२॥

\twolineshloka
{ते ययुर्हतविक्रान्ताः क्षीराब्धेस्तटमुत्तमम्}
{तत्राराध्य हरिं देवतास्तस्थुः प्राञ्जलयस्तदा} %॥१३॥

\twolineshloka
{ब्रह्मा च विष्णुमाराध्य गन्धपुष्पादिभिः शुभैः}
{प्राञ्जलिः प्रणतो भूत्वा वासुदेवमथास्तुवत्} %॥१४॥

\uvacha{ब्रह्मोवाच}

\twolineshloka
{नमः क्षीराब्धिवासाय नागपर्यङ्कशायिने}
{नमः श्रीकरसंस्पृष्टदिव्यपादाय विष्णवे} %॥१५॥

\twolineshloka
{नमस्ते योगनिद्राय योगान्तर्भाविताय च}
{तार्क्ष्यासनाय देवाय गोविन्दाय नमो नमः} %॥१६॥

\twolineshloka
{नमः क्षीराब्धिकल्लोलस्पृष्टमात्राय शार्ङ्गिणे}
{नमोऽरविन्दपादाय पद्मनाभाय विष्णवे} %॥१७॥

\twolineshloka
{भक्तार्चितसुपादाय नमो योगाप्रियाय वै}
{शुभाङ्गाय सुनेत्राय माधवाय नमो नमः} %॥१८॥

\twolineshloka
{सुकेशाय सुनेत्राय सुललाटाय चक्रिणे}
{सुवक्त्राय सुकर्णाय श्रीधराय नमो नमः} %॥१९॥

\twolineshloka
{सुवक्षसे सुनाभाय पद्मनाभाय वै नमः}
{सुभ्रुवे चारुदेहाय चारुदन्ताय शार्ङ्गिणे} %॥२०॥

\twolineshloka
{चारुजङ्घाय दिव्याय केशवाय नमो नमः}
{सुनखाय सुशान्ताय सुविद्याय गदाभृते} %॥२१॥

\twolineshloka
{धर्माप्रियाय देवाय वामनाय नमो नमः}
{असुरघ्नाय चोग्राय रक्षोघ्नाय नमो नमः} %॥२२॥

\twolineshloka
{देवानामार्तिनाशाय भीमर्ककृते नमः}
{नमस्ते लोकनाथाय रावणान्तकृते नमः} %॥२३॥

\uvacha{मार्कण्डेय उवाच}

\twolineshloka
{इति स्तुतो हषीकेशस्तुतोष परमेष्ठिना}
{स्वरुपं दर्शयित्वा तु पितामहमुवाच ह} %॥२४॥

\twolineshloka
{किमर्थं तु सुरैः सार्धमागतस्त्वं पितामह}
{यत्कार्य ब्रूहि मे ब्रह्मन् यदर्थं संस्तुतस्त्वया} %॥२५॥

\twolineshloka
{इत्युक्तो देवदेवेन विष्णुना प्रभविष्णुना}
{सर्वदेवगणैः सार्धं ब्रह्मा प्राह जनार्दनम्} %॥२६॥

\uvacha{ब्रह्मोवाच}

\twolineshloka
{नाशितं तु जगत्सर्वं रावणेन दुरात्मना}
{सेन्द्राः पराजितास्तेन बहुशो रक्षसा विभो} %॥२७॥

\twolineshloka
{राक्षसैर्भक्षिता मर्त्या यज्ञाश्चापि विदूषिताः}
{देवकन्या हतास्तेन बलाच्छतसहस्त्रशः} %॥२८॥

\twolineshloka
{त्वामृते पुण्डरीकाक्ष रावणस्य वधं प्रति}
{न समर्था यतो देवास्त्वमतस्तद्वधं कुरु} %॥२९॥

\twolineshloka
{इत्युक्तो ब्रह्मणा विष्णुर्ब्रह्माणमिदमब्रवीत्}
{श्रृणुष्वावहितो ब्रह्मन् यद्वदामि हितं वचः} %॥३०॥

\twolineshloka
{सूर्यवंशोद्भवः श्रीमान् राजाऽऽसीद्भुवि वीर्यवान्}
{नाम्ना दशरथख्यातस्तस्य पुत्रो भवाम्यहम्} %॥३१॥

\twolineshloka
{रावणस्य वधार्थाय चतुर्धांशेन सत्तम}
{स्वांशैर्वानररुपेण सकला देवतागणाः} %॥३२॥

\twolineshloka
{वतार्यन्तां विश्वकर्तः स्यादेवं रावणक्षयः}
{इत्युक्तो देवदेवेन ब्रह्मा लोकपितामहः} %॥३३॥

\twolineshloka
{देवाश्च ते प्रणम्याथ मेरुपृष्ठं तदा ययुः}
{स्वांशैर्वानररुपेण अवतेरुश्च भूतले} %॥३४॥

\twolineshloka
{अथापुत्रो दशरथो मुनिभिर्वेदपारगैः}
{इष्टिं तु कारयामास पुत्रप्राप्तिकरी नृपः} %॥३५॥

\twolineshloka
{ततः सौवर्णपात्रस्थं हविरादाय पायसम्}
{वह्निः कुण्डात् समुत्तस्थौ नूनं देवेन नोदितः} %॥३६॥

\twolineshloka
{आदाय मुनयो मन्त्राच्चक्रुः पिण्डद्वयं शुभम्}
{दत्ते कौशल्यकैकेय्योर्द्वे पिण्डे मन्त्रमन्त्रिते} %॥३७॥

\twolineshloka
{ते पिण्डप्राशने काले सुमित्राया महामते}
{पिण्डाभ्यामल्पमल्पं तु सुभागिन्याः प्रयच्छतः} %॥३८॥

\twolineshloka
{ततस्ताः प्राशयामासू राजपत्न्यो यथाविधि}
{पिण्डान् देवकृतान् प्राश्य प्रापुर्गर्भाननिन्दितान्} %॥३९॥

\twolineshloka
{एवं विष्णुर्दशरथाज्जातस्तत्पत्निषु त्रिषु}
{स्वांशैर्लोकहितायैव चतुर्धा जगतीपते} %॥४०॥

\twolineshloka
{रामश्च लक्ष्मणश्चैव भरतः शत्रुघ्न एव च}
{जातकर्मादिकं प्राप्य संस्कारं मुनिसंस्कृतम्} %॥४१॥

\twolineshloka
{मन्त्रपिण्डवशाद्योगं प्राप्य चेरुर्यथार्भकाः}
{रामश्च लक्ष्मणश्चैव सह नित्यं विचेरतुः} %॥४२॥

\twolineshloka
{जन्मादिकृतसंस्कारौ पितुः प्रीतिकरौ नृप}
{ववृधाते महावीर्यौ श्रुतिशब्दातिलक्षणौ} %॥४३॥

\twolineshloka
{भरतः कैकयो राजन् भ्रात्रा सह गृहेऽवसत्}
{वेदशास्त्राणि बुबुधे शस्त्रशास्त्रं नृपोत्तम} %॥४४॥

\twolineshloka
{एतस्मिन्नेव काले तु विश्वामित्रो महातपाः}
{यागेन यष्टुमारेभे विधिना मधुसूदनम्} %॥४५॥

\twolineshloka
{स तु विघ्नेन यागोऽभूद्राक्षसैर्बहुशः पुरा}
{नेतुं स यागरक्षार्थं सम्प्रातो रामलक्ष्मणौ} %॥४६॥

\twolineshloka
{विश्वामित्रो नृपश्रेष्ठ तत्पितुर्मन्दिरं शुभम्}
{दशरथस्तु तं दृष्ट्वा प्रत्युत्थाय महामतिः} %॥४७॥

\twolineshloka
{अर्घ्यपाद्यादि विधिना विश्वामित्रमपूजयत्}
{स पूजितो मुनिः प्राह राजानं राजसन्निधौ} %॥४८॥

\twolineshloka
{श्रृणु राजन् दशरथ यदर्थमहमागतः}
{तत्कार्यं नृपशार्दूल कथयामि तवाग्रतः} %॥४९॥

\twolineshloka
{राक्षसैर्नाशितो यागो बहुशो मे दुरासदैः}
{यज्ञस्य रक्षणार्थं मे देहि त्वं रामक्ष्मणौ} %॥५०॥

\twolineshloka
{राजा दशरथः श्रुत्वा विश्वामित्रवचो नृप}
{विषण्णवदनो भूत्वा विश्वामित्रमुवाच ह} %॥५१॥

\twolineshloka
{बालाभ्यां मम पुत्राभ्यां किं ते कार्यं भविष्यति}
{अहं त्वया सहागत्य शक्त्या रक्षामि ते मखम्} %॥५२॥

\twolineshloka
{राज्ञस्तु वचनं श्रुत्वा राजानं मुनिरब्रवीत्}
{रामोऽपि शक्नुते नूनं सर्वान्नशयितुं नृप} %॥५३॥

\twolineshloka
{रामेणैव हि ते शक्या न त्वया राक्षसा नृप}
{अतो मे देहि रामं च न चिन्तां कर्तुमर्हसि} %॥५४॥

\twolineshloka
{इत्युक्तो मुनिना तेन विश्वामित्रेण धीमता}
{तूष्णीं स्थित्वा क्षणं राजा मुनिवर्यमुवाच ह} %॥५५॥

\twolineshloka
{यद्ववीमि मुनिश्रेष्ठ प्रसन्नस्त्वं निबोध मे}
{राजीवलोचनं राममहं दास्ये सहानुजम्} %॥५६॥

\twolineshloka
{किं त्वस्य जननी ब्रह्मन् अदृष्टैनं मरिष्यति}
{अतोऽहं चतुरङ्गेण बलेन सहितो मुने} %॥५७॥

\twolineshloka
{आगत्य राक्षसान् हन्मीत्येबं मे मनसि स्थितम्}
{विश्वामित्रः पुनः प्राह राजानममितौजसम्} %॥५८॥

\twolineshloka
{नाज्ञो रामो नृपश्रेष्ठ स सर्वज्ञः समः क्षमः}
{शेषनारायणावेतौ तव पुत्रौ न संशयः} %॥५९॥

\twolineshloka
{दुष्टानां निग्रहार्थाय शिष्टानां पालनाय च}
{अवतीर्णो न सन्देहो गृहे तव नराधिप} %॥६०॥

\twolineshloka
{न मात्रा न त्वया राजन् शोकः कार्योऽत्र चाण्वपि}
{निः क्षेपे च महाराज अर्पयिष्यामि ते सुतौ} %॥६१॥

\twolineshloka
{इत्युक्तो दशरथस्तेन विश्वामित्रेण धीमता}
{तच्छापभीतो मनसा नीयतामित्यभाषत्} %॥६२॥

\twolineshloka
{कृच्छ्रात्पित्रा विनिर्मुक्तं राममादाय सानुजम्}
{ततः सिद्धाश्रमं राजन् सम्प्रतस्थे स कौशिकः} %॥६३॥

\twolineshloka
{तं प्रस्थितमथालोक्य राजा दशरथस्तदा}
{अनुव्रज्याब्रवीदेतद् वचो दशरथस्तदा} %॥६४॥

\twolineshloka
{अपुत्रोऽहं पुरा ब्रह्मन् बहुभिः काम्यकर्मभिः}
{मुनिप्रसादादधुना पुत्रवानस्मि सत्तम} %॥६५॥

\twolineshloka
{मनसा तद्वियोगं तु न शक्ष्यामि विशेषतः}
{त्वमेव जानासि मुने नीत्वा शीघ्रं प्रयच्छ मे} %॥६६॥

\twolineshloka
{इत्येवमुक्तो राजानं विश्वामित्रोऽब्रवीत्पुनः}
{समाप्तयज्ञश्च पुनर्नेष्ये रामं च लक्ष्मणम्} %॥६७॥

\twolineshloka
{सत्यपूर्वं तु दास्यामि न चिन्तां कर्तुमर्हसि}
{इत्युक्तः प्रेषयामास रामं लक्ष्मणसंयुतम्} %॥६८॥

\twolineshloka
{अनिच्छन्नपि राजासौ मुनिशापभयान्नृपः}
{विश्वामित्रस्तु तौ गृह्य अयोध्याया ययौ शनैः} %॥६९॥

\twolineshloka
{सरय्वास्तीरमासाद्य गच्छन्नेव स कौशिकः}
{तयोः प्रीत्या स राजेन्द्र द्वे विद्ये प्रथमं ददौ} %॥७०॥

\twolineshloka
{बलामतिबलां चैव समन्त्रे च ससङ्ग्रहे}
{क्षुत्पिपासापनयने पुनश्चैव महामतिः} %॥७१॥

\twolineshloka
{अस्त्रग्राममशेषं तु शिक्षयित्वा तु तौ तदा}
{आश्रमाणि च दिव्यानि मुनीनां भावितात्मनाम्} %॥७२॥

\twolineshloka
{दर्शयित्वा उषित्वा च पुण्यस्थानेषु सत्तमः}
{गङ्गामुत्तीर्य शोणस्य तीरमासाद्य पश्चिमम्} %॥७३॥

\twolineshloka
{मुनिधार्मिकसिद्धांश्च पश्यन्तौ रामलक्ष्मणौ}
{ऋषिभ्यश्च वरान् प्राप्य तेन नीतौ नृपात्मजौ} %॥७४॥

\twolineshloka
{ताटकाया वनं घोरं मृत्योर्मुखमिवापरम्}
{गते तत्र नृपश्रेष्ठ विश्वामित्रो महातपाः} %॥७५॥

\twolineshloka
{राममक्लिष्टकर्माणमिदं वचनमब्रवीत्}
{राम राम महाबाहो ताटका नाम राक्षसी} %॥७६॥

\twolineshloka
{रावणस्य नियोगेन वसत्यस्मिन् महावने}
{तया मनुष्या बहवो मुनिपुत्रा मृगास्तथा} %॥७७॥

\twolineshloka
{निहता भक्षिताश्चैव तस्मात्तां वध सत्तम}
{इत्येवमुक्तो मुनिना रामस्तं मुनिमब्रवीत्} %॥७८॥

\twolineshloka
{कथं हि स्त्रीवधं कुर्यामहमद्य महामुने}
{स्त्रीवधे तु महापापं प्रवदन्ति मनीषिणः} %॥७९॥

\twolineshloka
{इति रामवचः श्रुत्वा विश्वामित्र उवाच तम्}
{तस्यास्तु निधनाद्राम जनाः सर्वे निराकुलाः} %॥८०॥

\twolineshloka
{भवन्ति सततं तस्मात् तस्याः पुण्यप्रदो वधः}
{इत्येवं वादिनि मुनौ विश्वामित्रे निशाचरी} %॥८१॥

\twolineshloka
{आगता सुमहाघोरा ताटका विवृतानना}
{मुनिना प्रेरितो रामस्तां दृष्ट्वा विवृताननाम्} %॥८२॥

\twolineshloka
{उद्यतैकभुजयष्टिमायतीं श्रोणिलम्बिपुरुषान्त्रमेखलाम्}
{तां विलोक्य वनितावधे घृणां पत्रिणा सह मुमोच राघवः} %॥८३॥

\twolineshloka
{शरं सन्धाय वेगेन तेन तस्या उरः स्थलम्}
{विपाटितं द्विधा राजन् सा पपात ममार च} %॥८४॥

\twolineshloka
{घातयित्वा तु तामेवं तावानीय मुनिस्तु तौ}
{प्रापयामास तं तत्र नानाऋषिनिषेवितम्} %॥८५॥

\twolineshloka
{नानाद्रुमलताकीर्णं नानापुष्पोपशोभितम्}
{नानानिर्झरतोयाढ्यं विन्ध्यशैलान्तरस्थितम्} %॥८६॥

\twolineshloka
{शकमूलफलोपेतं दिव्यं सिद्धाश्रमं स्वकम्}
{रक्षार्थं तावुभौ स्थाप्य शिक्षयित्वा विशेषतः} %॥८७॥

\twolineshloka
{ततश्चारब्धवान् यागं विश्वामित्रो महातपाः}
{दीक्षां प्रविष्टे च मुनौ विश्वामित्रे महात्मनि} %॥८८॥

\twolineshloka
{यज्ञे तु वितते तत्र कर्म कुर्वन्ति ऋत्विजः}
{मारीचश्च सुबाहुश्च बहवश्चान्यराक्षसाः} %॥८९॥

\twolineshloka
{आगता यागनाशाय रावणेन नियोजिताः}
{तानागतान् स विज्ञाय रामः कमललोचनः} %॥९०॥

\twolineshloka
{शरेण पातयामास सुबाहुं धरणीतले}
{असृक्प्रवाहं वर्षन्तं मारीचं भल्लकेन तु} %॥९१॥

\twolineshloka
{प्रताङ्य नीतवानब्धिं यथा पर्णं तु वायुना}
{शेषांस्तु हतवान् रामो लक्ष्मणश्च निशाचरान्} %॥९२॥

\twolineshloka
{रामेण रक्षितमखो विश्वामित्रो महायशाः}
{समाप्य यागं विधिवत् पूजयामास ऋत्विजान्} %॥९३॥

\twolineshloka
{सदस्यानपि सम्पूज्य यथार्हं च ह्यरिन्दम}
{रामं च लक्ष्मणं चैव पूजयामास भक्तितः} %॥९४॥

\twolineshloka
{ततो देवगणस्तुष्टो यज्ञभागेन सत्तम}
{ववर्ष पुष्पवर्षं तु रामदेवस्य मूर्धनि} %॥९५॥

\twolineshloka
{निवार्य राक्षसभयं कारयित्वा तु तन्मखम्}
{श्रुत्वा नानाकथाः पुण्या रामो भ्रातृसमन्वितः} %॥९६॥

\twolineshloka
{तेन नीतो विनीतात्मा अहल्या यत्र तिष्ठति}
{व्यभिचारान्महेन्द्रेण भर्त्रा शप्ता हि सा पुरा} %॥९७॥

\twolineshloka
{पाषाणभूता राजेन्द्र तस्य रामस्य दर्शनात्}
{अहल्या मुक्तशापा च जगाम गौतमं प्रति} %॥९८॥

\twolineshloka
{विश्वामित्रस्ततस्तत्र चिन्तयामास वै क्षणम्}
{कृतदारो मया नेयो रामः कमललोचनः} %॥९९॥

\twolineshloka
{इति सञ्चिन्त्य तौ गृह्य विश्वामित्रो महातपाः}
{शिष्यैः परिवृतोऽनेकैर्जगाम मिथिलां प्रति} %॥१००॥

\twolineshloka
{नानादेशादथायाता जनकस्य निवेशनम्}
{राजपुत्रा महावीर्याः पूर्वं सीताभिकाङ्क्षिणः} %॥१०१॥

\twolineshloka
{तान् दृष्ट्वा पूजयित्वा तु जनकश्च यथार्हतः}
{यत्सीतायाः समुत्पन्नं धनुर्माहेश्वरं महत्} %॥१०२॥

\twolineshloka
{अर्चितं गन्धमालाभी रम्यशोभासमन्विते}
{रङ्गे महति विस्तीर्णे स्थापयामास तद्धनुः} %॥१०३॥

\twolineshloka
{उवाच च नृपान् सर्वांस्तदोच्चैर्जनको नृपः}
{आकर्षणादिदं येन धनुर्भग्नं नृपात्मजाः} %॥१०४॥

\twolineshloka
{तस्येयं धर्मतो भार्या सीता सर्वाङ्गशोभना}
{इत्येवं श्राविते तेन जनकेन महात्मना} %॥१०५॥

\twolineshloka
{क्रमादादाय ते तत्तु सज्यीकर्तुमथाभवन्}
{धनुषा ताडिताः सर्वे क्रमात्तेन महीपते} %॥१०६॥

\twolineshloka
{विधूय पतिता राजन् विलजास्तत्र पार्थिवाः}
{तेषु भग्नेषु जनकस्तद्धनुस्त्र्यम्बकं नृप} %॥१०७॥

\twolineshloka
{संस्थाप्य स्थितवान् वीरो रामागमनकाङ्क्षया}
{विश्वामित्रस्ततः प्राप्तो मिथिलाधिपतेर्गृहम्} %॥१०८॥

\twolineshloka
{जनकोऽपि च तं दृष्टवा विश्वामित्रं गृहागतम्}
{रामलक्ष्मणसंयुक्तं शिष्यैश्चाभिगतं तदा} %॥१०९॥

\twolineshloka
{तं पूजयित्वा विधिवत्प्राज्ञं विप्रानुयायिनम्}
{रामं रघुपतिं चापि लावण्यादिगुणैर्युतम्} %॥११०॥

\twolineshloka
{शीलाचारगुणोपेतं लक्ष्मणं च महामतिम्}
{पूजयित्वा यथान्यायं जनकः प्रीतमानसः} %॥१११॥

\twolineshloka
{हेमपीठे सुखासीनं शिष्यैः पूर्वापरैर्वृतम्}
{विश्वामित्रमुवाचाथ किं कर्तव्यं मयेति सः} %॥११२॥

\uvacha{मार्कण्डेय उवाच}

\twolineshloka
{इति श्रुत्वा वचस्तस्य मुनिः प्राह महीपतिम्}
{एष रामो महाराज विष्णुः साक्षान्महीपतिः} %॥११३॥

\twolineshloka
{रक्षार्थं विष्टपानां तु जातो दशरथात्मजः}
{अस्मै सीतां प्रयच्छ त्वं देवकन्यामिव स्थिताम्} %॥११४॥

\twolineshloka
{अस्या विवाहे राजेन्द्र धनुर्भङ्गमुदीरितम्}
{तदानय भवधनुरर्चयस्व जनाधिप} %॥११५॥

\twolineshloka
{तथेत्युक्त्वा च राजा हि भवचापं तदद्भुतम्}
{अनेक भूभुजां भङ्गि स्थापयामास पूर्ववत्} %॥११६॥

\twolineshloka
{ततो दशरथसुतो विश्वामित्रेण चोदितः}
{तेषां मध्यात्समुत्थाय रामः कमललोचनः} %॥११७॥

\twolineshloka
{प्रणम्य विप्रान् देवांश्च धनुरादाय तत्तदा}
{सज्यं कृत्वा महाबाहुर्ज्याघोषमकरोत्तदा} %॥११८॥

\twolineshloka
{आकृष्यमाणं तु बलात्तेन भग्नं महद्धनुः}
{सीता च मालामादाय शुभां रामस्य मूर्धनि} %॥११९॥

\twolineshloka
{क्षिप्त्वा संवरयामास सर्वक्षत्रियसन्निधौ}
{ततस्ते क्षत्रियाः क्रुद्धा राममासाद्य सर्वतः} %॥१२०॥

\twolineshloka
{मुमुचुः शरजालानि गर्जयन्तो महाबलाः}
{तान्निरीक्ष्य ततो रामो धनुरादाय वेगवान्} %॥१२१॥

\twolineshloka
{ज्याघोषतलघोषेण कम्पयामास तान्नृपान्}
{चिच्छेद शरजालानि तेषां स्वास्त्रै रथांस्ततः} %॥१२२॥

\twolineshloka
{धनूंषि च पताकाश्च रामश्चिच्छेद लीलया}
{सन्नह्य स्वबलं सर्वं मिथिलाधिपतिस्ततः} %॥१२३॥

\twolineshloka
{जामातरं रणे रक्षन् पार्ष्णिग्राहो बभूव ह}
{लक्ष्मणश्च महावीरो विद्राव्य युधि तान्नृपान्} %॥१२४॥

\twolineshloka
{हस्त्यश्वाञ्जगृहे तेषां स्यन्दनानि बहूनि च}
{वाहनानि परित्यज्य पलायनपरान्नृपान्} %॥१२५॥

\twolineshloka
{तान्निहन्तुं च धावत्स पृष्ठतो लक्ष्मणस्तदा}
{मिथिलाधिपतिस्तं च वारयामास कौशिकः} %॥१२६॥

\twolineshloka
{जितसेनं महावीरं रामं भ्रात्रा समन्वितम्}
{आदाय प्रविवेशाथ जनकः स्वगृहं शुभम्} %॥१२७॥

\twolineshloka
{दूतं च प्रेषयामास तदा दशरथाय सः}
{श्रुत्वा दूतमुखात् सर्वं विदितार्थः स पार्थिवः} %॥१२८॥

\twolineshloka
{सभार्यः ससुतः श्रीमान् हस्त्यश्वरथवाहनः}
{मिथिलामाजगामाशु स्वबलेन समन्वितः} %॥१२९॥

\twolineshloka
{जनकोऽप्यस्य सत्कारं कृत्वा स्वां च सुतां ततः}
{विधिवत्कृतशुल्कां तां ददौ रामाय पार्थिव} %॥१३०॥

\twolineshloka
{अपराश्च सुतास्तिस्त्रो रुपवत्यः स्वलडकृताः}
{त्रिभ्यस्तु लक्ष्मणादिभ्यः स्वकन्या विधिवद्ददौ} %॥१३१॥

\twolineshloka
{एवं कृतविवाहोऽसौ रामः कमललोचनः}
{भ्रातृभिर्मातृभिः सार्धं पित्रा बलवता सह} %॥१३२॥

\threelineshloka
{दिनानि कतिचित्तत्र स्थितो विविधभोजनैः}
{ततोऽयोध्यापुरीं गन्तुमुत्सुकं ससुतं नृपम्}
{दृष्ट्वा दशरथं राजा सीतायाः प्रददौ वसु} %॥१३३॥

\fourlineindentedshloka
{रत्नानि दिव्यानि बहूनि दत्त्वा}
{रामाय वस्त्राण्यतिशोभनानि}
{हस्त्यश्वदासानपि कर्मयोग्यान्}
{दासीजनांश्च प्रवराः स्त्रियश्च} %॥१३४॥

\fourlineindentedshloka
{सीतां सुशीलां बहुरत्नभूषितां}
{रथं समारोप्य सुतां सुरुपाम्}
{वेदादिघोषैर्बहुमङ्गलैश्च}
{सम्प्रेषयामास स पार्थिवो बली} %॥१३५॥

\twolineshloka
{प्रेषयित्वा सुतां दिव्यां नत्वा दशरथं नृपम्}
{विश्वामित्रं नमस्कृत्य जनकः सन्निवृत्तवान्} %॥१३६॥

\twolineshloka
{तस्य पल्यो महाभागाः शिक्षयित्वा सुतां तदा}
{भर्तृभक्तिं कुरु शुभे श्वश्रूणां श्वशुरस्य च} %॥१३७॥

\twolineshloka
{श्वश्रूणामर्पयित्वा तां निवृत्ता विविशुः पुरम्}
{ततस्तु रामं गच्छन्तमयोध्यां प्रबलान्वितम्} %॥१३८॥

\twolineshloka
{श्रुत्वा परशुरामो वै पन्थानं संरुरोध ह}
{तं दृष्ट्वा राजपुरुषाः सर्वे ते दीनमानसाः} %॥१३९॥

\twolineshloka
{आसीद्दशरथश्चापि दुःखशोकपरिप्लुतः}
{सभार्यः सपरीवारो भार्गवस्य भयान्नृप} %॥१४०॥

\twolineshloka
{ततोऽब्रवीज्जनान् सर्वान् राजानं च सुदुः खितम्}
{वसिष्ठश्चोर्जिततपा ब्रह्मचारी महामुनिः} %॥१४१॥

\uvacha{वसिष्ठ उवाच}

\onelineshloka
{युष्माभिरत्र रामार्थं न कार्य दुःखमण्वपि} %॥१४२॥

\twolineshloka
{पित्रा वा मातृभिर्वापि अन्यैर्भृत्यजनैरपि}
{अयं हि नृपते रामः साक्षाद्विष्णुस्तु ते गृहे} %॥१४३॥

\twolineshloka
{जगतः पालनार्थाय जन्मप्राप्तो न संशयः}
{यस्य सकीर्त्य नामपि भवभीतिः प्रणश्चति} %॥१४४॥

\twolineshloka
{ब्रह्म मूर्तं स्वयं यत्र भयादेस्तत्र का कथा}
{यत्र सकीर्त्यते रामकथामात्रमपि प्रभो} %॥१४५॥

\twolineshloka
{नोपसर्गभयं तत्र नाकालमरणं नृणाम्}
{इत्युक्ते भार्गवो रामो राममाहाग्रतः स्थितम्} %॥१४६॥

\twolineshloka
{त्यज त्वं रामसज्ञां तु मया वा सगरं कुरु}
{इत्युक्ते राघवः प्राह भार्गवं तं पथि स्थितम्} %॥१४७॥

\twolineshloka
{रामसज्ञां कुतस्त्यक्ष्ये त्वया योत्स्ये स्थिरो भव}
{इत्युक्त्वा तं पृथक् स्थित्वा रामो राजीवलोचनः} %॥१४८॥

\twolineshloka
{ज्याघोषमकरोद्वीरो वीरस्यैवाग्रतस्तदा}
{ततः परशुरामस्य देहान्निष्क्रम्य वैष्णवम्} %॥१४९॥

\twolineshloka
{पश्यतां सर्वभूतानां तेजो राममुखेऽविशत्}
{दृष्ट्वा तं भार्गवो रामः प्रसन्नवदनोऽब्रवीत्} %॥१५०॥

\twolineshloka
{राम राम महाबाहो रामस्त्वं नात्र संशयः}
{विष्णुरेव भवाञ्जातो ज्ञातोऽस्यद्य मया विभो} %॥१५१॥

\twolineshloka
{गच्छ वीर यथाकामं देवकार्यं च वै कुरु}
{दुष्टानां निधनं कृत्वा शिष्टांश्च परिपालय} %॥१५२॥

\twolineshloka
{याहि त्वं स्वेच्छया राम अहं गच्छे तपोवनम्}
{इत्युक्त्वा पूजितस्तैस्तु मुनिभावेन भार्गवः} %॥१५३॥

\twolineshloka
{महेन्द्राद्रिं जगामाथ तपसे धृतमानसः}
{ततस्तु जातहर्षास्ते जना दशरथश्च ह} %॥१५४॥

\twolineshloka
{पुरीमयोध्यां सम्प्राप्य रामेण सह पार्थिवः}
{दिव्यशोभां पुरीं कृत्वा सर्वतो भद्रशालिनीम्} %॥१५५॥

\twolineshloka
{प्रत्युत्थाय ततः पौराः शङ्खतूर्यादिभिः स्वनैः}
{विशन्तं राममागत्य कृतदारं रणेऽजितम्} %॥१५६॥

\twolineshloka
{तं वीक्ष्य हर्षिताः सन्तो विविशुस्तेन वै पुरीम्}
{तौ दृष्ट्वा स मुनिः प्राप्तौ रामं लक्ष्मणमन्तिके} %॥१५७॥

\threelineshloka
{दशरथाय तत्पित्रे मातृभ्यश्च विशेषतः}
{तौ समर्प्य मुनिश्रेष्ठस्तेन राज्ञा च पूजितः}
{विश्वामित्रश्च सहसा प्रतिगन्तुं मनो दधे} %॥१५८॥

\fourlineindentedshloka
{समर्प्य राम स मुनिः सहानुजं}
{सभार्यमग्ने पितुरेकवल्लभम्}
{पुनः पुनः श्राव्य हसन्महामतिर्-}
{जगाम सिद्धाश्रममेवमात्मनः} %॥१५९॥

॥इति श्रीनरसिंहपुराणे रामप्रादुर्भावे सप्तचत्वारिंशोऽध्यायः॥४७॥

\sect{अष्टचत्वारिंशोऽध्यायः --- अयोध्या-काण्डः}

\uvacha{मार्कण्डेय उवाच}

\twolineshloka
{कृतदारो महातेजा रामः कमललोचनः}
{पित्रे सुमहतीं प्रीतिं जनानामुपपादयन्} %॥१॥

\twolineshloka
{अयोध्यायां स्थितो रामः सर्वभोगसमन्वितः}
{प्रीत्या नन्दत्ययोध्यायां रामे रघुपतौ नृप} %॥२॥

\twolineshloka
{भ्राता शत्रुघ्नसहितो भरतो मातुलं ययौ}
{ततो दशरथो राजा प्रसमीक्ष्य सुशोभनम्} %॥३॥

\twolineshloka
{युवानं बलिनं योग्यं भूपसिद्ध्यै सुतं कविम्}
{अभिषिच्य राज्यभारं रामे संस्थाप्य वैष्णवम्} %॥४॥

\twolineshloka
{पदं प्राप्तुं महद्यत्नं करिष्यामीत्यचिन्तयत्}
{सचिन्त्य तत्परो राजा सर्वदिक्षु समादिशत्} %॥५॥

\twolineshloka
{प्राज्ञान् भृत्यान महीपालान्मन्त्रिणश्च त्वरान्वितः}
{रामाभिषेकद्रव्याणि ऋषिप्रोक्तानि यानि वै} %॥६॥

\twolineshloka
{तानि भृत्याः समाहत्य शीघ्रमागन्तुमर्हथ}
{दूतामात्याः समादेशात्सर्वदिक्षु नराधिपान्} %॥७॥

\twolineshloka
{आहूय तान् समाहत्य शीघ्रमागन्तुमर्हथ}
{अयोध्यापुरमत्यर्थं सर्वशोभासमन्वितम्} %॥८॥

\twolineshloka
{जनाः कुरुत सर्वत्र नृत्यगीतादिनन्दितम्}
{पुरवासिजनानन्दं देशवासिमनः प्रियम्} %॥९॥

\twolineshloka
{रामाभिषेकं विपुलं श्वो भविष्यति जानथ}
{श्रुत्वेत्थं मन्त्रिणः प्राहुस्तं नृपं प्रणिपत्य च} %॥१०॥

\twolineshloka
{शोभनं ते मतं राजन् यदिदं परिभाषितम्}
{रामाभिषेकमस्माकं सर्वेषां च प्रियकरम्} %॥११॥

\twolineshloka
{इत्युक्तो दशरथस्तैस्तान् सर्वान् पुनरब्रवीत्}
{आनीयन्तां द्रुतं सर्वे सम्भारा मम शासनात्} %॥१२॥

\twolineshloka
{सर्वतः सारभूता च पुरी चेयं समन्ततः}
{अद्य शोभान्विता कार्या कर्तव्यं यागमण्डलम्} %॥१३॥

\twolineshloka
{इत्येवमुक्ता राज्ञा ते मन्त्रिणः शीघ्रकारिणः}
{तथैव चक्रुस्ते सर्वे पुनः पुनरुदीरिताः} %॥१४॥

\twolineshloka
{प्राप्तहर्षः स राजा च शुभं दिनमुदीक्षयन्}
{कौशल्या लक्ष्मणश्चैव सुमित्रा नागरो जनः} %॥१५॥

\twolineshloka
{रामाभिषेकमाकर्ण्य मुदं प्राप्यातिहर्षितः}
{श्वश्रूश्वशुरयोः सम्यक् शुश्रूषपणपरा तु सा} %॥१६॥

\twolineshloka
{मुदान्विता सिता सीता भर्तुराकर्ण्य शोभनम्}
{श्वोभाविन्यभिषेके तु रामस्य विदितात्मनः} %॥१७॥

\twolineshloka
{दासी तु मन्थरानाम्नी कैकेय्याः कुब्जरुपिणी}
{स्वां स्वामिनीं तु कैकेयीमिदं वचनमब्रवीत्} %॥१८॥

\twolineshloka
{श्रृणु राज्ञि महाभागे वचनं मम शोभनम्}
{त्वत्पतिस्तु महाराजस्तव नाशाय चोद्यतः} %॥१९॥

\twolineshloka
{रामोऽसौ कौसलीपुत्रः श्वो भविष्यति भूपतिः}
{वसुवाहनकोशादि राज्यं च सकलं शुभे} %॥२०॥

\twolineshloka
{भविष्यत्यद्य रामस्य भरतस्य न किचन}
{भरतोऽपि गतो दूरं मातुलस्य गृहं प्रति} %॥२१॥

\twolineshloka
{हा कष्टं मन्दभाग्यासि सापल्याद्दुःखिता भृशम्}
{सैवमाकर्ण्य कैकेयी कुब्जामिदमथाब्रवीत्} %॥२२॥

\twolineshloka
{पश्य मे दक्षतां कुब्जे अद्यैव त्वं विचक्षणे}
{यथा तु सकलं राज्यं भरतस्य भविष्यति} %॥२३॥

\twolineshloka
{रामस्य वनवासश्च तथा यत्नं करोम्यहम्}
{इत्युक्त्वा मन्थरां सा तु उन्मुच्य स्वाङ्गभूषणम्} %॥२४॥

\twolineshloka
{वस्त्रं पुष्पाणि चोन्मुच्य स्थूलवासोधराभवत्}
{निर्माल्यपुष्पधृक्कष्टा कश्मलाङ्गी विरुपिणी} %॥२५॥

\twolineshloka
{भस्मधूल्यादिनिर्दिग्धा भस्मधूल्या तथा श्रिते}
{भूभागे शान्तदीपे सा सन्ध्याकाले सुदुःखिता} %॥२६॥

\twolineshloka
{ललाटे श्वेतचैलं तु बद्ध्वा सुष्वाप भामिनी}
{मन्त्रिभिः सह कार्याणि सम्मन्त्र्य सकलानि तु} %॥२७॥

\twolineshloka
{पुण्याहः स्वस्तिमाङ्गल्यैः स्थाप्य रामं तु मण्डले}
{ऋषिभस्तु वसिष्ठाद्यैः सार्धं सम्भारमण्डपे} %॥२८॥

\twolineshloka
{वृद्धिजागरणीयैश्च सर्वतस्तूर्यनादिते}
{गीतनृत्यसमाकीर्णे शङ्खकाहलनिः स्वनैः} %॥२९॥

\twolineshloka
{स्वयं दशरथस्तत्र स्थित्वा प्रत्यागतः पुनः}
{कैकेया वेश्मनो द्वारं जरद्भिः परिरक्षितम्} %॥३०॥

\twolineshloka
{रामाभिषेकं कैकेयीं वक्तुकामः स पार्थिवः}
{कैकेयीभवनं वीक्ष्य सान्धकारमथाब्रवीत्} %॥३१॥

\twolineshloka
{अन्धकारमिदं कस्मादद्य ते मन्दिरे प्रिये}
{रामाभिषेकं हर्षाय अन्त्यजा अपि मेनिरे} %॥३२॥

\twolineshloka
{गृहालकरणं कुर्वन्त्यद्य लोका मनोहरम्}
{त्वयाद्य न कृतं कस्मादित्युक्त्वा च महीपतिः} %॥३३॥

\twolineshloka
{ज्वालायित्वा गृहे दीपान् प्रविवेश गृहं नृपः}
{अशोभनाङ्गीं कैकेयीं स्वपन्तीं पतितां भुवि} %॥३४॥

\twolineshloka
{दृष्ट्वा दशरथः प्राह तस्याः प्रियमिदं त्विति}
{आश्लिष्योत्थाय तां राजा श्रृणु मे परमं वचः} %॥३५॥

\twolineshloka
{स्वमातुरधिकां नित्यं यस्ते भक्तिं करोति वै}
{तस्याभिषेकं रामस्य श्वो भविष्यति शोभने} %॥३६॥

\twolineshloka
{इत्युक्ता पार्थिवेनापि किचिन्नोवाच सा शुभा}
{मुञ्चन्ती दीर्घमुष्णं च रोषोस्च्छ्वासं मुहुर्मुहुः} %॥३७॥

\twolineshloka
{तस्थावाश्लिष्य हस्ताभ्यां पार्थिवः प्राह रोषिताम्}
{किं ते कैकेयि दुःखस्य कारणं वद शोभने} %॥३८॥

\twolineshloka
{वस्त्राभरणरत्नादि यद्यदिच्छसि शोभने}
{तत्त्वं गृह्णीष्व निश्शङ्कं भाण्डारात् सुखिनी भव} %॥३९॥

\twolineshloka
{भाण्डारेण मम शुभे श्वोऽर्थसिद्धिर्भविष्यति}
{यदाभिषेकं सम्प्राप्ते रामे राजीवलोचने} %॥४०॥

\twolineshloka
{भाण्डागारस्य मे द्वारं मया मुक्तं निरर्गलम्}
{भविष्यति पुनः पूर्णं रामे राज्यं प्रशासति} %॥४१॥

\twolineshloka
{बहु मानय रामस्य अभिषेकं महात्मनः}
{इत्युक्ता राजवर्य्येण कैकेयी पापलक्षणा} %॥४२॥

\twolineshloka
{कुमतिर्नर्घुणा दुष्टा कुब्जया शिक्षिताब्रवीत्}
{राजानं स्वपतिं वाक्यं क्रूरमत्यन्तनिष्ठुरम्} %॥४३॥

\twolineshloka
{रत्नादि सकलं यत्ते तन्ममैव न संशयः}
{देवासुरमहायुद्धे प्रीत्या यन्मे वरद्वयम्} %॥४४॥

\twolineshloka
{पुरा दत्तं त्वया राजंस्तदिदानीं प्रयच्छ मे}
{इत्युक्तः पार्थिवः प्राह कैकेयीमशुभां तदा} %॥४५॥

\twolineshloka
{अदत्तमप्यहं दास्ये तव नान्यस्य वा शुभे}
{किं मे प्रतिश्रुतं पूर्वं दत्तमेव मया तव} %॥४६॥

\twolineshloka
{शुभाङ्गी भव कल्याणि त्यज कोपमनर्थकम्}
{रामाभिषेकजं हर्षं भजोत्तिष्ठ सुखी भव} %॥४७॥

\twolineshloka
{इत्युक्ता राजवर्येण कैकेयी कलहप्रिया}
{उवाच परुषं वाक्यं राज्ञो मरणकारणम्} %॥४८॥

\twolineshloka
{वरद्वयं पूर्वदत्तं यदि दास्यसि मे विभो}
{श्वोभूते गच्छतु वनं रामोऽयं कोशलात्मजः} %॥४९॥

\twolineshloka
{द्वादशाब्दं निवसतु त्वद्वाक्याद्दण्डके वने}
{अभिषेकं च राज्यं च भरतस्य भविष्यति} %॥५०॥

\twolineshloka
{इत्याकर्ण्य स कैकेया वचनं घोरमप्रियम्}
{पपात भुवि निस्सज्ञो राजा सापि विभूषिता} %॥५१॥

\twolineshloka
{रात्रिशेषं नयित्वा तु प्रभाते सा मुदावती}
{दूतं सुमन्त्रमाहैवं राम आनीयतामिति} %॥५२॥

\twolineshloka
{रामस्तु कृतपुण्याहः कृतस्वस्त्ययनो द्विजैः}
{यागमण्डपमध्यस्थः शङ्खतूर्यरवान्वितः} %॥५३॥

\twolineshloka
{तमासाद्य ततो दूतः प्रणिपत्य पुरः स्थितः}
{राम राम महाबाहो आज्ञापयति ते पिता} %॥५४॥

\twolineshloka
{द्रुतमुत्तिष्ठ गच्छ त्वं यत्र तिष्ठति ते पिता}
{इत्युक्तस्तेन दूतेन शीघ्रमुत्थाय राघवः} %॥५५॥

\twolineshloka
{अनुज्ञाप्य द्विजान् प्राप्तः कैकेय्या भवनं प्रति}
{प्रविशन्तं गृहं रामं कैकेयी प्राह निर्घृणा} %॥५६॥

\twolineshloka
{पितुस्तव मतं वत्स इदं ते प्रब्रवीम्यहम्}
{वने वस महाबाहो गत्वा त्वं द्वादशाब्दकम्} %॥५७॥

\twolineshloka
{अद्यैव गम्यतां वीर तपसे धृतमानसः}
{न चिन्त्यमन्यथा वत्स आदरात् कुरु मे वचः} %॥५८॥

\twolineshloka
{एतच्छुत्वा पितुर्वाक्यं रामः कमललोचनः}
{तथेत्याज्ञां गृहीत्वासौ नमस्कृत्य च तावुभौ} %॥५९॥

\twolineshloka
{निष्क्रम्य तदगृहाद्रामो धनुरादाय वेश्मतः}
{कौशल्यां च नमस्कृत्य सुमित्रां गन्तुमुद्यतः} %॥६०॥

\twolineshloka
{तच्छुत्वा तु ततः पौरा दुःखशोकपरिप्लुताः}
{विव्यथुश्चाथ सौमित्रिः कैकेयीं प्रति रोषितः} %॥६१॥

\twolineshloka
{ततस्तं राघवो दृष्ट्वा लक्ष्मणं रक्तलोचनम्}
{बारयामास धर्मज्ञो धर्मवाग्भिर्महामतिः} %॥६२॥

\twolineshloka
{ततस्तु तत्र ये वृद्धास्तान प्रणम्य मुनींश्च सः}
{रामो रथं खिन्नसूतं प्रस्थानायारुरोह वै} %॥६३॥

\twolineshloka
{आत्मीयं सकलं द्रव्यं ब्राह्मणेभ्यो नृपात्मजः}
{श्रद्धया परया दत्त्वा वस्त्राणि विविधानि च} %॥६४॥

\twolineshloka
{तिस्त्रः श्वश्रूः समामन्त्र्य श्वशुरं च विसज्ञितम्}
{मुञ्चन्तमश्रुधाराणि नेत्रयोः शोकजानि च} %॥६५॥

\twolineshloka
{पश्यती सर्वतः सीता चारुरोह तथा रथम्}
{रथमारुह्य गच्छन्तं सीतया सह राघवम्} %॥६६॥

\twolineshloka
{दृष्ट्वा सुमित्रा वचनं लक्ष्मणं चाह दुःखिता}
{रामं दशरथं विद्धि मां विद्धि जनकात्मजाम्} %॥६७॥

\twolineshloka
{अयोध्यामटर्वी विद्धि व्रज ताभ्यां गुणाकर}
{मात्रैवमुक्तो धर्मात्मा स्तनक्षीरार्द्रदेहया} %॥६८॥

\twolineshloka
{तां नत्वा चारुयानं तमारुरोह स लक्ष्मनः}
{गच्छतो लक्ष्मणो भ्राता सीता चैव पतिव्रताः} %॥६९॥

\twolineshloka
{रामस्य पृष्ठतो यातौ पुराद्धीरौ महामते}
{विधिच्छिन्नाभिषेकं तं रामं राजीवलोचनम्} %॥७०॥

\twolineshloka
{अयोध्याया विनिष्क्रान्तमनुयाताः पुरोहिताः}
{मन्त्रिणः पौरमुख्याश्च दुःखेन महतान्विताः} %॥७१॥

\twolineshloka
{तं च प्राप्य हि गच्छन्तं राममूचुरिदं वचः}
{राम राम महाबाहो गन्तुं नार्हसि शोभन} %॥७२॥

\twolineshloka
{राजन्नत्र निवर्तस्व विहायास्मान् क्व गच्छसि}
{इत्युक्तो राघवस्तैस्तु तानुवाच दृढव्रतः} %॥७३॥

\twolineshloka
{गच्छध्वं मन्त्रिणः पौरा गच्छध्वं च पुरोधसः}
{पित्रादेशं मया कार्यमभियास्यामि वै वनम्} %॥७४॥

\twolineshloka
{द्वादशाब्दं व्रतं चैतन्नीत्वाहं दण्डके वने}
{आगच्छामि पितुः पादं मातृणां द्रष्टुमञ्जसा} %॥७५॥

\twolineshloka
{इत्युक्त्वा ताञ्जगामाथ रामः सत्यपरायणः}
{तं गच्छन्तं पुनर्याताः पृष्ठतो दुःखिता जनाः} %॥७६॥

\twolineshloka
{पुनः प्राह स काकुत्स्थो गच्छध्वं नगरीमिमाम्}
{मातृश्च पितरं चैव शत्रुघ्नं नगरीमिमाम्} %॥७७॥

\twolineshloka
{प्रजाः समस्तास्त्रत्रस्था राज्यं भरतमेव च}
{पालयध्वं महाभागास्तपसे याम्यहं वनम्} %॥७८॥

\twolineshloka
{अथ लक्ष्मणमाहेदं वचनं राघवस्तदा}
{सीतामर्पय राजानं जनकं मिथिलेश्वरम्} %॥७९॥

\twolineshloka
{पितृमातृवशे तिष्ठ गच्छ लक्ष्मण याम्यहम्}
{इत्युक्तः प्राह धर्मात्मा लक्ष्मणो भ्रातृवत्सलः} %॥८०॥

\twolineshloka
{मैवामाज्ञापाय विभो मामद्य करुणाकर}
{गन्तुमिच्छसि यत्र त्वमवश्यं तत्र याम्यहम्} %॥८१॥

\twolineshloka
{इत्युक्तो लक्ष्मणेनासौ सीतां तामाह राघवः}
{सीते गच्छ ममादेशात् पितरं प्रति शोभने} %॥८२॥

\twolineshloka
{सुमित्राया गृहे चापि कौशल्यायाः सुमध्यमे}
{निवर्तस्व हि तावत्त्वं यावदागमनं मम} %॥८३॥

\twolineshloka
{इत्युक्ता राघवेनापि सीता प्राह कृताञ्जलिः}
{यत्र गत्वा वने वासं त्वं करोषि महाभुज} %॥८४॥

\twolineshloka
{तत्र गत्वा त्वया सार्धं वसाम्यहमरिन्दम}
{वियोगं नो सहे राजंस्त्वया सत्यवता क्वचित्} %॥८५॥

\twolineshloka
{अतस्त्वां प्रार्थयिष्यामि दयां कुरु मम प्रभो}
{गन्तुमिच्छसि यत्र त्वमवश्यं तत्र याम्यहम्} %॥८६॥

\twolineshloka
{नानायानैरुपगताञ्जनान् वीक्ष्य स पृष्ठतः}
{योषितां च गणान् रामो वारयामास धर्मवित्} %॥८७॥

\twolineshloka
{निवृत्त्य स्थीयतां स्वैरमयोध्यायां जनाः स्त्रियः}
{गत्वाहं दण्डकारण्यं तपसे धृतमानसः} %॥८८॥

\twolineshloka
{कतिपयाब्दादायास्ये नान्यथा सत्यमीरितम्}
{लक्ष्मणेन सह भ्रात्रा वैदेह्या च स्वभार्यया} %॥८९॥

\twolineshloka
{जनान्निवर्त्य रामोऽसौ जगाम च गुहाश्रमम्}
{गुहस्तु रामभक्तोऽसौ स्वभावादेव वैष्णवः} %॥९०॥

\twolineshloka
{कृताञ्जलिपुटो भूत्वा किं कर्तव्यमिति स्थितः}
{महता तपसाऽऽनीता गुरुणा या हि वः पुरा} %॥९१॥

\twolineshloka
{भगीरथेन या भूमिं सर्वपापहरा शुभा}
{नानामुनिजनैर्जुष्टा कूर्ममत्स्यसमाकुला} %॥९२॥

\twolineshloka
{गङ्गा तुङ्गोर्मिमालाढ्या स्फटिकाभजलावहा}
{गुहोपनीतनावा तु तां गङ्गां स महाद्युतिः} %॥९३॥

\twolineshloka
{उत्तीर्य भगवान् रामो भरद्वाजाश्रमं शुभम्}
{प्रयागे तु ततस्तस्मिन् स्त्रात्वा तीर्थे यथाविधि} %॥९४॥

\twolineshloka
{लक्ष्मणेन सह भ्रात्रा राघवः सीतया सह}
{भरद्वाजाश्रमे तत्र विश्रान्तस्तेन पूजितः} %॥९५॥

\twolineshloka
{ततः प्रभाते विमले तमनुज्ञाप्य राघवः}
{भरद्वाजोक्तमार्गेण चित्रकूटं शनैर्ययौ} %॥९६॥

\twolineshloka
{नानाद्रुमलताकीर्णं पुण्यतीर्थमनुत्तमम्}
{तापसं वेषमास्थाय जह्नुकन्यामतीत्य वै} %॥९७॥

\twolineshloka
{गते रामे सभार्ये तु सह भ्रात्रा ससारथौ}
{अयोध्यामवसन् भूप नष्टशोभां सुदुःखिताः} %॥९८॥

\twolineshloka
{नष्टसज्ञो दशरथः श्रुत्वा वचनमप्रियम्}
{रामप्रवासजननं कैकेय्या मुखनिस्सृतम्} %॥९९॥

\twolineshloka
{लब्धसज्ञः क्षणाद्राजा रामरामेति चुक्रुशे}
{कैकेय्युवाच भूपालं भरतं चाभिषेचय} %॥१००॥

\twolineshloka
{सीतालक्ष्मणसंयुक्तो रामचन्द्रो वनं गतः}
{पुत्रशोकाभिसन्तप्तो राजा दशरथस्तदा} %॥१०१॥

\twolineshloka
{विहाय देहं दुःखेन देवलोकं गतस्तदा}
{ततस्तस्य महापुर्य्यामयोध्यायामरिन्दम} %॥१०२॥

\twolineshloka
{रुरुदुर्दुःखशोकार्त्ता जनाः सर्वे च योषितः}
{कौशल्या च सुमित्रा च कैकेयी कष्टकारिणी} %॥१०३॥

\twolineshloka
{परिवार्य मृतं तत्र रुरुदुस्ताः पतिं ततः}
{ततः पुरोहितस्तत्र वसिष्ठः सर्वधर्मवित्} %॥१०४॥

\twolineshloka
{तैलद्रोण्यां विनिक्षिप्य मृतं राजकलेवरम्}
{दूत वैं प्रेषयामास सहमन्त्रिगणैः स्थितः} %॥१०५॥

\twolineshloka
{स गत्वा यत्र भरतः शत्रुघ्नेन सह स्थितः}
{तत्र प्राप्य तथा वार्तां सन्निवर्त्य नृपात्मजौ} %॥१०६॥

\twolineshloka
{तावानीय ततः शीघ्रमयोध्यां पुनरागतः}
{क्रूराणि दृष्ट्वा भरतो निमित्तानि च वै पथि} %॥१०७॥

\twolineshloka
{विपरीतं त्वयोध्यामिति मेने स पार्थिवः}
{निश्शोभां निर्गतश्रीकां दुःखशोकान्वितां पुरीम्} %॥१०८॥

\twolineshloka
{कैकेय्याग्निविनिर्दग्धामयोध्यां प्रविवेश सः}
{दुःखान्विता जनाः सर्वे तौ दृष्ट्वा रुरुदुर्भृशम्} %॥१०९॥

\twolineshloka
{हा तात राम हा सीते लक्ष्मणेति पुनः पुनः}
{रुरोद भरतस्तत्र शत्रुघ्नश्च सुदुःखितः} %॥११०॥

\twolineshloka
{कैकेय्यास्तत्क्षणाच्छुत्वा चुक्रोध भरतस्तदा}
{दुष्टा त्वं दुष्टचित्ता च यया रामः प्रवासितः} %॥१११॥

\twolineshloka
{लक्ष्मणेन सह भ्रात्रा राघवः सीतया वनम्}
{साहसं किं कृतं दुष्टे त्वया सद्यो‍ऽल्पभाग्यया} %॥११२॥

\twolineshloka
{उद्वास्य सीतया रामं लक्ष्मणेन महात्मना}
{ममैव पुत्रं राजानं करोत्विति मतिस्तव} %॥११३॥

\twolineshloka
{दुष्टाया नष्टभाग्यायाः पुत्रोऽहं भाग्यवर्जितः}
{भ्रात्रा रामेण रहितो नाहं राज्यं करोमि वै} %॥११४॥

\twolineshloka
{यत्र रामो नरव्याध्रः पद्यपत्रायतेक्षणः}
{धर्मज्ञः सर्वशास्त्राज्ञो मतिमान् बन्धुवत्सलः} %॥११५॥

\twolineshloka
{सीता च यत्र वैदेही नियमव्रतचारिणी}
{पतिव्रता महाभागा सर्वलक्षणसंयुता} %॥११६॥

\twolineshloka
{लक्ष्मणश्च महावीर्यो गुणवान् भ्रातृवत्सलः}
{तत्र यास्यामि कैकेयि महत्पापं त्वया कृतम्} %॥११७॥

\twolineshloka
{राम एव मम भ्राता ज्येष्ठो मतिमतां वरः}
{स एव राजा दुष्टात्मे भृत्यो‍ऽहं तस्य वै सदा} %॥११८॥

\twolineshloka
{इत्युक्त्वा मातरं तत्र रुरोद भृशदुःखितः}
{हा राजन् पृथिवीपाल मां विहाय सुदुःखितम्} %॥११९॥

\twolineshloka
{क्व गतोऽस्यद्य वै तात किं करोमीह तद्वद}
{भ्राता पित्रा समः क्वास्ते ज्येष्ठो मे करुणाकरः} %॥१२०॥

\twolineshloka
{सीता च मातृतुल्या मे क्व गतो लक्ष्मणश्चह}
{इत्येवं विलपन्तं तं भरतं मन्त्रिभिः सह} %॥१२१॥

\twolineshloka
{वसिष्ठो भगवानाह कालकर्मविभागवित्}
{उत्तिष्ठोत्तिष्ठ वत्स त्वं न शोकं कर्तुमर्हसि} %॥१२२॥

\twolineshloka
{कर्मकालवशादेव पिता ते स्वर्गमास्थितः}
{तस्य संस्कारकार्याणि कर्माणि कुरु शोभन} %॥१२३॥

\twolineshloka
{रामोऽपि दुष्टनाशाय शिष्टानां पालनाय च}
{अवतीर्णो जगत्स्वामी स्वांशेन भुवि माधवः} %॥१२४॥

\twolineshloka
{प्रायस्तत्रास्ति रामेण कर्तव्यं लक्ष्मणेन च}
{यत्रासौ भगवान् वीरः कर्मणा तेन चोदितः} %॥१२५॥

\twolineshloka
{तत्कृत्वा पुनरायाति रामः कमललोचनः}
{इत्युक्तो भरतस्तेन वसिष्ठेन महात्मना} %॥१२६॥

\twolineshloka
{संस्कारं लम्भयामास विधिदृष्टेन कर्मणा}
{अग्निहोत्राग्निना दग्ध्वा पितुर्देहं विधानतः} %॥१२७॥

\twolineshloka
{स्नात्वा सरय्वाः सलिले कृत्वा तस्योदकक्रियाम्}
{शत्रुघ्नेन सह श्रीमान्तातृभिर्बान्धवैः सह} %॥१२८॥

\twolineshloka
{तस्यौर्ध्वदेहिकं कृत्वा मन्त्रिणा मन्त्रिनायकः}
{हस्त्यश्वरथपत्तीभिः सह प्रायान्महामतिः} %॥१२९॥

\twolineshloka
{भरतो राममन्वेष्टुं राममार्गेण सत्तमः}
{तमायान्तं महासेनं रामस्यानुविरोधिनम्} %॥१३०॥

\twolineshloka
{मत्वा तं भरतं शत्रुं रामभक्तो गुहस्तदा}
{स्वं सैन्यं वर्तुलं कृत्वा सन्नद्धः कवची रथी} %॥१३१॥

\onelineshloka
{महाबलपरीवारो रुरोध भरतं पथि} %॥१३२॥

\twolineshloka
{सभ्रातृकं सभा र्यं मे राम स्वामिनमुत्तमम्}
{प्रापयस्त्वं वनं दुष्टं साम्प्रतं हन्तुमिच्छसि} %॥१३३॥

\twolineshloka
{गमिष्यसि दुरात्मंस्त्वं सेनया सह दुर्मते}
{इत्युक्तो भरतस्तत्र गुहेन नृपनन्दनः} %॥१३४॥

\twolineshloka
{तमुवाच विनीतात्मा रामायाथ कृताञ्जलिः}
{यथा त्वं रामभक्तोऽमि तथाहमपि भक्तिमान्} %॥१३५॥

\twolineshloka
{प्रोषिते मयि कैकेय्या कृतमेतन्महामते}
{रामस्यानयनार्थाय व्रजाम्यद्य महामते} %॥१३६॥

\twolineshloka
{सत्यपूर्वं गमिष्यामि पन्थानं देहि मे गुह}
{इति विश्वासमानीय जाह्नवीं तेन तारितः} %॥१३७॥

\twolineshloka
{नौकावृन्दैरनेकैस्तु स्त्रात्वासौ जाह्नवीजले}
{भरद्वाजाश्रमं प्राप्तो भरतस्तं महामुनिम्} %॥१३८॥

\twolineshloka
{प्रणम्य शिरसा तस्मै यथावृत्तमुवाच ह}
{भरद्वाजोऽपि तं प्राह कालेन कृतमीदृशम्} %॥१३९॥

\twolineshloka
{दुःखं न तावत् कर्तव्यं रामार्थेऽपि त्वयाधुना}
{वर्तते चित्रकूटेऽसौ रामः सत्यपराक्रमः} %॥१४०॥

\twolineshloka
{त्वयि तत्र गते वापि प्रायोऽसौ नागमिष्यति}
{तथापि तत्र गच्छ त्वं यदसौ वक्ति तत्कुरु} %॥१४१॥

\twolineshloka
{रामस्तु सीतया सार्धं वनखण्डे स्थितः शुभे}
{लक्ष्मणस्तु महावीर्यो दुष्टालोकनतत्परः} %॥१४२॥

\twolineshloka
{इत्युक्तो भरतस्तत्र भरद्वाजेन धीमता}
{उत्तीर्य यमुनां यातश्चित्रकूटं महानगम्} %॥१४३॥

\twolineshloka
{स्थितोऽसौ दृष्टवान्दूरात्सधूलीं चोत्तरां दिशम्}
{रामाय कथियित्वाऽऽस तदादेशात्तु लक्ष्मणः} %॥१४४॥

\twolineshloka
{वृक्षमारुह्य मेधावी वीक्षमाणः प्रयत्नतः}
{स ततो दृष्टवान् हष्टामायान्तीं महतीं चमूम्} %॥१४५॥

\twolineshloka
{हस्त्यश्वरथसंयुक्तां दृष्ट्वा राममथाब्रवीत्}
{हे भ्रातस्त्वं महाबाहो सीतापार्श्वे स्थिरो भव} %॥१४६॥

\twolineshloka
{भूपोऽस्ति बलवान् कश्चिद्धस्त्यश्वरथपत्तिभिः}
{इत्याकर्ण्य वचस्तस्य लक्ष्मणस्य महात्मनः} %॥१४७॥

\twolineshloka
{रामस्तब्रवीद्वीरो वीरं सत्यपराक्रमः}
{प्रायेण भरतोऽस्माकं द्रष्टुमायाति लक्ष्मण} %॥१४८॥

\twolineshloka
{इत्येवं वदतस्तस्य रामस्य विदितात्मनः}
{आरात्संस्थाप्य सेनां तां भरतो विनयान्वितः} %॥१४९॥

\twolineshloka
{ब्राह्मणैर्मन्त्रिभिः सार्धं रुदन्नागत्य पादयोः}
{रामस्य निपपाताथ वैदेह्या लक्ष्मणस्य च} %॥१५०॥

\twolineshloka
{मन्त्रिणो मातृवर्गश्च स्निग्धबन्धुसुहज्जनाः}
{परिवार्य ततो रामं रुरुदुः शोककातराः} %॥१५१॥

\twolineshloka
{स्वर्यातं पितरं ज्ञात्वा ततो रामो महामतिः}
{लक्ष्मणेन सह भ्रात्रा वैदोह्याथ समन्वितः} %॥१५२॥

\twolineshloka
{स्त्रात्वा मलापहे तीर्थे दत्त्वा च सलिलाञ्जलिम्}
{मात्रादीनभिवाद्याथ रामो दुःखसमन्वितः} %॥१५३॥

\twolineshloka
{उवाच भरतं राजन् दुःखेन महतान्वितम्}
{अयोध्यां गच्छ भरत इतः शीघ्रं महामते} %॥१५४॥

\twolineshloka
{राज्ञा विहीनां नगरीं अनाथां परिपालय}
{इत्युक्तो भरतः प्राह रामं राजीवलोलचनम्} %॥१५५॥

\twolineshloka
{त्वामृते पुरुषव्याघ्र न यास्येऽहमितो ध्रुवम्}
{यत्र त्वं तत्र यास्यामि वैदेही लक्ष्मणो यथा} %॥१५६॥

\twolineshloka
{इत्याकर्ण्य पुनः प्राह भरतं पुरतः स्थितम्}
{नृणां पितृसमो ज्येष्ठः स्वधर्ममनुवर्तिनाम्} %॥१५७॥

\twolineshloka
{यथा न लङ्ह्यं वचनं मया पितृमुखेरितम्}
{तथा त्वया न लङ्ह्यं स्याद्वचनं मम सत्तम} %॥१५८॥

\twolineshloka
{मत्समीपादितो गत्वा प्रजास्त्वं परिपालय}
{द्वादशाब्दिकमेतन्मे व्रतं पितृमुखेरितम्} %॥१५९॥

\twolineshloka
{तदरण्ये चरित्वा तु आगामिष्यामि तेऽन्तिकम्}
{गच्छ तिष्ठ ममादेशे न दुःखं कर्तुमर्हसि} %॥१६०॥

\twolineshloka
{इत्युक्तो भरतः प्राह बाष्पपर्याकुलेक्षणः}
{यथा पिता तथा त्वं मे नात्र कार्या विचारणा} %॥१६१॥

\twolineshloka
{तवादेशान्मया कार्यं देहि त्वं पादुके मम}
{नन्दिग्रामे वसिष्येऽहं पादुके द्वादशाब्दिकम्} %॥१६२॥

\twolineshloka
{त्वद्वेषमेव मद्वेषं त्वदव्रतं मे महाव्रतम्}
{त्वं द्वादशाब्दिकादूर्ध्वं यदि नायासि सत्तम} %॥१६३॥

\twolineshloka
{ततो हविर्यथा चाग्नौ प्रधक्ष्यामि कलेवरम्}
{इत्येवं शपथं कृत्वा भरतो हि सुदुःखितः} %॥१६४॥

\twolineshloka
{बहु प्रदक्षिणं कृत्वा नमस्कृत्य च राघवम्}
{पादुके शिरसा स्थाप्य भरतः प्रस्थितः शनैः} %॥१६५॥

\twolineshloka
{स कुर्वन् भ्रातुरादेशं नन्दिग्रामे स्थितो वशी}
{तपस्वी नियताहारः शाकमूलफलाशनः} %॥१६६॥

\fourlineindentedshloka
{जटाकलापं शिरसा च बिभ्रत्}
{त्वचश्च वाक्षीः किल वन्यभोजी}
{रामस्य वाक्यादरतो हदि स्थितं}
{बभार भूभारमनिन्दितात्मा} %॥१६७॥

॥इति श्रीनरसिंहपुराणे रामप्रादुर्भावे अष्टचत्वारिंशोऽध्यायः॥४८॥

\sect{एकोनपञ्चाशोऽध्यायः --- अरण्य-काण्डः}

\uvacha{मार्कण्डेय उवाच}

\twolineshloka
{गतेऽथ भरते तस्मिन् रामः कमललोचनः}
{लक्ष्मणेन सह भ्रात्रा भार्यया सीतया सह} %॥१॥

\twolineshloka
{शाकमूलफलाहारो विचचार महावने}
{कदाचिल्लक्ष्मणमृते रामदेवः प्रतापवान्} %॥२॥

\twolineshloka
{चित्रकूटवनोद्देशे वैदेह्युत्सङ्गमाश्रितः}
{सुष्वाप स मुहूर्तं तु ततः काको दुरात्मवान्} %॥३॥

\twolineshloka
{सीताभिमुखमभ्येत्य विददार स्तनान्तरम्}
{विदार्य वृक्षमारुह्य स्थितोऽसौ वायसाधमः} %॥४॥

\twolineshloka
{ततः प्रबुद्धो रामोऽसौ दृष्ट्वा रक्तं स्तनान्तरे}
{शोकाविष्टां तु सीतां तामुवाच कमलेक्षणः} %॥५॥

\twolineshloka
{वद स्तनान्तरे भद्रे तव रक्तस्य कारणम्}
{इत्युक्ता सा च तं प्राह भर्तारं विनयान्विता} %॥६॥

\twolineshloka
{पश्य राजेन्द्र वृक्षाग्रे वायसं दुष्टचेष्टितम्}
{अनेनैव कृतं कर्म सुप्ते त्वयि महामते} %॥७॥

\twolineshloka
{रामोऽपि दृष्टवान् काकं तस्मिन् क्रोधमथाकरोत्}
{इषीकास्त्रं समाधाय ब्रह्मास्त्रेणाभिमन्त्रितम्} %॥८॥

\twolineshloka
{काकमुद्दिश्य चिक्षेप सोऽप्यधावद्भयान्वितः}
{स त्विन्द्रस्य सुतो राजन्निन्द्रलोकं विवेश ह} %॥९॥

\twolineshloka
{रामास्त्रं प्रज्वलद्दीप्तं तस्यानु प्रविवेश वै}
{विदितार्थश्च देवेन्द्रो देवैः सह समन्वितः} %॥१०॥

\twolineshloka
{निष्क्रामयच्च तं दुष्टं राघवस्यापकारिणम्}
{ततोऽसौ सर्वदेवैस्तु देवलोकाद्वहिः कृतः} %॥११॥

\twolineshloka
{पुनः सोऽप्यपतद्रामं राजानं शरणं गतः}
{पाहि राम महाबाहो अज्ञानादपकारिणम्} %॥१२॥

\twolineshloka
{इति ब्रुवन्तं तं प्राह रामः कमललोचनः}
{अमोघं च ममैवास्त्रमङ्गमेकं प्रयच्छ वै} %॥१३॥

\twolineshloka
{ततो जीवसि दुष्ट त्वमपकारो महान् कृतः}
{इत्युक्तोऽसौ स्वकं नेत्रमेकमस्त्राय दत्तवान्} %॥१४॥

\twolineshloka
{अस्त्रं तन्नेत्रमेकं तु भस्मीकृत्य समाययौ}
{ततः प्रभृति काकानां सर्वेषामेकनेत्रता} %॥१५॥

\twolineshloka
{चक्षुषैकेन पश्यन्ति हेतुना तेन पार्थिव}
{उषित्वा तत्र सुचिरं चित्रकूटे स राघवः} %॥१६॥

\twolineshloka
{जगाम दण्डकारण्यं नानामुनिनिषेवितम्}
{सभ्रातृकः सभार्यश्च तापसं वेषमास्थितः} %॥१७॥

\twolineshloka
{धनुः पर्वसुपाणिश्च सेषुधिश्च महाबलः}
{ततो ददर्श तत्रस्थानम्बुभक्षान्महामुनीन्} %॥१८॥

\twolineshloka
{अश्मकुट्टाननेकांश्च दन्तोलूखलिनस्तथा}
{पञ्चाग्निमध्यगानन्यानन्यानुग्रतपश्चरान्} %॥१९॥

\twolineshloka
{तान् दृष्ट्वा प्रणिपत्योच्चै रामस्तैश्चाभिनन्दितः}
{ततोऽखिलं वनं दृष्ट्वा रामः साक्षाज्जनार्दनः} %॥२०॥

\twolineshloka
{भ्रातृभार्यासहायश्च सम्प्रतस्थे महामतिः}
{दर्शयित्वा तु सीतायै वनं कुसुमितं शुभम्} %॥२१॥

\twolineshloka
{नानाश्चर्यसमायुक्तं शनैर्गच्छन् स दृष्टवान्}
{कृष्णाङ्गं रक्तनेत्रं तु स्थूलशैलसमानकम्} %॥२२॥

\twolineshloka
{शुभ्रदंष्ट्रं महाबाहुं सन्ध्याघनशिरोरुहम्}
{मेघस्वनं सापराधं शरं सन्धाय राघवः} %॥२३॥

\twolineshloka
{विव्याध राक्षसं क्रोधाल्लक्ष्मणेन सह प्रभुः}
{अन्यैरवध्यं हत्वा तं गिरिगर्ते महातनुम्} %॥२४॥

\twolineshloka
{शिलाभिश्छाद्य गतवाज्शरभङ्गाश्रमं ततः}
{तं नत्वा तत्र विश्रम्य तत्कथातुष्टमानसः} %॥२५॥

\twolineshloka
{तीक्ष्णाश्रममुपागम्य दुष्टवांस्तं महामुनिम्}
{तेनादिष्टेन मार्गेण गत्वागस्त्यं ददर्श ह} %॥२६॥

\twolineshloka
{खङ्गं तु विमलं तस्मादवाप रघुनन्दनः}
{इषुधिं चाक्षयशरं चापं चैव तु वैष्णवम्} %॥२७॥

\twolineshloka
{ततोऽगस्त्याश्रमाद्रामो भ्रातृभार्यासमन्वितः}
{गोदावर्याः समीपे तु पञ्चवट्यामुवास सः} %॥२८॥

\twolineshloka
{ततो जटायुरभ्येत्य रामं कमललोचनम्}
{नत्वा स्वकुलमाख्याय स्थितवान् गृध्रनायकः} %॥२९॥

\twolineshloka
{रामोऽपि तत्र तं दृष्ट्वा आत्मवृत्तं विशेषतः}
{कथयित्वा तु तं प्राह सीतां रक्ष महामते} %॥३०॥

\twolineshloka
{इत्युक्तोऽसौ जतायुस्तु राममालिङग्य सादरम्}
{कार्यार्थं तु गते रामे भ्रात्रा सह वनान्तरम्} %॥३१॥

\twolineshloka
{अहं रक्ष्यामि ते भार्यां स्थीयतामत्र शोभन}
{इत्युक्त्वा गतवान्रामं गृध्रराजः स्वमाश्रमम्} %॥३२॥

\twolineshloka
{समीपे दक्षिणे भागे नानापक्षिनिषेविते}
{वसन्तं राघवं तत्र सीतया सह सुन्दरम्} %॥३३॥

\twolineshloka
{मन्मथाकारसदृशं कथयन्तं महाकथाः}
{कृत्वा मायामयं रुपं लावण्यगुणसंयुतम्} %॥३४॥

\twolineshloka
{मदनाक्रान्तहदया कदाचिद्रावणानुजा}
{गायन्ती सुस्वरं गीतं शनैरागत्य राक्षसी} %॥३५॥

\twolineshloka
{ददर्श राममासीनं कानने सीतया सह}
{अथ शूर्पणखा घोरा मायारुपधरा शुभा} %॥३६॥

\twolineshloka
{निश्शङ्का दुष्टचित्ता सा राघवं प्रत्यभाषत}
{भज मां कान्त कल्याणीं भजन्तीं कामिनीमिह} %॥३७॥

\twolineshloka
{भजमानां त्यजेद्यस्तु तस्य दोषो महान् भवेत्}
{इत्युक्तः शूर्पणखया रामस्तामाह पार्थिवः} %॥३८॥

\twolineshloka
{कलत्रवानहं बाले कनीयांसं भजस्व मे}
{इति श्रुत्वा ततः प्राह राक्षसी कामरुपिणी} %॥३९॥

\twolineshloka
{अतीव निपुणा चाहं रतिकर्मणि राघव}
{त्यक्त्वैनामनभिज्ञां त्वं सीतां मां भज शोभनाम्} %॥४०॥

\twolineshloka
{इत्याकर्ण्य वचः प्राह रामस्तां धर्मतत्परः}
{परस्त्रियं न गच्छेऽहं त्वमितो गच्छ लक्ष्मणम्} %॥४१॥

\twolineshloka
{तस्य नात्र वने भार्या त्वामसौ सग्रहीष्यति}
{इत्युक्ता सा पुनः प्राह रामं राजीवलोचनम्} %॥४२॥

\twolineshloka
{यथा स्याल्लक्ष्मणो भर्ता तथा त्वं देहि पत्रकम्}
{तथैवमुक्त्वा मतिमान् रामः कमललोचनः} %॥४३॥

\twolineshloka
{छिन्ध्यस्या नासिकामिति मोक्तव्या नात्र संशयः}
{इति रामो महाराजो लिख्य पत्रं प्रदत्तवान्} %॥४४॥

\twolineshloka
{सा गृहीत्वा तु तत्पत्रं गत्वा तस्मान्मुदान्विता}
{गत्वा दत्तवती तद्वल्लक्ष्मणाय महात्मने} %॥४५॥

\twolineshloka
{तां दृष्ट्वा लक्ष्मणः प्राह राक्षसीं कामरुपिणीम्}
{न लङ्घ्यं राघववचो मया तिष्ठात्मकश्मले} %॥४६॥

\twolineshloka
{तां प्रगृह्य ततः खङ्गमुद्यम्य विमलं सुधीः}
{तेन तत्कर्णनासां तु चिच्छेद तिलकाण्डवत्} %॥४७॥

\twolineshloka
{छिन्ननासा ततः सा तु रुरोद भृशदुः खिता}
{हा दशास्य मम भ्रातः सर्वदेवविमर्दक} %॥४८॥

\twolineshloka
{हा कष्टं कुम्भकर्णाद्यायाता मे चापदा परा}
{हा हा कष्टं गुणनिधे विभीषण महामते} %॥४९॥

\twolineshloka
{इत्येवमार्ता रुदती सा गत्वा खरदूषणौ}
{त्रिशिरसं च सा दृष्ट्वा निवेद्यात्मपराभवम्} %॥५०॥

\twolineshloka
{राममाह जनस्थाने भ्रात्रा सह महाबलम्}
{ज्ञात्वा ते राघवं क्रुद्धाः प्रेषयामासुरुर्जितान्} %॥५१॥

\twolineshloka
{चतुर्दशसहस्त्राणि राक्षसानां बलीयसाम्}
{अग्रे निजग्मुस्तेनैव रक्षसां नायकास्त्रयः} %॥५२॥

\twolineshloka
{रावणेन नियुक्तास्ते पुरैव तु महाबलाः}
{महाबलपरीवारा जनस्थानमुपागताः} %॥५३॥

\twolineshloka
{क्रोधेन महताऽऽविष्टा दृष्ट्वा तां छिन्ननासिकाम्}
{रुदतीमश्रुदिग्धाङ्गीं भगिनीं रावणस्य तु} %॥५४॥

\twolineshloka
{रामोऽपि तद्वलं दृष्ट्वा राक्षसानां बलीयसाम्}
{संस्थाप्य लक्ष्मणं तत्र सीताया रक्षणं प्रति} %॥५५॥

\twolineshloka
{गत्वा तु प्रहितैस्तत्र राक्षसैर्बलदर्पितैः}
{चतुर्दशसहस्त्रं तु राक्षसानां महाबलम्} %॥५६॥

\twolineshloka
{क्षणेन निहतं तेन शरैरग्निशिखोपमैः}
{खरश्च निहतस्तेन दूषणश्च महाबलः} %॥५७॥

\twolineshloka
{त्रिशिराश्च महारोषाद रणे रामेण पातितः}
{हत्वा तान् राक्षसान् दुष्टान् रामश्चाश्रममाविशत्} %॥५८॥

\twolineshloka
{शूर्पणखा च रुदती रावणान्तिकमागता}
{छिन्ननासां च तां दृष्ट्वा रावणो भगिनीं तदा} %॥५९॥

\twolineshloka
{मारीचं प्राह दुर्बुद्धिः सीताहरणकर्मणि}
{पुष्पकेण विमानेन गत्वाहं त्वं च मातुल} %॥६०॥

\twolineshloka
{जनस्थानसमीपे तु स्थित्वा तत्र ममाज्ञया}
{सौवर्णमृगरुपं त्वमास्थाय तु शनैः शनैः} %॥६१॥

\twolineshloka
{गच्छ त्वं तत्र कार्यार्थं यत्र सीता व्यवस्थिता}
{दृष्ट्वा सा मृगपोतं त्वां सौवर्णं त्वयि मातुल} %॥६२॥

\twolineshloka
{स्पृहां करिष्यते रामं प्रेषयिष्यति बन्धने}
{तद्वाक्यात्तत्र गच्छन्तं धावस्व गहने वने} %॥६३॥

\twolineshloka
{लक्ष्मणस्यापकर्षार्थं वक्तव्यं वागुदीरणम्}
{ततः पुष्पकमारुह्य मायारुपेण चाप्यहम्} %॥६४॥

\twolineshloka
{तां सीतामहमानेष्ये तस्यामासक्तमानसः}
{त्वमपि स्वेच्छया पश्चादागमिष्यसि शोभन} %॥६५॥

\twolineshloka
{इत्युक्ते रावणेनाथ मारीचो वाक्यमब्रवीत्}
{त्वमेव गच्छ पापिष्ठ नाहं गच्छामि तत्र वै} %॥६६॥

\twolineshloka
{पुरैवानेन रामेण व्यथितोऽहं मुनेर्मखे}
{इत्युक्तवति मारीचे रावणः क्रोधमूर्च्छितः} %॥६७॥

\twolineshloka
{मारीचं हन्तुमारेभे मारीचोऽप्याह रावणम्}
{तव हस्तवधाद्वीर रामेण मरणं वरम्} %॥६८॥

\twolineshloka
{अहं गमिष्यामि तत्र यत्र त्वं नेतुमिच्छसि}
{अथ पुष्पकमारुह्य जनस्थानमुपागतः} %॥६९॥

\twolineshloka
{मारीचस्तत्र सौवर्णं मृगमास्थाय चाग्रतः}
{जगाम यत्र सा सीता वर्तते जनकात्मजा} %॥७०॥

\twolineshloka
{सौवर्णं मृगपोतं तु दृष्ट्वा सीता यशस्विनी}
{भाविकर्मवशाद्रामुवाच पतिमात्मनः} %॥७१॥

\twolineshloka
{गृहीत्वा देहि सौवर्णं मृगपोतं नृपात्मज}
{अयोध्यायां तु मद्रेहे क्रीडनार्थमिदं मम} %॥७२॥

\twolineshloka
{तयैवमुक्तो रामस्तु लक्ष्मणं स्थाप्य तत्र वै}
{रक्षणार्थ तु सीताया गतोऽसौ मृगपृष्ठतः} %॥७३॥

\twolineshloka
{रामेण चानुयातोऽसौ अभ्यधावद्वने मृगः}
{ततः शरेण विव्याध रामस्तं मृगपोतकम्} %॥७४॥

\twolineshloka
{हा लक्ष्मणेति चोक्त्वासौ निपपात महीतले}
{मारीचः पर्वताकारस्तेन नष्टो बभूव सः} %॥७५॥

\twolineshloka
{आकर्ण्य रुदतः शब्दं सीता लक्ष्मणमब्रवीत्}
{गच्छ लक्ष्मण पुत्र त्वं यत्रायं शब्द उत्थितः} %॥७६॥

\twolineshloka
{भ्रातुर्ज्येष्ठस्य तत्त्वं वै रुदतः श्रूयते ध्वनिः}
{प्रायो रामस्य सन्देहं लक्षयेऽहं महात्मनः} %॥७७॥

\twolineshloka
{इत्युक्तः स तथा प्राह लक्ष्मणस्तामनिन्दिताम्}
{न हि रामस्य सन्देहो न भयं विद्यते क्वचित्} %॥७८॥

\twolineshloka
{इति ब्रुवाणं तं सीता भाविकर्मबलाद्भृतम्}
{लक्ष्मणं प्राह वैदेही विरुद्धवचनं तदा} %॥७९॥

\twolineshloka
{मृते रामे तु मामिच्छन्नतस्त्वं न गामिष्यसि}
{इत्युक्तः स विनीतात्म असहन्नप्रियं वचः} %॥८०॥

\twolineshloka
{जगाम राममन्वेष्टुं तदा पार्थिवनन्दनः}
{सन्न्यासवेषमास्थाय रावणोऽपि दुरात्मवान्} %॥८१॥

\twolineshloka
{स सीतापार्श्वमासाद्य वचनं चेदमुक्तवान्}
{आगतो भरतः श्रीमानयोध्याया महामतिः} %॥८२॥

\twolineshloka
{रामेण सह सम्भाष्य स्थितवांस्तत्र कानने}
{मां च प्रेषितवान् रामो विमानमिदमारुह} %॥८३॥

\twolineshloka
{अयोध्यां याति रामस्तु भरतेन प्रसादितः}
{मृगबालं तु वैदेहि क्रीडार्थं ते गृहीतवान्} %॥८४॥

\twolineshloka
{क्लेशितासि महारण्ये बहुकालं त्वमीदृशम्}
{सम्प्राप्तराज्यस्ते भर्ता रामः स रुचिराननः} %॥८५॥

\twolineshloka
{लक्ष्मणश्च विनीतात्मा विमानमिदमारुह}
{इत्युक्ता सा तथा गत्वा नीता तेन महात्मना} %॥८६॥

\twolineshloka
{आरुरोह विमानं तु छद्मना प्रेरिता सती}
{तज्जगाम ततः शीघ्रं विमानं दक्षिणां दिशम्} %॥८७॥

\twolineshloka
{ततः सीता सुदुःखार्ता विललाप सुदुःखिता}
{विमाने खेऽपि रोदन्त्याश्चक्रे स्पर्शं न राक्षसः} %॥८८॥

\twolineshloka
{रावणः स्वेन रुपेण बभूवाथ महातनुः}
{दशग्रीवं महाकायं दृष्ट्वा सीता सुदुःखिता} %॥८९॥

\twolineshloka
{हा राम वञ्चिताद्याहं केनापिच्छद्मरुपिणा}
{रक्षसा घोररुपेण त्रायस्वेति भयार्दिता} %॥९०॥

\twolineshloka
{हे लक्ष्मण महाबाहो मां हि दुष्टेन रक्षसा}
{द्रुतमागत्य रक्षस्व नीयमानामथाकुलाम्} %॥९१॥

\twolineshloka
{एवं प्रलपमानायाः सीतायास्तन्महस्त्स्वनम्}
{आकर्ण्य गृध्रराजस्तु जटायुस्तत्र चागतः} %॥९२॥

\twolineshloka
{तिष्ठ रावण दुष्टात्मन मुञ्च मुञ्चात्र मैथिलीम्}
{इत्युक्त्वा युयुधे तेन जटायुस्तत्र वीर्यवान्} %॥९३॥

\twolineshloka
{पक्षाभ्यां ताडयामास जटायुस्तस्य वक्षसि}
{ताडयन्तं तु तं मत्वा बलवानिति रावणः} %॥९४॥

\twolineshloka
{तुण्डचञ्जुप्रहारैस्तु भृशं तेन प्रपीडितः}
{तत उत्थाप्य वेगेन चन्द्रहासमसिं महत्} %॥९५॥

\twolineshloka
{जघान तेन दुष्टात्मा जटायुं धर्मचारिणम्}
{निपपात महीपृष्ठे जटायुः क्षीणचेतनः} %॥९६॥

\twolineshloka
{उवाच च दशग्रीवं दुष्टात्मन् न त्वया हतः}
{चन्द्रहासस्य वीर्येण हतोऽहं राक्षसाधम} %॥९७॥

\twolineshloka
{निरायुधं को हनेन्मूढ सायुधस्त्वामृते जनः}
{सीतापहरणं विद्धि मृत्युस्ते दुष्ट राक्षस} %॥९८॥

\twolineshloka
{दुष्ट रावण रामस्त्वां वधिष्यति न संशयः}
{रुदती दुःखशोकार्ता जटायुं प्राह मैथिली} %॥९९॥

\twolineshloka
{मत्कृते मरणं यस्मात्त्वया प्राप्तं द्विजोत्तम}
{तस्माद्रामप्रसादेन विष्णुलोकमवाप्स्यसि} %॥१००॥

\twolineshloka
{यावद्रामेण सङ्गस्ते भविष्यति महाद्विज}
{तावत्तिष्ठन्तु ते प्राणा इत्युक्त्वा तु खगोत्तमम्} %॥१०१॥

\twolineshloka
{ततस्तान्यर्पितान्यङ्गाद्भूषणानि विमुच्य सा}
{शीघ्रं निबध्य वस्त्रेण रामहस्तं गमिष्यथ} %॥१०२॥

\twolineshloka
{इत्युक्त्वा पातयामास भूमौ सीता सुदुःखिता}
{एवं हत्वा स सीतां तु जटायुं पात्य भूतले} %॥१०३॥

\twolineshloka
{पुष्पकेण गतः शीघ्रं लङ्कां दुष्टनिशाचरः}
{अशोकवनिकामध्ये स्थापयित्वा स मैथिलीम्} %॥१०४॥

\twolineshloka
{इमामत्रैव रक्षध्वं राक्षस्यो विकृताननाः}
{इत्यादिश्य गृहं यातो रावणो राक्षसेश्वरः} %॥१०५॥

\twolineshloka
{लङ्कानिवासिनश्चोचुरेकान्तं च परस्परम्}
{अस्याः पुर्या विनाशार्थं स्थापितेयं दुरात्मना} %॥१०६॥

\twolineshloka
{राक्षसीभिर्विरुपाभी रक्ष्यमाणा समन्ततः}
{सीता च दुःखिता तत्र स्मरन्ती राममेव सा} %॥१०७॥

\twolineshloka
{उवास सा सुदुःखार्ता दुःखिता रुदती भृशम्}
{यथा ज्ञानखले देवी हंसयाना सरस्वती} %॥१०८॥

\twolineshloka
{सुग्रीवभृत्या हरयश्चतुरश्च यदृच्छया}
{वस्त्रबद्धं तयोत्सृष्टं गृहीत्वा भूषणं द्रुतम्} %॥१०९॥

\twolineshloka
{स्वभर्त्रे विनिवेद्योचुः सुग्रीवाय महात्मने}
{अरण्येऽभून्महायुद्धं जटायो रावणस्य च} %॥११०॥

\twolineshloka
{अथ रामश्च तं हत्वा मारीचं माययाऽऽगतम्}
{निवृत्तो लक्ष्मणं दृष्ट्वा तेन गत्वा स्वमाश्रमम्} %॥१११॥

\twolineshloka
{सीतामपश्यन्दुः खार्तः प्ररुरोद स राघवः}
{लक्ष्मणश्च महातेजा रुरोद भृशदुःखितः} %॥११२॥

\twolineshloka
{बहुप्रकारमस्वस्थं रुदन्तं राघवं तदा}
{भूतले पतितं धीमानुत्थाप्याश्वास्य लक्ष्मणः} %॥११३॥

\twolineshloka
{उवाच वचनं प्राप्तं तदा यत्तच्छृणुष्व मे}
{अतिवेलं महाराज न शोकं कर्तुमर्हसि} %॥११४॥

\twolineshloka
{उत्तिष्ठोत्तिष्ठ शीघ्रं त्वं सीतां मृगयितुं प्रभो}
{इत्येवं वदता तेन लक्ष्मणेन महात्मना} %॥११५॥

\twolineshloka
{उत्थापितो नरपतिर्दुःखितो दुःखितेन तु}
{भ्रात्रा सह जगामाथ सीतां मृगयितुं वनम्} %॥११६॥

\fourlineindentedshloka
{वनानि सर्वाणि विशोध्य राघवो}
{गिरीन् समस्तान् गिरिसानुगोचरान्}
{तथा मुनीनामपि चाश्रमान् बहूंस्-}
{तृणादिवल्लीगहनेषु भूमिषु} %॥११७॥

\fourlineindentedshloka
{नदीतटे भूविवरे गुहायां}
{निरीक्षमाणोऽपि महानुभावः}
{प्रियामपश्यन् भृशदुःखितस्तदा}
{जटायुषं वीक्ष्य च घातितं नृपः} %॥११८॥

\fourlineindentedshloka
{अहो भवान् केन हतस्त्वमीदृशीं}
{दशामवाप्तोऽसि मृतोऽसि जीवसि}
{ममाद्य सर्वं समदुःखितस्य भोः}
{पत्नीवियोगादिह चागतस्य वै} %॥११९॥

\fourlineindentedshloka
{इत्युक्तमात्रे विहगोऽथ कृच्छ्रा-}
{दुवाच वाचं मधुरां तदानीम्}
{श्रृणुष्व राजन् मम वृत्तमत्र}
{वदामि दृष्टं च कृतं च सद्यः} %॥१२०॥

\fourlineindentedshloka
{दशाननस्तामपनीय मायया}
{सीतां समारोप्य विमानमुत्तमम्}
{जगाम खे दक्षिणदिङ्मुखोऽसौ}
{सीता च माता विललाप दुःखिता} %॥१२१॥

\fourlineindentedshloka
{आकर्ण्य सीतास्वनमागतोऽहं}
{सीतां विमोक्तुं स्वबलेन राघव}
{युद्धं च तेनाहमतीव कृत्वा}
{हतः पुनः खङ्गबलेन रक्षसा} %॥१२२॥

\fourlineindentedshloka
{वैदेहिवाक्यादिह जीवता मया}
{दृष्टो भवान् स्वर्गमितो गमिष्ये}
{मा राम शोकं कुरु भूमिपाल}
{जह्यद्य दुष्टं सगणं तु नैऋतम्} %॥१२३॥

\twolineshloka
{रामो जटायुषेत्युक्तः पुनस्तं चाह शोकतः}
{स्वस्त्यस्तु ते द्विजवर गतिस्तु परमास्तु ते} %॥१२४॥

\twolineshloka
{ततो जटायुः स्वं देहं विहाय गतवान्दिवन्}
{विमानेन तु रम्येण सेव्यमानोऽप्सरोगणैः} %॥१२५॥

\twolineshloka
{रामोऽपि दग्ध्वा तद्देहं स्नातो दत्त्वा जलाञ्जलिम्}
{भ्रात्रा सगच्छन् दुःखार्तो राक्षसी पथि दृष्टवान्} %॥१२६॥

\twolineshloka
{उद्वमन्तीं महोल्काभां विवृतास्यां भयकरीम्}
{क्षयं नयन्तीं जन्तून् वै पातयित्वा गतो रुषा} %॥१२७॥

\twolineshloka
{गच्छन् वनान्तरं रामः स कबन्धं ददर्श ह}
{विरुपं जठरमुखं दीर्घबाहुं घनस्तनम्} %॥१२८॥

\twolineshloka
{रुन्धानं राममार्गं तु दृष्ट्वा तं दग्धवाज्शनैः}
{दग्धोऽसौ दिव्यरुपी तु खस्थो राममभाषत} %॥१२९॥

\twolineshloka
{राम राम महाबाहो त्वया मम महामते}
{विरुपं नाशितं वीर मुनिशापाच्चिरागतम्} %॥१३०॥

\twolineshloka
{त्रिदिवं यामि धन्योऽस्मि त्वत्प्रसादान्न संशयः}
{त्वं सीताप्राप्तये सख्यं कुरु सूर्यसुतेन भोः} %॥१३१॥

\twolineshloka
{वानरेन्द्रेण गत्वा तु सुग्रीवे स्वं निवेद्य वै}
{भविष्यति नृपश्रेष्ठ ऋष्यमूकगिरि व्रज} %॥१३२॥

\twolineshloka
{इत्युक्त्वा तु गते तस्मिन् रामो लक्ष्मणसंयुतः}
{सिद्धैस्तु मुनिभिः शून्यमाश्रमं प्रविवेश ह} %॥१३३॥

\twolineshloka
{तत्रस्थां तापसीं दृष्ट्वा तया संलाप्य संस्थितः}
{शबरीं मुनिमुख्यानां सपर्याहतकल्मषाम्} %॥१३४॥

\twolineshloka
{तया सम्पूजितो रामो बदरादिभिरीश्वरः}
{साप्येनं पूजयित्वा तु स्वामवस्थां निवेद्य वै} %॥१३५॥

\twolineshloka
{सीतां त्वं प्राप्स्यसीत्युक्त्वा प्रविश्याग्निं दिवगता}
{दिवं प्रस्थाप्य तां चापि जगामान्यत्र राघवः} %॥१३६॥

\fourlineindentedshloka
{ततो विनीतेन गुणान्वितेन}
{भ्रात्रा समेतो जगदेकनाथः}
{प्रियावियोगेन सुदुःखितात्मा}
{जगाम याम्यां स तु रामदेवः} %॥१३७॥

॥इति श्रीनरसिंहपुराणे रामप्रादुर्भावे एकोनपञ्चाशोऽध्यायः॥४९॥

\sect{पञ्चाशत्तमोऽध्यायः --- किष्किन्धा-काण्डः}

\uvacha{मार्कण्डेय उवाच}

\twolineshloka
{बालिना कृतवैरोऽथ दुर्गवर्ती हरीश्वरः}
{सुग्रीवो दृष्टवान् दूराद् दृष्ट्वाऽऽह पवनात्मजम्} %॥१॥

\twolineshloka
{कस्येमौ सुधनुः पाणी चीरवल्कलधारिणौ}
{पश्यन्तौ सरसीं दिव्यां पद्मोत्पलसमावृताम्} %॥२॥

\twolineshloka
{नानारुपधरावेतौ तापसं वेषमास्थितौ}
{बालिदूताविह प्राप्ताविति निश्चित्य सूर्यजः} %॥३॥

\twolineshloka
{उत्पपात भयत्रस्तः ऋष्यमूकाद् वनान्तरम्}
{वानरैः सहितः सर्वैरगस्त्यश्रममुत्तमम्} %॥४॥

\twolineshloka
{तत्र स्थित्वा स सुग्रीवः प्राह वायुसुतं पुनः}
{हनूमन् पृच्छ शीघ्रं त्वं गच्छ तापसवेषधृक्} %॥५॥

\twolineshloka
{कौ हि कस्य सुतौ जातौ किमर्थं तत्र संस्थितौ}
{ज्ञात्वा सत्यं मम ब्रूहि वायुपुत्र महामते} %॥६॥

\twolineshloka
{इत्युक्तो हनुमान् गत्वा पम्पातटमनुत्तमम्}
{भिक्षुरुपी स तं प्राह रामं भ्रात्रा समन्वितम्} %॥७॥

\twolineshloka
{को भवानिह सम्प्राप्तस्तथ्यं ब्रूहि महामते}
{अरण्ये निर्जने घोरे कुतस्त्वं किं प्रयोजनम्} %॥८॥

\twolineshloka
{एवं वदन्तं तं प्राह लक्ष्मणो भ्रातुराज्ञया}
{प्रवक्ष्यामि निबोध त्वं रामवृत्तान्तमादितः} %॥९॥

\twolineshloka
{राजा दशरथो नाम बभूव भुवि विश्रुतः}
{तस्य पुत्रो महाबुद्धे रामो ज्येष्ठो ममाग्रजः} %॥१०॥

\twolineshloka
{अस्याभिषेक आरब्धः कैकेय्या तु निवारितः}
{पितुराज्ञामयं कुर्वन् रामो भ्राता ममाग्रजः} %॥११॥

\twolineshloka
{मया सह विनिष्क्रम्य सीतया सह भार्यया}
{प्रविष्टो दण्डकारण्यं नानामुनिसमाकुलम्} %॥१२॥

\twolineshloka
{जनस्थाने निवसतो रामस्यास्य महात्मनः}
{भार्या सीता तत्र वने केनापि पाप्पना हता} %॥१३॥

\twolineshloka
{सीतामन्वेषयन् वीरो रामः कमललोचनः}
{इहायातस्त्वया दृष्ट इति वृत्तान्तमीरितम्} %॥१४॥

\twolineshloka
{श्रुत्वा ततो वचस्तस्य लक्ष्मणस्य महात्मनः}
{अव्याञ्जितात्मा विश्वासाद्धनूमान् मारुतात्मजः} %॥१५॥

\twolineshloka
{त्वं मे स्वामी इति वदन् रामं रघुपतिं तदा}
{आश्वास्यानीय सुग्रीवं तयोः सख्यमकारयत्} %॥१६॥

\twolineshloka
{शिरस्यारोप्य पादाब्जं रामस्य विदितात्मनः}
{सुग्रीवो वानरेन्द्रस्तु उवाच मधुराक्षरम्} %॥१७॥

\twolineshloka
{अद्यप्रभृति राजेन्द्र त्वं मे स्वामी न संशयः}
{अहं तु तव भृत्यश्च वानरैः सहितः प्रभो} %॥१८॥

\twolineshloka
{त्वच्छत्रुर्मम शत्रुः स्यादद्यप्रभृति राघव}
{मित्रं ते मम सन्मित्रं त्वददुःखं तन्ममापि च} %॥१९॥

\twolineshloka
{त्वत्प्रीतिरेव मत्प्रीतिरित्युक्त्वा पुनराह तम्}
{वाली नाम मम ज्येष्ठो महाबलपराक्रमः} %॥२०॥

\twolineshloka
{भार्यापहारी दुष्टात्मा मदनासक्तमानसः}
{त्वामृते पुरुषव्याघ्र नास्ति हन्ताद्य वालिनम्} %॥२१॥

\twolineshloka
{युगपत्सप्ततालांस्तु तरुन् यो वै वधिष्यति}
{स तं वधिष्यतीत्युक्तं पुराणज्ञैर्नृपात्मज} %॥२२॥

\twolineshloka
{तत्प्रियार्थं हि रामोऽपि श्रीमांश्छित्त्वा महातरुन्}
{अर्धाकृष्टेन बाणेन युगप्रदघुनन्दनः} %॥२३॥

\twolineshloka
{विदध्वा महातरुन् रामः सुग्रीवं प्राह पार्थिवम्}
{वालिना गच्छ युध्यस्व कृतचिह्नो रवेः सुत} %॥२४॥

\twolineshloka
{इत्युक्तः कृतचिह्नोऽयं युद्धं चक्रेऽथ वालिना}
{रामोऽपि तत्र गत्वाथ शरेणैकेन वालिनम्} %॥२५॥

\twolineshloka
{विव्याध वीर्यवान् वाली पपात च ममार च}
{वित्रस्तं वालिपुत्रं तु अङ्गदं विनयान्वितम्} %॥२६॥

\twolineshloka
{रणशौण्डं यौवराज्ये नियुक्त्वा राघवस्तदा}
{तां च तारां तथा दत्त्वा रामश्च रविसूनवे} %॥२७॥

\twolineshloka
{सुग्रीवं प्राहं धर्मात्मा रामः कमललोचनः}
{राज्यमन्वेषय स्वं त्वं कपीनां पुनराव्रज} %॥२८॥

\twolineshloka
{त्वं सीतान्वेषणे यत्नं कुरु शीघ्रं हरीश्वर}
{इत्युक्तः प्राह सुग्रीवो रामं लक्ष्मणसंयुतम्} %॥२९॥

\twolineshloka
{प्रावृट्कालो महान् प्राप्तः साम्प्रतं रघुनन्दन}
{वानराणां गतिर्नास्ति वने वर्षति वासवे} %॥३०॥

\twolineshloka
{गते तमिंस्तु राजेन्द्र प्राप्ते शरदि निर्मले}
{चारान् सम्प्रेषयिष्यामि वानरान्दिक्षु राघव} %॥३१॥

\twolineshloka
{इत्युक्त्वा रामचन्द्रं स तं प्रणम्य कपीश्वरः}
{पम्पापुरं प्रविश्याथ रेमे तारासमन्वितः} %॥३२॥

\twolineshloka
{रामोऽपि विधिवदभ्रात्रा शैलसानौ महावने}
{निवासं कृतवान् शैले नीलकण्ठे महामतिः} %॥३३॥

\twolineshloka
{प्रावृट्काले गते कृच्छ्रात् प्राप्ते शरदि राघवः}
{सीतावियोगाद्व्यथितः सौमित्रिं प्राह लक्ष्मणम्} %॥३४॥

\twolineshloka
{उल्लङ्घितस्तु समयः सुग्रीवेण ततो रुषा}
{लक्ष्मणं प्राह काकुत्स्थो भ्रातरं भ्रातृवत्सलः} %॥३५॥

\twolineshloka
{गच्छ लक्ष्मण दुष्टोऽसौ नागतः कपिनायकः}
{गते तु वर्षाकालेऽहमागमिष्यामि तेऽन्तिकम्} %॥३६॥

\twolineshloka
{अनेकैर्वानरैः सार्धमित्युक्त्वासौ तदा गतः}
{तत्र गच्छ त्वरा युक्तो यत्रास्ते कपिनायकः} %॥३७॥

\twolineshloka
{तं दुष्टमग्रतः कृत्वा हरिसेनासमन्वितम्}
{रमन्तं तारया सार्धं शीघ्रमानय मां प्रति} %॥३८॥

\twolineshloka
{नात्रागच्छति सुग्रीवो यद्यसौ प्राप्तभूतिकः}
{तदा त्वयैवं वक्तव्यः सुग्रीवोऽनृतभाषकः} %॥३९॥

\twolineshloka
{वालिहन्ता शरो दुष्ट करे मेऽद्यापि तिष्ठति}
{स्मृत्वैतदाचर कपे रामवाक्यं हितं तव} %॥४०॥

\threelineshloka
{इत्युक्तस्तु तथेत्युक्त्वा रामं नत्वा च लक्ष्मणः}
{पम्पापुरं जगामाथ सुग्रीवो यत्र तिष्ठति}
{दृष्ट्वा स तत्र सुग्रीवं कपिराजं बभाष वै} %॥४१॥

\twolineshloka
{ताराभोगविषक्तस्त्वं रामकार्यपराङ्मुखः}
{किं त्वया विस्मृतं सर्वं रामाग्रे समयं कृतम्} %॥४२॥

\twolineshloka
{सीतामन्विष्य दास्यामि यत्र क्वापीति दुर्मते}
{हत्वा तु वालिनं राज्यं येन दत्तं पुरा तव} %॥४३॥

\twolineshloka
{त्वामृते कोऽवमन्येत कपीन्द्र पापचेतस}
{प्रतिश्रुत्य च रामस्य भार्याहीनस्य भूपते} %॥४४॥

\twolineshloka
{साहाय्यं ते करोमिति देवाग्निजलसन्निधौ}
{ये ये च शत्रवो राजंस्ते ते च मम शत्रवः} %॥४५॥

\twolineshloka
{मित्राणि यानि ते देव तानि मित्राणि मे सदा}
{सीतामन्वेषितुं राजन् वानरैर्बहुभिर्वृतः} %॥४६॥

\twolineshloka
{सत्यं यास्यामि ते पार्श्वमित्युक्त्वा कोऽन्यथाकरोत्}
{त्वामृते पापिनं दुष्टं रामदेवस्य सन्निधौ} %॥४७॥

\twolineshloka
{कारयित्वा तु तेनैवं स्वकार्यं दुष्टवानर}
{ऋषीणां सत्यवद्वाक्यं त्वयि दृष्टं मयाधुना} %॥४८॥

\twolineshloka
{सर्वस्य हि कृतार्थस्य मतिरन्या प्रवर्तते}
{वत्सः क्षीरक्षयं दृष्ट्वा परित्यजति मातरम्} %॥४९॥

\twolineshloka
{जनवृत्तविदां लोके सर्वज्ञानां महात्मनाम्}
{न तं पश्यामि लोकेऽस्मिन् कृतं प्रतिकरोति यः} %॥५०॥

\twolineshloka
{शास्त्रेषु निष्कृतिर्दृष्टा महापातकिनामपि}
{कृतघ्नस्य कपे दुष्ट न दृष्टा निष्कृतिः पुरा} %॥५१॥

\twolineshloka
{कृतघ्रना न कार्या ते त्वत्कृतं समयं स्मर}
{एह्येह्यागच्छ शरणं काकुत्स्थं हितपालकम्} %॥५२॥

\twolineshloka
{यदि नायासि च कपे रामवाक्यामिदं श्रृणु}
{नयिष्ये मृत्युसदनं सुग्रीवं वालिनं यथा} %॥५३॥

\twolineshloka
{स शरो विद्यतेऽस्माकं येन वाली हतः कपिः}
{लक्ष्मणेनैवमुक्तोऽसौ सुग्रीवः कपिनायकः} %॥५४॥

\twolineshloka
{निर्गत्य तु नमश्चके लक्ष्मणं मन्त्रिणोदितः}
{उवाच च महात्मानं लक्ष्मणं वानराधिपः} %॥५५॥

\twolineshloka
{अज्ञानकृतपापानामस्माकं क्षन्तुमर्हसि}
{समयः कृतो मया राज्ञा रामेणामिततेजसा} %॥५६॥

\twolineshloka
{यस्तदानीं महाभाग तमद्यापि न लङ्घये}
{यास्यामि निखिलरैद्य कपिभिर्नृपनन्दन} %॥५७॥

\twolineshloka
{त्वया सह महावीर रामपार्श्वं न संशयः}
{मां दृष्ट्वा तत्र काकुत्स्थो यद्वक्ष्यति च मां प्रति} %॥५८॥

\twolineshloka
{तत्सर्वं शिरसा गृह्य करिष्यासि न संशयः}
{सन्ति मे हरयः शूराः सीतान्वेषणकर्मणि} %॥५९॥

\twolineshloka
{तान्यहं प्रेषयिष्यामि दिक्षु सर्वासु पार्थिव}
{इत्युक्तः कपिराजेन सुग्रीवेण स लक्ष्मणः} %॥६०॥

\twolineshloka
{इहि शीघ्रं गमिष्यामो रामपार्श्वमितोऽधुना}
{सेना चाहूयतां वीर ऋक्षाणां हरिणामपि} %॥६१॥

\twolineshloka
{यां दृष्ट्वा प्रीतिमभ्येति राघवस्ते महामते}
{इत्युक्तो लक्ष्मणेनाथ सुग्रीवः स तु वीर्यवान्} %॥६२॥

\twolineshloka
{पार्श्वस्यं युवराजानमङ्गदं सज्ञयाब्रवीत्}
{सोऽपि निर्गत्य सेनानीमाह सेनापतिं तदा} %॥६३॥

\twolineshloka
{तेनाहूताः समागत्य ऋक्षवानरकोटयः}
{गुहास्थाश्च गिरिस्थाश्च वृक्षस्थाश्चैव वानराः} %॥६४॥

\twolineshloka
{तैः सार्धं पर्वताकारैर्वानरैर्भीमविक्रमैः}
{सुग्रीवः शीघ्रमागत्य ववन्दे राघवं तदा} %॥६५॥

\twolineshloka
{लक्ष्मणोऽपि नमस्कृत्य रामं भ्रातरमब्रवीत्}
{प्रसादं कुरु सुग्रीवे विनीते चाधुना नृप} %॥६६॥

\twolineshloka
{इत्युक्तो राघवस्तेन भ्रात्रा सुग्रीवमब्रवीत्}
{आगच्छात्र महावीर सुग्रीव कुशलं तव} %॥६७॥

\twolineshloka
{श्रुत्वेत्थं रामवचनं प्रसन्नं च नराधिपम}
{शिरस्यञ्जलिमाधाय सुग्रीवो राममब्रवीत्} %॥६८॥

\twolineshloka
{तदा मे कुशलं राजन् सीतादेवी तव प्रभो}
{अन्विष्य तु यदा दत्ता मया भवति नान्यथा} %॥६९॥

\twolineshloka
{इत्युक्ते वचने तेन हनूमान्मारुतात्मजः}
{नत्वा रामं बभाषैनं सुग्रीवं कपिनायकम्} %॥७०॥

\twolineshloka
{श्रृणु सुग्रीव मे वाक्यं राजायं दुःखितो भृशम्}
{सीतावियोगेन च सदा नाश्नाति च फलादिकम्} %॥७१॥

\twolineshloka
{अस्य दुःखेन सततं लक्ष्मणोऽयं सुदुःखितः}
{एतयोरत्र यावस्था तां श्रुत्वा भरतोऽनुजः} %॥७२॥

\twolineshloka
{दुःखी भवति तददुः खाददुः खं प्राप्नोति तज्जनः}
{यत एवमतो राजन् सीतान्वेषणमाचर} %॥७३॥

\twolineshloka
{इत्युक्ते वचने तत्र वायुपुत्रेण धीमता}
{जाम्बवानतितेजस्वी नत्वा रामं पुरः स्थितः} %॥७४॥

\twolineshloka
{स प्राह कपिराजं तं नीतिमान् नीतिमद्वचः}
{यदुक्तं वायुपुत्रेण तत्तथेत्यवगच्छ भोः} %॥७५॥

\twolineshloka
{यत्र क्वापि स्थिता सीता रामभार्या यशस्विनी}
{पतिव्रता महाभागा वैदेही जनकात्मजा} %॥७६॥

\twolineshloka
{अद्यापि वृत्तसम्पन्ना इति मे मनसि स्थितम्}
{न हि कल्याणचित्तायाः सीतायाः केनचिद्भुवि} %॥७७॥

\twolineshloka
{पराभवोऽस्ति सुग्रीव प्रेषयाद्यैव वानरान्}
{इत्युक्तस्तेन सुग्रीवः प्रीतामा कपिनायकः} %॥७८॥

\twolineshloka
{पश्चिमायां दिशि तदा प्रेषयामास तान् कपीन्}
{अन्वेष्टुं रामभार्यां तां महाबलपराक्रमः} %॥७९॥

\twolineshloka
{उत्तरस्यां दिशि तदा नियुतान् वानरानसौ}
{प्रेषयामास धर्मात्मा सीतान्वेषणकर्मणि} %॥८०॥

\twolineshloka
{पूर्वस्यां दिशि कर्पीश्च कपिराजः प्रतापवान्}
{प्रेषयामास रामस्य सुभार्यान्वेषणाय वै} %॥८१॥

\twolineshloka
{इति तान् प्रेषयामास वानरान् वानराधिपः}
{सुग्रीवो वालिपुत्रं तमङ्गदं प्राह बुद्धिमान्} %॥८२॥

\twolineshloka
{त्वं गच्छ दक्षिणं देशं सीतान्वेषणकर्मणि}
{जाम्बवांश्च हनूमांश्च मैन्दो द्विविद एव च} %॥८३॥

\twolineshloka
{नीलाद्याश्चैव हरयो महाबलपराक्रमाः}
{अनुयास्यन्ति गच्छन्तं त्वामद्य मम शासनात्} %॥८४॥

\twolineshloka
{अचिरादेव यूयं तां दृष्ट्वा सीतां यशस्विनीम्}
{स्थानतो रुपतश्चैव शीलतश्च विशेषतः} %॥८५॥

\twolineshloka
{केन नीता च कुत्रास्ते ज्ञात्वात्रागच्छ पुत्रक}
{इत्युक्तः कपिराजेन पितृव्येण महात्मना} %॥८६॥

\twolineshloka
{अङ्गदस्तूर्णमुत्थाय तस्याज्ञां शिरसा दधे}
{इत्युक्ते दूरतः स्थाप्य वानरानथ जाम्बवान्} %॥८७॥

\twolineshloka
{रामं च लक्ष्मणं चैव सुग्रीवं मारुतात्मजम्}
{एकतः स्थाप्य तानाह नीतिमान् नीतिमद्वचः} %॥८८॥

\twolineshloka
{श्रूयतां वचनं मेऽद्य सीतान्वेषणकर्मणि}
{श्रुत्वा च तदगृहाण त्वं रोचते यन्नृपात्मज} %॥८९॥

\twolineshloka
{रावणेन जनस्थानान्नीयमाना तपस्विनी}
{जटायुषा तु सा दृष्ट्वा शक्त्या युद्धं प्रकुर्वता} %॥९०॥

\twolineshloka
{भूषणानि च दृष्टानि तया क्षिप्तानि तेन वै}
{तान्यस्माभिः प्रदृष्टानि सुग्रीवायार्पितानि च} %॥९१॥

\twolineshloka
{जटायुवाक्याद्राजेन्द्र सत्यमित्यवधारय}
{एतस्मात्कारणात्सीता नीता तेनैव रक्षसा} %॥९२॥

\twolineshloka
{रावणेन महाबाहो लङ्कायां वर्तते तु सा}
{त्वां स्मरन्ती तु तत्रस्था त्वद्दुःखेन सुदुःखिता} %॥९३॥

\twolineshloka
{रक्षन्ती यत्नतो वृत्तं तत्रपि जनकात्मजा}
{त्वद्ध्यानेनैव स्वान् प्राणान्धारयन्ती शुभानना} %॥९४॥

\twolineshloka
{स्थिता प्रायेण ते देवी सीता दुःखपरायणा}
{हितमेव च ते राजन्नुदधेर्लङ्घने क्षमम्} %॥९५॥

\twolineshloka
{वायुपुत्रं हनूमन्तं त्वमत्रादोष्टुमर्हसि}
{त्वं चाप्यर्हसि सुग्रीव प्रेषितुं मारुतात्मजम्} %॥९६॥

\twolineshloka
{तमृते सागरं गन्तुं वानराणां न विद्यते}
{बलं कस्यापि वा वीर इति मे मनसि स्थितम्} %॥९७॥

\twolineshloka
{क्रियतां मव्दचः क्षिप्रं हितं पथ्यं च नः सदा}
{उक्ते जाम्बवतैवं तु नीतिस्वल्पाक्षरान्विते} %॥९८॥

\twolineshloka
{वाक्ये वानरराजोऽसौ शीघ्रमुत्थाय चासनात्}
{वायुपुत्रसमीपं तु तं गत्वा वाक्यमब्रवीत्} %॥९९॥

\twolineshloka
{श्रृणु मद्वचनं वीर हनुमन्मारुतात्मज}
{अयमिक्ष्वाकुतिलको राजा रामः प्रतापवान्} %॥१००॥

\twolineshloka
{पितुरादेशमादाय भ्रातृभार्यासमन्वितः}
{प्रविष्टो दण्डकारण्यं साक्षाद्धर्मपरायणः} %॥१०१॥

\twolineshloka
{सर्वात्मा सर्वलोकेशो विष्णुर्मानुषरुपवान्}
{अस्य भार्या हता तेन दुष्टेनापि दुरात्मना} %॥१०२॥

\twolineshloka
{तद्वियोगजदुःखार्तो विचिन्वंस्तां वने वने}
{त्वया दृष्टो नृपः पूर्वमयं वीरः प्रतापवान्} %॥१०३॥

\twolineshloka
{एतेन सह सगम्य समयं चापि कारितम्}
{अनेन निहतः शत्रुर्मम वालिर्महाबलः} %॥१०४॥

\twolineshloka
{अस्य प्रसादेन कपे राज्यं प्राप्तं मयाधुना}
{मया च तत्प्रतिज्ञातमस्य साहाय्यकर्मणि} %॥१०५॥

\twolineshloka
{तत्सत्यं कर्तुमिच्छामि त्वद्वलान्मारुतात्मज}
{उत्तीर्य सागरं वीर दृष्टा सीतामनिन्दिताम्} %॥१०६॥

\twolineshloka
{भूयस्तर्तुं बलं नास्ति वानराणां त्वया विना}
{अतस्त्वमेव जानासि स्वामिकार्यं महामते} %॥१०७॥

\twolineshloka
{बलवान्नीतिमांश्चैव दक्षस्त्वं दौत्यकर्मणि}
{तेनैवमुक्तो हनुमान् सुग्रीवेण महात्मना} %॥१०८॥

\twolineshloka
{स्वामिनोऽर्थं न किं कुर्यामीदृशं किं नु भाषसे}
{इत्युक्तो वायुपुत्रेण रामस्तं पुरतः स्थितम्} %॥१०९॥

\twolineshloka
{प्राह वाक्यं महाबाहुर्वाष्पसम्पूर्णलोचनः}
{सीतां स्मृत्वा सुदुःखार्तः कालयुक्तममित्रजित्} %॥११०॥

\twolineshloka
{त्वयि भारं समारोप्य समुद्रतरणादिकम्}
{सुग्रीवः स्थाप्यते ह्यत्र मया सार्धं महामते} %॥१११॥

\twolineshloka
{हनुमंस्तत्र गच्छ त्वं मत्प्रीत्यै कृतनिश्चयः}
{ज्ञातीनां च तथा प्रीत्यै सुग्रीवस्य विशेषतः} %॥११२॥

\twolineshloka
{प्रायेण रक्षसा नीता भार्या मे जनकात्मजा}
{तत्र गच्छ महावीर यत्र सीता व्यवस्थिता} %॥११३॥

\twolineshloka
{यदि पृच्छति सादृश्यं मदाकारमशेषतः}
{अतो निरीक्ष्य मां भूयो लक्ष्मणं च ममानुजम्} %॥११४॥

\twolineshloka
{ज्ञात्वा सर्वाङ्गगं लक्ष्म सकलं चावयोरिह}
{नान्यथा विश्वसेत्सीता इति मे मनसि स्थितम्} %॥११५॥

\twolineshloka
{इत्युक्तो रामदेवेन प्रभञ्जनसुतो बली}
{उत्थाय तत्पुरः स्थित्वा कृताञ्जलिरुवाच तम्} %॥११६॥

\twolineshloka
{जानामि लक्षणं सर्वं युवयोस्तु विशेषतः}
{गच्छामि कपिभिः सार्धं त्वं शोकं मा कुरुष्व वै} %॥११७॥

\twolineshloka
{अन्यच्च देह्यभिज्ञानं विश्वासो येन मे भवेत्}
{सीतायास्तव देव्यास्तु राजन् राजीवलोचन} %॥११८॥

\twolineshloka
{इत्युक्तो वायुपुत्रेण रामः कमललोचनः}
{अङ्गुलीयकमुन्मुच्य दत्तवान् रामचिह्नितम्} %॥११९॥

\twolineshloka
{तदगृहीत्वा तदा सोऽपि हनुमान्मारुतात्मजः}
{रामं प्रदक्षिणीकृत्य लक्ष्मणं च कपीश्वरम्} %॥१२०॥

\twolineshloka
{नत्वा ततो जगामाशु हनुमानञ्जनीसुतः}
{सुग्रीवोऽपि च ताञ्छुत्वा वानरान् गन्तुमुद्यतान्} %॥१२१॥

\twolineshloka
{आज्ञेयानाज्ञापयति वानरान् बलदर्पितान्}
{श्रृण्वन्तु वानराः सर्वे शासनं मम भाषितम्} %॥१२२॥

\twolineshloka
{विलम्बनं न कर्तव्यं युष्माभिः पर्वतादिषु}
{द्रुतं गत्वा तु तां वीक्ष्य आगन्तव्यमनिन्दिताम्} %॥१२३॥

\twolineshloka
{रामपत्नीं महाभागां स्थास्येऽहं रामसन्निधौ}
{कर्तनं वा करिष्यामि अन्यथा कर्णनासयोः} %॥१२४॥

\twolineshloka
{एवं तान् प्रेषयित्वा तु आज्ञापूर्वं कपीश्वरः}
{अथ ते वानरा याताः पश्चिमादिषु दिक्षु वै} %॥१२५॥

\twolineshloka
{ते सानुषु समस्तेषु गिरीणामपि मूर्धसु}
{नदीतीरेषु सर्वेषु मुनीनामाश्रमेषु च} %॥१२६॥

\twolineshloka
{कन्दरेषु च सर्वेषु वनेषूपवनेषु च}
{वृक्षेषु वृक्षगुल्मेषु गुहासु च शिलासु च} %॥१२७॥

\twolineshloka
{सह्यपर्वतपार्श्वेषु विन्ध्यसागरपार्श्वयोः}
{हिमवत्यपि शैले च तथा किम्पुरुषादिषु} %॥१२८॥

\twolineshloka
{मनुदेशेषु सर्वेषु सप्तपातालकेषु च}
{मध्यदेशेषु सर्वेषु कश्मीरेषु महाबलाः} %॥१२९॥

\twolineshloka
{पूर्वदेशेषु सर्वेषु कामरुपेषु कोशले}
{तीर्थस्थानेषु सर्वेषु सप्तकोङ्कणकेषु च} %॥१३०॥

\twolineshloka
{यत्र तत्रैव ते सीतामदृष्ट्वा पुनरागताः}
{आगत्य ते नमस्कृत्य रामलक्ष्मणपादयोः} %॥१३१॥

\twolineshloka
{सुग्रीवं च विशेषेण नास्माभिः कमलेक्षणा}
{दृष्टा सीता महाभागेत्युक्त्वा तांस्तत्र तस्थिरे} %॥१३२॥

\twolineshloka
{ततस्तं दुःखितं प्राह रामदेवं कपीश्वरः}
{सीता दक्षिणदिग्भागे स्थिता द्रष्टुं वने नृप} %॥१३३॥

\twolineshloka
{शक्या वानरसिंहेन वायुपुत्रेण धीमता}
{दृष्टा सीतामिहायाति हनुमान्नात्र संशयः} %॥१३४॥

\twolineshloka
{स्थिरो भव महाबाहो राम सत्यमिदं वचः}
{लक्ष्मणोऽप्याह शकुनं तत्र वाक्यमिदं तदा} %॥१३५॥

\twolineshloka
{सर्वथा दृष्टसीतस्तु हनुमानागमिष्यति}
{इत्याश्वास्य स्थितौ तत्र रामं सुग्रीवलक्ष्मणौ} %॥१३६॥

\twolineshloka
{अथाङ्गदं पुरस्कृत्य ये गता वानरोत्तमाः}
{यत्नादन्वेषणार्थाय रामपत्नीं यशस्विनीम्} %॥१३७॥

\twolineshloka
{अदृष्ट्वा श्रममापन्नाः कृच्छ्रभूतास्तदा वने}
{भक्षणेन विहीनास्ते क्षुधया च प्रपीडिताः} %॥१३८॥

\twolineshloka
{भ्रमद्भिर्गहनेऽरण्ये क्वापि दृष्ट्वा च सुप्रभा}
{गुहानिवासिनी सिद्धा ऋषिपत्नी ह्यनिन्दिता} %॥१३९॥

\twolineshloka
{सा च तानागतान्दृष्ट्वा स्वाश्रमं प्रति वानरान्}
{आगताः कस्य यूयं तु कुतः किं नु प्रयोजनम्} %॥१४०॥

\twolineshloka
{इत्युक्ते जाम्बवानाह तां सिद्धां सुमहामतिः}
{सुग्रीवस्य वयं भृत्या आगता ह्यत्र शोभने} %॥१४१॥

\twolineshloka
{रामभार्यार्थमनघे सीतान्वेषणकर्मणि}
{कां दिग्भूता निराहारा अदृष्टा जनकात्मजाम्} %॥१४२॥

\twolineshloka
{इत्युक्ते जाम्बवत्यत्र पुनस्तानाह सा शुभा}
{जानामि रामं सीतां च लक्ष्मणं च कपीश्वरम्} %॥१४३॥

\twolineshloka
{भुञ्जीध्वमत्र मे दत्तमाहारं च कपीश्वराः}
{रामकार्यागतास्त्वत्र यूयं रामसमा मम} %॥१४४॥

\twolineshloka
{इत्युक्त्वा चामृतं तेषां योगाद्दत्वा तपस्विनी}
{भोजयित्वा यथाकामं भूयस्तानाह तापसी} %॥१४५॥

\twolineshloka
{सीतास्थानं तु जानाति सम्पातिर्नाम पक्षिराट्}
{आस्थितो वै वने सोऽपि महेन्द्रे पर्वते द्विजः} %॥१४६॥

\twolineshloka
{मार्गेणानेन हरयस्तत्र यूयं गमिष्यथ}
{स वक्ति सीतां सम्पातिर्दूरदर्शी तु यः खगः} %॥१४७॥

\twolineshloka
{तेनादिष्टं तु पन्थानं पुनरासाद्य गच्छथ}
{अवश्यं जानकीं सीतां द्रक्ष्यते पवनात्मजः} %॥१४८॥

\twolineshloka
{तयैवमुक्ताः कपयः परां प्रीतिमुपागताः}
{ह्यष्टास्तेजनमापन्नास्तां प्रणम्य प्रतस्थिरे} %॥१४९॥

\twolineshloka
{महेन्द्राद्रिं गता वीरा वानरास्तद्दिदृक्षया}
{तत्र सम्पातिमासीनं दृष्टवन्तः कपीश्वराः} %॥१५०॥

\twolineshloka
{तानुवाचाथ सम्पातिर्वानरानागतान्द्विजः}
{के यूयमिति सम्प्राप्ताः कस्य वा ब्रूत मा चिरम्} %॥१५१॥

\twolineshloka
{इत्युक्ते वानरा ऊचुर्यथावृत्तमनुक्रमात्}
{रामदूता वयं सर्वे सीतान्वेषणकर्मणि} %॥१५२॥

\twolineshloka
{प्रेषिताः कपिराजेन सुग्रीवेण महात्मना}
{त्वां द्रष्टुमिह सम्प्राप्ताः सिद्धाया वचनादद्विज} %॥१५३॥

\twolineshloka
{सीतास्थानं महाभाग त्वं नो वद महामते}
{इत्युक्तो वानरैः श्येनो वीक्षाचक्रे सुदक्षिणाम्} %॥१५४॥

\twolineshloka
{सीतां दृष्ट्वा स लङ्कायामशोकाख्ये महावने}
{स्थितेति कथितं तेज जटायुस्तु मृतस्तव} %॥१५५॥

\twolineshloka
{भ्रातेति चोचुः स स्नात्वा दत्त्वा तस्योदकाञ्जलिम्}
{योगमास्थाय स्वं देहं विससर्ज महामतिः} %॥१५६॥

\twolineshloka
{ततस्तं वानरा दग्ध्वा दत्त्वा तस्योदकाञ्जलिम्}
{गत्वा महेन्द्रश्रृङ्गं ते तमारुह्य क्षणं स्थिताः} %॥१५७॥

\twolineshloka
{सागरं वीक्ष्य ते सर्वे परस्परमथाब्रुवन्}
{रावणेनैव भार्या सा नीता रामस्य निश्चितम्} %॥१५८॥

\twolineshloka
{सम्पातिवचनादद्य सज्ञातं सकलं हि तत्}
{वानराणां तु कश्चात्र उत्तीर्य लवणोदधिम्} %॥१५९॥

\twolineshloka
{लङ्कां प्रविश्य दृष्ट्वा तां रामपत्नीं यशस्विनीम्}
{पुनश्चोदधितरणे शक्तिं ब्रूत हि शोभनाः} %॥१६०॥

\twolineshloka
{इत्युक्तो जाम्बवान् प्राह सर्वे शक्तास्तु वानराः}
{सागरोत्तरणे किन्तु कार्यमन्यस्य सम्भवेत्} %॥१६१॥

\twolineshloka
{तत्र दक्षोऽयमेवात्र हनुमानिति मे मतिः}
{कालक्षेपो न कर्तव्यो मासार्धमधिकं गतम्} %॥१६२॥

\twolineshloka
{यद्यदृष्ट्वा तु गच्छामो वैदेहीं वानरर्षभाः}
{कर्णनासादि नः स्वाङ्गं निकृन्तति कपीश्वरः} %॥१६३॥

\twolineshloka
{तस्मात् प्रार्थ्यः स चास्माभिर्वायुपुत्रस्तु मे मतिः}
{इत्युक्तास्ते तथेत्यूचुर्वानरा वृद्धवानरम्} %॥१६४॥

\twolineshloka
{ततस्ते प्रार्थयामासुर्वानराः पवनात्मजम्}
{हनुमन्तं महाप्राज्ञं दक्षं कार्येषु चाधिकम्} %॥१६५॥

\threelineshloka
{गच्छ त्वं रामभृत्यस्त्वं रावणस्य भयाय च}
{रक्षस्व वानरकुलमस्माकमञ्जनीसुत}
{इत्युक्तस्तांस्तथेत्याह वानरान् पवनात्मजः} %॥१६६॥

\fourlineindentedshloka
{रामप्रयुक्तश्च पुनः स्वभर्तृणा}
{पुनर्महेन्द्रे कपिभिश्च नोदितः}
{गन्तुं प्रचक्रे मतिमञ्जनीसुतः}
{समुद्रमुत्तीर्य निशाचरालयम्} %॥१६७॥

॥इति श्रीनरसिंहपुराणे रामप्रादुर्भावे पञ्चाशत्तमोऽध्यायः॥५०॥

\sect{एकपञ्चाशत्तमोऽध्यायः --- सुन्दर-काण्डः}

\uvacha{मार्कण्डेय उवाच}

\twolineshloka
{स तु रावणनीतायाः सीतायाः परिमार्गणम्}
{इयेष पदमन्वेष्टुं चारणाचरिते पथि} %॥१॥

\twolineshloka
{अञ्जलिं प्राङ्मुखं कृत्वा सगणायात्मयोनये}
{मनसाऽऽवन्द्य रामं च लक्ष्मणं च महारथम्} %॥२॥

\twolineshloka
{सागरं सरितश्चैव प्रणम्य शिरसा कपिः}
{ज्ञातीश्चैव परिष्वज्य कृत्वा चैव प्रदक्षिणाम्} %॥३॥

\twolineshloka
{अरिष्टं गच्छ पन्थानं पुण्यवायुनिषेवितम्}
{पुनरागमनायेति वानरैरभिपूजितः} %॥४॥

\twolineshloka
{अञ्जसा स्वं तथा वीर्यमाविवेशाथ वीर्यवान्}
{मार्गमालोकयन् दूरादूर्ध्वं प्रणिहितेक्षणः} %॥५॥

\twolineshloka
{सम्पूर्णमिव चात्मानं भावयित्वा महाबलः}
{उत्पपात गिरेः श्रृङ्गान्निष्पीड्य गिरिमम्बरम्} %॥६॥

\twolineshloka
{पितुर्मार्गेण यातस्य वायुपुत्रस्य धीमतः}
{रामकार्यपरस्यास्य सागरेण प्रचोदितः} %॥७॥

\twolineshloka
{विश्रामार्थं समुत्तस्थौ मैनाको लवणोदधेः}
{तं निरीक्ष्य निपीड्याथ रयात्सम्भाष्य सादरम्} %॥८॥

\twolineshloka
{उत्पतंश्च वने वीरः सिंहिकास्यं महाकपिः}
{आस्यप्रान्तं प्रविश्याथ वेगेनान्तर्विनिस्सृतः} %॥९॥

\twolineshloka
{निस्सृत्य गतवाञ्शीघ्रं वायुपुत्रः प्रतापवान्}
{लङ्घयित्वा तु तं देशं सागरं पवनात्मजः} %॥१०॥

\twolineshloka
{त्रिकूटशिखरे रम्ये वृक्षाग्रे निपपात ह}
{तस्मिन् स पर्वतश्रेष्ठे दिनं नीत्वा दिनक्षये} %॥११॥

\twolineshloka
{सन्ध्यामुपास्य हनुमान् रात्रौ लङ्कां शनैर्निशि}
{लङ्काभिधां विनिर्जित्य देवतां प्रविवेश ह} %॥१२॥

\twolineshloka
{लङ्कामनेकरत्नाढ्यां बह्वाश्चर्यसमन्विताम्}
{राक्षसेषु प्रसुप्तेषु नीतिमान् पवनात्मजः} %॥१३॥

\twolineshloka
{रावणस्य ततो वेश्म प्रविवेशाथ ऋद्धिमत्}
{शयानं रावणं दृष्ट्वा तल्पे महति वानरः} %॥१४॥

\twolineshloka
{नासापुटैर्घोरकारैर्विशद्भिर्वायुमोचकैः}
{तथैव दशभिर्वक्त्रैर्दंष्टोपेतैस्तु संयुतम्} %॥१५॥

\twolineshloka
{स्त्रीसहस्थैस्तु दृष्ट्वा तं नानाभरणभूषितम्}
{तस्मिन् सीतामदृष्ट्वा तु रावणस्य गृहे शुभे} %॥१६॥

\twolineshloka
{तथा शयानं स्वगृहे राक्षसानां च नायकम्}
{दुःखितो वायुपुत्रस्तु सम्पातेर्वचनं स्मरन्} %॥१७॥

\twolineshloka
{अशोकवनिकां प्राप्तो नानापुष्पसमन्विताम्}
{जुष्टां मलयजातेन चन्दनेन सुगन्धिना} %॥१८॥

\twolineshloka
{प्रविश्य शिंशपावृक्षमाश्रितां जनकात्मजाम्}
{रामपत्नीं समद्राक्षीद राक्षसीभिः सुरक्षिताम्} %॥१९॥

\twolineshloka
{अशोकवृक्षमारुह्य पुष्पितं मधुपल्लवम्}
{आसाचक्रे हरिस्तत्र सेयं सीतेति संस्मरन्} %॥२०॥

\twolineshloka
{सीतां निरीक्ष्य वृक्षाग्रे यावदास्तेऽनिलात्मजः}
{स्त्रीभिः परिवृतस्तत्र रावणस्तावदागतः} %॥२१॥

\twolineshloka
{आगत्य सीतां प्राहाथ प्रिये मां भज कामुकम्}
{भूषिता भव वैदेहि त्यज रामगतं मनः} %॥२२॥

\twolineshloka
{इत्येवं भाषमाणं तमन्तर्धाय तृणं ततः}
{प्राह वाक्यं शनैः सीता कम्पमानाथ रावणम्} %॥२३॥

\twolineshloka
{गच्छ रावण दुष्ट त्वं परदारपरायण}
{अचिराद्रामबाणास्ते पिबन्तु रुधिरं रणे} %॥२४॥

\twolineshloka
{तथेत्यक्तो भर्त्सितश्च राक्षसीराह राक्षसः}
{द्विमासाभ्यन्तरे चैनां वशीकुरुत मानुषीम्} %॥२५॥

\twolineshloka
{यदि नेच्छति मां सीता ततः खादत मानुषीम्}
{इत्युक्त्वा गतवान् दुष्टो रावणः स्वं निकेतनम्} %॥२६॥

\twolineshloka
{ततो भयेन तां प्राहू राक्षस्यो जनकात्मजाम्}
{रावणं भज कल्याणी सधनं सुखिनी भव} %॥२७॥

\twolineshloka
{इत्युक्ता प्राह ताः सीता राघवोऽलघुविक्रमः}
{निहत्य रावणं युद्धे सगणं मां नयिष्यति} %॥२८॥

\twolineshloka
{नाहमन्यस्य भार्या स्यामृते रामं रघूत्तमम्}
{स ह्यागत्य दशग्रीवं हत्वा मां पालयिष्यति} %॥२९॥

\twolineshloka
{इत्याकर्ण्य वचस्तस्या राक्षस्यो ददृशुर्भयम्}
{हन्यतां हन्यतामेषा भक्ष्यतां भक्ष्यतामियम्} %॥३०॥

\twolineshloka
{ततस्तास्त्रिजटा प्राह स्वप्ने दृष्टमनिन्दिता}
{श्रृणुध्वं दुष्टराक्षस्यो रावणस्य विनाशनः} %॥३१॥

\twolineshloka
{रक्षोभिः सह सर्वेस्तु रावणस्य मृतिप्रदः}
{लक्ष्मणेन सह भ्रात्रा रामस्य विजयप्रदः} %॥३२॥

\twolineshloka
{स्वप्नः शुभो मया दृष्टः सीतायाश्च पतिप्रदः}
{त्रिजटावाक्यमाकर्ण्य सीतापार्श्वं विसृज्य ताः} %॥३३॥

\twolineshloka
{राक्षस्यस्ता ययुः सर्वाः सीतामाहाञ्जनीसुतः}
{कीर्तयन् रामवृत्तान्तं सकलं पवनात्मजः} %॥३४॥

\twolineshloka
{तस्यां विश्वासमानीय दत्त्वा रामाङ्गुलीयकम्}
{सम्भाष्य लक्षणं सर्वं रामलक्ष्मणयोस्ततः} %॥३५॥

\twolineshloka
{महत्या सेनया युक्तः सुग्रीवः कपिनायकः}
{तेन सार्धमिहागत्य रामस्तव पतिः प्रभुः} %॥३६॥

\twolineshloka
{लक्ष्मणश्च महावीरो देवरस्ते शुभानने}
{रावणं सगणं हत्वा त्वामितोऽऽदाय गच्छति} %॥३७॥

\twolineshloka
{इत्युक्ते सा तु विश्वस्ता वायुपुत्रमथाब्रवीत्}
{कथमत्रागतो वीर त्वमुत्तीर्य महोदधितम्} %॥३८॥

\twolineshloka
{इत्याकर्ण्य वचस्तस्याः पुनस्तामाह वानरः}
{गोष्पदवन्मयोत्तीर्णः समुद्रोऽयं वरानने} %॥३९॥

\twolineshloka
{जपतो रामरामेति सागरो गोष्पदायते}
{दुःखमग्नासि वैदेहि स्थिरा भव शुभानने} %॥४०॥

\twolineshloka
{क्षिप्रं पश्यसि रामं त्वं सत्यमेतदब्रवीमि ते}
{इत्याश्वास्य सतीं सीतां दुःखितां जनकात्मजाम्} %॥४१॥

\twolineshloka
{ततश्चूडामणिं प्राप्य श्रुत्वा काकपराभवम्}
{नत्वा तां प्रस्थितो वीरो गन्तुं कृतमतिः कपिः} %॥४२॥

\twolineshloka
{ततो विमृश्य तद्भड्क्त्वा क्रीडावनमशेषतः}
{तोरणस्थो ननादोच्चै रामो जयति वीर्यवान्} %॥४३॥

\twolineshloka
{अनेकान् राक्षसान् हत्वा सेनाः सेनापतींश्च सः}
{तदा त्वक्षकुमारं तु हत्वा रावणसैनिकम्} %॥४४॥

\twolineshloka
{साश्वं ससारथिं हत्वा इन्द्रजित्तं गृहीतवान्}
{रावणस्य पुरः स्थित्वा रामं सकीर्त्य लक्ष्मणम्} %॥४५॥

\twolineshloka
{सुग्रीवं च महावीर्यं दग्ध्वा लङ्कामशेषतः}
{निर्भर्त्य्स्य रावण दुष्टं पुनः सम्भाष्य जानकीम्} %॥४६॥

\twolineshloka
{भूयः सागरमुत्तीर्य ज्ञातीनासाद्य वीर्यवान्}
{सीतादर्शनमावेद्य हनूमांश्चैव पूजितः} %॥४७॥

\twolineshloka
{वानरैः सार्धमागत्य हनुमान्मधुवनं महत्}
{निहत्य रक्षपालांस्तु पाययित्वा च तन्मधु} %॥४८॥

\twolineshloka
{सर्वे दधिमुखं पात्य हर्षितो हरिभिः सह}
{खमुत्पत्य च सम्प्राप्य रामलक्ष्मणपादयोः} %॥४९॥

\twolineshloka
{नत्वा तु हनुमांस्तत्र सुग्रीवं च विशेषतः}
{आदितः सर्वमावेद्य समुद्रतरणादिकम्} %॥५०॥

\twolineshloka
{कथयामास रामाय सीता द्रुष्टा मयेति वै}
{अशोकवनिकामध्ये सीता देवी सुदुःखिता} %॥५१॥

\twolineshloka
{राक्षसीभिः परिवृत्ता त्वां स्मरन्ती च सर्वदा}
{अश्रुपूर्णमुखी दीना तव पत्नी वरानना} %॥५२॥

\twolineshloka
{शीलवृत्तसमायुक्ता तत्रापि जनकात्मजा}
{सर्वत्रान्वेषमाणेन मया दुष्टा पतिव्रता} %॥५३॥

\twolineshloka
{मया सम्भाषिता सीता विश्वस्ता रघुनन्दन}
{अलङ्कारश्च सुमणिस्तया ते प्रेषितः प्रभो} %॥५४॥

\twolineshloka
{इत्युक्त्वा दत्तवांस्तस्मै चूडामणिमनुत्तमम्}
{इदं च वचनं तुभ्यं पल्या सम्प्रेषितं श्रृणु} %॥५५॥

\twolineshloka
{चित्रकूटे मदङ्के तु सुप्ते त्वयि महाव्रत}
{वायसाभिभवं राजंस्तत्किल स्मर्तुमर्हसि} %॥५६॥

\twolineshloka
{अल्पापराधे राजेन्द्र त्वया बलिभुजि प्रभो}
{यत्कृतं तन्न कर्तुं च शक्यं देवासुरैरपि} %॥५७॥

\threelineshloka
{ब्रह्मास्त्रं तु तदोत्सृष्टं रावणं किं न जेष्यसि}
{इत्येवमादि बहुशः प्रोक्त्वा सीता रुरोद ह}
{एवं तु दुःखिता सीता तां मोक्तुं यत्नमाचर} %॥५८॥

\fourlineindentedshloka
{इत्येवमुक्ते पवनात्मजेन}
{सीतावचस्तच्छुभभूषणं च}
{श्रुत्वा च दृष्ट्वा च रुरोद रामः}
{कपिं समालिङ्य शनैः प्रतस्थे} %॥५९॥

॥इति श्रीनरसिंहपुराणे रामप्रादुर्भावे एकपञ्चाशत्तमोऽध्यायः॥५१॥

\sect{द्विपञ्चाशोऽध्यायः --- युद्ध-काण्डः}

\uvacha{मार्कण्डेय उवाच}

\twolineshloka
{इति श्रुत्वा प्रियावार्तां वायुपुत्रेण कीर्तिताम्}
{रामो गत्वा समुद्रान्तं वानरैः सह विस्तृतैः} %॥१॥

\twolineshloka
{सागरस्य तटे रम्ये तालीवनविराजिते}
{सुग्रीवो जाम्बवांश्चाथ वानरैरतिहर्षितैः} %॥२॥

\twolineshloka
{सख्यातीतैर्वृतः श्रीमान् नक्षत्रैरिव चन्द्रमाः}
{अनुजेन च धीरेण वीक्ष्य तस्थौ सरित्पतिम्} %॥३॥

\twolineshloka
{रावणेनाथ लङ्कायां स सूक्तौ भर्त्सितोऽनुजः}
{विभीषणो महाबुद्धिः शास्त्रज्ञैर्मन्त्रिभिः सह} %॥४॥

\twolineshloka
{नरसिंहे महादेवे श्रीधरे भक्तवत्सले}
{एवं रामेऽचलां भक्तिमागत्य विनयात्तदा} %॥५॥

\twolineshloka
{कृताञ्जलिरुवाचेदं राममक्लिष्टकारिणम्}
{राम राम महाबाहो देवदेव जनार्दन} %॥६॥

\twolineshloka
{विभीषणोऽस्मि मां रक्ष अहं ते शरणं गतः}
{इत्युक्त्वा निपपाताथ प्राञ्जली रामपादयोः} %॥७॥

\twolineshloka
{विदितार्थोऽथ रामस्तु तमुत्थाप्य महामतिम्}
{समुद्रतोयैस्तं वीरमभिषिच्य विभीषणम्} %॥८॥

\twolineshloka
{लङ्काराज्यं तवैवेति प्रोक्तः सम्भाष्य तस्थिवान्}
{ततो विभीषणेनोक्तं त्वं विष्णुर्भुवनेश्वरः} %॥९॥

\twolineshloka
{अब्धिर्ददातु मार्गं ते देव तं याचयामहे}
{इत्युक्तो वानरैः सार्धं शिश्ये तत्र स राघवः} %॥१०॥

\twolineshloka
{सुप्ते रामे गतं तत्र त्रिरात्रमतितद्युतौ}
{ततः क्रुद्धो जगन्नाथो रामो राजीवलोचनः} %॥११॥

\twolineshloka
{संशोषणमपां कर्तुमस्त्रमाग्नेयमाददे}
{तदोत्थाय वचः प्राह लक्ष्मणश्च रुषान्वितम्} %॥१२॥

\twolineshloka
{क्रोधस्ते लयकर्ता हि एनं जहि महामते}
{भूतानां रक्षणार्थाय अवतारस्त्वया कृतः} %॥१३॥

\twolineshloka
{क्षन्तव्यं देवदेवेश इत्युक्त्वा धृतवान् शरम्}
{ततो रात्रित्रये याते कुद्धं राममवेक्ष्य सः} %॥१४॥

\twolineshloka
{आग्नेयास्त्राच्च सन्त्रस्तः सागरोऽभ्येत्य मूर्तिमान्}
{आह रामं महादेवं रक्ष मामपकारिणम्} %॥१५॥

\twolineshloka
{मार्गो दत्तो मया तेऽद्य कुशलः सेतुकर्मणि}
{नलश्च कथितो वीरस्तेन कारय राघव} %॥१६॥

\twolineshloka
{यावदिष्टं तु विस्तीर्ण सेतुबन्धमुत्तमम्}
{ततो नलमुखैरन्यैर्वानैररमितौजसैः} %॥१७॥

\twolineshloka
{बन्धयित्वा महासेतुं तेन गत्वा स राघवः}
{सुवेलाख्यं गिरिं प्राप्तः स्थितोऽसौ वानरैर्वृतः} %॥१८॥

\twolineshloka
{हर्म्यस्थलास्थितं दुष्टं रावणं वीक्ष्य चाङ्गध}
{रामादेशादथोत्प्लुत्य दूतकर्मसु तत्परः} %॥१९॥

\twolineshloka
{प्रादात्पादप्रहारं तु रोषाद्रावणमूर्धनि}
{विस्मितं तैः सुरगणैर्वीक्षितः सोऽतिवीर्यवान्} %॥२०॥

\twolineshloka
{साधयित्वा प्रतिज्ञां तां सुवेलं पुनरागतः}
{ततो वानरसेनाभिः सख्यातिताभिरच्युतः} %॥२१॥

\twolineshloka
{रुरोध रावणपुरीं लङ्कां तत्र प्रतापवान्}
{रामः समन्तादालोक्य प्राह लक्ष्मणमन्तिके} %॥२२॥

\fourlineindentedshloka
{तीर्णोऽर्णवः कवलितेव कपीश्वरस्य}
{सेनाभटैर्झटिति राक्षसराजधानीम्}
{यत्पौरुषोचितामिहाङ्कुरितं मया तद्}
{दैवस्य वश्यमपरं धनुषोऽथ वास्य} %॥२३॥

\onelineshloka*
{लक्ष्मणः प्राह--- कातरजनमनोऽवलम्बिना किं दैवेन।}

\fourlineindentedshloka
{यावल्ललाटशिखरं भ्रुकुटिर्नयाति}
{यावन्न कार्मुकशिखामधिरोहति ज्या}
{तावन्निशाचरपतेः पटिमानमेतु}
{त्रैलोक्यमूलविभुजेषु दर्पः} %॥२४॥


तदा लक्ष्मणः रामस्य कर्णे लगित्वा पितृवधवैरस्मरणे अथ 
तद्भक्तिवीर्यपरीक्षणाय लक्षणविज्ञानायादिश्यतामङ्गदाय दूत्यम्
रामः साधु इति भणित्वा अङ्गदं सबहुमानमवलोक्य आदिशति॥२५॥

अङ्गद! पिता ते यद्वाली बलिनि दशकण्ठे
कलितवान्नशक्तास्तद्वक्तुं वयमपि मुदा तेन पुलकः
स एव त्वं व्यावर्त्तयसि तनुजत्वेन पितृतां 
ततः किं वक्तव्यं तिलकयति सृष्टार्थपदवीम्॥२६॥

अङ्गदो मौलिमण्डलमिलत्करयुगलेन प्रणम्य---\\
यदाज्ञापयति देवः। अवधार्यताम्॥२७॥

किं प्राकारविहारतोरणवतीं लङ्कामिहैवानये
किं वा सैन्यमहं द्रुतं रघुपते तत्रैव सम्पादये
अत्यल्पं कुलपर्वतैरविरलैर्बध्नामि वा सागरं
देवादेशय किं करोमि सकलं दोर्द्दण्डसाध्यं मम॥२८॥

श्रीरामस्तद्वचनमात्रेणैव तद्भक्तिं सामर्थ्य चावेक्ष्य वदति॥२९॥

अज्ञानादथवाधिपत्यरभसा वास्मत्परोक्षे ह्नता सीतेयं
प्रविमुच्यतामिति वचो गत्वा दशास्यं वद
नो चेल्लोक्ष्मणमुक्तमार्गणगणच्छेदोच्छलच्छोणित-
च्छत्रच्छन्नदिगन्तमन्तकपुरीं पुत्रैर्वृतो यास्यसि॥३०॥

अङ्गदः--- देव!॥३१॥

\addtocounter{shlokacount}{7}
\twolineshloka
{सन्धौ वा विग्रहे वापि मयि दूते दशाननी}
{अक्षता वाक्षता वापि क्षितिपीठे लुठिष्यति} %॥३२॥

\twolineshloka
{तदा श्रीरामचन्द्रेण प्रशस्य प्रहितोऽङ्गदः}
{उक्तिप्रत्युक्तिचात्यर्यैः पराजित्यागतो रिपुम्} %॥३३॥

\twolineshloka
{राघवस्य बलं ज्ञात्वा चारैस्तदनुजस्य च}
{वानराणां च भीतोऽपि निर्भीरिव दशाननः} %॥३४॥

\twolineshloka
{लङ्कापुरस्य रक्षार्थमादिदेश स राक्षसान्}
{आदिश्य सर्वतो दिक्षु पुत्रानाह दशाननः} %॥३५॥

\threelineshloka
{धूम्राक्षं धूम्रपानं च राक्षसा यात मे पुरीम्}
{पाशैर्बध्नीत तौ मर्यौ अमित्रान्तकवीर्यवान्}
{कुम्भकर्णोऽपि मदभ्राता तुर्यनादैः प्रबोधितः} %॥३६॥

\twolineshloka
{राक्षसाश्चैव सन्दिष्टा रावणेन महाबलाः}
{तस्याज्ञां शिरसाऽऽदाय युयुधुर्वानरैः सह} %॥३७॥

\twolineshloka
{युध्यमाना यथाशक्त्या कोटिसख्यास्तु राक्षसाः}
{वानरैर्निधनं प्राप्ताः पुनरन्यान् यथाऽऽदिशत्} %॥३८॥

\twolineshloka
{पूर्वद्वारे दशग्रीवो राक्षसानमितौजसः}
{ते चापि युध्य हरिभिर्नीलाद्यैर्निधनं गताः} %॥३९॥

\twolineshloka
{अथ दक्षिणदिग्भागे रावणेन नियोजिताः}
{ते सर्वे वानरवरैर्दारितास्तु यमं गताः} %॥४०॥

\twolineshloka
{पश्चिमेऽङ्गदमुख्यैश्च वानरैरतिगर्वितैः}
{राक्षसाः पर्वताकाराः प्रापिता यमसादनम्} %॥४१॥

\twolineshloka
{तदुत्तरे तु दिग्भागे रावणेन निवेशिताः}
{पेतुस्ते राक्षसाः क्रूरा मैन्दाद्यैर्वानरैर्हताः} %॥४२॥

\twolineshloka
{ततो वानरसङ्घास्तु लङ्काप्राकारमुच्छ्रितम्}
{उत्प्लुत्याभ्यन्तरस्थांश्च राक्षसान् बलदर्पितान्} %॥४३॥

\twolineshloka
{हत्वा शीघ्रं पुनः प्राप्ताः स्वसेनामेव वानराः}
{एवं हतेषु सर्वेषु राक्षसेषु दशाननः} %॥४४॥

\twolineshloka
{रोदमानासु तस्त्रीषु निर्गतः क्रोधमूर्च्छितः}
{द्वारे स पश्चिमे वीरो राक्षसैर्बहुभिर्वृतः} %॥४५॥

\twolineshloka
{क्वासौ रामेति च वदन् धनुष्पाणीः प्रतापवान्}
{रथस्थः शरवर्षं च विसृजन् वानरेषु सः} %॥४६॥

\twolineshloka
{ततस्तद्वाणछिन्नाङ्गा वानरा दुद्रुवुस्तदा}
{पलायमानांस्तान् दृष्ट्वा वानरान् राघवस्तदा} %॥४७॥

\twolineshloka
{कस्मात्तु वानरा भग्नाः किमेषां भयमागतम्}
{इति रामवचः श्रुत्वा प्राह वाक्यं विभीषणः} %॥४८॥

\twolineshloka
{श्रृणु राजन् महाबाहो रावणो निर्गतोऽधुना}
{तद्वाणाछिन्ना हरयः पलायन्ते महामते} %॥४९॥

\twolineshloka
{इत्युक्तो राघवस्तेन धनुरुद्यम्य रोषितः}
{ज्याघोषतलघोषाभ्यां पूरयामास खं दिशः} %॥५०॥

\twolineshloka
{युयुधे रावणेनाथ रामः कमललोचनः}
{सुग्रीवो जाम्बवांश्चैव हनूमानङ्गदस्तथा} %॥५१॥

\twolineshloka
{विभीषणो वानराश्च लक्ष्मणश्चापि वीर्यवान्}
{उपेत्य रावणीं सेनां वर्षन्तीं सर्वसायकान्} %॥५२॥

\twolineshloka
{हस्त्यश्वरथसंयुक्तां ते निजघ्नुर्महाबलाः}
{रामरावणयोर्युद्धमभूत् तत्रापि भीषणम्} %॥५३॥

\twolineshloka
{रावणेन विसृष्टानि शस्त्रास्त्राणि च यानि वै}
{तानि छित्त्वाथ शस्त्रैस्तु राघवश्च महाबलः} %॥५४॥

\twolineshloka
{शरेण सारथिं हत्वा दशभिश्च महाहयान्}
{रावणस्य धनुश्छित्त्वा भल्लेनैकेन राघवः} %॥५५॥

\twolineshloka
{मुकुटं पञ्चदशभिश्छित्त्वा तन्मस्तकं पुनः}
{सुवर्णपुङ्खैर्दशभिः शरैर्विव्याध वीर्यवान्} %॥५६॥

\twolineshloka
{तदा दशास्यो व्यथितो रामबाणैर्भृशं तदा}
{विवेश मन्त्रिभिर्नीतः स्वपुरीं देवमर्दकः} %॥५७॥

\twolineshloka
{बोधितस्तूर्यनादैस्तु गजयूथक्रमैः शनैः}
{पुनः प्राकारमुल्लङ्घ्य कुम्भकर्णो विनिर्गतः} %॥५८॥

\twolineshloka
{उत्तुङ्गस्थूलदेहोऽसौ भीमदृष्टिर्महाबलः}
{वानरान् भक्षयन् दुष्टो विचचार क्षुधान्वितः} %॥५९॥

\twolineshloka
{तं दृष्टोत्पत्य सुग्रीवः शूलेनोरस्यताडयत्}
{कर्णद्वयं कराभ्यां तुच्छित्त्वा वक्त्रेण नासिकाम्} %॥६०॥

\twolineshloka
{सर्वतो युध्यमानांश्च रक्षोनाथान् रणेऽधिकान्}
{राघवो घातयित्वा तु वानरेन्दैः समन्ततः} %॥६१॥

\twolineshloka
{चकर्त विशिखैस्तीक्ष्णैः कुम्भकर्णस्य कन्धराम्}
{विजित्येन्द्रजितं साक्षादगरुडेनागतेन सः} %॥६२॥

\twolineshloka
{रामो लक्ष्मणसंयुक्तः शुशुभे वानरैर्वृतः}
{व्यर्थं गते चेन्द्रजिति कुम्भकर्णे निपातिते} %॥६३॥

\twolineshloka
{लङ्कानाथस्ततः कुद्धः पुत्रं त्रिशिरसं पुनः}
{अतिकायमहाकायौ देवान्तकनरान्तकौ} %॥६४॥

\twolineshloka
{यूयं हत्वा तु पुत्राद्या तौ नरौ युधि निघ्रत}
{तान्नियुज्य दशग्रीवः पुत्रानेवं पुनर्ब्रवीत्} %॥६५॥

\twolineshloka
{महोदरमहापार्श्वो सार्धमेतैर्महाबलैः}
{सग्रामेऽस्मिन् रिपून हन्तुं युवां व्रजतमुद्यतौ} %॥६६॥

\twolineshloka
{दृष्टा तानागतांश्चैव युध्यमानान् रणे रिपून्}
{अनयल्लक्ष्मणः षड्भिः शरैस्तीक्ष्णैर्यमालयम्} %॥६७॥

\twolineshloka
{वानराणां समूहश्च शिष्टांश्च रजनीचरान्}
{सुग्रीवेण हतः कुम्भो राक्षसो बलदर्पितः} %॥६८॥

\twolineshloka
{निकुम्भो वायुपुत्रेण निहतो देवकण्टकः}
{विरुपाक्षं युध्यमानं गदया तु विभीषणः} %॥६९॥

\twolineshloka
{भीममैन्दौ च श्वपतिं वानरेन्दौ निजघ्रतुः}
{अङ्गदो जाम्बवांश्चाथ हरयोऽन्यान्निशाचरान्} %॥७०॥

\twolineshloka
{युध्यमानस्तु समरे महालक्षं महाचलम्}
{जघान रामोऽथ रणे बाणवृष्टिकरं नृप} %॥७१॥

\twolineshloka
{इन्द्रजिन्मन्त्रलब्धं तु रथमारुह्य वै पुनः}
{वानरेषु च सर्वेषु शरवर्षं ववर्ष सः} %॥७२॥

\twolineshloka
{रात्रौ तद्वाणाभिन्नं तु बलं सर्वं च राघवम्}
{निश्चेष्टमखिलं दृष्ट्वा जाम्बवत्प्रेरितस्तदा} %॥७३॥

\twolineshloka
{वीर्यादौषधमानीय हनुमान मारुतात्मजः}
{भूम्यां शयानमुत्थाप्य रामं हरिगणांस्तथा} %॥७४॥

\twolineshloka
{तैरेव वानरैः सार्धं ज्वलितोल्काकरैर्निशि}
{दाहयामास लङ्कां तां हस्त्यश्वरथरक्षसाम्} %॥७५॥

\twolineshloka
{वर्षन्तं शरजालानि सर्वदिक्षु घनो यथा}
{स भ्रात्रा मेघनादं तं घातयामास राघवः} %॥७६॥

\twolineshloka
{घातितेष्वथ रक्षस्सु पुत्रमित्रादिबन्धुषु}
{कारितेष्वथ विघ्नेषु होमजप्यादिकर्मणाम्} %॥७७॥

\twolineshloka
{ततः क्रुद्धो दशग्रीवो लङ्काद्वारे विनिर्गतः}
{क्वासौ राम इति ब्रूते मानुषस्तापसाकृतिः} %॥७८॥

\twolineshloka
{योद्धा कपिबलीत्युच्चैर्व्याहरद्राक्षसाधिपः}
{वेगवद्भिर्विनीतैश्च अश्वैश्चित्ररथे स्थितः} %॥७९॥

\twolineshloka
{अथायान्तं तु तं दृष्टा रामः प्राह दशाननम्}
{रामोऽहमत्र दुष्टात्मत्रेहि रावण मां प्रति} %॥८०॥

\twolineshloka
{इत्युक्ते लक्ष्मणः प्राह रामं राजीवलोचनम्}
{अनेन रक्षसा योत्स्ये त्वं तिष्ठेति महाबल} %॥८१॥

\twolineshloka
{ततस्तु लक्ष्मणो गत्वा रुरोध शरवृष्टिभिः}
{विंशद्वाहुविसृष्टैस्तु शस्त्रास्त्रैर्लक्ष्मणं युधि} %॥८२॥

\twolineshloka
{रुरोध स दशग्रीवः तयोर्युद्धमभून्महत्}
{देवा व्योम्नि विमानस्था वीक्ष्य तस्थुर्महाहवम्} %॥८३॥

\twolineshloka
{ततो रावणशस्त्राणिच्छित्वा स्वैस्तीक्ष्णसायकैः}
{लक्ष्मणः सारथिं हत्वा स्याश्वानपि भल्लकैः} %॥८४॥

\twolineshloka
{रावणस्य धनुश्छित्तआ ध्वजं च निशितैः शरैः}
{वक्षः स्थलं महावीर्यो विव्याध परवीरहा} %॥८५॥

\twolineshloka
{ततो रथान्निपत्याधः क्षिप्रं राक्षसनायकः}
{शक्तिं जग्राह कुपितो घण्टानादविनादिनीम्} %॥८६॥

\twolineshloka
{अग्निज्वालाज्वलज्जिह्वां महोल्कासदृशद्युतिम्}
{दृढमुष्ट्या तु निक्षिप्ता शक्तिः सा लक्ष्मणोरसि} %॥८७॥

\twolineshloka
{विदार्यान्तः प्रविष्टाथ देवास्त्रस्तास्ततोऽम्बरे}
{लक्ष्मणं पतितं दृष्ट्वा रुदद्भिर्वानरेश्वरैः} %॥८८॥

\twolineshloka
{दुःखितः शीघ्रमागम्य तत्पार्शं प्राह राघवः}
{क्व गतो हनुमान वीरो मित्रो मे पवनात्मजः} %॥८९॥

\twolineshloka
{यदि जीवति मे भ्राता कथचित्पतितो भुवि}
{इत्युक्ते हनुमान राजन् वीरो विख्यातपौरुषः} %॥९०॥

\twolineshloka
{बदध्वाञ्जलिं बभाषेदं देह्यनुज्ञां स्थितोऽस्मि भोः}
{रामः प्राह महावीर विशल्यकरणी मम} %॥९१॥

\twolineshloka
{अनुजं विरुजं शीघ्रं कुरु मित्र महाबल}
{ततो वेगात्समुत्पत्य गत्वा द्रोणागिरिं कपिः} %॥९२॥

\twolineshloka
{बदध्वा च शीघ्रमानीय लक्ष्मणं नीरुजं क्षणात्}
{चकार देवदेवेशां पश्यतां राघवस्य च} %॥९३॥

\twolineshloka
{ततः कुद्धो जगन्नाथो रामः कमललोचनः}
{रावण्यस्य बलं शिष्टं हस्त्यश्वरथराक्षसम्} %॥९४॥

\twolineshloka
{हत्वा क्षणेन रामस्तु तच्छरीरं तु सायकैः}
{तीक्ष्णैर्जर्जरित्म कृत्वा रस्थिवान् वानरैर्वृतः} %॥९५॥

\twolineshloka
{अस्तचेष्टो दशग्रीवः सज्ञां प्राप्य शनैः पुनः}
{उत्थाय रावणः कुद्धः सिंहनादं ननाद च} %॥९६॥

\twolineshloka
{तत्रादश्रवणैर्व्योनि वित्रस्तो देवतागणः}
{एतस्मिन्नेव काले तु रामं प्राप्य महामुनिः} %॥९७॥

\twolineshloka
{रावणे बद्धवैरस्तु अगस्त्यो वै जयप्रदम्}
{आदित्यहदयं नाम मन्त्रं प्रादाज्जयप्रदम्} %॥९८॥

\twolineshloka
{रामोऽपि जप्त्वा तन्मत्रमगस्त्योक्तं जयप्रदम्}
{तद्दत्तं वैष्णवं चापमतुलं सद्गुणं दृढम्} %॥९९॥

\twolineshloka
{पूजायित्वा तदादाय सज्यं कृत्वा महाबलः}
{सौवर्णपुङ्खैस्तीक्ष्णैस्तु शरैर्मर्मविदारणेः} %॥१००॥

\twolineshloka
{युयुधे राक्षसेन्द्रेण रघुनाथः प्रतापवान्}
{तयोस्तु युध्यतोस्तत्र भीमशक्त्योर्महामते} %॥१०१॥

\twolineshloka
{परस्परविसृष्टस्तु व्योम्नि संवर्द्धितोऽनलः}
{समुत्थितो नृपश्रेष्ठ रामरावणयोर्युधि} %॥१०२॥

\twolineshloka
{सगरे वर्तमाने तु रामो दाशरथिस्तदा}
{पदातिर्युयुधे वीरो रामोऽनुक्तपराक्रमः} %॥१०३॥

\twolineshloka
{सहस्त्राश्वयुतं दिव्यं रथं मातलिमेव च}
{प्रेषयामास देवेन्द्रो महान्तं लोकविश्रुतम्} %॥१०४॥

\twolineshloka
{रामस्तं रथमारुह्य पूज्यमानः सुरोत्तमैः}
{मातल्युक्तोपदेशस्तु रामचन्द्रः प्रतापवान्} %॥१०५॥

\twolineshloka
{ब्रह्मदत्तवरं दुष्टं ब्रह्मास्त्रेण दशाननम्}
{जघान वैरिणं क्रूरं रामदेवः प्रतापवान्} %॥१०६॥

\twolineshloka
{रामेण निहते तत्र रावणे सगणे रिपौ}
{इन्द्राद्या देवताः सर्वाः परस्परमथाबुवन्} %॥१०७॥

\twolineshloka
{रामो भूत्वा हरिर्यस्मादस्माकं वैरिणं रणे}
{अन्यैरवध्यमप्येनं जघान युधि रावणम्} %॥१०८॥

\twolineshloka
{तस्मात्तं रामनामानमनन्तमपराजितम्}
{पूजयामोऽवतीर्यैनमित्युक्त्वा ते दिवौकसः} %॥१०९॥

\twolineshloka
{नानाविमानैः श्रीमद्भिरवतीर्य महीतले}
{रुद्रेन्द्रवसुचन्द्राद्या विधातारं सनातनम्} %॥११०॥

\twolineshloka
{विष्णुं जिष्णुं जगन्मूर्तिं सानुजं राममव्ययम्}
{तं पूजयित्वा विधिवत्परिवार्योपतास्थिरे} %॥१११॥

\twolineshloka
{रामोऽयं दृश्यतां देवा लक्ष्मणोऽयं व्यवस्थितः}
{सुग्रीवो रविपुत्रोऽयं वायुपुत्रोऽयमास्थितः} %॥११२॥

\twolineshloka
{अङ्गदाद्या इमे सर्वे इत्यूचुस्ते दिवौकसः}
{गन्धामोदितदिक्चक्रा भ्रमरालिपदानुगा} %॥११३॥

\twolineshloka
{देवस्त्रीकरनिर्मुक्ता राममूर्धनि शोभिता}
{पपात पुष्पवृष्टिस्तु लक्ष्मणस्य च मूर्धनि} %॥११४॥

\twolineshloka
{ततो ब्रह्मा समागत्य हंसयानेन राघवम्}
{अमोघाख्येन स्तोत्रेण स्तुत्वा राममवोचत} %॥११५॥

\uvacha{ब्रह्मोवाच}

\twolineshloka
{त्वं विष्णुरादिर्भूतानामनन्तो ज्ञानदृक्प्रभुः}
{त्वमेव शाश्वतं ब्रह्म वेदान्ते विदितं परम्} %॥११६॥

\twolineshloka
{त्वया यदद्य निहतो रावणो लोकरावणः}
{तदाशु सर्वलोकानां देवानां कर्म साधितम्} %॥११७॥

\twolineshloka
{इत्युक्ते पद्मयोनौ तु शङ्करः प्रीतिमास्थितः}
{प्रणम्य रामं तस्मै तं भूयो दशरथं नृपम्} %॥११८॥

\twolineshloka
{दर्शयित्वा गतो देवः सीता शुद्धेति कीर्तयन्}
{ततो बाहुबलप्राप्तं विमानं पुष्पकं शुभम्} %॥११९॥

\twolineshloka
{पूतामारोप्य सीतां तामादिष्टः पवनात्मजः}
{ततस्तु जानकीं देवीं विशोकां भूषणान्विताम्} %॥१२०॥

\twolineshloka
{वन्दितां वानरेन्दैस्तु सार्धं भ्रात्रा महाबलः}
{प्रतिष्ठाप्य महादेवं सेतुमध्ये स राघवः} %॥१२१॥

\twolineshloka
{लब्धवान् परमां भक्तिं शिवे शम्भोरनुग्रहात्}
{रामेश्वर इति ख्यातो महादेवः पिनाकधृक्} %॥१२२॥

\twolineshloka
{तस्य दर्शनमात्रेण सर्वहत्यां व्यपोहति}
{रामस्तीर्णप्रतिज्ञोऽसौ भरतासक्तमानसः} %॥१२३॥

\threelineshloka
{ततोऽयोध्यां पुरीं दिव्यां गत्वा तस्यां द्विजोत्तमैः}
{अभिषिक्तो वसिष्ठाद्यैर्भरतेन प्रसादितः}
{अकरोद्धर्मतो राज्यं चिरं रामः प्रतापवान्} %॥१२४॥

\sixlineindentedshloka
{यज्ञादिकं कर्म निजं च कृत्वा}{पौरेस्तु रामो दिवमारुरोह}
{राजन्मया ते कथितं समासतो}{रामस्य भूम्यां चरितं महात्मनः}
{इदं सुभक्त्या पठतां च श्रृण्वतां}{ददाति रामः स्वपदं जगत्पतिः} %॥१२५॥

॥इति श्रीनरसिंहपुराणे रामप्रादुर्भावे द्विपञ्चाशोऽध्यायः॥५२॥

\closesection
    \chapt{पद्म-पुराणम्}
    \input{rama-charitam/padma-puranam/pura-kalpa-ramayana-varnanam}
    \sect{द्विचत्वारिंशदधिक-द्विशततमोऽध्यायः --- रामस्यायोध्याप्रवेशः}

\src{पद्म-पुराणम्}{सृष्टिखण्डम्}{अध्यायः २४२--२४४}{}
% \tags{concise, complete}
\notes{}
\textlink{https://sa.wikisource.org/wiki/पद्मपुराणम्/खण्डः_५_(पातालखण्डः)/अध्यायः_००१}
\translink{https://www.wisdomlib.org/hinduism/book/the-padma-purana/d/doc365826.html}

\storymeta


\uvacha{रुद्र उवाच}

\twolineshloka
{स्वायम्भुवो मनुः पूर्वं द्वाशार्णं महामनुम्}
{जजाप गोमतीतीरे नैमिषे विमले शुभे}% १

\twolineshloka
{तेन वर्षसहस्रेण पूजितः कमलापतिः}
{मत्तो वरं वृणीष्वेति तं प्राह भगवान्हरिः}% २

\onelineshloka*
{ततः प्रोवाच हर्षेण मनुः स्वायम्भुवो हरिम्}

\uvacha{मनुरुवाच}
\onelineshloka
{पुत्रत्वं भज देवेश त्रीणि जन्मानि चाच्युत}% ३

\onelineshloka*
{त्वां पुत्रलालसत्वेन भजामि पुरुषोत्तमम्}

\uvacha{रुद्र उवाच}
\onelineshloka
{इत्युक्तस्तेन लक्ष्मीशः प्रोवाच सुमहागिरा}% ४

\uvacha{विष्णुरुवाच}

\twolineshloka
{भविष्यति नृपश्रेष्ठ यत्ते मनसि काङ्क्षितम्}
{ममैव च महत्प्रीतिस्तव पुत्रत्वहेतवे}% ५

\twolineshloka
{स्थितिप्रयोजने काले तत्र तत्र नृपोत्तम}
{त्वयि जाते त्वहमपि जातोस्मि तव सुव्रत}% ६

\twolineshloka
{परित्राणाय साधूनां विनाशाय च दुष्कृताम्}
{धर्म्मसंस्थापनार्थाय सम्भवामि तवानघ}% ७

\uvacha{रुद्र उवाच}

\twolineshloka
{एवं दत्वा वरं तस्मै तत्रैवान्तर्दधे हरिः}
{अस्याभूत्प्रथमं जन्म मनोः स्वायम्भुवस्य च}% ८

\twolineshloka
{रघूणामन्वये पूर्वं राजा दशरथो ह्यभूत्}
{द्वितीयो वसुदेवोऽभूद्वृष्णीनामन्वये विभुः}% ९

\twolineshloka
{कलेर्दिव्यसहस्राब्दप्रमाणस्यान्त्यपादयोः}
{शम्भलग्रामकं प्राप्य ब्राह्मणः सञ्जनिष्यति}% १०

\twolineshloka
{कौशल्या समभूत्पत्नी राज्ञो दशरथस्य हि}
{यदोर्वंशस्य सेवार्थं देवकी नाम विश्रुता}% ११

\twolineshloka
{हरिव्रतस्य विप्रस्य भार्य्या देवप्रभा पुनः}
{एवं मातृत्वमापन्ना त्रीणि जन्मानि शार्ङ्गिणः}% १२

\twolineshloka
{पूर्वं रामस्य चरितं वक्ष्यामि तव सुव्रते}
{यस्य स्मरणमात्रेण विमुक्तिः पापिनामपि}% १३

\twolineshloka
{हिरण्यकहिरण्याक्षौ द्वितीयं जन्मसंश्रितौ}
{कुम्भकर्ण दशग्रीवावजायेतां महाबलौ}% १४

\twolineshloka
{पुलस्त्यस्य सुतो विप्रो विश्रवा नाम धार्मिकः}
{तस्य पत्नी विशालाक्षी राक्षसेन्द्र सुताऽनघे}% १५

\twolineshloka
{सुकेशितनया सा स्यात्सुमाली दानवस्य च}
{केकसी नाम कन्यासीत्तस्य भार्या दृढव्रता}% १६

\twolineshloka
{कामोद्रिक्ता तु सा देवी सन्ध्याकाले महामुनिम्}
{रमयामास तन्वङ्गी यथेष्टं शुभदर्शना}% १७

\twolineshloka
{कामात्सन्ध्याभवाद्यत्वात्तस्यां जातौ महाबलौ}
{रावणः कुम्भकर्णश्च राक्षसौ लोकविश्रुतौ}% १८

\twolineshloka
{कन्या शूर्पणखा नाम जातातिविकृतानना}
{कस्यचित्त्वथ कालस्य तस्यां जातो विभीषणः}% १९

\twolineshloka
{सुशीलो भगवद्भक्तः सत्यवाग्धर्म्मवाञ्शुचिः}
{रावणः कुम्भकर्णश्च हिमवत्पर्वतोत्तमे}% २०

\twolineshloka
{महोग्रतपसा मां वै पूजयामासतुर्भृशम्}
{रावणस्त्वथ दुष्टात्मा स्वशिरःकमलैः शुभैः}% २१

\twolineshloka
{पूजयामास मां देवि दारुणेनैव कर्म्मणा}
{ततस्तमब्रुवं सुभ्रूः प्रहृष्टेनान्तरात्मना}% २२

\twolineshloka
{वरं वृणीष्व मे वत्स यत्ते मनसि वर्त्तते}
{ततः प्रोवाच दुष्टात्मा देवदानव रक्षसाम्}% २३

\twolineshloka
{अवध्यत्वं प्रदेहीति सर्वलोकजिगीषया}
{ततोऽहं दत्तवांस्तस्मै राक्षसाय दुरात्मने}% २४

\twolineshloka
{देवदानवयक्षाणामवध्यत्वं वरानने}
{राक्षसोऽसौ महावीर्यो वरदानात्तु गर्वितः}% २५

\twolineshloka
{त्रींल्लोकान्पीडयामास देवदानवमानुषान्}
{तेन सम्बाध्यमानाश्च देवा ब्रह्मपुरोगमाः}% २६

\twolineshloka
{भयार्त्ताः शरणं जग्मुरीश्वरं कमलापतिम्}
{ज्ञात्वाथ वेदनां तेषामभयाय सनातनः}% २७

\onelineshloka*
{उवाच त्रिदशान्सर्वान्ब्रह्मरुद्रपुरोगमान्}

\uvacha{श्रीभगवानुवाच}
\onelineshloka
{राज्ञो दशरथस्याहमुत्पत्स्यामि रघोः कुले}% २८

\twolineshloka
{वधिष्यामि दुरात्मानं रावणं सह बान्धवम्}
{मानुषं वपुरास्थाय हन्मि दैवतकण्टकम्}% २९

\twolineshloka
{नन्दिशापाद्भवन्तोऽपि वानरत्वमुपागताः}
{कुरुध्वं मम साहाय्यं गन्धर्वाप्सरसोत्तमाः}% ३०

\uvacha{रुद्र उवाच}

\twolineshloka
{इत्युक्ता देवतास्सर्वा देवदेवेन विष्णुना}
{वानरत्वमुपागम्य जज्ञिरे पृथिवीतले}% ३१

\twolineshloka
{भार्गवेण प्रदत्ता तु महीसागरमेखला}
{दत्ता महर्षिभिः पूर्वं रघूणां सुमहात्मनाम्}% ३२

\twolineshloka
{वैवस्वतमनोः पुत्रो राज्ञां श्रेष्ठो महाबलः}
{इक्ष्वाकुरिति विख्यातस्सर्वधर्म्मविदांवरः}% ३३

\twolineshloka
{तदन्वये महातेजा राजा दशरथो बली}
{अजस्य नृपतेः पुत्रः सत्यवान्शीलवान्शुचिः}% ३४

\twolineshloka
{स राजा पृथिवीं सर्वां पालयामास वीर्य्यतः}
{राज्येषु स्थापयामास सर्वान्पार्थिवसत्तमान्}% ३५

\twolineshloka
{कोशलस्य नृपस्याथ पुत्री सर्वाङ्गशोभना}
{कौशल्या नाम तां कन्यामुपयेमे स पार्थिवः}% ३६

\twolineshloka
{मागधस्य नृपस्याथ तनया च शुचिस्मिता}
{सुमित्रा नाम नाम्ना च द्वितीया तस्य भामिनी}% ३७

\twolineshloka
{तृतीया केकयस्याथ नृपतेर्दुहिता तथा}
{भार्य्याभूत्पद्मपत्राक्षी केकयी नाम नामतः}% ३८

\twolineshloka
{ताभिः स्म राजा भार्याभिस्तिसृभिर्धर्मसंयुतः}
{रमयामास काकुत्स्थः पृथिवीं चानुपालयन्}% ३९

\twolineshloka
{अयोध्या नाम नगरी सरयूतीर संस्थिता}
{सर्वरत्नसुसम्पूर्णा धनधान्यसमाकुला}% ४०

\twolineshloka
{प्राकारगोपुरैर्जुष्टा हेमप्राकारसङ्कुला}
{उत्तमैर्नागतुरगैर्महेन्द्रस्य यथा पुरी}% ४१

\twolineshloka
{तस्यां राजा स धर्मात्मा उवास मुनिसत्तमैः}
{पुरोहितेन विप्रेण वसिष्ठेन महात्मना}% ४२

\twolineshloka
{राज्यं चकारयामास सर्वं निहतकण्टकम्}
{यस्मादुत्पत्स्यते तस्यां भगवान्पुरुषोत्तमः}% ४३

\twolineshloka
{तस्मात्तु नगरी पुण्या साप्ययोध्येति कीर्तिता}
{नगरस्य परं धाम्नो नाम तस्याप्यभूच्छुभे}% ४४

\twolineshloka
{यत्रास्ते भगवान्विष्णुस्तदेव परमं पदम्}
{तत्र सद्यो भवेन्मोक्षः सर्वकर्म्मनिकृन्तनः}% ४५

\twolineshloka
{जाते तत्र महाविष्णौ नराः सर्वे मुदं ययुः}
{स राजा पृथिवीं सर्वां पालयित्वा शुभानने}% ४६

\twolineshloka
{अयजद्वैष्णवेष्ट्या च पुत्रार्थी हरिमच्युतम्}
{तेन सम्पूजितः श्रीशो राजा सर्वगतो हरिः}% ४७

\twolineshloka
{वैष्णवेन तु यज्ञेन वरदः प्राह केशवः}
{तस्मिन्नाविरभूदग्नौ यज्ञरूपो हरिस्तदा}% ४८

\twolineshloka
{शुद्धजाम्बूनदप्रख्यः शङ्खचक्रगदाधरः}
{शुक्लाम्बरधरः श्रीमान्सर्वभूषणभूषितः}% ४९

\twolineshloka
{श्रीवत्सकौस्तुभोरस्को वनमालाविभूषितः}
{पद्मपत्रविशालाक्षश्चतुर्बाहुरुदारधीः}% ५०

\twolineshloka
{सव्याङ्कस्थ श्रिया सार्द्धमाविरासीद्रमेश्वरः}
{वरदोस्मीति तं प्राह राजानं भक्तवत्सलः}% ५१

\twolineshloka
{तं दृष्ट्वा सर्वलोकेशं राजा हर्षसमाकुलः}
{ववन्दे भार्य्यया सार्द्धं प्रहृष्टेनान्तरात्मना}% ५२

\twolineshloka
{प्राञ्जलिः प्रणतो भूत्वा हर्षगद्गदया गिरा}
{पुत्रत्वं मे भजेत्याह देवदेवं जनार्दनम्}% ५३

\onelineshloka*
{ततः प्रसन्नो भगवान्प्राह राजानमच्युतः}

\uvacha{विष्णुरुवाच}
\onelineshloka
{उत्पत्स्येऽहं नृपश्रेष्ठ देवलोकहिताय वै}% ५४

\twolineshloka
{परित्राणाय साधूनां राक्षसानां वधाय च}
{मुक्तिं प्रदातुं लोकानां धर्म्मसंस्थापनाय च}% ५५

\uvacha{महादेव उवाच}

\twolineshloka
{इत्युक्त्वा पायसं दिव्यं हेमपात्रस्थितं शृतम्}
{लक्ष्म्याहस्तस्थितं शुभ्रं पार्थिवाय ददौ हरिः}% ५६

\uvacha{विष्णुरुवाच}

\twolineshloka
{इदं वै पायसं राजन्पत्नीभ्यस्तव सुव्रत}
{देहि ते तनयास्तासु उत्पत्स्यन्ते मदङ्गजाः}% ५७

\uvacha{महादेव उवाच}

\twolineshloka
{इत्युक्त्वा मुनिभिः सर्वैः स्तूयमानो जनार्दनः}
{स्वात्मानं दर्शयित्वाथ तथैवान्तरधीयत}% ५८

\twolineshloka
{स राजा तत्र दृष्ट्वा च पत्नीं ज्येष्ठां कनीयसीम्}
{विभज्य पायसं दिव्यं प्रददौ सुसमाहितः}% ५९

\twolineshloka
{एतस्मिन्नन्तरे पत्नी सुमित्रा तस्य मध्यमा}
{तत्समीपं प्रयाता सा पुत्रकामा सुलोचना}% ६०

\twolineshloka
{तां दृष्ट्वा तत्र कौशल्या कैकेयी च सुमध्यमा}
{अर्द्धमर्द्धं प्रददतुस्ते तस्यै पायसं स्वकम्}% ६१

\twolineshloka
{तत्प्राश्य पायसं दिव्यं राजपत्न्यः सुमध्यमाः}
{सम्पन्नगर्भाः सर्वास्ता विरेजुः शुभ्रवर्च्चसः}% ६२

\twolineshloka
{तासां स्वप्नेषु देवेशः पीतवासा जनार्दनः}
{शङ्खचक्रगदापाणिराविर्भूतस्तदा हरिः}% ६३

\twolineshloka
{अस्मिन्काले मनोरम्ये मधुमासि शुचिस्मिते}
{शुक्ले नवम्यां विमले नक्षत्रेऽदितिदैवते}% ६४

\twolineshloka
{मध्याह्नसमये लग्ने सर्वग्रहशुभान्विते}
{कौसल्या जनयामास पुत्रं लोकेश्वरं हरिम्}% ६५

\twolineshloka
{इन्दीवरदलश्यामं कोटिकन्दर्प्पसन्निभम्}
{पद्मपत्रविशालाक्षं सर्वाभरणशोभितम्}% ६६

\twolineshloka
{श्रीवत्सकौस्तुभोरस्कं सर्वाभरणभूषितम्}
{उद्यद्दिनकरप्रख्यकुण्डलाभ्यां विराजितम्}% ६७

\twolineshloka
{अनेकसूर्य्यसङ्काशं तेजसा महता वृतम्}
{परेशस्य तनो रम्यं दीपादुत्पन्नदीपवत्}% ६८

\twolineshloka
{ईशानं सर्वलोकानां योगिध्येयं सनातनम्}
{सर्वोपनिषदामर्थमनन्तं परमेश्वरम्}% ६९

\twolineshloka
{जगत्सर्गस्थितिलये हेतुभूतमनामयम्}
{शरण्यं सर्वभूतानां सर्वभूतमयं विभुम्}% ७०

\twolineshloka
{समुत्पन्ने जगन्नाथे देवदुन्दुभयो दिवि}
{विनेदुः पुष्पवर्षाणि ववर्षुः सुरसत्तमाः}% ७१

\twolineshloka
{प्रजापतिमुखा देवा विमानस्था नभस्तले}
{तुष्टुवुर्मुनिभिः सार्द्धं हर्षपूर्णाङ्गविह्वलाः}% ७२

\twolineshloka
{जगुर्गन्धर्वपतयो ननृतुश्चाप्सरोगणाः}
{ववुः पुण्यशिवा वाताः सुप्रभोभूद्दिवाकरः}% ७३

\twolineshloka
{जज्वलुश्चाग्नयः शान्ता विमलाश्च दिशोदश}
{ततस्स राजा हर्षेण पुत्रं दृष्ट्वा सनातनम्}% ७४

\twolineshloka
{पुरोधसा वसिष्ठेन जातकर्म्मतदाऽकरोत्}
{नाम चास्मै ददौ रम्यं वसिष्ठो भगवांस्तदा}% ७५

\twolineshloka
{श्रियः कमलवासिन्या रमणोऽयं महान्प्रभुः}
{तस्माच्छ्रीराम इत्यस्य नाम सिद्धं पुरातनम्}% ७६

\twolineshloka
{सहस्रनाम्नां श्रीशस्य तुल्यं मुक्तिप्रदं नृणाम्}
{विष्णुना स समुत्पन्नो विष्णुरित्यभिधीयते}% ७७

\twolineshloka
{एवं नामास्य दत्वाथ वसिष्ठो भगवानृषिः}
{परिणीय नमस्कृत्य स्तुत्वा स्तुतिभिरेव च}% ७८

\twolineshloka
{सङ्कीर्त्य नामसाहस्रं मङ्गलार्थं महात्मनः}
{विनिर्ययौ महातेजास्तस्मात्पुण्यतमाद्गृहात्}% ७९

\twolineshloka
{राजाथ विप्रमुख्येभ्यो ददौ बहुधनं मुदा}
{गवामयुतदानं च कारयामास धर्म्मतः}% ८०

\twolineshloka
{ग्रामाणां शतसाहस्रं ददौ रघुकुलोत्तमः}
{वस्त्रैराभरणैर्दिव्यैरसङ्ख्येयैर्धनैरपि}% ८१

\twolineshloka
{विष्णोस्सन्तुष्टये तत्र तर्प्पयामास भूसुरान्}
{कौसल्या च सुतं दृष्ट्वा रामं राजीवलोचनम्}% ८२

\twolineshloka
{फुल्लहस्तारविन्दाभं पद्महस्ताम्बुजान्वितम्}
{तस्य श्रीपादकमले पद्माब्जे च वरानने}% ८३

\twolineshloka
{शङ्खचक्रगदापद्मध्वजवस्त्रादिचिह्निते}
{दृष्ट्वा वक्षसि श्रीवत्सं कौस्तुभं वनमालया}% ८४

\twolineshloka
{तस्याङ्गे सा जगत्सर्वं सदेवासुरमानुषम्}
{स्मितवक्त्रे विशालाक्षी भुवनानि चतुर्दश}% ८५

\twolineshloka
{निश्वासे तस्य वेदांश्च सेतिहासान्महात्मनः}
{द्वीपानब्धीन्गिरींस्तस्य जघने वरवर्णिनि}% ८६

\twolineshloka
{नाभ्यां ब्रह्मशिवौ तस्य कर्णयोश्च दिशः शुभाः}
{नेत्रयोर्वह्निसूर्यौ च घ्राणे वायुं महाजवम्}% ८७


\threelineshloka
{सर्वोपनिषदामर्थं दृष्ट्वा तस्य विभूतयः}
{कृत्स्ना भीता वरारोहा प्रणम्य च पुनः पुनः}
{हर्षाश्रुपूर्णनयना प्राञ्जलिर्वाक्यमब्रवीत्}% ८८

\uvacha{कौशल्योवाच}

\twolineshloka
{धन्यास्मि देवदेवेश लब्ध्वा त्वां तनयं प्रभो}
{प्रसीद मे जगन्नाथ पुत्रस्नेहं प्रदर्शय}% ८९

\uvacha{ईश्वर उवाच}

\twolineshloka
{एवमुक्तो हृषीकेशो मात्रा सर्वगतो हरिः}
{मायामानुषतां प्राप्य शिशुभावाद्रुरोद सः}% ९०

\twolineshloka
{अथ प्रमुदिता देवी कौशल्या शुभलक्षणा}
{पुत्रमालिङ्ग्य हर्षेण स्तन्यं प्रादात्सुमध्यमा}% ९१

\twolineshloka
{तस्याः स्तन्यं पपौ देवो बालभावात्सनातनः}
{उवास मातुरुत्सङ्गे जगद्भर्ता महाविभुः}% ९२

\twolineshloka
{देशे तस्मिञ्छुशुभे रम्ये सर्वकामप्रदे नृणाम्}
{उत्सवं चक्रिरे पौरा हृष्टा जनपदा नराः}% ९३

\twolineshloka
{कैकेय्यां भरतो जज्ञे पाञ्चजन्यांशचोदितः}
{सुमित्रा जनयामास लक्ष्मणं शुभलक्षणम्}% ९४

\twolineshloka
{शत्रुघ्नं च महाभागा देवशत्रुप्रतापनम्}
{अनन्तांशेन सम्भूतो लक्ष्मणः परवीरहा}% ९५

\twolineshloka
{सुदर्शनांशाच्छत्रुघ्नः सञ्जज्ञेऽमितविक्रमः}
{ते सर्वे ववृधुस्तत्र वैवस्वतमनोः कुले}% ९६

\twolineshloka
{संस्कृतास्ते सुताः सम्यग्वसिष्ठेन महौजसा}
{अधीतवेदास्ते सर्वे श्रुतवन्तस्तथा नृपाः}% ९७

\twolineshloka
{सर्वशास्त्रार्थतत्वज्ञा धनुर्वेदे च निष्ठिताः}
{बभूवुः परमोदारा लोकानां हर्षवर्द्धनाः}% ९८

\twolineshloka
{युग्मं बभूवतुस्तत्र राजानौ रामलक्ष्मणौ}
{तथा भरतशत्रुघ्नौ तयोर्युग्मं बभूव ह}% ९९

\twolineshloka
{अथ लोकेश्वरी लक्ष्मीर्जनकस्य निवेशने}
{शुभक्षेत्रे हलोद्धाते सुनासीरे शुभेक्षणे}% १००

\twolineshloka
{बालार्ककोटिसङ्काशा रक्तोत्पलकराम्बुजा}
{सर्वलक्षणसम्पन्ना सर्वाभरणभूषिता}% १०१

\twolineshloka
{धृत्वा वक्षसि चार्वङ्गी मालामम्लानपङ्कजाम्}
{सीतामुखे समुत्पन्ना बालभावेन सुन्दरी}% १०२

\twolineshloka
{तां दृष्ट्वा जनको राजा कन्यां वेदमयीं शुभाम्}
{उद्धृत्यापत्यभावेन पुपोष मिथिलापतिः}% १०३

\twolineshloka
{जनकस्य गृहे रम्ये सर्वलोकेश्वरप्रिया}
{ववृधे सर्वलोकस्य रक्षणार्थं सुरेश्वरी}% १०४

\twolineshloka
{एतस्मिन्नन्तरे देवि कौशिको लोकविश्रुतः}
{सिद्धाश्रमे महापुण्ये भागीरथ्यास्तटे शुभे}% १०५

\twolineshloka
{क्रतुप्रवरमारेभे यष्टुं तत्र महामुनिः}
{वर्त्तमानस्य तस्यास्य यज्ञस्याथ द्विजन्मनः}% १०६

\twolineshloka
{क्रतुविध्वंसिनोऽभूवन्रावणस्य निशाचराः}
{कौशिकश्चिन्तयित्वाथ रघुवंशोद्भवं हरिम्}% १०७

\twolineshloka
{आनेतुमैच्छद्धर्मात्मा लोकानां हितकाम्यया}
{स गत्वा नगरीं रम्यामयोध्यां रघुपालिताम्}% १०८

\twolineshloka
{नृपश्रेष्ठं दशरथं ददर्श मुनिसत्तमः}
{राजापि कौशिकं दृष्ट्वा प्रत्युत्थाय कृताञ्जलिः}% १०९

\twolineshloka
{पुत्रैः सह महातेजा ववन्दे मुनिसत्तमम्}
{धन्योऽहमस्मीति वदन्हर्षेण रघुनन्दनम्}% ११०

\twolineshloka
{अर्चयामास विधिना निवेश्य परमासने}
{परिणीय नमस्कृत्य किं करोमीत्युवाच तम्}% १११

\onelineshloka*
{ततः प्रोवाच हृष्टात्मा विश्वामित्रो महातपाः}

\uvacha{विश्वामित्र उवाच}
\onelineshloka
{देहि मे राघवं राजन्रक्षणार्थं क्रतोर्मम}% ११२

\twolineshloka
{साफल्यमस्तु मे यज्ञे राघवस्य समीपतः}
{तस्माद्रामं रक्षणार्थं दातुमर्हसि भूपते}% ११३

\uvacha{ईश्वर उवाच}

\twolineshloka
{तच्छ्रुत्वा मुनिवर्य्यस्य वाक्यं सर्वविदां वरः}
{प्रददौ मुनिवर्य्याय राघवं सह लक्ष्मणम्}% ११४

\twolineshloka
{आदाय राघवं तत्र विश्वामित्रो महातपाः}
{स्वमाश्रममभिप्रीतः प्रययौ द्विजसत्तमः}% ११५

\twolineshloka
{ततः प्रहृष्टास्त्रिदशाः प्रयाते रघुसत्तमे}
{ववृषुः पुष्पवर्षाणि तुष्टुवुश्च महौजसः}% ११६

\twolineshloka
{अथाजगाम हृष्टात्मा वैनतेयो महाबलः}
{अदृश्यभूतो भूतानां सम्प्राप्य रघुसत्तमम्}% ११७

\twolineshloka
{ताभ्यां दिव्ये च धनुषी तूणौ चाक्षयसायकौ}
{दिव्यान्यस्त्राणि शस्त्राणि दत्वा च प्रययौ द्विजः}% ११८

\twolineshloka
{तौ रामलक्ष्मणौ वीरौ कौशिकेन महात्मना}
{गच्छन्ती ज्ञापितारण्ये राक्षसी घोरदर्शना}% ११९

\twolineshloka
{नाम्ना तु ताडका देवि भार्या सुन्दस्य रक्षसः}
{जघ्नतुस्तां महावीरौ बाणैर्दिव्यधनुश्च्युतैः}% १२०

\twolineshloka
{निहता राघवेणाथ राक्षसी घोरदर्शना}
{त्यक्त्वा तनुं घोररूपां दिव्यरूपा बभूव सा}% १२१

\twolineshloka
{जाज्वल्यमानावपुषा सर्वाभरणविभूषिता}
{प्रययौ वैष्णवं लोकं प्रणम्य च रघूत्तमौ}% १२२

\twolineshloka
{तां हत्वा राघवः श्रीमान्कौशिकस्याश्रमं शुभम्}
{प्रविवेश महातेजा लक्ष्मणेन महात्मना}% १२३

\twolineshloka
{ततः प्रहृष्टा मुनयः प्रत्युद्गम्य रघूत्तमम्}
{निवेश्य पूजयामासुरर्घाद्यैः परमात्मने}% १२४

\twolineshloka
{कौशिकः कृतदीक्षस्तु यंष्टुं यज्ञमनुत्तमम्}
{आरेभे मुनिभिः सार्द्धं विधिना मुनिसत्तमः}% १२५

\twolineshloka
{वर्त्तमाने महायज्ञे मारीचो नाम राक्षसः}
{भ्रात्रा सुबाहुना तत्र विघ्नं कर्तुमवस्थितः}% १२६

\twolineshloka
{दृष्ट्वा तौ राक्षसौ घोरौ राघवः परवीरहा}
{जघानैकेन बाणेन सुबाहुं राक्षसेश्वरम्}% १२७

\twolineshloka
{पवनास्त्रेण महता मारीचं स निशाचरम्}
{सागरे पातयामास शुष्कपर्णमिवानिलः}% १२८

\twolineshloka
{स रामस्य महावीर्य्यं दृष्ट्वा राक्षससत्तमः}
{न्यस्तशस्त्रस्तपस्तप्तुं प्रययौ महादाश्रमम्}% १२९

\twolineshloka
{विश्वामित्रो महातेजाः समाप्ते महति क्रतौ}
{प्रहृष्टमनसा तत्र पूजयामास राघवम्}% १३०

\twolineshloka
{समाश्लिष्य महात्मानं काकपक्षधरं हरिम्}
{नीलोत्पलदलश्यामं पद्मपत्रायतेक्षणम्}% १३१

\twolineshloka
{उपाघ्राय तदा मूर्ध्नि तुष्टाव मुनिसत्तमः}
{एतस्मिन्नन्तरे राजा मिथिलाया अधीश्वरः}% १३२

\twolineshloka
{वाजपेयं क्रतुं यष्टुमारेभे मुनिसत्तमैः}
{तं द्रष्टुं प्रययुस्सर्वे विश्वामित्रपुरोगमाः}% १३३

\twolineshloka
{मुनयो रघुशार्दूल सहिताः पुण्यचेतसः}
{गच्छतस्तस्य रामस्य पदाब्जेन महात्मनः}% १३४

\twolineshloka
{अभूत्सा पावनी भूमिः समाक्रान्ता महाशिला}
{सापि शप्ता पुरा भर्त्रा गौतमेन द्विजन्मना}% १३५

\twolineshloka
{अहल्या रघुनाथस्य पादस्पर्शाच्छुभाऽभवत्}
{अथ सम्प्राप्य नगरीं मिथिलां मुनिसत्तमाः}% १३६

\twolineshloka
{राघवाभ्यां तु सहिता बभूवुः प्रीतमानसाः}
{समागतान्महाभागान्दृष्ट्वा राजा महाबलः}% १३७

\twolineshloka
{प्रत्युद्गम्य प्रणम्याथ पूजयामास मैथिलः}
{रामं पद्मविशालाक्षमिन्दीवरदलप्रभम्}% १३८

\twolineshloka
{पीताम्बरधरं श्लक्ष्णं कोमलावयवोज्ज्वलम्}
{अवधीरित कन्दर्प्पकोटिलावण्यमुत्तमम्}% १३९

\twolineshloka
{सर्वलक्षणसम्पन्नं सर्वाभरणभूषितम्}
{स्वस्य हृत्पद्ममध्ये यः परेशस्य तनुर्हरिः}% १४०

\twolineshloka
{उत्पन्नो दीपवद्दीपात्सौशील्यादिगुणैः परैः}
{तं दृष्ट्वा रघुनाथं स जनको हृष्टमानसः}% १४१

\twolineshloka
{परेशमेव तं मेने रामं दशरथात्मजम्}
{पूजयामास काकुत्स्थं धन्योस्मीति ब्रुवन्नृपः}% १४२

\twolineshloka
{प्रसादं वासुदेवस्य विष्णोर्मेने नरेश्वरः}
{प्रदातुं दुहितां तस्मै मनसा चिन्तयन्प्रभुः}% १४३

\twolineshloka
{आत्मजौ रघुवंशस्य ज्ञात्वा तत्र नृपोत्तमः}
{पूजयामास धर्मेण वस्त्रैराभरणैः शुभैः}% १४४

\twolineshloka
{ऋषीन्समर्चयामास मधुपर्कादिपूजनैः}
{ततोऽवसाने यज्ञस्य रामो राजीवलोचनः}% १४५

\twolineshloka
{भङ्क्त्वा शैवं धनुर्दिव्यं जितवान्जनकात्मजाम्}
{अथासौ वीर्यशुल्केन महता परितोषितः}% १४६

\twolineshloka
{मुदा धरणिजां तस्मै प्रददौ मिथिलाधिपः}
{केशवाय श्रियमिव यथापूर्वं महार्णवः}% १४७

\twolineshloka
{स दूतं प्रेषयामास राघवं मिथिलाधिपः}
{पुत्राभ्यां सह धर्मात्मा मिथिलायां विवेश ह}% १४८

\twolineshloka
{वसिष्ठवामदेवाद्यैः प्रीतैः सह महीपतिः}
{उवास नगरे रम्ये जनकस्य रघूत्तमः}% १४९

\twolineshloka
{तस्मिन्नेव शुभे काले रामस्य धरणीसुताम्}
{विवाहमकरोद्राजा मिथिलेन समर्चितः}% १५०

\twolineshloka
{लक्ष्मणस्योर्मिलां नाम कन्यां जनकसम्भवाम्}
{जनकस्यानुजस्याथ तनये शुभवर्चसी}% १५१

\twolineshloka
{माण्डवी श्रुतकीर्त्तिश्च सर्वलक्षणलक्षिते}
{भरतस्य च सौमित्रेर्विवाहमकरोन्नृपः}% १५२

\twolineshloka
{निर्वर्त्यौद्वाहिकं तत्र राजा दशरथो बली}
{अयोध्यां प्रस्थितः श्रीमान्पौरैर्जनपदैर्वृतः}% १५३

\twolineshloka
{पारिबर्हं समादाय मैथिलेन च पूजितः}
{ससुतः सस्नुषः साश्वः सगजः सबलानुगः}% १५४

\twolineshloka
{तदध्वनि महावीर्य्यो जामदग्निः प्रतापवान्}
{गृहीत्वा परशुं चापं सङ्क्रुद्ध इव केसरी}% १५५

\twolineshloka
{अभ्यधावच्च काकुत्स्थं योद्धुकामो नृपान्तकः}
{सम्प्राप्य राघवं दृष्ट्वा वचनं प्राह भार्गवः}% १५६

\uvacha{परशुराम उवाच}

\twolineshloka
{रामराम महाबाहो शृणुष्व वचनं मम}
{बहुशः पार्थिवान्हत्वा संयुगे भूरिविक्रमान्}% १५७

\twolineshloka
{ब्राह्मणेभ्यो महीं दत्वा तपस्तप्तुमहं गतः}
{तव वीर्यबलं श्रुत्वा त्वया योद्धुमिहागतः}% १५८

\twolineshloka
{इक्ष्वाकवो न वध्या मे मातामहकुलोद्भवाः}
{वीर्य्यं क्षत्रबलं श्रुत्वा न शक्यं सहितुं मम}% १५९

\twolineshloka
{रौद्रं चापं दुराधर्षं भज्यमानां त्वया नृप}
{तस्माद्वदान्य युद्धं मे दीयतां रघुसत्तम}% १६०

\twolineshloka
{इदं तु वैष्णवं चापं तेन तुल्यमरिन्दम}
{आरोपय स्ववीर्येण निर्जितोस्मि त्वयैव हि}% १६१

\twolineshloka
{अथवा त्यज शस्त्राणि पुरस्ताद्बलिनो मम}
{शरणं भज काकुत्स्थ कातरोस्यथ चेतसी}% १६२

\uvacha{ईश्वर उवाच}

\twolineshloka
{एवमुक्तस्तु काकुत्स्थो भार्गवेण प्रतापवान्}
{तच्चापं तस्य जग्राह तच्छक्तिं वैष्णवीमपि}% १६३

\twolineshloka
{शक्त्या वियुक्तस्स तदा जामदग्निः प्रतापवान्}
{निर्वीर्यो नष्टतेजाभूत्कर्म्महीनो यथा द्विजः}% १६४

\twolineshloka
{विनष्टतेज सन्दृष्ट्वा भार्गवं नृपसत्तमाः}
{साधुसध्विति काकुत्स्थं प्रशशंसुर्मुहुर्मुहुः}% १६५

\twolineshloka
{काकुत्स्थस्तन्महच्चापं गृहीत्वारोप्य लीलया}
{सन्धाय बाणं तच्चापे भार्गवं प्राह विस्मितम्}% १६६

\uvacha{राम उवाच}

\twolineshloka
{अनेन शरमुख्येन किं कर्त्तव्यं तव द्विज}
{छेद्मि लोकमिमं चाधः स्वर्गं वा हन्मि ते तपः}% १६७

\uvacha{ईश्वर उवाच}

\twolineshloka
{तन्दृष्ट्वा घोरसङ्काशं बाणं रामस्य भार्गवः}
{ज्ञात्वा तं परमात्मानं प्रहृष्टो राममब्रवीत्}% १६८

\uvacha{परशुराम उवाच}

\twolineshloka
{रामराम महाबाहो न वेद्मि त्वां सनातनम्}
{जानाम्यद्यैव काकुत्स्थ तव वीर्य्यगुणादिभिः}% १६९

\twolineshloka
{त्वमादिपुरुषः साक्षात्परब्रह्मपरोऽव्ययः}
{त्वमनन्तो महाविष्णुर्वासुदेवः परात्परः}% १७०

\twolineshloka
{नारायणस्त्वं श्रीशस्त्वमीश्वरस्त्वं त्रयीमयः}
{त्वं कालस्त्वं जगत्सर्वमकाराख्यस्त्वमेव च}% १७१

\twolineshloka
{स्रष्टा धाता च संहर्त्ता त्वमेव परमेश्वरः}
{त्वमचिन्त्यो महद्भूतरूपस्त्वं तु मनुर्महान्}% १७२

\twolineshloka
{चतुःषट्पञ्चगुणवांस्त्वमेव पुरुषोत्तमः}
{त्वं यज्ञस्त्वं वषट्कारस्त्वमोङ्कारस्त्रयीमयः}% १७३

\twolineshloka
{व्यक्ताव्यक्तस्वरूपस्त्वं गुणभृन्निर्ग्गुणः परः}
{स्तोतुं त्वाहमशक्तश्च वेदानामप्यगोचरम्}% १७४

\twolineshloka
{यच्चापलत्वं कृतवांस्त्वां युयुत्सुतया प्रभो}
{तत्क्षन्तव्यं त्वया नाथ कृपया केवलेन तु}% १७५

\twolineshloka
{तव शक्त्या नृपान्सर्वाञ्जित्वा दत्वा महीं द्विजान्}
{त्वत्प्रसादवशादेव शान्तिमाप्नोति नैष्ठिकीम्}% १७६

\uvacha{ईश्वर उवाच}

\twolineshloka
{एवमुक्त्वा तु काकुत्स्थं जामदग्निर्महातपाः}
{परिणीय नमस्कृत्वा राघवं लोकरक्षकम्}% १७७

\twolineshloka
{शतक्रतुकृतं स्वर्गं तदस्त्राय न्यवेदयत्}
{राघवोऽथ महातेजा ववन्दे तं महामुनिम्}% १७८

\twolineshloka
{विधिवत्पूजयामास पाद्यार्घाचमनादिभिः}
{तेन सम्पूजितस्तत्र जामदग्निर्महातपाः}% १७९

\twolineshloka
{तपस्तप्तुं ययौ रम्यं नरनारायणाश्रमम्}
{राजा दशरथः सोऽथ पुत्रैर्दारसमन्वितैः}% १८०

\twolineshloka
{स्वां पुरीं सुमुहूर्त्तेन प्रविवेश महाबलः}
{राघवो लक्ष्मणश्चैव शत्रुघ्नो भरतस्तथा}% १८१

\twolineshloka
{स्वान्स्वान्दारानुपागम्य रेमिरे हृष्टमानसाः}
{तत्र द्वादश वर्षाणि सीतया सह राघवः}% १८२

\twolineshloka
{रमयामास धर्मात्मा नारायण इव श्रिया}
{तस्मिन्नेव तु राजाथ काले दशरथः सुतम्}% १८३

\twolineshloka
{ज्येष्ठं राज्येन संयोक्तुमैच्छत्प्रीत्या महीपतिः}
{तस्य भार्याथ कैकेयी पुरा दत्तवरा प्रिया}% १८४

\twolineshloka
{अयाचत नृपश्रेष्ठं भरतस्याभिषेचनम्}
{विवासनं च रामस्य वत्सराणि चतुर्दश}% १८५

\twolineshloka
{स राजा सत्यवचनाद्रामं राज्यादथोः सुतम्}
{विवासयामास तदा दुःखेन हतचेतनः}% १८६

\twolineshloka
{शक्तोऽपि राघवस्तस्मिन्राज्यं सन्त्यज्य धर्मतः}
{दशग्रीववधार्थाय पितुर्वचनहेतुना}% १८७

\twolineshloka
{वनं जगाम काकुत्स्थो लक्ष्मणेन च सीतया}
{राजा पुत्रवियोगार्त्तः शोकेन च ममार सः}% १८८

\twolineshloka
{नियुज्यमानो भरतस्तस्मिन्राज्ये समन्त्रिभिः}
{नैच्छद्राज्यं स धर्म्मात्मा सौभ्रात्रमनुदर्शयन्}% १८९

\twolineshloka
{वनमागम्य काकुत्स्थमयाचद्भ्रातरं ततः}
{रामस्तु पितुरादेशान्नैच्छद्राज्यमरिन्दमः}% १९०

\twolineshloka
{स्वपादुके ददौ तस्मै भक्त्या सोऽप्यग्रहीत्तथा}
{रामस्य पादुके राज्यमवाप्य भरतः शुभे}% १९१

\twolineshloka
{प्रत्यहं गन्धपुष्पैश्च पूजयन्कैकयीसुतः}
{तपश्चरणयुक्तेन तस्मिंस्तस्थौ नृपोत्तमः}% १९२

\twolineshloka
{यावदागमनं तस्य राघवस्य महात्मनः}
{तावद्व्रतपराः सर्वे बभूवुः पुरवासिनः}% १९३

\twolineshloka
{राघवश्चित्रकूटाद्रौ भरद्वाजाश्रमे शुभे}
{रमयामास वैदेह्या मन्दाकिन्या जले शुभे}% १९४

\twolineshloka
{कदाचिदङ्के वैदेह्याः शेते रामो महामनाः}
{ऐन्द्रिः काकस्समागम्य तस्मिन्नेव चचार ह}% १९५

\twolineshloka
{स दृष्ट्वा जानकीं तत्र कन्दर्प्पशरपीडितः}
{विददार नखैस्तीक्ष्णैः पीनोन्नतपयोधरम्}% १९६

\twolineshloka
{तं दृष्ट्वा वायसं रामः कुशं जग्राह पाणिना}
{ब्रह्मणास्त्रेण संयोज्य चिक्षेप धरणीधरः}% १९७

\twolineshloka
{तं तृणं घोरसङ्काशं ज्वालारचितविग्रहम्}
{दृष्ट्वा काकः प्रदुद्राव विमुञ्चन्कातरं स्वरम्}% १९८

\twolineshloka
{तं काकं प्रत्यनुययौ रामस्यास्त्रं सुदारुणम्}
{वायसस्त्रिषुलोकेषु बभ्राम भयपीडितः}% १९९

\twolineshloka
{यत्र यत्र ययौ काकः शरणार्थी स वायसः}
{तत्र तत्र तदस्त्रं तु प्रविवेश भयावहम्}% २००

\twolineshloka
{ब्रह्माणमिन्द्रं रुद्रं च यमं वरुणमेव च}
{शरणार्थी जगामाशु वायसः शस्त्रपीडितः}% २०१


\threelineshloka
{तं दृष्ट्वा वायसं सर्वे रुद्राद्या देव दानवाः}
{न शक्ताः स्म वयं त्रातुमिति प्राहुर्मनीषिणः}
{अथ प्रोवाच भगवान्ब्रह्मा त्रिभुवनेश्वरः}% २०२

\uvacha{ब्रह्मोवाच}

\twolineshloka
{भो भो बलिभुजां श्रेष्ठ तमेव शरणं व्रज}
{स एव रक्षकः श्रीमान्सर्वेषां करुणानिधिः}% २०३

\twolineshloka
{रक्षत्येव क्षमासारो वत्सलं शरणागतान्}
{ईश्वरः सर्वभूतानां सौशील्यादिगुणान्वितः}% २०४

\twolineshloka
{रक्षिता जीवलोकस्य पिता माता सखा सुहृत्}
{शरणं व्रज देवेशं नान्यत्र शरणं द्विज}% २०५

\uvacha{महादेव उवाच}

\twolineshloka
{इत्युक्तस्तेन बलिभुग्ब्रह्मणा रघुनन्दनम्}
{उपेत्य सहसा भूमौ निपपात भयातुरः}% २०६

\twolineshloka
{प्राणसंशयमापन्नं दृष्ट्वा सीताथ वायसम्}
{त्राहित्राहीति भर्तारमुवाच विनयाद्विभुम्}% २०७

\twolineshloka
{पुरतः पतितं देवी धरण्यां वायसं तदा}
{तच्छिरः पादयोस्तस्य योजयामास जानकी}% २०८

\twolineshloka
{समुत्थाप्य करेणाथ कृपापीयूषसागरः}
{ररक्ष रामो गुणवान् वायसं दययार्दितः}% २०९

\twolineshloka
{तमाह वायसं रामो मा भैरिति दयानिधिः}
{अभयं ते प्रदास्यामि गच्छ गच्छ यथासुखम्}% २१०

\twolineshloka
{प्रणम्य राघवायाथ सीतायै च मुहुर्मुहुः}
{स्वर्ल्लोकं प्रययावाशु राघवेण च रक्षितः}% २११

\twolineshloka
{ततो रामस्तु वैदेह्या लक्ष्मणेन च धीमता}
{उवास चित्रकूटाद्रौ स्तूयमानो महर्षिभिः}% २१२

\twolineshloka
{तस्मिन्सम्पूज्यमानस्तु भरद्वाजेन राघवः}
{जगामात्रेस्सुविपुलमाश्रमं रघुसत्तमः}% २१३

\twolineshloka
{समागतं रघुवरं दृष्ट्वा मुनिवरोत्तमः}
{भार्यया सह धर्म्मात्मा प्रत्युद्गम्य मुदा युतः}% २१४

\twolineshloka
{आसने सुशुभे मुख्ये निवेश्य सह सीतया}
{अर्घ्यपाद्याचमनीयं च वस्त्राणि विविधानि च}% २१५

\twolineshloka
{मधुपर्कन्ददौ प्रीत्या भूषणं चानुलेपनम्}
{तस्य पत्न्यनसूया तु दिव्याम्बरमनुत्तमम्}% २१६

\twolineshloka
{सीतायै प्रददौ प्रीत्या भूषणानि द्युमन्ति च}
{दिव्यान्नपानभक्षाद्यैर्भोजयामास राघवम्}% २१७

\twolineshloka
{तेन सम्पूजितस्तत्र भक्त्या परमया नृपः}
{उवास दिवसं तत्र प्रीत्या रामस्सलक्ष्मणः}% २१८

\twolineshloka
{प्रभाते विमले रामः समुत्थाय महामुनिम्}
{परिणीय प्रणम्याथ गमनायोपचक्रमे}% २१९

\twolineshloka
{अनुज्ञातस्ततस्तेन रामो राजीवलोचनः}
{प्रययौ दण्डकारण्यं महर्षिकुलसङ्कुलम्}% २२०

\twolineshloka
{तत्रातिभीषणं घोरं विराधं नाम राक्षसम्}
{हत्वाथ शरभङ्गस्य प्रविवेशाश्रमं शुभम्}% २२१

\twolineshloka
{स तु दृष्ट्वाथ काकुत्स्थं सद्यः सङ्क्षीणकल्मषः}
{प्रययौ ब्रह्मलोकं तु गन्धर्वाप्सरसान्वितम्}% २२२

\twolineshloka
{सुतीक्ष्णस्याप्यगस्त्यस्य ह्यगस्त्यभ्रातुरेव च}
{क्रमेण प्रययौ रामस्तैश्च सम्पूजितस्तथा}% २२३

\twolineshloka
{पञ्चवट्यां ततो रामो गोदावर्यास्तटे शुभे}
{उवास सुचिरं कालं सुखेन परमेण च}% २२४

\twolineshloka
{तत्र गत्वा मुनिश्रेष्ठास्तापसा धर्मचारिणः}
{पूजयामासुरात्मेशं रामं राजीवलोचनम्}% २२५

\twolineshloka
{भयं विज्ञापयामासुस्तं च रक्षोगणेरितम्}
{तानाश्वास्य तु काकुस्थो ददौ चाभयदक्षिणाम्}% २२६

\twolineshloka
{ते तु सम्पूजितास्तेन स्वाश्रमान्सम्प्रपेदिरे}
{तस्मिंस्त्रयोदशाब्दानि रामस्य परिनिर्य्ययुः}% २२७

\twolineshloka
{गोदावर्य्यास्तटे रम्ये पञ्चवट्यां मनोरमे}
{कस्यचित्त्वथ कालस्य राक्षसी घोररूपिणी}% २२८

\twolineshloka
{रावणस्य स्वसा तत्र प्रविवेश दुरासदा}
{सा तु दृष्ट्वा रघुवरं कोटिकन्दर्प्पसन्निभम्}% २२९

\twolineshloka
{इन्दीवरदलश्यामं पद्मपत्रायतेक्षणम्}
{प्रोन्नतांसं महाबाहुं कम्बुग्रीवं महाहनुम्}% २३०

\twolineshloka
{सम्पूर्णचन्द्रसदृशं सस्मिताननपङ्कजम्}
{भृङ्गावलिनिभैः स्निग्धैः कुटिलैः शीर्षजैर्वृतम्}% २३१

\twolineshloka
{रक्तारविन्दसदृशं पद्महस्ततलाङ्कितम्}
{निष्कलङ्केन्दुसदृशं नखपङ्क्तिविराजितम्}% २३२

\twolineshloka
{स्निग्धकोमलदूर्वाभं सौकुमार्य्यनिधिं शुभम्}
{पीतकौशेयवसनं सर्वाभरणभूषितम्}% २३३

\twolineshloka
{युवाकुमारवयसं जगन्मोहनविग्रहम्}
{दृष्ट्वा तं राक्षसी रामं कन्दर्प्पशरपीडिता}% २३४

\onelineshloka*
{अब्रवीत्समुपेत्याथ रामं कमललोचनम्}

\uvacha{राक्षस्युवाच}
\onelineshloka
{कस्त्वं तापसवेषेण वर्त्तसे दण्डके वने}% २३५

\twolineshloka
{आगतोऽसि किमर्थं च राक्षसानां दुरासदे}
{शीघ्रमाचक्ष्व तत्त्वेन नानृतं वक्तुमर्हसि}% २३६

\uvacha{महेश्वर उवाच}

\onelineshloka*
{इत्युक्तः स तदा रामः सम्प्रहस्याब्रवीद्वचः}

\uvacha{राम उवाच}

\twolineshloka
{राज्ञो दशरथस्याहं पुत्रो राम इतीरितः}
{असौ ममानुजो धन्वी लक्ष्मणो नाम चानघः}% २३७

\twolineshloka
{पत्नी चेयं च मे सीता जनकस्यात्मजा प्रिया}
{पितुर्वचननिर्देशादहं वनमिहागतः}% २३८

\twolineshloka
{विचरामो महारण्यमृषीणां हितकाम्यया}
{आगतासि किमर्थं त्वमाश्रमं मम सुन्दरि}% २३९

\onelineshloka*
{का त्वं कस्य कुले जाता सर्वं सत्यं वदस्व मे}

\uvacha{महेश्वर उवाच}
\onelineshloka
{इत्युक्ता सा तु रामेण प्राह वाक्यमशङ्किता}% २४०

\uvacha{राक्षस्युवाच}

\twolineshloka
{अहं विश्रवसः पुत्री रावणस्य स्वसा नृप}
{नाम्ना शूर्पणखा नाम त्रिषु लोकेषु विश्रुता}% २४१

\twolineshloka
{इदं च दण्डकारण्यं भ्रात्रा दत्तं मम प्रभो}
{भक्षयन्नृषिसङ्घान्वै विचरामि महावने}% २४२

\twolineshloka
{त्वां तु दृष्ट्वा मुनिवरं कन्दर्पशरपीडिता}
{रन्तुकामा त्वया सार्द्धमागतास्मि सुनिर्भया}% २४३

\twolineshloka
{मम त्वं नृपशार्दूल भर्ता भवितुमर्हसि}
{इमां तव सतीं सीतां ग्रसितुं भूप कामये}% २४४

\onelineshloka*
{वनेषु गिरिमुख्येषु रमयामि त्वया सह}

\uvacha{महेश्वर उवाच}

\onelineshloka
{इत्युक्त्वा राक्षसी सीतां ग्रसितुं वीक्ष्य चोद्यताम्}% २४५

\onelineshloka
{श्रीरामः खड्गमुद्यम्य नासाकर्णौ प्रचिच्छिदे}% २४६

\twolineshloka
{रुदन्ती सभयं शीघ्रं राक्षसी विकृतानना}
{खरालयं प्रविश्याह तस्य रामस्य चेष्टितम्}% २४७

\twolineshloka
{स तु राक्षससाहस्रैर्दूषणत्रिशिरो वृतः}
{आजगाम भृशं योद्धुं राघवं शत्रुसूदनः}% २४८

\twolineshloka
{तान्रामः कानने घोरे बाणः कालान्तकोपमैः}
{निजघान महाकायान्राक्षसांस्तत्र लीलया}% २४९

\twolineshloka
{खरं त्रिशिरसं चैव दूषणं तु महाबलम्}
{रणे निपातयामास बाणैराशीविषोपमैः}% २५०

\twolineshloka
{निहत्य राक्षसान्सर्वान्दण्डकारण्यवासिनः}
{पूजितः सुरसङ्घैश्च स्तूयमानो महर्षिभिः}% २५१

\twolineshloka
{उवास दण्डकारण्ये सीतया लक्ष्मणेन च}
{राक्षसानां वधं श्रुत्वा रावणः क्रोधमूर्च्छितः}% २५२

\twolineshloka
{आजगाम जनस्थानं मारीचेन दुरात्मना}
{सम्प्राप्य पञ्चवट्यां तु दशग्रीवः स राक्षसः}% २५३

\twolineshloka
{मायाविना मरीचेन मृगरूपेण रक्षसः}
{अपहृत्याश्रमाद्दूरे तौ तु दशरथात्मजौ}% २५४

\twolineshloka
{जहार सीतां रामस्य भार्यां स्ववधकाङ्क्षया}
{ह्रियमाणां तु तां दृष्ट्वा जटायुर्गृध्रराड्बली}% २५५

\twolineshloka
{रामस्य सौहृदात्तत्र युयुधे तेन रक्षसा}
{तं हत्वा बाहुवीर्येण रावणं शत्रुवारणः}% २५६

\twolineshloka
{प्रविवेश पुरीं लङ्कां राक्षसैर्बहुभिर्वृताम्}
{अशोकवनिकामध्ये निःक्षिप्य जनकात्मजाम्}% २५७

\twolineshloka
{निधनं रामबाणेन काङ्क्षयन्स्वगृहं विशत्}
{रामस्तु राक्षसं हत्वा मारीचं मृगरूपिणम्}% २५८

\twolineshloka
{पुनराविश्य तत्राथ भ्रात्रा सौमित्रिणा ततः}
{राक्षसापहृतां भार्यां ज्ञात्वा दशरथात्मजः}% २५९

\twolineshloka
{प्रभूतशोकसन्तप्तो विललाप महामतिः}
{मार्गमाणो वने सीतां पथि गृध्रं महाबलम्}% २६०

\twolineshloka
{विच्छिन्नपादपक्षं च पतितं धरणीतले}
{रुधिरापूर्णसर्वाङ्गं दृष्ट्वा विस्मयमागतः}% २६१

\twolineshloka
{पप्रच्छ राघवं श्रीमान्केन किं त्वं जिघांसितः}
{गृध्रस्तु राघवं दृष्ट्वा मन्दमन्दमुवाच ह}% २६२

\uvacha{गृध्र उवाच}

\twolineshloka
{रावणेन हृता राम तव भार्यां बलीयसा}
{तेन राक्षसमुख्येन सङ्ग्रामे निहतोस्म्यहम्}% २६३

\uvacha{महेश्वर उवाच}

\twolineshloka
{इत्युक्त्वा राघवस्याग्रे सहसा त्यक्तजीवितः}
{संस्कारमकरोद्रामस्तस्य ब्रह्मविधानतः}% २६४

\twolineshloka
{स्वपदं च ददौ तस्मै योगिगम्यं सनातनम्}
{राघवस्य प्रसादेन स गृध्रः परमं पदम्}% २६५

\twolineshloka
{हरेः सामान्यरूपेण मुक्तिं प्राप खगोत्तमः}
{माल्यवन्तं ततो गत्वा मतङ्गस्याश्रमे शुभे}% २६६

\twolineshloka
{अभिगम्य महाभागां शबरीं धर्मचारिणीम्}
{सा तु भागवतश्रेष्ठा दृष्ट्वा तौ रामलक्ष्मणौ}% २६७

\twolineshloka
{प्रत्युद्गम्य नमस्कृत्वा निवेश्य कुशविष्टरे}
{पादप्रक्षालनं कृत्वा वन्यैः पुष्पैः सुगन्धिभिः}% २६८

\twolineshloka
{अर्चयामास भक्त्या वै हर्षनिर्भरमानसा}
{फलानि च सुगन्धीनि मूलानि मधुराणि च}% २६९

\twolineshloka
{निवेदयामास तदा राघवाभ्यां दृढव्रता}
{फलान्यास्वाद्य काकुत्स्थस्तस्यै मुक्तिं ददौ पराम्}% २७०

\twolineshloka
{ततः पम्पासरो गत्वा राघवः शत्रुसूदनः}
{जघान राक्षसं तत्र कबन्धं घोररूपिणम्}% २७१

\twolineshloka
{तं निहत्य महावीर्यो ददाह स्वर्गतश्च सः}
{ततो गोदावरीं गत्वा रामो राजीवलोचनः}% २७२

\twolineshloka
{पप्रच्छ सीतां गङ्गे त्वं किं तां जानासि मे प्रियाम्}
{न शशंस तदा तस्मै सा गङ्गा तमसावृता}% २७३

\twolineshloka
{शशाप राघवः क्रोधाद्रक्ततोया भवेति ताम्}
{ततो भयात्समुद्विग्ना पुरस्कृत्य महामुनीन्}% २७४

\twolineshloka
{कृताञ्जलिपुटा दीना राघवं शरणं गता}
{ततो महर्षयस्सर्वे रामं प्राहुस्सनातनम्}% २७५

\uvacha{ऋषय ऊचुः}

\twolineshloka
{त्वत्पादकमलोद्भूता गङ्गा त्रैलोक्यपावनी}
{त्वमेव हि जगन्नाथ तां शापान्मोक्तुमर्हसि}% २७६

\uvacha{महेश्वर उवाच}

\onelineshloka*
{ततः प्रोवाच धर्मात्मा रामः शरणवत्सलः}

\uvacha{राम उवाच}

\twolineshloka
{शबर्याः स्नानमात्रेण सङ्गता शुभवारिणा}
{मुक्ता भवतु मच्छापाद्गङ्गेयं पापनाशिनी}% २७७

\twolineshloka
{एवमुक्त्वा तु काकुत्स्थः शबरीतीर्थमुत्तमम्}
{गङ्गा गयासमं चक्रे शार्ङ्गकोट्या महाबलः}% २७८

\twolineshloka
{महाभागवतानां च तीर्थं यस्योदकेऽभवत्}
{तच्छरीरं जगद्वन्द्यं भविष्यति न संशयः}% २७९

\twolineshloka
{एवमुक्त्वा तु काकुत्स्थ ऋष्यमूकं गिरिं ययौ}
{ततः पम्पासरस्तीरे वानरेण हनूमता}% २८०

\twolineshloka
{सङ्गतस्तस्य वचनात्सुग्रीवेण समागतः}
{सुग्रीववचनाद्धत्वा वालिनं वानरेश्वरम्}% २८१

\twolineshloka
{सुग्रीवमेव तद्राज्ये रामोसावभ्यषेचयत्}
{स तु सम्प्रेषयामास दिदृक्षुर्जनकात्मजाम्}% २८२

\twolineshloka
{हनुमत्प्रमुखान्वीरान्वानरान्वानराधिपः}
{स लङ्घयित्वा जलधिं हनूमान्मारुतात्मजः}% २८३

\twolineshloka
{प्रविश्य नगरीं लङ्कां दृष्ट्वा सीतां दृढव्रताम्}
{उपवासकृशां दीनां भृशं शोकपरायणाम्}% २८४

\twolineshloka
{मलपङ्केन दिग्धाङ्गीं मलिनाम्बरधारिणीम्}
{निवेदयित्वाऽभिज्ञानं प्रवृत्तिं च निवेद्य ताम्}% २८५

\twolineshloka
{सप्तमन्त्रिसुतांस्तत्र रावणस्य सुतं तथा}
{तोरणस्तम्भमुत्पाट्य निजघान स्वयं कपिः}% २८६

\twolineshloka
{समाश्वास्य च वैदेहीं बभञ्जोपवनं तदा}
{वनपालान्किङ्करांश्च पञ्चसेनाग्रनायकान्}% २८७

\twolineshloka
{रावणस्य सुतेनाथ निगृहीतो यदृच्छया}
{दृष्ट्वा च राक्षसेन्द्रं तु सम्भाषित्वा तथैव च}% २८८

\twolineshloka
{ददाह नगरीं लङ्कां स्वलाङ्गूलाग्निना कपिः}
{तया दत्तमभिज्ञानं गृहीत्वा पुनरागमत्}% २८९

\twolineshloka
{सोऽभिगम्य महातेजा रामं कमललोचनम्}
{न्यवेदयद्वानरेन्द्रो दृष्टा सीतेति तत्वतः}% २९०

\twolineshloka
{सुग्रीवसहितो रामो वानरैर्बहुभिर्वृतः}
{महोदधेस्तटं गत्वा तत्रानीकं न्यवेशयत्}% २९१

\twolineshloka
{रावणस्यानुजो भ्राता विभीषण इतीरितः}
{धर्मात्मा सत्यसन्धश्च महाभागवतोत्तमः}% २९२

\twolineshloka
{ज्ञात्वा समागतं रामं परित्यज्य स्वपूर्वजम्}
{राज्यं सुतांश्च दारांश्च राघवं शरणं ययौ}% २९३

\twolineshloka
{परिगृह्य च तं रामो मारुतेर्वचनात्प्रभुः}
{तस्मै दत्वाऽभयं सौम्यं रक्षो राज्येऽभ्यषेचयत्}% २९४

\twolineshloka
{ततस्समुद्रं काकुत्स्थस्तर्तुकामः प्रपद्य वै}
{सुप्रसन्नजलं तं तु दृष्ट्वा रामो महाबलः}% २९५

\twolineshloka
{शार्ङ्गमादाय बाणौघैः शोषयामास वारिधिम्}
{ततस्तु सरितामीशः काकुत्स्थं करुणानिधिम्}% २९६

\twolineshloka
{प्रपद्य शरणं देवमर्चयामास वारिधिः}
{पुनरापूर्य जलधिं वरुणास्त्रेण राघवः}% २९७

\twolineshloka
{उदधेर्वचनात्सेतुं सागरे मकरालये}
{गिरिभिर्वानरानीतैर्नलः सेतुमकारयत्}% २९८

\twolineshloka
{ततो गत्वा पुरीं लङ्कां सन्निवेश्य महाबलम्}
{सम्यगायोधनं चक्रे वानराणां च रक्षसाम्}% २९९

\twolineshloka
{ततो दशास्यतनयः शक्रजिद्राक्षसो बली}
{बबन्ध नागपाशैश्च तावुभौ रामलक्ष्मणौ}% ३००

\twolineshloka
{वैनतेयः समागत्य तान्यस्त्राणि प्रमोचयत्}
{राक्षसा निहतास्सर्वे वानरैश्च महाबलैः}% ३०१

\twolineshloka
{रावणस्यानुजं वीरं कुम्भकर्णं महाबलम्}
{निजघान रणे रामो बाणैरग्निशिखोपमैः}% ३०२

\twolineshloka
{ब्रह्मास्त्रेणेन्द्रजित्क्रुद्धः पातयामास वानरान्}
{हनूमता समानीतो महौषधि महीधरः}% ३०३

\twolineshloka
{तस्यानीतस्य च स्पर्शात्सर्व एव समुत्थिताः}
{ततो रामानुजो वीरः शक्रजेतारमाहवे}% ३०४

\twolineshloka
{निपातयामास शरैर्वृत्रं वज्रधरो यथा}
{निर्ययावथ पौलस्त्यो योद्धुं रामेण संयुगे}% ३०५

\twolineshloka
{चतुरङ्गबलैः सार्द्धं मन्त्रिभिश्च महाबलः}
{समन्ततोभवद्युद्धं वानराणां च रक्षसाम्}% ३०६

\twolineshloka
{रामरावणयोश्चैव तथा सौमित्रिणा सह}
{शक्त्या निपातयामास लक्ष्मणं राक्षसेश्वरः}% ३०७

\twolineshloka
{ततः क्रुद्धो महातेजा राघवो राक्षसान्तकः}
{जघान राक्षसान्वीराञ्शरैः कालान्तकोपमैः}% ३०८

\twolineshloka
{प्रदीप्तैर्बाणसाहस्रैः कालदण्डोपमैर्भृशम्}
{छादयामास काकुत्स्थो दशग्रीवं च राक्षसम्}% ३०९

\twolineshloka
{स तु निर्भिन्नसर्वाङ्गो राघवास्त्रैर्निशाचरः}
{भयात्प्रदुद्राव रणाल्लङ्कां प्रति निशाचरः}% ३१०

\twolineshloka
{जगद्राममयं पश्यन्निर्वेदाद्गृहमाविशत्}
{ततो हनूमता नीतो महौषधिमहागिरिः}% ३११

\twolineshloka
{तेन रामानुजस्तूर्णं लब्धसंज्ञोऽभवत्तदा}
{दशग्रीवस्ततो होममारेभे जयकाङ्क्षया}% ३१२

\twolineshloka
{ध्वंसितं वानरेन्द्रैस्तदभिचारात्मकं रिपोः}
{पुनर्युद्धाय पौलस्त्यो रामेण सह निर्ययौ}% ३१३

\twolineshloka
{दिव्यस्यन्दनमारुह्य राक्षसैर्बहुभिर्युतः}
{ततः शतमखो दिव्यं रथं हर्यश्वसंयुतम्}% ३१४

\twolineshloka
{राघवाय ससूतं हि प्रेषयामास बुद्धिमान्}
{रथं मातलिना नीतं समारुह्य रघूत्तमः}% ३१५

\twolineshloka
{स्तूयमानं सुरगणैर्युयुधे तेन रक्षसा}
{ततो युद्धमभूद्धोरं रामरावणयोर्महत्}% ३१६

\twolineshloka
{सप्ताह्निकमहोरात्रं शस्त्रास्त्रैरतिभीषणम्}
{विमानस्थाः सुरास्सर्वे ददृशुस्तत्र संयुगम्}% ३१७

\twolineshloka
{दशग्रीवस्य चिच्छेद शिरांसि रघुसत्तमः}
{समुत्थितानि बहुशो वरदानात्कपर्दिनः}% ३१८

\twolineshloka
{ब्राह्ममस्त्रं महारौद्रं वधायास्य दुरात्मनः}
{ससर्ज राघवस्तूर्णं कालाग्निसदृशप्रभम्}% ३१९

\twolineshloka
{तदस्त्रं राघवोत्सृष्टं रावणस्य स्तनान्तरम्}
{विदार्य धरणीं भित्त्वा रसातलतले गतम्}% ३२०

\twolineshloka
{सम्पूज्यमानं भुजगै राघवस्य करं ययौ}
{स गतासुर्महादैत्यः पपात च ममार च}% ३२१

\twolineshloka
{ततो देवगणास्सर्वे हर्षनिर्भरमानसाः}
{ववृषुः पुष्पवर्षाणि महात्मनि जगद्गुरौ}% ३२२

\twolineshloka
{जगुर्गन्धर्वपतयो ननृतुश्चाप्सरोगणाः}
{ववुः पुण्यास्तथा वाताः सुप्रभोऽभूद्दिवाकरः}% ३२३

\twolineshloka
{तुष्टुवुर्मुनयः सिद्धा देवगन्धर्वकिन्नराः}
{लङ्कायां राक्षसश्रेष्ठमभिषिच्य विभीषणम्}% ३२४

\twolineshloka
{कृतकृत्यमिवात्मानं मेने रघुकुलोत्तमः}
{रामस्तत्राब्रवीद्वाक्यमभिषिच्य विभीषणम्}% ३२५

\uvacha{राम उवाच}

\twolineshloka
{यावच्चन्द्रश्च सूर्यश्च यावत्तिष्ठति मेदिनी}
{यावन्ममकथालोके तावद्राज्यं विभीषणे}% ३२६

\twolineshloka
{गत्वा मम पदं दिव्यं योगिगम्यं सनातनम्}
{सपुत्रपौत्रः सगणः सम्प्राप्नुहि महाबलः}% ३२७

\uvacha{ईश्वर उवाच}

\twolineshloka
{एवं दत्वा वरं तस्मै राक्षसाय महाबलः}
{सम्प्राप्य मैथिलीं तत्र परुषं जनसंसदि}% ३२८

\twolineshloka
{उवाच राघवः सीतां गर्हितं वचनं बहु}
{सा तेन गर्हिता साध्वी विवेश चानलं महत्}% ३२९


\threelineshloka
{ततो देवगणास्सर्वे शिवब्रह्मपुरोगमाः}
{दृष्ट्वा तु मातरं वह्नौ प्रविशन्तीं भयातुराः}
{समागम्य रघुश्रेष्ठं सर्वे प्राञ्जलयोऽब्रुवन्}% ३३०

\uvacha{देवा ऊचुः}

\twolineshloka
{रामराम महाबाहो शृणु त्वं चातिविक्रम}
{सीतातिविमला साध्वी तव नीत्यानपायिनी}% ३३१

\twolineshloka
{अत्याज्या तु वृथा सा हि भास्करेण प्रभा यथा}
{सेयं लोकहितार्थाय समुत्पन्ना महीतले}% ३३२

\twolineshloka
{माता सर्वस्य जगतः समस्तजगदाश्रया}
{रावणः कुम्भकर्णश्च भृत्यौ पूर्वपरायणौ}% ३३३

\twolineshloka
{शापात्तौ सनकादीनां समुत्पन्नौ महीतले}
{तयोर्विमुक्त्यै वैदेही गृहीता दण्डके वने}% ३३४

\twolineshloka
{तावुभौ वै वधं प्राप्तौ त्वया राक्षसपुङ्गवौ}
{तौ विमुक्तौ दिवं यातौ पुत्रपौत्रसहानुगौ}% ३३५

\twolineshloka
{त्वं विष्णुस्त्वं परं ब्रह्म योगिध्येयः सनातनः}
{त्वमेव सर्वदेवानामनादिनिधनोऽव्ययः}% ३३६

\twolineshloka
{त्वं हि नारायणः श्रीमान्सीता लक्ष्मीः सनातनी}
{माता सा सर्वलोकानां पिता त्वं परमेश्वर}% ३३७

\twolineshloka
{नित्यैवैष जगन्माता तव नित्यानपायिनी}
{यथा सर्वगतस्त्वं हि तथा चेयं रघूत्तम}% ३३८

\twolineshloka
{तस्माच्छुद्धसमाचारां सीतां साध्वीं दृढव्रताम्}
{गृहाण सौम्य काकुत्स्थ क्षीराब्धेरिव मा चिरम्}% ३३९

\uvacha{ईश्वर उवाच}


\threelineshloka
{एतस्मिन्नन्तरे तत्र लोकसाक्षी स पावकः}
{आदाय सीतां रामाय प्रददौ सुरसन्निधौ}
{अब्रवीत्तत्र काकुत्स्थं वह्निः सर्वशरीरगः}% ३४०

\uvacha{वह्निरुवाच}

\twolineshloka
{इयं शुद्धसमाचारा सीता निष्कल्मषा विभो}
{गृहाण मा चिरं राम सत्यं सत्यं तवाब्रुवन्}% ३४१

\uvacha{ईश्वर उवाच}

\twolineshloka
{ततोऽग्निवचनात्सीतां परिगृह्य रघूद्वहः}
{बभूव रामः संहृष्टः पूज्यमानः सुरोत्तमैः}% ३४२

\twolineshloka
{राक्षसैर्निहता ये तु सङ्ग्रामे वानरोत्तमाः}
{पितामहवरात्तूर्णं जीवमानाः समुत्थिताः}% ३४३

\twolineshloka
{ततस्तु पुष्पकं नाम विमानं सूर्यसन्निभम्}
{भ्रात्रा गृहीतं सङ्ग्रामे कौबेरं राक्षसेश्वरः}% ३४४

\twolineshloka
{तद्राघवाय प्रददौ वस्त्राण्याभरणानि च}
{तेन सम्पूजितः श्रीमान्रामचन्द्रः प्रतापवान्}% ३४५

\twolineshloka
{आरुरोह विमानाग्र्यं वैदेह्या भार्यया सह}
{लक्ष्मणेन च शूरेण भ्रात्रा दशरथात्मजः}% ३४६

\twolineshloka
{ऋक्षवानरसङ्घातैः सुग्रीवेण महात्मना}
{विभीषणेन शूरेण राक्षसैश्च महाबलैः}% ३४७

\twolineshloka
{यथाविमाने वैकुण्ठे नित्यमुक्तैर्महात्मभिः}
{तथा सर्वे समारुह्य ऋक्षवानरराक्षसाः}% ३४८

\twolineshloka
{अयोध्यां प्रस्थितो रामः स्तूयमानः सुरोत्तमैः}
{भरद्वाजाश्रमं गत्वा रामः सत्यपराक्रमः}% ३४९

\twolineshloka
{भरतस्यान्तिके तत्र हनूमन्तं व्यसर्जयत्}
{स निषादालयं गत्वा गुहं दृष्ट्वाऽथ वैष्णवम्}% ३५०

\twolineshloka
{राघवागमनं तस्मै प्राह वानरपुङ्गवः}
{नन्दिग्रामं ततो गत्वा दृष्ट्वा तं राघवानुजम्}% ३५१

\twolineshloka
{न्यवेदयत्तथा तस्मै रामस्यागमनोत्सवम्}
{भरतश्चागतं श्रुत्वा वानरेण रघूत्तमम्}% ३५२

\twolineshloka
{प्रर्हर्षमतुलं लेभे सानुजः ससुहृज्जनः}
{पुनरागत्य काकुत्स्थं हनूमान्मारुतात्मजः}% ३५३

\twolineshloka
{सर्वं शशंस रामाय भरतस्य च वर्तितम्}
{राघवस्तु विमानाग्र्यादवरुह्य सहानुजः}% ३५४

\twolineshloka
{ववन्दे भार्यया सार्द्धं भारद्वाजं तपोनिधिम्}
{स तु सम्पूजयामास काकुत्स्थं सानुजं मुनिः}% ३५५

\twolineshloka
{पक्वान्नैः फलमूलाद्यैर्वस्त्रैराभरणैरपि}
{तेन सम्पूजितस्तत्र प्रणम्य मुनिसत्तमम्}% ३५६

\twolineshloka
{अनुज्ञातः समारुह्य पुष्पकं सानुगस्तदा}
{नन्दिग्रामं ययौ रामः पुष्पकेण सुहृद्वृतः}% ३५७

\twolineshloka
{मन्त्रिभिः पौरमुख्यैश्च सानुजः केकयीसुतः}
{प्रत्युद्ययौ नृपवरैः सबलैः पूर्वजं मुदा}% ३५८

\twolineshloka
{सम्प्राप्य रघुशार्दूलं ववन्दे सानुगैर्वृतः}
{पुष्पकादवरुह्याथ राघवः शत्रुतापनः}% ३५९

\twolineshloka
{भरतं चैव शत्रुघ्नमुपसम्परिषस्वजे}
{पुरोहितं वसिष्ठं च मातृवृद्धांश्च बान्धवान्}% ३६०

\twolineshloka
{प्रणनाम महातेजाः सीतया लक्ष्मणेन च}
{विभीषणं च सुग्रीवं जाम्बवन्तं तथाङ्गदम्}% ३६१

\twolineshloka
{हनुमन्तं सुषेणं च भरतः परिषस्वजे}
{भ्रातृभिः सानुगैस्तत्र मङ्गलस्नानपूर्वकम्}% ३६२

\twolineshloka
{दिव्यमाल्याम्बरधरो दिव्यगन्धानुलेपनः}
{आरुरोह रथं दिव्यं सुमन्त्राधिष्ठितं शुभम्}% ३६३

\twolineshloka
{संस्तूयमानस्त्रिदशैर्वैदेह्या लक्ष्मणेन च}
{भरतश्चैव सुग्रीवः शत्रुघ्नश्च विभीषणः}% ३६४

\twolineshloka
{अङ्गदश्च सुषेणश्च जाम्बवान्मारुतात्मजः}
{नीलो नलश्च सुभगः शरभो गन्धमादनः}% ३६५

\twolineshloka
{अन्ये च कपयः शूरा निषादाधिपतिर्गुहः}
{राक्षसाश्च महावीर्याः पार्थिवेन्द्रा महाबलाः}% ३६६

\twolineshloka
{गजानश्वानथो सम्यगारुह्य बहुशः शुभान्}
{नानामङ्गलवादित्रैः स्तुतिभिः पुष्कलैस्तथा}% ३६७

\twolineshloka
{ऋक्षवानररक्षोभिर्निषादवरसैनिकैः}
{प्रविवेश महातेजाः साकेतं पुरमव्ययम्}% २६८

\twolineshloka
{आलोक्य राजनगरीं पथि राजपुत्रो राजानमेव पितरं परिचिन्तयानः}
{सुग्रीवमारुतिविभीषणपुण्यपादसञ्चारपूतभवनं प्रविवेश रामः}% ३६९

{॥इति श्रीपाद्मे महापुराणे पञ्चपञ्चाशत्साहस्र्यां संहितायामुत्तरखण्डे उमामहेश्वरसंवाद रामस्यायोध्याप्रवेशो नाम द्विचत्वारिंशदधिकद्विशततमोऽध्यायः॥२४२॥}

\sect{त्रिचत्वारिंशदधिक-द्विशततमोऽध्यायः --- विश्वदर्शनम्}

\uvacha{शङ्कर उवाच}

\twolineshloka
{अथ तस्मिन्दिने पुण्ये शुभलग्ने शुभान्विते}
{मङ्गलस्याभिषेकार्थं मङ्गलं चक्रिरे जनाः}% १

\twolineshloka
{वसिष्ठो वामदेवश्च जाबालिरथ कश्यपः}
{मार्कण्डेयश्च मौद्गल्यः पर्वतो नारदस्तथा}% २

\twolineshloka
{एते महर्षयस्तत्र जपहोमपुरस्सरम्}
{अभिषेकं शुभं चक्रुर्मुनयो राजसत्तमम्}% ३

\twolineshloka
{नानारत्नमये दिव्ये हेमपीठे शुभान्विते}
{निवेश्य सीतया सार्द्धं श्रिया इव जनार्दनम्}% ४

\twolineshloka
{सौवर्णकलशैर्दिव्यैर्नानारत्नमयैः शुभैः}
{सर्वतीर्थोदकैः पुण्यैर्माङ्गल्यद्रव्यसंयुतैः}% ५

\twolineshloka
{दूर्वाग्रतुलसीपत्रपुष्पगन्धसमन्वितैः}
{मन्त्रपूतजलैः शुद्धैर्मुनयः संशितव्रताः}% ६

\twolineshloka
{अजपन्वैष्णवान्सूक्तान्चतुर्वेदमयान्शुभान्}
{अभिषेकं शुभं चक्रुः काकुत्स्थं जगतः पतिम्}% ७

\twolineshloka
{तस्मिन्शुभतमे लग्ने देवदुन्दुभयो दिवि}
{विनेदुः पुष्पवर्षाणि ववृषुश्च समन्ततः}% ८

\twolineshloka
{दिव्याम्बरैर्भूषणैश्च दिव्यगन्धानुलेपनैः}
{पुष्पैर्नानाविधैर्दिव्यैर्देव्या सह रघूद्वहः}% ९

\twolineshloka
{अलङ्कृतश्च शुशुभे मुनिभिर्वेदपारगैः}
{छत्रं च चामरं दिव्यं धृतवान्लक्ष्मणस्तदा}% १०

\twolineshloka
{पार्श्वे भरतशत्रुघ्नौ तालवृन्तौ विरेजतुः}
{दर्पणं प्रददौ श्रीमान्राक्षसेन्द्रो विभीषणः}% ११

\twolineshloka
{दधार पूर्णकलशं सुग्रीवो वानरेश्वरः}
{जाम्बवांश्च महातेजाः पुष्पमालां मनोहराम्}% १२

\twolineshloka
{वालिपुत्रस्तु ताम्बूलं सकर्पूरं ददौ हरेः}
{हनुमान्दीपकां दिव्यां सुषेणश्च ध्वजं शुभम्}% १३

\twolineshloka
{परिवार्य महात्मानं मन्त्रिणः समुपासिरे}
{सृष्टिर्जयन्तो विजयः सौराष्ट्रो राष्ट्रवर्द्धनः}% १४

\twolineshloka
{अकोपो धर्मपालश्च सुमन्त्रो मन्त्रिणः स्मृताः}
{राजानश्च नरव्याघ्रा नानाजनपदेश्वराः}% १५

\twolineshloka
{पौराश्च नैगमा वृद्धा राजानं पर्युपासत}
{ऋक्षैश्च वानरेन्द्रैश्च मन्त्रिभिः पृथिवीश्वरैः}% १६

\twolineshloka
{राक्षसैर्द्विजमुख्यैश्च किङ्करैश्च समावृतः}
{परे व्योम्नि यथा लीनो दैवतैः कमलापतिः}% १७

\twolineshloka
{तथा नृपवरः श्रीमान्साकेते शुशुभे तदा}
{इन्दीवरदलश्यामं पद्मपत्रनिभेक्षणम्}% १८

\twolineshloka
{आजानुबाहुं काकुत्स्थं पीतवस्त्रधरं हरिम्}
{कम्बुग्रीवं महोरस्कं विचित्राभरणैर्युतम्}% १९

\twolineshloka
{देव्या सह समासीनमभिषिक्तं रघूत्तमम्}
{विमानस्थाः सुरगणा हर्षनिर्भरमानसाः}% २०

\twolineshloka
{तुष्टुवुर्जयशब्देन गन्धर्वाप्सरसां गणाः}
{अभिषिक्तस्ततो रामो वसिष्ठाद्यैर्महर्षिभिः}% २१

\twolineshloka
{शुशुभे सीतया देव्या नारायण इव श्रिया}
{अतिमर्त्यतयाभीत उपासितुं पदाम्बुजम्}% २२

\threelineshloka
{दृष्ट्वा तुष्टाव हृष्टात्मा शङ्करो हृष्टमागतः}
{कृताञ्जलिपुटो भूत्वा सानन्दो गद्गदाकुलः}
{हर्षयन्सकलान्देवान्मुनीनपि च वानरान्}% २३

\uvacha{महादेव उवाच}

\twolineshloka
{नमो मूलप्रकृतये नित्याय परमात्मने}
{सच्चिदानन्दरूपाय विश्वरूपाय वेधसे}% २४

\twolineshloka
{नमो निरन्तरानन्द कन्दमूलाय विष्णवे}
{जगत्त्रयकृतानन्द मूर्त्तये दिव्यमूर्त्तये}% २५

\twolineshloka
{नमो ब्रह्मेन्द्रपूज्याय शङ्कराभयदाय च}
{नमो विष्णुस्वरूपाय सर्वरूपनमोनमः}% २६

\twolineshloka
{उत्पत्तिस्थितिसंहारकारिणे त्रिगुणात्मने}
{नमोस्तु निर्गतोपाधिस्वरूपाय महात्मने}% २७

\twolineshloka
{अनया विद्यया देव्या सीतयोपाधिकारिणे}
{नमः पुम्प्रकृतिभ्यां च युवाभ्यां जगतां कृते}% २८

\twolineshloka
{जगन्मातापितृभ्यां च जनन्यै राघवाय च}
{नमः प्रपञ्चरूपिण्यै निष्प्रपञ्चस्वरूपिणे}% २९

\twolineshloka
{नमो ध्यानस्वरूपिण्यै योगिध्येयात्ममूर्त्तये}
{परिणामापरीणामरिक्ताभ्यां च नमोनमः}% ३०

\twolineshloka
{कूटस्थबीजरूपिण्यै सीतायै राघवाय च}
{सीता लक्ष्मीर्भवान्विष्णुः सीता गौरी भवान्शिवः}% ३१

\twolineshloka
{सीता स्वयं हि सावित्रि भवान्ब्रह्मा चतुर्मुखः}
{सीता शची भवान्शक्रः सीता स्वाहानलो भवान्}% ३२

\twolineshloka
{सीता संहारिणी देवी यमरूपधरो भवान्}
{सीता हि सर्वसम्पत्तिः कुबेरस्त्वं रघूत्तम}% ३३

\twolineshloka
{सीता देवी च रुद्राणी भवान्रुद्रो महाबलः}
{सीता तु रोहिणी देवी चन्द्रस्त्वं लोकसौख्यदः}% ३४

\twolineshloka
{सीता संज्ञा भवान्सूर्यः सीता रात्रिर्दिवा भवान्}
{सीतादेवी महाकाली महाकालो भवान्सदा}% ३५

\twolineshloka
{स्त्रीलिङ्गेषु त्रिलोकेषु यत्तत्सर्वं हि जानकी}
{पुन्नाम लाञ्छितं यत्तु तत्सर्वं हि भवान्प्रभो}% ३६

\twolineshloka
{सर्वत्र सर्वदेवेश सीता सर्वत्र धारिणी}
{तदात्वमपिचत्रातुन्तच्छक्तिर्विश्वधारिणी}% ३७

\twolineshloka
{तस्मात्कोटिगुणं पुण्यं युवाभ्यां परिचिह्नितम्}
{चिह्नितं शिवशक्तिभ्यां चरितं तव शान्तिदम्}% ३८

\twolineshloka
{आवां राम जगत्पूज्यौ मम पूज्यौ सदा युवाम्}
{त्वन्नामजापिनी गौरी त्वन्मन्त्रजपवानहम्}% ३९

\twolineshloka
{मुमूर्षोर्मणिकर्ण्यां तु अर्द्धोदकनिवासिनः}
{अहं दिशामि ते मन्त्रं तारकं ब्रह्मदायकम्}% ४०

\twolineshloka
{अतस्त्वं जानकीनाथ परब्रह्मासि निश्चितम्}
{त्वन्मायामोहितास्सर्वे न त्वां जानन्ति तत्वतः}% ४१

\uvacha{ईश्वर उवाच}

\twolineshloka
{इत्युक्तः शम्भुना रामः प्रसादप्रवणोऽभवत्}
{दिव्यरूपधरः श्रीमानद्भुताद्भुतदर्शनः}% ४२

\twolineshloka
{तथा तं रूपमालोक्य नरवानरदेवताः}
{न द्रष्टुमपिशक्तास्ते तेजसं महदद्भुतम्}% ४३


\threelineshloka
{भयाद्वै त्रिदशश्रेष्ठाः प्रणेमुश्चातिभक्तितः}
{भीता विज्ञाय रामोऽपि नरवानरदेवताः}
{मायामानुषतां प्राप्य स देवानब्रवीत्पुनः}% ४४

\uvacha{रामचन्द्र उवाच}

\twolineshloka
{शृणुध्वं देवता यो मां प्रत्यहं संस्तुविष्यति}
{स्तवेन शम्भुनोक्तेन देवतुल्यो भवेन्नरः}% ४५

\twolineshloka
{विमुक्तः सर्वपापेभ्यो मत्स्वरूपं समश्नुते}
{रणे जयमवाप्नोति न क्वचित्प्रतिहन्यते}% ४६

\twolineshloka
{भूतवेतालकृत्याभिर्ग्रहैश्चापि न बाध्यते}
{अपुत्रो लभते पुत्रं पतिं विन्दति कन्यका}% ४७

\twolineshloka
{दरिद्रः श्रियमाप्नोति सत्ववाञ्शीलवान्भवेत्}
{आत्मतुल्यबलः श्रीमाञ्जायते नात्र संशयः}% ४८

\twolineshloka
{निर्विघ्नं सर्वकार्येषु सर्वारम्भेषु वै नृणाम्}
{यंयं कामयते मर्त्यः सुदुर्लभमनोरथम्}% ४९


\threelineshloka
{षण्मासात्सिद्धिमाप्नोति स्तवस्यास्य प्रसादतः}
{यत्पुण्यं सर्वतीर्थेषु सर्वयज्ञेषु यत्फलम्}
{तत्फलं कोटिगुणितं स्तवेनानेन लभ्यते}% ५०

\uvacha{ईश्वर उवाच}

\twolineshloka
{इत्युक्त्वा रामचन्द्रोऽसौ विससर्ज महेश्वरम्}
{ब्रह्मादि त्रिदशान्सर्वान्विससर्ज समागतान्}% ५१

\twolineshloka
{अर्चिता मानवाः सर्वे नरवानरदेवताः}
{विसृष्टा रामचन्द्रेण प्रीत्या परमया युताः}% ५२

\twolineshloka
{इत्थं विसृष्टाः खलु ते च सर्वे सुखं तदा जग्मुरतीवहृष्टाः}
{परं पठन्तः स्तवमीश्वरोक्तं रामं स्मरन्तो वरविश्वरूपम्}% ५३

{॥इति श्रीपाद्मे महापुराणे पञ्चपञ्चाशत्साहस्र्यां संहितायामुत्तरखण्डे उमामहेश्वर संवादे विश्वदर्शनं नाम त्रिचत्वारिंशदधिकद्विशततमोऽध्यायः॥२४३॥}

\sect{चतुश्चत्वारिंशदधिक-द्विशततमोऽध्यायः --- श्रीरामचरितकथनम्}

\uvacha{शङ्कर उवाच}

\twolineshloka
{अथ रामस्तु वैदेह्या राज्यभोगान्मनोरमान्}
{बुभुजे वर्षसाहस्रं पालयन्सर्वतोदिशः}% १

\twolineshloka
{अन्तःपुरजनास्सर्वे राक्षसस्य गृहे स्थिताम्}
{गर्हयन्ति स्म वैदेहीं तथा जानपदा जनाः}% २

\twolineshloka
{लोकापवादभीत्या च रामः शत्रुनिवारकः}
{दर्शयन्मानुषं धर्ममन्तर्वत्नीं नृपात्मजाम्}% ३

\twolineshloka
{वाल्मीकेराश्रमे पुण्ये गङ्गातीरे महावने}
{विससर्ज महातेजा गर्भिणीं मुनिसंसदि}% ४

\twolineshloka
{सा भर्तुः परतन्त्रा हि उवास मुनिवेश्मनि}
{अर्चिता मुनिपत्नीभिर्वाल्मीकमुनि रक्षिता}% ५

\twolineshloka
{तत्रैवासूत यमलौ नाम्ना कुशलवौ सुतौ}
{तौ च तत्रैव मुनिना संस्कृतौ च ववर्धतुः}% ६

\twolineshloka
{रामोऽपि भ्रातृभिस्सार्द्धं पालयामास मेदिनीम्}
{यमादिगुणसम्पन्नस्सर्वभोगविवर्जितः}% ७

\twolineshloka
{अर्चयन्सततं विष्णुमनादिनिधनं हरिम्}
{ब्रह्मचर्यपरो नित्यं शशास पृथिवीं नृपः}% ८

\twolineshloka
{शत्रुघ्नो लवणं हत्वा मथुरां देवनिर्मिताम्}
{पालयामास धर्मात्मा पुत्राभ्यां सह राघवः}% ९

\twolineshloka
{गन्धर्वान्भरतो हत्वा सिन्धोरुभयपार्श्वतः}
{स्वात्मजौ स्थापयामास तस्मिन्देशे महाबलौ}% १०

\twolineshloka
{पश्चिमे मद्रदेशे तु मद्रान्हत्वा च लक्ष्मणः}
{स्वसुतौ च महावीर्यौ अभिषिच्य महाबलः}% ११

\twolineshloka
{गत्वा पुनरयोध्यां तु रामपादावुपस्पृशत्}
{ब्राह्मणस्य मृतं बालं कालधर्ममुपागतम्}% १२

\twolineshloka
{जीवयामास काकुत्स्थः शूद्रं हत्वा च तापसम्}
{ततस्तु गौतमीतीरे नैमिषे जनसंसदि}% १३

\twolineshloka
{इयाज वाजिमेधं च राघवः परवीरहा}
{काञ्चनीं जानकीं कृत्वा तया सार्द्धं महाबलः}% १४

\twolineshloka
{चकार यज्ञान्बहुशो राघवः परमार्थवित्}
{अयुतान्यश्वमेधानि वाजपेयानि च प्रभुः}% १५

\twolineshloka
{अग्निष्टोमं विश्वजितं गोमेधं च शतक्रतुम्}
{चकार विविधान्यज्ञान्परिपूर्णसदक्षिणान्}% १६

\twolineshloka
{एतस्मिन्नन्तरे तत्र वाल्मीकिः सुमहातपाः}
{सीतामानीय काकुत्स्थमिदं वचनमब्रवीत्}% १७

\uvacha{वाल्मीकिरुवाच}


\threelineshloka
{अपापां मैथिलीं राम त्यक्तुं नार्हसि सुव्रत}
{इयं तु विरजा साध्वी भास्करस्य प्रभा यथा}
{अनन्या तव काकुत्स्थ कस्मात्त्यक्ता त्वयानघ}% १८

\uvacha{राम उवाच}

\twolineshloka
{अपापां मैथिलीं ब्रह्मन्जानामि वचनात्तव}
{रावणेन हृता साध्वी दण्डके विजने पुरा}% १९

\twolineshloka
{तं हत्वा समरे सीतां शुद्धामग्निमुखागताम्}
{पुनर्यातोस्म्ययोध्यायां सीतामादाय धर्मतः}% २०

\twolineshloka
{लोकापवादः सुमहानभूत्पौरजनेषु च}
{त्यक्ता मया शुभाचारा तद्भयात्तव सन्निधौ}% २१

\twolineshloka
{तस्माल्लोकस्य सन्तुष्ट्यै सीता मम परायणा}
{पार्थिवानां महर्षीणां प्रत्ययं कर्तुमर्हति}% २२

\uvacha{महेश्वर उवाच}

\twolineshloka
{एवमुक्ता तदा सीता मुनिपार्थिवसंसदि}
{चकारप्रत्ययं देवी लोकाश्चर्यकरं सती}% २३

\twolineshloka
{दर्शयंस्तस्य लोकस्य रामस्यानन्यतां सती}
{अब्रवीत्प्राञ्जलिः सीता सर्वेषां जनसंसदि}% २४

\uvacha{सीतोवाच}

\twolineshloka
{यथाऽहं राघवादन्यं मनसापि न चिन्तये}
{तथा मे धरणी देवी विवरन्दातुमर्हति}% २५

\twolineshloka
{यथैव सत्यमुक्तं मे वेद्मि रामात्परं न च}
{तथा स्वपुत्र्यां वैदेह्यां धरणी सहसा इयात्}% २६

\uvacha{महेश्वर उवाच}

\twolineshloka
{ततो रत्नमयं पीठं पृष्ठे धृत्वा खगेश्वरः}
{रसातलात्तदा वीरो विज्ञाय जननीं तदा}% २७

\twolineshloka
{ततस्तु धरणीदेवी हस्ताभ्यां गृह्य मैथिलीम्}
{स्वागतेनाभिनन्द्यैनामासने सन्न्यवेशयत्}% २८

\twolineshloka
{सीतां समागतां दृष्ट्वा दिवि देवगणा भृशम्}
{पुष्पवृष्टिमविच्छिन्नां दिव्यां सीतामवाकिरन्}% २९

\twolineshloka
{सापि दिव्याप्सरोभिस्तु पूज्यमाना सनातनी}
{वैनतेयं समारुह्य तस्मान्मार्गाद्दिवं ययौ}% ३०

\twolineshloka
{दासीगणैः पूर्वभागे संवृता जगदीश्वरी}
{सम्प्राप्य परमं धाम योगिगम्यं सनातनम्}% ३१

\twolineshloka
{रसातलप्रविष्टां तु तां दृष्ट्वा सर्वमानुषाः}
{साधुसाध्विति सीतेयमुच्चैः सर्वे प्रचुक्रुशुः}% ३२

\twolineshloka
{रामः शोकसमाविष्टः सङ्गृह्य तनयावुभौ}
{मुनिभिः पार्थिवेन्द्रैश्च साकेतं प्रविवेश ह}% ३३

\twolineshloka
{अथ कालेन महता मातरः संशितव्रताः}
{कालधर्मं समापन्ना भर्तुः स्वर्गं प्रपेदिरे}% ३४

\twolineshloka
{दशवर्षसहस्राणि दशवर्षशतानि च}
{चकार राज्यं धर्मेण राघवः संशितव्रतः}% ३५

\twolineshloka
{कस्यचित्त्वथकालस्य राघवस्य निवेशनम्}
{कालस्तापसरूपेण सम्प्राप्तो वाक्यमब्रवीत्}% ३६

\uvacha{काल उवाच}

\twolineshloka
{राम राम महाबाहो धात्रा सम्प्रेषितोऽस्म्यहम्}
{यद्ब्रवीमि रघुश्रेष्ठ तच्छृणुष्व महामते}% ३७

\twolineshloka
{द्वन्द्वमेव हि कार्यं स्यादावयोः परिभाषितम्}
{तदन्तरे प्रविष्टोयस्स वद्ध्यो हि भविष्यति}% ३८

\uvacha{महेश्वर उवाच}


\threelineshloka
{तथेति च प्रतिश्रुत्य रामो राजीवलोचनः}
{द्वास्थं कृत्वा तु सौमित्रिं कालो वाक्यमभाषत}
{वैवस्वतोऽब्रवीद्वाक्यं रामं दशरथात्मजम्}% ३९

\uvacha{काल उवाच}

\twolineshloka
{शृणु राम यथावृत्तं समागमनकारणात्}
{दशवर्षसहस्राणि दशवर्षशतानि च}% ४०

\twolineshloka
{वसामि मानुषे लोके हत्वा राक्षसपुङ्गवौ}
{एवमुक्तः सुरगणैरवतीर्णोसि भूतले}% ४१

\twolineshloka
{तदयं समयः प्राप्तः स्वर्लोकं गमितुं त्वया}
{सनाथा हि सुरास्सर्वे भवन्त्वद्य त्वयानघ}% ४२

\uvacha{महेश्वर उवाच}

\twolineshloka
{एवमस्त्विति काकुत्स्थो रामः प्राह महामुनिम्}
{एतस्मिन्नन्तरे तत्र दुर्वासास्तु महातपाः}% ४३

\onelineshloka*
{राजद्वारमुपागम्य लक्ष्मणं वाक्यमब्रवीत्}

\uvacha{दुर्वासा उवाच}
\onelineshloka
{मां निवेदय काकुत्स्थं शीघ्रं गत्वा नृपात्मज}% ४४

\uvacha{महेश्वर उवाच}

\twolineshloka
{तमब्रवील्लक्ष्मणस्तु असान्निध्यमिति द्विज}
{ततः क्रोधसमाविष्टः प्राह तं मुनिसत्तमः}% ४५

\uvacha{दुर्वासा उवाच}

\onelineshloka*
{शापं दास्यामि काकुत्स्थं रामं न यदि दर्शये}

\uvacha{महेश्वर उवाच}

\twolineshloka
{तस्माच्छापभयाद्विप्रं राघवाय न्यवेदयत्}
{तत्रैवान्तर्दधे कालः सर्वभूतभयावहः}% ४६

\twolineshloka
{पूजयामास तं प्राप्तमृषिं दुर्वाससं नृपः}
{अग्रजस्य प्रतिज्ञा तं विज्ञाय रघुसत्तमः}% ४७

\twolineshloka
{तत्याज मानुषं रूपं लक्ष्मणः सरयूजले}
{विसृज्य मानुषं रूपं प्रविवेश स्वकां तनुम्}% ४८

\twolineshloka
{फणासहस्रसंयुक्तः कोटीन्दुसमवर्चसः}
{दिव्यमाल्याम्बरधरो दिव्यगन्धानुलेपनः}% ४९

\twolineshloka
{नागकन्यासहस्रैस्तु संवृतः समलङ्कृतः}
{विमानं दिव्यमारुह्य प्रययौ वैष्णवं पदम्}% ५०

\twolineshloka
{लक्ष्मणस्य गतिं सर्वां विदित्वा रघुसत्तमः}
{स्वयमप्यथ काकुत्स्थः स्वर्गं गन्तुमभीप्सितः}% ५१

\twolineshloka
{अभिषिच्याथ काकुत्स्थः स्वात्मजौ च कुशीलवौ}
{विभज्य रथनागाश्वं सधनं प्रददौ तयोः}% ५२

\twolineshloka
{कुशवत्यां कुशं तं च शरवत्यां लवं तथा}
{स्थापयामास धर्मेण राज्ये स्वे रघुसत्तमः}% ५३

\twolineshloka
{अभिप्रायं तु विज्ञाय रामस्य विदितात्मनः}
{आजग्मुर्वानराः सर्वे राक्षसाः सुमहाबलाः}% ५४

\twolineshloka
{विभीषणोऽथ सुग्रीवो जाम्बवान्मारुतात्मजः}
{नीलो नलः सुषेणश्च निषादाधिपतिर्गुहः}% ५५

\twolineshloka
{अभिषिच्य सुतौ वीरौ शत्रुघ्नश्च महामनाः}
{सर्व एते समाजग्मुरयोध्यां रामपालिताम्}% ५६

\onelineshloka*
{ते प्रणम्य महात्मानमूचुः प्राञ्जलयस्तथा}

\uvacha{वानरप्रभृतय ऊचुः}

\onelineshloka
{स्वर्लोकं गन्तुमुद्युक्तं ज्ञात्वा त्वां रघुसत्तम}% ५७


\threelineshloka
{आगताः स्म वयं सर्वे तवानुगमनं प्रति}
{न शक्ताः स्म क्षणं राम जीवितुं त्वां विना प्रभो}
{तस्मात्त्वया विशालाक्ष गच्छामस्त्रिदशालयम्}% ५८

\uvacha{महेश्वर उवाच}

\twolineshloka
{तैरेवमुक्तः काकुत्स्थो बाढमित्यब्रवीत्ततः}
{अथोवाच महातेजा राक्षसेन्द्रं विभीषणम्}% ५९

\uvacha{राम उवाच}

\onelineshloka*
{राज्यं प्रशास धर्मेण मा प्रतिज्ञां वृथा कृथाः}

\twolineshloka
{यावच्चन्द्रश्च सूर्यश्च यावत्तिष्ठति मेदिनी}
{तावद्रमस्व सुप्रीतो काले मम पदं व्रज}% ६०

\uvacha{महेश्वर उवाच}

\twolineshloka
{इत्युक्त्वाथ स काकुत्स्थः स्वाड्गं विष्णुं सनातनम्}
{श्रीरङ्गशायिनं सौम्यमिक्ष्वाकुकुलदैवतम्}% ६१

\twolineshloka
{सम्प्रीत्या प्रददौ तस्मै रामो राजीवलोचनः}
{हनुमन्तमथोवाच राघवः शत्रुसूदनः}% ६२

\uvacha{राम उवाच}

\twolineshloka
{मत्कथाः प्रचरिष्यन्ति यावल्लोके हरीश्वर}
{तावत्त्वमास मेदिन्यां काले मां व्रज सुव्रत}% ६३

\uvacha{महेश्वर उवाच}

\onelineshloka*
{तमेवमुक्त्वा काकुत्स्थो जाम्बवन्तमथाब्रवीत्}

\uvacha{राम उवाच}
\onelineshloka
{द्वापरे समनुप्राप्ते यदूनामन्वये पुनः}% ६४

\twolineshloka
{भूभारस्य विनाशाय समुत्पत्स्याम्यहं भुवि}
{करिष्ये तत्र सङ्ग्रामं स्वयं भल्लूकसत्तम}% ६५

\uvacha{महेश्वर उवाच}

\twolineshloka
{तमेवमुक्त्वा काकुत्स्थः सर्वांस्तानृक्षवानरान्}
{उवाच वाचा गच्छध्वमिति रामो महाबलः}% ६६

\twolineshloka
{मन्त्रिणो नैगमाश्चैव भरतः कैकयीसुतः}
{राघवस्यानुगमने निश्चितास्ते समाययुः}% ६७

\twolineshloka
{ततः शुक्लाम्बरधरो ब्रह्मचारी ययौ परम्}
{कुशान्गृहीत्वा पाणिभ्यां संसक्तः प्रययौ परम्}% ६८

\twolineshloka
{रामस्य दक्षिणे पार्श्वे पद्महस्ता रमा गता}
{तथैव धरणीदेवी दक्षिणेतरगा तथा}% ६९

\twolineshloka
{वेदाः साङ्गाः पुराणानि सेतिहासानि सर्वतः}
{ॐकारोऽथ वषट्कारः सावित्री लोकपावनी}% ७०

\twolineshloka
{अस्त्रशस्त्राणि च तदा धनुराद्यानि पार्वति}
{अनुजग्मुस्तथा रामं सर्वे पुरुषविग्रहाः}% ७१

\twolineshloka
{भरतश्चैव शत्रुघ्नः सर्वे पुरनिवासिनः}
{सपुत्रदाराः काकुत्स्थमनुजग्मुः सहानुगाः}% ७२

\twolineshloka
{मन्त्रिणो भृत्यवर्गाश्च किङ्करा नैगमास्तथा}
{वानराश्चैव ऋक्षाश्च सुग्रीवसहितास्तदा}% ७३

\twolineshloka
{सपुत्रदाराः काकुत्स्थमन्वगच्छन्महामतिम्}
{पशवः पक्षिणश्चैव सर्वे स्थावरजङ्गमाः}% ७४

\twolineshloka
{अनुजग्मुर्महात्मानं समीपस्था नरोत्तमाः}
{ये च पश्यन्ति काकुत्स्थं स्वपथान्तर्गतं प्रभुम्}% ७५

\twolineshloka
{ते तथानुगता रामं निवर्त्तन्ते न केचन}
{अथ त्रियोजनं गत्वा नदीं पश्चान्मुखीं स्थिताम्}% ७६

\twolineshloka
{सरयूं पुण्यसलिलां प्रविवेश सहानुगः}
{ततः पितामहो ब्रह्मा सर्वदेवगणावृतः}% ७७

\twolineshloka
{तुष्टाव रघुशार्दूलमृषिभिः सार्द्धमक्षरैः}
{अब्रवीत्तत्र काकुत्स्थं प्रविष्टं सरयूजले}% ७८

\uvacha{ब्रह्मोवाच}

\twolineshloka
{आगच्छ विष्णो भद्रं ते दिष्ट्या प्राप्तोऽसि मानद}
{भ्रातृभिस्सहदेवाभैः प्रविशस्व निजां तनुम्}% ७९

\twolineshloka
{वैष्णवीं तां महातेजां देवाकारां सनातनीम्}
{त्वं हि लोकगतिर्देव न त्वां केचित्तु जानते}% ८०

\twolineshloka
{त्वामचिन्त्यं महात्मानमक्षरं सर्वसङ्ग्रहम्}
{यमिच्छसि महातेजस्तां तनुं प्रविशस्व भोः}% ८१

\uvacha{महेश्वर उवाच}

\twolineshloka
{तस्मिन्सूर्यकराकीर्णे पुष्पवृष्टिनिपातिते}
{उत्सृज्य मानुषं रूपं स्वां तनुं प्रविवेश ह}% ८२

\twolineshloka
{अंशाभ्यां शङ्खचक्राभ्यां शत्रुघ्नभरतावुभौ}
{तदा तेन महात्मानौ दिव्यतेजस्समन्वितौ}% ८३

\twolineshloka
{शङ्खचक्रगदाशार्ङ्गपद्महस्तश्चतुर्भुजः}
{दिव्याभरणसम्पन्नो दिव्यगन्धानुलेपनः}% ८४

\twolineshloka
{दिव्यपीताम्बरधरः पद्मपत्रनिभेक्षणः}
{युवा कुमारः सौम्याङ्गः कोमलावयवोज्ज्वलः}% ८५

\twolineshloka
{सुस्निग्धनीलकुटिलकुन्तलः शुभलक्षणः}
{नवदूर्वाङ्कुरः श्यामः पूर्णचन्द्र निभाननः}% ८६

\twolineshloka
{देवीभ्यां सहितः श्रीमान्विमानमधिरुह्य च}
{तस्मिन्सिंहासने दिव्ये मूले कल्पतरोः प्रभुः}% ८७

\twolineshloka
{निषसाद महातेजाः सर्वदेवैरभिष्टुतः}
{राघवानुगता ये च ऋक्षवानरमानुषाः}% ८८

\twolineshloka
{स्पृष्ट्वैव सरयूतोयं सुखेन त्यक्तजीविताः}
{रामप्रसादात्ते सर्वे दिव्यरूपधराः शुभाः}% ८९

\twolineshloka
{दिव्यमाल्याम्बरधरा दिव्यमङ्गलवर्चसः}
{आरुरोह विमानं तदसङ्ख्यास्तत्र देहिनः}% ९०

\twolineshloka
{सर्वैः परिवृतः श्रीमान्रामो राजीवलोचनः}
{पूजितः सुरसिद्धौघैर्मुनिभिस्तु महात्मभिः}% ९१

\twolineshloka
{आययौ शाश्वतं दिव्यमक्षरं स्वपदं विभुः}
{यः पठेद्रामचरितं श्लोकं श्लोकार्धमेव वा}% ९२

\twolineshloka
{शृणुयाद्वा तथा भक्त्या स्मरेद्वा शुभदर्शने}
{कोटिजन्मार्जितात्पापाज्ज्ञानतोऽज्ञानतः कृतात्}% ९३

\twolineshloka
{विमुक्तो वैष्णवं लोकं पुत्रदारसबान्धवैः}
{समाप्नुयाद्योगगम्यमनायासेन वै नरः}% ९४


\onelineshloka
{एतत्ते कथितं देवि रामस्य चरितं महत्}
{धन्योऽस्म्यहं त्वया देवि रामचन्द्रस्य कीर्त्तनात्}
{किमन्यच्छ्रोतुकामासि तद्ब्रवीमि वरानने}% ९५

{॥इति श्रीपाद्मे महापुराणे पञ्चपञ्चाशत्साहस्र्यां संहितायामुत्तरखण्डे उमामहेश्वर संवादे श्रीरामचरितकथनं नाम चतुश्चत्वारिंशदधिकद्विशततमोऽध्यायः॥२४४॥}



    \input{rama-charitam/brahma-puranam/ananta-vasudeva-mahatmyam}
    \sect{विष्णोर्प्रादुर्भावः --- रामावतारः}

\src{ब्रह्म-पुराणम्}{पूर्वखण्डः}{अध्यायः २१३}{श्लोकाः १२४---१५८}
\tags{concise, complete}
\notes{Summary of Ramayana, during the narration of various Vishnu avataras.}
\textlink{https://sa.wikisource.org/wiki/ब्रह्मपुराणम्/अध्यायः_२१३}
\translink{}

\storymeta

\uvacha{व्यास उवाच}

\addtocounter{shlokacount}{123}

\twolineshloka
{चतुर्विंशे युगे वाऽपि विश्वामित्रपुरःसरः}
{जज्ञे दशरथस्याथ पुत्रः पद्मयतेक्षणः} %॥१२४॥

\twolineshloka
{कृत्वाऽत्मानं महाबाहुश्चतुर्धा प्रभुरीश्वरः}
{लोके राम इति ख्यातस्तेजसा भास्करोपमः} %॥१२५॥

\twolineshloka
{प्रसादनार्थं लोकस्य रक्षसां निग्रहाय च}
{धर्मस्य च विवृद्ध्यर्थं जज्ञे तत्र महयशाः} %॥१२६॥

\twolineshloka
{तमप्याहुर्मनुष्येन्द्रं सर्वभूतहिते रतम्}
{यः समाः सर्वधर्मज्ञश्चतुर्दश वनेऽवसत्} %॥१२७॥

\twolineshloka
{लक्ष्मणानुचरो रामः सर्वभूतहिते रतः}
{चतुर्दश वने तप्त्वा तपो वर्षणि राघवः} %॥१२८॥

\twolineshloka
{रूपिणी तस्य पार्श्वस्था सीतेति प्रथिता जने}
{पूर्वोदिता तु या लक्ष्मीर्भर्तारमनुगच्छति} %॥१२९॥

\twolineshloka
{जनस्थाने वसन्कार्यं त्रिदशानां चकार सः}
{तस्यापकारिणं क्रूरं पौलस्त्यं मनुजर्षभः} %॥१३०॥

\twolineshloka
{सीतायाः पदमन्विच्छन्निजघान महायशाः}
{देवासुरगणानां च यक्षराक्षसभोगिनाम्} %॥१३१॥

\twolineshloka
{यत्रावध्यं राक्षसेन्द्रं रावणं युधि दुर्जयम्}
{युक्तं राक्षसकोटीभिर्नीलाञ्जनचयोपमम्} %॥१३२॥

\twolineshloka
{त्रैलाक्यद्रावणं क्रूरं रावणं राक्षसेश्वरम्}
{दुर्जयं दुर्धरं दृप्तं शार्दूलसमविक्रमम्} %॥१३३॥

\twolineshloka
{दुर्निरीक्ष्यं सुरगणैर्वरदानेन दर्पितम्}
{जघान सचिवैः सार्धं ससैन्यं रावणं युधि} %॥१३४॥

\twolineshloka
{महाभ्रगणसङ्काशं महाकायं महाबलम्}
{रावणं निजघानाऽऽशु रामो भूतपतिः पुरा} %॥१३५॥

\twolineshloka
{सुग्रीवस्य कृते येन वानरेन्द्रो महाबलः}
{वाली विनिहतः सङ्ख्ये सुग्रीवश्चाभिषेचितः} %॥१३६॥

\twolineshloka
{मधोश्च तनयो दृप्तो लवणो नाम दानवः}
{हतो मधुवने वीरो वरमत्तो महासुरः} %॥१३७॥

\twolineshloka
{यज्ञविघ्नकरौ येन मुनीनां भावितात्मनाम्}
{मारीचश्च सुबाहुश्च बलेन बलिनां वरौ} %॥१३८॥

\twolineshloka
{निहतौ च निराशौ च कृतौ तेन महात्मना}
{समरे युद्धशौण्डेन तथाऽन्ये चापि राक्षसाः} %॥१३९॥

\twolineshloka
{विराधश्च कबन्धश्च राक्षसौ भीमविक्रमौ}
{जघान पुरुषव्याघ्रो गन्धवौ शापमोहितौ} %॥१४०॥

\twolineshloka
{हुताशनार्कांशुतडिद्‌गुणाभैः प्रतप्तजाम्बूनदचित्रपुङ्खैः}
{महेन्द्रवज्राशनितुल्यसारै रिपून्स रामः समरे निजघ्ने} %॥१४१॥

\twolineshloka
{तस्मै दत्तानि शस्त्राणि विश्वामित्रेण धीमता}
{वधार्थं देवशत्रूणां दुर्धर्षाणां सुरैरपि} %॥१४२॥

\twolineshloka
{वर्तमाने मखे येन जनकस्य महात्मनः}
{भग्नं माहेश्वरं चापं क्रीडता लीलया पुरा} %॥१४३॥

\twolineshloka
{एतानि कृत्वा कर्माणि रामो धर्मभृतां वरः}
{दशाश्वमेधाञ्जारूथ्यानाजहार निरर्गलान्} %॥१४४॥

\twolineshloka
{नाश्रूयन्ताशुभा वाचो नाऽऽकुलं मारुतो ववौ}
{न वित्तहरणं चाऽऽसीद्रामे राज्यं प्रशासति} %॥१४५॥

\twolineshloka
{परिदेवन्ति विधवा नानर्थाश्च कदाचन}
{सर्वमासीच्छुभं तत्र रामे राज्यं प्रशासति} %॥१४६॥

\twolineshloka
{न प्राणिनां भयं चाऽऽसीज्जलाग्न्यनिलघातजम्}
{न चापि वृद्धा बालानां प्रेतकार्याणि चक्रिरे} %॥१४७॥

\twolineshloka
{ब्रह्मचर्यपरं क्षत्रं विशस्तु क्षत्रिये रताः}
{शूद्राश्चैव हि वर्णास्त्रीञ्शुश्रूषन्त्यनहङ्कृताः} %॥१४८॥

\twolineshloka
{नार्यो नात्यचरन्भर्तॄन्भार्यां नात्यचरत्पतिः}
{सर्वमासीज्जगद्दान्तं निर्दस्युरभवन्मही} %॥१४९॥

\twolineshloka
{राम एकोऽभवद्भर्ता रामः पालयिताऽभवत्}
{आसन्वर्षसहस्राणि तथा पुत्रसहस्रिणः} %॥१५०॥

\twolineshloka
{अरोगाः प्राणिनश्चाऽऽसन्रामे राज्यं प्रशासति}
{देवतानामृषीणां च मनुष्याणां च सर्वशः} %॥१५१॥

\twolineshloka
{पृथिव्यां समवायोऽभूद्रामे राज्यं प्रशासति}
{गाथामप्यत्र गायन्ति ये पुराणविदो जनाः} %॥१५२॥

\twolineshloka
{रामे निबद्धतत्त्वार्था माहात्म्यं तस्य धीमतः}
{श्यामो युवा लोहिताक्षो दीप्तास्यो मितभाषितः} %॥१५३॥

\twolineshloka
{आजानुबाहुः सुमुखः सिंहस्कन्धो महाभुजः}
{दश वर्षसहस्राणि रामो राज्यमकारयत्} %॥१५४॥

\twolineshloka
{ऋक्सामयजुषां घोषो ज्याघोषश्च महात्मनः}
{अव्युच्छिन्नोऽभवद्राष्ट्रे दीयतां भुज्यतामिति} %॥१५५॥

\twolineshloka
{सत्त्ववान्गुणसम्पन्नो दीप्यमानः स्वतेजसा}
{अतिचन्द्रं च सूर्यं च रामो दाशरथिर्बभौ} %॥१५६॥

\twolineshloka
{ईजे क्रतुशतैः पुण्यैः समाप्तवरदक्षिणैः}
{हित्वाऽयोध्यां दिवं यातो राघवो हि महाबलः} %॥१५७॥

\twolineshloka
{एवमेव महाबाहुरिक्ष्वाकुकुलनन्दनः}
{रावणं सगणं हत्वा दिवमाचक्रमे विभुः} %॥१५८॥

॥इति श्रीमहापुराणे आदिब्राह्मे विष्णोः प्रादुर्भावानुकीर्तनं नाम त्रयोदशाधिकद्विशततमोऽध्यायः॥२१३॥

    \input{rama-charitam/brahmavaivarta-puranam/ramacharitam}
    \chapt{शिव-पुराणम्}

\sect{रामेश्वरमाहात्म्यम्}

\src{शिव-पुराणम्}{पूर्वखण्डः}{अध्यायः ३१}{श्लोकाः १---४५}
\tags{concise, complete}
\notes{Summary of Ramayana.}
\textlink{https://sa.wikisource.org/wiki/शिवपुराणम्/संहिता_४_(कोटिरुद्रसंहिता)/अध्यायः_३१}
\translink{https://www.wisdomlib.org/hinduism/book/shiva-purana-english/d/doc226532.html}

\storymeta

\uvacha{सूत उवाच}

\twolineshloka
{अतः परं प्रवक्ष्यामि लिङ्गं रामेश्वराभिधम्} 
{उत्पन्नं च यथा पूर्वमृषयश्शृणुतादरात्}

\onelineshloka
{पुरा विष्णुः पृथिव्यां चावततार सतां प्रियः} %॥२॥

\twolineshloka
{तत्र सीता हृता विप्रा रावणेनोरुमायिना} 
{प्रापिता स्वगृहं सा हि लङ्कायां जनकात्मजा} %॥३॥

\twolineshloka
{अन्वेषणपरस्तस्याः किष्किन्धाख्यां पुरीमगात्} 
{सुग्रीवहितकृद्भूत्वा वालिनं सञ्जघान ह} %॥४॥

\twolineshloka
{तत्र स्थित्वा कियत्कालं तदन्वेषणतत्परः} 
{सुग्रीवाद्यैर्लक्ष्मणेन विचारं कृतवान्स वै} %॥५॥

\twolineshloka
{कपीन्सम्प्रेषयामास चतुर्दिक्षु नृपात्मजः} 
{हनुमत्प्रमुखान्रामस्तदन्वेषणहेतवे} %॥६॥

\twolineshloka
{अथ ज्ञात्वा गतां लङ्कां सीतां कपिवराननात्} 
{सीताचूडामणिं प्राप्य मुमुदे सोऽति राघवः} %॥७॥

\twolineshloka
{सकपीशस्तदा रामो लक्ष्मणेन युतो द्विजाः} 
{सुग्रीवप्रमुखैः पुण्यैर्वानरैर्बलवत्तरैः} %॥८॥

\twolineshloka
{पद्मैरष्टादशाख्यैश्च ययौ तीरं पयोनिधेः}
{दक्षिणे सागरे यो वै दृश्यते लवणाकरः}

\twolineshloka
{तत्रागत्य स्वयं रामो वेलायां संस्थितो हि सः} 
{वानरैस्सेव्यमानस्तु लक्ष्मणेन शिवप्रियः} %॥4॥

\twolineshloka
{हा जानकि कुतो याता कदा चेयं मिलिष्यति} 
{अगाधस्सागरश्चैवातार्या सेना च वानरी} %॥११॥

\twolineshloka
{राक्षसो गिरिधर्त्ता च महाबलपराक्रमः}
{लङ्काख्यो दुर्गमो दुर्ग इन्द्रजित्तनयोस्य वै} %॥१२॥

\twolineshloka
{इत्येवं स विचार्यैव तटे स्थित्वा सलक्ष्मणः}
{आश्वासितो वनौकोभिरङ्गदादिपुरस्सरैः} %॥१३॥

\twolineshloka
{एतस्मिन्नन्तरे तत्र राघवश्शैवसत्तमः}
{उवाच भ्रातरं प्रीत्या जलार्थी लक्ष्मणाभिधम्} %॥१४॥

\uvacha{राम उवाच}

\twolineshloka
{भ्रातर्लक्ष्मण वीरेशाहं जलार्थी पिपासितः}
{तदानय द्रुतं पाथो वानरैः कैश्चिदेव हि} %॥१५॥

\uvacha{सूत उवाच}

\twolineshloka
{तच्छ्रुत्वा वानरास्तत्र ह्यधावन्त दिशो दश} 
{नीत्वा जलं च ते प्रोचुः प्रणिपत्य पुरः स्थिताः} %॥१६॥

\uvacha{वानरा ऊचुः}

\twolineshloka
{जलं च गृह्यतां स्वामिन्नानीतं तत्त्वदाज्ञया}
{महोत्तमं च सुस्वादु शीतलं प्राणतर्पणम्} %॥१७॥

\uvacha{सूत उवाच}

\twolineshloka
{सुप्रसन्नतरो भूत्वा कृपादृष्ट्या विलोक्य तान् }
{तच्छ्रुत्वा रामचन्द्रोऽसौ स्वयं जग्राह तज्जलम्} %॥१८॥

\twolineshloka
{स शैवस्तज्जलं नीत्वा पातुमारब्धवान्यदा} 
{तदा च स्मरणं जातमित्थमस्य शिवेच्छया} %॥१९॥

\twolineshloka
{न कृतं दर्शनं शम्भोर्गृह्यते च जलं कथम्}
{स्वस्वामिनः परेशस्य सर्वानन्दप्रदस्य वै} %॥२०॥

\twolineshloka
{इत्युक्त्वा च जलं पीतं तदा रघुवरेण च}
{पश्चाच्च पार्थिवीं पूजां चकार रघुनन्दनः} %॥२१॥

\twolineshloka
{आवाहनादिकांश्चैव ह्युपचारान्प्रकल्प्य वै}
{विधिवत्षोडश प्रीत्या देवमानर्च शङ्करम्} %॥२२॥

\twolineshloka
{प्रणिपातैस्स्तवैर्दिव्यैश्शिवं सन्तोष्य यत्नतः}
{प्रार्थयामास सद्भक्त्या स रामश्शङ्करं मुदा} %॥२३॥

\uvacha{राम उवाच}

\twolineshloka
{स्वामिञ्छम्भो महादेव सर्वदा भक्तवत्सल} 
{पाहि मां शरणापन्नं त्वद्भक्तं दीनमानसम्} %॥२४॥

\twolineshloka
{एतज्जलमगाधं च वारिधेर्भवतारण}
{रावणाख्यो महावीरो राक्षसो बलवत्तरः} %॥२५॥

\twolineshloka
{वानराणां बलं ह्येतच्चञ्चलं युद्धसाधनम्}
{ममकार्यं कथं सिद्धं भविष्यति प्रियाप्तये} %॥२६॥

\twolineshloka
{तस्मिन्देव त्वया कार्यं साहाय्यं मम सुव्रत} 
{साहाय्यं ते विना नाथ मम कार्य्यं हि दुर्लभम्} %॥२७॥

\twolineshloka
{त्वदीयो रावणोऽपीह दुर्ज्जयस्सर्वथाखिलैः} 
{त्वद्दत्तवरदृप्तश्च महावीरस्त्रिलोकजित्} %॥२८॥

\twolineshloka
{अप्यहं तव दासोऽस्मि त्वदधीनश्च सर्वथा} 
{विचार्येति त्वया कार्यः पक्षपातस्सदाशिव} %॥२९॥

\uvacha{सूत उवाच}

\twolineshloka
{इत्येवं स च सम्प्रार्थ्य नमस्कृत्य पुनःपुनः} 
{तदा जयजयेत्युच्चैरुद्धोषैश्शङ्करेति च} %॥4॥

\twolineshloka
{इति स्तुत्वा शिवं तत्र मन्त्रध्यानपरायणः}
{पुनः पूजां ततः कृत्वा स्वाम्यग्रे स ननर्त ह} %॥३१॥

\twolineshloka
{प्रेमी विक्लिन्नहृदयो गल्लनादं यदाकरोत्} 
{तदा च शङ्करो देवस्सुप्रसन्नो बभूव ह} %॥३२॥

\twolineshloka
{साङ्गस्सपरिवारश्च ज्योतीरूपो महेश्वरः}
{यथोक्तरूपममलं कृत्वाविरभवद्द्रुतम्} %॥३३॥

\twolineshloka
{ततस्सन्तुष्टहृदयो रामभक्त्या महेश्वरः} 
{शिवमस्तु वरं ब्रूहि रामेति स तदाब्रवीत्} %॥३४॥

\twolineshloka
{तद्रूपं च तदा दृष्ट्वा सर्वे पूतास्ततस्स्वयम्} 
{कृतवान्राघवः पूजां शिवधर्मपरायणः} %॥३५॥

\twolineshloka
{स्तुतिं च विविधां कृत्वा प्रणिपत्य शिवं मुदा} 
{जयं च प्रार्थयामास रावणाजौ तदात्मनः} %॥३६॥

\twolineshloka
{ततः प्रसन्नहृदयो रामभक्त्या महेश्वरः} 
{जयोस्तु ते महाराज प्रीत्या स पुनरब्रवीत्} %॥३७॥

\twolineshloka
{शिवदत्तं जयं प्राप्य ह्यनुज्ञां समवाप्य च} 
{पुनश्च प्रार्थयामास साञ्जलिर्नतमस्तकः} %॥३८॥

\uvacha{राम उवाच}

\twolineshloka
{त्वया स्थेयमिह स्वामिंल्लोकानां पावनाय च} 
{परेषामुपकारार्थं यदि तुष्टोऽसि शङ्कर} %॥३९॥

\uvacha{सूत उवाच}

\twolineshloka
{इत्युक्तस्तु शिवस्तत्र लिङ्गरूपोऽभवत्तदा} 
{रामेश्वरश्च नाम्ना वै प्रसिद्धो जगतीतले} %॥४०॥

\twolineshloka
{रामस्तु तत्प्रभावाद्वै सिन्धुमुत्तीर्य चाञ्जसा}
{रावणादीन्निहत्याशु राक्षसान्प्राप तां प्रियाम्} %॥४१॥

\twolineshloka
{रामेश्वरस्य महिमाद्भुतोऽभूद्भुवि चातुलः} 
{भुक्तिमुक्तिप्रदश्चैव सर्वदा भक्तकामदः} %॥४२॥

\twolineshloka
{दिव्यगङ्गाजलेनैव स्नापयिष्यति यश्शिवम्} 
{रामेश्वरं च सद्भक्त्या स जीवन्मुक्त एव हि} %॥४३॥

\twolineshloka
{इह भुक्त्वाखिलान्भोगान्देवानां दुर्लभानपि} 
{अन्ते प्राप्य परं ज्ञानं कैवल्यं प्राप्नुयाद्ध्रुवम्} %॥४४॥

\twolineshloka
{इति वश्च समाख्यातं ज्योतिर्लिगं शिवस्य तु}
{रामेश्वराभिधं दिव्यं शृण्वतां पापहारकम्} %॥४५।

॥इति श्रीशिवमहापुराणे चतुर्थ्यां कोटिरुद्रसन्तायां रामेश्वरमाहात्म्यवर्णनं नामैकत्रिंशोऽध्यायः॥

\closesection
    \chapt{स्कन्द-पुराणम्}
    \input{rama-charitam/skanda-puranam/rama-katha-varnanam}
    \sect{रामचरित्रवर्णनम्}

\src{स्कन्दपुराणम्}{खण्डः ३ (ब्रह्मखण्डः)}{धर्मारण्य खण्डः}{अध्यायः ३०}
\vakta{}
\shrota{}
\tags{}
\notes{}
\textlink{https://sa.wikisource.org/wiki/स्कन्दपुराणम्/खण्डः_३_(ब्रह्मखण्डः)/धर्मारण्य_खण्डः/अध्यायः_३०}
\translink{https://www.wisdomlib.org/hinduism/book/the-skanda-purana/d/doc423651.html}

\storymeta




\uvacha{व्यास उवाच}

\twolineshloka
{पुरा त्रेतायुगे प्राप्ते वैष्णवांशो रघूद्वहः}
{सूर्यवंशे समुत्पन्नो रामो राजीवलोचनः}%॥ १ ॥

\twolineshloka
{स रामो लक्ष्मणश्चैव काकपक्षधरावुभौ}
{तातस्य वचनात्तौ तु विश्वामित्रमनुव्रतौ}%॥ २ ॥

\twolineshloka
{यज्ञसंरक्षणार्थाय राज्ञा दत्तौ कुमारकौ}
{धनुःशरधरौ वीरौ पितुर्वचनपालकौ}%॥ ३ ॥

\twolineshloka
{पथि प्रव्रजतो यावत्ताडकानाम राक्षसी}
{तावदागम्य पुरतस्तस्थौ वै विघ्नकारणात्}%॥ ४ ॥


\twolineshloka
{ऋषेरनुज्ञया रामस्ताडकां समघातयत्॥}
{प्रादिशच्च धनुर्वेदविद्यां रामाय गाधिजः}%॥५॥

\twolineshloka
{तस्य पादतलस्पर्शाच्छिला वासवयोगतः}
{अहल्या गौतमवधूः पुनर्जाता स्वरूपिणी}%॥ ६ ॥


\twolineshloka
{विश्वामित्रस्य यज्ञे तु सम्प्रवृत्ते रघूत्तमः}
{मारीचं च सुबाहुं च जघान परमेषुभिः}%॥७॥

\twolineshloka
{ईश्वरस्य धनुर्भग्नं जनकस्य गृहे स्थितम्}
{रामः पञ्चदशे वर्षे षड्वर्षां चैव मैथिलीम्}%॥ ८ ॥

\twolineshloka
{उपयेमे तदा राजन्रम्यां सीतामयोनिजाम्}
{कृतकृत्यस्तदा जातः सीतां सम्प्राप्य राघवः}%॥ ९ ॥

\twolineshloka
{अयोध्यामगमन्मार्गे जामदग्न्यमवेक्ष्य च}
{सङ्ग्रामोऽभूत्तदा राजन्देवानामपि दुःसहः}%॥ १० ॥

\twolineshloka
{ततो रामं पराजित्य सीतया गृहमागतः}
{ततो द्वादशवर्षाणि रेमे रामस्तया सह}%॥ ११ ॥

\twolineshloka
{एकविंशतिमे वर्षे यौवराज्यप्रदायकम्}
{राजानमथ कैकेयी वरद्वयमयाच त}%॥ १२ ॥

\twolineshloka
{तयोरेकेन रामस्तु ससीतः सहलक्ष्मणः}
{जटाधरः प्रव्रजतां वर्षाणीह चतुर्दश}%॥ १३ ॥

\twolineshloka
{भरतस्तु द्वितीयेन यौवराज्याधिपोस्तु मे}
{मन्थरावचनान्मूढा वरमेतमयाचत}%॥ १४ ॥

\twolineshloka
{जानकीलक्ष्मणसखं रामं प्राव्राजयन्नृपः}
{त्रिरात्रमुदकाहारश्चतुर्थेह्नि फलाशनः}%॥ १५ ॥

\twolineshloka
{पञ्चमे चित्रकूटे तु रामो वासमकल्पयत्}
{तदा दशरथः स्वर्गं गतो राम इति ब्रुवन्}%॥ १६ ॥

\twolineshloka
{ब्रह्मशापं तु सफलं कृत्वा स्वर्गं जगाम किम्}
{ततो भरत शत्रुघ्नौ चित्रकूटे समागतौ}%॥ १७ ॥

\twolineshloka
{स्वर्गतं पितरं राजन्रामाय विनिवेद्य च}
{सान्त्वनं भरतस्यास्य कृत्वा निवर्तनं प्रति}%॥ १८ ॥

\twolineshloka
{ततो भरत शत्रुघ्नौ नन्दिग्रामं समागतौ}
{पादुकापूजनरतौ तत्र राज्यधरावुभौ}%॥ १९ ॥

\twolineshloka
{अत्रिं दृष्ट्वा महात्मानं दण्डकारण्यमागमत}
{रक्षोगणवधारम्भे विराधे विनिपातिते}%॥ २० ॥

\threelineshloka
{अर्द्धत्रयोदशे वर्षे पञ्चवट्यामुवास ह}
{ततो विरूपयामास शूर्पणखां निशाचरीम्}
{वने विचरतरतस्य जानकीसहितस्य च}%॥ २१ ॥

\twolineshloka
{आगतो राक्षसो घोरः सीतापहरणाय सः}
{ततो माघासिताष्टम्यां मुहूर्ते वृन्दसंज्ञके}%॥ २२ ॥

\twolineshloka
{राघवाभ्यां विना सीतां जहार दश कन्धरः}
{मारीचस्याश्रमं गत्वा मृगरूपेण तेन च}%॥ २३ ॥

\twolineshloka
{नीत्वा दूरं राघवं च लक्ष्मणेन समन्वितम्}
{ततो रामो जघानाशु मारीचं मृगरू पिणम्}%॥ २४ ॥

\twolineshloka
{पुनः प्राप्याश्रमं रामो विना सीतां ददर्श ह}
{तत्रैव ह्रियमाणा सा चक्रन्द कुररी यथा}%॥ २५ ॥

\twolineshloka
{रामरामेति मां रक्ष रक्ष मां रक्षसा हृताम्}
{यथा श्येनः क्षुधायु्क्तः क्रन्दन्तीं वर्तिकां नयेत्}%॥ २६ ॥

\twolineshloka
{तथा कामवशं प्राप्तो राक्षसो जनकात्मजाम्}
{नयत्येष जनकजां तच्छ्रुत्वा पक्षिराट् तदा}%॥ २७ ॥

\twolineshloka
{युयुधे राक्षसेन्द्रेण रावणेन हतोऽपतत्}
{माघासितनवम्यां तु वसन्तीं रावणालये}%॥ २८

\onelineshloka
{मार्गमाणौ तदा तौ तु भ्रातरौ रामलक्ष्मणौ}%॥ २९ ॥

\twolineshloka
{जटायुषं तु दृष्ट्वैव ज्ञात्वा राक्षससंहृताम्}
{सीतां ज्ञात्वा ततः पक्षी संस्कृतस्तेन भक्तितः}%॥ ३० ॥

\twolineshloka
{अग्रतः प्रययौ रामो लक्ष्मणस्तत्पदानुगः}
{पम्पाभ्याशमनुप्राप्य शबरीमनुगृह्य च}%॥ ३१ ॥

\twolineshloka
{तज्जलं समुपस्पृश्य हनुमद्दर्शनं कृतम्}
{ततो रामो हनुमता सह सख्यं चकार ह}%॥ ३२ ॥

\twolineshloka
{ततः सुग्रीवमभ्येत्य अहनद्वालिवानरम्}
{प्रेषिता रामदेवेन हनुमत्प्रमुखाः प्रियाम्}%॥ ३३ ॥

\twolineshloka
{अङ्गुलीयकमादाय वायुसूनुस्तदागतः}
{सम्पातिर्दशमे मासि आचख्यौ वानराय ताम्}%॥ ३४ ॥

\twolineshloka
{ततस्तद्वचनादब्धिं पुप्लुवे शतयोजनम्}
{हनुमान्निशि तस्यां तु लङ्कायां परितोऽचिनोत्}%॥ ३५ ॥

\twolineshloka
{तद्रात्रिशेषे सीताया दर्शनं तु हनूमतः}
{द्वादश्यां शिंशपावृक्षे हनुमान्पर्यवस्थितः}%॥ ३६ ॥

\twolineshloka
{तस्यां निशायां जानक्या विश्वासायाह सङ्कथाम्}
{अक्षादिभिस्त्रयोदश्यां ततो युद्धमवर्त्तत}%॥ ३७ ॥

\twolineshloka
{ब्रह्मास्त्रेण त्रयोदश्यां बद्धः शक्रजिता कपिः}
{दारुणानि च रूक्षाणि वाक्यानि राक्षसाधिपम्}%॥ ३८ ॥

\twolineshloka
{अब्रवीद्वायुसूनुस्तं बद्धो ब्रह्मास्त्रसंयुतः}
{वह्निना पुच्छयुक्तेन लङ्काया दहनं कृतम्}%॥ ३९ ॥

\twolineshloka
{पूर्णिमायां महेन्द्राद्रौ पुनरागमनं कपेः}
{मार्गशीर्षप्रतिपदः पञ्चभिः पथि वासरैः}%॥ ४० ॥

\twolineshloka
{पुनरागत्य वर्षेह्नि ध्वस्तं मधुवनं किल}
{सप्तम्यां प्रत्यभिज्ञानदानं सर्वनिवेदनम्}%॥ ४१ ॥

\twolineshloka
{मणिप्रदानं सीतायाः सर्वं रामाय शंसयत्}
{अष्टम्युत्तरफाल्गुन्यां मुहूर्ते विजयाभिधे}%॥ ४२ ॥

\twolineshloka
{मध्यं प्राप्ते सहस्रांशौ प्रस्थानं राघवस्य च}
{रामः कृत्वा प्रतिज्ञां हि प्रयातुं दक्षिणां दिशम्}%॥ ४३ ॥

\twolineshloka
{तीर्त्वाहं सागरमपि हनिष्ये राक्षसेश्वरम्}
{दक्षिणाशां प्रयातस्य सुग्रीवोऽथाभव त्सखा}%॥ ४४ ॥

\threelineshloka
{वासरैः सप्तभिः सिन्धोस्तीरे सैन्यनिवेशनम्}
{पौषशुक्लप्रतिपदस्तृतीयां यावदम्बुधौ}
{उपस्थानं ससैन्यस्य राघवस्य बभूव ह}%॥ ४५ ॥

\twolineshloka
{विभीषणश्चतुर्थ्यां तु रामेण सह सङ्गतः}
{समुद्रतरणार्थाय पञ्चम्यां मन्त्र उद्यतेः}%॥ ४६ ॥

\twolineshloka
{प्रायोपवेशनं चक्रे रामो दिनचतुष्टयम्}
{समुद्राद्वरलाभश्च सहोपायप्रदर्शनः}%॥ ४७ ॥

\twolineshloka
{सेतोर्दशम्यामारम्भस्त्रयोदश्यां समापनम्}
{चतुर्दश्यां सुवेलाद्रौ रामः सेनां न्यवे शयत्}%॥ ४८ ॥

\twolineshloka
{पूर्णिमास्या द्वितीयायां त्रिदिनैः सैन्यतारणम्}
{तीर्त्वा तोयनिधिं रामः शूरवानरसैन्यवान्}%॥ ४९ ॥

\twolineshloka
{रुरोध च पुरीं लङ्कां सीतार्थं शुभलक्षणः}
{तृतीयादिदशम्यन्तं निवेशश्च दिनाष्टकः}%॥ ५० ॥

\twolineshloka
{शुकसारणयोस्तत्र प्राप्तिरेकादशीदिने}
{पौषासिते च द्वादश्यां सैन्यसङ्ख्यानमेव च}%॥ ५१ ॥

\twolineshloka
{शार्दूलेन कपीन्द्राणां सारासारोपवर्णनम्}
{त्रयोदश्याद्यमान्ते च लङ्कायां दिवसैस्त्रिभिः}%॥ ५२ ॥

\twolineshloka
{रावणः सैन्यसं ख्यानं रणोत्साहं तदाऽकरोत्}
{प्रययावङ्गदो दौत्ये माघशुक्लाद्यवासरे}%॥ ५३ ॥

\twolineshloka
{सीतायाश्च तदा भर्तुर्मायामूर्धादिदर्शनम्}
{माघशुक्लद्वितीया यां दिनैः सप्तभिरष्टमीम्}%॥ ५४ ॥

\twolineshloka
{रक्षसां वानराणां च युद्धमासीच्च सङ्कुलम्}
{माघशुक्लनवम्यां तु रात्राविन्द्रजिता रणे}%॥ ५५ ॥

\twolineshloka
{रामलक्ष्मणयोर्ना गपाशबन्धः कृतः किल}
{आकुलेषु कपीशेषु हताशेषु च सर्वशः}%॥ ५६ ॥

\twolineshloka
{वायूपदेशाद्गरुडं सस्मार राघवस्तदा}
{नागपाशविमोक्षार्थं दशम्यां गरु डोऽभ्यगात्}%॥ ५७ ॥

\twolineshloka
{अवहारो माघशुक्लैस्यैकादश्यां दिनद्वयम्}
{द्वादश्यामाञ्जनेयेन धूम्राक्षस्य वधः कृतः}%॥ ५८ ॥

\twolineshloka
{त्रयोदश्यां तु तेनैव निहतोऽकम्पनो रणे}
{मायासीतां दर्शयित्वा रामाय दशकन्धरः}%॥ ५९ ॥

\twolineshloka
{त्रासयामास च तदा सर्वान्सैन्यगतानपि}
{माघशुक्लचतुर्द्दश्यां यावत्कृष्णादिवासरम्}%॥ ६० ॥

\twolineshloka
{त्रिदिनेन प्रहस्तस्य नीलेन विहितो वधः}
{माघकृष्णद्वितीयायाश्चतुर्थ्यन्तं त्रिभिर्दिनैः}%॥ ६१ ॥

\twolineshloka
{रामेण तुमुले युद्धे रावणो द्रावितो रणात्}
{पञ्चम्या अष्टमी यावद्रावणेन प्रबोधितः}%॥ ६२ ॥

\twolineshloka
{कुम्भकर्णस्तदा चक्रेऽभ्यवहारं चतुर्दिनम्}
{कुम्भकर्णोकरोद्युद्धं नवम्यादिचतुर्दिनैः}%॥ ६३ ॥

\twolineshloka
{रामेण निहतो युद्धे बहुवानरभक्षकः}
{अमावास्यादिने शोकाऽभ्यवहारो बभूव ह}%॥ ६४ ॥

\twolineshloka
{फाल्गुनप्रतिपदादौ चतुर्थ्यन्तैश्चतुर्दिनैः}
{नरान्तकप्रभृतयो निहताः पञ्च राक्षसाः}%॥ ६५ ॥

\twolineshloka
{पञ्चम्याः सप्तमीं यावदतिकायवधस्त्र्यहात्}
{अष्टम्या द्वादशीं यावन्निहतो दिनपञ्चकात्}%॥ ६६ ॥

\twolineshloka
{निकुम्भकुम्भौ द्वावेतौ मकराक्षश्चतुर्दिनैः}
{फाल्गुनासितद्वितीयाया दिने वै शक्रजिज्जितः}%॥ ६७ ॥

\twolineshloka
{तृतीयादौ सप्तम्यन्तदिनपञ्चकमेव च}
{ओषध्यानयवैयग्र्यादवहारो बभूव ह}%॥ ६८ ॥

\twolineshloka
{अष्टम्यां रावणो मायामैथिलीं हतवान्कुधीः}
{शोकावेगात्तदा रामश्चक्रे सैन्यावधारणम्}%॥ ६९ ॥

\twolineshloka
{ततस्त्रयोदशीं यावद्दिनैः पञ्चभिरिन्द्रजित्}
{लक्ष्मणेन हतो युद्धे विख्यातबलपौरुषः}%॥ ७० ॥

\twolineshloka
{चतुर्द्दश्यां दशग्रीवो दीक्षामापावहारतः}
{अमावास्यादिने प्रागाद्युद्धाय दशकन्धरः}%॥ ७१ ॥

\twolineshloka
{चैत्रशुक्लप्रतिपदः पञ्चमीदिनपञ्चके}
{रावणो युध्यमानो ऽभूत्प्रचुरो रक्षसां वधः}%॥ ७२ ॥

\twolineshloka
{चैत्रशुक्लाष्टमीं यावत्स्यन्दनाश्वादिसूदनम्}
{चैत्रशुक्लनवम्यां तु सौमित्रेः शक्तिभेदने}%॥ ७३ ॥

\twolineshloka
{कोपाविष्टेन रामेण द्रावितो दशकन्धरः}
{विभीषणोपदेशेन हनुमद्युद्धमेव च}%॥ ७४ ॥

\twolineshloka
{द्रोणाद्रेरोषधीं नेतुं लक्ष्मणार्थमुपागतः}
{विशल्यां तु समादाय लक्ष्मणं तामपाययत्}%॥ ७५ ॥

\twolineshloka
{दशम्यामवहारोऽभूद्रात्रौ युद्धं तु रक्षसाम्}
{एकादश्यां तु रामाय रथो मातलिसारथिः}%॥ ७६ ॥

\twolineshloka
{प्राप्तो युद्धाय द्वादश्यां यावत्कृष्णां चतुर्दशीम्}
{अष्टादशदिने रामो रावणं द्वैरथेऽवधीत्}%॥ ७७ ॥

\twolineshloka
{संस्कारा रावणादीनाममावा स्यादिनेऽभवन्}
{सङ्ग्रामे तुमुले जाते रामो जयमवाप्तवान्}%॥ ७८ ॥

\twolineshloka
{माघशुक्लद्वितीयादिचैत्रकृष्णचतुर्द्दशीम्}
{सप्ताशीतिदिनान्येवं मध्ये पंवदशा हकम्}%॥ ७९ ॥

\threelineshloka
{युद्धावहारः सङ्ग्रामो द्वासप्ततिदिनान्यभूत्}
{वैशाखादि तिथौ राम उवास रणभूमिषु}
{अभिषिक्तो द्वितीयायां लङ्काराज्ये विभी षणः}%॥ ८० ॥

\twolineshloka
{सीताशुद्धिस्तृतीयायां देवेभ्यो वरलम्भनम्}
{दशरथस्यागमनं तत्र चैवानुमोदनम्}%॥ ८१ ॥

\twolineshloka
{हत्वा त्वरेण लङ्केशं लक्ष्मणस्याग्रजो विभुः}
{गृहीत्वा जानकीं पुण्यां दुःखितां राक्षसेन तु}%॥ ८२ ॥

\twolineshloka
{आदाय परया प्रीत्या जानकीं स न्यवर्तत}
{वैशाखस्य चतुर्थ्यां तु रामः पुष्पकमा श्रितः}%॥ ८३ ॥

\twolineshloka
{विहायसा निवृत्तस्तु भूयोऽयोध्यां पुरीं प्रति}
{पूर्णे चतुर्दशे वर्षे पञ्चम्यां माधवस्य च}%॥ ८४ ॥

\twolineshloka
{भारद्वाजाश्रमे रामः सगणः समु पाविशत्}
{नन्दिग्रामे तु षष्ठ्यां स पुष्पकेण समागतः}%॥ ८५ ॥

\twolineshloka
{सप्तम्यामभिषिक्तोऽसौ भूयोऽयोध्यायां रघूद्वहः}
{दशाहाधिकमासांश्च चतुर्दश हि मैथिली}%॥ ८५ ॥

\twolineshloka
{उवास रामरहिता रावणस्य निवेशने}
{द्वाचत्वारिंशके वर्षे रामो राज्यमकारयत्}%॥ ८७ ॥

\twolineshloka
{सीतायास्तु त्रयस्त्रिंशद्वर्षाणि तु तदा भवन्}
{स चतुर्दशवर्षान्ते प्रविष्टः स्वां पुरीं प्रभुः}%॥ ८८ ॥

\twolineshloka
{अयोध्यां नाम मुदितो रामो रावणदर्पहा}
{भ्रातृभिः सहितस्तत्र रामो राज्यमकार यत्}%॥ ८९ ॥

\twolineshloka
{दशवर्षसहस्राणि दशवर्षशतानि च}
{रामो राज्यं पालयित्वा जगाम त्रिदिवालयम्}%॥ ९० ॥

\twolineshloka
{रामराज्ये तदा लोका हर्षनिर्भरमा नसाः}
{बभूवुर्धनधान्याढ्याः पुत्रपौत्रयुता नराः}%॥ ९१ ॥

\twolineshloka
{कामवर्षी च पर्जन्यः सस्यानि गुणवन्ति च}
{गावस्तु घटदोहिन्यः पादपाश्च सदा फलाः}%॥ ९२ ॥

\twolineshloka
{नाधयो व्याधयश्चैव रामराज्ये नराधिप}
{नार्यः पतिव्रताश्चासन्पितृभक्तिपरा नराः}%॥ ९३ ॥

\twolineshloka
{द्विजा वेदपरा नित्यं क्षत्रिया द्विज सेविनः}
{कुर्वते वैश्यवर्णाश्च भक्तिं द्विजगवां सदा}%॥ ९४ ॥

\twolineshloka
{न योनिसङ्करश्चासीत्तत्र नाचारसङ्करः}
{न वन्ध्या दुर्भगा नारी काकवन्ध्या मृत प्रजा}%॥ ९५ ॥

\twolineshloka
{विधवा नैव काप्यासीत्सभर्तृका न लप्यते}
{नावज्ञां कुर्वते केपि मातापित्रोर्गुरोस्तथा}%॥ ९६ ॥

\twolineshloka
{न च वाक्यं हि वृद्धानामुल्लं घयति पुण्यकृत्}
{न भूमिहरणं तत्र परनारीपराङ्मुखाः}%॥ ९७ ॥

\twolineshloka
{नापवादपरो लोको न दरिद्रो न रोगभाक्}
{न स्तेयो द्यूतकारी च मैरेयी पापिनो नहि}%॥ ९८ ॥

\twolineshloka
{न हेमहारी ब्रह्मघ्नो न चैव गुरुतल्पगः}
{न स्त्रीघ्नो न च बालघ्नो न चैवानृतभाषणः}%॥ ९९ ॥

\twolineshloka
{न वृत्तिलोपकश्चासीत्कूट साक्षी न चैव हि}
{न शठो न कृतघ्नश्च मलिनो नैव दृश्यते}%॥ १०० ॥

\twolineshloka
{सदा सर्वत्र पूज्यन्ते ब्राह्मणा वेदपारगाः}
{नावैष्णवोऽव्रती राजन्राम राज्येऽतिविश्रुते}%॥ १०१ ॥

\twolineshloka
{राज्यं प्रकुर्वतस्तस्य पुरोधा वदतां वरः}
{वसिष्ठो मुनिभिः सार्द्धं कृत्वा तीर्थान्यनेकशः}%॥ २ ॥

\twolineshloka
{आजगाम ब्रह्मपुत्रो महाभागस्तपोनिधिः}
{रामस्तं पूजयामास मुनिभिः सहितं गुरुम्}%॥ ३ ॥

\twolineshloka
{अभ्युत्थानार्घपाद्यैश्च मधुपर्कादिपूजया}
{प्रपच्छ कुशलं रामं वसिष्ठो मुनिपुङ्गवः}%॥ ४ ॥

\twolineshloka
{राज्ये चाश्वे गजे कोशे देशे सद्भ्रातृभृत्ययोः}
{कुशलं वर्त्तते राम इति पृष्टे मुनेस्तदा}%॥ ५ ॥

\uvacha{राम उवाच}

\twolineshloka
{सर्वत्र कुशलं मेऽद्य प्रसादाद्भवतः सदा}
{पप्रच्छ कुशलं रामो वसिष्ठं मुनिपुङ्गवम्}%॥ ६ ॥

\twolineshloka
{सर्वतः कुशली त्वं हि भार्यापुत्रसमन्वितः}
{स सर्वं कथयामास यथा तीर्थान्यशेषतः}%॥ ७ ॥

\twolineshloka
{सेवितानि धरापृष्ठे क्षेत्राण्यायतनानि च}
{रामाय कथयामास सर्वत्र कुशलं तदा}%॥ ८ ॥

\twolineshloka
{ततः स विस्मयाविष्टो रामो राजीवलोचनः}
{पप्रच्छ तीर्थमाहात्म्यं यत्तीर्थेषूत्तमोत्तमम्}%॥ १०९ ॥

॥इति श्रीस्कान्दे महापुराण एकाशीतिसाहस्र्यां संहितायां तृतीये ब्रह्मखण्डे पूर्वभागे धर्मारण्यमाहात्म्ये रामचरित्रवर्णनं नाम त्रिंशोऽध्यायः॥३०॥

    \sect{अष्टादशोऽध्यायः --- बलिनिग्रहवृत्तान्तवर्णने रामावतारवर्णनम्}

\src{स्कन्दपुराणम्}{खण्डः ७ (प्रभासखण्डः)}{वस्त्रापथक्षेत्रमाहात्म्यम्}{अध्यायः १८}
\vakta{वामनः}
\shrota{नारदः}
\tags{}
\notes{Vāmana Bhagavān narrates the context/story of different avatāras to Nārada.}
\textlink{https://sa.wikisource.org/wiki/स्कन्दपुराणम्/खण्डः_७_(प्रभासखण्डः)/वस्त्रापथक्षेत्रमाहात्म्यम्/अध्यायः_१८}
\translink{https://www.wisdomlib.org/hinduism/book/the-skanda-purana/d/doc627173.html}

\storymeta

\addtocounter{shlokacount}{180}

\twolineshloka
{लङ्कायां रावणो राज्यं करिष्यति महाबलः}
{त्रैलोक्यकण्टकं नाम यदासौ धारयिष्यति}%॥ १८१ ॥

\twolineshloka
{तदा दाशरथी रामः कौसल्यानन्दवर्द्धनः}
{भविष्ये भ्रातृभिः सार्द्धं गमिष्ये यज्ञमण्डपे}%॥ १८२ ॥

\twolineshloka
{ताडकां ताडयित्वाहं सुबाहुं यज्ञमन्दिरे}
{नीत्वा यज्ञाद्गमिष्यामि सीतायास्तु स्वयंवरे}%॥ १८३ ॥

\twolineshloka
{परिणेष्याभि तां सीतां भङ्क्त्वा माहेश्वरं धनुः}
{त्यक्त्वा राज्यं गमिष्यामि वने वर्षांश्चतुर्दश}%॥ १८४ ॥

\twolineshloka
{सीताहरणजं दुःखं प्रथमं मे भविष्यति}
{नासाकर्णविहीनां तां करिष्ये राक्षसीं वने}%॥ १८५ ॥

\twolineshloka
{चतुर्द्दशसहस्राणि त्रिशिरःखरदूषणान्}
{द्हत्वा हनिष्ये मारीचं राक्षसं मृगरूपिणम्}%॥ १८६ ॥

\twolineshloka
{हृतदारो गमिष्यामि दग्ध्वा गृध्रं जटायुषम्}
{सुग्रीवेण समं मैत्रीं कृत्वा हत्वाऽथ वालिनम्}%॥ १८७ ॥

\twolineshloka
{समुद्रं बन्धयिष्यामि नलप्रमुखवानरैः}
{लङ्कां संवेष्टयिष्यामि मारयिष्यामि राक्षसान्}%॥ १८८ ॥

\twolineshloka
{कुम्भकर्णं निहत्याजौ मेघनादं ततो रणे}
{निहत्य रावणं रक्षः पश्यतां सर्वरक्षसाम्}%॥ १८९ ॥

\twolineshloka
{विभीषणाय दास्यामि लङ्कां देवविनिर्मिताम्}
{अयोध्यां पुनरागत्य कृत्वा राज्यमकण्टकम्}%॥ १९० ॥

\twolineshloka
{कालदुर्वाससोश्चित्रचरित्रेणामरावतीम्}
{यास्येऽहं भ्रातृभिः सार्धं राज्यं पुत्रे निवेद्य च}%॥ १९१ ॥

॥इति श्रीस्कान्दे महापुराण एकाशीतिसाहस्र्यां संहितायां सप्तमे प्रभासखण्डे द्वितीये वस्त्रापथक्षेत्रमाहात्म्ये बलिनिग्रहवृत्तान्तवर्णनं नामाष्टादशोऽध्यायः॥१८॥

\closesection
    \chapt{सौरपुराणम्}

\src{सौरपुराणम्}{}{अध्यायः ३०}{श्लोकाः ४८--६९}
\vakta{}
\shrota{}
\notes{This chapter briefly recounts the life of Lord Rama---His divine birth, marriage to Sita, exile, Sita’s abduction by Ravana, the alliance with Hanuman and Sugriva, the war in Lanka, and His triumphant return and coronation. It concludes with a lineage of Rama's descendants from Lava/Kuśa.}
\textlink{https://archive.org/details/saurapurana1924compl/page/97/mode/2up}
\translink{}

\storymeta


\sect{त्रिंशोऽध्यायः --- इक्ष्वाकुकुलसम्भवनृपमालिका-कथनम्}

\addtocounter{shlokacount}{47}

\twolineshloka
{दीर्घबाहुस्ततो जज्ञे रघुस्तस्याभवत्सुतः}
{रघोरजस्तु विख्यातो राजा दशरथस्ततः} %॥४८॥

\twolineshloka
{तस्य पुत्राश्च चत्वारो धर्मज्ञा लोकविश्रुताः}
{रामोऽथ भरतश्चैव तृतीयो लक्ष्मणः स्मृतः} %॥४९॥

\twolineshloka
{चतुर्थश्चैव शत्रुघ्नो रामो नारायणः स्वयम्}
{धर्मज्ञः सत्यसङ्कल्पो महादेवपरायणः} %॥५०॥

\twolineshloka
{सीता तस्याभवद्भार्या पार्वत्यंशसमुद्भवा}
{जनकेन पुरा गौरी तपसा तोषिता यतः} %॥५१॥

\twolineshloka
{जनकाय ददौ शम्भुः प्रीतो धनुरनुत्तमम्}
{तद्धनुर्भञ्जयामास जनकस्य गृहे स्थितम्} %॥५२॥

\twolineshloka
{दृष्ट्वा पराक्रमं तस्य रामस्य गुणशालिनः}
{जनकः प्रददौ तस्मै सीतां ब्रह्मविदां वरः} %॥५३॥

\twolineshloka
{पित्रा कृतोऽभिषेकार्थं रामो राज्यस्य वै यदा}
{वारयामास कैकेयी तदा राज्ञः प्रिया वधूः} %॥५४॥

\twolineshloka
{राजंस्त्वया वरो दत्तः पूर्वमेव यतः प्रभो}
{राजानं मत्सुतं तस्माद्भरतं कर्तुमर्हसि} %॥५५॥

\twolineshloka
{इति तस्या वचः श्रुत्वा राज्ये तमभिषिच्य सः}
{प्रेषयामास तं रामं वनं प्रति सलक्ष्मणम्} %॥५६॥

\twolineshloka
{वनं गत्वा निवसतो भार्यां दृष्ट्वाऽथ राक्षसः}
{रावणो नाम पौलस्त्यो नीत्वा लङ्कां पुनर्ययौ} %॥५७॥

\twolineshloka
{अदृष्ट्वा तां ततः सीतां दुःखितौ रामलक्ष्मणौ}
{सख्यं वानरराजेन गत्वा दाशरथी द्विजाः} %॥५८॥

\twolineshloka
{सुग्रीवस्य सखा वीरो हनुमान्नाम वानरः}
{गत्वाऽथ रावणपुरीमपश्यज्जनकात्मजाम्} %॥५९॥

\twolineshloka
{अश्रुपूर्णेक्षणां सीतामिन्दीवरनिभाननाम्}
{विश्वासार्थं ददौ तस्यै रामस्यैवाङ्गुलीयकम्} %॥६०॥

\twolineshloka
{दृष्ट्वाऽङ्गुलीयकं सीता प्रहृष्टा च तदाऽभवत्}
{समाश्वास्य ततः सीतां प्रययौ राघवान्तिकम्} %॥६१॥

\twolineshloka
{रामस्तमागतं दृष्ट्वा प्रहर्षोत्फुल्ललोचनः}
{श्रुत्वा तद्वचनाद्वृत्तं युद्धाय कृतनिश्चयः} %॥६२॥

\twolineshloka
{सेतुं कृत्वाऽथ रक्षोभिर्युद्धं कृत्वा महामनाः}
{निहत्य रावणं रामो भ्रातृभिः सह सुव्रतः} %॥६३॥

\twolineshloka
{आनयामास तां सीतामशोकवनमध्यगाम्}
{प्रतिष्ठाप्य महादेवं सेतुमध्येऽथ राघवः} %॥६४॥

\twolineshloka
{लब्धवान्परमां भक्तिं शिवे शिवपराक्रमः}
{रामेश्वर इति ख्यातो महादेवः पिनाकधृक्} %॥६५॥

\twolineshloka
{तस्य दर्शनमात्रेण ब्रह्महत्यां व्यपोहति}
{अभिषिक्तस्ततो राज्ये रामो राजीवलोचनः} %॥६६॥

\twolineshloka
{पालयन्पृथिवीं सर्वां धर्मेण मुनिपुंगवाः}
{अयजद्देवदेवेशमश्वमेधेन शङ्करम्} %॥६७॥

\twolineshloka
{तस्य प्रसादात्स्वपदं प्राप्तवानथ राघवः}
{एवं सङ्क्षेपतः प्रोक्तं रामस्य चरितं मया} %॥६८॥

\twolineshloka
{इदं विस्तरतो विप्राः प्रोक्तं वाल्मीकिना पुनः}
{कुशश्चैको लवश्वान्यः पुत्रौ रामस्य सुव्रतौ} %॥६९॥

\threelineshloka
{सत्यसन्धौ महावीर्यौ महादेवपरायणौ}
{अतिथिश्च कुशाज्जज्ञे निषधस्तत्सुतोऽभवत्}
{नलस्तस्याभवत्पुत्रो नभस्तस्याभवत्सुतः} %॥७०॥

\twolineshloka
{ततश्चन्द्रावलोकश्च तारापीडस्ततोऽभवत्}
{ततश्चन्द्रगिरिर्नाम भानुजित्तत्सुतोऽभवत्} %॥७१॥

\twolineshloka
{एते सर्वे नृपाः प्रोक्ता इक्ष्वाकुकुलसम्भवाः}
{धर्मात्मानो महासत्त्वाः कीर्तिमन्तो दृढव्रताः} %॥७२॥

\twolineshloka
{इमं यः पठते नित्यमिक्ष्वाकोर्वंशमुत्तमम्}
{सर्वपापविनिर्मुक्तः सूर्यलोके महीयते} %॥७३॥

॥इति श्रीब्रह्मपुराणोपपुराणे श्रीसौरे सुतशौनकसंवादे प्रह्लादराज्यारोहणादीक्ष्वाकुकुलसम्भवनृपमालिकान्तकथनं नाम त्रिंशोऽध्यायः॥३०॥

\closesection
    \input{rama-charitam/vishnu-puranam/vishnu-puranam-ramayanam}
    \chapt{हरिवंशः}

\src{हरिवंशः}{हरिवंशपर्व}{अध्यायः ४१}{श्लोकाः १२१--१५५}
\notes{This chapter is part of the Harivamsha Purana, which is a supplement to the Mahabharata. Along with the rest of the avataras, this extract from the 41st chapter narrates the life and exploits of Bhagavān Rāma.}
\textlink{https://sa.wikisource.org/wiki/हरिवंशपुराणम्/पर्व_१_(हरिवंशपर्व)/अध्यायः_४१}
\translink{https://www.wisdomlib.org/hinduism/book/harivamsha-purana-dutt/d/doc485519.html}

\storymeta


\sect{रामकथा}

\addtocounter{shlokacount}{120}

\twolineshloka
{चतुर्विंशे युगे चापि विश्वामित्रपुरस्सरः}
{राज्ञो दशरथस्याथ पुत्रः पद्मायतेक्षणः}%।। १२१।।

\twolineshloka
{कृत्वाऽऽत्मानं महाबाहुश्चतुर्धा प्रभुरीश्वरः}
{लोके राम इति ख्यातस्तेजसा भास्करोपमः}%।। १२२।।

\twolineshloka
{प्रसादनार्थं लोकस्य रक्षसां निधनाय च}
{धर्मस्य च विवृद्धद्यर्थं जज्ञे तत्र महायशाः}%।। १२३।।

\twolineshloka
{तमप्याहुर्मनुष्येन्द्रं सर्वभूतपतेस्तनुम्}
{यस्मै दत्तानि चास्त्राणि विश्वामित्रेण धीमता}%।। १२४।।

\twolineshloka
{वधार्थं देवशत्रूणां दुर्धराणि सुरैरपि}
{यज्ञविध्नकरो येन मुनीनां भावितात्मनाम्}%।। १२५।।

\twolineshloka
{मारीचश्च सुबाहुश्च बलेन बलिनां वरौ}
{निहतौ च निराशी च कृतौ तेन महात्मना}%।। १२६।।

\twolineshloka
{वर्तमाने मखे येन जनकस्य महात्मनः} 
{भग्नं माहेश्वरं चापं क्रीडता लीलया पुरा} %।। १२७।

\twolineshloka
{यः समाः सर्वधर्मज्ञश्चतुर्दश वनेऽवसत्}
{लक्ष्मणानुचरो रामः सर्वभूतहिते रतः}%।। १२८।।

\twolineshloka
{रूपिणी यस्य पार्श्वस्था सीतेति प्रथिता जनैः}
{पूर्वोचिता तस्य लक्ष्मीर्भर्तारमनुगच्छति}%।। १२९।।

\twolineshloka
{चतुर्दश तपस्तप्त्वा वने वर्षाणि राघवः}
{जनस्थाने वसन् कार्यं त्रिदशानां चकार ह}%।।१३०।।


\threelineshloka
{सीतायाः पदमन्विच्छल्लँक्ष्मणानुचरो विभुः}
{विराधं च कबन्धं च राक्षसौ भीमविक्रमौ}
{जघान पुरुषव्याघ्रौ गन्धर्वौ शापवीक्षितौ}%।। १३१।।

\twolineshloka
{हुताशनार्केन्दुतडिद्घनाभैः प्रतप्तजाम्बूनदचित्रपुङ्खैः}
{महेन्द्रवज्राशनितुल्यसारैः शरैः शरीरेण वियोजितौ बलात्}%।। १३२।।

\twolineshloka
{सुग्रीवस्य कृते येन वानरेन्द्रो महाबलः}
{वाली विनिहतो युद्धे सुग्रीवश्चाभिषेचितः}%।। १३३।।

\twolineshloka
{देवासुरगणानां हि यक्षगन्धर्वभोगिनाम्}
{अवध्यं राक्षसेन्द्रं तं रावणं युद्धदुर्मदम्}%।। १३४।।

\twolineshloka
{युक्तं राक्षसकोटीभिर्नीलाञ्जनचयोपमम्}
{त्रैलोक्यरावणं घोरं रावणं राक्षसेश्वरम्}%।। १३५।।

\twolineshloka
{दुर्जयं दुर्धरं दृप्तं शार्दूलसमविक्रमम्}
{दुर्निरीक्ष्यं सुरगणैर्वरदानेन दर्पितम्}%।।१३६।।

\twolineshloka
{जघान सचिवैः सार्द्धं ससैन्यं रावणं युधि}
{महाभ्रघनसङ्काशं महाकायं महाबलम्}%।। १३७।।

\twolineshloka
{तमागस्कारिणं घोरं पौलस्त्यं युधि दुर्जयम्}
{सभ्रातृपुत्रसचिवं ससैन्यं क्रूरनिश्चयम्}%।। १३८ ।।

\twolineshloka
{रावणं निजघानाशु रामो भूतपतिः पुरा}
{मधोश्च तनयो दृप्तो लवणो नाम दानवः}%।।१३९।।

\twolineshloka
{हतो मधुवने वीरो वरदृप्तो महासुरः}
{समरे युद्धशौण्डेन तथा चान्येऽपि राक्षसाः}%।।१४०।।

\twolineshloka
{एतानि कृत्वा कर्माणि रामो धर्मभृतां वरः}
{दशाश्वमेधाञ्जारूथ्यानाजहार निरर्गलान्}%।।१४१।।

\twolineshloka
{नाश्रूयन्ताशुभा वाचो नाकुलं मारुतो ववौ}
{न वित्तहरणं त्वासीद् रामे राज्यं प्रशासति}%।। १४२।।

\twolineshloka
{पर्यदेवन्न विधवा नानर्थास्ताभवंस्तदा}
{सर्वमासीज्जगद् दान्तं रामे राज्यं प्रशासति}%।। १४३ ।।

\twolineshloka
{न प्राणिनां भयं चापि जलानलनिघातजम्}
{न च स्म वृद्धा बालानां प्रेतकार्याणि कुर्वते}%।। १४४।।


\threelineshloka
{ब्रह्म पर्यचरत् क्षत्र विशः क्षत्रमनुव्रताः}
{शूद्राश्चैव हि वर्णांस्त्रीञ्छुश्रूषन्त्यनहङ्कृताः}
{नार्यो नात्यचरन्भर्तॄन् भार्यां नात्यचरत् पतिः}%।। १४५।।

\twolineshloka
{सर्वमासीञ्जगद् दान्तं निर्दस्युरभवन्मही}
{राम एकोऽभवद् भर्त्ता रामः पालयिताभवत्}%।।१४६।।

\twolineshloka
{आयुर्वर्षसहस्राणि तथा पुत्रसहस्रिणः}
{अरोगाः प्राणिनश्चासन् रामे राज्यं प्रशासति}%।।१४७।।

\twolineshloka
{देवतानामृषीणां च मनुष्याणां च सर्वशः}
{पृथिव्यां समवायोऽभूद्रामे राज्यं प्रशासति}%।। १४८।।

\twolineshloka
{गाथा अप्यत्र गायन्ति ये पुराणविदो जनाः}
{रामे निबद्धतत्त्वार्था माहात्म्यं तस्य धीमतः}%।।१४९।।

\twolineshloka
{श्यामो युवा लोहिताक्षो दीप्तास्यो मितभाषिता}
{आजानुबाहुः सुमुखः सिंहस्कन्धो महाभुजः}%।।१५०।।

\twolineshloka
{दश वर्षसहस्राणि दश वर्षशतानि च}
{अयोध्याधिपतिर्भूत्वा रामो राज्यमकारयत्}%।। १५१।।

\twolineshloka
{ऋक्सामयजुषां घोषो ज्याघोषश्च महात्मनः}
{अव्युच्छिन्नोऽभवद्राज्ये दीयतां भुज्यतामिति}%।। १५२।।

\twolineshloka
{सत्त्ववान् गुणसम्पन्नो दीप्यमानः स्वतेजसा}
{अति चन्द्रं च सूर्यं च रामो दाशरथिर्बभौ}%।।१५३।।

\twolineshloka
{ईजे क्रतुशतैः पुण्यैः समाप्तवरदक्षिणैः}
{हित्वायोध्यां दिवं यातो राघवः समहाबलः}%।।१५४।।

\twolineshloka
{एवमेष महाबाहुरिक्ष्वाकुकुलनन्दनः}
{रावणं सगणं हत्वा दिवमाचक्रमे प्रभुः}%।। १५५।।

\ldots

॥इति श्रीमहाभारते खिलभागे हरिवंशे हरिवंशपर्वणि प्रादुर्भावानुसङ्ग्रहो नामैकचत्वारिंशोऽध्यायः॥४१॥


\closesection
    \chapt{नारदीय-पुराणम्}
\sect{लक्ष्मणाचलमाहात्म्यम् --- पञ्चसप्ततितमोऽध्यायः}

\src{नारदीय-पुराणम्}{}{}{}
\tags{concise, complete}
\notes{}
\textlink{}
\translink{}

\storymeta


\uvacha{मोहिन्युवाच}

\twolineshloka
{श्रुतं गोकर्णमाहात्म्यं वसो पापविनाशनम्}
{लक्ष्मणस्यापि माहात्म्यं वक्तुमर्हसि साम्प्रतम्}% १

\uvacha{वसुरुवाच}

\twolineshloka
{शृणु देवि प्रवक्ष्यामि माहात्म्यं लक्ष्मणस्य च}
{यं दृष्ट्वा मनुजो देवं मुच्यते सर्वपातकैः}% २

\twolineshloka
{चतुर्व्यूहावतारे यो देवः सकर्षणः स्वयय्}
{सर्वभूमण्डलं ह्येतत्सहस्रवदनः स्वराट्}% ३

\twolineshloka
{एकस्मिञ्छिरसि न्यस्तं नावैत्सिद्धार्थकोपमम्}
{देवो नारायणः साक्षाद्रा मो ब्रह्मादिवन्दितः}% ४

\twolineshloka
{प्रद्युम्नो भरतो भद्रे शत्रुघ्नो ह्यनिरुद्धकः}
{लक्ष्मणस्तु महाभागे स्वयं सङ्कर्षणः शिवः}% ५

\twolineshloka
{ब्रह्माद्यैः प्रार्थितः पूर्वं साक्षाद्देवो रमापतिः}
{रामादिनामभिर्जज्ञे चतुर्द्धा दिग्ग्रथान्नृपात्}% ६

\twolineshloka
{ततः कालान्तरे देवि विश्वामित्रो मुनीश्वरः}
{यज्ञरक्षार्थमागत्य प्रार्थयद्रा मलक्ष्मणौ}% ७

\twolineshloka
{ततो राजा दशरथः प्राणेभ्योऽपि प्रियौ सुतौ}
{मुनेः शापभयाद्भीतो ददौ तौ रामलक्ष्मणौ}% ८

\twolineshloka
{गत्वा यज्ञं मुनीन्द्र स्य गाधिपस्य ररक्षतुः}
{सताडकं सुबाहुं तु हत्वा प्रक्षिप्य दूरतः}% ९

\twolineshloka
{मारीचं मानवास्त्रेण विश्वामित्रमतोषयत्}
{ततः प्रीतान्मुनिश्रेष्ठादस्त्रग्राममवाप्य च}% १०

\twolineshloka
{उवाच स कियत्कालं सानुजस्तेन सत्कृतः}
{वैदेहनगरं नीतो विश्वामित्रेण तत्परम्}% ११

\twolineshloka
{ततस्तु राजा जनको विश्वामित्रं सुसत्कृतम्}
{पप्रच्छ बालकावेतौ कस्य क्षत्रकुलेशितुः}% १२

\twolineshloka
{ततस्तस्मै मुनिवरो राज्ञो दशरथस्य तौ}
{पुत्रौ निवेदयामास भ्रातरौ रामलक्ष्मणौ}% १३

\twolineshloka
{ततो विदेहः सप्रीतो दृष्ट्वा रामं च लक्ष्मणम्}
{सीतोर्मिलाख्ययोः पुत्र्! योश्चेतसाकल्पयत्पती}% १४

\twolineshloka
{त्रिकालज्ञस्तु स मुनिर्ज्ञात्वा तस्य मनोगतम्}
{मोदमानोऽथ जनकं प्राह दर्शय तद्धनुः}% १५

\twolineshloka
{सीतास्वयंवरे न्यस्तं न्यासभूतं महेशितुः}
{राजा श्रुत्वा तु तद्वाक्यं विश्वामित्रस्य सत्वरम्}% १६

\twolineshloka
{भृत्यत्रिशत्यानाय्यास्मै दर्शयामास सादरम्}
{रामश्चण्डीशचापं तद्वामदोष्णोद्धरन् क्षणात्}% १७

\twolineshloka
{सज्यं विकृष्य सहसा बभञ्जेक्षुमिवेभराट्}
{ततोऽति मिथिलः प्रीतः स्वे कन्ये रामलक्ष्मणौ}% १८

\twolineshloka
{समभ्यर्च्यार्पयामास ताभ्यां ते विधिपूर्वकम्}
{ज्ञात्वा मुनिवरादन्यौ राज्ञो दशरथस्य तु}% १९

\twolineshloka
{ताभ्यां सह तमाहूय भ्रातृकन्ये अदापयत्}
{ततः स कृतदारैस्तु चतुर्भिस्तनयैः सह}% २०

\twolineshloka
{समर्चितो विदेहेनायोध्यां मुन्याज्ञया ययौ}
{मार्गे भृगुपतेर्दर्पं शमयित्वा स राघवः}% २१

\twolineshloka
{पितृभ्रातृयुतः श्रीमान्मुमुदे बहुवत्सरान्}
{पण्डितैस्तु वसिष्ठाद्यैर्बोधितोऽसौ निजं महः}% २२

\twolineshloka
{ब्रह्माख्यं बुबुधे रामो मानुषत्वं विडम्बयन्}
{ततो दशरथो राजा ज्ञातज्ञेयं निजं सुतम्}% २३

\twolineshloka
{रामं समुद्यतो हृष्टो यौवराज्येऽभिषेचितुम्}
{यज्ज्ञात्वा कैकयी देवी राज्ञः प्रेष्ठा कनीयसी}% २४

\twolineshloka
{सन्निवार्य हठात्तस्य पुत्रस्य तदरोचत}
{ततो रामो मुदे तस्याः पित्राननुमतो ययौ}% २५

\twolineshloka
{सभार्यः सः ससौमित्रिश्चित्रकूटं गिरिं शुभे}
{कियत्कालमुवासासौ तत्रैव मुनिवेषधृक्}% २६

\twolineshloka
{मातामहगृहात्तच्च श्रुत्वाऽयातः पितुर्वधम्}
{स विज्ञाय मृतं तातं हा रामेति विराविणम्}% २७

\twolineshloka
{धिक्कृत्य कैकयीं यातो रामं स विनिवर्तितुम्}
{ततः स्वपादुके दत्वा भरतं विनिवर्त्य च}% २८

\twolineshloka
{रामोऽत्रेश्चाप्यगस्त्यस्य सुतीक्ष्णस्याश्रमेष्वगात्}
{तेषु द्वादश वर्षाणि गमयित्वा रघूद्वहः}% २९

\twolineshloka
{भार्यानुजान्वितः श्रीमांस्ततः पञ्चवटीमगात्}
{तत्रावसज्जनस्थाने त्रिशिरःखरदूषणान्}% ३०

\twolineshloka
{शूर्पणख्या विकृतया प्रेरितान्स व्यनाशयत्}
{ततो रक्षःसहस्रैश्च चतुर्द्दशभिरागतान्}% ३१

\twolineshloka
{विचित्रवाजैर्नाराचैर्यमक्षयमनीनयत्}
{यच्छ्रुत्वा रक्षसां राजा मारीचं काञ्चनं मृगम्}% ३२

\twolineshloka
{दर्शयित्वापवाह्यैतौ सीतां हृत्वा जटायुषम्}
{रुन्धानं मार्गमाहत्य लङ्कायां समुपानयत्}% ३३

\twolineshloka
{आगत्य तौ हृतां सीतां मार्गमाणौ समन्ततः}
{दृष्ट्वा जटायुषं शान्तं दग्ध्वा हत्वा कबन्धकम्}% ३४

\twolineshloka
{शबरीमनुकम्प्याथ ऋष्यमूकमुपागतौ}
{ततस्तु हनुमद्वाक्यात्स्वसख्युः प्लवगेशितुः}% ३५

\twolineshloka
{विद्विषं वालिनं हत्वा सुग्रीवमकरोन्नृपम्}
{तदाज्ञप्तास्तु ते कीशाः सर्वतः समुपागताः}% ३६

\twolineshloka
{हनुमत्प्रमुखाः सीतां मार्गन्तो दक्षिणोदधिम्}
{प्राप्य सम्पातिवचनाल्लङ्कायां निश्चयं गताः}% ३७

\twolineshloka
{ततस्तु हनुमानेकः प्राप्य लङ्कां पुरीं कपिः}
{समुद्र स्य परे पारेऽपश्यद्रा मप्रियां सतीम्}% ३८

\twolineshloka
{दत्त्वा रामाङ्गुलीरत्नं विश्वासमुपपाद्य ताम्}
{तयोः कुशलमाश्राव्य लब्ध्वा चूडामणिं ततः}% ३९

\twolineshloka
{भङ्क्त्वा चाशोकवनिकां हत्वा चाक्षं ससैन्यकम्}
{इन्द्र जिद्बन्धनात्प्राप्य सम्भाष्यापि च रावणम्}% ४०

\twolineshloka
{दग्ध्वा लङ्कां पुरीं कृत्स्नां पुनर्दृष्ट्वा तु मैथिलीम्}
{लब्धाज्ञोऽणवमुल्लङ्घ्य रामायैनां न्यवेदयत्}% ४१

\twolineshloka
{श्रुत्वा रामोऽपि तां सीतां राक्षसस्य निवासगाम्}
{सार्द्धं स कपिसैन्येन सम्प्राप्तो मकरालयम्}% ४२

\twolineshloka
{सागरानुमतेनासौ सेतुं बद्ध्वा महोदधौ}
{अद्रि कूटेः परं तीरं प्राप्य सेनां न्यवेशयत्}% ४३

\twolineshloka
{ततोऽसौ रावणो भ्रात्रा बोधितोऽपि कनीयसा}
{प्रदानं तत्र मैथिल्यास्तद्भर्त्रे न त्वरोचयत्}% ४४

\twolineshloka
{पदा हतस्ततस्तेन रावणेन विभीषणः}
{सम्प्राप्तः शरणं रामं रामो लङ्कामुपारुणत्}% ४५

\twolineshloka
{ततस्तु मन्त्रिणोऽमात्याः पुत्रा भृत्याः प्रचोदिताः}
{युद्धाय ते क्षयं नीतास्ताभ्यां सङ्ख्ये कपीश्वरैः}% ४६

\twolineshloka
{लक्ष्मणः शक्रजेतारं जघ्निवान्निशितैः शरैः}
{रामोऽपि कुम्भश्रवणं रावणं चाप्यजीघनत्}% ४७

\twolineshloka
{विभीषणेन तत्कृत्यं कारयित्वा निजां प्रियाम्}
{वह्नौ संशोध्य दत्वास्मै रामो रक्षोगणेशताम्}% ४८

\twolineshloka
{लङ्कामायुश्च कल्पान्तं ययौ चीर्णव्रतः पुरीम्}
{पुष्पकेण विमानेन ससुग्रीवविभीषणः}% ४९

\twolineshloka
{नन्दिग्रामस्थभरतं नीत्वायोध्यां समाविशत्}
{मातॄः प्रणम्य ताः सर्वा भ्रातरस्ते पुरोधसा}% ५०

\twolineshloka
{वसिष्ठेनानुविज्ञाप्य रामं राज्येऽभ्यषेचयन्}
{ततो रामोऽपि भगवान्प्रजाः शासन्निवौरसान्}% ५१

\twolineshloka
{लोकापवादात्सन्त्रस्तः सीतां तत्याज धर्मवित्}
{सा तु सम्प्राप्य वाल्मीकेराश्रमं न्यवसत्सुखम्}% ५२

\twolineshloka
{पुत्रौ च सुषुवे तत्र नाम्ना ख्यातौ कुशीलवौ}
{वाल्मीकिस्तु तयोः कृत्वा यथा समुदिताः क्रियाः}% ५३

\twolineshloka
{रामायणं विरच्यैतावध्यापयदुदारधीः}
{तौ गायमानौ सत्रेषु मुनीनां ख्यातिमागतौ}% ५४

\twolineshloka
{यज्ञे रामस्य सम्प्राप्तौ वाजिमेधे प्रवर्तिते}
{तत्र ताभ्यां तु तद्गीतं स्वचरित्रं प्रसन्नधीः}% ५५

\twolineshloka
{मुनिमाकारयामास ससीतं तत्र संसदि}
{सा तु रामाय तौ पुत्रौ निवेद्य जगतीजनिः}% ५६

\twolineshloka
{जगत्या विवरं भूयो विवेशासीत्तदद्भुतम्}
{ततः परं ब्रह्मचर्यं यज्ञमेव त्रयोदश}% ५७

\twolineshloka
{सहस्राब्दान्प्रकुवार्णस्तस्थौ भुवि रघूत्तमः}
{ततस्तु काले दुर्वासाः सम्प्राप्तो राघवं प्रति}% ५८

\twolineshloka
{ब्रह्मणा प्रेषितो भद्रे वैकुण्ठगमनाय च}
{स एकान्तगतो रामं प्राह कोऽपीह नाऽव्रजेत्}% ५९

\twolineshloka
{आगतो वध्यतां यातु रामस्तत्प्रतिजज्ञिवान्}
{स लक्ष्मणं समाहूय प्रोवाच रघुनन्दनः}% ६०

\twolineshloka
{द्वारि तिष्ठात्र निर्विष्टो वध्यतां मे प्रयास्यति}
{स तथेति प्रतिज्ञाय रामस्याज्ञां समाचरन्}% ६१

\twolineshloka
{प्रवेशनं न कस्यापि प्रददौ रामसन्निधौ}
{एवमेकान्तगं रामं कालसंविदमास्थितम्}% ६२

\twolineshloka
{ज्ञात्वाथ द्वारि दुर्वासा लक्ष्मणं समुपागमत्}
{तमागतं तु सम्प्रेक्ष्य सौमित्रिः प्रणिपत्य च}% ६३

\twolineshloka
{मुहूर्तं पालयेत्याह मन्त्रव्यग्रोऽस्ति राघवः}
{दुर्वासास्तद्वचः श्रुत्वा कालस्यार्थविधायकः}% ६४

\twolineshloka
{क्रुद्धः प्रोवाच सौमित्रिं देहि मेऽन्तप्रवेशनम्}
{नो चेत्त्वां भस्मसात्सद्यः करिष्यामि विचारय}% ६५

\twolineshloka
{वचो दुर्वाससः श्रुत्वा लक्ष्मणो जातसम्भ्रमः}
{मुनेर्भीतो विवेशान्तर्विज्ञापयितुमग्रजम्}% ६६

\twolineshloka
{दृष्ट्वा तु लक्ष्मणं काल उत्थाय कृतमन्त्रकः}
{प्रतिज्ञां पालयेत्युक्त्वा ययौ रामविसर्ज्जितः}% ६७

\twolineshloka
{ततो निष्क्रम्य भगवान् रामो धर्मभृतां वरः}
{प्रतोष्य तं मुनिं प्रीतो दुर्वाससमभोजयत्}% ६८

\twolineshloka
{भोजयित्वा प्रणम्यैनं विसृज्य प्राह लक्ष्मणम्}
{भ्रातर्लक्ष्मण सम्प्राप्तं सङ्कटं धर्मकारणात्}% ६९

\twolineshloka
{यत्त्वं मे वध्यतां प्राप्तो दैवं हि बलवत्तरम्}
{मया त्यक्तस्ततो वीर यथेच्छं गच्छ साम्प्रतम्}% ७०

\twolineshloka
{ततः प्रणम्य तं रामं सत्यधर्मे व्यवस्थितम्}
{दक्षिणां दिशमाश्रित्य तपश्चक्रे नगोपरि}% ७१

\twolineshloka
{ततो रामोऽपि भगवान्ब्रह्मप्रार्थनया पुनः}
{स्वधामाविशदव्यग्रः ससाकेतः सकोशलः}% ७२

\twolineshloka
{गोप्रतारे सरय्वां ये रामं सञ्चिन्त्य सम्प्लुताः}
{ते रामधाम विविशुर्दिव्याङ्गा योगिदुर्लभम्}% ७३

\twolineshloka
{लक्ष्मणस्तु कियत्कालं तपोयोगबलान्वितः}
{रामानुगमनेनैव स्वधामाविशदव्ययम्}% ७४

\twolineshloka
{सान्निध्यं पर्वते तस्मिन्दत्त्वा सौमित्रिरन्वहम्}
{चक्रं निजाधिकारं स ततस्तत्क्षेत्रमुत्तमम्}% ७५

\twolineshloka
{ये पश्यन्ति नरा भक्त्या लक्ष्मणं लक्ष्मणाचले}
{ते कृतार्था न सन्देहो गच्छन्ति हरिमन्दिरम्}% ७६

\twolineshloka
{तत्र दानं प्रशंसन्ति स्वर्णगोभूमिवाजिनाम्}
{दत्तं तत्राक्षयं सर्वं हुतं जप्तं कृतं तथा}% ७७

\twolineshloka
{बहुना किमिहोक्तेन दर्शनं तस्य दुर्लभम्}
{अगस्त्याज्ञान्तरा देवि दृष्टे मुक्तिर्न संशयः}% ७८

\twolineshloka
{एतद्रा मचरित्रं तु लक्ष्मणाख्यानंसयुतम्}
{श्रावयेद्योऽपि शृणुयात्स्यातां तौ रामवल्लभौ}% ७९

॥इति श्रीबृहन्नारदीयपुराणे बृहदुपाख्याने उत्तरभागे वसुमोहिनीसंवादे रामलक्ष्मणचरित्रसहितलक्ष्मणाचलमाहात्म्यं नाम पञ्चसप्ततितमोऽध्यायः॥७५॥

    
    \part{रामायणान्तर्गताः कथाः}
    \input{katha/devi-bhagavatam/vedavati-charitram}
    \sect{हनुमच्चरित्रम् --- एकोनाशीतितमोऽध्यायः}

\uvacha{सनत्कुमार उवाच}

\twolineshloka
{अथापरं वायुसूनोश्चरितं पापनाशनम्}
{यदुक्तं स्वासु रामेण आनन्दवनवासिना}% १

\twolineshloka
{सद्योजाते महाकल्पे श्रुतवीर्ये हनूमति}
{मम श्रीरामचन्द्र स्य भक्तिरस्तु सदैव हि}% २

\twolineshloka
{शृणुष्व गदतो मत्तः कुमारस्य कुमारक}
{चरितं सर्वपापघ्नं शृण्वतां पठतां सदा}% ३

\twolineshloka
{वाञ्छाम्यहं सदा विप्र सङ्गमं कीशरूपिणा}
{रहस्यं रहसि स्वस्य ममानन्दवनोत्तमे}% ४

\twolineshloka
{परीतेऽत्र सखायो मे सख्यश्च विगतज्वराः}
{क्रीडन्ति सर्वदा चात्र प्राकट्येऽपि रहस्यपि}% ५

\twolineshloka
{कस्मिंश्चिदवतारे तु यद्वृत्तं च रहो मम}
{तदत्र प्रकटं तुभ्यं करोमि प्रीतमानसः}% ६

\twolineshloka
{आविर्भूतोऽस्म्यहं पूर्वं राज्ञो दशरथक्षये}
{चतुर्व्यूहात्मकस्तत्र तस्य भार्यात्रये मुने}% ७

\twolineshloka
{ततः कतिपयैरब्दैरागतो द्विजपुङ्गवः}
{क्श्विमित्रोऽथयामास पितरं मम भूपतिम्}% ८

\twolineshloka
{यक्षरक्षोविघातार्थं लक्ष्मणेन सहैव माम्}
{प्रेषयामास धर्मात्मा सिद्धाश्रममरण्यकम्}% ९

\twolineshloka
{तत्र गत्वाश्रममृषेर्दूषयन्तौ निशाचरौ}
{ध्वस्तौ सुबाहुमारीचौ प्रसन्नोऽभूत्तदा मुनिः}% १०

\twolineshloka
{अस्त्रग्रामं ददौ मह्यं मासं चावासयत्तथा}
{ततो गाधिसुतो धीमान् ज्ञात्वा भाव्यर्थमादरात्}% ११

\twolineshloka
{मिथिलामनयत्तत्र रौद्रं चादर्शयद्धनुः}
{तस्य कन्यां पणीभूतां सीतां सुरसुतोपमाम्}% १२

\twolineshloka
{धनुर्विभज्य समिति लब्धवान्मानिनोऽस्य च}
{ततो मार्गे भृगुपतेर्दर्प्पमूढं चिरं स्मयन्}% १३

\twolineshloka
{व्यपनीयागमं पश्चादयोध्यां स्वपितुः पुरीम्}
{ततो राज्ञाहमाज्ञाय प्रजाशीलनमानसः}% १४

\twolineshloka
{यौवराज्ये स्वयं प्रीत्या सम्मन्त्र्याप्तैर्विकल्पितः}
{तच्छ्रुत्वा सुप्रिया भार्या कैकेयी भूपतिं मुने}% १५

\twolineshloka
{देवकार्यविधानार्थं विदूषितमतिर्जगौ}
{पुत्रो मे भरतो नाम यौवराज्येऽभिषिच्यताम्}% १६

\twolineshloka
{रामश्चतुर्दशसमा दण्डकान्प्रविवास्यताम्}
{तदाकर्ण्याहमुद्युक्तोऽरण्यं भार्यानुजान्वितः}% १७

\twolineshloka
{गन्तुं नृपतिनानुक्तोऽप्यगमं चित्रकूटकम्}
{तत्र नित्यं वन्यफलैर्मांसैश्चावर्तितक्रियः}% १८

\twolineshloka
{निवसन्नेव राज्ञस्तु निधनं चाप्यवागमम्}
{ततो भरतशत्रुघ्नौ भ्रातरौ मम मानदौ}% १९

\twolineshloka
{मातृवर्गयुतौ दीनौ साचार्यामात्यनागरौ}
{व्यजिज्ञपतमागत्य पञ्चवट्यां निजाश्रमम्}% २०

\twolineshloka
{अकल्पयं भ्रातृभार्यासहितश्च त्रिवत्सरम्}
{ततस्त्रयोदशे वर्षे रावणो नाम राक्षसः}% २१

\twolineshloka
{मायया हृतवान्सीतां प्रियां मम परोक्षतः}
{ततोऽहं दीनवदनः ऋष्यमूकं हि पर्वतम्}% २२

\twolineshloka
{भार्यामन्वेषयन्प्राप्तः सख्यं हर्यधिपेन च}
{अथ वालिनमाहत्य सुग्रीवस्तत्पदे कृतः}% २३

\twolineshloka
{सह वानरयूथैश्च साहाय्यं कृतवान्मम}
{विरुध्य रावणेनालं मम भक्तो विभीषणः}% २४

\twolineshloka
{आगतो ह्यभिषिच्याशु लङ्केशो हि विकल्पितः}
{हत्वा तु रावणं सङ्ख्ये सपुत्रामात्यबान्धवम्}% २५

\twolineshloka
{सीतामादाय संशुद्धामयोध्यां समुपागतः}
{ततः कालान्तरे विप्र सुग्रीवश्च विभीषणः}% २६

\twolineshloka
{निमन्त्रितौ पितुः श्राद्धे षट्कुलाश्च द्विजोत्तमाः}
{अयोध्यायां समाजग्मुस्ते तु सर्वे निमन्त्रिताः}% २७

\twolineshloka
{ऋते विभीषणं तत्र चिन्तयाने रघूत्तमे}
{शम्भुर्बाह्मणरूपेण षट्कुलैश्च सहागतः}% २८

\twolineshloka
{अथ पृष्टो मया शम्भुर्विभीषणसमागमे}
{नीत्वा मां द्र विडे देशे मोचय द्विजबन्धनात्}% २९

\twolineshloka
{मया निमन्त्रिताः श्राद्धे ह्यगस्त्याद्या मुनीश्वराः}
{सम्भोजितास्तु प्रययुः स्वस्वमाश्रममण्डलम्}% ३०

\twolineshloka
{ततः कालान्तरे विप्रा देवा दैत्या नरेश्वराः}
{गौतमेन समाहूताः सर्वे यज्ञसभाजिताः}% ३१

\twolineshloka
{ते सर्वे स्फाटिकं लिङ्गं त्र्! यम्बकाद्रौ निवेशितम्}
{सम्पूज्य न्यवंसस्तत्र देवदैत्यनृपाग्रजाः}% ३२

\twolineshloka
{तस्मिन्समाजे वितते सर्वैर्लिङ्गे समर्चिते}
{गौतमोऽप्यथ मध्याह्ने पूजयामास शङ्करम्}% ३३

\twolineshloka
{सर्वे शुक्लाम्बरधरा भस्मोद्धूलितविग्रहाः}
{सितेन भस्मना कृत्वा सर्वस्थाने त्रिपुण्ड्रकम्}% ३४

\twolineshloka
{नत्वा तु भार्गवं सर्वे भूतशुद्धिं प्रचक्रमुः}
{हृत्पद्ममध्ये सुषिरं तत्रैव भूतपञ्चकम्}% ३५

\twolineshloka
{तेषां मध्ये महाकाशमाकाशे निर्मलामलम्}
{तन्मध्ये च महेशानं ध्यायेद्दीप्तिमयं शुभम्}% ३६

\twolineshloka
{अज्ञानसंयुतं भूतं समलं कर्मसङ्गतः}
{तं देहमाकाशदीपे प्रदहेज्ज्ञानवह्निना}% ३७

\twolineshloka
{आकाशस्यावृतिं चाहं दग्ध्वाकाशमथो दहेत्}
{दग्ध्वाकाशमथो वायुमग्निभूतं तथा दहेत्}% ३८

\twolineshloka
{अब्भूतं च ततो दग्ध्वा पृथिवीभूतमेव च}
{तदाश्रितान्गुणान्दग्ध्वा ततो देहं प्रदाहयेत्}% ३९

\twolineshloka
{एवं प्रदग्ध्वा भूतादि देही तज्ज्ञानवह्निना}
{शिखामध्यस्थितं विष्णुमानन्दरसनिर्भरम्}% ४०

\twolineshloka
{निष्पन्नचन्द्र किरणसङ्काशकिरणं शिवम्}
{शिवाङ्गोत्पन्नकिरणैरमृतद्र वसंयुतैः}% ४१
सुशीतला ततो ज्वाला प्रशान्ता चन्द्र रश्मिवत्

\twolineshloka
{प्रसारितसुधारुग्भिः सान्द्री भूतश्च सम्प्लवः}
{अनेन प्लावितं भूतग्रामं सञ्चिन्तयेत्परम्}% ४२

\twolineshloka
{इत्थं कृत्वा भूतशुद्धिं क्रियार्हो मर्त्यः शुद्धो जायते ह्येव सद्यः}
{पूजां कर्तुं जप्यकर्मापि पश्चादेवं ध्यायेद्ब्रह्महत्यादिशुद्ध्यै}% ४३

\twolineshloka
{एवं ध्यात्वा चन्द्र दीप्तिप्रकाशं ध्यानेनारोप्याशु लिङ्गे शिवस्य}
{सदाशिवं दीपमध्ये विचिन्त्य पञ्चाक्षरेणार्चनमव्ययं तु}% ४४

\twolineshloka
{आवाहनादीनुपचारांस्तथापि कृत्वा स्नानं पूर्ववच्छङ्करस्य}
{औदुम्बरं राजतं स्वर्णपीठं वस्त्रादिच्छन्नं सर्वमेवेह पीठम्}% ४५

\twolineshloka
{अन्ते कृत्वा बुद्बुदाभ्यां च सृष्टिं पीठे पीठे नागमेकं पुरस्तात्}
{कुर्यात्पीठे चोर्द्ध्वके नागयुग्मं देवाभ्याशे दक्षिणे वामतश्च}% ४६

\twolineshloka
{जपापुष्पं नागमध्ये निधाय मध्ये वस्त्रं द्वादशप्रातिगुण्ये}
{सुश्वेतेन तस्य मध्ये महेशं लिङ्गाकारं पीठयुक्तं प्रपूज्यम्}% ४७

\twolineshloka
{एवं कृत्वा साधकास्ते तु सर्वे दत्त्वा दत्त्वा पञ्चगन्धाष्टगन्धम्}
{पुष्पैः पत्रैः श्रीतिलैरक्षतैश्च तिलोन्मिश्रैः केवलैश्च प्रपूज्य}% ४८

\twolineshloka
{धूपं दत्त्वा विधिवत्सम्प्रयुक्तं दीपं दत्त्वा चोक्तमेवोपहारम्}
{पूजाशेषं ते समाप्याथ सर्वे गीतं नृत्यं तत्र तत्रापि चक्रुः}% ४९}

\onelineshloka
{काले चास्मिन्सुव्रते गौतमस्य शिष्यः प्राप्तः शङ्करात्मेति नाम्ना}% ५०

\twolineshloka
{उन्मत्तवेषो दिग्वासा अनेकां वृत्तिमास्थितः}
{क्वचिद्द्विजातिप्रवरः क्वचिच्चण्डालसन्निभः}% ५१

\twolineshloka
{क्वच्छ्द्र समो योगी तापसः क्वचिदप्युत}
{गर्जत्युत्पतते चैव नृत्यति स्तौति गायति}% ५२

\twolineshloka
{रोदिति शृणुतेऽत्युक्तं पतत्युत्तिष्ठति क्वचित्}
{शिवज्ञानैकसम्पन्नः परमानन्दनिर्भरः}% ५३

\twolineshloka
{सम्प्राप्तो भोज्यवेलायां गौतमस्यान्तिकं ययौ}
{बुभुजे गुरुणा साकं क्वचिदुच्छिष्टमेव च}% ५४

\twolineshloka
{क्वचिल्लिहति तत्पात्रं तूष्णीमेवाभ्यगात्क्वचित्}
{हस्तं गृहीत्वैव गुरोः स्वयमेवाभुनक्क्वचित्}% ५५

\twolineshloka
{क्वचिद् गृहान्तरे मूत्रं क्वचित्कर्दमलेपनम्}
{सर्वदा तं गुरुर्दृष्ट्वा करमालम्ब्य मन्दिरम्}% ५६

\twolineshloka
{प्रविश्य स्वीयपीठे तमुपवेश्याप्यभोजयत्}
{स्वयं तदस्य पात्रेण बुभुजे गौतमो मुनिः}% ५७

\threelineshloka
{तस्य चित्तं परिज्ञातुं कदाचिदथ सुन्दरी}
{अहल्या शिष्यमाहूय भुङ्क्ष्वेति प्राह तं मुदा}
{निर्दिष्टो गुरुपत्न्या तु बुभुजे सोऽविशेषतः}% ५८

\twolineshloka
{यथा पपौ हि पानीयं तथा वह्निमपि द्विज}
{कन्टकानन्नवद्भुक्त्वा यथापूर्वमतिष्ठत}% ५९

\twolineshloka
{पुरो हि मुनिकन्याभिराहूतो भोजनाय च}
{दिने दिने तत्प्रदत्तं लोष्टमम्बु च गोमयम्}% ६०

\twolineshloka
{कर्दमं काष्ठदण्डं च भुक्त्वा पीत्वाथ हर्षितः}
{एतादृशो मुनिरसौ चण्डालसदृशाकृतिः}% ६१

\twolineshloka
{सुजीर्णोपानहौ हस्ते गृहीत्वा प्रलपन्हसन्}
{अन्त्यजोचितवेषश्च वृषपर्वाणमभ्यगात्}% ६२

\twolineshloka
{वृषपर्वेशयोर्मध्ये दिग्वासाः समतिष्ठत}
{वृषपर्वा तमज्ञात्वा पीडयित्वा शिरोऽच्छिनत्}% ६३

\twolineshloka
{हते तस्मिन्द्विजश्रेष्ठे जगदेतच्चराचरम्}
{अतीव कलुषं ह्यासीत्तत्रस्था मुनयस्तथा}% ६४

\twolineshloka
{गौतमस्य महाशोकः सञ्जातः सुमहात्मनः}
{निर्ययौ चक्षुषो वारि शोकं सन्दर्शयन्निव}% ६५

\twolineshloka
{गौतमः सर्वदैत्यानां सन्निधौ वाक्यमुक्तवान्}
{किमनेन कृतं पापं येन च्छिन्नमिदं शिरः}% ६६

\twolineshloka
{मम प्राणाधिकस्येह सर्वदा शिवयोगिनः}
{ममापि मरणं सत्यं शिष्यच्छद्मा यतो गुरुः}% ६७

\twolineshloka
{शैवानां धर्मयुक्तानां सर्वदा शिववर्तिनाम्}
{मरणं यत्र दृष्टं स्यात्तत्र नो मरणं ध्रुवम्}% ६८

\twolineshloka
{तच्छ्रुत्वा ह्यसुराचार्यः शुक्रः प्राह विदांवरः}
{एनं सञ्जीवयिष्यामि भार्गवं शङ्करप्रियम्}% ६९

\twolineshloka
{किमर्थं म्रियते ब्रह्मन्पश्य मे तपसो बलम्}
{इति वादिनि विप्रेन्द्रे गौतमोऽपि ममार ह}% ७०

\twolineshloka
{तस्मिन्मृतेऽथ शुक्रोऽपि प्राणांस्तत्याज योगतः}
{तस्यैवं हतिमाज्ञाय प्रह्लादाद्या दितीश्वराः}% ७१

\twolineshloka
{देवा नृपा द्विजाः सर्वे मृता आसंस्तदद्भुतम्}
{मृतमासीदथ बलं तस्य बाणस्य धीमतः}% ७२

\twolineshloka
{अहल्याशोकसन्तप्ता रुरोदोच्चैः पुनः पुनः}
{गौतमेन महेशस्य पूजया पूजितो विभुः}% ७३

\twolineshloka
{वीरभद्रो महायोगी सर्वं दृष्ट्वा चुकोप ह}
{अहो कष्टमहो कष्टं महेशा बहवो हताः}% ७४

\twolineshloka
{शिवं विज्ञापयिष्यामि तेनोक्तं करवाण्यथ}
{इति निश्चित्य गतवान्मन्दराचलमव्ययम्}% ७५

\twolineshloka
{नमस्कृत्वा विरूपाक्षं वृत्तं सर्वमथोक्तवान्}
{ब्रह्माणं च हरिं तत्र स्थितौ प्राह शिवो वचः}% ७६

\twolineshloka
{मद्भक्तैः साहसं कर्म कृतं ज्ञात्वा वरप्रदम्}
{गत्वा पश्यामि हे विष्णो सर्वं तत्कृतसाहसम्}% ७७

\twolineshloka
{इत्युक्त्वा वृषमारुह्य वायुना धूतचामरः}
{नन्दिकेन सुवेषेण धृते छत्रेऽतिशोभने}% ७८

\twolineshloka
{सुश्वेते हेमदण्डे च नान्ययोग्ये धृते विभो}
{महेशानुमतिं लब्ध्वा हरिर्नागान्तके स्थितः}% ७९

\twolineshloka
{आरक्तनीलच्छत्राभ्यां शुशुभे लक्ष्मकौस्तुभः}
{शिवानुमत्या ब्रह्मापि हंसारूढोऽभवत्तदा}% ८०

\twolineshloka
{इन्द्र गोपप्रभाकारच्छत्राभ्यां शुशुभे विधिः}
{इन्द्रा दिसर्वदेवाश्च स्वस्ववाहनसंयुताः}% ८१

\twolineshloka
{अथ ते निर्ययुः सर्वे नानावाद्यानुमोदिताः}
{कोटिकोटिगणाकीर्णा गौतमस्याश्रमं गताः}% ८२

\twolineshloka
{ब्रह्मविष्णुमहेशाना दृष्ट्वा तत्परमाद्भुतम्}
{स्वभक्तं जीवयामास वामकोणनिरीक्षणात्}% ८३

\twolineshloka
{शङ्करो गौतमं प्राह तुष्टोऽहं ते वरं वृणु}
{तदाकर्ण्य वचस्तस्य गौतमः प्राह सादरम्}% ८४

\twolineshloka
{यदि प्रसन्नो देवेश यदि देयो वरो मम}
{त्वल्लिङ्गार्चनसामर्थ्यं नित्यमस्तु ममेश्वर}% ८५

\twolineshloka
{वृतमेतन्मया देव त्रिनेत्र शृणु चापरम्}
{शिष्योऽय मे महाभागो हेयादेयादिवर्जितः}% ८६

\twolineshloka
{प्रेक्षणीयं ममत्वेन न च पश्यति चक्षुषा}
{न घ्राणग्राह्यं देवेश न पातव्यं न चेतरम्}% ८७

\twolineshloka
{इति बुद्ध्या तथा कुर्वन्स हि योगी महायशाः}
{उन्मत्तविकृताकारः शङ्करात्मेति कीर्तितः}% ८८

\twolineshloka
{न कश्चित्तं प्रति द्वेषी न च तं हिंसयेदपि}
{एतन्मे दीयतां देव मृतानाममृतिस्तथा}% ८९

\twolineshloka
{तच्छ्रुत्वोमापतिः प्रीतो निरीक्ष्य हरिमव्ययः}
{स्वांशेन वायुना देहमाविशज्जगदीश्वरः}% ९०

\twolineshloka
{हरिरूपः शङ्करात्मा मारुतिः कपिसत्तमः}
{पर्यायैरुच्यतेऽधीशः साक्षाद्विष्णुः शिवः परः}% ९१

\twolineshloka
{आकल्पमेष प्रत्येकं कामरूपमुपाश्रितः}
{ममाज्ञाकारको रामभक्तः पूजितविग्रहः}% ९२

\twolineshloka
{अनन्तकल्पमीशानः स्थास्यति प्रीतमानसः}
{त्वया कृतमिदं वेश्म विस्तृतं सुप्रतिष्ठितम्}% ९३

\twolineshloka
{नित्यं वै सर्वरूपेण तिष्ठामः क्षणमादरात्}
{समर्चिताः प्रयास्यामः स्वस्ववासं ततः परम्}% ९४

\twolineshloka
{अथाबभाषे विश्वेशं गौतमो मुनिपुङ्गवः}
{अयोग्यं प्रार्थयामीश ह्यर्थी दोषं न पश्यति}% ९५

\twolineshloka
{ब्रह्माद्यलभ्यं देवेश दीयतां यदि रोचते}
{अथेशो विष्णुमालोक्य गृहीत्वा तत्करं करे}% ९६

\twolineshloka
{प्रहसन्नम्बुजाभाक्षमित्युवाच सदाशिवः}
{क्षामोदरोऽसि गोविन्द देयं ते भोजनं किमु}% ९७

\twolineshloka
{स्वयं प्रविश्य यदि वा स्वयं भुङ्क्ष्व स्वगेहवत्}
{गच्छ वा पार्वतीगेहं या कुक्षिं पूरयिष्यति}% ९८

\twolineshloka
{इत्युक्त्वा तत्करालम्बी ह्येकान्तमगमद्विभुः}
{आदिश्य नन्दिनं देवो द्वाराध्यक्षं यथोक्तवत्}% ९९

\twolineshloka
{स गत्वा गौतमं वाथ ह्युक्तवान्विष्णुभाषणम्}
{सम्पादयान्नं देवेशा भोक्तुकामा वयं मुने}% १००

\twolineshloka
{इत्युक्त्वैकान्तमगमद्वासुदेवेन शङ्करः}
{मृदुशय्यां समारुह्य शयितौ देवतोत्तमौ}% १०१

\twolineshloka
{अन्योन्यं भाषणं कृत्वा प्रोत्तस्थतुरुभावपि}
{गत्वा तडागं गम्भीरं स्नास्यन्तौ देवसत्तमौ}% १०२

\twolineshloka
{कराम्बुपातमन्योन्यं पृथक्कृत्वोभयत्र च}
{मुनयो राक्षसाश्चैव जलक्रीडां प्रचक्रिरे}% १०३

\twolineshloka
{अथ विष्णुर्महेशश्च जलपानानि शीघ्रतः}
{चक्रतुः शङ्करः पद्मकिञ्जल्काञ्जलिना हरेः}% १०४

\twolineshloka
{अवाकिरन्मुखे तस्य पद्मोत्फुल्लविलोचने}
{नेत्रे केशरसम्पातात्प्रमीलयत केशवः}% १०५

\twolineshloka
{अत्रान्तरे हरेः स्कन्धमारुरोह महेश्वरः}
{हर्युत्तमाङ्गं बाहुभ्यां गृहीत्वा सन्न्यमज्जयत्}% १०६

\twolineshloka
{उन्मज्जयित्वा च पुनः पुनश्चापि पुनः पुनः}
{पीडितं स हरिः सूक्ष्मं पातयामास शङ्करम्}% १०७

\twolineshloka
{अथ पादौ गृहीत्वा तं भ्रामयन्विचकर्ष ह}
{अताडयद्धरेर्वक्षः पातयामास चाच्युतम्}% १०८

\twolineshloka
{अथोत्थितो हरिस्तोयमादायाञ्जलिना ततः}
{शीर्षे चैवाकिरच्छम्भुमथ शम्भुरथो हरेः}% १०९

\twolineshloka
{जलक्रीडैवमभवदथ चर्षिगणान्तरे}
{जलक्रीडासम्भ्रमेण विस्रस्तजटबन्धनाः}% ११०

\twolineshloka
{अथ सम्भ्रमतां तेषामन्योन्यजटबन्धनम्}
{इतरेतरबद्धासु जटासु च मुनीश्वराः}% १११

\twolineshloka
{शक्तिमन्तोऽशक्तिमत आकर्षन्ति च सव्यथम्}
{पातयन्तोऽन्यतश्चापि क्रोशन्तो रुदतस्तथा}% ११२

\twolineshloka
{एवं प्रवृत्ते तुमुले सम्भूते तोयकर्मणि}
{आकाशे वानरेशस्तु ननर्त च ननाद च}% ११३

\twolineshloka
{विपञ्चीं वादयन्वाद्यं ललितां गीतिमुज्जगौ}
{सुगीत्या ललिता यास्तु अगायत विधा दश}% ११४

\twolineshloka
{शुश्राव गीतिं मधुरां शङ्करो लोकभावतः}
{स्वयं गातुं हि ललितं मन्दं मन्दं प्रचक्रमे}% ११५

\twolineshloka
{स्वयं गायति देवेशे विश्रामं गलदेशिकम्}
{स्वरं ध्रुवं समादाय सर्वलक्षणसंयुतम्}% ११६

\twolineshloka
{स्वधारामृतसंयुक्तं गानेनैवमपोनयन्}
{वासुदेवो मर्दलं च कराभ्यामप्यवादयत्}% ११७

\twolineshloka
{अम्बुजाङ्गश्चतुर्वक्त्रस्तुम्बुरुर्मुखरो बभौ}
{तानका गौतमाद्यास्तु गायको वायुजोऽभवत्}% ११८

\twolineshloka
{गायके मधुरं गीतं हनूमति कपीश्वरे}
{म्लानमम्लानमभवत्कृशाः पुष्टास्तदाभवन्}% ११९

\twolineshloka
{स्वां स्वां गीतिमतः सर्वे तिरस्कृत्यैव मूर्च्छिताः}
{तूष्णीभूतं समभवद्देवर्षिगणदानवम्}% १२०

\threelineshloka
{एकः स हनुमान् गाता श्रोतारः सर्व एव ते}
{मध्याह्नकाले वितते गायमाने हनूमति}
{स्वस्ववाहनमारुह्य निर्गताः सर्वदेवताः}% १२१

\twolineshloka
{गानप्रियो महेशस्तु जग्राह प्लवगेश्वरम्}
{प्लवग त्वं मयाज्ञप्तो निःशङ्को वृषमारुह}% १२२

\twolineshloka
{मम चाभिमुखो भूत्वा गायस्वानेकगायनम्}
{अथाह कपिशार्दूलो भगवन्तं महेश्वरम्}% १२३

\twolineshloka
{वृषभारोहसामर्थ्यं तव नान्यस्य विद्यते}
{तव वाहनमारुह्य पातकी स्यामहं विभो}% १२४

\twolineshloka
{मामेवारुह देवेश विहङ्गः शिवधारणः}
{तव चाभिमुखं गानं करिष्यामि विलोकय}% १२५

\twolineshloka
{अथेश्वरो हनूमन्तमारुरोह यथा बृषम्}
{आरूढे शङ्करे देवे हनुमत्कन्धरां शिवः}% १२६

\twolineshloka
{छित्वा त्वचं परावृत्य सुखं गायति पूर्ववत्}
{शृण्वन्गीतिसुधां शम्भुर्गौतमस्य गृहं ततः}% १२७

\twolineshloka
{सर्वे चाप्यागतास्तत्र देवर्षिगणदानवाः}
{पूजिता गौतमेनाथ भोजनावसरे सति}% १२८

\twolineshloka
{यच्छुष्कं दारुसम्भूतं गृहोपकरणादिकम्}

\twolineshloka
{प्ररूढमभवत्सर्वं गायमाने हनूमति}% १२९}
{तस्मिन्गाने समस्तानां चित्रदृष्टिरतिष्ठत}% १३०

\twolineshloka
{द्विबाहुरीशस्य पदाभिवन्दनः समस्तगात्राभरणोपपन्नः}
{प्रसन्नमूर्तिस्तरुणः सुमध्ये विन्यस्तमूर्द्धाञ्जलिभिः शिरोभिः}% १३१

\twolineshloka
{शिरः कराभ्यां परिगृह्य शङ्करो हनुमतं पूर्वमुखं चकार}
{पद्मासनासीनहनूमतोऽञ्जलौ निधाय पादं त्वपरं मुखे च}% १३२

\twolineshloka
{पादाङ्गुलीभ्यामथ नासिकां विभुः स्नेहेन जग्राह च मन्दमन्दम्}
{स्कन्धे मुखे त्वंसतले च कण्ठे वक्षस्थले च स्तनमध्यमे हृदि}% १३३

\twolineshloka
{ततश्च कुक्षावथ नाभिमण्डले पादं द्वितीयं विदधाति चाञ्जलौ}
{शिरो गृहीत्वाऽवनमय्य शङ्करः पस्पर्श पृष्ठं चिबुकेन सोऽध्वनि}% १३४}

\onelineshloka
{हारं च मुक्तापरिकल्पितं शिवो हनूमतः कण्ठगतं चकार}% १३५

\twolineshloka
{अथ विष्णुर्महेशानमिदं वचनमुक्तवान्}
{हनूमता समो नास्ति कृत्स्नब्रह्माण्डमण्डले}% १३६

\twolineshloka
{श्रुतिदेवाद्यगम्यं हि पदं तव कपिस्थितम्}
{सर्वोपनिषदव्यक्तं त्वत्पदं कपिसर्वयुक्}% १३७

\twolineshloka
{यमादिसाधनैर्योगैर्न क्षणं ते पदं स्थिरम्}
{महायोगिहृदम्भोजे परं स्वस्थं हनूमति}% १३८

\twolineshloka
{वर्षकोटिसहस्रं तु सहस्राब्दैरथान्वहम्}
{भक्त्या सम्पूज्!ऽपीश पादो नो दर्शितस्त्वया}% १३९

\twolineshloka
{लोके वादो हि सुमहाञ्छम्भुर्नारायणप्रियः}
{हरिप्रियस्तथा शम्भुर्न तादृग्भाग्यमस्ति मे}% १४०

\twolineshloka
{तच्छ्रुत्वा वचनं शम्भुर्विष्णोः प्राह मुदान्वितः}
{न त्वया सदृशो मह्यं प्रियोऽन्योऽस्ति हरे क्वचित्}% १४१

\twolineshloka
{पार्वती वा त्वया तुल्या वर्तते नैव भिद्यते}
{अथ देवाय महते गौतमः प्रणिपत्य च}% १४२

\twolineshloka
{व्यजिज्ञपदमेयात्मन्देवैर्हि करुणानिधे}
{मध्याह्नोऽय व्यतिक्रान्तो भुक्तिवेलाखिलस्य च}% १४३

\twolineshloka
{अथाचम्य महादेवो विष्णुना सहितो विभुः}
{प्रविश्य गौतमगृहं भोजनायोपचक्रमे}% १४४

\twolineshloka
{रत्नाङ्गुलीयैरथ नूपुराभ्यां दुकूलबन्धेन तडित्सुकाञ्च्या}
{हारैरनेकैरथ कण्ठनिष्कयज्ञोपवीतोत्तरवाससी च}% १४५

\twolineshloka
{विलम्बिचञ्चन्मणिकुडण्लेन सुपुष्पधम्मिल्लवरेण चैव}
{पञ्चागगन्धस्य विलेपनेन बाह्वङ्गदैः कङ्कणकाङ्गुलीयैः}% १४६

\twolineshloka
{अथो विभूषितः शिवो निविष्ट उत्तमासने}
{स्वसम्मुखं हरिं तथा न्यवेशयद्वरासने}% १४७

\twolineshloka
{देव श्रेष्ठौ हरीशौ तावन्योन्याभिमुखस्थितौ}
{सुवर्णभाजनस्थान्नं ददौ भक्त्या स गौतमः}% १४८

\twolineshloka
{त्रिंशत्प्रभेदान्भक्ष्यांस्तु पायसं च चतुर्विधम्}
{सुपक्वं पाकजातं च कल्पितं यच्छतद्वयम्}% १४९

\twolineshloka
{अपक्वं मिश्रकं तद्वत्त्रिंशतं परिकल्पितम्}
{शतं शतं सुकन्दानां शाकानां च प्रकल्पितम्}% १५०

\twolineshloka
{पञ्चविंशतिधा सर्पिःसंस्कृतं व्यञ्जनं तथा}
{शर्कराद्यं तथा चूतमोचाखर्जूरदाडिमम्}% १५१

\twolineshloka
{द्रा क्षेक्षुनागरङ्गं च मिष्टं पक्वं फलोत्करम्}
{प्रियालकं जम्बुफलं विकङ्कतफलं तथा}% १५२

\twolineshloka
{एवमादीनि चान्यानि द्र व्याणीशे समर्प्य च}
{दत्त्वापोशानकं विप्रो भुञ्जध्वमिति चाब्रवीत्}% १५३

\twolineshloka
{भुञ्जानेषु च सर्वेषु व्यजनं सूक्ष्मविस्तृतम्}
{गौतमः स्वयमादाय शिवविष्णू अवीजयत्}% १५४

\twolineshloka
{परिहासमथो कर्तुमियेष परमेश्वरः}
{पश्य विष्णो हनूमन्तं कथं भुङ्क्ते स वानरः}% १५५

\twolineshloka
{वानरं पश्यति हरौ मण्डकं विष्णुभाजने}
{चिक्षेप मुनिसङ्घेषु पश्यत्स्वपि महेश्वरः}% १५६

\twolineshloka
{हनूमते दत्तवांश्च स्वोच्छिष्टं पायसादिकम्}
{त्वदुच्छिष्टमभोज्यं तु तवैव वचनाद्विभो}% १५७

\twolineshloka
{अनर्हं मम नैवेद्यं पत्रं पुष्पं फलादिकम्}
{मह्यं निवेद्यसकलं कूप एव विनिःक्षिपेत्}% १५८

\twolineshloka
{अभुक्ते त्वद्वचो नूनं भुक्ते चापि कृपा तव}
{बाणलिङ्गे स्वयम्भूते चन्द्र कान्ते हृदि स्थिते}% १५९

\twolineshloka
{चान्द्रा यणसमं ज्ञेयं शम्भोर्नैवैद्यभक्षणम्}
{भुक्तिवेलेयमधुना तद्वैरस्यं कथान्तरात्}% १६०

\twolineshloka
{भुक्त्वा तु कथयिष्यामि निर्विशङ्कं विभुङ्क्ष्व तत्}
{अथासौ जलसंस्कारं कृतवान् गौतमो मुनिः}% १६१

\twolineshloka
{आरक्तसुस्निग्धसुसूक्ष्मगात्राननेकधाधौतसुशोभिताङ्गान्}
{तडागतोयैः कतबीजघर्षितैर्विशौधितैस्तैः करकानपूरयत्}% १६२

\fourlineindentedshloka*
{नद्याः सैकतवेदिकां नवतरां सञ्छाद्य सूक्ष्माम्बरैः}
{शुद्धैः श्वेततरैरथोपरि घटांस्तोयेन पूर्णान्क्षिपेत्}
{लिप्त्वा नालकजातिमास्तपुटकं तत्कौलकं कारिका}
{चूर्णं चन्दनचन्द्र रश्मिविशदां मालां पुटान्तं क्षिपेत्}

\twolineshloka
{यामस्यापि पुनश्च वारिवसनेनाशोध्य कुम्भेन तच्चन्द्र ग्रन्थिमथो-}
{निधाय बकुलं क्षिप्त्वा तथा पाटलम्}% १६३
शेफालीस्तबकमथो जलं च तत्रै
विन्यस्य प्रथमत एव तोयशुद्धिम्

\twolineshloka
{कृत्वाथो मृदुतरसूक्ष्मवस्त्रखण्डे-}
{नावेष्टेत्सृणिकमुखं च सूक्ष्मचन्द्र म्}% १६४

\twolineshloka
{अनातपप्रदेशे तु निधाय करकानथ}
{मन्दवातसमोपेते सूक्ष्मव्यजनवीजिते}% १६५

\twolineshloka
{सिञ्चेच्छीतैर्जलैश्चापि वासितैः सृणिकामपि}
{संस्कृताः स्वायतास्तत्र नरा नार्योऽथवा नृपाः}% १६६

\twolineshloka
{तत्कन्या वा क्षालिताङ्गा धौतपादास्सुवाससः}
{मधुपिङ्गमनिर्यासमसान्द्र मगुरूद्भवम्}% १६७

\twolineshloka
{बाहुमूले च कण्ठे च विलिप्यासान्द्र मेव च}
{मस्तकेजापकं न्यस्य पञ्चगन्धविलेपनम्}% १६८

\twolineshloka
{पुष्पनद्धसुकेशास्तु ताः शुभाः स्युः सुनिर्मलाः}
{एवमेवार्चिता नार्य आप्तकुङ्कुमविग्रहाः}% १६९

\twolineshloka
{युवत्यश्चारुसर्वाङ्ग्यो नितरां भूषणैरपि}
{एतादृग्वनिताभिर्वा नरैर्वा दापयेज्जलम्}% १७०

\twolineshloka
{तेऽपि प्रादानसमये सूक्ष्मवस्त्राल्पवेष्टनम्}
{अथ वामकरे न्यस्य करकं प्रेक्ष्य तत्र हि}% १७१

\twolineshloka
{दोरिकान्यस्तमुन्मुच्य ततस्तोयं प्रदापयेत्}
{एवं स कारयामास गौतमो भगवान्मुनिः}% १७२

\twolineshloka
{महेशादिषु सर्वेषु भुक्तवत्सु महात्मसु}
{प्रक्षालिताङिघ्रहस्तेषु गन्धोद्वर्तितपाणिषु}% १७३

\twolineshloka
{उच्चासनसमासीने देवदेवे महेश्वरे}
{अथ नीचसमासीना देवाः सर्षिगणास्तथा}% १७४

\twolineshloka
{मणिपात्रेषु संवेष्ट्य पूगखण्डान्सुधूपितान्}
{अकोणान्वर्तुलान्स्थूलानसूक्ष्मानकृशानपि}% १७५

\twolineshloka
{श्वेतपत्राणि संशोध्य क्षिप्त्वा कर्पूरखण्डकम्}
{चूर्णं च शङ्करायाथ निवेदयति गौतमे}% १७६

\twolineshloka
{गृहाण देव ताम्बूलमित्युक्तवचने मुनौ}
{कपे गृहाण ताम्बूलं प्रयच्छ मम खण्डकान्}% १७७

\twolineshloka
{उवाच वानरो नास्ति मम शुद्धिर्महेश्वर}
{अनेकफलभोक्तृत्वाद्वानरस्तु कथं शुचिः}% १७८

\twolineshloka
{तच्छ्रुत्वा तु विरूपाक्षः प्राह वानरसत्तमम्}
{मद्वाक्यादखिलं शुद्ध्य्न्मेद्वाक्यादमृतं विषम्}% १७९

\twolineshloka
{मद्वाक्यादखिला वेदा मद्वाक्याद्देवतादयः}
{मद्वाक्याद्धर्मविज्ञानं मद्वाक्यान्मोक्ष उच्यते}% १८०

\twolineshloka
{पुराणान्यागमाश्चैव स्मृतयो मम वाक्यतः}
{अतो गृहाण ताम्बूलं मम देहि सुखण्डकान्}% १८१

\twolineshloka
{हरिर्वामकरेणाधात्ताम्बूलं पूगखण्डकम्}
{ततः पत्राणि सङ्गृह्य तस्मै खण्डान्समर्पयत्}% १८२

\twolineshloka
{कर्पूरमग्रतो दत्तं गृहीत्वाभक्षयच्छिवः}
{देवे तु कृतताम्बूले पार्वती मन्दराचलात्}% १८३

\twolineshloka
{जयाविजययोर्हस्तं गृहीत्वायान्मुनेर्गृहम्}
{देवपादौ ततो नत्वा विनम्रवदनाभवत्}% १८४

\twolineshloka
{उन्नमय्यमुखे तस्या इदमाह त्रिलोचनः}
{त्वदर्थं देवदेवेशि अपराधः कृतो मया}% १८५

\twolineshloka
{यत्त्वां विहाय भुक्तं हि तथान्यच्छृणु सुन्दरी}
{यत्त्वां स्वमन्दिरे त्यक्त्वा महदेनो मया कृतम्}% १८६

\twolineshloka
{क्षन्तुमर्हसि देवेशि त्यक्तकोपा विलोकय}
{न बभाषेऽप्येवमुक्ता सारुन्धत्या विनिर्ययौ}% १८७

\twolineshloka
{निर्गच्छन्तीं मुनिर्ज्ञात्वा दण्डवत्प्रणनाम ह}
{अथोवाच शिवा तं च गौतम त्वं किमिच्छसि}% १८८

\twolineshloka
{अथाह गौतमो देवीं पार्वतीं प्रेक्ष्य संस्मिताम्}
{कृतकृत्यो भवेयं वै भुक्तायां मद्गृहे त्वयि}% १८९

\twolineshloka
{ततः प्राह शिवा विप्रं गौतमं रचिताञ्जलिम्}
{भोक्ष्यामि त्वद्गृहे विप्र शङ्करानुमतेन वै}% १९०

\twolineshloka
{अथ गत्वा शिवं विंशे लब्धानुज्ञस्त्वरागतः}
{भोजयामास गिरिजां देवीं चारुन्धतीं तथा}% १९१

\twolineshloka
{भुक्त्वाथ पार्वती सर्वगन्धपुष्पाद्यलङ्कृता}
{सहानुचरकन्याभिः सहस्राभिर्हरं ययौ}% १९२

\twolineshloka
{अथाह शङ्करो देवीं गच्छ गौतममन्दिरम्}
{सन्ध्योपास्तिमहं कृत्वा ह्यागमिष्ये तवान्तिकम्}% १९३

\twolineshloka
{इत्युक्त्वा प्रययौ देवी गौतमस्यैव मन्दिरम्}
{सन्ध्यावन्दनकामास्तु सर्व एव विनिर्गताः}% १९४

\twolineshloka
{कृतसन्ध्यास्तडागे तु महेशाद्याश्च कृत्स्नशः}
{अथोत्तरमुखः शम्भुर्न्यास कृत्वा जजाप ह}% १९५

\twolineshloka
{अथ विष्णुर्महातेजा महेशमिदमब्रवीत्}
{सर्वैर्नमस्यते यस्तु सर्वैरेव समर्च्यते}% १९६

\twolineshloka
{हूयते सर्वयज्ञेषु स भवान्किं जपिष्यति}
{रचिताञ्जलयः सर्वे त्वामेवैकमुपासते}% १९७

\twolineshloka
{स भवान्देवदेवेशः कस्मै विरचिताञ्जलि}
{नमस्कारादिपुण्यानां फलदस्त्वं महेश्वर}% १९८

\twolineshloka
{तव कः फलदो वन्द्यः को वा त्वत्तोऽधिको वद}
{तच्छ्रुत्वा शङ्करः प्राह देवदेवं जनार्दनम्}% १९९

\twolineshloka
{ध्याये न किञ्चिद्गोविन्द नमस्ये ह न किञ्चन}
{किन्तु नास्तिकजन्तूनां प्रवृत्त्यर्थमिदं मया}% २००

\twolineshloka
{दर्शनीयं हरे चैतदन्यथा पापकारिणः}
{तस्माल्लोकोपकारार्थमिदं सर्वं कृतं मया}% २०१

\twolineshloka
{ओमित्युक्त्वा हरिरथ तं नत्वा समतिष्ठत}
{अथ ते गौतमगृहं प्राप्ता देवर्षयस्तदा}% २०२

\twolineshloka
{सर्वे पूजामथो चक्रुर्देवदेवाय शूलिने}
{देवो हनूमता सार्द्धं गायन्नास्ते मुनीश्वर}% २०३

\twolineshloka
{पञ्चाक्षरीं महाविद्यां सर्व एव तदाऽजपन्}
{हनुमत्करमालम्ब्य देवाभ्यां सङ्गतो हरः}% २०४

\twolineshloka
{एकशय्यासमासीनौ तावुभौ देवदम्पती}
{गायन्नास्ते च हनुमांस्तुम्बुरुप्रमुखास्तथा}% २०५

\twolineshloka
{नानाविधविलासांश्च चकार परमेश्वरः}
{आहूय पार्वतीमीश इदं वाक्यमुवाच ह}% २०६

\twolineshloka
{रचयिष्यामि धम्मिल्लमेहि मत्पुरतः शुभे}
{देव्याह न च युक्तं तद्भर्त्रा शुश्रूषणं स्त्रियः}% २०७

\twolineshloka
{केशप्रसाधनकृतावनर्थान्तरमापतेत्}
{केशप्रसाधने देव तत्त्वं सर्वं न चेप्सितम्}% २०८

\twolineshloka
{अथ बन्धेकृते पश्चादंसप्रान्तप्रमार्जनम्}
{ततश्चरमसंलग्नकेशपुष्पादिमार्जनम्}% २०९

\twolineshloka
{एतस्मिन्वर्तमाने तु महात्मानो यदागमन्}
{तदा किमुत्तरं वाच्यं तव देवादिवन्दित}% २१०

\twolineshloka
{नायान्ति चेदथ विभो भीतिर्नाशमुपैष्यति}
{एवं हि भाषमाणां तां करेणाकृष्य शङ्करः}% २११

\twolineshloka
{स्वोर्वोः संस्थापयित्वैव विस्रस्य कचबन्धनम्}
{विभज्य च कराभ्यां स प्रससार नखैरपि}% २१२

\twolineshloka
{विष्णुदत्तां पारिजातस्रजं कचगतामपि}
{कृत्वा धम्मिल्लमकरोदथ मालां करागताम्}% २१३

\twolineshloka
{मल्लिकास्रजमादाय बबन्ध कचबन्धने}
{कल्पप्रसूनमालां च ब्रह्मदत्तां महेश्वरः}% २१४

\twolineshloka
{पार्वतीवसने गूढगन्धाढ्ये च समाददात्}
{अथांसपृष्टसंलग्नमार्जनं कृतवान् विभुः}% २१५

\twolineshloka
{श्लथन्नीवेरधो देव्या वस्त्रवेष्टादधोगतः}
{किमिदं देवि चेत्युक्त्वा नीवीबन्धं चकार ह}% २१६

\twolineshloka
{नासा भूषणमेतत्ते सत्यमेव वदामि ते}
{ततः प्राह शिवा शम्भुं स्मित्वा पर्वतनन्दिनी}% २१७

\twolineshloka
{अहो त्वन्मन्दिरे शम्भो सर्ववस्तु समृद्धिमत्}
{पूर्वमेव मया सर्वं ज्ञातप्रायमभूत्किल}% २१८

\twolineshloka
{सर्वद्र विणसम्पत्तिर्भूषणैरवगम्यते}
{शिरो विभूषितं देव ब्रह्मशीर्षस्य मालया}% २१९

\twolineshloka
{नरकस्य तथा माला वक्षस्थलविभूषणम्}
{शेषश्च वासुकिश्चैव सविषौ तव कङ्कणौ}% २२०

\twolineshloka
{दिशोऽम्बरं जटाः केशा भसितं चाङ्गरागकम्}
{महोक्षो वाहनं गोत्रं कुलं चाज्ञातमेव च}% २२१

\twolineshloka
{ज्ञायेते पितरौ नैव विरूपाक्षं तथा वपुः}
{एवं वदन्तीं गिरिजां विष्णुः प्राहातिकोपनः}% २२२

\twolineshloka
{किमर्थं निन्दसे देवि देवदेवं जगत्पतिम्}
{दुष्प्राणा न प्रिया भद्रे तव नूनमसंयमात्}% २२३

\twolineshloka
{यत्रेशनिन्दनं भद्रे तत्र नो मरणव्रतम्}
{इत्युक्त्वाथ नखाभ्यां हि हरिश्छेत्तुं शिरो गतः}% २२४

\twolineshloka
{महेशस्तत्करं गृह्य प्राह मा साहसं कृथाः}
{पार्वतीवचनं सर्वं प्रियं मम न चाप्रियम्}% २२५

\twolineshloka
{ममाप्रियं हृषीकेश कर्तुं यत्किञ्चिदिष्यते}
{ओमित्युक्त्वाथ भगवांस्तूष्णीम्भूतोऽभवद्धरिः}% २२६

\twolineshloka
{हनुमानथ देवाय व्यज्ञापयदिदं वचः}
{अर्थयामि विनिष्क्रामं मम पूजाव्रतं तथा}% २२७

\twolineshloka
{पूजार्थमप्यहं गच्छे मामनुज्ञातुमर्हसि}
{तच्छ्रुत्वा शङ्करो देवः स्मित्वा प्राह कपीश्वरम्}% २२८

\twolineshloka
{कस्य पूजा क्व वा पूजा किं पुष्पं किं दलं वद}
{को गुरुः कश्च मन्त्रस्ते कीदृशं पूजनं तथा}% २२९

\twolineshloka
{एवं वदति दैवेशे हनुमान्नीतिसंयुतः}
{वेपमानसमस्ताङ्गः स्तोतुमेवं प्रचक्रमे}% २३०

\twolineshloka
{नमो देवाय महते शङ्करायामितात्मने}
{योगिने योगधात्रे च योगिनां गुरवे नमः}% २३१

\twolineshloka
{योगगम्याय देवाय ज्ञानिनां पतये नमः}
{वेदानां पतये तुभ्यं देवानां पतये नमः}% २३२

\twolineshloka
{ध्यानाय ध्यानगम्याय ध्यातॄणां गुरवे नमः}
{अष्टमूर्ते नमस्तुभ्यं पशूनां पतये नमः}% २३३

\twolineshloka
{अम्बकाय त्रिनेत्राय सोमसूर्याग्निचक्षुषे}
{सुभृङ्गराजधत्तूरद्रो णपुष्पप्रियस्य ते}% २३४

\twolineshloka
{बृहतीपूगपुन्नागचम्पकादिप्रियाय च}
{नमस्तेऽस्तु नमस्तेऽस्तु भूय एव नमो नमः}% २३५

\twolineshloka
{शिवो हरिमथ प्राह मा भैषीर्वद मेऽखिलम्}
{ततस्त्यक्त्वा भयं प्राह हनुमान् वाक्यकोविदः}% २३६

\twolineshloka
{शिवलिङ्गार्चनं कार्यं भस्मोद्धूलितदेहिना}
{दिवा सम्पादितैस्तोयैः पुष्पाद्यैरपि तादृशैः}% २३७

\twolineshloka
{देव विज्ञापयिष्यामि शिवपूजाविधिं शुभम्}
{सायङ्काले तु सम्प्राप्ते आशिरःस्नानमाचरेत्}% २३८

\twolineshloka
{क्षालितं वसनं शुष्कं धृत्वाचम्य त्रिरन्यधीः}
{अथ भस्म समादाय आग्नेयं स्नानमाचरेत्}% २३९

\twolineshloka
{प्रणवेन समामम्त्र्य अष्टवारमथापि वा}
{पञ्चाक्षरेण मन्त्रेण नाम्ना वा येन केनचित्}% २४०

\twolineshloka
{सप्ताभिमन्त्रितं भस्म दर्भपाणिः समाहरेत्}
{ईशानं सर्वविद्यानामुक्त्वा शिरसि पातयेत्}% २४१

\twolineshloka
{तत्पुरुषाय विद्महे मुखे भस्म प्रसेचयेत्}
{अघोरेभ्योऽथ घोरेभ्यो भस्म वक्षसि निक्षिपेत्}% २४२

\twolineshloka
{वामदेवाय नमः इति गुह्यस्थाने विनिक्षिपेत्}
{सद्योजातं प्रपद्यामि निक्षिपेदथ पादयोः}% २४३

\twolineshloka
{उद्धूलयेत्समस्ताङ्गं प्रणवेन विचक्षणः}
{त्रैवर्णिकानामुदितः स्नानादिविधिरुत्तमः}% २४४

\twolineshloka
{शूद्रा दीनां प्रवक्ष्यामि यदुक्तं गुरुणा तथा}
{शिवेति पदमुच्चार्य भस्म सम्मन्त्रयेत्सुधीः}% २४५

\twolineshloka
{सप्त वारमथादाय शिवायेति शिरस्यथ}
{शङ्कराय मुखे प्रोक्तं सर्वज्ञाय हृदि क्षिपेत्}% २४६

\twolineshloka
{स्थाणवे नम इत्युक्त्वा मुखे चापि स्वयम्भुवे}
{उच्चार्य पादयोः क्षिप्त्वा भस्म शुद्धमतः परम्}% २४७

\twolineshloka
{नमः शिवायेत्युच्चार्य सर्वाङ्गोद्धूलनं स्मृतम्}
{प्रक्षाल्य हस्तावाचम्य दर्भपाणिः समाहितः}% २४८

\twolineshloka
{दर्भाभावे सुवर्णं स्यात्तदभावे गवालुकाः}
{तदभावे तु दूर्वाः स्युस्तदभावे तु राजतम्}% २४९

\twolineshloka
{सन्ध्योपास्तिं जपं देव्याः कृत्वा देवगृहं व्रजेत्}
{देववेदीमथो वापि कल्पितं स्थण्डिलं तु वा}% २५०

\twolineshloka
{मृण्मयं कल्पितं शुद्धं पद्मादिरचनायुतम्}
{चातुर्वर्णकरङ्गैश्च श्वेतेनैकेन वा पुनः}% २५१

\twolineshloka
{विचित्राणि च पद्मानि स्वस्तिकादि तथैव च}
{उत्पलादिगदाशङ्खत्रिशूलडमरूंस्तथा}% २५२

\twolineshloka
{शरोक्तपञ्च प्रासादं शिवलिङ्गमथैव च}
{सर्वकामफलं वृक्षं कुलकं कोलकं तथा}% २५३

\twolineshloka
{षट्कोणं च त्रिकोणं च नवकोणमथापि वा}
{कोणे द्वादशकान्दोलापादुकाव्यजनानि च}% २५४

\twolineshloka
{चामरच्छत्र्रयुगलं विष्णुब्रह्मादिकांस्तथा}
{चूर्णैर्विरचयेद्वेद्यां धीमान्देवालयेऽपि वा}% २५५

\twolineshloka
{यत्रापि देवपूजा स्यात् तत्रैव कल्पयेद्बुधः}
{स्वहस्तरचितं मुख्यं क्रीतं चैव तु मध्यमम्}% २५६

\twolineshloka
{याचितं तु कनिष्ठं स्याद्बलात्कारमथोऽधमम्}
{अर्हेषु यत्त्वनर्हेषु बलात्कारात्तु निष्फलम्}% २५७

\twolineshloka
{रक्तशालिजपाशाणकलमासितरक्तकैः}
{तन्दुलैर्वीहिमात्रोत्थैः कणैश्चैव यथाक्रमम्}% २५८

\twolineshloka
{उत्तमैर्मध्यमैश्चैव कनिष्ठैरधमैस्तथा}
{पद्मादिस्थापनैरेव तत्सम्यग्यागमाचरेत्}% २५९

\twolineshloka
{प्रागुत्तरमुखो वापि यदि वा प्राङ्मुखो भवेत्}
{आसनं च प्रवक्ष्यामि यथादृष्टं यथा श्रुतम्}% २६०

\twolineshloka
{कौशं चार्मं चैलतल्पे दारवं तालपत्रकम्}
{काम्बलं काञ्चनं चैव राजतं ताम्रमेव च}% २६१

\twolineshloka
{गोकरीषार्कजैर्वापि ह्यासनं परिकल्पयेत्}
{वैयाघ्रं रौरवं चैव हारिणं मार्गमेव च}% २६२

\twolineshloka
{चार्मं चतुर्विधं ज्ञेयमथ बन्धुकमेव च}
{यथासम्भवमेतेषु ह्यासनं परिकल्पयेत्}% २६३

\twolineshloka
{कृतपद्मासनो वापि स्वस्तिकासन एव च}
{दर्भभस्मसमासीनः प्राणानायम्य वाग्यतः}% २६४

\twolineshloka
{तावत्स देवतारूपो ध्यानं चान्तः समाचरेत्}
{शिखान्ते द्वादशाङ्गुल्ये स्थितं सूक्ष्मतनुं शिवम्}% २६५

\twolineshloka
{अन्तश्चरन्तं भूतेषु गुहायां विश्वतोमुखम्}
{सर्वाभरणसंयुक्तमणिमादिगुणान्वितम्}% २६६

\twolineshloka
{ध्यात्वा तं धारयेच्चिते तद्दीप्त्या पूरयेत्तनुम्}
{तया दीप्त्या शरीरस्थं पापं नाशमुपागतम्}% २६७

\twolineshloka
{स्वर्णपादैरसम्पर्काद्र क्तं श्वेतं यथा भवेत्}
{तद्द्वादशदलावृत्तमष्ट पञ्च त्रिरेव वा}% २६८

\twolineshloka
{परिकल्प्यासनं शुद्धं तत्र लिङ्ग निधाय च}
{गुहास्थितं महेशानं लिङ्गेशं चिन्तयेत्तथा}% २६९

\twolineshloka
{शोधिते कलशे तोयं शोधितं गन्धवासितम्}
{सुगन्धपुष्पं निक्षिप्य प्रणवेनाभिमन्त्रितम्}% २७०

\twolineshloka
{प्राणायामश्च प्रणवः शूद्रे षु न विधीयते}
{प्राणायामपदे ध्यानं शिवेत्यॐकारमन्त्रितम्}% २७१

\twolineshloka
{गन्धपुष्पाक्षतादीनि पूजाद्र व्याणि यानि च}
{तानि स्थाप्य समीपे तु ततः सङ्कल्पमाचरेत्}% २७२

\twolineshloka
{शिवपूजां करिष्यामि शिवतुष्ट्यर्थमेव च}
{इति सङ्कल्पयित्वा तु तत आवाहनादिकम्}% २७३

\twolineshloka
{कृत्वा तु स्नानपर्यन्तं ततः स्नानं प्रकल्पयेत्}
{नमस्तेत्यादिमन्त्रेण शतरुद्र विधानतः}% २७४

\twolineshloka
{अविच्छिन्ना तु या धारा मुक्तिधारेति कीर्तिता}
{तया यः स्नापयेन्मासं जपन् रुद्र मुखांश्च वा}% २७५

\twolineshloka
{एकवारं त्रिवारं च पञ्च सप्त नवापि वा}
{एकादश तथा वारमथैकादशधान्वितम्}% २७६

\twolineshloka
{मुक्तिस्नानमिदं ज्ञेयं मासं मोक्षप्रदायकम्}
{शैवया विद्यया स्नानं केवलं प्रणवेन वा}% २७७

\twolineshloka
{मृण्मयैर्नालिकेरस्य शकलैश्चोर्मिभिस्तथा}
{कांस्येन मुक्ताशुक्त्या च पुष्पादिकेसरेण वा}% २७८

\twolineshloka
{स्नापयेद्देवदेवेशं यथासम्भवमीरितैः}
{शृङ्गस्य च विधिं वक्ष्ये स्नानयोग्यं यथा भवेत्}% २७९

\twolineshloka
{पूर्वमन्तस्तु संशोध्य बहिरन्तस्तु शोधयेत्}
{सुस्निग्धं लघु कृत्वाथ नाङ्गं छिन्द्यात्कथञ्चन}% २८०

\twolineshloka
{नीचैकदेशविन्यस्तद्वारद्रो ण्या सुहृत्तया}
{कृशानुयुक्तं स्नानं तु देवाय परिकल्पयेत्}% २८१

\twolineshloka
{एवं गवयशृङ्गस्य जलपूर्तिरथोच्यते}
{द्वारे निषिद्धलोहार्द्धसन्धिद्वारासमन्विते}% २८२

\twolineshloka
{योगवक्रं नागदण्डं नागाकारं प्रकल्पयेत्}
{फलस्थाने तु चषकं दण्डेन समरन्ध्रकम्}% २८३

\twolineshloka
{तत्रैव पातयेत्तोयं मूर्द्धयन्त्रघटे स्थितम्}
{पातयेदथ चान्येन वामेनैव करेण वा}% २८४

\twolineshloka
{मुक्तिधारा कृता तेन पवित्रं पापनाशनम्}
{एवं संस्नाप्य देवेशं पञ्चगव्यैस्तथैव च}% २८५

\twolineshloka
{पञ्चामृतैरथ स्नाप्य मधुरत्रितयेन च}
{विभूष्य भूषणैर्देवं पुनः स्नाप्यमहेश्वरम्}% २८६

\twolineshloka
{शीतोपचारं कृत्वाथ तत आचमनादिकम्}
{वस्त्रं तथोपवीतं च गन्धद्र व्यकमेव च}% २८७


\twolineshloka*
{कर्पूरमगरुं चापि पाटीरमथवा भवेत्}
{उभयमिश्रितं चापि शिवलिङ्गं प्रपूजयेत्}

\twolineshloka
{कृत्स्नं पीठं गन्धपूर्णं यद्वा विभवसारतः}
{तूष्णीमथोपचारं वा कालीयं पुष्पमेव च}% २८८

\twolineshloka
{श्रीपत्रमरुचित्याज्यं यथाशक्त्यखिलं यथा}
{अनेकद्र व्यधूपं च गुग्गुलं केवलं तथा}% २८९

\twolineshloka
{कपिलाघृतसंयुक्तं सर्वधूपं प्रशस्यते}
{धूपं दत्वा यथाशक्ति कपिलाघृतदीपकान्}% २९०

\twolineshloka
{अथवा पूजामात्रेण दीपान्दत्वोपहारकम्}
{नैवेद्यमुपपन्नं च दत्वा पुष्पसमन्वितम्}% २९१

\twolineshloka
{मुखशुद्धिं ततः कृत्वा दत्त्वा ताम्बूलमादरात्}
{प्रदक्षिणानमस्कारौ पूजैवं हि समाप्यते}% २९२

\twolineshloka
{गीत्यङ्गपञ्चकं पश्चात्तानि विज्ञापयामि ते}
{गीतिर्वाद्यं पुराणं च नृत्यं हासोक्तिरेव च}% २९३

\twolineshloka
{नीराजनं च पुष्पाणामञ्जलिश्चाखिलार्पणम्}
{क्षमापनं चोद्वसनं प्रोक्तं पञ्चोपचारकम्}% २९४

\twolineshloka
{भूषणं च तथा छत्रं चामरव्यजने अपि}
{उपवीतं च कैकर्यं षोडशानुपचारकान्}% २९५

\twolineshloka
{द्वात्रिंशदुपचारैस्तु यः समाराधयेच्छिवम्}
{एकेनाह्ना समस्तानां पातकानां क्षयो भवेत्}% २९६

\twolineshloka
{एतच्छ्रुत्वा हनुमतो वचनं प्राह शङ्करः}
{एवमेतत्कपिश्रेष्ठः यदुक्तं पूजनं मम}% २९७

\twolineshloka
{सारभूतमहं तुभ्यमुपदेक्ष्यामि साम्प्रतम्}
{आराधनं यथा लिङ्गे विस्तरेण त्वयोदितम्}% २९८

\twolineshloka
{मत्पादयुगलं प्रार्च्य पूजाफलमवाप्नुहि}
{ततः प्राह कपिश्रेष्ठो देवदेवमुमापतिम्}% २९९

\twolineshloka
{गुरुणा लिङ्गपूजैव नियता परिकल्पिता}
{तां करोमि पुरा देव पश्चात्त्वत्पादपूजनम्}% ३००

\twolineshloka
{इत्युक्त्वेशं नमस्कृत्य शिवलिङ्गार्चनाय च}
{सरसस्तीरमागत्य कृत्वा सैकतवेदिकाम्}% ३०१

\twolineshloka
{तालपत्रैर्विरचितमासनं पर्यकल्पयत्}
{प्रक्षाल्य पादौ हस्तौ च समाचम्य समाहितः}% ३०२

\twolineshloka
{भस्मस्नानमथो चक्रे पुनराचम्य वाग्यतः}
{देववेद्यामथो चक्रे पद्मं च सुमनोहरम्}% ३०३

\twolineshloka
{अनन्तरं तालपत्रे पद्मासनगतः कपिः}
{प्राणानायम्य सन्न्यस्य शुक्लध्यानसमन्वितः}% ३०४

\twolineshloka
{प्रणम्य गुरुमीशानं जपन्नासीदतः परम्}
{अथ देवार्चनं कर्त्तुं यत्नमास्थितवान्कपिः}% ३०५

\twolineshloka
{पलाशपत्रपुटकद्वयानीतजलं शुचि}
{शिरः कमण्डलुगतं निधायाग्निनिमन्त्रितम्}% ३०६

\twolineshloka
{आवाहनादि कृत्वाथ स्नानपर्यन्तमेव च}
{अथ स्नापयितुं देवमादाय करसम्पुटे}% ३०७

\twolineshloka
{कृत्वा निरीक्षणं देवपीठं नो दृष्टवान्कपिः}
{लिङ्गमात्रं करगतं दृष्ट्वा भीतिसमन्वितः}% ३०८

\twolineshloka
{इदमाह महायोगी किं वा पापं मया कृतम्}
{यदेतत्पीठरहितं शिवलिङ्गं करस्थितम्}% ३०९

\twolineshloka
{ममाद्य मरणं सिद्धं न पीठं चागमिष्यति}
{अथ रुद्रं जपिष्यामि तदायाति महेश्वरः}% ३१०

\twolineshloka
{इति निश्चित्य मनसा जजाप शतरुद्रि यम्}
{यदा तु न समायातो महेशोऽथ कपीश्वरः}% ३११

\twolineshloka
{रुद्रं न्यपातयद्भूमौ वीरभद्र ः! समागतः}
{किमर्थ रुद्यते भद्र रुदिते कारणं वद}% ३१२

\twolineshloka
{तच्छ्रुत्वा प्राह हनुमान्वीरभद्रं मनोगतम्}
{पीठहीनमिदं लिङ्गं पश्य मे पापसञ्चयम्}% ३१३

\twolineshloka
{वीरभद्र स्ततः प्राह श्रुत्वा कपिसमीरितम्}
{यदि नायाति पीठं ते लिङ्गे मा साहसं कुरु}% ३१४

\twolineshloka
{दाहयिष्याम्यहं लोकान्यदि नायाति पीठकम्}
{पश्य दर्शय मे लिङ्गं पीठं चात्रागतं न वा}% ३१५

\twolineshloka
{अथ दृष्ट्वा वीरभद्रो लिङ्गे पीठमनागतम्}
{दग्धुकामोऽखिलाँल्लोकान्वीरभद्रः प्रतापवान्}% ३१६

\twolineshloka
{अनलं भुवि चिक्षेप क्षणाद्दग्धा मही तदा}
{अथ सप्ततलान्दग्ध्वा पुनरूर्द्ध्वमवर्तत}% ३१७

\twolineshloka
{पञ्चोर्द्ध्वलोकानदहज्जनलोकनिवासिनः}
{ललाटनेत्रसम्भूतं नखेनादाय चानलम्}% ३१८

\twolineshloka
{जम्बीरफलसङ्काशं कृत्वा करतले विभुः}
{तपः सत्यं च सन्दग्धुमुद्यतोऽभून्मुनीश्वरः}% ३१९

\twolineshloka
{ततस्तु मुनयो दृष्ट्वा तपोलोकनिवासिनः}
{दग्धुकामं वीरभद्रं गौतमाश्रममागताः}% ३२०

\twolineshloka
{न दृष्ट्वा तत्र देवेशं शङ्करं स्वात्मनि स्थितम्}
{अस्तुवन्भक्तिसंयुक्ताः स्तोत्रैर्वेदसमुद्भवैः}% ३२१

\twolineshloka
{ॐ वेदवेद्याय देवाय तस्मै शुद्धप्रभाचिन्त्यरूपाय कस्मै}
{ब्रह्माद्यधीशाय सृष्ट्यादिकर्त्रे विष्णुप्रियायार्तिहन्त्रेऽन्तकर्त्रे}% ३२२

\twolineshloka
{नमस्तेऽखिलधीश्वरायाम्बराय नमस्ते चरस्थावरव्यापकाय}
{नमो वेदगुह्याय भक्तप्रियाय नमः पाकभोक्त्रे मखेशाय तुभ्यम्}% ३२३

\twolineshloka
{नमस्ते शिवायादिदेवाय कुर्मो नमो व्यालयज्ञोपवीतप्रधर्त्रे}
{नमस्ते सुराबिन्दुवर्षापनाय त्रयीमूर्तये कालकालाय नाथ}% ३२४

\twolineshloka
{धरित्रीमरुद्व्योमतोयेन्दुवह्निप्रभामण्डलात्माष्टधामूर्तिधर्त्रे}
{शिवायाशिवघ्नाय वीराय भूयात्सदा नः प्रसन्नो जगन्नाथकेज्यः}% ३२५

\twolineshloka
{कलानाथभालाय आत्मा महात्मा मनो ह्यग्रयानो निरूप्यो न वाग्भिः}
{जगज्जाढ्यविध्वंसनो भुक्तिमुक्तिप्रदः स्तात्प्रसन्नः सदा शुद्धकीर्तिः}% ३२६

\twolineshloka
{यतः सम्प्रसूतं जगज्जातमीशात्स्थितं येन रक्षावता भावितं च}
{लयं यास्यते यत्र वाचां विदूरे स वै नः प्रसन्नोऽस्तु कालत्रयात्मा}% ३२७

\twolineshloka
{यदादिं च मध्यं तथान्तं न केऽपि विजानन्ति विज्ञा अपि स्वानुमानाः}
{स वै सर्वमूर्तिः सदा नो विभूत्यै प्रसन्नोऽस्तु किं ज्ञापयामोऽत्र कृत्यम्}% ३२८

\twolineshloka
{एतां स्तुतिमथाकर्ण्य भगनेत्रप्रदः शिवः}
{विष्णुमाह मुनीनेतानानयस्व मदन्तिकम्}% ३२९

\twolineshloka
{अथ विष्णुः समागत्य तपोलोकनिवासिनः}
{मुनीन्सान्त्वय्य विश्वेशं दर्शयामास शङ्करम्}% ३३०

\twolineshloka
{तानाह शङ्करो वाक्यं किमर्थं यूयमागताः}
{तपोलोकाद्भूमिलोकं मुनयो मुक्तकिल्बिषाः}% ३३१

\twolineshloka
{तच्छ्रुत्वा शूलिनो वाक्यं प्रोचुस्ते मुनिसत्तमाः}
{देव द्वादशलोकानां दृश्यन्ते भस्मराशयः}% ३३२

\twolineshloka
{स्थितमेकं वनमिदं पश्य तल्लोकसङ्क्षयम्}
{तच्छ्रुत्वा गिरिशः प्राह तान्मुनीनूर्द्ध्वरेतसः}% ३३३

\twolineshloka
{भूर्लोकस्य तु सन्दाहे पातालानां तथैव च}
{सन्देहो नास्ति मुनयः स्थितानां नो रहः स्थले}% ३३४

\twolineshloka
{ऊर्द्ध्वपञ्चकलोकानां दाहे सन्देह एव नः}
{कथमङ्गारवृष्टिश्च कथं नो वा महाध्वनिः}% ३३५

\twolineshloka
{तदाकर्ण्य विभोर्वाक्यं शङ्करस्य मुनीश्वराः}
{प्रोचुः प्राञ्जलयो देवं ब्रह्मादिसुरसङ्गतम्}% ३३६

\twolineshloka
{भीतिरस्माकमधुना वर्तते वीरभद्र तः}
{स एवाङ्गारवृष्टिं च पिपासुरपिबद्विभोः}% ३३७

\twolineshloka
{देवोऽथ वीरमाहूय किं वीरेत्यब्रवीद्वचः}
{वीरोऽप्याह कपेर्लिङ्गे पीठाभावादिदं कृतम्}% ३३८

\twolineshloka
{तच्छ्रुत्वाह शिवो देवो मुनींस्तान्भयविह्वलान्}
{कपेश्चित्तं परिज्ञातुं मया कृतमिदं द्विजाः}% ३३९

\twolineshloka
{मा भैष्ट भवतां सौख्यं सदा सम्पादयाम्यहम्}
{इत्युक्त्वा तु यथापूर्वं देवदेवः कृपानिधिः}% ३४०

\twolineshloka
{दग्धानप्यखिलाँ ल्लोकान्पूर्वतः शोभनान्विभुः}
{कल्पयामास विश्वात्मा वीरभद्र मथाब्रवीत्}% ३४१

\twolineshloka
{साधु वत्स यतो भद्रं भक्तानामीहसे सदा}
{ततस्ते विपुला कीर्तिर्लोके स्थास्यति शाश्वती}% ३४२

\twolineshloka
{इत्युक्त्वालिङ्ग्य शिरसि समाघ्राय महेश्वरः}
{ताम्बूलं वीरभद्रा य दत्तवान्प्रीतमानसः}% ३४३

\twolineshloka
{अथासौ हनुमानीशपूजनं कृतवान्यथा}
{समाप्तायां तु पूजायां हनुमान्प्रीतमानसः}% ३४४

\twolineshloka
{एकं वनचरं तत्र गन्धर्वं सविपञ्चकम्}
{ददर्श तमथाभ्याह वीणा मे दीयतामिति}% ३४५

\twolineshloka
{गन्धर्वोऽप्याह न मया त्याज्या वीणा प्रिया मम}
{ममापीष्टेह गन्धर्व वीणेत्याह कपीश्वरः}% ३४६

\twolineshloka
{यदा न दत्ते गन्धर्वो वल्लकीं कपये प्रियाम्}
{तदा मुष्टिप्रहारेण गन्धर्वः पातितः क्षितौ}% ३४७

\twolineshloka
{वीणामादाय महतीं स्वरतन्तुसमन्विताम्}
{हनुमान्वानरश्रेष्ठो गायन्प्रागाच्छिवान्तिकम्}% ३४८

\twolineshloka
{ततो गानेन महता प्रसाद्य जगदीश्वरम्}
{बृहतीकुसुमैः शुद्धैर्देवपादावपूजयत्}% ३४९

\twolineshloka
{ततः प्रसन्नो विश्वात्मा मुनीनां सन्निधौ तदा}
{दैत्यानां देवतानां च नृपाणां शङ्करोऽपि च}% ३५०

\twolineshloka
{तस्मै वरमथ प्रादात्कल्पान्तं जीवितं पुनः}
{समुद्र लङ्घने शक्तिं शास्त्रज्ञत्वं बलोन्नतिम्}% ३५१

\twolineshloka
{एवं दत्तं वरं प्राप्य महेशेन महात्मना}
{प्रत्यक्षं मम विप्रेन्द्र हनुमान्हर्षमागतः}% ३५२

\twolineshloka
{समस्तभूषासुविभूषिताङ्गः स्वदीप्तिमन्दीकृतदेवदीप्तिः}
{प्रसन्नमूर्तिस्तरुणः शिवांशः सम्भावयामास समस्तदेवान्}% ३५३

\twolineshloka
{आज्ञप्तो हनुमांस्तत्र मत्सेवायै मुनीश्वरः}
{महेशेनाहमप्येनं शशिमौलिमवैमि च}% ३५४

\twolineshloka
{किं बहूक्तेन विप्रर्षे यादृशो वानरेश्वरः}
{बुद्धौ न्याये च वै धैर्ये तादृगन्योऽस्ति न क्वचित्}% ३५५

\twolineshloka
{इति ते सर्वमाख्यातं चरितं पापनाशनम्}
{पठतां शृण्वतां चैव गच्छ विप्र यथासुखम्}% ३५६

\twolineshloka
{तच्छ्रुत्वा रामभद्र स्य रघुनाथस्य धीमतः}
{वचनं दक्षिणीकृत्य नत्वा चागां यथागतः}% ३५७

\twolineshloka
{एतत्तेऽभिहितं विप्र चरितं च हनूमतः}
{सुखदं मोक्षदं सारं किमन्यच्छ्रोतुमिच्छसि}% ३५८

॥इति श्रीबृहन्नारदीयपुराणे पूर्वभागे तृतीयपादे बृहदुपाख्याने तृतीयपादे हनुमच्चरित्रं नाम एकोनाशीतितमोऽध्यायः॥७९॥

    \chapt{ब्रह्म-पुराणम्}

\sect{रामतीर्थवर्णनम्}

\src{ब्रह्म-पुराणम्}{}{अध्यायः १२३}{}
% \tags{concise, complete}
\notes{This chapter describes the significance of Rama Tirtha, a sacred place associated with that helped Dasharatha expiate his sins.}
\textlink{https://sa.wikisource.org/wiki/ब्रह्मपुराणम्/अध्यायः_१२३}
\translink{}

\storymeta


\twolineshloka
{रामतीर्थमिति ख्यातं भ्रूणहत्याविनाशनम्}
{तस्य श्रवणमात्रेण सर्वपापैः प्रमुच्यते} %॥१॥

\twolineshloka
{इक्ष्वाकुवंशप्रभवः क्षत्रियो लोकविश्रुतः}
{बलवान्मतिमाञ्शूरो यथा शक्रः पुरन्दरः} %॥२॥

\twolineshloka
{पितृपैतामहं राज्यं कुर्वन्नास्ते यथा बलिः}
{तस्य तिस्रो महिष्यः स्यू राज्ञो दशरथस्य हि} %॥३॥

\twolineshloka
{कौशल्या च सुमित्रा च कैकेयी च महामते}
{एताः कुलीनाः सुभगा रूपलक्षणसंयुताः} %॥४॥

\twolineshloka
{तस्मिन् राजनि राज्ये तु स्थितेऽयोध्यापतौ मुने}
{वसिष्ठे ब्रह्मविच्छ्रेष्ठे पुरोधसि विशेषतः} %॥५॥

\twolineshloka
{न च व्याधिर्न दुर्भिक्षं न चावृष्टिर्न चाधयः}
{ब्रह्मक्षत्रविशां नित्यं शूद्राणां च विशेषतः} %॥६॥

\twolineshloka
{आश्रमाणां तु सर्वेषामानन्दोऽभूत्पृथक्पृथक्}
{तस्मिञ्शासति राजेन्द्र इक्ष्वाकूणां कुलोद्वहे} %॥७॥

\twolineshloka
{देवानां दानवानां तु राज्यार्थे विग्रहोऽभवत्}
{क्वापि तत्र जयं प्रापुर्देवाः क्वापि तथेतरे} %॥८॥

\twolineshloka
{एवं प्रवर्तमाने तु त्रैलोक्यमतिपीडितम्}
{अभून्नारद तत्राहमवदं दैत्यदानवान्} %॥९॥

\twolineshloka
{देवांश्चापि विशेषेण न कृतं तैर्मदीरितम्}
{पुनश्च सङ्गरस्तेषां बभूव सुमहान्मिथः} %॥१०॥

\twolineshloka
{विष्णुं गत्वा सुराः प्रोचुस्तथेशानं जगन्मयम्}
{तावूचतुरुभौ देवानसुरान् दैत्यदानवान्} %॥११॥

\twolineshloka
{तपसा बलिनो यान्तु पुनः कुर्वन्तु सङ्गरम्}
{तथेत्याहुर्ययुः सर्वे तपसे नियतव्रताः} %॥१२॥

\twolineshloka
{ययुस्तु राक्षसान् देवाः पुनस्ते मत्सरान्विताः}
{देवानां दानवानां च सङ्गरोऽभूत्सुदारुणः} %॥१३॥

\twolineshloka
{न तत्र देवा जेतारो नैव दैत्याश्च दानवाः}
{संयुगे वर्तमाने तु वागुवाचाशरीरिणी} %॥१४॥

\uvacha{आकाशवागुवाच}

\onelineshloka
{येषां दशरथो राजा ते जेतारो न चेतरे} %}%॥* १५॥

\uvacha{ब्रह्मोवाच}

\twolineshloka
{इति श्रुत्वा जयायोभौ जग्मतुर्देवदानवौ}
{तत्र वायुस्त्वरन् प्राप्तो राजानमवदत्तदा} %॥१६॥

\uvacha{वायुरुवाच}


\twolineshloka
{आगन्तव्यं त्वया राजन् देवदानवसङ्गरे}
{यत्र राजा दशरथो जयस्तत्रेति विश्रुतम्} %॥१७॥

\onelineshloka
{तस्मात्त्वं देवपक्षे स्या भवेयुर्जयिनः सुराः}%॥* १८॥

\uvacha{ब्रह्मोवाच}


\twolineshloka
{तद्वायुवचनं श्रुत्वा राजा दशरथो नृपः}
{आगम्यते मया सत्यं गच्छ वायो यथासुखम्} %॥१९॥

\twolineshloka
{गते वायौ तदा दैत्या आजग्मुर्भूपतिं प्रति}
{तेऽप्यूचुर्भगवन्नस्मत्साहाय्यं कर्तुमर्हसि} %॥२०॥

\twolineshloka
{राजन् दशरथ श्रीमन् विजयस्त्वयि संस्थितः}
{तस्मात्त्वं वै दैत्यपतेः साहाय्यं कर्तुमर्हसि} %॥२१॥

\twolineshloka
{ततः प्रोवाच नृपतिर्वायुना प्रार्थितः पुरा}
{प्रतिज्ञातं मया तच्च यान्तु दैत्याश्च दानवाः} %॥२२॥

\twolineshloka
{स तु राजा तथा चक्रे गत्वा चैव त्रिविष्टपम्}
{युद्धं चक्रे तथा दैत्यैर्दानवैः सह राक्षसैः} %॥२३॥

\twolineshloka
{पश्यत्सु देवसङ्घेषु नमुचेर्भ्रातरस्तदा}
{विविधुर्निशितैर्बाणैरथाक्षं नृपतेस्तथा} %॥२४॥

\twolineshloka
{भिन्नाक्षं तं रथं राजा न जानाति स सम्भ्रमात्}
{राजान्तिके स्थिता सुभ्रूः कैकेय्याज्ञायि नारद} %॥२५॥

\twolineshloka
{न ज्ञापितं तया राज्ञे स्वयमालोक्य सुव्रता}
{भग्नमक्षं समालक्ष्य चक्रे हस्तं तदा स्वकम्} %॥२६॥

\twolineshloka
{अक्षवन्मुनिशार्दूल तदेतन्महदद्भुतम्}
{रथेन रथिनां श्रेष्ठस्तया दत्तकरेण च} %॥२७॥

\twolineshloka
{जितवान् दैत्यदनुजान् देवैः प्राप्य वरान् बहून्}
{ततो देवैरनुज्ञातस्त्वयोध्यां पुनरभ्यगात्} %॥२८॥

\twolineshloka
{स तु मध्ये महाराजो मार्गे वीक्ष्य तदा प्रियाम्}
{कैकेय्याः कर्म तद्दृष्ट्वा विस्मयं परमं गतः} %॥२९॥

\twolineshloka
{ततस्तस्यै वरान् प्रादात्त्रींस्तु नारद सा अपि}
{अनुमान्य नृपप्रोक्तं कैकेयी वाक्यमब्रवीत्} %॥३०॥

\uvacha{कैकेय्युवाच}


\onelineshloka
{त्वयि तिष्ठन्तु राजेन्द्र त्वया दत्ता वरा अमी}%॥* ३१॥

\uvacha{ब्रह्मोवाच}


\twolineshloka
{विभूषणानि राजेन्द्रो दत्त्वा स प्रियया सह}
{रथेन विजयी राजा ययौ स्वनगरं सुखी} %॥३२॥

\twolineshloka
{योषितां किमदेयं हि प्रियाणामुचितागमे}
{स कदाचिद्दशरथो मृगयाशीलिभिर्वृतः} %॥३३॥

\twolineshloka
{अटन्नरण्ये शर्वर्यां वारिबन्धमथाकरोत्}
{सप्तव्यसनहीनेन भवितव्यं तु भूभुजा} %॥३४॥

\twolineshloka
{इति जानन्नपि च तच्चकार तु विधेर्वशात्}
{गर्तं प्रविश्य पानार्थमागतान्निशितैः शरैः} %॥३५॥

\twolineshloka
{मृगान् हन्ति महाबाहुः शृणु कालविपर्ययम्}
{गर्तं प्रविष्टे नृपतौ तस्मिन्नेव नगोत्तमे} %॥३६॥

\twolineshloka
{वृद्धो वैश्रवणो नाम न शृणोति न पश्यति}
{तस्य भार्या तथाभूता तावब्रूतां तदा सुतम्} %॥३७॥

\uvacha{मातापितरावूचतुः॒}


\twolineshloka
{आवां तृषार्तौ रात्रिश्च कृष्णा चापि प्रवर्तते}
{वृद्धानां जीवितं कृत्स्नं बालस्त्वमसि पुत्रक} %॥३८॥

\twolineshloka
{अन्धानां बधिराणां च वृद्धानां धिक्च जीवितम्}
{जराजर्जरदेहानां धिग्धिक्पुत्रक जीवितम्} %॥३९॥

\twolineshloka
{तावत्पुम्भिर्जीवितव्यं यावल्लक्ष्मीर्दृढं वपुः}
{यावदाज्ञाप्रतिहता तीर्थादावन्यथा मृतिः} %॥४०॥

\uvacha{ब्रह्मोवाच}

\twolineshloka
{इत्येतद्वचनं श्रुत्वा वृद्धयोर्गुरुवत्सलः}
{पुत्रः प्रोवाच तद्दुःखं गिरा मधुरया हरन्} %॥४१॥

\uvacha{पुत्र उवाच}

\twolineshloka
{मयि जीवति किं नाम युवयोर्दुःखमीदृशम्}
{न हरत्यात्मजः पित्रोर्यश्चरित्रैर्मनोरुजम्} %॥४२॥

\onelineshloka
{तेन किं तनुजेनेह कुलोद्वेगविधायिना}%॥* ४३॥

\uvacha{ब्रह्मोवाच}


\twolineshloka
{इत्युक्त्वा पितरौ नत्वा तावाश्वास्य महामनाः}
{तरुस्कन्धे समारोप्य वृद्धौ च पितरौ तदा} %॥४४॥

\twolineshloka
{हस्ते गृहीत्वा कलशं जगाम ऋषिपुत्रकः}
{स ऋषिर्न तु राजानं जानाति नृपतिर्द्विजम्} %॥४५॥

\twolineshloka
{उभौ सरभसौ तत्र द्विजो वारि समाविशत्}
{सत्वरं कलशे न्युब्जे वारि गृह्णन्तमाशुगैः} %॥४६॥

\twolineshloka
{द्विजं राजा द्विपं मत्वा विव्याध निशितैः शरैः}
{वनद्विपोऽपि भूपानामवध्यस्तद्विदन्नपि} %॥४७॥

\twolineshloka
{विव्याध तं नृपः कुर्यान्न किं किं विधिवञ्चितः}
{स विद्धो मर्मदेशे तु दुःखितो वाक्यमब्रवीत्} %॥४८॥

\uvacha{द्विज उवाच}


\twolineshloka
{केनेदं दुःखदं कर्म कृतं सद्ब्राह्मणस्य मे}
{मैत्रो ब्राह्मण इत्युक्तो नापराधोऽस्ति कश्चन} %॥४९॥

\uvacha{ब्रह्मोवाच}


\twolineshloka
{तदेतद्वचनं श्रुत्वा मुनेरार्तस्य भूपतिः}
{निश्चेष्टश्च निरुत्साहो शनैस्तं देशमभ्यगात्} %॥५०॥

\twolineshloka
{तं तु दृष्ट्वा द्विजवरं ज्वलन्तमिव तेजसा}
{असावप्यभवत्तत्र सशल्य इव मूर्च्छितः} %॥५१॥

\onelineshloka
{आत्मानमात्मना कृत्वा स्थिरं राजाब्रवीदिदम्}%॥* ५२॥

\uvacha{राजोवाच}


\twolineshloka
{को भवान् द्विजशार्दूल किमर्थमिह चागतः}
{वद पापकृते मह्यं वद मे निष्कृतिं पराम्} %॥५३॥

\twolineshloka
{ब्रह्महा वर्णिभिः किन्तु श्वपचैरपि जातुचित्}
{न स्प्रष्टव्यो महाबुद्धे द्रष्टव्यो न कदाचन} %॥५४॥

\uvacha{ब्रह्मोवाच}


\onelineshloka
{तद्राजवचनं श्रुत्वा मुनिपुत्रोऽब्रवीद्वचः}%॥* ५५॥

\uvacha{मुनिपुत्र उवाच}


\twolineshloka
{उत्क्रमिष्यन्ति मे प्राणा अतो वक्ष्यामि किञ्चन}
{स्वच्छन्दवृत्तिताज्ञाने विद्धि पाकं च कर्मणाम्} %॥५६॥

\twolineshloka
{आत्मार्थं तु न शोचामि वृद्धौ तु पितरौ मम}
{तयोः शुश्रूषकः कः स्यादन्धयोरेकपुत्रयोः} %॥५७॥

\twolineshloka
{विना मया महारण्ये कथं तौ जीवयिष्यतः}
{ममाभाग्यमहो कीदृक्पितृशुश्रूषणे क्षतिः} %॥५८॥

\twolineshloka
{जाता मेऽद्य विना प्राणैर्हा विधे किं कृतं त्वया}
{तथापि गच्छ तत्र त्वं गृहीतकलशस्त्वरन्} %॥५९॥

\onelineshloka
{ताभ्यां देह्युदपानं त्वं यथा तौ न मरिष्यतः}%॥* ६०॥

\uvacha{ब्रह्मोवाच}


\twolineshloka
{इत्येवं ब्रुवतस्तस्य गताः प्राणा महावने}
{विसृज्य सशरं चापमादाय कलशं नृपः} %॥६१॥

\twolineshloka
{तत्रागात्स तु वेगेन यत्र वृद्धौ महावने}
{वृद्धौ चापि तदा रात्रौ तावन्योन्यं समूचतुः} %॥६२॥

\uvacha{वृद्धावूचतुः॒}


\twolineshloka
{उद्विग्नः कुपितो वा स्यादथवा भक्षितः कथम्}
{न प्राप्तश्चावयोर्यष्टिः किं कुर्मः का गतिर्भवेत्} %॥६३॥

\twolineshloka
{न कोपि तादृशः पुत्रो विद्यते सचराचरे}
{यः पित्रोरन्यथा वाक्यं न करोत्यपि निन्दितः} %॥६४॥

\twolineshloka
{वज्रादपि कठोरं वा जीवितं तमपश्यतोः}
{शीघ्रं न यान्ति यत्प्राणास्तदेकायत्तजीवयोः} %॥६५॥

\uvacha{ब्रह्मोवाच}


\twolineshloka
{एवं बहुविधा वाचो वृद्धयोर्वदतोर्वने}
{तदा दशरथो राजा शनैस्तं देशमभ्यगात्} %॥६६॥

\onelineshloka
{पादसञ्चारशब्देन मेनाते सुतमागतम्}%॥* ६७॥

\uvacha{वृद्धावूचतुः॒}


\twolineshloka
{कुतो वत्स चिरात्प्राप्तस्त्वं दृष्टिस्त्वं परायणम्}
{न ब्रूषे किन्तु रुष्टोऽसि वृद्धयोरन्धयोः सुतः} %॥६८॥

\uvacha{ब्रह्मोवाच}


\twolineshloka
{सशल्य इव दुःखार्तः शोचन् दुष्कृतमात्मनः}
{स भीत इव राजेन्द्रस्तावुवाचाथ नारद} %॥६९॥

\twolineshloka
{उदपानं च कुरुतां तच्छ्रुत्वा नृपभाषितम्}
{नायं वक्ता सुतोऽस्माकं को भवांस्तत्पुरा वद} %॥७०॥

\onelineshloka
{पश्चात्पिबावः पानीयं ततो राजाब्रवीच्च तौ}%॥* ७१॥

\uvacha{राजोवाच}


\onelineshloka
{तत्र तिष्ठति वां पुत्रो यत्र वारिसमाश्रयः}%॥* ७२॥

\uvacha{ब्रह्मोवाच}


\twolineshloka
{तच्छ्रुत्वोचतुरार्तौ तौ सत्यं ब्रूहि न चान्यथा}
{आचचक्षे ततो राजा सर्वमेव यथातथम्} %॥७३॥

\twolineshloka
{ततस्तु पतितौ वृद्धौ तत्रावां नय मा स्पृश}
{ब्रह्मघ्नस्पर्शनं पापं न कदाचिद्विनश्यति} %॥७४॥

\twolineshloka
{निन्ये वै श्रवणं वृद्धं सभार्यं नृपसत्तमः}
{यत्रासौ पतितः पुत्रस्तं स्पृष्ट्वा तौ विलेपतुः} %॥७५॥

\uvacha{वृद्धावूचतुः॒}


\twolineshloka
{यथा पुत्रवियोगेन मृत्युर्नौ विहितस्तथा}
{त्वं चापि पाप पुत्रस्य वियोगान्मृत्युमाप्स्यसि} %॥७६॥

\uvacha{ब्रह्मोवाच}


\twolineshloka
{एवं तु जल्पतोर्ब्रह्मन् गताः प्राणास्ततो नृपः}
{अग्निना योजयामास वृद्धौ च ऋषिपुत्रकम्} %॥७७॥

\twolineshloka
{ततो जगाम नगरं दुःखितो नृपतिर्मुने}
{वसिष्ठाय च तत्सर्वं न्यवेदयदशेषतः} %॥७८॥

\twolineshloka
{नृपाणां सूर्यवंश्यानां वसिष्ठो हि परा गतिः}
{वसिष्ठोऽपि द्विजश्रेष्ठैः सम्मन्त्र्याह च निष्कृतिम्} %॥७९॥

\uvacha{वसिष्ठ उवाच}


\twolineshloka
{गालवं वामदेवं च जाबालिमथ कश्यपम्}
{एतानन्यान् समाहूय हयमेधाय यत्नतः} %॥८०॥

\onelineshloka
{यजस्व हयमेधैश्च बहुभिर्बहुदक्षिणैः}%॥* ८१॥

\uvacha{ब्रह्मोवाच}


\twolineshloka
{अकरोद्धयमेधांश्च राजा दशरथो द्विजैः}
{एतस्मिन्नन्तरे तत्र वागुवाचाशरीरिणी} %॥८२॥

\uvacha{आकाशवाण्युवाच}

पूतं शरीरमभवद्राज्ञो दशरथस्य हि।

\twolineshloka
{व्यवहार्यश्च भविता भविष्यन्ति तथा सुताः}
{ज्येष्ठपुत्रप्रसादेन राजापापो भविष्यति} %॥८३॥

\uvacha{ब्रह्मोवाच}


\twolineshloka
{ततो बहुतिथे काले ऋष्यशृङ्गान्मुनीश्वरात्}
{देवानां कार्यसिद्ध्यर्थं सुता आसन् सुरोपमाः} %॥८४॥

\twolineshloka
{कौशल्यायां तथा रामः सुमित्रायां च लक्ष्मणः}
{शत्रुघ्नश्चापि कैकेय्यां भरतो मतिमत्तरः} %॥८५॥

\twolineshloka
{ते सर्वे मतिमन्तश्च प्रिया राज्ञो वशे स्थिताः}
{तं राजानमृषिः प्राप्य विश्वामित्रः प्रजापतिः} %॥८६॥

\twolineshloka
{रामं च लक्ष्मणं चापि अयाचत महामते}
{यज्ञसंरक्षणार्थाय ज्ञाततन्महिमा मुनिः} %॥८७॥

\onelineshloka
{चिरप्राप्तसुतो वृद्धो राजा नैवेत्यभाषत}%॥* ८८॥

\uvacha{राजोवाच}


\twolineshloka
{महता दैवयोगेन कथञ्चिद्वार्धके मुने}
{जातावानन्दसन्दोह दायकौ मम बालकौ} %॥८९॥

\onelineshloka
{सशरीरमिदं राज्यं दास्ये नैव सुताविमौ}%॥* ९०॥

\uvacha{ब्रह्मोवाच}


\onelineshloka
{वसिष्ठेन तदा प्रोक्तो राजा दशरथस्त्विति}%॥* ९१॥

\uvacha{वसिष्ठ उवाच}


\onelineshloka
{रघवः प्रार्थनाभङ्गं न राजन् क्वापि शिक्षिताः}%॥* ९२॥

\uvacha{ब्रह्मोवाच}


\onelineshloka
{रामं च लक्ष्मणं चैव कथञ्चिदवदन्नृपः}%॥* ९३॥

\uvacha{राजोवाच}


\onelineshloka
{विश्वामित्रस्य ब्रह्मर्षेः कुरुतां यज्ञरक्षणम्}%॥* ९४॥

\uvacha{ब्रह्मोवाच}


\twolineshloka
{वदन्निति सुतौ सोष्णं निश्वसन् ग्लपिताधरः}
{पुत्रौ समर्पयामास विश्वामित्रस्य शास्त्रकृत्} %॥९५॥

\twolineshloka
{तथेत्युक्त्वा दशरथं नमस्य च पुनः पुनः}
{जग्मतू रक्षणार्थाय विश्वामित्रेण तौ मुदा} %॥९६॥

\twolineshloka
{ततः प्रहृष्टः स मुनिर्मुदा प्रादात्तदोभयोः}
{माहेश्वरीं महाविद्यां धनुर्विद्यापुरःसराम्} %॥९७॥

\twolineshloka
{शास्त्रीमास्त्रीं लौकिकीं च रथविद्यां गजोद्भवाम्}
{अश्वविद्यां गदाविद्यां मन्त्राह्वानविसर्जने} %॥९८॥

\twolineshloka
{सर्वविद्यामथावाप्य उभौ तौ रामलक्ष्मणौ}
{वनौकसां हितार्थाय जघ्नतुस्ताटकां वने} %॥९९॥

\twolineshloka
{अहल्यां शापनिर्मुक्तां पादस्पर्शाच्च चक्रतुः}
{यज्ञविध्वंसनायाताञ्जघ्नतुस्तत्र राक्षसान्} %॥१००॥

\twolineshloka
{कृतविद्यौ धनुष्पाणी चक्रतुर्यज्ञरक्षणम्}
{ततो महामखे वृत्ते विश्वामित्रो मुनीश्वरः} %॥१०१॥

\twolineshloka
{पुत्राभ्यां सहितो राज्ञो जनकं द्रष्टुमभ्यगात्}
{चित्रामदर्शयत्तत्र राजमध्ये नृपात्मजः} %॥१०२॥

\twolineshloka
{रामः सौमित्रिसहितो धनुर्विद्यां गुरोर्मताम्}
{तत्प्रीतो जनकः प्रादात्सीतां लक्ष्मीमयोनिजाम्} %॥१०३॥

\twolineshloka
{तथैव लक्ष्मणस्यापि भरतस्यानुजस्य च}
{शत्रुघ्नभरतादीनां वसिष्ठादिमते स्थितः} %॥१०४॥

\twolineshloka
{राजा दशरथः श्रीमान् विवाहमकरोन्मुने}
{ततो बहुतिथे काले राज्यं तस्य प्रयच्छति} %॥१०५॥

\twolineshloka
{नृपतौ सर्वलोकानामनुमत्या गुरोरपि}
{मन्थरात्मकदुर्दैव प्रेरिता मत्सराकुला} %॥१०६॥

\twolineshloka
{कैकेयी विघ्नमातस्थे वनप्रव्राजनं तथा}
{भरतस्य च तद्राज्यं राजा नैव च दत्तवान्} %॥१०७॥

\twolineshloka
{पितरं सत्यवाक्यं तं कुर्वन् रामो महावनम्}
{विवेश सीतया सार्धं तथा सौमित्रिणा सह} %॥१०८॥

\twolineshloka
{सतां च मानसं शुद्धं स विवेश स्वकैर्गुणैः}
{तस्मिन् विनिर्गते रामे वनवासाय दीक्षिते} %॥१०९॥

\twolineshloka
{समं लक्ष्मणसीताभ्यां राज्यतृष्णाविवर्जिते}
{तं रामं चापि सौमित्रिं सीतां च गुणशालिनीम्} %॥११०॥

\twolineshloka
{दुःखेन महताविष्टो ब्रह्मशापं च संस्मरन्}
{तदा दशरथो राजा प्राणांस्तत्याज दुःखितः} %॥१११॥

\twolineshloka
{कृतकर्मविपाकेन राजा नीतो यमानुगैः}
{तस्मै राज्ञे महाप्राज्ञ यावत्स्थावरजङ्गमे} %॥११२॥

\twolineshloka
{यमसद्मन्यनेकानि तामिस्रादीनि नारद}
{नरकाण्यथ घोराणि भीषणानि बहूनि च} %॥११३॥

\twolineshloka
{तत्र क्षिप्तस्तदा राजा नरकेषु पृथक्पृथक्}
{पच्यते छिद्यते राजा पिष्यते चूर्ण्यते तथा} %॥११४॥

\twolineshloka
{शोष्यते दश्यते भूयो दह्यते च निमज्ज्यते}
{एवमादिषु घोरेषु नरकेषु स पच्यते} %॥११५॥

\twolineshloka
{रामोऽपि गच्छन्नध्वानं चित्रकूटमथागमत्}
{तत्रैव त्रीणि वर्षाणि व्यतीतानि महामते} %॥११६॥

\twolineshloka
{पुनः स दक्षिणामाशामाक्रामद्दण्डकं वनम्}
{विख्यातं त्रिषु लोकेषु देशानां तद्धि पुण्यदम्} %॥११७॥

\twolineshloka
{प्राविशत्तन्महारण्यं भीषणं दैत्यसेवितम्}
{तद्भयादृषिभिस्त्यक्तं हत्वा दैत्यांस्तु राक्षसान्} %॥११८॥

\twolineshloka
{विचरन् दण्डकारण्ये ऋषिसेव्यमथाकरोत्}
{तत्रेदं वृत्तमाख्यास्ये शृणु नारद यत्नतः} %॥११९॥

\twolineshloka
{तावच्छनैस्त्वगाद्रामो यावद्योजनपञ्चकम्}
{गौतमीं समनुप्राप्तो राजापि नरके स्थितः} %॥१२०॥

\twolineshloka
{यमः स्वकिङ्करानाह रामो दशरथात्मजः}
{गौतमीमभितो याति पितरं तस्य धीमतः} %॥१२१॥

\twolineshloka
{आकर्षन्त्वथ राजानं नरकान्नात्र संशयः}
{उत्तीर्य गौतमीं याति यावद्योजनपञ्चकम्} %॥१२२॥

\twolineshloka
{रामस्तावत्तस्य पिता नरके नैव पच्यताम्}
{यदेतन्मद्वचः पुण्यं न कुर्युर्यदि दूतकाः} %॥१२३॥

\twolineshloka
{ततश्च नरके घोरे यूयं सर्वे निमज्जथ}
{या काप्युक्ता परा शक्तिः शिवस्य समवायिनी} %॥१२४॥

\twolineshloka
{तामेव गौतमीं सन्तो वदन्त्यम्भःस्वरूपिणीम्}
{हरिब्रह्ममहेशानां मान्या वन्द्या च सैव यत्} %॥१२५॥

\twolineshloka
{निस्तीर्यते न केनापि तदतिक्रमजं त्वघम्}
{पापिनोऽप्यात्मजः कश्चिद्यश्च गङ्गामनुस्मरेत्} %॥१२६॥

\twolineshloka
{सोऽनेकदुर्गनिरयान्निर्गतो मुक्ततां व्रजेत्}
{किं पुनस्तादृशः पुत्रो गौतमीनिकटे स्थितः} %॥१२७॥

\twolineshloka
{यस्यासौ नरके पक्तुं न कैरपि हि शक्यते}
{दक्षिणाशापतेर्वाक्यं निशम्य यमकिङ्कराः} %॥१२८॥

\twolineshloka
{नरके पच्यमानं तमयोध्याधिपतिं नृपम्}
{उत्तार्य घोरनरकाद्वचनं चेदमब्रुवन्} %॥१२९॥

\uvacha{यमकिङ्करा ऊचुः॒}


\twolineshloka
{धन्योऽसि नृपशार्दूल यस्य पुत्रः स तादृशः}
{इह चामुत्र विश्रान्तिः सुपुत्रः केन लभ्यते} %॥१३०॥

\uvacha{ब्रह्मोवाच}


\onelineshloka
{स विश्रान्तः शनै राजा किङ्करान् वाक्यमब्रवीत्}%॥* १३१॥

\uvacha{राजोवाच}


\twolineshloka
{नरकेष्वथ घोरेषु पच्यमानः पुनः पुनः}
{कथं त्वाकर्षितः शीघ्रं तन्मे वक्तुमिहार्हथ} %॥१३२॥

\uvacha{ब्रह्मोवाच}


\onelineshloka
{तत्र कश्चिच्छान्तमना राजानमिदमब्रवीत्}%॥* १३३॥

\uvacha{यमदूत उवाच}


\twolineshloka
{वेदशास्त्रपुराणादावेतद्गोप्यं प्रयत्नतः}
{प्रकाश्यते तदपि ते सामर्थ्यं पुत्रतीर्थयोः} %॥१३४॥

\twolineshloka
{रामस्तव सुतः श्रीमान् गौतमीतीरमागतः}
{तस्मात्त्वं नरकाद् घोरादाकृष्टोऽसि नरोत्तम} %॥१३५॥
यदि त्वां तत्र गौतम्यां स्मरेद्रामः सलक्ष्मणः।

\twolineshloka
{स्नानं कृत्वाथ पिण्डादि ते दद्यात्स नृपोत्तम}
{ततस्त्वं सर्वपापेभ्यो मुक्तो यासि त्रिविष्टपम्} %॥१३६॥

\uvacha{राजोवाच}


\twolineshloka
{तत्र गत्वा भवद्वाक्यमाख्यास्ये स्वसुतौ प्रति}
{भवन्त एव शरणमनुज्ञां दातुमर्हथ} %॥१३७॥

\uvacha{ब्रह्मोवाच}


\twolineshloka
{तद्राजवचनं श्रुत्वा कृपया यमकिङ्कराः}
{आज्ञां च प्रददुस्तस्मै राजा प्रागात्सुतौ प्रति} %॥१३८॥

\twolineshloka
{भीषणं यातनादेहमापन्नो निःश्वसन्मुहुः}
{निरीक्ष्य स्वं लज्जमानः कृतं कर्म च संस्मरन्} %॥१३९॥

\twolineshloka
{स्वेच्छया विहरन् गङ्गामाससाद च राघवः}
{गौतम्यास्तटमाश्रित्य रामो लक्ष्मण एव च} %॥१४०॥

\twolineshloka
{सीतया सह वैदेह्या सस्नौ चैव यथाविधि}
{नैव तत्राभवद्भोज्यं भक्ष्यं वा गौतमीतटे} %॥१४१॥

\twolineshloka
{तद्दिने तत्र वसतां गौतमीतीरवासिनाम्}
{तद्दृष्ट्वा दुःखितो भ्राता लक्ष्मणो राममब्रवीत्} %॥१४२॥

\uvacha{लक्ष्मण उवाच}


\twolineshloka
{पुत्रौ दशरथस्यावां तवापि बलमीदृशम्}
{नास्ति भोज्यमथास्माकं गङ्गातीरनिवासिनाम्} %॥१४३॥

\uvacha{राम उवाच}


\twolineshloka
{भ्रातर्यद्विहितं कर्म नैव तच्चान्यथा भवेत्}
{पृथिव्यामन्नपूर्णायां वयमन्नाभिलाषिणः} %॥१४४॥

\twolineshloka
{सौमित्रे नूनमस्माभिर्न ब्राह्मणमुखे हुतम्}
{अवज्ञया महीदेवांस्तर्पयन्त्यर्चयन्ति न} %॥१४५॥
ते ये लक्ष्मण जायन्ते सर्वदैव बुभुक्षिताः।

\twolineshloka
{स्नात्वा देवानथाभ्यर्च्य होतव्यश्च हुताशनः}
{ततः स्वसमये देवो विधास्यत्यशनं तु नौ} %॥१४६॥

\uvacha{ब्रह्मोवाच}


\twolineshloka
{भ्रात्रोः सञ्जल्पतोरेवं पश्यतोः कर्मणो गतिम्}
{शनैर्दशरथो राजा तं देशमुपजग्मिवान्} %॥१४७॥

\twolineshloka
{तं दृष्ट्वा लक्ष्मणः शीघ्रं तिष्ठ तिष्ठेति चाब्रवीत्}
{धनुराकृष्य कोपेन रक्षस्त्वं दानवोऽथवा} %॥१४८॥

\twolineshloka
{आसन्नं च पुनर्दृष्ट्वा याहि याह्यत्र पुण्यभाक्}
{रामो दाशरथी राजा धर्मभाक्पश्य वर्तते} %॥१४९॥

\twolineshloka
{गुरुभक्तः सत्यसन्धो देवब्राह्मणसेवकः}
{त्रैलोक्यरक्षादक्षोऽसौ वर्तते यत्र राघवः} %॥१५०॥

\twolineshloka
{न तत्र त्वादृशामस्ति प्रवेशः पापकर्मणाम्}
{यदि प्रविशसे पाप ततो वधमवाप्स्यसि} %॥१५१॥
तत्पुत्रवचनं श्रुत्वा शनैराहूय वाचया।

\twolineshloka
{उवाचाधोमुखो भूत्वा स्नुषां पुत्रौ कृताञ्जलिः}
{मुहुरन्तर्विनिध्यायन् गतिं दुष्कृतकर्मणः} %॥१५२॥

\uvacha{राजोवाच}

अहं दशरथो राजा पुत्रौ मे शृणुतं वचः।

\twolineshloka
{तिसृभिर्ब्रह्महत्याभिर्वृतोऽहं दुःखमागतः}
{छिन्नं पश्यत मे देहं नरकेषु च पातितम्} %॥१५३॥

\uvacha{ब्रह्मोवाच}


\twolineshloka
{ततः कृताञ्जली रामः सीतया लक्ष्मणेन च}
{भूमौ प्रणेमुस्ते सर्वे वचनं चैतदब्रुवन्} %॥१५४॥

\uvacha{सीतारामलक्ष्मणा ऊचुः॒}


\onelineshloka
{कस्येदं कर्मणस्तात फलं नृपतिसत्तम}%॥* १५५॥

\uvacha{ब्रह्मोवाच}


\onelineshloka
{स च प्राह यथावृत्तं ब्रह्महत्यात्रयं तथा}%॥* १५६॥

\uvacha{राजोवाच}


\onelineshloka
{निष्कृतिर्ब्रह्महन्तॄणां पुत्रौ क्वापि न विद्यते}%॥* १५७॥

\uvacha{ब्रह्मोवाच}


\twolineshloka
{ततो दुःखेन महता वृताः सर्वे भुवं गताः}
{राजानं वनवासं च मातरं पितरं तथा} %॥१५८॥
दुःखागमं कर्मगतिं नरके पातनं तथा।

\twolineshloka
{एवमाद्यथ संस्मृत्य मुमोह नृपतेः सुतः}
{विसंज्ञं नृपतिं दृष्ट्वा सीता वाक्यमथाब्रवीत्} %॥१५९॥

\uvacha{सीतोवाच}


\twolineshloka
{न शोचन्ति महात्मानस्त्वादृशा व्यसनागमे}
{चिन्तयन्ति प्रतीकारं दैव्यमप्यथ मानुषम्} %॥१६०॥

\twolineshloka
{शोचद्भिर्युगसाहस्रं विपत्तिर्नैव तीर्यते}
{व्यामोहमाप्नुवन्तीह न कदाचिद्विचक्षणाः} %॥१६१॥

\twolineshloka
{किमनेनात्र दुःखेन निष्फलेन जनेश्वर}
{देहि हत्यां प्रथमतो या जाता ह्यतिभीषणा} %॥१६२॥

\twolineshloka
{पितृभक्तः पुण्यशीलो वेदवेदाङ्गपारगः}
{अनागा यो हतो विप्रस्तत्पापस्यात्र निष्कृतिम्} %॥१६३॥

\twolineshloka
{आचरामि यथाशास्त्रं मा शोकं कुरुतं युवाम्}
{द्वितीयां लक्ष्मणो हत्यां गृह्णातु त्वपरां भवान्} %॥१६४॥

\uvacha{ब्रह्मोवाच}


\twolineshloka
{एतद्धर्मयुतं वाक्यं सीतया भाषितं दृढम्}
{तथेति चाहतुरुभौ ततो दशरथोऽब्रवीत्} %॥१६५॥

\uvacha{दशरथ उवाच}


\twolineshloka
{त्वं हि ब्रह्मविदः कन्या जनकस्य त्वयोनिजा}
{भार्या रामस्य किं चित्रं यद्युक्तमनुभाषसे} %॥१६६॥

\twolineshloka
{न कोपि भवतां किन्तु श्रमः स्वल्पोऽपि विद्यते}
{गौतम्यां स्नानदानेन पिण्डनिर्वपणेन च} %॥१६७॥

\twolineshloka
{तिसृभिर्ब्रह्महत्याभिर्मुक्तो यामि त्रिविष्टपम्}
{त्वया जनकसम्भूते स्वकुलोचितमीरितम्} %॥१६८॥

\twolineshloka
{प्रापयन्ति परं पारं भवाब्धेः कुलयोषितः}
{गोदावर्याः प्रसादेन किं नामास्त्यत्र दुर्लभम्} %॥१६९॥

\uvacha{ब्रह्मोवाच}


\twolineshloka
{तथेति क्रियमाणे तु पिण्डदानाय शत्रुहा}
{नैवापश्यद्भक्ष्यभोज्यं ततो लक्ष्मणमब्रवीत्} %॥१७०॥

\twolineshloka
{लक्ष्मणः प्राह विनयादिङ्गुद्याश्च फलानि च}
{सन्ति तेषां च पिण्याकमानीतं तत्क्षणादिव} %॥१७१॥

\twolineshloka
{पिण्याकेनाथ गङ्गायां पिण्डं दातुं तथा पितुः}
{मनः कुर्वंस्ततो रामो मन्दोऽभूद्दुःखितस्तदा} %॥१७२॥

\twolineshloka
{दैवी वागभवत्तत्र दुःखं त्यज नृपात्मज}
{राज्यभ्रष्टो वनं प्राप्तः किं वै निष्किञ्चनो भवान्} %॥१७३॥

\twolineshloka
{अशठो धर्मनिरतो न शोचितुमिहार्हसि}
{वित्तशाठ्येन यो धर्मं करोति स तु पातकी} %॥१७४॥

\twolineshloka
{श्रूयते सर्वशास्त्रेषु यद्राम शृणु यत्नतः}
{यदन्नः पुरुषो राजंस्तदन्नास्तस्य देवताः} %॥१७५॥

\twolineshloka
{पिण्डे निपतिते भूमौ नापश्यत्पितरं तदा}
{शवं च पतितं यत्र शवतीर्थमनुत्तमम्} %॥१७६॥

\twolineshloka
{महापातकसङ्घात विघातकृदनुस्मृतिः}
{तत्रागच्छंल्लोकपाला रुद्रादित्यास्तथाश्विनौ} %॥१७७॥

\twolineshloka
{स्वं स्वं विमानमारूढास्तेषां मध्येऽतिदीप्तिमान्}
{विमानवरमारूढः स्तूयमानश्च किन्नरैः} %॥१७८॥

\twolineshloka
{आदित्यसदृशाकारस्तेषां मध्ये बभौ पिता}
{तमदृष्ट्वा स्वपितरं देवान् दृष्ट्वा विमानिनः} %॥१७९॥

\twolineshloka
{कृताञ्जलिपुटो रामः पिता मे क्वेत्यभाषत}
{इति दिव्याभवद्वाणी रामं सम्बोध्य सीतया} %॥१८०॥

\twolineshloka
{तिसृभिर्ब्रह्महत्याभिर्मुक्तो दशरथो नृपः}
{वृतं पश्य सुरैस्तात देवा अप्यूचिरे च तम्} %॥१८१॥

\uvacha{देवा ऊचुः॒}


\twolineshloka
{धन्योऽसि कृतकृत्योऽसि राम स्वर्गं गतः पिता}
{नानानिरयसङ्घातात्पूर्वजानुद्धरेत्तु यः} %॥१८२॥

\twolineshloka
{स धन्योऽलङ्कृतं तेन कृतिना भुवनत्रयम्}
{एनं पश्य महाबाहो मुक्तपापं रविप्रभम्} %॥१८३॥

\twolineshloka
{सर्वसम्पत्तियुक्तोऽपि पापी दग्धद्रुमोपमः}
{निष्किञ्चनोऽपि सुकृती दृश्यते चन्द्रमौलिवत्} %॥१८४॥

\uvacha{ब्रह्मोवाच}


\onelineshloka
{दृष्ट्वाब्रवीत्सुतं राजा आशीर्भिरभिनन्द्य च}%॥* १८५॥

\uvacha{राजोवाच}


\twolineshloka
{कृतकृत्योऽसि भद्रं ते तारितोऽहं त्वयानघ}
{धन्यः स पुत्रो लोकेऽस्मिन् पितॄणां यस्तु तारकः} %॥१८६॥

\uvacha{ब्रह्मोवाच}

\onelineshloka*
{ततः सुरगणाः प्रोचुर्देवानां कार्यसिद्धये}

\twolineshloka
{रामं च पुरुषश्रेष्ठं गच्छ तात यथासुखम्}
{ततस्तद्वचनं श्रुत्वा रामस्तानब्रवीत्सुरान्} %॥१८७॥

\uvacha{राम उवाच}


\onelineshloka
{गुरौ पितरि मे देवाः किं कृत्यमवशिष्यते}%॥* १८८॥

\uvacha{देवा ऊचुः॒}


\twolineshloka
{नदी न गङ्गया तुल्या न त्वया सदृशः सुतः}
{न शिवेन समो देवो न तारेण समो मनुः} %॥१८९॥
त्वया राम गुरूणां च कार्यं सर्वमनुष्ठितम्।

\twolineshloka
{तारिताः पितरो राम त्वया पुत्रेण मानद}
{गच्छन्तु सर्वे स्वस्थानं त्वं च गच्छ यथासुखम्} %॥१९०॥

\uvacha{ब्रह्मोवाच}


\twolineshloka
{तद्देववचनाद्धृष्टः सीतया लक्ष्मणाग्रजः}
{तद्दृष्ट्वा गङ्गामाहात्म्यं विस्मितो वाक्यमब्रवीत्} %॥१९१॥

\uvacha{राम उवाच}


\twolineshloka
{अहो गङ्गाप्रभावोऽयं त्रैलोक्ये नोपमीयते}
{वयं धन्या यतो गङ्गा दृष्टास्माभिस्त्रिपावनी} %॥१९२॥

\uvacha{ब्रह्मोवाच}


\twolineshloka
{हर्षेण महता युक्तो देवं स्थाप्य महेश्वरम्}
{तं षोडशभिरीशानमुपचारैः प्रयत्नतः} %॥१९३॥

\twolineshloka
{सम्पूज्यावरणैर्युक्तं षट्त्रिंशत्कलमीश्वरम्}
{कृताञ्जलिपुटो भूत्वा रामस्तुष्टाव शङ्करम्} %॥१९४॥

\uvacha{राम उवाच}


\fourlineindentedshloka
{नमामि शम्भुं पुरुषं पुराणं}
{नमामि सर्वज्ञमपारभावम्} 
{नमामि रुद्रं प्रभुमक्षयं तं} 
{नमामि शर्वं शिरसा नमामि}% १९५

\fourlineindentedshloka
{नमामि देवं परमव्ययं तम्} 
{उमापतिं लोकगुरुं नमामि}
{नमामि दारिद्र्यविदारणं तं}
{नमामि रोगापहरं नमामि}% १९६

\fourlineindentedshloka
{नमामि कल्याणमचिन्त्यरूपं}
{नमामि विश्वोद्भवबीजरूपम्}
{नमामि विश्वस्थितिकारणं तं}
{नमामि संहारकरं नमामि}% १९७

\fourlineindentedshloka
{नमामि गौरीप्रियमव्ययं तं}
{नमामि नित्यं क्षरमक्षरं तम्}
{नमामि चिद्रूपममेयभावं}
{त्रिलोचनं तं शिरसा नमामि}% १९८

\fourlineindentedshloka
{नमामि कारुण्यकरं भवस्य}
{भयङ्करं वापि सदा नमामि}
{नमामि दातारमभीप्सितानां}
{नमामि सोमेशमुमेशमादौ}% १९९

\fourlineindentedshloka
{नमामि वेदत्रयलोचनं तं}
{नमामि मूर्तित्रयवर्जितं तम्}
{नमामि पुण्यं सदसद्व्यतीतं}
{नमामि तं पापहरं नमामि}% २००

\fourlineindentedshloka
{नमामि विश्वस्य हिते रतं तं}
{नमामि रूपाणि बहूनि धत्ते}
{यो विश्वगोप्ता सदसत्प्रणेता}
{नमामि तं विश्वपतिं नमामि}% २०१

\fourlineindentedshloka
{यज्ञेश्वरं सम्प्रति हव्यकव्यं}
{तथा गतिं लोकसदाशिवो यः}
{आराधितो यश्च ददाति सर्वं}
{नमामि दानप्रियमिष्टदेवम्}% २०२

\fourlineindentedshloka
{नमामि सोमेश्वरमस्वतन्त्रम्}
{उमापतिं तं विजयं नमामि}
{नमामि विघ्नेश्वरनन्दिनाथं}
{पुत्रप्रियं तं शिरसा नमामि}% २०३

\fourlineindentedshloka
{नमामि देवं भवदुःखशोक}
{विनाशनं चन्द्रधरं नमामि}
{नमामि गङ्गाधरमीशमीड्यम्}
{उमाधवं देववरं नमामि}% २०४

\fourlineindentedshloka
{नमाम्यजादीशपुरन्दरादि}
{सुरासुरैरर्चितपादपद्मम्}
{नमामि देवीमुखवादनानाम्}
{ईक्षार्थमक्षित्रितयं य ऐच्छत्}% २०५

\fourlineindentedshloka
{पञ्चामृतैर्गन्धसुधूपदीपैर्}
{विचित्रपुष्पैर्विविधैश्च मन्त्रैः}
{अन्नप्रकारैः सकलोपचारैः}
{सम्पूजितं सोममहं नमामि}% २०६

\uvacha{ब्रह्मोवाच}


\twolineshloka
{ततः स भगवानाह रामं शम्भुः सलक्ष्मणम्}
{वरान् वृणीष्व भद्रं ते रामः प्राह वृषध्वजम्} %॥२०७॥

\uvacha{राम उवाच}


\twolineshloka
{स्तोत्रेणानेन ये भक्त्या तोष्यन्ति त्वां सुरोत्तम}
{तेषां सर्वाणि कार्याणि सिद्धिं यान्तु महेश्वर} %॥२०८॥

\twolineshloka
{येषां च पितरः शम्भो पतिता नरकार्णवे}
{तेषां पिण्डादिदानेन पूता यान्तु त्रिविष्टपम्} %॥२०९॥

\twolineshloka
{जन्मप्रभृति पापानि मनोवाक्कायिकं त्वघम्}
{अत्र तु स्नानमात्रेण तत्सद्यो नाशमाप्नुयात्} %॥२१०॥

\twolineshloka
{अत्र ये भक्तितः शम्भो ददत्यर्थिभ्य अण्वपि}
{सर्वं तदक्षयं शम्भो दातॄणां फलकृद्भवेत्} %॥२११॥

\uvacha{ब्रह्मोवाच}


\twolineshloka
{एवमस्त्विति तं रामं शङ्करो हृषितोऽब्रवीत्}
{गते तस्मिन् सुरश्रेष्ठे रामोऽप्यनुचरैः सह} %॥२१२॥

\twolineshloka
{गौतमी यत्र चोत्पन्ना शनैस्तं देशमभ्यगात्}
{ततः प्रभृति तत्तीर्थं रामतीर्थमुदाहृतम्} %॥२१३॥

\twolineshloka
{दयालोरपतत्तत्र लक्ष्मणस्य कराच्छरः}
{तद्बाणतीर्थमभवत्सर्वापद्विनिवारणम्} %॥२१४॥

\twolineshloka
{यत्र सौमित्रिणा स्नानं शङ्करस्यार्चनं कृतम्}
{तत्तीर्थं लक्ष्मणं जातं तथा सीतासमुद्भवम्} %॥२१५॥

\twolineshloka
{नानाविधाशेषपाप सङ्घनिर्मूलनक्षमम्}
{यदङ्घ्रिसङ्गादभवद्गङ्गा त्रैलोक्यपावनी} %॥२१६॥

\twolineshloka
{स यत्र स्नानमकरोत्तद्वैशिष्ट्यं किमुच्यते}
{तद्रामतीर्थसदृशं तीर्थं क्वापि न विद्यते} %॥२१७॥

॥इति श्रीमहापुराणे आदिब्राह्मे तीर्थमाहात्म्ये रामतीर्थवर्णनं नाम त्रयोविंशत्यधिकशततमोऽध्यायः॥१२३॥

    \sect{सहस्रकुण्डाख्यतीर्थवर्णनम्}

\src{ब्रह्म-पुराणम्}{गौतमीमाहात्म्यम्}{अध्यायः १५४}{}
% \tags{concise, complete}
\notes{This chapter describes the significance of the Sahasrakunda Tirtha, where Lord Rama performed rituals and established a sacred site after defeating Ravana. It also narrates the events leading to the establishment of this Tirtha, including Rama's return to Ayodhya with Sita.}
\textlink{https://sa.wikisource.org/wiki/ब्रह्मपुराणम्/अध्यायः_१५४}
\translink{}

\storymeta

\uvacha{ब्रह्मोवाच}

\twolineshloka
{सहस्रकुण्डमाख्यातं तीर्थं वेदविदो विदुः}
{यस्य स्मरणमात्रेण सुखी सम्पद्यते नरः} %॥१॥

\twolineshloka
{पुरा दाशरथी रामः सेतुं बद्‌ध्वा महार्णवे}
{लङ्कां दग्ध्वा रिपून्हत्वा रावणादीन्रणे शरैः} %॥२॥

\twolineshloka
{वैदेहीं च समासाद्य रामो वचनमब्रवीत्}
{पश्यत्सु लोकपालेषु तस्याऽऽचार्ये पुरः स्थिते} %॥३॥

\twolineshloka
{अग्नौ शुद्धिगतां सीतां रामो लक्ष्मणसन्निधौ}
{एहि वैदेहि शुद्धऽसि अङ्कमारोढुमर्हसि} %॥४॥

\twolineshloka
{नेत्युवाच तदा श्रीमानङ्गदो हनुमांस्तथा}
{अयोध्यायां तु वैदेहि सार्धं यामः सुहृज्जनैः} %॥५॥

\twolineshloka
{तत्र शुद्धिमवाप्याथ पुनर्भातृषु मातृषु}
{लौकिकेष्वपि पश्यत्सु ततः शुद्धा नृपात्मजा} %॥६॥

\twolineshloka
{अयोध्यायां सुपुण्येऽह्नि अङ्कमारोढुमर्हंसि}
{अस्याश्चरित्रविषये सन्देहः कस्य जायते} %॥७॥

\twolineshloka
{लोकापवादस्तदपि निरस्यः स्वजनेषु हि}
{तयोर्वाक्यमनादृत्य लक्ष्मणः सविभीषणः} %॥८॥

\twolineshloka
{रामश्च जाम्बवांश्चैव तामाह्वयन्नृपात्मजाम्}
{स्वस्तीत्युक्ता देवताभी राज्ञोङ्कं चाऽऽरुरोह सा} %॥९॥

\twolineshloka
{मुदतिस्ते ययुः शीघ्रं पुष्पकेण विराजता}
{अयोध्यां नगरीं प्राप्य तथा राज्यं स्वकं तु यत्} %॥१०॥

\twolineshloka
{मुदितास्तेऽभवन्सर्वे सदा रामानुवर्तिनः}
{ततः कतिपयाहेषु अनार्येभ्यो विरूपिकाम्} %॥११॥

\twolineshloka
{वाचं श्रुत्वा स तत्याज गुर्विणीं तामयोनिजाम्}
{मिथ्यापवादमपि हि न सहन्ते कुलोन्नताः} %॥१२॥

\twolineshloka
{वाल्मीकेर्मुनिमुख्यस्य आश्रमस्य समीपतः}
{तत्याज लक्ष्मणः सीतामदुष्टां रुदतीं रुदन्} %॥१३॥

\twolineshloka
{नोल्लङ्घ्याऽऽज्ञा गुरूणामित्यसौ तदकरोद्भिया}
{ततः कतिपयाहेतु व्यतीतेषु नृपात्मजः} %॥१४॥

\twolineshloka
{रामः सौमित्रिणा सार्धं हयमेधाय दीक्षितः}
{तत्रैवाऽऽजग्मतुरुभौ रामपुत्रौ यशस्विनौ} %॥१५॥

\twolineshloka
{लवः कुशश्च विख्यातौ नारदाविव गायकौ}
{रामायणं समग्रं तद्‌गन्धर्वाविव सुस्वरौ} %॥१६॥

\twolineshloka
{रामाय चरितं सर्वं गायमानौ समीयतुः}
{यज्ञवाटं राजसुतौ हेतुभिर्लक्षितौ तदा} %॥१७॥

\twolineshloka
{रामपुत्रावुभौ शूरौ वैदेह्यास्तनयाविति}
{तावानीय ततः पुत्रावभिषच्य यथाक्रमम्} %॥१८॥

\twolineshloka
{अङ्कारूढौ ततः कृत्वा सस्वजे तौ पुनः पुनः}
{संसारदुःखिन्नानामगतीनां शरीरिणाम्} %॥१९॥

\twolineshloka
{पुत्रालिङ्गनमेवात्र परं विश्रान्तिकारणम्}
{मुहुरालिङ्ग्य तौ पुत्रौ मुहुः स्वजति चुम्बति} %॥२०॥

\twolineshloka
{किमप्यन्तर्ध्याति च निःश्वसत्यपि वै मुहुः}
{एतस्मिन्नन्तरे प्राप्ता राक्षसा लङ्कवासिनः} %॥२१॥

\twolineshloka
{सुग्रीवो हनुमांश्चैव अङ्गदो जाम्बवांस्तथा}
{अन्ये च वानराः सर्वे विभीषणपुरः सराः} %॥२२॥

\twolineshloka
{ते चाऽऽगत्य नृपं प्राप्ताः सिंहासनमुपस्थितम्}
{सीतामदृष्ट्वा हनुमानङ्गदः कनकाङ्गदः} %॥२३॥

\twolineshloka
{क्व गताऽयोनिजा माता एको रामोऽत्र दृश्यते}
{रामेण सा परित्यक्ता इत्यूचुर्द्वारपालकाः} %॥२४॥

\twolineshloka
{पश्यत्सु लोकपालेषु आर्ये तत्र प्रवादिनि}
{अग्नौ शुद्धिगतां(ता)सीतां(ता)किन्तु राजा निरङ्कुशः} %॥२५॥

\twolineshloka
{उत्पन्नैर्लौकिकैर्वाक्यै रामस्त्यजति तां प्रियाम्}
{मरिष्याव इति ह्युक्त्वा गौतमीं पुनरीयतुः} %॥२६॥

\twolineshloka
{रामस्तौ पृष्ठतोऽभ्येत्य(?)अयोध्यावासिभिः सह}
{आगत्य गौतमीं तत्राकुर्वंस्त परमं तपः} %॥२७॥

\twolineshloka
{स्मारं स्मारं निश्वसन्तस्तां सीतां लोकमातरम्}
{संसारास्थाविरहिता गौतमीसेवनोत्सुकाः} %॥२८॥

\twolineshloka
{लोकत्रयपतिः साक्षाद्रामोऽनुजसमन्वितः}
{प्राप्तं स्नात्वा च गौतम्यां शिवाराधनतत्परः} %॥२९॥

\twolineshloka
{परितापं हजौ सर्वं सहस्रपरिवारितः}
{यत्र चाऽऽसीत्स वृत्तान्तः सहस्रकुण्डमुच्यते} %॥३०॥

\twolineshloka
{दशापराणि तीर्थानि तत्र सर्वार्थदानि च}
{तत्र स्नानं च दानं च सहस्रफलदायकम्} %॥३१॥

\twolineshloka
{यत्र श्रीगौतमीतीरे वसिष्ठादिमुनीश्वरैः}
{सर्वापत्तारकं होममकारयदघान्तकम्} %॥३२॥

\twolineshloka
{सहस्रसङ्ख्यायुक्तेषु कुण्डेषु वसुधारया}
{सर्वानपेक्षितान्कामानवापासौ महातपाः} %॥३३॥

\twolineshloka
{गौतम्याः सरिदम्बायाः प्रसादाद्राक्षसान्तकः}
{सहस्रकुण्डाभिधं तदभूत्तीर्थं महाफलम्} %॥३४॥

॥इति श्रीमहापुराणे आदिब्राह्मे तीर्थमाहात्म्ये सहस्रकुण्डादिदशतीर्थवर्णनं नाम चतुष्पञ्चाशदधिकशततमोऽध्यायः॥१५४॥

॥गौतमीमाहात्म्ये पञ्चाशीतितमोऽध्यायः॥८५॥
    \chapt{ब्रह्मवैवर्त-पुराणम्}

\sect{वेदवती-प्रस्तावः}

\src{ब्रह्मवैवर्त-पुराणम्}{खण्डः २ (प्रकृतिखण्डः)}{अध्यायः १४}{}
% \tags{concise, complete}
\notes{This chapter briefly describes the stories of Vedavati, Sita and Draupadi, across the three yugas.}
\textlink{https://sa.wikisource.org/wiki/ब्रह्मवैवर्तपुराणम्/खण्डः_२_(प्रकृतिखण्डः)/अध्यायः_१४}
\translink{https://archive.org/details/brahma-vaivarta-purana-all-four-kandas-english-translation/page/n255/mode/2up}

\storymeta


\uvacha{नारायण उवाच}

\twolineshloka
{लक्ष्मीं तौ च समाराध्य चोग्रेण तपसा मुने}
{प्रत्येकं वरमिष्टं च सम्प्रापतुरभीप्सितम्}% ।। १ ।।

\twolineshloka
{महालक्ष्म्या वरेणैव तौ पृध्वीशौ बभूवतुः}
{धनवन्तौ पुत्रवन्तौ धर्मध्वजकुशध्वजौ}% ।। २ ।।

\twolineshloka
{कुशध्वजस्य पत्नी च देवी मालावती सती}
{सा सुषाव च कालेन कमलांशां सुतां सतीम्}% ।। ३ ।।

\twolineshloka
{सा च भूतलसम्बन्धाज्ज्ञानयुक्ता बभूव ह}
{कृत्वा वेदध्वनिं स्पष्टमुत्तस्थौ सूतिकागृहे।}% ।। ४ ।।

\twolineshloka
{वेदध्वनिं सा चकार जातमात्रेण कन्यका}
{तस्मात्तां ते वेदवतीं प्रवदन्ति मनीषिणः}% ।। ५ ।।

\twolineshloka
{जातमात्रेण सुस्नाता जगाम तपसे वनम्}
{सर्वैर्निषिद्धा यत्नेन नारायणपरायणा}% ।। ६ ।।

\twolineshloka
{एकमन्वन्तरं चैव पुष्करे च तपस्विनी}
{अत्युग्रां वै तपस्यां तु लीलया च चकार सा}% ।। ७ ।।

\twolineshloka
{तथाऽपि पुष्टा न कृशा नवयौवनसंयुता}
{शुश्राव खे च सहसा सा वाचमशरीरिणीम्}% ।। ।। ८ ।।

\twolineshloka
{जन्मान्तरे ते भर्त्ता च भविष्यति हरिः स्वयम्}
{ब्रह्मादिभिर्दुराराध्यं पतिं लप्स्यसि सुन्दरि}% ।। ९ ।।

\twolineshloka
{इति श्रुत्वा तु सा रुष्टा चकार च पुनस्तपः}
{अतीव निर्जनस्थाने पर्वते गन्धमादने}% ।। 2.14.१० ।।

\twolineshloka
{तत्रैवं सुचिरं तप्त्वा विश्वस्य समुवास सा}
{ददर्श पुरतस्तत्र रावणं दुर्निवारणम्}% ।। ११ ।।

\twolineshloka
{दृष्ट्वा साऽतिथिभक्त्या च पाद्यं तस्मै ददौ किल}
{सुस्वादु फलमूलं च जलं चापि सुशीतलम्}% ।। १२ ।।

\twolineshloka
{तच्च भुक्त्वा स पापिष्ठश्चावात्सीत्तत्समीपतः}
{चकार प्रश्नमिति तां का त्वं कल्याणि चेति च}% ।। १३ ।।

\twolineshloka
{तां च दृष्ट्वा वरारोहां पीनोन्नतपयोधराम्}
{शरत्पद्मनिभास्यां च सस्मितां सुदतीं सतीम्}% ।। १४ ।।

\twolineshloka
{मूर्च्छामवाप कृपणः कामबाणप्रपीडितः}
{तां करेण समाकृष्य सम्भोगं कर्तुमुद्यतः}% ।। १५ ।।

\twolineshloka
{सा सती कोपदृष्ट्या च स्तम्भितं तं चकार ह}
{स जडो हस्तपादैश्च किञ्चिद्वक्तुं न च क्षमः}% ।। १६ ।।

\twolineshloka
{तुष्टाव मनसा देवीं पद्मांशां पद्मलोचनाम्}
{सा तत्स्तवेन सन्तुष्टा प्राकृतं तं मुमोच ह}% ।। १७ ।।

\twolineshloka
{शशाप च मदर्थे त्वं विनश्यसि सबान्धवः}
{स्पृष्टाऽहं च त्वया कामाद्विसृजाम्यवलोकय}% ।। १८ ।।

\twolineshloka
{इत्युक्त्वा सा च योगेन देहत्यागं चकार ह}
{गङ्गायां तां च सन्न्यस्य स्व गृहं रावणो ययौ}% ।। १९ ।।

\twolineshloka
{अहो किमद्भुतं दृष्टं किं कृतं वा मयाऽधुना}
{इति सञ्चिन्त्य संस्मृत्य विललाप पुनः पुनः}% ।। ।। 2.14.२० ।।

\twolineshloka
{सा च कालान्तरे साध्वी बभूव जनकात्मजा}
{सीतादेवीति विख्याता यदर्थे रावणो हतः}% ।। २१ ।।

\twolineshloka
{महातपस्विनी सा च तपसा पूर्वजन्मनः}
{लेभे रामं च भर्त्तारं परिपूर्णतमं हरिम्}% ।। २२ ।।

\twolineshloka
{सम्प्राप्य तपसाऽऽराध्य स्वामिनं च जगत्पतिम्}
{सा रमा सुचिरं रेमे रामेण सह सुन्दरी}% ।। २३ ।।

\twolineshloka
{जातिस्मरा स्म स्मरति तपसश्च क्रमं पुरा}
{सुखेन तज्जहौ सर्वं दुःखं चापि सुखं लभेत्}% ।। ।। २४ ।।

\twolineshloka
{नानाप्रकारविभवं चकार सुचिरं सती}
{सम्प्राप्य सुकुमारं तमतीव नवयौवनम्}% ।। २५ ।।

\twolineshloka
{गुणिनं रसिकं शान्तं कान्तवेषमनुत्तमम्}
{स्त्रीणां मनोज्ञं रुचिरं तथा लेभे यथेप्सितम्}% ।। २६ ।।

\twolineshloka
{पितुर्वचःपालनार्थं सत्यसन्धो रघूत्तमः}
{जगाम काननं पश्चात्कालेन च बलीयसा}% ।। २७ ।।

\twolineshloka
{तस्थौ समुद्रनिकटे सीतया लक्ष्मणेन च}
{ददर्श तत्र वह्निं च विप्ररूपधरं हरिः}% ।। २८ ।।

\twolineshloka
{तं रामं दुःखितं दृष्ट्वा स च दुःखी बभूव ह}
{उवाच किञ्चित्सत्येष्टं सत्यं सत्यपरायणः}% ।। २९ ।।
वह्निरुवाच ।।

\twolineshloka
{भगवञ्छ्रूयतां वाक्यं कालेन यदुपस्थितम्}
{सीताहरणकालोऽयं तवैव समुपस्थितः}% ।। 2.14.३० ।।

\twolineshloka
{दैवं च दुर्निवार्य्यं वै न च दैवात्परं बलम्}
{मत्प्रसू मयि सन्न्यस्य च्छायां रक्षान्तिकेऽधुना}% ।। ३१ ।।

\twolineshloka
{दास्यामि सीतां तुभ्यं च परीक्षासमये पुनः}
{देवैः प्रस्थापितोऽहं च न च विप्रो हुताशनः}% ।। ३२ ।।

\twolineshloka
{रामस्तद्वचनं श्रुत्वा न प्रकाश्य च लक्ष्मणम्}
{स्वच्छन्दं स्वीचकारासौ हृदयेन विदूयता}% ।। ३३ ।।

\twolineshloka
{वह्निर्योगेन सीतावन्मायासीतां चकार ह}
{तत्तुल्यगुणरूपां तां ददौ रामाय नारद}% ।। ३४ ।।

\twolineshloka
{सीतां गृहीत्वा स ययौ गोप्यं वक्तुं निषेध्य च}
{लक्ष्मणो नैव बुबुधे गोप्यमन्यस्य का कथा}% ।। ३९ ।।

\twolineshloka
{एतस्मिन्नन्तरे रामो ददर्श कनकं मृगम्}
{सीता तं प्रेरयामास तदर्थे यत्नपूर्वकम्}% ।। ३६ ।।

\twolineshloka
{सन्न्यस्य लक्ष्मणं रामो जानक्या रक्षणे वने}
{स्वयं जगाम हन्तुं तं विव्यधे सायकेन च}% ।। ३७ ।।

\twolineshloka
{लक्ष्मणेति च शब्दं वै कृत्वा मायामृगस्तदा}
{प्राणांस्तत्याज सहसा पुरो दृष्ट्वा हरिं स्मरन्}% ।। ३८ ।।

\twolineshloka
{मृगरूपं परित्यज्य दिव्यरूपं विधाय च}
{रत्ननिर्मितयानेन वैकुण्ठं स जगाम ह}% ।। ३९ ।।

\twolineshloka
{वैकुण्ठस्य महाद्वारं किङ्करो द्वारपालयोः}
{जय विजययोश्चैव बलवांश्च जयाभिधः}% ।। 2.14.४० ।।

\twolineshloka
{शापेन सनकादीनां सम्प्राप्तो राक्षसीं तनुम्}
{पुनर्जगाम तद्द्वारमादौ स द्वारपालयोः}% ।। ४१ ।।

\twolineshloka
{अथ शब्दं च सा श्रुत्वा लक्ष्मणेति च विक्लवम्}
{सीता तं प्रेरयामास लक्ष्मणं रामसन्निधौ}% ।। ४२ ।।

\twolineshloka
{गते च लक्ष्मणे रामं रावणो दुर्निवारणः}
{सीतां गृहीत्वा प्रययौ लङ्कामेव स्वलीलया}% ।। ४३ ।।

\twolineshloka
{विषसाद च रामश्च वने दृष्ट्वा च लक्ष्मणम्}
{तूर्णं च स्वाश्रमं गत्वा सीतां नैव ददर्श सः}% ।। ४४ ।।

\twolineshloka
{मूर्च्छां सम्प्राप्य सुचिरं विललाप भृशं पुनः}
{पुनर्बभ्राम गहने तदन्वेषणपूर्वकम्}% ।। ४५ ।।

\twolineshloka
{काले सम्प्राप्य तद्वार्तां गृधद्वारा नदीतटे}
{सहायं वानरं कृत्वा चाबध्नात्सागरं हरिः}% ।। ।। ४६ ।।

\twolineshloka
{लङ्कां गत्वा रघुश्रेष्ठश्चावधीत्सायकेन च}
{सबान्धवं रावणं च सीतां सम्प्राप दुःखिताम्}% ।। ४७ ।।

\twolineshloka
{तां च वह्निपरीक्षां व कारयामास सत्वरम्}
{हुताशनस्तत्र काले वास्तवीं जानकीं ददौ}% ।। ४८ ।।

\twolineshloka
{छाया चोवाच वह्निं च रामं च विनयान्विता}
{करिष्यामीति किमहं तदुपायं वदस्व मे}% ।। ४९ ।।
वह्निरुवाच ।।

\twolineshloka
{त्वं गच्छ तपसे देवि पुष्करं च सुपुण्यदम्}
{कृत्वा तपस्यां तत्रैव स्वर्गलक्ष्मीर्भविष्यसि}% ।। 2.14.५० ।।

\twolineshloka
{सा च तद्वचनं श्रुत्वा प्रतेपे पुष्करे तपः}
{दिव्यं त्रिलक्षवर्ष च स्वर्गे लक्ष्मीर्बभूव ह}% ।। ५१ ।।

\twolineshloka
{सा च कालेन तपसा यज्ञकुण्डसमुद्भवा}
{कामिनी पाण्डवानां च द्रौपदी द्रुपदात्मजा}% ।। ५२ ।।

\twolineshloka
{कृते युगे वेदवती कुशध्वजसुता शुभा}
{त्रेतायां रामपत्नी च सीतेति जनकात्मजा}% ।। ५३ ।।

\twolineshloka
{तच्छाया द्रौपदी देवी द्वापरे द्रुपदात्मजा}
{त्रिहायणीति सा प्रोक्ता विद्यमाना युगत्रये}% ।। ५४ ।।

\uvacha{नारद उवाच}

\twolineshloka
{प्रियाः पञ्च कथं तस्या बभूवुर्मुनिपुङ्गव}
{इति वै चित्तसन्देहं दूरीकुरु महाप्रभो}% ।। ५५ ।।

\uvacha{नारायण उवाच}

\twolineshloka
{लङ्कायां वस्तुतः सीता रामं सम्प्राप नारद}
{रूपयौवनसम्पन्ना छाया सा बहुविह्वला}% ।। ५६ ।।

\twolineshloka
{रामाग्न्योराज्ञया तप्त्वा ययाचे शङ्करं वरम्}
{कामातुरा पतिव्यग्रा प्रार्थयन्ती पुनः पुनः}% ।। ५७ ।।

\twolineshloka
{पतिं देहि पतिं देहि पतिं देहि त्रिलोचन}
{पतिं देहि पतिं देहि पञ्चवारं पतिव्रता}% ।। ५८ ।।

\twolineshloka
{शिवस्तत्प्रार्थनां श्रुत्वा सस्मितो रसिकेश्वरः}
{प्रिये तव प्रियाः पञ्च भवन्तीति वरं ददौ}% ।। ५९ ।।

\twolineshloka
{तेनासीत्पाण्डवानां च पञ्चानां कामिनी प्रिया}
{इत्येवं कथितं सर्वं प्रस्तुतं वस्तुतः शृणु}% ।। 2.14.६० ।।

\twolineshloka
{अथ सम्प्राप्य लङ्कायां सीतां रामो मनोहराम्}
{विभीषणाय तां लङ्कां दत्त्वाऽयोध्यां ययौ पुनः}% ।। ६१ ।।

\twolineshloka
{एकादशसहस्राब्दं कृत्वा राज्यं च भारते}
{जगाम सर्वैर्लोकैश्च सार्द्धं वैकुण्ठमेव च}% ।। ६२ ।।

\twolineshloka
{कमलांशा वेदवती कमलायां विवेश सा}
{कथितं पुण्यमाख्यानं पुण्यदं पापनाशनम्}% ।। ६३ ।।

\twolineshloka
{सततं मूर्तिमन्तश्च वेदाश्चत्वार एव च}
{सन्ति यस्याश्च जिह्वाग्रे सा च वेदवती स्मृता}% ।। ६४ ।।

\twolineshloka
{कुशध्वजसुताख्यानमुक्तं सङ्क्षेपतस्तव}
{धर्मध्वजसुताख्यानं निबोध कथयामि ते}% ।। ६५ ।।

॥इति श्रीब्रह्मवैवर्त्ते महापुराणे द्वितीये प्रकृतिखण्डे नारदनारायणसंवादे तुलस्युपाख्याने वेदवतीप्रस्तावे चतुर्दशोऽध्यायः॥१४॥

    \chapt{पद्म-पुराणम्}

\sect{मार्कण्डेयाश्रमदर्शनम्}

\src{पद्म-पुराणम्}{सृष्टिखण्डम्}{अध्यायः ३३}{१--१८५}
% \tags{concise, complete}
\notes{This chapter describes the visit of Rama to the Markandeya Ashrama, the story of Markandeya, and the significance of the Pushkara Tirtha. It also includes a most interesting episode of Rama performing Shrāddha, and Sita actually seeing Dasharatha descended in the Brāhmaṇas.}
\textlink{https://sa.wikisource.org/wiki/पद्मपुराणम्/खण्डः_१_(सृष्टिखण्डम्)/अध्यायः_३३}
\translink{https://www.wisdomlib.org/hinduism/book/the-padma-purana/d/doc364155.html}

\storymeta


\uvacha{भीष्म उवाच}

\twolineshloka
{मार्कण्डेयेन वै रामः कथमत्र प्रबोधितः}
{कथं समागमो भूतः कस्मिन्काले कदा मुने} %॥१।

\twolineshloka
{मार्कण्डेयः कस्य सुतः कथं जातो महातपाः}
{नाम्नोऽस्य निगमं ब्रूहि यथाभूतं महामुने} %॥२।
\uvacha{पुलस्त्य उवाच}

\twolineshloka
{अथ ते सम्प्रवक्ष्यामि मार्कण्डेयोद्भवं पुनः}
{पुराकल्पे मुनिः पूर्वं मृकण्डुर्नाम विश्रुतः} %॥३।

\twolineshloka
{भृगोः पुत्रो महाभागः सभार्यस्तप्तवांस्तपः}
{तस्य पुत्रस्तदा जातो वसतस्तु वनान्तरे} %॥४।

\twolineshloka
{सपञ्चवार्षिको भूतो बाल एव गुणाधिकः}
{ज्ञानिना स तदा दृष्टो भ्रमन्बालस्तदाङ्गणे} %॥५।

\twolineshloka
{स्थित्वा स सुचिरं कालं भाव्यर्थं प्रत्यबुध्यत}
{तस्य पित्रा स वै पृष्टः कियदायुः सुतस्य मे} %॥६।

\twolineshloka
{सङ्ख्यायाचक्ष्व वर्षाणि तस्याल्पान्यधिकानि वा}
{मृकण्डुनैवमुक्तस्तु स ज्ञानी वाक्यमब्रवीत्} %॥७।

\twolineshloka
{षण्मासमायुः पुत्रस्य धात्रा सृष्टं मुनीश्वर}
{नैव शोकस्त्वया कार्यः सत्यमेतदुदाहृतम्} %॥८।

\twolineshloka
{स तच्छ्रुत्वा वचो भीष्म ज्ञानिना यदुदाहृतम्}
{अथोपनयनं चक्रे बालकस्य पिता तदा} %॥९।

\twolineshloka
{आह चैनं पितापुत्रमृषींस्त्वमभिवादय}
{एवमुक्तः स वै पित्रा प्रहृष्टश्चाभिवादने} %॥१०।

\twolineshloka
{न वर्णा वर्णतां वेत्ति सर्ववर्णाभिवादनः}
{पञ्चमासास्त्वतिक्रान्ता दिवसाः पञ्चविंशतिः} %॥११।

\twolineshloka
{मार्गेणाथ समायाता ऋषयस्तत्र सप्त वै}
{बालेन तेन ते दृष्टाः सर्वे चाप्यभिवादिताः} %॥१२।

\twolineshloka
{आयुष्मान्भव तैरुक्तः स बालो दण्डमेखली}
{उक्त्वैवं ते पुनर्बालमपश्यन्क्षीणजीवितम्} %॥१३।

\twolineshloka
{दिनानि पञ्च तस्यायुर्ज्ञात्वा भीताश्च ते नृप}
{तं गृहीत्वा बालकं च गतास्ते ब्रह्मणोन्तिकम्} %॥१४।

\twolineshloka
{प्रतिमुच्य च तं राजन्प्रणिपेतुः पितामहम्}
{अयमावेदितस्तैस्तु तेन ब्रह्माभिवादितः} %॥१५।

\twolineshloka
{चिरायुर्ब्रह्मणा बालः प्रोक्तः स ऋषिसन्निधौ}
{ततस्ते मुनयः प्रीताः श्रुत्वा वाक्यं पितामहात्} %॥१६।

\twolineshloka
{पितामह ऋषीन्दृष्ट्वा प्रोवाच विस्मयान्वितः}
{कार्येण येन चायातः कोयं बालो निवेद्यताम्} %॥१७।

\twolineshloka
{ततस्त ऋषयो राजन्सर्वं तस्मै न्यवेदयन्}
{पुत्रो मृकण्डोः क्षीणायुः सायुषं कुरु बालकम्} %॥१८।

\twolineshloka
{अल्पायुषस्त्वस्य मुनिर्बध्वेमां चापि मेखलाम्}
{यज्ञोपवीतं दण्डं च दत्वा चैनमबोधयत्} %॥१९।

\twolineshloka
{यं कञ्चित्पश्यसे बाल भ्रमन्तं भूतले जनम्}
{तस्याभिवादः कर्तव्य एवमाह पिता वचः} %॥२०।

\twolineshloka
{अभिवादनशीलोयं क्षितौ दृष्टः परिभ्रमन्}
{तीर्थयात्राप्रसङ्गेन दैवयोगात्पितामह} %॥२१।

\twolineshloka
{चिरायुर्भव पुत्रेति प्रोक्तोसौ तत्र बालकः}
{कथं वचो भवेत्सत्यमस्माकं भवता सह} %॥२२।

\twolineshloka
{एवमुक्तस्तदा तैस्तु ब्रह्मा लोकपितामहः}
{ऋतवाक्यादियं भूमिः संस्थिता सर्वतोभया} %॥२३।


\uvacha{ब्रह्मोवाच}

\twolineshloka
{मत्समश्चायुषा बालो मार्कण्डेयो भविष्यति}
{कल्पस्यादौ तथाचान्ते मतो मे मुनिसत्तमः} %॥२४।

\twolineshloka
{एवं ते मुनयो बालं ब्रह्मलोके पितामहात्}
{संसाध्य प्रेषयामासुर्भूयोप्येनं धरातलम्} %॥२५।

\twolineshloka
{तीर्थयात्रां गता विप्रा मार्कण्डेयो निजं गृहम्}
{जगाम तेषु यातेषु पितरं स्वमथाब्रवीत्} %॥२६।

\twolineshloka
{ब्रह्मलोकमहं नीतो मुनिभिर्ब्रह्मवादिभिः}
{दीर्घायुश्च कृतश्चास्मि वरान्दत्वा विसर्जितः} %॥२७।

\twolineshloka
{एतदन्यच्च मे दत्तं गतं चिन्ताकरं तव}
{कल्पस्यादौ तथा चान्ते भविष्ये समनन्तरे} %॥२८।

\twolineshloka
{लोककर्तुर्ब्रह्मणोहं प्रसादात्तस्य वै पितः}
{पुष्करं वै गमिष्यामि तपस्तप्तुं समुद्यतः} %॥२९।

\twolineshloka
{तत्राहं देवदेवेशमुपासिष्ये पितामहम्}
{सर्वकामावाप्तिकरं सर्वारातिनिबर्हणम्} %॥३०।

\twolineshloka
{सर्वसौख्यप्रदं देवमिन्द्रादीनां परायणम्}
{ब्रह्माणं तोषयिष्यामि सर्वलोकपितामहम्} %॥३१।

\twolineshloka
{मार्कण्डेयवचः श्रुत्वा मृकण्डुर्मुनिसत्तमः}
{जगाम परमं हर्षं क्षणमेकं समुच्छ्वसन्} %॥३२।

\twolineshloka
{धैर्यं सुमनसा स्थाय इदं वचनमब्रवीत्}
{अद्य मे सफलं जन्म जीवितं च सुजीवितम्} %॥३३।

\twolineshloka
{सर्वस्य जगतां स्रष्टा येन दृष्टः पितामहः}
{त्वया दायादवानस्मि पुत्रेण वंशधारिणा} %॥३४।

\twolineshloka
{त्वं गच्छ पश्य देवेशं पुष्करस्थं पितामहम्}
{दृष्टे तस्मिन्जगन्नाथे न जरामृत्युरेव च} %॥३५।

\twolineshloka
{नृणां भवति सौख्यानि तथैश्वर्यं तपोऽक्षयम्}
{त्रीणि शृङ्गाणि शुभ्राणि त्रीणि प्रस्रवणानि च} %॥३६।

\twolineshloka
{पुष्कराणि तथा त्रीणि नविद्मस्तत्र कारणम्}
{कनीयांसं मध्यमं च तृतीयं ज्येष्ठपुष्करम्} %॥३७।

\twolineshloka
{शृङ्गशब्दाभिधानानि शुभप्रस्रवणानि च}
{ब्रह्माविष्णुस्तथा रुद्रो नित्यं सन्निहितास्त्रयः} %॥३८।

\twolineshloka
{पुष्करेषु महाराजा नातः पुण्यतमं भुवि}
{विरजं विमलं तोयं त्रिषु लोकेषु विश्रुतम्} %॥३९।

\twolineshloka
{ब्रह्मलोकस्य पन्थानं धन्याः पश्यन्ति पुष्करं}
{यस्तु वर्षशतं साग्रमग्निहोत्रमुपासते} %॥४०।

\twolineshloka
{कार्तिकीं वा वसेदेकां पुष्करे सममेव च}
{कर्तुम्मया न शकितं कर्मणा नैव साधितम्} %॥४१।

\twolineshloka
{तदयत्नात्त्वया तात मृत्युस्सर्वहरो जितः}
{तत्र दृष्टस्स देवेशो ब्रह्मा लोकपितामहः} %॥४२।

\twolineshloka
{नान्यो मर्त्यस्त्वया तुल्यो भविता जगतीतले}
{अहं वै तोषितो येन पञ्चवार्षिकजन्मना} %॥४३।

\twolineshloka
{वरेण त्वं मदीयेन उपमां चिरजीविनाम्}
{गमिष्यसि न सन्देहस्तथाशीर्वचनम्मम} %॥४४।

\twolineshloka
{एवं वदन्ति ते सर्वे व्रज लोकान्यथेप्सितान्}
{एवं लब्धप्रसादेन मृकण्डुतनयेन च} %॥४५।

\twolineshloka
{आश्रमः स्थापितस्तेन मार्कण्डाश्रम इत्युत}
{तत्र स्नात्वा शुचिर्भूत्वा वाजपेयफलं लभेत्} %॥४६।

\onelineshloka*
{सर्वपापविशुद्धात्मा चिरायुर्जायते नरः}


\uvacha{पुलस्त्य उवाच}

\onelineshloka
{तथान्यं ते प्रवक्ष्यामि इतिहासं पुरातनम्} %॥४७।

\twolineshloka
{यथा रामेण वै तीर्थं पुष्करं तु विनिर्मितम्}
{चित्रकूटात्पुरा रामो मैथिल्या लक्ष्मणेन च} %॥४८।

\onelineshloka*
{अत्रेराश्रममासाद्य पप्रच्छ मुनिसत्तमम्}


\uvacha{राम उवाच}

\onelineshloka
{कानि पुण्यानि तीर्थानि किं वा क्षेत्रं महामुने} %॥४९।

\twolineshloka
{यत्र गत्वा नरो योगिन्वियोगं सह बन्धुभिः}
{नैव प्राप्नोति भगवन्तन्ममाचक्ष्व सुव्रत} %॥५०।

\twolineshloka
{अनेन वनवासेन राज्ञस्तु मरणेन च}
{भरतस्य वियोगेन परितप्ये ह्यहं त्रिभिः} %॥५१।

\twolineshloka
{तद्वाक्यं राघवेणोक्तं श्रुत्वा विप्रर्षभस्तदा}
{ध्यात्वा च सुचिरं कालमिदं वचनमब्रवीत्} %॥५२।


\uvacha{अत्रिरुवाच}

\twolineshloka
{साधु पृष्टं त्वया वीर रघूणां वंशवर्धन}
{मम पित्रा कृतं तीर्थं पुष्करं नाम विश्रुतम्} %॥५३।

\twolineshloka
{पर्वतौ द्वौ च विख्यातौ मर्यादा यज्ञपर्वतौ}
{कुण्डत्रयं तयोर्मध्ये ज्येष्ठमध्यकनिष्ठकम्} %॥५४।

\twolineshloka
{तेषु गत्वा दशरथं पिण्डदानेन तर्पय}
{तीर्थानां प्रवरं तीर्थं क्षेत्राणामपि चोत्तमम्} %॥५५।

\twolineshloka
{अवियोगा च सुरसा वापी रघुकुलोद्वह}
{तथा सौभाग्यकूपोन्यः सुजलो रघुनन्दन} %॥५६।

\twolineshloka
{तेषु पिण्डप्रदानेन पितरो मोक्षमाप्नुयुः}
{आभूतसम्प्लवं कालमेतदाह पितामहः} %॥५७।

\twolineshloka
{तत्र राघव गच्छस्व भूयोप्यागमनं क्रियाः}
{तथेति चोक्त्वा रामोपि गमनाय मनो दधे} %॥५८।

\twolineshloka
{ऋक्षवन्तमभिक्रम्य नगरं वैदिशं तथा}
{चर्मण्वतीं समुत्तीर्य प्राप्तोसौ यज्ञपर्वतम्} %॥५९।

\twolineshloka
{तमतिक्रम्य वेगेन मध्यमे पुष्करे स्थितः}
{पितॄन्सन्तर्पयामास अद्भिर्देवांश्च सर्वशः} %॥६०।

\twolineshloka
{स्नानावसाने रामेण मार्कण्डो मुनिपुङ्गवः}
{आगच्छन्शिष्यसंयुक्तो दृष्टस्तत्रैव धीमता} %॥६१।

\twolineshloka
{गत्वा वै सम्मुखं तस्य प्रणिपत्य च सादरम्}
{पृष्टोऽवियोगदः कूपः कतमस्यां दिशि प्रभो} %॥६२।

\twolineshloka
{सुतो दशरथस्याहं रामो नाम जनैः स्मृतः}
{सौभाग्यवापीं तां द्रष्टुमहं प्राप्तोत्रिशासनात्} %॥६३।

\twolineshloka
{तत्स्थानं तौ च वै कूपौ भगवान्प्रब्रवीतु मे}
{एवमुक्तश्च रामेण मार्कण्डः प्रत्युवाच ह} %॥६४।
\uvacha{मार्कण्डेय उवाच}

\twolineshloka
{साधु राघव भद्रं ते सुकृतं भवता कृतम्}
{तीर्थयात्राप्रसङ्गेन यत्प्राप्तोसीह साम्प्रतम्} %॥६५।

\twolineshloka
{एह्यागच्छस्व पश्य स्ववापीं तामवियोगदाम्}
{अवियोगश्च सर्वैश्च कूप एवात्र जायते} %॥६६।

\twolineshloka
{आमुष्मिके चैहिके च जीवतोपि मृतस्य वा}
{एतद्वाक्यं मुनीन्द्रस्य श्रुत्वा लक्ष्मणपूर्वजः} %॥६७।

\twolineshloka
{सस्मार रामो राजानं तदा दशरथं नृप}
{भरतं सह शत्रुघ्न्रं भातॄनन्यांश्चनागरान्} %॥६८।

\twolineshloka
{एवं चिन्तयतस्तस्य सन्ध्याकालो व्यजायत}
{उपास्य पश्चिमां सन्ध्यां मुनिभिः सह राघवः} %॥६९।

\twolineshloka
{सुष्वाप तां निशां तत्र भ्रातृभार्यासमन्वितः}
{विभावर्यवसाने तु स्वप्नान्ते रघुनन्दनः} %॥७०।

\twolineshloka
{पित्रा मात्रा तथा चान्यैरयोध्यायां स्थितः किल}
{विवाहमङ्गले वृत्ते बहुभिर्बान्धवैः सह} %॥७१।

\twolineshloka
{समासीनः सभार्योऽसावृषिभिः परिवारितः}
{लक्ष्मणेनाप्येवमेव दृष्टोऽसौ सीतया तथा} %॥७२।

\twolineshloka
{प्रभाते तु मुनीनां तत्सर्वमेव प्रकीर्तितम्}
{ऋषिभिश्च तथेत्युक्तः सत्यमेतद्रघूत्तम} %॥७३।

\twolineshloka
{मृतस्य दर्शने श्राद्धं कार्यमावश्यकं स्मृतम्}
{वृद्धिकामास्तु पितरस्तथा चैवान्नकाङ्क्षिणः} %॥७४।

\twolineshloka
{ददन्ति दर्शनं स्वप्ने भक्तियुक्तस्य राघव}
{अवियोगस्तु ते भ्रात्रा पित्रा च भरतेन च} %॥७५।

\twolineshloka
{चतुर्दशानां वर्षाणां भविता राघव ध्रुवम्}
{कुरु श्राद्धं तथा वीर राज्ञो दशरथस्य च} %॥७६।

\twolineshloka
{अमी च ऋषयः सर्वे तव भक्ताः कृतक्षणाः}
{अहं च जमदग्निश्च भारद्वाजश्च लोमशः} %॥७७।

\twolineshloka
{देवरातः शमीकश्च षडेते वै द्विजोत्तमाः}
{श्राद्धे च ते महाबाहो सम्भारांस्त्वमुपाहर} %॥७८।

\twolineshloka
{मुख्यं चेङ्गुदिपिण्याकं बदरामलकैः सह}
{श्रीफलानि च पक्वानि मूलं चोच्चावचं बहु} %॥७९।

\twolineshloka
{मार्गेण चाथ मांसेन धान्येन विविधेन च}
{तृप्तिं प्रयच्छ विप्राणां श्राद्धदानेन सुव्रत} %॥८०।

\twolineshloka
{पुष्करारण्यमासाद्य नियतो नियताशनः}
{पितॄंस्तर्पयते यस्तु सोश्वमेधमवाप्नुयात्} %॥८१।

\twolineshloka
{स्नानार्थं तु वयं राम गच्छामो ज्येष्ठपुष्करम्}
{इत्युक्त्वा ते गताः सर्वे मुनयो राघवं नृप} %॥८२।

\twolineshloka
{लक्ष्मणं चाब्रवीद्रामो मेध्यमाहर मे मृगम्}
{शुद्धेक्षणं च शशकं कृष्णशाकं तथा मधु} %॥८३।

\twolineshloka
{जम्बीराणि च मुख्यानि मूलानि विविधानि च}
{पक्वानि च कपित्थानि फलान्यन्यानि यानि च} %॥८४।

\twolineshloka
{तान्याहरस्व वै श्राद्धे क्षिप्रमेवास्तु लक्ष्मण}
{तथा तत्कृतवान्सर्वं रामादेशाच्च राघवः} %॥८५।

\twolineshloka
{बदरेङ्गुदिशाकानि मूलानि विविधानि च}
{तत्राहृत्य च रामेण कूटाकारः कृतो महान्} %॥८६।

\twolineshloka
{परिपक्वं च जानक्या सिद्धं रामे निवेदितम्}
{स्नात्वा रामो योगवाप्यां मुनींस्ताननुपालयन्} %॥८७।

\twolineshloka
{मध्याह्नाच्चलिते सूर्ये काले कुतपके तथा}
{आयाता ऋषयः सर्वे ये रामेणानुमन्त्रिताः} %॥८८।

\twolineshloka
{तानागतान्मुनीन्दृष्ट्वा वैदेही जनकात्मजा}
{रामान्तिकं परित्यज्य व्रीडिताऽन्यत्र संस्थिता} %॥८९।

\twolineshloka
{विस्मयोत्फुल्लनयना चिन्तयाना च वेपती}
{ब्राह्मणा नेह जानन्ति श्राद्धकाले ह्युपस्थिताः} %॥९०।

\twolineshloka
{रामेण भोजिता विप्राः स्मृत्युक्तेन यथाविधि}
{वैदिक्यश्च कृतास्सर्वाः सत्क्रिया यास्समीरिताः} %॥९१।

\twolineshloka
{पुराणोक्तो विधिश्चैव वैश्वदेविकपूर्वकः}
{भुक्तवत्सु च विप्रेषु दत्वा पिण्डान्यथाक्रमम्} %॥९२।

\twolineshloka
{प्रेषितेषु यथाशक्ति दत्वा तेषु च दक्षिणाम्}
{गतेषु विप्रमुख्येषु प्रियां रामोऽब्रवीदिदम्} %॥९३।

\twolineshloka
{किमर्थं सुभ्रु नष्टासि मुनीन्दृष्ट्वा त्विहागतान्}
{तत्सर्वं त्वमिदं तत्वं कारणं वद माचिरम्} %॥९४।

\twolineshloka
{भवितव्यं कारणेन तच्च गोप्यं न मे कुरु}
{शापितासि मम प्राणैर्लक्ष्मणस्य शुचिस्मिते} %॥९५।

\twolineshloka
{एवमुक्ता तदा भर्त्रा त्रपयाऽवाङ्मुखी स्थिता}
{विमुञ्चन्ती साऽश्रुपातं राघवं वाक्यमब्रवीत्} %॥९६।

\twolineshloka
{शृणु त्वं नाथ यद्दृष्टमाश्चर्यमिह यादृशम्}
{राम त्वयाऽचिन्त्यमानो राजेन्द्रस्त्विह चागतः} %॥९७।

\twolineshloka
{सर्वाभरणसंयुक्तौ द्वौ चान्यौ च तथाविधौ}
{द्विजानां देहसंयुक्तास्त्रयस्ते रघुनन्दन} %॥९८।

\twolineshloka
{पितरस्तु मया दृष्टा ब्राह्मणाङ्गेषु राघव}
{दृष्ट्वा त्रपान्विता चाहमपक्रान्ता तवान्तिकात्} %॥९९।

\twolineshloka
{त्वया वै भोजिता विप्राः कृतं श्राद्धं यथाविधि}
{वल्कलाजिनसंवीता कथं राज्ञः पुरःसरा} %॥१००।

\twolineshloka
{भवामि रिपुवीरघ्न सत्यमेतदुदाहृतम्}
{कौशेयानि च वस्त्राणि कैकेय्यापहृतानि च} %॥१०१।

\twolineshloka
{ततः प्रभृति चैवाहं चीरिणी तु वनाश्रयम्}
{ज्ञात्वाहं न वदे किञ्चिन्मा ते दुःखं भवत्विति} %॥१०२।

\twolineshloka
{नाहं स्मरामि वै मातुर्न पितुश्च परन्तप}
{कदा भविष्यतीहान्तो वनवासस्य राघव} %॥१०३।

\twolineshloka
{एतदेवानिशं राम चिन्तयन्त्याः पुनः पुनः}
{व्रजन्ति दिवसा नाथ तव पद्भ्यां शपाम्यहम्} %॥१०४।

\twolineshloka
{स्वहस्तेन कथं राज्ञो दास्ये वै भोजनं त्विदम्}
{दासानामपि यो दासो नोपभुञ्जीतयत्क्वचित्} %॥१०५।

\twolineshloka
{एतादृशी कथं त्वस्मै सम्प्रदातुं समुत्सहे}
{याहं राज्ञा पुरा दृष्टा सर्वालङ्कारभूषिता} %॥१०६।

\twolineshloka
{बालव्यजनहस्ता च वीजयन्ती नराधिपम्}
{सा स्वेदमलदिग्धाङ्गी कथं पश्यामि भूमिपम्} %॥१०७।

\twolineshloka
{व्यक्तं त्रिविष्टपं प्राप्तस्त्वया पुत्रेण तारितः}
{दृष्ट्वा मां दुःखितां बालां वने क्लिष्टामनागसम्} %॥१०८।

\twolineshloka
{शोकः स्यात्पार्थिवस्यास्य तेन नष्टास्मि राघव}
{भवान्प्राणसमो राम न ते गोप्यं ममत्विह} %॥१०९।

\twolineshloka
{सत्येन तेन चैवाथ स्पृशामि चरणौ तव}
{तच्छ्रुत्वा राघवः प्रीतः प्रियां तां प्रियवादिनीम्} %॥११०।

\twolineshloka
{अङ्कमानीय सुदृढं परिष्वज्य च सादरम्}
{भुक्तौ भोज्यं तदा वीरौ पश्चाद्भुक्ता च जानकी} %॥१११।

\twolineshloka
{एवं स्थितौ तदा सा च तां रात्रिं तत्र राघवौ}
{उदिते च सहस्रांशौ गमनाय मनो दधुः} %॥११२।

\twolineshloka
{प्रत्यङ्मुखं गतः क्रोशं ज्येष्ठं यावच्च पुष्करम्}
{पूर्वभागे पुष्करस्य यावत्तिष्ठति राघवः} %॥११३।

\twolineshloka
{शुश्राव च ततो वाचं देवदूतेन भाषितम्}
{भो भो राघव भद्रं ते तीर्थमेतत्सुदुर्लभम्} %॥११४।

\twolineshloka
{अस्मिन्स्थाने स्थितो वीर आत्मनः पुण्यतां कुरु}
{देवकार्यं त्वया कार्यं हन्तव्या देवशत्रवः} %॥११५।

\twolineshloka
{ततो हृष्टमना वीरो ह्यब्रवील्लक्ष्मणं वचः}
{सौमित्रेऽनुगृहीतोहं देवदेवेन ब्रह्मणा} %॥११६।

\twolineshloka
{अत्राश्रमपदं कृत्वा मासमेकं च लक्ष्मण}
{व्रतं चरितुमिच्छामि कायशोधनमुत्तमम्} %॥११७।

\twolineshloka
{तथेति लक्ष्मणेनोक्ते व्रतं परिसमाप्यतु}
{पिण्डदानादिभिर्दानैः श्राद्धैश्चैव पितामहान्} %॥११८।

\twolineshloka
{पुष्करे तु तदा रामोऽतर्पयद्विधिवत्तदा}
{कनका सुप्रभा चैव नन्दा प्राची सरस्वती} %॥११९।

\twolineshloka
{पञ्चस्रोताः पुष्करेषु पितॄणां तुष्टिदायिनी}
{दैनन्दिनीं पितॄणां तु पूजां तां पितृपूर्विकाम्} %॥१२०।

\twolineshloka
{रचयित्वा तदा रामो लक्ष्मणं वाक्यमब्रवीत्}
{एहि लक्ष्मण शीघ्रं त्वं पुष्कराज्जलमानय} %॥१२१।

\twolineshloka
{पादप्रक्षालनं कृत्वा शयनं कुरु संस्तरे}
{विभावर्यां निवृत्तायां यास्यामो दक्षिणां दिशम्} %॥१२२।

\twolineshloka
{लक्ष्मणस्त्वब्रवीद्वाक्यं सीतयानीय तां पयः}
{नाहं राम सर्वकाले दासभावं करोमि ते} %॥१२३।

\twolineshloka
{इयम्पुष्टाचसुभृशम्पीवरीचममाप्युत}
{किं त्वं करिष्यस्यनया भार्यया वद साम्प्रतम्} %॥१२४।

\twolineshloka
{किं वा मृतस्य वै पृष्ठ इयं यास्यति ते प्रिया}
{रक्षसे त्वं सदा कालं सुपुष्टां चैव सर्वदा} %॥१२५।

\twolineshloka
{हृष्टा चैषा क्लेशयति सततं मां रघूत्तम}
{त्वं च क्लेशयसे राम परत्र जायते क्षतिः} %॥१२६।

\twolineshloka
{त्वत्कृते च सदा चाहं पिपासां क्षुधया सह}
{संसहामि न सन्देहः परत्र च निशामय} %॥१२७।

\twolineshloka
{मृतानां पृष्ठतः कश्चिद्गतो नैव च दृश्यते}
{भार्य्या पुत्रो धनं चापि एवमाहुर्मनीषिणः} %॥१२८।

\twolineshloka
{मृतश्च ते पिता राम त्यक्त्वा राज्यमकण्टकम्}
{विनिक्षिप्य वने त्वां च कैकेय्याः प्रियकाम्यया} %॥१२९।

\twolineshloka
{इहस्थिता सा कैकेयी धनं सर्वे च बान्धवाः}
{महाराजो दशरथ एक एव गतो गतिम्} %॥१३०।

\twolineshloka
{मन्येहं न त्वया सार्धं सीता यास्यति वै ध्रुवम्}
{करिष्यसे किमनया वद राघव साम्प्रतम्} %॥१३१।

\twolineshloka
{श्रुत्वा चाश्रुतपूर्वं हि वाक्यं लक्ष्मणभाषितम्}
{विमना राघवस्तस्थौ सीता चापि वरानना} %॥१३२।

\twolineshloka
{यदुक्तं लक्ष्मणेनाथ सीता सर्वं चकार ह}
{स्नात्वा भुक्त्वा ततो वीरौ पुष्करे पुष्करेक्षणौ} %॥१३३।

\twolineshloka
{नीत्वा विभावरीं तत्र गमनाय मनो दधुः}
{एह्युत्तिष्ठ च सौमित्रे व्रजामो दक्षिणां दिशम्} %॥१३४।

\twolineshloka
{सौमित्रिरब्रवीद्राम नाहं यास्ये कथञ्चन}
{व्रज त्वमनया सार्धं भार्यया कमलेक्षण} %॥१३५।

\twolineshloka
{नान्यद्वनं गमिष्यामि नैवायोध्यां च राघव}
{अस्मिन्वने वसिष्यामि वर्षाणीह चतुर्दश} %॥१३६।

\twolineshloka
{मया विना त्वयोध्यायां यदि त्वं न गमिष्यसि}
{अनेन वर्त्मना भूप आगन्तव्यं त्वया विभो} %॥१३७।

\twolineshloka
{यदि जीवामि तत्कालं पुनर्यास्ये पितुः पुरम्}
{तपस्सम्भावयिष्यामि मया त्वं किं करिष्यसि} %॥१३८।

\twolineshloka
{व्रज सौम्य शिवः पन्थामा च ते परिपन्थिनः}
{पश्यामि त्वां पुनः प्राप्तं सभार्यं कमलेक्षणम्} %॥१३९।

\twolineshloka
{पितृपैतामहं राज्यमयोध्यायां नराधिप}
{शत्रुघ्नभरतौ चोभौ त्वदाज्ञाकरणे स्थितौ} %॥१४०।

\twolineshloka
{अहं ते प्रतिकूलस्तु वनवासे विशेषतः}
{अनारतं दिवा चाहं रात्रौ चैव परन्तप} %॥१४१।

\twolineshloka
{कर्मकर्तुं न शक्रोमि व्रज सौम्य यथासुखम्}
{एवं ब्रुवाणं सौमित्रिमुवाच रघुनन्दनः} %॥१४२।

\twolineshloka
{कथं पूर्वमयोध्याया निर्गतोसि मया सह}
{वने वत्स्याम्यहं राम नववर्षाणि पञ्च च} %॥१४३।

\twolineshloka
{न तु त्वया विरहितः स्वर्गेपि निवसे क्वचित्}
{या गतिस्ते नरव्याघ्र मम सापि भविष्यति} %॥१४४।

\twolineshloka
{प्रसादः क्रियतां मह्यं नय मामपि राघव}
{इदानीमर्धमार्गे त्वं कथं स्थास्यसि शत्रुहन्} %॥१४५।

\twolineshloka
{लक्ष्मणस्त्वब्रवीद्रामं नाहं गन्ता वने पुनः}
{लक्ष्मणं संस्थितं ज्ञात्वा रामो वचनमब्रवीत्} %॥१४६।

\twolineshloka
{मामनुव्रज सौमित्र एको यास्यामि काननम्}
{द्वितीया मे त्वियं सीता रामेणोक्तस्तु लक्ष्मणः} %॥१४७।

\twolineshloka
{गृहीत्वाऽथ समुत्तस्थौ रामवाक्यं स लक्ष्मणः}
{मर्यादापर्वतं प्राप्तौ क्षेत्रसीमां परन्तपौ} %॥१४८।

\twolineshloka
{अजगन्धं च देवेशं देवदेवं पिनाकिनम्}
{अष्टाङ्गप्रणिपातेन नत्वा रामस्त्रिलोचनम्} %॥१४९।

\twolineshloka
{तुष्टाव प्रयतः स्थित्वा शङ्करं पार्वतीप्रियम्}
{कृताञ्जलिपुटो भूत्वा रोमाञ्चितशरीरकः} %॥१५०।

\twolineshloka
{सात्विकं भावमापन्नो विनिर्धूतरजस्तमाः}
{लोकानां कारणं देवं बुबुधे विबुधाधिपम्} %॥१५१।
\uvacha{राम उवाच}

\twolineshloka
{कृत्स्नस्य योऽस्य जगतः स चराचरस्य कर्ता कृतस्य च पुनः सुखदुःखदश्च}
{संहारहेतुरपि यः पुनरन्तकाले तं शङ्करं शरणदं शरणं व्रजामि} %॥१५२।

\twolineshloka
{योऽयं सकृद्विमलचारुविलोलतोयां गङ्गां महोर्मिविषमां गगनात्पतन्तीम्}
{मूर्ध्ना दधेऽस्रजमिव प्रविलोलपुष्पां तं शङ्करं शरणदं शरणं व्रजामि} %१५३ ।

\twolineshloka
{कैलासशैलशिखरं परिकम्प्यमानं कैलासशृङ्गसदृशेन दशाननेन}
{यत्पादपद्मविधृतं स्थिरतां दधार तं शङ्करं शरणदं शरणं व्रजामि} %॥१५४।

\twolineshloka
{येनासकृद्दनुसुताः समरे निरस्ता विद्याधरोरगगणाश्च वरैः समग्रैः}
{संयोजिता मुनिवराः फलमूलभक्षास्तं शङ्करं शरणदं शरणं व्रजामि} %॥१५५।

\twolineshloka
{दक्षाध्वरे च नयने च तथा भगस्य पूष्णस्तथा दशनपङ्क्तिमपातयच्च}
{तस्तम्भयः कुलिशयुक्तमथेन्द्रहस्तं तं शङ्करं शरणदं शरणं व्रजामि} %॥१५६।

\twolineshloka
{एनःकृतोपिविषयेष्वपिसक्तचित्ताज्ञानान्वयश्रुतगुणैरपिनैवयुक्ताः}
{यं संश्रिताः सुखभुजः पुरुषा भवन्ति तं शङ्करं शरणदं शरणं व्रजामि} %॥१५७।

\twolineshloka
{अत्रिप्रसूतिरविकोटिसमानतेजाः सन्त्रासनं विबुधदानवसत्तमानाम्}
{यः कालकूटमपिबत्प्रसभं सुदीप्तं तं शङ्करं शरणदं शरणं व्रजामि} %॥१५८।

\twolineshloka
{ब्रह्मेन्द्ररुद्रमरुतां च सषण्मुखानां दद्याद्वरं सुबहुशो भगवान्महेशः}
{नन्दिं च मृत्युवदनात्पुनरुज्जहार तं शङ्करं शरणदं शरणं व्रजामि} %॥१५९।

\twolineshloka
{आराधितः सुतपसा हिमवन्निकुञ्जे धूमव्रतेन मनसापि परैरगम्ये}
{सञ्जीवनीमकथयद्भृगवे महात्मा तं शङ्करं शरणदं शरणं व्रजामि} %॥१६०।

\twolineshloka
{नानाविधैर्गजबिडालसमानवक्त्रैर्दक्षाध्वरप्रमथनैर्बलिभिर्गणैन्द्रैः }
{योभ्यर्चितोमरगणैश्च सलोकपालैस्तं शङ्करं शरणदं शरणं व्रजामि} %॥१६१।

\twolineshloka
{शङ्खेन्दुकुन्दधवलं वृषभं प्रवीरमारुह्य यः क्षितिधरेन्द्रसुतानुयातः}
{यात्यम्बरं प्रलयमेघविभूषितं च तं शङ्करं शरणदं शरणं व्रजामि} %॥१६२।

\twolineshloka
{शान्तं मुनिं यमनियोगपरायणैस्तैर्भीमैर्महोग्रपुरुषैः प्रतिनीयमानम्}
{भक्त्यानतं स्तुतिपरं प्रसभं ररक्ष तं शङ्करं शरणदं शरणं व्रजामि} %॥१६३।

\twolineshloka
{यः सव्यपाणि कमलाग्रनखेन देवस्तत्पञ्चमं प्रसभमेव पुरस्सुराणाम्}
{ब्राह्मं शिरस्तरुणपद्मनिभं चकर्त्त तं शङ्करं शरणदं शरणं व्रजामि} %॥१६४।

\twolineshloka
{यस्य प्रणम्य चरणौ वरदस्य भक्त्या स्तुत्वा च वाग्भिरमलाभिरतन्द्रितात्मा}
{दीप्तस्तमांसि नुदते स्वकरैर्विवस्वांस्तं शङ्करं शरणदं शरणं व्रजामि} %॥१६५।

\twolineshloka
{ये त्वां सुरोत्तमगुरुं पुरुषा विमूढा जानन्ति नास्य जगतः सचराचरस्य}
{ऐश्वर्यमाननिगमानुशयेन पश्चात्ते यातनामनुभवन्त्यविशुद्धचित्ताः} %॥१६६।

\twolineshloka
{तस्यैवं स्तुवतोऽवोचच्छूलपाणिर्वृषध्वजः}
{उवाच वचनं हृष्टो राघवं तुष्टमानसः} %॥१६७।
\uvacha{रुद्र उवाच}

\twolineshloka
{राम हृष्टोस्मि भद्रं ते जातस्त्वं निर्मले कुले}
{त्वं चापि जगतां वन्द्यो देवो मानुषरूपधृत्} %॥१६८।

\twolineshloka
{त्वया नाथेन वै देवाः सुखिनः शाश्वतीः समा}
{सेविष्यन्ते चिरं कालं गते वर्षे चतुर्दशे} %॥१६९।

\twolineshloka
{अयोध्यामागतं त्वां ये द्रक्ष्यन्ति भुवि मानवाः}
{सुखं तेऽत्र भजिष्यन्ति स्वर्गे वासन्तथाक्षयम्} %॥१७०।

\twolineshloka
{देवकार्यं महत्कृत्वा आगच्छेथाः पुनः पुरीम्}
{राघवस्तु तथा देवं नत्वा शीघ्रं विनिर्गतः} %॥१७१।

\twolineshloka
{इन्द्रमार्गां नदीं प्राप्य जटाजूटं नियम्य च}
{अब्रवील्लक्ष्मणं राम इदमर्पय मे धनुः} %॥१७२।

\twolineshloka
{रामवाक्यं तु तच्छ्रुत्वा सीतां वै लक्ष्मणोऽब्रवीत्}
{किमर्थं देवि रामेण त्यक्तोहं कारणं विना} %॥१७३।

\twolineshloka
{अपराधं न जानामि कुपितो यन्महाभुजः}
{रामेणाहं परित्यक्तः प्राणांस्त्यक्ष्याम्यसंशयम्} %॥१७४।

\twolineshloka
{नैव मे जीवितेनार्थो धिग्धिङ्मां कुलपांसनम्}
{आर्यस्य येन वै मन्युर्जनितः पापकारिणा} %॥१७५।

\twolineshloka
{कांस्तु लोकान्गमिष्यामि अपध्यातो महात्मना}
{उभौ हस्तौ मुखे कृत्वा साश्रुकण्ठोऽब्रवीदिदम्} %॥१७६।

\twolineshloka
{नापराध्यामि रामस्य कर्मणा मनसा गिरा}
{स्पृष्टौ ते चरणौ देवि मम नान्या गतिर्भवेत्} %॥१७७।

\twolineshloka
{ततः सीताऽब्रवीद्रामं त्यक्तः किमनुजस्त्वया}
{वैषम्यं त्यज्यतां बाले लक्ष्मणे लक्ष्मिवर्धने} %॥१७८।

\twolineshloka
{राघवस्त्वब्रवीत्सीतां नाहं त्यक्ष्यामि लक्ष्मणम्}
{न कदाचिदपि स्वप्ने लक्ष्मणस्य मतं प्रिये} %॥१७९।

\twolineshloka
{श्रुतपूर्वं च सुश्रोणि क्षेत्रस्यास्य विचेष्टितम्}
{अत्र क्षेत्रे जनास्सत्यं सर्वे हि स्वार्थतत्पराः} %॥१८०।

\twolineshloka
{परस्परं न पश्यन्ति स्वात्मनश्च हितं वचः}
{न शृण्वन्ति पितुः पुत्राः पुत्राणां पितरस्तथा} %॥१८१।

\twolineshloka
{न शिष्या हि गुरोर्वाक्यं शिष्यस्यापि तथा गुरुः}
{अर्थानुबन्धिनीप्रीतिर्न कश्चित्कस्यचित्प्रियः} %॥१८२।

\twolineshloka
{इत्येवं कथयन्नेव प्राप्तो रेवां महानदीम्}
{चक्रेभिषेकं काकुत्स्थः सानुजः सह सीतया} %॥१८३।

\twolineshloka
{तर्पयित्वा च सलिलैः स्वान्पितॄन्दैवतान्यपि}
{उदीक्ष्य च मुहुः सूर्यं देवताश्च समाहितः} %॥१८४।

\twolineshloka
{कृताभिषेकस्तु रराज रामः सीता द्वितीयः सह लक्ष्मणेन}
{कृताभिषेकः सह शैलपुत्र्या गुहेन सार्धं भगवानिवेशः} %॥१८५।

॥इति श्रीपाद्मपुराणे प्रथमे सृष्टिखण्डे मार्कण्डेयाश्रमदर्शनं नाम त्र्यस्त्रिंशोऽध्यायः॥३३॥
    \input{katha/padma-puranam/shudra-tapasa-vadha}
    \input{katha/padma-puranam/rama-agastya-samvada}
    \sect{यज्ञनिवारणम्}

\src{पद्म-पुराणम्}{सृष्टिखण्डम्}{अध्यायः ३७}{१--१७१}
% \tags{concise, complete}
\notes{This chapter describes Rāma's Abstaining from the Performance of Rājasūya yajna.}
\textlink{https://sa.wikisource.org/wiki/पद्मपुराणम्/खण्डः_१_(सृष्टिखण्डम्)/अध्यायः_३७}
\translink{https://www.wisdomlib.org/hinduism/book/the-padma-purana/d/doc364160.html}

\storymeta


\uvacha{पुलस्त्य उवाच}

\twolineshloka
{तदद्भुततमं वाक्यं श्रुत्वा च रघुनन्दनः}
{गौरवाद्विस्मयाच्चापि भूयः प्रष्टुं प्रचक्रमे}% १

\uvacha{राम उवाच}

\twolineshloka
{भगवंस्तद्वनं घोरं यत्रासौ तप्तवांस्तपः}
{श्वेतो वैदर्भको राजा तदद्भुतमभूत्कथम्}% २

\twolineshloka
{विषमं तद्वनं राजा शून्यं मृगविवर्जितम्}
{प्रविष्टस्तप आस्थातुं कथं वद महामुने}% ३

\twolineshloka
{समन्ताद्योजनशतं निर्मनुष्यमभूत्कथम्}
{भवान्कथं प्रविष्टस्तद्येन कार्येण तद्वद}% ४

\uvacha{अगस्त्य उवाच}

\twolineshloka
{पुरा कृतयुगे राजा मनुर्दण्डधरः प्रभुः}
{तस्य पुत्रोथ नाम्नासीदिक्ष्वाकुरमितद्युतिः}% ५

\twolineshloka
{तं पुत्रं पूर्वजं राज्ये निक्षिप्य भुविसम्मतम्}
{पृथिव्यां राजवंशानां भव राजेत्युवाच ह}% ६

\twolineshloka
{तथेति च प्रतिज्ञातं पितुः पुत्रेण राघव}
{ततःपरमसंहृष्टः पुनस्तं प्रत्यभाषत}% ७

\twolineshloka
{प्रीतोस्मि परमोदार कर्मणा ते न संशयः}
{दण्डेन च प्रजा रक्ष न च दण्डमकारणम्}% ८

\twolineshloka
{अपराधिषु यो दण्डः पात्यते मानवैरिह}
{स दण्डो विधिवन्मुक्तः स्वर्गं नयति पार्थिवम्}% ९

\twolineshloka
{तस्माद्दण्डे महाबाहो यत्नवान्भव पुत्रक}
{धर्मस्ते परमो लोके कृत एवं भविष्यति}% १०

\twolineshloka
{इति तं बहुसन्दिश्य मनुः पुत्रं समाधिना}
{जगाम त्रिदिवं हृष्टो ब्रह्मलोकमनुत्तमम्}% ११

\twolineshloka
{जनयिष्ये कथं पुत्रानिति चिन्तापरोऽभवत्}
{कर्मभिर्बहुभिस्तैस्तैस्ससुतैस्संयुतोऽभवत्}% १२

\twolineshloka
{तोषयामास पुत्रैस्स पितॄन्देवसुतोपमैः}
{सर्वेषामुत्तमस्तेषां कनीयान्रघुनन्दन}% १३

\twolineshloka
{शूरश्च कृतविद्यश्च गुरुश्च जनपूजया}
{नाम तस्याथ दण्डेति पिता चक्रे स बुद्धिमान्}% १४

\twolineshloka
{भविष्यद्दण्डपतनं शरीरे तस्य वीक्ष्य च}
{सम्पश्यमानस्तं दोषं घोरं पुत्रस्य राघव}% १५

\twolineshloka
{स विन्ध्यनीलयोर्मध्ये राज्यमस्य ददौ प्रभुः}
{स दण्डस्तत्र राजाभूद्रम्ये पर्वतमूर्द्धनि}% १६

\twolineshloka
{पुरं चाप्रतिमं तेन निवेशाय तथा कृतम्}
{नाम तस्य पुरस्याथ मधुमत्तमिति स्वयम्}% १७

\twolineshloka
{तथादेशेन सम्पन्नः शूरो वासमथाकरोत्}
{एवं राजा स तद्राज्यं चकार सपुरोहितः}% १८

\twolineshloka
{प्रहृष्ट सुप्रजाकीर्णं देवराजो यथा दिवि}
{ततः स दण्डः काकुत्स्थ बहुवर्षगणायुतम्}% १९

\twolineshloka
{अकारयत्तु धर्मात्मा राज्यं निहतकण्टकम्}
{अथ काले तु कस्मिंश्चिद्राजा भार्गवमाश्रमम्}% २०

\twolineshloka
{रमणीयमुपाक्रामच्चैत्रमासे मनोरमे}
{तत्र भार्गवकन्यां तु रूपेणाप्रतिमां भुवि}% २१

\twolineshloka
{विचरन्तीं वनोद्देशे दण्डोऽपश्यदनुत्तमाम्}
{उत्तुङ्गपीवरीं श्यामां चन्द्राभवदनां शुभाम्}% २२

\twolineshloka
{सुनासां चारुसर्वाङ्गीं पीनोन्नतपयोधराम्}
{मध्ये क्षामां च विस्तीर्णां दृष्ट्वा तां कुरुते मुदम्}% २३

\twolineshloka
{एकवस्त्रां वने चैकां प्रथमे यौवने स्थिताम्}
{स तां दृष्ट्वात्वधर्मेण अनङ्गशरपीडितः}% २४

\twolineshloka
{अभिगम्य सुविश्रान्तां कन्यां वचनमब्रवीत्}
{कुतस्त्वमसि सुश्रोणि कस्य चासि सुशोभने}% २५

\twolineshloka
{पीडतोहमनङ्गेन पृच्छामि त्वां सुशोभने}
{त्वया मेऽपहृतं चित्तं दर्शनादेव सुन्दरि}% २६

\twolineshloka
{इदं ते वदनं रम्यं मुनीनां चित्तहारकम्}
{यद्यहं न लभे भोक्तुं मृतं मामवधारय}% २७

\twolineshloka
{त्वया हृता मम प्राणा मां जीवय सुलोचने}
{दासोस्मि ते वरारोहे भक्तं मां भज शोभने}% २८

\twolineshloka
{तस्यैवं तु ब्रुवाणस्य मदोन्मत्तस्य कामिनः}
{भार्गवी प्रत्युवाचेदं वचः सविनयं नृपम्}% २९

\twolineshloka
{भार्गवस्य सुतां विद्धि शुक्रस्याक्लिष्टकर्मणः}
{अरजां नाम राजेन्द्र ज्येष्ठामाश्रमवासिनः}% ३०

\twolineshloka
{शुक्रः पिता मे राजेन्द्र त्वं च शिष्यो महात्मनः}
{धर्मतो भगिनी चाहं भवामि नृपनन्दन}% ३१

\twolineshloka
{एवंविधं वचो वक्तुं न त्वमर्हसि पार्थिव}
{अन्येभ्योपि सुदुष्टेभ्यो रक्ष्या चाहं सदा त्वया}% ३२

\twolineshloka
{क्रोधनो मे पिता रौद्रो भस्मत्वं त्वां समानयेत्}
{अथवा राजधर्मेणासम्बन्धं कुरुषे बलात्}% ३३

\twolineshloka
{पितरं याचयस्व त्वं धर्मदृष्टेन कर्मणा}
{वरयस्व नृपश्रेष्ठ पितरं मे महाद्युतिम्}% ३४

\twolineshloka
{अन्यथा विपुलं दुःखं तव घोरं भवेद्ध्रुवम्}
{क्रुद्धो हि मे पिता सर्वं त्रैलोक्यमभिनिर्दहेत्}% ३५

\twolineshloka
{ततोऽशुभं महाघोरं श्रुत्वा दण्डः सुदारुणम्}
{प्रत्युवाच मदोन्मत्तः शिरसाभिनतः पुनः}% ३६

\twolineshloka
{प्रसादं कुरु सुश्रोणि कामोन्मत्तस्य कामिनि}
{त्वया रुद्धा मम प्राणा विशीर्यन्ति शुभानने}% ३७

\twolineshloka
{त्वां प्राप्य वैरं मेऽत्रास्तु वधो वापि महत्तरः}
{भक्तं भजस्व मां भीरु त्वयि भक्तिर्हि मे परा}% ३८

\twolineshloka
{एवमुक्त्वा तु तां कन्यां बलात्सङ्गृह्य बाहुना}
{अन्येन राज्ञा हस्तेन विवस्त्रा सा तथा कृता}% ३९

\twolineshloka
{अङ्गमङ्गे समाश्लेष्य मुखे चैव मुखं कृतम्}
{विस्फुरन्तीं यथाकामं मैथुनायोपचक्रमे}% ४०

\twolineshloka
{तमनर्थं महाघोरं दण्डः कृत्वा सुदारुणम्}
{नगरं स्वं जगामाशु मदोन्मत्त इव द्विपः}% ४१

\twolineshloka
{भार्गवी रुदती दीना आश्रमस्याविदूरतः}
{प्रत्यपालयदुद्विग्ना पितरं देवसम्मितम्}% ४२

\twolineshloka
{स मुहूर्तादुपस्पृश्य देवर्षिरमितद्युतिः}
{स्वमाश्रमं शिष्यवृतं क्षुधार्तः सन्यवर्तत}% ४३

\twolineshloka
{सोपश्यदरजां दीनां रजसा समभिप्लुताम्}
{चन्द्रस्य घनसंयुक्तां ज्योत्स्नामिव पराजिताम्}% ४४

\twolineshloka
{तस्य रोषः समभवत्क्षुधार्तस्य महात्मनः}
{निर्दहन्निव लोकांस्त्रींस्तान्शिष्यान्समुवाच ह}% ४५

\twolineshloka
{पश्यध्वं विपरीतस्य दण्डस्यादीर्घदर्शिनः}
{विपत्तिं घोरसङ्काशां दीप्तामग्निशिखामिव}% ४६

\twolineshloka
{यन्नाशं दुर्गतिं प्राप्तस्सानुगश्च न संशयः}
{यस्तु दीप्तहुताशस्य अर्चिः संस्पृष्टवानिह}% ४७

\twolineshloka
{यस्मात्स कृतवान्पापमीदृशं घोरसम्मितम्}
{तस्मात्प्राप्स्यति दुर्मेधाः पांसुवर्षमनुत्तमम्}% ४८

\twolineshloka
{कुराजा देशसंयुक्तः सभृत्यबलवाहनः}
{पापकर्मसमाचारो वधं प्राप्स्यति दुर्मतिः}% ४९

\twolineshloka
{समन्ताद्योजनशतं विषयं चास्य दुर्मतेः}
{धुनोतु पांसुवर्षेण महता पाकशासनः1.37.}% ५०

\twolineshloka
{सर्वसत्वानि यानीह जङ्गमस्थावराणि वै}
{सर्वेषां पांसुवर्षेण क्षयः क्षिप्रं भविष्यति}% ५१

\twolineshloka
{दण्डस्य विषयो यावत्तावत्सवनमाश्रमम्}
{पांसुवर्षमिवाकस्मात्सप्तरात्रं भविष्यति}% ५२

\twolineshloka
{इत्युक्त्वा क्रोधसन्तप्तस्तमाश्रमनिवासिनम्}
{जनं जनपदस्यान्ते स्थीयतामित्युवाच ह}% ५३

\twolineshloka
{उक्तमात्रे उशनसा आश्रमावसथो जनः}
{क्षिप्रं तु विषयात्तस्मात्स्थानं चक्रे च बाह्यतः}% ५४

\twolineshloka
{तं तथोक्त्वा मुनिजनमरजामिदमब्रवीत्}
{आश्रमे त्वं सुदुर्मेधे वस चेह समाहिता}% ५५

\twolineshloka
{इदं योजनपर्यन्तमाश्रमं रुचिरप्रभम्}
{अरजे विरजास्तिष्ठ कालमत्र समाश्शतम्}% ५६

\twolineshloka
{श्रुत्वा नियोगं विप्रर्षेररजा भार्गवी तदा}
{तथेति पितरं प्राह भार्गवं भृशदुःखिता}% ५७

\twolineshloka
{इत्युक्त्वा भार्गवो वासं तस्मादन्यमुपाक्रमत्}
{सप्ताहे भस्मसाद्भूतं यथोक्तं ब्रह्मवादिना}% ५८

\twolineshloka
{तस्माद्दण्डस्य विषयो विन्ध्यशैलस्य मानुष}
{शप्तो ह्युशनसा राम तदाभूद्धर्षणे कृते}% ५९

\twolineshloka
{ततःप्रभृति काकुत्स्थ दण्डकारण्यमुच्यते}
{एतत्ते सर्वमाख्यातं यन्मां पृच्छसि राघव}% ६०

\twolineshloka
{सन्ध्यामुपासितुं वीर समयो ह्यतिवर्तते}
{एते महर्षयो राम पूर्णकुम्भाः समन्ततः}% ६१

\twolineshloka
{कृतोदका नरव्याघ्र पूजयन्ति दिवाकरम्}
{सर्वैरॄषिभिरभ्यस्तैः स्तोत्रैर्ब्रह्मादिभिः कृतैः}% ६२

\twolineshloka
{रविरस्तङ्गतो राम गत्वोदकमुपस्पृश}
{ॠषेर्वचनमादाय रामः सन्ध्यामुपासितुम्}% ६३

\twolineshloka
{उपचक्राम तत्पुण्यं ससरोरघुनन्दनः}
{अथ तस्मिन्वनोद्देशे रम्ये पादपशोभिते}% ६४

\twolineshloka
{नदपुण्ये गिरिवरे कोकिलाशतमण्डिते}
{नानापक्षिरवोद्याने नानामृगसमाकुले}% ६५

\twolineshloka
{सिंहव्याघ्रसमाकीर्णे नानाद्विजसमावृते}
{गृध्रोलूकौ प्रवसितौ बहून्वर्षगणानपि}% ६६

\twolineshloka
{अथोलूकस्य भवनं गृध्रः पापविनिश्चयः}
{ममेदमिति कृत्वाऽसौ कलहं तेन चाकरोत्}% ६७

\twolineshloka
{राजा सर्वस्य लोकस्य रामो राजीवलोचनः}
{तं प्रपद्यावहै शीघ्रं कस्यैतद्भवनं भवेत्}% ६८

\twolineshloka
{गृध्रोलूकौ प्रपद्येतां जातकोपावमर्षिणौ}
{रामं प्रपद्यतौ शीघ्रं कलिव्याकुलचेतसौ}% ६९

\twolineshloka
{तौ परस्परविद्वेषौ स्पृशतश्चरणौ तथा}
{अथ दृष्ट्वा राघवेन्द्रं गृध्रो वचनमब्रवीत्}% ७०

\twolineshloka
{सुराणामसुराणां च त्वं प्रधानो मतो मम}
{बृहस्पतेश्च शुक्राच्च त्वं विशिष्टो महामतिः}% ७१

\twolineshloka
{परावरज्ञो भूतानां मर्त्ये शक्र इवापरः}
{दुर्निरीक्षो यथा सूर्यो हिमवानिव गौरवे}% ७२

\twolineshloka
{सागरश्चासि गाम्भीर्ये लोकपालो यमो ह्यसि}
{क्षान्त्या धरण्या तुल्योसि शीघ्रत्वे ह्यनिलोपमः}% ७३

\twolineshloka
{गुरुस्त्वं सर्वसम्पन्नो विष्णुरूपोसि राघव}
{अमर्षी दुर्जयो जेता सर्वास्त्रविधिपारगः}% ७४

\twolineshloka
{शृणु त्वं मम देवेश विज्ञाप्यं नरपुङ्गव}
{ममालयं पूर्वकृतं बाहुवीर्येण वै प्रभो}% ७५

\twolineshloka
{उलूको हरते राजंस्त्वत्समीपे विशेषतः}
{ईदृशोयं दुराचारस्त्वदाज्ञा लङ्घको नृप}% ७६

\twolineshloka
{प्राणान्तिकेन दण्डेन राम शासितुमर्हसि}
{एवमुक्ते तु गृध्रेण उलूको वाक्यमब्रवीत्}% ७७

\twolineshloka
{शृणु देव मम ज्ञाप्यमेकचित्तो नराधिप}
{सोमाच्छक्राच्च सूर्याच्च धनदाच्च यमात्तथा}% ७८

\twolineshloka
{जायते वै नृपो राम किञ्चिद्भवति मानुषः}
{त्वं तु सर्वमयो देवो नारायणपरायणः}% ७९

\twolineshloka
{प्रोच्यते सोमता राजन्सम्यक्कार्ये विचारिते}
{सम्यग्रक्षसि तापेभ्यस्तमोघ्नो हि यतो भवान्}% ८०

\twolineshloka
{दोषे दण्डात्प्रजानां त्वं यतः पापभयापहः}
{दाता प्रहर्ता गोप्ता च तेनेन्द्र इव नो भवान्}% ८१

\twolineshloka
{अधृष्यः सर्वभूतेषु तेजसा चानलो मतः}
{अभीक्ष्णं तपसे पापांस्तेन त्वं राम भास्करः}% ८२

\twolineshloka
{साक्षाद्वित्तेशतुल्यस्त्वमथवा धनदाधिकः}
{चित्तायत्ता तु पत्नीश्रीर्नित्यं ते राजसत्तम}% ८३

\twolineshloka
{धनदस्य तु कोशेन धनदस्तेन वैभवान्}
{समः सर्वेषु भूतेषु स्थावरेषु चरेषु च}% ८४

\twolineshloka
{शत्रौ मित्रे च ते दृष्टिः समन्ताद्याति राघव}
{धर्मेण शासनं नित्यं व्यवहारविधिक्रमैः}% ८५

\twolineshloka
{यस्य रुष्यसि वै राम मृत्युस्तस्याभिधीयते}
{गीयसे तेन वै राजन्यम इत्यभिविश्रुतः}% ८६

\twolineshloka
{यश्चासौ मानुषो भावो भवतो नृपसत्तम}
{आनृशंस्यपरो राजा सर्वेषु कृपयान्वितः}% ८७

\twolineshloka
{दुर्बलस्य त्वनाथस्य राजा भवति वै बलम्}
{अचक्षुषो भवेच्चक्षुरमतेषु मतिर्भवेत्}% ८८

\twolineshloka
{अस्माकमपि नाथस्त्वं श्रूयतां मम धार्मिक}
{भवता तत्र मन्तव्यं यथैते किल पक्षिणः}% ८९

\twolineshloka
{योस्मन्नाथः स पक्षीन्द्रो भवतो विनियोज्यकः}
{अस्वाम्यं देव नास्माकं सन्निधौ भवतः प्रभो}% ९०

\twolineshloka
{भवतैव कृतं पूर्वं भूतग्रामं चतुर्विधम्}
{ममालयप्रविष्टस्तु गृध्रो मां बाधते नृप}% ९१

\twolineshloka
{भवान्देवमनुष्येषु शास्ता वै नरपुङ्गव}
{एतच्छ्रुत्वा तु वै रामः सचिवानाह्वयत्स्वयम्}% ९२

\twolineshloka
{विष्टिर्जयन्तो विजयः सिद्धार्थो राष्ट्रवर्धनः}
{अशोको धर्मपालश्च सुमन्त्रश्च महाबलः}% ९३

\twolineshloka
{एते रामस्य सचिवा राज्ञो दशरथस्य च}
{नीतियुक्ता महात्मानः सर्वशास्त्रविशारदाः}% ९४

\twolineshloka
{सुशान्ताश्च कुलीनाश्च नये मन्त्रे च कोविदाः}
{तानाहूय स धर्मात्मा पुष्पकादवरुह्य च}% ९५

\twolineshloka
{गृध्रोलूकौ विवदन्तौ पृच्छति स्म रघूत्तमः}
{कति वर्षाणि भो गृध्र तवेदं निलयं कृतम्}% ९६

\twolineshloka
{एतन्मे कौतुकं ब्रूहि यदि जानासि तत्त्वतः}
{एतच्छ्रुत्वा वचो गृध्रो बभाषे राघवं स्थितम्}% ९७

\twolineshloka
{इयं वसुमती राम मानुषैर्बहुबाहुभिः}
{उच्छ्रितैराचिता सर्वा तदाप्रभृति मद्गृहम्}% ९८

\twolineshloka
{उलूकस्त्वब्रवीद्रामं पादपैरुपशोभिता}
{यदैव पृथिवी राजंस्तदाप्रभृति मे गृहम्}% ९९

\twolineshloka
{एतच्छ्रुत्वा तु रामो वै सभासद उवाचह}
{न सा सभा यत्र न सन्ति वृद्धा वृद्धा न ते ये न वदन्ति धर्मम्}% १००

\twolineshloka
{नासौ धर्मो यत्र न चास्ति सत्यं न तत्सत्यं यच्छलमभ्युपैति}
{ये तु सभ्याः सभां गत्वा तूष्णीं ध्यायन्त आसते}% १०१

\twolineshloka
{यथाप्राप्तं न ब्रुवते सर्वे तेऽनृतवादिनः}
{न वक्ति च श्रुतं यश्च कामात्क्रोधात्तथा भयात्}% १०२

\twolineshloka
{सहस्रं वारुणाः पाशाः प्रतिमुञ्चन्ति तं नरम्}
{तेषां संवत्सरे पूर्णे पाश एकः प्रमुच्यते}% १०३

\twolineshloka
{तस्मात्सत्यं तु वक्तव्यं जानता सत्यमञ्जसा}
{एतच्छ्रुत्वा तु सचिवा राममेवाब्रुवंस्तदा}% १०४

\twolineshloka
{उलूकः शोभते राजन्न तु गृध्रो महामते}
{त्वं प्रमाणं महाराज राजा हि परमा गतिः}% १०५

\twolineshloka
{राजमूलाः प्रजाः सर्वा राजा धर्मः सनातनः}
{शास्ता राजा नृणां येषां न ते गच्छन्ति दुर्गतिम्}% १०६

\twolineshloka
{वैवस्वतेन मुक्ताश्च भवन्ति पुरुषोत्तमाः}
{सचिवानां वचः श्रुत्वा रामो वचनमब्रवीत्}% १०७

\twolineshloka
{श्रूयतामभिधास्यामि पुराणं यदुदाहृतम्}
{द्यौः सचन्द्रार्कनक्षत्रा सपर्वतमहीद्रुमम्}% १०८

\twolineshloka
{सलिलार्णवसम्मग्नं त्रैलोक्यं सचराचरम्}
{एकमेव तदा ह्यासीत्सर्वमेकमिवाम्बरम्}% १०९

\twolineshloka
{पुनर्भूः सह लक्ष्म्या च विष्णोर्जठरमाविशत्}
{तां निगृह्य महातेजाः प्रविश्य सलिलार्णवम्}% ११०

\twolineshloka
{सुष्वाप हि कृतात्मा स बहुवर्षशतान्यपि}
{विष्णौ सुप्ते ततो ब्रह्मा विवेश जठरं ततः}% १११

\twolineshloka
{बहुस्रोतं च तं ज्ञात्वा महायोगी समाविशत्}
{नाभ्यां विष्णोः समुद्भूतं पद्मं हेमविभूषितम्}% ११२

\twolineshloka
{स तु निर्गम्य वै ब्रह्मा योगी भूत्वा महाप्रभुः}
{सिसृक्षुः पृथिवीं वायुं पर्वतांश्च महीरुहान्}% ११३

\twolineshloka
{तदन्तराः प्रजाः सर्वा मानुषांश्च सरीसृपान्}
{जरायुजाण्डजान्सर्वान्ससर्ज स महातपाः}% ११४

\twolineshloka
{तस्य गात्रसमुत्पन्नः कैटभो मधुना सह}
{दानवौ तौ महावीर्यौ घोरौ लब्धवरौ तदा}% ११५

\twolineshloka
{दृष्ट्वा प्रजापतिं तत्र क्रोधाविष्टावुभौ नृप}
{वेगेन महता भोक्तुं स्वयम्भुवमधावताम्}% ११६

\twolineshloka
{दृष्ट्वा सत्वानि सर्वाणि निस्सरन्ति पृथक्पृथक्}
{ब्रह्मणा संस्तुतो विष्णुर्हत्वा तौ मधुकैटभौ}% ११७

\twolineshloka
{पृथिवीं वर्धयामास स्थित्यर्थं मेदसा तयोः}
{मेदोगन्धा तु धरणी मेदिनीत्यभिधां गता}% ११८

\twolineshloka
{तस्माद्गृध्रस्त्वसत्यो वै पापकर्मापरालयम्}
{स्वीयं करोति पापात्मा दण्डनीयो न संशयः}% ११९

\twolineshloka
{ततोऽशरीरिणीवाणी अन्तरिक्षात्प्रभाषते}
{मा वधी राम गृध्रं त्वं पूर्वन्दग्धं तपोबलात्}% १२०

\twolineshloka
{पुरा गौतम दग्धोऽयं प्रजानाथो जनेश्वर}
{ब्रह्मदत्तस्तु नामैष शूरः सत्यव्रतः शुचिः}% १२१

\twolineshloka
{गृहमागत्य विप्रर्षेर्भोजनं प्रत्ययाचत}
{साग्रं वर्षशतं चैव भुक्तवान्नृपसत्तम}% १२२

\twolineshloka
{ब्रह्मदत्तस्य वै तस्य पाद्यमर्घ्यं स्वयं ततः}
{आत्मनैवाकरोत्सम्यग्भोजनार्थं महाद्युते}% १२३

\twolineshloka
{समाविश्य गृहं तस्य आहारे तु महात्मनः}
{नारीं पूर्णस्तनीं दृष्ट्वा हस्तेनाथ परामृशत्}% १२४

\twolineshloka
{अथ क्रुद्धेन मुनिना शापो दत्तः सुदारुणः}
{गृध्रत्वं गच्छ वै मूढ राजा मुनिमथाब्रवीत्}% १२५

\twolineshloka
{कृपां कुरु महाभाग शापोद्धारो भविष्यति}
{दयालुस्तद्वचः श्रुत्वा पुनराह नराधिप}% १२६

\twolineshloka
{उत्पत्स्यते रघुकुले रामो नाम महायशाः}
{इक्ष्वाकूणां महाभागो राजा राजीवलोचनः}% १२७

\twolineshloka
{तेन दृष्टो विपापस्त्वं भविता नरपुङ्गव}
{दृष्टो रामेण तच्छ्रुत्वा बभूव पृथिवीपतिः}% १२८

\twolineshloka
{गृध्रत्वं त्यज्य वै शीघ्रं दिव्यगन्धानुलेपनः}
{पुरुषो दिव्यरूपोऽसौ बभाषे तं नराधिपम्}% १२९

\twolineshloka
{साधु राघव धर्मज्ञ त्वत्प्रसादादहं विभो}
{विमुक्तो नरकाद्घोरादपापस्तु त्वया कृतः}% १३०

\twolineshloka
{विसर्जितं मया गार्ध्यं नररूपी महीपतिः}
{उलूकं प्राह धर्मज्ञ स्वगृहं विश कौशिक}% १३१

\twolineshloka
{अहं सन्ध्यामुपासित्वा गमिष्ये यत्र वै मुनिः}
{अथोदकमुपस्पृश्य सन्ध्यामन्वास्य पश्चिमाम्}% १३२

\twolineshloka
{आश्रमं प्राविशद्रामः कुम्भयोनेर्महात्मनः}
{तस्यागस्त्यो बहुगुणं फलमूलं च सादरम्}% १३३

\twolineshloka
{रसवन्ति च शाकानि भोजनार्थमुपाहरत्}
{सभुक्तवान्नरव्याघ्रस्तदन्नममृतोपमम्}% १३४

\twolineshloka
{प्रीतश्च परितुष्टश्च तां रात्रिं समुपावसत्}
{प्रभाते काल्यमुत्थाय कृत्वाह्निकमरिन्दम}% १३५

\twolineshloka
{ॠषिं समभिचक्राम गमनाय रघूत्तमः}
{अभिवाद्याब्रवीद्रामो महर्षिं कुम्भसम्भवम्}% १३६

\twolineshloka
{आपृच्छे साधये ब्रह्मन्ननुज्ञातुं त्वमर्हसि}
{धन्योस्म्यनुगृहीतोस्मि दर्शनेन महामुने}% १३७

\twolineshloka
{दिष्ट्या चाहं भविष्यामि पावनात्मा महात्मनः}
{एवं ब्रुवति काकुत्स्थे वाक्यमद्भुतदर्शनम्}% १३८

\twolineshloka
{उवाच परमप्रीतो बाष्पनेत्रस्तपोधनः}
{अत्यद्भुतमिदं वाक्यं तव राम शुभाक्षरम्}% १३९

\twolineshloka
{पावनं सर्वभूतानां त्वयोक्तं रघुनन्दन}
{मुहूर्तमपि राम त्वां मैत्रेणेक्षन्ति ये नराः}% १४०

\twolineshloka
{पावितास्सर्वसूक्तैस्ते कथ्यन्ते त्रिदिवौकसः}
{ये च त्वां घोरचक्षुर्भिरीक्षन्ते प्राणिनो भुवि}% १४१

\twolineshloka
{ते हता ब्रह्मदण्डेन सद्यो नरकगामिनः}
{ईदृशस्त्वं रघुश्रेष्ठ पावनः सर्वदेहिनाम्}% १४२

\twolineshloka
{कथयन्तश्च लोकास्त्वां सिद्धिमेष्यन्ति राघव}
{गच्छस्वानातुरोऽविघ्नं पन्थानमकुतोभयः}% १४३

\twolineshloka
{प्रशाधि राज्यं धर्मेण गतिस्तु जगतां भवान्}
{एवमुक्तस्तु मुनिना प्राञ्जलि प्रग्रहो नृपः}% १४४

\twolineshloka
{अभिवादयितुं चक्रे सोऽगस्त्यमृषिसत्तमम्}
{अभिवाद्य मुनिश्रेष्ठंस्तांश्च सर्वांस्तपोधिकान्}% १४५

\twolineshloka
{अथारोहत्तदाव्यग्रः पुष्पकं हेमभूषितम्}
{तं प्रयान्तं मुनिगणा आशीर्वादैस्समन्ततः}% १४६

\twolineshloka
{अपूपुजन्नरेन्द्रं तं सहस्राक्षमिवामराः}
{ततोऽर्धदिवसे प्राप्ते रामः सर्वार्थकोविदः}% १४७

\twolineshloka
{अयोध्यां प्राप्य काकुत्स्थः पद्भ्यां कक्षामवातरत्}
{ततो विसृज्य रुचिरं पुष्पकं कामवाहितम्}% १४८

\twolineshloka
{कक्षान्तराद्विनिष्क्रम्य द्वास्थान्राजाऽब्रवीदिदम्}
{लक्ष्मणं भरतं चैव गच्छध्वं लघुविक्रमाः}% १४९

\twolineshloka
{ममागमनमाख्याय समानयत मा चिरम्}
{श्रुत्वाथ भाषितं द्वास्था रामस्याक्लिष्टकर्मणः1.37.}% १५०

\twolineshloka
{गत्वा कुमारावाहूय राघवाय न्यवदेयन्}
{द्वास्थैः कुमारावानीतौ राघवस्य निदेशतः}% १५१

\twolineshloka
{दृष्ट्वा तु राघवः प्राप्तौ प्रियौ भरतलक्ष्मणौ}
{समालिङ्ग्य तु रामस्तौ वाक्यं चेदमुवाच ह}% १५२

\twolineshloka
{कृतं मया यथातथ्यं द्विजकार्यमनुत्तमम्}
{धर्महेतुमतो भूयः कर्तुमिच्छामि राघवौ}% १५३

\twolineshloka
{भवद्भ्यामात्मभूताभ्यां राजसूयं क्रतूत्तमम्}
{सहितो यष्टुमिच्छामि यत्र धर्मश्च शाश्वतः}% १५४

\twolineshloka
{पुष्करस्थेन वै पूर्वं ब्रह्मणा लोककारिणा}
{शतत्रयेण यज्ञानामिष्टं षष्ट्याधिकेन च}% १५५

\twolineshloka
{इष्ट्वा हि राजसूयेन सोमो धर्मेण धर्मवित्}
{प्राप्तः सर्वेषु लोकेषु कीर्तिस्थानमनुत्तमम्}% १५६

\twolineshloka
{इष्ट्वा हि राजसूयेन मित्रः शत्रुनिबर्हणः}
{मुहूर्तेन सुशुद्धेन वरुणत्वमुपागतः}% १५७

\onelineshloka*
{तस्माद्भवन्तौ सञ्चिन्त्य कार्येस्मिन्वदतं हि तत्}

\uvacha{भरत उवाच}

\onelineshloka
{त्वं धर्मः परमः साधो त्वयि सर्वा वसुन्धरा}% १५८

\twolineshloka
{प्रतिष्ठिता महाबाहो यशश्चामितविक्रम}
{महीपालाश्च सर्वे त्वां प्रजापतिमिवामराः}% १५९

\twolineshloka
{निरीक्षन्ते महात्मानो लोकनाथ तथा वयम्}
{प्रजाश्च पितृवद्राजन्पश्यन्ति त्वां महामते}% १६०

\twolineshloka
{पृथिव्यां गतिभूतोसि प्राणिनामिह राघव}
{सत्वमेवंविधं यज्ञं नाहर्त्तासि परन्तप}% १६१

\twolineshloka
{पृथिव्यां सर्वभूतानां विनाशो दृश्यते यतः}
{श्रूयते राजशार्दूल सोमस्य मनुजेश्वर}% १६२

\twolineshloka
{ज्योतिषां सुमहद्युद्धं सङ्ग्रामे तारकामये}
{तारा बृहस्पतेर्भार्या हृता सोमेनकामतः}% १६३

\twolineshloka
{तत्र युद्धं महद्वृत्तं देवदानवनाशनम्}
{वरुणस्य क्रतौ घोरे सङ्ग्रामे मत्स्यकच्छपाः}% १६४

\twolineshloka
{निवृत्ते राजशार्दूल सर्वे नष्टा जलेचराः}
{हरिश्चन्द्रस्य यज्ञान्ते राजसूयस्य राघव}% १६५

\twolineshloka
{आडीबकम्महद्युद्धं सर्वलोकविनाशनम्}
{पृथिव्यां यानि सत्वानि तिर्यग्योनिगतानि वै}% १६६

\twolineshloka
{दिव्यानां पार्थिवानां च राजसूये क्षयः श्रुतः}
{स त्वं पुरुषशार्दूल बुद्ध्या सञ्चिन्त्य पार्थिव}% १६७

\twolineshloka
{प्राणिनां च हितं सौम्यं पूर्णधर्मं समाचर}
{भरतस्य वचः श्रुत्वा राघवः प्राह सादरम्}% १६८

\twolineshloka
{प्रीतोस्मि तव धर्मज्ञ वाक्येनानेन शत्रुहन्}
{निवर्तिता राजसूयान्मतिर्मे धर्मवत्सल}% १६९

\twolineshloka
{पूर्णं धर्मं करिष्यामि कान्यकुब्जे च वामनम्}
{स्थापयिष्याम्यहं वीर सा मे ख्यातिर्दिवं गता}% १७०

\onelineshloka
{भविष्यति न सन्देहो यथा गङ्गा भगीरथात्}% १७१

{॥इति श्रीपाद्मपुराणे प्रथमे सृष्टिखण्डे यज्ञनिवारणं नाम सप्तत्रिंशोऽध्यायः॥३७॥}

    \sect{वामनप्रतिष्ठा}

\src{पद्म-पुराणम्}{सृष्टिखण्डम्}{अध्यायः ३८}{१--१९४}
% \tags{concise, complete}
\notes{This chapter describes the conversation between Rama and Agastya. It narrates how Rama, after defeating Ravana, meets Agastya in the forest. Agastya explains the significance of the divine ornaments given to Rama.}
\textlink{https://sa.wikisource.org/wiki/पद्मपुराणम्/खण्डः_१_(सृष्टिखण्डम्)/अध्यायः_३८}
\translink{https://www.wisdomlib.org/hinduism/book/the-padma-purana/d/doc364161.html}

\storymeta


\uvacha{भीष्म उवाच}

\twolineshloka
{कथं रामेण विप्रर्षे कान्यकुब्जे तु वामनः}
{स्थापितः क्व च लब्धोसौ विस्तरान्मम कीर्तय}% १

\twolineshloka
{तथा हि मधुरा चैषा या वाणी रामकीर्तने}
{कीर्तिता भगवन्मह्यं हृता कर्णसुखावह}% २

\twolineshloka
{अनुरागेण तं लोकाः स्नेहात्पश्यन्ति राघवम्}
{धर्मज्ञश्च कृतज्ञश्च बुद्ध्या च परिनिष्ठितः}% ३

\twolineshloka
{प्रशास्ति पृथिवीं सर्वां धर्मेण सुसमाहितः}
{तस्मिन्शासति वै राज्यं सर्वकामफलाद्रुमाः}% ४

\twolineshloka
{रसवन्तः प्रभूताश्च वासांसि विविधानि च}
{अकृष्टपच्या पृथिवी निःसपत्ना महात्मनः}% ५

\twolineshloka
{देवकार्यं कृतं तेन रावणो लोककण्टकः}
{सपुत्रोमात्यसहितो लीलयैव निपातितः}% ६

\twolineshloka
{तस्यबुद्धिस्समुत्पन्ना पूर्णे धर्मे द्विजोत्तम}
{तस्याहं चरितं सर्वं श्रोतुमिच्छामि वै मुने}% ७

\uvacha{पुलस्त्य उवाच}

\twolineshloka
{कस्यचित्त्वथ कालस्य रामो धर्मपथे स्थितः}
{यच्चकार महाबाहो शृणुष्वैकमना नृप}% ८

\twolineshloka
{सस्मार राक्षसेन्द्रं तं कथं राजा विभीषणः}
{लङ्कायां संस्थितो राज्यं करिष्यति च राक्षसः}% ९

\twolineshloka
{गीर्वाणेषु प्रातिकूल्यं विनाशस्य तु लक्षणम्}
{मया तस्य तु तद्दत्तं राज्यं चन्द्रार्ककालिकम्}% १०

\twolineshloka
{तस्याविनाशतः कीर्तिः स्थिरा मे शाश्वती भवेत्}
{रावणेन तपस्तप्तं विनाशायात्मनस्त्विह}% ११

\twolineshloka
{विध्वस्तः स च पापिष्ठो देवकार्ये मयाधुना}
{तदिदानीं मयान्वेष्यः स्वयं गत्वा विभीषणः}% १२

\twolineshloka
{सन्देष्टव्यं हितं तस्य येन तिष्ठेत्स शाश्वतम्}
{एवं चिन्तयतस्तस्य रामस्यामिततेजसः}% १३

\twolineshloka
{आजगामाथ भरतो रामं दृष्ट्वाब्रवीदिदम्}
{किं त्वं चिन्तयसे देव न रहस्यं वदस्व मे}% १४

\twolineshloka
{देवकार्ये धरायां वा स्वकार्ये वा नरोत्तम}
{एवं ब्रुवन्तं भरतं ध्यायमानमवस्थितम्}% १५

\twolineshloka
{अब्रवीद्राघवो वाक्यं रहस्यं तु न वै तव}
{भवान्बहिश्चरः प्राणो लक्ष्मणश्च महायशाः}% १६

\twolineshloka
{अवेद्यं भवतो नास्ति मम सत्यं विधारय}
{एषा मे महती चिन्ता कथं देवैर्विभीषणः}% १७

\twolineshloka
{वर्तते यद्धितार्थं वै दशग्रीवो निपातितः}
{गमिष्ये तदहं लङ्कां यत्र चासौ विभीषणः}% १८

\twolineshloka
{तं च दृष्ट्वा पुरीं तां तु कार्यमुक्त्वा च राक्षसम्}
{आलोक्य सर्ववसुधां सुग्रीवं वानरेश्वरम्}% १९

\twolineshloka
{महाराजं च शत्रुघ्नं भातृपुत्रांश्च सर्वशः}
{एवं वदति काकुत्स्थे भरतः पुरतः स्थितः}% २०

\twolineshloka
{उवाच राघवं वाक्यं गमिष्ये भवता सह}
{एवं कुरु महाबाहो सौमित्रिरिह तिष्ठतु}% २१

\twolineshloka
{इत्युक्त्वा भरतं रामः सौमित्रं चाह वै पुरे}
{रक्षाकार्या त्वया वीर यावदागमनं हि नौ}% २२

\twolineshloka
{एवं लक्ष्मणमादिश्य ध्यात्वा वै पुष्पकं नृप}
{आरुरोह स वै यानं कौसल्यानन्दवर्धनः}% २३

\twolineshloka
{पुष्पकं तु ततः प्राप्तं गान्धारविषयो यतः}
{भरतस्य सुतौ दृष्ट्वा जगन्नीतिं निरीक्ष्य च}% २४

\twolineshloka
{पूर्वां दिशं ततो गत्वा लक्ष्मणस्य सुतौ यतः}
{पुरेषु तेषु षड्रात्रमुषित्वा रघुनन्दनौ}% २५

\twolineshloka
{गतौ तेन विमानेन दक्षिणामभितो दिशम्}
{गङ्गायामुनसम्भेदं प्रयागमृषिसेवितम्}% २६

\twolineshloka
{अभिवाद्य भरद्वाजमत्रेराश्रममीयतुः}
{सम्भाष्य च मुनींस्तत्र जनस्थानमुपागतौ}% २७

\uvacha{राम उवाच}

\twolineshloka
{अत्र पूर्वं हृता सीता रावणेन दुरात्मना}
{हत्वा जटायुषं गृध्रं योसौ पितृसखो हि नौ}% २८

\twolineshloka
{अत्रास्माकं महद्युद्धं कबन्धेन कुबुद्धिना}
{हतेन तेन दग्धेन सीतास्ते रावणालये}% २९

\twolineshloka
{ॠष्यमूके गिरिवरे सुग्रीवो नाम वानरः}
{स ते करिष्यते साह्यं पम्पां व्रज सहानुजः}% ३०

\twolineshloka
{पम्पासरः समासाद्य शबरीं गच्छ तापसीम्}
{इत्युक्तो दुःखितो वीर निराशो जीविते स्थितः}% ३१

\twolineshloka
{इयं सा नलिनी वीर यस्यां वै लक्ष्मणोवदत्}
{मा कृथाः पुरुषव्याघ्र शोकं शत्रुविनाशन}% ३२

\twolineshloka
{आज्ञाकारिणि भृत्ये च मयि प्राप्स्यसि मैथिलीम्}
{अत्र मे वार्षिका मासा गता वर्षशतोपमाः}% ३३

\twolineshloka
{अत्रैव निहतो वाली सुग्रीवार्थे परन्तप}
{एषा सा दृश्यते नूनं किष्किन्धा वालिपालिता}% ३४

\twolineshloka
{यस्यां वै स हि धर्मात्मा सुग्रीवो वानरेश्वरः}
{वानरैः सहितो वीर तावदास्ते समाः शतम्}% ३५

\twolineshloka
{वानरैस्सह सुग्रीवो यावदास्ते सभां गतः}
{तावत्तत्रागतौ वीरौ पुर्यां भरतराघवौ}% ३६

\twolineshloka
{दृष्ट्वा स भ्रातरौ प्राप्तौ प्रणिपत्याब्रवीदिदम्}
{क्व युवां प्रस्थितौ वीरौ कार्यं किं नु करिष्यथः}% ३७

\twolineshloka
{विनिवेश्यासने तौ च ददावर्घ्ये स्वयं तदा}
{एवं सभास्थिते तत्र धर्मिष्टे रघुनन्दने}% ३८

\twolineshloka
{अङ्गदोथ हनूमांश्च नलो नीलश्च पाटलः}
{गजो गवाक्षो गवयः पनसश्च महायशाः}% ३९

\twolineshloka
{पुरोधसो मन्त्रिणश्च दैवज्ञो दधिवक्रकः}
{नीलश्शतबलिर्मैन्दो द्विविदो गन्धमादनः}% ४०

\twolineshloka
{वीरबाहुस्सुबाहुश्च वीरसेनो विनायकः}
{सूर्याभः कुमुदश्चैव सुषेणो हरियूथपः}% ४१

\twolineshloka
{ॠषभो विनतश्चैव गवाख्यो भीमविक्रमः}
{ॠक्षराजश्च धूम्रश्च सहसैन्यैरुपागताः}% ४२

\twolineshloka
{अन्तःपुराणि सर्वाणि रुमा तारा तथैव च}
{अवरोधोङ्गदस्यापि तथान्याः परिचारिकाः}% ४३

\twolineshloka
{प्रहर्षमतुलं प्राप्य साधुसाध्विति चाब्रुवन्}
{वानराश्च महात्मानः सुग्रीवसहितास्तदा}% ४४

\twolineshloka
{वानर्यश्च महाभागास्ताराद्यास्तत्र राघवम्}
{अभिप्रेक्ष्याश्रुकण्ठ्यश्च प्रणिपत्येदमब्रुवन्}% ४५

\twolineshloka
{क्व सा देवी त्वया देव या विनिर्जित्यरावणम्}
{शुद्धिं कृत्वा हि ते वह्नौ पितुरग्र उमापतेः}% ४६

\twolineshloka
{त्वयानीता पुरीं राम न तां पश्यामि तेग्रतः}
{न विना त्वं तया देव शोभसे रघुनन्दन}% ४७

\twolineshloka
{त्वया विनापि साध्वी सा क्व नु तिष्ठति जानकी}
{अन्यां भार्यां न ते वेद्मि भार्याहीनो न शोभसे}% ४८

\twolineshloka
{क्रौञ्चयुग्मं मिथो यद्वच्चक्रवाकयुगं यथा}
{एवं वदन्तीं तां तारां ताराधिपसमाननाम्}% ४९

\twolineshloka
{प्राह प्रवचसां श्रेष्ठो रामो राजीवलोचनः}
{चारुदंष्ट्रे विशालाक्षि कालो हि दुरतिक्रमः1.38.}% ५०

\twolineshloka
{सर्वं कालकृतं विद्धि जगदेतच्चराचरम्}
{विसृज्यताः स्त्रियः सर्वाः सुग्रीवोभिमुखः स्थितः}% ५१

\uvacha{सुग्रीव उवाच}

\twolineshloka
{भवन्तौ येन कार्येण इहायातौ नरेश्वरौ}
{तच्चापि कथ्यतां शीघ्रं कृत्यकालो हि वर्तते}% ५२


\threelineshloka
{ब्रुवाणमेवं सुग्रीवं भरतो रामचोदितः}
{आचचक्षे च गमनं लङ्कायां राघवस्य तु}
{तौ चाब्रवीच्च सुग्रीवो भवद्भ्यां सहितः पुरीम्}% ५३

\twolineshloka
{गमिष्ये राक्षसं देव द्रष्टुं तत्र विभीषणम्}
{सुग्रीवेणैवमुक्ते तु गच्छस्वेत्याह राघवः}% ५४

\twolineshloka
{सुग्रीवो राघवौ तौ च पुष्पके तु स्थितास्त्रयः}
{तावत्प्राप्तं विमानं तु समुद्रस्योत्तरं तटम्}% ५५

\twolineshloka
{अब्रवीद्भरतं रामो ह्यत्र मे राक्षसेश्वरः}
{चतुर्भिः सचिवैः सार्धं जीवितार्थे विभीषणः}% ५६

\twolineshloka
{प्राप्तस्ततो लक्ष्मणेन लङ्काराज्येभिषेचितः}
{अत्र चाहं समुद्रस्य परेपारे स्थितस्त्र्यहम्}% ५७

\twolineshloka
{दर्शनं दास्यते मेऽसौ ज्ञातिकार्यं भविष्यति}
{तावन्न दर्शनं मह्यं दत्तमेतेन शत्रुहन्}% ५८

\twolineshloka
{ततः कोपः सुमद्भूतश्चतुर्थेहनि राघव}
{धनुरायम्य वेगेन दिव्यमस्त्रं करे धृतम्}% ५९

\twolineshloka
{दृष्ट्वा मां शरणान्वेषी भीतो लक्ष्मणमाश्रितः}
{सुग्रीवेणानुनीतोऽस्मि क्षम्यतां राघव त्वया}% ६०

\twolineshloka
{ततो मयोत्क्षिप्तशरो मरुदेशे ह्यपाकृतः}
{ततस्समुद्रराजेन भृशं विनयशालिना}% ६१

\twolineshloka
{उक्तोहं सेतुबन्धेन लङ्कां त्वं व्रज राघव}
{लङ्घयित्वा नरव्याघ्र वारिपूर्णं महोदधिम्}% ६२

\twolineshloka
{एष सेतुर्मया बद्धः समुद्रे वरुणालये}
{त्रिभिर्दिनैः समाप्तिं वै नीतो वानरसत्तमैः}% ६३

\twolineshloka
{प्रथमे दिवसे बद्धो योजनानि चतुर्दश}
{द्वितीयेहनि षट्त्रिंशत्तृतीयेर्धशतं तथा}% ६४

\twolineshloka
{इयं सा दृश्यते लङ्का स्वर्णप्राकारतोरणा}
{अवरोधो महानत्र कृतो वानरसत्तमैः}% ६५

\twolineshloka
{अत्र युद्धं महद्वृत्तं चैत्राशुक्लचतुर्दशि}
{अष्टचत्वारिंशद्दिनं यत्रासौ रावणो हतः}% ६६

\twolineshloka
{अत्र प्रहस्तो नीलेन हतो राक्षसपुङ्गवः}
{हनूमता च धूम्राक्षो ह्यत्रैव विनिपातितः}% ६७

\twolineshloka
{महोदरातिकायौ च सुग्रीवेण महात्मना}
{अत्रैव मे कुम्भकर्णो लक्ष्मणेनेन्द्रजित्तथा}% ६८

\twolineshloka
{मया चात्र दशग्रीवो हतो राक्षसपुङ्गवः}
{अत्र सम्भाषितुं प्राप्तो ब्रह्मा लोकपितामहः}% ६९

\twolineshloka
{पार्वत्या सहितो देवः शूलपाणिर्वृषध्वजः}
{महेन्द्राद्याः सुरगणाः सगन्धर्वास्स किन्नराः}% ७०

\twolineshloka
{पिता मे च समायातो महाराजस्त्रिविष्टपात्}
{वृतश्चाप्सरसां सङ्घैर्विद्याधरगणैस्तथा}% ७१

\twolineshloka
{तेषां समक्षं सर्वेषां जानकी शुद्धिमिच्छता}
{उक्ता सीता हव्यवाहं प्रविष्टा शुद्धिमागता}% ७२

\twolineshloka
{लङ्काधिपैः सुरैर्दृष्टा गृहीता पितृशासनात्}
{अथाप्युक्तोथ राज्ञाहमयोध्यां गच्छ पुत्रकम्}% ७३

\twolineshloka
{न मे स्वर्गो बहुमतस्त्वया हीनस्य राघव}
{तारितोहं त्वया पुत्र प्राप्तोऽस्मीन्द्रसलोकताम्}% ७४

\twolineshloka
{लक्ष्मणं चाब्रवीद्राजा पुत्र पुण्यं त्वयार्जितम्}
{भ्रात्रासममथो दिव्यांल्लोकान्प्राप्स्यसि चोत्तमान्}% ७५

\twolineshloka
{आहूय जानकीं राजा वाक्यं चेदमुवाच ह}
{न च मन्युस्त्वया कार्यो भर्तारं प्रति सुव्रते}% ७६

\twolineshloka
{ख्यातिर्भविष्यत्येवाग्र्या भर्तुस्ते शुभलोचने}
{एवं वदति रामे तु पुष्पके च व्यवस्थिते}% ७७

\twolineshloka
{तत्र ये राक्षसवरास्ते गत्वाशु विभीषणम्}
{प्राप्तो रामः ससुग्रीवश्चारा इत्थं तदाऽवदन्}% ७८

\twolineshloka
{विभीषणस्तु तच्छ्रुत्वा रामागमनमन्तिके}
{चारांस्तान्पूजयामास सर्वकामधनादिभिः}% ७९

\twolineshloka
{अलङ्कृत्य पुरीं तां तु निष्क्रान्तः सचिवैः सह}
{दृष्ट्वा रामं विमानस्थं मेराविव दिवाकरम्}% ८०

\twolineshloka
{अष्टाङ्गप्रणिपातेन नत्वा राघवमब्रवीत्}
{अद्य मे सफलं जन्म प्राप्ताः सर्वे मनोरथाः}% ८१

\twolineshloka
{यद्दृष्टौ देवचरणौ जगद्वन्द्यावनिन्दितौ}
{कृतः श्लाघ्योस्म्यहं देव शक्रादीनां दिवौकसाम्}% ८२

\twolineshloka
{आत्मानमधिकं मन्ये त्रिदशेशात्पुरन्दरात्}
{रावणस्य गृहे दीप्ते सर्वरत्नोपशोभिते}% ८३

\twolineshloka
{उपविष्टे तु काकुत्स्थे अर्घं दत्वा विभीषणः}
{उवाच प्राञ्जलिर्भूत्वा सुग्रीवं भरतं तथा}% ८४

\twolineshloka
{इहागतस्य रामस्य यद्दास्ये न तदस्ति मे}
{इयं च लङ्का रामेण रिपुं त्रैलोक्यकण्टकम्}% ८५

\twolineshloka
{हत्वा तु पापकर्माणं दत्ता पूर्वं पुरी मम}
{इयं पुरी इमे दारा अमी पुत्रास्तथा ह्यहम्}% ८६

\twolineshloka
{सर्वमेतन्मया दत्तं सर्वमक्षयमस्तु ते}
{ततः प्रकृतयः सर्वा लङ्कावासिजनाश्च ये}% ८७

\twolineshloka
{आजग्मू राघवं द्रष्टुं कौतूहलसमन्विताः}
{उक्तो विभीषणस्तैस्तु रामं दर्शय नः प्रभो}% ८८

\twolineshloka
{विभीषणेन कथिता राघवाय महात्मने}
{तेषामुपायनं सर्वं भरतो रामचोदितः}% ८९

\twolineshloka
{जग्राह वानरेन्द्रश्च धनरत्नौघसञ्चयम्}
{एवं तत्र त्र्यहं रामो ह्यवसद्राक्षसालये}% ९०

\twolineshloka
{चतुर्थेहनि सम्प्राप्ते रामे चापि सभास्थिते}
{केकसी पुत्रमाहेदं रामं द्रक्ष्यामि पुत्रक}% ९१

\twolineshloka
{दृष्टे तस्मिन्महत्पुण्यं प्राप्यते मुनिसत्तमैः}
{विष्णुरेष महाभागश्चतुर्मूर्तिस्सनातनः}% ९२

\twolineshloka
{सीता लक्ष्मीर्महाभाग न बुद्धा साग्रजेन ते}
{पित्रा ते पूर्वमाख्यातं देवानां दिविसङ्गमे}% ९३

\twolineshloka
{कुले रघूणां वै विष्णुः पुत्रो दशरथस्य तु}
{भविष्यति विनाशाय दशग्रीवस्य रक्षसः}% ९४

\uvacha{विभीषण उवाच}

\twolineshloka
{एवं कुरुष्व वै मातर्गृहाण नवमं वरम्}
{पात्रं चन्दनसंयुक्तं दधिक्षौद्राक्षतैः सह}% ९५

\twolineshloka
{दूर्वयार्घं सह कुरु राजपुत्रस्य दर्शनम्}
{सरमामग्रतः कृत्वा याश्चान्या देवकन्यकाः}% ९६

\twolineshloka
{व्रजस्व राघवाभ्याशं तस्मादग्रे व्रजाम्यहम्}
{एवमुक्त्वा गतं रक्षो यत्र रामो व्यवस्थितः}% ९७

\twolineshloka
{उत्सार्य दानवान्सर्वान्रामं द्रष्टुं समागतान्}
{सभां तां विमलां कृत्वा रामं स्वाभिमुखे स्थितम्}% ९८

\uvacha{विभीषण उवाच}

\twolineshloka
{विज्ञाप्यं शृणु मे देव वदतश्च विशाम्पते}
{दशग्रीवं कुम्भकर्णं या च मां चाप्यजीजनत्}% ९९

\twolineshloka
{इयं सा देवमाता नः पादौ ते द्रष्टुमिच्छति}
{तस्यास्तु त्वं कृपां कृत्वा दर्शनं दातु मर्हसि1.38.}% १००

\uvacha{राम उवाच}

\twolineshloka
{अहं तस्याः समीपं तु मातृदर्शनकाङ्क्षया}
{गमिष्ये राक्षसेन्द्र त्वं शीघ्रं याहि ममाग्रतः}% १०१

\twolineshloka
{प्रतिज्ञाय तु तं वाक्यमुत्तस्थौ च वरासनात्}
{मूर्ध्नि चाञ्जलिमाधाय प्रणाममकरोद्विभुः}% १०२

\twolineshloka
{अभिवादयेहं भवतीं माता भवसि धर्मतः}
{महता तपसा चापि पुण्येन विविधेन च}% १०३

\twolineshloka
{इमौ ते चरणौ देवि मानवो यदि पश्यति}
{पूर्णस्स्यात्तदहं प्रीतो दृष्ट्वेमौ पुत्रवत्सले}% १०४

\twolineshloka
{कौसल्या मे यथा माता भवती च तथा मम}
{केकसी चाब्रवीद्रामं चिरं जीव सुखी भव}% १०५

\twolineshloka
{भर्त्रा मे कथितं वीर विष्णुर्मानुषरूपधृत्}
{अवतीर्णो रघुकुले हितार्थेत्र दिवौकसाम्}% १०६

\twolineshloka
{दशग्रीव विनाशाय भूतिं दातुं विभीषणे}
{वालिनो निधनं चैव सेतुबन्धं च सागरे}% १०७

\twolineshloka
{पुत्रो दशरथस्यैव सर्वं स च करिष्यति}
{इदानीं त्वं मया ज्ञातः स्मृत्वा तद्भर्तृभाषितम्}% १०८

\twolineshloka
{सीता लक्ष्मीर्भवान्विष्णुर्देवा वै वानरास्तथा}
{गृहं पुत्र गमिष्यामि स्थिरकीर्तिमवाप्नुहि}% १०९

\uvacha{सरमोवाच}

\twolineshloka
{इहैव वत्सरं पूर्णमशोकवनिकास्थिता}
{सेविता जानकी देव सुखं तिष्ठति ते प्रिया}% ११०

\twolineshloka
{नित्यं स्मरामि वै पादौ सीतायास्तु परन्तप}
{कदा द्रक्ष्यामि तां देवीं चिन्तयाना त्वहर्निशम्}% १११

\twolineshloka
{किमर्थं देवदेवेन नानीता जानकी त्विह}
{एकाकी नैव शोभेथा योषिता च तया विना}% ११२

\twolineshloka
{समीपे शोभते सीता त्वं च तस्याः परन्तप}
{एवं ब्रुवन्त्यां भरतः केयमित्यब्रवीद्वचः}% ११३

\twolineshloka
{ततश्चेङ्गितविद्रामो भरतं प्राह सत्वरम्}
{विभीषणस्य भार्या वै सरमा नाम नामतः}% ११४

\twolineshloka
{प्रिया सखी महाभागा सीतायास्सुदृढं मता}
{सर्वङ्कालकृतं पश्य न जाने किं करिष्यति}% ११५

\twolineshloka
{गच्छ त्वं सुभगे भर्तृगेहं पालय शोभने}
{मां त्यक्त्वा हि गता देवी भाग्यहीनं गतिर्यथा}% ११६

\twolineshloka
{तया विरहितः सुभ्रु रतिं विन्दे न कर्हिचित्}
{शून्या एव दिशः सर्वाः पश्यामीह पुनर्भ्रमन्}% ११७

\twolineshloka
{विसृज्यतां च सरमां सीतायास्तु प्रियां सखीम्}
{गतायामथ केकस्यां रामः प्राह विभीषणम्}% ११८

\twolineshloka
{दैवतेभ्यः प्रियं कार्यं नापराध्यास्त्वया सुराः}
{आज्ञया राजराजस्य वर्तितव्यं त्वयानघ}% ११९

\twolineshloka
{लङ्कायां मानुषो यो वै समागच्छेत्कथञ्चन}
{राक्षसैर्न च हन्तव्यो द्रष्टव्योसौ यथा त्वहम्}% १२०

\uvacha{विभीषण उवाच}

\twolineshloka
{आज्ञयाहं नरव्याघ्र करिष्ये सर्वमेव तु}
{विभीषणे हि वदति वायू राममुवाच ह}% १२१

\twolineshloka
{इहास्तिवैष्णवी मूर्तिः पूर्वं बद्धो बलिर्यया}
{तां नयस्व महाभाग कान्यकुब्जे प्रतिष्ठय}% १२२

\twolineshloka
{विदित्वा तदभिप्रायं वायुना समुदाहृतम्}
{विभीषणस्त्वलङ्कृत्य रत्नैः सर्वैश्च वामनम्}% १२३

\twolineshloka
{आनीय चार्पयद्रामे वाक्यं चेदमुवाच ह}
{यदा वै निर्जितः शक्रो मेघनादेन राघव}% १२४

\twolineshloka
{तदा वै वामनस्त्वेष आनीतो जलजेक्षण}
{नयस्व तमिमं देव देवदेवं प्रतिष्ठय}% १२५

\twolineshloka
{तथेति राघवः कृत्वा पुष्पकं च समारुहत्}
{धनं रत्नमसङ्ख्येयं वामनं च सुरोत्तमम्}% १२६

\twolineshloka
{गृह्य सुग्रीवभरतावारूढौ वामनादनु}
{व्रजन्नेवाम्बरे रामस्तिष्ठेत्याह विभीषणम्}% १२७

\twolineshloka
{राघवस्य वचः श्रुत्वा भूयोप्याह स राघवम्}
{करिष्ये सर्वमेतद्धि यदाज्ञप्तं विभो त्वया}% १२८

\twolineshloka
{सेतुनानेन राजेन्द्र पृथिव्यां सर्वमानवाः}
{आगत्य प्रतिबाधेरन्नाज्ञाभङ्गो भवेत्तव}% १२९

\twolineshloka
{कोत्र मे नियमो देव किन्नु कार्यं मया विभो}
{श्रुत्वैतद्राघवो वाक्यं राक्षसोत्तमभाषितम्}% १३०

\twolineshloka
{कार्मुकं गृह्य हस्तेन रामः सेतुं द्विधाच्छिनत्}
{त्रिर्विभज्य च वेगेन मध्ये वै दशयोजनम्}% १३१

\twolineshloka
{छित्वा तु योजनं चैकमेकं खण्डत्रयं कृतम्}
{वेलावनं समासाद्य रामः पूजां रमापतेः}% १३२

\twolineshloka
{कृत्वा रामेश्वरं नाम्ना देवदेवं जनार्दनम्}
{अभिषिच्याथ सङ्गृह्य वामनं रघुनन्दनः}% १३३

\twolineshloka
{दक्षिणादुदधेश्चैव निर्जगाम त्वरान्वितः}
{अन्तरिक्षादभूद्वाणी मेघगम्भीरनिःस्वना}% १३४

\uvacha{रुद्र उवाच}

\twolineshloka
{भो भो रामास्तु भद्रं ते स्थितोऽहमिह साम्प्रतम्}
{यावज्जगदिदं राम यावदेषा धरा स्थिता}% १३५

\twolineshloka
{तावदेव च ते सेतु तीर्थं स्थास्यति राघव}
{श्रुत्वैवं देवदेवस्य गिरं ताममृतोपमाम्}% १३६

\uvacha{राम उवाच}

\twolineshloka
{नमस्ते देवदेवेश भक्तानामभयङ्कर}
{गौरीकान्त नमस्तुभ्यं दक्षयज्ञविनाशन}% १३७

\twolineshloka
{नमो भवाय शर्वाय रुद्राय वरदाय च}
{पशूनाम्पतये नित्यं चोग्राय च कपर्दिने}% १३८

\twolineshloka
{महादेवाय भीमाय त्र्यम्बकाय दिशाम्पते}
{ईशानाय भगघ्नाय नमोस्त्वन्धकघातिने}% १३९

\twolineshloka
{नीलग्रीवाय घोराय वेधसे वेधसा स्तुत}
{कुमारशत्रुनिघ्नाय कुमारजननाय च}% १४०

\twolineshloka
{विलोहिताय धूम्राय शिवाय क्रथनाय च}
{नमो नीलशिखण्डाय शूलिने दैत्यनाशिने}% १४१

\twolineshloka
{उग्राय च त्रिनेत्राय हिरण्यवसुरेतसे}
{अनिन्द्यायाम्बिकाभर्त्रे सर्वदेवस्तुताय च}% १४२

\twolineshloka
{अभिगम्याय काम्याय सद्योजाताय वै नमः}
{वृषध्वजाय मुण्डाय जटिने ब्रह्मचारिणे}% १४३

\twolineshloka
{तप्यमानाय तप्याय ब्रह्मण्याय जयाय च}
{विश्वात्मने विश्वसृजे विश्वमावृत्य तिष्ठते}% १४४

\twolineshloka
{नमो नमोस्तु दिव्याय प्रपन्नार्तिहराय च}
{भक्तानुकम्पिने देव विश्वतेजो मनोगते}% १४५

\uvacha{पुलस्त्य उवाच}

\twolineshloka
{एवं संस्तूयमानस्तु देवदेवो हरो नृप}
{उवाच राघवं वाक्यं भक्तिनम्रं पुरास्थितम्}% १४६

\uvacha{रुद्र उवाच}

\twolineshloka
{भो भो राघव भद्रं ते ब्रूहि यत्ते मनोगतम्}
{भवान्नारायणो नूनं गूढो मानुषयोनिषु}% १४७

\twolineshloka
{अवतीर्णो देवकार्यं कृतं तच्चानघ त्वया}
{इदानीं स्वं व्रजस्थानं कृतकार्योसि शत्रुहन्}% १४८

\twolineshloka
{त्वया कृतं परं तीर्थं सेत्वाख्यं रघुनन्दन}
{आगत्य मानवा राजन्पश्येयुरिह सागरे}% १४९

\twolineshloka
{महापातकयुक्ता ये तेषां पापं विलीयते}
{ब्रह्मवध्यादिपापानि यानि कष्टानि कानिचित्1.38.}% १५०

\twolineshloka
{दर्शनादेव नश्यन्ति नात्र कार्या विचारणा}
{गच्छ त्वं वामनं स्थाप्य गङ्गातीरे रघूत्तम}% १५१

\twolineshloka
{पृथिव्यां सर्वशः कृत्वा भागानष्टौ परन्तप}
{श्वेतद्वीपं स्वकं स्थानं व्रज देव नमोस्तु ते}% १५२

\twolineshloka
{प्रणिपत्य ततो रामस्तीर्थं प्राप्तश्च पुष्करम्}
{विमानं तु न यात्यूर्ध्वं वेष्टितं तत्तु राघवः}% १५३

\twolineshloka
{किमिदं वेष्टितं यानं निरालम्बेऽम्बरे स्थितम्}
{भवितव्यं कारणेन पश्येत्याह स्म वानरम्}% १५४

\twolineshloka
{सुग्रीवो रामवचनादवतीर्य धरातले}
{स च पश्यति ब्रह्माणं सुरसिद्धसमन्वितम्}% १५५

\twolineshloka
{ब्रह्मर्षिसङ्घसहितं चतुर्वेदसमन्वितम्}
{दृष्ट्वाऽऽगत्याब्रवीद्रामं सर्वलोकपितामहः}% १५६

\twolineshloka
{सहितो लोकपालैश्च वस्वादित्यमरुद्गणैः}
{तं देवं पुष्पकं नैव लङ्घयेद्धि पितामहम्}% १५७

\twolineshloka
{अवतीर्य ततो रामः पुष्पकाद्धेमभूषितात्}
{नत्वा विरिञ्चनं देवं गायत्र्या सह संस्थितम्}% १५८

\twolineshloka
{अष्टाङ्गप्रणिपातेन पञ्चाङ्गालिङ्गितावनिः}
{तुष्टाव प्रणतो भूत्वा देवदेवं विरिञ्चनम्}% १५९

\uvacha{राम उवाच}

\twolineshloka
{नमामि लोककर्तारं प्रजापतिसुरार्चितम्}
{देवनाथं लोकनाथं प्रजानाथं जगत्पतिम्}% १६०

\twolineshloka
{नमस्ते देवदेवेश सुरासुरनमस्कृत}
{भूतभव्यभवन्नाथ हरिपिङ्गललोचन}% १६१

\twolineshloka
{बालस्त्वं वृद्धरूपी च मृगचर्मासनाम्बरः}
{तारणश्चासि देवस्त्वं त्रैलोक्यप्रभुरीश्वरः}% १६२

\twolineshloka
{हिरण्यगर्भः पद्मगर्भः वेदगर्भः स्मृतिप्रदः}
{महासिद्धो महापद्मी महादण्डी च मेखली}% १६३

\twolineshloka
{कालश्च कालरूपी च नीलग्रीवो विदांवरः}
{वेदकर्तार्भको नित्यः पशूनां पतिरव्ययः}% १६४

\twolineshloka
{दर्भपाणिर्हंसकेतुः कर्ता हर्ता हरो हरिः}
{जटी मुण्डी शिखी दण्डी लगुडी च महायशाः}% १६५

\twolineshloka
{भूतेश्वरः सुराध्यक्षः सर्वात्मा सर्वभावनः}
{सर्वगः सर्वहारी च स्रष्टा च गुरुरव्ययः}% १६६

\twolineshloka
{कमण्डलुधरो देवः स्रुक्स्रुवादिधरस्तथा}
{हवनीयोऽर्चनीयश्च ॐकारो ज्येष्ठसामगः}% १६७

\twolineshloka
{मृत्युश्चैवामृतश्चैव पारियात्रश्च सुव्रतः}
{ब्रह्मचारी व्रतधरो गुहावासी सुपङ्कजः}% १६८

\twolineshloka
{अमरो दर्शनीयश्च बालसूर्यनिभस्तथा}
{दक्षिणे वामतश्चापि पत्नीभ्यामुपसेवितः}% १६९

\twolineshloka
{भिक्षुश्च भिक्षुरूपश्च त्रिजटी लब्धनिश्चयः}
{चित्तवृत्तिकरः कामो मधुर्मधुकरस्तथा}% १७०

\twolineshloka
{वानप्रस्थो वनगत आश्रमी पूजितस्तथा}
{जगद्धाता च कर्त्ता च पुरुषः शाश्वतो ध्रुवः}% १७१

\twolineshloka
{धर्माध्यक्षो विरूपाक्षस्त्रिधर्मो भूतभावनः}
{त्रिवेदो बहुरूपश्च सूर्यायुतसमप्रभः}% १७२

\twolineshloka
{मोहकोवन्धकश्चैवदानवानांविशेषतः}
{देवदेवश्च पद्माङ्कस्त्रिनेत्रोऽब्जजटस्तथा}% १७३

\twolineshloka
{हरिश्मश्रुर्धनुर्धारी भीमो धर्मपराक्रमः}
{एवं स्तुतस्तु रामेण ब्रह्मा ब्रह्मविदांवरः}% १७४

\twolineshloka
{उवाच प्रणतं रामं करे गृह्य पितामहः}
{विष्णुस्त्वं मानुषे देहेऽवतीर्णो वसुधातले}% १७५

\twolineshloka
{कृतं तद्भवता सर्वं देवकार्यं महाविभो}
{संस्थाप्य वामनं देवं जाह्नव्या दक्षिणे तटे}% १७६

\twolineshloka
{अयोध्यां स्वपुरीं गत्वा सुरलोकं व्रजस्व च}
{विसृष्टो ब्रह्मणा रामः प्रणिपत्य पितामहम्}% १७७

\twolineshloka
{आरूढः पुष्पकं यानं सम्प्राप्तो मधुरां पुरीम्}
{समीक्ष्य पुत्रसहितं शत्रुघ्नं शत्रुघातिनम्}% १७८

\twolineshloka
{तुतोष राघवः श्रीमान्भरतः स हरीश्वरः}
{शत्रुघ्नो भ्रातरौ प्राप्तौ शक्रोपेन्द्राविवागतौ}% १७९

\twolineshloka
{प्रणिपत्य ततो मूर्ध्ना पञ्चाङ्गालिङ्गितावनिः}
{उत्थाप्य चाङ्कमारोप्य रामो भ्रातरमञ्जसा}% १८०

\twolineshloka
{भरतश्च ततः पश्चात्सुग्रीवस्तदनन्तरम्}
{उपविष्टोऽथ रामाय सोऽर्घमादाय सत्वरम्}% १८१

\twolineshloka
{राज्यं निवेदयामास चाष्टाङ्गं राघवे तदा}
{श्रुत्वा प्राप्तं ततो रामं सर्वो वै माथुरो जनः}% १८२

\twolineshloka
{वर्णा ब्राह्मणभूयिष्ठा द्रष्टुमेनं समागताः}
{सम्भाष्य प्रकृतीः सर्वा नैगमान्ब्राह्मणैः सह}% १८३

\twolineshloka
{दिनानि पञ्चोषित्वाऽत्र रामो गन्तुं मनो दधे}
{शत्रुघ्नश्च ततो रामे वाजिनोथ गजांस्तथा}% १८४

\twolineshloka
{कृताकृतं च कनकं तत्रोपायनमाहरत्}
{रामस्त्वाह ततः प्रीतः सर्वमेतन्मया तव}% १८५

\twolineshloka
{दत्तं पुत्रौ तेऽभिषिञ्च राजानौ माथुरे जने}
{एवमुक्त्वा ततो रामः प्राप्तो मध्यन्दिने रवौ}% १८६

\twolineshloka
{महोदयं समासाद्य गङ्गातीरे स वामनम्}
{प्रतिष्ठाप्य द्विजानाह भाविनः पार्थिवांस्तथा}% १८७

\twolineshloka
{मया कृतोऽयं धर्मस्य सेतुर्भूतिविवर्धनः}
{प्राप्ते काले पालनीयो न च लोप्यः कथञ्चन}% १८८

\twolineshloka
{प्रसारितकरेणैवं प्रार्थनैषा मया कृता}
{नृपाः कृते मयार्थित्वे यत्क्षेमं क्रियतामिह}% १८९

\twolineshloka
{नित्यं दैनन्दिनीपूजा कार्या सर्वैरतन्द्रितैः}
{ग्रामान्दत्वा धनं तच्च लङ्काया आहृतं च यत्}% १९०

\twolineshloka
{प्रेषयित्वा च किष्किन्धां सुग्रीवं वानरेश्वरम्}
{अयोध्यामागतो रामः पुष्पकं तमथाब्रवीत्}% १९१

\twolineshloka
{नागन्तव्यं त्वया भूयस्तिष्ठ यत्र धनेश्वरः}
{कृतकृत्यस्ततो रामः कर्तव्यं नाप्यमन्यत}% १९२

\uvacha{पुलस्त्य उवाच}

\twolineshloka
{एवन्ते भीष्म रामस्य कथायोगेन पार्थिव}
{उत्पत्तिर्वामनस्योक्ता किं भूयः श्रोतुमिच्छसि}% १९३

\twolineshloka
{कथयामि तु तत्सर्वं यत्र कौतूहलं नृप}
{सर्वं ते कीर्त्तयिष्यामि येनार्थी नृपनन्दन}% १९४

{॥इति श्रीपाद्मपुराणे प्रथमे सृष्टिखण्डे वामनप्रतिष्ठानामाष्टत्रिंशोऽध्यायः॥३८॥}


    \input{katha/padma-puranam/rama-ashvamedha-prakaranam}
    \chapt{विष्णुधर्मोत्तर-पुराणम्}

\src{विष्णुधर्मोत्तर-पुराणम्}{प्रथमखण्डः}{अध्यायः २०२}{}
\tags{concise, complete}
\notes{Bharata is sent by Rāma with a vast army to drive out the Gandharvas from Yudhājit’s lands. He marches in splendour, crossing sacred rivers and reaching the rich city of Takṣaśilā. After a fierce battle, the Gandharvas are defeated, Yudhājit’s sons are crowned, and Bharata returns to Ayodhyā in triumph.}
\textlink{}
\translink{}

\storymeta

\sect{भरतप्रस्थानवर्णनम् --- द्व्यधिकद्विशततमोऽध्यायः}

\uvacha{वज्र उवाच}

\twolineshloka
{शैलूषपुत्रा गन्धर्वा भरतेन कथं हताः}
{किमर्थं तु महाभाग तन्ममाचक्ष्व पृच्छतः}%।।१।।

\uvacha{मार्कण्डेय उवाच}

\twolineshloka
{अयोध्यायामयोध्यायां रामे दशरथात्मजे}
{कैकेयाधिपतिः श्रीमान्युधाजिन्नाम पार्थिवः}%।। २।।

\twolineshloka
{रामाय प्रेषयामास दूतं भरत मातुलः}
{वृद्धं पुरोहितं गार्ग्यं येन कार्येण तच्छृणु}%।।३।।

\twolineshloka
{सिन्धोरुभयकूलेषु रामदेशो मनोहरः}
{हत्वा रणे मनुष्येन्द्रान्गन्धर्वैर्विनिवेशित}%।।४।।

\twolineshloka
{गन्धर्वास्ते च राजेन्द्र राज्ञां विप्रियकारकाः}
{लक्ष्मणं भरतं वापि शत्रुघ्नमथवा नृप}%।। ५ ।।

\twolineshloka
{विसर्जयित्वा गन्धर्वांस्तान्विनाशय राघव}
{राजानो निर्भयाः सन्तु देशश्चास्तु तथा तव}%।। ६ ।।

\twolineshloka
{दूतस्य वचनं श्रुत्वा चिन्तयामास राघवः}
{मेघनादवधे कर्म लक्ष्मणेन महत्कृतम्}%।। ७ ।।

\twolineshloka
{शत्रुघ्नेन कृतं कर्म लवणं च विनिघ्नता}
{प्रेषयिष्यामि भरतं गन्धर्वस्य च निग्रहे}%।।८।।

\twolineshloka
{इत्येवं मनसि ध्यात्वा रामो भरतमब्रवीत्}
{गच्छ गार्ग्यं पुरस्कृत्य वत्स राजगृहं स्वयम्}%।।९।।

\twolineshloka
{मातुलेन समायुक्तः कैकयेन्द्रगृहात्ततः}
{जहि शैलूषतनयान्गन्धर्वान्पापनिश्चयान्}% ।।१० ।।

\twolineshloka
{एवमुक्तः स धर्मात्मा भरतो भ्रातृवत्सलः}
{रामस्य पादौ शिरसा चाभिवन्द्य कृताञ्जलिः}%।। ११ ।।

\twolineshloka
{गृहं गत्वा चकाराथ सर्वं प्रास्थानिकं विधिम्}
{ओषधीनां कषायेण तदोत्प्लावितविग्रहः}%।। १२ ।।

\twolineshloka
{गौरसर्षपकल्केन प्रसादितशिरोरुहः}
{तीर्थसारसनादेयैः सलिलैश्च स सागरैः}%।। १३ ।।

\twolineshloka
{चन्दनस्रावसम्मिश्रैः कुङ्कुमाक्षोदसंयुतैः}
{सर्वौषधिसमायुक्तैः सर्वगन्धसमन्वितैः}%।। १४ ।।

\twolineshloka
{मन्त्रपूतैर्महातेजाः सस्नौ राघववर्धनः}
{शङ्खभेरी निनादेन पणवानां स्वनेन च}%।। १५ ।।

\twolineshloka
{आनकानां च शब्देन निस्वनेन च बन्दिनाम्}
{सूतमागधशब्देन जयकारैस्तथैव च}%।। १६ ।।

\twolineshloka
{तुष्टवुः स्नानकाले तं स्तवैर्मङ्गलपाठकाः}
{तथोपतस्थुर्गीतेन गन्धर्वाप्सरसां गणाः}%।। १७ ।।

\twolineshloka
{स्नातः स भरतो लक्ष्म्या युवराजाभिरूपया}
{विलिप्य चारुसर्वाङ्गं चन्दनेन सुगन्धिना}%।। १८ ।।

\twolineshloka
{अहताम्बरसंवीतः श्वेतमाल्यविभूषणः}
{कुण्डली साङ्गदी मौली सर्वरत्नविभूषितः}%।। १९ ।।

\twolineshloka
{अन्तस्थं पूजयामास देवदेवं त्रिवक्रमम्}
{गन्धमाल्यनमस्कारधूपदीपादिकर्मणा} %॥२०॥

\twolineshloka
{पूजयित्वा जगन्नाथमुपतस्थे हुताशनम्}
{सुहुतं ब्राह्मणेन्द्रेण राघवाणां पुरोधसा}%।। २१ ।।

\twolineshloka
{गोभिर्वस्त्रैर्हिरण्यैश्च तुरङ्गकरिपुङ्गवैः}
{मोदकैः सफलैर्दध्ना गन्धैर्माल्यैस्तथाक्षतैः}%।। २२ ।।

\twolineshloka
{स्वस्तिवाद्यांस्तथा विप्रान्महात्मा भूरिदक्षिणः}
{व्याघ्रचर्मोत्तरे रम्ये सूपविष्टो वरासने}%।। २३ ।।

\twolineshloka
{आयुधानन्तरं चक्रे ध्वजच्छत्राभिपूजनम्}
{स्वस्तिकान्वर्धमानांश्च नन्द्यावर्तास्तथैव च}%।। २४ ।।

\twolineshloka
{नद्यः काञ्चनविन्यस्ताः शङ्खः सत्कमलाञ्जनम्}
{पूर्णकुम्भं गजमदं दूर्वाः सार्द्रं च गोमयम्}%।। २५ ।।

\twolineshloka
{रत्नान्यादाय बिल्वं च चापमादाय सत्वरः}
{सशरं राघवश्रेष्ठः पदद्वात्रिंशकं ययौ}%।। २६ ।।

\twolineshloka
{श्रेष्ठमश्वं सुचन्द्राभं हेमभाण्डपरिच्छदम्}
{आरुह्य निर्ययौ श्रीमाञ्जयकाराभिपूजितः}%।। २७ ।।

\twolineshloka
{पौरजानपदामात्यैर्वाद्यघोषेण भूरिणा}
{ह्रादेन गजघण्टानां बृंहितेन पुनःपुनः}%।। २८।।

\twolineshloka
{हेषितेन तुरङ्गाणां नराणां क्ष्वेडितेन च}
{द्विजपुण्याहघोषेण प्रयातो भूरिदक्षिणः}%।। २९ ।।

\twolineshloka
{भरतस्य प्रयाणे तु देवाः शक्पुरोगमाः}
{मुमुचुः पुष्पवर्षाणि वाच ऊचुः शुभास्तथा} %॥३०॥

\twolineshloka
{एष विग्रहवान्धर्म एष सत्यवतां वरः}
{एष वीर्यवतां श्रेष्ठो रूपेणाप्रतिमो भुवि}%।। ३१ ।।

\twolineshloka
{अनेन यत्कृतं कर्म रामे वनमुपागते}
{न तस्य कर्ता लोकेऽस्मिन्दिवि वा विद्यते क्वचित्}%।। ३२ ।।

\twolineshloka
{अनेन राज्यं सन्त्यक्तं गृहं दग्धमिवाग्निना}
{अनेन दुःखशय्यासु शयितं समहात्मना}%।। ३३ ।।

\twolineshloka
{नित्यमासन्नभोगेन जटावल्कलधारिणा}
{फलमूलाशिनानेन रामराज्यं हि पालितम्}%।। ३४ ।।

\twolineshloka
{शृण्वन्सुवाक्यानि सुरेरितानि रामानुजो रामगृहं जगाम}
{शूरार्यविद्वत्पुरुषोपकीर्णं रत्नैर्यथा सागरमप्रमेयम्}%।। ३५ ।।

॥इति श्रीविष्णुधर्मोत्तरे प्रथमखण्डे मार्कण्डेयवज्रसंवादे भरतप्रस्थानवर्णनं नाम द्व्यधिकद्विशततमोऽध्यायः॥२०२॥

\sect{आनुयात्रिकवर्णनम् --- त्र्यधिकद्विशततमोऽध्यायः}

\uvacha{मार्कण्डेय उवाच}

\twolineshloka
{गृहात्प्रयाते भरते प्रस्थानार्थिनि यादव}
{प्रागेव लक्ष्मणं रामो नित्योद्युक्तमभाषत}%।। १ ।।

\twolineshloka
{अनुयानं कुमारस्य भरतस्य महात्मनः}
{प्रेषयाश्वसहस्राणां शतानि त्रीणि राघव}%।। २ ।।

\twolineshloka
{दश दन्तिसहस्राणि रथानां षड्गुणानि च}
{कोट्यः पञ्च पदातीनां समरेष्वनिवर्तिनाम्}%।। ३ ।।

\twolineshloka
{धनाध्यक्षास्तथा वत्समनुगच्छन्तु लक्ष्मण}
{गोरथैश्च तथा पुष्टैर्गोभिरुष्ट्रैस्तथैव च}%।। ४ ।।

\twolineshloka
{व्यायतैः पुरुषैरश्वैर्गर्दभैश्च तथा वरैः}
{वस्त्ररूप्यसुवर्णानां मणीनामपि भागशः}%।। ५ ।।

\twolineshloka
{विसर्गाय कुमारस्य परिपूर्णा यथासुखम्}
{ब्राह्मणाः कथया मुख्यास्तथैव नटवर्तकाः}%।। ६ ।।

\twolineshloka
{गीते नृत्ते तथा लास्ये प्रवीणाश्च वराङ्गनाः}
{प्रास्थानिकाश्च ये केचित्पानविक्रयिणश्च ये}%।। ७ ।।

\twolineshloka
{रूपाजीवाश्च वणिजो नानापण्योपजीविनः}
{नानारूपमुपादाय बहुपण्यं व्रजन्तु वै}%।।८।।

\twolineshloka
{विषवैद्याः शल्यवैद्यास्तथा कायचिकित्सकाः}
{कर्मन्तिका स्थपतयो मार्गिणो वृक्षरोपकाः}%।। ९ ।।

\twolineshloka
{कूपकाराः सुधाकारा वंशकर्मकतस्तथा}
{वाणिक्काराश्च ये केचित्कूर्चकाराश्च शोभनाः} %॥१०॥

\twolineshloka
{परिकर्मकृतश्चैव तथा वस्त्रोपजीविनः}
{मायूरिकास्तैत्तिरिकाश्चेतका भेदकाश्च ये}%।। ११ ।।

\twolineshloka
{रञ्जका दन्तकाराश्च तथा दन्तोपजीविनः}
{एरण्डवेत्रकाराश्च कटकाराश्च शोभनाः}%।। १२ ।।

\twolineshloka
{आरकूटकृतश्चैव तात्रकूटास्तथैव च}
{भूर्जकूटाः खड्गकारा गुडसीधुप्रपाचकाः}%।। १३ ।।

\twolineshloka
{औरभ्रका माहिषकास्तुन्नवायाश्च लक्ष्मण}
{ये चाभिष्टावकाः केचित्सूतमागधबन्दिनः}%।। १४ ।।

\twolineshloka
{चैलनिर्णेजकाश्चैव चर्मकारास्तथैव च}
{अङ्गारकोराश्च तथा लुब्धका ये च धीवराः}%।। १५ ।।

\twolineshloka
{कबन्धधारिणो ये च ये च काष्ठप्रपाटकाः}
{वस्त्रसीवनसक्ताश्च गृहकाराश्च ये नराः}%।। १६ ।।

\twolineshloka
{कुम्भकाराश्च ये केचिच्छ्मश्रुवर्धकिनश्च ये}
{लेखका गणका ये च तथा तन्दुलकारकाः}%।। १७ ।।

\twolineshloka
{सक्तुकाराश्च ये केचिच्छाकपण्योपजीविनः}
{तैलिका गान्धिकाश्चैव तीर्थसंशुद्धिकारकाः}%।।१६।।

\twolineshloka
{चित्रकर्मविदो ये च ये च लाङ्गूलिका जनाः}
{सूताः पौरोगवाश्चैव सौविदल्लाश्च लक्ष्मण}%।। १९ ।।

\twolineshloka
{गोपा वनचरा ये च नदीतीरविचारणाः}
{गोसङ्घैर्महिषीसङ्घैस्तेऽनुयान्तु यथासुखम्}% ।।२० ।।

\twolineshloka
{श्रेणीमहत्तरा ये च ग्रामघोषमहत्तराः}
{तथैवाटविका ये च ये च शैलविचारिणः}%।। २१ ।।

\twolineshloka
{सौमित्रे सर्व एवैते सुभृताश्च सुपूजिताः}
{अनुयान्तु कुमारं मे भरतं भ्रातृवत्सलम्}%।। २२ ।।

\twolineshloka
{इत्येवमुक्तः स तु तान्समस्तानाज्ञापयामास नरेन्द्रवाक्यात्}
{आमन्त्र्य रामं शिरसा च सर्वे विनिर्ययुस्ते नगरात्प्रहृष्टाः}%।।२३।।

॥इति श्रीविष्णुधर्मोत्तरे प्रथमखण्डे मार्कण्डेयवज्रसंवादे आनुयात्रिकवर्णनन्नाम त्र्यधिकद्विशततमोऽध्यायः॥२०३॥

\sect{भरतनिर्याणवर्णनम् --- चतुरधिकद्विशततमोऽध्यायः}

\uvacha{मार्कण्डेय उवाच}

\twolineshloka
{एतस्मिन्नेव काले तु रामः शुश्राव तन्महत्}
{शङ्खवाद्यरवोन्मिश्रं भरतस्यानुयात्रिकम्}%।। १ ।।

\twolineshloka
{राजद्वारमुपागत्य भरतोऽपि महायशाः}
{पदभ्यां जगाम राजानमवतीर्य तुरङ्गमात्}%।।२।।

\twolineshloka
{स ददर्श तदा रामं रत्नसिंहासनस्थितम्}
{अनुलिप्तं परार्ध्येन चन्दनेन सुगन्धिना}%।।३।।

\twolineshloka
{सूक्ष्मं वसानं वसनं सर्वाभरणभूषितम}
{तेजसा भास्कराकारं सौन्दर्येणोडुपोपमम्}%।। ४ ।।

\twolineshloka
{क्षमया पृथिवीतुल्यं क्रोधे कालानलोपमम्}
{बृहस्पतिसमं बुद्ध्या विष्णुतुल्यं पराक्रमे}%।।५।।

\twolineshloka
{सत्ये दानेऽप्यनौपम्यं दमे शीले च राघवम्}
{पुरोहितैरमात्यैश्च युतं प्रकृतिभिस्तथा}%।। ६ ।।

\twolineshloka
{दृष्ट्वा तं भरतः श्रीमाञ्जगाम शिरसा महीम्}
{भरतं युवराजानं रामाय विदितात्मने}%।। ७ ।।

\twolineshloka
{न्यवेदयत धर्मात्मा प्रतीहारः सलक्ष्मणम्}
{उत्थाय कण्ठे जग्राह रामोऽपि भरतं तदा}%।। ८ ।।

\twolineshloka
{मूर्ध्नि चैनमुपाघ्राय आदिदेशास्य शासनम्}
{भरतं तु सुखासीनं रामो वचनमब्रवीत्}%।। ९ ।।

\twolineshloka
{गन्धर्वपुत्रांस्तान्हत्वा कर्तव्यं नगद्वयम्}
{सिन्धो रुभयपार्श्वे तु पुत्रयोरुभयोः कृते} %॥१०॥

\twolineshloka
{अभिषिच्य तदा वत्स पुत्रौ नगरयोस्तयोः}
{युधाजिति परीधाय क्षिप्रमागन्तुमर्हसि}%।।११।।

\twolineshloka
{त्वया विना नरव्याघ्र नाहं जीवितुमुत्सहे}
{क्षत्रधर्मं पुरस्कृत्य तत्र त्वं प्रेषितो मया}%।।१२।।

\twolineshloka
{स त्वं गच्छ महाभाग मा ते कालात्ययो भवेत्}
{स्वस्त्यस्तु तेन्तरिक्षेभ्यः पार्थिवेभ्यश्च गच्छतः}%।। १३}

\onelineshloka
{दिव्येभ्यश्चैव भूतेभ्यः समरे च तथानघ}%।। १४ ।।

\twolineshloka
{स्वांस्वां दिशमधिष्ठाय दिक्पाला दीप्ततेजसः}
{पालयन्तु सदा तुभ्यं दीप्तविग्रहधारिणः}%।। १५ ।।

\twolineshloka
{ब्रह्मा विष्णुश्च रुद्रश्च साध्याश्च समरुद्गणाः}
{आदित्या वसवो रुद्रा अश्विनौ च भिषग्वरौ}%।। १६ ।।

\twolineshloka
{भृगवोङ्गिरसश्चैव कालस्यावयवास्तथा}
{सरितः सागराः शैलाः समुद्राश्च सरांसि च}%।। १७ ।।

\twolineshloka
{दैत्यदानवगन्धर्वा पिशाचोरगराक्षसाः}
{देवपत्न्यस्तथा सर्वा देवमातर एव च}%।। १८ ।।

\twolineshloka
{शस्त्राण्यस्त्राणि शास्त्राणि मङ्गलाय भवन्तु ते}
{विजयं दीर्घमायुश्च भोगांश्चान्यान्दिशन्तु ते}%।। १९ ।।

\twolineshloka
{इति स्वस्त्ययनं श्रुत्वा राज्ञा स समदीरितम्}
{रामस्य पादौ शिरसा त्वभिवन्द्य धनुर्धरः} %॥२०॥

\twolineshloka
{निर्गत्य राजभवनाद्रामाज्ञाकल्पितं जगत}
{हिमाद्रिकूटसङ्काशं चारुदंष्ट्रोज्ज्वलाननम्}%।। २१ ।।

\twolineshloka
{मदेन सिञ्चमानं च नृपवेश्माजिरं नृप}
{समाक्रान्तकटं चापि पानलुब्धशिलीमुखैः}%।। २२ ।।

\twolineshloka
{स्तब्धचारुमहाकर्णं मधुपकृपयैव तु}
{दीर्घाग्रमध्वक्षकृतं कृतशृङ्गावतंसकम्}%।। २३ ।।

\twolineshloka
{स्वासनं व्यूढकुम्भं च तथोदग्रं महाबलम्}
{नक्षत्रमालां शिरसा धारयानं तु काञ्चनीम्}%।। २४ ।।

\twolineshloka
{पटुस्वने तथा घण्टे दर्शनीये मनोहरे}
{कुथं विचित्रं रम्यं च कोविदारं महाध्वजं}%।। २५ ।।

\twolineshloka
{वैजयन्त्यः पताकाश्च किङ्किणीजालमालिताः}
{समारूढं नयविदा महामात्रेण धीमता}%।। २६ ।।

\twolineshloka
{वैडूर्यदण्डतीक्ष्णाग्र काञ्चनाङ्कुशधारिणा}
{जघनस्थेन चान्येन वरतोमरधारिणा}%।। २७ ।।

\twolineshloka
{तथा वैजयिकैर्मन्त्रैर्देवज्ञेनाभिमन्त्रितम्}
{आरुरोह महातेजा जयत्काराभिनन्दिनः}%।।२८।।

\twolineshloka
{पूर्णेन्दुमण्डलाकारं रुक्मदण्डं मनोहरम्}
{छत्रमादाय तं प्रेम्णा चारुरोह स लक्ष्मणः}%।। २९ ।।

\twolineshloka
{चामरौ द्वौ समादाय चन्द्ररश्मिमप्रभौ}
{आरूढं योषितोर्युग्मं रूपेणाऽप्रतिमं भुवि} %॥३०॥

\twolineshloka
{तं समारुह्य नागेन्द्रं मदलेखाभिगामिनम्}
{जगाम सह गार्ग्येण रथारूढेन यादव}%।। ३१ ।।

\twolineshloka
{तमन्वयौ महाभाग चतुरङ्गमहाबलम्}
{पताकाध्वजसम्बाधं कल्पयन्तं वसुन्धराम्}%।। ३२ ।।

\twolineshloka
{श्येनाः काकवहाः कङ्काः पिशाचा यक्षराक्षसाः}
{ययुः पुरःसरास्तस्य भरतस्य महात्मनः}%।। ३३ ।।

\twolineshloka
{गन्धर्वपुत्रमांसानां लुब्धा मांसोपजीविनः}
{तूर्यघोषेण महता बन्दिनां निःस्वनेन च}%।। ३४ ।।

\twolineshloka
{वायुना चानुलोमेन सेव्यमानः सुगन्धिना ।। ।}
{मङ्गलानां च मुख्यानां दर्शनाद्धृष्टमानसः}%।। ३९ ।।

\twolineshloka
{निर्ययौ राजमार्गेण जनसम्बाधशालिना}
{तेजस्विनां च तेजांसि हृदयानि च योषि ताम्}%।। ३६ ।।

\twolineshloka
{आददानो महातेजा नगरात्स विनिर्ययौ}
{क्रोशमात्रं ततो गत्वा समे देशे च सोदके}%।। ३७ ।।

\twolineshloka
{प्रशस्तद्रुमसङ्कीर्णे शिबिरं प्राङ्निषेवितम्}
{सेनाध्यक्षैः सुनिपुणैर्विवेश भरतस्तदा}%।। ३८ ।।

\twolineshloka
{स प्रविश्य महातेजाः शिबिरं स्वं निवेशनम्}
{मङ्गलालम्भनं कृत्वा वरासनगतः प्रभुः}%।। ३९ ।।

\twolineshloka
{पौरजानपदं सर्वं प्रेषयामा यादव}
{परिष्वज्य ततः पश्चाल्लक्ष्मणं शुभलक्षणम्} %॥४०॥

\twolineshloka
{मूर्ध्नि चैवमुपाघ्राय प्रेषयामास धर्मवित्}
{प्रायाणकं च श्वोभूते दुन्दुभिस्ताड्यतां मम}%।। ४१ ।।

\twolineshloka
{दैशिकाः पुरतो यान्तु ये च वृक्षावरोहकाः}
{आज्ञाप्य सकलं चैव शिबिरं च तथाविधम्}%।। ४२ ।।

\twolineshloka
{अवृक्षेषु तु देशेषु रोपयन्तु द्रुमाञ्जनाः}
{द्रुमाः कण्टकिनश्चैव ये च मार्गप्ररोधकाः}%।। ४३ ।।

\twolineshloka
{छिन्दन्तु गत्वा तानद्य तीक्ष्णैः शीघ्रं परश्वधैः}
{तोयहीनेषु देशेषु कूपान्कुर्वन्तु मे तथा}%।। ४४ ।।

\twolineshloka
{विषमांश्च तथा देशान्समान्कुर्वन्तु मे पथि}
{तीराणि सरितां चैव कुर्वन्तु पुलिनानि च}%।। ४५ ।।

\twolineshloka
{न भवेद्येन सङ्घट्टो जनस्य पथि यास्यतः}
{पुरः प्रयान्तु सैन्येन विजये रतिवर्धनाः}%।। ४६ ।।

\twolineshloka
{नीलश्च नक्रदेवश्च वसुमान्मुनयस्तथा}
{मध्यप्रयाणे गान्धारो जयनाभो रणोत्कटः}%।। ४७ ।।

\twolineshloka
{सुशीलः कामपालश्च यान्तु सैन्येन दंशिताः}
{जघनं कटकस्याहं पालयानो यथासुखम्}%।। ४८ ।।

\twolineshloka
{सैन्येन सह यास्यामि विजयाय नृपस्य तु}
{पश्यन्तु दैशिकाः स्थानं प्रभूतयवसेन्धनम् ।।}%।। ४९ ।।

\twolineshloka
{सोदकं च समं यत्र सेनावासो वेन्मम}
{एवमाज्ञाप्य भरतो विजहार यथासुखम्} %॥५०॥

\onelineshloka
{सुष्वाप च निशां तत्र घर्मकाले मनोहराम्}%।।५१।।

\twolineshloka
{सुप्तस्य सा तत्र रघूद्वहस्य पुण्या ययौ रात्रिरदीनसत्त्वा}
{सम्पूर्णचन्द्राभरणा प्रतीता ज्येष्ठस्य मासस्य रणोत्सुकस्य}%।। ५२।।

॥इति श्रीविष्णुधर्मोत्तरे प्रथमखण्डे मार्कण्डेयवज्रसंवादे भरतनिर्याणवर्णनं नाम चतुरधिकद्विशततमोऽध्यायः॥२०४॥

\sect{प्रयाणवर्णनम् --- पञ्चाधिकद्विशततमोऽध्यायः}

\uvacha{मार्कण्डेय उवाच}

\twolineshloka
{प्रभातायां तु शर्वर्यां दुन्दुभिः समहन्यत}
{प्रयाणिको महाराज भरतस्य महास्वनः}%।। १ ।।

\twolineshloka
{तस्य शब्देन महता विबुद्धः कटके जनः}
{अवश्यकरणीयानि कृत्वा राजंस्त्वरान्विताः}%।। २ ।।

\twolineshloka
{त्वरिता गमनार्थाय समाहूयेतरेतरम्}
{पटवेश्मानि रम्याणि सहन्तुमुपचक्रमुः}%।। ३ ।।

\twolineshloka
{महान्ति सुमनोज्ञानि वर्तितोर्णाकृतानि च}
{चक्रुस्तानि च राजेन्द्र सुखवाह्यान्ययत्नतः}%।। ४ ।।

\twolineshloka
{ततस्त्वारोपयाञ्चक्रुः करभेषु खरेषु च}
{गोरथेषु तु मुख्येषु तथा दन्तिषु सत्वराः}%।। ५ ।।

\twolineshloka
{भाण्डमुच्चावचं चैव शयनानि मृदूनि च}
{आसनानि च मुख्यानि भाण्डं यच्च महानसे}%।। ६ ।।

\twolineshloka
{पेयं च यवसं चैव शस्त्राणि विविधानि च}
{कवचानि तुरङ्गाणां शिल्पभाण्डानि यानि च}%।। ७ ।।

\twolineshloka
{धनं च विविधं राजन्सर्वोप करणानि च}
{आरोप्यमाणे भाण्डे तु करभाणां विकृष्यताम्}%।। ८ ।।

\twolineshloka
{शुश्रुवे तुमुलः शब्दः खराणां च खरस्वनः}
{गजानां युज्यमानानां तुरङ्गाणां रथैः सह}%।। ९ ।।

\twolineshloka
{वाद्यानां हन्यमानानां शुश्रुवे तुमुलंस्वनम्}
{नादेन गजघण्टानां बृंहितेन च पार्थिव} %॥१०॥

\twolineshloka
{ह्रेषितेन तुरङ्गाणां बभूव तुमुलः स्वनः}
{प्रायाणिकमुपादाय ताडयन्नेव दुन्दुभिम्}%।। ११ ।।

\twolineshloka
{अग्रे प्रयाणमान्येन ययौ दुन्दुभिभिः सह}
{पण्यानि च समादाय वणिजस्त्वरिता ययुः}%।। १२ ।।

\twolineshloka
{ग्रहीतुकामाश्चान्यानि सोदकानि समानि च}
{स्थानानि वरमुख्यानां ययुश्चाग्रेसरा नराः}%।। १३ ।।

\twolineshloka
{महानसिकमुख्यास्तु त्वरिताश्च तथा ययुः}
{सुखयानासु रम्यासु तथैवाश्वतरीषु च}%।। १४ ।।

\twolineshloka
{आरोप्य योषितो जग्मुः प्रत्यूषे मुदिता जनाः}
{आरूढाश्चापरा नार्यः सवितानाः करेणवः}%।।१५।।

\twolineshloka
{कञ्चुकोष्णीषिभिर्गुप्ता गुप्ता वर्षवरैस्तथा}
{ययुः ससैन्या राजेन्द्र गीतवाद्य पुरस्सराः}%।। १६ ।।

\twolineshloka
{दीनान्धकृपणानाथांस्तर्पयन्त्यो धनेन ताः}
{नरेन्द्रयोषितो राज्ञां दिव्यालङ्गारभूषिताः}%।। १७ ।।

\twolineshloka
{तथान्ये बद्धनिस्त्रिंशाः पुरुषाश्च कलापिनः}
{आदीप्य तृणवेश्मानि जग्मुस्त्वरितमानसाः}%।। १८ ।।

\twolineshloka
{भरतोऽपि समारुह्य शिबिकां रत्नभूषिताम्}
{विनिर्ययौ महातेजा स्तूर्यघोषपुरःसरः}%।। १९ ।।

\twolineshloka
{शून्यं च शिबिरस्थानं गृध्रमण्डसङ्कुलम्}
{बहुक्रव्यादसङ्कीर्णं क्षणेन समपद्यत} %॥२०॥

\twolineshloka
{गजोष्ट्रगर्दभाश्वानां शरीरावयवैर्युतम्}
{भग्नभाण्डसमाकीर्णं शरीरावयवैर्युतम्}%।।२१।।

\twolineshloka
{बहुक्रव्यादसङ्कीर्णं करीषोत्करसंयुतम्}
{खातैर्महानसस्थानैर्दग्धमृत्तिकया युतैः ।। ।}%।।२२।।

\twolineshloka
{समण्डकर्दमोपेतैर्मक्षिकासहितैर्युतम्}
{सन्त्यज्य निर्ययुः सर्वे भरतस्य तु सैनिकाः}%।।२३।।

\twolineshloka
{प्रयाणे तस्य सैन्यस्य बलीवर्दाञ्छ्रमान्वितान्}
{नागानुत्थापयामासुरुपविष्टान्प्रयत्नतः}%।।२४।।

\twolineshloka
{केचिदुष्ट्रपरित्रस्तातान्गार्दभेन निपातितान्}
{भाण्डमारोपयाञ्चक्रुर्भूय एव नरोत्तमाः}%।। २५ ।।

\twolineshloka
{नद्युत्तारेषु महिषान्केचित्सूर्यांशुतापितान्}
{निषण्णान्सह भारेण ताडयाञ्चक्रिरे जनाः}%।। २६ ।। 

\twolineshloka
{केचिदश्वतरांस्त्रस्तान्नागबृंहितनिस्वनैः}
{आरूढयोषितो यत्नाज्जगृहुर्नृप रश्मिषु}%।। २७ ।।

\twolineshloka
{केचिदश्वैर्गजत्रस्तैराक्षिप्ता भुवि मानवाः}
{जानुविश्रमणार्थाय वाजिग्रीवकृताङघ्रयः}%।। २८ ।।

\twolineshloka
{तुरङ्गांश्च तदोद्भ्रान्तान्स्रस्तचर्माश्च सादिनः}
{केचिदाक्रम्य वेगेन जगृहुस्तत्र यादव}%।। २९ ।।

\twolineshloka
{विशश्रमुस्तथा केचिद्वृक्षच्छायास मानवाः}
{केचिच्चोदकतीरेषु चक्रिरे भोजनक्रियाम्} %॥३०॥

\twolineshloka
{केचित्सन्त्रस्ततुरगसन्निकृष्टसमुत्थितैः}
{द्रुतान्कापिञ्जलैर्यत्नाज्जगृहुस्तांस्तुरङ्गमान्}%।। ३१ ।।
 
\twolineshloka
{केचित्कटकसन्त्रस्तान्मृगयूथान्प्रधावतः}
{वेगेनाक्रम्य विशिखैर्जघ्नुर्यदुकुलोद्वह}%।। ३२ ।।

\twolineshloka
{केचिच्च यवसं चक्रुः केचिच्चक्रुरथेन्धनम्}
{तथान्यैर्द्विगुणीभूतं तथा दुन्दुभिनिस्वनम्}%।। ३३ ।।

\twolineshloka
{प्रायाणिकं जहुः शीघ्रं श्रमं यदुकुलोद्वह}
{केचिदापणवीथ्यग्रमहावंशसमुच्छ्रितम्}%।।३४ ।।

\twolineshloka
{सपताकं नरा दृष्ट्वा प्राप्ताः स्म इति मेनिरे}
{चक्रुः केचिच्च च्छन्दांसि सहायानां पुनःपुनः}%।।३५ ।।

\twolineshloka
{पुरोगतानां राजेन्द्र स्थानलब्धिचिकीर्षया}
{केचित्पटकुटीं दृष्ट्वा स्वकीयां त्वरिता ययुः}%।। ३६ ।।

\twolineshloka
{वर्तितोर्णाकृतं दृष्ट्वा गृहांश्चान्ये ययुर्द्रुतम्}
{केषाञ्चित्तत्र वेश्मानि तीर्णानि नृपसत्तम}%।। ३७ ।।

\twolineshloka
{कृतानि क्रियमाणानि ददृशुस्तत्र मानवाः}
{द्रुमैर्विश्राम्यमाणैश्च क्रियमाणैस्तथा कटैः}%।। ३८ ।।

\twolineshloka
{गृहैरारोप्यमाणैश्च पटोर्णातृणसंस्कृतैः}
{शुशुभे तन्महाराज कटकं शुभकर्मणः}%।। ३९ ।।

\twolineshloka
{शस्त्रेण संशोधयतां भुवं भूमिपत नृणाम्}
{राजसाक्रान्तवपुषामप्रकाशं वपुर्बभौ} %॥४०॥

\twolineshloka
{पार्श्वस्थतोयसम्पूर्णदृतयश्च तथा जनाः}
{धावमानाः प्रदृश्यन्ते वर्धिता गृहशोधने}%।। ४१ ।।

\twolineshloka
{अभ्युक्षयन्ति चाप्यन्ये तृणवेश्मानि पार्थिव}
{तप्तानां शीतकामानां दृतिवक्त्रोद्गतैर्जलैः}%।। ४२ ।।

\twolineshloka
{अभिश्रयेण धूपेन समन्तादाकुलीकृतम्}
{बभूव तस्य कटकं नीहारेणैव संयुतम्}%।। ४३ ।।

\twolineshloka
{अवरोपितभाण्डानां दान्तानां यदुनन्दन}
{पृष्ठान्यभ्युक्षयामासुर्गोमयेन जलेन च}%।। ४४}

\onelineshloka
{खरोष्ट्रसबलीवर्द चरणार्थे विनिर्गतम्}%।। ४५ ।।

\twolineshloka
{ददृशे बहुसाहस्रं कटके भरतस्य तु}
{रथेभ्यस्तुरगानन्ये विमुच्य हयकोविदाः}%।। ४६ ।।

\twolineshloka
{अवरोपितभाण्डानि सान्त्वयाञ्चक्रिरे शनैः}
{समुत्थाय रजो भौमं तुरङ्गपरिवर्तनैः}%।। ४७ ।।

\twolineshloka
{खमारुरोह राजेन्द्र कपोतारुणपाण्डुरम्}
{स्थापयाञ्चक्रिरे चान्ये जलस्थाने च वाजिनः}%।। ४८ ।।

\twolineshloka
{श्रेणी चक्रुस्तथैवान्ये कटच्छायासु पार्थिव}
{वितानाधस्तथा केचित्स्थापयाञ्चक्रिरे हयान्}%।। ४९ ।।

\twolineshloka
{हयेभ्यो यवसं दत्त्वा केचिद्बुभुजरे जनाः}
{कटच्छायाश्च नागानां चक्रुश्चान्ये सहस्रशः} %॥५०॥

\twolineshloka
{वितानानि च मुख्यानि सूर्यतापप्रशान्तये}
{स्नाताञ्जलाशये नागाँल्लब्धतोयाञ्जनाधिप}%।। ५१।। 

\twolineshloka
{सच्छन्नान्स्वपरीधानकुम्भान्निन्युः स्वमालयम्}
{कटकाद्दूरतश्चक्रुरालानं नृपदन्तिनाम्}%।। ५२।।

\twolineshloka
{आलानानि महाराज महावृक्षेषु मानवाः}
{आदाय गोपिनस्तत्र कटकाच्च विदूरतः}%।। ५३ ।।

\twolineshloka
{गोसङ्घान्महिषीसङ्घांश्चक्रुर्व्यस्तान्यथासुखम्}
{कटके च तथा निन्युर्गोरसानि नराधिप}%।। ५४ ।।

\twolineshloka
{कटकापण्यवीथीं च सर्वपण्यविभूषिताम्}
{ददृशुः पुरुषास्तत्र अयोध्यामिव चापराम्}%।। ५५ ।।

\twolineshloka
{स्थानानि सर्ववैद्यानांसध्वजानि नराधिप}
{सागदानि च दृश्यन्ते कटके भरतस्य तु}%।।५६।।

\twolineshloka
{सेनाध्यक्षेण वीरेण विजयेन महात्मना}
{कृतं शास्त्रानुसारेण स्कन्धावारनिवेशनम्}%।। ५७ ।।

\twolineshloka
{विवेश भरतः श्रीमांश्चतुरङ्गबलान्वितः}
{अन्वीयमानो वीराभ्यां पुत्राभ्यां यदुनन्दन}%।। ५८ ।।

\twolineshloka
{पुष्करेण च वीरेण तक्षेण सुमहात्मना}
{बलमुख्यैस्तथैवान्यैः सूतमागधबन्दिभिः}%।। ५९ ।।

\twolineshloka
{शङ्खवादित्रशब्देन पटहानां स्नेन च}
{भरतस्य गृहद्वारतोरणान्तिकमागताः} %॥६०॥

\twolineshloka
{बभूबुर्बलमुख्यास्ते दिक्षु ये यदुनन्दन}
{सन्त्यज्य मध्यमां वीथीं प्रवेशाय महामनाः}%।। ६१ ।।

\twolineshloka
{बन्दिभिः ख्याप्यमानांस्तान्नामकर्मावदानतः}
{शिरःकम्पेन भरतः प्रैरयत्स्वान्निवेशनान्}%।। ६२ ।।

\twolineshloka
{बलमुख्यान्विवेशाथ स्वगृहं सर्वऋद्धिमत्}
{द्वियोजनाध्वना श्रान्ता भरतस्य तु सैनिकाः}%।। ६३ ।। 

\twolineshloka
{विविशुर्भवनान्स्वान्स्वान्भेजिरे शयनानि च}
{प्रविश्य वेश्मप्रवरं भरतोऽपि यथासुखम्}%।। ६४ ।।


\threelineshloka
{विजहार महाराज देवराजसमद्युतिः}
{क्रमेणानेन धर्मात्मा भूमिपाल दिनेदिने}
{आससाद व्रजन्नेव गङ्गां त्रिपथगां नदीम्}%।। ६५}

\twolineshloka
{सुराङ्गनापीनपयोधरस्थसच्चन्दनक्षालनलब्धलक्ष्मीम्} 
{ग्रीष्मार्कतापाद्विगलत्तुषारविवृद्धशीतोदपटोत्तरीयाम्}%।। ६६ ।।

॥इति श्रीविष्णुधर्मोत्तरे प्रथमखण्डे मार्कण्डेयवज्रसंवादे प्रयाणवर्णनं नाम पञ्चाधिकद्विशततमोऽध्यायः॥२०५॥

\sect{गङ्गावतरण वर्णनम् --- षडुत्तरद्विशततमोऽध्यायः}

\uvacha{मार्कण्डेय उवाच}

\twolineshloka
{निवेशमकरोद्राजन् गङ्गातीरे स राघवः}
{भरतस्य तदा चक्रुर्गङ्गायां जलमण्डपान्}%।। १ ।।

\twolineshloka
{तथैव बलमुख्यानां प्राधान्येन नराधिप}
{सैनिकश्च जनस्तस्य प्राप्य तां सुरनिम्नगाम्}%।। २ ।।

\twolineshloka
{साफल्यं जन्मनो मेने तत्याज च तथा क्लमम्}
{सस्नौ पपौ पयः कामं हर्षेण महता युतः}%।। ३ ।।।

\twolineshloka
{स्नापयाञ्चक्रिरे तत्र तुरगांस्तुरगप्रियाः}
{कुञ्जरान्स्नापयमासुर्महामात्रास्तथैव च}%।। ४ ।।

\twolineshloka
{मज्जद्भिर्बहुसाहस्रैः कुञ्जरैर्जाह्नवी नदी}
{किरातविषयैः सैव विरराज यथोपलैः}%।। ५ ।।

\twolineshloka
{समुत्थितमहामात्रान्दृष्ट्वाभ्युक्षणछद्मना}
{चिक्रीडुर्दृतिभिः केचिद्विविशुर्बाहुभिस्तथा}%।। ६ ।।

\twolineshloka
{गङ्गामासाद्य संहृष्टा भरतस्य तु सैनिकाः}
{भरतोऽपि तथा स्नातः कृतदैवतपूजनः}%।।७।।

\twolineshloka
{श्राद्धं चक्रे महातेजा ददौ दानं तथैव च}
{दत्त्वा तत्र महादानानाज्ञापयति यादव}%।।८।।

\twolineshloka
{शिल्पिनो मम कुर्वन्तु कूटागारान्मनोरमान्}
{रात्रौ दीपनिवेशार्थं शतशोऽथ सहस्रशः}%।।९।।

\twolineshloka
{आनीयन्तां तथा नावः सैन्यस्य तणाय मे}
{क्रियन्तां चर्मनावश्च प्लवाश्च शतशस्तथा} %॥१०॥

\twolineshloka
{उत्तरं पारमासाद्य निवेशः क्रियतां तथा}
{परं पारं जना यान्तु उदयास्तमयान्तरे}%।।११।।

\twolineshloka
{एवमाज्ञाय भरतो विजहार यथासुखम्}
{प्रसारितकरो राजन्सर्वत्रैव दिवाकरः}%।। १२ ।।

\twolineshloka
{ययावदर्शनं तत्र भरतस्यैव लज्जया}
{पद्मपत्रदलाग्राभा प्रतीची चाभवत्क्षणात्}%।।१३।।

\twolineshloka
{आदित्येस्तमनुप्राप्ते सन्ध्यारागानुरञ्जिता}
{ततस्तु तमसा व्याप्ते न प्राज्ञायत किञ्चन}%।। १४ ।।

\twolineshloka
{भरताज्ञाकृतान्पूर्वं कूटागारान्सदीपकान्}
{ध्वजमालापरिक्षिप्तांश्चिक्षिपुर्जाह्नवीजले}%।।१५।।

\twolineshloka
{सैनिकैश्च तथा मुख्यैर्भरतस्य पृथक्पृथक् ।}
{गङ्गाम्भसि परिक्षिप्ता दीपवृक्षाः सहस्रशः}%।।१६।।

\twolineshloka
{सा दीपमालिनी गङ्गा तीरद्योतितविग्रही}
{जहास फेनहासेन जिह्वेव गगनप्रिया}%।। १७ ।।

\twolineshloka
{जलान्तरागतै र्दीपैर्दीपवृक्षैर्मनोहरैः}
{तथा कल्लोलसङ्क्रान्तैर्गङ्गा दीप्तेव लक्ष्यते}%।। १८ ।।

\twolineshloka
{एवं हि क्रीडतां तत्र गङ्गातीरे तथा नृणाम्}
{रात्रावेवाञ्जसा जग्मुर्नावो बहुविधा नृप}%।। १९ ।।

\twolineshloka
{शशाङ्कराजहंसेन दृष्ट्वा खससीं जनाः}
{आक्रम्यमाणा सुषुपुर्निशीथे निद्रयान्विताः} %॥२०॥

\twolineshloka
{शशाङ्कोदयसंसुप्तबलं पद्मवनोपमम्}
{निबोधयामास तथा दिवाकरकरोत्करम्}%।। २१ ।।

\twolineshloka
{चकृषुस्ते तदा नावः कर्णधारा यथा स्वकाः}
{बलं च सकलं पारं तारया ञ्चक्रिरे तदा}%।। २२ ।।

\twolineshloka
{नावस्तां किङ्किणीजालैः पताकाभिश्च राजिता}
{निन्युर्बलं परं पारं कर्णधारस्फिगाहताः}%।। २३ ।।

\twolineshloka
{भाण्डैः पूर्णास्तथा काश्चित्काश्चित्पूर्णा जनेन च}
{तुरङ्गमैस्तथा पूर्णा गोखरोष्ट्रैस्तथा पराः}%।। २४ ।।

\twolineshloka
{काश्चित्सकुञ्जरा नावो ययुः पारं तदा परम्}
{तथा पद्मदलाक्षीणां स्त्रीणां नावश्च पूरिताः}%।। २५ ।।

\twolineshloka
{विमानाभाः प्रदृश्यन्ते गङ्गाम्भसि नरेश्वर}
{कर्णधारवरोपेता दण्डिभिः पुरुषैर्वृताः}%।। २६ ।।

\twolineshloka
{लोकसन्तारणार्थाय भूयो जग्मुस्तथा पराः}
{चर्मनौभिस्तथा केचित्प्लवैः केचित्सुयन्त्रितैः}%।। २७ ।।

\twolineshloka
{जग्मुरादाय भाण्डानि दृतिभिश्च तथा क्वचित्}
{नरैर्दृतिसमारूढैः कृष्यमाणास्तुरङ्गमाः}%।। २८ ।।

\twolineshloka
{जग्मुः केचित्परं पारं बाहुभिर्मनुजेश्वर}
{जग्मुरन्ये परं पारं सङ्गृहीताश्च रश्मिभिः ।।}%।। २९ ।।

\twolineshloka
{नौस्थैरेव परैः शीघ्रं तुरगा नपबाहुभिः}
{महिषीणां च सङ्घानि गवां च यदुनन्दन} %॥३०॥

\twolineshloka
{जग्मुर्गोपालकैः सार्धं परं पारं च बाहुभिः}
{उष्ट्रगर्दभसङ्घानि तार्यमाणानि बाहुभिः}%।। ३१ ।।

\twolineshloka
{भूयोभूयो न्यवर्तन्त तत्र रावो महानभूत्}
{तीर्णस्य तार्यमाणस्य नरस्य यदुसत्तम}%।। ३२ ।।

\twolineshloka
{आसीत्कोलाहलो घोरस्तीरयोरुभयोरपि}
{भरतोऽपि महातेजाः स्नातो हुतहुताशनः}%।। ३३ ।।

\twolineshloka
{मत्स्यरूपधरं विष्णुं पटे सम्पूज्य यादव}
{स्वस्तिवाच्यांस्ततो विप्रान्गोभिर्वस्त्रैर्धनेन च}%।। ३४ ।।

\twolineshloka
{सम्पूज्य जाह्नवीं देवीं गन्धमाल्यानुलेपनैः}
{सम्पूजयामास तदा भूय एव द्विजोत्तमान्}%।। ३५ ।।

\twolineshloka
{गोभिरश्वैस्तथा निष्कैर्वस्त्रैर्गन्धैस्तथैव च}
{आरुरोह तदा नावं पाण्डुकम्बलसंवृताम्}%।।३६।।

\twolineshloka
{किङ्किणीजालविततां पताकाध्वजमालिनीम्}
{भरते तु समारूढे कर्णधारस्फिगाहता}%।। ३७ ।।

जगाम सा परं पारं नौर्विमानोपमा नृप ।।

\twolineshloka
{सम्प्राप्य स परं पारं सम्पूर्णकटकस्तदा}
{उवास राजंस्तत्रैव दिवि देवेश्वरो यथा}%।। ३८ ।।

\twolineshloka
{देवीं महादेवजटातटस्थां शशाङ्कसंसर्गविवृद्ध शीताम्}
{उत्तीर्य राजा भरतः स चैनां निन्ये क्षयं शक्रसमप्रभावः}%।। ३९ ।।

॥इति श्रीविष्णुधर्मोत्तरे प्रथमखण्डे मार्कण्डेयवज्रसंवादे गङ्गावतरणवर्णनं नाम षडुत्तरद्विशततमोऽध्यायः॥२०६॥

\sect{राजगृहगमनो --- सप्तोत्तरद्विशततमोऽध्यायः}

\uvacha{मार्कण्डेय उवाच}

\twolineshloka
{भरते तु समुत्तीर्णे गङ्गां गगनमेखलाम्}
{विवेश कटकं तस्य तमन्यः कौरवो नृप}%।। १ ।।

\twolineshloka
{बलेन चतुरङ्गेण समुद्राभेन संयुतः}
{ईश्वरश्च किरातानां दमनो नाम पार्थिवः}%।। २ ।।

\twolineshloka
{गन्धिनां बहुसाहस्रैर्बलैर्युक्तो व्यदृश्यत}
{यथार्हं पूजयित्वा तं भरतो धर्मवत्सलः}%।। ३।।

\twolineshloka
{आससादार्कतनयां यमुनां पापनाशिनीम्}
{यमस्य भगिनीं पुण्यां नीलमालां मनोहराम्}%।। ४ ।।

\twolineshloka
{यत्र क्वचन नद्यां हि कृत्वा श्राद्धन्नराधिप}
{अक्षयं फलमाप्नोति नाकपृष्ठे च मोदते}%।।५।।

\twolineshloka
{यत्र कृष्णचतुर्दश्यां स्नातः सम्पूज्य भानुजाम्}
{मुच्यते पातकैः सर्वैर्नाकलोकं स गच्छति}%।। ६ ।।

\twolineshloka
{अनर्काभ्युदिते काले माघकृष्णचतुर्दशीम्}
{यस्यां स्नातस्तु सम्पूज्य धर्मराजं तिलाम्भसा}%।। ७ ।।

\twolineshloka
{न दुर्गतिमवाप्नोति कुलं चैव समुद्धरेत्}
{यत्र क्वचन नद्यां हि माघकृष्णचतुर्दशीम्}%।। ८ ।।

\twolineshloka
{नादेयाम्भसि सर्वस्मिन्स्नातः पापैर्विमुच्यते}
{यमुना तु विशेषेण यमस्य भगिनीति सा}%।। ९ ।।

\twolineshloka
{स्नातश्च यामुने तोये सन्तर्प्य पतृदेवताः}
{न दुर्गतिमवाप्नोति नाकलोकं च गच्छति} %॥१०॥

\twolineshloka
{तां समासाद्य यमुनां यथा गङ्गाजलं तथा}
{विहृत्य भरतस्तद्वदुत्ततार महानदीम्}%।। ११ ।।

\twolineshloka
{समुत्तीर्णस्य यमुनां भरतस्य महात्मनः}
{विवेश कटकं राजा मत्स्यानां सुरथस्तदा}%।। १२ ।।

\twolineshloka
{गोरसेनश्च साल्वानां शिबीनां च प्रभद्रकः}
{स तैर्नृपतिभिः सार्धं कुरुक्षेत्रमुपाययौ}%।। १३ ।।

\twolineshloka
{यदर्थमेषा चरति लोके गाथा पुरातनी}
{पांसवोऽपि कुरुक्षेत्रे वायुना सममीरिताः}%।। १४ ।।

\twolineshloka
{अपि दुष्कृतकर्माणो नयन्ति परमां गतिम्}
{समन्तपञ्चके पुण्ये ये मृता मनुजेश्वर}%।। १५ ।।

\twolineshloka
{ते सर्वे नाकमासाद्य राजन्ते दिवि देववत्}
{सन्नीतिर्यत्र राजेन्द्र तीर्थं त्रैलोक्यविश्रुतम्}%।। १६ ।।

\twolineshloka
{तीर्थसन्नयनादेव सन्नीतिरिति विश्रुतम्}
{पृथिव्यां यानि तीर्थानि आसमुद्रसरांसि च}%।। १७ ।।

\twolineshloka
{मासान्ते सततं तत्र नित्यमायान्ति यादव}
{तत्र श्राद्धं तु यः कुर्याद्राहुग्रस्ते दिवाकरे}%।। १८ ।।

\twolineshloka
{अश्वमेधशतस्याग्रं फलं विन्दति मानवः}
{भरतस्तु समासाद्य तत्रोवास सुखी तदा}%।।१९।।

\twolineshloka
{पप्रच्छ ब्राह्मणांस्तत्र तीर्थसन्नीतकारणम्}
{पृष्टस्तु भरतेनाथ ब्राह्मणस्तु घटोदरः} %॥२०॥

\onelineshloka*
{उवाच भरतं तत्र कथां पापप्रणाशिनीम्}

\uvacha{घटोदर उवाच}
\onelineshloka
{शक्रे वृत्रवधाक्रान्ते त्रैलोक्ये दैत्यसाद्गते}%।। २१ ।।

\twolineshloka
{सब्रह्मकाः सुराः सर्वे विष्णुं शरणमाययुः}
{तानुवाच हरिर्देवश्च्यवनस्यात्मसम्भवः}%।। २२ ।।

\twolineshloka
{भार्गवो ब्राह्मणः श्रीमान्दधीच इति विश्रुतः}
{अस्थिभिः क्रियतां तस्य देवेन्द्रस्य वरायुधम्}%।। २३ ।।

\twolineshloka
{प्रविश्य देवांस्तत्राहं हन्ता वृत्रमसंशयम्}
{एवमुक्ताः सुराः सर्वे दधीचस्याश्रमं ययुः}%।। २४ ।।

\twolineshloka
{ददृशुश्च महाभागं दधीचं तपसां निधिम्}
{पूजयित्वा महाभागं तमूचुः संहताः सुरा.}%।। २५ ।।

\twolineshloka
{त्वदस्थिभिः करिष्यामो वज्रं दैत्यनिबर्हणम्}
{अन्यानि च तथास्त्राणि सुरकार्यार्थसिद्धये}%।। २६ ।।

\twolineshloka
{तत्र त्वं देवकार्यार्थं सन्न्यासं द्विज रोचय}
{एवमुक्तो दधीचस्तु प्रत्युवाच स तान्सुरान्}%।। २७ ।।

\twolineshloka
{तीर्थयात्रा प्रतिज्ञाता सर्वतीर्थेषु वै मया}
{तां तु कृत्वा करिष्यामि देहन्यासं सुरोत्तमाः}%।। २८ ।।

\uvacha{देवा ऊचुः}

\twolineshloka
{शक्तस्त्वं सर्वतीर्थानामाह्वाने द्विजपुङ्गव}
{इहाद्यैव समायान्तु सर्वतीर्थानि तेऽनघ}%।। २९ ।।

\onelineshloka*
{तेजसा च त्वदीयेन तथास्माकं च भार्गव}

\uvacha{घटोदर उवाच}
\onelineshloka
{एवमुक्तः सुरैः सर्वैस्तीर्थानि सरितस्तथा}%।।३० ।।

\twolineshloka
{सरांसि च समुद्राश्च तत्राजग्मुर्नराधिप}
{काम्येन महता राजन्योगेन परमेण च}%।। ३१ ।।

\twolineshloka
{ज्ञात्वा च तेषां सान्निध्यं तत्र स्नातो द्विजोत्तमः}
{तर्पणं च तथा कृत्वा सुराणां पितृभिः सह}%।। ३२ ।।

\onelineshloka
{उवाच देवांस्तत्रस्थानिदं वचनमर्थवत्}

\uvacha{दधीच उवाच}
\onelineshloka*
{अद्य प्रभृति मासान्ते भवद्भिः सततं सुराः}%।। ३३ ।।

\onelineshloka*
{सान्निध्यमिह कर्तव्यं तीर्थैश्चैव यथागतैः}

\uvacha{घटोदर उवाच}
\onelineshloka
{एवमुक्तैस्तथेत्युक्तो देवैस्तीर्थैश्च स त्वथ}%।। ३४ ।।

\twolineshloka
{त्यक्त्वा देहं दिवं यातो दधीचः स्वेन तेजसा}
{विश्वकर्मा च तस्यास्थ्नां भागैर्वज्रमथा करोत्}%।। ३५ ।।

\twolineshloka
{आयुधानि च देवानां तथैव च पृथक्पृथक्}
{तेन वज्रेण महता वृत्रं हत्वा महासुरम्}%।। ३६ ।।

\twolineshloka
{जघान दैत्यमुख्यानां नवतिर्नवतिस्तथा}
{ततः प्रभृति मासान्ते नित्यमेव रघूद्वह}%।। ३७ ।।

\onelineshloka*
{सान्निध्यमिह तीर्थानां देवतानां च कल्पितम्} 

\uvacha{मार्कण्डेय उवाच}
\onelineshloka
{एतच्छ्रुत्वा महातेजा दत्त्वा दानानि राघवः}%।। ३८ ।।

\twolineshloka
{ययौ सैन्येन महता भरतोऽमरकण्टकम्}
{तत्र विष्णुपदं प्राप्य पूजयित्वा च राघवः}%।।३९।।

\twolineshloka
{आससाद नदीं गौरीं पुण्यां पापयापहाम्}
{विश्वामित्राज्ञया रक्षः प्रविवेश पुरा नृपम्} %॥४०॥

\twolineshloka
{कशाहतेन मार्गस्थं शप्तं भूपाल शक्तिना}
{सौदासं स प्रविष्टस्तु भक्षयामास शक्तिनम्}%।। ४१ ।।

\twolineshloka
{ततः पुत्रशतं राजन्वसिष्ठस्यैव सत्वरः}
{हृते पुत्रशते दुखाद्वसिष्ठो भगवानृषिः}%।। ४२ ।।

\twolineshloka
{विवेश निम्नगां गौरीं प्राणत्यागचिकीर्षया}
{विप्रस्य भूप यातस्य विद्रुता सा तदाभवत्}%।। ४३ ।।

\twolineshloka
{ततः प्रभृति लोकेऽस्मिञ्छतद्रुरिति शब्दिता ।। शतद्रु}
{सर्वपापप्रशमनी सर्वकल्याणकारिणी}%।।४४।।

\twolineshloka
{स्नातानां च तथा राजन्दशधेनुफलप्रदा}
{तां समासाद्य तत्रापि दत्त्वा दानं स राघवः}%।।४५।।

\twolineshloka
{उत्तीर्य तां ययौ तत्र सा चान्त्या यत्र निम्नगा}
{यस्यां स्नात्वा विमुच्यन्ते सर्वपापभयैर्नराः}%।।४६।।

\twolineshloka
{यस्यां स्नानादवाप्नोति दशगोदानजं फलम्}
{आषाढे च तथा कृत्वा गोसहस्रफलं लभेत्}%।। ४७ ।।

\twolineshloka
{गौरीतोयाद्विनिर्मुक्तो वसिष्ठो भगवानृषिः}
{पाशबन्धभरैर्यस्यां पपात सहसा नृप}%।।४८।।

\twolineshloka
{विपाशश्च तथा देव्या कृत्वा तीरे विसर्जितः ।। विपाशा}
{शक्तिपुत्रं ततो दृष्ट्वा वसिष्ठोऽपि पराशरम्}%।। ४९ ।।

\twolineshloka
{अस्ति सन्तानमित्युक्त्वा मरणाद्विनवर्तत}
{सा चान्त्या च तथा लोके विपाशेत्यभिधीयते} %॥५०॥

\twolineshloka
{विपाशां च समुत्तीर्णे भरते धर्मवत्सले}
{विवेश कटकं तस्य कुणिदेशो महोदयः}%।। ५१ ।।

\twolineshloka
{त्रैगर्तो वसुधानश्च कुलूताधिपतिर्जयः}
{दाशेरकस्तथा राजा गोवाशन इति श्रुतः}%।। ५२ ।।

\twolineshloka
{इरावतीं शीघ्रगमां भरतोऽपि तदा ययौ}
{दशधेनुफलं यत्र स्नात एव समश्नुते}%।। ५३ ।।

\twolineshloka
{षष्टितीर्थसहस्राणि वहत्येका इरावती}
{अष्टम्यां तु विशेषेण यत्र युज्येत रेवती}%।। ५४ ।।

\twolineshloka
{तत्र दत्त्वा बहुविधं दानं रघुकुलोद्वहः}
{उत्तीर्य तां ययौ शीघ्रं देविकां पापनाशिनीम्}%।। ५९ ।।

\twolineshloka
{दृष्टमात्रैव या देवी सर्वकल्मषनाशिनी}
{शरीरबहुला सा तु हरस्य दयिता उमा}%।। ५६ ।।

\twolineshloka
{तत्रापि दत्त्वा दानानि पूजयामास शङ्करम्}
{भरते चाथ तत्रस्थे विविशुः पञ्च पार्थिवाः}%।। ५७ ।।

\twolineshloka
{पार्वतीया महाराज पदातिगणसङ्कुलाः}
{कुमारः श्रेणिमाञ्छूरो बलबन्धुः सुयोधनः}%।। ५८ ।।

\twolineshloka
{मद्रराजोंऽशुमांश्चैव तथैव च महाबलः}
{पूजितो मद्रराजेन शाकलेन नरोत्तमः}%।। ५९ ।।

\twolineshloka
{त्रिरात्रमुषितः श्रीमांश्चन्द्रभागां दीं ययौ}
{चन्द्रांशुशीतलजलां सर्वपापप्रणाशिनीम्} %॥६०॥

\twolineshloka
{बहुतीर्थसमायुक्तां स्नानात्सर्वप्रदां नृणाम् ।। ।}
{विशेषेण महाराज माघपुष्यत्रयोदशीम्}%।। ६१ ।।

\twolineshloka
{भरते तां समुत्तीर्णे विवेश कटकं ततः}
{एतच्छ्रुतञ्जयः श्रीमानभिचारः कृतञ्जयः}%।। ६२ ।। ।

\twolineshloka
{काश्मीरकश्च धर्मात्मा सुबाहुरिति विश्रुतः}
{आससाद स तैः सार्द्धं वितस्तां तु महानदीम्}%।।६३।।

\twolineshloka
{स्वर्गलोकप्रदां स्नाने सर्वकल्मषनाशिनीम्}
{प्रोष्ठपादस्य मासस्य शुक्लपक्षत्रयोदशीम्}%।। ६४ ।।

\twolineshloka
{विशेषेण महाराज पुण्यां परमपावनीम्}
{भरतस्तां समुत्तीर्य सुदामां चैव निम्नगाम्}%।। ६५ ।।

\twolineshloka
{आससाद महातेजाः कैकेयान्यदुनन्दन}
{बलेन चतुरङ्गेण युधाजित्कैकयाधिपः}%।। ६६ ।।

\twolineshloka
{निर्ययौ भरतं प्राप्तं श्रुत्वा भरतमातुलः}
{नागराश्च तथा मुख्या राजगृहनिवासिनः}%।। ६७ ।।

\twolineshloka
{ब्राह्मणाः क्षत्त्रिया वैश्या ये च वर्णवरा जनाः}
{यानैरुच्चावचैः सर्वै नगरात्तु विनिर्गताः}%।। ६८ ।।

\twolineshloka
{वादित्रान्पुरतः कृत्वा गणिकाश्च विनिर्गताः}
{प्रतिग्रहनिमित्तं तु राघवस्य महात्मनः}%।। ६९ ।।

\twolineshloka
{भरतोऽपि महातेजाः स समेत्ययुधाजिता}
{अभ्यधावत तत्प्रीत्या मातुलं कैकयाधिपम्} %॥७०॥

\twolineshloka
{कण्ठे गृहीत्वा भरतं मूर्ध्न्युपाघ्राय चासकृत्}
{भरतोऽपि समादाय राजगेहमुपागमत्}%।।७१।। ।।

\twolineshloka
{समे मनोरमे देशे प्रभूतयवसेन्धने}
{शिबिरं भरतश्चक्रे नगरस्याविदूरतः}%।। ७२ ।।

\twolineshloka
{पृथक्पृथक्तदा चक्रुः स्कन्दावारनिवेशनम्}
{मनःप्रियेषु देशेषु नानादेश्या नराधिपाः}%।। ७३ ।।

\twolineshloka
{कृत्वा निवेशान्मनुजेश्वराणामादिश्य भोगान्सकलान्स तेषाम्}
{विवेश नागेन स मातुलस्य पुरं प्रहृष्टो रघुवंशचन्द्रः}%।। ७४ ।।

॥इति श्रीविष्णुधर्मोत्तरे प्रथमखण्डे मार्कण्डेयवज्रसंवादे राजगृहगमनो नाम सप्तोत्तरद्विशततमोऽध्यायः॥२०७॥

\sect{भरतस्य राजगृहप्रवेशवर्णनम् --- अष्टोत्तरद्विशततमोऽध्यायः}

\uvacha{मार्कण्डेय उवाच}

\twolineshloka
{ततः स्वल्पपरीवारो भरतो धर्मवत्सलः}
{अभ्यधाय ततः प्रीत्या मातुलः केकयाधिपः}%।। १ ।।

\twolineshloka
{कण्ठे गृहीत्वा भरतं मूर्ध्न्युपाघ्राय चासकृत्}
{भरतोऽपि समादाय राजन् गृहमुपागतम्}%।।२ ।।

\twolineshloka
{प्रविवेश पुरं हृष्टो मातुलस्य युधाजितः}
{पताकाध्वजसम्बाधं सिक्तं चन्दनवारिणा}%।।३ ।।

\twolineshloka
{धूपेनाऽगुरुसाराणां समन्तादाकुलीकृतम्}
{कृतोपहारं सर्वत्र कुसुमैः पञ्चवर्णकैः}%।। ४ ।।

\twolineshloka
{सर्वपण्यविसंयुक्तं समालङ्कृतबालकम्}
{खमुल्लिखद्भिः प्रासादैः श्वेतमेघगणैर्युतम्}%।।५ ।।

\twolineshloka
{विन्यस्तबलिपूजञ्च देवतायतनेषु च}
{महावादित्रघोषेण समन्तात्कृतनिःस्वनम्}%।।६।।

\twolineshloka
{पौरैर्दिदृश्रुभिः सवैर्राकीर्णापणवीथिकम्}
{राजमार्गे महाराज पिण्डीकृतमहाजनम्}%।।७।।)

\twolineshloka
{राजमार्गादपाकृष्टशून्यसर्वनिवेशनम्}
{योषिद्वृन्दसमाक्रान्त राजमार्गमहागृहम्}%।। ८ ।।

\twolineshloka
{प्रविवेश पुरं श्रीमान्भरतो नागधूर्गतः}
{तस्य कामाभवपुषः प्रवेशे भरतस्य तुं}%।। ९ ।।

\twolineshloka
{गृहकार्याणि सन्त्यज्य ययुर्नार्यो गवक्षकान्}
{काश्चिदर्धानुलिप्ताङ्ग्यः काश्चिदेकाञ्जितेक्षणाः} %॥१०॥

\twolineshloka
{केशैः संयमितैकार्धैः काश्चिदर्धनिवेशितैः}
{एकस्मिंश्चरणे काश्चिद्गृहीत्वा काष्ठपादुकाः}%।।११।।

\twolineshloka
{त्वरिताशा ययुर्नार्यो द्वितीये चर्मपादुकाः}
{तथा पराः समाक्षिप्य पूर्वाक्रान्तगवाक्षकाः}%।।१२।।

\twolineshloka
{व्रजन्तीषु तथान्यासु काश्चिन्नार्यो गवाक्षकात्}
{वेगवत्यो ययुः शीघ्रं सन्त्यक्त्वा चर्मपादुकाः}%।। १३ ।।

\twolineshloka
{नार्यः स्ववदनैश्चक्रुः सुवक्त्राँस्तान्गवाक्षकान्}
{अर्धप्रविष्टसत्कम्बुपाणिवारिजकुड्मलाः}%।। १४ ।।

\twolineshloka
{द्वितीयपाणिसन्दर्शसमाक्रान्तैर्ययुः करैः}
{नीवीबन्धनविश्लेषसमाकुलितचेतनाः}%।। १५ ।।

\twolineshloka
{ययुरेवापरास्तत्र पाणिसंश्लिष्टनीवयः}
{कुसुमप्रकरं काश्चिदूहमानाः शिरोगतैः}%।। १६ ।।

\twolineshloka
{ययुरेवांशुकैदीर्घैस्त्वरमाणा गवाक्षकान्}
{सितासितेन रामाणां रमणीयेन राघवः}%।। १७ ।।

\twolineshloka
{ययौ दृष्टिनिपातेन रज्यमान इवांशुमान्}
{स चकर्ष तदा तासां पतितैर्नेत्ररश्मिभिः}%।।१८ ।।

\twolineshloka
{हृदयान्निगृहीत्वेव गच्छमानः स राघवः}
{स तु दृग्विषये यासां यासां तस्मात्परागतः}%।।१९।।

\twolineshloka
{न ता बुबुधिरे काञ्चित्क्रियां चित्रता इव}
{भरते दूरयातेऽपि निश्चेष्टाः काश्चिदेव ताः} %॥२०॥

\twolineshloka
{आकृष्टास्ता ययुः क्षोणीं प्रेर्यमाणैः सखीजनैः}
{प्रद्युम्नांशसमुत्पन्नः प्रद्युम्नसमदर्शनः}%।। २१ ।।

\twolineshloka
{आदाय तासां चेतांसि ययौ गजगृहं द्रुतम्}
{तस्मिन्राजगृहे राजन्राघवो राजमन्दिरम्}%।। २२ ।।

\twolineshloka
{आससाद महातेजाः कैलासमिव चापरम्}
{प्रविश्य स गृहं मुख्यं प्रेषयामास भूभुजम्}%।। २३ ।।

पानभोजनवासांसि सैनिकानां च यादव ।।

\twolineshloka
{उवास स सुखं तत्र पूजितश्च युधाजिता}
{जगाम चास्तं सविता जपापुष्पोत्करप्रभः}%।। २४ ।।

\twolineshloka
{जाम्बूनदे रत्नसहस्रचित्रे स राजपुत्रः शयने विचित्रे}
{सुष्वाप रात्रौ भवने विचित्रे निदाघरात्रिं पवने विचित्रे}%।। २५ ।।

॥इति श्रीविष्णुधर्मोत्तरे प्रथमखण्डे मार्कण्डेयवज्रसंवादे भरतस्य राजगृहप्रवेशवर्णनं नामाष्टोत्तरद्विशततमोऽध्यायः॥२०८॥

\sect{युद्धप्रसङ्गवर्णनम् --- नवोत्तरद्विशततमोऽध्यायः}

\uvacha{मार्कण्डेय उवाच}

\twolineshloka
{तस्मिन्रात्र्यवसाने तु भूमिपानां पृथक्पृथक्}
{कटकेष्वभ्यहन्यन्त संज्ञातूर्याणि यादव}%।। १ ।।।

\twolineshloka
{बहूनां बलमुख्यानां नानालिङ्गानि भागशः}
{तेषां कोलाहलः शब्दो बभूव गगनङ्गमः}%।। २ ।।

\twolineshloka
{विविशुर्भरतस्यापि ततः प्राबोधिका जनाः}
{वेणिका गायना मुख्या वंशवाद्यविदश्च ये}%।। ३ ।।

\twolineshloka
{मार्दङ्गिका पाणविकाः शङ्खवादनकाश्च ये}
{रक्तकण्ठाः सुमधुरा ये च मङ्गलपाठकाः}%।। ४ ।।

\twolineshloka
{सूतमागधमुख्याश्च बन्दिनश्च नराधिप}
{तुष्टुवुर्भरतं वीरं सुखसुप्तं महामतिम्}%।। ५ ।।

\twolineshloka
{सुप्रभातं समुत्तिष्ठ प्रभाता रजनी शुभा}
{दिक्प्राची रघुशार्दूल वर्ततेऽरुणरञ्जिता}%।। ६ ।।

\twolineshloka
{नूनमस्यां हि वेलायां मातलिस्त्रिदशाधिपम्}
{विबोधयति राजेन्द्र सुखाय जगतां विभुम्}%।। ७ ।।

\twolineshloka
{त्वयि सुप्ते जगत्सुप्तं विबुद्धे च सुखान्वितम्}
{तस्मादुत्तिष्ठ लोकानां शिवाय रघुनन्दन}%।। ८ ।।

\twolineshloka
{त्वं हि सर्वगुणारामो यथा रामो महीपतिः}
{गुणैः शशाङ्करश्म्याभैस्त्वया वै रञ्जितं जगत्}%।।९।।

\twolineshloka
{अत्याश्चर्यं महाबाहो यशसा सुसतेन ते}
{मुखान्यरातिवृन्दानां क्रियन्ते मलिनानि यत्} %॥१०॥

\twolineshloka
{कृपाणधारापानीयं दृष्ट्वाऽरातिगणस्तव}
{तृष्णयैव भवत्याशु रञ्जितानिलयो भयात्}%।। ११ ।।

\twolineshloka
{अक्षोभ्यश्चातिगम्भीरो भवान्रत्नाकरस्तथा ।।।}
{अग्राह्यत्वात्समुद्रेण न समः प्रतिभाति नः}%।।१२।।

\twolineshloka
{सौम्यः कलावाँल्लक्ष्मीवान्नयनानन्दकारकः}
{(दोषाकरेण क्षयिणा नौपम्यं ते नराधिप)}%।।१३।।

\twolineshloka
{बाहुभोज्यातिविस्तीर्णा सर्वाश्रयवती दृढा}
{नौपम्यं याति ते सम्यक् क्षोणी विन्ध्येन कम्पिना}%।। १४ ।।

\twolineshloka
{यत्तेऽस्ति तदवश्यं त्वं ददासि रिपुसूदन}
{अविद्यमाना भीर्दत्ता भवतारिगणे कुतः}%।।१५ ।।

\twolineshloka
{निमेषमपि यो दृष्टस्त्वयापाङ्गनिरीक्षणैः}
{समग्रदृष्ट्या य दृष्टो नित्यमेवेक्षणैः श्रियः ।}%।। १६ ।।

\twolineshloka
{क्षुरपर्यन्तधारेण चक्रेणारिगणस्य ते}
{शिरांस्यपहरत्याजौ देवदेवो जनार्दनः}%।। १७ ।।

\twolineshloka
{कपालमाली खट्वाङ्गी शशाङ्ककृतभूषणः ।}
{वामार्धदयिताकारः शङ्करः शं करोतु ते}%।।१८।।

\twolineshloka
{पद्मासनः पद्मजन्मा सर्वलोकपितामहः}
{ऋद्धिं मेधां धृतिं लक्ष्मीं बलं च विदधातु ते}%।। १९।।

\twolineshloka
{नभश्चरोऽम्बुजो देवो दिग्वधूरपूरकः}
{ब्रह्माण्डमण्डपे दीपः सुप्रभातं करोतु ते} %॥२०॥

\twolineshloka
{केशयक्षेशदेवेशप्रेतेश्वरनिशाचराः}
{सर्वदेवगणैः सार्धं सुप्रभातं दिशन्तु ते}%।। २१ ।।

\twolineshloka
{ऋषयः सरितः शैलाः सागराश्च दिशो दश}
{कालस्यावयवाश्चैव सुप्रभातं दिशन्तु ते}%।। २२ ।।

\twolineshloka
{इति शृण्वन्गिरं पुण्यामेषां मङ्गलवादिनाम्}
{महापुण्याहघोषेण तत्याज शयनं तदा}%।।२३।।

\twolineshloka
{आयव्ययं स शुश्राव लेखकैर्गणकैः सह}
{वेगोत्सर्गं ततः कृत्वा ययौ स्नानगृहं शुभम्}%।। २४ ।।

\twolineshloka
{स्नानारम्भं ततश्चक्रे दन्तधावनपूर्वकम्}
{उत्सादितः कषायेण बलवद्भिर्नरैस्तदा}%।। २५ ।।

\twolineshloka
{सुखं भद्रासनासीनः सूक्ष्माम्बरधरः प्रभुः}
{सस्नौ स विविधैस्तोयैर्नदीसागरजैः शुभैः}%।। २६ ।।

\twolineshloka
{कुम्भैः सुवर्णमाहेयैस्ताम्रै रौप्यमयैस्तथा}
{शताधिकैर्महाराज सर्वौषधि समन्वितैः}%।। २७ ।।

\twolineshloka
{क्षीरप्रवाहसंयुक्तैर्माल्यकण्ठैः सुपूजितैः}
{आवर्जितैर्महीपाल सुस्नातालङ्कृतैर्नरैः}%।। २८ ।।

\twolineshloka
{चन्दनस्रावसम्पूर्णैर्द्विजमन्त्रानुमन्त्रितैः}
{बभार वसनं चारु स्नानतोयौघसङ्कुलम्}%।। २९ ।।

\twolineshloka
{शशाङ्कमण्डलं पूर्णं तन्वभ्रैरिवसंवृतम्}
{वाद्यपुण्याहघोषेण तथा गीतस्वनेन च} %॥३०॥

\twolineshloka
{स्नात्वोपस्पृश्य विधिवत्पूर्वां सन्ध्यां समाहितः}
{ददर्श वदनं चारु दर्पणे चाथ सर्पिषि}%।। ३१ ।।

\twolineshloka
{स सुवर्णे महाराज दैवज्ञेनाभिमन्त्रिते}
{दिनेशं तिथिनक्षत्रं ततः शुश्राव राघवः}%।। ३२ ।।

\twolineshloka
{सांवत्सरमुखोद्गीर्णं कलिदुःस्वप्ननाशनम्}
{चकाराभ्यर्चनं चाथ देवदेवस्य चक्रिणः}%।। ३३ ।।

\twolineshloka
{गन्धमाल्यनमस्कारधूपूदीपान्नसम्पदा}
{स्तवैर्बलिप्रदानैश्च गीतवाद्यस्वनेन च}%।। ३४ ।।

\twolineshloka
{सम्पूज्य देवदेवेशं विवेशाग्निगृहं शुभम्}
{तत्राग्निं समुपस्थाय हुतं पूर्वं पुरोधसा}%।। ३५ ।।

\twolineshloka
{दुःस्वप्ननाशनं कर्म सिद्धिवृद्धिजयप्रदम्}
{प्रागेव तस्य कृतवान्विद्वान्नृपपुरोहितः}%।। ३६ ।।

\twolineshloka
{ततस्त्वौपसमे वह्नौ भरतः प्रीतमानसः}
{श्रीसूक्तं पौरुषं सूक्तं जुहाव प्रयतस्तदा}%।। ३७ ।।

\twolineshloka
{आज्येन मन्त्रपूतेन विधिना सुसमाहितः}
{दत्त्वा पूर्णाहुतिं चाग्नौ कृतजप्यो महामतिः}%।। ३८ ।।

\twolineshloka
{उदकेनार्चनं चक्रे देवानां पितृभिः सह}
{ततः स निर्ययौ तस्माद्भरतो वह्निवेश्मनः}%।। ३९ ।।

\twolineshloka
{निष्क्रम्य पूजयामास ब्राह्मणान्वसना तदा}
{गोभिरश्वैः सुवर्णेन दधिपुष्पफलान्वितैः} %॥४०॥

\twolineshloka
{मोदकैश्च तथा रत्नैर्वस्त्रैश्च रघुनन्दनः}
{पूजितानां द्विजेन्द्राणां भरतेन महात्मना}%।। ४१ ।।

\twolineshloka
{पुण्याहघोषस्त्रिदिवं जगाम मधुरस्वरः}
{स पूजयित्वा विप्रेन्द्रान्प्रविश्य च तथा गृहम्}%।। ४२ ।।

\twolineshloka
{नित्यकर्म च कृत्वेदं चन्दनेन सुगन्धिना}
{सूक्ष्मशुक्लपरीधानो वरधूपेन धूपितः}%।। ४३ ।।

\twolineshloka
{आभूष्य सर्वगात्राणां भूषणानि रघूद्वहः}
{शुक्लं सुगन्धि माल्यं च स्रजश्च विविधास्तथा}%।। ४४ ।।

\twolineshloka
{मङ्गलालम्भनं कृत्वा निश्चक्राम सभागृहम्}
{क्लृप्तं शय्यासनं तत्र वररत्नविभूषितम्}%।। ४५ ।।

\twolineshloka
{महार्घतोरणोपेतं सोत्तरच्छदमृद्धिमत्}
{वितानञ्च तथा दत्तं तस्योपरि महर्द्धिमत्}%।। ४६ ।।

\twolineshloka
{आसीनमासने तस्न्निभरतं सत्यसङ्गरम्}
{प्रांशवो बद्धनिस्त्रिंशाः कवचोत्तमभूषिताः}%।। ४७ ।।

\twolineshloka
{रक्ताम्बरधरा वीरा ररक्षुः पृष्ठसंस्थिताः}
{तथैवोभयपार्श्वस्थाः पूर्णचन्द्रनिभाननाः}%।। ४८ ।।

\twolineshloka
{वारमुख्याः सुवेशास्तमुपासां चक्रिरे तदा}
{बालव्यजनधारिण्यस्तालवृन्तकराः पराः}%।। ४९।।

\twolineshloka
{ताम्बूलभाण्डधारिण्यो नीलनीजलोचनाः}
{कुण्डली बद्धनिस्त्रिंशो दण्डपाणिः सुवेशवान्} %॥५०॥

\twolineshloka
{उवाच भरतं क्षत्ता भूमिविन्यस्तजानुकः}
{दिदृक्षवस्ते सम्प्राप्ता ब्राह्मणाः संशयच्छिदः}%।। ५१ ।।

\twolineshloka
{श्रेणीमहत्तरा ये च बलमुख्यास्तथैव च}
{प्रवेशयैनानित्युक्तो द्वास्थान्क्षत्ता ततोऽब्रवीत}%।। ५२ ।।

\twolineshloka
{ब्राह्मणान्बलमुख्यांश्च प्रवेशयत सत्वरम्}
{आशीर्भिरभिनन्द्यैनं सम्प्रविष्टा द्विजोत्तमाः}%।। ५३ ।।

\twolineshloka
{बृसीषु दन्तपीठेषु विविशुश्च यथासुखम्}
{ततस्तु बलमुख्यानां नमतां भरतं तदा}%।। ५४ ।।

\twolineshloka
{क्षत्ता जग्राह नामानि कण्ठारक्तस्वरान्वितः}
{ततस्तेषूपविष्टेषु द्वारेषु विवृतेषु च}%।। ५५ ।।

\twolineshloka
{प्रविवेश जनः सर्वो न न्यवार्येत कश्चन}
{एतस्मिन्नेव काले तु शुश्रुवे तुमुलो ध्वनिः}%।। ५६ ।।।

\twolineshloka
{भरतं द्रष्टुकामानां भूमिपानां महात्मनाम्}
{ह्रादेन गजघण्टानां बृंहितेन तथैव च}%।। ५७ ।।

\twolineshloka
{ह्रेषितेन तुरङ्गाणां रथनेमिस्वनेन च}
{नामभिः कीर्त्यमानैश्च बन्दिभिः पृथिवीक्षिताम्}%।। ५८ ।।

\twolineshloka
{शङ्खवादित्रघोषेण पटहानां स्वनेन च}
{( आजग्मुर्भरतं द्रष्टुं नरेन्द्राः प्रियदर्शनाः}%।। ५९ ।।

\twolineshloka
{तोरणादवतीर्यैव वाहनेभ्यो हीक्षितः}
{सर्वे स्वल्पपरीवारा विविशुस्ते सभां शुभाम्} %॥६०॥

\twolineshloka
{शिरःकम्पेन भरतं नमस्कृत्य निवेदिताः}
{प्रतीहारेण दक्षेण तेन चकुर्वरासनम्}%।। ६१ ।।

\twolineshloka
{सिंहासनस्थान्नृपतीन्भरतस्त्वनुरूपयन्}
{गिरा पप्रच्छ कुशलं पूजयामास चाप्यथ}%।। ६२ ।।

\twolineshloka
{तुष्टुवुर्वन्दिनस्तत्र नानादेश्यान्नराधिपान्}
{निवेदयन्तः स्तुत्यन्ते भरताय महात्मने}%।।६३।।

\twolineshloka
{स्तुवतां भरतं तत्र बन्दिनां स महास्वनः}
{प्रासादभोगसंरुद्धो विपुलः समपद्यत}%।। ६४ ।। 

\twolineshloka
{क्ष्मापालमौलिमाणिक्यमरीचिविकटोज्ज्वलम्}
{बालातपांशुच्छुरितं बभूव च सभागृहम्}%।। ६५ ।।

\twolineshloka
{ततः स भरतः श्रीमान्विससर्ज नराधिपान्}
{स्वहस्तदत्तताम्बूलान्प्रतीहारनिवेदितान् ।।}%।। ६६ ।।

\twolineshloka
{ततः समुत्थाय ययौ द्वितीयगृहमुत्तमम्}
{तत्र चक्रे तदा मन्त्रं मातुलेन युधाजिता}%।। ६७ ।।

\onelineshloka*
{पुरोधसा च गार्ग्येण स्वेन कालविदा तथा}

{युधाजिदुवाच}
\onelineshloka
{काले त्वमीप्सिते प्राप्तो गन्धर्वाणां वधेच्छया}%।। ६८ ।।

\twolineshloka
{तत्र यावन्न जानन्ति गन्धर्वास्ते त्वदागमम्}
{अवस्कन्देन तानद्य तावद्रात्रौ जहि प्रभो}%।। ६९ ।।

\onelineshloka*
{अवस्कन्देन निधनं सुखं तेषां भविष्यति}

\uvacha{कालविदुवाच}

\onelineshloka
{सुप्ते प्रमत्ते विश्वस्ते तथा राजञ्च्छ्रमान्विते}%।।७०।।

\twolineshloka
{अवतीर्णे बले चैव सरितं वहतीं तथा}
{रात्रौ जागरणश्रान्ते छद्मयुद्धं विधीयते}%।। ७१ ।।

\twolineshloka
{रात्रौ विहारशीलास्ते गन्धर्वाः सततं प्रभो}
{न तेऽवस्कन्दमर्हन्ति रात्रौ कैकेयिनन्दन}%।। ७२ ।।

\onelineshloka
{विजयश्च दिवा युद्धे भरतस्य प्रदृश्यते}


\uvacha{गार्ग्य उवाच}
\onelineshloka*
{आदौ दूतेन वक्तव्यं गन्धर्वाणां प्रयोजनम्}%।। ७३ ।।

\twolineshloka
{यथा देशमिमं त्यक्त्वा व्रजध्वं तुहिनाचलम्}
{गन्धर्वाणां निवासस्तु हिमवत्यचलोत्तमे}%।। ७४ ।।

\twolineshloka
{पूर्वमेव कृतस्तेन येनेदं निर्मितं जगत्}
{स्थानमेतन्मनुष्याणां त्यक्तुमर्हथ मा चिरम्}%।। ७५ ।।

\twolineshloka
{ते च दूतवचः श्रुत्वा स्थानं दद्युर्नराधिप}
{असंशयमुपारुह्य चैतन्मम मतं भवेत्}%।। ७६ ।।

\uvacha{भरत उवाच}

\twolineshloka
{राघवाः सत्यसन्धास्तु कूटयुद्धं न शिक्षिताः}
{तस्मात्तेषां वधः कार्यः सुयुद्धेन मया नृप}%।। ७७ ।।

\twolineshloka
{गार्ग्यवाक्यं तथा बुद्ध्या रोचतेऽतिशयेन मे}
{प्रयातु तेषां दौत्येन गार्ग्य एव महायशाः}%।। ७८ ।।

\uvacha{युधाजिदुवाच}

\twolineshloka
{गच्छ गार्ग्य महाभाग गन्धर्वाधिपतिं प्रति}
{तं च श्रावय वाक्यानि त्वयोक्तानीह यानि मे}%।। ७९ ।।

\twolineshloka
{अक्रियायां तथा तेषां श्रावयाऽग्र्याणि म चिरम्}
{अस्यैव नाम्ना धर्मज्ञ भरतस्य महात्मनः} %॥८०॥

\uvacha{मार्कण्डेय उवाच}

\twolineshloka
{एवमस्त्वित्यथोक्तोऽसौ ययौ गार्ग्यो महायशाः}
{रथेन काञ्चनाङ्गेन गन्धर्वनगरं प्रति}%।। ८१ ।।

\twolineshloka
{गते पुरोहिते गार्ग्ये भरतोऽपि महाशयाः}
{हस्तिपृष्ठे रथे चाश्वे शस्त्रे शास्त्रे तथैव च}%।। ८२ ।।

\twolineshloka
{व्यायामं च तथा चक्रे नियुद्धे च यदूत्तम}
{उत्सारितस्तथा पदभ्यां धावद्भिः कुशलैर्जनैः}%।।८३।।

\twolineshloka
{स्नातैः सम्पूजितो विष्णुर्विधिवत्सात्त्वतोत्तमैः}
{नमस्कृत्य तथैवाग्निं हुतं सुष्ठु पुरोधसा}%।। ८४ ।।

\twolineshloka
{तथा भुक्तवतां श्रुत्वा पूजितानां द्विजन्मनाम्}
{पुण्याहघोषं वित्तेन भूमिपाल विसर्ज्य तान्}%।। ८५ ।।

\twolineshloka
{परार्ध्यचन्दनाक्ताङ्गस्तनुचारुसिताम्बरः}
{सर्वालङ्करणोपेतः स्रग्वी धूपेन धूपितः}%।। ८६ ।।

\twolineshloka
{आसीनस्त्वासने दिव्ये बुभुजे स्वजनैर्युतः}
{मातुलस्य महार्घाणि शुचीनि गुणवन्ति च}%।। ८७ ।।

\twolineshloka
{अपरीक्षितपूर्वाणि पुरुषैराप्तकारिभिः}
{नरपक्षिमृगाणान्तु लिङ्गैर्वह्नौ तथैव च}%।। ८८ ।।

\twolineshloka
{भक्ष्यं भोज्यं च लेह्यं च पेयं चोष्यं तथैव च}
{पात्रेषु रुक्मरौप्येषु तथा मणिमयेषु च}%।। ६९ ।।

\twolineshloka
{भुक्त्वान्नं गीतशब्देन चाप्तैः कतियैः सह}
{तथोपस्पृश्य धर्मात्मा दन्तधावनपूर्वकम्} %॥९०॥

\twolineshloka
{चक्रस्य शयनं भेजे वामपार्श्वेन शत्रुहा}
{इतिहासं स शुश्राव तत्रस्थः पुरुषोत्तमः}%।। ९१ ।।

\twolineshloka
{ततः स शयनं त्यक्त्वा शास्त्राभ्यासं महायशाः}
{चकार रघुशार्दूलः सतां मार्गमनुव्रजन्}%।। ९२ ।।

\twolineshloka
{एतस्मिन्नेव काले तु सह गार्ग्यो युधाजिता}
{विवेश भरतं द्रष्टुं रथरेणुपरिप्लुतः}%।। ९३ ।।

\onelineshloka*
{सुखासीनश्च भरतं वाक्यमेतत्ततोऽब्रवीत्}

\uvacha{गार्ग्य उवाच}
\onelineshloka
{ततो वाक्यानि सर्वाणि शैलूषः श्रावितो मया}%।। ९४ ।।

\twolineshloka
{न तानि तस्य रोचन्ते सङ्ग्रामस्तस्य रोचते}
{भरतेन समागम्य श्वोभूते द्विजपुङ्गव}%।। ९५ ।।

\twolineshloka
{भरतं नाशयिष्यामि नीहारं चन्द्रमा यथा}
{इत्युक्त्वा स तु मां राजा प्रेषयामास सत्वरः}%।। ९६ ।।

\twolineshloka
{आह्वानदुन्दुभिस्तत्र निष्क्रान्ते मयि चाहतः}
{एतज्ज्ञात्वा स युद्धाय प्रातः सज्जो भवेत्तव}%।। ९७ ।।

\uvacha{मार्कण्डेय उवाच}

\twolineshloka
{इति गार्ग्यवचः श्रुत्वा भरतो गार्ग्यमब्रवीत्}
{गच्छ शीघ्रं गृहं ब्रह्मञ्छ्रान्तो रथबलाध्वतः}%।।९८।।

\twolineshloka
{अहमाज्ञापयिष्यामि सर्वं साङ्ग्रामिकं विधिम्}
{एतावदुक्त्वा विजयं सेनाध्यक्षमथाऽब्रवीत्}%।।९९।।

\twolineshloka
{ममोष्ट्रवाहिभिः शीघ्रं शिबिरेषु महीक्षिताम्}
{योधानाज्ञापयत्वद्य श्वोभूताय रणाय वै} %॥१००॥

\twolineshloka
{युद्धावधानिकं सर्वं कर्तव्यं च तथा त्वया}
{एवमाज्ञाप्य नागानां तुरगाणां तथैव च}%।। १०१ ।।

\twolineshloka
{प्रत्यावेक्षां ततः कृत्वा सन्ध्यामन्वास्य पश्चिमाम्}
{रहोगतः स शुश्राव नराणां मूढभाषितम्}%।।१०२ ।।

\twolineshloka
{आरुरोह तदा श्रुत्वा प्रासादं हिमपाण्डुरम्}
{कैलासशिखराकारं निर्वातं रजनीमुखे ।।}%।। १०३ ।।

\twolineshloka
{ततस्तु सैनिकः कश्चिच्छिबिरे भरतस्य तु}
{तलं तलेनाभ्यहनत्पवनार्थं यदृच्छया}%।। १०४ ।।

\twolineshloka
{ततस्तु सैनिकैः सर्वैस्तलतालैर्महास्वनम्}
{चक्रिरे पुरुषव्याघ्र तस्मिन्काले दिवं गतम्}%।। १०५ ।।

\twolineshloka
{प्रववौ च तदानीतो वायुर्मनुजपुङ्गव}
{एतस्मिन्नेव काले तु मद्रराजस्तदांशुमान्}%।। १०६ ।।

\twolineshloka
{प्रासादवरमारूढो ज्ञातवान्भरतं तदा}
{दीपालोकेन लक्ष्म्या च प्रासादस्य विवृद्धया}%।। १०७ ।।।

\twolineshloka
{स जगाम तदा राजा प्रहसन्सैनिकं जनम्}
{भरतस्य प्रदास्यामि युक्त्यैव बलदर्शनम्}%।। १०८ ।।

\twolineshloka
{तृणमुष्टिमुपादीप्य सर्वोऽपि कटके जनः}
{पाणावादाय मुदितः क्ष्वेडाशब्दं करोतु वै}%।। १०९ ।।

\twolineshloka
{पार्थिवेनैवमुक्ते तु कटके तस् धीमतः}
{सोल्काहस्तो जनः सर्वः क्षणेन समपद्यत} %॥११०॥

\twolineshloka
{शिबिरं मद्रराज्ञस्तु द्वितीयमिव चाम्बरम्}
{बभूव तारकाचित्रमुल्काहस्तैस्तदा नरैः}%।। १११ ।।

\twolineshloka
{तद्बलौघमपर्यन्तं सोल्काहस्तैर्जनैर्वृतम्}
{दृष्ट्वा जगाम धर्मात्मा परां प्रीतिं रघूद्वहः}%।। ११२ ।।

\twolineshloka
{क्ष्वेडाः किलकिलाश्चैव श्रुत्वा हर्षमुपागतः}
{तस्मिन्प्रशान्ते ज्वलने मन्दीभूते च निस्वने}%।। ११३ ।।

\twolineshloka
{अभ्यहन्यन्त भूपानां शिबिरेषु पृथक्पृथक्}
{संज्ञातूर्याणि रम्याणि नानालिङ्गानि चाप्यथ}%।। ११४ ।।

\twolineshloka
{सुबहूनि महाराज तेन कोलाहलं महत्}
{बभूव प्रीतिजननं भरतस्य महात्मनः}%।। ११५ ।।

मन्त्रयित्वा ततः श्रीमान्क्षणमात्रं युधाजिता।।

\twolineshloka
{तेनैव सह भुक्त्वा च सुष्वाप शयनोत्तमे}
{मधुरेण सुगीतेन वीणावेणुरवेण च}%।। ११६ ।।

\twolineshloka
{हिमावदाते वरमाल्यचित्रे वितानकाधोविहिते मनोज्ञे}
{सुष्वाप रात्रिं स महानुभावो भोगीन्द्रभोगे मधुजिद्यथैव}%।। ११७ ।।

॥इति श्रीविष्णुधर्मोत्तरे प्रथमखण्डे मार्कण्डेयवज्रसंवादे युद्धप्रसङ्गवर्णनं नाम नवोत्तरद्विशततमोऽध्यायः॥२०९॥

    \chapt{शिव-पुराणम्}

\sect{रामपरीक्षा-वर्णनम् --- चतुर्विंशोऽध्यायः}

\src{शिव-पुराणम्}{द्वितीयायां रुद्रसंहितायां}{द्वितीये सतीखण्डे}{अध्यायः २४--२५}
\tags{concise, complete}
\notes{While wandering with Satī, Śiva bows to the grief-stricken Rāma in the forest, telling Her He is Viṣṇu incarnate. Still doubtful, Satī takes the form of Sītā to test him, but Rāma immediately recognises her as Satī, proving His divinity and dispelling Her doubts.}
\textlink{}
\translink{https://www.wisdomlib.org/hinduism/book/shiva-purana-english/d/doc226044.html}

\storymeta

\uvacha{नारद उवाच}

\twolineshloka
{ब्रह्मन् विधे प्रजानाथ महाप्राज्ञ कृपाकर}
{श्रावितं शङ्करयशस्सतीशङ्करयोः शुभम्} %।। १ ।।

\twolineshloka
{इदानीं ब्रूहि सत्प्रीत्या परं तद्यश उत्तमम्}
{किमकार्ष्टां हि तत्स्थौ वै चरितं दम्पती शिवौ} %।। २ ।।

\uvacha{ब्रह्मोवाच}

\twolineshloka
{सतीशिवचरित्रं च शृणु मे प्रेमतो मुने}
{लौकिकीं गतिमाश्रित्य चिक्रीडाते सदान्वहम्} %।। ३ ।।

\twolineshloka
{ततस्सती महादेवी वियोगमलभन्मुने}
{स्वपतश्शङ्करस्येति वदन्त्येके सुबुद्धयः} %।। ४ ।।

\twolineshloka
{वागर्थाविव सम्पृक्तौ शक्तोशौ सर्वदा चितौ}
{कथं घटेत च तयोर्वियोगस्तत्त्वतो मुने} %।।५।।

\twolineshloka
{लीलारुचित्वादथ वा सङ्घटेताऽखिलं च तत्}
{कुरुते यद्यदीशश्च सती च भवरीतिगौ} %।। ६ ।।

\twolineshloka
{सा त्यक्ता दक्षजा दृष्ट्वा पतिना जनकाध्वरे}
{शम्भोरनादरात्तत्र देहं तत्याज सङ्गता} %।। ७ ।।

\twolineshloka
{पुनर्हिमालये सैवाविर्भूता नामतस्सती}
{पार्वतीति शिवं प्राप तप्त्वा भूरि विवाहतः} %।। ८ ।।

\uvacha{सूत उवाच}

\twolineshloka
{इत्याकर्ण्य वचस्तस्य ब्रह्मणस्स तु नारदः}
{पप्रच्छ च विधातारं शिवाशिवमहद्यशः} %।।९।।

\uvacha{नारद उवाच}

\twolineshloka
{विष्णुशिष्य महाभाग विधे मे वद विस्तरात्}
{शिवाशिवचरित्रं तद्भवाचारपरानुगम्} %।। १० ।।

\twolineshloka
{किमर्थं शङ्करो जायां तत्याज प्राणतः प्रियाम्}
{तस्मादाचक्ष्व मे तात विचित्रमिति मन्महे} %।। ११ ।।

\twolineshloka
{कुतोऽह्यध्वरजः पुत्रां नादरोभूच्छिवस्य ते}
{कथं तत्याज सा देहं गत्वा तत्र पितृक्रतौ} %।। १२ ।।

\twolineshloka
{ततः किमभवत्तत्र किमकार्षीन्महेश्वरः}
{तत्सर्वं मे समाचक्ष्व श्रद्धायुक् तच्छुतावहम्} %।। १३ ।।

\uvacha{ब्रह्मोवाच}

\twolineshloka
{शृणु तात परप्रीत्या मुनिभिस्सह नारद}
{सुतवर्य महाप्राज्ञ चरितं शशिमौलिनः} %।। १४ ।।

\twolineshloka
{नमस्कृत्य महेशानं हर्यादिसुरसेवितम्}
{परब्रह्म प्रवक्ष्यामि तच्चरित्रं महाद्भुतम्} %।। १५ ।।

\twolineshloka
{सर्वेयं शिवलीला हि बहुलीलाकरः प्रभुः}
{स्वतन्त्रो निर्विकारी च सती सापि हि तद्विधा} %।। १६ ।।

\twolineshloka
{अन्यथा कस्समर्थो हि तत्कर्मकरणे मुने}
{परमात्मा परब्रह्म स एव परमेश्वरः} %।। १७ ।।

\twolineshloka
{यं सदा भजते श्रीशोऽहं चापि सकलाः सुराः}
{मुनयश्च महात्मानः सिद्धाश्च सनकादयः} %।। १८ ।।

\twolineshloka
{शेषस्सदा यशो यस्य मुदा गायति नित्यशः}
{पारं न लभते तात स प्रभुश्शङ्करः शिवः} %।। १९ ।।

\twolineshloka
{तस्यैव लीलया सर्वोयमिति तत्त्वविभ्रमः}
{तत्र दोषो न कस्यापि सर्वव्यापी स प्रेरकः} %।। २० ।।

\twolineshloka
{एकस्मिन्समये रुद्रस्सत्या त्रिभुवने भवः}
{वृषमारुह्य पर्याटद्रसां लीलाविशारदः} %।। २१ ।।

\twolineshloka
{आगत्य दण्डकारण्यं पर्यटन् सागराम्बराम्}
{दर्शयन् तत्र गां शोभां सत्यै सत्यपणः प्रभुः} %।। २२ ।।

\twolineshloka
{तत्र रामं ददर्शासौ लक्ष्मणेनान्वितं हरः}
{अन्विष्यन्तं प्रियां सीतां रावणेन हृता छलात्} %।। २३ ।।

\twolineshloka
{हा सीतेति प्रोच्चरन्तं विरहाविष्टमानसम्}
{यतस्ततश्च पश्यन्तं रुदन्तं हि मुहुर्मुहुः} %।। २४ ।।

\twolineshloka
{समिच्छन्तं च तत्प्राप्तिं पृच्छन्तं तद्गतिं हृदा}
{कुजादिभ्यो नष्टधियमत्रपं शोकविह्वलम्} %।।२५।।

\twolineshloka
{सूर्यवंशोद्भवं वीरं भूपं दशरथात्मजम्}
{भरताग्रजमानन्दरहितं विगतप्रभम्} %।।२६।।

\twolineshloka
{पूर्णकामो वराधीनं प्राणमत्स्म मुदा हरः}
{रामं भ्रमन्तं विपिने सलक्ष्मणमुदारधीः} %।।२७।।

\twolineshloka
{जयेत्युक्त्वाऽन्यतो गच्छन्नदात्तस्मै स्वदर्शनम्}
{रामाय विपिने तस्मिच्छङ्करो भक्तवत्सलः} %।।२८।।

\twolineshloka
{इतीदृशीं सतीं दृष्ट्वा शिवलीलां विमोहनीम्}
{सुविस्मिता शिवं प्राह शिवमायाविमोहिता} %।।२९।।

\uvacha{सत्युवाच}

\twolineshloka
{देव देव परब्रह्म सर्वेश परमेश्वर}
{सेवन्ते त्वां सदा सर्वे हरिब्रह्मादयस्सुराः} % ।।2.2.24.३०।।

\twolineshloka
{त्वं प्रणम्यो हि सर्वेषां सेव्यो ध्येयश्च सर्वदा}
{वेदान्तवेद्यो यत्नेन निर्विकारी परप्रभुः} %।।३१।।

\twolineshloka
{काविमौ पुरुषौ नाथ विरहव्याकुलाकृती}
{विचरन्तौ वने क्लिष्टौ दीनौ वीरौ धनुर्धरौ} %।।३२।।

\twolineshloka
{तयोर्ज्येष्ठं कञ्जश्यामं दृष्ट्वा वै केन हेतुना}
{सुदितस्सुप्रसन्नात्माऽभवो भक्त इवाऽधुना} %।।३३।।

\twolineshloka
{इति मे संशयं स्वामिञ्शङ्कर छेत्तुमर्हसि}
{सेव्यस्य सेवकेनैव घटते प्रणतिः प्रभो} %।।३४।।

\uvacha{ब्रह्मोवाच}

\twolineshloka
{आदिशक्तिस्सती देवी शिवा सा परमेश्वरी}
{शिवमायावशीभूत्वा पप्रच्छेत्थं शिवं प्रभुम्} %।। ३५ ।।

\twolineshloka
{तदाकर्ण्य वचस्सत्याश्शङ्करः परमेश्वरः}
{तदा विहस्य स प्राह सतीं लीलाविशारदः} %।।३६

\uvacha{परमेश्वर उवाच}

\twolineshloka
{शृणु देवि सति प्रीत्या यथार्थं वच्मि नच्छलम्}
{वरदानप्रभावात्तु प्रणामं चैवमादरात्} %।।३७।।

\twolineshloka
{रामलक्ष्मणनामानौ भ्रातरौ वीरसम्मतौ}
{सूर्यवंशोद्भवौ देवि प्राज्ञौ दशरथात्मजौ} %।।३८।।

\twolineshloka
{गौरवर्णौ लघुर्बन्धुश्शेषेशो लक्ष्मणाभिधः}
{ज्येष्ठो रामाभिधो विष्णुः पूर्णांशो निरुपद्रवः} %।।३९।।

\twolineshloka
{अवतीर्णं क्षितौ साधुरक्षणाय भवाय नः}
{इत्युक्त्वा विररामाऽसौ शम्भुस्मृतिकरः प्रभुः} %।। ४० ।।

\twolineshloka
{श्रुत्वापीत्थं वचश्शम्भोर्न विशश्वास तन्मनः}
{शिवमाया बलवती सैव त्रैलोक्यमोहिनी} %।। ४१ ।।

\twolineshloka
{अविश्वस्तं मनो ज्ञात्वा तस्याश्शम्भुस्सनातनः}
{अवोचद्वचनं चेति प्रभुलीलाविशारदः} %।। ४२ ।।

\uvacha{शिव उवाच}

\twolineshloka
{शृणु मद्वचनं देवि न विश्वसिति चेन्मनः}
{तव रामपरिक्षां हि कुरु तत्र स्वया धिया} %।। ४३ ।।

\twolineshloka
{विनश्यति यथा मोहस्तत्कुरु त्वं सति प्रिये}
{गत्वा तत्र स्थितस्तावद्वटे भव परीक्षिका}% । ४४ ।।

\uvacha{ब्रह्मोवाच}

\twolineshloka
{शिवाज्ञया सती तत्र गत्वाचिन्तयदीश्वरी}
{कुर्यां परीक्षां च कथं रामस्य वनचारिणः} %।।४५।।

\twolineshloka
{सीतारूपमहं धृत्वा गच्छेयं रामसन्निधौ}
{यदि रामो हरिस्सर्वं विज्ञास्यति न चान्यथा} %।।४६।।

\twolineshloka
{इत्थं विचार्य सीता सा भूत्वा रामसमीपतः}
{आगमत्तत्परीक्षार्थं सती मोहपरायणा} %।। ४७ ।।

\twolineshloka
{सीतारूपां सतीं दृष्ट्वा जपन्नाम शिवेति च}
{विहस्य तत्प्रविज्ञाय नत्वावोचद्रघूद्वहः} %।।४८।।

\uvacha{राम उवाच}

\twolineshloka
{प्रेमतस्त्वं सति ब्रूहि क्व शम्भुस्ते नमोगतः}
{एका हि विपिने कस्मादागता पतिना विना} %।। ४९ ।।

\twolineshloka
{त्यक्त्वा स्वरूपं कस्मात्ते धृतं रूपमिदं सति}
{ब्रूहि तत्कारणं देवि कृपां कृत्वा ममोपरि}%।। 2.2.24.५०।।

\uvacha{ब्रह्मोवाच}

\twolineshloka
{इति रामवचः श्रुत्वा चकितासीत्सती तदा}
{स्मृत्वा शिवोक्तं मत्वा चावितथं लज्जिता भृशम्} %।। ५१ ।।

\twolineshloka
{रामं विज्ञाय विष्णुं तं स्वरूपं संविधाय च}
{स्मृत्वा शिवपदं चित्ते सत्युवाच प्रसन्नधीः} %।। ५२ ।।

\twolineshloka
{शिवो मया गणैश्चैव पर्यटन् वसुधां प्रभुः}
{इहागच्छच्च विपिने स्वतन्त्रः परमेश्वरः} %।। ५३ ।।

\twolineshloka
{अपश्यदत्र स त्वां हि सीतान्वेषणतत्परम्}
{सलक्ष्मणं विरहिणं सीतया श्लिष्टमानसम्} %।। ५४ ।।

\twolineshloka
{नत्वा त्वां स गतो मूले वटस्य स्थित एव हि}
{प्रशंसन् महिमानं ते वैष्णवं परमं मुदा} %।। ५५ ।।

\twolineshloka
{चतुर्भुजं हरिं त्वां नो दृष्ट्वेव मुदितोऽभवत्}
{यथेदं रूपममलं पश्यन्नानन्दमाप्तवान्} %।।५६।।

\twolineshloka
{तच्छ्रुत्वा वचनं शम्भौर्भ्रममानीय चेतसि}
{तदाज्ञया परीक्षां ते कृतवत्य स्मि राघव} %।। ५७ ।।

\twolineshloka
{ज्ञातं मे राम विष्णुस्त्वं दृष्टा ते प्रभुताऽखिला}
{निःसशंया तदापि तच्छृणु त्वं च महामते} %।। ५८ ।।

\twolineshloka
{कथं प्रणम्यस्त्वं तस्य सत्यं ब्रूहि ममाग्रतः}
{कुरु निस्संशयां त्वं मां शमलं प्राप्नुहि द्रुतम्} %।।५९।।

\uvacha{ब्रह्मोवाच}

\twolineshloka
{इत्याकर्ण्य वचस्तस्या रामश्चोत्फुल्ललोचनः}
{अस्मरत्स्वं प्रभुं शम्भुं प्रेमाभूद्धृदि चाधिकम्} %।।2.2.24.६०।।

\twolineshloka
{सत्या विनाज्ञया शम्भुसमीपं नागमन्मुने}
{संवर्ण्य महिमानं च प्रावोचद्राघवस्सतीम्} %।। ६१ ।।

॥इति श्रीशिवमहापुराणे द्वितीयायां रुद्रसंहितायां द्वितीये सतीखण्डे रामपरीक्षावर्णनं नाम चतुर्विंशोऽध्यायः॥२४॥


\sect{सतीवियोगः --- पञ्चविंशोऽध्यायः}

\uvacha{राम उवाच}

\twolineshloka
{एकदा हि पुरा देवि शम्भुः परमसूतिकृत्}
{विश्वकर्माणमाहूय स्वलोके परतः परे} %।।१।।

\twolineshloka
{स्वधेनुशालायां रम्यं कारयामास तेन च}
{भवनं विस्तृतं सम्यक् तत्र सिंहासनं वरम्} %।। २ ।।

\twolineshloka
{तत्रच्छत्रं महादिव्यं सर्वदाद्भुत मुत्तमम्}
{कारयामास विघ्नार्थं शङ्करो विश्वकर्मणा} %।। ३ ।।

\twolineshloka
{शक्रादीनां जुहावाशु समस्तान्देवतागणान्}
{सिद्धगन्धर्वनागानुपदे शांश्च कृत्स्नशः} %।।४।।

\twolineshloka
{देवान् सर्वानागमांश्च विधिं पुत्रैर्मुनीनपि}
{देवीः सर्वा अप्सरोभिर्नानावस्तुसमन्विताः} %।। ५ ।।

\twolineshloka
{देवानां च तथर्षीणां सिद्धानां फणिनामपि}
{आनयन्मङ्गलकराः कन्याः षोडशषोडश} %।।६।।

\twolineshloka
{वीणामृदङ्गप्रमुखवाद्यान्नानाविधान्मुने}
{उत्सवं कारयामास वादयित्वा सुगायनैः} %।।७।।

\twolineshloka
{राजाभिषेकयोग्यानि द्रव्याणि सकलौषधैः}
{प्रत्यक्षतीर्थपाथोभिः पञ्चकुभांश्च पूरितान्} %।।८।।

\twolineshloka
{तथान्यास्संविधा दिव्या आनयत्स्वगणैस्तदा}
{ब्रह्मघोषं महारावं कारयामास शङ्करः} %।।९।।

\twolineshloka
{अथो हरिं समाहूय वैकुण्ठात्प्रीतमानसः}
{तद्भक्त्या पूर्णया देवि मोदतिस्म महेश्वरः} %।। १० ।।

\twolineshloka
{सुमुहूर्ते महादेवस्तत्र सिंहासने वरे}
{उपवेश्य हरिं प्रीत्या भूषयामास सर्वशः} %।।११।।

\twolineshloka
{आबद्धरम्यमुकुटं कृतकौतुकमङ्गलम्}
{अभ्यषिञ्चन्महेशस्तु स्वयं ब्रह्माण्डमण्डपे} %।। १२ ।।

\twolineshloka
{दत्तवान्निखिलैश्वर्यं यन्नैजं नान्यगामि यत्}
{ततस्तुष्टाव तं शम्भुस्स्वतन्त्रो भक्तवत्सलः} %।। १३ ।।

\twolineshloka
{ब्रह्माणं लोककर्तारमवोचद्वचनं त्विदम्}
{व्यापयन्स्वं वराधीनं स्वतन्त्रं भक्तवत्सलः} %।। १४।।

\uvacha{महेश उवाच}

\twolineshloka
{अतः प्रभृति लोकेश मन्निदेशादयं हरिः}
{मम वन्द्य स्वयं विष्णुर्जातस्सर्वश्शृणोति हि} %।। १५ ।।

\twolineshloka
{सर्वैर्देवादिभिस्तात प्रणमत्वममुं हरिम्}
{वर्णयन्तु हरिं वेदा ममैते मामिवाज्ञया}%।। १६ ।

\uvacha{राम उवाच}

\twolineshloka
{इत्युक्त्वाथ स्वयं रुद्रोऽनमद्वै गरुडध्वजम्}
{विष्णुभक्तिप्रसन्नात्मा वरदो भक्तवत्सलः} %।। १७।।

\twolineshloka
{ततो ब्रह्मादिभिर्देवैः सर्वरूपसुरैस्तथा}
{मुनिसिद्धादिभिश्चैवं वन्दितोभूद्धरिस्तदा} %।। १८ ।।

\twolineshloka
{ततो महेशो हरयेशंसद्दिविषदां तदा}
{महावरान् सुप्रसन्नो धृतवान्भक्तवत्सलः} %।। १९ ।।

\uvacha{महेश उवाच}

\twolineshloka
{त्वं कर्ता सर्वलोकानां भर्ता हर्ता मदाज्ञया}
{दाता धर्मार्थकामानां शास्ता दुर्नयकारिणाम्} %।। २० ।।

\twolineshloka
{जगदीशो जगत्पूज्यो महाबलपराक्रमः}
{अजेयस्त्वं रणे क्वापि ममापि हि भविष्यसि} %।। २१ ।।

\twolineshloka
{शक्तित्रयं गृहाण त्वमिच्छादि प्रापितं मया}
{नानालीलाप्रभावत्वं स्वतन्त्रत्वं भवत्रये} %।। २२ ।।

\twolineshloka
{त्वद्द्वेष्टारो हरे नूनं मया शास्याः प्रयत्नतः}
{त्वद्भक्तानां मया विष्णो देयं निर्वाणमुत्तमम्} %।। २३ ।।

\twolineshloka
{मायां चापि गृहाणेमां दुःप्रणोद्यां सुरादिभिः}
{यया सम्मोहितं विश्वमचिद्रूपं भविष्यति} %।। २४ ।।

\twolineshloka
{मम बाहुर्मदीयस्तं दक्षिणोऽसौ विधिर्हरे}
{अस्यापि हि विधेः पाता जनितापि भविष्यसि} %।। २५ ।।

\twolineshloka
{हृदयं मम यो रुद्रस्स एवाहं न संशयः}
{पूज्यस्तव सदा सोपि ब्रह्मादीनामपि ध्रुवम्} %।। २६ ।।

\twolineshloka
{अत्र स्थित्वा जगत्सर्वं पालय त्वं विशेषतः}
{नानावतारभेदैश्च सदा नानोति कर्तृभिः} %।। २७ ।।

\twolineshloka
{मम लोके तवेदं व स्थानं च परमर्द्धिमत्}
{गोलोक इति विख्यातं भविष्यति महोज्ज्वलम्} %।। २८ ।।

\twolineshloka
{भविष्यन्ति हरे ये तेऽवतारा भुवि रक्षकाः}
{मद्भक्तास्तान् ध्रुवं द्रक्ष्ये प्रीतानथ निजाद्वरात} %।। २९ ।।

\uvacha{राम उवाच}

\twolineshloka
{अखण्डैश्वर्यमासाद्य हरेरित्थं हरः स्वयम्}
{कैलासे स्वगणैस्तस्मिन् स्वैरं क्रीडत्युमापतिः} %।। ३० ।।

\twolineshloka
{तदाप्रभृति लक्ष्मीशो गोपवेषोभवत्तथा}
{अयासीत्तत्र सुप्रीत्या गोपगोपोगवां पतिः} %।। ३१ ।।

\twolineshloka
{सोपि विष्णुः प्रसन्नात्मा जुगोप निखिलं जगत्}
{नानावतारस्सन्धर्ता वनकर्ता शिवाज्ञया} %।। ३२ ।।

\twolineshloka
{इदानीं स चतुर्द्धात्रावातरच्छङ्कराज्ञया}
{रामोहं तत्र भरतो लक्ष्मणश्शत्रुहेति च} %।। ३३ ।।

\twolineshloka
{अथ पित्राज्ञया देवि ससीतालक्ष्मणस्सति}
{आगतोहं वने चाद्य दुःखितौ दैवतो ऽभवम्} %।। ३४ ।।

\twolineshloka
{निशाचरेण मे जाया हृता सीतेति केनचित्}
{अन्वेष्यामि प्रियां चात्र विरही बन्धुना वने} %।। ३५ ।।

\twolineshloka
{दर्शनं ते यदि प्राप्तं सर्वथा कुशलं मम}
{भविष्यति न सन्देहो मातस्ते कृपया सति} %।। ३६ ।।

\twolineshloka
{सीताप्राप्तिवरो देवि भविष्यति न संशयः}
{तं हत्वा दुःखदं पापं राक्षसं त्वदनुग्रहात्} %।। ३७ ।।

\twolineshloka
{महद्भाग्यं ममाद्यैव यद्यकार्ष्टां कृपां युवाम्}
{यस्मिन् सकरुणौ स्यातां स धन्यः पुरुषो वरः} %।। ३८ ।।

\twolineshloka
{इत्थमाभाष्य बहुधा सुप्रणम्य सतीं शिवाम्}
{तदाज्ञया वने तस्मिन् विचचार रघूद्वहः} %।। ३९ ।।

\twolineshloka
{अथाकर्ण्य सती वाक्यं रामस्य प्रयतात्मनः}
{हृष्टाभूत्सा प्रशंसन्ती शिवभक्तिरतं हृदि} %।। ४० ।।

\twolineshloka
{स्मृत्वा स्वकर्म मनसाकार्षीच्छोकं सुविस्तरम्}
{प्रत्यागच्छदुदासीना विवर्णा शिवसन्निधौ} %।।४१।।

\twolineshloka
{अचिन्तयत्पथि सा देवी सञ्चलन्ती पुनः पुनः}
{नाङ्गीकृतं शिवोक्तं मे रामं प्रति कुधीः कृता} %।।४२।।

\twolineshloka
{किमुत्तरमहं दास्ये गत्वा शङ्करसन्निधौ}
{इति सञ्चिन्त्य बहुधा पश्चात्तापोऽभवत्तदा} %।।४३।।

\twolineshloka
{गत्वा शम्भुसमीपं च प्रणनाम शिवं हृदा}
{विषण्णवदना शोकव्याकुला विगतप्रभा} %।।४४।।

\twolineshloka
{अथ तां दुःखितां दृष्ट्वा पप्रच्छ कुशलं हरः}
{प्रोवाच वचनं प्रीत्या तत्परीक्षा कृता कथम्} %।। ४५ ।।

\twolineshloka
{श्रुत्वा शिववचो नाहं किमपि प्रणतानना}
{सती शोकविषण्णा सा तस्थौ तत्र समीपतः} %।। ४६ ।।

\twolineshloka
{अथ ध्यात्वा महेशस्तु बुबोध चरितं हृदा}
{दक्षजाया महायोगी नानालीला विशारदः} %।। ४७ ।।

\twolineshloka
{सस्मार स्वपणं पूर्वं यत्कृतं हरिकोपतः}
{तत्प्रार्थितोथ रुद्रोसौ मर्यादा प्रतिपालकः} %।। ४८ ।।

\twolineshloka
{विषादोभूत्प्रभोस्तत्र मनस्येवमुवाच ह}
{धर्मवक्ता धर्मकर्त्ता धर्मावनकरस्सदा} %।। ४९ ।।

\uvacha{शिव उवाच}

\twolineshloka
{कुर्यां चेद्दक्षजायां हि स्नेहं पूर्वं यथा महान्}
{नश्येन्मम पणः शुद्धो लोकलीलानुसारिणः} %।। ५० ।।

\uvacha{ब्रह्मोवाच}

\twolineshloka
{इत्थं विचार्य बहुधा हृदा तामत्यजत्सतीम्}
{पणं न नाशयामास वेदधर्मप्रपालकः} %।। ५१ ।।

\twolineshloka
{ततो विहाय मनसा सतीं तां परमेश्वरः}
{जगाम स्वगिरि भेदं जगावद्धा स हि प्रभुः} %।। ५२ ।।

\twolineshloka
{चलन्तं पथि तं व्योमवाण्युवाच महेश्वरम्}
{सर्वान् संश्रावयन् तत्र दक्षजां च विशेषतः} %।। ५३ ।।

\uvacha{व्योमवाण्युवाच}

\twolineshloka
{धन्यस्त्वं परमेशान त्वत्त्समोद्य तथा पणः}
{न कोप्यन्यस्त्रिलोकेस्मिन् महायोगी महाप्रभुः} %।। ५४ ।।

\uvacha{ब्रह्मोवाच}

\twolineshloka
{श्रुत्वा व्योमवचो देवी शिवं पप्रच्छ विप्रभा}
{कं पणं कृतवान्नाथ ब्रूहि मे परमेश्वर} %।। ५५ ।।

\twolineshloka
{इति पृष्टोपि गिरिशस्सत्या हितकरः प्रभुः}
{नोद्वाहे स्वपणं तस्यै कहर्यग्रेऽकरोत्पुरा} %।।५६।।

\twolineshloka
{तदा सती शिवं ध्यात्वा स्वपतिं प्राणवल्लभम्}
{सर्वं बुबोध हेतुं तं प्रियत्यागमयं मुने} %।।५७।।

\twolineshloka
{ततोऽतीव शुशोचाशु बुध्वा सा त्यागमात्मनः}
{शम्भुना दक्षजा तस्मान्निश्वसन्ती मुहुर्मुहुः} %।।५८।।

\twolineshloka
{शिवस्तस्याः समाज्ञाय गुप्तं चक्रे मनोभवम्}
{सत्ये पणं स्वकीयं हि कथा बह्वीर्वदन्प्रभुः} %।।५९।।

\twolineshloka
{सत्या प्राप स कैलासं कथयन् विविधाः कथा}
{वरे स्थित्वा निजं रूपं दधौ योगी समाधिभृत्} %।।2.2.25.६०।।

\twolineshloka
{तत्र तस्थौ सती धाम्नि महाविषण्णमानसा}
{न बुबोध चरित्रं तत्कश्चिच्च शिवयोर्मुने} %।।६१।।

\twolineshloka
{महान्कालो व्यतीयाय तयोरित्थं महामुने}
{स्वोपात्तदेहयोः प्रभ्वोर्लोकलीलानुसारिणोः} %।। ६२ ।।

\twolineshloka
{ध्यानं तत्याज गिरिशस्ततस्स परमार्तिहृत्}
{तज्ज्ञात्वा जगदम्बा हि सती तत्राजगाम सा} %।। ६३ ।।

\twolineshloka
{ननामाथ शिवं देवी हृदयेन विदूयता}
{आसनं दत्तवाञ्शम्भुः स्वसन्मुख उदारधीः} %।। ६४ ।।

\twolineshloka
{कथयामास सुप्रीत्या कथा बह्वीर्मनोरमाः}
{निश्शोका कृतवान्सद्यो लीलां कृत्वा च तादृशीम्} %।।६५।।

\twolineshloka
{पूर्ववत्सा सुखं लेभे तत्याज स्वपणं न सः}
{नेत्याश्चर्यं शिवे तात मन्तव्यं परमेश्वरे} %।।६६।।

\twolineshloka
{इत्थं शिवाशिवकथां वदन्ति मुनयो मुने}
{किल केचिदविद्वांसो वियोगश्च कथं तयोः} %।।६७।।

\twolineshloka
{शिवाशिवचरित्रं को जानाति परमार्थतः}
{स्वेच्छया क्रीडतस्तो हि चरितं कुरुतस्सदा} %।। ६८ ।।

\twolineshloka
{वागर्थाविव सम्पृक्तौ सदा खलु सतीशिवौ}
{तयोर्वियोगस्सम्भाव्यस्सम्भवेदिच्छया तयोः} %।।६९।।

॥इति श्रीशिवमहापुराणे द्द्वितीयायां रुद्रसंहितायां द्वितीये सतीखण्डे सतीवियोगो नाम पञ्चविंशोऽध्यायः॥२५॥

\closesection
    \chapt{स्कन्द-पुराणम्}
    \input{katha/skanda-puranam/setu-nirmanam}
    \sect{रामनाथलिङ्गप्रतिष्ठाविधिवर्णनम्}

\src{स्कन्दपुराणम्}{खण्डः ३ (ब्रह्मखण्डः)}{सेतुखण्डः}{अध्यायः ४४}
\vakta{}
\shrota{}
\tags{}
\notes{Notably includes a short stotram by the Rishis, praising the glory of the Bhagavan Rāma.}
\textlink{https://sa.wikisource.org/wiki/स्कन्दपुराणम्/खण्डः_३_(ब्रह्मखण्डः)/सेतुखण्डः/अध्यायः_४४}
\translink{https://www.wisdomlib.org/hinduism/book/the-skanda-purana/d/doc423612.html}

\storymeta




\uvacha{ऋषय ऊचुः}

\twolineshloka
{सर्ववेदार्थतत्त्वज्ञ पुराणार्णवपारग}
{व्यासपादाम्बुजद्वन्द्वनमस्कारहृताशुभ}%॥ १ ॥

\twolineshloka
{पुराणार्थोपदेशेन सर्वप्राण्युपका रक}
{त्वया ह्यनुगृहीताः स्म पुराणकथनाद्वयम्}%॥ २ ॥

\twolineshloka
{अधुना सेतुमाहात्म्यकथनात्सुतरां मुने}
{वयं कृतार्थाः सञ्जाता व्यासशिष्य महामते}%॥ ३ ॥

\twolineshloka
{यथा प्रातिष्ठिपल्लिङ्गं रामो दशरथात्मजः}
{तच्छ्रोतुं वयमिच्छामस्त्वमिदानीं वदस्व नः}%॥ ४ ॥

\uvacha{श्रीसूत उवाच}

\twolineshloka
{यदर्थं स्थापितं लिङ्गं गन्धमादनपर्वते}
{रामचन्द्रेण विप्रेन्द्र तदिदानीं ब्रवीमि वः}%॥ ५ ॥

\twolineshloka
{हृतभार्यो वनाद्रामो रावणेन बलीयसा}
{कपिसेनायुतो धीरः ससौमि त्रिर्महाबलः}%॥६॥

\twolineshloka
{महेन्द्रं गिरिमासाद्य व्यलोकयत वारिधिम्}
{तस्मिन्नपारे जलधौ कृत्वा सेतुं रघूद्वहः}%॥७॥

\twolineshloka
{तेन गत्वा पुरीं लङ्कां रावणेनाभिरक्षि ताम्}
{अस्तङ्गते सहस्रांशौ पौर्णमास्यां निशामुखे}%॥८॥

\twolineshloka
{रामः ससैनिको विप्राः सुवेलगिरिमारुहत्}
{ततः सौधस्थितं रात्रौ दृष्ट्वा लङ्केश्वरं बली}%॥ ९ ॥

\twolineshloka
{सूर्यपुत्रोऽस्य मुकुटं पातयास भूतले}
{राक्षसो भग्नमुकुटः प्रविवेश गृहोदरम्}%॥ १० ॥

\twolineshloka
{गृहं प्रविष्टे लङ्केशे रामः सुग्रीवसंयुतः}
{सानुजः सेनया सार्द्धमवरुह्य गिरेस्तटात्}%॥ ११ ॥

\twolineshloka
{सेनां न्यवेशयद्वीरो रामो लङ्कासमीपतः}
{ततो निवेशमानांस्तान्वानरान्रावणानुगाः}%॥ १२ ॥

\twolineshloka
{अभिजग्मुर्महाकायाः सायुधाः सहसैनिकाः}
{पर्वणः पूतनो जृम्भः खरः क्रोधवशो हरिः}%॥ १३ ॥

\twolineshloka
{प्रारुजश्चारुजश्चैव प्रहस्तश्चेतरे तथा}
{ततोऽभिपततां तेषामदृश्यानां दुरात्मनाम्}%॥ १४ ॥

\twolineshloka
{अन्तर्धानवधं तत्र चकार स्म विभीषणः}
{ते दृश्यमाना बलिभिर्हरिभिर्दूरपातिभिः}%॥ १५ ॥

\twolineshloka
{निहताः सर्वतश्चैते न्यपतन्वै गतासवः}
{अमृष्यमाणः सबलो रावणो निर्ययावथ}%॥ १६ ॥

\twolineshloka
{व्यूह्य तान्वानरान्सर्वान्न्यवारयत सायकैः}
{राघवस्त्वथ निर्याय व्यूढानीको दशाननम्}%॥ १७ ॥

\twolineshloka
{प्रत्ययुध्यत वेगेन द्वन्द्वयुद्धमभूत्तदा}
{युयुधे लक्ष्मणेनाथ इन्द्रजिद्रावणात्मजः}%॥ १८ ॥

\twolineshloka
{विरूपाक्षेण सुग्रीवस्तारेयेणापि खर्वटः}
{पौण्ड्रेण च नलस्तत्र पुटेशः पनसेन च}%॥ १९ ॥

\twolineshloka
{अन्येपि कपयो वीरा राक्षसैर्द्वन्द्वमेत्य तु}
{चक्रुर्युद्धं सुतुमुलं भीरूणां भयवर्द्धनम्}%॥ २० ॥

\twolineshloka
{अथ रक्षांसि भिन्नानि वानरैर्भीमविक्रमैः}
{प्रदुद्रुवू रणादाशु लङ्कां रावणपालिताम्}%॥ २१ ॥

\twolineshloka
{भग्नेषु सर्वसैन्येषु रावणप्रेरितेन वै}
{पुत्रेणेन्द्रजिता युद्धे नागास्त्रैरतिदारुणैः}%॥ २२ ॥

\twolineshloka
{विद्धौ दाशरथी विप्रा उभौ तौ रामलक्ष्मणौ}
{मोचितौ वैनतेयेन गरुडेन महात्मना}%॥ २३ ॥

\twolineshloka
{तत्र प्रहस्तस्तरसा समभ्येत्य विभीषणम्}
{गदया ताडयामास विनद्य रणकर्कशः}%॥ २४ ॥

\twolineshloka
{स तयाभिहतो धीमान्गदया भामिवेगया}
{नाकम्पत महाबाहुर्हिमवानिव सुस्थितः}%॥ २५ ॥

\twolineshloka
{ततः प्रगृह्य विपुलामष्टघण्टां विभीषणः}
{अभिमन्त्र्य महाशक्तिं चिक्षे पास्य शिरः प्रति}%॥ २६ ॥

\twolineshloka
{पतन्त्या स तया वेगाद्राक्षसोऽशनिना यथा}
{हृतोत्तमाङ्गो ददृशे वातरुग्ण इव द्रुमः}%॥ २७ ॥

\twolineshloka
{तं दृष्ट्वा निहतं सङ्ख्ये प्रहस्तं क्षणदाचरम्}
{अभिदुद्राव धूम्राक्षो वेगेन महता कपीन्}%॥ २८ ॥

\twolineshloka
{कपिसैन्यं समालोक्य विद्रुतं पवनात्मजः}
{धूम्राक्षमाजघानाशु शरेण रणमूर्धनि}%॥ २९ ॥

\twolineshloka
{धूम्राक्षं निहतं दृष्ट्वा हतशेषा निशाचराः}
{सर्वं राज्ञे यथावृत्तं रावणाय न्यवेदयन्}%॥ ३० ॥

\twolineshloka
{ततः शयानं लङ्केशः कुम्भकर्णमबोधयत्}
{प्रबुद्धं प्रेषयामास युद्धाय स च रावणः}%॥ ३१ ॥

\twolineshloka
{आगतं कुम्भकर्णं तं ब्रह्मास्त्रेण तु लक्ष्मणः}
{जघान समरे क्रुद्धो गतासुर्न्यपतच्च सः}%॥ ३२ ॥

\twolineshloka
{दूषणस्यानुजौ तत्र वत्रवेगप्रमाथिनौ}
{हनुमन्नीलनिहतौ रावणप्रतिमौ रणे}%॥ ३३ ॥

\twolineshloka
{वज्रदंष्ट्रं समवधीद्विश्वकर्मसुतो नलः}
{अकम्पनं च न्यहनत्कुमुदो वानरर्षभः}%॥ ३४ ॥

\twolineshloka
{षष्ठ्यां पराजितो राजा प्राविशच्च पुरीं ततः}
{अतिकायो लक्ष्मणेन हतश्च त्रिशिरास्तथा}%॥ ३५ ॥

\twolineshloka
{सुग्रीवेण हतौ युद्धे देवान्त कनरान्तकौ}
{हनूमता हतौ युद्धे कुम्भकर्णसुतावुभौ}%॥ ३६ ॥

\twolineshloka
{विभीषणेन निहतो मकराक्षः खरात्मजः}
{तत इन्द्रजितं पुत्रं चोदयामास रावणः}%॥ ३७ ॥

\twolineshloka
{इन्द्रजिन्मोहयित्वा तौ भ्रातरौ रामलक्षमणौ}
{घोरैः शरैरङ्गदेन हतवाहो दिवि स्थितः}%॥ ३८ ॥

\twolineshloka
{कुमुदाङ्गदसुग्रीवनलजाम्बवदादिभिः}
{सहिता वानराः सर्वे न्यपतंस्तेन घातिताः}%॥ ३९ ॥

\twolineshloka
{एवं निहत्य समरे ससैन्यौ रामलक्ष्मणौ}
{अन्तर्दधे तदा व्योम्नि मेघनादो महाबलः}%॥ ४० ॥

\twolineshloka
{ततो विभीषणो राममिक्ष्वाकुकुलभूषणम्}
{उवाच प्राञ्जलिर्वाक्यं प्रणम्य च पुनःपुनः}%॥ ४१ ॥

\twolineshloka
{अयमम्भो गृहीत्वा तु राजराजस्य शासनात्}
{गुह्यकोऽभ्यागतो राम त्वत्सकाशमरिन्दम}%॥ ४२ ॥

\twolineshloka
{इदमम्भः कुबेरस्ते महाराज प्रयच्छति}
{अन्तर्हितानां भूतानां दर्शनार्थं परं तप}%॥ ४३ ॥

\twolineshloka
{अनेन स्पृष्टनयनो भूतान्यन्तर्हितान्यपि}
{भवान्द्रक्ष्यति यस्मै वा भवानेतत्प्रदास्यति}%॥ ४४ ॥

\twolineshloka
{सोऽपि द्रक्ष्यति भूतानि वियत्त्यन्तर्हितानि वै}
{तथेति रामस्तद्वारि प्रतिगृह्याथ सत्कृतम्}%॥ ४५ ॥

\twolineshloka
{चकार नेत्रयोः शौचं लक्ष्मणश्च महाबलः}
{सुग्रीवजाम्बवन्तौ च हनुमानङ्गदस्तथा}%॥ ४६ ॥

\twolineshloka
{मैन्दद्विविदनीलाश्च ये चान्ये वानरास्तथा}
{ते सर्वे रामदत्तेन वारिणा शुद्धचक्षुषः}%॥ ४७ ॥

\twolineshloka
{आकाशेन्तर्हितं वीरमपश्यन्रावणा त्मजम्}
{ततस्तमभिदुद्राव सौमित्रिर्दृष्टिगोचरम्}%॥ ४८ ॥

\twolineshloka
{ततो जघान सङ्कुद्धो लक्ष्मणः कृतलक्षणः}
{कुवेरप्रेषितजलैः पवित्रीकृतलोचनः}%॥ ४९ ॥

\twolineshloka
{ततः समभवद्युद्धं लक्ष्मणेन्द्रजितोर्महत्}
{अतीव चित्रमाश्चर्यं शक्रप्रह्लादयोरिव}%॥ ५० ॥

\twolineshloka
{ततस्तृतीयदिवसे यत्नेन महता द्विजाः}
{इन्द्रजिन्निहतो युद्धे लक्ष्मणेन बलीयसा}%॥ ५१ ॥

\twolineshloka
{ततो मूलबलं सर्वं हतं रामेण धीमता}
{अथ क्रुद्धो दशग्रीवः प्रियपुत्रे निपातिते}%॥ ५२ ॥

\twolineshloka
{निर्ययौ रथमास्थाय नगराद्बहुसैनिकः}
{रावणो जानकीं हन्तुमुद्युक्तो विन्ध्यवारितः}%॥ ५३ ॥

\twolineshloka
{ततो हर्यश्वयुक्तेन रथेनादित्यवर्चसा}
{उपतस्थे रणे रामं मातलिः शक्रसारथिः}%॥ ५४ ॥

\twolineshloka
{ऐन्द्रं रथं समारुह्य रामो धर्मभृतां वरः}
{शिरांसि राक्षसेन्द्रस्य ब्रह्मास्त्रेणावधीद्रणे}%॥ ५५ ॥

\twolineshloka
{ततो हतदशग्रीवं रामं दशरथात्मजम्}
{आशीर्भिर्जययुक्ताभिर्देवाः सर्षिपुरोगमाः}%॥ ५६ ॥

\twolineshloka
{तुष्टुवुः परिसन्तुष्टाः सिद्धविद्याधरास्तथा}
{रामं कमलपत्राक्षं पुष्प वर्षेरवाकिरन्}%॥ ५७ ॥

\twolineshloka
{रामस्तैः सुरसङ्घातैः सहितः सैनिकैर्वृतः}
{सीतासौमित्रिसहितः समारुह्य च पुष्पकम्}%॥ ५८ ॥

\twolineshloka
{तथाभिषिच्य राजानं लङ्कायां च विभीषणम्}
{कपिसेनावृतो रामो गन्धमादनमन्वगात्}%॥ ५९ ॥

\twolineshloka
{परिशोध्य च वैदेहीं गन्धमादनपर्वते}
{रामं कमलपत्राक्षं स्थितवानर संवृतम्}%॥ ६० ॥

\twolineshloka
{हतलङ्केश्वरं वीरं सानुजं सविभीषणम्}
{सभार्यं देववृन्दैश्च सेवितं मुनिपुङ्गवैः}%॥ ६१ ॥

\twolineshloka
{मुनयोऽभ्यागता द्रष्टुं दण्डकारण्य वासिनः}
{अगस्त्यं ते पुरस्कृत्य तुष्टुवुर्मैथिलीपतिम्}%॥ ६२ ॥

\uvacha{मुनय ऊचुः}

\twolineshloka
{नमस्ते रामचन्द्राय लोकानुग्रहकारिणे}
{अरावणं जगत्कर्तुमवतीर्णाय भूतले}%॥ ६३ ॥

\twolineshloka
{ताटिकादेहसंहर्त्रे गाधिजाध्वररक्षिणे}
{नमस्ते जितमारीच सुबाहुप्राणहारिणे}%॥ ६४ ॥

\twolineshloka
{अहल्यामुक्तिसन्दायिपादपङ्कजरेणवे}
{नमस्ते हरकोदण्डलीलाभञ्जनकारिणे}%॥ ६५ ॥

\twolineshloka
{नमस्ते मैथिलीपाणिग्रहणोत्सवशालिने}
{नमस्ते रेणुकापुत्रपराजयविधायिने}%॥ ६६ ॥

\twolineshloka
{सहलक्ष्मणसीताभ्यां कैकेय्यास्तु वरद्वयात्}
{सत्यं पितृवचः कर्तुं नमो वनमुपेयुषे}%॥ ६७ ॥

\twolineshloka
{भरतप्रार्थनादत्तपादुकायुगुलाय ते}
{नमस्ते शरभङ्गस्य स्वर्गप्राप्त्यैकहेतवे}%॥ ६८ ॥

\twolineshloka
{नमो विराधसंहर्त्रे गृधराजसखाय ते}
{मायामृगमहाक्रूरमारीचाङ्गविदारिणे}%॥ ६९ ॥

\twolineshloka
{सीतापहारिलोकेशयुद्धत्यक्तकलेवरम्}
{जटायुषं तु सन्दह्य तत्कैवल्यप्रदायिने}%॥ ७० ॥

\twolineshloka
{नमः कबन्धसंहर्त्रे शबरीपूजिताङ्घ्रये}
{प्राप्तसुग्रीवसख्याय कृतवालिवधाय ते}%॥ ७१ ॥

\twolineshloka
{नमः कृतवते सेतुं समुद्रे वरुणालये}
{सर्वराक्षससंहर्त्रे रावणप्राणहारिणे}%॥ ७२ ॥

\twolineshloka
{संसाराम्बुधिसन्तारपोतपादाम्बुजाय ते}
{नमो भक्तार्तिसंहर्त्रे सच्चिदानन्दरूपिणे}%॥ ७३ ॥

\twolineshloka
{नमस्ते रामभद्राय जगतामृद्धिहेतवे}
{रामादिपुण्यनामानि जपतां पापहारिणे}%॥ ७४ ॥

\twolineshloka
{नमस्ते सर्वलोकानां सृष्टिस्थित्यन्तकारिणे}
{नमस्ते करुणामूर्ते भक्तरक्षणदीक्षित}%॥ ७८५ ॥

\twolineshloka
{ससीताय नमस्तुभ्यं विभीषणसुखप्रद}
{लङ्केश्वरवधाद्राम पालितं हि जगत्त्वया}%॥ ७६ ॥

\twolineshloka
{रक्ष रक्ष जगन्नाथ पाह्यस्माञ्जानकीपते}
{स्तुत्वैवं मुनयः सर्वे तूष्णीं तस्थुर्द्विजोत्तमाः}%॥ ७७ ॥

\uvacha{श्रीसूत उवाच}

\twolineshloka
{य इदं रामचन्द्रस्य स्तोत्रं मुनिभिरीरितम्}
{त्रिसन्ध्यं पठते भक्त्या भुक्तिं मुक्तिं च विन्दति}%॥ ७८ ॥

\twolineshloka
{प्रयाणकाले पठतो न् भीतिरुपजायते}
{एतत्स्तोत्रस्य पठनाद्भूतवेतालकादयः}%॥ ७९ ॥

\twolineshloka
{नश्यन्ति रोगा नश्यन्ति नश्यते पापसञ्चयः}
{पुत्रकामो लभेत्पुत्रं कन्या विन्दति सत्पतिम्}%॥ ८० ॥

\twolineshloka
{मोक्षकामो लभेन्मोक्षं धनकामो धनं लभेत्}
{सर्वान्कामानवाप्नोति पठन्भक्त्या त्विमं स्तवम्}%॥ ८१ ॥

\twolineshloka
{ततो रामो मुनीन्प्राह प्रणम्य च कृताञ्जलिः}
{अहं विशुद्धये प्राप्यः सकलैरपि मानवैः}%॥ ८२ ॥

\twolineshloka
{मद्दृष्टिगोचरो जन्तुर्नित्यमोक्षस्य भाजनम्}
{तथाऽपि मुनयो नित्यं भक्तियुक्तेन चेतसा}%॥ ८३ ॥

\twolineshloka
{स्वात्मलाभेन सन्तुष्टान्साधून्भूतसुहृत्तमान्}
{निरहङ्कारिणः शान्तान्नमस्याम्यूर्ध्वरेतसः}%॥ ८४ ॥

\twolineshloka
{यस्माद्ब्रह्मण्यदेवोऽहमतो विप्रान्भजे सदा}
{युष्मान्पृच्छाम्यहं किञ्चित्तद्वदध्वं विचार्य तु}%॥ ८५ ॥

\threelineshloka
{रावणस्य वधाद्विप्रा यत्पापं मम वर्तते}
{तस्य मे निष्कृतिं ब्रूत पौलस्त्यवधजस्य हि}
{यत्कृत्वा तेन पापे न मुच्येऽहं मुनिपुङ्गवाः}%॥ ८६ ॥


\uvacha{मुनय ऊचुः}

\onelineshloka
{सत्यव्रत जगन्नाथ जगद्रक्षाधुरन्धर}%॥ ८७ ॥

\twolineshloka
{सर्वलोकोपकारार्थं कुरु राम शिवार्चनम्}
{गन्धमादनशृङ्गेऽस्मिन्महापुण्ये विमुक्तिदे}%॥ ८८ ॥

\twolineshloka
{शिवलिङ्गप्रतिष्ठां त्वं लोकसङ्ग्रहकाम्यया}
{कुरु राम दशग्रीववधदोषापनुत्तये}%॥ ८९ ॥


\twolineshloka
{लिङ्गस्थापनजं पुण्यं चतुर्वक्त्रोऽपि भाषितुम्॥}
{न शक्नोति ततो वक्तुं किं पुनर्मनुजेश्वर}%॥९॥

\twolineshloka
{यत्त्वया स्थाप्यते लिगं गन्धमादनपर्वते}
{अस्य सन्दर्शनं पुंसां काशीलिङ्गावलोकनात्}%॥ ९१ ॥

\twolineshloka
{अधिकं कोटिगुणितं फलवत्स्यान्न संशयः}
{तव नाम्ना त्विदं लिङ्गं लोके ख्यातिं समश्नुताम्}%॥ ९२ ॥

\twolineshloka
{नाशकं पुण्यपापाख्यकाष्ठानां दहनोपमम्}
{इदं रामेश्वरं लिङ्गं ख्यातं लोके भविष्यति}%॥ ९३ ॥

\twolineshloka
{मा विलम्बं कुरुष्वातो लिङ्गस्थापनकर्मणि}
{रामचन्द्र महाभाग करुणापूर्णविग्रह}%॥ ९४ ॥

\uvacha{श्रीसूत उवाच}

\twolineshloka
{इति श्रुत्वा वचो रामो मुनीनां तं मुनीश्वराः}
{पुण्यकालं विचार्याथ द्विमुहूर्तं जगत्पतिः}%॥ ९५ ॥

\twolineshloka
{कैलासं प्रेषयामास हनुमन्तं शिवालयम्}
{शिवलिङ्गं समानेतुं स्थापनार्थं रघूद्वहः}%॥ ९६ ॥

\uvacha{राम उवाच}

\twolineshloka
{हनूमन्नञ्जनीसूनो वायुपुत्र महाबल}
{कैलासं त्वरितो गत्वा लिङ्गमानय मा चिरम्}%॥ ९७ ॥

\twolineshloka
{इत्याज्ञप्तस्स रामेण भुजावास्फाल्य वीर्यवान्}
{मुहूर्तद्वितयं ज्ञात्वा पुण्यकालं कपीश्वरः}%॥ ९८ ॥

\twolineshloka
{पश्यतां सर्वदेवानामृषीणां च महात्मनाम्}
{उत्पपात महावेगश्चालयन्गन्धमादनम्}%॥ ९९ ॥

\twolineshloka
{लङ्घयन्स वियन्मार्गं कैलासं पर्वतं ययौ}
{न ददर्श महादेवं लिङ्गरूपधरं कपिः}%॥ १०० ॥

\twolineshloka
{कैलासे पर्वते तस्मिन्पुण्ये शङ्करपालिते}
{आञ्जनेयस्तपस्तेपे लिङ्गप्राप्त्यर्थमादरात्}%॥ १ ॥

\twolineshloka
{प्रागग्रेषु समासीनः कुशेषु मुनिपुङ्गवाः}
{ऊर्ध्वबाहुर्निरालम्बो निरुच्छ्वासो जितेन्द्रियः}%॥ २ ॥

\twolineshloka
{प्रसादयन्महादेवं लिङ्गं लेभे स मारुतिः}
{एतस्मिन्नन्तरे विप्रा मुनिभिस्तत्त्वदर्शिभिः}%॥ ३ ॥

\twolineshloka
{अनागतं हनूमन्तं कालं स्वल्पावशेषितम्}
{ज्ञात्वा प्रकथितं तत्र रामं प्रति महामतिम्}%॥ ४ ॥

\twolineshloka
{रामराम महाबाहो कालो ह्यत्येति साम्प्रतम्}
{जानक्या यत्कृतं लिङ्गं सैकतं लीलया विभो}%॥ ५ ॥

\twolineshloka
{तल्लिङ्गं स्थापयस्वाद्य महालिङ्गमनुत्तमम्}
{श्रुत्वैतद्वचनं रामो जानक्या सह सत्वरम्}%॥ ६ ॥

\twolineshloka
{मुनिभिः सहितः प्रीत्या कृतकौतुकमङ्गलः}
{ज्येष्ठे मासे सिते पक्षे दशम्यां बुधहस्तयोः}%॥ ७ ॥

\twolineshloka
{गरानन्दे व्यतीपाते कन्या चन्द्रे वृषे रवौ}
{दशयोगे महापुण्ये गन्धमादनपर्वते}%॥ ८ ॥

\twolineshloka
{सेतुमध्ये महादेवं लिङ्गरूपधरं हरम्}
{ईशानं कृत्तिवसनं गङ्गाचन्द्रकलाधरम्}%॥ ९ ॥

\twolineshloka
{रामो वै स्थापयामास शिवलिङ्गमनुत्तमम्}
{लिङ्गस्थं पूजयामास राघवः साम्बमीश्वरम्}%॥ ११० ॥

\twolineshloka
{लिङ्गस्थः स महादेवः पार्वत्या सह शङ्करः}
{प्रत्यक्षमेव भगवान्दत्तवान्वरमुत्तमम्}%॥ ११ ॥

\twolineshloka
{सर्वलोकशरण्याय राघवाय महात्मने}
{त्वयात्र स्थापितं लिङ्गं ये पश्यन्ति रघूद्वह}%॥ १२ ॥

\twolineshloka
{महापातकयुक्ताश्च तेषां पापं प्रणश्यति}
{सर्वाण्यपि हि पापानि धनुष्कोटौ निमज्जनात्}%॥ १३ ॥

\twolineshloka
{दर्शनाद्रामलिङ्गस्य पातकानि महान्त्यपि}
{विलयं यान्ति राजेन्द्र रामचन्द्र न संशयः}%॥ १४ ॥

\twolineshloka
{प्रादादेवं हि रामाय वरं देवोंऽबिकापतिः}
{तदग्रे नन्दिकेशं च स्थापयामास राघवः}%॥ १५ ॥

\twolineshloka
{ईश्वरस्याभिषेकार्थं धनुष्कोट्याथ राघवः}
{एकं कूपं धरां भित्त्वा जनयामास वै द्विजाः}%॥ १६ ॥

\twolineshloka
{तस्माज्जलमुपादाय स्नापयामास शङ्करम्}
{कोटितीर्थमिति प्रोक्तं तत्तीर्थं पुण्यमुत्तमम्}%॥ १७ ॥

\threelineshloka
{उक्तं तद्वैभवं पूर्वमस्माभिर्मुनिपुङ्गवाः}
{देवाश्च मुनयो नागा गन्धर्वाप्स रसां गणाः}
{सर्वेपि वानरा लिङ्गमेकैकं चक्रुरादरात्}%॥ १८ ॥


\uvacha{श्रीसूत उवाच}

\onelineshloka
{एवं वः कथितं विप्रा यथा रामेण धीमत}%॥ ५९ ॥

\twolineshloka
{स्थापितं शिवलिङ्गं वै भुक्तिमुक्तिप्रदायकम्}
{इमां लिङ्गप्रतिष्ठां यः शृणोति पठतेऽथवा}%॥ १२० ॥

\twolineshloka
{स रामेश्वरलिङ्गस्य सेवाफलमवाप्नुयात्}
{सायुज्यं च समाप्नोति रामनाथस्य वैभवात्}%॥ १२१ ॥

॥इति श्रीस्कान्दे महापुराण एकाशीतिसाहस्र्यां संहितायां तृतीये ब्रह्मखण्डे सेतुमाहात्म्ये रामनाथलिङ्गप्रतिष्ठाविधिवर्णनं नाम चतुश्चत्वारिंशोऽध्यायः॥४४॥

\sect{रामचन्द्रतत्त्वज्ञानोपदेशवर्णनम्}

\src{स्कन्दपुराणम्}{खण्डः ३ (ब्रह्मखण्डः)}{सेतुखण्डः}{अध्यायः ४५}
\vakta{}
\shrota{}
\tags{}
\notes{}
\textlink{https://sa.wikisource.org/wiki/स्कन्दपुराणम्/खण्डः_३_(ब्रह्मखण्डः)/सेतुखण्डः/अध्यायः_४५}
\translink{https://www.wisdomlib.org/hinduism/book/the-skanda-purana/d/doc423613.html}

\storymeta




\uvacha{श्रीसूत उवाच}

\twolineshloka
{एवं प्रतिष्ठिते लिङ्गे रामेणाक्लिष्टकारिणा}
{लिङ्गं वरं समादाय मारुतिः सहसाऽऽययौ}%॥ १ ॥

\twolineshloka
{रामं दाशरथिं वीरमभिवाद्य स मारुतिः}
{वैदेहीलक्ष्मणौ पश्चात्सुग्रीवं प्रणनाम च}%॥ २ ॥

\twolineshloka
{सीता सैकतलिङ्गं तत्पूजयन्तं रघूद्वहम्}
{दृष्ट्वाथ मुनिभिः सार्द्धं चुकोप पवनात्मजः}%॥ ३ ॥

\twolineshloka
{अत्यन्तं खेदखिन्नः सन्वृथाकृतपरिश्रमः}
{उवाच रामं धर्मज्ञं हनूमानञ्जनात्मजः}%॥ ४ ॥

\uvacha{हनूमानुवाच}

\twolineshloka
{दुर्जातोऽहं वृथा राम लोके क्लेशाय केवलम्}
{खिन्नोऽस्मि बहुशो देव राक्षसैः क्रूरकर्मभिः}%॥ ५ ॥

\twolineshloka
{मा स्म सीमन्तिनी काचिज्जनयेन्मादृशं सुतम्}
{यतोऽनुभूयते दुःखमनन्तं भवसागरे}%॥ ६ ॥

\twolineshloka
{खिन्नोऽस्मि सेवया पूर्वं युद्धेनापि ततोधिकम्}
{अनन्तं दुःखमधुना यतो मामवमन्यसे}%॥ ७ ॥

\twolineshloka
{सुग्रीवेण च भार्यार्थं राज्यार्थं राक्षसेन च}
{रावणावरजेन त्वं सेवितो ऽसि रघूद्वह}%॥ ८ ॥

\twolineshloka
{मया निर्हेतुकं राम सेवितोऽसि महामते}
{वानराणामनेकेषु त्वयाज्ञप्तोऽहमद्य वै}%॥ ९ ॥

\twolineshloka
{शिवलिङ्गं समानेतुं कैलासात्पर्वतो त्तमात्}
{कैलासं त्वरितो गत्वा न चापश्यं पिनाकिनम्}%॥ १० ॥

\twolineshloka
{तपसा प्रीणयित्वा तं साम्बं वृषभवाहनम्}
{प्राप्तलिङ्गो रघुपते त्वरितः समु पागतः}%॥ ११ ॥

\twolineshloka
{अन्यलिङ्गं त्वमधुना प्रतिष्ठाप्य तु सैकतम्}
{मुनिभिर्देवगन्धर्वैः साकं पूजयसे विभो}%॥ १२ ॥

\twolineshloka
{मयानीतमिदं लिङ्गं कैलासा त्पर्वताद्वृथा}
{अहो भाराय मे देहो मन्दभाग्यस्यजायते}%॥ १३ ॥

\twolineshloka
{भूतलस्य महाराज जानकीरमण प्रभो}
{इदं दुःखमहं सोढुं न शक्नोमि रघूद्वह}%॥ १४ ॥

\twolineshloka
{किं करिष्यामि कुत्राहं गमिष्यामि न मे गतिः}
{अतः शरीरं त्यक्ष्यामि त्वयाहमवमानितः}%॥ १५ ॥

\uvacha{श्रीसूत उवाच}

\twolineshloka
{एवं स बहुशो विप्राः क्रुशित्वा पवनात्मजः}
{दण्डवत्प्रणतो भूमौ क्रोधशोकाकुलोऽभवत्}%॥ १६ ॥

\threelineshloka
{तं दृष्ट्वा रघुनाथोऽपि प्रहसन्निदमब्रवीत्}
{पश्यतां सवदेवानां मुनीनां कपिरक्षसाम्}
{सान्त्वयन्मारुतिं तत्र दुःखं चास्य प्रमार्जयन्}%॥ १७ ॥

\uvacha{श्रीराम उवाच}

\onelineshloka
{सर्वं जानाम्यहं कार्यमात्मनोऽपि परस्य च}%॥ १८ ॥

\twolineshloka
{जातस्य जायमानस्य मृतस्यापि सदा कपे}
{जायते म्रियते जन्तुरेक एव स्वकर्मणा} % १९ ॥

\twolineshloka
{प्रयाति नरकं चापि परमात्मा तु निर्गुणः}
{एवं तत्त्वं विनिश्चित्य शोकं मा कुरु वानर}%॥ २० ॥

\twolineshloka
{लिङ्गत्रयविनिर्मुक्तं ज्योतिरेकं निरञ्जनम्}
{निराश्रयं निर्विकारमात्मानं पश्य नित्यशः}%॥ २१ ॥

\twolineshloka
{किमर्थं कुरुषे शोकं तत्त्वज्ञानस्य बाधकम्}
{तत्त्वज्ञाने सदा निष्ठां कुरु वानरसत्तम}%॥ २२ ॥

\twolineshloka
{स्वयम्प्रकाशमात्मानं ध्यायस्व सततं कपे}
{देहादौ ममतां मुञ्च तत्त्वज्ञानविरोधिनीम्}%॥ २३ ॥

\twolineshloka
{धर्मं भजस्व सततं प्राणिहिंसां परित्यज}
{सेवस्व साधुपुरुषाञ्जहि सर्वेन्द्रियाणि च}%॥ २४ ॥

\twolineshloka
{परित्यजस्व सततमन्येषां दोषकीर्तनम्}
{शिवविष्ण्वादिदेवानामर्चां कुरु सदा कपे}%॥ २५ ॥

\twolineshloka
{सत्यं वदस्व सततं परित्यज शुचं कपे}
{प्रत्यग्ब्रह्मैकताज्ञानं मोहवस्तुसमुद्गतम्}%॥ २६ ॥

\twolineshloka
{शोभनाशोभना भ्रान्तिः कल्पि तास्मिन्यथार्थवत्}
{अध्यास्ते शोभनत्वेन पदार्थे मोहवैभवात्}%॥ २७ ॥

\twolineshloka
{रोगो विजायते नृणां भ्रान्तानां कपिसत्तम}
{रागद्वेषबलाद्बद्धा धर्मा धर्मवशङ्गताः}%॥ २८ ॥

\twolineshloka
{देवतिर्यङ्मनुष्याद्या निरयं यान्ति मानवाः}
{चन्दनागरुकर्पूरप्रमुखा अतिशोभनाः}%॥ २९ ॥

\twolineshloka
{मलं भवन्ति यत्स्पर्शात्तच्छरीरं कथं सुखम्}
{भक्ष्यभोज्यादयः सर्वे पदार्था अतिशोभनाः}%॥ ३० ॥

\twolineshloka
{विष्ठा भवन्ति यत्सङ्गात्तच्छरीरं कथं सुखम्}
{सुगन्धि शीतलं तोयं मूत्रं यत्सङ्गमाद्भवेत्}%॥ ३१ ॥

\twolineshloka
{तत्कथं शोभनं पिण्डं भवेद्ब्रूहि कपेऽधुना}
{अतीव धवलाः शुद्धाः पटा यत्सङ्गमेनहि}%॥ ३२ ॥

\twolineshloka
{भवन्ति मलिनाः स्वेदात्तत्कथं शोभनं भवेत}
{श्रूयतां परमार्थो मे हनूमन्वायुनन्दन}%॥ ३३ ॥

\twolineshloka
{अस्मिन्संसारगर्ते तु किञ्चित्सौख्यं न विद्यते}
{प्रथमं जन्तुराप्नोति जन्म बाल्यं ततः परम्}%॥ ३४ ॥

\twolineshloka
{पश्चाद्यौवनमाप्नोति ततो वार्धक्यमश्नुते}
{पश्चान्मृत्युमवाप्नोति पुनर्जन्म तदश्नुते}%॥ ३५ ॥

\twolineshloka
{अज्ञानवैभवादेव दुःखमाप्नोति मानवः}
{तदज्ञान निवृत्तौ तु प्राप्नोति सुखमुत्तमम्}%॥ ३६ ॥

\twolineshloka
{अज्ञानस्य निवृत्तिस्तु ज्ञानादेव न कर्मणा}
{ज्ञानं नाम परं ब्रह्म ज्ञानं वेदान्तवाक्यजम्}%॥ ३७ ॥

\twolineshloka
{तज्ज्ञानं च विरक्तस्य जायते नेतरस्य हि}
{मुख्याधिकारिणः सत्यमाचार्यस्य प्रसादतः}%॥ ३९ ॥

\twolineshloka
{यदा सर्वे प्रमुच्यन्ते कामा येऽस्य हृदि स्थिताः}
{तदा मर्त्योऽमृतोऽत्रैव परं ब्रह्म समश्नुते}%॥ ३९ ॥

\twolineshloka
{जाग्रतं च स्वपन्तं च भुञ्जन्तं च स्थितं तथा}
{इमं जनं सदा क्रूरः कृतान्तः परिकर्षति}%॥ ४० ॥

\twolineshloka
{सर्वे क्षयान्ता निचयाः पतनान्ताः समुच्छ्रयाः}
{संयोगा विप्रयोगान्ता मरणान्तं च जीवितम्}%॥ ४१ ॥

\twolineshloka
{यथा फलानां पक्वानां नान्यत्र पतनाद्भयम्}
{यथा नराणां जातानां नान्यत्र पतनाद्भयम्}%॥ ४२ ॥

\twolineshloka
{यथा गृहं दृढस्तम्भं जीर्णं काले विनश्यति}
{एवं विनश्यन्ति नरा जरामृत्युवशङ्गताः}%॥ ४३ ॥

\twolineshloka
{अहोरात्रस्य गमनान्नृणामायुर्विनश्यति}
{आत्मानमनुशोच त्वं किमन्यमनुशोचसि}%॥ ४४ ॥

\twolineshloka
{नश्यत्यायुः स्थितस्यापि धावतोऽपि कपीश्वर}
{सहैव मृत्युर्व्रजति सह मृत्युर्निषीदति}%॥ ४५ ॥

\twolineshloka
{चरित्वा दूरदेशं च सह मृत्युर्निवर्तते}
{शरीरे वलयः प्राप्ताः श्वेता जाताः शिरोरुहाः}%॥ ४६ ॥

\twolineshloka
{जीर्यते जरया देहः श्वासकासादिना तथा}
{यथा काष्ठं च काष्ठं च समेयातां महोदधौ}%॥ ४७ ॥

\twolineshloka
{समेत्य च व्यपेयातां कालयोगेन वानर}
{एवं भार्या च पुत्रश्च वधुक्षेत्रधनानि च}%॥ ४८ ॥

\twolineshloka
{क्वचित्सम्भूय गच्छन्ति पुनरन्यत्र वानर}
{यथा हि पान्थं गच्छन्तं पथि कश्चित्पथि स्थितः}%॥ ४९ ॥

\twolineshloka
{अहमप्या गमिष्यामि भवद्भिः साकमित्यथ}
{कञ्चित्कालं समेतौ तौ पुनरन्यत्र गच्छतः}%॥ ५० ॥

\twolineshloka
{एवं भार्यासुतादीनां सङ्गमो नश्वरः कपे}
{शरीरजन्मना साकं मृत्युः सञ्जायते ध्रुवम्}%॥ ५१ ॥

\twolineshloka
{अवश्यम्भाविमरणे न हि जातु प्रतिक्रिया}
{एतच्छरीरपाते तु देही कर्मगतिं गतः}%॥ ५२ ॥

\twolineshloka
{प्राप्य पिण्डान्तरं वत्स पूर्वपिण्डं त्यजत्यसौ}
{प्राणिनां न सदैकत्र वासो भवति वानर}%॥ ५३ ॥

\twolineshloka
{स्वस्वकर्मवशात्सर्वे वियुज्यन्ते पृथक्पृथक्}
{यथा प्राणिशरीराणि नश्यन्ति च भवन्ति च}%॥ ५४ ॥

\twolineshloka
{आत्मनो जन्ममरणे नैव स्तः कपिसत्तम}
{अतस्त्वमञ्जनासूनो विशोकं ज्ञानमद्वयम्}%॥ ५५ ॥

\twolineshloka
{सद्रूपममलं ब्रह्म चिन्तयस्व दिवानिशम्}
{त्वत्कृतं मत्कृतं कर्म मत्कृतं त्वाकृतं तथा}%॥ ५६ ॥

\twolineshloka
{मल्लिङ्गस्थापनं तस्मात्त्वल्लिङ्ग स्थापनं कपे}
{मुहूर्तातिक्रमाल्लिङ्गं सैकतं सीतया कृतम्}%॥ ५७ ॥

\twolineshloka
{मयात्र स्थापितं तस्मात्कोपं दुःखं च मा कुरु}
{कैलासादागतं लिङ्गं स्थापयास्मिच्छुभे दिने}%॥ ५८ ॥

\twolineshloka
{तव नाम्ना त्विदं लिङ्गं यातु लोकत्रये प्रथाम्}
{हनूमदीश्वरं दृष्ट्वा द्रष्टव्यो राघवेश्वरः}%॥ ५९ ॥

\twolineshloka
{ब्रह्मराक्षसयूथानि हतानि भवता कपे}
{अतः स्वनाम्ना लिङ्गस्य स्थापनात्त्वं प्रमोक्ष्यसे}%॥ ६० ॥

\twolineshloka
{स्वयं हरेण दत्तं तु हनूमन्नामकं शिवम्}
{सम्पश्यन्रामनाथं च कृतकृत्यो भवेन्नरः}%॥ ६१ ॥

\twolineshloka
{योजनानां सहस्रेऽपि स्मृत्वा लिङ्गं हनूमतः}
{रामनाथेश्वरं चापि स्मृत्वा सायुज्यमाप्नुयात्}%॥ ६२ ॥

\twolineshloka
{तेनेष्टं सर्वयज्ञैश्च तपश्चाकारि कृत्स्नशः}
{येन दृष्टौ महादेवौ हनूमद्राघवेश्वरौ}%॥ ६३ ॥

\twolineshloka
{हनूमता कृतं लिङ्गं यच्च लिङ्गं मया कृतम्}
{जानकीयं च यल्लिङ्गं यल्लिङ्गं लक्ष्मणेश्वरम्}%॥ ६४ ॥

\twolineshloka
{सुग्रीवेण कृतं यच्च सेतुकर्त्रा नलेन च}
{अङ्गदेन च नीलेन तथा जाम्बवता कृतम्}%॥ ६५ ॥

\twolineshloka
{विभीषणेन यच्चापि रत्नलिङ्गं प्रतिष्ठितम्}
{इन्द्राद्यैश्च कृतं लिङ्गं यच्छेषाद्यैः प्रतिष्ठितम्}%॥ ६६ ॥

\twolineshloka
{इत्येकादशरूपोऽयं शिवः साक्षाद्विभासते}
{सदा ह्येतेषु लिङ्गेषु सन्निधत्ते महेश्वरः}%॥ ६७ ॥

\twolineshloka
{तत्स्वपापौघशुद्ध्यर्थं स्थापयस्व महेश्वरम्}
{अथ चेत्त्वं महाभाग लिङ्गमुत्सादयिष्यसि}%॥ ६८ ॥

\twolineshloka
{मयात्र स्थापितं वत्स सीतया सैकतं कृतम्}
{स्थापयिष्यामि च ततो लिङ्गमेतत्त्वया कृतम्}%॥ ६९ ॥

\twolineshloka
{पातालं सुतलं प्राप्य वितलं च रसातलम्}
{तलातलं च तदिदं भेदयित्वा तु तिष्ठति}%॥ ७० ॥

\twolineshloka
{प्रतिष्ठितं मया लिङ्गं भेत्तुं कस्य बलं भवेत्}
{उत्तिष्ठ लिङ्गमुद्वास्य मयैतत्स्थापितं कपे}%॥ ७१ ॥

\twolineshloka
{त्वया समाहृतं लिङ्गं स्थापयस्वाशु मा शुचः}
{इत्युक्तस्तं प्रणम्याथाज्ञातसत्त्वोऽथ वानरः}%॥ ७२ ॥

\twolineshloka
{उद्वासयामि वेगेन सैकतं लिङ्गमुत्त मम्}
{संस्थापयामि कैलासादानीतं लिङ्गमादरात्}%॥ ७३ ॥

\twolineshloka
{उद्वासने सैकतस्य कियान्भारो भवेन्मम}
{चेतसैवं विचार्यायं हनूमान्मारुता त्मजः}%॥ ७४ ॥

\twolineshloka
{पश्यतां सर्वदेवानां मुनीनां कपिरक्षसाम्}
{पश्यतो रामचन्द्रस्य लक्ष्मणस्यापि पश्यतः}%॥ ७५ ॥

\twolineshloka
{पश्यन्त्या अपि वैदेह्या लिङ्गं तत्सैकतं बलात्}
{पाणिना सर्वयत्नेन जग्राह तरसा बली}%॥ ७६ ॥

\twolineshloka
{यत्नेन महता चायं चालयन्नपि मारुतिः}
{नालं चालयितुं ह्यासीत्सैकतं लिङ्गमोजसा}%॥ ७७ ॥

\twolineshloka
{ततः किलकिलाशब्दं कुर्वन्वानरपुङ्गवः}
{पुच्छमुद्यम्य पाणिभ्यां निरास्थत्तन्निजौजसा}%॥ ७८ ॥

\twolineshloka
{इत्यनेकप्रकारेण चाल यन्नपि वानरः}
{नैव चालयितुं शक्तो बभूव पवनात्मजः}%॥ ७९ ॥

\twolineshloka
{तद्वेष्टयित्वा पुच्छेन पाणिभ्यां धरणीं स्पृशन्}
{उत्पपाताथ तरसा व्योम्नि वायुसुतः कपिः}%॥ ८० ॥

\twolineshloka
{कम्पयन्स धरां सर्वां सप्तद्वीपां सपर्वतम्}
{लिङ्गस्य क्रोशमात्रे तु मूर्च्छितो रुधिरं वमन्}%॥ ८१ ॥

\twolineshloka
{पपात हनुमान्विप्राः कम्पिताङ्गो धरातले}
{पततो वायुपुत्रस्य वक्त्राच्च नयनद्वयात्}%॥ ८२ ॥

\twolineshloka
{नासापुटाच्छ्रोत्ररन्ध्रादपानाच्च द्विजोत्तमाः}
{रुधिरौघः प्रसुस्राव रक्तकुण्ड मभूच्च तत्}%॥ ८३ ॥

\twolineshloka
{ततो हाहाकृतं सर्वं सदेवासुरमानुषम्}
{धावन्तौ कपिभिः सार्द्धमुभौ तौ रामलक्ष्मणौ}%॥ ८४ ॥

\twolineshloka
{जानकीसहितौ विप्रा ह्यास्तां शोकाकुलौ तदा}
{सीतया सहितौ वीरौ वानरैश्च महाबलौ}%॥ ८५ ॥

\twolineshloka
{रुरुचाते तदा विप्रा गन्धमादनपर्वते}
{यथा तारागणयुतौ रजन्यां शशि भास्करौ}%॥ ८६ ॥

\twolineshloka
{ददर्शतुर्हनूमन्तं चूर्णीकृतकलेवरम्}
{मूर्च्छितं पतितं भूमौ वमन्तं रुधिरं मुखात्}%॥ ८७ ॥

\twolineshloka
{विलोक्य कपयः सर्वे हाहाकृत्वाऽपतन्भुवि}
{कराभ्यां सदयं सीता हनूमन्तं मरुत्सुतम्}%॥ ८८ ॥

\twolineshloka
{ताततातेति पस्पर्श पतितं धरणीतले}
{रामोऽपि दृष्ट्वा पतितं हनूमन्तं कपीश्वरम्}%॥ ८९ ॥

\twolineshloka
{आरोप्याङ्कं स्वपाणिभ्यामाममर्श कलेवरम्}
{विमुञ्चन्नेत्रजं वारि वायुजं चाव्रवीद्द्विजाः}%॥ ९० ॥
॥इति श्रीस्कान्दे महापुराण एकाशीतिसाहस्र्यां संहितायां तृतीये ब्रह्मखण्डे सेतुमाहात्म्ये रामचन्द्रतत्त्वज्ञानोपदेशवर्णनं नाम पञ्चचत्वारिंशोऽध्यायः॥४५॥

\sect{षट्चत्वारिंशोऽध्यायः --- रामनाथलिङ्गप्रतिष्ठाकारणवर्णनम्}

\src{स्कन्दपुराणम्}{खण्डः ३ (ब्रह्मखण्डः)}{सेतुखण्डः}{अध्यायः ४६}
\vakta{}
\shrota{}
\tags{}
\notes{}
\textlink{https://sa.wikisource.org/wiki/स्कन्दपुराणम्/खण्डः_३_(ब्रह्मखण्डः)/सेतुखण्डः/अध्यायः_४६}
\translink{https://www.wisdomlib.org/hinduism/book/the-skanda-purana/d/doc423614.html}

\storymeta




\uvacha{श्रीराम उवाच}


\twolineshloka
{पम्पारण्ये वयं दीनास्त्वया वानरपुङ्गव॥}
{आश्वासिताः कारयित्वा}%सख्यमादित्यसूनुना॥१॥

\twolineshloka
{त्वां दृष्ट्वा पितरं बन्धून्कौसल्यां जननीमपि}
{न स्मरामो वयं सर्वान्मे त्वयोपकृतं बहु}%॥ २ ॥

\twolineshloka
{मदर्थं सागरस्तीर्णो भवता बहु योजनः}
{तलप्रहाराभिहतो मैनाकोऽपि नगोत्तमः}%॥ ३ ॥

\twolineshloka
{नागमाता च सुरसा मदर्थं भवता जिता}
{छायाग्रहां महाक्रूराम वधीद्राक्षसीं भवान्}%॥ ४ ॥

\twolineshloka
{सायं सुवेलमासाद्य लङ्कामाहत्य पाणिना}
{अयासी रावणगृहं मदर्थं त्वं महाकपे}%॥ ५ ॥

\twolineshloka
{सीतामन्विष्य लङ्कायां रात्रौ गतभयो भवान्}
{अदृष्ट्वा जानकीं पश्चादशोकवनिकां ययौ}%॥ ६ ॥

\twolineshloka
{नमस्कृत्य च वैदेहीमभिज्ञानं प्रदाय च}
{चूडामणिं समादाय मदर्थं जानकीकरात्}%॥ ७ ॥

\twolineshloka
{अशोकवनिकावृक्षानभाङ्क्षीस्त्वं महाकपे}
{ततस्त्वशीतिसाहस्रान्किङ्करान्नाम राक्षसान्}%॥ ८ ॥

\twolineshloka
{रावणप्रतिमान्युद्धे पत्यश्वेभरथाकुलान्}
{अवधीस्त्वं मदर्थे वै महाबलपराक्रमान्}%॥ ९ ॥

\twolineshloka
{ततः प्रहस्ततनयं जम्बुमालिनमागतम्}
{अवधीन्मन्त्रितनयान्सप्त सप्तार्चिवर्चसः}%॥ १० ॥

\twolineshloka
{पञ्च सेनापतीन्पश्चादनयस्त्वं यमालयम्}
{कुमारमक्षमवधीस्ततस्त्वं रणमूर्धनि}%॥ ११ ॥

\twolineshloka
{तत इन्द्रजिता नीतो राक्षसेन्द्र सभां शुभाम्}
{तत्र लङ्केश्वरं वाचा तृणीकृत्यावमन्य च}%॥ १२ ॥

\twolineshloka
{अभाङ्क्षीस्त्वं पुरीं लङ्कां मदर्थं वायुनन्दन}
{पुनः प्रतिनिवृत्तस्त्वमृष्यमूकं महागिरिम्}%॥ १३ ॥

\twolineshloka
{एवमादि महादुःखं मदर्थं प्राप्तवानसि}
{त्वमत्र भूतले शेषे मम शोकमुदीरयन्}%॥ १४ ॥

\twolineshloka
{अहं प्राणान्परित्यक्ष्ये मृतोऽसि यदि वायुज}
{सीतया मम किं कार्यं लक्ष्मणेनानुजेन वा}%॥ १५ ॥

\twolineshloka
{भरतेनापि किं कार्यं शत्रुघ्नेन श्रियापि वा}
{राज्येनापि न मे कार्यं परेतस्त्वं कपे यदि}%॥ १६ ॥

\twolineshloka
{उत्तिष्ठ हनुमन्वत्स किं शेषेऽद्य महीतले}
{शय्यां कुरु महाबाहो निद्रार्थं मम वानर}%॥ १७ ॥

\twolineshloka
{कन्दमूलफलानि त्वमाहारार्थं ममाहर}
{स्नातुमद्य गमिष्यामि शीघ्रं कलशमानय}%॥ १८ ॥

\twolineshloka
{अजिनानि च वासांसि दर्भांश्च समुपाहर}
{ब्रह्मास्त्रेणावबद्धोऽहं मोचितश्च त्वया हरे}%॥ १९ ॥

\twolineshloka
{लक्ष्मणेन सह भ्रात्रा ह्यौषधानयनेन वै}
{लक्ष्मणप्राणदाता त्वं पौलस्त्यमदनाशनः}%॥ २० ॥

\twolineshloka
{सहायेन त्वया युद्धे राक्षसा न्रावणादिकान्}
{निहत्यातिबलान्वीरानवापं मैथिलीमहम्}%॥ २१ ॥

\twolineshloka
{हनूमन्नञ्जनासूनो सीताशोकविनाशन}
{कथमेवं परित्यज्य लक्ष्मणं मां च जानकीम्}%॥ २२ ॥

\twolineshloka
{अप्रापयित्वायोध्यां त्वं किमर्थं गतवानसि}
{क्व गतोसि महावीर महाराक्षसकण्टक}%॥ २३ ॥

\twolineshloka
{इति पश्यन्मुखं तस्य निर्वाक्यं रघुनन्दनः}
{प्ररुदन्नश्रुजालेन सेचयामास वायुजम्}%॥ २४ ॥

\twolineshloka
{वायुपुत्रस्ततो मूर्च्छामपहाय शनैर्द्विजाः}
{पौलस्त्यभयसन्त्रस्तलोकरक्षार्थमागतम्}%॥ २५ ॥

\twolineshloka
{आश्रित्य मानुषं भावं नारायणमजं विभुम्}
{जानकीलक्ष्मणयुतं कपिभिः परिवारितम्}%॥ २६ ॥

\twolineshloka
{कालाम्भोधरसङ्काशं रणधूलिसमुक्षितम्}
{जटामण्डलशोभाढ्यं पुण्डरीकायतेक्षणम्}%॥ २७ ॥

\twolineshloka
{खिन्नं च बहुशो युद्धे ददर्श रघुनन्दनम्}
{स्तूयमानममित्रघ्नं देवर्षिपितृकिन्नरैः}%॥ २८ ॥

\twolineshloka
{दृष्ट्वा दाशरथिं रामं कृपाबहुलचेतसम्}
{रघुनाथकरस्पर्शपूर्णगात्रः स वानरः}%॥ २९ ॥

\twolineshloka
{पतित्वा दण्डवद्भूमौ कृताञ्जलिपुटो द्विजाः}
{अस्तौषीज्जानकीनाथं स्तोत्रैः श्रुतिमनोहरैः}%॥ ३० ॥

\uvacha{हनूमानुवाच}

\twolineshloka
{नमो रामाय हरये विष्णवे प्रभविष्णवे}
{आदिदेवाय देवाय पुराणाय गदाभृते}%॥ ३१ ॥

\twolineshloka
{विष्टरे पुष्पकं नित्यं निविष्टाय महात्मने}
{प्रहृष्टवानरानीकजुष्टपादाम्बुजाय ते}%॥ ३२ ॥

\twolineshloka
{निष्पिष्ट राक्षसेन्द्राय जगदिष्टविधायिने}
{नमः सहस्रशिरसे सहस्रचरणाय च}%॥ ३३ ॥

\twolineshloka
{सहस्राक्षाय शुद्धाय राघवाय च विष्णवे}
{भक्तार्तिहारिणे तुभ्यं सीतायाः पतये नमः}%॥ ३४ ॥

\twolineshloka
{हरये नारसिंहाय दैत्यराजविदारिणे}
{नमस्तुभ्यं वराहाय दंष्ट्रोद्धृतवसुन्धर}%॥ ३५ ॥

\twolineshloka
{त्रिविक्रमाय भवते बलियज्ञ विभेदिने}
{नमो वामनरूपाय नमो मन्दरधारिणे}%॥ ३६ ॥

\twolineshloka
{नमस्ते मत्स्यरूपाय त्रयीपालनकारिणे}
{नमः परशुरामाय क्षत्रियान्तकराय ते}%॥ ३७ ॥

\twolineshloka
{नमस्ते राक्षसघ्नाय नमो राघवरूपिणे}
{महादेव महाभीम महाकोदण्डभेदिने}%॥ ३८ ॥

\twolineshloka
{क्षत्रियान्तकरक्रूरभार्गवत्रासकारिणे}
{नमोऽस्त्वहिल्या सन्तापहारिणे चापहारिणे}%॥ ३९ ॥

\twolineshloka
{नागायुतवलोपेतताटकादेहहारिणे}
{शिलाकठिनविस्तारवालिवक्षोविभेदिने}%॥ ४० ॥

\twolineshloka
{नमो माया मृगोन्माथकारिणेऽज्ञानहारिणे}
{दशस्यन्दनदुःखाब्धिशोषणागस्त्यरूपिणे}%॥ ४१ ॥

\twolineshloka
{अनेकोर्मिसमाधूतसमुद्रमदहारिणे}
{मैथिलीमानसां भोजभानवे लोकसाक्षिणे}%॥ ४२ ॥

\twolineshloka
{राजेन्द्राय नमस्तुभ्यं जानकीपतये हरे}
{तारकब्रह्मणे तुभ्यं नमो राजीवलोचन}%॥ ४३ ॥

\twolineshloka
{रामाय रामचन्द्राय वरेण्याय सुखात्मने}
{विश्वामित्रप्रियायेदं नमः खरविदारिणे}%॥ ४४ ॥

\twolineshloka
{प्रसीद देवदेवेश भक्तानामभयप्रद}
{रक्ष मां करु णासिन्धो रामचन्द्र नमोऽस्तु ते}%॥ ४५ ॥

\twolineshloka
{रक्ष मां वेदवचसामप्यगोचर राघव}
{पाहि मां कृपया राम शरणं त्वामुपैम्यहम्}%॥ ४६ ॥

\twolineshloka
{रघुवीर महामोहमपाकुरु ममाधुना}
{स्नाने चाचमने भुक्तो जाग्रत्स्वप्नसुषुप्तिषु}%॥ ४७ ॥

\twolineshloka
{सर्वावस्थासु सर्वत्र पाहि मां रघुनन्दन}
{महिमानं तव स्तोतुं कः समर्थो जगत्त्रये}%॥ ४८ ॥

\twolineshloka
{त्वमेव त्वन्महत्त्वं वै जानासि रघुनन्दन}
{इति स्तुत्वा वायुपुत्रो रामचन्द्रं घृणानिधिम्}%॥ ४९ ॥

\twolineshloka
{सीतामप्यभितुष्टाव भक्तियुक्तेन चेतसा}
{जानकि त्वां नमस्यामि सर्वपापप्रणाशिनीम्}%॥ ५० ॥

\twolineshloka
{दारिद्र्यरणसंहर्त्रीं भक्तानामिष्टदायिनीम्}
{विदेहराजतनयां राघवानन्दकारिणीम्}%॥ ५१ ॥

\twolineshloka
{भूमेर्दुहितरं विद्यां नमामि प्रकृतिं शिवाम्}
{पौलस्त्यैश्वर्यसंहर्त्रीं भक्ताभीष्टां सरस्वतीम्}%॥ ५२ ॥

\twolineshloka
{पतिव्रताधुरीणां त्वां नमामि जनकात्मजाम्}
{अनुग्रहपरामृद्धिमनघां हरिवल्लभाम्}%॥ ५३ ॥

\twolineshloka
{आत्मविद्यां त्रयीरूपामुमारूपां नमाम्य हम्}
{प्रसादाभिमुखीं लक्ष्मीं क्षीराब्धितनयां शुभाम्}%॥ ५४ ॥

\twolineshloka
{नमामि चन्द्रभगिनीं सीतां सर्वाङ्गसुन्दरीम्}
{नमामि धर्मनिलयां करुणां वेदमातरम्}%॥ ५५ ॥

\twolineshloka
{पद्मालयां पद्महस्तां विष्णुवक्षस्थलालयाम्}
{नमामि चन्द्रनिलयां सीतां चन्द्रनिभाननाम्}%॥ ५६ ॥

\threelineshloka
{आह्लादरूपिणीं सिद्धिं शिवां शिवकरीं सतीम्}
{नमामि विश्वजननीं रामचन्द्रेष्टवल्लभाम्}
{सीतां सर्वानवद्याङ्गीं भजामि सततं हृदा}%॥ ५७ ॥


\uvacha{श्रीसूत उवाच}

\onelineshloka
{स्तुत्वैवं हनुमान्सीतारामचन्द्रौ सभक्तिकम्}%॥ ५८ ॥

\twolineshloka
{आनन्दाश्रुपरिक्लिन्नस्तूष्णीमास्ते द्विजोत्तमाः}
{य इदं वायुपुत्रेण कथितं पापनाशनम्}%॥ ५९ ॥

\twolineshloka
{स्तोत्रं श्रीरामचन्द्रस्य सीतायाः पठतेऽन्वहम्}
{स नरो महदैश्वर्यमश्नुते वाञ्छितं स दा}%॥ ६० ॥

\twolineshloka
{अनेकक्षेत्रधान्यानि गाश्च दोग्ध्रीः पयस्विनीः}
{आयुर्विद्याश्च पुत्रांश्च भार्यामपि मनोरमाम्}%॥ ६१ ॥

\twolineshloka
{एतत्स्तोत्रं सकृ द्विप्राः पठन्नाप्नोत्यसंशयः}
{एतत्स्तोत्रस्य पाठेन नरकं नैव यास्यति}%॥ ६२ ॥

\twolineshloka
{ब्रह्महत्यादिपापानि नश्यन्ति सुमहान्त्यपि}
{सर्वपापविनिर्मुक्तो देहान्ते मुक्तिमाप्नुयात्}%॥ ६३ ॥

\twolineshloka
{इति स्तुतो जगन्नाथो वायुपुत्रेण राघवः}
{सीतया सहितो विप्रा हनूमन्तमथाब्रवीत्}%॥ ६४ ॥

\uvacha{श्रीराम उवाच}

\twolineshloka
{अज्ञानाद्वा नरश्रेष्ठ त्वयेदं साहसं कृतम्}
{ब्रह्मणा विष्णुना वापि शक्रादित्रिदशैरपि}%॥ ६५ ॥

\twolineshloka
{नेदं लिङ्गं समुद्धर्तुं शक्यते स्थापितं मया}
{महादेवापराधेन पतितोऽस्यद्य मूर्च्छितः}%॥ ६६ ॥

\twolineshloka
{इतः परं मा क्रियतां द्रोहः साम्बस्य शूलिनः}
{अद्यारभ्य त्विदं कुण्डं तव नाम्ना जगत्त्रये}%॥ ६७ ॥

\twolineshloka
{ख्यातिं प्रयातु यत्र त्वं पतितो वानरोत्तम}
{महापातकसङ्घानां नाशः स्यादत्र मज्जनात्}%॥ ६८ ॥

\twolineshloka
{महादेवजटाजाता गौतमी सरितां वरा}
{अश्वमेधसहस्रस्य फलदा स्नायिनां नृणाम्}%॥ ६९ ॥

\twolineshloka
{ततः शतगुणा गङ्गा यमुना च सरस्वती}
{एतन्नदीत्रयं यत्र स्थले प्रवहते कपे}%॥ ७० ॥

\twolineshloka
{मिलित्वा तत्र तु स्नानं सहस्रगुणितं स्मृतम्}
{नदीष्वेतासु यत्स्नानात्फलं पुंसां भवेत्कपे}%॥ ७१ ॥

\twolineshloka
{तत्फलं तव कुण्डेऽस्मिन्स्नानात्प्राप्नोत्यसंशयम्}
{दुर्लभं प्राप्य मानुष्यं हनूमत्कुण्डतीरतः}%॥ ७२ ॥

\twolineshloka
{श्राद्धं न कुरुते यस्तु भक्तियुक्तेन चेतसा}
{निराशास्तस्य पितरः प्रयान्ति कुपिताः कपे}%॥ ७३ ॥

\twolineshloka
{कुप्यन्ति मुनयोऽप्यस्मै देवाः सेन्द्राः सचारणाः}
{न दत्तं न हुतं येन हनूमत्कुण्डतीरतः}%॥ ७४ ॥

\threelineshloka
{वृथाजीवित एवासाविहामुत्र च दुःखभाक्}
{हनूमत्कुण्डसविधे येन दत्तं तिलोदकम्}
{मोदन्ते पितरस्तस्य घृतकुल्याः पिबन्ति च}%॥ ७५ ॥


\uvacha{श्रीसूत उवाच}

\onelineshloka
{श्रुत्वैतद्वचनं विप्रा रामेणोक्तं स वायुजः}%॥ ७६ ॥

\twolineshloka
{उत्तरे रामनाथस्य लिङ्गं स्वेनाहृतं मुदा}
{आज्ञया रामचन्द्रस्य स्थापयामास वायुजः}%॥ ७७ ॥

\threelineshloka
{प्रत्यक्षमेव सर्वेषां कपिलाङ्गूलवेष्टितम्}
{हरोपि तत्पुच्छजा तं बिभर्ति च वलित्रयम्}
{तदुत्तरायां ककुभि गौरीं संस्थापयन्मुदा}%॥ ७८ ॥

\uvacha{श्रीसूत उवाच}

\twolineshloka
{एवं वः कथितं विप्रा यदर्थं राघवेण तु}
{लिङ्गं प्रतिष्ठितं सेतौ भुक्तिमुक्तिप्रदं नृणाम्}%॥ ७९ ॥

\twolineshloka
{यः पठेदिममध्यायं शृणुयाद्वा समाहितः}
{स विधूयेह पापानि शिवलोके महीयते}%॥ ८० ॥
॥इति श्रीस्कान्दे महापुराण एकाशीतिसाहस्र्यां संहितायां तृतीये ब्रह्मखण्डे सेतुमाहात्म्ये रामनाथलिङ्गप्रतिष्ठाकारणवर्णनं नाम षट्चत्वारिंशोऽध्यायः॥४६॥


\sect{सप्तचत्वारिंशोऽध्यायः --- रामस्य ब्रह्महत्योत्पत्तिहेतुनिरूपणम्}

\src{स्कन्दपुराणम्}{खण्डः ३ (ब्रह्मखण्डः)}{सेतुखण्डः}{अध्यायः ४७}
\vakta{}
\shrota{}
\tags{}
\notes{}
\textlink{https://sa.wikisource.org/wiki/स्कन्दपुराणम्/खण्डः_३_(ब्रह्मखण्डः)/सेतुखण्डः/अध्यायः_४७}
\translink{https://www.wisdomlib.org/hinduism/book/the-skanda-purana/d/doc423615.html}

\storymeta




\uvacha{ऋषय ऊचुः}

\twolineshloka
{राक्षसस्य वधात्सूत रावणस्य महामुने}
{ब्रह्महत्या कथमभूद्राघवस्य महात्मनः}%॥ १ ॥

\twolineshloka
{ब्राह्मणस्य वधात्सूत ब्रह्महत्याभिजायते}
{न ब्राह्मणो दशग्रीवः कथं तद्वद नो मुने}%॥ २ ॥

\twolineshloka
{ब्रह्महत्या भवेत्क्रूरा रामचन्द्रस्य धीमतः}
{एतन्नः श्रद्दधानानां वद कारुण्यतोऽधुना}%॥ ३ ॥

\twolineshloka
{इति पृष्टस्ततः सूतो नैमिषारण्यवासिभिः}
{वक्तुं प्रचक्रमे तेषां प्रश्नस्योत्तरमुत्तमम्}%॥ ४ ॥

\uvacha{श्रीसूत उवाच}

\twolineshloka
{ब्रह्मपुत्रो महातेजाः पुलस्त्योनाम वै द्विजाः}
{बभूव तस्य पुत्रोऽभूद्विश्रवा इति विश्रुतः}%॥ ५ ॥

\twolineshloka
{तस्य पुत्रः पुलस्त्य स्य विश्रवा मुनिपुङ्गवाः}
{चिरकालं तपस्तेपे देवैरपि सुदुष्करम्}%॥ ६ ॥

\twolineshloka
{तपः कुर्वति तस्मिंस्तु सुमाली नाम राक्षसः}
{पाताललोकाद्भूलोकं सर्वं वै विचचार ह}%॥ ७ ॥

\twolineshloka
{हेमनिष्काङ्गदधरः कालमेघनिभच्छविः}
{समादाय सुतां कन्यां पद्महीनामिव श्रियम्}%॥ ८ ॥

\twolineshloka
{विचरन्स महीपृष्ठे कदाचित्पुष्पकस्थितम्}
{दृष्ट्वा विश्रवसः पुत्रं कुबेरं वै धनेश्वरम्}%॥ ९ ॥

\twolineshloka
{चिन्तयामास विप्रेन्द्राः सुमाली स तु राक्षसः}
{कुबेरसदृशः पुत्रो यद्यस्माकं भविष्यति}%॥ १० ॥

\twolineshloka
{वयं वर्द्धामहे सर्वे राक्षसा ह्यकुतोभयाः}
{विचार्यैवं निजसुतामब्रवीद्राक्षसेश्वरः}%॥ ११ ॥

\twolineshloka
{सुते प्रदानकालोऽद्य तव कैकसि शोभने}
{अद्य ते यौवनं प्राप्तं तद्देया त्वं वराय हि}%॥ १२ ॥

\twolineshloka
{अप्रदानेन पुत्रीणां पितरो दुःखमाप्नुयुः}
{किं च सर्वगुणोत्कृष्टा लक्ष्मीरिव सुते शुभे}%॥ १३ ॥

\twolineshloka
{प्रत्याख्यानभयात्पुम्भिर्न च त्वं प्रार्थ्यसे शुभे}
{कन्यापितृत्वं दुःखाय सर्वेषां मानकाङ्क्षिणाम्}%॥ १४ ॥

\twolineshloka
{न जानेऽहं वरः को वा वरयेदिति कन्यके}
{सा त्वं पुलस्त्यतनयं मुनिं विश्रवसं द्विजम्}%॥ १५ ॥

\twolineshloka
{पितामहकुलोद्भूतं वरयस्व स्वयङ्गता}
{कुबेरतुल्यास्तनया भवेयुस्ते न संशयः}%॥ १६ ॥

\twolineshloka
{कैकसी तद्वचः श्रुत्वा सा कन्या पितृगौरवात्}
{अङ्गीचकार तद्वाक्यं तथास्त्विति शुचिस्मिता}%॥ १७ ॥

\twolineshloka
{पर्णशालां मुनिश्रेष्ठा गत्वा विश्रवसो मुनेः}
{अतिष्ठदन्तिके तस्य लज्जमाना ह्यधोमुखी}%॥ १८ ॥

\twolineshloka
{तस्मिन्नवसरे विप्राः पुलस्त्यतनयः सुधीः}
{अग्निहोत्रमुपास्ते स्म ज्वलत्पावकसन्निभः}%॥ १९ ॥

\twolineshloka
{सन्ध्याकालमतिक्रूरमविचिन्त्य तु कैकसी}
{अभ्येत्य तं मुनिं सुभ्रूः पितुर्वचनगौरवात्}%॥ २० ॥

\threelineshloka
{तस्थावधोमुखी भूमिं लिखत्यङ्गुष्ठकोटिना}
{विश्रवास्तां विलोक्याथ कैकसीं तनुमध्यमाम्}
{उवाच सस्मितो विप्राः पूर्णचन्द्रनिभाननाम्}%॥ २१ ॥

\uvacha{विश्रवा उवाच}

\onelineshloka
{शोभने कस्य पुत्री त्वं कुतो वा त्वमिहागता}%॥ २२ ॥

\twolineshloka
{कार्यं किं वा त्वमुद्दिश्य वर्तसेऽत्र शुचिस्मिते}
{यथार्थतो वदस्वाद्य मम सर्वमनिन्दिते}%॥ २३ ॥

\twolineshloka
{इतीरिता कैकसी सा कन्या बद्धाञ्जलिर्द्विजाः}
{उवाच तं मुनिं प्रह्वा विनयेन समन्विता}%॥ २४ ॥

\twolineshloka
{तपः प्रभावेन मुने मदभिप्रायमद्य तु}
{वेत्तुमर्हसि सम्यक्त्वं पुलस्त्यकुलदीपन}%॥ २५ ॥

\twolineshloka
{अहं तु कैकसी नाम सुमालिदुहिता मुने}
{मत्तातस्याज्ञया ब्रह्मंस्तवान्तिकमुपागता}%॥ २६ ॥

\twolineshloka
{शेष त्वं ज्ञानदृष्ट्याद्य ज्ञातुमर्हस्यसंशयः}
{क्षणं ध्यात्वा मुनिः प्राह विश्रवाः स तु कैकसीम्}%॥ २७ ॥

\twolineshloka
{मया ते विदितं सुभ्रूर्मनोगतमभीप्सितम्}
{पुत्राभिलाषिणी सा त्वं मामगाः साम्प्रतं शुभे}%॥ २८ ॥

\twolineshloka
{सायङ्कालेऽधुना क्रूरे यस्मान्मां त्वमुपागता}
{पुत्राभिलाषिणी भूत्वा तस्मात्त्वां प्रब्रवीम्यहम्}%॥ २९ ॥

\twolineshloka
{शृणुष्वावहिता रामे कैकसी त्वमनिन्दिते}
{दारुणान्दारुणाकारान्दारुणाभिजनप्रियान्}%॥ ३० ॥

\twolineshloka
{जनयिष्यसि पुत्रांस्त्वं राक्षसान्क्रूरकर्मणः}
{श्रुत्वा तद्वचनं सा तु कैकसी प्रणिपत्य तम्}%॥ ३१ ॥

\twolineshloka
{पुलस्त्यतनयं प्राह कृताञ्जलिपुटा द्विजाः}
{भगवदीदृशाः पुत्रास्त्वत्तः प्राप्तुं न युज्यते}%॥ ३२ ॥

\twolineshloka
{इत्युक्तः स मुनिः प्राह कैकसीं तां सुमध्यमाम्}
{मद्वंशानुगुणः पुत्रः पश्चिमस्ते भविष्यति}%॥ ३३ ॥

\twolineshloka
{धार्मिकः शास्त्रविच्छान्तो न तु राक्षसचेष्टितः}
{इत्युक्ता कैकसी विप्राः काले कतिपये गते}%॥ ३४ ॥

\twolineshloka
{सुषुवे तनयं क्रूरं रक्षोरूपं भयङ्करम्}
{द्विपञ्चशीर्षं कुमतिं विंशद्बाहुं भयानकम्}%॥ ३५ ॥

\twolineshloka
{ताम्रोष्ठं कृष्णवदनं रक्तश्मश्रु शिरोरुहम्}
{महादंष्ट्रं महाकायं लोकत्रासकरं सदा}%॥ ३६ ॥

\twolineshloka
{दशग्रीवाभिधः सोऽभूत्तथा रावण नामवान्}
{रावणानन्तरं जातः कुम्भकर्णाभिधः सुतः}%॥ ३७ ॥

\twolineshloka
{ततः शूर्पणखा नाम्ना क्रूरा जज्ञे च राक्षसी}
{ततो बभूव कैकस्या विभीषण इति श्रुतः}%॥ ३८ ॥

\twolineshloka
{पश्चिमस्तनयो धीमान्धार्मिको वेदशास्त्रवित्}
{एते विश्रवसः पुत्रा दशग्रीवादयो द्विजाः}%॥ ३९ ॥

\twolineshloka
{अतो दशग्रीववधात्कुम्भकर्णवधादपि}
{ब्रह्महत्या समभवद्रामस्याक्लिष्टकर्मणः}%॥ ४० ॥

\twolineshloka
{अतस्तच्छान्तये रामो लिङ्गं रामेश्वराभिधम्}
{स्थापयामास विधिना वैदिकेन द्विजोत्तमाः}%॥ ४१ ॥

\twolineshloka
{एवं रावणघातेन ब्रह्महत्यासमुद्भवः}
{समभूद्रामचन्द्रस्य लोककान्तस्य धीमतः}%॥ ४२ ॥

\twolineshloka
{तत्सहैतुकमाख्यातं भवतां ब्रह्मघातजम्}
{पापं यच्छान्तये रामो लिङ्गं प्रातिष्ठिपत्स्वयम्}%॥ ४३ ॥

\twolineshloka
{एवं लिङ्गं प्रतिष्ठाप्य रामचन्द्रोऽतिधार्मिकः}
{मेने कृतार्थमात्मानं ससीता वरजो द्विजाः}%॥ ४४ ॥

\twolineshloka
{ब्रह्महत्या गता यत्र रामचन्द्रस्य भूपतेः}
{तत्र तीर्थमभूत्किञ्चिद्ब्रह्महत्याविमोचनम्}%॥ ४५ ॥

\twolineshloka
{तत्र स्नानं महापुण्यं ब्रह्महत्याविनाशनम्}
{दृश्यते रावणोऽद्यापि छायारूपेण तत्र वै}%॥ ४६ ॥

\twolineshloka
{तदग्रे नागलोकस्य बिलमस्ति महत्तरम्}
{दशग्रीववधोत्पन्नां ब्रह्महत्यां बलीयसीम्}%॥ ४७ ॥

\twolineshloka
{तद्बिलं प्रापयामास जानकीरमणो द्विजाः}
{तस्योपरि बिलस्याथ कृत्वा मण्डपमुत्तमम्}%॥ ४८ ॥

\twolineshloka
{भैरवं स्थापयामास रक्षार्थं तत्र राघवः}
{भैरवाज्ञापरित्रस्ता ब्रह्महत्या भयङ्करी}%॥ ४९ ॥

\twolineshloka
{नाशक्नोत्तद्बिलादूर्ध्वं निर्गन्तुं द्विजसत्तमाः}
{तस्मिन्नेव बिले तस्थौ ब्रह्महत्या निरुद्यमा}%॥ ५० ॥

\twolineshloka
{रामनाथमहालिङ्गं दक्षिणे गिरिजा मुदा}
{वर्तते परमानन्दशिवस्यार्धशरीरिणी}%॥ ५१ ॥

\twolineshloka
{आदित्यसोमौ वर्तेते पार्श्वयोस्तत्र शूलिनः}
{देवस्य पुरतो वह्नी रामनाथस्य वर्तते}%॥ ५२ ॥

\twolineshloka
{आस्ते शतक्रतुः प्राच्यामाग्नेयां च तथाऽनलः}
{आस्ते यमो दक्षिणस्यां रामनाथस्य सेवकः}%॥ ५३ ॥

\twolineshloka
{नैर्ऋते निर्ऋतिर्विप्रा वर्तते शङ्करस्य तु}
{वारुण्यां वरुणो भक्त्या सेवते राघवेश्वरम्}%॥ ५४ ॥

\twolineshloka
{वायव्ये तु दिशो भागे वायुरास्ते शिवस्य तु}
{उत्तरस्यां च धनदो रामनाथस्य वर्तते}%॥ ५५ ॥

\twolineshloka
{ईशान्येऽस्य च दिग्भागे महेशो वर्तते द्विजाः}
{विनायक कुमारौ च महादेवसुतावुभौ}%॥ ५६ ॥

\twolineshloka
{यथाप्रदेशं वर्तेते रामनाथालयेऽधुना}
{वीरभद्रादयः सर्वे महेश्वरगणेश्वराः}%॥ ५७ ॥

\twolineshloka
{यथाप्रदेशं वर्तन्ते रामनाथालये सदा}
{मुनयः पन्नगाः सिद्धा गन्धर्वाप्सरसां गणाः}%॥ ५८ ॥

\twolineshloka
{सन्तुष्यमाणहृदया यथेष्टं शिवसन्निधौ}
{वर्तन्ते रामनाथस्य सेवार्थं भक्तिपूर्वकम्}%॥ ५९ ॥

\twolineshloka
{रामनाथस्य पूजार्थं श्रोत्रियान्ब्राह्मणान्बहून्}
{रामेश्वरे रघुपतिः स्थापयामास पूजकान्}%॥ ६० ॥

\twolineshloka
{रामप्रतिष्ठितान्विप्रान्हव्यकव्यादिनार्चयेत्}
{तुष्टास्ते तोषिताः सर्वा पितृभिः सहदेवताः}%॥ ६१ ॥

\twolineshloka
{तेभ्यो बहुधनान्ग्रामान्प्रददौ जानकीपतिः}
{रामनाथमहादेव नैवेद्यार्थमपि द्विजाः}%॥ ६२ ॥

\twolineshloka
{बहून्ग्रामान्बहुधनं प्रददौ लक्ष्मणाग्रजः}
{हारकेयूरकटकनिष्काद्याभरणानि च}%॥ ६३ ॥

\twolineshloka
{अनेकपट्ट वस्त्राणि क्षौमाणि विविधानि च}
{रामनाथाय देवाय ददौ दशरथात्मजः}%॥ ६४ ॥

\twolineshloka
{गङ्गा च यमुना पुण्या सरयूश्च सरस्वती}
{सेतौ रामेश्वरं देवं भजन्ते स्वाघशान्तये}%॥ ६५ ॥

\twolineshloka
{एतदध्यायपठनाच्छ्रवणादपि मानवः}
{विमुक्तः सर्वपापेभ्यः सायुज्यं लभते हरेः}%॥ ६६ ॥
॥इति श्रीस्कान्दे महापुराण एकाशीतिसाहस्र्यां संहितायां तृतीये ब्रह्मखण्डे सेतुमाहात्म्ये रामस्य ब्रह्महत्योत्पत्तिहेतुनिरूपणं नाम सप्तचत्वारिंशोऽध्यायः॥४७॥
    \sect{धर्मारण्यतीर्थक्षेत्रजीर्णोद्धारवर्णनम्}


\src{स्कन्दपुराणम्}{खण्डः ३ (ब्रह्मखण्डः)}{धर्मारण्य खण्डः}{अध्यायाः ३१--३५}
\vakta{}
\shrota{}
\tags{}
\notes{These five chapters describe Rāma’s pilgrimage to Dharmāraṇya, the establishment of Satyamandira, Rāma's Return to Ayodhyā, Rāma’s copper-plate grant to Brāhmaṇas and ultimately the repair of the ruins of Dharmāraṇya.}
\textlink{https://sa.wikisource.org/wiki/स्कन्दपुराणम्/खण्डः_३_(ब्रह्मखण्डः)/धर्मारण्य_खण्डः/अध्यायः_३१}
\translink{https://www.wisdomlib.org/hinduism/book/the-skanda-purana/d/doc423652.html}

\storymeta

\dnsub{एकत्रिंशोऽध्यायः --- दूतागमनम्}\resetShloka

\uvacha{श्रीराम उवाच}

\twolineshloka
{भगवन्यानि तीर्थानि सेवितानि त्वया विभो}
{एतेषां परमं तीर्थं तन्ममाचक्ष्व मानद}%॥ १ ॥

\twolineshloka
{मया तु सीताहरणे निहता ब्रह्मराक्षसाः}
{तत्पापस्य विशुदयर्थं वद तीर्थोत्तमोत्तमम्}%॥ २ ॥

\uvacha{वसिष्ठ उवाच}

\twolineshloka
{गङ्गा च नर्मदा तापी यमुना च सरस्वती}
{गण्डकी गोमती पूर्णा एता नद्यः सुपावनाः}%॥ ३ ॥

\twolineshloka
{एतासां नर्मदा श्रेष्ठा गङ्गा त्रिपथगामिनी}
{दहते किल्बिषं सर्वं दर्शनादेव राघव}%॥ ४ ॥

\twolineshloka
{दृष्ट्वा जन्मशतं पापं गत्वा जन्मशतत्रयम्}
{स्नात्वा जन्मसहस्रं च हन्ति रेवा कलौ युगे}%॥ ५ ॥

\twolineshloka
{नर्मदातीरमाश्रित्य शाकमूलफलैरपि}
{एकस्मिन्भोजिते विप्रे कोटि भोजफलं लभेत}%॥ ६ ॥

\twolineshloka
{गङ्गा गङ्गेति यो ब्रूयाद्योजनानां शतैरपि}
{मुच्यते सर्वपापेभ्यो विष्णुलोकं स गच्छति}%॥ ७ ॥

\twolineshloka
{फाल्गुनान्ते कुहूं प्राप्य तथा प्रौष्ठपदेऽसिते}
{पक्षे गङ्गामधि प्राप्य स्नानं च पितृतर्पणम्}%॥ ८ ॥

\twolineshloka
{कुरुते पिण्डदानानि सोऽक्षयं फलमश्नुते}
{शुचौ मासे च सम्प्राप्ते स्नानं वाप्यां करोति यः}%॥ ९ ॥

\twolineshloka
{चतुरशीतिनरकान्न पश्यति नरो नृप}
{तपत्याः स्मरणे राम महापातकिनामपि}%॥ १० ॥

\twolineshloka
{उद्धरेत्सप्तगोत्राणि कुलमेकोत्तरं शतम्}
{यमुनायां नरः स्नात्वा सर्वपापैः प्रमुच्यते}%॥ ११ ॥

\twolineshloka
{महापातकयुक्तोऽपि स गच्छेत्परमां गतिम्}
{कार्त्तिक्यां कृत्तिकायोगे सरस्वत्यां निमज्जयेत्}%॥ १२ ॥

\twolineshloka
{गच्छेत्स गरुडारूढः स्तूयमानः सुरोत्तमैः}
{स्नात्वा यः कार्तिके मासि यत्र प्राची सरस्वती}%॥ १३ ॥

\twolineshloka
{प्राचीं माधवमास्तूय स गच्छेत्परमां गतिम्}
{गण्डकीपुण्यतीर्थे हि स्नानं यः कुरुते नरः}%॥ १४ ॥

\twolineshloka
{शालग्रामशिलामर्च्य न भूयः स्तनपो भवेत्}
{गोमतीजलकल्लोलैर्मज्जयेत्कृष्णसन्निधौ}%॥ १५ ॥

\twolineshloka
{चतुर्भुजो नरो भूत्वा वैकुण्ठे मोदते चिरम्}
{चर्मण्वतीं नमस्कृत्य अपः स्पृशति यो नरः}%॥ १६ ॥

\twolineshloka
{स तारयति पूर्वजान्दश पूर्वान्दशापरान्}
{द्वयोश्च सङ्गमं दृष्ट्वा श्रुत्वा वा सागरध्वनिम्}%॥ १७ ॥

\twolineshloka
{ब्रह्महत्यायुतो वापि पूतो गच्छेत्परां गतिम्}
{माघमासे प्रयागे तु मज्जनं कुरुते नरः}%॥ १८ ॥

\twolineshloka
{इह लोके सुखं भुक्त्वा अन्ते विष्णुपदं व्रजेत्}
{प्रभासे ये नरा राम त्रिरात्रं ब्रह्मचारिणः}%॥ १९ ॥

\twolineshloka
{यमलोकं न पश्येयुः कुम्भीपाकादिकं तथा}
{नैमिषारण्यवासी यो नरो देवत्वमाप्नुयात्}%॥ २० ॥

\twolineshloka
{देवानामालयं यस्मात्तदेव भुवि दुर्लभम्}
{कुरुक्षेत्रे नरो राम ग्रहणे चन्द्रसूर्ययोः}%॥ २१ ॥

\twolineshloka
{हेमदानाच्च राजेन्द्र न भूयः स्तनपो भवेत्}
{श्रीस्थले दर्शनं कृत्वा नरः पापात्प्रमुच्यते}%॥ २२ ॥

\twolineshloka
{सर्वदुःखविनाशे च विष्णुलोके महीयते}
{काश्यपीं स्पर्शयेद्यो गां मानवो भुवि राघव}%॥ २३ ॥

\twolineshloka
{सर्वकामदुघावासमृषिलोकं स गच्छति}
{उज्जयिन्यां तु वैशाखे शिप्रायां स्नानमाचरेत्}%॥ २४ ॥

\twolineshloka
{मोचयेद्रौरवाद् घोरात्पूर्वजांश्च सहस्रशः}
{सिन्धुस्नानं नरो राम प्रकरोति दिनत्रयम्}%॥ २५ ॥

\twolineshloka
{सर्वपापविशुद्धात्मा कैलासे मोदते नरः}
{कोटितीर्थे नरः स्नात्वा दृष्ट्वा कोटीश्वरं शिवम्}%॥ २६ ॥

\twolineshloka
{ब्रह्महत्यादिभिः पापैर्लिप्यते न च स क्वचित्}
{अज्ञानामपि जन्तूनां महाऽमेध्ये तु गच्छताम्}%॥ २७ ॥

\twolineshloka
{पादोद्भूतं पयः पीत्वा सर्वपापं प्रणश्यति}
{वेदवत्यां नरो यस्तु स्नाति सूर्योदये शुभे}%॥ २८ ॥

\twolineshloka
{सर्वरोगात्प्रमुच्येत परं सुखमवाप्नुयात्}
{तीर्थानि राम सर्वत्र स्नानपानावगाहनैः}%॥ २९ ॥

\twolineshloka
{नाशयन्ति मनुष्याणां सर्वपापानि लीलया}
{तीर्थानां परमं तीर्थं धर्मारण्यं प्रचक्षते}%॥ ३० ॥

\twolineshloka
{ब्रह्मविष्णुशिवाद्यैर्यदादौ संस्थापितं पुरा}
{अरण्यानां च सर्वेषां तीर्थानां च विशेषतः}%॥ ३१ ॥

\twolineshloka
{धर्मारण्यात्परं नास्ति भुक्तिमुक्तिप्रदायकम्}
{स्वर्गे देवाः प्रशंसन्ति धर्मारण्यनिवासिनः}%॥ ३२ ॥

\twolineshloka
{ते पुण्यास्ते पुण्यकृतो ये वसन्ति कलौ नराः}
{धर्मारण्ये रामदेव सर्वकिल्बिषनाशने}%॥ ३३ ॥

\twolineshloka
{ब्रह्महत्यादिपापानि सर्वस्तेयकृतानि च}
{परदारप्रसङ्गादि अभक्ष्यभक्षणादि वै}%॥ ३४ ॥

\twolineshloka
{अगम्यागमना यानि अस्पर्शस्पर्शनादि च}
{भस्मीभवन्ति लोकानां धर्मारण्यावगाहनात्}%॥ ३५ ॥

\twolineshloka
{ब्रह्मघ्नश्च कृतघ्नश्च बालघ्नोऽनृतभाषणः}
{स्त्रीगोघ्नश्चैव ग्रामघ्रो धर्मारण्ये विमुच्यते}%॥ ३६ ॥

\twolineshloka
{नातः परं पावनं हि पापिनां प्राणिनां भुवि}
{स्वर्ग्यं यशस्यमायुष्यं वाञ्छितार्थप्रदं शुभम्}%॥ ३७ ॥

\twolineshloka
{कामिनां कामदं क्षेत्रं यतीनां मुक्तिदायकम्}
{सिद्धानां सिद्धिदं प्रोक्तं धर्मारण्यं युगेयुगे}%॥ ३८ ॥

\uvacha{ब्रह्मोवाच}

\twolineshloka
{वसिष्ठवचनं श्रुत्वा रामो धर्मभृतां वरः}
{परं हर्षमनुप्राप्य हृदयानन्दकारकम्}%॥ ३९ ॥

\twolineshloka
{प्रोत्फुल्लहृदयो रामो रोमाचिन्ततनूरुहः}
{गमनाय मतिं चक्रे धर्मारण्ये शुभव्रतः}%॥ ४० ॥

\twolineshloka
{यस्मिन्कीटपतङ्गादिमानुषाः पशवस्तथा}
{त्रिरात्रसेवनेनैव मुच्यन्ते सर्वपातकैः}%॥ ४१ ॥

\twolineshloka
{कुशस्थली यथा काशी शूलपाणिश्च भैरवः}
{यथा वै मुक्तिदो राम धर्मारण्यं तथोत्तमम्}%॥ ४२ ॥

\twolineshloka
{ततो रामो महेष्वासो मुदा परमया युतः}
{प्रस्थितस्तीर्थयात्रायां सीतया भ्रातृभिः सह}%॥ ४३ ॥

\twolineshloka
{अनुजग्मुस्तदा रामं हनुमांश्च कपीश्वरः}
{कौशल्या च सुमित्रा च कैकेयी च मुदान्विता}%॥ ४४ ॥

\twolineshloka
{लक्ष्मणो लक्षणोपेतो भरतश्च महामतिः}
{शत्रुघ्नः सैन्यसहितोप्ययोध्यावासिनस्तथा}%॥ ४५ ॥

\twolineshloka
{प्रकृतयो नरव्याघ्र धर्मारण्ये विनिर्ययुः}
{अनुजग्मुस्तदा रामं मुदा परमया युताः}%॥ ४६ ॥

\twolineshloka
{तीर्थयात्राविधिं कर्तुं गृहात्प्रचलितो नृपः}
{वसिष्ठं स्वकुलाचार्यमिदमाह महीपते}%॥ ४७ ॥

\uvacha{श्रीराम उवाच}

\twolineshloka
{एतदाश्चर्यमतुलं किमादि द्वारकाभवत्}
{कियत्कालसमुत्पन्ना वसिष्ठेदं वदस्व मे}%॥ ४८ ॥

\uvacha{वसिष्ठ उवाच}

\twolineshloka
{न जानामि महाराज कियत्कालादभूदिदम्}
{लोमशो जाम्बवांश्चैव जानातीति च कारणम्}%॥ ४९ ॥

\twolineshloka
{शरीरे यत्कृतं पापं नानाजन्मान्तरेष्वपि}
{प्रायश्चितं हि सर्वेषामेतत्क्षेत्र परं स्मृतम्}%॥ ५० ॥

\twolineshloka
{श्रुत्वेति वचनं तस्य रामं ज्ञानवतां वरः}
{गन्तुं कृतमतिस्तीर्थं यात्राविधिमथाचरत्}%॥ ५१ ॥

\twolineshloka
{वसिष्ठं चाग्रतः कृत्वा महामाण्डलिकैर्नृपैः}
{पुनश्चरविधिं कृत्वा प्रस्थितश्चोत्तरां दिशम्}%॥ ५२ ॥

\twolineshloka
{वसिष्ठं चाग्रतः कृत्वा प्रतस्थे पश्चिमां दिशम्}
{ग्रामाद्ग्राममतिक्रम्य देशाद्देशं वनाद्वनम्}%॥ ५३ ॥

\twolineshloka
{विमुच्य निर्ययौ रामः ससैन्यः सपरिच्छदः}
{गजवाजिसहस्रौघै रथैर्यानैश्च कोटिभिः}%॥ ५४ ॥

\twolineshloka
{शिबिकाभिश्चासङ्ख्याभिः प्रययौ राघवस्तदा}
{गजारूढः प्रपश्यंश्च देशान्विविधसौहृदान्}%॥ ५५ ॥

\twolineshloka
{श्वेतातपत्रं विधृत्य चामरेण शुभेन च}
{वीजितश्च जनौघेन रामस्तत्र समभ्यगात्}%॥ ५६ ॥

\twolineshloka
{वादित्राणां स्वनैघोरैर्नृत्यगीतपुरःसरैः}
{स्तूयमानोपि सूतैश्च ययौ रामो मुदान्वितः}%॥ ५७ ॥

\twolineshloka
{दशमेऽहनि सम्प्राप्तं धर्मारण्यमनुत्तमम्}
{अदूरे हि ततो रामो दृष्ट्वा माण्डलिकं पुरम्}%॥ ५८ ॥

\twolineshloka
{तत्र स्थित्वा ससैन्यस्तु उवास निशि तां पुरीम्}
{श्रुत्वा तु निर्जनं क्षेत्रमुद्वसं च भयानकम्}%॥ ५९ ॥

\threelineshloka
{व्याघ्रसिंहाकुलं तत्र यक्षराक्षससेवितम्}
{श्रुत्वा जनमुखाद्रामो धर्मारण्यमरण्यकम्}
{तच्छ्रुत्वा रामदेवस्तु न चिन्ता क्रियतामिति}%॥ ६०

\onelineshloka
{तत्रस्थान्वणिजः शूरान्दक्षान्स्वव्यवसायके}%॥ ६१ ॥

\twolineshloka
{समर्थान्हि महाकायान्महाबलपराक्रमान्}
{समाहूय तदा काले वाक्यमेतदथाब्रवीत्}%॥ ६२ ॥

\twolineshloka
{शिबिकां सुसुवणां मे शीघ्रं वाहयताचिरम्}
{यथा क्षणेन चैकेन धर्मरण्यं व्रजाम्यहम्}%॥ ६३ ॥

\twolineshloka
{तत्र स्नात्वा च पीत्वा च सर्वपापात्प्रमुच्यते}
{एवं ते वणिजः सर्वै रामेण प्रेरितास्तदा}%॥ ६४ ॥

\twolineshloka
{तथेत्युक्त्वा च ते सर्वे ऊहुस्तच्छिबिकां तदा}
{क्षेत्रमध्ये यदा रामः प्रविष्टः सहसैनिकः}%॥ ६५ ॥

\twolineshloka
{तद्यानस्य गतिर्मन्दा सञ्जाता किल भारत}
{मन्दशब्दानि वाद्यानि मातङ्गा मन्दगामिनः}%॥ ६६ ॥

\twolineshloka
{हयाश्च तादृशा जाता रामो विस्मय मागतः}
{गुरुं पप्रच्छ विनयाद्वशिष्ठं मुनिपुङ्गवम्}%॥ ६७ ॥

\twolineshloka
{किमेतन्मन्दगतयश्चित्रं हृदि मुनीश्वर}
{त्रिकालज्ञो मुनिः प्राह धर्मक्षेत्रमुपागतम्}%॥ ६८ ॥

\twolineshloka
{तीर्थे पुरातने राम पादचारेण गम्यते}
{एवं कृते ततः पश्चात्सैन्यसौख्यं भविष्यति}%॥ ६९ ॥

\twolineshloka
{पादचारी ततौ रामः सैन्येन सह संयुतः}
{मधुवासनके ग्रामे प्राप्तः परमभावनः}%॥ ७० ॥

\twolineshloka
{गुरुणा चोक्तमार्गेण मातॄणां पूजनं कृतम्}
{नानोपहारैर्विविधैः प्रतिष्ठाविधिपूर्वकम्}%॥ ७३ ॥

\twolineshloka
{ततो रामो हरिक्षेत्रं सुवर्णादक्षिणे तटे}
{निरीक्ष्य यज्ञयोग्याश्च भूमीर्वै बहुशस्तथा}%॥ ७२ ॥

\twolineshloka
{कृतकृत्यं तदात्मानं मेने रामो रघूद्वहः}
{धर्मस्थानं निरीक्ष्याथ सुवर्णाक्षोत्तरे तटे}%॥ ७३ ॥

\twolineshloka
{सैन्यसङ्घं समुत्तीर्य्य बभ्राम क्षेत्रमध्यतः}
{तत्र तीर्थेषु सर्वेषु देवतायतनेषु च}%॥ ७४ ॥

\twolineshloka
{यथोक्तानि च कर्माणि रामश्चक्रे विधानतः}
{श्राद्धानि विधिवच्चक्रे श्रद्धया परया युतः}%॥ ७५ ॥

\twolineshloka
{स्थापयामास रामेशं तथा कामेश्वरं पुनः}
{स्थानाद्वायुप्रदेशे तु सुवर्णो भयतस्तटे}%॥ ७६ ॥

\twolineshloka
{कृत्वैवं कृतकृत्योऽभूद्रामो दशरथात्मजः}
{कृत्वा सर्वविधिं चैव सभायां समुपाविशत्}%॥ ७७ ॥

\twolineshloka
{तां निशां स नदीतीरे सुष्वाप रघुनन्दनः}
{ततोऽर्द्धरात्रे सञ्जाते रामो राजीवलोचनः}%॥ ७८ ॥

\twolineshloka
{जागृतस्तु तदा काल एकाकी धर्मवत्सलः}
{अश्रौषीच्च क्षणे तस्मिन्रामो नारीविरोदनम्}%॥ ७९ ॥

\twolineshloka
{निशायां करुणैर्वाक्यै रुदन्तीं कुररीमिव}
{चारैर्विलोकयामास रामस्तामतिसम्भ्रमात्}%॥ ८० ॥

\twolineshloka
{दृष्ट्वातिविह्वलां नारीं क्रन्दन्तीं करुणैः स्वरैः}
{पृष्टा सा दुःखिता नारी रामदूतैस्तदानघ}%॥ ८१ ॥

\uvacha{दूता ऊचुः}

\twolineshloka
{कासि त्वं सुभगे नारि देवी वा दानवी नु किम्}
{केन वा त्रासितासि त्वं मुष्टं केन धनं तव}%॥ ८२ ॥

\twolineshloka
{विकला दारुणाञ्छब्दानुद्गिरन्ती मुहुर्मुहुः}
{कथयस्व यथातथ्यं रामो राजाभिपृच्छति}%॥ ८३ ॥

\twolineshloka
{तयोक्तं स्वामिनं दूताः प्रेषयध्वं ममान्तिकम्}
{यथाहं मानसं दुःखं शान्त्यै तस्मै निवेदये}%॥ ८४

\onelineshloka
{तथेत्युक्त्वा ततो दूता राममागत्य चाब्रुवन्}%॥ ८५ ॥

॥इति श्रीस्कान्दे महापुराण एकाशीतिसाहस्र्यां संहितायां तृतीये ब्रह्मखण्डे पूर्वभागे धर्मारण्यमाहात्म्ये दूतागमनं नामैकत्रिंशोऽध्यायः॥३१॥

\dnsub{द्वात्रिंशोऽध्यायः --- सत्यमन्दिरस्थापनम्}\resetShloka

\uvacha{व्यास उवाच}

\twolineshloka
{ततश्च रामदूतास्ते नत्वा राममथाब्रुवन्}
{रामराम महाबाहो वरनारी शुभानना}%॥ १ ॥

\twolineshloka
{सुवस्त्रभूषाभरणां मृदुवाक्यपरायणाम्}
{एकाकिनीं क्रदमानाम दृष्ट्वा तां विस्मिता वयम्}%॥ २ ॥

\twolineshloka
{समीपवर्तिनो भूत्वा पृष्टा सा सुरसुन्दरी}
{का त्वं देवि वरारोहे देवी वा दानवी नु किम्}%॥ ३ ॥

\twolineshloka
{रामः पृच्छति देवि त्वां ब्रूहि सर्वं यथातथम्}
{तच्छ्रुत्वा वचनं रामा सोवाच मधुरं वचः}%॥ ४

\onelineshloka
{रामं प्रेषयत भद्रं वो मम दुःखापहं परम्}%॥ ५ ॥

\threelineshloka
{तदाकर्ण्य ततो रामः सम्भ्रमात्त्वरितो ययौ}
{दृष्ट्वा तां दुःखसन्तप्तां स्वयं दुःखमवाप सः}
{उवाच वचनं रामः कृताञ्जलिपुटस्तदा}%॥ ६ ॥

\uvacha{श्रीराम उवाच}

\twolineshloka
{का त्वं शुभे कस्य परिग्रहो वा केनावधूता विजने निरस्ता}
{मुष्टं धनं केन च तावकीनमाचक्ष्व मातः सकलं ममाग्रे}%॥ ७ ॥

\twolineshloka
{इत्युक्त्वा चातिदुःखार्तो रामो मतिमतां वरः}
{प्रणामं दण्डवच्चक्रे चक्रपाणिरिवापरः} %॥ ८ ॥

\twolineshloka
{तयाभिवन्दितो रामः प्रगम्य च पुनःपुनः}
{तुष्टया परया प्रीत्या स्तुतो मधुरया गिरा}%॥ ९ ॥

\twolineshloka
{परमात्मन्परेशान दुःखहारिन्सनातन}
{यदर्थमवतारस्ते तच्च कार्यं त्वया कृतम्}%॥ १० ॥

\twolineshloka
{रावणः कुम्भकर्णश्च शक्रजित्प्रमुखास्तथा}
{खरदूषणत्रिशिरोमारीचाक्षकुमारकाः}%॥ ११

\onelineshloka
{असङ्ख्या निर्जिता रौद्रा राक्षसाः समराङ्गणे}%॥ १२ ॥

\twolineshloka
{किं वच्मि लोकेश सुकीर्त्तिमद्य ते वेधास्त्वदीयाङ्गजपद्मसम्भवः}
{विश्वं निविष्टं च ततो ददर्श वटस्य पत्रे हि यथो वटो मतः}%॥ १३ ॥

\twolineshloka
{धन्यो दशरथो लोके कौशल्या जननी तव}
{ययोर्जातोसि गोविन्द जगदीश परः पुमान्}%॥ १४ ॥

\twolineshloka
{धन्यं च तत्कुलं राम यत्र त्वमागतः स्वयम्}
{धन्याऽयोध्यापुरी राम धन्यो लोकस्त्वदाश्रयः}%॥ १५ ॥

\twolineshloka
{धन्यः सोऽपि हि वाल्मीकिर्येन रामायणं कृतम्}
{कविना विप्रमुख्येभ्य आत्मबुद्ध्या ह्यनागतम्}%॥ १६

\onelineshloka
{त्वत्तोऽभवत्कुलं चेदं त्वया देव सुपावितम्}%॥ १७ ॥

\twolineshloka
{नरपतिरिति लोकैः स्मर्यते वैष्णवांशः स्वयमसि रमणीयैस्त्वं गुणैर्विष्णुरेव}
{किमपि भुवनकार्यं यद्विचिन्त्यावतीर्य तदिह घटयतस्ते वत्स निर्विघ्नमस्तु}%॥ १८ ॥

\twolineshloka
{स्तुत्वा वाचाथ रामं हि त्वयि नाथे नु साम्प्रतम्}
{शून्या वर्ते चिरं कालं यथा दोषस्तथैव हि}%॥ १९ ॥

\twolineshloka
{धर्मारण्यस्य क्षेत्रस्य विद्धि मामधिदेवताम्}
{वर्षाणि द्वादशेहैव जातानि दुःखि तास्म्यहम्}%॥ २० ॥

\twolineshloka
{निर्जनत्वं ममाद्य त्वमुद्धरस्व महामते}
{लोहासुरभयाद्राम विप्राः सर्वे दिशो दश}%॥ २१ ॥

\twolineshloka
{गताश्च वणिजः सर्वे यथास्थानं सुदुःखिताः}
{स दैत्यो घातितो राम देवैः सुरभयङ्करः}%॥ २२ ॥

\twolineshloka
{आक्रम्यात्र महामायो दुराधर्षो दुरत्ययः}
{न ते जनाः समायान्ति तद्भयादति शङ्किताः}%॥ २३ ॥

\twolineshloka
{अद्य वै द्वादश समाः शून्यागारमनाथवत्}
{यस्माच्च दीर्घिकायां मे स्नानदानोद्यतो जनः}%॥ २४ ॥

\twolineshloka
{राम तस्यां दीर्घिकायां निपतन्ति च शूकराः}
{यत्राङ्गना भर्तृयुता जलक्रीडापरायणाः}%॥ २५ ॥

\twolineshloka
{चिक्रीडुस्तत्र महिषा निपतन्ति जलाशये}
{यत्र स्थाने सुपुष्पाणां प्रकरः प्रचुरोऽभवत्}%॥ २६ ॥

\twolineshloka
{तद्रुद्धं कण्टकैर्वृक्षैः सिंहव्याघ्रसमाकुलैः}
{सञ्चिक्रीडुः कुमाराश्च यस्यां भूमौ निरन्तरम्}%॥ २७ ॥

\twolineshloka
{कुमार्यश्चित्रकाणां च तत्र क्रीडं ति हर्षिताः}
{अकुर्वन्वाडवा यत्र वेदगानं तिरन्तरम्}%॥ २८ ॥

\twolineshloka
{शिवानां तत्र फेत्काराः श्रूयन्तेऽतिभयङ्कराः}
{यत्र धूमोऽग्निहोत्राणां दृश्यते वै गृहेगृहे}%॥ २९ ॥

\twolineshloka
{तत्र दावाः सधूमाश्च दृश्यन्तेऽत्युल्बणा भृशम्}
{नृत्यन्ते नर्त्तका यत्र हर्षिता हि द्विजाग्रतः}%॥ ३० ॥

\twolineshloka
{तत्रैव भूतवेताला प्रेताः नृत्यन्ति मोहिताः}
{नृपा यत्र सभायां तु न्यषीदन्मन्त्रतत्पराः}%॥ ३१ ॥

\twolineshloka
{तस्मिन्स्थाने निषीदन्ति गवया ऋक्षशल्लकाः}
{आवासा यत्र दृश्यन्ते द्विजानां वणिजां तथा}%॥ ३२ ॥

\twolineshloka
{कुट्टिमप्रतिमा राम दृश्यन्तेत्र बिलानि वै}
{कोटराणीह वृक्षाणां गवाक्षाणीह सर्वतः}%॥ ३३ ॥

\twolineshloka
{चतुष्का यज्ञवेदिर्हि सोच्छ्राया ह्यभवत्पुरा}
{तेऽत्र वल्मीकनिचयैर्दृश्यन्ते परिवेष्टिताः}%॥ ३४ ॥

\twolineshloka
{एवंविधं निवासं मे विद्धि राम नृपोत्तम}
{शून्यं तु सर्वतो यस्मान्निवासाय द्विजा गताः}%॥ ३५ ॥

\twolineshloka
{तेन मे सुमहद्दुःखं तस्मात्त्राहि नरेश्वर}
{एतच्छ्रुत्वा वचो राम उवाच वदतां वरः}%॥ ३६ ॥

\uvacha{श्रीराम उवाच}

\twolineshloka
{न जाने तावकान्विप्रांश्चतुर्दिक्षु समाश्रितान्}
{न तेषां वेद्म्यहं सङ्ख्यां नामगोत्रे द्विजन्मनाम्}%॥ ३७ ॥

\twolineshloka
{यथा ज्ञातिर्यथा गोत्रं याथातथ्यं निवेदय}
{तत आनीय तान्सर्वान्स्वस्थाने वासयाम्यहम्}%॥ ३८ ॥

\uvacha{श्रीमातोवाच}

\twolineshloka
{ब्रह्मविष्णुमहेशैश्च स्थापिता ये नरेश्वर}
{अष्टादश सहस्राणि ब्राह्मणा वेदपारगाः}%॥ ३९ ॥

\twolineshloka
{त्रयीविद्यासु विख्याता लोकेऽस्मिन्नमितद्युते}
{चतुष्षष्टिकगोत्राणां वाडवा ये प्रतिष्ठिताः}%॥ ४० ॥

\twolineshloka
{श्रीमातादात्त्रयीविद्यां लोके सर्वे द्विजोत्तमाः}
{षट्त्रिंशच्च सहस्राणि वैश्या धर्मपरायणाः}%॥ ४१ ॥

\twolineshloka
{आर्यवृत्तास्तु विज्ञेया द्विजशुश्रूषणे रताः}
{बहुलार्को नृपो यत्र संज्ञया सह राजते}%॥ ४२ ॥

\twolineshloka
{कुमारावश्विनौ देवौ धनदो व्ययपूरकः}
{अधिष्ठात्री त्वहं राम नाम्ना भट्टारिका स्मृता}%॥ ४३ ॥

\uvacha{श्रीसूत उवाच}

\twolineshloka
{स्थानाचाराश्च ये केचित्कुलाचारास्तथैव च}
{श्रीमात्रा कथितं सर्वं रामस्याग्रे पुरातनम्}%॥ ४४ ॥

\twolineshloka
{तस्यास्तु वचनं श्रुत्वा रामो मुदमवाप ह}
{सत्यंसत्यं पुनः सत्यं सत्यं हि भाषितं त्वया}%॥ ४५ ॥

\twolineshloka
{यस्मात्सत्यं त्वया प्रोक्तं तन्नाम्ना नगरं शुभम्}
{वासयामि जगन्मातः सत्यमन्दिरमेव च}%॥ ४६

\onelineshloka
{त्रैलोक्ये ख्यातिमाप्नोतु सत्यमन्दिरमु त्तमम्}%॥ ४७ ॥

\twolineshloka
{एतदुक्त्वा ततो रामः सहस्रशतसङ्ख्यया}
{स्वभृत्यान्प्रेषयामास विप्रानयनहेतवे}%॥ ४८ ॥

\twolineshloka
{यस्मिन्देशे प्रदेशे वा वने वा सरि तस्तटे}
{पर्यन्ते वा यथास्थाने ग्रामे वा तत्रतत्र च}%॥ ४९ ॥

\twolineshloka
{धर्मारण्यनिवासाश्च याता यत्र द्विजोत्तमाः}
{अर्घपाद्यैः पूजयित्वा शीघ्रमानयतात्र तान्}%॥ ५०

\onelineshloka
{अहमत्र तदा भोक्ष्ये यदा द्रक्ष्ये द्विजोत्तमान्}%॥ ५१ ॥

\twolineshloka
{विमान्य च द्विजानेतानागमिष्यति यो नरः}
{स मे वध्यश्च दण्ड्यश्च निर्वास्यो विषयाद्बहिः}%॥ ५२ ॥

\twolineshloka
{तच्छ्रुत्वा दारुणं वाक्यं दुःसहं दुःप्रधर्षणम्}
{रामाज्ञाकारिणो दूता गताः सर्वे दिशो दश}%॥ ५३ ॥

\twolineshloka
{शोधिता वाडवाः सर्वे लब्धाः सर्वे सुहर्षिताः}
{यथोक्तेन विधानेन अर्घपाद्यैरपूजयन्}%॥ ५४ ॥

\twolineshloka
{स्तुतिं चक्रुश्च विधिवद्विनयाचारपूर्वकम्}
{आमन्त्र्य च द्विजान्सर्वान्रामवाक्यं प्रकाशयन्}%॥ ५५ ॥

\twolineshloka
{ततस्ते वाडवाः सर्वे द्विजाः सेवकसंयुताः}
{गमनायोद्यताः सर्वे वेदशास्त्रपरायणाः}%॥ ५६ ॥

\twolineshloka
{आगता रामपार्श्वं च बहुमानपुरःसराः}
{समागतान्द्विजान्दृष्ट्वा रोमाञ्चिततनूरुहः}%॥ ५७ ॥

\twolineshloka
{कृतकृत्यमिवात्मानं मेने दाशरथिर्नृपः}
{स सम्भ्रमात्समुत्थाय पदातिः प्रययौ पुरः}%॥ ५८ ॥

\twolineshloka
{करसम्पुटकं कृत्वा हर्षाश्रु प्रतिमुञ्चयन्}
{जानुभ्यामवनिं गत्वा इदं वचनमब्रवीत्}%॥ ५९ ॥

\twolineshloka
{विप्रप्रसादात्कमलावरोऽहं विप्रप्रसादाद्धरणीधरोऽहम्}
{विप्रप्रसादाज्जगतीपतिश्च विप्रप्रसादान्मम रामनाम}%॥ ६० ॥

\twolineshloka
{इत्येवमुक्ता रामेण वाड वास्ते प्रहर्षिताः}
{जयाशीर्भिः प्रपूज्याथ दीर्घायुरिति चाब्रुवन्}%॥ ६१ ॥

\twolineshloka
{आवर्जितास्ते रामेण पाद्यार्घ्यविष्टरादिभिः}
{स्तुतिं चकार विप्राणां दण्डवत्प्रणिपत्य च}%॥ ६२ ॥

\twolineshloka
{कृताञ्जलिपुटः स्थित्वा चक्रे पादाभिवन्दनम्}
{आसनानि विचित्राणि हैमान्याभरणानि च}%॥ ६३ ॥

\twolineshloka
{समर्पयामास ततो रामो दशरथात्मजः}
{अङ्गुलीयकवासांसि उपवीतानि कर्णकान्}%॥ ६४ ॥

\twolineshloka
{प्रददौ विप्रमुख्येभ्यो नानावर्णाश्च धेनवः}
{एकैकशत सङ्ख्याका घटोध्नीश्च सवत्सकाः}%॥ ६५ ॥

\twolineshloka
{सवस्त्रा बद्धघण्टाश्च हेमशृङ्गविभूषिताः}
{रूप्यखुरास्ताम्रपृष्ठीः कांस्यपात्रसमन्विताः}%॥ ६६ ॥

॥इति श्रीस्कान्दे महापुराण एकाशीतिसाहस्र्यां संहितायां तृतीये ब्रह्मखण्डे पूर्वभागे धर्मारण्यमाहात्म्ये ब्रह्मनारदसंवादे सत्यमन्दिरस्थापन वर्णनोनाम द्वात्रिंशोऽध्यायः॥३२॥

\dnsub{त्रयस्त्रिंशोऽध्यायः --- श्रीरामचन्द्रस्य पुरप्रत्यागमनवर्णनम्}\resetShloka

\uvacha{राम उवाच}

\twolineshloka
{जीर्णोद्धारं करिष्यामि श्रीमातुर्वचनादहम्}
{आज्ञा प्रदीयतां मह्यं यथादानं ददामि वः}%॥ १ ॥

\twolineshloka
{पात्रे दानं प्रदातव्यं कृत्वा यज्ञवरं द्विजाः}
{नापात्रे दीयते किञ्चिद्दत्तं न तु सुखावहम्}%॥ २ ॥

\twolineshloka
{सुपात्रं नौरिव सदा तारयेदुभयोरपि}
{लोहपिण्डोपमं ज्ञेयं कुपात्रं भञ्जनात्मकम्}%॥ ३ ॥

\twolineshloka
{जातिमात्रेण विप्रत्वं जायते न हि भो द्विजाः}
{क्रिया बलवती लोके क्रियाहीने कुतः फलम्}%॥ ४ ॥

\twolineshloka
{पूज्यास्तस्मात्पूज्यतमा ब्राह्मणाः सत्यवादिनः}
{यज्ञकार्ये समुत्पन्ने कृपां कुर्वन्तु सर्वदा}%॥ ५ ॥

\uvacha{ब्रह्मोवाच}

\twolineshloka
{ततस्तु मिलिताः सर्वे विमृश्य च परस्परम्}
{केचिदूचुस्तदा रामं वयं शिलोञ्छजीविकाः}%॥ ६ ॥

\twolineshloka
{सन्तोषं परमास्थाय स्थिता धर्मपरायणाः}
{प्रतिग्रहप्रयोगेण न चास्माकं प्रयोजनम्}%॥ ७ ॥

\twolineshloka
{दशसूनासमश्चक्री दशचक्रिसमो ध्वजः}
{दशध्वजसमा वेश्या दशवेश्यासमो नृपः}%॥ ८ ॥

\twolineshloka
{राजप्रतिग्रहो घोरो राम सत्यं न संशयः}
{तस्माद्वयं न चेच्छामः प्रतिग्रहं भया वहम्}%॥ ९ ॥

\twolineshloka
{एकाहिका द्विजाः केचित्केचित्स्वामृतवृत्तयः}
{कुम्भीधान्या द्विजाः केचित्केचित्षट्कर्मतत्पराः}%॥ १० ॥

\twolineshloka
{त्रिमूर्तिस्थापिताः सर्वे पृथग्भावाः पृथग्गुणाः}
{केचिदेवं वदन्ति स्म त्रिमूर्त्याज्ञां विना वयम्}%॥ ११ ॥

\twolineshloka
{प्रतिग्रहस्य स्वीकारं कथं कुर्याम ह द्विजाः}
{न ताम्बूलं स्त्रीकृतं नो ह्यद्मो दानेन भषितम्}%॥ १२ ॥

\twolineshloka
{विमृश्य स तदा रामो वसिष्ठेन महात्मना}
{ब्रह्मविष्णुशिवादीनां सस्मार गुरुणा सह}

\twolineshloka
{स्मृतमात्रास्ततो देवास्तं देशं समुपागमन्}
{सूर्यकोटिप्रतीकाशीवमानावलिसंवृताः}%॥ १४}

\twolineshloka
{रामेण ते यथान्यायं पूजिताः परया मुदा}%॥
{निवेदितं तु तत्सर्वं रामेणातिसुबुद्धिना}%॥ १५ ॥

\twolineshloka
{अधिदेव्या वचनतो जीर्णोद्धारं करोम्यहम्}
{धर्मारण्ये हरिक्षेत्रे धर्मकूपसमीपतः}%॥ १६ ॥

\twolineshloka
{ततस्ते वाडवाः सर्वे त्रिमूर्त्तीः प्रणिपत्य च}
{महता हर्षवृन्देन पूर्णाः प्राप्तमनोरथाः}%॥ १७ ॥

\twolineshloka
{अर्घ्यपाद्यादिविधिना श्रद्धया तानपूजयन्}
{क्षणं विश्रम्य ते देवा ब्रह्मविष्णुशिवादयः}%॥ १८

\onelineshloka
{ऊचू रामं महाशक्तिं विनयात्कृतसम्पुटम्}%॥ १९ ॥

\uvacha{देवा ऊचुः}

\twolineshloka
{देवद्रुहस्त्वया राम ये हता रावणादयः}
{तेन तुष्टा वयं सर्वे भानुवंशविभूषण}%॥ २०

\onelineshloka
{उद्धरस्व महास्थानं महतीं कीर्तिमाप्नुहि}%॥ २१ ॥

\twolineshloka
{लब्ध्वा स तेषामाज्ञां तु प्रीतो दशरथात्मजः}
{जीर्णोद्धारेऽनन्तगुणं फलमिच्छन्निलापतिः}%॥ २२ ॥

\twolineshloka
{देवानां सन्निधौ तेषां कार्यारम्भमथाकरोत्}
{स्थण्डिलं पूर्वतः कृत्वा महागिरि समं शुभम्}%॥ २३ ॥

\twolineshloka
{तस्योपरि बहिःशाला गृहशाला ह्यनेकशः}
{ब्रह्मशालाश्च बहुशो निर्ममे शोभनाकृतीः}%॥ २४ ॥

\twolineshloka
{निधानैश्च समायुक्ता गृहोपकरणै र्वृताः}
{सुवर्णकोटिसम्पूर्णा रसवस्त्रादिपूरिताः}%॥ २५ ॥

\twolineshloka
{धनधान्यसमृद्धाश्च सर्वधातुयुतास्तथा}
{एतत्सर्वं कारयित्वा ब्राह्मणेभ्यस्तदा ददौ}%॥ २६ ॥

\twolineshloka
{एकैकशो दशदश ददौ धेनूः पयस्विनीः}
{चत्वारिंशच्छतं प्रादाद्ग्रामाणां चतुराधिकम्}%॥ २७ ॥

\twolineshloka
{त्रैविद्यद्विजविप्रेभ्यो रामो दशरथात्मजः}
{काजेशेन त्रयेणैव स्थापिता द्विजसत्तमाः}%॥ २८ ॥

\twolineshloka
{तस्मात्त्रयीविद्य इति ख्यातिर्लोके बभूव ह}
{एवंविधं द्विजेभ्यः स दत्त्वा दानं महाद्भुतम्}%॥ २९ ॥

\twolineshloka
{आत्मानं चापि मेने स कृतकृत्यं नरेश्वरः}
{ब्रह्मणा स्थापिताः पूर्वं विष्णुना शङ्करेण ये}%॥ ३० ॥

\twolineshloka
{ते पूजिता राघवेण जीर्णोद्धारे कृते सति}
{षट्त्रिंशच्च सहस्राणि गोभुजा ये वणिग्वराः}%॥ ३१ ॥

\twolineshloka
{शुश्रूषार्थं प्रदत्ता वै देवैर्हरिहरादिभिः}
{सन्तुष्टेन तु शर्वेण तेभ्यो दत्तं तु चेत नम्}%॥ ३२ ॥

\twolineshloka
{श्वेताश्वचामरौ दत्तौ खङ्गं दत्तं सुनिर्मलम्}
{तदा प्रबोधितास्ते च द्विजशुश्रूषणाय वै}%॥ ३३ ॥

\twolineshloka
{विवाहादौ सदा भाव्यं चामरै मङ्गलं वरम्}
{खङ्गं शुभं तदा धार्य्यं मम चिह्नं करे स्थितम्}%॥ ३४ ॥

\twolineshloka
{गुरुपूजा सदा कार्या कुलदेव्याः पुनःपुनः}
{वृद्ध्यागमेषु प्राप्तेषु वृद्धि दायकदक्षिणा}%॥ ३५ ॥

\twolineshloka
{एकादश्यां शनेर्वारे दानं देयं द्विजन्मने}
{प्रदेयं बालवृद्धेभ्यो मम रामस्य शासनात्}%॥ ३६ ॥

\twolineshloka
{मण्डलेषु च ये शुद्धा वणिग्वृत्तिरताः पराः}
{सपादलक्षास्ते दत्ता रामशासनपालकाः}%॥ ३७ ॥

\twolineshloka
{माण्डलीकास्तु ते ज्ञेया राजानो मण्डलेश्वराः}
{द्विज शुश्रूषणे दत्ता रामेण वणिजां वराः}%॥ ३८ ॥

\twolineshloka
{चामरद्वितयं रामो दत्तवान्खड्गमेव च}
{कुलस्य स्वामिनं सूर्यं प्रतिष्ठाविधिपूर्वकम्}%॥ ३९ ॥

\twolineshloka
{ब्रह्माणं स्थापयामास चतुर्वेदसमन्वितम्}
{श्रीमातरं महाशक्तिं शून्यस्वामिहरिं तथा}%॥ ४० ॥

\twolineshloka
{विघ्नापध्वंसनार्थाय दक्षिणद्वारसंस्थितम्}
{गणं संस्थापयामास तथान्याश्चैव देवताः}%॥ ४१ ॥

\twolineshloka
{कारितास्तेन वीरेण प्रासादाः सप्तभूमिकाः}
{यत्किं चित्कुरुते कार्यं शुभं माङ्गल्यरूपकम्}%॥ ४२ ॥

\twolineshloka
{पुत्रे जाते जातके वान्नाशने मुण्डनेऽपि वा}
{लक्षहोमे कोटिहोमे तथा यज्ञक्रियासु च}%॥ ४३ ॥

\twolineshloka
{वास्तुपूजाग्रहशान्त्योः प्राप्ते चैव महोत्सवे}
{यत्किञ्चित्कुरुते दानं द्रव्यं वा धान्यमुत्तमम्}%॥ ४४ ॥

\twolineshloka
{वस्त्रं वा धेनवो नाथ हेम रूप्यं तथैव च}
{विप्राणामथ शूद्राणां दीनानाथान्धकेषु च}%॥ ४५ ॥

\twolineshloka
{प्रथमं बकुलार्कस्य श्रीमातुश्चैव मानवः}
{भागं दद्याच्च निर्विघ्नकार्यसिद्ध्यै निरन्तरम्}%॥ ४६ ॥

\twolineshloka
{वचनं मे समुल्लङ्घ्य कुरुते योऽन्यथा नरः}
{तस्य तत्कर्मणो विघ्नं भविष्यति न संशयः}%॥ ४७ ॥

\twolineshloka
{एवमुक्त्वा ततो रामः प्रहृष्टेनान्तरात्मना}
{देवानामथ वापीश्च प्राकारांस्तु सुशोभनान्}%॥ ४८ ॥

\twolineshloka
{दुर्गोपकरणैर्युक्तान्प्रतोलीश्च सुविस्तृताः}
{निर्ममे चैव कुण्डानि सरांसि सरसीस्तथा}%॥ ४९ ॥

\twolineshloka
{धर्मवापीश्च कूपांश्च तथान्यान्देवनिर्मितान्}
{एतत्सर्वं च विस्तार्य धर्मारण्ये मनोरमे}%॥ ५० ॥

\twolineshloka
{ददौ त्रैविद्यमुख्येभ्यः श्रद्धया परया पुनः}
{ताम्रपट्टस्थितं रामशासनं लोपयेत्तु यः}%॥ ५१ ॥

\twolineshloka
{पूर्वजास्तस्य नरके पतन्त्यग्रे न सन्ततिः}
{वायुपुत्रं समाहूय ततो रामोऽब्रवीद्वचः}%॥ ५२ ॥

\twolineshloka
{वायुपुत्र महावीर तव पूजा भविष्यति}
{अस्य क्षेत्रस्य रक्षायै त्वमत्र स्थितिमाचर}%॥ ५३ ॥

\twolineshloka
{आञ्जनेयस्तु तद्वाक्यं प्रणम्य शिरसादधौ}
{जीर्णोद्धारं तदा कृत्वा कृतकृत्यो बभूव ह}%॥ ५४ ॥

\twolineshloka
{श्रीमातरं तदाभ्यर्च्य प्रसन्नेनान्तरात्मना}
{श्रीमातरं नमस्कृत्य तीर्थान्यन्यानि राघवः}%॥ ५५

\onelineshloka
{तेऽपि देवाः स्वकं स्थानं ययुर्बह्मपुरोगमाः}%॥ ५६ ॥

\twolineshloka
{दत्त्वाशिषं तु रामाय वाञ्छितं ते भविष्यति}
{रम्यं कृतं त्वया राम विप्राणां स्थापनादिकम्}%॥ ५७ ॥

\twolineshloka
{अस्माकमपि वात्सल्यं कृतं पुण्यवता त्वया}
{इति स्तुवन्तस्ते देवाः स्वानि स्थानानि भेजिरे}%॥ ५८ ॥
॥इति श्रीस्कान्दे महापुराण एकाशीतिसाहस्र्यां संहितायां तृतीये ब्रह्मखण्डे पूर्वार्धे धर्मारण्यमाहात्म्ये श्रीरामचन्द्रस्य पुरप्रत्यागमनवर्णनं नाम त्रयस्त्रिंशोऽध्यायः॥३३॥

\dnsub{चतुस्त्रिंशोऽध्यायः --- श्रीरामेण ब्राह्मणेभ्यः शासनपट्टप्रदानवर्णनम्}\resetShloka

\uvacha{व्यास उवाच}

\twolineshloka
{एवं रामेण धर्मज्ञ जीर्णोद्धारः पुरा कृतः}
{द्विजानां च हितार्थाय श्रीमातुर्वचनेन च}%॥ १ ॥

\uvacha{युधिष्ठिर उवाच}

\twolineshloka
{कीदृशं शासनं ब्रह्मन्रामेण लिखितं पुरा}
{कथयस्व प्रसादेन त्रेतायां सत्यमन्दिरे}%॥ २ ॥

\uvacha{व्यास उवाच}

\twolineshloka
{धर्मारण्ये वरे दिव्ये बकुलार्के स्वधिष्ठिते}
{शून्यस्वामिनि विप्रेन्द्र स्थिते नारायणे प्रभौ}%॥ ३ ॥

\twolineshloka
{रक्षणाधिपतौ देवे सर्वज्ञे गुणनायके}
{भवसागर मग्नानां तारिणी यत्र योगिनी}%॥ ४ ॥

\twolineshloka
{शासनं तत्र रामस्य राघवस्य च नामतः}
{शृणु ताम्राश्रयं तत्र लिखितं धर्मशास्त्रतः}%॥ ५ ॥

\twolineshloka
{महाश्चर्यकरं तच्च ह्यनेकयुगसंस्थितम्}
{सर्वो धातुः क्षयं याति सुवर्णं क्षयमेति च}%॥ ६ ॥

\twolineshloka
{प्रत्यक्षं दृश्यते पुत्र द्विजशासनमक्षयम्}
{अविनाशो हि ताम्रस्य कारणं तत्र विद्यते}%॥ ७ ॥

\twolineshloka
{वेदोक्तं सकलं यस्माद्विष्णुरेव हि कथ्यते}
{पुराणेषु च वेदेषु धर्मशास्त्रेषु भारत}%॥ ८ ॥

\twolineshloka
{सर्वत्र गीयते विष्णुर्नाना भावसमाश्रयः}
{नानादेशेषु धर्मेषु नानाधर्मनिषेविभिः}%॥ ९ ॥

\twolineshloka
{नानाभेदैस्तु सर्वत्र विष्णुरेवेति चिन्त्यते}
{अवतीर्णः स वै साक्षात्पुराणपुरुषो त्तमः}%॥ १० ॥

\twolineshloka
{देववैरिविनाशाय धर्मसंरक्षणाय च}
{तेनेदं शासनं दत्तमविनाशात्मकं सुत}%॥ ११ ॥

\twolineshloka
{यस्य प्रतापादृषद(य)स्तारिता जलमध्यतः}
{वानरैर्वेष्टिता लङ्का हेलया राक्षसा हताः}%॥ १२ ॥

\twolineshloka
{मुनिपुत्रं मृतं रामो यमलोकादुपानयत्}
{दुन्दुभिर्निहतो येन कबन्धोऽभिहतस्तथा}%॥ १३ ॥

\twolineshloka
{निहता ताडका चैव सप्तताला विभेदिताः}
{खरश्च दूषणश्चैव त्रिशिराश्च महासुरः}%॥ १४ ॥

\twolineshloka
{चतुर्दशसहस्राणि जवेन निहता रणे}
{तेनेदं शासनं दत्तमक्षयं न कथं भवेत्}%॥ १५ ॥

\twolineshloka
{स्ववंशवर्णनं तत्र लिखित्वा स्वयमेव तु}
{देशकालादिकं सर्वं लिलेख विधिपूर्वकम्}%॥ १६ ॥

\twolineshloka
{स्वमुद्राचिह्नितं तत्र त्रैविद्येभ्यस्तथा ददौ}
{चतुश्चत्वारिंशवर्षो रामो दशरथात्मजः}%॥ १७ ॥

\twolineshloka
{तस्मिन्काले महाश्चर्यं सन्दत्तं किल भारत}
{तत्र स्वर्णोपमं चापि रौप्योपमम थापि च}%॥ १८ ॥

\twolineshloka
{उवाह सलिलं तीर्थे देवर्षिपितृतृप्तिदम्}
{स्ववंशनायकस्याग्रे सूर्येण कृतमेव तत्}%॥ १९ ॥

\twolineshloka
{तद्दृष्ट्वा महदाश्चर्यं रामो विष्णुं प्रपूज्य च}
{रामलेखविचित्रैस्तु लिखितं धर्मशासनम्}%॥ २० ॥

\twolineshloka
{यद्दृष्ट्वाथ द्विजाः सर्वे संसारभयबन्धनम्}
{कुर्वते नैव यस्माच्च तस्मान्निखिलरक्षकम्}%॥ २१ ॥

\twolineshloka
{ये पापिष्ठा दुराचारा मित्रद्रोहरताश्च ये}
{तेषां प्रबोधनार्थाय प्रसिद्धिमकरोत्पुरा}%॥ २२ ॥

\twolineshloka
{रामलेखविचित्रैस्तु विचित्रे ताम्रपट्टके}
{वाक्यानीमानि श्रूयन्ते शासने किल नारद}%॥ २३ ॥

\twolineshloka
{आस्फोटयन्ति पितरः कथयन्ति पितामहाः}
{भूमिदोऽस्मत्कुले जातः सोऽस्मान्सन्तारयिष्यति}%॥ २४ ॥

\twolineshloka
{बहुभिर्बहुधा भुक्ता राजभिः पृथिवी त्वियम्}
{यस्ययस्य यदा भूमिस्तस्यतस्य तदा फलम्}%॥ २५ ॥

\twolineshloka
{षष्टिवर्षसहस्राणि स्वर्गे वसति भूमिदः}
{आच्छेत्ता चानुमन्ता च तान्येव नरकं व्रजेत्}%॥ २६ ॥

\twolineshloka
{सन्दंशैस्तुद्यमानस्तु मुद्गरैर्विनिहत्य च}
{पाशैः सुबध्यमानस्तु रोरवीति महास्वरम्}%॥ २७ ॥

\twolineshloka
{ताड्यमानः शिरे दण्डैः समालिङ्ग्य विभावसुम्}
{क्षुरिकया छिद्यमानो रोरवीति महास्वनम्}%॥ २८ ॥

\twolineshloka
{यमदूतैर्महाघोरैर्ब्रह्मवृत्तिविलोपकः}
{एवंविधैर्महादुष्टैः पीड्यन्ते ते महागणैः}%॥ २९ ॥

\twolineshloka
{ततस्तिर्यक्त्वमाप्नोति योनिं वा राक्षसीं शुनीम्}
{व्यालीं शृगालीं पैशाचीं महाभूतभयङ्करीम्}%॥ ३० ॥

\twolineshloka
{भूमेरङ्गुलहर्ता हि स कथं पापमाचरेत्}
{भूमेरङ्गुलदाता च स कथं पुण्यमाचरेत्}%॥ ३१ ॥

\twolineshloka
{अश्वमेधसहस्राणां राजसूयशतस्य च}
{कन्याशतप्रदानस्य फलं प्राप्नोति भूमिदः}%॥ ३२ ॥

\twolineshloka
{आयुर्यशः सुखं प्रज्ञा धर्मो धान्यं धनं जयः}
{सन्तानं वर्द्धते नित्यं भूमिदः सुखमश्मुते}%॥ ३३ ॥

\threelineshloka
{भूमेरङ्गुलमेकं तु ये हरन्ति खला नराः}
{वन्ध्याटवीष्वतोयासु शुष्ककोटरवासिनः}
{कृष्णसर्पाः प्रजायन्ते दत्तदायापहारकाः}%॥ ३४ ॥

\twolineshloka
{तडागानां सहस्रेण अश्वमेधशतेन वा}
{गवां कोटिप्रदानेन भूमिहर्त्ता विशुध्यति}%॥ ३५ ॥

\twolineshloka
{यानीह दत्तानि पुनर्धनानि दानानि धर्मार्थयशस्कराणि}
{औदार्यतो विप्रनिवेदितानि को नाम साधुः पुनराददीत}%॥ ३६ ॥

\twolineshloka
{चलदलदललीलाचञ्चले जीवलोके तृणलवलघुसारे सर्वसंसारसौख्ये}
{अपहरति दुराशः शासनं ब्राह्मणानां नरकगहनगर्त्तावर्तपातोत्सुको यः}%॥ ३७ ॥

\twolineshloka
{ये पास्यन्ति महीभुजः क्षितिमिमां यास्यन्ति भुक्त्वाखिलां नो याता न तु याति यास्यति न वा केनापि सार्द्धं धरा}
{यत्किञ्चिद्भुवि तद्विनाशि सकलं कीर्तिः परं स्थायिनी त्वेवं वै वसुधापि यैरुपकृता लोप्या न सत्कीर्तयः}%॥ ३८ ॥

\twolineshloka
{एकैव भगिनी लोके सर्वेषामेव भूभुजाम्}
{न भोज्या न करग्राह्या विप्रदत्ता वसुन्धरा}%॥ ३९ ॥

\twolineshloka
{दत्त्वा भूमिं भाविनः पार्थिवेशान्भूयोभूयो याचते रामचन्द्रः}
{सामान्योऽयं धर्मसेतुर्नृपाणां स्वे स्वे काले पालनीयो भवद्भिः}%॥ ४० ॥

\twolineshloka
{अस्मिन्वंशे क्षितौ कोपि राजा यदि भविष्यति}
{तस्याहं करलग्नोस्मि मद्दत्तं यदि पाल्यते}%॥ ४१ ॥

\twolineshloka
{लिखित्वा शासनं रामश्चातुर्वेद्यद्विजोत्तमान्}
{सम्पूज्य प्रददौ धीमान्वसिष्ठस्य च सन्निधौ}%॥ ४२ ॥

\twolineshloka
{ते वाडवा गृहीत्वा तं पट्टं रामाज्ञया शुभम्}
{ताम्रं हैमाक्षरयुतं धर्म्यं धर्मविभूषणम्}%॥ ४३ ॥

\twolineshloka
{पूजार्थं भक्तिकामार्थास्तद्रक्षणमकुर्वत}
{चन्दनेन च दिव्येन पुष्पेण च सुगन्धिना}%॥ ४४ ॥

\twolineshloka
{तथा सुवर्णपुष्पेण रूप्यपुष्पेण वा पुनः}
{अहन्यहनि पूजां ते कुर्वते वाडवाः शुभाम्}%॥ ४५ ॥

\twolineshloka
{तदग्रे दीपकं चैव घृतेन विमलेन हि}
{सप्तवर्तियुतं राजन्नर्घ्यं प्रकुर्वते द्विजाः}%॥ ४६ ॥

\twolineshloka
{नैवेद्यं कुर्वते नित्यं भक्तिपूर्वं द्विजोत्तमाः}
{रामरामेति रामेति मन्त्रमप्युच्चरन्ति हि}%॥ ४७ ॥

\twolineshloka
{अशने शयने पाने गमने चोपवेशने}
{सुखे वाप्यथवा दुःखे राममन्त्रं समुच्चरेत्}%॥ ४८ ॥

\twolineshloka
{न तस्य दुःखदौर्भाग्यं नाधिव्याधिभयं भवेत्}
{आयुः श्रियं बलं तस्य वर्द्धयन्ति दिने दिने}%॥ ४९ ॥

\twolineshloka
{रामेति नाम्ना मुच्येत पापाद्वै दारुणादपि}
{नरकं नहि गच्छेत गतिं प्राप्नोति शाश्वतीम्}%॥ ५० ॥

\uvacha{व्यास उवाच}

\twolineshloka
{इति कृत्वा ततो रामः कृतकृत्यममन्यत}
{प्रदक्षिणीकृत्य तदा प्रणम्य च द्विजान्बहून्}%॥ ५१ ॥

\twolineshloka
{दत्त्वा दानं भूरितरं गवाश्वमहिषीरथम्}
{ततः सर्वान्निजांस्तांश्च वाक्यमेतदुवाच ह}%॥ ५२ ॥

\twolineshloka
{अत्रैव स्थीयतां सर्वैर्यावच्चन्द्रदिवाकरौ}
{यावन्मेरुर्महीपृष्ठे सागराः सप्त एव च}%॥ ५३ ॥

\twolineshloka
{तावदत्रैव स्थातव्यं भवद्भिर्हि न संशयः}
{यदा हि शासनं विप्रा न मन्यन्ते नृपा भुवि}%॥ ५४ ॥

\twolineshloka
{अथवा वणिजः शूरा मदमायाविमोहिताः}
{मदाज्ञां न प्रकुर्वन्ति मन्यन्ते वा न ते जनाः}%॥ ५५ ॥

\twolineshloka
{तदा वै वायुपुत्रस्य स्मरणं क्रियतां द्विजाः}
{स्मृतमात्रो हनूमान्वै समागत्य करिष्यति}%॥ ५६ ॥

\twolineshloka
{सहसा भस्म तान्सत्यं वचनान्मे न संशयः}
{य इदं शासनं रम्यं पालयिष्यति भूपतिः}%॥ ५७ ॥

\twolineshloka
{वायुपुत्रः सदा तस्य सौख्यमृद्धिं प्रदास्यति}
{ददाति पुत्रान्पौत्रांश्च साध्वीं पत्नीं यशो जयम्}%॥ ५८ ॥

\twolineshloka
{इत्येवं कथयित्वा च हनुमन्तं प्रबोध्य च}
{निवर्तितो रामदेवः ससैन्यः सपरिच्छदः}%॥ ५९ ॥

\threelineshloka
{वादित्राणां स्वनैर्विष्वक्सूच्यमानशुभागमः}
{श्वेतातपत्रयुक्तोऽसौ चामरैर्वी जितो नरैः}
{अयोध्यां नगरीं प्राप्य चिरं राज्यं चकार ह}%॥ ६० ॥
॥इति श्रीस्कान्दे महापुराण एकाशीतिसाहस्र्यां संहितायां तृतीये ब्रह्मखण्डे पूर्वभागे धर्मारण्यमाहात्म्ये ब्रह्मनारदसंवादे श्रीरामेण ब्राह्मणेभ्यः शासनपट्टप्रदानवर्णनं नाम चतुस्त्रिंशोऽध्यायः॥३४॥

\dnsub{पञ्चत्रिंशोऽध्यायः --- श्रीरामरुद्रकृतधर्मारण्यतीर्थक्षेत्रजीर्णोद्धारवर्णनम्}\resetShloka

\uvacha{नारद उवाच}

\twolineshloka
{भगवन्देवदेवेश सृष्टिसंहारकारक}
{गुणातीतो गुणैर्युक्तो मुक्तीनां साधनं परम्}%॥ १ ॥

\twolineshloka
{संस्थाप्य वेदभवनं विधिवद्द्विज सत्तमान्}
{किं चक्रे रघुनाथस्तु भूयोऽयोध्यां गतस्तदा}%॥ २ ॥

\uvacha{ब्रह्मोवाच}

\twolineshloka
{स्वस्थाने ब्राह्मणास्तत्र कानि कर्माणि चक्रिरे}
{इष्टापूर्तरताः शान्ताः प्रतिग्रहपराङ्मुखाः}%॥ ३ ॥

\twolineshloka
{राज्यं चक्रुर्वनस्यास्य पुरोधा द्विजसत्तमः}
{उवाच रामपुरतस्तीर्थमाहात्म्यमुत्तमम्}%॥ ४ ॥

\twolineshloka
{प्रयागस्य च माहात्म्यं त्रिवेणीफलमुत्तमम्}
{प्रयागतीर्थमहिमा शुक्लतीर्थस्य चैव हि}%॥ ५ ॥

\twolineshloka
{सिद्धक्षेत्रस्य काश्याश्च गङ्गाया महिमा तथा}
{वसिष्ठः कथया मास तीर्थान्यन्यानि नारद}%॥ ६ ॥

\twolineshloka
{धर्मारण्यसुवर्णाया हरिक्षेत्रस्य तस्य च}
{स्नानदानादिकं सर्वं वाराणस्या यवाधिकम्}%॥ ७ ॥

\twolineshloka
{एतच्छ्रुत्वा रामदेवः स चमत्कृतमानसः}
{धर्मारण्ये पुनर्यात्रां कर्त्तुकामः समभ्यगात्}%॥ ८ ॥

\twolineshloka
{सीतया सह धर्मज्ञो गुरुसैन्यपुरःसरः}
{लक्ष्मणेन सह भ्रात्रा भरतेन सहायवान्}%॥ ९ ॥

\twolineshloka
{शत्रुघ्नेन परिवृतो गतो मोहेरके पुरे}
{तत्र गत्वा वसिष्ठं तु पृच्छतेऽसौ महामनाः}%॥ १० ॥

\uvacha{राम उवाच}

\twolineshloka
{धर्मारण्ये महाक्षेत्रे किं कर्त्तव्यं द्विजोत्तम}
{दानं वा नियमो वाथ स्नानं वा तप उत्तमम्}%॥ ११ ॥

\twolineshloka
{ध्यानं वाथ क्रतुं वाथ होमं वा जपमुत्तमम्}
{दानं वा नियमं वाथ स्नानं वा तप उत्तमम्}%॥ १२ ॥

\twolineshloka
{येन वै क्रियमाणेन तीर्थेऽस्मिन्द्विजसत्तम}
{ब्रह्महत्यादिपापेभ्यो मुच्यते तद्ब्रवीहि मे}%॥ १३ ॥

\uvacha{वसिष्ठ उवाच}

\twolineshloka
{यज्ञं कुरु महाभाग धर्मारण्ये त्वमुत्तमम्}
{दिनेदिने कोटिगुणं यावद्वर्षशतं भवेत्}%॥ १४ ॥

\twolineshloka
{तच्छ्रुत्वा चैव गुरुतो यज्ञारम्भं चकार सः}
{तस्मिन्नवसरे सीता रामं व्यज्ञापयन्मुदा}%॥ १५ ॥

\twolineshloka
{स्वामिन्पूर्वं त्वया विप्रा वृता ये वेदपारगाः}
{ब्रह्मविष्णुमहेशेन निर्मिता ये पुरा द्विजाः}%॥ १६ ॥

\twolineshloka
{कृते त्रेतायुगे चैव धर्मारण्यनिवासिनः}
{विप्रांस्तान्वै वृणुष्व त्वं तैरेव साधकोऽध्वरः}%॥ १७ ॥

\twolineshloka
{तच्छ्रुत्वा रामदेवेन आहूता ब्राह्मणास्तदा}
{स्थापिताश्च यथापूर्वमस्मिन्मोहे रके पुरे}%॥ १८ ॥

\twolineshloka
{तैस्त्वष्टादशसङ्ख्याकैस्त्रैविद्यैर्मेहिवाडवैः}
{यज्ञं चकार विधिवत्तैरेवायतबुद्धिभिः}%॥ १९ ॥

\twolineshloka
{कुशिकः कौशिको वत्स उपमन्युश्च काश्यपः}
{कृष्णात्रेयो भरद्वाजो धारिणः शौनको वरः}%॥ २० ॥

\twolineshloka
{माण्डव्यो भार्गवः पैङ्ग्यो वात्स्यो लौगाक्ष एव च}
{गाङ्गायनोथ गाङ्गेयः शुनकः शौनकस्तथा}%॥ २१ ॥

\uvacha{ब्रह्मोवाच}

\twolineshloka
{एभिर्विप्रैः क्रतुं रामः समाप्य विधिवन्नृपः}
{चकारावभृथं रामो विप्रान्सम्पूज्य भक्तितः}%॥ २२ ॥

\twolineshloka
{यज्ञान्ते सीतया रामो विज्ञप्तः सुविनीतया}
{अस्याध्वरस्य सम्पत्ती दक्षिणां देहि सुव्रत}%॥ २३ ॥

\twolineshloka
{मन्नाम्ना च पुरं तत्र स्थाप्यतां शीघ्रमेव च}
{सीताया वचनं श्रुत्वा तथा चक्रे नृपोत्तमः}%॥ २४ ॥

\twolineshloka
{तेषां च ब्राह्मणानां च स्थानमेकं सुनिर्भयम्}
{दत्तं रामेण सीतायाः सन्तोषाय महीभृता}%॥ २५ ॥

\twolineshloka
{सीतापुरमिति ख्यातं नाम चक्रे तदा किल}
{तस्याधिदेव्यौ वर्त्तेते शान्ता चैव सुमङ्गला}%॥ २६ ॥

\twolineshloka
{मोहेरकस्य पुरतो ग्रामद्वादशकं पुरः}
{ददौ विप्राय विदुषे समुत्थाय प्रहर्षितः}%॥ २७ ॥

\twolineshloka
{तीर्थान्तरं जगामाशु काश्यपीसरितस्तटे}
{वाडवाः केऽपि नीतास्ते रामेण सह धर्मवित्}%॥ २८ ॥

\twolineshloka
{धर्मालये गतः सद्यो यत्र माला कमण्डलुः}
{पुरा धर्मेण सुमहत्कृतं यत्र तपो मुने}%॥ २९ ॥

\twolineshloka
{तदारभ्य सुविख्यातं धर्मालयमिति श्रुतम्}
{ददौ दाशरथिस्तत्र महादानानि षोडश}%॥ ३० ॥

\twolineshloka
{पञ्चाशत्तदा ग्रामाः सीतापुरसमन्विताः}
{सत्यमन्दिरपर्यन्ता रघुना थेन वै तदा}%॥ ३१ ॥

\twolineshloka
{सीताया वचनात्तत्र गुरुवाक्येन चैव हि}
{आत्मनो वंशवृद्ध्यर्थं द्विजेभ्योऽदाद्रघूत्तमः}%॥ ३२ ॥

\twolineshloka
{अष्टादशसहस्राणां द्विजानामभवत्कुलम्}
{वात्स्यायन उपमन्युर्जातूकर्ण्योऽथ पिङ्गलः}%॥ ३३ ॥

\twolineshloka
{भारद्वाजस्तथा वत्सः कौशिकः कुश एव च}
{शाण्डिल्यः कश्यपश्चैव गौतमश्छान्धनस्तथा}%॥ ३४ ॥

\twolineshloka
{कृष्णात्रेयस्तथा वत्सो वसिष्ठो धारणस्तथा}
{भाण्डिलश्चैव विज्ञेयो यौवनाश्वस्ततः परम्}%॥ ३५ ॥

\twolineshloka
{कृष्णायनोपमन्यू च गार्ग्यमुद्गलमौखकाः}
{पुशिः पराशरश्चैव कौण्डिन्यश्च ततः परम्}%॥ ३६ ॥

\twolineshloka
{पञ्चपञ्चाशद्ग्रामाणां नामान्येवं यथाक्रमम्}
{सीतापुरं श्रीक्षेत्रं च मुशली मुद्गली तथा}%॥ ३७ ॥

\twolineshloka
{ज्येष्ठला श्रेयस्थानं च दन्ताली वटपत्रका}
{राज्ञः पुरं कृष्णवाटं देहं लोहं चनस्थनम्}%॥ ३८ ॥

\twolineshloka
{कोहेचं चन्दनक्षेत्रं थलं च हस्तिनापुरम्}
{कर्पटं कन्नजह्नवी वनोडफनफावली}%॥ ३९ ॥

\twolineshloka
{मोहोधं शमोहोरली गोविन्दणं थलत्यजम्}
{चारणसिद्धं सोद्गीत्राभाज्यजं वटमालिका}%॥ ४० ॥

\twolineshloka
{गोधरं मारणजं चैव मात्रमध्यं च मातरम्}
{बलवती गन्धवती ईआम्ली च राज्यजम्}%॥ ४१ ॥

\twolineshloka
{रूपावली बहुधनं छत्रीटं वंशञ्जं तथा}
{जायासंरणं गोतिकी च चित्रलेखं तथैव च}%॥ ४२ ॥

\twolineshloka
{दुग्धावली हंसावली च वैहोलं चैल्लजं तथा}
{नालावली आसावली सुहाली कामतः परम्}%॥ ४३ ॥

\twolineshloka
{रामेण पञ्चपञ्चाशद्ग्रामाणि वसनाय च}
{स्वयं निर्माय दत्तानि द्विजेभ्यस्तेभ्य एव च}%॥ ४४ ॥

\twolineshloka
{तेषां शुश्रूषणार्थाय वैश्यान्रामो न्यवे दयत्}
{षट्त्रिंशच्च सहस्राणि शूद्रास्तेभ्यश्चतुर्गुणान्}%॥ ४५ ॥

\twolineshloka
{तेभ्यो दत्तानि दानानि गवाश्ववसनानि च}
{हिरण्यं रजतं ताम्रं श्रद्धया परया मुदा}%॥ ४६ ॥

\uvacha{नारद उवाच}

\threelineshloka
{अष्टादशसहस्रास्ते ब्राह्मणा वेदपारगाः}
{कथं ते व्यभजन्ग्रामान्द्रामोत्पन्नं तथा वसु}
{वस्त्राद्यं भूषणाद्यं च तन्मे कथय सुव्र तम्}%॥ ४७ ॥

\uvacha{ब्रह्मोवाच}

\twolineshloka
{यज्ञान्ते दक्षिणा यावत्सर्त्विग्भिः स्वीकृता सुत}
{महादानादिकं सर्वं तेभ्य एव समर्पितम्}%॥ ४८ ॥

\twolineshloka
{ग्रामाः साधारणा दत्ता महास्थानानि वै तदा}
{ये वसन्ति च यत्रैव तानि तेषां भवन्त्विति}%॥ ४९ ॥

\twolineshloka
{वशिष्ठवचनात्तत्र ग्रामास्ते विप्रसात्कृताः}
{रघूद्वहेन धीरेण नोद्व सन्ति यथा द्विजाः}%॥ ५० ॥

\twolineshloka
{धान्यं तेषां प्रदत्तं हि विप्राणां चामितं वसु}
{कृताञ्जलिस्ततो रामो ब्राह्मणानिदमब्रवीत्}%॥ ५१ ॥

\twolineshloka
{यथा कृतयुगे विप्रास्त्रेतायां च यथा पुरा}
{तथा चाद्यैव वर्त्तव्यं मम राज्ये न संशयः}%॥ ५२ ॥

\twolineshloka
{यत्किञ्चिद्धनधान्यं वा यानं वा वसनानि वा}
{मणयः काञ्चनादींश्च हेमादींश्च तथा वसु}%॥ ५३ ॥

\twolineshloka
{ताम्राद्यं रजतादींश्च प्रार्थयध्वं ममाधुना}
{अधुना वा भविष्ये वाभ्यर्थनीयं यथोचितम्}%॥ ५४ ॥

\twolineshloka
{प्रेषणीयं वाचिकं मे सर्वदा द्विजसत्तमाः}
{यंयं कामं प्रार्थयध्वं तं तं दास्याम्यहं विभो}%॥ ५५ ॥

\twolineshloka
{ततो रामः सेवकादीनादरात्प्रत्यभाषत}
{विप्राज्ञा नोल्लङ्घनीया सेव नीया प्रयत्नतः}%॥ ५६ ॥

\twolineshloka
{यंयं कामं प्रार्थयन्ते कारयध्वं ततस्ततः}
{एवं नत्वा च विप्राणां सेवनं कुरुते तु यः}%॥ ५७ ॥

\twolineshloka
{स शूद्रः स्वर्गमाप्नोति धनवान्पुत्रवान्भवेत्}
{अन्यथा निर्धनत्वं हि लभते नात्र संशयः}%॥ ५८ ॥

\twolineshloka
{यवनो म्लेच्छजातीयो दैत्यो वा राक्षसोपि वा}
{योत्र विघ्नं करोत्येव भस्मीभवति तत्क्षणात्}%॥ ५९ ॥

\uvacha{ब्रह्मोवाच}

\twolineshloka
{ततः प्रदक्षिणीकृत्य द्विजान्रामोऽतिहर्षितः}
{प्रस्थानाभिमुखो विप्रैराशीर्भिरभिनन्दितः}%॥ ६० ॥

\twolineshloka
{आसीमान्तमनुव्रज्य स्नेहव्याकुललोचनाः}
{द्विजाः सर्वे विनिर्वृत्ता धर्मारण्ये विमोहिताः}%॥ ६१ ॥

\twolineshloka
{एवं कृत्वा ततो रामः प्रतस्थे स्वां पुरीं प्रति}
{काश्यपाश्चैव गर्गाश्च कृतकृत्या दृढव्रताः}%॥ ६२ ॥

\twolineshloka
{गुर्वासनसमाविष्टाः सभार्या ससुहृत्सुताः}
{राजधानीं तदा प्राप रामोऽयोध्यां गुणान्विताम्}%॥ ६३ ॥

\twolineshloka
{दृष्ट्वा प्रमुदिताः सर्वे लोकाः श्रीरघुनन्दनम्}
{ततो रामः स धर्मात्मा प्रजापालनतत्परः}%॥ ६४ ॥

\twolineshloka
{सीतया सह धर्मात्मा राज्यं कुर्वंस्तदा सुधीः}
{जानक्या गर्भमाधत्त रविवंशोद्भवाय च}%॥ ६५ ॥

॥इति श्रीस्कान्दे महापुराण एकाशीतिसाहस्र्यां संहितायां तृतीये ब्रह्मखण्डे पूर्वभागे धर्मारण्यमाहात्म्ये श्रीरामरुद्रकृतधर्मारण्यतीर्थक्षेत्रजीर्णोद्धारवर्णनं नाम पञ्चत्रिंशोऽध्यायः॥३५॥
    \input{katha/skanda-puranam/hanumatkeshvara-mahatmyam}
    \sect{त्र्यशीतितमोऽध्यायः --- हनूमन्तेश्वरतीर्थमाहात्म्यवर्णनम्}

\src{स्कन्दपुराणम्}{खण्डः ५ (अवन्तीखण्डः)}{रेवा खण्डम्}{अध्यायः ०८३}
\vakta{}
\shrota{}
\tags{}
\notes{}
\textlink{https://sa.wikisource.org/wiki/स्कन्दपुराणम्/खण्डः_५_(अवन्तीखण्डः)/रेवा_खण्डम्/अध्यायः_०८३}
\translink{https://www.wisdomlib.org/hinduism/book/the-skanda-purana/d/doc425812.html}

\storymeta




\uvacha{श्रीमार्कण्डेय उवाच}


\threelineshloka
{ततो गच्छेन्महाराज तीर्थं परमशोभनम्}
{ब्रह्महत्याहरं प्रोक्तं रेवातटसमाश्रयम्}
{हनूमताभिधं ह्यत्र विद्यते लिङ्गमुत्तमम्}%॥ १ ॥

\uvacha{युधिष्ठिर उवाच}

\twolineshloka
{हनूमन्तेश्वरं नाम कथं जातं वदस्व मे}
{ब्रह्महत्याहरं तीर्थं रेवादक्षिणसंस्थितम्}%॥ २ ॥

\uvacha{श्रीमार्कण्डेय उवाच}

\twolineshloka
{साधु साधु महाबाहो सोमवंशविभूषण}
{गुह्याद्गुह्यतरं तीर्थं नाख्यातं कस्यचिन्मया}%॥ ३ ॥

\twolineshloka
{तव स्नेहात्प्रवक्ष्यामि पीडितो वार्द्धकेन तु}
{पूर्वं जातं महद्युद्धं रामरावणयोरपि}%॥ ४ ॥

\twolineshloka
{पुलस्त्यो ब्रह्मणः पुत्रो विश्रवास्तस्य वै सुतः}
{रावणस्तेन सञ्जातो दशास्यो ब्रह्मराक्षसः}%॥ ५ ॥

\twolineshloka
{त्रैलोक्यविजयी भूतः प्रसादाच्छूलिनः स च}
{गीर्वाणा विजिताः सर्वे रामस्य गृहिणी हृता}%॥ ६ ॥

\twolineshloka
{वारितः कुम्भकर्णेन सीतां मोचय मोचय}
{विभीषणेन वै पापो मन्दोदर्या पुनःपुनः}%॥ ७ ॥

\twolineshloka
{त्वं जितः कार्तवीर्येण रैणुकेयेन सोऽपि च}
{स रामो रामभद्रेण तस्य सङ्ख्ये कथं जयः}%॥ ८ ॥

\uvacha{रावण उवाच}

\twolineshloka
{वानरैश्च नरैरृक्षैर्वराहैश्च निरायुधैः}
{देवासुरसमूहैश्च न जितोऽहं कदाचन}%॥ ९ ॥

\uvacha{श्रीमार्कण्डेय उवाच}

\twolineshloka
{सुग्रीवहनुमद्भ्यां च कुमुदेनाङ्गदेन च}
{एतैरन्यैः सहायैश्च रामचन्द्रेण वै जितः}%॥ १० ॥

\twolineshloka
{रामचन्द्रेण पौलस्त्यो हतः सङ्ख्ये महाबलः}
{वनं भग्नं हताः शूराः प्रभञ्जनसुतेन च}%॥ ११ ॥

\twolineshloka
{रावणस्य सुतो जन्ये हतश्चाक्षकुमारकः}
{आयामो रक्षसां भीमः सम्पिष्टो वानरेण तु}%॥ १२ ॥

\twolineshloka
{एवं रामायणे वृत्ते सीतामोक्षे कृते सति}
{अयोध्यां तु गते रामे हनुमान्स महाकपिः}%॥ १३ ॥

\twolineshloka
{कैलासाख्यं गतः शैलं प्रणामाय महेशितुः}
{तिष्ठ तिष्ठेत्यसौ प्रोक्तो नन्दिना वानरोत्तमः}%॥ १४ ॥

\twolineshloka
{ब्रह्महत्यायुतस्त्वं हि राक्षसानां वधेन हि}
{भैरवस्य सभा नूनं न द्रष्टव्या त्वया कपे}%॥ १५ ॥

\uvacha{हनुमानुवाच}

\twolineshloka
{नन्दिनाथ हरं पृच्छ पातकस्योपशान्तिदम्}
{पापोऽहं प्लवगो यस्मात्सञ्जातः कारणान्तरात्}%॥ १६ ॥

\uvacha{नन्द्युवाच}

\twolineshloka
{रुद्रदेहोद्भवा किं ते न श्रुता भूतले स्थिता}
{श्रवणाज्जन्मजनितं द्विगुणं कीर्तनाद्व्रजेत्}%॥ १७ ॥

\twolineshloka
{त्रिंशज्जन्मार्जितं पापं नश्येद्रेवावगाहनात्}
{तस्मात्त्वं नर्मदातीरं गत्वा चर तपो महत्}%॥ १८ ॥

\twolineshloka
{गन्धर्वाहसुतोऽप्येवं नन्दिनोक्तं निशम्य च}
{प्रयातो नर्मदातीरमौर्व्यादक्षिणसङ्गमम्}%॥ १९ ॥

\twolineshloka
{दध्यौ सुदक्षिणे देवं विरूपाक्षं त्रिशूलिनम्}
{जटामुकुटसंयुक्तं व्यालयज्ञोपवीतिनम्}%॥ २० ॥

\twolineshloka
{भस्मोपचितसर्वाङ्गं डमरुस्वरनादितम्}
{उमार्द्धाङ्गहरं शान्तं गोनाथासनसंस्थितम्}%॥ २१ ॥

\twolineshloka
{वत्सरान् सुबहून् यावदुपासाञ्चक्र ईश्वरम्}
{तावत्तुष्टो महादेव आजगाम सहोमया}%॥ २२ ॥

\twolineshloka
{उवाच मधुरां वाणीं मेघगम्भीरनिस्वनाम्}
{साधु साध्वित्युवाचेशः कष्टं वत्स त्वया कृतम्}%॥ २३ ॥

\twolineshloka
{न च पूर्वं त्वया पापं कृतं रावणसङ्क्षये}
{स्वामिकार्यरतस्त्वं हि सिद्धोऽसि मम दर्शनात्}%॥ २४ ॥

\threelineshloka
{हनुमांश्च हरं दृष्ट्वा उमार्द्धाङ्गहरं स्थिरम्}
{साष्टाङ्गं प्रणतोऽवोचज्जय शम्भो नमोऽस्तु ते}
{जयान्धकविनाशाय जय गङ्गाशिरोधर}%॥ २५ ॥

\twolineshloka
{एवं स्तुतो महादेवो वरदो वाक्यमब्रवीत्}
{वरं प्रार्थय मे वत्स प्राणसम्भवसम्भव}%॥ २६ ॥

\uvacha{श्रीहनुमानुवाच}

\twolineshloka
{ब्रह्मरक्षोवधाज्जाता मम हत्या महेश्वर}
{न पापोऽहं भवेदेव युष्मत्सम्भाषणे क्षणात्}%॥ २७ ॥

\uvacha{ईश्वर उवाच}

\twolineshloka
{नर्मदातीर्थमाहात्म्याद्धर्मयोगप्रभावतः}
{मन्मूर्तिदर्शनात्पुत्र निष्पापोऽसि न संशयः}%॥ २८ ॥

\twolineshloka
{अन्यं च ते प्रयच्छामि वरं वानरपुङ्गव}
{उपकाराय लोकानां नामानि तव मारुते}%॥ २९ ॥

\twolineshloka
{हनूमानं जनिसुतो वायुपुत्रो महाबलः}
{रामेष्टः फाल्गुनो गोत्रः पिङ्गाक्षोऽमितविक्रमः}%॥ ३० ॥

\twolineshloka
{उदधिक्रमणश्रेष्ठो दशग्रीवस्य दर्पहा}
{लक्ष्मणप्राणदाता च सीताशोकनिवर्तनः}%॥ ३१ ॥

\twolineshloka
{इत्युक्त्वान्तर्दधे देव उमया सह शङ्करः}
{हनूमानीश्वरं तत्र स्थापयामास भक्तितः}%॥ ३२ ॥

\threelineshloka
{आत्मयोगबलेनैव ब्रह्मचर्यप्रभावतः}
{ईश्वरस्य प्रसादेन लिङ्गं कामप्रदं हि तत्}
{अच्छेद्यमप्रतर्क्यं च विनाशोत्पत्तिवर्जितम्}%॥ ३३ ॥

\ldots

॥इति श्रीस्कान्दे महापुराण एकाशीतिसाहस्र्यां संहितायां पञ्चम आवन्त्यखण्डे रेवाखण्डे हनूमन्तेश्वरतीर्थमाहात्म्यवर्णनं नाम त्र्यशीतितमोऽध्यायः॥८३॥
    \sect{लक्ष्मणादिप्रासादपञ्चकनिर्माणप्रतिष्ठापनवर्णनम्}

\src{स्कन्दपुराणम्}{खण्डः ६ (नागरखण्डः)}{}{अध्यायः ०९९}
\vakta{}
\shrota{}
\tags{}
\notes{}
\textlink{https://sa.wikisource.org/wiki/स्कन्दपुराणम्/खण्डः_६_(नागरखण्डः)/अध्यायः_०९९}
\translink{https://www.wisdomlib.org/hinduism/book/the-skanda-purana/d/doc493463.html}

\storymeta


\dnsub{एकोनशततमोऽध्यायः --- रामेश्वरस्थापनप्रस्तावे श्रीरामं प्रति दुर्वासः समागमनवृत्तान्तवर्णनम्}\resetShloka

\uvacha{ऋषय ऊचुः}

\twolineshloka
{यदेतद्भवता प्रोक्तं तत्र रामेण निर्मितः}
{रामेश्वरस्तथा सीता तेन तत्र विनिर्मिता}%॥ १ ॥

\twolineshloka
{तथा च लक्ष्मणार्थाय निर्मितस्तेन संश्रयः}
{एतन्महद्विरुद्धं ते प्रतिभाति वचोऽखिलम्}%॥ २ ॥

\twolineshloka
{त्वया सूत पुरा प्रोक्तं रामो लक्ष्मणसंयुतः}
{सीतया सहितः प्राप्तः क्षेत्रेऽत्र प्रस्थितो वने}%॥ ३ ॥

\twolineshloka
{श्राद्धं कृत्वा गयाशीर्षे लक्ष्मणेन विरुद्ध्य च}
{पुनः सम्प्रस्थितोऽरण्यं क्रोधाविष्टश्च तं प्रति}%॥ ४ ॥

\twolineshloka
{यत्त्वयोक्तं तदा तेन निर्मितोऽत्र महेश्वरः}
{एतच्च सर्वमाचक्ष्व सन्देहं सूतनन्दन}%॥ ५ ॥

\uvacha{सूत उवाच}


\threelineshloka
{अत्र मे नास्ति सन्देहो युष्माकं च पुनः स्थितः}
{ततो वक्ष्याम्यशेषेण श्रूयतां द्विजसत्तमाः}
{एतत्क्षेत्रं पुनश्चाद्यं न क्षयं याति कुत्रचित्}%॥ ६ ॥

\twolineshloka
{अन्यस्मिन्दिवसे प्राप्ते स तदा रघुनन्दनः}
{यदा विरोधमापन्नः सार्धं सौमित्रिणा सह}%॥ ७ ॥

\twolineshloka
{एतत्पुनर्दिनं चान्यद्यत्र तेन प्रतिष्ठितः}
{रामेश्वरः स्वयं भक्त्या दुःखितेन महात्मना}%॥ ८ ॥

\uvacha{ऋषय ऊचुः}

\twolineshloka
{अन्यस्मिन्दिवसे तत्र कस्मिन्काले रघूत्तमः}
{सम्प्राप्तस्तस्य किं दुःखं सञ्जातं तत्प्रकीर्तय}%॥ ९ ॥

\uvacha{सूत उवाच}

\twolineshloka
{कृत्वा सीतापरित्यागं रामो राजीवलोचनः}
{लोकापवादसन्त्रस्तस्ततो राज्यं चकार सः}%॥ १० ॥

\twolineshloka
{कृत्वा स्वर्णमयीं सीतां पत्नीं यज्ञप्रसिद्धये}
{न स चक्रे महाभागो भार्यामन्यां कथञ्चन}%॥ ११ ॥

\twolineshloka
{दशवर्षसहस्राणि दशवर्षशतानि च}
{ब्रह्मचर्येण चक्रे स राज्यं निहतकण्टकम्}%॥ १२ ॥

\twolineshloka
{ततो वर्षसहस्रान्ते प्राप्ते चैकादशे द्विजाः}
{देवदूतः समायातो रामस्य सदनं प्रति}%॥ १३ ॥

\twolineshloka
{तेनोक्तं देवराजेन प्रेषितोऽहं तवान्तिकम्}
{तस्मात्कुरु समालोकं विजने त्वं मया सह}%॥ १४ ॥

\twolineshloka
{एवमुक्तस्तदा तेन दूतेन रघुनन्दनः}
{परं रहः समासाद्य मन्त्रं चक्रे ततः परम्}%॥ १५ ॥

\twolineshloka
{तस्यैवमुपविष्टस्य मन्त्रस्थाने महात्मनः}
{बहुत्वादिष्टलोकस्य न रहस्यं प्रजायते}%॥ १६ ॥

\twolineshloka
{ततः कोपपरीतात्मा दूतः प्रोवाच सादरम्}
{विहस्य जनसंसर्गं दृष्ट्वैकान्तेऽपि संस्थिते}%॥ १७ ॥

\twolineshloka
{यथा दंष्ट्राच्युतः सर्पो नागो वा मदवर्जितः}
{आज्ञाहीनस्तथा राजा मानवैः परिभूयते}%॥ १८ ॥

\twolineshloka
{सेयं तव रघुश्रेष्ठ नाज्ञास्ति प्रतिवेद्म्यहम्}
{शक्रालापमपि त्वं च नैकान्ते श्रोतुमर्हसि}%॥ १९ ॥

\twolineshloka
{तस्य तद्वचनं श्रुत्वा कोपसंरक्तलोचनः}
{त्रिशाखां भृकुटीं कृत्वा ततः स प्राह लक्ष्मणम्}%॥ २० ॥

\threelineshloka
{ममात्र सन्निविष्टस्य सहानेन प्रजल्पतः}
{यदि कश्चिन्नरो मोहादागमिष्यति लक्ष्मण}
{स्वहस्तेन न सन्देहः सूदयिष्यामि तं द्रुतम्}%॥ २१ ॥

\twolineshloka
{न हन्मि यदि तं प्राप्तमत्र मे दृष्टिगोचरम्}
{तन्मा भून्मे गतिः श्रेष्ठा धर्मिणां या प्रपद्यते}%॥ २२ ॥

\twolineshloka
{एवं ज्ञात्वा प्रयत्नेन त्वया भाव्यमसंशयम्}
{राजद्वारि यथा कश्चिन्न मया वध्यतेऽधुना}%॥ २३ ॥

\twolineshloka
{तमोमित्येव सम्प्रोच्य लक्ष्मणः शुभलक्षणः}
{राजद्वारं समासाद्य चकार विजनं ततः}%॥ २४ ॥

\twolineshloka
{देवदूतोऽपि रामेण समं चक्रे ततः परम्}
{मन्त्रं शक्रसमादिष्टं तथान्यैः स्वर्गवासिभिः}%॥ २५ ॥

\uvacha{देवदूत उवाच}

\twolineshloka
{त्वं रावणविनाशार्थमवतीर्णो धरातले}
{स च व्यापादितो दुष्टः पापस्त्रैलोक्यकण्टकः}%॥ २६ ॥

\twolineshloka
{कृतं सर्वं महाभाग देव कृत्यं त्वयाऽधुना}
{तस्मात्सन्तु सनाथास्ते देवाः शक्रपुरोगमाः}%॥ २७ ॥

\threelineshloka
{यदि ते रोचते चित्ते नोपरोधेन साम्प्रतम्}
{प्रसादं कुरु देवानां तस्मादागच्छ सत्वरम्}
{स्वर्गलोकं परित्यज्य मर्त्यलोकं सुनिन्दितम्}%॥ २८ ॥

\uvacha{सूत उवाच}

\twolineshloka
{एतस्मिन्नन्तरे प्राप्तो दुर्वासा मुनिसत्तमः}
{प्रोवाचाथ क्षुधाविष्टः क्वासौ क्वासौ रघूत्तमः}%॥ २९ ॥

\uvacha{लक्ष्मण उवाच}

\twolineshloka
{व्यग्रः स पार्थिवश्रेष्ठो देवकार्येण केनचित्}
{तस्मादत्रैव विप्रेन्द्र मुहूर्तं परिपालय}%॥ ३० ॥

\twolineshloka
{यावत्सान्त्वयते रामो दूतं शक्रसमुद्भवम्}
{ममोपरि दयां कृत्वा विनयावनतस्य हि}%॥ ३१ ॥

\uvacha{दुर्वासा उवाच}

\twolineshloka
{यदि यास्यति नो दृष्टिं मम द्राक्स रघूत्तमः}
{शापं दत्त्वा कुलं सर्वं तद्धक्ष्यामि न संशयः}%॥ ३२ ॥

\twolineshloka
{ममापि दर्शनादन्यन्न किञ्चिद्विद्यते गुरु}
{कृत्यं लक्ष्मण यावत्त्वमन्यन्मूढ़ प्रकत्थसे}%॥ ३३ ॥

\twolineshloka
{तच्छ्रुत्वा लक्ष्मणश्चित्ते चिन्तयामास दुःखितः}
{वरं मे मृत्युरेकस्य मा भूयात्कुलसङ्क्षयः}%॥ ३४ ॥

\twolineshloka
{एवं स निश्चयं कृत्वा ततो राममुपाद्रवत्}
{उवाच दण्डवद्भूमौ प्रणिपत्य कृताञ्जलिः}%॥ ३५ ॥

\twolineshloka
{दुर्वासा मुनिशार्दूलो देव ते द्वारि तिष्ठति}
{दर्शनार्थी क्षुधाविष्टः किं करोमि प्रशाधि माम्}%॥ ३६ ॥

\threelineshloka
{तस्य तद्वचनं श्रुत्वा ततो दूतमुवाच तम्}
{गत्वेमं ब्रूहि देवेशं मम वाक्यादसंशयम्}
{अहं संवत्सरस्यान्त आगमिष्यामि तेंऽतिके}%॥ ३७ ॥

\twolineshloka
{एवमुक्त्वा विसृज्याथ तं दूतं प्राह लक्ष्मणम्}
{प्रवेशय द्रुतं वत्स तं त्वं दुर्वाससं मुनिम्}%॥ ३८ ॥

\twolineshloka
{ततश्चार्घ्यं च पाद्यं च गृहीत्वा सम्मुखो ययौ}
{रामदेवः प्रहृष्टात्मा सचिवैः परिवारितः}%॥ ३९ ॥

\twolineshloka
{दत्त्वार्घ्यं विधिवत्तस्य प्रणिपत्य मुहुर्मुहुः}
{प्रोवाच रामदेवोऽथ हर्षगद्गदया गिरा}%॥ ४० ॥

\twolineshloka
{स्वागतं ते मुनिश्रेष्ठ भूयः सुस्वागतं च ते}
{एतद्राज्यममी पुत्रा विभवश्च तव प्रभो}%॥ ४१ ॥

\threelineshloka
{कृत्वा मम प्रसादं च गृहाण मुनिसत्तम}
{धन्योऽस्म्यनुगृहीतोऽस्मि यत्त्वं मे गृहमागतः}
{पूज्यो लोकत्रयस्यापि निःशेषतपसान्निधिः}%॥ ४२ ॥

\uvacha{मुनिरुवाच}

\twolineshloka
{चातुर्मास्यव्रतं कृत्वा निराहारो रघूत्तम}
{अद्य ते भवनं प्राप्य आहारार्थं बुभुक्षितः}%॥ ४३ ॥

\twolineshloka
{तस्मात्त्वं यच्छ मे शीघ्रं भोजनं रघुनन्दन}
{नान्येन कारणं किञ्चित्सन्न्यस्तस्य धनादिना}%॥ ४४ ॥

\twolineshloka
{ततस्तं भोजयामास श्रद्धापूतेन चेतसा}
{स्वयमेवाग्रतः स्थित्वा मृष्टान्नैर्विविधैः शुभैः}%॥ ४५ ॥

\twolineshloka
{लेह्यैश्चोष्यैस्तथा चर्व्यैः खाद्यैरेव पृथग्विधैः}
{यावदिच्छा मुनेस्तस्य तथान्नैर्विविधैरपि}%॥ ४६ ॥

॥इति श्रीस्कान्दे महापुराण एकाशीतिसाहस्र्यां संहितायां षष्ठे नागरखण्डे हाटकेश्वरक्षेत्रमाहात्म्ये रामेश्वरस्थापनप्रस्तावे श्रीरामम्प्रति दुर्वासः समागमनवृत्तान्तवर्णनं नामैकोनशततमोऽध्यायः॥९९॥

\dnsub{शततमोऽध्यायः --- श्रीरामेश्वरस्थापनप्रस्तावे\\लक्ष्मणनिर्वाणोत्तरं श्रीरामस्य सुग्रीवनगरीं प्रति गमनवर्णनम्}\resetShloka

\uvacha{सूत उवाच}

\twolineshloka
{एवं भुक्त्वा स विप्रर्षिर्वाञ्छया राममन्दिरे}
{दत्ताशीर्निर्गतः पश्चादामन्त्र्य रघुनन्दनम्}%॥ १ ॥

\twolineshloka
{अथ याते मुनौ तस्मिन्दुर्वाससि तदन्तिकात्}
{लक्ष्मणः खङ्गमादाय रामदेवमुवाच ह}%॥ २ ॥

\twolineshloka
{एतत्खङ्गं गृहीत्वाशु मां प्रभो विनिपातय}
{येन ते स्यादृतं वाक्यं प्रतिज्ञातं च यत्पुरा}%॥ ३ ॥

\twolineshloka
{ततो रामश्चिरात्स्मृत्वा तां प्रतिज्ञां स्वयं कृताम्}
{वधार्थं सम्प्रविष्टस्य समीपे पुरुषस्य च}%॥ ४ ॥

\twolineshloka
{ततोऽतिचिन्तयामास व्याकुलेनान्तरात्मना}
{बाष्पव्याकुलनेत्रश्च निःष्वसन्पन्नगो यथा}%॥ ५ ॥

\twolineshloka
{तं दीनवदनं दृष्ट्वा निःष्वसन्तं मुहुर्मुहुः}
{भूयः प्रोवाच सौमित्रिर्विनयावनतः स्थितः}%॥ ६ ॥

\twolineshloka
{एष एव परो धर्मो भूपतीनां विशेषतः}
{यथात्मीयं वचस्तथ्यं क्रियते निर्विकल्पितम्}%॥ ७ ॥

\twolineshloka
{तस्मात्त्वया प्रभो प्रोक्तं स्वयमेव ममाग्रतः}
{तस्यैव देवदूतस्य तारनादेन कोपतः}%॥ ९ ॥

\twolineshloka
{योऽत्रागच्छति सौमित्रे मम दूतस्य सन्निधौ}
{तं चेद्धन्मि स्वहस्तेन नाहं तस्मात्सुपापकृत्}%॥ ९ ॥

\twolineshloka
{तदहं चागतस्तात भयाद्दुर्वाससो मुनेः}
{निषिद्धोऽपि त्वयातीव तस्माच्छीघ्रं तु घातय}%॥ १० ॥

\twolineshloka
{ततः सम्मन्त्र्य सुचिरं मन्त्रिभिः सहितो नृपः}
{ब्राह्मणैर्धर्मशास्त्रज्ञैस्तथान्यैर्वेदपारगैः}%॥ ११ ॥

\twolineshloka
{प्रोवाच लक्ष्मणं पश्चाद्विनयावनतं स्थितम्}
{वाष्पक्लिन्नमुखो रामो गद्गदं निःश्वसन्मुहुः}%॥ १२ ॥

\twolineshloka
{व्रज लक्ष्मण मुक्तस्त्वं मया देशातरं द्रुतम्}
{त्यागो वाथ वधो वाथ साधूनामुभयं समम्}%॥ १३ ॥

\twolineshloka
{न मया दर्शनं भूयस्तव कार्यं कथञ्चन}
{न स्थातव्यं च देशेऽपि यदि मे वाञ्छसि प्रियम्}%॥ १४ ॥

\twolineshloka
{तस्य तद्वचनं श्रुत्वा प्रणिपत्य ततः परम्}
{निर्ययौ नगरात्तस्मात्तत्क्षणादेव लक्ष्मणः}%॥ १५ ॥

\twolineshloka
{अकृत्वापि समालापं केनचिन्निजमन्दिरे}
{मात्रा वा भार्यया वाथ सुतेन सुहृदाथवा}%॥ १६ ॥

\twolineshloka
{ततोऽसौ सरयूं गत्वाऽवगाह्याथ च तज्जलम्}
{शुचिर्भूत्वा निविष्टोथ तत्तीरे विजने शुभे}%॥ १७ ॥

\twolineshloka
{पद्मासनं विधायाथ न्यस्यात्मानं तथात्मनि}
{ब्रह्मद्वारेण तं पश्चात्तेजोरूपं व्यसर्जयत्}%॥ १८ ॥

\twolineshloka
{अथ तद्राघवो दृष्ट्वा महत्तेजो वियद्गतम्}
{विस्मयेन समायुक्तोऽचिन्तयत्किमिदं ततः}%॥ १९ ॥

\twolineshloka
{अथ मर्त्ये परित्यक्ते तेजसा तेन तत्क्षणात्}
{वैष्णवेन तुरीयेण भागेन द्विजसत्तमाः}%॥ २० ॥

\twolineshloka
{पपात भूतले कायं काष्ठलोष्टोपमं द्रुतम्}
{लक्ष्मणस्य गतश्रीकं सरय्वाः पुलिने शुभे}%॥ २१ ॥

\twolineshloka
{ततस्तु राघवः श्रुत्वा लक्ष्मणं गतजीवितम्}
{पतितं सरितस्तीरे विललाप सुदुःखितः}%॥ २२ ॥

\twolineshloka
{स्वयं गत्वा तमुद्देशं सामात्यः ससुहृज्जनः}
{लक्ष्मणं पतितं दृष्ट्वा करुणं पर्यदेवयत्}%॥ २३ ॥

\twolineshloka
{हा वत्स मां परित्यज्य किं त्वं सम्प्रस्थितो दिवम्}
{प्राणेष्टं भ्रातरं श्रेष्ठं सदा तव मते स्थितम्}%॥ २४ ॥

\twolineshloka
{तस्मिन्नपि महारण्ये गच्छमानः पुरादहम्}
{अपि सन्धार्यमाणेन अनुयातस्त्वया तदा}%॥ २५ ॥

\twolineshloka
{सम्प्राप्तेऽपि कबन्धाख्ये राक्षसे बलवत्तरे}
{त्वया रात्रिमुखे घोरे सभार्योऽहं प्ररक्षितः}%॥ २६ ॥

\twolineshloka
{येनेन्द्रजिद्धतो युद्धे तादृग्रूपो निशाचरः}
{स एष पतितः शेते गतासुर्धरणीतले}%॥ ९७ ॥

\twolineshloka
{येन शूर्पणखा ध्वस्ता राक्षसी सा च दारुणा}
{लीलयापि ममादेशात्सोयमेवंविधः स्थितः}%॥ २८ ॥

\twolineshloka
{यद्बाहुबलमाश्रित्य मया ध्वस्ता निशाचराः}
{सोऽयं निपतितः शेते मम भ्राता ह्यनाथवत्}%॥ २९ ॥

\twolineshloka
{हा वत्स क्व गतो मां त्वं विमुच्य भ्रातरं निजम्}
{ज्येष्ठं प्राणसमं किं ते स्नेहोऽद्य विगतः क्वचित्}%॥ ३० ॥

\uvacha{सूत उवाच}

\twolineshloka
{एवं बहुविधान्कृत्वा प्रलापान्रघुनन्दनः}
{मातृभिः सहितो दीनः शोकेन महतान्वितः}%॥ ३१ ॥

\twolineshloka
{ततस्ते मन्त्रिणस्तस्य प्रोचुस्तं वीक्ष्य दुःखितम्}
{विलपन्तं रघुश्रेष्ठं स्त्रीजनेन समन्वितम्}%॥ ३२ ॥

\uvacha{मन्त्रिण ऊचुः}

\twolineshloka
{मा शोकं कुरु राजेन्द्र यथान्यः प्राकृतः स्थितः}
{कुरुष्व च यथेदं स्यात्साम्प्रतं चौर्ध्वदैहिकम्}%॥ ३३ ॥

\twolineshloka
{नष्टं मृतमतीतं च ये शोचन्ति कुबुद्धयः}
{धीराणां तु पुरा राजन्नष्टं नष्टं मृतं मृतम्}%॥ ३४ ॥

\twolineshloka
{एवं ते मन्त्रिणः प्रोच्य ततस्तस्य कलेवरम्}
{लक्ष्मणस्य विलप्यौच्चैश्चन्दनोशीरकुङ्कुमैः}%॥ ३५ ॥

\twolineshloka
{कर्पूरागुरुमिश्रैश्च तथान्यैः सुसुगन्धिभिः}
{परिवेष्ट्य शुभैर्वस्त्रैः पुष्पैः सम्भूष्य शोभनैः}%॥ ३६ ॥

\twolineshloka
{चन्दनागुरुकाष्ठैश्च चितिं कृत्वा सुविस्तराम्}
{न्यदधुस्तस्य तद्गात्रं तत्र दक्षिणदिङ्मुखम्}%॥ ३७ ॥

\twolineshloka
{एतस्मिन्नन्तरे जातं तत्राश्चर्यं द्विजोत्तमाः}
{तन्मे निगदतः सर्वं शृण्वन्तु सकलं द्विजाः}%॥ ३८ ॥

\twolineshloka
{यावत्तेंऽतः समारोप्य चितां तस्य कलेवरम्}
{प्रयच्छन्ति हविर्वाहं तावन्नष्टं कलेवरम्}%॥ ३९ ॥

\twolineshloka
{एतस्मिन्नन्तरे वाणी निर्गता गगनाङ्गणात्}
{नादयन्ती दिशः सर्वाः पुष्पवर्षादनन्तरम्}%॥ ४० ॥

\twolineshloka
{रामराम महाबाहो मा त्वं शोकपरो भव}
{न चास्य युज्यते वह्निर्दातुं चैव कथञ्चन}%॥ ४१ ॥

\twolineshloka
{ब्रह्मज्ञानप्रयुक्तस्य सन्न्यस्तस्य विशेषतः}
{अग्निदानं न युक्तं स्यात्सर्वेषामपि योगिनाम्}%॥ ४२ ॥

\twolineshloka
{तवायं बान्धवो राम ब्रह्मणः सदनं गतः}
{ब्रह्मद्वारेण चात्मानं निष्क्रम्य सुमहायशाः}%॥ ४३ ॥

\threelineshloka
{अथ ते मन्त्रिणः प्रोचुस्तच्छ्रुत्वाऽऽकाशगं वचः}
{अशोच्यो यं महाराज संसिद्धिं परमां गतः}
{लक्ष्मणो गम्यतां शीघ्रं तस्मात्स्वभवने विभो}%॥ ४४ ॥

\twolineshloka
{चिन्त्यन्तां राजकार्याणि तथा यच्चौर्ध्वदैहिकम्}
{कुरु स्नेहोचितं तस्य पृष्ट्वा ब्राह्मणसत्तमान्}%॥ ४५ ॥

\uvacha{राम उवाच}

\twolineshloka
{नाहं गृहं गमिष्यामि लक्ष्मणेन विनाऽधुना}
{प्राणानत्र विहास्यामि यथा तेन महात्मना}%॥ ४६ ॥

\twolineshloka
{एष पुत्रो मया दत्तः कुशाख्यो मम सम्मतः}
{युष्मभ्यं क्रियतां राज्ये मदीये यदि रोचते}%॥ ४७ ॥

\twolineshloka
{एवमुक्त्वा ततो रामो गन्तुकामो दिवालयम्}
{चिन्तयामास भूयोऽपि स्मृत्वा मित्रं विभीषणम्}%॥ ४८ ॥

\twolineshloka
{मया तस्य तदा दत्तं लङ्कायां राज्यमक्षयम्}
{बहुभक्तिप्रतुष्टेन यावच्चन्द्रार्कतारकाः}%॥ ४९ ॥

\twolineshloka
{अतिक्रूरतरा जाती राक्षसानां यतः स्मृता}
{विशेषाद्वरपुष्टानां जायतेऽत्र धरातले}%॥ ५० ॥

\twolineshloka
{तच्चेद्राक्षसभावेन स महात्मा विभीषणः}
{करिष्यति सुरैः सार्धं विरोधं रावणो यथा}%॥ ५१ ॥

\twolineshloka
{तं देवाः सूदयिष्यन्ति उपायैः सामपूर्वकैः}
{त्रैलोक्यकण्टको यद्वत्तस्य भ्राता दशाननः}%॥ ५२ ॥

\twolineshloka
{ततो मे स्यान्मृषा वाणी तस्माद्गत्वा तदन्तिकम्}
{शिक्षां ददामि तस्याहं यथा देवान्न दूषयेत्}%॥ ५३ ॥

\twolineshloka
{तथा मे परमं मित्रं द्वितीयं वानरः स्थितः}
{सुग्रीवाख्यो महाभागो जाम्बवांश्च तथाऽपरः}%॥ ५४ ॥

\twolineshloka
{सभृत्यो वायुपुत्रश्च वालिपुत्रसमन्वितः}
{कुमुदाख्यश्च तारश्च तथान्येऽपि च वानराः}%॥ ५५ ॥

\twolineshloka
{तस्मात्तानपि सम्भाष्य सर्वान्सम्मन्त्र्य सादरम्}
{ततो गच्छामि देवानां कृतकृत्यो गृहं प्रति}%॥ ५६ ॥

\twolineshloka
{एवं सञ्चिन्त्य सुचिरं समाहूय च पुष्पकम्}
{तत्रारुह्य ययौ तूर्णं किष्किन्धाख्यां पुरीं प्रति}%॥ ५७ ॥

\twolineshloka
{अथ ते वानरा दृष्ट्वा प्रोद्द्योतं पुष्पकोद्भवम्}
{विज्ञाय राघवं प्राप्तं सत्वरं सम्मुखा ययुः}%॥ ५८ ॥

\twolineshloka
{ततः प्रणम्य ते दूराज्जानुभ्यामवनिं गताः}
{जयेति शब्दमादाय मुहुर्मुहुरितस्ततः}%॥ ५९ ॥

\twolineshloka
{ततस्तेनैव संयुक्ताः किष्किन्धां तां महापुरीम्}
{विविशुः सत्पताकाभिः समन्तात्समलङ्कृताम्}%॥ ६० ॥

\twolineshloka
{अथोत्तीर्य विमानाग्र्यात्सुग्रीवभवने शुभे}
{प्रविवेश द्रुतं रामः सर्वतः सुविभूषिते}%॥ ६१ ॥

\twolineshloka
{तत्र रामं निविष्टं ते विश्रान्तं वीक्ष्य वानराः}
{अर्घ्यादिभिश्च सम्पूज्य पप्रच्छुस्तदनन्तरम्}%॥ ६२ ॥

\uvacha{वानरा ऊचुः}

\twolineshloka
{तेजसा त्वं विनिर्मुक्तो दृश्यसे रघुनन्दन}
{कृशोऽस्यतीव चोद्विग्नः कच्चित्क्षेमं गृहे तव}%॥ ६३ ॥

\twolineshloka
{काये वाऽनुगतो नित्यं तथा ते लक्ष्मणोऽनुजः}
{न दृश्यते समीपस्थः किमद्य तव राघव}%॥ ६४ ॥

\twolineshloka
{तथा प्राणसमाऽभीष्टा सीता तव प्रभो}
{दृश्यते किं न पार्श्वस्था एतन्नः कौतुकं परम्}%॥ ६५ ॥

\uvacha{सूत उवाच}

\twolineshloka
{तेषां तद्वचनं श्रुत्वा चिरं निःश्वस्य राघवः}
{वाष्पपूर्णेक्षणो भूत्वा सर्वं तेषां न्यवेदयत्}%॥ ६६ ॥

\twolineshloka
{अथ सीता परित्यक्ता तथा भ्राता स लक्ष्मणः}
{यदर्थं तत्र सम्प्राप्तः स्वयमेव द्विजोत्तमाः}%॥ ६७ ॥

\twolineshloka
{तच्छ्रुत्वा वानराः सर्वे सुग्रीवप्रमुखास्ततः}
{रुरुदुस्ते सुदुःखार्ताः समालिङ्ग्य ततः परम्}%॥ ६८ ॥

\twolineshloka
{एवं चिरं प्रलप्योच्चैस्ततः प्रोचू रघूत्तमम्}
{आदेशो दीयतां राजन्योऽस्माभिरिह सिध्यति}%॥ ६९ ॥

\twolineshloka
{धन्या वयं धरापृष्ठे येषां त्वं रघुसत्तम}
{ईदृक्स्नेहसमायुक्तः समागच्छसि मन्दिरे}%॥ ७० ॥

\uvacha{राम उवाच}

\twolineshloka
{उषित्वा रजनीमेकां सुग्रीव तव मन्दिरे} 
{प्रातर्लङ्कां गमिष्यामि यत्रास्ते स विभीषणः}% ७१ ॥

\twolineshloka
{प्रधानामात्ययुक्तेन त्वयापि कपिसत्तम}
{आगन्तव्यं मया सार्धं विभीषणगृहं प्रति}%॥ ७२ ॥

॥इति श्रीस्कान्दे महापुराण एकाशीतिसाहस्र्यां संहितायां षष्ठे नागरखण्डे हाटकेश्वरक्षेत्रमाहात्म्ये श्रीरामेश्वरस्थापनप्रस्तावे लक्ष्मणनिर्वाणोत्तरं श्रीरामस्य सुग्रीवनगरीं प्रति गमनवर्णनं नाम शततमोऽध्यायः॥१००॥

\dnsub{एकोत्तरशततमोऽध्यायः --- सेतुमध्ये\\श्रीरामकृतरामेश्वरप्रतिष्ठावर्णनम्}\resetShloka

\uvacha{सूत उवाच}

\twolineshloka
{एवं तां रजनीं तत्र स उषित्वा रघूत्तमः}
{उपास्यमानः सर्वैस्तैः सद्भक्त्या वानरोत्तमैः}%॥ १ ॥

\twolineshloka
{ततः प्रभाते विमले प्रोद्गते रविमण्डले}
{कृत्वा प्राभातिकं कर्म समाहूयाथ पुष्पकम्}%॥ २ ॥

\twolineshloka
{सुग्रीवेण सुषेणेन तारेण कुमुदेन च}
{अङ्गदेनाथ कुण्डेन वायुपुत्रेण धीमता}%॥ ३ ॥

\twolineshloka
{गवाक्षेण नलेनेव तथा जाम्बवतापि च}
{दशभिर्वानरैः सार्धं समारूढः स पुष्पके}%॥ ४ ॥

\twolineshloka
{ततः सम्प्रस्थितः काले लङ्कामुद्दिश्य राघवः}
{मनोजवेन तेनैव विमानेन सुवर्चसा}%॥ ५ ॥

\twolineshloka
{सम्प्राप्तस्तत्क्षणादेव लङ्काख्यां च महापुरीम्}
{वीक्षयंस्तान्प्रदेशांश्च यत्र युद्धं पुराऽभवत्}%॥ ६ ॥

\threelineshloka
{ततो विभीषणो दृष्ट्वा प्रोद्द्योतं पुष्पकोद्भवम्}
{रामं विज्ञाय सम्प्राप्तं प्रहृष्टः सम्मुखो ययौ}
{मन्त्रिभिः सकलैः सार्धं तथा भृत्यैः सुतैरपि}%॥ ७ ॥

\twolineshloka
{अथ दृष्ट्वा सुदूरात्तं रामदेवं विभीषणः}
{पपात दण्डवद्भूमौ जयशब्दमुदीरयन्}%॥ ८ ॥

\twolineshloka
{तथागतं परिष्वज्य सादरं स विभीषणम्}
{तेनैव सहितः पश्चाल्लङ्कां तां प्रविवेश ह}%॥ ९ ॥

\twolineshloka
{विभीषणगृहं प्राप्य तत्र सिंहासने शुभे}
{निविष्टो वानरैस्तैश्च समन्तात्परिवारितः}%॥ १० ॥

\twolineshloka
{ततो निवेदयामास तस्मै सर्वं विभीषणः}
{राज्यं पुत्रकलत्रादि यच्चान्यदपि किञ्चन}%॥ ११ ॥

\twolineshloka
{ततः प्रोवाच विनयात्कृताञ्जलिपुटः स्थितः}
{आदेशो दीयतां देव ब्रूहि कृत्यं करोमि किम्}%॥ १२ ॥

\twolineshloka
{अकस्मादेव सम्प्राप्तः किमर्थं वद मे प्रभो}
{किं नायातः स सौमित्रिस्त्वया सार्ध च जानकी}%॥ १३ ॥

\uvacha{सूत उवाच}

\twolineshloka
{निवेद्य राघवस्तस्मै सर्वं गद्गदया गिरा}
{वाष्पपूरप्रतिच्छन्नवक्त्रो भूयो विनिःश्वसन्}%॥ १४ ॥

\twolineshloka
{ततः प्रोवाच सत्यार्थं विभीषणकृते हितम्}
{तं चापि शोकसन्तप्तं सम्बोध्य रघुनन्दनः}%॥ १५ ॥

\twolineshloka
{अहं राज्यं परित्यज्य साम्प्रतं राक्षसोत्तम}
{यास्यामि त्रिदिवं तूर्णं लक्ष्मणो यत्र संस्थितः}%॥ १६ ॥

\twolineshloka
{न तेन रहितो मर्त्ये मुहूर्तमपि चोत्सहे}
{स्थातुं राक्षसशार्दूल बान्धवेन महात्मना}%॥ १७ ॥

\twolineshloka
{अहं शिक्षापणार्थाय तव प्राप्तो विभीषण}
{तस्मादव्यग्रचित्तेन संशृणुष्व कुरुष्व च}%॥ १८ ॥

\twolineshloka
{एषा राज्योद्भवा लक्ष्मीर्मदं सञ्जनयेन्नृणाम्}
{मद्यवत्स्वल्पबुद्धीनां तस्मात्कार्यो न स त्वया}%॥ १९ ॥

\twolineshloka
{शक्राद्या अमराः सर्वे त्वया पूज्याः सदैव हि}
{मान्याश्च येन ते राज्यं जायते शाश्वतं सदा}%॥ २० ॥

\twolineshloka
{मम सत्यं भवेद्वाक्य मेतस्मादहमागतः}
{प्राप्तराज्यप्रतिष्ठोऽपि तव भ्राता महाबलः}%॥ २१ ॥

\threelineshloka
{विनाशं सहसा प्राप्तस्तस्मान्मान्याः सुराः सदा}
{यदि कश्चित्समायाति मानुषोऽत्र कथञ्चन}
{मत्काय एव द्रष्टव्यः सर्वैरेव निशाचरैः}%॥ २२ ॥

\twolineshloka
{तथा निशाचराः सर्वे त्वया वार्या विभीषण}
{मम सेतुं समुल्लङ्घ्य न गन्तव्यं धरातले}%॥ २३ ॥

\uvacha{विभीषण उवाच}

\twolineshloka
{एवं विभो करिष्यामि तवादेशमसंशयम्}
{परं त्वया परित्यक्ते मर्त्ये मे जीवितं व्रजेत्}%॥ २४ ॥

\twolineshloka
{तस्मान्मामपि तत्रैव त्वं विभो नेतुमर्हसि}
{आत्मना सह यत्रास्ते प्राग्गतो लक्ष्मणस्तव}%॥ २५ ॥

\uvacha{श्रीराम उवाच}

\twolineshloka
{मया तेऽक्षयमादिष्टं राज्यं राक्षससत्तम}
{तस्मान्नार्हसि मां कर्तुं मिथ्याचारं कथञ्चन}%॥ २६ ॥

\threelineshloka
{अहमस्मिन्स्वके सेतौ शङ्करत्रितयं शुभम्}
{स्थापयिष्यामि कीर्त्यर्थं तत्पूज्यं भवता सदा}
{भक्तिमान्प्रतिसन्धाय यावच्चन्द्रार्कतारकम्}%॥ २७ ॥

\twolineshloka
{एवमुक्त्वा रघुश्रेष्ठो राक्षसेन्द्रं विभीषणम्}
{दशरात्रं तत्र तस्थौ लङ्कायां वानरैः सह}%॥ २८ ॥

\twolineshloka
{कुर्वन्युद्धकथाश्चित्रा याः कृताः पूर्वमेव हि}
{पश्यन्युद्धस्य सर्वाणि स्थानानि विविधानि च}%॥ २९ ॥

\twolineshloka
{शंसमानः प्रवीरांस्तान्राक्षसान्बलवत्तरान्}
{कुम्भकर्णेन्द्रजित्पूर्वान्सङ्ख्ये चाभिमुखागतान्}%॥ ३० ॥

\twolineshloka
{ततश्चैकादशे प्राप्ते दिवसे रघुनन्दनः}
{पुष्पकं तत्समारुह्य प्रस्थितः स्वपुरीं प्रति}%॥ ३१ ॥

\twolineshloka
{वानरैस्तैः समोपेतो विभीषणपुरःसरः}
{ततः संस्थापयामास सेतुप्रान्ते महेश्वरम्}%॥ ३२ ॥

\twolineshloka
{मध्ये चैव तथादौ च श्रद्धापूतेन चेतसा}
{रामेश्वरत्रयं राम एवं तत्र विधाय सः}%॥ ३३ ॥

\twolineshloka
{सेतुबन्धं तथासाद्य प्रस्थितः स्वगृहं प्रति}
{तावद्विभीषणेनोक्तः प्रणिपत्य मुहुर्मुहुः}%॥ ३४ ॥

\uvacha{विभीषण उवाच}

\twolineshloka
{अनेन सेतुमार्गेण रामेश्वरदिदृक्षया}
{मानवा आगमिष्यन्ति कौतुकाच्छ्रद्धयाविताः}%॥ ३५ ॥

\twolineshloka
{राक्षसानां महाराज जातिः क्रूरतमा मता}
{दृष्ट्वा मानुषमायान्तं मांसस्येच्छा प्रजायते}%॥ ३६ ॥

\twolineshloka
{यदा कश्चिज्जनं कश्चिद्राक्षसो भक्षयिष्यति}
{आज्ञाभङ्गो ध्रुवं भावी मम भक्तिरतस्य च}%॥ ३७ ॥

\twolineshloka
{भविष्यन्ति कलौ काले दरिद्रा नृपमानवाः}
{तेऽत्र स्वर्णस्य लोभेन देवतादर्शनाय च}%॥ ३८ ॥

\twolineshloka
{नित्यं चैवागमिष्यन्ति त्यक्त्वा रक्षःकृतं भयम्}
{तेषां यदि वधं कश्चिद्राक्षसात्प्रापयिष्यति}%॥ ३९ ॥

\threelineshloka
{भविष्यति च मे दोषः प्रभुद्रोहोद्भवः प्रभो}
{तस्मात्कञ्चिदुपायं त्वं चिन्तयस्व यथा मम}
{आज्ञाभङ्गकृतं पापं जायते न गुरो क्वचित्}%॥ ४० ॥

\twolineshloka
{तस्य तद्वचनं श्रुत्वा ततः स रघुसत्तमः}
{बाढमित्येव चोक्त्वाथ चापं सज्जीचकार सः}%॥ ४१ ॥

\twolineshloka
{ततस्तं कीर्तिरूपं च मध्यदेशे रघूत्तमः}
{अच्छिनन्निशितैर्बाणैर्दशयोजनविस्तृतम्}%॥ ४२ ॥

\twolineshloka
{तेन संस्थापितो यत्र शिखरे शङ्करः स्वयम्}
{शिखरं तत्सलिङ्गं च पतितं वारिधेर्जले}%॥ ४३ ॥

\twolineshloka
{एवं मार्गमगम्यं तं कृत्वा सेतुसमुद्भवम्}
{वानरै राक्षसैः सार्धं ततः सम्प्रस्थितो गृहम्}%॥ ४४ ॥

॥इति श्रीस्कान्दे महापुराण एकाशीतिसाहस्र्यां संहितायां षष्ठे नागरखण्डे हाटकेश्वरक्षेत्रमाहात्म्ये सेतुमध्ये श्रीरामकृतरामेश्वरप्रतिष्ठावर्णनं नामैकोत्तरशततमोऽध्यायः॥१०१॥

\dnsub{द्व्युत्तरशततमोऽध्यायः --- श्रीरामचन्द्रेण हाटकेश्वरक्षेत्रे लक्ष्मणादिप्रासादपञ्चकनिर्माणप्रतिष्ठापनवर्णनम्}\resetShloka

\uvacha{सूत उवाच}

\twolineshloka
{सम्प्रस्थितस्य रामस्य स्वकीयं सदनं प्रति}
{यदाश्चर्यमभून्मार्गे श्रूयतां द्विजसत्तमाः}%॥ १ ॥

\twolineshloka
{नभोमार्गेण गच्छत्तद्विमानं पुष्पकं द्विजाः}
{अकस्मादेव सञ्जातं निश्चलं चित्रकृन्नृणाम्}%॥ २ ॥

\twolineshloka
{अथ तन्निश्चलं दृष्ट्वा पुष्पकं गगनाङ्गणे}
{रामो वायुसुतस्येदं वचनं प्राह विस्मयात्}%॥ ३ ॥

\twolineshloka
{त्वं गत्वा मारुते शीघ्रं भूमिं जानीहि कारणम्}
{किमेतत्पुष्पकं व्योम्नि निश्चलत्वमुपागतम्}%॥ ४ ॥

\twolineshloka
{कदाचिद्धार्यते नास्य गतिः कुत्रापि केनचित्}
{ब्रह्मदृष्टिप्रसूतस्य पुष्पकस्य महात्मनः}%॥ ५ ॥

\twolineshloka
{बाढमित्येव स प्रोच्य हनूमान्धरणीतलम्}
{गत्वा शीघ्रं पुनः प्राह प्रणिपत्य रघूत्तमम्}%॥ ६ ॥

\twolineshloka
{अत्रास्याधः शुभं क्षेत्रं हाटकेश्वर संज्ञितम्}
{यत्र साक्षाज्जगत्कर्ता स्वयं ब्रह्मा व्यवस्थितः}%॥ ७ ॥

\twolineshloka
{आदित्या वसवो रुद्रा देववैद्यौ तथाश्विनौ}
{तत्र तिष्ठन्ति ते सर्वे तथान्ये सिद्धकिन्नराः}%॥ ८ ॥

\twolineshloka
{एतस्मात्कारणान्नैतदतिक्रामति पुष्पकम्}
{तत्क्षेत्रं निश्चलीभूतं सत्यमेतन्मयोदितम्}%॥ ९ ॥

\uvacha{सूत उवाच}

\twolineshloka
{तस्य तद्वचनं श्रुत्वा कौतूहलसमवितः} 
{पुष्पकं प्रेरयामास तत्क्षेत्रं प्रति राघवः} 

\twolineshloka
{सर्वैस्तैर्वानरैः सार्धं राक्षसैश्च पृथग्विधैः}
{अवतीर्य ततो हृष्टस्तस्मिन्क्षेत्रे समन्ततः}%॥ ११ ॥

\threelineshloka
{तीर्थमालोकयामास पुण्यान्यायतनानि च}
{ततो विलोकयामास पितामहविनिर्मिताम्}
{चामुण्डां तत्र च स्नात्वा कुण्डे कामप्रदायिनि}%॥ १२ ॥

\twolineshloka
{ततो विलोकयामास पित्रा तस्य विनिर्मितम्}
{रामः स्वमिव देवेशं दृष्ट्वा देवं चतुर्भुजम्}%॥ १३ ॥

\twolineshloka
{राजवाप्यां शुचिर्भूत्वा स्नात्वा तर्प्य निजान्पितॄन्}
{ततश्च चिन्तयामास क्षेत्रे त्र बहुपुण्यदे}%॥ १४ ॥

\twolineshloka
{लिङ्गं संस्थापयाम्येव यद्वत्तातेन केशवः}
{तथा मे दयितो भ्राता लक्ष्मणो दिवमाश्रितः}%॥ १५ ॥

\threelineshloka
{यस्तस्य नामनिर्दिष्टं लिङ्गं संस्थापयाम्यहम्}
{तं चापि मूर्तिमन्तं च सीतया सहितं शुभम्}
{क्षेत्रे मेध्यतमे चात्र तथात्मानं दृषन्मयम्}%॥ १६ ॥

\twolineshloka
{एवं स निश्चयं कृत्वा प्रासादानां च पञ्चकम्}
{स्थापयामास सद्भक्त्या रामः शस्त्रभृतां वरः}%॥ १७ ॥

\twolineshloka
{ततस्ते वानराः सर्वे राक्षसाश्च विशेषतः}
{लिङ्गानि स्थापयामासुः स्वानिस्वानि पृथक्पृथक्}%॥ १८ ॥

\twolineshloka
{तत्रैव सुचिरं कालं स्थितास्ते श्रद्धयाऽन्विताः}
{ततो जग्मुरयोध्यायां विमानवरमाश्रिताः}%॥ १९ ॥

\twolineshloka
{एतद्वः सर्वमाख्यातं यथा रामेश्वरो महान्}
{लक्ष्मणेश्वरसंयुक्तस्तस्मिंस्तीर्थे सुशोभने}%॥ २० ॥

\twolineshloka
{यस्तौ प्रातः समुत्थाय सदा पश्यति मानवः}
{स कृत्स्नं फलमाप्नोति श्रुते रामायणेऽत्र यत्}%॥ २१ ॥

\twolineshloka
{अथाष्टम्यां चतुर्दश्यां यो रामचरितं पठेत्}
{तदग्रे वाजिमेधस्य स कृत्स्नं लभते फलम्}%॥ २२ ॥

॥इति श्रीस्कान्दे महापुराण एकाशीतिसाहस्र्यां संहितायां षष्ठे नागरखण्डे हाटकेश्वरक्षेत्रमाहात्म्ये श्रीरामचन्द्रेण हाटकेश्वरक्षेत्रे लक्ष्मणादिप्रासादपञ्चकनिर्माणप्रतिष्ठापनवर्णनं नाम द्व्युत्तरशततमोऽध्यायः॥१०२॥
    \sect{रामेश्वरक्षेत्रमाहात्म्यवर्णनम्}

\src{स्कन्दपुराणम्}{खण्डः ७ (प्रभासखण्डः)}{प्रभासक्षेत्र माहात्म्यम्}{अध्यायः १११}
\vakta{}
\shrota{}
\tags{}
\notes{}
\textlink{https://sa.wikisource.org/wiki/स्कन्दपुराणम्/खण्डः_७_(प्रभासखण्डः)/प्रभासक्षेत्र_माहात्म्यम्/अध्यायः_१११}
\translink{https://www.wisdomlib.org/hinduism/book/the-skanda-purana/d/doc626899.html}

\storymeta




\uvacha{ईश्वर उवाच}

\twolineshloka
{ततो गच्छेन्महादेवि पुष्करारण्यमुत्तमम्}
{तस्मादीशानकोणस्थं धनुषां षष्टिभिः स्थितम्}%॥ १ ॥

\twolineshloka
{तत्र कुण्डं महादेवि ह्यष्टपुष्करसंज्ञितम्}
{सर्व पापहरं देवि दुष्प्राप्यमकृतात्मभिः}%॥ २ ॥

\twolineshloka
{तत्र कुण्डसमीपे तु पुरा रामेशधीमता}
{स्थापितं तन्महालिङ्गं रामेश्वर इति स्मृतम्}%॥ ३

\onelineshloka
{तस्य पूजनमात्रेण मुच्यते ब्रह्महत्यया}%॥ ४ ॥

\uvacha{श्रीदेव्युवाच}

\twolineshloka
{भगवन्विस्तराद्ब्रूहि रामेश्वरसमुद्भवम्}
{कथं तत्रागमद्रामः ससीतश्च सलक्ष्मणः}%॥ ५ ॥

\twolineshloka
{कथं प्रतिष्ठितं लिङ्गं पुष्करे पापतस्करे}
{एतद्विस्तरतो ब्रूहि फलं माहात्म्यसंयुतम्}%॥ ६ ॥

\uvacha{ईश्वर उवाच}

\twolineshloka
{चतुर्विंशयुगे रामो वसिष्ठेन पुरोधसा}
{पुरा रावणनाशार्थं जज्ञे दशरथात्मजः}%॥ ७ ॥

\twolineshloka
{ततः कालान्तरे देवि ऋषिशापान्महातपाः}
{ययौ दाशरथी रामः ससीतः सहलक्ष्मणः}%॥ ८ ॥

\twolineshloka
{वनवासाय निष्क्रान्तो दिव्यैर्ब्रह्मर्षिभिर्वृतः}
{ततो यात्राप्रसङ्गेन प्रभासं क्षेत्रमागतः}%॥ ९ ॥

\twolineshloka
{तं देशं तु समासाद्य सुश्रान्तो निषसाद ह}
{अस्तं गते ततः सूर्ये पर्णान्यास्तीर्य भूतले}%॥ १० ॥

\twolineshloka
{सुष्वापाथ निशाशेषे ददृशे पितरं स्वकम्}
{स्वप्ने दशरथं देवि सौम्यरूपं महाप्रभम्}%॥ ११ ॥

\twolineshloka
{प्रातरुत्थाय तत्सर्वं ब्राह्मणेभ्यो न्यवेदयत्}
{यथा दशरथः स्वप्ने दृष्टस्तेन महात्मना}%॥ १२ ॥

\uvacha{ब्राह्मणा ऊचुः}

\twolineshloka
{वृद्धिकामाश्च पितरो वरदास्तव राघव}
{दर्शनं हि प्रयच्छन्ति स्वप्नान्ते हि स्ववंशजे}%॥ १३ ॥

\twolineshloka
{एतत्तीर्थं महापुण्यं सुगुप्तं शार्ङ्गधन्वनः}
{पुष्करेति समाख्यातं श्राद्धमत्र प्रदीयताम्}%॥ १४ ॥

\twolineshloka
{नूनं दशरथो राजा तीर्थे चास्मिन्समीहते}
{त्वया दत्तं शुभं पिण्डं ततः स दर्शनं गतः}%॥ १५ ॥

\uvacha{ईश्वर उवाच}

\twolineshloka
{तेषां तद्वचनं श्रुत्वा रामो राजीवलोचनः}
{निमन्त्रयामास तदा श्राद्धार्हान्ब्राह्मणाञ्छुभान्}%॥ १६ ॥

\twolineshloka
{अब्रवील्लक्ष्मणं पार्श्वे स्थितं विनतकन्धरम्}
{फलार्थं व्रज सौमित्रे श्राद्धार्थं त्वरयाऽन्वितः}%॥ १७ ॥

\twolineshloka
{स तथेति प्रतिज्ञाय जगाम रघुनन्दनः}
{आनयामास शीघ्रं स फलानि विविधानि च}%॥ १८ ॥

\twolineshloka
{बिल्वानि च कपित्थानि तिन्दुकानि च भूरिशः}
{बदराणि करीराणि करमर्दानि च प्रिये}%॥ १९ ॥

\twolineshloka
{चिर्भटानि परूषाणि मातुलिङ्गानि वै तथा}
{नालिकेराणि शुभ्राणि इङ्गुदीसम्भवानि च}%॥ २० ॥

\twolineshloka
{अथैतानि पपाचाशु सीता जनकनन्दिनी}
{ततस्तु कुतपे काले स्नात्वा वल्कलभृच्छुचिः}%॥ २१ ॥

\twolineshloka
{ब्राह्मणानानयामास श्राद्धार्हान्द्विजसत्तमान्}
{गालवो देवलो रैभ्यो यवक्रीतोऽथ पर्वतः}%॥ २२ ॥

\twolineshloka
{भरद्वाजो वसिष्ठश्च जावालिर्गौतमो भृगुः}
{एते चान्ये च बहवो ब्राह्मणा वेदपारगाः}%॥ २३ ॥

\twolineshloka
{श्राद्धार्थं तस्य सम्प्राप्ता रामस्याक्लिष्टकर्मणः}
{एतस्मिन्नेव काले तु रामः सीतामभाषत}%॥ २४ ॥

\twolineshloka
{एहि वैदेहि विप्राणां देहि पादावनेजनम्}
{एतच्छ्रुत्वाऽथ सा सीता प्रविष्टा वृक्षमध्यतः}%॥ २५ ॥

\twolineshloka
{गुल्मैराच्छाद्य चात्मानं रामस्यादर्शने स्थिता}
{मुहुर्मुहुर्यदा रामः सीतासीतामभाषत}%॥ २६ ॥

\twolineshloka
{ज्ञात्वा तां लक्ष्मणो नष्टां कोपाविष्टं च राघवम्}
{स्वयमेव तदा चक्रे ब्राह्मणार्ह प्रतिक्रियाम्}%॥ २७ ॥

\twolineshloka
{अथ भुक्तेषु विप्रेषु कृत पिण्डप्रदानके}
{आगता जानकी सीता यत्र रामो व्यवस्थितः}%॥ २८ ॥

\threelineshloka
{तां दृष्ट्वा परुषैर्वाक्यैर्भर्त्सयामास राघवः}
{धिग्धिक्पापे द्विजांस्त्यक्त्वा पितृकृत्यमहोदयम्}
{क्व गताऽसि च मां हित्वा श्राद्धकाले ह्युपस्थिते}%॥ २९ ॥

\uvacha{ईश्वर उवाच}

\onelineshloka
{तस्य तद्वचनं श्रुत्वा भयभीता च जानकी}%॥ ३० ॥

\twolineshloka
{कृताञ्जलिपुटा भूत्वा वेपमाना ह्यभाषत}
{मा कोपं कुरु कल्याण मा मां निर्भर्त्सय प्रभो}%॥ ३१ ॥

\twolineshloka
{शृणु यस्माद्विभोऽन्यत्र गता त्यक्त्वा तवान्तिकम्}
{दृष्टस्तत्र पिता मेऽद्य तथा चैव पितामहः}%॥ ३२ ॥

\twolineshloka
{तस्य पूर्वतरश्चापि तथा मातामहादयः}
{अङ्गेषु ब्राह्मणेन्द्राणामाक्रान्तास्ते पृथक्पृथक्}%॥ ३३ ॥

\twolineshloka
{ततो लज्जा समभवत्तत्र मे रघुनन्दन}
{पित्रा तत्र महाबाहो मनोज्ञानि शुभानि च}%॥ ३४ ॥

\threelineshloka
{भक्ष्याणि भक्षितान्येव यानि वै गुणवन्ति च}
{स कथं सुकषायाणि क्षाराणि कटुकानि च}
{भक्षयिष्यति राजेन्द्र ततो मे दुःखमाविशत्}%॥ ३६ ॥

\twolineshloka
{एतस्मात्कारणान्नष्टा लज्जयाऽहं रघूद्वह}
{दृष्ट्वा श्वशुरवर्गं स्वं तस्मात्कोपं परित्यज}%॥ ३६ ॥

\twolineshloka
{तस्यास्तद्वचनं श्रुत्वा विस्मितो राघवोऽभवत्}
{विशेषेण ददौ तस्मिञ्छ्राद्धं तीर्थे तु पुष्करे}%॥ ३७ ॥

\twolineshloka
{तत्र पुष्करसान्निध्ये दक्षिणे धनुषां त्रये}
{लिङ्गं प्रतिष्ठयामास रामेश्वरमिति श्रुतम्}%॥ ३५ ॥

\twolineshloka
{यस्तं पूजयते भक्त्या गन्धपुष्पादिभिः क्रमात्}
{स प्राप्नोति परं स्थानं य्रत्र देवो जनार्दनः}%॥ ३९ ॥

\twolineshloka
{किमत्र बहुनोक्तेन द्वादश्यां यत्प्रदापयेत्}
{न तत्र परिसङ्ख्यानं त्रिषु लोकेषु विद्यते}%॥ ४० ॥

\twolineshloka
{शुक्राङ्गारकसंयुक्ता चतुर्थी या भवेत्क्वचित्}
{षष्ठी वात्र वरारोहे तत्र श्राद्धे महत्फलम्}%॥ ४१ ॥

\twolineshloka
{यावद्द्वादशवर्षाणि पितरश्च पितामहाः}
{तर्पिता नान्यमिच्छन्ति पुष्करे स्वकुलोद्भवे}%॥ ४२ ॥

\twolineshloka
{तत्र यो वाजिनं दद्यात्सम्यग्भक्तिसमन्वितः}
{अश्वमेधस्य यज्ञस्य फलं प्राप्नोति मानवः}%॥ ४३ ॥

\twolineshloka
{इति ते कथितं सम्यङ्माहात्म्यं पापनाशनम्}
{रामेश्वरस्य देवस्य पुष्करस्य च भामिनि}%॥ ४४ ॥

॥इति श्रीस्कान्दे महापुराण एकाशीतिसाहस्र्यां संहितायां सप्तमे प्रभासखण्डे प्रथमे प्रभासक्षेत्रमाहात्म्ये पुष्करमाहात्म्ये रामेश्वरक्षेत्रमाहात्म्यवर्णनं नामैकादशोत्तरशततमोऽध्यायः॥१११॥
    \sect{द्वादशोत्तरशततमोऽध्यायः --- लक्ष्मणेश्वरमाहात्म्यवर्णनम्}

\src{स्कन्दपुराणम्}{खण्डः ७ (प्रभासखण्डः)}{प्रभासक्षेत्र माहात्म्यम्}{अध्यायः ११२}
\vakta{}
\shrota{}
\tags{}
\notes{}
\textlink{https://sa.wikisource.org/wiki/स्कन्दपुराणम्/खण्डः_७_(प्रभासखण्डः)/प्रभासक्षेत्र_माहात्म्यम्/अध्यायः_११२}
\translink{https://www.wisdomlib.org/hinduism/book/the-skanda-purana/d/doc626900.html}

\storymeta




\uvacha{ईश्वर उवाच}

\twolineshloka
{ततो गच्छेन्महादेवि लक्ष्मणेश्वरमुत्तमम्}
{रामेशात्पूर्वदिग्भागे धनुस्त्रिंशकसंस्थितम्}%॥ १ ॥

\twolineshloka
{स्थापितं लक्ष्मणेनैव तत्र यात्रागतेन वै}
{महापापहरं देवि तल्लिङ्गं सुरपूजितम्}%॥ २ ॥

\twolineshloka
{यस्तं पूजयते भक्त्या नृत्यगीतादिवादनैः}
{होमजाप्यैः समाधिस्थः स याति परमां गतिम्}%॥ ३ ॥

\twolineshloka
{अन्नोदकं हिरण्यं च तत्र देयं द्विजातये}
{सम्पूज्य देवदेवेशं गन्धपुष्पादिभिः क्रमात्}%॥ ४ ॥

\twolineshloka
{माघे कृष्णचतुर्दश्यां विशेषस्तत्र पूजने}
{स्नानं दानं जपस्तत्र भवेदक्षयकारकम्}%॥ ५ ॥
॥इति श्रीस्कान्दे महापुराण एकाशीतिसाहस्र्यां संहितायां सप्तमे प्रभासखण्डे प्रथमे प्रभासक्षेत्रमाहात्म्ये रामेश्वरक्षेत्रमाहात्म्ये लक्ष्मणेश्वरमाहात्म्यवर्णनं नाम द्वादशोत्तरशततमोऽध्यायः॥११२॥
    \sect{जानकीश्वरमाहात्म्यवर्णनम्}

\src{स्कन्दपुराणम्}{खण्डः ७ (प्रभासखण्डः)}{प्रभासक्षेत्र माहात्म्यम्}{अध्यायः ११३}
\vakta{}
\shrota{}
\tags{}
\notes{}
\textlink{https://sa.wikisource.org/wiki/स्कन्दपुराणम्/खण्डः_७_(प्रभासखण्डः)/प्रभासक्षेत्र_माहात्म्यम्/अध्यायः_११३}
\translink{https://www.wisdomlib.org/hinduism/book/the-skanda-purana/d/doc626901.html}

\storymeta




\uvacha{ईश्वर उवाच}

\twolineshloka
{ततो गच्छेन्महादेवि जानकीश्वरमुत्तमम्}
{रामेशान्नैऋते भागे धनुस्त्रिंशकसंस्थितम्}%॥ १ ॥

\twolineshloka
{पापघ्नं सर्वजन्तूनां जानक्याऽऽराधितं पुरा}
{प्रतिष्ठितं विशेषेण सम्यगाराध्यशङ्करम्}%॥ २ ॥

\twolineshloka
{पूर्वं तस्यैव लिङ्गस्य वसिष्ठेशेति नाम वै}
{तत्पश्चाज्जानकीशेति त्रेतायां प्रथितं क्षितौ}%॥ ३ ॥

\twolineshloka
{ततः षष्टिसहस्राणि वालखिल्या महर्षयः}
{तत्र सिद्धिमनुप्राप्तास्तेन सिद्धेश्वरेति च}%॥ ४ ॥

\twolineshloka
{ख्यातं कलौ महादेवि युगलिङ्गं महाप्रभम्}
{तद्दृष्ट्वा मुच्यते पापैर्दुःखदौर्भाग्यसम्भवैः}%॥ ५ ॥

\twolineshloka
{यस्तं पूजयते भक्त्या नारी वा पुरुषोऽपि वा}
{संस्नाप्य विधिवद्भक्त्या स मुक्तः पातकैर्भवेत्}%॥ ६ ॥

\twolineshloka
{स्नात्वा च पुष्करे तीर्थे यस्तल्लिगं प्रपूजयेत्}
{नियतो नियताहारो मासमेकं निरन्तरम्}%॥ ७ ॥

\threelineshloka
{दिनेदिने भवेत्तस्य वाजिमेधाधिकं फलम्}
{माघे मासि तृतीयायां या नारी तं प्रपूजयेत्}
{तदन्वयेऽपि दौर्भाग्यं दुःखं शोकश्च नो भवेत्}%॥ ८ ॥

\twolineshloka
{इति ते कथितं देवि माहात्म्यं पापनाशनम्}
{श्रुतं हरति पापानि सौभाग्यं सम्प्रयच्छति}%॥ ९ ॥

॥इति श्रीस्कान्दे महापुराण एकाशीतिसाहस्र्यां संहितायां सप्तमे प्रभासखण्डे प्रथमे प्रभासक्षेत्रमाहात्म्ये जानकीश्वरमाहात्म्यवर्णनं नाम त्रयोदशोत्तरशततमोऽध्यायः॥११३॥
    \sect{दशरथेश्वरमाहात्म्यवर्णनम्}

\src{स्कन्दपुराणम्}{खण्डः ७ (प्रभासखण्डः)}{प्रभासक्षेत्र माहात्म्यम्}{अध्यायः १७१}
\vakta{}
\shrota{}
\tags{}
\notes{}
\textlink{https://sa.wikisource.org/wiki/स्कन्दपुराणम्/खण्डः_७_(प्रभासखण्डः)/प्रभासक्षेत्र_माहात्म्यम्/अध्यायः_१७१}
\translink{https://www.wisdomlib.org/hinduism/book/the-skanda-purana/d/doc626959.html}

\storymeta




\uvacha{ईश्वर उवाच}

\twolineshloka
{ततो गच्छेन्महादेवि देवीमेकल्लवीरिकाम्}
{एकल्लवीरायाम्ये तु नातिदूरे व्यवस्थिताम्}%॥ १ ॥

\twolineshloka
{पूर्वं दशरथो योऽसौ सूर्यवंशविभूषणः}
{प्रभासं क्षेत्रमासाद्य तपश्चक्रे सुदुश्चरम्}%॥ २ ॥

\twolineshloka
{लिङ्गं तत्र प्रतिष्ठाप्य तोषयामास शाङ्करम्}
{स देवं प्रार्थयामास पुत्रं चैवामितौजसम्}%॥ ३ ॥

\twolineshloka
{ददौ तस्य तदा पुत्रं देवं त्रैलोक्यपूजितम्}
{रामेति नाम यस्यासीत्त्रैलोक्ये प्रथितं यशः}%॥ ४ ॥

\twolineshloka
{यस्याद्यापीह गायन्ति भूर्भुवःस्वर्नि वासिनः}
{देवदैत्यासुराः सर्वे वाल्मीक्याद्या महर्षयः}%॥ ५ ॥

\threelineshloka
{तल्लिङ्गस्य प्रभावेन प्राप्तं राज्ञा महद्यशः}
{कार्तिक्यां कार्तिके मासि विधिना यस्तमर्चयेत्}
{दीपपूजोपहारेण यशस्वी सोऽपि जायते}%॥ ६ ॥

॥इति श्रीस्कान्दे महापुराण एकाशीतिसाहस्र्यां संहितायां सप्तमे प्रभासखण्डे प्रथमे प्रभासक्षेत्रमाहात्म्ये दशरथेश्वरमाहात्म्यवर्णनं नामैकसप्तत्युत्तरशततमोऽध्यायः॥१७१॥

    \part{रामायणान्तर्गत-स्तोत्राणि}
    \input{stotram/nama-ramayanam}
    \chapt{वाल्मीकि-रामायणम्}
    \input{stotram/valmiki-ramayanam/brahma-krta-rama-stava}
    \chapt{स्कन्द-पुराणम्}
    \input{stotram/skanda-puranam/rama-stotram.tex}


    
    \part{अनुबन्धाः}
    \begingroup
    \chapt{ध्यान-मङ्गल-श्लोकाः}
    \sect{वाल्मीकि-रामायणम्}
    \input{anubandha/valmiki-ramayanam/ramayana-dhyanam}
    \input{anubandha/valmiki-ramayanam/ramayana-mangala-shloka}
    \sect{वराह-पुराणम्}
    \dnsub{ध्यान-श्लोकाः}

\twolineshloka
{नारायणं नमस्कृत्य नरं चैव नरोत्तमम्}
{देवीं सरस्वतीं चैव ततो जयमुदीरयेत्}

\twolineshloka
{नमस्तस्मै वराहाय लीलयोद्धरते महीम्}
{खुरमध्यगतो यस्य मेरुः खणखणायते}

\fourlineindentedshloka
{दंष्ट्राग्रेणोद्धृता गौरुदधिपरिवृता पर्वतैर्निम्नगाभिः}
{साकं मृत्पिण्डवत् प्राग्बृहदुरुवपुषाऽनन्तरूपेण येन}
{सोऽयं कंसासुरारिर्मुरनरकदशास्यान्तकृत्सर्वसंस्थः}
{कृष्णो विष्णुः सुरेशो नुदतु मम रिपूनादिदेवो वराहः}

\fourlineindentedshloka
{यः संसारार्णवे नौरिव मरणजराव्याधिनक्रोर्मिभीमे}
{भक्तानां भीतिहर्ता मुरनरकदशास्यान्तकृत् कोलरूपी}
{विष्णुः सर्वेश्वरोऽयं यमिह कृतधियो लीलया प्राप्नुवन्ति}
{मुक्तात्मानो न पापं प्रभवमनुदिनारातिपक्षः क्षितीशः}

    \closesection
    \chapt{नामावल्यः}
    \let\chapt\sect
    \input{../namavali-manjari/100/Rama_108}
    \input{../namavali-manjari/100/Sita_108}
    \input{../namavali-manjari/100/Anjaneya_108}
    \input{../namavali-manjari/100/Rama_Ramarahasya_108}
    \input{../namavali-manjari/100/Sita_Ramarahasya_108}
    \input{../namavali-manjari/100/Anjaneya_Ramarahasya_108}
    \endgroup
    
\end{center}


\newpage\mbox{}
\clearemptydoublepage

\end{document}
